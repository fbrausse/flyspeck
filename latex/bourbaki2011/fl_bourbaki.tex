\documentclass[brochure,english,12pt]{bourbaki}
\usepackage[matrix,arrow]{xy}
\usepackage{amssymb,amsfonts,amsmath,footnote}
\usepackage[francais]{babel}
\addressindent 100mm

\date{Avril 2011}
\bbkannee{63\`eme ann\'ee, 2010-2011}
\bbknumero{1035}
\title{The fundamental lemma and the Hitchin fibration}
\subtitle{after Ng\^o Bao Ch\^au}
\author{Thomas HALES}
\address{University of Pittsburgh\\
Department of Mathematics\\
Pittsburgh, PA 15260 -- U.S.A.}
\email{tchales@gmail.com}


% Math notation.
\def\op#1{{\operatorname{#1}}}
\newcommand{\ring}[1]{\mathbb{#1}}


\begin{document}
\maketitle

\noindent{\bf INTRODUCTION}

\section{Introduction}
% 7 pages.



{\narrower \it

``Mais m\^eme apr\`es avoir
v\'erifi\'e que les facteurs de
transfert existent, il reste \`a v\'erifier ce que j'appelle le
lemme fondamental, qui affirme que pour des $G$, $H$ et $\phi_H$
non-ramifi\'es, on a $f\mapsto c \phi_H^*(f)$ [pour toute fonction $f\in {\mathcal H}_G$]. -- p.49 (Les deb.)  
}

\bigskip

``For other problems the correspondence is not given by a function, and the notions of 
stabilization and endoscopy necessary to circumvent the attendant difficulties are perhaps 
the most startling that the trace formula has suggested to harmonic analysis,'' 

from p21 Eisenstein series, the trace formula, and the modern theory of automorphic forms, http://publications.ias.edu/sites/default/files/eisen6-ps.pdf

"Le but le plus profond du principe de fonctorialit\'e ... est de remplacer chaque fonction L avec une partie galoisienne, par exemple chaque fonction L d'Artin, par une fonction L enti\`erement automorphe.'' Les deb. p 1.





\subsection{Origins of the fundamental lemma}

We find an early glimpse of the nascent theory in an exchange between Langlands and Singer.  Singer
expressed interest to Langlands in a particular alternating sum of
dimensions of spaces of cusp forms of $G=SL_2$ over a totally real
extension $F$ of $\ring{Q}$.  In 1974, Langlands replied to Singer in
a letter describing then unpublished joint work with
Labesse~\cite{Singer},~\cite{LL}.  The letter gives a formula for
this alternating sum, with terms indexed by the groups of norm 1
elements of quadratic extensions $E$ of $F$.  From these early
calculations of Labesse and Langlands, the general idea  developed that one should account for
alternating sums that appear in the
harmonic analysis on a reductive group $G$ in terms of the harmonic analysis on groups $H$ of smaller dimension.  
These smaller groups $H$ are called endoscopic groups.  
%Letter to Singer 1974.

The trace formula

XX Insert similar to NOTICES.

Local and Global Calculations.

The ring of adeles is a restricted direct product.  Tate's thesis,
calculations over the adeles by local calculations at each place.
Local calculations by approximation: $F$ is dense in $F_v$.


Calculations of orbital integrals.  Let us briefly review how Labesse and Langlands rearranged the terms of trace formula for $SL_2$ to make the quadratic extensions $E$ of $F$ appear.  Many conjugacy classes of $SL_2(F)$ have the same characteristic polynomial.  For example, if $F=\ring{Q}$,
the 

\section{Proof} % 7 pages.

\section{Applications of the Fundamental Lemma}  
% 6 pages.




\section{Appendix: Reductions and Supplementary Results (3) pages}



This appendix describes some theorems related to the fundamental lemma.

Twisted Fundamental Lemma, Lie algebra, Weighted Fundamental Lemma, Reduction to Units, Smooth transfer, descent, transfer to char 0.

\subsection{The full Hecke algebra}

The fundamental lemma, as originally stated by Langlands, is a statement
about the transfer of all functions in the spherical Hecke algebra.
A global argument based on the trace formula shows that the fundamental
lemma holds
for the full Hecke algebra for an arbitrary nonarchimedean local
field, provided it holds for the unit element of the Hecke algebra for
local fields of sufficiently large residual characteristic.

\subsection{Smooth Transfer}



\subsection{Transfer to Characteristic Zero}

The fundamental lemma for nonarchimedean local fields in
characteristic zero can be derived from the fundamental lemma in
positive characteristic [Wald], [CHL].  At the risk of
oversimplification, the basic idea of Cluckers and Loeser is that the
process of integration of large class of integrals can be described
abstractly as a result of more primitive operations such as taking the
volume of a ball of given radius, counting the number of points on a
variety over the finite residue field, summing a hypergeometric
$q$-series, and making a change of variables.  Since each of these
more primitive operations depends only on the residue field and not on
the field itself, many identities of integrals can be transferred from
one field to another.  The FL lemma is a collection of identities that
fall within the scope of this theory.

\subsection{Descent of identities of orbital integrals to the Lie algebra.}

A lemma of Harish-Chandra's asserts the transfer of an orbital
integral on $G$ near a singular semisimple element $\gamma_0$ to an
orbital integral on $C_G(\gamma_0)$, the centralizer of $\gamma_0$.
This is called the descent of orbital integrals.  Langlands and
Shelstad made difficult calculations in Galois cohomology to prove
that the transfer factors are compatible with Harish-Chandra's descent
of orbital integrals.  A particular consequence of this work is that
the fundamental lemma follows from corresponding identities in the Lie
algebra (together with the fundamental lemmas on smaller Lie algebras
obtained by descent).  See [Hales].

The original fundamental lemma has been supplemented by a twisted
fundamental lemma by Kottwitz and Shelstad, where the data is twisted
by a nontrivial outer automorphism $\theta$ of the group $G$.  The
corresponding difficult calculations in Galois cohomology that
establish descent for the twisted fundamental lemma have been carried
out by Waldspurger.  In the untwisted case, the centralizer is all of
$G$ precisely when $\gamma_0$ is central.  By contrast, a twisted
centralizer (with respect to a nontrivial outer automorphism) is never
all of $G$.  This means that descent always ``untwists'' the twisted
fundamental lemma into some ``nonstandard'' form.  After descent, the twisted FL
takes the form of identities of stable orbital
integrals on the Lie algebra (from which the automorphism has entirely
vanished).  It is in this untwisted form that NBC proved the twisted
fundamental lemma.


\end{document}