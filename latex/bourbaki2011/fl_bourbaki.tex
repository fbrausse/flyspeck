\documentclass[brochure,english,12pt]{bourbaki}
\usepackage[matrix,arrow]{xy}
\usepackage{amssymb,amsfonts,amsmath,footnote}
\usepackage[francais]{babel}
\addressindent 100mm

\date{Avril 2011}
\bbkannee{63\`eme ann\'ee, 2010-2011}
\bbknumero{1035}
\title{The fundamental lemma and the Hitchin fibration}
\subtitle{after Ng\^o Bao Ch\^au}
\author{Thomas HALES}
\address{University of Pittsburgh\\
Department of Mathematics\\
Pittsburgh, PA 15260 -- U.S.A.}
\email{tchales@gmail.com}

\newtheorem{example}[equation]{Example}
% Math notation.
\def\op#1{{\operatorname{#1}}}
\newcommand{\ring}[1]{\mathbb{#1}}
\newcommand{\NBC}{Ng\^o Bao Ch\^au}

\begin{document}
\maketitle

\noindent{\bf INTRODUCTION}

\section{Introduction}
% 7 pages.

I am here today to describe some identities of integrals that have been established
by \NBC.  In mathematics, we have an endless supply of integrals and of identities of integrals.  Part of my task today will be to describe why these identities took nearly 30 years to prove, and why they have particular importance for investigations in the theory of automorphic representations.






\subsection{Origins of the fundamental lemma}

\subsection{Examples from $SL_2$}

To orient ourselves to the fundamental lemma, we 
briefly describe two examples.

XX SL2 discrete series representations.

\begin{example}  For each $n\ge 2$, 
$SL_2(\ring{R})$ acts on the vector space of homomorphic functions on the
upper half plane by the formula by the formula:
\[
XX
\]
There is a corresponding action on functions on the lower half plane.  These
two infinite dimensional representations are irreducible and inequivalent to
one another.  The characters of these representations are represented by functions
$\chi_{n,\pm}$.   The difference of these two characters is given by the
formula
\[
XX
\]
It is striking that up to the factor $D$, the difference of two characters of infinite dimension representations collapses to a finite-dimensional representation of the circle group $H$.  Shelstad gives general characters identities of this sort~\cite{Sh}.
\end{example}

\begin{example}
We find another early glimpse of the nascent theory in an exchange between Langlands and Singer.  Singer
expressed interest to Langlands in a particular alternating sum of
dimensions of spaces of cusp forms of $G=SL_2$ over a totally real
extension $F$ of $\ring{Q}$.  In 1974, Langlands replied to Singer in
a letter describing then unpublished joint work with
Labesse~\cite{Singer},~\cite{LL}.  The letter gives a formula for
this alternating sum, with terms indexed by the groups $H$ of norm 1
elements of quadratic extensions of $F$.  
\end{example}


From these early
calculations of Labesse and Langlands, the general idea  developed that one should account for
alternating sums that appear in the
harmonic analysis on a reductive group $G$ in terms of the harmonic analysis on groups $H$ of smaller dimension.  
These smaller groups $H$ are called endoscopic groups.  
%Letter to Singer 1974.

The trace formula

XX Insert similar to NOTICES.

Local and Global Calculations.

The ring of adeles is a restricted direct product.  Tate's thesis,
calculations over the adeles by local calculations at each place.
Local calculations by approximation: $F$ is dense in $F_v$.


Calculations of orbital integrals.  Let us briefly review how Labesse and Langlands rearranged the terms of trace formula for $SL_2$ to make the quadratic extensions $E$ of $F$ appear.  Many conjugacy classes of $SL_2(F)$ have the same characteristic polynomial.  For example, if $F=\ring{Q}$,
the 

\section{Proof} % 7 pages.

\section{Applications of the Fundamental Lemma}  
% 6 pages.




\section{Appendix: Reductions and Supplementary Results (3) pages}



This appendix describes some theorems related to the fundamental lemma.

Twisted Fundamental Lemma, Lie algebra, Weighted Fundamental Lemma, Reduction to Units, Smooth transfer, descent, transfer to char 0.

\subsection{Descent of identities of orbital integrals to the Lie algebra.}

A lemma of Harish-Chandra's asserts the transfer of an orbital
integral on $G$ near a singular semisimple element $z\gamma_0$, with
$z$ central, to an orbital integral on $C_G(\gamma_0)$, the
centralizer of $\gamma_0$.  This is called the descent of orbital
integrals.  Langlands and Shelstad made long calculations in
Galois cohomology to prove that the transfer factors are compatible
with Harish-Chandra's descent of orbital integrals.  The purpose of
their calculations was to make it possible to prove results about
orbital integrals by induction on the dimension of the group, and
thereby reduce smooth transfer and the fundamental lemma to identities
in a neighborhood of $\gamma_0=1$ contained in the image of the
exponential map.  Because of descent, identities can be proved in the
Lie algebra rather than in the group.


The original fundamental lemma has been supplemented by a twisted
fundamental lemma, conjectured by Kottwitz and Shelstad, where the
data is twisted by a nontrivial outer automorphism $\theta$ of the
group $G$.  The corresponding long calculations in Galois cohomology
that establish descent for the twisted fundamental lemma have been
carried out by Waldspurger.  In the untwisted case, the centralizer of
an element fails to give a group of smaller dimension precisely when
the element is central.  By contrast, a twisted centralizer (with
respect to a nontrivial outer automorphism) always has dimension less
than $G$.  As a consequence, descent always {\it untwists} the twisted
fundamental lemma into some {\it nonstandard} form.  After descent, the
twisted FL of Kottwitz and Shelstad takes the form of identities of
stable orbital integrals on the Lie algebra (from which the
automorphism and the character $\kappa$ have entirely vanished).  NBC
has proved the general twisted fundamental lemma in its untwisted form
on the Lie algebra.


\subsection{The Hecke algebra}

The fundamental lemma is a statement about the transfer of all
functions in the spherical Hecke algebra.  A global argument based on
the trace formula shows that the fundamental lemma holds for the full
Hecke algebra for an arbitrary nonarchimedean local field of
characteristic zero, provided it holds for the unit element of the
Hecke algebra for local fields of sufficiently large residual
characteristic.  The idea of the proof is to choose suitable global
functions for which the comparison of stable trace formulas yields an
obstruction to the fundamental lemma.  This obstruction takes the form
of a linear map $L:{\mathcal H}\to V$ on the local spherical Hecke
algebra ${\mathcal H}$ of the endoscopic group into a finite
dimensional complex vector space $V$.  By purely local arguments, it
can be shown that no nonzero linear map $L$ exists of the form
prescribed by the global theory.  In fact, the global argument forces $L$
to be tempered, but a local argument shows that if nonzero it is not.
By the vanishing of the obstruction,
the fundamental lemma is seen to hold on the full spherical Hecke algebra.


\subsection{Smooth Transfer}

Langlands's book on the stabilization of the trace formula contains
two separate conjectures.  The first is the transfer of smooth
functions.  The second is the fundamental lemma.  An important result
of Waldspurger links the two conjectures, by proving that the
fundamental lemma implies the transfer of smooth functions.  
His key local lemma shows how to obtain simultaneous control over the
orbital integrals of test functions $f$ and the orbital integrals of
their Fourier transform $\hat f$~\cite[Prop.~8.2]{W}.  
In view of the uncertainty principle, it is a remarkable feat to
control both $f$ and $\hat f$ as he does.
% push up against the limits imposed by the
%uncertainty principle.
%is generally deemed nearly impossible by the
% Prop 8.2, p198.
His proof is a global argument, based on a stable Poisson summation trace
formula.  The key local lemma allows Waldspurger to pick global test
functions for which the comparison of trace formulas asserts a
local identity: the Fourier transform of a semisimple
$\kappa$-orbit on $G$ equals the Fourier transform of the corresponding
stable orbit on $H$.  By a purely local argument, this identity of Fourier
transforms of orbits implies smooth transfer.

\subsection{Weighted orbital integrals}

\subsection{Transfer to Characteristic Zero}

The fundamental lemma for nonarchimedean local fields in
characteristic zero can be deduced from the fundamental lemma in
positive characteristic [Wald], [CHL].  Cluckers and Loeser develop a
general abstract theory of integration as a conglomerate of primitive
operations such as taking the volume of a ball of given radius,
enumerating points on a variety over the residue field, summing a
hypergeometric $q$-series, and making a change of variables.  Since each
of the primitive operations manifestly depends only on the residue
field and not on the field itself, their theory allows many identities
of integrals to be transferred from one field to another with the same
residue field.  The FL lemma is a collection of identities that fall
within the scope of this theory.



\section{Cut}


{\narrower \it

``Mais m\^eme apr\`es avoir
v\'erifi\'e que les facteurs de
transfert existent, il reste \`a v\'erifier ce que j'appelle le
lemme fondamental, qui affirme que pour des $G$, $H$ et $\phi_H$
non-ramifi\'es, on a $f\mapsto c \phi_H^*(f)$ [pour toute fonction $f\in {\mathcal H}_G$]. -- p.49 (Les deb.)  
}

\bigskip

``For other problems the correspondence is not given by a function, and the notions of 
stabilization and endoscopy necessary to circumvent the attendant difficulties are perhaps 
the most startling that the trace formula has suggested to harmonic analysis,'' 

from p21 Eisenstein series, the trace formula, and the modern theory of automorphic forms, http://publications.ias.edu/sites/default/files/eisen6-ps.pdf

"Le but le plus profond du principe de fonctorialit\'e ... est de remplacer chaque fonction L avec une partie galoisienne, par exemple chaque fonction L d'Artin, par une fonction L enti\`erement automorphe.'' Les deb. p 1.


\end{document}