\documentclass[brochure,english,12pt]{bourbaki}
\usepackage[matrix,arrow]{xy}
\usepackage{amssymb,amsfonts,amsmath,footnote}
\usepackage[francais]{babel}
\usepackage{url}
%\usepackage{pxfonts}
\addressindent 100mm

\date{Avril 2011}
\bbkannee{63\`eme ann\'ee, 2010-2011}
\bbknumero{1035}
\title{The fundamental lemma and the Hitchin fibration}
\subtitle{after Ng\^o Bao Ch\^au}
\author{Thomas HALES}
\address{University of Pittsburgh\\
Department of Mathematics\\
Pittsburgh, PA 15260 -- U.S.A.}
\email{hales@pitt.edu}

% sectioning.
\newtheorem{example}[equation]{Example}
\newtheorem{definition}[equation]{Definition}
\newtheorem{theorem}[equation]{Theorem}
\newtheorem{lemma}[equation]{Lemma}
\newtheorem{corollary}[equation]{Corollary}


% Math notation.
\def\op#1{{\operatorname{#1}}}
\newcommand{\ring}[1]{\mathbb{#1}}
\newcommand{\NBC}{Ng\^o Bao Ch\^au}

\def\AD{\ring{A}}
%general
\def\b{\backslash }
%\def\mass{{\mathbf\mu}}
\def\card{\op{card}}
\def\a{{\scriptsize\text{ani}}}
\def\good{{\scriptsize\text{good}}}
\def\diamond{{\blacklozenge}}
\def\SO{{\mathbf {SO}}}
\def\OO{{\mathbf O}}


%frak
\def\so{\frak{so}}
\def\sp{\frak{sp}}
\def\gl{\frak{gl}}
\def\sl{\frak{sl}}
\def\g{\frak{g}}
\def\t{\frak{t}}
\def\cc{\frak{c}}
\def\DIV{{\frak{D}}}
\def\RDIV{{\frak{R}}}


%mathcal
\def\A{{\mathcal A}}
\def\C{{\mathcal C}}
\def\M{{\mathcal M}}
\def\P{{\mathcal P}}
\def\O{{\mathcal O}}
\def\tA{{\tilde{\mathcal A}}}
\def\tP{{\tilde{\mathcal P}}}
\def\tM{{\tilde{\mathcal M}}}
\def\tU{{\tilde{\mathcal U}}}






\begin{document}



\maketitle



{

\narrower{\it The study of orbital integrals on $p$-adic groups has turned
out to be singularly difficult. -- R. P. Langlands, 1992} % Real Igusa

}

\bigskip

%\section{introduction}
% 7 pages.


This report describe some remarkable identities of integrals that have
been established by \NBC.  In mathematics, we have an unlimited supply
of integrals and identities of integrals.  Part of my task 
will be to describe why these identities took nearly thirty years to
prove, and why they have particular importance for investigations in
the theory of automorphic representations.

\begin{abstract}
About thirty years ago, R. P. Langlands conjectured a collection of
identities to hold among integrals over conjugacy classes in
reductive groups.  Ng\^o Bao Ch\^au has proved these identities
(collectively called the fundamental lemma) by interpreting the
integrals in terms of the cohomology of the fibers of the Hitchin
fibration.  The fundamental lemma has profound consequences for the
theory of automorphic representations. Significant recent theorems in
number theory use the fundamental lemma as an ingredient in their
proofs.
\end{abstract}


\section{basic concepts}

\subsection{origins of the FL}


To orient ourselves, we give special examples of
behavior that the theory is designed to explain.


\begin{example}\label{ex:sl2}  We recall the definition of the holomorphic discrete series representations
of $SL_2(\ring{R})$.  For each natural number $n\ge 2$, let $V^+_n$ be the vector space
of all holomorphic functions $f$ on the upper half plane ${\frak h}$ such that
\[
\int_{\frak h} |f|^2 y^{n-2} dx\, dy < \infty.
\]
$SL_2(\ring{R})$ acts on $V^+_n$:
\[
\begin{pmatrix} a & b \\ c & d \end{pmatrix} \cdot f(z) = 
(-b z + d ) ^{-n} f (\frac{\phantom{-}a z - c}{-b z + d}).
\]
For each $n\ge 2$, $SL_2$ also acts on the vector space $V^-_n$ of
complex conjugates of functions in $V^+_n$.  These infinite
dimensional representations have characters that exist as integrable
functions $\chi_{n,\pm}$.  The characters are equal
$\chi_{n,+}(g)=\chi_{n,-}(g)$, except when $g$ is conjugate to a
rotation
\[
t = \begin{pmatrix} \phantom{-}\cos\theta & \sin\theta \\ -\sin\theta & \cos\theta\end{pmatrix}.
\] 
When $g$ is conjugate to $t$, a remarkable character identity holds:
\[
\chi_{n,-}(t) - \chi_{n,+}(t) = 
\frac{e^{i n \theta} + e^{- i n \theta}}{e^{i\theta}-e^{-i\theta}}.
\]
It is striking that numerator of the difference of two characters of
infinite dimension representations collapses to the character of a two
dimensional representation $t\mapsto t^n$ of the group $H$ of
rotations.  Shelstad gives general characters identities of this
sort~\cite{XX}. % Shelstad
\end{example}

%\begin{example}\label{ex:singer}
  We find another early glimpse of the theory in an exchange
  between Langlands and Singer.  Singer expressed interest to
  Langlands in a particular alternating sum of dimensions of spaces of
  cusp forms of $G=SL_2$ over a totally real extension of
  $\ring{Q}$.  In 1974, Langlands replied to Singer in a letter
  describing then unpublished joint work with
  Labesse~\cite{L:singer:1974},~\cite{LL:1979}.  Without going into details, we remark
 %The exact details do not concern
 % us here.  The important point for us is 
that in the formula for the
  alternating sum, there again is a collapse in complexity from the
  three dimensional group $SL_2$ to a sum indexed by one-dimensional
  groups $H$ (of norm $1$ elements of quadratic extensions of the totally real number field).
%\end{example}
%Letter to Singer 1974.
%Labess and Langlands

These two examples fit into a general framework that have now led to major
results in the theory of automorphic representations and number
theory, as described in Section~\ref{sec:uses}.  Langlands holds that methods
should be developed that are adequate for the theory of automorphic
representations in its full natural generality.  In ``full natural
generality'' means going from $SL_2$ (or even a torus) to all
reductive groups, from one local field to all local fields, from local
fields to global fields and back again, from the geometric side of the
trace formula to the spectral side and back again.  Moreover,
interconnections between different reductive groups and Galois groups
should be included, as predicted by his general principle of
functoriality.


Thus, from these early calculations of Labesse and Langlands, the
general idea developed that one should account for alternating sums
(or $\kappa$-sums as we shall call them because occasionally they
involve roots of unity other than $\pm1$) that appear in the harmonic
analysis on a reductive group $G$ in terms of the harmonic analysis on
groups $H$ of smaller dimension.  The FL is a concrete
expression of this idea.


\subsection{stable conjugacy}\label{sec:stable}

At the root of these $\kappa$-sum formulas is the distinction between
{\it ordinary conjugacy} and {\it stable conjugacy}.  
\begin{example}
A clockwise rotation and counterclockwise rotation
\[
\begin{pmatrix}\cos\theta &-\sin\theta\\\sin\theta &\phantom{-}\cos\theta\end{pmatrix},\quad
\begin{pmatrix}\phantom{-}\cos\theta &\sin\theta\\-\sin\theta &\cos\theta\end{pmatrix}
\]
in $SL_2(\ring{R})$ are conjugate by the complex matrix $\begin{pmatrix}i&0\\0&-i\end{pmatrix}$, 
but they are not conjugate
in the group $SL_2(\ring{R})$ when $\theta\not\in\ring{Z}\pi$.  Indeed,
% if conjugate in $SL_2(\ring{R})$ they are
%also conjugate in $GL_2(\ring{R})$, but 
a matrix calculation shows
that every element of $GL_2(\ring{R})$ carrying the rotation to
counter-rotation has odd determinant, thereby falling outside $SL_2(\ring{R})$.
\end{example}
%For example, the diagonal
%matrix with eigenvalues $1$ and $-1$ conjugates the clockwise rotation
%to a counterclockwise rotation.  
%The sign of the determinant gives an
%invariant in a group of order two that distinguishes conjugacy within
%stable conjugacy for $SL_2$.

\begin{definition} Let $G$ be a reductive group defined over a field
  $k$ with algebraic closure $\bar k$.  An element
  $\gamma'\in G(k)$ is said to be {\it stably conjugate} to a given regular
  semisimple element $\gamma\in G(k)$ if $\gamma'$ is conjugate to $\gamma$ in
  the group $G(\bar k)$.
\end{definition}

%Each stable conjugacy class is a union of ordinary conjugacy classes in $G(k)$.

There is a Galois cohomology group that can be used to study the conjugacy classes within
a given stable conjugacy class.
Write $\gamma'=g\gamma g^{-1}$, for $g\in G(\bar k)$.
For every element $\sigma$ of the Galois group $\op{Gal}(\bar k/k)$,  we have
 $\sigma(g)g^{-1}\in T(\bar k)$ and defines a cohomology class in
$H^1(\op{Gal}(\bar k/k),T(\bar k))$, where $T$ is the centralizer of
$\gamma$.  The cohomology class defined by the cocycle does not depend
on the choice of $g$.  It is the trivial class when
$\gamma'$ is conjugate to $\gamma$.

%When $k$ is a local field, the cohomology group $H^1$ is finite.  When $k=\ring{R}$,
%it is an elementary abelian $2$-group.
%The origin of the alternating signs in Examples 
%$\kappa:H^1(\bar k/k,T)\to\ring{C}$.  
%For fields $k\ne\ring{R}$, the cohomology
%group is needn't be a $2$-group, so we will henceforth refer to $\kappa$-sums rather
%than alternating sums. 

\begin{example} The centralizer $T$ of a noncentral rotation $\gamma$ is  the subgroup of all
rotations in  $SL_2(\ring{R})$.  The group $T(\ring{C})$ is isomorphic
  to $\ring{C}^\times$.   Each cocycle is identified with the value $r\in T(\ring{C})=\ring{C}^\times$ 
   of the cocycle on generator of  $\op{Gal}(\ring{C}/\ring{R})$.  It is a cocycle when $r\in \ring{R}^\times$
  and is the trivial class in cohomology when $r$ is positive.  This identifies the cohomology group:
\[
H^1(\op{Gal}(\ring{C}/\ring{R}),T(\ring{C})) = \ring{R}^\times/\ring{R}^\times_+ = \ring{Z}/2\ring{Z}.
\]
 This cyclic group of order two classifies the two conjugacy classes within the stable conjugacy class
 of a rotation.
\end{example}

When $k$ is a local field, $A=H^1(\op{Gal}(\bar k/k),T(\bar k))$ is a
finite abelian group.  Every function on this abelian group has an
Fourier expansion as a linear combination of characters $\kappa$ on
$A$.  
%Hence by understanding individual characters $\kappa$, we can
%reconstruct arbitrary functions on $A$.  
The {\it theory of endoscopy}
is the subject that studies stable conjugacy through the separate
characters $\kappa$ on $A$.  Allowing ourselves to be deliberately
vague for a moment, the idea of endoscopy is that the Fourier mode of
$\kappa$ (for given $T$ and $G$) produces oscillations that cause some
of the roots of $G$ to cancel away.  The remaining roots are
reinforced by the oscillations and become more pronounced.  The root
system consisting of the pronounced roots defines a group $H$ of smaller
dimension than $G$.  With respect to the harmonic analysis on the two
groups, the mode of $\kappa$ on the group $G$ should
be related to the dominant mode on $H$.  The FL is a precise
conjecture for nonarchimedean local fields that stems from this line of thought.





\subsection{endoscopic groups}\label{sec:endoscopy}

The smaller group $H$, formed from a ``pronounced'' subset of the roots of $G$, are called an endoscopic group.
Hints about how to define $H$ come from various sources.
\begin{itemize}
\item It should be constructed from the data $(G,T,\kappa)$.
\item Its roots should be a subset of the roots of $G$ (although $H$ need not
be a subgroup of $G$).
%\item Over $\ring{R}$, the presence or absence of a purely imaginary root in $H$ is determined by whether the normalized $\kappa$-orbital integral jumps across the hyperplane $\alpha=0$.
\item $H$ should have a Cartan subgroup $T_H\subset H$ isomorphic over $F$ to the Cartan subgroup
$T$ of $G$, compatible with the Weyl groups of the two groups $H$ and $G$.
\item Over a nonarchimedean local field, the spherical Hecke algebra on $G$ should be related to the
spherical algebra on $H$.
\item It should generalize the example of Labesse and Langlands.
\end{itemize}

In formulating the general conjecture of functoriality for automorphic
representations, Langlands defined a group $\hat G$ that is dual to
$G$.  The character group of a Cartan subgroup in the dual group is
the character group of a Cartan subgroup in $G$.  The roots of the
dual are the coroots of $G$.  The dual of a semisimple simply
connected semisimple group is an adjoint group, and vice versa.  For
example, we have dualities $\hat GL(n) = PGL(n)$ and
$\hat Sp(2n)=SO(2n+1)$.  Note that there is a bijection between the
root systems of $Sp(2n)$ and $SO(2n+1)$ that interchanges short and
long roots.  They have isomorphic Weyl groups.  In fact, there is a somewhat larger
dual group ${}^LG$ that is defined as the semidirect product of $\hat G$ with the Galois group of the splitting field of $G$.

There are indications that the groups $H$ should be defined through  
the Langlands dual $\hat G$ (or more precisely, ${}^LG$) of $G$.
\begin{itemize}
\item Langlands's principle of functoriality is a collection of conjectures, asserting that the
  representation theory of groups should be related when their dual
  groups are related.  
  Since the example in $SL_2$ are representation theoretic, we should look to the dual.
\item The Satake transform identifies the spherical Hecke algebra with
  ring of regular functions on the set of $\sigma$-conjugacy classes of the dual torus $\hat T$.
\item The Kottwitz-Tate-Nakayama isomorphism identifies the group of characters on $H^1(F,T)$
with a subquotient $\pi_0(\hat T^\Gamma)$ of the dual torus $\hat T$.
\end{itemize}

\begin{definition}[endoscopic group]
  The endoscopic group $H$ associated with $(G,T,\kappa)$ is defined
  as follows.  By the Kottwitz-Tate-Nakayama isomorphism just
  mentioned, $\kappa$ is represented by an element of the dual torus,
  $\hat T$.  By an abuse of notation, we will also write $\kappa\in
  \hat T$ for this element.  The connected component of the
  centralizer of $\kappa$ is the dual $\bar H$ of a quasi-split
  reductive group $H$ over $F$.  The choice of a particular quasi-split form $H$
  among its outer forms is determined by the condition that there should
  be an isomorphism over $F$ of a Cartan subgroup $T_H$ of $H$ with
  $T$ in $G$, compatible with their respective Weyl group actions.
\end{definition}

We write $\rho$ for the choice of quasi-split form $H$ among its outer forms and refer to
the pair $(\kappa,\rho)$ as endoscopic data for $H$.

One of the challenging aspects of the FL is that it is an assertion of
direct relation between groups that are defined by an indirect
relation, through dual groups.  Very limited information (such as
Cartan subgroups, root systems, and Weyl groups) can be transmitted
from the endoscopic group $H$ to $G$ through the dual group.

The FL is a statement about reductive groups over nonarchimedean local
fields.  The corresponding identities (transfer of Schwartz functions)
over the field of real numbers were taken up and solved by
Shelstad~\cite{XX}.  Her work gives a precise form to the vague idea
mentioned above that the oscillations of a character $\kappa$ cause
certain roots to cancel away and others to become more pronounced.  A
root of a real reductive group is said to be pure imaginary if [XX FINISH PARAGRAPH].   
Her
methods are based on Harish-Chandra's invariant differential
operators.  In this sense, NBC's use of perverse sheaves can be viewed
as $p$-adic substitute for these differential operators.



\section{a bit of Lie theory}



\subsection{characteristic polynomials}\label{sec:chevalley}



Let $G$ be a split reductive group over a finite field $k$ and let
$\g$ be its lie algebra, with split Cartan subalgebra $\t$ and Weyl
group $W$.  We assume that the characteristic of $k$ is sufficiently
large (more than twice the Coxeter number, to be precise).  The group
$G$ acts on $\g$ by the adjoint action.  By Chevalley,  the restriction  of regular
functions from $\g$ to $\t$ induces an isomorphism
\[
k[\g]^G = k[\t]^W.
\]
We let $\cc =  \op{Spec}\,(k[\t]^W)$, and let $\g\to\cc$ be the morphism deduced from
Chevalley's isomorphism.  The following example shows that $\g\to\cc$ is a generalization
of the characteristic polynomial of a matrix.

\begin{example}
If $G=GL(n)$, the ring $k[\g]^G$ is generated by the coefficients $c_i$ of the characteristic
polynomial 
\begin{equation}
p(t)=t^n + c_{n_1} t^{n-1} +\cdots c_0
\end{equation}
of a matrix $a\in \g$.  The morphism $\g\to\cc$ can be identified with
the ``characteristic map'' that sends $a$ to $(c_0,\ldots,c_{n-1})$.
\end{example}

The multiplicative group $\ring{G}_m$ on $\g$ by scalar multiplication.  This extends to an action
of $\ring{G}_m$ on $\cc$ for which $\g\to\cc$ is equivariant.  For example, in
the case of $GL(n)$, the coefficients $c_i$ of the characteristic
polynomial are homogeneous polynomials on $\g$ and $\ring{G}_m$ acts
according to the degrees of the homogeneous polynomial.

\subsection{Kostant section}

Kostant constructs a section $\epsilon:\cc\to\g$ of $\g\to\cc$ whose image lies in the set $\g^{reg}$ 
of regular elements
of $\g$.  In simplified terms, this can be viewed as a chosen matrix with a given characteristic polynomial.


\begin{example}
If $\g=\sl(2)$, the characteristic polynomial of $a\in\g$ has the form $t^2  +c_0$.
The Kostant section is the matrix
\[
\begin{pmatrix} 0 & -c_0\\ 1 & 0\end{pmatrix}
\]
\end{example}


\begin{example}
  If $\g=\gl(n)$, we can construct the companion matrix of a given
  characteristic polynomial $p$, by taking the endomorphism $t$ of
  $R=k[t]/(p(t))$, expressed as a matrix with respect to the standard
  basis $1,t,t^2,\ldots,t^{n-1}$ of $R$.  The companion matrix is a
  section $\cc\to\g$ that is somewhat different from the Kostant
  section.  Nevertheless, the Kostant section can be viewed as a
  generalization of this that works uniformly for all Lie algebras
  $\g$.
\end{example}


\subsection{centralizers}

Each element $\gamma\in\g^{reg}$ has a centralizer $I_\gamma$ in $G$.  If two
elements of $\g^{reg}$ have the same image $a$ in $\cc$, then their
centralizers are canonically isomorphic.  By descent, there is a
regular centralizer $J_a$ on $\cc$ that pulls back to $I_\gamma$ under
$\gamma\mapsto a$.



\begin{example}  Suppose $G=SL(2)$.  By identifying $J_a$ with the centralizer of
  $\epsilon(a)$, the centralizer $J_a$ consists of
  matrices of the form
\[
\begin{pmatrix} x & y c_0\\ y & x
\end{pmatrix}
\quad x^2 - c_0 y^2 = 1.
\]
\end{example}

\begin{example}
  If $\g=\gl(n)$, then the centralizer of the companion matrix with
  characteristic polynomial $p$ can be identified with the centralizer
  of $t$ in $GL(R)$, where $R=k[t]/(p(t))$.  An
  element of $\gl(R)$ centralizes the regular element $t$ if and only if it is a polynomial
  in $t$.  Thus, the centralizer in $\gl(R)$ is $R$ and the centralizer in $GL(R)$
 is $J_a = R^\times$.
\end{example}


\subsection{discriminant and resultant}

Let $\Phi$ be the root system of a split group $G$.  The differentials $d\alpha$ of each
root defines a  polynomial called the discriminant:
\begin{equation}\label{eqn:disc}
\prod_{\alpha\in\Phi} d\alpha
\end{equation}
on $\t$.  The polynomial is invariant under the action of the Weyl group 
$W$ and descends to a function on $\cc$.
The divisor $\DIV_G$ of this polynomial $\cc$ is called the discriminant divisor.

\begin{example} Let $G=GL(n)$.  The Lie algebra of $T$ can be identified with the diagonal matrix
algebra with coordinates $t_1,\ldots,t_n$ along the diagonal.  The discriminant the form
\[\prod_{i\ne j} (t_i - t_j).\]
This  is invariant under the action of the symmetric group on $n$ letters 
and can be expressed as a polynomial in the coefficients $c_i$ of the characteristic polynomial.
In particular, when $n=2$, the discriminant of the polynomial $x^2 + b x + c$ is $b^2 - 4 c$.
\end{example}

If $H$ is an endoscopic group of $G$, there is a morphism $\nu:\cc_H\to \cc$ (see Section~\ref{XX}).  There exists
a resultant divisor $\RDIV$ such that
\[
\nu^*\DIV_G = \DIV_H + 2\,\, \RDIV.
\]

\begin{example} Let $H=GL(2)\times GL(2)$, embedded as a block
  diagonal subgroup of $GL(4)$.  The morphism $\nu:\cc_H\to \cc$,
  viewed in terms of characteristic polynomials, maps the pair
  $(p_1,p_2)$ of quadratic polynomials to the quartic $p_1p_2$.  Let
  $t^1_i,t^2_i$ be the roots of $p_i$, for $i=1,2$.  The resultant is
\[
\prod_{j\ne k} (t^j_1 - t^k_2).
\]
  The
resultant is symmetric in the roots of $p_1$ and in the roots of $p_2$
and can be expressed as a polynomial $R(p_1,p_2)$ in the coefficients
of $p_1$ and $p_2$.  It vanishes exactly when $p_1$ and $p_2$ have a common root.
\end{example}



\section{the statement of the FL}\label{sec:statement}

Let $G$ be a reductive group scheme over the ring of integers $\O$ of
a nonarchimedean local field $F$ in positive characteristic.  Two
regular semisimple elements in $\g(F)$ are stably conjugate exactly
when they have the same image in $\cc(F)$.  (That is, the
characteristic polynomial of an element in $\g$ determines its
conjugacy class over $\bar F$.)  Using the Kostant section
$\epsilon:\cc\to\g$, and then applying the results of
Section~\ref{sec:stable}, each element $\gamma$ stably conjugate to
$\epsilon(a)$ carries a cohomological invariant in $H^1(\op{Gal}(\bar
F/F),J_a(\bar k))$.

For each regular semisimple element $a\in \cc(F)$ and character 
\[
\kappa:H^1(\op{Gal}(\bar F/F),J_a(\bar k))\to\ring{C}^\times,
\]
we define the pairing
 $\langle\kappa,\gamma\rangle$
of $\kappa$ with the cohomological invariant of $\gamma$.
A 
$\kappa$-orbital integral is defined to be
\[
\OO_\kappa(a) = \sum_ {\gamma\mapsto a} 
\int_{G/I_\gamma} \langle\kappa,\gamma\rangle 1_{\g(\O)} (\op{Ad}\, g(\gamma)) dg,
\]
where $I_\gamma$ centralizes $\gamma$, the sum runs over the finite
number of preimages of $a$ in $\g(F)$.  Here $1_{\g(\O)}$ is the
characteristic function of $\g(\O)$.  When $\kappa$ is trivial, we
write $\SO$ for $\OO_\kappa$.

The character $\kappa$ determines a reductive group scheme $H$ over $\O$,
according to the construction of Section~\ref{sec:endoscopy}.
  Let $\cc_H$ be the
Chevalley quotient of the Lie algebra of $H$.  It comes equipped with a morphism $\nu:\cc_H\to\cc$.
We also add a subscript $H$ to indicate
quantities coming from the group $H$.



\begin{theorem}[NBC's fundamental lemma]
Assume that the characteristic of $F$ is greater than twice the Coxeter number of $G$.
For all regular semisimple elements $a\in \cc_H(F)$ whose image
$\nu(a)$ in $\cc$ is also regular semisimple, the $\kappa$ orbital integral of $\nu(a)$ in $G$ is
equal to the stable orbital integral of $a$ in $H$, up to a power of $q$:
\[
\OO_\kappa(\nu(a)) = q^{r_v(a)}\SO_H(a).
\]
\end{theorem}

The  exponent $r_v(a)$ is defined as the valuation of the resultant of the characteristic polynomial:
\[
r_v(a) = \deg_v(a^*\RDIV).
\]
A sketch of NBC's proof of the fundamental
lemma appears in Section~\ref{sec:lmf}. 


Over the years from the time that Langlands first conjectured the FL
until the time that NBC gave its proof, the FL has been transformed
into simpler form.  The statement of the FL appears here in its
simplified form.  Section~\ref{sec:reduce} makes a series of comments
about Langlands's original form of fundamental lemma and its reduction
to this simple form.  Except for that section, our discussion is based
on this simplified form of the FL.  In particular, we assume that the
field $F$ has positive characteristic and that the stable conjugacy
classes live in the Lie algebra rather than the group.


\section{affine Springer fibers}

\subsection{geometry}

Calculations in special cases show why the FL is
essentially geometric in nature, rather purely analytic or
combinatorial. We recall a favorite old calculation of mine of the
orbital integrals for $\so(5)$ and $\sp(4)$, the rank two odd
orthogonal and symplectic Lie algebras~\cite{hyperelliptic-curves}.  
Let $F$ be a nonarchimedean
local field of residual characteristic greater than $2$.  Let $k$ be
the residue field with $q$ elements.  Let $a$ be an element of
$\so(5)\subset \gl(5)$ with eigenvalues $0,\pm t_1,\pm t_2$.  Assume
that there is an odd natural number $r$ such
\[
|\alpha(a)| = q^{-r/2},
\]
for every root $\alpha$ of $\so(5)$. 
There is an elliptic curve $E_a$ over the finite field $k$ given by
$y^2 = (1-x^2\tau_1)(1-x^2\tau_2)$, where $\tau_i$ is the image
of $t_i^2/\varpi^r$ in the residue field.  There are test functions $f$ 
such that the integral of $f$ over the stable orbit
of $a$ has the form
\begin{equation}\label{eqn:elliptic}
A(q) + B(q) | E_a(k)|,
\end{equation}
for some rational functions $A,B$, depending on $f$ and the normalization of measures.

Similarly, in the group $\sp(4)$, there is an element ${a_H}$ with
related eigenvalues $\pm t_1,\pm t_2$.  According to the general
framework of (twisted) endoscopy, there should be a corresponding
function $f'$ on $\sp(4)$ such that the integral over the stable orbit
of ${a_H}$ in $\sp(4)$ is equal to (\ref{eqn:elliptic}).  A
calculation of the orbital integral of $f'$ gives a similar formula,
with a different elliptic curve $E'_{a_H}$, but otherwise identical to
(\ref{eqn:elliptic}).  The elliptic curves $E_a$ and $E'_{a_H}$ have
different $j$-invariants (which vary with $a$ and ${a_H}$).  The proof
of the desired identities of orbital integrals in this case is
obtained by producing an isogeny between $E_a$ and $E'_{a_H}$.  (The
identities of orbital integrals are quite nontrivial, even though the
lie algebras $\so(5)$ and $\sp(4)$ are abstractly isomorphic.)

In a similar way in higher rank,  the spectral curves
\[
y^2 = (1-x^2 \tau_1)(1-x^2 \tau_2)\cdots (1-x^2 \tau_n).
\]
appear, for example, in calculations of orbital integrals for
$\so(2n+1)$.  When orbital integrals are computed by brute force,
these curves appear as freaks of nature.  As it turns
out, they aren't freaks at all, merely perverse.  One of the major
challenges of the proof of the FL and one of the major
triumphs of NBC has been to find the natural geometrical setting that combines
orbital integrals and spectral curves.


\subsection{cosets}\label{sec:coset}

An orbital integral can be computed by solving a coset counting
problem.  The value of the integrand is unchanged if $g$ is replaced
with $k g$, for $k\in G(\O_v)$.  The integral thus breaks into a
discrete sum over cosets of $G(\O_v)$ in $G$ modulo the group action
by $I_\gamma$.  With suitable normalization of measures, each coset
$G(\O_v)g$ contributes $0$ or a root of unity $\langle\kappa,\gamma\rangle$
to the value of the integral depending on whether $\op{Ad}g\,(a) \in
\g(\O_v)$ (again modulo symmetries $I_\gamma$).  This interpretation as a
coset counting problem makes the FL appear to be combinatorial.
However, purely combinatorial attempts to prove the FL have failed.

Let $\M_v(a,k)$ be the set of cosets that meet the support and give nonzero contribution:
\[
\M_v(a,k) = \{g\in G(F_v)/G(\O_v) \mid \op{ad}\,g^{-1} \gamma_0 \in \g(\O_v)\}.
\]

Kazhan and Lusztig showed that the coset space $G(F_v)/G(\O_v)$ is the
set of $k$-points of an inductive limit of schemes called the affine
Grassmanian.  Moreover, $\M_v(a,\bar k)$ is the set of points of a
locally finite union $\M_v(a)$ of projective varieties~\cite{KL:1988}.
This variety $\M_v(a)$ is called the affine Springer fiber.

Each irreducible component of $\M_v(a)$ has the same dimension.  This
dimension, $\delta_v(a)$, is given by a formula of
Berukavnikov~\cite{Bezrukavnikov}.  From that formula, it follows that
the dimension of the affine Springer fiber of $\nu(a)$ in $G$ exceeds
the dimension of the affine Springer fiber of $a$ in $H$ by precisely
$r_v(a)$.  The factor $q^{r_v(a)}$ that appears in the fundamental
lemma is forced to be what it is because of this simple dimensional
analysis.

Goresky, Kottwitz, and MacPherson made an extensive investigation of
the affine Springer fibers and conjectured that their equivariant
cohomology groups are pure.  Assuming this conjecture, they prove the
FL for elements whose centralizer is an unramified
Cartan subgroup~\cite{GKM:2004}.  They prove the purity result in particular cases by
constructing affine pavings of the Springer fibers~\cite{GKM:2006}.

Laumon has made a systematic investigation of
the affine Springer fibers for unitary groups.  NBC joined the effort, and together they
succeeded in giving a complete proof of the FL for unitary groups~\cite{LN:08}.

NBC encounterd two major obstacles in trying to generalize the work of
Goresky, Kottwitz, and MacPherson to an arbitrary reductive group.
Their approach computes computes the cohomology by computing the fixed
point set in $\M_v(a)$ under the action of a torus on $\M_v(a)$.  Laumon and
Ng\^o's proof of the FL for unitary groups also relies
the action of a torus on the affine Springer fibers.  (In the case of
unitary groups, over a quadratic extension, each
endoscopic group becomes isomorphic to a Levi subgroup of $GL(n)$.
The torus action comes from the center of this Levi.)  However, in
general, a nontrivial torus action on the affine Springer fiber simply
doesn't exist.

The second serious obstacle comes from the purity conjecture itself.
In accordance with Deligne's work, NBC believed that the task of
proving purity results should become much easier when the affine
Springer fibers are combined into families rather than treated in
isolation.  With this in mind, he started to investigate families of
affine Springer fibers, varying over a base curve $X$.  This brings us
back from local geometry over the field $F$ to the global geometry of
$X$.  What he found was that a family of affine Springer fibers can be
interpreted as a fiber in the Hitchin fibration.  He was able to use
Deligne's purity theorem in this setting~\cite{Deligne:Weil2}.  The
Hitchin fibers will be described in the next section.  But before
moving on, we need a few more facts about affine Springer fibers.




\section{Hitchin fibration}

The Hitchin fibration was introduced in 1987 in the context of
completely integrable systems~\cite{Hitchin:87}.  Roughly, the Hitchin
fibration is the stack obtained when we allow the characteristic map
$\g\to\cc$ to vary over a curve $X$.

Fix a smooth projective curve $X$ of genus $g$ over $k$.  We now shift
perspective and notation, allowing the constructions in Lie theory
from the previous section to vary over the base curve $X$.  In
particular, we now let $G$ be a quasi-split reductive over $X$ that is
locally trivial in the etale topology on $X$.  Let $\g$ be its Lie
algebra $G$ and $\cc$ the characteristic, both now schemes over $X$.

Let $D$ be a line bundle on $X$. For technical reasons (stemming from
the constant $2$ that appears in the structure constants for the Lie
algebra $\sl_2$) we assume that $D$ is the square of another line
bundle.  In Ng\^o's proof of the FL, it is important in one step of
the proof Section~\ref{sec:lmf} to allow the degree of $D$ to become
arbitrarily large.  We place a subscript $D$ to indicate the tensor
product with $D$: $\g_D = \g\otimes_{\O_X} D$, etc.

We let $\A$ be the space of global sections on $X$ with values in
$\cc_D=\cc\otimes_{\O_X} D$.  The group $G$ acts on $\g$ by the
adjoint action.  We can twist $\g$ by any $G$-torsor $E$ to get a
vector bundle $\op{ad}(E)$ over $X$.

\begin{definition}
The Hitchin fibration $\M$ is the stack
such that for any $k$-scheme $S$, the groupoid $\M(S)$ of points consists
of pairs $(E,\phi)$, where $E$ is $G$-torsor over $X\times S$ and $\phi$ is a section of
$\op{ad}(E)_D$.
\end{definition}

The characteristic polynomial of a section $\phi$ is a section of
$X\times S$ with values in $\cc_D$; that is, a section of $\A(S)$.
The characteristic map $\g\to\cc$ thus gives a morphism $f:\M\to\A$.
There is a section $\A\to\M$ obtained from the Kostant section.  In
particular, the fibers of the Hitchin fibration are nonempty.

The centralizers $J_a$ as we vary $a\in \cc$ define a smooth group
scheme $J$ over $\cc$.  Take $a:S\to\A$ now to be more generally any
$S$-point of $\A$.  There is a groupoid $\P_a(S)$ whose objects are
$J_a$-torsors on $X\times S$.  Moreover, $\P_a(S)$ acts on $M(S)$ by
twisting a pair $(E,\phi)$ by a $J_a$-torsor.  As $a$ varies, we
obtain a Picard stack $\P$ acting fiberwise on the Hitchin fibration
$\M$.

At this point in the development, it would be most appropriate to
insert a book-length discussion of the geometry of the Hitchin
fibration, with a full development and many examples.  As Langlands
speculates in his review of Ng\^o's poof, ``an exposition genuinely
accessible not alone to someone of my generation, but to
mathematicians of all ages eager to contribute to the arithmetic
theory of automorphic representations, would be, perhaps, \ldots close
to 700 pages'' \cite{L:Ngo}.

To cut  $700$-pages short, what are the absolutely key strategic ideas?  

First, as mentioned above, the Hitchin fibration is the correct global
analogue of the (local) affine Springer fiber.  The relationship
between the $k$-points of the Hitchin fiber $\M(a,k)$ and $k$-points
of the affine Springer category $\M_v(a,k)$ can be expressed in
precise terms as a factorization of categories, $\M(a,k)$ modulo
symmetries as a product of $\M_v(a,k)$ modulo their symmetries as $v$
runs over closed points of $X$ (Theorem~\ref{lemma:product}).  Through
the affine Springer fibers, the Hitchin fibration can be used to study
orbital integrals and the FL.

Second, the Hitchin fibration should be understood insofar as possible
through its symmetries $\P$.  The obvious reason for this is that it
is generally a good idea to study symmetry groups.  The deeper reason
for this has to do with endoscopy.  The Picard stack is defined via
$J_a$-torsors.  Although the, relation between $G$ and $H$ is mediated
throught dual groups, the relationship between centralizers is direct:
over $\cc_H$, there is a canonical homormophism from the regular
centralizer $J$ of $G$ to the regular centralizer $J_H$ of $H$:
\[
\nu^*J\to J_H.
\]
Thus, the Picard stacks $\P$ and $\P_H$ for $G$ and $H$ are also
directly related and information passes fluently between them.  The FL
should be proved largely at the level of Picard stacks.

Third, by working directly with the Hitchin fibration, the difficult
purity conjecture of Kottwitz, Goresky, and MacPherson can be
bypassed.

The dimension of the affine Springer fiber $\M_v(a)$ grows linearly
with $\deg_v(a^* \DIV)$.  As the dimension increases, so does the
complexity of the geometry and attendant computations of orbital
integrals.  So locally we have unbounded unpleasantness.  Globally, we
can hope to use continuity arguments: a global section $a\in\A$ of
$\cc_D$ over $X$ that has a large intersection multiplicity
$\deg_v(a^* \DIV)$ with the divisor $\DIV$ at some place $v$ can lie
in the closure of a set on which the intersection multiplicities are
all $0$ or $1$.  In the case of multiplicity $0$ or $1$, the affine
Springer fibers are directly computable.  The BBDG decomposition
theorem provides the infrastructure for the continuity arguments.


\subsection{perverse cohomology sheaves}

We give a brief summary without proofs of some of the main results
proved by NBC about the perverse cohomology sheaves of the Hitchin
fibration.

There is an etale open subset $\tA$ of $\A\otimes_k\bar k$ that has
the technical advantage of killing unwanted monodromy.  The tilde will
be used consistently to mark quantities over $\tA$.  For example, if
we write $f^\a:\M^\a\to\A^\a$ for the Hitchin fibration, restricted to
the open set of anisotropic elements of $\A$, then $\tilde
f^\a:\tM^\a\to\tA^\a$ is the corresponding Hitchin fibration over the
anisotropic part of $\tA$.

The anisotropic locus $\tA^\a$ admits a stratification by a numerical
invariant $\delta:\tA\to\ring{N}$:
\[
\tA^\a = \coprod_{\delta\in\ring{N}}\tA^\a_\delta.
\]
There is an open set $\tA^\good$ of $\tA^\a$, given as a union of some
strata $\tA^\a_\delta$ that satisfy:
\begin{equation}\label{eqn:codim}
\op{codim}(\tA_\delta^\a) \ge \delta.
\end{equation}

%Let $g^\a:\P^\a\to\A^\a$ be the Picard stack over the anisotropic locus.  
%Using theorems of Faltings, NBC proves that $\P^\a$  is a smooth separated Deligne-Mumford
%stack of finite type over $\A^\a$ and that $\M^\a$ is a smooth separated Deligne-Mumford stack
%over $k$.  

NBC shows that the conditions of Deligne's purity theorem are satisfied, so that $\tilde f_*^\a 
\bar{\ring{Q}}_\ell$ is isomorphic to a direct sum of perverse cohomology sheaves:
\[
{}^p\!H^n(\tilde f^\a_* \bar{\ring{Q}}_\ell)[-n].
\]

The action of $\tilde\P^\a$ on $\tilde\M^\a$ gives an action on the
the perverse cohomology sheaves, which factors
through the sheaf of components $\pi_0=\pi_0(\tilde\P^\a)$. 
% NBC gives an
%explicit description of the sheaf $\pi_0(\tilde\P^\a)$ in terms of
%data in the dual group.  
He shows that $\pi_0$ is an explicit quotient of the constant sheaf $X_*$ of
cocharacters, and hence $X_*$ acts on the perverse cohomology sheaves
through $\pi_0$.  This allows him to decompose the perverse cohomology
sheaves into a direct sum of $\kappa$-isotypic pieces
\[
L_\kappa = {}^p\!H^n(\tilde f^\a_* \bar{\ring{Q}}_\ell)_\kappa,
\]
as $\kappa$ runs
over elements in the dual torus $\hat T$.  (The character group of $\hat T$ is $X_*$, which
gives the pairing between $\hat T$ and $X_*$.)

For each $\kappa$, there is a closed subspace of $\tA_\kappa$ of $\tA$
consisting of elements $a$ whose ``geometric monodromy'' lies in the centralizer
of $\kappa$ in the dual group ${}^LG$.  Each subspace $\tA_\kappa$ is in fact
the disjoint union of the images of closed immersions $\nu:\tA_H\to\tA$ 
coming from endoscopic groups $H$
with compatible endoscopic data $(\kappa,\rho)$.  (We use the same symbol
$\nu$ for several slightly different but closely related morphisms,
such as $\nu:\A_H\to\A$ and $\nu:\cc_H\to\cc$.) 
The support $L_\kappa$  lies in $\tA^\a_\kappa$.

\subsection{support theorem}

The proof of the following  theorem about the support of the
perverse cohomology sheaves of the Hitchin fibration constitutes the
deepest part of the proof of the FL.  



\begin{theorem}[support theorem]
Let $Z$ be the support of a geometrically irreducible factor of $L_\kappa$.  
If $Z$ meets $\nu(\tA_H^\good)$ for some endoscopic group $H$  with data $(\kappa,\rho)$, 
then $Z=\nu(\tA_H^\a)$.  In fact, there is a unique such $H$.
\end{theorem}

An entire chapter of the book-length proof of the FL is devoted to the
proof of the support theorem.  The strategy of the proof is to show
that every support $Z$ also appears as the support of some factor in
the ordinary cohomology of highest degree of the Hitchin fibration.
To move cohomology classes from one degree to another, he uses
Poincar\'e duality and Pontryagin product operations on cohomology
coming from the action of the connected component of the identity
$\tP^{0,\a}$ on $\tM^\a$.  This action factors through the action of
an abelian variety, a quotient of the Picard stack $\tP^{0,\a}$.  To
show that the support $Z$ can be pushed all the way to the top degree
cohomology, NBC proves that the dimension of this abelian variety is
sufficiently large and that the cohomology of the abelian variety acts
freely on the cohomology of the Hitchin fiber.  The required estimate on the
dimension of the abelian variety comes from the inequality (\ref{eqn:codim}):
on the stratum
$\tA_\delta$, the abelian variety has dimension $d-\delta$, where $d$
is the dimension of the Hitchin fiber.
Freeness relies on a polarization of the abelian variety.

Once the support $Z$ is known to appear as a support in the top
degree, he shows that the action of $\tP^\a$ on the Hitchin fibration
leads to an explicit description of the top degree ordinary cohomology
as the sheaf associated with the presheaf
\[
 U\mapsto \bar{\ring{Q}}_\ell^{\pi_0(\tP^\a)(U)}.
\]
The supports of $\pi_0$ can be described explicitly in terms
of data in the dual group, in the style of duality theorems of
Kottwitz, Tate, and Nakayama.  By checking that the conclusion of the
support theorem holds for the particular sheaf $\pi_0$, the general support
theorem follows.
% Prop 6.5.1.


%In general, $\nu^*(\tA_H)$ can
%have high codimension in $\tA$, and precise information about supports
%is hard to obtain.


When $\kappa$ is taken to be the trivial character, the only
endoscopic group $H$ with stable data is $H=G$ itself.  In that case,
$\tA_H=\tA$ and $\nu$ is the identity.  The support theorem takes the
following form in this case.

\begin{corollary}
  Suppose that $\kappa=1$.  Let $Z$ be the support of a geometrically
  irreducible factor of $L_\kappa$.  If $Z$ meets $\tA^\good$, then
  $Z=\tA^\a$.
\end{corollary}





\section{mass formulas}

\subsection{groupoid cardinality (or mass)}

Let ${\C}$ be a groupoid (that is, a category in which every arrow is
invertible) which has only finitely many objects up to isomorphism and
in which every object has a finite automorphism group.  Define the
mass (or groupoid cardinality) of $\C$  to be the rational
number
\[
\mu(\C)= \sum_{x\in \op{obj}(\C)/\sim} \frac{1}{\op{card}(\op{Aut}(x))}.
\]

\begin{example}
  For example, if ${\C}$ is the category whose objects are the
  elements of a finite group $G$ and arrows are given by $x \mapsto g
  x g^{-1}$, for $g\in G$.  Then the set of objects up to isomorphism
  is in bijection with the set of conjugacy classes, the automorphism
  group of $x$ is the centralizer of $x$, and the mass is
\[
\mu(\C) = \sum_{x/\sim} \frac{1}{\op{card}(\op{Aut}(x))} = 
\sum_{x/\sim} \frac{\op{card}(O_x)}{\card{\,G}} = 1.
\]
\end{example}

\begin{example}\label{ex:groupoid}
  The following less trivial example appears in NBC.  Let $P$ be the
  group $\ring{G}_m\times \ring{Z}$ defined over a finite field $k$ of
  cardinality $q$.  Let $M = (\ring{P}^1\times\ring{Z})/\sim$, where
  the equivalence relation identifies the point $(\infty,j)$ with
  $(0,j+1)$ for all $j$.  Thus, $M$ is an infinite string of
  projective lines, with the point at infinity of each line joined to
  the zero point of the next line.  The group $P$ acts on $M$ by
  $(p_0,i)\cdot (m_0,j) = (p_0 m_0,i+j)$, where $p_0m_0$ is given by
  the standard action of $\ring{G}_m$ on $\ring{P}^1$ fixing $0$ and
  $\infty$.  Let $\sigma$ be the Frobenius automorphism of $\bar k$,
  and define an twisted automorphism of $P(\bar k)$ and $M(\bar k)$ by
  $\sigma(x_0,i) = (\sigma x_0^{-1},-i)$.  Define a category $\mathcal
  C$ with objects given by pairs
\begin{equation}\label{eqn:objects}
(m,p)\in M(\bar k)\times P(\bar k) \text{ such that } \sigma(m) = p
m.
\end{equation}
Define  arrows $h\in P(\bar k)$ 
\begin{equation}\label{eqn:arrows}
h(m,p) = (m',p'),    \text{ provided } hm = m' \text{ and } h p = p'\sigma(h).
\end{equation}  
Then it can be checked by a simple
calculation that there are two isomorphism classes of objects in this
category, representated by the objects
\[
((0,0),(1,1))\text{ and }  ((1,0),(1,0))\in M(\bar k)\times P(\bar k) = 
(\ring{P}^1\times\ring{Z}) \times (\ring{G}_m\times\ring{Z}).
\]
The group $P(\bar k)^\sigma$ of order $q+1$ acts as automorphisms of the first object,
and the group of automorphisms of the second object is trivial.  The mass of the category
is therefore
\[
\mu(\C) = \frac{1}{q+1} + 1.
\]
\end{example}

More generally, suppose there exists a function $\op{Obj}(\C)\to  A$ from the objects
of a groupoid into in a finite abelian group $A$ that
depends only on the isomorphism class of an object $x$.  Then for every
character $\kappa$ of $A$, we can define a $\kappa$-mass:
\[
\mu_\kappa(\C)= \sum_{x\in \op{obj}(\C)/\sim} \frac{\langle\kappa,x\rangle}{\op{card}(\op{Aut}(x))}.
\]

\begin{example}
  For example, in the previous example, if $(m,p)$ is an object and
  $p=(p_0,j)\in \ring{G}_m\times\ring{Z}$, then the image of $j$ in
  $A=\ring{Z}/2\ring{Z}$ depends only on the isomorphism class of the
  object $(m,p)$.  If $\kappa$ is the nontrivial character of $A$,
  then the $\kappa$-mass of this groupoid is
\[
\mu_\kappa(\C) = -\frac{1}{q+1} + 1.
\]
\end{example}

%\begin{example}  Let $G$ be a reductive group over $F$ and $T$ a Cartan subgroup of $G$.
%Let the set of objects be the set of cosets of $T$ in $G$ that are defined over $F$ in the sense
%that $\sigma(Tg) = Tg$ for all $\sigma\in \op{Gal(\bar F/F)$.  Let arrows $h(Tg) = Tg'$ be
%given by elements $h\in G(F)$.
%\end{example}


\subsection{mass formula for orbital integrals}

The description of orbital integrals in terms of affine Springer
fibers takes the following form in NBC.  It is a variant of KGM, KL,
expressed in a stacky language.  We note that all geometry in NBC is
carried out in the language of stacks, as developed in
\cite{LMB:2000}.  In regards to stacks, he makes no compromise.  In
particular, the collection of points of a stack form a groupoid (the
set of points together with isomorphisms between points marked with
arrows).

Let $\M_v(a)$ be the affine Springer fiber for the element $a$ and let
$J_a$ be its centralizer.  We write $\P_v(J_a)$ for the group of
symmetries of the affine Springer fiber.  (The notation for this group
will be explained below.)  Let $\C$ be the groupoid of $k$-points of
the quotient $[\M_v(a)/\P_v(J_a)]$ with objects $(m,p)$ and morphisms
and $h$ defined by the earlier formulas (\ref{eqn:objects}) and (\ref{eqn:arrows}).

For each character of $H^1(k,\P_v(J_a))$ we can naturally associate a
character $\kappa$ of $H^1(F_v,J_a)$ as well as a character (also
called $\kappa$) on a finite abelian group $A$ as above.

\begin{theorem}\label{lemma:orbital-mass}
The $\kappa$-mass of the category $\C$ is equal to the
$\kappa$-orbital integral of $a$:
\[
\mu_\kappa(\C) = c\,\, \OO_\kappa(a),
\]
up to a constant $c=\op{vol}(J^0_a(\O_v),dt_v)$ used to normalize measures.
\end{theorem}
% Prop 8.2.7.

%Because of this result, I will use the terms ``$\kappa$-mass'' and
%``$\kappa$-orbital integral'' more or less interchangeably.  

Let us write $\mu_{\kappa,v}(a)$ for this mass and $\mu_{H,v}(a_H)$
for the mass of the affine Springer fiber of $M_{H,v}(a_H)$.



\subsection{product formula for masses} 

Let $\A_H$ be the base of the Hitchin fibration for an endoscopic group $H$.  There
is a closed immersion $\nu:\A_H\to\A$.

%% XX GIve J'_a, etc. 

Let $\mu_\kappa(a)$ be defined as the $\kappa$-mass of the groupoid of
$k$ points of the quotient $[\M_a/\P(J_a')]$ of the Hitchin fiber by
its Picard stack $\P_a'$ of symmetries.  This depends on the usual
data $G,X,D$.  On the endoscopic side, let $\mu_H(a_H)$ be the mass of
the of the groupoid of $k$-points of the quotient
$[\M_H(a)/\P(J'_a)]$.

\begin{theorem}\label{lemma:product}
 The masses satisfy a product formula over all places of $X$ in terms
of the masses of the individual affine Springer fibers:
\[
\mu_\kappa(a) =\prod_v \mu_{\kappa,v}(a), \quad \mu_H(a_H) = \prod_v \mu_{H,v}(a_H).
\]
The local factors are $1$ for almost all $v$ so that the products are in fact finite.
\end{theorem}

\begin{proof}[proof sketch]
This is a geometric version of the factorization of $\kappa$-orbital
integrals over the adele group into a product of local
$\kappa$-orbital integrals in \cite{Langlands:debuts}.

The proof choses an open set of $X$ over which $J'_a$ is isomorphic to
$J_a$. For a given $a$, at a possibly smaller open set $U$ of $X$, the
action of $\P_a$ on $\M_a$ induces an isomorphism of $\P_a(J_a)$ with
$\M_a$.  The product in the lemma can be taken as extending over the
finite set of places $X\setminus U$.  The lemma is based on a product
formula for stacks
\[
[\M_a/\P_a(J_a)] = \prod_{X\setminus U} [\M_{v,a}/\P_{v,a}].
\]
and a similar formula for $H$.
\end{proof}

\subsection{global mass formula}

The following is the key global ingredient of the proof of the FL.

\begin{theorem}[Global Mass Formula]\label{lemma:gmf}
  Assume $\deg(D)>2g$, where $g$ is the genus of $X$.  Then for all
  $a\in \tA_H^\good(k)$ with image $\nu(a)$ in $\tA$, the following
  mass formula holds:
\[
\mu_\kappa(\nu(a)) = q^{r(a)} \mu_H(a).
\]
\end{theorem}

\begin{proof}[proof sketch]
  The proof first defines a particularly nice open set $\tU$ of
  $\tA_H^{\good}$.  The idea is to impose as many conditions as
  possible on $\tU$ to make it as nice as possible, without imposing
  so many conditions that it fails to be open.  There exists an open
  set $\tU$ of the good locus on which all of the following conditions
  hold:
\begin{itemize}
\item It is a subset of the anisotropic locus.
\item For each $n$, the restriction to $\tU$ of the perverse cohomology sheaves 
    $\tilde\nu^*\, {}^p\!H^n(\tilde f^\a_*\bar {\ring{Q}}_\ell)_\kappa$ and 
   ${}^p\!H^{n+2r}(\tilde f_{H,*}\bar{\ring{Q}}_\ell)_{st}(-r)$ are pure local systems of weight $n$.
\item Each $a_H\in \tU(\bar k)$ cuts the divisor $\DIV_{H,D}+\RDIV^G_{H,D}$ transversally.
\end{itemize}
NBC's support theorem is used to see that $\tU$ is contained in the
support of of every geometrically irreducible factor of the perverse
cohomology sheaves.  By the BBDG description of irreducibles, on some
open subset they become local systems~\cite{BBDG:1982}.

After choosing $\tU$, the proof of the lemma establishes the global
mass formula on $\tU$, then extends it to all of $\tA_H^{\good}$.  By
imposing such nice conditions on $\tU$, NBC is able to prove the mass
formula on this subset by explict local calculations.  By the
transversality condition on $a$, at any given place $v$, the local
intersection multiplicities $(d_{H,v}(a_H),r_{v}(a))$ must be $(0,0)$,
$(1,0)$, or $(0,1)$.  From Bezrukavnikov's dimension formula,
mentioned in Section~\ref{sec:coset}, the dimension of the endoscopic
affine Springer fiber $\M_{H,v}(a)$ is $0$.  In fact, the symmetry
group $\P_v(J_{a})$ acts simply transitively on the affine Springer
fiber, and the mass of the groupoid of $k$-points is $1$.

It is therefore enough to show that the $\kappa$-mass of $\nu(a)$ is also $1$.
The transversality condition also determines the
possibilities for the local intersection multiplicities of $\nu(a)$ in $G$.
The affine Springer fiber in this case is at most one and the
$\kappa$-masses of the groupoids of $[\M_v(a)/\P_v(J_a)]$ can be
computed directly.  In fact, Example~\ref{eqn:groupoid} is a typical example of the
computations involved.  

The result of these calculations is that for every point
$a$ in $\tA_H$, with image $\nu(a)\in \tA$, a local mass formula holds:
\[
\mu_{\kappa ,v}(\nu(a)) = q^{\deg(v) r_v(a)} \mu_{H,v} (a), \quad \text{ for all places } v.
\]
The exponents satisfy
\[
r(a) = \sum_v \deg(v) r_v(a).
\]
These two identities, together with the product formula for the global mass, give
the lemma for elements $a$ of $\tU$.  

The extension from $\tU$ to all of $\A_H^{\good}$ is a
global argument.  Using the Grothendieck-Lefschetz trace formula (adapted to stacks), this
identity of global masses over $\tU$ can be expressed as an identity of alternating sums
of trace of Frobenius on local systems.  These calculations can be
repeated for all finite extensions $k'/k$ and as we vary $k'$,
Cebotarev implies that the semisimplifications of the local systems
are isomorphic on $G$ and $H$.

By the NBC support theorem and BBDG, this isomorphism of local systems
on $\tU$ extends to an isomorphism (of their intermediate extensions)
over all of $\A_H^{\good}$.  This isomorphism of perverse cohomology
sheaves, again by Grothendieck-Lefschetz, translates back into
counting points on the Hitchin fibration, and hence the global mass
formula.
\end{proof}


\subsection{local mass formula and the FL}\label{sec:lmf}

Let $k$ be a finite field with $q$ elements, let $\O_w$ be the ring of
formal Laurent series over $k$ and $F_w$ its field of fractions.  Let
$G_w$ be a reductive group scheme over $\O_w$.  Let $(\kappa,\rho)$ be
endoscopic data defining the endoscopic group $H_w$.  Let $a \in
\cc_H(F)$ and $\nu(a)$ be its image in $\cc(F)$.  Assume that $\nu(a)$
is regular semisimple

Let $J_w$ be a group scheme over $\O_w$ that has a connected special
fiber and that is isomorphic over $F$ to the centralizer $J_a$.

Let $r_w(a)\in\ring{N}$ be the local invariant of Section~\ref{sec:statement}.

Assume that the characteristic of $k$ is large (larger than twice the
Coxeter number of $G$).  By
standard descent arguments (see Section~\ref{sec:descent}), we also assume
without loss of generality that the center of $H_w$ does not contain a
split torus.



\begin{theorem}[Local Mass Formula]\label{lemma:lmf}
  The following local mass formula holds for general anisotropic
  affine Springer fibers (both masses being computed with respect to
  the same symmetry group $\P(J_w)$ acting on the fibers):
\[
\mu_{\kappa,w}(\nu(a)) = q^{\deg(v) r_w(a)}\mu_{H,w}(a).
\]
\end{theorem}

\begin{corollary}[Fundamental Lemma]\label{lemma:fl}
$$\OO_\kappa(\nu(a)) = q^{r_w(a)}\SO_H(a).$$
\end{corollary}

The corollary follows from the theorem by
Lemma~\ref{lemma:orbital-mass}, which expresses orbital integrals as
local masses.

\begin{proof}[proof sketch]
  The proof of the FL is a global argument based on the global mass
  formula on $\A_H^{\good}$ (Theorem~\ref{lemma:gmf}).  We can use
  standard strategies to embed the local setting into a global
  setting; that is, to put the affine Springer fiber into a Hitchin
  fiber.  NBC checks that we pick a curve $X$, such that a completion
  of the function field at some place $w$ is isomorphic to $F_w$.  We
  choose a global endoscopic groups $H$ of a reductive group $G$, a
  divisor $D$ on $X$, a global element\footnote{More accurately, NBC
    shows that a suitable element $a'$ exists over every sufficiently
    large finite field extension $k'/k$.  He makes the global
    arguments over the extensions $k'$ and uses a Frobenius eigenvalue
    argument at the end to go back to $k$.}  $a'$ in the Hitchin base
  $\A_H$ of $H$.  If the degree of $D$ is sufficiently large, then
  $a'$ lies in $\A_H^{\good}$.  The element $a'$ and its image in
  $\nu(a')\in\A$ are chosen to approximate the given local elements
  $a$ and $\nu(a)$ so closely that at $w$ their affine Springer fibers
  together with Picard symmetries are unaffected.

The global mass formula for $a'$ asserts:
\[
\mu_\kappa(\nu(a')) = q^{r(a')} \mu_H(a').
\]
 By the product formula,
each global mass is a product of local masses:  
\[
\mu_\kappa(\nu(a')) = \prod_v \mu_{\kappa,v}(\nu(a')),\quad
\mu_H(a') = \prod_{H,v}\mu_{v}(a').
\]

The global data is chosen so that at every
place $v$ other than the chosen place $w$, the transversality conditions hold, so that
the calculations of the previous section give the local mass formula at $v$:
\[
\mu_{\kappa,v}(\nu(a')) = q^{r_v(a')}\mu_{H,v}(a'),\quad v\ne w.
\]  
These masses are nonzero and  can be canceled from
the product formula.  What remains is a single term on each side of the product formula:
\[
\mu_{\kappa,w}(\nu(a')) = q^{r_w(a')}\mu_{w}(a').
\]
Since, $a'$ was chosen as a  close approximation of $a$ at the place $w$, we also have
\[
\mu_{\kappa,w}(\nu(a)) = q^{r_w(a)}\mu_{w}(a).
\]
This is the desired local mass formula at the place $w$.
\end{proof}

After completing the proof of the FL,
in one final zigzag between the local and global, NBC extends the global mass formula
from $\tA_H^{\good}$ to all of $\tA^{\a}$, as well as giving an identity of the underlying
perverse cohomology sheaves.   This proof is similar to the local calculations
used to establish the global mass formula on $\tU$, but relies on the  added local
force of the Lemma~\ref{lemma:lmf}.


\section{uses of the FL}  \label{sec:uses}
% 6 pages.

The name ``lemma'' is also misleading because it went decades
without a proof, and its depth goes far beyond what would ordinarily be
called a lemma.  

Yet the name FL is apt both because it is fundamental and
because it is expected to be used widely as an intermediate result in
proving many different results.  This section mentions some major
theorems that have been proved recently that contain the FL as an
intermediate result.  In each case, the FL appears to be
an unavoidable ingredient.

The FL appears as a specific collection of identities
that are needed to stabilize the Arthur-Selberg trace formula.  By
stabilization, of the trace formula we mean that the terms on the
geometric side of the trace formula that associated with a given
stable conjugacy class in a reductive group are gathered together.
These terms are rearranged into $\kappa$-orbital integrals, which are
then transferred to stable orbital integrals on the endoscopic groups
by the FL.  Applications of the FL come through the stabilized
Arthur-Selberg trace formula.  Since a recent Bourbaki talk
contains a detailed description of the FL in relation to the trace
formula, I will not repeat it here~\cite{Dat:2004}.

Before going into recent uses of the FL, we might also mention various
special cases of the FL that have been known for years.  These special
cases of the FL already give abundant evidence of the usefulness of
the lemma.  For example, Langlands proves the FL for
cyclic base change for $GL(2)$ in his book \cite[Lemma~5.10]{LBC:1980}.
From there, it enters into the proof of the tetrahedral and
octahedral cases of the Artin conjecture (the Langlands-Tunnell theorem),
which in turn is used by Wiles in the proof of Fermat's Last theorem.
Waldspurger's proof of the FL for $SL(n)$ is taken up by Henniart and Herb in their
proof of local automorphic induction for $GL(n)$, which becomes part of the
proof of the proof of the local Langlands correspondence for $GL(n)$
in Harris and Taylor~\cite{Wald:1991},\cite{Herb:Autoinduct},\cite{Harris:Taylor:local}.  


For Langlands, Shimura varieties provided much of the early motivation
for endoscopy and the FL~\cite{LZ:1979}.  When writing the 
Hasse-Weil zeta function of Shimura varieties as a product of
automorphic $L$-functions, the formula involves the $L$-functions
associated with endoscopic groups $H$ as well as those of $G$.  This
can be most clearly through a comparison of the stable trace formula (using the FL)
with the Grothendick-Lefschetz trace formula of the Hasse-Weil zeta
function.
% On the Zeta-Functions of Some Simple Shimura Varieties.
% http://publications.ias.edu/sites/default/files/simpshim-ps.pdf
An early application of the FL carries this out for
Picard modular
varieties.~\cite{LPicard:1992}.  From there, the FL becomes relevant to the theory of Galois representations
through the representations associated with Shimura varieties.
%  Picard modular varieties.

Let's turn to more recent uses of the FL.  
For most applications to
date, the FL for unitary groups is used as well as the twisted FL
between $GL(n)$ and unitary groups.  Applications of the trace formula
to Shimura varieties often rely on a base change FL.  This base change FL arises
because of the description of that Kottwitz gives of points on certain
Shimura varieties in terms of twisted orbital
integrals~\cite{Kott:1990}.

% Kottwitz R.: Shimura varieties and $\lambda$-adic representations, in Automor- 

The original proof by Clozel, Harris, Sheppard-Baron and Taylor
of the Sato-Tate conjecture for elliptic curves over
$\ring{Q}$ was restricted to elliptic curves with non-integral
$j$-invariants~\cite{Car:Bourbaki}.
With the advent of the general FL is has become possible to remove
the non-integrality assumption and to greatly extend the theorem, in particular 
to all elliptic modular forms of positive weight \cite{BGHT:2010}, etc.


% Carayol's Bourbaki:
% http://www.mathematik.hu-berlin.de/gradkoll/Carayol_Exp.977.H.C4.pdf
% Harris's survey
%http://www.math.unipd.it/~algant/IHP_SMF.pdf

%\subsection{Shimura varieties}

Shin and Morel use the FL in their recent work on the cohomology of
Shimura varieties and associated Galois
representations~\cite{Shin:2010}~\cite{Morel:2010}.  Several of the
other recent uses of the FL rely on their work.  In particular,
%\subsection{Iwasawa main conjecture for $GL(2)$}
Skinner and Urban have proved the Iwasawa-Greenberg main conjecture
for many modular forms and in particular for the newforms associated
with many elliptic curves over
$\ring{Q}$~\cite{Skinner-Urban:2010},\cite{Skinner:2010}.  Their work
ultimately relies on the work of Shin and Morel and on the FL to prove
the existence of certain Galois representations.  It uses the FL for
unitary groups, the twisted FL relating $GL(n)$ to unitary groups, and
some base change identitites.

%\subsection{average ranks of elliptic curves}

Last year, Bhargava and Shankar proved that when elliptic curves $E$
over $\ring{Q}$ are ordered by height, a positive fraction of them
satisfy the Birch and Swinnerton-Dyer conjecture~\cite{BS:2010}.
Specifically, a positive fraction of them have rank $0$ and analytic
rank $0$.

First they construct a set (of positive
density) of elliptic curves with rank $0$.  Second, they construct a
subset (again of positive density) of the rank $0$ set, consisting of
elliptic curves with analytic rank $0$.  This second step relies on
conditions in Skinner and Urban for the analytic rank to be zero, and hence
indirectly on the FL.

%\subsection{local Langlands for unitary groups}

Moeglin classifies the discrete series representations of unitary
groups over a nonarchimedean local field~\cite{Moeglin:2007}.  Again, this
relies on the FL for unitary groups and twisted FL for
$GL(n)$. Finally, we mention that Arthur's forthcoming project uses
the twisted FL between $GL(n)$ and the classical
groups~\cite{Arthur:2011}.  His work uses the trace formula to give a
classification of the discrete automorphic representations of
classical groups in terms of cuspidal automorphic representations of
$GL(n)$.  It also gives a classification locally, for $p$-adic fields.

I will leave a further discussion of the uses of the FL to those who research
in this area is fresher than my own!

%
%







\section{reductions}\label{sec:reduce}

Langlands first expressed the FL in these words:
``Mais m\^eme apr\`es avoir
v\'erifi\'e que les facteurs de
transfert existent, il reste \`a v\'erifier ce que j'appelle le
lemme fondamental, qui affirme que pour des $G$, $H$ et $\phi_H$
non-ramifi\'es, on a $f\mapsto c\, \phi_H^*(f)$ [pour toute fonction $f\in {\mathcal H}_G$].''
 -- p.49 (Les deb.)  \cite[p.49]{Langlands:debuts}

In Langlands's notation $\phi_H^*$ is the homomorphism given by the
Satake transform, from the spherical Hecke algebra ${\mathcal H}_G$ on
$G$ to the spherical Hecke algebra on $H$ with respect to a hyperspecial maximal compact
subgroup.  The arrow $f\mapsto
c\,\phi_H^*(f)$ is his assertion of that for every strongly $G$-regular element $\gamma$
in $H$, the transfer (via transfer factors) of each $\kappa$-orbital integral
of a spherical function $f$ on $G$ (over a stable conjugacy class in $G$ matching $\gamma$) is
equal to the stable orbital integral of $\phi_H^*(f)$ on the stable conjugacy class of $\gamma$ in $H$.

This final section describes some theorems related to the fundamental
lemma that simplify it from the form in which it was initially
conjectured, to the final form in which it was proved by NBC.
Waldspurger's work has been particularly significant in transforming the conjecture into a 
friendlier form.  In Langlands's
initial conjecture, the existence of transfer factors was part of the conjecture.
The first progress was to define the transfer factors explicitly.  
This was done in \cite{LS:1987}.


%Twisted Fundamental Lemma, Lie algebra, Weighted Fundamental Lemma,
%Reduction to Units, Smooth transfer, descent, transfer to char 0.

\subsection{descent to the Lie algebra}\label{sec:descent}

A lemma of Harish-Chandra's asserts the transfer of an orbital
integral on $G$ near a singular semisimple element $z\gamma_0$, with
$z$ central, to an orbital integral on $C_G(\gamma_0)$, the
centralizer of $\gamma_0$.  This is called the descent of orbital
integrals.  Langlands and Shelstad made hard calculations in
Galois cohomology to prove that the transfer factors are compatible
with Harish-Chandra's descent of orbital integrals~\cite{LS:1990}.  The point of
their calculations was to make it possible to prove results about
orbital integrals by induction on the dimension of the group, and
thereby reduce smooth transfer and the FL to identities
in a neighborhood of $\gamma_0=1$.  In a neighborhood of $\gamma_0=1$,
identities can be pushed to the Lie algebra, using the exponential map.

The original FL has been supplemented by a twisted
FL, conjectured by Kottwitz and Shelstad, where the
data is twisted by a nontrivial outer automorphism $\theta$ of the
group $G$~\cite{KS:1999}.  In the untwisted case, the centralizer of an element fails
to give a group of smaller dimension precisely when the element is
central.  By contrast, a twisted centralizer (with respect to a
nontrivial outer automorphism) always has dimension less than $G$.  As
a consequence, descent always {\it untwists} the twisted fundamental
lemma into identities of ordinary orbital integrals.  If the
(standard) FL is then applied, each $\kappa$-orbital
integral can be replaced with a stable orbital integral.  By combining both
descent and stabilization, the twisted FL of Kottwitz and Shelstad
takes the form of identities of stable orbital integrals on the Lie
algebra (from which the automorphism and the character $\kappa$ have
entirely vanished).  The corresponding long calculations in Galois
cohomology that establish descent properties of the transfer factors
for the twisted FL have been carried out by
Waldspurger \cite{Wald:2008}.  NBC proves the general twisted FL in
its untwisted stable form on the Lie algebra.



\subsection{Hecke algebras}

%The FL as originally formulated by Langlands
%is a statement about the transfer of all
%functions in the spherical Hecke algebra. 
 A global argument based on
the trace formula shows that the FL holds for the full
Hecke algebra for an arbitrary nonarchimedean local field of
characteristic zero, provided it holds for the unit element of the
Hecke algebra for local fields of sufficiently large residual
characteristic (and for groups of smaller dimension)~\cite{FLSE}.  
The idea of the proof is to choose suitable global
functions for which the comparison of stable trace formulas yields an
obstruction to the FL.  This obstruction, which comes from the
spectral side of the trace formula, takes the form
of a set of linear functionals  
\[
L:{\mathcal H}\to\ring{C},\quad L(f) = \sum_\pi a(\pi) \op{trace}\,\pi(f),
\]
on the local spherical Hecke
algebra ${\mathcal H}$ of the reductive group $G$, each given by a finite sum over
irreducible admissible representations with an Iwahori fixed vector.
By purely local arguments, it
can be shown that no nonzero linear map $L$ exists of the form
prescribed by the global theory.  
Because the obstructions $L$ are zero,
the FL can be shown to hold on the full spherical Hecke algebra.


\subsection{smooth transfer}

Langlands's book on the stabilization of the trace formula contains
two separate conjectures~\cite{Langlands:debuts}.  The first is the transfer of smooth
functions.  The second is the FL.  An important result
of Waldspurger links the two conjectures, by proving that the
FL implies the transfer of smooth functions.  
His key local lemma shows how to obtain simultaneous control over the
orbital integrals of test functions $f$ and the orbital integrals of
their Fourier transform $\hat f$~\cite[Prop.~8.2]{Wald:transfert}.  
In view of the uncertainty principle, it is a remarkable feat to
control both $f$ and $\hat f$ as he does.
% push up against the limits imposed by the
%uncertainty principle.
%is generally deemed nearly impossible by the
% Prop 8.2, p198.
His proof is a global argument, based on a stable Poisson summation
trace formula over the ring of adeles.  The key local lemma allows
Waldspurger to pick global test functions for which the comparison of
trace formulas asserts a local identity: the Fourier transform of a
semisimple $\kappa$-orbit on $G$ equals the Fourier transform of the
corresponding stable orbit on $H$.  By a purely local argument, this
identity of Fourier transforms of orbits implies smooth transfer.

% FL implique le transfert.

\subsection{weighted orbital integrals}

Langlands's book is called ``Les d\'ebuts'' for a reason.  In the
stabilization of the trace formula, he only stabilizes the terms that
come from regular elliptic conjugacy classes.  This is insufficient
for general applications of the trace formula.  Kottwitz extended the
analysis to singular elliptic conjugacy classes~\cite{Kott:singular}.  Arthur has
completed the full stabilization without restrictions on the conjugacy
classes.  The non-elliptic conjugacy classes lead to significant
complications.  Arthur truncates the trace formula to obtain the convergence
of the non-elliptic terms.  Because of truncation, the non-elliptic
terms bear ``weights,'' non-invariant factors that appears in the
integrand of orbital integrals.  Arthur conjectured a weighted
FL needed for stabilization of the non-elliptic
terms~\cite{Arthur:2002}.  Chaudouard and Laumon have used the
Hitchin fibration to prove Arthur's weighted FL~\cite{CL:2009:I},\cite{CL:2009:II}.


\subsection{transfer to characteristic zero}

The FL for nonarchimedean local fields in
characteristic zero can be deduced from the FL in
positive characteristic~\cite{Wald:2006},~\cite{CHL:2010}.  Cluckers and Loeser have developed a
general abstract theory of integration as a combination of primitive
operations such as taking the volume of a ball of given radius,
enumerating points on a variety over the residue field, summing an
infinite $q$-series, and making a change of variables.  Since
each of the primitive operations manifestly depends only on the
residue field rather than the field itself, their theory allows many
identities of integrals to be transfered from one field to another
with the same residue field.  The FL lemma and its weighted and
twisted variants are identities that fall within the scope of this
theory.

These separate variations on the FL can be considered in concert: a
weighted twisted fundamental lemma, the twisted fundamental lemma on
the full Hecke algebra, transfer of the weighted FL to characteristic zero, and so forth.
Most combinations have now been proved.

\section{literature}

I recommend Ben-Zvi's video lecture for mathematicians in other fields
who want a one hour presentation of the big ideas of the proof.
Drinfeld's lecture notes contain many worked examples and exercises
that are helpful in learning the basic geometric ideas.  I also
recommend Nadler's survey~\cite{Nadler:2010}, Casselman for an
in-depth treatment of $SL_2$ with history going back to
Hecke~\cite{Cass:2010}, my summer school lecture for the detailed
statement of the FL~\cite{Hales:FL-statement},  Arthur's laudation
for the FL in the context of the trace formula~\cite{Arthur:2010}, and \cite{CHLaumon:2010}.
Several articles in the 
book project \cite{Harris:book-project} deal with the FL~\cite{Harris:book-project}, 
particularly~\cite{DN:2010}.


NBC's book is superb, both as mathematics and as
exposition~\cite{NBC:2010}!  It is helpful to read it with his earlier
paper~\cite{NBC:2006}.  He has several supplementary accounts,
especially~\cite{NBC:report}, his article in the book project, Hyderabad ICM report,
and report at the Rapoport conference in Bonn.


%% Book project.

While there have been numerous applications that quote the fundamental
lemma as a finished product, Yun, Chaudouard, and Laumon are
noteworthy in following Ng\^o in their direct use of the Hitchin fibration to
prove new results in harmonic analysis~\cite{Yun:2009}.

% Yuan.

I wish to thank M. Bhargava, M. Harris, NBC, and C. Skinner for comments that
have helped me to prepare this and my earlier report~\cite{thales:NBC:2011}.
 

\raggedright
\bibliographystyle{plain} %plainnat
\bibliography{/Users/thomashales/Desktop/googlecode/flyspeck/latex/bibliography/all}


\end{document}