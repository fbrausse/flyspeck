\documentclass[brochure,english,12pt]{bourbaki}
\usepackage[matrix,arrow]{xy}
\usepackage{amssymb,amsfonts,amsmath,footnote}
\usepackage[francais]{babel}
%\usepackage{pxfonts}
\addressindent 100mm

\date{Avril 2011}
\bbkannee{63\`eme ann\'ee, 2010-2011}
\bbknumero{1035}
\title{The fundamental lemma and the Hitchin fibration}
\subtitle{after Ng\^o Bao Ch\^au}
\author{Thomas HALES}
\address{University of Pittsburgh\\
Department of Mathematics\\
Pittsburgh, PA 15260 -- U.S.A.}
\email{hales@pitt.edu}

% sectioning.
\newtheorem{example}[equation]{Example}
\newtheorem{definition}[equation]{Definition}
\newtheorem{theorem}[equation]{Theorem}
\newtheorem{lemma}[equation]{Lemma}
\newtheorem{corollary}[equation]{Corollary}


% Math notation.
\def\op#1{{\operatorname{#1}}}
\newcommand{\ring}[1]{\mathbb{#1}}
\newcommand{\NBC}{Ng\^o Bao Ch\^au}

\def\AD{\ring{A}}
%general
\def\b{\backslash }
%\def\mass{{\mathbf\mu}}
\def\card{\op{card}}
\def\a{{\scriptsize\text{ani}}}
\def\good{{\scriptsize\text{good}}}
\def\diamond{{\blacklozenge}}
\def\SO{{\mathbf {SO}}}
\def\OO{{\mathbf O}}


%frak
\def\so{\frak{so}}
\def\sp{\frak{sp}}
\def\gl{\frak{gl}}
\def\sl{\frak{sl}}
\def\g{\frak{g}}
\def\t{\frak{t}}
\def\c{\frak{c}}
\def\D{{\frak{D}}}
\def\R{{\frak{R}}}


%mathcal
\def\A{{\mathcal A}}
\def\C{{\mathcal C}}
\def\M{{\mathcal M}}
\def\P{{\mathcal P}}
\def\O{{\mathcal O}}
\def\tA{{\tilde{\mathcal A}}}
\def\tU{{\tilde{\mathcal U}}}





\begin{document}



\maketitle



{

\narrower{\it The study of orbital integrals on $p$-adic groups has turned
out to be singularly difficult. -- R. P. Langlands, 1992} % Real Igusa

}

\bigskip

%\section{introduction}
% 7 pages.


This report describe some identities of integrals that have
been established by \NBC.  In mathematics, we have an unlimited supply
of integrals and identities of integrals.  Part of my task 
will be to describe why these identities took nearly thirty years to
prove, and why they have particular importance for investigations in
the theory of automorphic representations.

\begin{abstract}
About thirty years ago, R. P. Langlands conjectured a collection of
identities to hold among integrals over conjugacy classes in
reductive groups.  Ng\^o Bao Ch\^au has proved these identities
(collectively called the fundamental lemma) by interpreting the
integrals in terms of the cohomology of the fibers of the Hitchin
fibration.  The fundamental lemma has profound consequences for the
theory of automorphic representations. Significant recent theorems in
number theory use the fundamental lemma as an ingredient in their
proofs.
\end{abstract}




\section{origins of the fundamental lemma}


To orient ourselves, we give two examples of
behavior that the theory is designed to explain.


\begin{example}  We recall the definition of the holomorphic discrete series representations
of $SL_2(\ring{R})$.  For each natural number $n\ge 2$, let $V^+_n$ be the vector space
of all holomorphic functions $f$ on the upper half plane ${\frak h}$ such that
\[
\int_{\frak h} |f|^2 y^{n-2} dx\, dy < \infty.
\]
$SL_2(\ring{R})$ acts on $V^+_n$:
\[
\begin{pmatrix} a & b \\ c & d \end{pmatrix} \cdot f(z) = 
(-b z + d ) ^{-n} f (\frac{\phantom{-}a z - c}{-b z + d}).
\]
For each $n\ge 2$, $SL_2$ also acts on the vector space $V^-_n$ of
complex conjugates of functions in $V^+_n$.  These infinite
dimensional representations have characters that exist as integrable
functions $\chi_{n,\pm}$.  The characters are equal
$\chi_{n,+}(g)=\chi_{n,-}(g)$, except when $g$ is conjugate to a
rotation
\[
t = \begin{pmatrix} \phantom{-}\cos\theta & \sin\theta \\ -\sin\theta & \cos\theta\end{pmatrix}.
\] 
When $g$ is conjugate to $t$, a remarkable character identity holds:
\[
\chi_{n,-}(t) - \chi_{n,+}(t) = 
\frac{e^{i n \theta} + e^{- i n \theta}}{e^{i\theta}-e^{-i\theta}}.
\]
It is striking that numerator of the difference of two characters of
infinite dimension representations collapses to the character of a two
dimensional representation $t\mapsto t^n$ of the group $H$ of
rotations.  Shelstad gives general characters identities of this
sort~\cite{Sh}.
\end{example}

\begin{example}
  We find another early glimpse of the theory in an exchange
  between Langlands and Singer.  Singer expressed interest to
  Langlands in a particular alternating sum of dimensions of spaces of
  cusp forms of $G=SL_2$ over a totally real extension of
  $\ring{Q}$.  In 1974, Langlands replied to Singer in a letter
  describing then unpublished joint work with
  Labesse~\cite{Singer},~\cite{LL}.  The exact details do not concern
  us here.  The important point for us is that in the formula for this
  alternating sum, there again is a collapse in complexity from the
  three dimensional group $SL_2$ to a sum indexed by one-dimensional
  groups $H$ (of norm $1$ elements of quadratic extensions of the totally real number field).
\end{example}
%Letter to Singer 1974.

These two examples fit into a general framework that has been under
active development decades.  Langlands holds that methods should be
developed that are adequate for the theory of automorphic
representations in its full natural generality.  In ``full natural
generality'' means going from $SL_2$ (or even a torus) to all
reductive groups, from one local field to all local fields, from local
fields to global fields and back again, from the geometric side of the
trace formula to the spectral side and back again.  Moreover,
interconnections between different reductive groups and Galois groups
should be included, as predicted by his general principle of
functoriality.


Thus, from these early calculations of Labesse and Langlands, the
general idea developed that one should account for alternating sums
(or $\kappa$-sums as we shall call them because occasionally they
involve roots of unity other than $\pm1$) that appear in the harmonic
analysis on a reductive group $G$ in terms of the harmonic analysis on
groups $H$ of smaller dimension.  The fundamental lemma is a concrete
expression of this idea.


\section{stable conjugacy}

At the root of these $\kappa$-sum formulas is the distinction between
{\it ordinary conjugacy} and {\it stable conjugacy}.  
\begin{example}
A clockwise rotation and counterclockwise rotation
\[
\begin{pmatrix}\cos\theta &-\sin\theta\\\sin\theta &\phantom{-}\cos\theta\end{pmatrix},\quad
\begin{pmatrix}\phantom{-}\cos\theta &\sin\theta\\-\sin\theta &\cos\theta\end{pmatrix}
\]
in $SL_2(\ring{R})$ are conjugate by the complex matrix $\begin{pmatrix}i&0\\0&-i\end{pmatrix}$, 
but they are not conjugate
in the group $SL_2(\ring{R})$ when $\theta\not\in\ring{Z}\pi$.  Indeed,
% if conjugate in $SL_2(\ring{R})$ they are
%also conjugate in $GL_2(\ring{R})$, but 
a matrix calculation shows
that every element of $GL_2(\ring{R})$ carrying the rotation to
counter-rotation has odd determinant, thereby falling outside $SL_2(\ring{R})$.
\end{example}
%For example, the diagonal
%matrix with eigenvalues $1$ and $-1$ conjugates the clockwise rotation
%to a counterclockwise rotation.  
%The sign of the determinant gives an
%invariant in a group of order two that distinguishes conjugacy within
%stable conjugacy for $SL_2$.

\begin{definition} Let $G$ be a reductive group defined over a field
  $k$ with algebraic closure $\bar k$.  An element
  $x'\in G(k)$ is said to be {\it stably conjugate} to a given regular
  semisimple element $x\in G(k)$ if $x'$ is conjugate to $x$ in
  the group $G(\bar k)$.
\end{definition}

%Each stable conjugacy class is a union of ordinary conjugacy classes in $G(k)$.

There is a Galois cohomology group that can be used to study the conjugacy classes within
a given stable conjugacy class.
Write $x'=gx g^{-1}$, for $g\in G(\bar k)$.
For every element $\sigma$ of the Galois group $\op{Gal}(\bar k/k)$,  we have
 $\sigma(g)g^{-1}\in T(\bar k)$ and defines a cohomology class in
$H^1(\op{Gal}(\bar k/k),T(\bar k))$, where $T$ is the centralizer of
$x$.  The cohomology class defined by the cocycle does not depend
on the choice of $g$.  It is the trivial class when
$x'$ is conjugate to $x$.

%When $k$ is a local field, the cohomology group $H^1$ is finite.  When $k=\ring{R}$,
%it is an elementary abelian $2$-group.
%The origin of the alternating signs in Examples XX and XX is a character
%$\kappa:H^1(\bar k/k,T)\to\ring{C}$.  
%For fields $k\ne\ring{R}$, the cohomology
%group is needn't be a $2$-group, so we will henceforth refer to $\kappa$-sums rather
%than alternating sums. 

\begin{example} The centralizer $T$ of a noncentral rotation $x$ is  the subgroup of all
rotations in  $SL_2(\ring{R})$.  The group $T(\ring{C})$ is isomorphic
  to $\ring{C}^\times$.   Each cocycle is identified with the value $r\in T(\ring{C})=\ring{C}^\times$ 
   of the cocycle on generator of  $\op{Gal}(\ring{C}/\ring{R})$.  It is a cocycle when $r\in \ring{R}^\times$
  and is the trivial class in cohomology when $r$ is positive.  This identifies the cohomology group:
\[
H^1(\op{Gal}(\ring{C}/\ring{R}),T(\ring{C})) = \ring{R}^\times/\ring{R}^\times_+ = \ring{Z}/2\ring{Z}.
\]
 This cyclic group of order two classifies the two conjugacy classes within the stable conjugacy class
 of a rotation.
\end{example}

When $k$ is a local field, $A=H^1(\op{Gal}(\bar k/k),T(\bar k))$ is a
finite abelian group.  Every function on this abelian group has an
Fourier expansion as a linear combination of characters $\kappa$ on
$A$.  
%Hence by understanding individual characters $\kappa$, we can
%reconstruct arbitrary functions on $A$.  
The {\it theory of endoscopy}
is the subject that studies stable conjugacy through the separate
characters $\kappa$ on $A$.  Allowing ourselves to be deliberately
vague for a moment, the idea of endoscopy is that the Fourier mode of
$\kappa$ (for given $T$ and $G$) produces oscillations that cause some
of the roots of $G$ to cancel away.  Other roots are
reinforced by the oscillations and become more pronounced.  The root
system consisting of the pronounced roots defines a group $H$ of smaller
dimension than $G$.  With respect to the harmonic analysis on the two
groups, the mode of $\kappa$ on the group $G$ should
be related to the dominant mode on $H$.  The fundamental lemma is a precise
conjecture for nonarchimedean local fields that stems from this line of thought.





\section{endoscopic groups}

The
smaller groups $H$ are called endoscopic groups.
Hints about how to define $H$ come from various sources.
\begin{itemize}
\item It should be constructed from $G,T,\kappa$.
\item Its roots are a subset of the roots of $G$ (although $H$ neededn't
be a subgroup of $G$).
\item Over $\ring{R}$, the presence or absence of a purely imaginary
root in $H$ is determined by whether the normalized $\kappa$-orbital
integral jumps across the hyperplane $\alpha=0$.
\item $H$ should have a Cartan subgroup $T_H\subset H$ isomorphic to
$T_G\subset G$, compatible with Weyl group actions $W_H\subset W_G$.
\item Over a nonarchimedean local field, there should be a Satake
homomorphism $f \mapsto \phi^*_Hf$, sending the spherical Hecke algebra
of $G$ to the spherical Hecke algebra of $H$.
\item It should generalize the example of Labesse and Langlands.
\end{itemize}

There are indications that the groups $H$ should be defined through  
the Langlands dual ${}^LG$ of $G$.
\begin{itemize}
\item Langlands's principle of functoriality asserts that the
  representation theory of groups should be related when their dual
  groups are related.  The examples above are relations in
  representation theory.
\item The Satake transform identifies the spherical Hecke algebra with
  a ring of regular functions on a dual object.
\item The Tate-Nakayama isomorphism identifies $H^1(Gal,T)$
with a dual object.
\end{itemize}

The groups $H$ are defined as follows.  Let $s\in \hat T$ map
to the element representing $\kappa$ under the Tate-Nakayama isomorphism.
The connected component of the centralizer of $s$ is the dual $\bar H$
of a quasi-split reductive group.  The choice of
quasi-split inner form is determined by the requirement of a naturally
defined isomorphism $T_H\to T_G$ over $F$.



\begin{definition}
(Rough definition)
An endoscopic group $H$ associated with character $\kappa$ is a group
with the properties
\begin{itemize}
\item $H$ has a Cartan subgroup $T_H$ isomorphic to $T$.
\item The roots of $H$ are in natural bijection with a subset of the roots of $G$.
\item For each imaginary root $\alpha$ of $T$ in $G$, the root
lies in the root system of $H$ exactly when the orbital integral (*)
for a generic Schwartz function has a discontinuous jump across 
the wall of $T$ determined by $\alpha$.
\end{itemize}
\end{definition}


\section{geometry}

Calculations in special cases show why the fundamental lemma is
essentially geometric in nature, rather purely analytic or
combinatorial. We recall a favorite old calculation of mine of the
orbital integrals for $\so(5)$ and $\sp(4)$, the rank two odd
orthogonal and symplectic Lie algebras.  Let $F$ be a nonarchimedean
local field of residual characteristic greater than $2$.  Let $k$ be
the residue field with $q$ elements.  Let $a$ be an element of
$\so(5)\subset \gl(5)$ with eigenvalues $0,\pm t_1,\pm t_2$.  Assume
that there is an odd natural number $r$ such
\[
|\alpha(a)| = q^{-r/2},
\]
for every root $\alpha$ of $\so(5)$. 
There is an elliptic curve $E_a$ over the finite field $k$ given by
$y^2 = (1-x^2\tau_1)(1-x^2\tau_2)$, where $\tau_i$ is the image
of $t_i^2/\varpi^r$ in the residue field.  There are test functions $f$ 
such that the integral of $f$ over the stable orbit
of $a$ has the form
\begin{equation}\label{eqn:elliptic}
A(q) + B(q) | E_a(k)|,
\end{equation}
for some rational functions $A,B$, depending on $f$ and the normalization of measures.

Similarly, in the group $\sp(4)$, there is an element ${a_H}$ with related eigenvalues $\pm
t_1,\pm t_2$.  According to the general framework of (twisted) endoscopy, there
should be a corresponding function $f'$ on $\sp(4)$ such that the
integral over the stable orbit of ${a_H}$ in $\sp(4)$ is equal to
(\ref{eqn:elliptic}).  A calculation of the orbital integral of $f'$ gives a similar formula, with
a different elliptic curve $E'_{a_H}$, but otherwise identical to (\ref{eqn:elliptic}).
The elliptic curves $E_a$ and $E'_{a_H}$
have different $j$-invariants (which vary with $a$ and ${a_H}$).  The
proof of the desired identities of orbital integrals in this case is
obtained by producing an isogeny between $E_a$ and $E'_{a_H}$.  (The identities of orbital integrals are quite
nontrivial, even though the lie algebras $\so(5)$ and $\sp(4)$ are abstractly isomorphic.)

In a similar way in higher rank,  the spectral curves
\[
y^2 = (1-x^2 \tau_1)(1-x^2 \tau_2)\cdots (1-x^2 \tau_n).
\]
appear, for example, in calculations of orbital integrals for
$\so(2n+1)$.  When orbital integrals are computed by brute force,
these curves appear as freaks of nature.  As it turns
out, they aren't freaks at all, merely perverse.  One of the major
challenges of the proof of the fundamental lemma and one of the major
triumphs of NBC has been to find the natural geometrical setting for
orbital integrals and spectral curves.

\section{a bit of Lie theory}



\subsection{characteristic polynomials}



Let $G$ be a split reductive group over a finite field $k$ and let
$\g$ be its lie algebra, with split Cartan subalgebra $\t$ and Weyl
group $W$.  We assume that the characteristic of $k$ is sufficiently
large (more than twice the Coxeter number, to be precise).  The group
$G$ acts on $\g$ by the adjoint action.  Chevalley proves that the restriction  of regular
functions from $\g$ to $\t$ induces an isomorphism
\[
k[\g]^G = k[\t]^W.
\]
We let $\c =  \op{Spec}\,(k[\t]^W)$, and let $\g\to\c$ be the morphism deduced from
Chevalley's isomorphism.  The following example shows that $\g\to\c$ is a generalization
of the characteristic polynomial of a matrix.

\begin{example}
If $G=GL(n)$, the ring $k[\g]^G$ is generated by the coefficients $c_i$ of the characteristic
polynomial 
\begin{equation}
p(t)=t^n + c_{n_1} t^{n-1} +\cdots c_0
\end{equation}
of a matrix $a\in \g$.  The morphism $\g\to\c$ can be identified with
the ``characteristic map'' that sends $a$ to $(c_0,\ldots,c_{n-1})$.
\end{example}

The multiplicative group $\ring{G}_m$ on $\g$ by scalar multiplication.  This extends to an action
of $\ring{G}_m$ on $\c$ for which $\g\to\c$ is equivariant.  For example, in
the case of $GL(n)$, the coefficients $c_i$ of the characteristic
polynomial are homogeneous polynomials on $\g$ and $\ring{G}_m$ acts
according to the degrees of the homogeneous polynomial.

\subsection{Kostant section}

Kostant constructs a section $\epsilon:\c\to\g$ of $\g\to\c$ whose image lies in the set $\g^{reg}$ 
of regular elements
of $\g$.  In simplified terms, this can be viewed as a chosen matrix with a given characteristic polynomial.


\begin{example}
If $\g=\sl(2)$, the characteristic polynomial of $a\in\g$ has the form $t^2  +c_0$.
The Kostant section is the matrix
\[
\begin{pmatrix} 0 & -c_0\\ 1 & 0\end{pmatrix}
\]
\end{example}


\begin{example}
  If $\g=\gl(n)$, we can construct the companion matrix of a given
  characteristic polynomial $p$, by taking the endomorphism $t$ of
  $R=k[t]/(p(t))$, expressed as a matrix with respect to the standard
  basis $1,t,t^2,\ldots,t^{n-1}$ of $R$.  The companion matrix is a
  section $\c\to\g$ that is somewhat different from the Kostant
  section.  Nevertheless, the Kostant section can be viewed as a
  generalization of this that works uniformly for all Lie algebras
  $\g$.
\end{example}


\subsection{centralizers}

Each element $x\in\g^{reg}$ has a centralizer $I_x$ in $G$.  If two
elements of $\g^{reg}$ have the same image $a$ in $\c$, then their
centralizers are canonically isomorphic.  By descent, there is a
regular centralizer $J_a$ on $\c$ that pulls back to $I_x$ under
$x\mapsto a$.



\begin{example}  Suppose $G=SL(2)$.  By identifying $J_a$ with the centralizer of
  $\epsilon(a)$, the centralizer $J_a$ consists of
  matrices of the form
\[
\begin{pmatrix} x & y c_0\\ y & x
\end{pmatrix}
\quad x^2 - c_0 y^2 = 1.
\]
\end{example}

\begin{example}
  If $\g=\gl(n)$, then the centralizer of the companion matrix with
  characteristic polynomial $p$ can be identified with the centralizer
  of $t$ in $GL(R)$, where $R=k[t]/(p(t))$.  An
  element of $\gl(R)$ centralizes the regular element $t$ if and only if it is a polynomial
  in $t$.  Thus, the centralizer in $\gl(R)$ is $R$ and the centralizer in $GL(R)$
 is $J_a = R^\times$.
\end{example}


\section{the statement of the fundamental lemma}

Let $G$ be a reductive group scheme over the ring of integers $\O$ of a nonarchimedean local field $F$
in positive characteristic.
Two regular semisimple elements in $\g(F)$ are stably conjugate exactly when they have the
same image in $\c(F)$.  (That is, the characteristic polynomial of an element in $\g$ determines
its conjugacy class over $\bar F$.)  Using the Kostant section $\epsilon:\c\to\g$,
and then applying the results of  Section~\ref{XX}, each element $x$ stably conjugate
to $\epsilon(a)$  carries
a cohomological invariant in $H^1(\op{Gal}(\bar F/F),J_a(\bar k))$.

For each regular semisimple element $a\in \c(F)$ and character 
\[
\kappa:H^1(\op{Gal}(\bar F/F),J_a(\bar k))\to\ring{C}^\times,
\]
we define the
$\kappa$-orbital integral:
\[
\OO_\kappa(a) = \sum_ {x\mapsto a} \int_{G/I_x} \langle\kappa,x\rangle 1_{\g(\O)} (\op{Ad}\, g(x)) dg,
\]
where $I_x$ centralizes $x$, the sum runs over the finite number of preimages of $a$ in $\g(F)$,
and  $\langle\kappa,x\rangle$
denotes the pairing of $\kappa$ with the cohomological invariant of $x$ classifying its stable conjugacy class. 
Here $1_{\g(\O)}$ is the characteristic function of $\g(\O)$.
When $\kappa$ is trivial, we write $\SO$ for $\OO_\kappa$.  

The character $\kappa$ determines a reductive group scheme $H$ over $\O$,
according to the construction of Section~XX.
  Let $\c_H$ be the
Chevalley quotient of the Lie algebra of $H$.  It comes equipped with a morphism $\nu:\c_H\to\c$.
We also add a subscript $H$ to indicate
quantities coming from the group $H$.


\begin{theorem}[fundamental lemma]
Assume that the characteristic of $F$ is greater than twice the Coxeter number of $G$.
For all regular semisimple elements $a\in \c(F)$ whose image
$\nu(a)$ is also regular semisimple, the $\kappa$ orbital integral of $\nu(a)$ in $G$ is
equal to the stable orbital integral of $a$ in $H$, up to a power of $q$:
\[
\OO_\kappa(\nu(a)) = q^{r(a)}\SO_H(a).
\]
\end{theorem}
The  natural number $r(a)$ is given in Section~\ref{XX}.  A sketch of NBC's proof of the fundamental
lemma appears in Section~\ref{XX}.


Over the years from the time that Langlands first conjectured the
fundamental lemma until the time that NBC gave its proof, the FL has
been transformed into simpler form.   The statement of the FL appears here in its simplified form.
Section~XX makes a series of comments about Langlands's original form of fundamental
lemma and its reduction to this simple form.  Except for that section, our discussion is
based on this simplified form of the fundamental lemma.  In particular, we assume that
the field $F$ has positive characteristic and that the stable conjugacy classes live in the
Lie algebra rather than the group.


\section{affine Springer fibers}

The  value of the orbital integral
can be interpreted as a coset counting problem. 
The value of the integrand is unchanged if
$g$ is replaced with $k g$, for $k\in G(\O_v)$.  The integral thus
breaks into a discrete sum over cosets of $G(\O_v)$ in $G$ modulo the
group action by $I_x$.  With suitable normalization of measures, each
coset $G(\O_v)g$ contributes $0$ or a root of unity $\langle\kappa,x\rangle$ to
the value of the integral depending on whether $\op{Ad}g\,(a) \in
\g(\O_v)$ (again modulo symmetries $I_x$).   This interpretation 
as a coset counting problem makes the fundamental lemma appear to be
combinatorial.  However, purely combinatorial attempts to prove have FL failed.

Let $\M_v(a,k)$ be the set of cosets that meet the support and give nonzero contribution:
\[
XX.
\]

Kazhan and Lusztig showed that the coset space $G(\O_v)\b G$ is the set of
$k$-points of an inductive limit of schemes called the affine Grassmanian.   
Moreover, $\M_v(a,\bar k)$ is the set of points of a locally finite union
$\M_v(a)$ of projective varieties.  This variety $\M_a$ is called the affine Springer
fiber.  Each irreducible component of $\M_a$ has the same dimension
$\delta_v(a)$, given by a formula conjectured by KL and later proven by Berukaznikov.

Goresky, Kottwitz, and MacPherson made an extensive investigation of
the affine Springer fibers and conjectured that their equivariant
cohomology groups are pure.  Assuming this conjecture, they prove the
fundamental lemma for elements whose centralizer is an unramified
Cartan subgroup.  They prove the purity result in particular cases by
constructing affine pavings of the Springer fibers.

Laumon has made a systematic investigation of
the affine Springer fibers for unitary groups.  NBC joined the effort, and together they
succeeded in giving a complete proof of the FL for unitary groups.

NBC encounterd two major obstacles in trying to generalize the work of
Goresky, Kottwitz, and MacPherson to an arbitrary reductive group.
Their approach computes computes the cohomology by computing the fixed
point set in $\M_v(a)$ under the action of a torus on $\M_v(a)$.  Laumon and
Ng\^o's proof of the fundamental lemma for unitary groups also relies
the action of a torus on the affine Springer fibers.  (In the case of
unitary groups, over a quadratic extension, each
endoscopic group becomes isomorphic to a Levi subgroup of $GL(n)$.
The torus action comes from the center of this Levi.)  However, in
general, a nontrivial torus action on the affine Springer fiber simply
doesn't exist.

The second serious obstacle comes from the purity conjecture itself.
In accordance with Deligne's work, NBC believed that the task of proving
purity results should become much easier when the affine Springer fibers are
combined into families rather than treated in isolation.  With this in
mind, he started to investigate families of affine Springer fibers,
varying over a base curve $X$.  This brings us back from local
geometry over the field $F$ to the global geometry of $X$.  What he
found was that a family of affine Springer fibers can be interpreted as 
a fiber in the Hitchin fibration.  He was able to use Deligne's purity
theorem in this setting.  The Hitchin fibers will be described in the
next section.  But before moving on, we need a few more facts about
affine Springer fibers.




\section{Hitchin fibration}


Fix a smooth projective curve $X$ of genus $g$ over $k$.    We allow the constructions in Lie theory from
the previous section to vary over the base curve $X$.  Now let
 $G$ be a reductive over $X$ that is locally split in the etale topology on $X$.
Let $\g$ be its Lie algebra.  Let $D$ be a divisor on $G$.




\section{mass formulas}

\subsection{groupoid cardinality or mass of a category}

Let ${\C}$ be a groupoid (that is, a category in which every arrow is
invertible) which has only finitely many objects up to isomorphism and
in which every object has a finite automorphism group.  Define the
mass (or groupoid cardinality) of $\C$  to be the rational
number
\[
\mu(\C)= \sum_{x\in \op{obj}(\C)/\sim} \frac{1}{\op{card}(\op{Aut}(x))}.
\]

\begin{example}
For example, if ${\C}$ is the category whose objects are the elements of a finite group $G$
and arrows are given by $x \mapsto g x g^{-1}$, for $g\in G$.  Then the set of objects up to
isomorphism is in bijection with the set of conjugacy classes, the automorphism group of $x$ is the
centralizer of $x$,  and the mass is
\[
\mu(\C) = \sum_{x/\sim} \frac{1}{\op{card}(C_G(x))} = 
\sum_{x/\sim} \frac{\op{card}(O_x)}{\card{G}} = 1.
\]
\end{example}

\begin{example}
The following less trivial example appears in NBC.  Let $P$ be the
group $\ring{G}_m\times \ring{Z}$ defined over a finite field $k$ of cardinality $q$.
Let $M = (\ring{P}^1\times\ring{Z})/\sim$, where the equivalence
relation identifies the point $(\infty,j)$ with $(0,j+1)$   for all $j$.
Thus, $M$ is an infinite string of projective lines, with the point at infinity of each line joined to the
zero point of the next line.  The group $P$ acts on $M$ by $(p_0,i)\cdot
(m_0,j) = (p_0 m_0,i+j)$, where $p_0m_0$ is given by the standard action of
$\ring{G}_m$ on $\ring{P}^1$ fixing $0$ and $\infty$.  
Let $\sigma$ be the Frobenius automorphism
of $\bar k$, and define an twisted automorphism of $P(\bar k)$ and $M(\bar k)$
by $\sigma(x_0,i) = (\sigma x_0^{-1},-i)$.  Define a category $\mathcal C$ with
objects given by pairs 
\begin{equation}
(m,p)\in M(\bar k)\times P(\bar k) \text{ such that } \sigma(m) = p
m.
\end{equation}  
Define  arrows $h\in P(\bar k)$ 
\begin{equation}
h(m,p) = (m',p'),    \text{ provided } hm = m' \text{ and } h p = p'\sigma(h).
\end{equation}  
Then it can be checked by a simple
calculation that there are two isomorphism classes of objects in this
category, representated by the objects
\[
((0,0),(1,1))\text{ and }  ((1,0),(1,0))\in M(\bar k)\times P(\bar k) = 
(\ring{P}^1\times\ring{Z}) \times (\ring{G}_m\times\ring{Z}).
\]
The group $P(\bar k)^\sigma$ of order $q+1$ acts as automorphisms of the first object,
and the group of automorphisms of the second object is trivial.  The mass of the category
is therefore
\[
\mu(\C) = \frac{1}{q+1} + 1.
\]
\end{example}

More generally, suppose there exists a function $\iota:\op{Obj}(\C)\to  A$ from the objects
of a groupoid into in a finite abelian group $A$ that
depends only on the isomorphism class of an object $x$.  Then for every
character $\kappa$ of $A$, we can define a $\kappa$-mass:
\[
\mu_\kappa(\C)= \sum_{x\in \op{obj}(\C)/\sim} \frac{\kappa(\iota(x))}{\op{card}(\op{Aut}(x))}.
\]

\begin{example}
For example, in the previous example, if $(m,p)$ is an object and $p=(p_0,j)\in \ring{G}_m\times\ring{Z}$,
then the image of $j$ in $A=\ring{Z}/2\ring{Z}$ depends only on the isomorphism class of the
object $(m,p)$.  If $\kappa$ is the nontrivial character of $A$, then the 
$\kappa$-mass of this groupoid is
\[
\mu_\kappa(\C) = -\frac{1}{q+1} + 1.
\]
\end{example}

%\begin{example}  Let $G$ be a reductive group over $F$ and $T$ a Cartan subgroup of $G$.
%Let the set of objects be the set of cosets of $T$ in $G$ that are defined over $F$ in the sense
%that $\sigma(Tg) = Tg$ for all $\sigma\in \op{Gal(\bar F/F)$.  Let arrows $h(Tg) = Tg'$ be
%given by elements $h\in G(F)$.
%\end{example}


\subsection{mass formula for orbital integrals}

The description of orbital integrals in terms of affine Springer
fibers takes the following form in NBC.  It is a variant of KGM, KL,
expressed in a stacky language.  We note that all geometry in NBC is
carried out in the language of stacks, as developed in
\cite{Laumon-MB}.  In regards to stacks, he makes no compromise.  In
particular, the collection of points of a stack form a groupoid (the
set of points together with isomorphisms between points marked with
arrows).

Let $\M_v(a)$ be the affine Springer fiber for the element $a$ and let
$J_a$ be its centralizer.  We write $\P_v(J_a)$ for the group of
symmetries of the affine Springer fiber.  (The notation for this group
will be explained below.)  Let $\C$ be the groupoid of $k$-points of
the quotient $[\M_v(a)/\P_v(J_a)]$ with objects $(m,p)$ and morphisms
and $h$ defined by the same formula as in Example XX.

For each character of $H^1(k,\P_v(J_a))$ we can naturally associate a
character $\kappa$ of $H^1(F_v,J_a)$ as well as a character (also
called $\kappa$) on a finite abelian group $A$ as above.

\begin{theorem}
The $\kappa$-mass of the category $\C$ is equal to the
$\kappa$-orbital integral of $a$:
\[
\mu_\kappa(\C) = c\,\, \OO_\kappa(a),
\]
up to a constant $c=\op{vol}(J^0_a(\O_v),dt_v)$ used to normalize measures.
\end{theorem}
% Prop 8.2.7.

%Because of this result, I will use the terms ``$\kappa$-mass'' and
%``$\kappa$-orbital integral'' more or less interchangeably.  

Let us write $\mu_{\kappa,v}(a)$ for this mass and $\mu_{H,v}(a_H)$
for the mass of the affine Springer fiber of $M_{H,v}(a_H)$.



\subsection{product formula for masses} % 7 pages.

Let $\A_H$ be the base of the Hitchin fibration for an endoscopic group $H$.  There
is a closed immersion $\nu:\A_H\to\A$.

XX Give $\kappa$ and $J'_a$, etc. $a_H$, $a$. $\tA$ etc.

Let $\mu_\kappa(a)$ be defined as the $\kappa$-mass of the groupoid of
$k$ points of the quotient $[\M_a/\P(J_a')]$ of the Hitchin fiber by
its Picard stack $\P_a'$ of symmetries.  This depends on the usual
data $G,X,D$.  On the endoscopic side, let $\mu_H(a_H)$ be the mass of
the of the groupoid of $k$-points of the quotient
$[\M_H(a)/\P(J'_a)]$.

\begin{theorem}  The masses satisfy a product formula over all places of $X$ in terms
of the masses of the individual affine Springer fibers:
\[
\mu_\kappa(a) =\prod_v \mu_{\kappa,v}(a), \quad \mu_H(a_H) = \prod_v \mu_{H,v}(a_H).
\]
The local factors are $1$ for almost all $v$ so that the products are in fact finite.
\end{theorem}

\begin{proof}[proof sketch]
This is a geometric version of the factorization of $\kappa$-orbital
integrals over the adele group into a product of local
$\kappa$-orbital integrals in \cite{Debuts}.

The proof choses an open set of $X$ over which $J'_a$ is isomorphic to
$J_a$. For a given $a$, at a possibly smaller open set $U$ of $X$, the
action of $\P_a$ on $\M_a$ induces an isomorphism of $\P_a(J_a)$ with
$\M_a$.  The product in the lemma can be taken as extending over the
finite set of places $X\setminus U$.  The lemma is based on a product
formula for stacks
\[
[\M_a/\P_a(J_a)] = \prod_{X\setminus U} [\M_{v,a}/\P_{v,a}].
\]
and a similar formula for $H$.
\end{proof}

\subsection{global mass formula}

The following is the key global ingredient of the proof of the fundamental lemma.

\begin{theorem}[Global Mass Formula]\label{lemma:gmf}
Assume $\deg(D)>2g$, where $g$ is the genus of $X$.  
Then for all $a\in \tA_H^\good(k)$ with image $\nu(a)$ in $\tA$, the following mass formula
holds:
\[
\mu_\kappa(\nu(a)) = q^{r(a)} \mu_H(a).
\]
\end{theorem}

\begin{proof}[proof sketch]
The proof first defines a particularly nice  open set $\tU$ of $\tA_H^{\good}$.
The idea is to impose as many conditions as possible on $\tU$ to make it as
nice as possible, without imposing so many conditions that it fails to be open.
There exists an open set $\tU$ of the good locus on which all of the following conditions hold:
\begin{itemize}
\item It is a subset of the anisotropic locus.
\item For each $n$, the restriction to $\tU$ of the perverse cohomology sheaves 
    $\tilde\nu^*\, {}^p\!H^n(\tilde f^\a_*\bar {\ring{Q}}_\ell)_\kappa$ and ${}^p\!H^{n+2r}(\tilde f_{H,*}\bar{\ring{Q}}_\ell)_{st}(-r)$ are pure local systems of weight $n$.
\item Each $a_H\in \tU(\bar k)$ cuts the divisor $\D_{H,D}+\R^G_{H,D}$ transversally.
\end{itemize}
NBC's support theorem is used in the proof that the condition on the
perverse cohomology sheaves can be satisfied on an open subset of
$\tA_H^\good$.

After choosing $\tU$, the proof of the lemma establishes the global
mass formula on $\tU$, then extends it to all of $\tA_H^{\good}$.  By
imposing such nice conditions on $\tU$, NBC is able to prove the mass
formula on this subset by explict local calculations.  By the
transversality condition on $a$, at any given place $v$, the local
intersection multiplicities $(d_{H,v}(a_H),r_{v}(a))$ must be $(0,0)$,
$(1,0)$, or $(0,1)$.  From Bezrukanikov's dimension formula~\ref{XX},
the dimension of the endoscopic affine Springer fiber $\M_{H,v}(a)$ is
$0$.  In fact, the symmetry group $\P_v(J_{a})$ acts simply
transitively on the affine Springer fiber, and the mass of the
groupoid of $k$-points is $1$.

It is therefore enough to show that the $\kappa$-mass of $\nu(a)$ is also $1$.
The transversality condition also determines the
possibilities for the local intersection multiplicities of $\nu(a)$ in $G$.
The affine Springer fiber in this case is at most one and the
$\kappa$-masses of the groupoids of $[\M_v(a)/\P_v(J_a)]$ can be
computed directly.  In fact, Example XX is a typical example of the
computations involved.  

The result of these calculations is that for every point
$a$ in $\tA_H$, with image $\nu(a)\in \tA$, a local mass formula holds:
\[
\mu_{\kappa ,v}(\nu(a)) = q^{\deg(v) r_v(a)} \mu_{H,v} (a), \quad \text{ for all places } v.
\]
The exponents satisfy
\[
r(a) = \sum_v \deg(v) r_v(a).
\]
These two identities, together with the product formula for the global mass, give
the lemma for elements $a$ of $\tU$.  

The extension from $\tU$ to all of $\A_H^{\good}$ is a
global argument.  Using the Grothendieck-Lefschetz trace formula, this
identity of global masses over $\tU$ can be expressed as an identity of alternating sums
of trace of Frobenius on local systems.  These calculations can be
repeated over all finite extensions $k'/k$ and as we vary $k'$,
Cebotarev implies that the semisimplifications of the local systems
are isomorphic on $G$ and $H$.

By the NBC support theorem and BBDG, this isomorphism of local systems
on $\tU$ extends to an isomorphism (of their intermediate extensions)
over all of $\A_H^{\good}$.  This isomorphism of perverse cohomology
sheaves, again by Grothendieck-Lefschetz, translates back into
counting points on the Hitchin fibration, and hence the global mass
formula.
\end{proof}


\subsection{local mass formula and the fundamental lemma}

Let $k$ be a finite field with $q$ elements, let $\O_w$ be the ring of formal Laurent series over $k$
and $F_w$ its field of fractions.  Let $G_w$ be a reductive group scheme over $\O_w$.  Let $\kappa,\ldots$
be endoscopic data defining the endoscopic group $H_w$.  
Let $a \in \c_H(F)$ and $\nu(a)$ be its image in $\c(F)$.  Assume that $\nu(a)$ is regular semisimple

Let $J_w$ be a group scheme over $\O_w$ that has a connected special fiber and that is 
isomorphic over $F$ to the centralizer $J_a$.

Let $r_w(a)\in\ring{N}$ be the local invariant of Section~\ref{XX}.

Assume that the characteristic of $k$ is large (larger than twice the
Coxeter number of $G$).  By
standard descent arguments (see Section~\ref{XX}), we also assume
without loss of generality that the center of $H_w$ does not contain a
split torus.


\begin{theorem}[Local Mass Formula]\label{lemma:lmf}
  The following local mass formula holds for general anisotropic
  affine Springer fibers (both masses being computed with respect to
  the same symmetry group $\P(J_w)$ acting on the fibers):
\[
\mu_{\kappa,w}(\nu(a)) = q^{\deg(v) r_w(a)}\mu_{H,w}(a).
\]
\end{theorem}

\begin{corollary}[Fundamental Lemma]\label{lemma:fl}
$$\OO_\kappa(\nu(a)) = q^{r_w(a)}\SO_H(a).$$
\end{corollary}

The corollary follows from the theorem by Lemma~\ref{lemma:XX}, which expresses orbital integrals
as local masses.

\begin{proof}[proof sketch]
The proof of the FL is a global argument based on the global mass formula
on $\A_H^{\good}$ (Theorem~\ref{lemma:gmf}).   We can use standard
strategies to embed the local setting into a global setting; that is, to put the affine
Springer fiber into a Hitchin fiber.  NBC checks
that we pick a curve $X$, such that a completion of the function field
at some place $w$ is isomorphic to $F_w$.  We choose a global
endoscopic groups $H$ of a reductive group $G$, a divisor $D$ on $X$,
a global element\footnote{More accurately, NBC shows that a suitable element $a'$ exists over every
sufficiently large finite field extension $k'/k$.  He makes the global arguments over the extensions
$k'$ and uses a Frobenius eigenvalue argument at the end to go back to $k$.}  
$a'$ in the Hitchin base $\A_H$ of $H$.  If the
degree of $D$ is sufficiently large, then $a'$ lies in
$\A_H^{\good}$.  The element $a'$ and its image in $\nu(a')\in\A$ are
chosen to approximate the given local elements $a$ and $\nu(a)$ so
closely that at $w$ their affine Springer fibers together with Picard symmetries
are unaffected.

The global mass formula for $a'$ asserts:
\[
\mu_\kappa(\nu(a')) = q^{r(a')} \mu_H(a').
\]
 By the product formula,
each global mass is a product of local masses:  
\[
\mu_\kappa(\nu(a')) = \prod_v \mu_{\kappa,v}(\nu(a')),\quad
\mu_H(a') = \prod_{H,v}\mu_{v}(a').
\]

The global data is chosen so that at every
place $v$ other than the chosen place $w$, the transversality conditions hold, so that
the calculations of the previous section give the local mass formula at $v$:
\[
\mu_{\kappa,v}(\nu(a')) = q^{r_v}\mu_{H,v}(a'),\quad v\ne w.
\]  
These masses are nonzero and  can be canceled from
the product formula.  What remains is a single term on each side of the product formula:
\[
\mu_{\kappa,w}(\nu(a')) = q^{r_v}\mu_{w}(a').
\]
Since, $a'$ was chosen as a  close approximation of $a$ at the place $w$, we also have
\[
\mu_{\kappa,w}(\nu(a)) = q^{r_v}\mu_{w}(a).
\]
This is the desired local mass formula at the place $w$.
\end{proof}

After completing the proof of the fundamental lemma,
in one final zigzag between the local and global, NBC extends the global mass formula
from $\tA_H^{\good}$ to all of $\tA^{\a}$.  This proof is similar to the local calculations
used to establish the global mass formula on $\tU$, but relies on the  added local
force of the Lemma~\ref{XX}.


\section{uses of the fundamental lemma}  
% 6 pages.

The name ``lemma'' is also misleading because it went decades
without a proof, and its depth goes far beyond what would ordinarily be
called a lemma.  

Yet the name FL is apt both because it is fundamental and
because it is expected to be used widely as an intermediate result in
proving many different results.  This section mentions some major
theorems that have been proved recently that contain the FL as an
intermediate result.  In each case, the FL appears to be
an unavoidable ingredient.

The fundamental lemma appears as a specific collection of identities
that are needed to stabilize the Arthur-Selberg trace formula.  By
stabilization, of the trace formula we mean that the terms on the
geometric side of the trace formula that associated with a given
stable conjugacy class in a reductive group are gathered together.
These terms are rearranged into $\kappa$-orbital integrals, which are
then transferred to stable orbital integrals on the endoscopic groups
by the FL.  Applications of the FL come through the stabilized
Arthur-Selberg trace formula.  Since Dat's recent Bourbaki talk
contains a detailed description of the FL in relation to the trace
formula, I will not repeat it here.

Before going into recent uses of the FL, we might also mention various
special cases of the FL have been known for years.  These special
cases of the FL already give abundant evidence of the usefulness of
the lemma.  For example, Langlands proves the fundamental lemma for
stable base change for $GL(2)$ in his book \cite[Lemma~5.10]{BC}.
From there, base change enters into the proof of the tetrahedral and
octahedral cases of the Artin conjecture (Langlands-Tunnell theorem),
which in turn is used by Wiles in the proof of Fermat's Last theorem.
Waldspurger's proof of the FL for $SL(n)$ is taken up by HH in their
proof of automorphic induction for $GL(n)$, which becomes part of the
proof of the proof of the local Langlands correspondence for $GL(n)$
in Harris and Taylor.  

For Langlands, Shimura varieties provided much of the early motivation
for endoscopy and the fundamental lemma~\cite{}.  When writing the 
Hasse-Weil zeta function of Shimura varieties as a product of
automorphic $L$-functions, the formula involves the $L$-functions
associated with endoscopic groups $H$ as well as those of $G$.  This
can be most clearly through a comparison of the stable trace formula (using the FL)
with the Grothendick-Lefschetz trace formula of the Hasse-Weil zeta
function.
% On the Zeta-Functions of Some Simple Shimura Varieties.
% Appeared in Can. J. Math., Vol. XXXI, No. 6, 1979, pp. 1121--1216. Included by permission of Canadian Mathematical Society.
% http://publications.ias.edu/sites/default/files/simpshim-ps.pdf
An early application of the fundamental lemma carries this out for
Picard modular
varieties.~\cite{Pic}.  From there, the FL becomes relevant to the theory of Galois representations
through the representations associated with Shimura varieties.

Let's turn to more recent uses of the FL.  
For most applications to
date, the FL for unitary groups is used as well as the twisted FL
between $GL(n)$ and unitary groups.  Applications of the trace formula
to Shimura varieties often rely on a base change FL.  This base change FL arises
because of the description of that Kottwitz gives of points on (XX)
Shimura varieties in terms of twisted orbital
integrals~\cite{Kottwitz}.



% Kottwitz R.: Shimura varieties and $\lambda$-adic representations, in Automor- 
% phic forms, Shimura varieties, and L-functions, Vol. I 161--209, Perspect. Math., 10, Academic Press, Boston, MA, 1990. 


\subsection{Sato-Tate conjecture}

The original proof by Clozel, Harris, Sheppard-Baron and Taylor
of the Sato-Tate conjecture for elliptic curves over
$\ring{Q}$ was restricted to elliptic curves with non-integral
$j$-invariants~\cite{XX}.
With the advent of the general FL is has become possible to remove
the non-integrality assumption and to greatly extend the theorem, in particular 
to all elliptic modular forms of positive weight \cite{Barnet-Lamb}, etc.


% Carayol's Bourbaki:
% http://www.mathematik.hu-berlin.de/gradkoll/Carayol_Exp.977.H.C4.pdf
% Harris's survey
%http://www.math.unipd.it/~algant/IHP_SMF.pdf

\subsection{Shimura varieties}

Shim and Morel use the FL in their recent work on the cohomology of Shimura varieties and associated
Galois representations.

\subsection{Iwasawa main conjecture for $GL(2)$}

Skinner and Urban have proved the Iwasawa-Greenberg main conjecture for
many modular forms and in particular for the newforms associated with
many elliptic curves over $\ring{Q}$~\cite{SU}.  Their work ultimately
relies on the work of Shin and Morel and on the FL to prove the
existence of certain Galois representations.  It uses the FL for unitary
groups, the twisted FL relating $GL(n)$ to unitary groups, and some
base change identitites.

\subsection{average ranks of elliptic curves}

Last year, Bhargava and Shankar proved~\cite{BS} that
when elliptic curves $E$ over $\ring{Q}$ are ordered
  by height, a positive fraction of them satisfy the Birch and
  Swinnerton-Dyer conjecture.  Specifically, a positive fraction of them
  have rank $0$ and analytic rank $0$.


First they construct a set (of positive
density) of elliptic curves with rank $0$.  Second, they construct a
subset (again of positive density) of the rank $0$ set, consisting of
elliptic curves with analytic rank $0$.  This second step relies on
conditions in Skinner and Urban for the analytic rank to be zero, and hence
indirectly on the FL~\cite[Theorem~2]{SU}.

\subsection{local Langlands for unitary groups}

Moeglin classifies the discrete series representations of unitary
groups over a nonarchimedean local field.  Again, this relies on the
FL for unitary groups and twisted FL for $GL(n)$.

% Classification et Changement de base pour les  s\'eries discr\`etes des groupes unitaires p-adiques, Pacific Journal of Math, 233-1, Novembre, 2007, pp 159-204

\subsection{classification of automorphic representations of classical groups}

Arthur

\subsection{potential automorphy}

Taylor, Harris.






\section{reductions}

Langlands first expressed the fundamental lemma in these words:
``Mais m\^eme apr\`es avoir
v\'erifi\'e que les facteurs de
transfert existent, il reste \`a v\'erifier ce que j'appelle le
lemme fondamental, qui affirme que pour des $G$, $H$ et $\phi_H$
non-ramifi\'es, on a $f\mapsto c\, \phi_H^*(f)$ [pour toute fonction $f\in {\mathcal H}_G$].''
 -- p.49 (Les deb.)  \cite[p.49]{Debuts}

In Langlands's notation $\phi_H^*$ is the homomorphism given by the
Satake transform, from the spherical Hecke algebra ${\mathcal H}_G$ on
$G$ to the spherical Hecke algebra on $H$.  The arrow $f\mapsto
c\,\phi_H^*(f)$ is his assertion of that for every strongly $G$-regular element $\gamma$
in $H$, the transfer (via transfer factors) of each $\kappa$-orbital integral
of a spherical function $f$ on $G$ (over a stable conjugacy class in $G$ matching $\gamma$) is
equal to the stable orbital integral of $\phi_H^*(f)$ on the stable conjugacy class of $\gamma$ in $H$.

This final section describes some theorems related to the fundamental
lemma that simplify it from the form in which it was initially
conjectured, to the final form in which it was proved by NBC.
Waldspurger's work has been particularly significant in transforming the conjecture into a 
friendlier form.  In Langlands's
Paris lectures, the transfer factors were still a matter of conjecture.
The first simplification was to define the transfer factors explicitly.  
This was given in \cite{LS}.


%Twisted Fundamental Lemma, Lie algebra, Weighted Fundamental Lemma,
%Reduction to Units, Smooth transfer, descent, transfer to char 0.

\subsection{descent to the Lie algebra}

A lemma of Harish-Chandra's asserts the transfer of an orbital
integral on $G$ near a singular semisimple element $z\gamma_0$, with
$z$ central, to an orbital integral on $C_G(\gamma_0)$, the
centralizer of $\gamma_0$.  This is called the descent of orbital
integrals.  Langlands and Shelstad made hard calculations in
Galois cohomology to prove that the transfer factors are compatible
with Harish-Chandra's descent of orbital integrals.  The point of
their calculations was to make it possible to prove results about
orbital integrals by induction on the dimension of the group, and
thereby reduce smooth transfer and the fundamental lemma to identities
in a neighborhood of $\gamma_0=1$.  In a neighborhood of $\gamma_0=1$,
identities can be pushed to the Lie algebra, using the exponential map.

The original fundamental lemma has been supplemented by a twisted
fundamental lemma, conjectured by Kottwitz and Shelstad, where the
data is twisted by a nontrivial outer automorphism $\theta$ of the
group $G$.  In the untwisted case, the centralizer of an element fails
to give a group of smaller dimension precisely when the element is
central.  By contrast, a twisted centralizer (with respect to a
nontrivial outer automorphism) always has dimension less than $G$.  As
a consequence, descent always {\it untwists} the twisted fundamental
lemma into identities of ordinary orbital integrals.  If the
(standard) fundamental lemma is then applied, each $\kappa$-orbital
integral can be replaced with a stable orbital integral.  By combining both
descent and stabilization, the twisted FL of Kottwitz and Shelstad
takes the form of identities of stable orbital integrals on the Lie
algebra (from which the automorphism and the character $\kappa$ have
entirely vanished).  The corresponding long calculations in Galois
cohomology that establish descent properties of the transfer factors
for the twisted fundamental lemma have been carried out by
Waldspurger \cite{twisted-w}.  NBC proves the general twisted fundamental lemma in
its untwisted stable form on the Lie algebra.



\subsection{Hecke algebras}

%The fundamental lemma as originally formulated by Langlands
%is a statement about the transfer of all
%functions in the spherical Hecke algebra. 
 A global argument based on
the trace formula shows that the fundamental lemma holds for the full
Hecke algebra for an arbitrary nonarchimedean local field of
characteristic zero, provided it holds for the unit element of the
Hecke algebra for local fields of sufficiently large residual
characteristic (and for groups of smaller dimension).  
The idea of the proof is to choose suitable global
functions for which the comparison of stable trace formulas yields an
obstruction to the fundamental lemma.  This obstruction, which comes from the
spectral side of the trace formula, takes the form
of a set of linear functionals  
\[
L:{\mathcal H}\to\ring{C},\quad L(f) = \sum_\pi a(\pi) \op{trace}\,\pi(f),
\]
on the local spherical Hecke
algebra ${\mathcal H}$ of the reductive group $G$, each given by a finite sum over
irreducible admissible representations with an Iwahori fixed vector.
By purely local arguments, it
can be shown that no nonzero linear map $L$ exists of the form
prescribed by the global theory.  
Because the obstructions $L$ are zero,
the fundamental lemma can be shown to hold on the full spherical Hecke algebra.


\subsection{smooth transfer}

Langlands's book on the stabilization of the trace formula contains
two separate conjectures.  The first is the transfer of smooth
functions.  The second is the fundamental lemma.  An important result
of Waldspurger links the two conjectures, by proving that the
fundamental lemma implies the transfer of smooth functions.  
His key local lemma shows how to obtain simultaneous control over the
orbital integrals of test functions $f$ and the orbital integrals of
their Fourier transform $\hat f$~\cite[Prop.~8.2]{W}.  
In view of the uncertainty principle, it is a remarkable feat to
control both $f$ and $\hat f$ as he does.
% push up against the limits imposed by the
%uncertainty principle.
%is generally deemed nearly impossible by the
% Prop 8.2, p198.
His proof is a global argument, based on a stable Poisson summation
trace formula over the ring of adeles.  The key local lemma allows
Waldspurger to pick global test functions for which the comparison of
trace formulas asserts a local identity: the Fourier transform of a
semisimple $\kappa$-orbit on $G$ equals the Fourier transform of the
corresponding stable orbit on $H$.  By a purely local argument, this
identity of Fourier transforms of orbits implies smooth transfer.

\subsection{weighted orbital integrals}

Langlands's book is called ``Les d\'ebuts'' for a reason.  In the
stabilization of the trace formula, he only stabilizes the terms that
come from regular elliptic conjugacy classes.  This is insufficient
for general applications of the trace formula.  Kottwitz extended the
analysis to singular elliptic conjugacy classes.  Arthur has
completed the full stabilization without restrictions on the conjugacy
classes.  The non-elliptic conjugacy classes lead to significant
complications.  Arthur truncates the trace formula to obtain the convergence
of the non-elliptic terms.  Because of truncation, the non-elliptic
terms bear ``weights,'' non-invariant factors that appears in the
integrand of orbital integrals.  Arthur conjectured a weighted
fundamental lemma needed for stabilization of the non-elliptic
terms.  Chaudouard and Laumon have used the
Hitchin fibration to prove Arthur's weighted fundamental lemma.


\subsection{transfer to characteristic zero}

The fundamental lemma for nonarchimedean local fields in
characteristic zero can be deduced from the fundamental lemma in
positive characteristic [Wald], [CHL].  Cluckers and Loeser develop a
general abstract theory of integration as a combination of primitive
operations such as taking the volume of a ball of given radius,
enumerating points on a variety over the residue field, summing an
infinite $q$-series, and making a change of variables.  Since
each of the primitive operations manifestly depends only on the
residue field rather than the field itself, their theory allows many
identities of integrals to be transfered from one field to another
with the same residue field.  The FL lemma and its weighted and
twisted variants are identities that fall within the scope of this
theory.

\section{literature}

I recommend Ben-Zvi's lecture for mathematicians in other fields who
want a one hour presentation of the big ideas of the proof.
Drinfeld's lecture notes contain many worked examples and exercises
that are helpful in learning the basic ideas geometric of the Hitchin
fibration.  Nadler's survey is for the non-specialist who wants to
learn what the general idea of the stable trace formula, the statement
of the fundamental lemma, and the main ideas of the proof.  My lecture
at a summer school for graduate students at the Field's Institute
gives a statement of the fundamental lemma, including necessary
background material, without going into the proof.  See also
\cite{Hales}.  Casselman describes the fundamental lemma for $SL_2$ in
detail, motivating it with examples in representation theory and
history going back to Hecke.  Arthur's laudation describes NBC's work
in the context of the trace formula.  Langlands's review describes the
origins of the fundamental lemma in the theory of Shimura varieties
and the harmonic analysis over real groups.  His review speculates
that ``an exposition [of the fundamental lemma] genuinely accessible
not alone to someone of my generation, but to mathematicians of all
ages eager to contribute to the arithmetic theory of automorphic
representations, would be, perhaps, \ldots close to 700 pages.''


Much more is available in French, starting with NBC's primary article,
which is superb, both as mathematics and as exposition!  Although they
can be read independently, the ideas develop more naturally if it is
read with his earlier paper.  He has several secondary accounts in
English \cite{XX}.  We mention in particular, several articles in the
Book Project \cite{Harris}, particularly the articles \cite{Dat}
and NBC.

While there have been numerous applications that quote the fundamental
lemma as a finished product, Yuan, Chaudouard, and Laumon are
noteworthy in following Ng\^o in using the Hitchin fibration directly to
prove new results in harmonic analysis.

I wish to thank M. Bhargava, M. Harris, NBC, and C. Skinner for comments that
have helped me to prepare this and my earlier report~\cite{XX}.



\end{document}