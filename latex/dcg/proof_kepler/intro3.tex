%% SPIII Intro

This paper is the third in a series of six papers devoted to the
proof of the Kepler conjecture, which asserts that no packing of
congruent balls in three dimensions has density greater than the
face-centered cubic packing.  In the previous paper in this series,
a continuous function $f$ on a compact space is defined, certain
points in the domain are conjectured to give the global maxima, and
the relation between this conjecture and the Kepler conjecture is
established. This paper shows that those points are indeed local
maxima. Various approximations to $f$ are developed, that will be
used in subsequent papers to bound the value of the function $f$.
The function $f$ can be expressed as a sum of terms, indexed by
regions on a unit sphere. Detailed estimates of the terms
corresponding to triangular and quadrilateral regions are developed.

This paper has three objectives. The first is dealing with the two
types of decomposition stars that attain the optimal Kepler
conjecture bound. The second is obtaining general upper bounds on
the score of decomposition star by truncation. The third is
obtaining various upper bounds on the score associated to individual
triangular and quadrilateral regions of a general decomposition
star.

The first section contains a proof that the decomposition stars
attached to the face-centered cubic and hexagonal-close packings
give local maxima to the scoring function on the space of all
decomposition stars.  The proof describes precisely determined
neighborhoods of these critical points.  These special decomposition
stars are shown to yield the global maximum of the scoring function
on these restricted neighborhoods.

The second section gives an approximation to a decomposition star
that provides an upper bound approximation to the scoring function
$\sigma$.  In the simplest cases, the approximation to the
decomposition star is obtained by truncating the decomposition star
at distance $t_0=1.255$  from the origin. More generally, we define
a collection of simplices (that do not overlap any simplices in the
$Q$-system), and define a somewhat different truncation for each
type of simplex in the collection. For want of a more suggestive
term, these simplices are said to form the $\CalS$-system.

When truncation at $t_0$ cuts too deeply, we reclaim a scrap of
volume that lies outside the ball of radius $t_0$ but still inside
the $V$-cell.  This scrap is called a {\it crown}. These scraps are
studied in that same section.

In a final section, we develop a series of bounds on the score
function in triangular and quadrilateral regions, for use in later
papers.
