% Template for an AMS article style
\documentclass{llncs}
\usepackage{graphicx}
\usepackage{amsfonts}
\usepackage{amscd}
\usepackage{amssymb}
\usepackage{alltt}
\usepackage{multicol}
\usepackage{amsmath}


\usepackage{xcolor}
\usepackage{tikz}
\usetikzlibrary{calc,through,chains,shapes,arrows,trees,matrix,positioning,decorations,backgrounds,fit}

% Math notation.
\def\op#1{{\hbox{#1}}}
\newcommand{\ring}[1]{\mathbb{#1}}

\def\tikzfig#1#2#3{%
\begin{figure}[htb]%
  \centering
\begin{tikzpicture}#3
\end{tikzpicture}
  \caption{#2}
  \label{fig:#1}%
\end{figure}%
}

% generate revision number by
% svn propset svn:keywords "LastChangedRevision" thisfile.tex

\def\svninfo{%
  \noindent
  Document TeXed on \today. \hfill\break
  LaTex Repository: https://flyspeck.googlecode.com/svn \hfill\break
  SVN $LastChangedRevision$\hfill\break
  PGF version: $\pgftypesetversion$
  }
\newtheorem{Exercise}[equation]{exercise}

\def\newterm#1{{\it #1}}
\def\t#1{\hbox{}^t#1}
\def\card#1{\op{card}{#1}}

%%%%%%%%%%%%%%%%%%%%%%%%%%%%%%%%%%

\begin{document}

\title{Math 3600: Lectures on the Langlands-Tunnell theorem}
\author{Thomas C. Hales}
\institute{\hbox{ }}
\maketitle

%\begin{abstract}  
%\end{abstract}

%%%%%%%%%%%%%%%%%%%%%%


\parindent=0pt
\parskip=\baselineskip
\def\seg{~~~}
\def\text{\hbox}

These are lectures from a course on the Langlands-Tunnell theorem, during
the Spring 2012 semester in the math department at the University of Pittsburgh.%
\footnote{\svninfo}
Every effort has been made to keep the prerequisites to a minimum.
For background in Algebra, we use Knapp's Basic Algebra \cite{knapp-basic}.
For algebraic number theory and class field theory, 
we use Milne's notes 
\cite{ANT}, \cite{CFT}.  For local Langlands, we use \cite{bushnell-henniart}.
All of these books go to significant lengths to give detailed and self-contained 
proofs.

\section{Fermat's Last Theorem}

This lecture was delivered on Monday, January 4, 2012.

Fermat's Last Theorem \cite{DDT} states the equation
\[
a^n + b^n = c^n,
\]
has no solution in the integers when $n>2$ and $abc\ne 0$.  There are
infinitely many solutions when $n=2$:
\[
3^2 + 4^2 = 5^2,\quad 5^2 + 12^2 = 13^2, \ldots
\]
Fermat himself did not write down a proof, and it eventually became
the most famous problem in the history of mathematics.  There have
been many attempts through the centuries.  Over 1000 false proofs of
the result were published between 1908 and 1912.  It was solved by
A. Wiles, with some assistance from R. Taylor.

If there is a proof for exponent $n$ and if $N = n r$, then
\[
(a^r)^n + (b^r)^n = (c^r)^n,
\]
settles the exponent $N$ as well.  Any exponent $N\ge3$, is divisible
by $4$, divisible by $3$, or divisible by a prime $n\ge 5$.  Fermat
gave a proof when the exponent $n$ is $4$.  Euler gave a proof for
exponent $3$.  The case that remains is a prime exponent $n\ge 5$.  We
assume $n\ge5$ in what follows.

Frey suggested that we argue by contradiction.  Let
$(A,B,C)=(a^n,b^n,c^n)$ be a solution: $A+B=C$, with $ABC\ne
0$. Consider the curve
\[
y^2 = x (x - A) (x + B).
\]
The cubic has distinct roots, exactly when $0$, $-A$ and $B$ are
distinct, which occurs if and only if $abc\ne 0$.  By factoring out
common powers of $2$, we may assume that at least one of $A$, $B$, $C$
is odd.  By considering parity, at least one of them is even.
Reordering, we may assume that $A$ is odd and $B$ is even.  In fact,
$B=b^n$, with $n\ge 5$, is divisible by $32$.


The Frey curve is an elliptic curve.  What was Frey's motivation?  Who
knows! Perhaps it was an act of desperation when nothing else seemed
to work. At least elliptic curves have been studied very extensively.

The set of complex solutions, including a point at infinity is a
topological torus, a product of two circles.  The set of points of
order $3$ is $\ring{F}_3^2$, a two dimensional vector space over the
field with three elements.  Each automorphism of the group is
necessarily linear, hence an invertible linear transformation of
$\ring{F}^2_3$.  All such automorphisms form a group
$GL(2,\ring{F}_3)$.

The group law can be expressed algebraically: there is a collection of
polynomials\footnote{To be entirely correct, we should take the
  polynomials to be homogeneous polynomials in the variables
  $x_i,y_i,z_i$, for $i=1,2,3$, where $[x_i:y_i:z_i]$ gives the
  homogeneous coordinates of the points in projective space.}  with
rational coefficients in six variables such that
$(x_1,y_1),(x_2,y_2),(x_3,y_3)$ is a simultaneous zero of the
polynomials if and only if $(x_1,y_1)$, $(x_2,y_2)$, and $(x_3,y_3)$
are points on the elliptic curve and $(x_3,y_3)$ is the sum of
$(x_1,y_1)$ and $(x_2,y)$.  By convention, the identity element is
taken to be point at infinity.  It is known that the set
\[
E[3]= \{(x,y) \in \ring{C}^2\mid y^2 = x (x-A) (x+B), 
\quad (x,y) \text{ has order } 3\}\cup \{\infty\}
\]
is the set of roots of polynomials with rational coefficients,
together with the point at infinity.  The set of coefficients $(x,y)$
of these nine points lie in a finite Galois extension $K$ of
$\ring{Q}$.

If $\sigma$ is an automorphism of $K/\ring{Q}$, then it permutes the
set $E[3]=\ring{F}_3^2$ and it respects the group law, hence is
expressed as an element of $GL(2,\ring{F}_3)$.  In fact, we obtain a
homomorphism
\[
\phi_1 : \op{Gal}(K/\ring{Q}) \to GL(2,\ring{F}_3).
\]
Part of Wiles's proof shows that we may assume without loss of
generality that this representation is irreducible; that is, there is
no $1$-dimensional vector space that is stable under
$\op{Gal}(K/\ring{Q})$.

The group $GL(2,\ring{F}_q)$ is a finite group of order
$(q^2-1)(q^2-q)$.  When $q=3$, we get a group of order $48$.  For a
moment, view $GL(2,\ring{F}_3)$ as an abstract group.  It is known
to have a $2$-dimensional faithful representation,
\[
\phi_2 : GL(2,\ring{F}_3) \to GL(2,\ring{C}).
\]
The composite of $\phi_1$ and $\phi_2$ gives a homomorphism
\[
\op{Gal}(K/\ring{Q}) \to GL(2,\ring{C}).
\]

It is the aim of this course to understand this homomorphism.  The
Langlands-Tunnell theorem implies that the strong Artin conjecture
holds for this homomorphism.  We will have much more to say about this
later.


\newpage
\section{Finite subgroups of $GL(2,\ring{C})$}

Lecture delivered Friday January 6, 2012.


Basic facts about the representations of finite groups can be reviewed
in \cite[p.~326, Sec. VII.4]{knapp-basic}; Facts about symmetric and
hermitian 
forms are found at \cite[p.~250,~Sec.~VI.2,~VI.4]{knapp-basic}.

If $V$ is a vector space over a field $F$, we write $GL(V)$ (the
general linear group) for the group of invertible linear maps $g:V\to
V$.  If $V=F^n$ for some field $F$, we write $GL(n,F)$ for
$GL(n,F^n)$.  It is identified with the set of $n$ by $n$ matrices
with coefficients in $F$ with nonzero determinant.  More generally, if
$R$ is a commutative ring with $1$, we write $GL(n,R)$ for the group
of $n$ by $n$ matrices with coefficients in $R$ whose determinant is a
unit in $R$.

If $V$ is an $R$ module, where $R$ is a commutative ring with $1$, 
we write $\op{End}_R(V)$ for the ring of $R$-module endomorphisms of $V$.

A \newterm{representation} is a homomorphism $\phi$ from a group $G$
into $GL(V)$ for some vector space $V$.  We write $(\phi,V)$ for the
representation, which we abbreviate at times to $\phi$ or $V$.  
The representation is \newterm{reducible} if there is
a subspace $W\ne 0,V$ of $V$ such that $\phi(g)w\in W$ for all $g\in
G$ and $w\in W$.  It is \newterm{irreducible} if it is not reducible.
The representation is \newterm{faithful} if the kernel of $\phi$ is
$1$.

We let $U(n)$ be the group of $n$ by $n$ unitary matrices:
\[
U(n) = \{ g \in GL(n,\ring{C})  \mid \t{\bar g} g = I\},
\]
where the $t$ marks transpose, $\bar g$ is the coefficient-wise
conjugate of $g$, and $I=I_n$ is the identity matrix.  We let $O(n,F)$
be the group of $n$ by $n$ orthogonal matrices:
\[
O(n,F) = \{ g \in GL(n,F) \mid \t{g} g = I\},
\]
We write $SL(n,F)$, $SL(V)$, $SU(n)$, and $SO(n,F)$ for the subgroups
of $GL(n,F)$, $GL(V)$, $U(n)$, $O(n,F)$ respectively of those linear
transformations with determinant $1$.  Note that if a matrix $g$ is orthogonal,
then $\det(g)^2 = \det(\t{g}g) = \det{I} = I$, so that its determinant is $\pm 1$.

For example $SU(2)$ is given by the set of matrices with complex entries satisfying
\[
\begin{pmatrix}{\bar a} & {\bar c} \\ \bar b & \bar d\end{pmatrix} = 
\begin{pmatrix}d & - b\\ -c & a \end{pmatrix},\quad ad - bc = 1;
\]
or equivalently, complex matrices of the form
\[
\begin{pmatrix}a & b\\ -\bar b & \bar a\end{pmatrix},\quad |a|^2 + |b|^2 = 1.
\]
As a topological space, the set $SU(2)$ is homeomorphic to
 the $3$-sphere 
$$\{(a,b)\in \ring{C}^2\mid |a|^2 + |b|^2 = 1\}.$$


\begin{lemma} If $G$ is any finite subgroup of $GL(2,\ring{C})$, then it is
conjugate to a subgroup of $U(2)$.
\end{lemma}

\begin{proof} Fix any positive definite hermitian form $v,w\mapsto \langle v,w\rangle $ on $\ring{C}^2$;
say 
\[\langle (z_1,z_2),(w_1,w_2)\rangle  = z_1 \bar w_1 + z_2 \bar w_2\]
Let $\langle v,w\rangle_1 = \sum_{g\in G} \langle v,w\rangle$.  This is again a positive definite
hermitian form.    Then $G$ preserves $\langle\cdot,\cdot\rangle_1$ and with respect 
to an orthonormal basis  of $\langle\cdot,\cdot\rangle_1$, the matrices representing $G$ are unitary.
\end{proof}

We define a representation $\phi$ of $U(2)$ in $GL(3,\ring{R})$ by the action of $U(2)$ on the real $3$-dimensional
 vector space $H$
of $2$ by $2$ Hermitian matrices.
\[
H = \{ X \mid \t{\bar X} = X,\quad \op{trace}(X)=0\}, and 
\]
  \[
   \phi(g) X = {g} X g^{-1} \in H, \quad X\in H.
  \]
We claim that the image $\phi(U(2))$ is conjugate to a subgroup of  $SO(3,\ring{R})$.
First we check that the image is conjugate to a subgroup of $O(3,\ring{R})$.  We define an inner product
on $H$ by  $X,Y\mapsto \op{trace}(XY)$.  It is symmetric by a basic property of the trace.  Note that 
\[
\bar {\op{trace}(XY)} = \op{trace}(\bar X \bar Y) = \op{trace}(\t{X}\t{Y}) = \op{trace}(\t{(YX)}) = \op{trace}(XY),
\]
so that the inner product takes real values.
Also, if we take a general element of $H$ of the form
\[
X = \begin{pmatrix} a & b + i c \\ b - i c & a\end{pmatrix},
\]
we compute $\op{trace}(X^2) = 2 (a^2 + b^2 + c^2)$.  This gives that the inner product is positive definite.  We may
pick an orthonormal basis for this inner product.  With respect to this basis, the image $\phi(U(2))$
is represented by orthogonal matrices.

Next we check that the image consists of matrices of determinant $1$.
Recall that the determinant of an orthogonal matrix is $\pm 1$.
The set $\det\phi(SU(2))$ is the continuous image of the $3$-sphere, which is a connected topological space.  The
image contains $1$ (image of the identity).  Hence the image of $SU(2)$ under $\det\phi$ is $1$.  Similarly,
the image of the set of scalar matrices in $U(2)$ is the continuous image of a circle, which is again a connected
topological space.  Since every unitary matrix is a product of a scalar matrix with a special unitary matrix,
the result follows.

\begin{lemma} Every matrix $g\in SO(3,\ring{R})$ has  eigenvalue $1$.
\end{lemma}

\begin{proof}
\[
\det (g-I) = \det(\t{g}-I) = \det(g^{-1}-I) = \det(g^{-1})\det(I - g) = 
\det(-I)\det(g - I) = -\det(g-I).
\]
So $\det(g-I)=0$, and $g-I$ is singular.
\end{proof}

\begin{theorem}
Let $G$ be a finite subgroup of $SO(3,\ring{R})$.  Then $G$ is cyclic, dihedral,
or the group of symmetries of a Platonic solid.
\end{theorem}

There are five Platonic solids: tetrahedron, octahedron, cube, dodecahedron, and icosahedron.
There are only three symmetry groups because the octahedron and cube have the same symmetry group,
and the dodecahedron and icosahedron have the same symmetry group.

\begin{proof}
\claim{If $1\ne g\in G$, then it has a eigenvalue $1$ with multiplicity $1$.}  Let $v$ be an eigenvector
of $g$ with eigenvalue $1$.  Then $g$ is a rotation about the line through the origin and $v$.  The other
two eigenvalues are $exp(\pm i \theta)$, where $\theta$ is the angle of rotation.  Hence another eigenvalue
is one if and only if the angle of rotation is $0$.

We call an eigenvector of $1\ne g\in G$, normalized to have length $1$, a pole of $g$.  let $P$ be the
set of all poles of all $1\ne g\in G$.  
The group $G$ acts on the set $P$: if $v\in P$ is a pole of $g$, then $h v$ is a pole of $h g h^{-1}\in G$.

Let $O_1,\ldots O_k$ be the orbits of $G$ in $P$, and let $r_i=\card(G_v)$, for $v\in O_i$,
where $G_v$ is the stabilizer of $v\in P$.
A basic property
of transitive group actions is that $O_i$ can be identified with the coset space $G/G_v$.
In particular, $\card(O_i) = n/r_i$, where $n=\card(G)$.

The set
\[
\{(g,v)\mid g\ne 1,\quad v \hbox{ is a pole of } g\}
\]
can be counted two ways: either grouping by the first coordinate, or by the second coordinate.
For every first coordinate there are two second coordinates.  Alternatively,
for every orbit $O_i$ and every second coordinate $v\in O_i$, there
are $r_i - 1$ first coordinates.  Counting both ways, we get
\[
2 (n-1) = \sum_{i=1}^k \card{(O_i)}\, (r_i - 1) =  \sum \frac{n}{r_i}(r_i-1),
\]
or after dividing by $n$:
\[
2 (1 - 1/n) = \sum_{i=1}^k (1-1/r_i).
\]
Each $r_i$ is the order of a nontrivial group, so $r_i\ge2$.
This gives
\[
2 > 2 (1-1/n) =\sum_{i=1}^k (1-1/r_i) \ge k/2.
\]
This implies $k\le 3$.  
In the other direction
\[
1 \le 2(1-1/n) = \sum_{i=1}^k (1-1/r_i) < \sum_{i=1}^k 1 \le k.
\]
This implies $2\le k$.

We consider each case $k=2,3$ in turn.

If $k=2$, then the equality can be written
\[
2/n = 1/r_1 + 1/r_2 \ge 1/n + 1/n.
\]
This implies $n=r_1 = r_2$ so that the order of the stabilizer of a
pole $v\in O_1$ is the order of the group.  This forces $G$ to be
cyclic, consisting of rotations about the line through $v$.

If $k=3$, the equality takes the form
\[
1< 1 + 2/n = 1/r_1 + 1/r_2 + 1/r_3.
\]
We make an exhaustive lexicographical search over all triples
$(r_1,r_2,r_3)$ that satisfy
\[
1 < 1/r_1 + 1/r_2 + 1/r_3, \hbox{ and } r_1 \le r_2 \le r_3,
\]
where $r_i \in \{2,3,\ldots\}$.
The only solutions are 
\[
\{ (2,2,r)\mid r\ge 2\} \cup \{(2,3,3),(2,3,4),(2,3,5)\}.
\]
The order $n$ of $G$ in these respective cases is $2r$, $12$, $24$,
and $60$.  From the explicit information about orbits and stabilizers
thus obtained, an easy exercise shows that these cases correspond to
the dihedral, tetrahedral, octahedral, and icosahedral symmetry groups
respectively.  For example, in the first case, the set $O_1$ is the
set of vertices of a regular $r$-gon. In the other cases, the three
orbits $O_1$, $O_2$, $O_3$ are (the projections to the unit sphere of)
the edge midpoints, vertices, and face centers of the Platonic
solid (or its dual).
\end{proof}

The structure of the groups of rotations of the Platonic solids are well-known.
Every symmetry of the tetrahedron acts on its set of four vertices, giving
an embedding of the group into $S_4$.  Elements of $SO(3,\ring{R})$ act as
even permutations, so that the group is isomorphic to $A_4$.

The group of the cube/octahedron acts as permutations of the four lines passing
from vertex to opposite vertex of a cube.  This injects the octahedral group
into $S_4$.  Since they have the same order, this is an isomorphism.

The $20$ vertices of the dodecahedron can be partitioned into $5$ sets of cardinality
four in such a way that each set is the set of vertices of a regular tetrahedron.
The group of the dodecahedron/icosahedron acts as a group of order $60$ 
on this set of five
regular tetrahedra.  This is a subgroup of index $2$ in $S_5$, which uniquely
characterizes  $A_5$.  Thus, the icosahedral group is isomorphic to $A_5$.

We have noted that $GL(2,\ring{F}_3)$ is isomorphic to a subgroup of
$GL(2,\ring{C})$.  Its image in $SO(3,\ring{R})$ is octahedral.  Let
$Z$ be the subgroup (of order $2$) consisting of scalar matrices.  We
show that $GL(2,\ring{F}_3)/Z$ is $S_4$ directly as follows.  Let
$\ring{P}^1(\ring{F}_3)$ be the projective line over the field with
three elements.  Concretely, the projective line over a field $F$ is
given by the set of equivalence classes of
\[
X(F)=F^2 \setminus \{(0,0)\},
\]
under the equivalence relation $(z_1,z_2)\sim (t z_1,t z_2)$, for
$t\in F^\times$.  When $F=\ring{F}_q$, the cardinality of $X(F)$ is
$q^2-1$, and each equivalence class has $q-1$ elements, so that
$\ring{P}^1(\ring{F}_q)$ has cardinality $(q^2-1)/(q-1)=(q+1)$.  The
group $GL(2,\ring{F}_q)$ acts on $\ring{P}^1(\ring{F}_q)$ through the
usual matrix multiplication on $X(F)$.  When $q=3$, the projective
line has cardinality $4$, the subgroup $Z$ acts trivially, and this
action gives an isomorphism of $GL(2,\ring{F}_3)/Z$ with $S_4$.


\newpage
\section{Cyclic and Dihedral subgroups}

Lecture Wednesday, January 11, 2012.


Let $G$ be a finite group and $V$ a vector space over a field $F$.  
Every representation $(\phi,V)$ of $G$
is a module of the group algebra $F[G]$ by the rule 
\[
[g].v = \phi(g)v, \text{ where } [g]\in F[G].
\]
Conversely, if $V$ is an $F[G]$ module, the same rule read in the opposite
direction defines a representation
\[
\phi:G\to GL(V).
\]
Thus, we may switch back and forth between the language of representation
theory and the language of modules, sometimes referring to $(\phi,V)$ as
a module.

Let $\phi:H\to GL(V)$ be a representation of a subgroup $H$ of a
finite group $G$, where $V$ is a vector space over a field $F$.  There
is a universal way to construct from $(\phi,V)$ a representation $\op{Ind}_H^G\phi$ of
$G$, called induction.\footnote{We write $\op{Ind}_H^G V$ for the vector space on
which it acts, or sometimes $\op{Ind}_H^G (\phi,V)$ for the ensemble.}  
It may be described as a tensor product
\[
\op{Ind}_H^G(V) = F[G]\otimes_{F[H]} V,
\]
where $F[\cdot]$ denotes the group algebra.  The group $G$ acts
naturally on the vector space $F[G]$ and hence on the tensor product.

\begin{exercise}
If $S$ is a set of representatives of $G/H$, then $\op{Ind}_H^G(V)$
has dimension $\card(G/H)\dim_F(V)$, and has basis $[s]\otimes v_i$,
for $s\in S$ and $v_1,\ldots $ a basis of $V$.
\end{exercise}

In categorical terms, induction is the left adjoint of the restriction functor
from the category of  $F[G]$-modules to the category of  $F[H]$-modules.
We leave this as an exercise to the reader.  The adjointness is expressed as a 
natural bijection
\begin{equation}\label{eqn:adjoint}
\op{Hom}_{F[G]}(Ind_H^G V,W) = \op{Hom}_{F[H]}(V,\op{Res}_H W),
\end{equation}
where $\op{Res}_H W$ is the restriction of the $F[G]$-module $W$ to $F[H]$.

Two representations $\phi_1:G\to GL(V)$, $\phi_1:G\to GL(W)$ are
\newterm{equivalent} if $V$ and $W$ are isomorphic as $F[G]$-modules.

\begin{theorem}
  Let $\phi:G\to GL(2,\ring{C})$ be an irreducible $2$-dimensional
  representation of a finite group $G$.  Assume that the image of $G$
  in $SO(3,\ring{R})$ is cyclic or dihedral.  Then exists a subgroup
  $H$ of $G$ of index $2$ and a $1$-dimensional representation
  $\theta$ of $H$, such that $\phi$ is equivalent to
  $\op{Ind}_H^G\theta$.
\end{theorem}

In an earlier lecture, we constructed a homomorphism $U(2)\to
SO(3,\ring{R})$ by having $U(2)$ act on the three dimensional real
vector space $H$ of Hermitian matrices in $\op{End}(\ring{C}^2)$ with
trace zero.  The complexification
$sl(2,\ring{C})=\ring{C}\otimes_{\ring{R}} H$ is the three dimensional
complex vector space of all $2$ by $2$
matrices of trace zero.  Over the complex vector space, the
homomorphism $U(2)\to SO(3,\ring{R})$ extends to a homomorphism
$GL(2,\ring{C})\to GL(3,\ring{C})$.  The action is the same as before
\[
(g,X)\mapsto g X g^{-1},\quad X\in sl(2,\ring{C}).
\]
If $g = \begin{pmatrix} a& b\\ c & d\end{pmatrix}$, then the inverse
of $g$ is given by a well-known formula
\[
g^{-1} = \frac{1}{\det{g}}\begin{pmatrix} d & - b \\ -c & a\end{pmatrix} 
= \det g^{-1} J \t{g} J^{-1}, \quad
\hbox{ where } J = \begin{pmatrix} 0 & -1 \\ 1 &0 \end{pmatrix}
\]
Let $S$ be the $3$-dimensional complex vector space of all symmetric matrices in $\op{End}(\ring{C}^2)$.
We have a linear isomorphism $sl(2,\ring{C})\to S$, given by $X\mapsto X J$:
\[
\begin{pmatrix} a& b\\c & -a\end{pmatrix} \begin{pmatrix} 0& -1 \\1 &0\end{pmatrix} = 
\begin{pmatrix} b & -a \\ -a & -c \end{pmatrix}
\]
If we carry the action of $GL(2,\ring{C})$ on $sl(2,\ring{C})$
over to $S$ by this isomorphism, we get by a simple calculation
\[
(g,A) \mapsto \det(g)^{-1} g A \t{g},\quad A \in S.
\]

We are ready to give a proof of the theorem.

\begin{proof} Identify (a factor group of) $G$ with a subgroup of
  $U(2)$ by means of $\phi$.  A cyclic or dihedral subgroup of
  $SO(3,\ring{R})$ has a line that is stable under the subgroup (that
  is, a common eigenvector). Using our explicit description of the
  three-dimensional vector space as Hermition matrices, the stable
  line corresponds to a non-zero Hermitian matrix $X$ such that
\[
g X g^{-1} = \lambda(g) X,
\]
for all $g\in G$ for some scalar $\lambda(g)$, depending on $g$.
The determinant of a general non-zero hermition matrix of trace $0$ is non-zero:
\[
\det\begin{pmatrix} a & b + i c\\ b - i c & -a\end{pmatrix} = -a^2 - b^2 - c^2 \ne 0.
\]
The symmetric matrix $A = X J$ also has non-zero determinant.  In
terms of the action of $G$ on symmetric matrices, the stable line
condition becomes
\[
g A \t{g} = \lambda(g)\det(g) A.
\]
(All such $g$ form a group of orthogonal similitudes.)  We define an
inner product on row vectors $\ring{C}^2$ by $v,w\mapsto \langle
v,w\rangle = v A \t{w}$.  Since $\det A\ne 0$, it is non-degenerate.
All non-degenerate inner products on $\ring{C}^2$ are equivalent by
basic facts of quadratic forms over $\ring{C}$ (see \cite{knapp-basic}).  Hence there is a
basis of $\ring{C}^2$ such that $A$ takes the form
\[
A_1 = \begin{pmatrix} 0 & 1 \\ 1 & 0\end{pmatrix}
\]
The stable line condition becomes
\begin{equation}
g A_1 \t{g}=\begin{pmatrix} a & b \\ c & d\end{pmatrix} 
\begin{pmatrix} 0 & 1 \\ 1 & 0\end{pmatrix}
\begin{pmatrix} a & b \\ c & d\end{pmatrix} = 
\begin{pmatrix} 2 a b & * \\ * & 2 c d\end{pmatrix} =
\begin{pmatrix} 0 & * \\ * & 0\end{pmatrix} = \lambda(g)\det(g) A_1.
\end{equation}
Hence $ab=0$ and $cd=0$.  This forces $g$ to be diagonal $b=c=0$ or
antidiagonal $a=d=0$.  

Let $N(T)$ be the group 
\[
\{\begin{pmatrix} a & b \\ c & d\end{pmatrix} \in GL(2,\ring{C}) \mid
b = c =0 \quad \text{ or } a = d = 0\}.
\]
The diagonal group $T$ has index two in $N(T)$.
The nontrivial coset is represented by an  antidiagonal matrix.

If $G\subset T$, then $\phi$ is reducible,
contrary to assumption.  Hence the composite,
\[
G\to N(T)\to N(T)/T = \ring{Z}/2\ring{Z}
\]
is onto, and its kernel is a subgroup $H\subset T$ of index $2$ in $G$.
Let
$\theta_i:H\to GL(1,\ring{C})$, for $i=1,2$, be the diagonal entries
of $H$.  Let $w$ be a representative of the nontrivial coset of $H$ in
$G$.  Conjugation of $H$ by $w$ permutes its diagonal entries, so
$\theta_1(w h w^{-1}) = \theta_2(h)$ for $h\in H$.  If
$\theta_1=\theta_2$, then $H$ consists of scalars, and again we find
that $\phi$ is reducible.  Hence $\theta_1\ne\theta_2$.  A short
exercise based on the adjointness property (\ref{eqn:adjoint}) now
shows that $\op{Ind}_H^G\theta_1$ is irreducible and $\phi$ is
equivalent to $\op{Ind}_H^G\theta_1$ via the morphism of modules:
\[
1\in\op{Hom}_{F[H]}(\theta_1,\theta_1)=\op{Hom}_{F[H]}(\theta_1,\theta_1\oplus\theta_2)=
\op{Hom}_{F[H]}(\theta_1,\op{Res}_H\phi) = \op{Hom}_{F[G]}(\op{Ind}_H^G\theta_1,\phi).
\]
\end{proof}

\newpage
\section{Topological Groups}

In this section, we assume basic familiarity with point-set topology,
including connectedness, compactness, and so forth. 

A topological group is a group $G$, together with a topology on that
group, such that the maps
\[
\mu:G\times G\to G,\quad \iota:G\to G,
\]
are continuous, where $\mu$ and $\iota$ are multiplication and
inverse, respectively.  The topology on $G\times G$ is the product
topology.

Warning -- the word {\it closed} now has two meanings, a closed subset
in topology, and in the sense of a set being closed under
multiplication.  If in doubt, we probably intend the topological
meaning.

The notion of topological group is particularly fruitful; we should
consider {\it every} group as a topological group, falling back to the
discrete topology on $G$ if nothing else comes to mind.

(More generally, we may consider topological rings, topological
modules, and so forth, where each is defined as an algebraic
structure, together with a topology, for which all the algebraic
operations are continuous.)

There are many examples of topological groups $GL(n,F)$, $SL(n,F)$,
$SO(n,F)$, vector spaces over $F$, where $F=\ring{R}$ or $\ring{C}$.
Also, $U(n)$, $SU(n)$ are topological groups.  In all of these cases the
continuity of matrix multiplication is obvious, since each matrix
coefficient of the product is polynomial in the matrix coefficients of
the factors.  The continuity of of the matrix inverse for invertible
matrices follows from Cramer's rule \cite[Prop.~5.4]{knapp-basic} for
the inverse of $g\in GL(n,F)$:
\[
\det(g) g^{-1} = \text{  matrix of cofactors of } \t{g}.
\]
Cofactors are determinants, hence polynomial, hence continuous.

A finite group is a topological group in the discrete topology.

A topological group inherits notions from topology: it can be
discrete, connected, compact, locally compact, etc.

Let's proof a few simple facts about topological groups.  (Proofs are
generally easy in this branch of mathematics.)

\begin{lemma}
Let $L_g:G\to G$ be left translation in a topological group $L_g(h) = g h$.
Then $L_g$ is homeomorphism.  Similarly, right translation is a homeomorphism.
\end{lemma}

\begin{proof}
It is continuous, given as 
a composite of continuous maps $\{g\}\times G \to G\times G \to G$.
Its inverse $L_{g^{-1}}$ is also continuous.
\end{proof}

\begin{lemma}  The neighborhoods of $g\in G$ have the form $g U$ (or $Ug$),
where $U$ is a neighborhood of $e\in G$.
\end{lemma}

\begin{proof} $L_g$ is a homeomorphism.
\end{proof}

\begin{lemma} If $U$ is a neighborhood of $e\in G$, there is an open neighborhood  $V$ of $e$
such that $V= V^{-1}$ and $VV \subset U$.
\end{lemma}

\begin{proof} By the definition of the product topology, $\mu^{-1}(U)$ contains an open
neighborhood of $e$ of the from $W\times W$.  The open set $V = W\cap W^{-1}$ works.
\end{proof}

\begin{lemma} Let $G$ be a topological group and $H$ a closed subgroup.  Fix $g\ne H$.
Then there exists disjoint open sets $V_1,V_2$ such that $H\subset V_1$ and $gH\subset V_2$.
\end{lemma}

\begin{proof}  Let $U$ be the open set $G\setminus g H$.  Let $V$ be as in the
previous lemma.  Then $V_1 = VH$ and $V_2 = VgH$ works.  To check disjointness,
\[
v_1 g h_1 = v_2 h_2 \Longrightarrow U \owns v_1^{-1}v_2 = g h_1 h_2^{-1} \in g H.
\]
\end{proof}

\begin{lemma} if $\{e\}$ is closed, then $G$ is Hausdorff.
\end{lemma}

\begin{proof} Take $H = \{e\}$ in the previous lemma.
\end{proof}

If $X$ is any topological space and $\sim$ is any equivalence relation,
we let $X/\sim$ be the set of equivalence classes.  There is a canonical
projection $p:X\to X/\sim$.  We put the quotient topology on $X/\sim$:
a set $U$ is declared open if and only if $p^{-1}U$ is open.

If $H$ is a subgroup of a topological group $G$, then the left cosets are the equivalence
classes of an equivalence relation.  Hen $G/H$ carries a quotient topology.

\begin{lemma} Let $H$ be a subgroup of a topological group.  The canonical
projection $p:G\to G/H$ is an open mapping.
\end{lemma}

\begin{proof} The image of $U$ is open iff 
\[
p^{-1}p (U) = UH =\cup_{h\in H} Uh
\]
is open, which is evidently a union of open sets.
\end{proof}

\begin{lemma} If $G$ is Hausdorff and $H$ is a closed subgroup, then $G/H$ is
Hausdorff.
\end{lemma}

\begin{proof}
An earlier lemma separated the cosets $H$ and $gH$ of a closed group $G$.
\end{proof}

\begin{lemma}  If $G$ is Hausdorff and $H$ is a closed normal subgroup, then
$G/H$ is a Hausdorff topological group.
\end{lemma}

\begin{proof} We leave it to the reader to check continuity.
\end{proof}

\begin{lemma} If $H$ is a subgroup of a topological group $G$, 
then its closure $\bar H$ is also a subgroup
of $G$.
\end{lemma}

\begin{proof} Suppose that $g_1,g_2$ are in the closure.  We need to
show that $g_1 g_2^{-1}$ is in the closure.  Take an arbitrary neighborhood
$g_1 U g_2^{-1}$ of $g_1 g_2^{-1}$.  Pick $V$ as above adapted to $U$.
There exist $h_i\in g_i V\cap H$.  Then 
\[
h_1 h_2^{-1}\in g_1VV^{-1}g_2^{-1}\subset g_1Ug_2^{-1} \cap H
\]
and
$g_1 g_2^{-1}$ is indeed in the closure of $H$.
\end{proof}

\begin{lemma} Let $H$ be an open subgroup of $G$.  Then $H$ is a closed subgroup
of $G$.
\end{lemma}

\begin{proof}  Its complement is the union of open sets $g H$, $g\not\in H$.
\end{proof}

 Recall that the connected
component of a point $x$ in a topological space $X$ is the maximal connected subset
containing $x$ in $X$.

\begin{lemma} Let $G^0$ be the connected component of the identity $e$ in $G$.
Then $G^0$ is a normal subgroup of $G$.
\end{lemma}

\begin{proof}
  The image of the connected set $G^0\times G^0$ under $\mu$ in $G$ is
  connected and contains $e$.  Hence the image is a subset of $G^0$.
  So $G^0$ is closed under multiplication.  The image of the connected
  set $G^0$ under $\iota$ in $G$ is connected and contains $e$.  Hence
  the image is a subset of $G^0$.  So $G^0$ is closed under inverses.
  For each $g\in G$, the image of the connected set $G^0$ under
  $h\mapsto g^{-1}hg$ in $G$ is connected and contains $e$.  Hence the
  image is a subset of $G^0$.  So $G^0$ is closed under inverses.
\end{proof}

\begin{lemma}  If $H$ is a normal subgroup of $G$, then $\bar H$ is a normal
subgroup of $G$.
\end{lemma}

\begin{proof}  Let $x\in \bar H$.  Let $g^{-1} x U g$ be an arbitrary neighborhood
of $g^{-1} x g$.  Pick $h\in x U\cap H$.  Then
\[
g^{-1} h g \in g^{-1}x U g \cap H.
\]
\end{proof}

\newpage
\section{Inverse Limits}

Lecture January 18, 2012

A directed system $(I,\le)$ is a set $I$ together with a relation $(\le)$ on $I$
 that is
\begin{enumerate}
\item reflexive: $i\le i$,
\item transitive: $i\le j$ and $j\le k$ implies $i\le k$, and
\item for all $i, j\in I$, there exists $k\in I$ such that $i\le k$ and $j\le k$.
\end{enumerate}

Let $G_i$, $i\in I$ be a family of groups indexed by a directed system $I$.
Assume that there are group homomorphisms $\pi^j_i:G_j\to G_i$, for $i\le j$,
such that
for all $i\le\j\le k$, we have $\pi^j_i\circ \pi^k_j = \pi^k_i$.

We define a new group (the inverse limit)
\[
G = \lim_{\leftarrow_{i\in I}} G_i
\]
as follows.
Let 
\[
G = \{(g_i)_i \in \Pi_{i\in I} G_i\mid \pi^j_i g_j = g_i, \quad i\le j\}.
\]

\begin{lemma}
$G$ is a subgroup of this product.
\end{lemma}

\begin{proof}
This is an easy consequence of the fact that each $\pi^j_i$ is a group
homomorphism.
\end{proof}

If $G_i$ is a family of topological spaces, then
$\Pi G_i$ is a topological space in the product topology.  Recall that
every open set in the product topology is a union of sets of the form
$\Pi U_i$, where each $U_i$ is open in $G_i$ and $U_i = G_i$ for all
but finitely many indices $i\in I$.

If each $G_i$ is a topological group and each $\pi^j_i$ is a continuous
group homomorphism, then $G$ carries the induced topology as a subset of the product.
Note that each condition $\pi^j_i g_j = g_i$ is a closed condition (the
graph of a continuous function is closed), and an intersection of closed
is closed.  Hence $G$ is a closed subgroup of the product.

\begin{exercise}  Check that $\Pi G_i$ is a topological group, rather
than just a group that carries a topology, by checking that multiplication and
inverse are continuous in the product topology.
\end{exercise}

\begin{exercise}
If each $G_i$ is Hausdorff, then $\Pi G_i$ is a Hausdorff topological
space, and the inverse limit $G$ is also Hausdorff.
\end{exercise}

\begin{theorem}[Tychonoff]  The cartesian product of family of compact topological
spaces is compact.
\end{theorem}

\begin{proof} See a topology textbook, such as Munkres or Kelley \cite[page 143]{Kelley}.
\end{proof}

In particular if each $G_i$ is a compact Hausdorff topological group, then the inverse limit
$G$ is a closed subset of a compact Hausdorff space and is thus also
a compact Hausdorff topological group.

This holds in particular if each $G_i$ is a finite group, considered as carrying
the discrete topology.  In this case, the inverse limit is called a \newterm{profinite}
group.  It is a compact Hausdorff group.

A particularly important example is the family of finite groups $\ring{Z}/n\ring{Z}$,
indexed by the natural numbers $n$.  If $m$ divides $n$, then we have a group
homomorphism
\[
\pi^m_n:\ring{Z}/n\ring{Z} \to \ring{Z}/m\ring{Z}.
\]
Define a directed system $\ring{N}$ by the binary relation $m | n$.
Put the discrete topology on each finite group.
The inverse limit is called $\hat{\ring{Z}}$, the profinite completion of $\ring{Z}$.
It is a compact group.

\begin{exercise}
In fact, we note that of $\ring{Z}/n\ring{Z}$ is actually a ring, and each $\pi^m_n$
is actually a ring homomorphism.  The construction of the inverse limit is compatible
with these ring operations, and $\hat{\ring{Z}}$ is a topological ring.
\end{exercise}

The inverse limit can be characterized by a universal property.  
For each $i$, let $\pi_i: \prod G_i\to G_i$ be the projection onto the $i$th factor.
Suppose that
$H$ is a group equipped with group homomorphisms $\phi_i:H \to G_i$ that are compatible
in the sense that if $i\le j$, we have $\pi^j_i\circ \phi_j = \phi_i$.  Then
there exists a unique homomorphism $\phi:H\to \lim_{\leftarrow} G_i$ such that
for all $i$, we have $\pi_i\phi = \phi_i$.

Moreover, if $H$ is a topological group, and if each $\phi_i$ is continuous, then
$\phi$ is a morphism of topological groups (that is, a continuous homomorphism)

\begin{proof} Map $H$ to $\prod G_i$ by $h\mapsto (\phi_i(h))_i$.  The
  compatibilities conditions are exactly what is needed for the image
  to lie in $\lim_{\leftarrow} G_i$.
\end{proof}

For example, we have compatible maps from $\ring{Z}\to \ring{Z}/n\ring{Z}$.
Hence the universal property gives a homomorphism
$\ring{Z}\to \hat{\ring{Z}}$.

\begin{exercise}
Show that the kernel of this map is trivial.
\end{exercise}

\begin{exercise}
Show that the image of $\ring{Z}$ in $\hat{\ring{Z}}$ is dense.
\end{exercise}


\newpage

\section{Completion}

\section{Haar measure}

A reference for Haar measure is chapter 11 of \cite{Halmos-measure}.




\raggedright
\bibliographystyle{plainnat}
\bibliography{../bibliography/all}




\end{document}




