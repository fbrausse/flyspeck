% 
% Author: Thomas C. Hales
% Affiliation: University of Pittsburgh
% email: hales@pitt.edu
%
% latex format

% History.  File started June 20, 2011
% Paper sollicted by Journal of Automated Reasoning,
% special issue on
% "Formal Mathematics for Mathematicians:
% Developing Large Repositories of Advanced Mathematics."
 
%% 

\documentclass{llncs}
\usepackage{verbatim}
\usepackage{graphicx}
\usepackage{amsfonts}
\usepackage{amscd}
\usepackage{amssymb}
\usepackage{alltt}
% automatically generate revision number by
% svn propset svn:keywords "LastChangedRevision" flext.tex
\def\svninfo{{\tt
  filename: flext.tex\hfill\break
  PDF generated from LaTeX sources on \today; \hfill\break
  Repository Root: https://flyspeck.googlecode.com/svn \hfill\break
  SVN $LastChangedRevision: 1552 $
  }
  }
%-%


\input /Users/thomashales/Desktop/googlecode/flyspeck/kepler_tex/kepmacros.tex

% Math notation.
\def\op#1{{\hbox{#1}}} 
%\def\tc{\hbox{:}}
%\newcommand{\ring}[1]{\mathbb{#1}}

% Flags

%%%%%%%%%%%%%%%%%%%%%%%%%%%%%%%%%%

\begin{document}

\title{A Formalization of the Proof Text of\\ the Kepler Conjecture}
\author{Mark Adams,\\ 
Dat Tat Dang,\\
 Thomas C. Hales, \\
John Harrison,\\
 Truong Le Hoang,\\
Thang Tat Nguyen, \\
Truong Quang Nguyen,\\ 
Jason Rute, \\
Alexey Solovyev, \\
Diep Trieu Thi, \\
Trung Nam Tran, \\
Ky Khac Vu, \\
Thanh Vu,\\
Quyen Anh Vuong}
\institute{University of Pittsburgh\\
Hanoi, etc.\\
\email{hales@pitt.edu}}
\maketitle

\abstract{ 
  {\bf WARNING AND DISCLAIMER: 
  This is a draft based on our expectation that this work
  will some day be complete.  The actual announcement and publication
  must wait for the work to be done.}

\smallskip
  This article announces the completion of a multiyear
  collaboration to formalize the textual portions of the proof of the
  Kepler conjecture.

  \smallskip
  The full formalization of the Kepler conjecture also requires the
  formalization of some large bodies of computer code and the
  integration of that code with the text. These computer portions of
  the proof are not covered by this announcement.  }

\section*{background}

Description of the flyspeck project, Kepler conjecture, and its purpose.

What is HOL Light.  The need for a formal proof, etc.

400 year anniversary, etc.  (progress report at JMM, but Marchal strategy proof 
completed only May 24 (preliminary), June 3 (corrected).
June 3, 2011, recently).

\section{design of the proof}

2 activities Bourbaki and the rising sea.

streamlining the proof along the lines of Marchal's strategy, (new proof), 
rewritten text to be published by CUP,



\section{Theorems Formalized}



\subsection{a statement of theorems}

Since this project depends on bodies of computer code that have not
yet been formally verified, we do not give a complete formal proof
of the Kepler conjecture.  Nevertheless, we can show how the statement
appears in the formal proof system.

XX R3.
XX Define packing, saturated packing, statement of the theorem.

The primary aim of the text portions of the proof is to show that if the KC is
false, then there is a finite packing of spheres with very special properties.

The following theorem has been formally verified.  It uses a special constant
$\hm=1.26$ that gets used as a truncation parameter throughout the proof.
We let $\BB$ be the closed annulus:
\[
\BB = \{\v\in\ring{R}^3 \mid 2\le \normo{\v}\le 2\hm\}.
\]

\begin{theorem}
If the Kepler conjecture is false, then there exists a finite set $V\subset\ring{R}^3$
with the following properties.
\begin{enumerate}
\item $V$ has cardinality $13$, $14$, or $15$.
\item $V$ is a packing.
\item ${\mathcal L}(V) > 12$ and no finite packing in $V$ gives a
  larger value of ${\mathcal L}$.
\item For every $\v\in V$, we have $2\le \normo{\v}\le 2\hm$.
\item $V$ is \newterm{totally surrounded} in the following sense.
\item $(V,E_{std})$ is a fan.
\item $\op{hyp}(V,E_{std})$ is a \newterm{tame hypermap}.
\item etc.
\end{enumerate}
\end{theorem}

The significance of this result is  tame hypermap is a purely combinatorial
notion.  One of the bodies of computer code 

\subsection{statistics}

Give general statistics about SLOC, number of lemmas, definitions, etc.
Length of book, etc.

Does this work scale?  Modules helped people could redefine tactics, etc.

Would be useful:
dependent types, better encapsulation (constants are global),
more low-level automation (  ),  REAL FIELD is great.

Slowdown in definition lookup.  Searching for theorems.  Integrative
theorems were difficult, those that a single person was able to formalize
in their own work ok.

Might be difficult for a larger project (such as Fermat or Poincare).
We envy Gonthier's design.

If I were to do it all again?  More work on developing general tools in HOL Light.


\subsection{beyond the scope}

The following parts of the text fall outside the scope of this work.

\begin{enumerate}
\item Introductory material, illustrations, and remarks in the text
  that do not contribute directly to the logical structure of the
  proof.
\item Inessential appendices and a final chapter that discusses
  theorems in discrete geometry beyond the Kepler conjecture.
\item Bodies of computer code used in the proof of the Kepler
  conjecture.
\item The three sections of the text that discuss how the bodies of
  computer code relate to the text portions of the proof
  (Hypermap--Generation, Local Fan--Main Estimate, Tame
  Hypermap--Linear Programs).
\end{enumerate}

Various lemmas throughout the text rely on nonlinear inequalities that
have been verified by computer.  For purposes of formalization of the
text, these externally verified inequalities have been accepted as
independently given axioms.

\subsection{Libraries}

Here are some of the specific libraries of theorems that were developed as part
of this project.

The credits section of flypaper will give details about individual contributions.
This section describes how the labor was divided among us.


\subsubsection{HOL Libraries} 
Harrison has contributed several libraries to the HOL Light library,
that are not specifically designed for this project, but that have been extremely
useful: topology, real analysis, geometry of $R^n$, and so forth.

\subsubsection{Trigonometry} 
Rute Harrison NQT some contributions by Ky and Diep.
spherical trigonometry, Euler's theorem, Euclidean geometry in 3D (dihedral
angles spherical coordinate systems,....)

\subsubsection{Volume} Harrison with some contributions by Thang.
gauge integration in n-dimensions.  properties of null sets.  Explict tables of
volumes for the classical solids, such as ball of radius r, cones, 

\subsubsection{Hypermap}  TNT.  The combinatorial theory of hypermaps.  Parts of this
is inspired by work done in Coq on hypermaps as part of the formalization 
of the 4Ct.


\subsubsection{Polyhedra} John Harrison.  general properties of convex sets,
Krein-Millman, general properties of polyhedra, facets and faces,
Euler formula.

\subsubsection{Fans}  HLT with polyhedra sections contributed by Harrison.  The geometry of fans.
planarity of fans,
perimeter estimate.

\subsubsection{Discrete Geometry} (Solovyev, with many supporting contributors).  Properties
of Voronoi cells, saturated packings, Rogers simplices, Marchall cells.

\subsubsection{Kepler conjecture proper}  local fans, Nguyen Quang Truong, Solovyev, dots.
Special theory of Tame Hypermaps (solovyev with other contributors).

What we managed to avoid:

Jordan curve theorem (but come close to reproving it), even though this was
Hales's initial initiation. Lexell theorem, 
surface integrals, mostly avoid integration (as opposed to volume) 
once we have a table of
volumes of classical solids.

Corollary strong form of the 13 spheres problem, and proofs of a number
of new conjectures in discrete geometry that will be reported elsewhere XX,XX.




\subsection{History}

1611, Kepler made the assertion

1900 Hilbert

1953 Fejes Toth outlines  a program.

Aug 8, 1998, Kepler proved, Hales/Ferguson.
Nov, Dodecahedral conjecture proved McLaughlin

2003? Flyspeck announced.

2004 Hales Jordan Curve theorem (feasibility study).

2006 Kepler published.

DCG errata published.

Flyspeck I, Bauer-Nipkow,

Flyspeck II, Obua,

Marchal strategy published.

Zumkeller,

McLaughlin, projects, 


Initial contact, sabbatical Hanoi fall 2007, discrete geometry course, intl masters.

Small training session Summer 2008.

Flyspeck Intl. workshop 2009. 25 participants, several non-vietnamese.

Workshop II, 2010, smaller group of the active participants.

Several of the students go on fellowship to Europe. Loss of cohesion.

Postdocs, and graduate students assistants came to Pitt.



Thanks to Sean McLaughlin, Roland Zumkeller, Tobias Nipkow, Joe Pleso,
Emily Kline, Ta Thi Hoi An, Beth Laurel Martin and cohort, Nguyen Duc Tam,

Institutional Support: NSF, Benter, Vien Toan Hoc.

Timing and coordination 
issue : Marchal strategy was not completed until June 2011. However,
work started at the other end in summer 2009, unfinished text, and so forth.
We assumed that it would all work out and it did.


\subsection{Other research}

To complete the Flyspeck project (of formalizing the proof of the
Kepler conjecture), two tasks remain: the verification of the computer
parts of the proof and the integration of the computer parts and proof
text.  There are three bodies of computer code: tame hypermap
classification, linear programs, and nonlinear inequality
verification.  The formal verification of the classification of tame
hypermaps has been completed~\cite{XX}.  The formal verification of
the linear programs is near completion~\cite{XX},~\cite{XX}.  The
formal verification of nonlinear inequalities is the primary component
that is still missing from the Flyspeck project.

\bigskip
\noindent
\svninfo
\end{document}

