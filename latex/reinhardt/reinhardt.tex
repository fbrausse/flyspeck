% On the Reinhardt Conjecture
% Author: Thomas C. Hales
% Affiliation: University of Pittsburgh
% email: hales@pitt.edu
%
% latex format

% History.  File started Feb 16, 2008
% return Feb 16, 2009.
% replaced sigma->psi in rank 2 calculus of variations.

%%  XX Unfinished Lipschitz section and Bolza section. 

%% 
% XX Nazarov says "domain" rather than body.
% delta in 2x2 matrix, as dirac, as density.
% XX u as multi-point, tau as rank-1 parameter.
%%%


\documentclass[11pt]{amsart}
\usepackage{graphicx}
\usepackage{amsfonts}
\usepackage{amscd}
\usepackage{amssymb}
\usepackage{alltt}
\usepackage{amsmath}

%\spnewtheorem{conj}[theorem]{Conjecture}{\bfseries}{\itshape} 
\newtheorem{definition}{Definition}
\newtheorem{thm}{Theorem}
\newtheorem{lemma}{Lemma}
\newtheorem{conj}{Conjecture}
%\newtheorem{assumption}[section]{Assumption}
\newtheorem{corollary}{Corollary}
\newtheorem{remark}{Remark}
\newtheorem{quasilemma}{Quasi-Lemma}


\parindent=0pt
\parskip=\baselineskip


% automatically generate revision number by
% svn propset svn:keywords "LastChangedRevision" reinhardt.tex
\def\svninfo{{\tt
  filename: revision.tex\hfill\break
  PDF generated from LaTeX sources on \today; \hfill\break
  Repository Root: https://flyspeck.googlecode.com/svn \hfill\break
  SVN $LastChangedRevision$
  }
  }




% Math notation.
\def\op#1{{\operatorname{#1}}}
\newcommand{\ring}[1]{\mathbb{#1}}
\def\deltalat{\mathbb\delta}  %_{\hbox{\scriptsize{lat}}}}
\def\delt{\delta_{\min}}
\def\rZ{{\ring{Z}}}
\def\rR{{\ring{R}}}
\def\rP{{\ring{P}}}
\def\HH{{\mathcal H}}
\def\aa{{\op{area}}}
\def\ta{{\tau}}
\def\mid{\,:\,}
%%%%%%%%%%%%%%%%%%%%%%%%%%%%%%%%%%

\begin{document}

\title{On the Reinhardt Conjecture: the problem of the miser's coin}
\author{Thomas C. Hales}
%\institute{University of Pittsburgh\\
\email{hales@pitt.edu}
\thanks{Research supported by NSF grants 0503447 and 0804189.  A preliminary version of these results was presented at a conference in honor of L. Fejes T\'oth in Budapest, June 2008.  I thank Nate Mays for drawing my attention to this problem}.


\begin{abstract}  
In 1934, Reinhardt asked for the centrally symmetric convex domain in the plane whose best lattice packing has the lowest density.  He conjectured that the unique solution up to an affine transformation is the smoothed octagon.  This article offers a detailed proof strategy.  In particular, we show that the problem
is a special instance of the classical problem of Bolza in the calculus
of variations.  A minimizing solution is known to exist and has sufficient regularity that is satisfies the standard necessary conditions (such as the Euler-Lagrange equations) of extremals.
% XX replace by the statement:
% This article proves that the solution to the Reinhardt conjecture is a smoothed polygon.
\end{abstract}

\maketitle


%%%%%%%%%%%%%%%%%%%%%%





\section{Introduction}

{\narrower\it A contract requires a miser to make payment with a tray of identical gold coins filling the tray as densely as possible.  The contract stipulates the coins to be convex and centrally symmetric.  What shape coin should the miser choose in order to part with as little gold as possible?}

Let $K$ be a centrally symmetric convex domain in the Euclidean plane.  
If $\Lambda$ is a lattice such that the translates of $K$ under $\Lambda$ have disjoint interiors, then the packing density of $\Lambda+K$ is the ratio of the area of $K$ to the co-area of the lattice $\Lambda$. 
Let $\deltalat(K)$ be the maximum density of any lattice packing of $K$.  A lattice realizing this density exists for each $K$.
% K. Mahler, On star-regions in number geometry, Proc. Roy. Soc. London. Ser. A vol. 187 (1946) pp. 151-187. Zentralblatt MATH: 0060.11710 ...
% Cited in Zassenhaus.
%(Reinhardt restricts his attention to centrally symmetric domains because of Minkowski's observation that if $K'$ is an abitrary convex domain with central symmetrization $K$, then the set of translates of $K'$ under $\Lambda$ is a packing iff the corresponding set of translates of $K$ is also.)

Let $\delt$ be the infimum of $\deltalat(K)$,
as $K$ ranges over all convex domains in the Euclidean plane.  
Reinhardt proves that there exists $K$ for which $\deltalat(K)=\delt$.
Reinhardt's problem
is to determine the constant $\delt$ and 
to  describe $K$ explicitly
for which $\deltalat(K)=\delt$.

Reinhardt conjectured that $\deltalat(K) =\delt$ when $K$ is a
smoothed octagon (Figure~\ref{XX}).   The smoothed octagon is constructed
by taking a regular octagon and clipping the corners with hyperbolic arcs.
The hyperbolic arcs are chosen so that the smoothed octagon has no corners; that is, so that there is a unique tangent at each point of the boundary.  The asymptotes of each hyperbola are lines extending two sides of the regular octagon.  The density of the smoothed octagon is
   $$\deltalat(K) = \frac{8 - \sqrt{32} - \ln{2}}{\sqrt8 - 1}\approx 0.902414.$$

Reinhardt's original article contains many useful facts about his conjecture.  The main facts
from his article have been summarized in Section~\ref{sec:rein}.  Beyond Reinhardt's article,
various lower bounds for $\delt$ have been published \cite{MH}, \cite{FT}, \cite{T}, \cite{E}.
The special case of centrally symmetric decagons is considered in \cite{LM}.  
Nazarov has proved the local optimality of the smoothed octagon \cite{N}. The problem is discussed further in \cite{AP}, where it is referred to as a ``famous conjecture.''  As we will see, famous or not, it is simply a special instance of the classical problem of Bolza in the calculus of variations and can be analyzed in a completely satisfactory way within that theory.   It is also known that non-lattice packings of centrally symmetric convex domains in the plane cannot have greater density than $\deltalat(K)$ \cite{FT50}.  The Ulam conjecture, which is the corresponding conjecture in three dimensions, posits the sphere as solution \cite{G}.  

In this article we take the following approach.  The perimeter of any optimal $K$ is a $C^1$ curve.
We express the boundary of $K$ in the calculus of variations. The circle is the unique solution to the Euler-Lagrange equations (up to an affine transformation), but an examination of second-order conditions shows that the circle does not minimize.  This means that optimal $K$ is not an interior point of the configuration space.


This leads to a study of the boundary of the configuration space.  Analytic points in the configuration-space boundary have a well-defined (local) invariant, called the rank.  Calculus of variations further reduces the problem to the study of rank one.  We show that rank-one configurations $K$ are structurally similar to the smoothed octagon.  In particular, the perimeter of such $K$ consists of finitely many linear segments and hyperbolic arcs.  We propose a nonlinear optimization problem in a small number of variables over certain rank-one configurations.  The successful solution of this nonlinear optimization problem (together with an assumption about analytic approximation)
implies the Reinhardt conjecture. 




\section{Reinhardt's Article}\label{sec:rein}

We give a brief review a series of lemmas in Reinhardt's original article.
The proofs are generally elementary.


\subsection{balanced hexagons}

%
\begin{definition}[balanced hexagon]
We call a centrally symmetric hexagon $G$  a {\it balanced hexagon} of 
a centrally symmetric convex domain $K$
if $G$ contains $K$ and if the midpoint of each side of $G$ is a 
point on the boundary of $K$.  These six points on
the boundary of $K$ are called the midpoints of $G$.
\end{definition}


\begin{lemma}\label{lemma:parallel} Let $G$ be a balanced hexagon of a centrally symmetric convex domain $K$ without corners.  Then $G$ does not degenerate to a parallelogram.  That is, the six vertices are distinct \cite{R}.
\end{lemma}

\begin{lemma}\label{lemma:mid1} 
Let $K$ be a  centrally symmetric convex domain.  Each
point of $K$ is a midpoint of at most one balanced hexagon \cite[p.228]{R}.
\end{lemma}



\begin{lemma}\label{lemma:8G} 
Let $G$ be a balanced hexagon of a centrally symmetric convex domain $K$.  
Assume that the center of symmetry of $K$ is the origin.  
Let $u_j$, $j\in \ring{Z}/6\ring{Z}$,  be the midpoints of the sides of $G$ 
listed in cyclic order around the hexagon.  Then 
 \begin{itemize}
 \item $u_j+u_{j+2}+u_{j+4}=0$.
 \item $u_{j+3} = -u_j$.
 \item The area of $G$ is $4/3$ the area of the hexagon 
     $H$ formed by the convex hull of $\{u_j\}$.
  \item The six segments from the origin to the six midpoints $u_j$ breaks
    $H$ into six congruent triangles.  In particular, the area of $G$
    is $8$ times the area of the triangle $\{0,u_j,u_{j+2}\}$.
  \end{itemize}
\end{lemma}

\begin{proof} If the vertices of the centrally symmetric hexagon 
$G$ are $w_j$, with $w_{j+3} = - w_j$,  then
$$u_j = (w_j+w_{j+1})/2.$$
The first two statements are then immediate.  The other statements appear in 
\cite[p.219,p.222]{R}.
\end{proof}


\subsection{miserly domains}

\begin{lemma}  There exists a centrally symmetric convex domain
$K$ for which $\deltalat(K) = \delt$.
\end{lemma}

\begin{proof} This follows by Blaschke's selection lemma \cite[p.220]{R}.
\end{proof}
%
\begin{definition}[miserly domain]
Any centrally symmetric convex domain $K$ that realizes the lower
bound $\deltalat(K) = \delt$ is called a {\it miserly domain}. %was sparse
\end{definition}

\begin{lemma}  If $K$ is a miserly domain, then it has no
corners.  That is, there is a unique tangent through each point
on the boundary.  \cite[p.221]{R}
\end{lemma}

\begin{lemma}\label{lemma:mid-min} \hbox{ }
\begin{itemize}
\item Let $K$ be a centrally symmetric domain without corners.  Assume that each point
on the boundary of $K$ is a midpoint of a balanced hexagon.  Assume further that
every balanced hexagon in this collection has the same area.  Then there are no other
balanced hexagons and
\begin{equation}\label{eqn:density}
\deltalat(K) = \aa(K)/\aa(G)
\end{equation}
for any balanced hexagon $G$.
\item If $K$ is any miserly domain, then it satisfies these assumptions.
\end{itemize}
\end{lemma}
%
\begin{proof} The facts asserting in this proof 
appear in \cite[pp.219--222]{R}.
Let $K$ be a centrally symmetric convex domain in the plane.
Let $G$ be a smallest centrally symmetric hexagon that contains
$K$.  Such a hexagon exists, and $\deltalat(K)$ equals
the ratio of the area of $K$ to that of $G$. %  \cite[p.219]{R}.
Call any such hexagon a fitting hexagon.
Every fitting hexagon of $K$ is a balanced hexagon. % \cite[p.219]{R}.
There are no other balanced hexagons by Lemma~\ref{lemma:mid1}.

Now let $K$ be a miserly domain.
Reinhardt proves that each boundary point $p$ of $K$ lies on a  balanced hexagon that is also a fitting hexagon of $K$, 
although $p$ is not necessarily 
a midpoint of the balanced hexagon.  % \cite[p.221]{R}.
Next, he shows that each boundary point of $K$ is in fact the midpoint
of a balanced hexagon that is also a 
fitting hexagon of $K$. % \cite[p.222]{R}.
Since there are no other balanced hexagons, there are no other fitting hexagons.  The set of balanced
hexagons coincides with the set of fitting hexagons. 
All fitting hexagons have the same area.  Thus, the assumptions of the first part of the lemma
are all satisfied for a miserly domain.
\end{proof}






\section{the Boundary Curve}


\subsection{hexameral domains}
% We assume for the rest of the paper that the origin is the center of symmetry of every hexameral domain $K$. 

If we combine the properties of miserly domains that were established by Reinhardt, we are led to the following definition.

\begin{definition}[hexameral domain]
We say that $K$ is a hexameral domain if the following conditions hold.
\begin{itemize}
\item $K$ is a centrally symmetric domain whose center of symmetry is the origin.
\item $K$ has no corners.
\item Each point on
the boundary of $K$ is a midpoint of a balanced
hexagon $G$.  Moreover, these balanced hexagons all have the same area.
\end{itemize}
%We call them hexameral domains.
% and their boundaries Reinhardt curves. 
\end{definition}

By the preceding lemmas, if $K$ is a miserly domain, then (after recentering at the origin) it is a hexameral domain.  The density $\deltalat(K)$ of a hexameral domain is computed by Formula~(\ref{eqn:density})
and Lemma~\ref{lemma:8G}.
The smoothed octagon and the
circle are examples of hexameral domains. 
The
class of hexameral domains is much larger than the class of
miserly domains.  We consider the optimization problem
of locating the miserly domains within the class of
hexameral domains.  If $K$ is a hexameral domain,
then each point of the boundary is a midpoint of a {\it unique} balanced hexagon. 

Let $K$ be a hexameral domain.   Give a continuous
parametrization $t\mapsto\sigma_0(t)$ of the boundary curve. 
We follow the convention of parametrizing the boundary in
a counterclockwise direction.
Since $K$ has no corners, we may assume that $\sigma_0$ is $C^1$.
At each time $t$, there is a uniquely determined balanced hexagon
with midpoint $\sigma_0(t)$. Let the other midpoints be 
listed in (counterclockwise) order
 as $\sigma_j(t)$, $j\in\rZ/6\rZ$.

If $u$ and $v$ are ordered pairs of real numbers, write
$u\land v$ for the $2\times 2$ determinant with columns $u$ and $v$.


\begin{lemma} Let $K$ be a hexameral domain with $C^1$ boundary
parametrization $\sigma_0$.  Then the curves $\sigma_2$ and $\sigma_4$
are also $C^1$ parametrizations of the boundary, oriented in the same
way as $\sigma_0$.
\end{lemma}

\begin{proof}  Reinhardt shows that the boundary parametrizations
$\sigma_2,\sigma_4$
are continuous if $\sigma_0$ is continuous, and that they
are oriented in the same
way as $\sigma_0$ \cite[p.222]{R}.  Let us check that $\sigma_2$
is $C^1$, whenever $\sigma_0$ is.  Since $K$ has no corners, the
unit tangent $n_2(t)$ to $\sigma_2(t)$, 
with the orientation given by $\sigma_0$, 
is a continuous function of $t$.  It is enough to check
that the speed of $\sigma_2$ is continuous in $t$.  
By the previous lemma, ${\sigma_0(t)}\land{\sigma_2(t)}$ is a fixed fraction
of the area of the balanced hexagon, and does not depend on $t$.

We claim that ${\sigma_0(t)}\land{n_2(t)}\ne 0$.
Let $H(t)$ be the hexagon given by the convex hull of $\{\sigma_j(t)\}$.
If ${\sigma_0(t)}\land{n_2(t)}=0$, then the tangent line to $\sigma_2$ at
$t$ contains the edge of $H(t)$ through $\sigma_2(t)$ and $\sigma_1(t)$.
Then also, ${\sigma'_0(t)}\land{\sigma_2(t)}=0$ and the tangent line
to $\sigma_0$ lies along another edge of $H(t)$.  This forces a corner
at $\sigma_1(t)$, which is a contradiction.

This nonvanishing result and the fact that ${\sigma_0(t)}\land{\sigma_2(t)}$ 
is independent of $t$ imply that there exists a function
$s_2(t)$ such that 
  \begin{equation}\label{eqn:sB}
  {\sigma'_0(t)}\land{\sigma_2(t)} + {\sigma_0(t)}\land{n_2(t)} s_2(t) = 0.
  \end{equation}
% take difference quotient of {sigma0}\land{sigma2} and take the limit of all other terms.
The function $s_2(t)$ is the speed, and from the form of this equation,
it is necessarily continuous in $t$.
\end{proof}



\subsection{multi-curve}

There is no harm in rescaling a hexameral domain so that its balanced
hexagon has  area $\sqrt{12}$, which is the area of a regular hexagon of inradius $1$.
For this normalization,
Lemma~\ref{lemma:8G} gives
%As noted above,
%the area of a balanced hexagon $G$ is a fixed multiple of ${u_0}\land{u_2}$.
%Thus, equivalently, we may assume that 
 \begin{equation}\label{eqn:AB}
 {\sigma_j(t)}\land{\sigma_{j+2}(t)} = \sqrt{3}/2.
 \end{equation}


This suggests the following definition.

\begin{definition}[multi-point, multi-curve]
A function $u:\rZ/6\rZ\to\ring{R}^2$ such that 
  \begin{equation}\label{eqn:uA}
  u_j+u_{j+2}+u_{j+4} = 0,\quad u_{j+3} = -u_j,\quad  {u_j}\land{u_{j+2}}=\sqrt{3}/2
  \end{equation}
is called a {\it multi-point}. 
An indexed set of $C^1$ curves
$$\sigma:\ring{Z}/6\ring{Z} \times [t_0,t_1]\to \ring{R}^2$$ 
is a {\it  multi-curve} if for all $t\in[t_0,t_1]$, $\sigma(t)$ is a multi-point.
That is,
\begin{itemize}
\item $\sigma_j(t) + \sigma_{j+2}(t) + \sigma_{j+4}(t) = 0$,
\item $\sigma_{j+3}(t) = -\sigma_j(t)$,
\item ${\sigma_j(t)}\land{\sigma_{j+2}(t)}=\sqrt{3}/2$.
\end{itemize}
\end{definition}



By differentiation, a multi-curve also satisfies for all $j$:
\begin{equation}\label{eqn:sigma'}
{\sigma_j(t)}\land{\sigma'_{j+2}(t)} 
+ {\sigma'_j(t)}\land{\sigma_{j+2}(t)} = 0.
\end{equation}
The boundary of a hexameral domain admits a parametrization as a triple
curve.  The converse does not hold because a multi-curve has no convexity
constraint and no constraint for the curves $\sigma_j$ to fit
seamlessly into a simple closed curve containing the origin in the interior.


\subsection{Lipschitz continuity}

\begin{lemma}
Let $K$ be a miserly domain and let $\sigma_j$ be a multi-curve parametrization on the boundary of $K$.  Assume that $\sigma_0$ is parametrized by arclength $s$.  Then $\sigma_0'$ is Lipschitz continuous.
\end{lemma}

\begin{proof} For each $s$, let $H_s$ be the
hyperbola through $\sigma_0(s)$ whose asymptotes are the lines in direction $\sigma'_j(s)$ through $\sigma_j(s)$, for $j=\pm 1$.  By Reinhardt~\cite{R}, near $\sigma_0(s)$, the an arc of $H_s$ lies
inside $K$.  As $s$ varies, by the compactness of the boundary, the curvatures of the hyperbolas $H_s$ at $\sigma_0(s)$ are bounded.  This means that
we can roll a sufficiently small disk along the boundary of $K$, while
keeping the disk within $K$.  By convexity, we can roll a line
around the boundary of $K$, while keeping the line exterior.

[XX FINISH]
\end{proof}


\begin{lemma}
Let $K$ be a miserly domain and let $\sigma_j$ be a multi-curve parametrization on the boundary of $K$.  Assume that $\sigma_0$ is parametrized by arclength $s$.  Then $\sigma_j'$ is Lipschitz continuous for all $j$.
\end{lemma}

[XX FINISH]

Since $\sigma_j'$ is Lipschitz continuous, Rademacher's theorem implies that
$\sigma_j'$ is differentiable almost everywhere (or directly we have that the argument of $\sigma_j'$ is monotonic, hence differentiable almost everywhere).  
Thus, we may express
the convexity constraint locally at $\sigma_j(t)$ by a second derivative:
$$
\sigma_j'(t)\land \sigma_j''(t) \ge 0.
$$



\subsection{special linear group}



%We may fix the area of $G$ and then minimize the area of $K$ subject
%to this constraint.  When $K$ is a circle of radius $1$, 
%then $G$ is a
%regular hexagon of area $\sqrt{12}$.  We use this as our normalization.
%Thus, we assume that $G$ has area $\sqrt{12}$.


The special linear group
$SL_2(\ring{R})$ acts on $\ring{R}^2$ by linear transformations and preserves the
exterior product:
   $$
    {g u}\land{ g v}= \det(g)({u\land v}) .
   $$
Conversely any affine transformation fixing the origin and fixing some
${u}\land{v}\ne 0$ must be given by some $g\in SL_2(\ring{R})$.  


Given a multi-curve $\sigma$ and multi-point
$u$, 
there 
exists a unique $C^1$ curve
$\phi(t)\in SL_2(\ring{R})$, $t_0\le t \le t_1$, such that 
   \begin{equation}\label{eqn:sigma-phi}
   \sigma_j(t) = \phi(t) u_j
   \end{equation}
for $j\in \ring{Z}/6\ring{Z}$.

The transformed multi-curve $\phi(t_0)^{-1} \sigma_j$
starts at 
$\sigma_j(t_0) = u_j$.
It is often convenient to use the multi-point formed by roots of unity:
\begin{equation}\label{eqn:roots}
u^*_j = \exp(\pi i j/3).
\end{equation}
In particular, any hexameral domain is equivalent under $SL_2(\ring{R})$
to a hexameral domain that starts at the multi-point $u^*$ on the
unit circle.  We call this a {\it circle representation} of the
hexameral domain or multi-curve.


%Let $u_j$  be the midpoints of a balanced hexagon of $K$.
%There is a unique special 
%linear transformation $g$ that carries $u_0$ and $u_2$
%to $\lambda (1,0)$ and $\lambda (-1/2,\sqrt{3}/2)$ for some $\lambda>0$.
%Our normalization and Equation~(\ref{eqn:AB}) then give $\lambda=1$.
%Thus, $K$ equivalent under $SL_2(\ring{R})$
%to a convex domain whose boundary contains the six vertices of a 
%regular hexagon $\pm (1,0)$, $(\pm 1/2, \pm \sqrt{3}/2)$.

%The initial condition is $\phi(t_0)=I$.  At some later time $t_1$, $\phi$ completes the trajectory
%to the next vertices of the regular hexagon:  $\sigma_0(t_1) = \sigma_1(t_0) = (1/2,\sqrt{3}/2)$;
%$\sigma_2(t_1) = \sigma_{3}(t_0) = (-1,0)$.
%   \begin{equation}\label{eqn:theta}
%     \sigma_0(t) = (r(t) \cos(\theta(t)),r(t)\sin(\theta(t))),\quad \hbox{ with }
%     0 \le \theta(t) \le \pi/6,\quad\hbox{ for } t_0 \le t \le t_1.
%   \end{equation}
%Also,
%   \begin{equation}\label{eqn:t1}
%   \phi(t_1) = \frac12 \begin{pmatrix} 1 & -\sqrt3 \\ \sqrt3 & 1 \end{pmatrix}
%   \end{equation}
%When we wish to do so, we may choose the parametrization of $\sigma_0$ so that we also match the derivative at time $t_1$:
%$\sigma_0'(t_1) = \sigma_1'(t_0)$.   It then follows from the conditions of a hexameral domain that
%   \begin{equation}\label{eqn:phi'}
%   \phi'(t_1) = \frac12 \phi'(t_0)\cdot \begin{pmatrix} 1 & -\sqrt3 \\ \sqrt3 & 1 \end{pmatrix}
%   \end{equation}

Let $\sigma$ be a multi-curve.  Define $X:[t_0,t_1]\to\mathfrak{gl}_2(\rR)$
by
$$\sigma'_j(t) = X(t)\sigma_j(t),\hbox{ for } j = 0,2.$$
Then also,
$$
\sigma'_j(t) = X(t)\sigma_j(t),\hbox{ for } j\in\rZ.
$$
and
\begin{equation}\label{eqn:Xt}
\phi'(t) = X(t) \phi(t),
\end{equation}
where $\phi$ is given by Equation~(\ref{eqn:sigma-phi}).
Equation (\ref{eqn:sigma'}) implies that $X(t)\in\mathfrak{sl}_2(\rR)$,
the Lie algebra of $SL_2(\rR)$.  The tangent lines to the curves
$\sigma_j$ are determined by the image of $X(t)$ in the projective
space $\rP(\mathfrak{sl}_2(\rR))$ over the vector space $\mathfrak{sl}_2(\rR)$.

If we transform $\sigma_j$ to $g\sigma_j$, for some $g\in SL_2(\rR)$,
then $X(t)$ transforms to $g X(t) g^{-1}\in \mathfrak{sl}_2(\rR)$.

We have seen that the parametrized boundary of a 
hexameral domain determines a curve in $SL_2(\ring{R})$.
Conversely, a curve in $SL_2$ determines a hexameral domain in the
following sense.

\begin{lemma}\label{lemma:sl2-rein}  
Let $K$ be a centrally symmetric convex domain with a multi-point
$u$ on the boundary.
Let $\phi:[t_0,t_1]\mapsto SL_2(\ring{R})$ be a $C^1$ curve. % with $\phi(t_0)=I$
%and satisfying Equations~(\ref{eqn:t1}),~(\ref{eqn:phi'}).  
Define curves $\sigma_j$ by Equation~(\ref{eqn:sigma-phi}).
%, and set $\sigma_{-j} = -\sigma_j$.  
Assume that $\sigma_j$ parameterizes
the boundary of $K$,
 for $j\in\ring{Z}/6\ring{Z}$.
%and that Condition~(\ref{eqn:theta}) holds.  
%Then the arcs
%   $$\sigma_0(t),\ \sigma_1(t),\ \sigma_2(t),\ \sigma_{3}(t),\ \sigma_4(t),%\ \sigma_{5}(t)$$
%for $t_0\le t\le t_1$ form a simple closed curve.  The simple closed curve boun%ds a centrally symmetric
% region $K$ with center of symmetry at the origin.  Assume further that $K$ is convex.  
Then
$K$ is a hexameral domain.
\end{lemma}

\begin{proof} 
We check the balanced hexagon condition.
At time $t$, let $w_j(t)$ be the point of intersection of the tangent line to
$\sigma_j(t)$ with the tangent line to $\sigma_{j+1}(t)$.  The condition that $w_j(t)$ are the vertices of a balanced
hexagon generates a  system of six linear equations and three unknowns.  Consistency of this system
of equations imposes  three  constraints:
   $$
   \begin{array}{lll}
   0 &= \sigma_0(t) + \sigma_2(t) + \sigma_4(t),\\
   0 &= {\sigma_0(t)}\land{\sigma_2'(t)}+ {\sigma_0'(t)}\land{\sigma_2(t)}.\\
   \end{array}
   $$
Integrating the final constraint, gives that ${\sigma_0(t)}\land{\sigma_2(t)}$ is constant.  These conditions
hold for a curve coming from $\phi$.  Thus, solving for $w_j(t)$, we find that each point of the simple closed curve is a midpoint of a 
balanced hexagon with vertices $w
_j(t)$.  
Since ${\sigma_0(t)}\land{\sigma_2(t)}$ is constant, these balanced hexagons have the
same area.  Thus, $K$ is a hexameral domain.
\end{proof}

\subsection{star conditions}\label{sec:star}

The convexity of a hexameral domain places certain constraints on the
element $X\in\mathfrak{sl}_2(\ring{R})$.  We normalize the curve by applying
an affine transformation so that $\phi(0)=I$ in the circle representation.
This means that the roots of unity $u^*_j$ lie on the boundary of 
the hexameral domain.  We form a hexagram through these six points.
Specifically, we construct the six equilateral triangles, each with three
vertices:
   $$
   u^*_j,\quad u^*_{j+1},\quad (u^*_j + u^*_{j+1}).
$$
For the boundary curve to be convex, the direction $X u^*_j$ must lie between
the the secant lines joining $u^*_j$ with $u^*_{j\pm 1}$, hence
must point into this triangle for each $j$.  If we write
$$X  = \left(\begin{matrix} a & b \\ c & -a \end{matrix}\right),$$
we have the following constraints on $X$:
$$
\sqrt{3} |a| < c,\quad 3 b + c < 0.
$$

\begin{lemma}
In this context,
$$\det(X) > 0.$$
\end{lemma}

\begin{proof} 
$$\det(X) = - b c - a^2 \ge - b c  - \frac{c^2}{3} = \frac{-c(3 b + c)}{3} >0.$$
\end{proof}



\section{Rank of Multi-Curves}


%The terms elliptic, parabolic, and hyperbolic are already overused in mathematics, and I hesitate
%to introduce yet another meaning to the words.  I particularly wish to avoid the meaning
%already in use for elliptic, parabolic, and hyperbolic elements of the special linear group
%$SL_2(\ring{R})$.    However, in the study of the Reinhardt problem,
%there is behavior related to the ellipse and hyperbola.  Moreover, there is a case that fall between the two.  With
%that in mind, we define {\it elliptic, parabolic, and hyperbolic types} as follows.

\begin{definition}[rank]\label{def:rank}
Let $\sigma$ be a multi-curve.  We say that
it has {\it well-defined rank} if the multi-curve is $C^2$,
parameterized by $[t_0,t_1]$, with
the property that for each curve $\sigma_j$ one of the two conditions
hold:
\begin{itemize}
\item It is a line segment.
\item The curvature of $\sigma_j$
is nonzero on the open interval $(t_0,t_1)$.
\end{itemize}
The {\it rank} of such a multi-curve is the number $k\in\{0,1,2,3\}$ of curves $\sigma_{j}$ ($j\in2\rZ/6\rZ$) 
that
are {\it not} line segments.
\end{definition}
For example, a multi-curve parametrizing a circle
has rank $3$.  The smoothed octagon is parameterized by finitely many
multi-curves of rank $1$.  

\begin{lemma}
No multi-curve has rank $0$.
\end{lemma}

\begin{proof}
The tangent lines to a multi-curve, by the argument in the proof of  Lemma~\ref{lemma:sl2-rein}, determines a balanced hexagon.  If the rank is zero,
the tangent lines, the balanced hexagon, and its midpoints are fixed.
Thus, the curve degnerates to a stationary curve at the fixed midpoints.
\end{proof}

It is natural to consider hexameral domains whose boundary is
parameterized by a finite number of  analytic multi-curves.  

\begin{lemma} Suppose that a hexameral domain $K$ has a parametrization
by  a finite number of analytic multi-curves $\sigma$.
Then $K$ also admits a parametrization by a finite number of triple
curves satisfying the hypotheses of Definition~\ref{def:rank}, each admitting
a well-defined rank.
\end{lemma}

\begin{proof} The curvature of an analytic curve vanishes identically,
or has at most finitely many zeroes on a compact interval $[t_0,t_1]$.
Subdividing the intervals at the finitely many zeroes, we may assume
that the only zeroes appear at the endpoints of the intervals.
\end{proof}




\section{Rank Three and the Ellipse}

The following sections analyze the multi-curves according to rank,
starting with rank three in this section.
The primary method will be the calculus of variations  
to search for a curve $\phi(t)$ in $SL_2(\ring{R})$ that 
minimizes the area of a hexameral domain $K$.  

\subsection{first variation}

We consider a  curve $\phi(t)\in SL_2(\ring{R})$, for
$t\in[t_0,t_1]$.  Form  corresponding
curves $\sigma_j(t) = \phi(t) u_j$ and $\sigma_{j+3}(t) = -\sigma_j(t)$,
for $j\in\ring{Z}/6\ring{Z}$ and $u_j$ satisfying Conditions~(\ref{eqn:uA}).
Consider the closed curve that follows the
 line segment from $(0,0)$ to $\sigma_j(t_0)$,  the curve $\sigma_j(t)$
from $t_0\le t\le t_1$, and then the line segment from $\sigma_j(t_1)$ to $(0,0)$.    Assume that this closed curve is simple, and let $I_j$ be the area
enclosed by the curve.  Set
$$I(\phi) =\sum_{j=0}^5 I_j.$$

Let 
$$
\phi(t) = \phi(t_0)\cdot \begin{pmatrix} \alpha(t) & \beta(t) \\ \gamma(t) & \delta (t) \end{pmatrix}.
$$
If we express the integrals $I_j$ in terms of $\phi$, a short calculation gives
\begin{equation}\label{eqn:area-int}
I(\phi) = 3\int_{t_0}^{t_1} (\alpha d\gamma - \gamma d\alpha) + (\beta d\delta - \delta d\beta).
\end{equation}



\begin{lemma}  Let $K$ be a miserly domain. 
Pick a $C^1$ parametrization $\phi(t)\in SL_2(\ring{R})$
of the boundary of $K$.  Suppose that for all $t\in[t_0,t_1]$, the
multi-curve associated with the curve $\phi$
has a well-defined rank $3$.  Then the first variation of $I(\phi)$ (with fixed boundary conditions) vanishes. 
\end{lemma}

\begin{proof}  Assume for a contradiction that the first variation is nonzero. Under the conditions necessary for the rank
to be defined, on any compact interval $[s_0,s_1]$ with $t_0 < s_0 < s_1 < t_1$,
the curvatures of the curves $\sigma_j$ are bounded away from zero. 
Thus, a sufficiently 
small $C^\infty$-variation  of the functional preserves the convexity condition.  We can assume the small variation
gives a simple curve.  By Lemma~\ref{lemma:sl2-rein}, the small variation is again the boundary
of a hexameral domain.  If the first variation is nonzero, the area of can be decreased,
holding the area of the balanced hexagon constant.  By Lemma~\ref{lemma:mid-min}, $K$ is not a miserly domain.
\end{proof}


Basic results about variations now imply that the Euler-Lagrange equations must hold on $(t_0,t_1)$.
As we are working in $SL_2(\ring{R})$, we may take the variation of the form
  \begin{equation}\label{eqn:ell-var}
   \exp(\epsilon X(t))\cdot \phi(t)
   \end{equation}
for some curve 
\begin{equation}\label{eqn:X}
 X(t)  = \begin{pmatrix} u(t) & w(t)+v(t) \\ w(t)-v(t) & - u(t)\end{pmatrix}
 \in \mathfrak{sl}_2(\ring{R}),
\end{equation}
the Lie algebra of $SL_2(\ring{R})$.
%$$
%\begin{array}{lll}
%0 &= (\delta^2 + \gamma^2)'\\
%0 &= (\alpha^2 + \beta^2)'\\
%0 &= (\gamma \alpha + \delta\beta)'
%\end{array}
%$$
The Euler-Lagrange equations are
$$
\begin{array}{lll}
0 &= (\delta^2 + \gamma^2)'\\
0 &= (\alpha^2 + \beta^2)'\\
0 &= (\gamma \alpha + \delta\beta)'\\
\end{array}
$$
%
with boundary conditions
$$
\begin{array}{lll}
1 &= \alpha(t_0) = \delta(t_0),\\
0 &= \beta(t_0) = \gamma(t_0)\\
\end{array}
$$
Integrating, we have 
$$
\begin{array}{lll}
 1 &= \delta^2 + \gamma^2\\
 1 &= \alpha^2 + \beta^2\\
 0 &= \gamma \alpha + \delta \beta.
\end{array}
$$
We also have the determinant condition $\alpha\delta-\beta\gamma=1$.
These are the defining conditions of a special orthogonal matrix.  Hence, there is a function $\theta:[t_0,t_1]\to\ring{R}$
such that
$$
\phi(t) = \phi(t_0)\cdot \begin{pmatrix} \cos\theta(t) & -\sin\theta(t) \\ \sin\theta(t) &
  \cos\theta(t) \end{pmatrix}.
$$
Thus, the curves $\sigma_j$ trace out arcs of an ellipse.
We summarize in the following lemma.

\begin{lemma}  Let $K$ be a miserly domain.  Suppose that some portion of the boundary
has well-defined rank $3$.  Then up to a special linear transformation, that portion of the boundary
consists of three arcs of a unit circle.
\end{lemma}


\subsection{second variation}

Next we study the second variation of the area functional $I(\phi)$.  For this,
we may confine our attention to the unit circle:
$$
\phi(t) = \begin{pmatrix} \cos t & -\sin t \\ \sin t & \cos t
\end{pmatrix}.
$$
Again, we consider variations of the form of Equations~(\ref{eqn:ell-var}) and (\ref{eqn:X}).
A calculation of the second variation around $\phi$ gives
$$
\int_{t_0}^{t_1} 4 u(t) w'(t) dt.
$$
%Where $w_1 = w+v$.
This does not have fixed sign.  Therefore, the solution to the Euler-Lagrange equations is a saddle point. 
It follows that an arc of a circle cannot form part of a miserly domain.

\begin{thm}  Let $K$ be a miserly domain.  Then there is no segment
of the boundary parametrized by a multi-curve of  rank  three.
\end{thm}



\section{Rank Two}


In this section, we consider the variation of a Reinhardt curve
of well-defined rank two. 
We prove that the first variation is never zero in the rank two situation.  This leads to the following theorem. 

\begin{thm}  Let $K$ be a miserly domain.  Then no segment of the boundary is parameterized by a rank two multi-curve.  In fact, the first variation in area is always non-zero (on the space of multi-curves).
\end{thm}

\begin{proof}  By applying a special linear transformation, we may assume that $\sigma_0$ moves along the line $y=-1$.  At any given time $t$, there is a skew transformation
\def\xm{\left(\begin{matrix} 1& x\\0&1\end{matrix}\right)}
\def\vc#1#2{\left(\begin{matrix}#1\\#2\end{matrix}\right)}
$$
\xm
$$
that makes the first coordinates of  $\sigma_1(t)$ and $\sigma_2(t)$ equal.   The vertices of the midpoint hexagon (after this transformation) are
$$
\pm (r-z,-1), \quad \pm (r,s),\quad \pm (r-z,1),
$$
with $z>0$, $r>z$, and $-1<s<1$.  This hexagon has area $\sqrt{12}$ provided
$$
z = 2 r - 2 \rho,
$$
where  $\rho = \sqrt3/2$.  We regard $r$ as a function of $z$ through this relation.
We may solve for the midpoints of this hexagon, then reapply the skew transformation to obtain the coordinates of $\sigma_j$.
We then have
$$
\begin{array}{lll}
\sigma_1(t) &= \frac12\xm\cdot\vc {2\rho}{-1+s},\\
\sigma_2(t) &= \frac12\xm\cdot\vc {2\rho}{+1+s}.\\
\end{array}
$$
The condition that $\sigma$ is a multi-curve forces
\begin{equation}\label{eqn:z}
x' (s^2 - 1) - s' z = 0.
\end{equation}
If $s'(t)=0$, then $x'(t)=0$ and the curve $\sigma_1$ is not regular at $t$, which is contrary to the definition of a multi-curve.  Thus, $s'$ is everywhere nonzero, and we can define $z$ in terms of $x$ and $s$ by the relation (\ref{eqn:z}).  We can pick the sign of $s'$ so that it is positive, and $z$ so that it is positive.  Then $x'$ is everywhere negative.

We let $I_j$ be the area bounded by the three curves: the segment from $0$ to $\sigma_j(t_0)$, the curve $\sigma_j$ on $[t_0,t_1]$, and the segment from $\sigma_j(t_1)$ to $0$.  A short calculation gives
$$
I_1 + I_2 = \int_{t_0}^{t_1} \frac{1}4 (\sqrt{3} s' - (1+s^2) x').
$$
The integral $I_0$ is not relevant for a compactly supported variation
since $\sigma_0$ remains linear under any small rank-preserving variation.  The first variation of this integral in $x$ clearly does not vanish,
under the assumption that $s'>0$.
\end{proof}

\section{Hyperbolic Links (Rank One)}

We have shown that the boundary of a miserly domain cannot
contain multi-curves of rank $\ne 1$.  This section analyses triple
curves of rank one.


\subsection{square representation}

Let $K$ be a hexameral domain with boundary parametrized by a multi-curve $\sigma$.
Suppose that for some index $j$, the curves $\sigma_{j+2}$ and $\sigma_{j+4}$ travel along straight lines.
By applying a special linear transformation, we may assume that $\sigma_{j+2}$ moves along
$x=a$ and $\sigma_{j+4}$ moves along $y=a$, for some $a>0$.  
By reparametrizing the curves, we may assume
that $\sigma_{j+2}(t) = a(1,t)$ and $\sigma_{j+4}(t) = a(s(t),1)$, for some function $s$.  The fixed area condition 
on balanced hexagons, 
\begin{equation}
{\sigma_{j+2}(t)}\land{\sigma_{j+4}(t)}=\frac{\sqrt{3}}2,
\end{equation}
gives $1- t s = k$, where $k=\sqrt3/(2a^2) > 0$.  This determines the function $s$.  The condition $\sigma_{j} = -\sigma_{j+2} -\sigma_{j+4}$
gives $\sigma_{j}(t) = a (-1-s,-1-t)$.  The curve $\sigma_{j}$ 
traces the hyperbola $(x+a)(y+a) = a^2(1-k)$,
whose asymptotes are lines $x=-a$ and $y=-a$ containing the curves $\sigma_{j+5}$ and $\sigma_{j+1}$.
(This calculation shows why hyperbolic arcs play a special role.)
The curve traced by $\sigma_{j}$ does not form the arc of a convex region containing the origin unless
   \begin{equation}
   0 < k < 1,\quad s<0,\quad t < 0.
   \end{equation}
We assume this condition.
Also, $k<1$ implies $a^2 > \sqrt3/2$.  The balanced hexagon $G$ degenerates
to a quadrilateral when $s \le -1$ or $t\le -1$.  We therefore assume
that 
\begin{equation}
-1<s < 0,\qquad -1 < t < 0.
\end{equation}
  This implies that
\begin{equation}
-1 < s < -t s = k-1,\qquad -1 < t < k-1.
\end{equation}  
The parameter $t$ thus ranges
over an interval $[t_0,t_1]$ with $-1 < t_0 < t_1 < k-1$.  The parameter
$s$ runs over $[s_0,s_1]$ with $s_i = (1-k)/t_i$.  We may
write 
\begin{equation}
t_1 = t_0 + \ta (k-1-t_0),
\end{equation}
for some $\ta\in (0,1)$.

In summary, up to a special linear transformation, and reparametrization of the curve, a rank one multi-curve is uniquely determined by the index $j$ of the hyperbolic arc,
the initial parameters $(a,t_0)$, and the terminal parameter $\ta$, where
  \begin{equation}\label{eqn:atu}
  a > \sqrt{3}/2,\quad  k = \sqrt{3}/(2a^2),\quad -1 < t_0 < k - 1,
  \quad 0 < \ta < 1.
  \end{equation}
We call this the {\it square representation} of the multi-curve.
The starting and terminal point $s_i$ for the curve $\sigma_{j+4}(t_i) = a(s_i,1)$ and the constant $k$ are determined by Equation~(\ref{eqn:AB}).  The curve $\sigma_{j}$, is determined by $\sigma_{j+2}$ and $\sigma_{j+4}$. 

Conversely, if we are given three parameters $a,t_0,\ta$ satisfying
the conditions (\ref{eqn:atu}), and given the hyperbolic index $j$, 
there exists a multi-curve whose
square representation has these parameters.  This gives us a convenient
way to construct multi-curves.  

\subsection{area}

Let $I_j(a,t_0,t_1)$ be the area bounded by the line segment from $(0,0)$ to $\sigma_j(t_0)$, the curve $\sigma_j$, $t_0\le t\le t_1$, and the line segment from $\sigma_j(t_1)$ to $(0,0)$.  An easy calculation gives
\begin{equation}\label{eqn:I}
\begin{array}{lll}
 I(a,t_0,t_1) &=
  a^2((s_0-s_1)+(t_1-t_0) - (1-k)\ln (s_1/s_0))\\
 &= a^2 ((1-k) (1/t_0-1/t_1) + (t_1-t_0) -(1-k) \ln (t_0/t_1)),
\end{array}
\end{equation}
where 
$I=I_0+I_2+I_4,$ and the parameters $s_1,s_2,k$ are given in terms
of $a,t_0,t_1$ as above.

For example, the boundary of the 
smoothed octagon is parameterized by eight multi-curves (one for
each hyperbolically rounded corner of the octagon).  The parameters
for each of the eight multi-curves of the smoothed octagon are
$$
\begin{array}{lll}
a &= \frac{12^{1/4}}{\sqrt{4-\sqrt{2}}},\\
t_0 &= -1/\sqrt{2},\\
t_1 &= -1/2,\\
I &= \frac{\sqrt{3} \left(8-8 \sqrt{2}+\sqrt{2} \log (2)\right)}{4
   \left(-4+\sqrt{2}\right)}\\
\end{array}
$$
This gives density
$$8 I/\sqrt{12} \approx 0.902414,$$
mentioned in the introduction to this article.

\subsection{the set of initial states}



The initial state for a multi-curve is specified by a matrix
$\phi(t_0)\in SL_2(\ring{R})$ and a velocity $X(t_0)\in\mathfrak{sl}_2(\ring{R})$, given by Equation~(\ref{eqn:Xt}).  We wish to allow reparametrizations of the curve, so that the velocity is
given only up to a scalar, giving a point in projective space: $[X(t_0)]\in\ring{P}(\mathfrak{sl}_2(\ring{R}))$.
The space of initial states then has dimension five:
  $$\dim\, S = 3 + 2,\quad\hbox{ where } S = SL_2(\ring{R}) \times\ring{P}(\mathfrak{sl}_2(\ring{R})).$$
These five dimensions correspond to the three dimensional group of
transformations that can be used to transform a rank-one multi-curve
to its square representation, together with the two parameters $(a,t_0)$
giving the initial state in the square representation.

Likewise, the terminal state for a multi-curve is given by a point
in the same five dimensional space. 





\subsection{hyperbolic chains and smoothed polygons}

\begin{definition}
An analytic multi-curve of rank one is called a {\it hyperbolic link} (because the of hyperbolic arc $\sigma_j$).  A piecewise analytic
multi-curve of rank one is called a 
{\it hyperbolic chain}.
A hexameral domain $K$ whose boundary is a hyperbolic chain is
 is called a smoothed polygon.
\end{definition}

We consider a multi-curve that consists of a finite number of rank one triples, joined one to another to form $C^1$ curves.
There is no variational problem here, because there are no functional degrees of freedom for segments of rank one.  When the an initial state for a 
hyperbolic link is fixed, there is exactly one degree of freedom, 
the parameter $\ta\in(0,1)$
in the square representation.

 Suppose that we have a multi-curve $\sigma$
consisting of a finite number of hyperbolic links.    Along each hyperbolic link,
exactly one of the three curves $\sigma_j$ follows a hyperbolic arc;
the other two are linear.  Set $\Delta=2\ring{Z}/6\ring{Z}$.
We break the domain of the multi-curve
into finitely many subintervals, each labeled with an index
$j\in\Delta$, according to which arc $\sigma_j$ is the hyperbola.
The first link of $\sigma$
is entirely specified by $(s_1,\ta_1,j_1)$, where $s_1\in S$ is an initial state,
$\ta_1\in(0,1)$ determines the length of the arc, and $j_1\in\Delta$.  
In particular, $(s_1,\ta_1,j_1)$
determine the terminal state $s_2\in S$
of the first hyperbolic link.  Similarly, 
the second link of $\sigma$ is determined
by $(s_2,\ta_2,j_2)$, for some $\ta_2\in (0,1)$ and $j_2\in\Delta$.  Working through the hyperbolic
chain, link by link, we obtain a sequence
\begin{equation}\label{eqn:param}
  s_0\in S,\quad ((\ta_0,j_0),(\ta_1,j_1),\ldots,(\ta_n,j_n)),\quad \ta_i\in(0,1),\quad j_i\in\Delta
\end{equation}
that uniquely determines the hyperbolic chain.

Conversely, given a sequence of parameters (\ref{eqn:param}),
an  induction on $n$ shows that there is a unique hyperbolic chain
with those parameters.  
We can extend the parameters $\ta\in (0,1)$ to include $\ta=0$,
with the understanding that this corresponds to a degenerate link consisting of a single point.  If two consecutive parameters $(\ta_i,j)$ and 
$(\ta_{i+1},j)$ have the same index $j\in\Delta$, then they can be
combined into a single hyperbolic link.  Thus, there is no loss in
generality in assuming that consecutive 
links carry distinct hyperbolic indices $j$.  In fact,
we may insert degenerate parameters $(\ta,j)$, with $\ta=0$ so that
the parameters take the special form
\begin{equation}\label{eqn:ji}
  (\ta_i,j_i),\hbox{ where } j_i = j_0 + 2 i.
\end{equation}

If we are given a hyperbolic chain $\sigma$ 
with parameters (\ref{eqn:param}),
we may extract the square representation $(a(i),t_0(i),t_1(i))$
of each
link $i$ from these parameters.  We may then sum Equation (\ref{eqn:I})
over the set of links to obtain the area
\begin{equation}\label{eqn:Isigma}
I(\sigma) = I(s_0,((u_0,j_0),\ldots)) = \sum_i I(a(i),t_0(i),t_1(i))
\end{equation}
represented by the entire chain.

\subsection{closed curves}

Consider a multi-curve $\sigma$ that gives the boundary of a hexameral domain.  In the circle
representation, 
write $\sigma_j(t) = \phi(t)u^*_j$.  We have
$$u^*_{j+1} = \rho u^*_j,\quad \rho=\left(\begin{array}{ccc} \cos\theta & -\sin\theta\\ \sin\theta & \cos\theta\end{array}\right),\quad \theta=\pi/3.$$
Let the initial point of the curve be given at time $t_0$ and let $t_1>t_0$ be the first time
at which $\sigma_0(t_1) = \phi_1(t_0)$:
\begin{equation}\label{eqn:t0}
t_1 = \min \{ t \mid \sigma_0(t) = \phi_1(t_0), \quad t>t_0\}.
\end{equation} 
Then for the boundary of the hexameral domain to be closed
and $C^1$, we must have
\begin{equation}\label{eqn:closed}
\phi(t_1) = \phi(t_0)\rho\quad\hbox{ and }\quad X(t_1) = X(t_0).
\end{equation}
The angular argument must also satisfy
\begin{equation}\label{eqn:arg}
0\le \arg (\phi(t_0)^{-1}\phi(t)u^*_0) \le \pi/3,\quad t\in[t_0,t_1].
\end{equation}

Given a choice $u$ of multi-point on the boundary, the parameters $s\in S$ and $(\ta_i,j_i)$
are uniquely determined.  Conversely, the parameters uniquely determine the boundary.

When we choose to do so, we may pick the starting multi-point on
the boundary in such a way
that the first parameter is
\begin{equation}\label{eqn:j0}
(\ta_0,j_0) \hbox{ with } j_0=0.
\end{equation}  
We may also arrange
that the multipoint is the endpoint of a hyperbolic link.

\begin{definition}[link length]
Assume that the boundary of a hexameral domain is a hyperbolic
chain that satisfies conditions (\ref{eqn:j0}), (\ref{eqn:ji}).
Let $t_0,t_1$ be given as in (\ref{eqn:t0}).  We may extend
$\sigma_j$ to a periodic function $\rR\to\rR^2$.  We may extend
the parameters $(\ta_i,j_i)$ to all of $i\in\rZ$ with $j_i = 2i$.
The curve $\sigma_j$, restricted to $[t_0,t_1]$ has parameters
$$
((\ta_0,0),(\ta_1,2),\ldots,(\ta_n,2n)),
$$
for some $n$.  When $n$ is chosen to give the shortest representation of this form, we call $n+1$ the {\it link length} of the hexameral domain.
\end{definition}

For example,
the boundary of the smoothed octagon contains
eight hyperbolic arcs, one at each corner of the octagon. They appear
in centrally symmetric pairs.
There exists a choice of initial multi-point such that the smoothed octagon is by the following data:
$$
(I,[X])\in S,\quad ((\ta,0),(\ta,2),(\ta,4),(\ta,0)),
$$
for some $\ta>0$ that is independent of the link and some
$X\in\mathfrak{sl}_2(\rR)$.
The link length is $4$.

\begin{lemma}\label{lemma:link}  Let $K$ be a hexameral domain whose boundary is
a hyperbolic chain.  Let $n+1$ be the link length of $K$.  Then
$$n\equiv 0 \mod 3.$$
(In particular, the smoothed octagon minimizes the link length.)
\end{lemma}

\begin{proof} If we consider the multi-curve $\sigma$, the link
$(\ta_{n+1},2n+2)$ lies along the same part of the multi-curve as $(\ta_0,0)$,
so that $\ta_{n+1}=\ta_0$.  However, the hyperbolic index shifts by
$1$ as we pass from $\sigma_0$ to $\sigma_1$.  Therefore,
$$(\ta_{n+1},2n+2) = (\ta_0,2)\in \rR\times \Delta.$$
The congruence $$2n+2 \equiv 2\in \Delta = 2\rZ/6\rZ$$
gives the result.
\end{proof}

\subsection{link reduction}

With this background in place, we now return to a discussion of the
Reinhardt conjecture.  Lemma~\ref{lemma:link} shows that the
conjectural solution to the Reinhardt problem minimizes link
length.  This leads to the intuition 
that decreases in $\deltalat(K)$ should have
the effect of shortening the link length.  This is the motivation
for the Conjecture~\ref{conj:defrag}, which asserts that we may
simultaneously decrease areas and eliminate links
from some hyperbolic chains.   

We consider all possible hyperbolic chains $\sigma$ with the same
initial and terminal states $s_0,s_1$.  These chains are given by
parameters
  $$
  s_0,\quad ((\ta_1,j_1),\ldots).
  $$
Since $s_1$ lies in a five-dimensional space, the terminal state
places five constraints on the parameters set $(\ta_1,\ta_2,\ldots)$.  
By counting dimensions, we
might guess that for generic parameters $s_0,s_1$, the terminal
state cuts out a set of hyperbolic chains of codimension five.
Generically, it should take at least five links to match the terminal state $s_1$.

If, instead of fixing both endpoints, we may impose the closed
curve condition (\ref{eqn:closed}). We may use the action of
the special linear group to force $\phi(t_0)=I$.  The free
parameters are $X$ and $(\ta_0,\ldots,\ta_n)$, or $n+3$ free
variables.  


Wishful thinking leads to the following conjecture.  I am not confident
of its exact form, but something along these lines should hold.

\begin{conj}[Link Reduction] \label{conj:defrag}
Let $K$ be a hexameral domain with multi-curve $\sigma$ around the
boundary. 
Let $t_0,t_1$ be the parameters (\ref{eqn:t0}).  Suppose
that a portion $[t'_0,t'_1]$ of the boundary (with $t_0\le t'_0\le t'_1\le t_1$) is a hyperbolic chain with six
links, given in the form (\ref{eqn:param}).  (We do not assume (\ref{eqn:j0}), (\ref{eqn:ji}).)  Let $s_0,s_1\in S$ be the initial and terminal states
for the curve at $t_0'$ and $t_1'$.
If the sequence of parameters $j_i$ is any of the three patterns
that follow, there is another hyperbolic chain with five links
that 
\begin{itemize}
\item fits the same initial and terminal states $s_0,s_1\in S$,
\item satisfies the angle condition (\ref{eqn:arg}),
\item has parameter sequence $j'_i$ given by the indicated pattern,
\item has no greater area $I(\sigma)$ as defined by (\ref{eqn:Isigma}),
\item and in fact has strictly smaller area, unless the chain is already (degenerately) a 
five-link chain with the parameter sequence $j'_i$.
\end{itemize}
The three patterns are:
$$
\begin{array}{llllllllllllllllll}
j_i: &4&0&2&4&0&2\\
j'_i: &4&0&2&-&0&2
\end{array}
~\left|~
\begin{array}{llllllllllllllllll}
4&0&2&0&2&4\\
4&-&2&0&2&4
\end{array}
~\right|~
\begin{array}{llllllllllllllllll}
0&2&4&2&0&2 \\
0&2&4&-&0&2
\end{array}
$$
\end{conj}

In other words, the conjecture claims that one of the links can degenerate, decreasing the number of links to five, while simultaneously
decreasing area.

The conditions on the parameters $t_0,t_1,t'_0,t'_1$ are there
to insure that the hyperbolic chain is short enough that the corresponding curves $\sigma_0,\ldots,\sigma_5$ parametrize distinct portions
of the boundary.

The condition that the hyperbolic chain should bound part of a hexameral
domain places constraints on $s_0,s_1$.  They cannot be arbitrary
states of $S$.

The number of parameters in the implied optimization is eight:
six links and the five-dimensional initial state, reduced by the three dimensional group $SL_2(\rR)$.
This conjecture may be explored by computer, but I have only done
so in a very limited way.

\bigskip

The following is a very special case of the Reinhardt conjecture.

\begin{conj}[Five Link]\label{conj:5} 
Let $K$ be a hexameral domain with multi-curve $\sigma$ around the
boundary. 
Let $t_0,t_1$ be the parameters (\ref{eqn:t0}).  Suppose
that the entire non-repeating boundary $[t_0,t_1]$ is a hyperbolic chain with five
links, given in the form 
$$
(\ta_0,0),(\ta_1,2),(\ta_2,4),(\ta_3,2),(\ta_4,0).
$$
Then $\deltalat(K)= \delt$ exactly when
$K$ is the smoothed octagon, up to 
a transformation by $SL_2(\rR)$.
\end{conj}

The smoothed octagon belongs to this family of hexameral
domains, with parameters $\ta_3=0$ and $\ta_i=\ta_j$, if $i,j\ne 3$.

Nazarov's proof of the local optimality of the smoothed octagon
gives the conjecture for hexameral domains sufficiently close
to the smoothed octagon \cite{N}.

This is an optimization problem on a seven dimensional space.
(There is the five dimensional initial state $s\in S$ and
five links $(\ta_i,j_i)$, reduced by the action of the three-dimensional
group $SL_2$.)  For a generic choice of parameters $s,\{(\ta_i,j_i)\}$
the hyperbolic chain will not form a simple closed curve, and can
be discarded.  

Again, we might hope to test this conjecture by computer, and perhaps
even to prove it with interval arithmetic.

\begin{lemma}\label{lemma:smooth}  Assume Conjectures~\ref{conj:defrag} and~\ref{conj:5}.
Then up to affine transformation, the smoothed octagon uniquely
minimizes $\deltalat(K)$ over the class of all smoothed polygons $K$.
\end{lemma}

\begin{proof}  Represent the boundary as in (\ref{eqn:param}),
(\ref{eqn:ji}), (\ref{eqn:j0}).  The parameters $j_i$ are then
$(0,2,4,0,2,4,0,2,4,\ldots)$.  Recall that the link length $n+1$
satisfies
$n\equiv 0\mod 3$.  If the link length is at least nine,
we can apply Conjecture~\ref{conj:defrag} three times to eliminate
three links and preserve the form of (\ref{eqn:ji}):
$$
\begin{array}{llllllllllll}
 \ldots &0&2 & 4 & 0 & 2 & 4 & 0 & 2 &4&\ldots \\
 \ldots &0&2 & 4 & 0 & 2 & - & 0 & 2 &4&\ldots \\
 \ldots &0&2 & 4 & - & 2 & - & 0 & 2 &4&\ldots \\
 \ldots &0&2 & 4 & - & - & - & 0 & 2 &4&\ldots \\
\end{array}
$$
We repeat until the the link length is seven.  After two more
applications of Conjecture~\ref{conj:defrag}, there remain
five links in the state described by Conjecture~\ref{conj:5}.
This state also includes as a degenerate case 
all hyperbolic chains of link length four.
The result follows.
\end{proof}


\section{Bolza}


Viewed as a problem in the calculus of variations or control theory, the Reinhardt problem may be catalogued as an example of the classical problem of Bolza with nonholonomic inequality constraints, autonomous (no explicit time dependence), fixed endpoints, and no isoperimetric constraints.  Lacking isoperimetric constraints, it falls within the subclass of problems known as the classical problem of Mayer with inequality constraints~\cite[Ch.7]{He}.    In this section, we consider the Reinhardt problem as an example in this framework.

According to Tonelli's direct method for problems in the calculus of variations, one first establishes the existence of a minimizer, then shows that it has sufficient regularity to meet the hypotheses of various theorems describing necessary conditions (such as the Euler-Lagrange equation) for extremal solutions.  The necessary conditions are used to limit the possibilities, and the minimizer is selected from among these possibilities.  

We have established the existence of a minimizer with Lipschitz continuous derivative (when considered as a second-order system;  the minimizer itself is Lipschitz continuous when converted to a first-order system).  This is sufficiently regular to match the hypotheses in Pontryagin's treatise~\cite{Po}.  In particular, the convexity constraints con be included through Lagrange multipliers.


We search for minimizers of the integral (\ref{eqn:area-int}):
\begin{equation}
I(\phi) = 3\int_{t_0}^{t_1} (\alpha d\gamma - \gamma d\alpha) + (\beta d\delta - \delta d\beta).
\end{equation}
over the class of curves $\phi:[t_0,t_1]\to SL_2(\ring{R})$ such that 
\begin{itemize}
\item the special linear condition holds: $\alpha\delta-\beta\gamma=1$, where
 $$\phi(t) = \phi(t_0)\left(\begin{matrix}\alpha&\beta\\\gamma&\delta\end{matrix}\right).$$
\item $\phi$ is $C^1$
\item The derivative $\phi'$ is Lipschitz continuous.
\item Convexity constraints hold:
  $$\sigma_j'(t) \land \sigma_j''(t) \ge 0,\quad j=0,2,4,\quad t\in[t_0,t_1]~a.e.$$
where $u^*_j$ is as in (\ref{eqn:roots}) and $\phi(t_0)\sigma_j(t) = \phi(t) u^*_j$.
\end{itemize}
By the Lipschitz condition on $\phi'$, the second derivatives appearing in the convexity constraints exist almost everywhere.  These second derivatives are Lebesgue integrable and the integral over an interval is the change in the first derivative.



The variational problem has several symmetries, which lead to conserved quantities by Noether's theorem.  The area and curvature constraints are invariant under reparametrization of $\phi:[t_0,t_1]\to SL_2(\ring{R})$.  The problem is autonomous.  Finally, the group $SL_2(\ring{R})$ acts on the set of minimizers.

We may convert to a first order problem by setting $\phi(t_0)Y = \phi'$.  The convexity
constraints become linear in $Y'$:
\begin{equation}\label{eqn:curvature}
  y_j \land y_j' \ge 0,
\end{equation}
where $y_j = Y u^*_j$.
[D'Alembert's principle applies to systems with nonholonomic constraints which are linear in the velocity terms.]

\subsection{Hestenes rank condition}

We show that the Reinhardt problem satisfies the rank condition of Hestenes~\cite[7.11]{He}.

\begin{lemma}  Let $\phi:[t_0,t_1]\to SL_2(\ring{R})$ have a Lipschitz
continuous derivative.  For every $t$, there exists a $j$ such that the curvature constraint (\ref{eqn:curvature}) is a strict inequality.
\end{lemma}

\begin{proof}  %(Done in Mathematica.)  
This is a small modification of the fact that there are no rank $0$ solutions.  Without of loss of generality, after applying an affine transformation, let the boundary be given by
$\phi u^*_j$, with $\phi(0)=I$.  Let $\phi' u^*_j = X\phi u^*_j$.
The sign of the curvature at $t=0$ is given by the sign of
$$
\phi' u^*_j \land \phi'' u^*_j = X u^*_j \land (X' + X^2) u^*_j.
$$

A short calculation
assuming the vanishing of curvature for $j=0,2$ gives for $j=4$:
$$
X u^*_4 \land (X' + X^2) u^*_4 = \frac{3\sqrt{3} (a^2 + b c)^2}{3 a + \sqrt{3} c}.
$$
The star conditions in Section~\ref{sec:star} imply that the numerator and denominator are both positive.

\end{proof}


By our choice of initial conditions $\alpha(t_0)=\delta(t_0)=1$; $\beta(t_0)=\gamma(t_0)=0$.  Thus, $\alpha$ is nonzero on some sufficiently small time
interval, and the constraint $\alpha\delta-\beta\gamma=1$ may be solved
for $\delta$.  In this way $\delta$ (and $\delta'$)
may be eliminated from the system of equations.

[XX] This section needs to be completed.  Prove the Hestenes rank conditions.  Give Pontryagin's version of the Euler-Lagrange equations with multipliers (as derived from the Pontryagin maximum principle), modified as in Hestenes (or Clarke) to deal with inequality constraints rather than equality.  Write down the conservation laws that come from Noether's theorem.  (Weierstrass gives the form the equations take when the problem is independent of reparametrization.) Write down the Weierstrass $E$ function.  The hope is when all this is written out, the conclusion will be that the optimal solution must be a hyperbolic chain, as treated in an earlier section. The Weierstrass $E$-function may even put constraints on the form of the hyperbolic chain, although I am not counting on this.











\begin{thebibliography}{}



\bibitem{E}  V. Ennola, On the Latice Constant of a Symmetric Convex
Domain, J. London Math. Soc. 36 (1961), 135--138.

\bibitem{FT} L. 
Fejes T\'oth,  Lagerungen in der Ebene, auf der Kugel und in Raum, 2nd ed. Berlin: Springer-Verlag, 1972.
%p104.

\bibitem{FT50} L. Fejes T\'oth, Some packing and convering theorems, Acta Sci. Math.
(Szeged) 12 (1950), pp. 62--67.

\bibitem{G} Gardner, M. "Packing Spheres." Ch. 10 in The Colossal Book of Mathematics: Classic Puzzles, Paradoxes, and Problems. New York: W. W. Norton, pp. 128-136, 2001.
%"In a 1972 personal communication to Martin Gardner, Ulam conjectured that in% their densest packing, spheres allow more empty space than the densest pack%ing of any other identical convex solids" (Gardner 2001, p. 135).
% Weisstein, Eric W. "Sphere Packing." From MathWorld--A Wolfram Web Resource. http://mathworld.wolfram.com/SpherePacking.html

\bibitem{GZ} E. Grinberg and G. Zhang, Convolutions, Transforms, and Convex Bodies.
% Convolutions_transforms_convex_bodies   (pdf on laptop)

\bibitem{He} M. R. Hestenes, Calculus of Variations and Optimal Control Theory, John Wiley and Sons, New York, 1966.

\bibitem{J} I. Juh\'asz, Research Problems, Periodica Math. Hungarica,
14, 1983, 309--314. 

\bibitem{LM}  W. Ledermann, K. Mahler, On lattice points in a convex
decagon,  % mahler.pdf on computer.

\bibitem{MH}  K. Mahler, The Theorem of Minkowski-Hlawka, Duke Math. J. 13 (1946), pp611-621.

\bibitem{M47a}  K. Mahler, On the area and the densest packing of convex domains,
Proc. Koninkl. Nederl. Akad. Wet., 50, 108--118 (1947).

\bibitem{M47b} K. Mahler, On the minimum determinant and the circumscribed hexagons of convex domain, Proc. Konink. Nederl. Akad. Wet., 50 (1947).

\bibitem{N}  F. L. Nazarov, On the Reinhardt problem of lattice packings
of convex regions, Local extremality of the Reinhardt octagon,
J. Soviet Math 43(5) (1988), 2687--2693.


\bibitem{AP} J. Pach and P. K. Agarwal, Combinatorial Geometry, Wiley, New York, 1995.

\bibitem{Po} L. S. Pontryagin et al., The Mathematical Theory of Optimal Processes, Selected Works IV., Gordon and Breach Science, 1986.

\bibitem{R} K. Reinhardt, \"Uber die dichteste gitterf\"ormige Lagerung
kongruenter Bereiche in der Ebene und eine besondere Art konvexer Kurven, Abh. Math. Sem., Hamburg, Hansischer Univ., Hamburg 10, 216-230, 1934.

\bibitem{T} P. P. Tammela, A bound on the critical determinant of a two-dimensional convex symmetric domain,  Izv. Vyssh. Uchebn. Zaved., Mat. No. 12 (103), 103--107 (1970).




\end{thebibliography}

\svninfo

\end{document}

