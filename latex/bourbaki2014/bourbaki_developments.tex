% Bourbaki talk.
% Developments in Formal Proofs
% Author Thomas C. Hales
% File created May 12, 2014.


\documentclass[brochure,english,12pt]{bourbaki}
\usepackage[matrix,arrow]{xy}
\usepackage{amssymb,amsfonts,amsmath,footnote}
\usepackage[francais]{babel}
\usepackage{MnSymbol}
\usepackage{url}
\usepackage{listings}
%\usepackage{pxfonts}
\addressindent 100mm



\date{June 2014}
\bbkannee{6x\`eme ann\'ee, 2013-2014}
\bbknumero{10xx}
\title{Developments in Formal Proofs}
\author{Thomas C. HALES}
\address{University of Pittsburgh\\
Department of Mathematics\\
Pittsburgh, PA 15260-2341\\
U.S.A.\\}
\email{hales@pitt.edu}

% sectioning.
\theoremstyle{plain}
\newtheorem{example}[equation]{Example}
\newtheorem{definition}[equation]{Definition}
\newtheorem{theorem}[equation]{Theorem}
\newtheorem{lemma}[equation]{Lemma}
\newtheorem{corollary}[equation]{Corollary}


% Math notation.
\def\op#1{{\operatorname{#1}}}
%\newcommand{\ring}[1]{\mathbb{#1}}
\def\ring#1{{\mathbb{#1}}}


\def\AD{\ring{A}}
%general
\def\b{\backslash }
%\def\mass{{\mathbf\mu}}
\def\card{\op{card}}
\def\a{{\scriptsize\text{ani}}}
\def\good{{\scriptsize\text{good}}}
\def\diamond{{\blacklozenge}}
\def\SO{{\mathbf {SO}}}
\def\OO{{\mathbf O}}
\def\st{{\text{st}}}


%frak
\def\so{\mathfrak{so}}
\def\sp{\mathfrak{sp}}
\def\gl{\mathfrak{gl}}
\def\sl{\mathfrak{sl}}
\def\g{\mathfrak{g}}
\def\t{\mathfrak{t}}
\def\cc{\mathfrak{c}}
\def\DIV{{\mathfrak{D}}}
\def\RDIV{{\mathfrak{R}}}


%mathcal
\def\A{{\mathcal A}}
\def\T{{\mathcal T}}
\def\M{{\mathcal M}}
\def\P{{\mathcal P}}
\def\O{{\mathcal O}}
\def\tA{{\tilde{\mathcal A}}}
\def\tP{{\tilde{\mathcal P}}}
\def\tM{{\tilde{\mathcal M}}}
\def\tU{{\tilde{\mathcal U}}}






\begin{document}



\maketitle



{

\narrower{\it Si la math\'ematique formalis\'ee \'etait
aussi simple que le jeu
d'\'echecs, \ldots il n'y aurait plus qu'\`a r\'ediger nos d\'emonstrations dans ce
language, comme l'auteur d'un trait\'e d'\'echecs \'ecrit dans sa notation.\ldots
Mais les choses sont loins d'\^etre aussi faciles, et point n'est besoin d'une
longue pratique pour s'appercevoir qu'un tel projet est absolument irr\'ealisable.
 -- N. Bourbaki, 1966} % page 5. T. d. Ensembles.




}

\bigskip



A proof assistant is interactive computer software that humans use to prepare scripts of mathematical proofs.
These proof scripts can be parsed and verified, directly from
the fundamental rules of logic and the foundational axioms of mathematics.  
The
technology underlying proof assistants and formal proofs has been under development for decades and grew out
of efforts in the early twentieth century to place mathematics on solid foundations.
Various proof assistants have been built upon various mathematical foundations, including
Zermelo-Fraenkel set theory (Mizar), Higher Order Logic (HOL), and dependent type theory (Coq)
\cite{Mizar}, \cite{HOL}, \cite{Coq}.
A {\it formal proof} is one that has been verified from first principles (generally by computer).  


This report
will focus on three proof particular assistants -- HOL, Coq, and Mizar -- and one recent technological advance for each of them.
The HOL Light proof assistant illustrates the design of a highly reliable system.
The Coq proof assistant will be used to show how advanced mathematics can be developed.  In particular, we will
discuss the formal proof in Coq of the Odd Order theorem in group theory.
Finally, we will turn to the the Mizar proof assistant to describe recent advances in the automation of formal proofs.

%\subsection{origins of the fundamental lemma (FL)}\label{sec:origin}





\section{Building a trustworthy system with HOL Light}

\def\bool{\op{bool}}
\def\Fun{\op{Fun}}

HOL Light is a lightweight implementation of a foundational system
based on Higher Order Logic (HOL).  Because it is such a lightweight system,
it is a natural system to use for explorations of the reliability of formal proof assistants.



\subsection{Naive type theory}



The foundational system of mathematics, HOL, that we describe in this section is based
on a $\lambda$-calculus with a simple type theory.  This subsection describes a simple type theory in
naive terms.  

A salient feature of set theory is that it is so amorphous; everything is a set:
ordered pairs are sets, elements of sets are sets,
and functions between sets are sets.
Thus, it is a meaningful in set theory to ask bizarre questions such as whether 
a Turing machine is a minimal surface.  In type theory,
the very syntax of the language prohibits such questions; and this helps to keep computers in line.

Naively, a simple type system is a countable collection of disjoint nonempty sets called types.
\footnote{We restrict our attention to a simple type theory; for dependent type theory.}
The collection of types satisfies a closure property: for every two types $A$ and $B$,
there  is a further type, denoted $A\to B$, that can be naively identified with the set of functions from
$A$ to $B$.  

In addition to types, there are terms, which are naively thought of as elements of types.
Each term $x$ has a unique type $A$.  This relationship between a term and its type is denoted $x:A$.
In particular, $f:A\to B$ denotes a term $f$ of type $A\to B$.

There are variables that range over types called {\it type variables}, and another
collection of variables that run over terms.


\subsection{models of HOL}

The naive interpretation of types as sets can be be made precise.
We build a model of HOL in Zermelo-Fraenkel-Choice (ZFC) set theory, to see that HOL is consistent assuming that ZFC is.
In this section, we review this routine exercise in model theory.  
At the same time, we will give some indications of the structure of HOL Light.
See Appendix XX for details of the design of HOL Light.

%The interpretations we construct will map every type of HOL to a nonempty set and every
%term of HOL to an element of a set.

%For mathematicians with a background in ZFC, it can be very helpful at first to think of the
%set of
%types of HOL as a countable collection of nonempty sets and the terms
%as elements of those nonempty sets.

\subsubsection{an interpretation of types as sets, terms as elements of sets}

The interpretation
of variable-free types as sets is recursively defined.  We use a superscript $M$ to mark the interpretation
of a type as a set.  Specifically, the types in HOL are generated by
the boolean type $\bool$ (which we interpret as a set $\bool^M=\{\downvdash ,\upvdash \}$ 
of cardinality two with labeled elements representing true and false),
the infinite type $I$ (which we interpret as a countably infinite set $I^M$),
and then recursively for any two variable-free types $A$ and $B$, the type $A\to B$ is interpreted
as the set $\Fun(A,B)^M$ of all functions from $A^M$ to $B^M$.
We can arrange that the sets interpreting these types are all disjoint.

%The interpretation extends to types with variables in the usual way by 
%parametrizing the interpretation over an valuation $v$ of variables to variables-free types.
%To interpret terms, we will also parametrize the interpretation $M$ by extending $v$ to
%include an
%valuation of term variables (of type $A$) to elements of the set $A^{M,v}$.

In summary so far,
we fix an interpretation $M$, determining a countable collection $\T =\{A^M\}$ of nonempty sets in 
ZFC.  
%HOL has two kinds of variables: type variables which appear in types and term variables
%which are terms.  
We now extend our interpretation $M$ to a valuation $v=(M,v_1,v_2)$, where $v_1$ is a function
from the set of type variables in HOL to $\T$, and $v_2$ is a function from the set of term
variables in HOL to $\bigcup\T$.
%We do this in the usual way by parameterizing the interpretation over an valuation $v$ that
%handles the variables.
%The valuation
%$v$  has two parts.  The first part of $v$ is a mapping
%from the set of type variables to $\T$; 
%and the second part is a mapping from the set of term variables to
%the union  $\bigcup\T$.
The valuation $v$ (or $v_1$) extends recursively to give a mapping that assigns
a set $A^{v}\in \T$ to every type $A$.  We require $v_2$ to be chosen so that whenever $x$ is a term variable of type $A$,
then $x^{v_2} \in A^v$.  The valuation $v$ extends recursively to give a mapping on all terms:
\[
t\mapsto t^v \in A^v \in \T, \quad \text{for all } t:A.
\]

For example,
for every type $A$,
there is a HOL term $(=)$ of type $A\to (A\to \bool)$ representing equality for that type. 
This term is interpreted
as the function in $\Fun(A,\Fun(A,\bool))^{v}$ that
takes $a\in A^{v}$ to the delta function $\delta_a$ supported at $a$
(where the support of the function means 
the preimage of $\downvdash$).\footnote{The convention in HOL is to curry functions: using the bijection
$X^{Y \times Z} = (X^Z)^Y$ to write a function whose domain is a product as a function
of a single argument taking values in a function space. In particular, 
equality is a curried function of type $\A\to (A \to\bool)$
rather than a relation on $A\times A$.}
%If $f$ has type $A\to B$ and $x$ has type $A$, there is a term
%$f(x)$, which can be interpreted as the value of the function $f^{v}\in \Fun(A,B)^{v}$ 
%at $x^{v}\in A^{v}$.
%If $x$ is a variable of type $A$, 
%and if $p$ is a term of type $B$, there is a term 
%$\lambda x.~ p$ of type $A\to B$, which can be interpreted as a function in
%$\Fun(A,B)^{v}$, defined to map $a\in A^{v}$ to the element $p^{v[a/x]}\in B^{v}$, where
%$v[a/x]$ is the modification of the valuation $v$ that sends $x$ to $a$, and that remains
%otherwise the same.


\subsubsection{sequents}

A {\it sequent} is a pair $(L,t)$, traditionally written $L\vdash t$, where $L$ is a finite set of terms
called the {\it assumptions}, and $t$ is a term called the {\it conclusion}.  The
terms of $L$ and $t$ must all have type $\bool$.   If $L$ is empty, it is omitted from the
notation.  A {\it theorem} in HOL is a sequent that is generated from the mathematical axioms
and rules of logic.  

\begin{theorem} HOL is consistent.
\end{theorem}

\begin{proof}[Proof sketch]
We give the proof in ZFC.  Here, HOL is treated purely syntactically as a set of strings in 
a formal language.

If $L$ is a finite set of boolean terms, and if $v$ is a valuation extending $M$,  
write $L^{v}$ for the corresponding set of elements of the set $\bool^M$.
We say a sequent $L \vdash t$ is {\it logically valid}
if for every valuation $v$ for which every element of $L^{v}$ is $\downvdash \in \bool^M$, we also have
$t^{v}=\downvdash $ in $\bool^M$.

We run through the rules of logic of HOL one by one and check that each one preserves
validity.  For example, the reflexive law of equality in HOL states that for any term $t$
of any type $A$, we have a theorem $\vdash t = t$.  By the interpretation of equality described above, under
any valuation $v$, this equality is interpreted as the value in $\bool^M$ of the delta function: 
$\delta_{t^{v}}(t^{v})$, which is $\downvdash $.  Hence the reflexive law preserves validity.
The other rules (transitivity of equality, and so forth) are checked similarly.

We may well-order
 each set in the collection $\T$. In what follows, we 
assume that this
has been done.  HOL posits a choice operator of type $(A\to\bool)\to A$ for every type $A$.  The well-ordering  
allows us to 
interpret HOL's choice operator, as a functional
in $\Fun(\Fun(A,\bool),A)^{v}$, 
which maps a function $f\in \Fun(A,\bool)^{v}$ with nonempty support to
the minimal element of its support.

We run through the mathematical axioms of HOL and check their validity. 
There are only three.  The {\it axiom
of infinity} positing the existence of an infinite type $I$
is logically valid by our requirement to interpret $I$ as a countably infinite set.  The {\it axiom of
choice} is logically valid by the well-ordering we have placed on the sets $A^M$.  The {\it axiom
of extensionality} also holds in this model, because it holds for sets.
Thus, every axiom is logically valid.  
Since all axioms are logically valid and every rule of inference
preserves validity, every theorem is logically valid.

There is a boolean constant $\op{\tt FALSE}$ in HOL.
An easy calculation based on its definition gives that $\op{\tt FALSE}^v = \upvdash$ for every valuation $v$
extending the interpretation $M$.  Hence,
$\vdash \op{\tt FALSE}$ is not a logically valid sequent, and hence not a theorem.
This proves HOL consistent.
\end{proof}

\subsection{Computer implementation}

The bare consistency proof is just the beginning.  We can push matters  much further when
the logic is implemented in computer code.

The HOL Light system is implemented in the Objective Caml programming language, which is one of the many dialects of the ML
language.  The language ML (an acronym for Meta Language) was originally designed as a
meta language to automate mathematical proof commands \cite{Gor}.  It is significant that the development
of the ML language and the development of proof assistants have progressed hand in hand, with many
of the same researchers participating in language design and formal proofs.  
The result is a programming language that can stand up to intense mathematical scrutiny.

This parallel development of ML and proof assistants\footnote{HOL is a descendant of the original LCF theorem prover that
spurred the development of ML} also means that there are striking similarities between
the syntax of ML and the syntax of HOL.    Listing~XX shows a few parallels.

\begin{lstlisting}[keepspaces=true,stringstyle=\tt,basicstyle=\small,frame=single,framesep=8pt,mathescape,morekeywords={type,let,and,in},columns=flexible,caption={Ocaml syntax compared with HOL syntax}]
Ocaml                          versus    HOL
3:int                                    3:num
[0;1;2;3]                                [0;1;2;3]  
let x = 3 and y = 4 in x + y             let x = 3 and y = 4 in x + y
map (fun x -> x + 1) [0;1;2]             map (\x. x + 1) [0;1;2]
\end{lstlisting}

%(3:int);;
%[0;1;2;3];;
%let x = 3 and y = 4 in x + y;;
%map (fun x -> x + 1) [0;1;2];;

%`3:num`;;
%`[0;1;2;3]`;;
%`let x = 3 and y = 4 in x + y`;;
%`map (\x. x + 1) [0;1;2]`;;

In ML as in HOL every term has a type and explicit functions $f:A\to B$ of the right type
are required to convert from data of one type $A$ to another $B$.
One of Milner's key ideas in the design of ML was to have an abstract datatype representing theorems.
The strict type system of the language ML prevents the construction of any theorems except through
a carefully secured kernel that decrees the axioms and rules of inference.  In a formal proof in HOL, 
every theorem -- no matter how long or how complex -- is checked exhaustively by the kernel,
rigidly enforced by the structure of the language.

\subsection{Verification of the code that implements HOL Light}

The code in the kernel that declares the rules of HOL is of critical importance.  
Even a minor bug in the kernel might be exploited to create an inconsistent system.  Fortunately,
there are good reasons to believe that the kernel does not have a single bug.

1.  The kernel is remarkably small.  It takes fewer than 1000 lines of computer code to express
all of the kernel functions, including the type system, the term constructors, sequents, the rules of inference,
the axioms, and theorems.  For example, it only takes seven lines of computer code to describe the datatypes
for HOL types, terms, and theorems.  See Listing~XX.

\begin{lstlisting}[keepspaces=true,stringstyle=\tt,basicstyle=\small,frame=single,framesep=8pt,mathescape,morekeywords={type},columns=flexible,caption={Code from the kernel of HOL Light \cite{HOLL}}]
  type hol_type = Tyvar of string
                | Tyapp of string *  hol_type list

  type term = Var of string * hol_type
            | Const of string * hol_type
            | Comb of term * term
            | Abs of term * term

  type thm = Sequent of (term list * term)
\end{lstlisting}

2.  The code has been written in a clean, readable style and has been scrutinized by many computer scientists,
logicians, and mathematicians (including me).

3.  The correctness of the kernel has been formally verified, using the HOL Light proof assistant itself (extended by a large
cardinal) \cite{HaSelf}.  Specifically,
a model of HOL can be built inside HOL itself\footnote{A large cardinal axiom 
 gives the existence of a large type corresponding to the set $\bigcup\T$ that we used above in the construction of a model.
By G\"odel, we do not expect to construct a model of HOL in HOL except by 
adding an axiom to strengthen the system.} along the same
lines as the model of HOL in ZFC described above.  

This formal verification of HOL in HOL goes  further than the construction of a model.
It also checks that the code implementing the logic is bug free.  The code verification is based on the 
parallels mentioned above between the meta language and HOL itself, allowing the Ocaml source code for the HOL kernel to be
translated back into HOL for verification.  A stricter standard of code verification, based
on the semantics of the programming language, is discussed next.

4.  As independent corroboration, the correctness proof of the kernel of HOL Light 
has been automatically translated into the HOL-Zero and HOL4 assistants
and reverified there.


The proof of HOL in HOL removes practical doubts about the correctness of the kernel, but there
are vulnerabilities that we discuss in the next subsection.
%In particular, there are gaps between the semantics of Ocaml as defined and real-world Ocaml.\footnote{Ocaml has mutable strings and object magic.}
%As Harrison puts it elsewhere, ``the final jump from an abstract function inside the logic to a 
%concrete implementation in a serious programming language which {\it appears to correspond to it} is a glaring leap of faith.'' 
%\cite{harrison:meta}.  



\subsection{HOL in Machine Code}

The formal verification of HOL in HOL does not settle the issue of trust once and for all.
The reliability of the proof assistant ultimately
rests on the entire computer environment in which the software operates, 
including the semantics of programming languages,
compilers, operating systems, and hardware.  These are serious concerns.


 The current working goal of researchers is to create an unbroken formally verified chain extended from the
HOL Light kernel all the way down to machine code.  Most of the links have been forged, but parts of the chain
are still unsecured.  
\footnote{A brilliant success has been the construction of a formally verified C compiler~\cite{CC}. Another remarkable
project that has led to the full formal an operating system kernel~\cite{sel4}.  In this survey article,
we focus on the work done on formal verification related to the ML programming language, because it fits more closely
with our narrative of building a trustworthy system in HOL.}

CakeML, which
is a dialect of ML with mathematically rigorous operational semantics.  We turn our
attention to this dialect.    According to the designers, ``Our overal goal for CakeML
is to provide the most secure system possible for running verified software and other programs that require a high-assurance
platform'' \cite{CakeML}.

References: \cite{CakeML}, \cite{VIH}, \cite{stateful-ML}, \cite{x86-CC}.

(XX rewrite this paragraph and add references)
Harrison's formal proof of HOL in HOL has been extended in the following ways.  1.  Support has been added for 
HOL Light kernel functions that extend the system, by adding new definitions, types, and axioms.
2.  The ML code implementing the kernel of HOL Light is generated automatically from the HOL4 specification of the kernel with
an automatically generated theorem asserting that the generated code conforms to the specification, according to the operational
semantics of ML.
There are future plans -- already in the works --
to give a fully verified theorem prover in ML that runs on a fully verified ML runtime.



\bigskip


\subsection{Disclaimers}

At the conclusion of this section, we make the usual disclaimers.
Without exception, all physical devices are prone to failure.  Soft errors (typically 
caused by alpha particle interactions between memory and its environment) 
produce a steady
stream of errors depending on complex factors such as hardware architecture and
the elevation above sea level at which the
calculation is performed.  A formal proof in HOL of the correctness of HOL carries
the evident dangers of self-justification.  Finally, proofs of correctness are made relative to 
mathematically precise idealized descriptions of things rather than the things
in themselves.

Notwithstanding all these disclaimers,  formalized mathematics reduces defect rates
in mathematical proofs to levels that are simply not possible by any other available process.
In particular, formalized mathematics can now claim to be orders of magnitude more reliable than 
traditionally refereed papers.  


In the previous section, we discussed the construction of a reliable proof assistant.  In the next section,
we turn to a different proof assistant, Coq, and look at the formalization of group theory.
We warn the reader that the Coq system, and its underlying logic -- Calculus of Inductive Constructions, 
are significantly different from the HOL system in the previous section~\cite{CiC}.


\section{Advanced Developments in Coq: The Feit-Thompson Theorem}


Feit and Thompson published their famous theorem in 1963.

\begin{theorem}[The Odd Order Theorem, Feit-Thompson (FT)]  All finite groups of odd order are solvable.
\end{theorem}

\begin{lstlisting}[keepspaces=true,stringstyle=\tt,basicstyle=\small,frame=single,framesep=8pt,mathescape,morekeywords={Theorem},columns=flexible]
Theorem Feit_Thompson (gT : finGroup Type) (G : {group gT}) :
  odd #|G| -> solvable G.
\end{lstlisting}

Solomon writes this about the significance of the Odd Order theorem, ``This short sentence and its long proof
were a moment in the evolution of finite group theory analogous to the emergence of fish onto dry
land.  Nothing like it had happened before; nothing quite like it has happened since.''

The Feit-Thompson paper broke through various barriers that cleared the way for a 
remarkably fruitful massive research collaboration that eventually led to the classification
of finite simple groups.  Significantly, the  255 page Feit-Thompson result
started an avalanche of long complex proofs related to the classification, 
culminating in the 1221 page classification of quasithin groups by Aschbacher and Smith.
% references at https://en.wikipedia.org/wiki/Quasithin_group. 


Background material for the proof appears in
textbooks {\it Finite Groups} by Gorenstein, and {\it Finite  Group Theory} by Aschbacher, and 
{\it Character Theory of finite groups} by Isaacs.
The necessary background includes a basic graduate-level understanding of rings, modules, linear and multilinear algebra (including 
direct sums, tensor
products  and determinants);
fields, algebraic closures, and basic Galois theory; the structure theorems of Sylow and Hall; Jordan-H\"older;
Wedderburn's structure theorem for semisimple algebras; representation theory with Schur's lemma, Maschke's theorem,
and Jacobson density (for finite groups); and character theory including Frobenius reciprocity, and orthogonality. 

I will not say much about the actual proof of the 
Feit-Thompson theorem.\footnote{I was a student of some of the great Cambridge group theorists at the peak of
the classification era, but Thompson's influence pushed me towards modular forms and Conway towards sphere packings,
leaving me as a failed finite group theorist.}  We have now had
more than 
50 years to assimilate the ideas of the proof.  There are numerous surveys of the proof [Goren, p 450], [Glauber,] [Thompson, ICM],
[Soloman, ].
The original proof of Feit and Thompson was later reworked  and simplified in two books cite[XXX].

In very brief terms, the proof starts by assuming a minimal counterexample to the theorem.  This
counterexample will be a finite simple group $G$ of odd order in which every proper subgroup is solvable.
Each maximal subgroup of $G$ is $p$-local; that is, the normalizer of a nontrivial subgroup
of $G$ of $p$-power order, for some prime $p$.  A major part of the proof of FT consists in establishing
restrictions on the structure of the maximal subgroups and their embeddings into $G$.
In the special case when $G$ is a $CN$-group (a group whose centralizers of non-identity elements are all nilpotent),
the maximal subgroups are {\it Frobenius groups}.  A Frobenius group is
a nontrivial semidirect product $K\rtimes H$, where $H$ is disjoint from its conjugates, and $H$ is its own normalizer [Gor, Th. 7.7].
Back in general case, the strategy is to prove that as many of the maximal subgroups as possible are as close as possible to
being Frobenius groups.  This strategy encounters many exceptions and detours, but eventually the local
analysis shows that the maximal subgroups are mostly rather Frobenius-like. 
This can be made precise [Loc An. Sec 16].

% Frobenius : http://ssr2.msr-inria.inria.fr/~jenkins/current/frobenius.html

A second major portion of the proof is uses the complex character theory of $G$ to obtain inequalities over the real numbers
that restrict and ultimately eliminate all possibilities for $G$. 

A final part of the proof uses generators and relations to remove a special case in which there
are maximal subgroups isomorphic to the group of all permutations of a finite field $\ring{F}$
of the form $x \mapsto a\, \sigma (x) + b$, where $\sigma$ is a field automorphism, $b\in\ring{F}$,
and $a$ is an element $\ring{F}$ of norm $1$.





\subsection{Formal verification}


The Feit-Thompson theorem has been formally verified in the Coq proof assistant by a team led by Georges Gonthier.
[Gont-FT].  This is an extraordinary milestone in the history of formal proofs.
What is particularly significant about the formalization itself?

1. Although the Feit-Thompson theorem itself has never been seriously questioned,
 the premature announcement of the classification of finite simple groups drew sustained criticism.
 Gorenstein wrote, ``In February 1981, the classification of the finite simple groups was
completed'' [cite G, page 1]; and yet essential work on the classification
continued for decades after that date [cite Aschbacher, Smith; Aschbacher's survey Notices.].
This formalization project plants a flag in the middle of those squabbles and provides a
methodology for conflict resolution that is beyond reproach.

% http://www.ams.org/notices/200407/fea-aschbacher.pdf


2. Traditional methods of refereeing mathematical research become strained, when
proofs are unusually long or computer assisted.  There is a wiki page listing numerous proofs
in mathematics that set records for length.
According to the list,
long papers cluster in certain areas
such as finite group theory related to the classification, 
the Langlands program and representation theory of Lie groups,
and graph theory.  Feit-Thompson, the four-color theorem (Gonthier's previous formalization
project), and the Kepler conjecture are all on that list. 
These three recent formalization projects send a clear defiant message  to mathematicians:
no matter how long or how complex your mathematical proofs may be, we can formalize them.
%\footnote{%The is a popular list of 100 well-known
%mathematical theorems
%that Wiedijk has used as benchmarks for formalization.  They range in difficult from the trivial (the 
%formula for the sum
%of an arithmetic series) to deep (Wiles's proof Fermat's Last theorem).
%89\% of them have been formalized.  
%For future formalization projects, the wiki's list of 27 long
%proofs would be a good place to start.} 

Absolutely no technological barriers 
prevent the formalization of large parts of the mathematical corpus.  The issues now
are how to make the technology more efficient, cost effective, and user friendly.

% Reference http://www.cs.ru.nl/F.Wiedijk/100/


\subsection{Constructive proof of FT}

The formal proof of the FT theorem is based on the second-generation proof described in
two books cited above [cite Glaube] [cite Peter].
If we were to translate the formal proof of Feit-Thompson back into a humanly readable book, 
the most notable difference would be that the original proof uses classical logic,
but the formal proof uses constructive logic.  In particular, a constructive proof avoids
 the law of excluded middle $\phi\lor \neg \phi$ and proofs by contradiction.

Several different strategies were used to translate the proof from book form to
constructive computer proof.

1.  Often, mathematicians use proof by contradiction when a direct proof
would work equally well.  
``Let $G$ be the finite group of minimal order that is a counterexample'' gets
replaced with an induction on the order of the group, and so forth.

2. In [Peter], the character theory of finite groups relies heavily on 
vector spaces over $\ring{C}$ and complex conjugation.  In the constructive formal proof, the corresponding
vector spaces over an algebraic clousure $\bar{\ring{Q}}$ are used.  An algebraic closure of $\ring{Q}$ 
is obtained as the union of an increasing tower of  
number fields $\ring{Q}(\alpha_i)$, for $i=0,\ldots$, each  an explicitly constructed splitting field
over the preceding one.
Complex conjugation is replaced with
conjugation with respect to a maximal real subfield of $\bar{\ring{Q}}$, also constructed
as an explicit union of real subfields of finite degree over $\ring{Q}$.

3.  Some constructions are justified by the decision procdure for the first-order theory
of algebraically closed fields~\cite{XX}.  
For example, the socle of a module is defined as the sum of its
simple submodules.  This decision procedure gives a constructive test for the simplicity of a submodule,
leading to a constructive definition of the socle  (for a finitely generated 
$k[G]$-module where $G$ is a finite group and $k$ is an algebraically closed field).  

On a related note, the construction of a simple submodule of a given module $M$ requires choice;
by confining attention to modules that are countable as sets, countable choice suffices, which 
is provable in Coq from the well-foundedness of the natural numbers.


4.  Various intermediate results in FT use classical logic, but the use of classical
logic is eliminated from the final statement of the Feit-Thompson theorem.  The wedding of constructive and
classical logics is 
seemlessly arranged
through a predicate $\op{\it classically}~\phi$ that marks every result proved by classical logic.
The definition of $\op{\it classically}~\phi$ is
\[
\op{classically}~\phi : Prop := \forall {b:\bool}, (\phi\to b) \to b,
\]
which is essentially the usual double negation translation of a classical formula into
constructive logic.  The definition of {\it classically} works in such a way that $\phi$ may
be used as an assumption in the proof of any boolean goal $b$.  Specifically, matching a goal
with $\op{\it classically}~\phi$ reduces the goal $b$ to subgoal $\phi\to b$, whereby the assumption
$\phi$ is introduced.

5.  Function extensionality does not hold in Coq; that is, it is not provable in Coq
that two functions are equal, if they are provably equal on every input.  This creates various
complications in Coq.  There are no general quotient types in Coq: types corresponding
to the set of classes on a type under an equivalence relation.  Instead, Coq introduces {\it setoids} 
(that is, a set together 
with an equivalence relation on that set) without passing to the quotient.  Unfortunately, setoids carry a certain
overhead, which was not acceptable in the FT proof, which makes ubiquitous use of quotient
groups $G/N$.   

To allow for genuiune quotients, the FT theorem is developed in the context of {\it finite types} with {\it decidable equality};
(that is, the type is equipped with boolean procedure that decides whether any two elements of that type are equal).
In this context, function extensionality holds and genuine quotients groups can be formed.


6.  For all its advantages, at times the type theory of Coq is best forgotten.  In particular,
in arguments involving a finite number of finite groups, there would be excessive overhead in
assigning a separate type, a separate binary operation on each group, 
and writing explicit homomorphisms embedding subgroups into groups.
 For such arguments, the finitely many finite groups are
often considered via Cayley as subgroups of a larger ambient finite group giving them a common binary
operation and type.  These ambient group arguments are one of the biggest stylistic differences
between the Coq proof and the original.


\subsection{Library of abstract algebra}

Most of the work in developing the proof of the Feit-Thompson theorem went into development of the libraries in
abstract algebra and the related computer infrastructure.  The libraries include all the formalization of all
of the necessary background material described at the beginning of this section, from algebraic
closures to Wedderburn's theorem.  From a software engineering point of view, it has
been a major undertaking to get the computer to understand algebra at a level
comparable to that of a working mathematician, and for it all to be formally
justified.  This includes notation, so that for example,  in a 
given context the computer is able to correctly infer the correct binary operation 
to be used in a given expression $g*h$.  

This includes much of the implicit knowledge
of proofs.  Mathematicians read a proof with a great deal of domain knowledge;
for example, they do not need to be reminded again and again that $K \cap H$ is a group when
$K$ and $H$ are subgroups of some group.  The FT team has worked very hard
to automate such details within their system.



\section{Automating Formalization}

One way to make proof assistants more useable is to increase the amount of
automation.   Ideally, the human should provide the high level structure of a proof
as it is done in a traditionally published paper, and the computer should use
search algorithms to insert the low-level reasoning.   

As technology has developed, computers have become capable of providing
more and more of the low-level reasoning.   This section describes some of
the most promising technologies in this field.

Before getting intot he details, we point out that these search algorithms
are based on fundamentally different design considerations than the proof
assistants themselves.  As we explained in Section XX, it is a matter of
immense importance for the underlying logic of the proof assistant to be sound
and for the implementation in code to be free of bugs.  

By contrast, proof search algorithms should be allowed to use whatever
wreckless methods they please, with the understanding that their results must
eventually verified by  the proof assistant before being admitted as part of
formal proof.  In particular, the soundness or completeness of these algorithms
is not a necessity.  It is more important to have fast algorithms with a high
success rate.  Our algorithms may make extensive use of
experimentation, statistical methods, machine learning, and artificial intelligence.



\subsection{First-order theorem provers and Vampire}


XX Write this section.

\subsection{Hammers}





\section{Final Remarks}

The aim of this report has been to describe some of the recent developments in formal proofs.
Space and time do not permit a comprehensive survey, but in this final section, I briefly mention
a few other projects.


\subsection{Homotopy Type Theory}

My report will be directly followed by a report by Thierry Coquand on dependent types and
the univalence axiom, so I will be brief in my remarks on homotopy type theory.

Homotopy type theory (HoTT) is a foundational system for mathematics that includes
dependent type theory, the univalence axiom, and higher inductive types.  
Introductions to homotopy type theory can be found in the [HoTT Book] and the survey [Bulletin], [Awodey], etc.
For models of HoTT, see [Kap.Lums.VV].

It goes without saying that as mathematicians, we construct the ground on which we stand;
the foundations of mathematics are of choosing, subject
to only mild constraints such as plausible consistency, expressive power, and a community of users.
In particular, nothing but our own limited imaginations prevents us from relocating the
foundations much closer to home.

By being a foundational system that is close to the actual practice of homotopy theory,
HoTT makes the formalization of this branch of mathematics a walk in the park.
In the last two years many new formal proofs and constructions have been obtained: 
loop spaces, computations of various
fundamental groups of spheres, the Freudenthal suspension theorem,
construction of Eilenberg-Mac Lane spaces, and the Blakers-Massey theorem,
Formalization of these results in other systems would have been much more labor intensive.
A new line of research, synthetic homotopy theory, develops homotopy theory on HoTT foundations.
% See Licata https://github.com/dlicata335/hott-agda, homotopy directory.

\begin{lstlisting}[keepspaces=true,stringstyle=\tt,basicstyle=\small,frame=single,framesep=8pt,mathescape,morekeywords={Definition,Lemma,Proof,Defined},columns=flexible,caption={Construction of classifying spaces as the type of a torsor in Univalent foundations.  The proof that the classifying space $BG$ is connected. --D. Grayson.}]
Definition ClassifyingSpace G := pointedType (Torsor G) (trivialTorsor G).
Definition E := PointedTorsor.
Definition B := ClassifyingSpace.
Definition $\pi$ {G:gr} := underlyingTorsor : E G -> B G.

Lemma isconnBG (G:gr) : isconnected (B G).
Proof. intros. apply (base_connected (trivialTorsor _)).
  intros X. apply (squash_to_prop (torsor_nonempty X)). { apply propproperty. }
  intros x. apply hinhpr. exact (torsor_eqweq_to_path (triviality_isomorphism X x)). 
Defined.
\end{lstlisting}
% https://github.com/UniMath/UniMath/blob/master/UniMath/Ktheory/GroupAction.v

I challenge any other system to pass from the foundations of math to classifying spaces so directly and elegantly.




\subsection{Bourbaki on Formalization}

Over the past generation, the mantle for Bourbaki-style mathematics has  passed to the formal proof community, in the way it
 deliberates carefully on matters of notation and terminology, finds
appropriate level of generalization of concepts and abstractions, and
situates different branches of mathematics within a coherent framework.

The opening quote claims that formalized mathematics is absolutely unrealizable.
This is correct in the strict sense that no human artefact is absolutely trustworthy.

Nevertheless,  technological barriers hindering formalization have fallen one
after another.  Bourbaki objected that formal proofs are too long (``{\it la moindre d\'emonstration \ldots
exigerait d\'ej\`a des centaines de signes}''). Today, computers verifications that
check the millions of inferences are routine.

Bourbaki argues that it would be a burden to forego the convenience of abuses of notation.
As Gonthier has convincingly shown, 
many abuses of notation can actually be described by precise rules and implemented as algorithms,
making the term {\it abuse of notation} really something of a misnomer, and
allowing mathematicians to work formally with with customary notation.
Finally, Bourbaki gives the utility of metamathemaical arguments and abbreviations as an argument against formalization.
The trend over the past decades has been to move more and more features out of the metatheory and into the theory, and to
rework metamathematical argument as reasoning in higher-order logic.  In particular, it is now
standard to treat abbreviations and definitions as part of the logic itself.

\subsection{Conclusion}

This report has described three major projects in the world of formal proofs:
developing trustworthy systems with HOL, formalizing advanced mathematical theorems
in Coq, and using sledgehammers to increase automation.  
We might anticipate that these developments might some day be available within a single proof assistant
that will be used routinely by mathematicians.


\section{Appendix. HOL Light axioms}

Give a summary.


\section{Appendix. Some Formally verified theorems}


This section gives a brief overview of some of the theorems that have been sucessfully formalized
in various proof assistants.  The purpose of these examples is to showcase the range
of what can be obtained by current technologies.


The four color theorem assserts that every planar map can be colored with a palette of four colors in such a way that adjacent
countries are painted different colors.

Starting in joint work with B. Werner and then eventually continuing on his own,
Georges Gonthier obtained a complete formal verification of the four-color theorem in 2003 in
the Coq proof assistant.

\begin{lstlisting}[keepspaces=true,stringstyle=\tt,basicstyle=\small,frame=single,framesep=8pt,morekeywords={Variable,Theorem,Proof,Qed},columns=flexible,caption={Four-color theorem in Coq}]
Variable R : real_model. 
Theorem four_color : (m : (map R))
     (simple_map m) -> (map_colorable (4) m). 
Proof.
    Exact (compactness_extension four_color_finite). 
Qed.
\end{lstlisting}


The elementary proof of Erd\"os and Selberg was formalized in \cite{XX}.  Later the analytic proof of Hadamard and de la Val\'ee Poussin
was formalized \cite{XX}.

\begin{lstlisting}[keepspaces=true,stringstyle=\tt,basicstyle=\small,frame=single,framesep=8pt,mathescape,morekeywords={theorem,fixes,assumes,defines,shows,Variable,Theorem,Proof,Qed},columns=flexible,caption={The prime number theorem}]
  ((\n. &(CARD {p | prime p /\ p <= n}) / (&n / log(&n)))
    ---> &1) sequentially
\end{lstlisting}

\begin{lstlisting}[keepspaces=true,stringstyle=\tt,basicstyle=\small,frame=single,framesep=8pt,mathescape,morekeywords={theorem,fixes,assumes,defines,shows,Variable,Theorem,Proof,Qed},columns=flexible,caption={Brouwer fixed point formula}]
  $\forall$ f:real^N->real^N s. 
  compact s $\land$ 
  convex s $\land$
  ~(s = {}) $\land$
  f continuous_on s $\land$
  IMAGE f s SUBSET s
  $\Longrightarrow$ $\exists$x. x IN s /\ f x = x
\end{lstlisting}

The formalization of the central limit theorem was carried out earlier this year.

\begin{lstlisting}[keepspaces=true,stringstyle=\tt,basicstyle=\small,frame=single,framesep=8pt,mathescape,morekeywords={theorem,fixes,assumes,defines,shows,Variable,Theorem,Proof,Qed},columns=flexible,caption={The central limit theorem in Isabelle}]
theorem (in prob_space) central_limit_theorem:
  fixes 
    X :: "nat $\Rightarrow$ 'a $\Rightarrow$ real" and
    $\mu$ :: "real measure" and
    $\sigma$ :: real and
    S :: "nat $\Rightarrow$ 'a $\Rightarrow$ real"
  assumes
    X_indep: "indep_vars ($\lambda$i. borel) X UNIV" and
    X_integrable: "$\bigwedge$n. integrable M (X n)" and
    X_mean_0: "$\bigwedge$n. expectation (X n) = 0" and
    $\sigma$_pos: "$\sigma$ > 0" and
    X_square_integrable: "$\bigwedge$n. integrable M ($\lambda$x. (X n x)$^2$)" and
    X_variance: "$\bigwedge$n. variance (X n) = $\sigma^2$" and
    X_distrib: "$\bigwedge$n. distr M borel (X n) = $\mu$"
  defines
    "S n $\equiv$ $\lambda$x. $\Sigma\,$i<n. X i x"
  shows
    "weak_conv_m ($\lambda$n. distr M borel ($\lambda$x. S n x / sqrt (n * $\sigma^2$))) 
        (density lborel standard_normal_density)"
\end{lstlisting}

% https://github.com/avigad/isabelle/blob/master/Analysis/Central_Limit_Theorem.thy
% https://github.com/avigad/isabelle/blob/master/Analysis/Central_Limit_Theorem.thy


The Kepler conjecture asserts that no packing of congruent balls in $\ring{R}^3$ can have density greater than
the face-centered cubic packing.  The Kepler conjecture is now a theorem.  The proof relies on many computer
calculations.  The noncomputer parts of the proof (as well as many of the computer parts of the proof) of the Kepler conjecture 
have been formalized in HOL Light.  This formalization has been a large collaborative effort.

\begin{lstlisting}[keepspaces=true,stringstyle=\tt,basicstyle=\small,frame=single,framesep=8pt,framextopmargin=10pt,mathescape,morekeywords={Variable,Theorem,Proof,Qed},columns=flexible,caption={The formal proof of the Kepler conjecture in HOL Light}]
    linear_programming_results $\land$      
    import_tame_classification $\land$      
    the_nonlinear_inequalities $\land$
    restricted_hypermaps_are_planegraphs    
    $\Longrightarrow$ the_kepler_conjecture
\end{lstlisting}


None of the work described in this report represents my own research, except for contributions to the
proof and formalization of the Kepler conjecture.  I would like to thank the many people who helped me
during the preparation of this report, including A. Mahboubi, (XX Add XX).

%I wish to thank Harrison, Mahboubi, Gonthier, Nipkow, Avigad, Urban...


\raggedright
\bibliographystyle{plain} %plainnat
\bibliography{/Users/flyspeck/Desktop/googlecode/flyspeck/latex/bibliography/all}


References: Bourbaki - Sets.

Aschbacher, Michael (2000), Finite Group Theory, Cambridge University Press.


\end{document}