\section{Appendix: Voronoi Hypermap System}\label{ap:A}

This appendix  lists the system of constraints determining
the Voronoi hypermap system.  As the main body of the text explains
a hypermap system is a pair $(H,\Phi)$, where $H$ is a hypermap
and $\Phi$ is a finite set of constraints on $H$.  To describe
the constraints $\Phi$ in precise terms requires some notation.

Let $H=(D,e,n,f)$ be a hypermap.   Greek letters
$\alpha,\beta \in D$ denote darts.  Let $V$ be the vector space 
of real-valued  functions on $D$.  A constraint $\phi$ is a boolean-valued function on $V^\ell$ for some $\ell$.  The suggestive
notation $(\optt{sol},\optt{mu},\optt{tau},\optt{yn},\optt{ye})\in V^\ell$  is used for $\ell$-tuples
of elements of $V$.  The main text relates each coordinate back
with its namesake.  For example, the linear constraints on
$\optt{sol}\in V$, mirror  nonlinear relations satisfied by the
nonlinear function $\sol$ in the main text.  This correspondence
between functions in $V$ and functions does not enter into the
definition of the Voronoi hypermap system.  For the purposes of this
appendix, this correspondence is simply an aid to the
intuition.

If $m$ is  a permutation on $D$, let $\op{ord}(m,\alpha,i)$
be the predicate that asserts that the cardinality of the $m$-orbit
of $\alpha$ is $i\in\ring{N}$.
