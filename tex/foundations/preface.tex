% Preface


%%  \label{part:intro}
   \newpage

   \setcounter{chapter}{-1}
   \chapter{Preface}
   \label{XX}  % for missing references. XX.

  A proof assistant is a piece of computer software that allows a user to check every logical inference of a mathematical proof by computer.  There are now a number of proof assistants based on HOL (Higher Order Logic).  I have been using the John Harrison's proof assistant (called HOL-Light) in my work for several years.

  Last year I taught a course on the Foundations of Mathematics, based on HOL-Light.  This book has grown out of the notes for that course.  Before teaching the course, John Harrison provided me with a list of references that he followed in the proofs in his system.  Harrison has also provided extensive documentation of his proof assistant.  

What I have done for the course and for this book is to translate Harrison's computer system into something that has more of the look and feel of mathematics, and less of the look of a computer program.  Where Harrison uses computer code implemented in Objective CAML, I use traditional mathematical functions.  

For me it has been a worthwhile exercise to rewrite a nontrivial body of computer code in rigorous mathematical form.  It is evidence of the great progress of computer language semantics that the code lends itself so readily to rigorous mathematical description.

It is intended for this book to be easier for math students to read and understand than computer code.  Of course, there are some serious drawbacks in replacing computer code with traditional mathematics.  Computer code is compiled; compilation catches bugs.  Traditional mathematics is not compiled.  Traditions change.  My hope is that the readers of this book will combine their study of this book with practical hands-on experience with proof assistants such as HOL, to obtain a deeper understanding of the foundations of mathematics.

   I thank the students of my undergraduate course: Introduction to the Foundations of Mathematics, for their many contributions.






