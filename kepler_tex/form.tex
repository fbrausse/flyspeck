% file started March 22, 2009
% Marchal objective function

\def\lam{\lambda}
\def\Lam{\Lambda}
\def\bl{{\underline{\lam}}}
\def\bm{{\underline{\mu}}}
\def\angle#1#2#3{{\op{angle}(#1,{#2,#3})}}

\chapter{Formulation}

\section{Statement}



\begin{theorem}[Sphere Packing Problem (Kepler Conjecture)]
\label{theorem:kepler}   No packing of congruent balls in
Euclidean three space has density greater than that of the
face-centered cubic packing.
\end{theorem}

\begin{remark}
This density is $\pi/\sqrt{18}\approx 0.74.$  There are other
packings, such as the hexagonal close packing, that attain this
same density.
\end{remark}

The proof of this result is presented in this book. Here, we
describe the top-level outline of the proof and give references to
the sources of the details of the proof.

By a {\it packing}, we mean an arrangement of congruent balls that
are nonoverlapping in the sense that the interiors of the balls are
pairwise disjoint. Consider a \indy{Index}{packing} packing of congruent
balls in Euclidean three space. There is no harm in assuming that
all the balls have unit radius. 

The density of a packing does not
decrease when balls are added to the packing. Thus, to answer a
question about the greatest possible density we may add
nonoverlapping balls until there is no room to add further balls.
Such a packing will be said to be {\it saturated}.

 \indy{Index}{saturated}
 \indy{Index}{overlap}

Let $\Lambda$ be the set of centers of the balls in a
packing. Our choice of radius for the
balls implies that any two points in $\Lambda$ have distance at
least $2$ from each other. We call the points of $\Lambda$ {\it
\indy{Index}{vertex} vertices}.
%A bijection
%$\varph:\ring{N}\to\Lambda$ is called an {\it enumerated\/}
%packing.

%\begin{lemma}\guid{DQLSGKR} A saturated packing $\Lambda$ is countably infinite.  Hence,
%there exists a bijection $\varphi:\ring{N}\to\Lambda$.
%\end{lemma}

%\begin{proof}  If $\Lambda$ is finite, there is room to add a ball
%far from the finite cluster $\Lambda$, so $\Lambda$ is not
%saturated.

%The map $\Lambda\to\ring{Z}^3$, $$(x,y,z)\mapsto (\lfloor 2x
%\rfloor, \lfloor 2y \rfloor, \lfloor 2z \rfloor)$$ is a one-one
%map from $\Lambda$ into a countable set.  Hence $\Lambda$ is
%countable.
%\end{proof}


  Let $B(p,r)$ denote the open ball in
Euclidean three space at center $p$ and radius $r$.  The open ball
is measurable, with measure $4\pi r^3/3$.



Let $\delta(\Lambda,p,r)$ be the finite density, defined as the
ratio of the volume of $B(\Lambda,p,r)$ to the volume of $B(p,r)$,
where $B(\Lambda,p,r)$ is defined as the intersection with
$B(p,r)$ of the union of all balls in the packing. Set
$\Lambda(p,r) = \Lambda \cap
B(p,r)$ and $\Lambda^*(p,r) = \Lambda(p,r)\setminus \{p\}$.
\indy{Greek}{ZZdelta@$\delta(\Lambda,p,r)$}
\indy{Greek}{ZZLamda@$\Lambda(p,r)$}
\indy{Greek}{ZZLamda@$\Lambda^*(p,r)$}

\begin{lemma}\guid{KIUMVTC}\label{lemma:Lambda-finite}
Let $\Lambda$ be a packing and let $p\in\ring{R}^3$.
Then the set $\Lambda(p,r)$ is finite.
\end{lemma}

\begin{proof}  Let $p = (p_x,p_y,p_z)$. The map
$$v=(v_x,v_y,v_z)\mapsto (\lfloor 2(v_x-p_x)
\rfloor, \lfloor 2(v_y-p_y) \rfloor, \lfloor 2(v_z-p_z) \rfloor)$$
is a one-one map from $\Lambda(p,r)$ into the set $\ring{Z}^3\cap B(0,2
r
 + 1)$.  By Lemma~\ref{lemma:Zcount}, this gives a one-to-one map
 into a finite set, hence $\Lambda(p,r)$ is finite.
\end{proof}


\begin{definition}[Voronoi~cell,~$\Omega$]\label{def:voronoi}\indy{Index}{Voronoi cell} Voronoi cell
$\Omega(v)=\Omega(\Lambda,v)$\indy{Greek}{ZZzomega@$\Omega(v)$} around a
vertex $v\in \Lambda$ is the set of points closer to $v$ than to
any other ball center. 
% Let $\Omega_t(\Lambda,v) = \Omega(\Lambda,v)
%\cap B(v,t)$ be the truncated Voronoi cell at radius $t$.
\end{definition}

\begin{lemma}\guid{DRUQUFE} Let $\Lambda$ be a saturated packing and let $t>0$.
Each Voronoi cell $\Omega(\Lambda,v)$, for $v\in\Lambda$, is
convex, open, bounded, and measurable.
\end{lemma}

\begin{proof}  This is elementary.
\end{proof}

\begin{definition}[negligible]\label{def:negligible}
Let $A:\Lambda\to\R$ be a function.  We say that $A$\indy{Index}{Az@$A$}
is
  {\it negligible\/}\indy{Index}{negligible}
if there is a constant $C_1$ such that for all $r\ge1$ and all
$p\in\ring{R}^3$,
   $$\sum_{v\in\Lambda(p,r)} A(v) \le C_1 r^2.$$
We say that the function $A:\Lambda\to\R$ is
  {\it fcc-compatible\/}\indy{Index}{fcc-compatible}
if for all $v\in\Lambda$ we have the inequality
$$\sqrt{32}\le \op{vol}(\Omega(v)) + A(v).$$
\end{definition}





\begin{remark}
The value $\op{vol}(\Omega(v)) + A(v)$ may be interpreted as a
{\it corrected\/} volume\indy{Index}{corrected volume} of the Voronoi
cell. Fcc-compatibility asserts that the corrected volume of the
Voronoi cell is always at least the volume of the Voronoi cells in
the face-centered cubic and hexagonal-close packings.
\end{remark}

%\begin{remark} In \cite{Hales:2006:DCG}, the full Voronoi cell $\Omega(v)$
%is used, rather than $\Omega(v)$.  The truncation at radius $2$
%is just a matter of convenience to guarantee the boundedness
%and hence the finite volume of the (truncated) Voronoi cell.
%In \cite{Hales:2006:DCG}, the same effect was achieved by requiring all packing%s
%to be saturated.  We have dropped the assumption of saturation
%on $\Lambda$.
%\end{remark}


\begin{lemma}\guid{JGXZYGW}
\label{lemma:deltabound} If there exists a \indy{Index}{negligible}
negligible \indy{Index}{fcc-compatible} fcc-compatible function
$A:\Lambda\to\R$ for a 
%saturated 
packing $\Lambda$, then there
exists a constant $C$ such that for all $r\ge1$ and all
$p\in\ring{R}^3$,
    $$
    \delta(\Lambda,p,r)
    \le \pi/\sqrt{18} + C/r.
    $$
The constant $C$ depends on $\Lambda$ only through the constant
$C_1$ of Definition~\ref{def:negligible}.
\end{lemma}



\begin{proof}
The numerator $\op{vol}\, B(\Lambda,p,r)$ of $\delta(\Lambda,p,r)$
is at most the product of the volume of a ball $4\pi/3$ with the
number $\card(\Lambda(p,r+1))$ of balls intersecting $B(p,r)$.  Hence
    \begin{equation}
    \op{vol}\, B(\Lambda,p,r) \le \card(\Lambda(p,r+1)) 4\pi/3.
    \label{eqn:Abound}
    \end{equation}

%In a %saturated packing 
Each truncated Voronoi cell is contained in a ball of
radius $2$ centered at the {\it center} of the cell.  The volume
of the ball $B(p,r+3)$ is at least the combined volume of 
truncated Voronoi
cells whose center lies in the ball $B(p,r+1)$. This observation,
combined with fcc-compatibility and negligibility, gives
    \begin{equation}
    \begin{split}
    \sqrt{32}\card(\Lambda(p,r+1))
    &\le \sum_{v\in\Lambda(p,r+1)} (A(v) +
    \op{vol}(\Omega(v))) \\
    &\le C_1 (r+1)^2 + \op{vol}\,B(p,r+3) \\
    &\le C_1 (r+1)^2 + (1+3/r)^3 \op{vol}\,B(p,r)
    \label{eqn:Bbound}
    \end{split}.
    \end{equation}
Recall that $\delta(\Lambda,p,r)=
\op{vol}\,B(\Lambda,p,r)/\op{vol}\,B(p,r)$. Divide Inequality
\ref{eqn:Abound} through by $\op{vol}\,B(p,r)$.  Use
Inequality~\ref{eqn:Bbound} to eliminate $\card(\Lambda(p,r+1))$ from the
resulting inequality.  This gives
    $$\delta(\Lambda,p,r)
        \le \frac{\pi}{\sqrt{18}} (1+3/r)^3 + C_1 \frac{(r+1)^2}{r^3\sqrt{32}}.
    $$
The result follows for an appropriately chosen constant $C$.
\end{proof}

\begin{remark} \label{remark:precise}
We take the precise meaning of the packing problem to be the
bound $\delta(\Lambda,p,r) \le \pi/\sqrt{18} + C/r$ of the lemma.
Thus, the solution to the packing problem follows, provided a negligible
fcc-compatible function can be found. The strategy will be to
define a negligible function, and then to solve an optimization
problem in finitely many variables to establish that it is
fcc-compatible.
\end{remark}

If $\Lambda$ is a %saturated 
packing, we have an ordered pair
$(\Lambda,v)$ for each vertex $v\in\Lambda$.  The pairs
$(\Lambda,v)$ are called {\it centered packings}.



\section{Rogers's partition}

\begin{definition}[saturated packing]
A packing $\Lam\subset \ring{R}^3$ is a set such that
$$\forall \lam~\mu\in \Lam.~  \norm{\lam}{\mu} < 2 \Rightarrow (\lam=\mu).$$
The packing is saturated if for every $v\in\ring{R}^3$,   there exists $\lam\in\Lam$
such that $\norm{\lam}{v}\le 2$.
\end{definition}

%We think of $\Lam$ as the set of centers of a  packing of congruent balls of radius $1$. To be saturated means that there is no room for further balls to be added to the packing. There is no loss in generality in assuming that the packing is saturated, when searching for the greatest possible density of a packing.

Given a packing $\Lam$, Rogers gives a partition of Euclidean space into
simplices with vertices at $\Lam$ \cite{Rogers:1958:Packing}.   This section describes his partition.

Let $\Omega(\lam)$  be the closed Voronoi cell centered at $\lam$:
$$
  \Omega(\lam) = \{p \mid  \norm{p}{\lam} 
\le \norm{p}{\mu},\ \lam\ne\mu\in\Lam\}.
$$
If $S\subset\Lam$, let $\Omega(S)$ be the intersection of the family:
$$\Omega(S)  = \bigcap \{\Omega(\lam)\mid \lam\in S \}.$$
We call $\Omega(S)$ the $S$-face of $\Omega(\lam)$, when $\lambda\in S$.

For $X\subset\ring{R}^3$, we let $\dim(X)$ be the dimension of the affine hull
of $X$: $\dim(X) = \dim(\op{aff}(X))$.

If $\bl=(\lam_0,\ldots,\lam_k)$ and $j\le k$ write $\bl[j] = (\lam_0,\ldots,\lam_j)$.
Let $\Lam(k)$ be the set of $k+1$-tuples $\bl=(\lam_0,\ldots,\lam_k)$, with
$\lambda_i\in\Lambda$ such
that 
\begin{equation}\label{eqn:omega-dim}
\dim(\Omega(\bl[j])) = 3-j,
\end{equation}
for all $0<j\le k$.
In particular, $\Lam(0)=\Lam$.  Also, $\Lam(1)$ is the
set of all pairs $(\lam,\mu)$ such that the Voronoi cells at $\lam$ and $\mu$ meet along
a face of codimension $1$, and
so forth.


Define $v(\bl)$ recursively on $\Lam(k)$ by
$$v(\lam[0]) = \lam[0]$$
and $v(\lam[j+1])$ is the closest point on $\Omega(\lam[j+1])$ to $v(\lam[j])$.  The point is defined whenever $\Omega(\lam[j+1])$ is nonempty.
The set $\Omega(\lam[j+1])$ is convex and compact.  Thus, the point $v(\lam[j+1])$ exists
uniquely.

For $\bl\in\Lam(k)$, let 
$$R(j,\bl) = \op{conv}\{v(\lam[j]), v(\lam[j+1]),\ldots,v(\lam[k])\}.$$  Set $R(\bl)=R(0,\bl)$.


\begin{lemma} 
For any saturated packing $\Lambda$, we have
$$\ring{R}^3 = \bigcup \{ R(\bl) \mid \bl\in \Lam(3)\}.$$
\end{lemma}

\begin{proof}
We have $$\ring{R}^3 = \bigcup \{\Omega(\lam)\mid \lam\in \Lam(0)\}.$$
So it is enough to show that
$$\Omega(\bl[0]) = \bigcup \{ R(\bl') \mid \bl'\in \Lam(3),~\bl[0]=\bl'[0]\}.$$
We have
$$\Omega(\bl[0]) = \bigcup \op{conv}(v(\bl[0]),\{\Omega(\bl[1])\mid \bl[1]\in \Lam(1)\}).$$
So it is enough to show that
$$\Omega(\bl[1]) = \bigcup \{ R(1,\bl') \mid \bl'[1]=\bl[1],~\bl'\in \Lam(3)\}.$$
Repeating, it is enough to show that
$$\Omega(\bl[3]) = \bigcup\{R(3,\bl')\mid\bl'[3]=\bl[3],~\bl'\in\Lambda(3)\}.$$
The right-hand side is the singleton $\{v(\bl[3])\}$.  The left-hand side
contains this point and is contained in a properly decreasing chain of affine spaces:
$\ring{R}^3$, the bisector of $\lam_0$ and $\lam_1$, and so forth.  This determines a point,
so the two sides are equal.
\end{proof}

\begin{lemma}  The intersection of any two distinct $R(\bl)$ and $R(\bm)$ is
contained in a plane (hence has measure zero).
\end{lemma}

Combined with the previous lemma, we see that the simplices $R(\bl)$ partition Euclidean
space.

\begin{proof}  Let $\bl = (\lambda_0,\ldots)$ and $\mu = (\mu_0,\ldots)$.
If $\lambda_0\ne\mu_0$, the intersection of the simplices lies in $\Omega(\lam_0,\mu_0)$
and we are done.  Let $k+1$ the first index such that
$$\op{conv}(v(\lam_0),\ldots,v(\lam_0,\ldots,\lam_{k+1}))\ne
\op{conv}(v(\mu_0),\ldots,v(\mu_0,\ldots,\mu_{k+1})).
$$
By induction $\Omega(\lam_0,\ldots,\lam_k) = \Omega(\mu_0,\ldots,\mu_k)$
but
$$\Omega\lam_0,\ldots,\lam_{k+1}) \ne \Omega(\mu_0,\ldots,\mu_{k+1}).$$
The intersection lies in the convex hull $C$ of
$\{v(\lam_0),\ldots,v(\lam_0,\ldots,\lam_k)\}$ and
$V=\Omega(\lam_0,\ldots,\lam_{k+1},\mu_{k+1})$.  The set $V$ lies in an affine space of 
codimension $k+2$
and $C$ has codimension $(k+2) - (k+1) = 1$.
\end{proof}

\begin{lemma}\label{lemma:v2} Let $\bl=(\lam_0,\ldots,\lam_3)\in\Lambda(3)$ and fix $0\le j\le 3$.  Assume that
$$
\norm{v(\bl[j]) }{ v(\bl[0])} < \sqrt2.
$$
Then $v(\bl[j])$ is the circumcenter of the points $S_j =\{\lam_0,\ldots,\lam_j\}$ and
lies in the convex hull of $S_j$.  The points $v(\bl[j])$, for $j=0,1,2,3$, are distinct.
\end{lemma}

\begin{proof} By definition, an element $\bl\in\Lambda(3)$ gives dimensions: 
$$\dim\Omega(\lam_0,\ldots,\lam_j) = 3-j.$$  The case $j=0$ of the lemma is trivially
satisfied.  Assume by induction the result holds for numbers less than $j$.

Now $j>0$.
The circumcenter $p=p_{j}$ of the set $S_{j}$ is the point on the plane $\op{aff}(S_{j})$
closest to $p_{j-1}=v(\bl[j-1])$.  We claim that $p$ lies in $\Omega(S_{j})$.  Otherwise
there is a point $\mu\in\Lambda$ closer to $p$ than any $\lambda\in S_{j}$.  
The angles $\angle{p}{\mu}{\lam_i}$ are obtuse for $i\le j$:
$$
\norm{p}{\mu} < \sqrt2,\quad \norm{p}{\lambda_1} <\sqrt2,\quad \norm{\mu}{\lambda_1} \ge 2.
$$  
We make a case-by-case
argument for each $j$.

If $j=1$, the points $p,\lam_0,\lam_1$ are collinear and cannot give two obtuse angles.

If $j=2$, let $\mu'$ be the projection of $\mu$ to the plane containing
$p,\lam_0,\lam_1,\lam_2$. The four points $\mu',\lam_0,\lam_1,\lam_2$ can be arranged
cyclically around $p$, each forming an obtuse angle with the next.  A circle around $p$
cannot give four obtuse angles.


If $j=3$, assume that $\lam_0,\ldots,\lam_3$ are labeled in cyclic order around the line
$\op{aff}\{p,\mu\}$.  Consider the dihedral angle 
  $$
  \gamma=\gamma_i=\dih(\{p,\mu\},\{\lam_i,\lam_{i+1}\}).
  $$
By the law of cosines the angle $\gamma$ is given in terms of the edges of the spherical
triangle $a,b,c$ by
$$
  \cos c - \cos a \cos b = \sin a \sin b \cos \gamma.
$$
All the terms on the left-hand side are negative, so $\gamma =\gamma_i > \pi/2$.
This is impossible, as the sum of the four dihedral angles $\gamma_i$ is $2\pi$.
This completes the proof that $v(\bl[j])$ is the circumcenter.

Next we show that this point lies in the convex hull of $S_j$.
If $j=0$, there is nothing to show.  If $j=1$, the point is the midpoint of the convex hull.
If $j=2$, the point is the circumcenter of an acute triangle.  If $j=3$, the point is the
circumcenter of a simplex such that every face has positive orientation (see XX).  Thus,
in every case the point lies in the convex hull.

The distances $\norm{v(\bl[j])}{v(\bl[0])}$ are increasing, so it is enough to show
that $v(\bl[j])\ne v(\bl[j+1])$.  If we had equality, then the circumcenter $v(\bl[j])$
would have an equally close packing point $\lambda_{j+1}\in\Lambda$, which is impossible by the first part of the proof.
\end{proof}

\begin{lemma}   Let $\bl\in\Lambda(k)$.  Assume that $\norm{v(\bl)}{v(\bl[0])}<\sqrt2$.
Let $\bm$ be any permutation of the components of $\bl$.  Then $\bm\in\Lambda(k)$ and
 $v(\bl) = v(\bm)$.
\end{lemma}

\begin{proof} 
Since the sets $\Omega(\bl[j])$ satisfy (\ref{eqn:omega-dim}), we have that
$\Omega(\bl)\cap \op{conv}\{\lambda_0,\lambda_j\}$ is a point consisting of the singleton $\{v(\bl)\}$, which is the
circumcenter of the simplex with vertices $\{\lam_0,\ldots,\lam_k\}$.  This describes
the point $v(\bl)$ in a way that is independent of the permutation.

We check that the condition (\ref{eqn:omega-dim}) holds for the permutation $\mu$.
The proof of Lemma~\ref{lemma:v2} shows that the midpoint of $\op{conv}(\mu_0,\mu_1)$
lies in $\Omega(\bm[1])$ and since the inequalities are strict, some neighborhood
of this midpoint in the bisecting plane of $\{\mu_0,\mu_1\}$ lies in $\Omega(\bm[1])$.
Thus, $\dim(\Omega(\bm[1]))=2$.  We continue in this fashion, showing that each dimension
$\dim(\Omega(\bm[j]))$ drops by exactly one.
\end{proof}

\begin{lemma}\label{lemma:Rconv} Let $\bl = (\lam_0,\ldots,\lam_k)\in\Lambda(k)$.  Assume that $\norm{v(\bl)}{v(\bl[0])}<\sqrt2$.
For any permutation $\pi$, let $\pi(\bl)\in\Lam(k)$ be the permuted entry.  Let
$S(k)$ be the group of all permutations on $k+1$ letters.   Then
$$
\op{conv}\{\lam_0,\ldots,\lam_k\} = \bigcup \{ R(\pi(\bl)) : \pi\in S(k)\}.
$$
\end{lemma}

\begin{proof} Let $L = \{\lam_0,\ldots,\lam_k\}$.  We prove it by induction on $k$.
When $k=0$, the result is trivial.  Now assume $k>0$.

The circumcenter $v(\bl)$ of $L$ lies in the convex hull of these
point (see the proof of Lemma~\ref{lemma:v2}).  Thus, the left-hand side is the union
of the cones over the (k+1)-faces:
$$
\op{conv}(S) = \bigcup\{ \op{conv}(v(\bl),L\setminus \{\lam_i\}\mid i=0,\ldots,k) \}.
$$
The sets $L\setminus \{\lam_i\}$ can be identified with  cosets of $S(k)/S(k-1)$.
By induction $L\setminus \{\lam_i\}$ is the union of $R(\bm)$ as $\bm$ runs
over all permutations $S(k-1)$ of $L\setminus \{\lam_i\}$.
The result follows by induction.
\end{proof}

We remark that the Rogers's simplices $R(\bl)$ are compatible with the Voronoi
decomposition of space (by construction).  Also, by Lemma~\ref{lemma:Rconv}, under
mild restrictions on the circumradius, they can also be reassembled into simplices
with vertices at the centers of the packing (the Delaunay simplices).  

\section{Marchal's partition}

C, Marchal has introduced a new approach to the Keper conjecture that is manifestly superior to the one in \cite{Hales:2006:DCG}.  If his approach can be pushed
to completion, it would cut years off the Flyspeck project.

His articles claim to give
a {\it demonstration} of the Kepler conjecture \cite{marchal:2007}, \cite{marchal:2008}.  However, the
mathematically rigorous part of the article only gives a reduction
of the problem to an optimization problem in a finite number of
variables.  The method of gradient descent is then used to explore
the local minima of the optimization problem in finitely many variables.

Marchal's partition of space is a variant of Rogers's partition into simplices
$R(\bl)$.  The main part of construction is 
the decomposition obtained by truncating the Voronoi cells
by a ball of radius $\sqrt2$.  In a few carefully chosen situations he assembles the simplices
into a larger convex hull along the lines of Lemma~\ref{lemma:Rconv}.

Let $h(\bl[i]) = \norm{v(\bl[i])}{\lam_0}$.  The numbers $h(\bl[i])$ are increasing with $i$.

\begin{definition} Consider $\bl=(\lam_0,\ldots,\lam_3)\in \Lambda(3)$.
\hfill\break\smallskip  
{\bf The $0$-cell} of $\bl$ is
$$
R(\bl)_0 = \{x\in R(\bl) \mid \norm{v(\lam_0)}{x} > \sqrt2.
$$
\bigskip
{\bf The $1$-cell} of $\bl$ is 
$$
R(\bl)_1 = \op{conv}(v(\lam_0),T_1),\quad T_1 = \{x \in R(\bl) \mid \norm{v(\lam_0)}{x}= \sqrt2\}.
$$
\bigskip
{\bf The $2$-cell} of $\bl$ is
$$
\begin{array}{rll}
R(\bl)_2 &= \op{conv}(v(\bl[0]),v(\bl[1]),T_2),\quad \\
  T_2 &= \{x \in R(\bl)\cap \Omega(\bl[1]) \mid \norm{\lam_0}{x}=\norm{\lam_1}{x} =\sqrt2\}.
\end{array}
$$
\bigskip
{\bf The $3$-cell} of $\bl$ is defined to be empty unless 
$$
h(\bl[2]) <\sqrt2 \le h(\bl[3]).
$$
When this inequality holds,
there is a a unique point $p(\bl[2])$ in
$\op{conv}(\bl[2],\bl[3])$ at distance exactly $\sqrt2$ from $\lam_0$.  
Define the $3$ cell to be
$$
R(\bl)_3 = \op{conv}\{p(\bl[2]),\lam_0,\lam_1,\lam_2\}.
$$
\bigskip
{\bf The $4$-cell} of $\bl$ is defined to be empty unless
$$
h(\bl[3]) <\sqrt2.
$$
When this inequality holds, define the $4$ cell to be
$$
R(\bl)_4 = \op{conv}\{\lam_0,\lam_1,\lam_2,\lam_3\}.
$$
\end{definition}

Note the the $0$, $1$, and $2$-cells are always subsets of $R$.  However, the $3$ and
$4$-cells are finite unions of Rogers simplices.

\begin{lemma}  The $0,\ldots,4$-cells of all Rogers simplices $R(\bl)$, for $\bl\in \Lam(3)$
give a partition of $\ring{R}^3$.  An $i$-cell is never equal to a $j$-cell when $i\ne j$.
Two $3$-cells $R(\bl)_3 = R(\bm)_3$ are equal exactly when
$(\lam_0,\lam_1,\lam_2)$ is a permutation of $(\mu_0,\mu_1,\mu_2)$.  Two $4$-cells are
equal exactly when $\bl$ is a permutation of $\bm$.
\end{lemma}

\begin{proof}  First we will prove that cells are either disjoint or equal.  At the same
time, we determine when $4$-cells or $3$-cells are equal to one another.
By Lemma~XX, the $4$-cell $R(\bl)_4$ is a union of the Rogers simplices
$R(\bm)$, as $\bm$ runs over permutations of $\bl$. 
 Two $4$-cells
that meet are equal. 
Every $4$-cell comes from a simplex
$R(\bl)$ with $h(\bl)<\sqrt2$.  This condition gives $R(\bl)_i=\emptyset$, for $i=0,1,2,3$.

Similarly, the $3$-cell is a union
of the convex hulls $\op{conv}(p(\bm[2]),R(\bm[2]))$ as $\bm[2]$ runs over permutations of $\bl[2]$.  Note that the point $p(\bm[2])=p(\bl[2])$ is independent of the permutation, since
it is determined as the point at distance $\sqrt2$ from $\lam_0$ along the line through
from $v(\bm[2])=v(\bl[2])$ perpendicular to the plane $\op{aff}\{\lam_0,\lam_1,\lam_2\}$
(in the half-space of $\bl[3]$). Two $3$-cells that meet are equal and come from parameters that are permutations of
one another as described.   The intersection $R(\bl)_3$ cannot meet $R(\bl)_i$, for $i<3$, in
a set of full dimension, because the plane $P(\bl[2])=\op{aff}\{\lam_0,\lam_1,p\}$ separates them.

The $2$-cell $R(\bl)_2$ is separated from the $0$ and $1$-cells $R(\bl)_i$
by the cone $P(\bl[1])$ with apex $\lam_0$
passing through $T_2$. The $1$-cell is separated from $0$-cells the the sphere $P(\bl[0])$
of radius
$\sqrt2$, centered at $\lam_0$.

Finally, we show that every point $x$ in $\ring{R}^3$ belongs to a cell.  By the Rogers
partition, the point $x$ belongs to a Rogers simplex $R(\bl)$.  If $h(\bl)<\sqrt2$, then
$x$ belongs to a $4$-cell.  Otherwise $x\in R(\bl)$ belongs to $R(\bl)_i$ according to
which side of the plane $P(\bl[2])$, cone $P(\bl[1])$, and sphere $P(\bl[0])$ the point
lies, as described above.
\end{proof}


\section{Kissing number estimates}

This section shows how Conjecture~\ref{conj:m1} would follow from an explicit estimate
related to the Gregory-Newton problem


We define the following constants and functions.
$$
\begin{array}{lll}
\alpha &= \arccos(1/3)\\
m_1 &= (3\alpha-\pi)\sqrt2/(12\pi - 30\alpha)\\ %% K Marchal
m_2  &= (18\alpha-7\pi)\sqrt2/(144\pi-360\alpha)\\ %% M Marchal
M(h) = &=
\begin{cases}
 (\sqrt2-h) (h-1.3254) (9h^2 - 17 h + 3)/(1.627 (\sqrt2-1))& h\le\sqrt2\\
 0 & h >\sqrt2.
\end{cases}
\\
\end{array}
$$
We have 
\begin{equation}\label{eqn:km}m_1 - 12m_2 = \sqrt{1/2}\end{equation}
and
\begin{equation}M(1) = 1,\quad M(\sqrt2) =0\end{equation}


\begin{conjecture}[Marchal-2]\label{conj:m1} For any saturated packing $\Lam$, and
any $\lam_0\in\Lam$, we have
$$
\sum_{\bl=(\lam_0,\lam_1)\in\Lam(1)} M(h(\bl)) \le 12.
$$
\end{conjecture}

\begin{theorem}\label{theorem:mk}
Conjecture~\ref{conj:m1} implies the Kepler conjecture.
\end{theorem}

\begin{proof} 
We show that $A(\lam_0)  = 8 m_1 - \sum 8 m_2 M(h(\bl)) -\op{vol}(\Omega(\lam_0))$ is fcc-compatible and negligeable.  The
result then follows from \cite[Lemma~3.3]{Hales:2006:DCG}.  This is fcc-compatible directly
by equation~(\ref{eqn:km})
and Conjecture~\ref{conj:m1}.  The issue is to prove it negligeable.  Explicitly, we need
to show there exists a constant  $C$ such that or all $h\ge 1$ and all $x\in\ring{R}^3$:
\begin{equation}\label{eqn:neg}
\sum 8m_1 - \sum (\sum 8 m_2 M(h)) - \sum \op{vol}(\Omega(\lam))) \le C h^2,
\end{equation}
where the outer sum runs over $\lam\in \Lam(x,h)$.


Define the {\it total solid angle} and the {\it basic dihedral angles} of $i$-cells as follows for
$i>0$.  The total solid angle is the sum of the solid angles of the $i$-cell, summed
over all the extreme points of the cell that belong to $\Lambda$.  For a $0$-cell
the total solid angle is zero. For $1$ and $2$-cells,
it is just the solid angle of the cell at $\lam_0$.  For a $3$-cell, it is the sum of the
solid angles at $\lam_0,\lam_1,\lam_2$.  And for a $4$-cell, it is the sum of the solid
angles at $\lam_0,\ldots,\lam_3$. Write $\op{tsol}(X)$ for the total solid angle of a cell $X$.


The  basic dihedral angles of a cell $X = R(\bl)_i$ are indexed by the set $D(X)$ of
edges $\op{conv}(\lam_0,v(\bl[1]))$ that lie on the boundary of the cell.  The value
$\dih(X,d)$ of
of the basic dihedral angle indexed by $d$ is just the dihedral angle along that edge
of the boundary.  The indexing set $X$ is empty for  $0$ and $1$-cells.
The (single) basic  dihedral angle of a $2$-cell is
the same as the radian angle subtended by the arc $T_2$ in the construction of $2$-cells.
The basic dihedral angles of a $3$-cell are the dihedral angles along the
six directed edges $(\lam_i,\lam_j)$, $i,j=0,1,2$.  The basic dihedral angles of a $4$-cell are
the dihedral angles along any directed edge.  Each element $d\in D(X)$ also determines
the real number $h(d) = \norm{v(\bl[1])}{\lam_0}$.

We have the Marchal's fundamental estimate for any cell $X$:
\begin{equation}\label{eqn:mfe}
\op{vol}(X) \ge \left(\frac{2m_1}{\pi}\right) \op{tsol}(X) - \left(\frac{4m_2}{\pi}\right)
\sum_{d\in D(X)} \dih(X,d) M(h(d)).
\end{equation}
We write $\op{vol}_e(X)$ for the volume estimate appearing on the 
right-hand-side, to write this in the form:
\begin{equation}\label{eqn:vole}
\op{vol}(X) \ge \op{vol}_e(X).
\end{equation}
(Note that this is an inequality in at most six variables; the most difficult case to prove
is that of a $4$-cell.)  The volumes and solid angles and so forth are given by the 
explicit formulas.

Sum the inequality~(\ref{eqn:mfe}) over all cells in a large ball $B(x,r)$ to get an
inequality of the form $0\ge T_1 + T_2 + T_3$ for three sums $T_i = T_i(r)$.  We compare this term-by-term
with the three terms $T'_i(r)$ of the desired equation~(\ref{eqn:neg}).  The solid angles around each point sum to
$4\pi$, and the dihedral angles around each directed edge sum to $2\pi$, so we see that
the three terms satisfy
$$T'_i(r) \le T_i(r) + C r^2,$$
for some constant $C$ and error term $C r^2$ coming from the boundary effects of the cells $X$ that meet the boundary of $B(x,r)$.  The result follows.
\end{proof}

\subsection{Old proof of negligeability}

For reference, we insert the old proof of negligeability.\FIXX{edit}

Let $B(x,r)$ be the closed ball of radius $r\in\ring{R}$ centered
at $x$.  Let $\Lambda(x,r)=\Lambda\cap B(x,r)$.

Recall from Definition~\ref{def:negligible} that a function
$A:\Lambda\to\ring{R}$ is said to be {\it negligible} if there is a
constant $C_1$ such that for all $r\ge1$,
   $$\sum_{v\in\Lambda(x,r) } A(v) \le C_1 r^2.$$
%
 \indy{Index}{negligible}


Recall the function $A: \Lambda\to\ring{R}$ given by
Equation~\ref{eqn:A}.  Explicitly, let
   $$A(v) = A_0(\Lambda,v),$$
where $A_0$ in turn depends on functions $A_1$ and $\sigma$, as
determined by Equations~\ref{eqn:A1} and \ref{eqn:a1-sigma}, and
Definition~\ref{def:sigma}.

\begin{claim}\label{claim:negbounds}
    The absolute values of the functions $\sol$, $\op{volan}$, and $\op{vol}$
are bounded on the set of all quarters and
    quasi-regular tetrahedra.\FIXX{Give a proof somewhere. Give a reference.}
\end{claim}


\begin{lemma}\guid{SVGACAQ}\label{lemma:A1bound}  $A_1(v,Q,c)$ is bounded by a constant that is
independent of the choice of packing $\Lambda$, simplex
$Q$, context $c$, and vertex $v\in\Lambda$.
\end{lemma}

\begin{proof} By the definition of $A_1$:
    $$
    A_1(v,Q,c) \le \sol(v,Q)/3 + |\sigma(v,Q,c)|/(4\doct).
    $$
By Claim~\ref{claim:negbounds}, it is enough to bound $|\sigma|$.
For all contexts and choices of simplex, we have
    $$|\sigma(v,Q,c)|\le \sum_{i=1}^4 |\op{svan}(v_i,S)| +
        |\op{sv}_0(v,S)| 
      + |\op{sv}_0(\hat v,S)|.
    $$
It is thus enough to bound $|\op{svan}|$ and $|\op{sv}_0|$.  Referring
to their definitions, it is enough to bound $|\sol(v,S)|$,
$|\op{volan}(v,S)|$, and $\op{vol}(\op{VC}(v,S))$. The first two
are bounded by Claim~\ref{claim:negbounds}.  The set
    $$\op{VC}(v,S) = \op{VCM}(\Lambda,v)\cap \op{conv}(S)$$
is the intersection of two measurable sets.  Hence it is a
measurable subset of $\op{conv}(S)$.  The volume of $\op{VC}(v,S)$ is
dominated by that of $\op{conv}(S)$, which is given by
Claim~\ref{claim:negbounds}.  The result follows.
\end{proof}

\begin{lemma}\guid{FNJPVVT} \label{lemma:quarterfiber}
The number of quarters at a vertex is bounded by a constant that
is independent of the packing and the vertex $v$.
\end{lemma}

\begin{proof}  If $Q=\{v,v_1,v_2,v_3\}$, then $Q\subset
\Lambda(v,2\sqrt2)$.  The set $\Lambda(v,2\sqrt2)$ is finite by
Lemma~\ref{lemma:Lambda-finite} and is bounded by a constant $n$
that is independent of $v$ and $\Lambda$.  So $Q$ is an element of
the powerset of $\Lambda(v,2\sqrt2)$, which has cardinality at
most $2^n$.
\end{proof}

\begin{lemma}\guid{DGLKVWJ}\label{lemma:qr2} There exists a constant $C$ (independent of the
packing and $p\in\ring{R}^3$) such that the number of
quarters in the $Q$-system that have at least one vertex in
$B(p,r)$ and at least one vertex out is at most $C r^2$, for all
$r\ge 1$.
\end{lemma}

\begin{proof}  For each such quarter, pick a vertex $v_Q=v=(v_x,v_y,v_z)\in
\Lambda(p,r)$.  By Lemma~\ref{lemma:quarterfiber}, The map
    $$
    Q\mapsto (\lfloor 2(v_x-p_x)\rfloor, \lfloor
    2(v_y-p_y)\rfloor,\lfloor 2(v_z-p_z)\rfloor)
    $$
is a finite to one map to $\ring{Z}^3 \cap ( B(0,2r+1) \setminus
B(0,2r-k))$ (for some $k\ge 0$) whose fibers have bounded
cardinality independent of the data.  The result now follows from
Lemma~\ref{lemma:Zr2}.
\end{proof}

\begin{lemma}\guid{PZQEATR}\label{lemma:negA1} There exists a constant $C$ that is independent of
the packing and the point $x\in\ring{R}^3$ such that for
all $r\ge 1$ we have
$$\sum_{v\in\Lambda(x,r)} \sum_{Q\in\CalQ(\Lambda,v)}
      A_1 (v,Q,c) \le C r^2$$
\end{lemma}

\begin{proof}
Each quarter $Q=\{v_1,v_2,v_3,v_4\}$ in the $Q$-system occurs in
four sets $\CalQ(\Lambda,v_i)$.  By
Lemma~\ref{lemma:A1-cancel} the sum cancels, except when some
vertex of $Q$ lies inside $\Lambda(x,r)$ and another lies outside.
The number of such quarters is at most quadratic in $r$ by
Lemma~\ref{lemma:qr2}.  The contribution of each such quarter at
each vertex is at most the constant of Lemma~\ref{lemma:A1bound}.
The result follows.
\end{proof}

\begin{theorem}\label{lemma:negligible}
The function $A$ of Equation~\ref{eqn:A} is negligible.
\end{theorem}

\begin{proof}   
Referring to the definition in Equation~\ref{eqn:A1}, we find that
the terms coming from $A_1$ are negligible by
Lemma~\ref{lemma:negA1}.  Thus, it is enough to prove the
negligibility of the function $A'$ defined by
      $$
      A'(\Lambda,v) = -\op{vol}(\Omega(\Lambda,v))+
         \op{vol}(\op{VCM}(\Lambda,v)).$$
%The Voronoi cells partition $\ring{R}^3$, as do the $V$-cells. 
We
have $\Omega(\Lambda,v)\subset B(v,2)$ and by
construction, 
$\op{VC}(\Lambda,v)\subset B(v,2)$. 

Let 
  $$
  B_2(\Lambda,x,r) = \{y \in B(x,r)\mid \exists v\in \Lambda.\ 
         \norm{v}{y} < 2\}.
  $$
Hence the Voronoi cells with
$v\in \Lambda(x,r)$ cover $B_2(\Lambda,x,r-2)$ (except for a null set
consisting of the union of planes equidistant from two vertices in
$\Lambda(x,r+2)$). Moreover, the $V$-cells $\op{VCM}(\Lambda,v)$
with $v\in \Lambda(x,r)$ are contained in $B_2(\Lambda,x,r+d)$.  Hence by
the volume formula for a ball, we have
   $$
   \begin{array}{lll}
   \sum_{v\in\Lambda(x,r)} A'(v) &\le 
     -\op{vol}\,B_2(\Lambda,x,r-2)
      +\op{vol}\,B_2(\Lambda,x,r+d)\\&\le 
       -\op{vol}\,B(x,r-2)
      +\op{vol}\,B(x,r+d)\\& \le 
      C r^2
      \end{array}
   $$
for some constant $C$. This completes the proof.
\end{proof}



\section{More kissing number estimates}

This section shows how to improve on the estimates of the previous section
by combining various Marchal cells into {\it supercells}.

\begin{definition}
Set
$$
\begin{array}{lll}
  h_0  &= 1.26\\
  h_+  &= 1.3254\\
\end{array}
$$
Let $L:[1,\sqrt{2}]\to\ring{R}^2$ be the linear function interpolating the
values:
$$
L(1) = 1\quad L(h_0) = 0.
$$
Let $h_- = 1.23175\ldots$ be the unique root of the quartic polynomial
$M(h)-L(h)$ lying in the interval $[1.2,1.3]$.
Define the function $M_2:[1,\sqrt{2}]\to\ring{R}$ by
$$
M_2(h) = \max(0,\min(M(h),L(h))) = 
\begin{cases}
  M(h) & h \in [1,h_-]\\
  L(h) & h \in [h_-,h_0]\\
  0 & h \in [h_0,\sqrt2]\\
\end{cases}
$$
\end{definition}

The function $M_2$ is continuous and piecewise smooth.  We have
$M_2(h)\le M(h)$ except when $h\in [h_-,h_+]$.  The aim of this section is to prove a variant of Theorem~\ref{theorem:mk} that uses the function $M_2$ rather than $M$.  For this, we need to combine cells into larger groups, called supercells.

\begin{definition}
Let $X$ be a $k$-cell, with $k\ge 2$.  Let $e$ be an edge of $X$ that
has an endpoint at a vertex of $\Lambda$.  The edge is determined by
a pair of point $\lambda_1,\lambda_2\in\Lambda$.  We call
$\norm{\lambda_1}{\lambda_2}/2$ the {\it reduced length} of the edge.
(For a $2$-cell this is the actual length of the edge, but for a $3$-cell
or $4$-cell this is half the length of the edge.)
We say the edge is {\it critical} if the reduced length lies in the
range $[h_-,h_+]$.
Let $X$ be a $k$-cell with a critical edge.  The weight of $X$  
is $1/m$, where $m:1..6$ is the
number of critical edges on $X$.
(A $2$-cell with a critical edge has weight $1$; a $3$-cell can have
weights $1$, $1/2$, $1/3$; a $4$-cell has possible weights $1/m$, $1\le m\le 6$.)
\end{definition}

\begin{definition}
Let $e$ be a critical edge of a $k$-cell for some $k\ge 1$.
A supercell is a formal linear combination
$$
\sum_X [X] w(X),
$$
where $w(X)$ is the weight of $X$ and the sum runs over all cells with an edge along $e$.  If $Z = \sum_X [X] w(X)$ is a formal linear combination of cells, define
$$
\begin{array}{lll}
\Gamma(Z) &= \sum_S \gamma(X) w(X),\quad\hbox{ and }\\
\gamma(X) &=  (\op{vol}(X)-\op{vol}_e(X)).
\end{array}
$$

\end{definition}

\begin{theorem}\label{lemma:superineq} 
Let $Z$ be any supercell.  Then $\Gamma(Z)\ge 0$.
\end{theorem}

The proof of this theorem will occupy the rest of the section.  Before giving the proof, we give its relation to the Kepler conjecture.


\begin{conjecture}\label{conj:m2} For any saturated packing $\Lam$, and
any $\lam_0\in\Lam$, we have
\begin{equation}\label{eqn:M12}
\sum_{\bl=(\lam_0,\lam_1)\in\Lam(1)} M_2(h(\bl)) \le 12.
\end{equation}
\end{conjecture}

\begin{theorem}\label{theorem:mk2}
Conjecture~\ref{conj:m1} implies the Kepler conjecture.
\end{theorem}

\begin{proof}  For cells $X$ that do not form part of a supercell,
the proof is just as in the proof of Theorem~\ref{theorem:mk2}.
Each cell $X$ that has a critical edge belongs to $1/w(X)$ different
supercells, each with weight $w(X)$.  In supercells, we replace inequality
(\ref{eqn:vole}) with Theorem~\ref{lemma:superineq}.
\end{proof}

Now we turn to the proof of Theorem~\ref{lemma:superineq}.
We fix one critical edge and consider the supercell of all cells around it.  Consider all the faces along the edge $\{\lambda_0,\lambda_1\}$, consisting of three vertices $\{\lambda_0,\lambda_1,\lambda_2\}$ with circumradius less than $\sqrt2$.  Call such faces {\it blades}.  The proof will be divided into cases, according to the number of blades along the critical edge.

We call a $4$-cell a quarter, when it has exactly one
critical edge and all other edges of the simplex have length
at most $2 h_-$.
By calculations\footnote{XX}, if $X$ is any cell, then
$$
 \gamma(X) \ge 0,
$$ 
except possibly when $X$ is a quarter.  Thus, it
suffices to prove the inequality~(\ref{eqn:superineq})
when there is at least one quarter along the critical
edge.  We note that the weight of any quarter is $1$.

Adjacent to a $4$-cell along a blade is always a $2$-cell or $3$-cell.
Adjacent to a $2$-cell is a $3$-cell.

The $2$-cells appear in pairs along a critical edge,
one stacked on the other along the edge.

\subsection{two blades}

We may assume that the two blades are edges of a quarter
$X_4$. 
The azimuth angle of what remains outside the quarter
is greater than $\pi$.  Thus, there must be a $3$-cell
along each face.  Let $X_3$ be one of these $3$-cells.
It has weight $1$.
Then $$\Gamma(Z)\ge \gamma(X_4)+\gamma(X_3)\ge 0,$$
by a calculation\footnote{XX}.

\subsection{three blades}

Every $4$-cell along the critical edge has azimuth
angle\footnote{XX} at most $2.8$.  Assume a quarter with $\gamma(X) <0$;
this implies\footnote{XX} 
its azimuth angle is at most $1.65$.
As
$$
2 (1.65) + 2.8 < 2 \pi,
$$
the quarter $X$ is unique.  If there is a $3$-cell adjacent
to it, we argue as in the case of two blades.  Thus, without loss of generality, there are $4$-cells adjacent to the quarter $X$.  One of the $4$-cells $Y$ has angle at least
$$
(2\pi - 1.65)/2 > 2.3.
$$
Such a $4$-cell has weight $1$.
We have\footnote{XX},
$$
\Gamma(Z)\ge \gamma(X) + \gamma(Y) \ge -0.0056 + 0.0057 \ge0.
$$

\subsection{six or more blades}

Let $X_1,\ldots,X_r$ be all the $k$-cells between a pair of
blades.  (This will be either a single $4$-cell, or a stacked pair of $2$-cells sandwiched between $3$-cells along
the two blades.)  Calculations\footnote{XX} give
$$\sum_{i=1}^r \gamma(X_i) w(X_i) \ge 0.2147 - 0.20495\sum_{i=1}^4 \op{azim},$$
where the azimuth angles runs between the two blades.
We sum this inequality over the supercell, blade by blade:
$$
\Gamma(Z) \ge 6 0.2147 - 0.20495 (2\pi) \ge 0.
$$


\subsection{five blades}


A quarter satisfies an inequality of the form\footnote{The details here need to be checked more fully.}
$$
\gamma(X) \ge a - b\, \op{azim}(X),
$$
and any other material $X_1,\ldots,X_r$
between a pair of blades
satisfies a similar inequality:
$$
\sum \gamma(X_i)w(X_i) \ge a' - b\,\op{azim}.
$$
If there are $j$-quarters $(j\ge1)$, we get
$$
\Gamma(Z) \ge j a + (5-j) a' - b\, (2\pi) \ge 0.
$$

\subsection{four blades}

We consider subcases, according to the number of quarters along the critical edge.

If there are four quarters, we use\footnote{The details here need to be checked more fully.}
$$
\gamma(X) \ge a - b\, \op{azim}(X),
$$
and
$$
\Gamma(Z) \ge 4 a  - b\, (2\pi) \ge 0.
$$

If there are three quarters. XX FINISH.

If there are two quarters, the quarters might be adjacent or not.  Assume first that they are not adjacent.  Then
on the alternate simplices, all but possibly the edge
opposite the critical edge has length in $[2,2h_-]$.
Calculations\footnote{XX} of the opposite edge shows that
one of these edges has length at most $2h_-$, making it a quarter, contrary to the assumption that there are only two quarters.

Now consider the case of two quarters that are adjacent.  We argue as in the previous case, if all of the edges of the blades, except the diagonal, have lengths in $[2,2h_-]$.  Assume an edge on a blade of length at least $2h_-$.  Then calculations\footnote{XX} give the result.

Finally, if there is a single quarter $X$, 
$$
\gamma(X) \ge a - b\, \op{azim}(X),
$$
and any other material $X_1,\ldots,X_r$
between a pair of blades
satisfies a similar inequality:
$$
\sum \gamma(X_i)w(X_i) \ge a' - b\,\op{azim}.
$$
We get
$$
\Gamma(Z) \ge  a + 3 a' - b\, (2\pi) \ge 0.
$$

\section{bounding the number of spheres}

The purpose of the remainder of this work is to prove the inequality~(\ref{eqn:M12}).   Let $L_2(h) = \max(M_2(h),L(h))$.  We will actually aim for the 
following stronger
inequality, although the weaker inequality would suffice for the proof of the Kepler conjecture.  This stronger inequality has the advantage of being piecewise linear.

\begin{conjecture}
\begin{equation}\label{eqn:L12}
\sum_{\bl=(\lam_0,\lam_1)\in\Lam(1)} L_2(h(\bl)) \le 12.
\end{equation}
\end{conjecture}
Since $L_2(h) = M_2(h) = 0$, for $h\ge h_0$, we may discard all summands for which $h(\bl)\ge h_0$.  Since $L_2(h)\le 1$, it is clear that the inequality holds whenever the number of summands is at most $12$. The following is a variant of a lemma of Marchal.


\begin{lemma}  If the number of nonzero summands is $15$ or greater, then inequality~\ref{eqn:L12} holds.
\end{lemma}

\begin{proof} 
Consider a configuration with $15$ or more nonzero summands, $\lambda_1,\ldots,\lambda_N$. 
We fix the origin $\lambda_0=0$.  Set $h_i = \normo{\lambda_i}/2$.  Set
$$
a(h) = \arccos(h/2) - \pi/6.
$$
On the unit sphere,  consider the disks $D_i$ of radii $a(h_i)$, centered at $\lambda_i/\normo{\lambda_i}$.  These disks do not overlap; this follows from the easy inequality 
$$
a(h_i) + a(h_j) \le \op{arc}(2h_i,2h_j,2).
$$
Let $P_i$ be the half-space containing the origin, bounded by the plane through the circular boundary of $D_i$.  The intersection of these half-spaces is a convex polytope with faces $F_i$.  The radial projection of $F_i$ to the unit sphere is a spherical polygon $R_i$ containing $D_i$.  Let $n_i$ be the number of sides to the polygon $R_i$.  The area of $R_i$ is at least the area $A(a(h_i),n_i)$ of the smallest $n_i$-gon containing $D_i$.  The smallest spherical $n$-gon containing a disk of radius $a$ is regular.  This leads to the formula
$$
A(a,n) = 2\pi - 2 n (\arcsin(\cos(a)\sin(\pi/n))).
$$
We can verify directly that
$$
A(a(h),n) \ge c_0 + c_1 n + c_2 L_2(h),\quad
n = 3,4,\ldots,\quad 1\le h\le h_0,
$$
where
$$c_0 = 0.6327,\quad c_1 = -0.0333,\quad c_2 = 0.4754.$$
The number of faces, edges, and vertices of the convex
polyhedron satisfy the Euler relation $\sum_i n_i \le (6N-12)$.
Summing over $i=1..N$, for $N\ge 15$, we get the follow
estimate on $\bar L_2 = \sum_i L_2(h_i)$:
$$
\begin{array}{lll}
4\pi &= \sum_i\op{area}(R_i)\\
     &\ge \sum_i A(a(h_i),n_i) \\
     &\ge c_0 N +c_1\sum_i n_i + c_2 \bar L_2\\
     &\ge c_0 N +c_1 (6N-12) + c_2 \bar L_2\\
     &\ge c_0 (15) +c_1 (78) + c_2 \bar L_2\\
11.94 &\ge \bar L_2. 
\end{array}
$$
\end{proof} 


\begin{lemma}\label{lemma:D'}  
If some $\lambda_1$ satisfies $\normo{\lambda_1}=2$ and
$\norm{\lambda_1}{\lambda'}\ge 2.52$ for all $\lambda'\ne0\in\Lambda$,
then  inequality~\ref{eqn:L12} holds.
\end{lemma}

\begin{proof}  We create
a larger disk $D_1'$ centered at $\lambda_1/2$ and argue otherwise
as in the previous proof.  We assume that we have $N\ge 13$.


By the conditions of the lemma, we may take 
$$a'=\arc(2,2,2.52)-a(1.26) \approx 0.797$$
for the arcradius of this disk.  We can verify directly that
$$A(a',n) \ge c_0 + c_1 n + c_2 L_2(1) + c_3$$
where $c_3 = 0.85$.
Then 
$$
\begin{array}{lll}
4\pi &= \sum_i\op{area}(R_i)\\
     &\ge A(a',n_1)+\sum_{i>1} A(a(h_i),n_i) \\
     &\ge  c_0 N +c_1\sum_i n_i + c_2 \bar L_2 + c_3\\
     &\ge c_0 N +c_1 (6N-12) + c_2 \bar L_2 + c_3\\
     &\ge  c_0 (13) +c_1 (66) + c_2 \bar L_2 + c_3\\
11.97 &> \bar L_2. 
\end{array}
$$
\end{proof}

