
%% XX issues.

\label{XX}\label{tarski:XX}

We mark warnings with the letters WW in 
the TeX file.

We mark serious issues with the letters XX.

\section{Notation}

We use the notation $]x,y[$ in trig.tex, then move away from it.

With functions $\sol$, $\op{sovo}$, $\op{sv}$, I am not
consistent about the order of the arguments.
The first argument should be $v$, then the simplex $S$,
then others such as truncation $t_0$ and $\lambda$.

\section{Spelling, Terminology}

Hyphenation on  quasi-, half-, semi-, non-, etc. needs to be made
consistent.  Two words or one, hyphenated or not?

The term sphere is used 243 times and ball 139 times. Which should it be?
Sphere packing problem, etc.  congruent balls, etc.
ball, sphere packing -> packing .

$$
\begin{array}{lll}
 \text{geometrical} &\to& \text{geometric}\\
\end{array}
$$


 1. Give exact citation on algorithms.

 2. Put archival material at arXiv, with better title, etc.



 non conventions:
 non-Euclidean, non-triangular, non-strict, non-star, non-aggregated
 rest omit the hyphen: nonoverlapping nonlinear, nonzero, nonpositive, nonnegative,
   nonempty, nonadjacent, nondegenerate


Spherical triangle area formula = Girard's theorem.

\section{Definitions}

Some definitions are repeated ($\Delta$, $\dtet$, etc.)
Mark these in some way.

Is the notation $(p_x,p_y,p_z)$ for coordinates 
of $p\in R^3$ consistent throughout?

Use the word simplex combinatorially for four vertices.
Use the word tetrahedron geometrically for the convex hull.

barrier should be closed. Intersecting segment should be open ended.
(Open ends prevent meeting at a vertex.)
Everything will be done up to a null set
anyway for measure related things.

Definition of clusters has changed.  The word should be deleted
in favor of standard component.  

The definition of corner has been deleted.

Distinguished edges and blades.  The adjective can modify either.




\section{Define before use}

Make sure circle, great circle are defined before use.
A good place would be in vector geoemtry chapter.

Define basic notions of $\op{aff}$, $\op{conv}$, $\op{cone}$
before first use, as in vector geometry chapter.


\section{Policy Things}

Here we list some things that have been changed.  Make sure that
all occurences have been eliminated.

There should be no mention of enumerated packings.

Indexing: Things before the ``AT'' symbol are used for ordering entries.
Things after used for display in the actual index.
Subentry is!  After vertical slash stuff gets passed to page
number.  \\index{Index}{policy|bold}  puts the page in bold.
\\index{Index}{policy|(} starts a subrange \\index{Index}{policy|)} ends a page range.

Make all Deltas take same args in same order (to simplify formal proofs).
$\chi$, $\Delta$, $\ups$ should appear only in tarski.tex. 

\section{Incomplete Things in Tarski}

Eventually, I want to separate out the Tarski stuff, quoting
only the results that are needed.  Quote by footnote. Index every use.

Need an introduction to Tarski. What it is all about.

Replace $y_i$ by $y_{jk}$ globally.


\section{Change in Tame Graph Definition}

We add a new type of aggregate: the pentagon-triangle quadrilateral.
It needs to be checked that all of the properties of tameness go
through for this quadrilateral.  This affects the linear programs.
There needs to be a new flag (pent-tri) on quadrilateral faces.
None of the other inequalities for quads are valid in this case.
It has to be run as a separate initial case.

We add a new condition on separated set.  We require that 
at each node of the separated set, we have $p_v\in\{3,4\}$.
This allows us to eliminate a couple of pages of analysis
that rule out $p_v=0,1,2$.  However, this is an incompatible
change in the tame graph spec.  The graph generator has to
be rerun.  It is possible that new tame graphs will arise.
This affects the Bauer-Nipkow formal proof.

\section{Things to do:}

\subsection{hypermap}
Describe the localization of a hypermap.  (Throw away everything
except for one contour loop.)

Describe the components of $Y(v_0,V,E)$ that lie interior to
a contour loop.  


\subsection{Ferguson's thesis}

Add this.



