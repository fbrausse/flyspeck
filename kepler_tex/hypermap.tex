\chapter{Hypermap}\label{chap:hypermap}
\indy{Index}{hypermap}%

\section{Background on Permutations}

\begin{definition} A \newterm{permutation} $f$ on a set $D$ is a bijection
$f:D\to D$.
\end{definition}

For example, the identity map $I_D$ on a set $D$,
\begin{displaymath}
I_D(x)=x \text{ for all } x \in D,
\end{displaymath}
 is a permutation.
If $f:D\to D$ is a permutation then there is an inverse function $f^{-1}:D\to D$
that is also a permuation.  
It satisfies
\begin{displaymath}
f f^{-1} = f^{-1} f = I_D.
\end{displaymath}
(This chapter uses product notation $f g$ for the composition of maps
$f\circ g$.)
If $D$ is a finite set, and two maps
$f,g:D\to D$ satisfy $f g = I_D$ on $D$, then $f$ and $g$ are permutations and are
inverses of one another:
\begin{displaymath}
f g = g f = I_D.
\end{displaymath}

Natural number powers  $f^k$ of a permutation $f:D\to D$ are defined
recursively by
\begin{displaymath}
f^0 = I_D,\quad\text{ and } f^{k+1} = f f^k.
\end{displaymath}
Integer powers $f^m$ of a permutation are defined as
$$f^m = f^i (f^{-1})^j,$$ where $m = i -j$.  This is well-defined.
The usual rule of exponents holds:
\begin{displaymath}
f^{a+b} = f^a f^b.
\end{displaymath}

If $f:D\to D$ is a permutation on a finite set $D$, then there is a smallest
positive integer $k$ such that $f^k=I_D$.  The integer $k$ is the \newterm{order}
of the permutation $f$.  If $f^m=I_D$ for any some $m$, then $m = k i$ for some
integer $i$, where $k$ is the order of $f$. The inverse $f^{-1} = f^k f^{-1} = f^{k-1}$ can be written as a
non-negative power of $f$.



\section{Definitions}



\begin{definition}[hypermap,~dart]\label{def:hypermap}  
  A hypermap is a finite set $D$, together with three functions
  $e,n,f:D\to D$ whose composition is the identity:
  \begin{displaymath}
e\ocirc n\ocirc f = I_D.
\end{displaymath} The
elements of $D$ are called \newterm{darts}.  The functions $e,n$ and
$f$ are called the \newterm{edge map}, the \newterm{node map}, and
the \newterm{face map}, respectively.  \indy{Index}{hypermap}%
\indy{Index}{dart}%
\indy{Index}{edge!map}%
\indy{Index}{node!map}%
\indy{Index}{face!map}%
\indy{Notation}{edgemapz@$e$ (edge map)}%
\indy{Notation}{nodemap@$n$ (node map)}%
\indy{Notation}{facemap@$f$ (face map)}%
\indy{Notation}{D@$D$ (dart)}%
\end{definition}

%\pdf{dart.pdf}{dart}{The arrowhead represents a dart.}
\begin{figure}[htb]
\centering
\szincludegraphics[width=2mm]{\pdfp/dart.eps}
\caption{This symbol represents a dart.}
\label{fig:dart}
\end{figure}

\begin{remark}\tlabel{rem:hypermap} A hypermap is an abstraction of
the concept of 
planar graph.  Place a dart at each angle of a planar graph.
One function, $f$, 
cycles counterclockwise around the angles of each face.  
Another function, $n$, 
rotates counterclockwise around the angles at each
node.  A third function, $e$, pairs angles at opposite ends of
each edge  (Figure~\ref{fig:hypermap_ex}).   The hypermap extracts
the data $(D,e,n,f)$ from the planar graph and discards the rest.
\indy{Index}{planar graph}%
\end{remark}

\begin{figure}[htb]
\centering
\szincludegraphics[width=80mm]{\pdfp/hypermap-ex.eps}
\caption{Darts mark the angles of a planar graph.  Darts may
be permuted about faces, nodes, and edges.}
\label{fig:hypermap_ex}
\end{figure}

By the background on permutations, $e,n,f$ are all permutations on $D$.
A hypermap satisfies 
\begin{equation}\tlabel{eqn:triality}
e \ocirc n\ocirc f = n\ocirc f\ocirc e = f\ocirc e\ocirc n = I_D.
\end{equation}
Inverted, this triality becomes
\begin{displaymath}
n^{-1} \ocirc e^{-1} \ocirc f^{-1} = (f \ocirc e \ocirc n)^{-1} = I_D.
\end{displaymath}
This inversion is the abstract form of the the duality between nodes
and faces in a planar graph.  Because of these symmetries in the
defining relation, there will be multiple versions of theorems about
hypermaps, all obtained from one proof by symmetry.


\begin{definition}[path] 
Let $D$ be a set (of darts), and let $S$ be a set of permutations of $D$.
A \newterm{path} with \newterm{steps} in $S$
from $x_0$ to $x_{k-1}$ is a list\footnote{The empty path $[]$ seems
to have an ancient origin: ``This is the path made known to me
when I had learned to remove all darts.'' --The Dhammapada} of
darts $[x_0;x_1;\ldots;x_{k-1}]$ such that for each $i$, $x_{i+1} = h_i x_i$,
for some $h_i \in S$  .    The
path is \newterm{injective} if $x_i=x_j$ implies $i=j$. 
The \newterm{dart set} of $L$ is $\{x_0,x_1,\ldots,x_{k-1}\}$.  A path \newterm{visits}
a dart $x$, if $x$ is an element of the dart set of $L$.
\end{definition}

\begin{notation}[$::$,~$\opat$]
Write $P[x_i:x_j]$ for $i<j$ for the subpath $[x_{i+1};\ldots;x_j]$ of
$P=[x_0;\ldots;x_{k-1}]$.  (The notation is ambiguous when the path is
not injective.)  Write $x::P$ to prepend to a list:
$[x;x_0;\ldots;x_{k-1}]$.  Write $\opat$ for the concatenation of
lists:
\begin{displaymath}
[a;\ldots;b] \opat [c;\ldots;d]  = [a;\ldots;b;c;\ldots;d].
\end{displaymath}
\end{notation}
\indy{Notation}{1@$::$ (list operation)}%
\indy{Notation}{P@$P$ (dart path)}%
\indy{Notation}{1@$[-:-]$ (dart subpath)}%
%\indy{Notation}{Z@$\opat$~(concatenation)}%


\begin{definition}[$\sim_S$]
Let $D$ be a set, and let $S$ be a 
set of permutations on $D$.
Define a relation on the set of darts by $x\sim y$ when there is a
path from $x$ to $y$ with steps in $S$.
\end{definition}

\begin{lemma}\guid{}\rating{50}\label{lemma:er} %\guid{QLPBIKV}
% wording changed by thales Jan 7, 2010.
Let $(D,e,n,f)$ be a hypermap and let $S$ be a  set of permutations.
Then for each $h_1,h_2\in S$, 
the relation $\sim_S$ is the same as the relation $\sim_T$, where
\begin{displaymath}
T = S \cup \{h_1h_2\}.
\end{displaymath}
Moreover, for each $h\in S$, 
the relation $\sim_S$ is the same as the relation $\sim_T$, where
\begin{displaymath}
T = S \cup \{h^{-1}\}.
\end{displaymath}
Also,  the relation $\sim_S$ (that is, $\sim_T$) is an equivalence relation.  
\indy{Index}{equivalence relation}%
\end{lemma}

\begin{proof} If $x\sim_S y$ then clearly $x\sim_T y$.  Conversely,
if $x\sim_T y$, where $T = S\cup\{h_1,h_2\}$, pick a path $P$ from $x$ to $y$ with steps
in $T$ that contains the fewest $h_1h_2$-steps.  

\claim{$P$ does not contain any $h_1h_2$-steps}.  Otherwise, a subpath $[\ldots;u;h_1h_2u;\ldots]$
of $P$ can be expanded to a path $[\ldots;u;h_2u;h_1u;\ldots]$ that contradicts the minimal
property of $P$.

This proves the first conclusion of the lemma.  Fix $h$ in a set of permutations $R$.
By an induction that uses the first conclusion,  for all $i$, $\sim_R$ equals the relation $\sim_{R[h,i]}$,
where $R[h,i] = R \cup \{h,h^2,\ldots,h^i\}$.  If $h\in S$ is an element of order $k$, 
and $T = S\cup\{h^{-1}\}$, then
the second conclusion follows because the following sets give the same relation:
\begin{displaymath}
S,\quad S[h,k-1] = T[h,k-2],\quad T.
\end{displaymath}

By repeated action of the previous conclusion, $\sim_S=\sim_T$, where 
$T = S\cup S^{-1}\cup \{I_D\}$, and where $S^{-1} = \{h^{-1}\mid h\in S\}$.
The unit path $[x]$ yields reflexivity of $\sim_T$.  Also, $T^{-1} = T$ gives the symmetry.  Finally, composition of paths gives transitivity.  Thus, $\sim_T$ (i.e., $\sim_S$) is an equivalence relation.
\end{proof}

\begin{definition}[combinatorial~component,~connected]
A \newterm{combinatorial component} of a hypermap $(D,e,n,f)$ is an 
equivalence class of the relation $\sim_S$, where
$S=\{e,f,n\}$. 
(See Lemma~\ref{lemma:er} for other sets that define the same equivalence classes.)  
Write $\#c$ for the
number of combinatorial components.  The hypermap is \newterm{connected} if
$\#c=1$.  \indy{Index}{Dhammapada}%
\indy{Index}{path}%
\indy{Index}{connected}%
\indy{Index}{component!combinatorial}%
\indy{Notation}{1@$\#c$~ (number of components)}%
\end{definition}





\begin{definition}[orbit,~node,~face,~edge] The \newterm{orbit} of $x\in D$ under a permutation $h$ on
a set $D$ is a set of the form $\{h^i x\mid i\in\ring{N}\}$.  A \newterm{node}
of a hypermap $(D,e,n,f)$ is the orbit of a dart $x\in D$ under $n$.  
A \newterm{face} is an orbit under $f$.  
An \newterm{edge} is an
orbit under $e$.  \indy{Index}{node}%
\indy{Index}{face}%
\indy{Index}{edge}%
\end{definition}

Write $\#h$ for the
number of orbits of a permutation $h$ on $D$.  
\indy{Notation}{h@$h$ (permutation)}%
\indy{Notation}{1@$\#h$~(number of orbits)}%


\begin{lemma}
Let $D$ be a finite set.  The orbit of $x\in D$ of a permutation $h:D\to D$
is the equivalence class of $x$ under the relation $\sim_S$, when $S=\{h\}$.
\end{lemma}

\begin{definition}[plain] A hypermap $(D,e,n,f)$ is \newterm{plain}
  (carefully note\footnote{It is deliberate play on the homophonous
    {\it plane} that  privileges writing over speech.  Every plane (hyper)graph may
be planar, but not all plain hypermaps are planar.}
  the  spelling!) when $e$ is an involution on $D$ (that is, $e^2 = I_D$).
  \indy{Index}{planar}%
\end{definition}




\begin{definition}[degenerate] A dart in a hypermap $(D,e,n,f)$ 
is degenerate if it is a
fixed point of one of the maps $e,n,f$; otherwise it is nondegenerate.  
%%It is nondegenerate otherwise.
\indy{Index}{dart!degenerate}%
\indy{Index}{dart!nondegenerate}%
\end{definition}

\begin{definition}[simple] 
A hypermap is \newterm{simple} if the intersection of each face with
each node contains at most one dart.  \indy{Index}{simple}%
\end{definition}


% Moved from cup05_tame.tex section on tame plane graphs. 9/5/07:
\begin{lemma}\guid{ZHQCZLX}\rating{50}\tlabel{lemma:nondegen} 
Let $(D,e,n,f)$ be a simple plain hypermap such that every face has
at least three darts.
Then $n$ has no fixed point.
\indy{Index}{fixed point}%
\end{lemma}

\begin{proof} For a contradiction, let $x$ be a fixed point of
$n$. 

\claim{The darts $e x$ and $f x$ lie in the same node and face, so are
equal in the simple hypermap.}  Indeed, they lie in the same node
because $n(f x) = e^{-1} x = e x$. They lie in the same face because
\begin{displaymath}f^2 (e x) = f (f e n x) = f x.\end{displaymath}
So $e x = f x$.

Thus, $f^2 (e x) = f x = e x$, and $e x$ lies on a
face with at most two darts.  This contradicts what is given.
\end{proof}




\section{Walkup}

To focus on a dart $x$ in a
hypermap, it can be useful to draw a hexagon around $x$ and place
the six darts $e x$,
$f x$, $e^{-1} x$, $n x$,  $f^{-1} x$, $e x$, $n^{-1} x$  at its corners
in Figure~\ref{fig:dart+}.  Some of these $7$ darts may be
equal to one another, even if the figure draws them apart.
Figure~\ref{fig:dart-fix} shows the layout of a degenerate dart.
\indy{Notation}{x@$x$ (dart)}%

\begin{figure}[htb]
\centering
\szincludegraphics[width=40mm]{\pdfp/dart+.eps}
\caption{A dart $x$ and its entourage}
\label{fig:dart+}
\end{figure}

\begin{figure}[htb]
\centering
\szincludegraphics[width=60mm]{\pdfp/dart-fix.eps}
\caption{A dart fixed under a face map.}
\label{fig:dart-fix}
\end{figure}

\subsection{single}

A \newterm{walkup} deletes
a dart from a hypermap and reforms the edge, node, and face
maps to produce a hypermap on the reduced set of darts.  Walkups
come in three flavors: edge walkups, face walkups,
and node walkups.

\begin{definition}[walkup,~degenerate]
The edge \newterm{walkup}
$W_e$ at  a dart $x\in D$ of a hypermap $(D,e,n,f)$ is the hypermap
$(D',e',n',f')$, where $D' = D\setminus\{x\}$ and the
the maps skip over $x$:
\begin{displaymath}
\begin{array}{lll}
f' y &= \text{ if } (f y =  x) \text{ then } f x \text{ else
} f y\\
n' y &= \text{ if } (n y = x) \text{ then } n x \text{ else
} n y\\
e' &= (n'\ocirc f')^{-1}
\end{array}
\end{displaymath}
A walkup at $x$ is said to be \newterm{degenerate} if the dart $x$ is
degenerate.  
\indy{Index}{walkup}%
\indy{Index}{edge!walkup}%
\indy{Index}{face!walkup}%
\indy{Index}{node!walkup}%
\indy{Notation}{Wh@$W_h$ (walkup)}%
\end{definition}

Figure~\ref{fig:walk} shows
the result of an edge walkup on the hexagon around a dart $x$.
The triality symmetry~\ref{eqn:triality}, applied to the definition
of edge walkups, yields the definition of
face walkup $W_f$ and node walkup $W_n$.  
% Figure~\ref{fig:walkfn} shows the result of the face and node
% walkups on the hexagon around a dart $x$.

At a degenerate dart $x$, all three walkups are equal:
$W=W_e=W_n=W_f$ (Figure~\ref{fig:walkdegen}).
\indy{Index}{walkup!degenerate}%
\indy{Notation}{x@$x$ (dart)}%

\begin{figure}[htb]
\centering
\szincludegraphics[width=80mm]{\pdfp/walk.eps}
\caption{The effect of a walkup at $x$}
\label{fig:walk}
\end{figure}


\begin{figure}[htb]
\centering
\szincludegraphics[width=80mm]{\pdfp/walkdegen.eps}
\caption{The effect of a walkup at a degenerate dart}
\label{fig:walkdegen}
\end{figure}


\begin{definition}[merge,~split]\tlabel{def:merge-split} 
Let $(D,e,n,f)$ be a hypermap, and let $h=n,e$, or $f$.  Let $\op{orbit}(h,x)$
denote the orbit of $x\in D$ under $h$.  Let $(D',e',n',f')$ be the hypermap obtained
from $(D,e,n,f)$ by the walkup $W_h$ at $x\in D$.
Let $h'=e',n',f'$, respectively, according to the choice of $h$.
The walkup $W_h$ at $x$ \newterm{merges} when the walkup joins the
orbit of $h$ through $x$ with another orbit.  That is, there is an orbit $O$ of some
$y\in D'$ under $h':D'\to D'$ of the form
\begin{displaymath}
O \cup\{x\} = \op{orbit}(h,x) \cup \op{orbit}(h,y),
\end{displaymath}
where $y\not\in \op{orbit}(h,x)$.
It \newterm{splits}
when the walkup splits the orbit at $x$ into two orbits.  That is, there are 
distinct orbits $O_1,O_2$ under $h'$ in the hypermap $(D',e',n',f')$ such that
\begin{displaymath}
\{x\}\cup O_1\cup O_2 = \op{orbit}(h,x).
\end{displaymath}
\indy{Index}{split}%
\indy{Index}{merge}%
\indy{Index}{orbit}%
\end{definition}

\begin{lemma}\guid{ZMFKZNH}\rating{150}\tlabel{lemma:merge-split} 
  Let $(D,e,n,f)$ be a hypermap and let $W_h$ be a nondegenerate
  walkup at a dart $x$.  Then $W_h$ merges or splits. Moreover, it merges if
  and only if $x$ and $y$ lie in distinct $h$-orbits, where
  $(h,y)=(f,e x)$,  $(e,n x)$, or $(n,f x)$.
\end{lemma}

\begin{proof} The walkup $W_f$ splits if and only if $f x$ 
(or $x$)
and $e x$ lie in the same $f$-orbit before the split. 
Figure~\ref{fig:split} makes this clear.
The other cases $h=e,n$ hold by triality.
\end{proof}


\begin{figure}[htb]
\centering
\szincludegraphics[height=90mm]{\pdfp/split.eps}
\caption{The face walkup at $x$ mixes $f$-orbits.  If it mixes a
single orbit, the orbit splits. If it mixes two separate orbits, the
orbits merge. }
\label{fig:split}
\end{figure}

The following is a useful way to tell if a walkup merges.


\begin{lemma}\guid{FKSNTKR}\rating{80}\tlabel{lemma:ng-merge}  
Suppose, in a simple plain hypermap $(D,e,n,f)$, that an edge $\{x,y\}$ consists
of two nondegenerate darts.  Then the walkup $W_f$ 
% (resp. $W_n$)  removed Jan 10, 2009.  Needed? Is it even true?
at $x$ merges.
\end{lemma}
\indy{Index}{merge}%

\begin{proof} 
The darts $f x$ and $e x$ lie in the same node: $n (f x) = e^{-1} x
= e x$. If they are also in the same face of a simple hypermap, then
$f x = e x = y$. So
\begin{displaymath}n y  = n f x = n f e y = y,\end{displaymath}
and $y$ is a fixed
point of $n$, hence degenerate, contrary to assumption.  
Thus, $f x$ and $e x$ are in different faces, and the walkup merges
by Lemma~\ref{lemma:merge-split}.  
\end{proof}


\subsection{double}
\indy{Index}{walkup!double}%

A double walkup is the composite of two walkups of the same type.  The
two darts for the two walkups are to be the members of an orbit of
size two (under $n$, $e$, or $f$).
%%XX?The first walkup is to be chosen so that it merges.  
By choosing the type of the walkups to be different from the type of
the orbit, the first walkup reduces the orbit to a singleton, forcing
the second walkup to be degenerate.

Here are some examples.
\begin{itemize}
\item A double $W_n$ along an edge deletes the edge and 
merges the two endpoints into
a single node (Figure~\ref{fig:doublenode}). 
\item A double $W_f$ along an edge 
deletes the edge and merges the two faces along the edge into
one (Figure~\ref{fig:doubleface}).
\item A double $W_e$ at a node of degree two
deletes the node and merges the two edges at the node into
one (Figure~\ref{fig:doubleedge}).
\end{itemize}


\begin{figure}[htb]
\centering
\szincludegraphics[width=90mm]{\pdfp/double-node-walkup.eps}
\caption{The double node walkup applied to an edge}
\label{fig:doublenode}
\end{figure}


\begin{figure}[htb]
\centering
\szincludegraphics[width=90mm]{\pdfp/double-face-walkup.eps}
\caption{The double face walkup applied to an edge}
\label{fig:doubleface}
\end{figure}


\begin{figure}[htb]
\centering
\szincludegraphics[width=80mm]{\pdfp/double-edge-walkup.eps}
\caption{The double edge walkup applied to a node}
\label{fig:doubleedge}
\end{figure}

\begin{figure}[htb]
\centering
\szincludegraphics[width=80mm]{\pdfp/double_edge.eps}
\caption{The double edge walkup preserves plainness.}
\label{fig:doubleplain}
\end{figure}


\begin{lemma}\guid{HOZKXVW}\rating{150}\tlabel{lemma:dwalk-planar}  
The three preceding double walkups carry plain
hypermaps into plain hypermaps.
\end{lemma}
\indy{Index}{hypermap!plain}%

\begin{proof} The walkups $W_n$ and $W_f$ preserve the orbit structure
of edges, except for dropping one dart.  By dropping both darts from
the same edge, one edge is lost and all others edges remain
unchanged.

Figure~\ref{fig:doubleplain} illustrates the double $W_e$.  The two
edges $\{x,e x\}$, $\{y, e y\}$ meeting the node are fused by the
double walkup into $\{e x, e y\}$, which is still an edge of size
two.
\end{proof}

\begin{remark}[reverse double walkup]\label{rem:reverse-double-walkup}
Double walkup transformations can be run in reverse.
Let $H'=(D',e',n',f')$ be
a hypermap and let $D\supset D'$ be a set that contains two additional elements
$x,y$.  Fix distinct elements $x',y'\in D'$.  Define $n,e:D\to D$ as follows.
\begin{displaymath}
\begin{cases} 
e x = y, &\\
e y = x,&\\
e z = z,&\text{otherwise.}\\
\end{cases}
\qquad\qquad
\begin{cases} 
n x' = x, &\\
n x =  n' x',&\\
n y' = y,&\\
n y =  n' y',&\\
n z = n' z, &\text{otherwise.}\\
\end{cases}
\end{displaymath}
Define $f$ by forcing the hypermap identity $e n f = I_D$.  
The edge $\{x,y\}$ has been inserted by a reverse double walkup.  The insertion points of the edge
into the hypermap
depend on the data $\{(x',x),(y',y)\}$.   Reverse double walkup transformations that
insert a node $\{x,y\}$ or a face $\{x,y\}$ into a hypermap are obtained similarly  by triality symmetry.
\indy{Index}{reverse double walkup}%
\end{remark}



\section{Planarity}
\indy{Index}{walkup}%
\indy{Index}{planarity}%

\begin{definition}[planar]  A hypermap is \newterm{planar} (note the
spelling!) when the Euler relation holds:
\begin{displaymath}\# n + \# e + \# f = \# D + 2\, \#c.\end{displaymath}
\indy{Index}{planar}%
\end{definition}


\begin{remark}\label{rem:Euler}   
The Euler relation for planar graphs can be translated into the
language of hypermaps.  Consider a connected planar graph that
satisfies the Euler relation for the alternating sum of Betti
numbers:
\begin{displaymath}b_0 - b_1 + b_2 = 2\end{displaymath} where $b_0$
is the number of vertices, $b_1$ the number of edges, and $b_2$ the
number of faces (including an unbounded face) of the planar
graph. The hypermap $(D,e,n,f)$, made from the planar graph in
Remark~\ref{rem:hypermap}, is plain, and the involution $e$ has no fixed points.  
Thus, $\# D = 2\#e$, according to the partition of $D$ into edges.  Moreover,
\begin{displaymath}\begin{array}{lll}
b_0 &= \# n\\
b_1 &= \# e\\
b_2 &= \# f\\
2b_1 &= \# D\\
1 &= \#c\\
b_0 - b_1 + b_2  &= \# n + (\#e - \#D) + \# f = 2\,\# c.
\end{array}
\end{displaymath}
Thus, the hypermap is also planar.
\indy{Index}{Euler relation} %
\end{remark}


\begin{lemma}\guid{TGJISOK}\rating{80}\label{lemma:dart-upper} 
Let $H$ be a connected plain planar hypermap such that every edge
has cardinality two.  Assume that there are at least three darts in
every node.  Then
\begin{displaymath}
\# D \le (6\, \#f - 12).
\end{displaymath}
\end{lemma}
\indy{Notation}{H@$H$ (hypermap)}%

\begin{proof}  In a plain planar hypermap, the Euler relation becomes
\begin{displaymath}6\, \#f - 12 = 3\,\#D - 6\,\#n,\end{displaymath}
so it is enough to show that
\begin{displaymath}
\# D \ge 3\,\#n.
\end{displaymath}
This follows directly by assumption: the set of darts can be
partitioned into nodes, with at least three darts per node.
\end{proof}


\begin{definition}[planar~index] The planar index of a hypermap is
\begin{displaymath}\iota = \# f + \# e + \# n - \# D - 2\,\#
c.\end{displaymath}
(A hypermap with null index is planar.)
\indy{Index}{hypermap!planar index}%
\indy{Notation}{ZZiota@$\iota$ (planar index)}%
\end{definition}

\begin{lemma}\guid{IUCLZYI}\rating{400}\tlabel{lemma:index} 
Let $x$ be a nondegenerate dart of a hypermap $(D,e,n,f)$. Let
$(D',e',n',f')$ be the result of the face walkup $W$ at $x$.  The
walkup changes the size of some orbits.
\begin{displaymath}
\begin{array}{lll}
%\text{\bf Non-degenerate dart $x$: }&\\
\# f' &=\# f +\op{split}_f  \\  
\# e'&=\# e \\
\# n'&=\# n \\
\# D'&=\# D - 1 \\
\#c'&=\# c + \op{split}_c\\
\iota' &= \iota + 1+\op{split}_f - 2\op{split}_c,\\
\end{array}
\end{displaymath}
where
\begin{displaymath}
\op{split}_f = \begin{cases}
1,&\text{if $W$ splits }\\
-1,&\text{if $W$ merges}\\
\end{cases}
\end{displaymath}
and $\op{split}_c=1$ if $e x$ and $f^{-1} x$ belong to different
combinatorial components after the walkup $W$, and $\op{split}_c=0$
otherwise. Moreover, a walkup at a degenerate dart preserves the
planar index.  \indy{Notation}{splitc@$\op{split}_c$}%
\indy{Notation}{splitf@$\op{split}_f$}%
\indy{Notation}{W@$W$ (walkup)}%
\end{lemma}

\begin{proof} The figures make this clear.
\end{proof}

\begin{lemma}\guid{BISHKQW}\rating{100}\tlabel{lemma:planar-index2}
Let $\iota$ be the index of a hypermap $(D,e,n,f)$, and let $\iota'$
be the index after a walkup $W_h$ at a dart $x$.  Then $\iota \le
\iota'$.
\end{lemma} 


\begin{proof} Without loss of generality, by triality symmetry, the
walkup is a face walkup.  If $\op{split}_c=0$, the inequality is
immediate by Lemma~\ref{lemma:index}.  If $\op{split}_c=1$, 
then $e x$ and $f^{-1} x$ lie in
different components after the walkup, hence also in different
faces.  Thus, the walkup splits by Lemma~\ref{lemma:merge-split}.
Hence  $\op{split}_f = 1$.  The result
follows by Lemma~\ref{lemma:index}.
\end{proof}


\begin{lemma}\guid{FOAGLPA}\rating{50}\tlabel{lemma:planar-nonpos}  
The planar index
of a hypermap is never positive.
\end{lemma}

\begin{proof}  An face walkup never decreases the index.  A sequence
of face walkups leads to the empty hypermap, which has
index zero.
\end{proof}


\begin{lemma}\guid{SGCOSXK}\rating{50}\tlabel{lemma:walkup-planar}
Walkups take planar hypermaps to planar
hypermaps.
\end{lemma}

\begin{proof}  
A planar hypermap has maximum index.  The walkup
can only increase the index, but never beyond its maximum.  
Thus, the index remains at its maximum value.
\end{proof}





\section{Path}

\subsection{contour}

\begin{definition}[contour~path,~contour~loop] A contour path from
$x_0$ to $x_{k-1}$ is a path $[x_0;x_1;\ldots;x_{k-1}]$ such that
$x_{i+1} = n^{-1} x_i$ or $f x_i$ for each $i<k$.  (That is, each
step in the path is clockwise step around a node or a
counterclockwise step around a face.)  If the contour path
$[x_1;\ldots;x_{k-1}]$ is injective and $x_0 = x_{k-1}$, then it is
a contour loop.  \indy{Index}{contour!path}%
\indy{Index}{contour!loop}%
\indy{Index}{loop}%
\end{definition}

\begin{remark} Figure~\ref{fig:hypermap_ex} constructs a hypermap from
a planar graph by drawing darts next to each angle.  In this
representation, the darts along a contour path lie to the left of
the corresponding planar graph edges.  For that reason, a shaded
region to the left of a curve depicts a contour path.
\end{remark}

\begin{figure}[htb]
\centering
\szincludegraphics[width=80mm]{\pdfp/shade_dart.eps}
\caption{A contour path as a sequence as dart is represented as a
shaded path.}
\label{fig:shade-dart}
\end{figure}

\begin{lemma}\guid{QZTPGJV}\rating{50} An injective contour path from
$x$ to $y$ can be constructed from an arbitrary contour path from
$x$ to $y$ by dropping some darts from the path.
\end{lemma}

\begin{proof} Repeatedly replace $[\ldots;a;b;\ldots;b;c;\ldots]$ with
$[\ldots;a;b;c;\ldots]$.
\end{proof}





\begin{lemma}\guid{KDAEDEX}\rating{100}\tlabel{lemma:connect-contour}  
Let $H$ be a hypermap.
If $x$ and $y$ are darts in the same combinatorial component of $H$ if and only if
there exists a contour path from $x$ to $y$.
\end{lemma}

\begin{proof} 
Combinatorial components are defined by an equivalence relation $\sim_S$, where
$S = \{e,n,f\}$.  By Lemma~\ref{XXD}, this is the same equivalence relation as
$\sim_T$, where $T = \{n^{-1},f\}$.  By the definition of the equivalence relation $T$,
$x\sim_T y$ if and only if there is a contour path from $x$ to $y$.
\end{proof}
\indy{Index}{component!combinatorial}%

\begin{definition}[complement] 
Let $(D,e,n,f)$ be a plain hypermap.
Let $P=[x;y;\ldots]$ be a contour loop that does not visit any node
twice in a plain hypermap.   (That is, the dart set of $P$ intersected with a node
is the dart set of a maximal subpath $[z;n^{-1}z;\ldots;n^{-k}z]$ of $n^{-1}$ steps.)
 Replace each maximal sublist of
$n^{-1}$-steps
\begin{displaymath}
[z;n^{-1} z; \ldots; n^{-k} z]
\end{displaymath}
with the sublist
\begin{displaymath}
[n^{-(k+1)} z;n^{-(k+2)} z;\ldots; n z]
\end{displaymath}
Concatenate these new sublists in reverse order.  By the relation $n f = f^{-1} n^{-1}$,
the transitions between the new sublists are $f$-steps.
The resulting contour loop $P^c$
is the \newterm{complement}. 
\end{definition}
\indy{Notation}{1@$*^c$ (complement)}

\begin{figure}[htb]
\centering
\szincludegraphics[width=70mm]{\pdfp/complement.eps}
\caption{The complement contour traces the remaining darts
at the same nodes as the original contour loop. }
\label{fig:contour-comp}
\end{figure}


\subsection{M\"obius}

\begin{definition}[M\"obius~contour] A M\"obius contour in a hypermap
$(D,e,n,f)$ is an
injective contour path $P=[x_0;\ldots]$ that satisfies
\begin{equation}
\tlabel{eqn:mobius}
x_j = n x_0\quad x_k = n x_i
\end{equation}
for some $0 < i\le j< k$ (Figure~\ref{fig:mobius}).
\indy{Index}{contour!M\"obius}%
\end{definition}


\begin{remark}
G. Gonthier devised the notion of M\"obius contour as a way to prove
the Four-Color theorem without appeal to topology.  (The Appel-Haken
proof of the Four-Color theorem relies on the Jordan curve theorem.)
This chapter uses a significant amount of material from ~\cite{Gonthier:2005:Four-Color}.
\end{remark}

\begin{figure}[htb]
\centering
\szincludegraphics[width=50mm]{\pdfp/mobius.eps}
\caption{A M\"obius contour}
\label{fig:mobius}
\end{figure}

\begin{figure}[htb]
\centering
\szincludegraphics[width=30mm]{\pdfp/3m.eps}
\caption{The face map on this hypermap gives a M\"obius contour with
three darts}
\label{fig:3m}
\end{figure}

\begin{remark} Heuristically, a M\"obius contour is a 
combinatorial M\"obius strip that
twists to make 
its left-hand side into
its right-hand side.  A planar hypermap has no such contour.  
Figure~\ref{fig:violate-jct}
redraws a violation of the Jordan curve theorem
as a M\"obius contour.   
\end{remark}

\begin{figure}[htb]
\centering
\szincludegraphics[width=80mm]{\pdfp/violate-jct2.eps}
\caption{A path that tunnels from the interior to the exterior
of a simple closed curve
is analogous to a M\"obius contour.}
\label{fig:violate-jct}
\end{figure}

\begin{figure}[htb]
\centering
\szincludegraphics[width=80mm]{\pdfp/mobius_contour.eps}
\caption{Some M\"obius contours}
\label{fig:mobius-contour}
\end{figure}






\begin{lemma}\guid{LIPYTUI}\rating{300}\tlabel{lemma:no-mobius}
A planar hypermap does not have a M\"obius contour.
\end{lemma}
\indy{Index}{hypermap!planar}%

\begin{proof} For a contradiction, assume that there exist planar
hypermaps with M\"obius contours.  An edge walkup carries
planar hypermaps into planar hypermaps. An edge walkup
at a dart that is not on the M\"obius contour carries the
M\"obius contour to a M\"obius contour 
and reduces the number of darts.  
In the M\"obius Condition~\ref{eqn:mobius},
a walkup at a dart that is not at position $0$, $i$, $j$, $k$
along the contour carries the M\"obius contour to a M\"obius contour
and reduces the number of darts. Thus, a counterexample with
the smallest possible number of darts contains no
darts except those on the M\"obius contour, and its only darts
are at positions $0$, $i=j=1$, $k=2$.

This is a three darted hypermap (Figure~\ref{fig:3m}.)  
The M\"obius condition, the
definition of contours, together with $e\ocirc n\ocirc f=I_D$ force
$e=n=f$, all permutations of order three.  This hypermap is not planar:
\begin{displaymath}\# e + \# n + \# f = 3~~\ne~~ 5 = \# D + 2\,
\#c.\end{displaymath}
\end{proof}



%\subsection{interior}
%\indy{Index}{interior}%
%
%\begin{definition}[interior]\label{def:interior} 
%A dart $y$ lies in the \newterm{interior} of a contour
%loop $L$ if there is a an injective contour path
%$x_0,x_1,\ldots,x_k=y$ such that $x_1 = f x_0$ (or $k=0$), and
%such that $x_i$ lies on the loop $L$ if and only if $i=0$.
%\indy{Index}{interior!contour loop}%
%\indy{Notation}{L@$L$ (contour loop)}%
%\indy{Notation}{y@$y$ (dart)}%
%\end{definition}
%

\begin{lemma}\guid{ILTXRQD}\rating{100}\tlabel{lemma:contour-path-type}
Suppose that a hypermap has no M\"obius contours. Let $L$ be a
contour loop.  Let $P$ be any injective contour path with at least
$3$ darts, that starts and ends on $L$, but visits no other darts of
$L$.  Then the first and last steps of $P$ are both of the same type
($n^{-1}$ or $f$).
\end{lemma}
\indy{Notation}{P@$P$ (contour path)}%

\begin{figure}[htb]
\centering
\szincludegraphics[width=80mm]{\pdfp/interior_nf.eps}
\caption{A path must enter and depart from a contour loop with the
same type of step.}
\label{fig:interior_nf}
\end{figure}


\begin{proof} The proof shows the contrapositive.  Suppose $P=[n x;f n
x;\ldots;n y;y]$.  The successor of $n x$ on $L$ is $x$.  Starting
at $x$, follow $L$ to $y$, and on to $n x$.  Follow $P$ back to $n
y$:
\begin{displaymath}
x::L[x:n x] \opat P[ n x;ny].
\end{displaymath}  
This is a M\"obius contour $x\ldots y\ldots n x\ldots n y$.

Suppose $P=[n x;x;\ldots;f^{-1} y;y]$.  Starting at $x$, follow $P$ to
$y$, then follow $L$ to $n x$, and on to $n y$.  This is a M\"obius
contour.\footnote{The second statement can also be deduced from the first statement
by the duality $(D,e,n,f)\leftrightarrow (D,e^{-1},f^{-1},n^{-1})$ that swaps
$f$-steps with $n^{-1}$-steps in a path.}
\end{proof}

%\begin{lemma}\guid{UMYSGDB}\rating{80}\tlabel{lemma:dart-interior}
%  Let $L$ be a contour loop on a plain hypermap without M\"obius
%  contours.  Assume a dart $x$ lies in the interior of the loop $L$.
%  Then every dart in its $f$-orbit lies in the interior of the loop.
%  Moreover, if the dart $x$ does not lie on the same node as any dart
%  in $L$, then every dart in the $n$-orbit of $x$ lies in the
%  interior of $L$.
%\end{lemma}

%\begin{proof} Let $P= [x_0;\ldots;x]$ be an injective path that
%  certifies that $x$ lies in the interior of $L$.  If $f x$ lies
%  along this path already or if it lies on $L$, then it is clearly
%  interior.  Otherwise, $[x_0;\ldots,x;f x]$ is a certifying path for
%  $f x$.  Similarly, use the certifying path $[x_0;\ldots;x;n^{-1}
%  x]$ for $n^{-1} x$.
%\end{proof}
%
%%
%\begin{definition}[interior~face,~node] A face or a node is interior
%  to a loop in a hypermap if all of its darts are interior.
%  \indy{Index}{interior!face}%
%  \indy{Index}{interior!node}%
%\end{definition}


\begin{lemma}\guid{ICJHAOQ}\rating{180}\tlabel{lemma:contour-f}
Suppose that a hypermap has no M\"obius contours.  Let $L$ be a
contour loop.  Then there does not exist a contour path
$[x_0;\ldots;x_k]$, for $k\ge 1$ with the following properties:
\begin{enumerate}
\item $x_i$ lies on $L$ if and only if $i=0$.
\item $x_1 = f x_0$.
\item $x_0$ and $x_k$ lie in different nodes.
\item Some dart of $L$ is at the node of $x_k$.
\end{enumerate}
\end{lemma}

%\begin{figure}[htb]
%  \centering
%  \szincludegraphics[width=40mm]{\pdfp/no_node_path.eps}
%  \caption{No path exists from a node of $L$ to the interior.}
%  \label{fig:no-node-path}
%\end{figure}

\begin{proof} Assume for a contradiction that the path $P$ exists.
Some subpath is injective and satisfies the same conditions.  Again,
without loss of generality, shrinking the path if needed, $k$ is the
smallest index for which the last two conditions are met.  Append
$n^{-1}$-steps to $P$ to reach a dart of $L$.  This is contrary to
Lemma~\ref{lemma:contour-path-type}.
\end{proof}

\begin{lemma}[three darts]\guid{EUXPBPO}\rating{ZZ}\label{lemma:3dart}  
Assume that each face of a hypermap  has at least three darts.
Then every contour loop that meets at least two nodes has at least
three darts.
\end{lemma}

\begin{proof} Let $P=[x;y]$ be a contour loop meeting two nodes.  Then
$y = f x$ and $x = f y$, so that the face has size two.
\end{proof}

\section{Generation}
\indy{Index}{generation}%

\subsection{quotient}
\indy{Index}{quotient}%

\begin{definition}[isomorphism] Two hypermaps $(D,e,n,f)$ and
$(D',e',n',f')$ are \newterm{isomorphic} when there is a bijection
$\varphi:D\to D'$ such that
\begin{displaymath}h'\ocirc \varphi = \varphi\ocirc h\end{displaymath}
for $(h,h')=(e,e'), (f,f'), (n,n')$.
\indy{Index}{isomorphic hypermaps}%
\indy{Notation}{ZZp@$\varphi$ (morphism of hypermaps)}%
\end{definition}


\begin{definition}[normal family]
Let $(D,e,n,f)$ be a hypermap. 
%Assume that 
%there are no darts fixed by $e$ 
%(so that $f x \ne n^{-1} x$ at each dart). 
Let $\cal L$ be a family of contour
loops.  The family $\cal L$ is  normal if the following
conditions hold of its loops. \begin{enumerate}
\item  No dart is visited by two different loops.
\item  Every loop visits at least two nodes.
\item  If a loop visits a node, then every dart at that node is visited
by some loop.
\end{enumerate}
\indy{Index}{normal family}%
\end{definition}

A normal family determines a new hypermap.  A dart in the new set $D'$
of darts is a maximal path $[x;n^{-1} x; n^{-2} x;\ldots;n^{-k}
x]$ of $n^{-1}$ steps appearing in some loop in $\cal L$. The map $f'$
takes the maximal path $[x;n^{-1}x;\ldots;y]$ to the maximal path (in
the same contour loop) starting at $f y$. The map ${n'}^{-1}$ takes
the maximal path $[\ldots;y]$ to the maximal sequence (in some other
contour loop) starting $[n^{-1}y;\ldots]$. Equivalently, $n'$ takes
the maximal path $[x;\ldots]$ to the maximal path ending $[\ldots;n
x]$. The map $e'$ is defined by $e'\ocirc n'\ocirc f' = I_{D'}$.
\indy{Index}{path!maximal} %
\indy{Notation}{D@$D$ (dart)}%

\begin{definition}[quotient] The hypermap $(D',e',n',f')$
constructed from the normal
family $\cal L$ of $H=(D,e,n,f)$ 
is called the \newterm{quotient} of $H$ by $\cal L$, and is denoted
$H/{\cal L}$.  
%The hypermap $H$ is said to be a \newterm{cover} of $H/{\cal L}$.  
\indy{Index}{quotient}%
\end{definition}
\indy{Notation}{L@$\cal L$}%
\indy{Notation}{H@$H/{\cal L}$}%
%\indy{Index}{cover}%
\indy{Index}{hypermap!quotient}%
\indy{Index}{normal family}%

Intuitively, the quotient hypermap is represented as a graph whose
cycles under $f'$ are precisely the contour loops in the normal family
(Figure~\ref{fig:quot}).


\begin{figure}[htb]
\centering
\szincludegraphics[width=70mm]{\pdfp/quot.eps}
\caption{The contour loops in a normal family become faces in the
quotient}
\label{fig:quot}
\end{figure}

\begin{lemma}[quotient bijection]\label{lemma:quotient-bijection}
Let ${\cal L}$ be a normal family of the hypermap $H$.  Then ${\cal L}$ is in
natural bijection with the set of faces
of the quotient $H/{\cal L}$.  If $x'=[x;\ldots;n^{-k}x]$ is a maximal path of $n^{-1}$ steps in
the contour loop
$L\in{\cal L}$, then the corresponding face $F(L)$ of $H/{\cal L}$ is the one containing the
quotient dart $x'$.
\end{lemma}

\begin{proof}  This is left as an exercise to the reader.
\end{proof}


\begin{example}[maximal normal family]\label{ex:Hall} 
Assume that $H=(D,e,n,f)$ is a hypermap. % with no fixed points under $e$. 
Assume that every face meets at least two nodes. Then the set of all faces
defines a normal family of contour loops: follow $f$ around each
face $[x;f x;\ldots]$.  If $e$ acts without fixed points, then each dart of the quotient is just a
unit path consisting of a single dart of $H$, and the quotient
is isomorphic to $H$ itself.
\end{example}

\begin{example}[minimal normal family]\label{ex:H2} 
  Assume that $H=(D,e,n,f)$ is a plain hypermap.  Let $F = \{x,f x,\ldots\}$ be a face
  that visits at least three nodes and that meets each node in at most
  one dart.  Let $\cal L$ be the family with two contour loops: $[x;f
  x;\ldots]$ and its complement $L^c = [n^{-1} x;\ldots]$.
%\begin{displaymath}
%  [n^{-1} x;n^{-2} x;\ldots;n x;f n x = y;n^{-1} y; n^{-2} y;\ldots; n y; f ny;\ldots]
%\end{displaymath}
The family $\cal L$ is normal. The quotient hypermap $H/{\cal L}$ has
two faces: $F$ and a back side $F'$ of the same cardinality $k$.
\indy{Notation}{F@$F$ (hypermap face)}%
\end{example}

\begin{example}[cyclic]\label{ex:H2k} 
There is a hypermap $H_{2k}$ with two face.  The set of darts is the
disjoint union of two copies of $Z_k$, a cyclic group of order $k$
with generator $1$.  Each cyclic group is a face.  Use the variable
$i$ to index the first cyclic group and $i'$ to index the second.
The face map is $i\mapsto i+1$ and $i'\mapsto (i-1)'$.  The node map
is the involution $i\leftrightarrow i'$.  The edge map is the
involution $i\leftrightarrow (i+1)'$.  The relation $e\ocirc n\ocirc
f = I_D$ is verified:
\begin{displaymath}
\begin{array}{llllllll}
enf(i) &= e n(i+1) &= e(i+1)' &= i\\
e n f (i') & e n (i-1)' & e (i-1) &= i'.\\
\end{array}
\end{displaymath}
If a hypermap is isomorphic to $H_{2k}$ for
some $k$, then it is \newterm{cyclic}.  In particular,
the hypermap constructed in the previous example is cyclic.
\indy{Index}{hypermap!cyclic}%
\indy{Notation}{Z@$Z_k$ (cyclic group)}%
\end{example}

\begin{lemma}[plain quotient]\guid{JMKRXLA}\rating{280}\tlabel{lemma:quotient-plain}
Let $H$ be a plain hypermap, and let $\cal L$ be a
normal family.  Then $H/{\cal L}$ is a plain hypermap.
\end{lemma}

\begin{proof}  Write $H=(D,e,n,f)$ and $H/{\cal L} = (D',e',n',f')$.  
Write $[\ldots; x]$ for the node in
the quotient ending in dart $x\in D$ and $[x;\ldots]$ for the node
in the quotient starting with dart $x\in D$.  Plainness gives $e^2 x
= x$, so that for any dart $[\ldots x]$ in the quotient:
\begin{displaymath}\begin{array}{lll}
{e'}^{-2} [\ldots; x] &= n' f' n' f' [\ldots; x] = n' f' n' [f x; \ldots] \\&=
n' f' [\ldots; n f x] = n' [f n f; \ldots] = [\ldots; n f n f x]\\ &=
[\ldots; e^{-2} x] = [\ldots; x].
\end{array}\end{displaymath}
Thus, $e'$ has order $2$ on the quotient.
\end{proof}

%\begin{lemma}\guid{ZOKKAOI}\tlabel{lemma:quotient-planar}
%Let $H$ be a plain planar hypermap, and let $\cal L$
%be a normal family.  Then $H/{\cal L}$ is a plain planar hypermap.
%\end{lemma}
%
%\begin{proof} Suppose $H/{\cal L}$ is not planar.
%Let $P$ be a M\"obius contour on $H/{\cal L}$.  It lifts uniquely to
%a contour on $H$ with the property that the darts visited on $H$ are
%precisely the darts that belong to a dart in the quotient.  This is
%compatible with the node map $n$.  So the contour path lifts to a
%M\"obius contour on $H$.  Thus, $H$ is not planar.
%\end{proof}


\subsection{flag}
\indy{Index}{flag}%

The remainder of the chapter presents an algorithm that generates all
simple, plain, planar hypermaps satisfying certain general conditions
(Definition~\ref{def:restricted}).  The algorithm proceeds by adding
edges and nodes to a hypermap by reverse double walkup
transformations.

The algorithm  marks certain faces as `true.'
Roughly, this  means that the the face cannot be modified
at any later stage of the algorithm.   When all of its faces
are true, the hypermap stands in final form.
The function that marks each face as true or false is a
\newterm{flag}.  For the algorithm to work properly, it is necessary
to impose some constraints, as captured in Definition~\ref{def:flag}.
\indy{Index}{hypermap!algorithm}%

\begin{definition}[restricted]\label{def:restricted}
A restricted hypermap $H = (D,e,n,f)$ is one with the following
properties.
\begin{enumerate}
\item $H$ is nonempty, connected, plain, planar, and simple.
\item The edge map $e$ has no fixed points.  % Needed in Lemma:[flag set quotient]
\item The node map $n$ has no fixed points.
\item The size of every face is at least $3$.
%%  (All hypoth. Needed?)
\end{enumerate}
\indy{Index}{hypermap!restricted}%
\indy{Notation}{H@$H$ (hypermap)}%
\end{definition}

\begin{remark}
The assumption that $e x \ne x$ implies that $f x \ne n^{-1} x$ so that $f$-steps of a 
path can be distinguished from $n^{-1}$-steps.
\end{remark}

\begin{definition}[flag]\label{def:flag} 
Let $S$ be a set of darts in a hypermap $H$.  An
$S$-\newterm{flag} on $H$ is a boolean-valued function on the set of faces that satisfies
the following two constraints. 
\begin{enumerate}
\item If darts $x,y$ belong to true faces,
then there is a contour path from $x$ to $y$ that remains
in true faces.
\item Each edge of the hypermap meets a true face or $S$.
\end{enumerate}
An $\emptyset$-flag is simply called a flag.
%An isomorphism of flagged hypermaps is an isomorphism of
%hypermaps that respects the flags.
\indy{Index}{flag}%
\indy{Notation}{S@$S$ (set ofdarts)}%
\end{definition}



\begin{example}[cyclic hypermap flag] 
The cyclic hypermap of Example~\ref{ex:H2k}, carries a
flag that marks one face true and the other false.
\end{example}

\begin{example}[maximal quotient flag]\label{ex:Hall-flag} 
Let $H$ be a connected hypermap, and let $\cal L$ be the example of
Example~\ref{ex:Hall}, then the canonical map takes value
$\op{true}$ on every face.  This is a flag. In fact,
Lemma~\ref{lemma:connect-contour} provides the contour paths that
are required in the definition of flag.
\end{example}

\begin{definition}[canonical boolean function] Let $H/{\cal L}$ be a
quotient hypermap with normal family $\cal L$.  The
\newterm{canonical boolean function} on the set of faces of the
quotient is the function that is true exactly when every quotient dart in the
face is a unit path $[x]$ in $H$.  
\indy{Index}{function!canonical boolean}%
\end{definition}

In other words, the face in the quotient is canonically true, exactly when the corresponding
contour loop $L\in {\cal L}$ has no $n^{-1}$ steps.  The dart set of such
a contour loop is a face of $H$.  Based on this observation, we make the following definition.

\begin{definition}[face specific]
A contour loop $L$ in a hypermap is \newterm{face specific} if its dart set is a face of $H$.
\end{definition}


\begin{lemma}[quotient isomorphism criterion]\guid{STKBEPH}\rating{100}\tlabel{lemma:all-dart}  
  Let $H$ be a hypermap in which $e$ acts without fixed points, and
  let ${\cal L}$ be a normal family of $H$. If the canonical boolean
  function on the set of faces of $H/{\cal L}$ has at least as many
  true values as there are faces of $H$, then $\cal L$ is the normal
  family in Example~\ref{ex:Hall}. In particular, $H/{\cal L}$ is
  isomorphic to $H$.
\end{lemma}

\begin{proof} If a face of  $H/{\cal L}$ is $\op{true}$,  then
its darts are unit paths, and the face of $H/{\cal L}$ is in natural
bijection with a face in $H$.  This is an injective map from the
set of true faces of $H/{\cal L}$ to the set of faces of $H$.  The
hypothesis of the lemma implies that this injective map is
bijective. All of the darts of $H$ are accounted for under this
bijection. Thus, the quotient has no false faces.  The result
follows.
\end{proof}

There is a standard way of constructing the sets $S$ of darts that
will be used in $S$-flags.  We call a set $S$ arising in this
standard way a \newterm{flag set}.

\begin{definition}[flag set]
Let $H$ be a hypermap, $L$ a contour loop of the hypermap,
and $x$ an element of the dart set of $L$.
If  $L$ is  face specific, then let $S=\emptyset$.
Otherwise,
let $m\ge0$ to be the largest $m$ 
such that 
\begin{displaymath}
[x;f x; f^2 x;\ldots;f^{m+1} x]
\end{displaymath}  
is a subpath of $L$, and
%Set $y = f^{m+1} x$
set $S = S(H,L,x) = \{f^i x \mid 1 \le i\le m\}$.
The set $S$ is called the \newterm{flag set} of $(H,L,x)$.
\end{definition}

\begin{lemma}[flag set quotient]\label{lemma:flag-set-quotient}
Let $H$ be a hypermap in which $e$ acts without fixed points, 
$L$ a contour loop, and $x$ and element of the dart set of $L$.
Let ${\cal L}$ be a normal family of $H$ that contains $L$.
Then $S(H,L,x)$ maps bijectively to a set $S'$ of darts in the quotient $H/{\cal L}$.
\end{lemma}

\begin{proof} The darts of the quotient are maximal subpaths $[y;n^{-1} y;\ldots;n^{-k} y]$
of contour loops $L'\in {\cal L}$ made entirely of $n^{-1}$ steps.  
Each $y\in S(H,L,x)$ is preceded by an $f$-step and is
followed by an $f$-step in $L$.  Hence the maximal subpath of $L$ containing $y$ 
is a unit path $[y]$.  The
bijection follows.
\end{proof}


\subsection{markup}\label{sec:face-insert}
\indy{Index}{extension}%


%This section describes the transition from $H/{\cal L}$ to $H/{\cal
%M}$.  The algorithm carries along various auxiliary data satisfying
%various assumptions, enumerated as follows:


\begin{definition}[marked hypermap]\label{def:marked}
Let $(H,{\cal L},L,x,\varphi)$ be a tuple, consisting of 
\begin{itemize}
\item a hypermap $H=(D,e,n,f)$ with no M\"obius contours and in which $e$ acts without 
fixed points.  % Mobius needed for HQY...
\item a
normal family ${\cal L}$, 
\item a contour loop $L\in{\cal L}$, 
\item a dart $x$ visited by $L$,
and 
\item a  boolean-valued function $\varphi$ on ${\cal L}$.
\end{itemize}
Such a tuple is a \newterm{marked hypermap} if
the following conditions hold.
\begin{enumerate}
%\item $H$ is a hypermap.
%\item $\cal L$ is a normal family of $H$.
\item The quotient $H'=(D',e',n',f')=H/{\cal L}$ is simple.  
%\item $L$ is a contour loop in ${\cal L}$.
%\item $x$ belongs to the dart set of $L$.
%\item $\varphi$ is a boolean function on the faces of $H'$.
\item $\varphi$ is false on the loop $L$.
\item There is a contour loop $L'\in {\cal L}$ that visits $e x$ and that is true (with respect to
$\varphi$).
%\item The flag set $S$ of $(H,L,x)$ maps bijectively to a set $S'$
%of darts in the quotient.
\item The image $x'$ of $x$ in $D'$ has the form $[\ldots;x]$.
\item 
  $\varphi$ is an $S'$-flag on $H'$, where $S'\subset D'$ is the set
  associated with $S(H,L,x)$ in Lemma~\ref{lemma:flag-set-quotient},
  under the identification of $\varphi$ with a boolean-valued
  function on $H'$ (see Lemma~\ref{lemma:quotient-bijection}).
\end{enumerate}
\end{definition}




\begin{example}[illustration]\label{ex:graph-gen}  
  This example illustrates the setup (Figure~\ref{fig:graph-gen}).  In
  the figure, the hypermap $H$ is represented as a planar graph.  The
  contour loops are represented by left-side shadings of the edges of
  the planar graph.  The shaded edges give the edges of a planar graph
  representing the quotient.  The function $\varphi$ is the canonical
  boolean function.  The polygons that are fully shaded are true.  Two
  polygons in the quotient are false.  A dart of $H$ that maps to a
  false face in $H'$ is marked $x$.  By inspection, $m=3$ and $S=\{f
  x,f^2 x,f^3 x\}$.  By inspection, $\varphi$ is an $S'$-flag, where
  $S'$ is the image of $S$ in $D'$.  (In fact, the darts in true faces
  form a connected region.  Every edge except one edge $\{f' x', e' f'
  x'\}$ meets a true face, and this one edge meets $S'$.)
%% XX recheck
\end{example}

\begin{figure}[htb]
\centering
\szincludegraphics[width=90mm]{\pdfp/graph_gen.eps}
\caption{An example of the current situation.}
\label{fig:graph-gen}
\end{figure}


\begin{lemma}\guid{HQYMRTX}\rating{200} \label{lemma:yz}
Let $(H,{\cal L},L,x,\varphi)$ be a marked hypermap.
Assume that  $L$ is not face specific.
Set  $y = f^{m+1} x$, where $m = \card(S(H,L,x))$.
  Set
$z=f^{p+1} y$, where $p$ be the smallest natural number 
%Then there is a smallest $p\ge0$,
such that $f^{p+1} y$ belongs to some loop of ${\cal L}$.
Then, $z$ is on the contour loop $L$ and is not
equal to $f^k x$ when $0 < k \le {m+1}$.
\end{lemma}

\begin{proof} 
  For a contradiction, suppose $f^{p+1} y = f^k x$. Then also, $f^p y
  = f^{k-1} x$.  If $p>0$, then this contradicts the minimality of
  $p$.  So $p=0$, and $y=f^{k-1} x = f^{m+1} x$.  Also, $0\le k-1 <
  {m+1}$, which forces  $L$ to be face specific, which is
  impossible by assumption.  This proves the second conclusion of the
  lemma.  In particular, $z\not\in S$, where $S = S(H,L,x)$.

For a contradiction, assume that $z = f^{p+1} y$ lies on a
contour loop $L'\in {\cal L}$, where $L'\ne L$.  
If $u$ is any dart of $D$, write $\varphi(u)$ for the value
of $\varphi$ on the face of $D'$ containing the image of $u$ in $D'$.
If $\varphi(z)$ is true, then $\varphi(y)$
and $\varphi(x)$ are also true, contrary to the properties of $x$
in a marked hypermap.
So $\varphi(z)$ is false.  Let $z' = [\ldots;z_1]\in D'$ be the
image of $z$ in $D'$.  
Also, $z'\not\in S'$, where  $S'$ is the image of $S$ in $D'$.  By the definition of
$S'$-flag, the dart $e'z'$ lies in a true face or $e'z'\in S'$.  This disjunction splits
splits the proof into two cases.
\begin{enumerate}
\item\claim{[$e'z'$ lies in a true face.]}  In this case, consider the following
path in $H$:
\begin{displaymath}
[y;fy;\ldots;z] @ [z;\ldots;z_1] @ [n^{-1} z_1;\ldots;f e x].
\end{displaymath}
The first segment consists of $f$-steps; the second of $n^{-1}$-steps;
and the third segment exists within true faces by the connectedness of
true faces (for flags).  This path satisfies all the enumerated
conditions of Lemma~\ref{lemma:contour-f}.  The lemma asserts that the
path does not exist.
\item 
\claim {[$e'z'\in S$.]}  In this case,  consider the following path in $H$:
\begin{displaymath}
[y;f y;\ldots;z].
\end{displaymath}
This path satisfies all the enumerated conditions of
Lemma~\ref{lemma:contour-f}.  The lemma asserts that the path does not
exist.
\end{enumerate}
\end{proof}


\begin{definition}[transform]
From one marked hypermap $(H,{\cal L},L,x,\varphi)$ in
which $L$ meets at least two faces of $H$, we construct a new tuple 
$T(H,{\cal L},L,x,\varphi)$.  We write it in the form
\begin{displaymath}
T(H,{\cal L},L,x,\varphi) = (H,{\cal M},L_1,x,\varphi').
\end{displaymath}
In particular, the hypermap $H$ and the dart $x$ are
the same for both tuples.   The data ${\cal M}$, $L_1$, and $\varphi'$ are specified
in the following paragraphs.
The new tuple is called the \newterm{transform} of $(H,{\cal L},L,x,\varphi)$.
\end{definition}

Let $m$, $p$, $y$, and $z$ be as given in
Lemma~\ref{lemma:yz}.
Set
\begin{displaymath}
L_1 = L[x:y] \opat P[y:z] \opat L[z:x],
\end{displaymath}
the contour loop in $H$ that follows $L$ from $x$ to $y$, then takes
$f$-steps from $y$ to $z$, then continues along $L$ back to $x$.  
Set 
\begin{displaymath}
L_2 = L[n^{-1}y:n z] \opat P^c[n z:n^{-1}y],
\end{displaymath}
the contour loop in $H$ that follows $L$ from $n^{-1} y$ to $n z$,
then complements the path of $L_1$ from $y$ to $z$, traveling instead
from $n z$ to $n^{-1} y$. 


Set
\begin{displaymath}{\cal M} = ({\cal L}\setminus \{L\}) \cup
\{L_1,L_2\}.\end{displaymath}

%  Under the bijection between the set of faces of a quotient $H/{\cal
%    L}$ and the family ${\cal L}$ (of
%  Lemma~\ref{lemma:quotient-bijection}), an $S$-flag on a quotient
%  $H/{\cal L}$ can be identified with a boolean function on ${\cal L}$
%  satisfying appropriate properties.  With this in mind, 
We define a boolean function $\varphi'$
on $\cal M$.
Define $\varphi'$ to be false on
the face $L_1$, the canonical function on $L_2$ (that is, true if $L_2$ is face specific), and equal to
$\varphi$ on other loops. (Once we verify that ${\cal M}$ is a
normal family,  $\varphi'$ will be considered as a function on
the faces of the quotient $H/{\cal M}$.)  
\indy{Notation}{ZZphi@$\varphi$ (boolean function)}%
\indy{Index}{function!boolean}%




\begin{figure}[htb]
\centering
\szincludegraphics[width=80mm]{\pdfp/L1L2.eps}
\caption{The loop $L$ is replaced with two loops $L_1, L_2$.}
\label{fig:L1L2}
\end{figure}



Each of the two
loops $L_i$ meets at least two nodes (those
of $y$ and $z$) and has length at least three by
Lemma~\ref{lemma:3dart}.  
The flag set $S(H,L_1,x)$ contains the proper subset $S(H,L,x)$.




\begin{lemma}[transform]\guid{AQIUNPP}\rating{600}\tlabel{lemma:flag} 
%${\cal M}$ is a normal family, the noncanonical function $\varphi$
%is an $S_{\cal M}$-flag, and 
Let $H$ be a restricted hypermap.
If $(H,{\cal L},L,x,\varphi)$ is a marked hypermap such that $L$
meets at least two faces of $H$,  then the transform
$(H,{\cal M},L_1,x,\varphi')$ (as constructed above)
is also a marked hypermap.
\end{lemma}

\begin{proof} The proof can be organized into parts, according
to the separate properties of a marked hypermap.
%establishing that ${\cal M}$ is normal, that $\varphi'$ is an $S(H,L_1,x)$-flag,
%and that the quotient is simple.  \indy{Index}{normal}%
The first part of the proof establishes that ${\cal M}$ is a normal family.

\case{normal-1} \claim{No dart is visited by two different loops.}
Indeed by construction, the sets of darts of $L_1$ and $L_2$
are disjoint from each other and disjont from the sets of darts of $L'\in
{\cal L}\setminus \{L\}$.  The result follows.

\case{normal-2} \claim{Every loop visits at least two nodes.}  Indeed, this
is true for $L_1$ and $L_2$ because they visit the nodes of $y$ and
$z$.  It is true of the other loops because they belong to the
normal family ${\cal L}$. It follows that ${\cal M}$ is normal.

\case{normal-3} \claim{If a loop visits a node, then every dart at
that node is visited by some loop.}  Indeed, the new nodes visited on the
subpath of $L_1$ from $y$ to $z$ do not contain any darts in any
loop in ${\cal L}$.  (By the definition of normal family, if a loop
visits a node, then every dart at that node is visited by some loop
of ${\cal L}$.)  These new nodes contain two darts, with one dart
along $L_1$ and the other along $L_2$.  The paths $L_1$ and $L_2$
visit every dart visited by $L$.  They visit every dart at the nodes
of $y,\ldots,z$.

\case{flag-1} \claim{The true faces are connected.}  Indeed, if $F(L_2)$ is
false, then the set of true faces is the same on $H/{\cal L}$ and
$H/{\cal M}$.  In this case the proof is immediate.  If $F(L_2)$ is
true, then the proof requires more argument.  Let $y'$ be the image
of $y$ in $H'$.  Since the canonical function on $H'$ is a flag, the
edge $\{y',e'y'\}$ meets a true face or $S'$.  However, $y'\not\in
S'\subset F(L)$, by the simplicity of $H$.  $e'y'\not\in S' \subset
F(L)$, by the simplicity of $H'$.  The only remaining possibility is
that $e'y'$ lies in a true face.  The dart $e'y'$ is naturally
identified with the dart $e_{\cal M} y'$ on in $H/{\cal M}$, and
hence a path exists from the true face $F(L_2)$ into another true
face, and from there any true face may be reached.

\case{flag-2} \claim{Each edge of $H'$ meets $S'$ or a true face.}  Indeed, if the
function $\varphi'$ is an $S(H,L_1,x)$-flag.  The edges of $H/{\cal
L}$ can be identified with a subset of the edges of $H/{\cal M}$.
For this subset, the flag condition on edges is immediate.  Any
other edge is one of the newly created edges, and they all share a
dart with $S(x,{\cal M})$.
% XX REWRITE .

\case{simple} To prove the simplicity of the quotient, it is enough to show that
none of the contour
paths in ${\cal M}$  ever return to a node after leaving it.
This is true of ${\cal L}\setminus\{L\}$ by assumption and true of
$L_1$ and $L_2$ by construction.  Note in particular, that the dart
$z$ is not at the same node as $y$ (by the simplicity of $H$).

The other verifications are routine.
\end{proof}

\subsection{sequence}

The aim is to prove that every restricted hypermap with a given bound
on the size of the dart set is generated by a particular algorithm.
The proof is a long induction argument.  The proof starts by showing
how to go from one partially constructed hypermap to another more
fully constructed hypermap.  The hypermap $H$ represents the fully
constructed hypermap and two quotients $H/{\cal L}$ and $H/{\cal M}$
represent the partially constructed hypermaps.  The algorithm involves
the transition from the hypermap $H/{\cal L}$ to $H/{\cal M}$.  The
transition from one quotient to another is given by the transform of
marked hypermaps.  The transform thus represents one step of the
algorithm.  \indy{Notation}{M@$\cal M$ (normal family)}%

\begin{lemma}  Let $H$ be a restricted hypermap.  
There is a sequence of marked hypermaps
$A_i = (H,{\cal L}_i,L_i,x_i,\phi_i)$ for $i=0,\ldots,k-1$ such that
\begin{itemize}
\item $\phi_i$ is the canonical boolean function on $H/{\cal L}_i$.
\item $H/{\cal L}_0$ is the cyclic hypermap of example~\ref{ex:H2}, with
chosen face $F$ given by the dart set of $L_i$.
\item $A_{i+1}$ equals the transform $(H,.,L,\ldots)$ of
$A_i$, whenever the contour loop $L$ is not face specific.
\item $A_{i+1}$ has the form $(H,{\cal M},M,y,\phi_{i+1})$, where $(H,{\cal M},L_1,x,\phi)$ is
the the transform of $A_i$, whenever the $L_1$ is face specific, and there exists some other contour loop of ${\cal M}$ that is not face specific.
\item The transform $(H,{\cal L}_k,\ldots)$ of $A_{k-1}$ has a quotient $H/{\cal L}_k$,
which is naturally isomorphic to $H$.  
  $H$.
\end{itemize}
\end{lemma}

\begin{proof}
Let $A_0=(H,{\cal L}_0,L_0,x_0,\phi_0)$ be given as follows.  Pick any face $F$ of $H$, and
construct the normal family {\cal L} of example~\ref{ex:H2}. 
XXD REWRITE this proof.

The construction of $A_{i+1}$ depends on the structure of the transform $(H,{\cal M},L,x,\varphi')$ of $A_i$.  We consider three
cases.
\begin{nomerate}
\item 
\claim{[Every contour loop of ${\cal M}$ is face specific.]}   In this case,  by
Lemma~\ref{lemma:all-dart}, the quotient map $H\to H/{\cal M}$ is an
isomorphism.  Set $k= i+1$. The sequence terminates.
\item 
\claim{[The contour loop $L$ is not face specific.]}  In this case,  from the inductive hypothesis that $\varphi_i$ is
the canonical function and the definition of $\varphi'$, it follows that $\varphi'$ is the canonical
function.  By Lemma~\ref{lemma:flag}, the transform is
a marked hypermap.  Let  $A_{i+1}$ equal the transform of $A_i$.
\item
\claim{[The contour loop $L$ is face specific, but there exists another $M\in {\cal M}$ that is not face specific.]}
In this case, the canonical
function on $H/{\cal M}$ is a flag (with $S=\emptyset$).  Pick any quotient dart $x'=[\ldots;y]$ in $H/{\cal M}$, that
lies in the face $F(M)$ corresponding to $M$.  By the definition of flag,
The dart $e' x'$ lies in a true face.  Let
$A_{i+1} = (H,{\cal M},M,y,\varphi_{i+1})$, where $\varphi_{i+1}$ is the canonical function.
The canonical function is a flag, hence an $S(H,M,x)$-flag.  $A_{i+1}$ is a marked hypermap.
\end{nomerate}

The sequence must terminate eventually, because each step
constructs a quotient of $H$ with more darts than the previous one and
the number of steps is bounded by the size of the dart set of
$H$.
\end{proof}

Next, we wish to describe the steps of the sequence in a way that relies on a lesser degree
on the structural details of the marked hypermaps.  (These details will not be available to us
in the algorithm of the next subsection.)  The next lemma gives a way of constructing a new
hypermap from a given hypermap $H'$ that does not require us to  represent it first as a
quotient $H' = H/{\cal L}$ for some normal family.

\begin{definition}[R]\label{def:R}  
Let $H'$ be a hypermap and let $x$ be dart of $H'$.  Let  $r$ be the
cardinality of the face of $x$,
and let $m,p,q$ be integers that
satisfy $m\ge 0$, $p\ge 0$, and $0 \le m < q < r$,
and the  constraint  $m+1\ne q$ or $p\ge1$.  
Construct a hypermap $R(H',x,m,q,p)$ from $H'$ as follows.
First
construct an edge from the node of $y = f^{m+1} x$ to the node of $z =
f^{q+1} x$ (by a reverse double walkup).  Then insert $p$ new nodes (of
degree $2$) along the constructed edge (again by reverse double walkup transformations).  
This is $R(H',x,m,q,p)$.  
\end{definition}
\indy{Notation}{R@$R$ (hypermap constructor)}

We can immediately relate this construction to the sequence of marked hypermaps that
we have constructed.

\begin{lemma} Let $H$ be a restricted hypermap.  Let $(H,{\cal L},L,x,\varphi)$ be
a marking of $H$, and let $(H,{\cal M},M,x,\varphi')$ be its transform.
Then $H/{\cal M}$ is isomorphic to $R(H/{\cal L},x',m,q,p)$, where $x'$ is the image
of $x$ in $H/{\cal L}$, and $m,q,p$ are integers satisfying the constraints of Definition~\ref{def:R}.
Specifically, $m$ and $p$ are the constants of Lemma~\ref{lemma:yz}.  XX FINISH.
\end{lemma}

\begin{proof}
The passage from $H/{\cal M}$ to $H/{\cal L}$ consists of a double
walkup on all of the nodes coming from the nodes of $f y$, $f^2 y,
\ldots, f^p y$, and then a double walkup on the edge from the node of
$y$ to the node of
$z$.  %The passage in the other direction from $H/{\cal L}$ to
%$H/{\cal M}$ comes by adding an edge from the node of $y$ to
%the node of $z$ and inserting $p$ new nodes (of degree two)
%along it.
If we play these double walkups in reverse,
one may also pass from $H/{\cal L}$ to $H/{\cal M}$.  
If
$m,q,p$ are chosen as above, then $R(H',x,m,q,p)$ is isomorphic to
$H/{\cal M}$.  
\end{proof}


\begin{figure}[htb]
\centering
\szincludegraphics[width=80mm]{\pdfp/L1L2dart.eps}
\caption{$H/{\cal L}$ is obtained from $H/{\cal M}$ by double walkup
transformations.}
\label{fig:L1L2dart}
\end{figure}



\subsection{algorithm}

This final section puts the algorithm a precise form, based on
the Knaster-Tarski fixed point theorem.  (The Knaster-Tarski fixed
point theorem is a standard way to give precise mathematical form to
an algorithm.)

\begin{lemma}[Knaster-Tarski]\guid{EAOGWLE}\rating{ZZ}   
Let $X$ be a set.  Let $f:\powerset(X)\to \powerset(X)$ be a
function from the powerset of $X$ to itself.  Assume that $f$ is
monotonic in the sense that whenever $Y\subset Z\subset X$, it
follows that $f(Y) \subset f(Z)$.  Then $f$ has a least fixed point.
That is there exists a set $\op{fix}(f,X)\subset X$ such that
$f(\op{fix}(f,X)) = \op{fix}(f,X)$ and such that the following
minimality condition holds: if $Y\subset X$ is any set such that
$f(Y) \subset Y$, then $Y\subset \op{fix}(f,X)$.
\end{lemma}
\indy{Notation}{X@$X$ (set)}%
\indy{Notation}{f@$f$ (function on powerset)}%
\indy{Notation}{fix@$\op{fix}$~(Knaster-Tarski fixed point)}%
\indy{Notation}{P@$\powerset(\cdot)$ (powerset)}%
\indy{Index}{Knaster-Tarski fixed point theorem}%

\begin{proof} Let $\op{fix}(f,X)$ be the intersection of all subsets
$Y$ of $X$ such that $f(Y)\subset Y$.  It is easily verified that
this set has the required properties.
\end{proof}


Let $D$ be a fixed finite set that will contain all of the darts from
all of the hypermaps to be constructed.  (The set of darts for some of
the hypermaps may be a proper subset of $D$.)  Fix a choice function
$\op{ch}:\powerset(D)\to D$ that picks an element from each nonempty
subset:
\begin{displaymath}
X\ne\emptyset\quad  \Rightarrow \quad  \op{ch}(X)\in X.
\end{displaymath}
\index{Notation}{D@$D$ (set of darts)}
\index{Notation}{ch@$\op{ch}$~(choice)} For example, when
$D\subset\ring{N}$, let $\op{ch}$ choose the least element of a subset
of $D$.

Fix $d\ge 3$, representing an upper bound on the size of the faces of
the hypermaps that will be constructed.  (In applications to the
sphere packing problem, $d=6$.)  The construction depends on $D$ and
$d$, although the notation does not reflect this.
\indy{Notation}{d3@$d$ (upper bound)}%

Define a set $X$ as the disjoint union of $X_1$ and $X_2$ as follows.
Let $X_1$ be the set of all hypermaps whose darts are elements of $D$.
Let $X_2$ be the set of four-tuples $(H,m,\varphi,x)$, where $H$ is a
hypermap whose darts are elements of $D$, $m\in\{0,\ldots,d-1\}$,
$\varphi$ is an $S$-flag on $H$, $x$ is a dart in a false face of $H$,
and $S = \{f^i x\mid 1 \le i \le m\}$.  (One may think of $X_1$ as
holding the output of the algorithm and $X_2$ as holding the partially
constructed hypermaps.)

Let $H$ be a fixed hypermap isomorphic to $H_{2d}$, with darts in $D$.
Let $\varphi$ be the canonical flag on $H$ (with one true face and one
false face).  Let $x$ be the value of the choice function on the false
face.  Set
\begin{displaymath}A = \{(H,m,\varphi,x) \mid 0\le m \le d-1
\}.\end{displaymath}

When a function 
$g:X_2 \to \powerset(X)$ is given, set 
\begin{displaymath}f(S) = A \cup (\bigcup \{g(s) \mid s\in S\cap
X_2\}).\end{displaymath} Any function $f :\powerset(X)\to
\powerset(X)$ defined in this way is monotonic.  \indy{Notation}{g@$g$
(function)}%

The function $g$ is defined as follows.  Let $(H,m,\varphi,x)\in X_2$.
Let $r$ be the cardinality of the face of $x$.  Consider $q,p$
satisfying the following constraints:
\begin{itemize}
\item $0\le m < q < r$.
\item $0\le p$.
\item if $m+1 = q$, then $p \ge 1$.
\item $m+p+2 \le d$.
\item if $q+1\ne r$, then $m+p+3\le d$.
\end{itemize}

For each such $q,p$, construct the hypermap $R(H,x,m,q,p)$.  Recall
that this hypermap is constructed from $H$ by adding two faces $F_1$
and $F_2$ and eliminating the face $F$ of $x$.  All other faces can be
naturally identified.  Say that a boolean function $\varphi'$ on
$R(H,x,m,q,p)$ is an \newterm{extension} of the boolean function
$\varphi$ on $H$ if $\varphi'(F') =\varphi(F')$, when $F'\ne F$.
\indy{Index}{function!extension}%

Let $g(H,m,\varphi,x)$ be the following subset of $X$:
\begin{itemize}
\item $H'\in X_1\cap g(H,m,\varphi,x)$ if and only if there exists
$q,p$ satisfying the given constraints such that $H'=R(H,x,m,q,p)$
and $\varphi(F')$ is true for all $F'\ne F$.
\item $(H',m',\varphi',x')\in X_2\cap g(H,m,\varphi,x)$ if and only if
there exists $q,p$ satisfying the given constraints such that
$H'=R(H,x,m,q,p)$ and such that $\varphi'$ is an extension of
$\varphi$, and one of the following two conditions hold:
\begin{itemize}
\item $\varphi'(F_1)$ is false;  $x' = x$; and  $p+m+1 \le m' \le r-1$.
\item $\varphi'(F_1)$ is true; $x'$ is the value of the choice
function on the union of false faces of $H'$; and $0 \le m' \le
r-1$.
\end{itemize}
\end{itemize}





\begin{lemma}\guid{BRGEFNH}\rating{2000}  
Define $f $ and $X$ as above (depending on $d\ge 3$ and $D\ne
\emptyset$) .  Let $F=\op{fix}(f,X)$ be the Knaster-Tarski fixed
point set of $f$ on $X$.  Then every restricted hypermap with at
most $\card(D)$ darts and such that every face has size at most $d$
is isomorphic to a hypermap in $F\cap X_1$.
\end{lemma}
\indy{Notation}{F@$F$ (fixed point set)}%
\indy{Notation}{card (cardinality)}

In other words, by starting with the \newterm{seed} hypermaps in $A$
one may find all restricted hypermaps (for given $D$ and $d$) by
applying the function $f$ repeatedly:
\begin{displaymath}
A_0 = A = f(\emptyset),\quad A_1 = f(A_0),\quad A_2 = f(A_1),\ldots
\end{displaymath}
and by looking at the output $A_i \cap X_1$.
\indy{Index}{seed}%

\begin{proof} Let $H$ be a restricted hypermap whose dart set belongs
to $D$ and whose greatest face size is $d$.  It has a quotient
$H/{\cal L}_0$ isomorphic to $H_{2d}$.  By repeating the
construction of Section~\ref{sec:face-insert}, one obtains a
sequence of hypermaps $H_i = H/{\cal L}_i$, $i=0,\ldots,N$,
terminating with a hypermap $H/{\cal L}_N$, which is isomorphic to
$H$.  The data $m_i,\varphi_i,x_i$ is also obtained for each $H_i$.

The tuple $(H_0,m_0,\varphi_0,x_0)$ is isomorphic to an element of
$A$.  By construction, $A\subset F$.  If the lemma is false, there
is a smallest $i>0$ for which $(H_i,m_i,\varphi_i,x_i)\not\in F$ (or
if $i=N$, for which $H\not\in F$).  There are isomorphisms
$R(H_i,x_i,m_i,q_i,p_i) \simeq H_{i+1}$ for appropriate choices of
$q_i,p_i$.  By construction, when the data belongs to $F$ for $i-1$,
the data belongs to $F$ for $i$.  Thus, $H$ belongs to $F\cap X_1$.
\end{proof}

%\subsection{old algorithm}
%
%If a hypermap is restricted and $x$ is any dart, then $x$ and $n x$
% lie on different faces.  In particular, a restricted hypermap has at
% least two faces.  To begin the process, take ${\cal L}$ to be the
% normal family of Example~\ref{ex:H2} with two contour loops, whose
% quotient hypermap is a polygon $H_{2d}$.  When the the initial
% contour loop is chosen on a face of maximal size, the natural number
% $d$ is an upper bound on size of a face.
%
% In summary, a process starts with a single polygon and then adds
% edges and nodes of degree two along the inserted edges, to obtain a
% restricted hypermap $H$.
%
% A modification of the process avoids explicit reference to the
% hypermap $H$ and to the normal family ${\cal L}$.  Let $D$ be a
% finite set that contains all the darts for all of the restricted
% hypermaps to be constructed.  For each $d=3,\ldots,\# D$, the
% process generates all restricted hypermaps with greatest face-size
% $d$, with darts in $D$.
%
% The algorithms performs the following initialization.  The initial
% hypermap is the polygonal hypermap $H_{2d}$.  A flag $\varphi$ marks
% one face true and the other false.  A distinguished dart $x$ is
% selected on the false face.  For each $m<d$, set $S=S_m = \{f^i
% x\mid 1\le i\le m\}$.
%
% Each iteration processes a finite list ${\cal H}$ of quadruples
% $(H,m,\varphi,x)$, where $H$ is a simple hypermap, $\varphi$ is an
% $S$-flag, $x$ lies in a false face, and $S = \{f^i x\mid 1\le i\le
% m\}$ for some $m$.  The algorithm terminates when every face of
% every hypermap in ${\cal H}$ is true.  At every step of the
% algorithm, one quadruple with some false face is removed from ${\cal
%   H}$ and finitely many quadruples are returned to ${\cal H}$.
%
% At each iteration the chosen $(H,m,\varphi,x)$ is modified in the
% following ways and each modification is placed back in the list
% ${\cal H}$.  As the natural number $m$ depends on the unknown normal
% family ${\cal L}$, all possible $m < d$ are used.  Similarly, the
% natural number $p < d$ depends on ${\cal L}$, and all possible $p$
% are used.  Any quadruple that is isomorphic to one previously
% considered is discarded as redundant.
%
% In summary, the algorithm constructs of hypermaps.  The process must
% terminate, because the set $D$ is finite, so there are only finitely
% many quadruples (or quadruples up to isomorphism) that construct
% their dart sets from $D$.
%
%\begin{lemma}\guid{BRGEFNH}\rating{2000}  Fix $d\ge 3$ and $D\ne \emptyset$.
%  This process constructs in a finite number of steps all restricted
%  hypermaps, up to isomorphism, such that the dart set belongs to the
%  finite set $D$, and whose greatest face size is $d$.
%\end{lemma}




%%%%%%%%%%%%%%%%%

