%% HYPERMAPS



\section{Basics}



\begin{definition}[hypermap]  A hypermap is a finite set $D$, together with
three functions $e,n,f:D\to D$ that satisfy
    $$e\circ n\circ f = I.$$
The elements of $D$ are called {\it darts}.  The functions $e,n,f$
are called the {\it edge map}, the {\it node map}, and the {\it
face map}, respectively.
\end{definition}


\begin{figure}[htb]
  \centering
  \myincludegraphics{\ps/dart.eps}
  \caption{This symbol represents a dart.}
  \label{fig:dart}
\end{figure}

\begin{remark}\tlabel{rem:hypermap} A hypermap is an abstraction of
the concept of 
planar graph.  Place a dart at each angle of a planar graph $G$.
One function, $f$, 
cycles counterclockwise around the angles of each face.  
Another function, $n$, 
rotates counterclockwise around the angles at each
node.  A third function, $e$, pairs angles at opposite ends of
each edge  (Figure~\ref{fig:hypermap_ex}).   The hypermap extracts
the data $(D,e,n,f)$ from the planar graph and discards the rest.
\end{remark}

\begin{figure}[htb]
  \centering
  \myincludegraphics{\ps/hypermap_ex.eps}
  \caption{Darts mark the angles of a planar graph.  We may
  permute darts about faces, nodes, and edges.}
  \label{fig:hypermap_ex}
\end{figure}

A hypermap satisfies 
  \begin{equation}\tlabel{eqn:triality}
  n\circ f\circ e = f\circ e\circ n = I.
  \end{equation}
Inverted, this triality becomes
   $$
   n^{-1} \circ e^{-1} \circ f^{-1} = f \circ e \circ n = I.
   $$
This inversion
is the abstract form of the
the duality between nodes and faces in a planar graph.  
Because of
these symmetries in the defining relation, 
there will be multiple versions of 
theorems about hypermaps,
all obtained from one proof by symmetry.

Each function $e,n,f$ is a permutation of $D$.  
We write $\#h$ for the
number of orbits of a permutation $h$ on $D$, and $\#\tangle{e,n,f}$
for the number of orbits of the combined action of functions $e,n,f$
on $D$.   

\begin{definition}[node,~face,~edge,~group~orbit]  A node is an orbit  under
$n$.  A face is an  orbit  under $f$.  An edge
is an orbit under $e$. An orbit of $D$ under $\tangle{e,n,f}$
(the combined action of $e,n,f$) is called a {\it group orbit}.
A hypermap has a single group orbit
if 
  $\#\tangle{e,n,f} = 1$. 
\end{definition}

\begin{definition}[plain,~planar] A hypermap is {\it plain} (note the spelling!) if
$e$ is an involution on $D$ (that is, $e\circ e = I$).  A hypermap
is {\it planar} (note the spelling!) if the Euler relation holds:
    $$\# n + \# e + \# f = \# D + 2 \#\tangle{e,n,f}.$$
\end{definition}

\begin{definition}[degenerate] A dart is degenerate if it is a
fixed point of one of the maps $e,n,f$; otherwise it's nondegenerate.  
%%It is nondegenerate otherwise.
\index{Index}{degenerate}\index{Index}{nondegenerate}
\end{definition}

\begin{definition}[simple] 
A hypermap is {\it simple} if the intersection of a face with
a node never contains more than one dart.
\index{Index}{simple}
\end{definition}


% Moved from cup05_tame.tex section on tame plane graphs. 9/5/07:
\begin{lemma}\tlabel{lemma:nondegen} 
Let $(D,e,n,f)$ be a simple plane hypermap such that every face has
at least three darts.
Then $n$ has no fixed point.
\end{lemma}

\begin{proof}  For a contradiction, let $x$ be a fixed point of
$n$. We show that $e x$ and $f x$ lie in the same node and
face, so are equal in the simple hypermap.  
They lie in the same node because
$n(f x) = e^{-1} x = e x$. They lie in the same face because
    $$f^2 (e x) =  f(n^{-1} x) = f x.$$
So $e x = f x$.   Thus, $f^2 (e x) = f x = e x$, and $e x$ lies on a
face with at most two darts.  This contradicts what is given.
\end{proof}

\begin{remark}  The Euler relation for hypermaps harks back
to the Euler relation for planar graphs.
Let $G$ be a planar graph with a single group orbit that satisfies the
Euler relation
    $$V - E + F = 2$$
where $V$ is the number of vertices, $E$ the number of edges, and
$F$ the number of faces of $G$ (including an unbounded face). The
hypermap $(D,e,n,f)$, made from $G$ in
Remark~\ref{rem:hypermap}, is plain.
Moreover,
    $$\begin{array}{lll}
    V &= \# n\\
    E &= \# e\\
    F &= \# f\\
    2E &= \# D\\
    1 &= \#\tangle{e,n,f}\\
    \# n + \#e + \# f &= 
    V + E + F = 2 E + 2 = \# D + 2 \#\tangle{e,n,f}.
    \end{array}
    $$
Thus, the hypermap is also planar. 
\end{remark}


%% WW Used??
%\begin{lemma}\tlabel{lemma:euler-alt}  Let $H$ be a plain planar hypermap with a single group orbit.
%Let $f_i$ be the number of faces with $i$ darts.  Then
%    $$2 \# n - 2 =  f_3 + 2 f_4 + 3 f_5 +\cdots$$
%\end{lemma}
%
%\begin{proof}  We have
%    $$
%    \begin{array}{lll}
%     \# D &= 2 \# e = 3 f_3 + 4 f_4 + \cdots\\
%    \# f &= f_3 + f_4 + \cdots.
%    \end{array}
%    $$
%Use these equations to eliminate $\#e$, $\#f$, $\#D$ from the Euler
%relation.  The result follows.
%\end{proof}
%


\subsection{walkup}

When we focus on a dart $x$ in a
hypermap, it can be useful to draw a hexagon around $x$ and place
the six darts $e x$,
$n x$, $f x$, $e^{-1} x$, $n^{-1} x$, $f^{-1} x$ at its corners
in Figure~\ref{fig:dart+}.  Some of these $7$ darts may be
equal to one another, even if the figure draws them apart.
Figure~\ref{fig:dart-fix} shows the layout of the darts, when 
a map $e$, $n$, or $f$ fixes $x$.

\begin{figure}[htb]
  \centering
  \myincludegraphics{noimage.eps}
  \caption{A dart $x$ and its entourage}
  \label{fig:dart+}
\end{figure}

\begin{figure}[htb]
  \centering
  \myincludegraphics{noimage.eps}
  \caption{A dart fixed under a face map.}
  \label{fig:dart-fix}
\end{figure}


A {\it walkup} deletes
a dart from a hypermap and remolds the edge, node, and face
maps to produce a hypermap on the reduced set of darts.  Walkups
come in three flavors: edge walkups, face walkups,
and node walkups.

\begin{definition}[walkup]
Let $x$ be a dart in a hypermap.  The edge walkup 
$W_e$ at $x$ of the hypermap is the hypermap
$(D',e',n',f')$, where $D' = D\setminus\{x\}$ and the
the maps skip over $x$:
    $$
    \begin{array}{lll}
    f' y &= \text{ if } (y = f^{-1} x) \text{ then } f x \text{ else
    } f y\\
    n' y &= \text{ if } (y = n^{-1} x) \text{ then } n x \text{ else
    } n y\\
    e' = (n'\circ f')^{-1}
    \end{array}
    $$
\index{Index}{walkup}
\index{Index}{edge walkup}
\end{definition}

Figure~\ref{fig:walk} shows
the result of an edge walkup on the hexagon around a dart $x$.
The Triality symmetry~\ref{eqn:triality}, applied to the definition
of edge walkups, yields the definition of
face walkup $W_f$ and node walkup $W_n$.  
Figure~\ref{fig:walkfn} shows the result of the face and node
walkups on the hexagon around a dart $x$.

A walkup at $x$ is said to be degenerate 
if the dart $x$ is degenerate.   
At a degenerate dart $x$, all three walkups
are equal: $W=W_e=W_n=W_f$ (Figure~\ref{fig:walkdeg}).

\begin{figure}[htb]
  \centering
  \myincludegraphics{noimage.eps}
  \caption{The effect of an edge walkup at $x$}
  \label{fig:walk}
\end{figure}

\begin{figure}[htb]
  \centering
  \myincludegraphics{noimage.eps}
  \caption{The effect of face and node walkups at $x$}
  \label{fig:walkfn}
\end{figure}

\begin{figure}[htb]
  \centering
  \myincludegraphics{noimage.eps}
  \caption{The effect of a walkup at a degenerate dart}
  \label{fig:walkdeg}
\end{figure}


\begin{definition}[merge,~split]\tlabel{def:merge-split} Let $h=n,e$, or $f$.
A walkup $W_h$ at $x$ is said to merge,
if the walkup joins the orbit of $h$ through $x$ with another orbit.  
It is said to split, if the walkup splits the
orbit at $x$ into two orbits.
\index{Index}{split}
\index{Index}{merge}
\end{definition}

\begin{lemma}\tlabel{lemma:merge-split} 
Every nondegenerate walkup merges or splits.
The walkup $W_h$ at $x$ merges if and only if $x$ and $y$  lie
in distinct $h$-orbits, where $(h,y)=(f,e x)$.  
(The lemma also holds for $(h,y)=(e,n x)$ and other cases generated
by triality.)
\end{lemma}

\begin{proof} The walkup $W_f$ splits if and only if $f x$ 
(or $x$)
and $e x$ lie in the same $f$-orbit before the split. 
Figure~\ref{fig:split} makes this clear.
\end{proof}


\begin{figure}[htb]
  \centering
  \myincludegraphics{noimage.eps}
  \caption{The face walkup at $x$ mixes $f$-orbits.  If it mixes
  two separate 
  orbits, the orbits merge.  If it mixes a single orbit, 
  the orbit splits.}
  \label{fig:split}
\end{figure}


\subsection{walkup and planarity}

\begin{definition}[planar~index] Define the planar index of a hypermap to be
$$\# f + \# e + \# n - \# D - 2\# \tangle{e,n,f}.$$
(A hypermap with null index is planar.)
\index{Index}{index}
\index{Index}{planar index}
\end{definition}

\begin{lemma}\tlabel{lemma:index} Let $x$ be a dart of a hypermap $(D,e,n,f)$. Let $(D',e',n',f')$ be the result of the edge walkup $W_e$ at
a nondegenerate dart $x$.  
The walkup changes the sizes of some orbits.
    $$
    \begin{array}{lll}
    %\text{\bf Non-degenerate dart $x$: }&\\
    \# f &= \# f'\\  
    \# e + \delta' &= \# e'\\
    \# n &= \# n'\\
    \# D - 1&= \# D' \\
    \# \tangle{e,n,f} + \delta&= \#\tangle{e',n',f'},\\
    \end{array}
    $$
where
   $$
   \delta' = \begin{cases}
     1 & W_e \text{ splits }\\
    -1 & W_e \text{ merges}\\
   \end{cases}
   $$
and
   $$
   \delta = \begin{cases}
    0 & W_e \text{ merges }\\
    0 & W_e \text{ splits and } x,nx 
      \text{ belong to the same group orbit}\\
    1 & \text{otherwise}.
     \end{cases}
   $$
Moreover, a walkup at a degenerate dart preserves the planar index.
\end{lemma}

\begin{proof} The figures make this clear.
\end{proof}

\begin{lemma}\tlabel{lemma:planar-index2}
Let $\iota$ be the index of a  hypermap $(D,e,n,f)$, and
let $\iota'$ be the index after a walkup $W_h$
at a dart $x$.  We have $\iota = \iota' + 2$ when
$x$ is nondegenerate, $W_h$ splits, but the walkup does not
split the group orbit through $x$ into two orbits.
Otherwise, $\iota'=\iota$.
\end{lemma} 


%If $x$ is degenerate, then $\iota=\iota'$.
%If $W_h$ merges, then $\iota=\iota'$.
%If $x$ is non-degenerate and and the walkup
%splits, the planar index is preserved iff the walkup splits the 
%group orbit through $x$ into two orbits. When the walkup
%does not preserve the planar index, $\iota+2=\iota'$.


\begin{proof}  By triality symmetry, 
we can assume the walkup is an edge walkup.  
Note that
if the group orbit splits then
the $e$-orbit also splits.
From Lemma~\ref{lemma:index},
which lists index data, the
relation between $\iota$ and $\iota'$ is easily calculated.
\end{proof}


\begin{lemma}\tlabel{lemma:planar-nonpos}  
The planar index
of a hypermap is never positive.
\end{lemma}

\begin{proof}  An edge walkup never decreases the index.  By a sequence
of edge walkups we reach the empty hypermap, which has
index zero.
\end{proof}


\begin{lemma}\tlabel{lemma:walkup-planar}
Walkups take planar hypermaps to planar
hypermaps.
\end{lemma}

\begin{proof}  
A planar hypermap has maximum index.  The walkup
can only increase the index, but never beyond its maximum.  
Thus, the index remains at its maximum value.
\end{proof}


\subsection{double walkup}

A double walkup is the composite of two walkups
of the same type.  The
two darts for the two walkups 
are to be the members of an orbit of size
two (under $n$, $e$, or $f$).  The first walkup is to be
chosen so that it merges.  The second walkup is
to be degenerate.
By choosing the type of the walkups to be different from the type of
the orbit, the first walkup reduces the orbit to a singleton,
forcing the second walkup to be degenerate. 
 
Here are some examples.
\begin{itemize}
    \item A double $W_n$ along an edge deletes the edge and 
   merges the two endpoints into
    a single node (Figure~\ref{fig:doublenode}). 
    \item A double $W_f$ along an edge 
    deletes the edge and merges the two faces along the edge into
    one (Figure~\ref{fig:doubleface}).
    \item A double $W_e$ at a node of degree two
    deletes the node and merges the two edges at the node into
    one (Figure~\ref{fig:doubleedge}).
\end{itemize}


\begin{figure}[htb]
  \centering
  \myincludegraphics{noimage.eps}
  \caption{The double node walkup applied to an edge}
  \label{fig:doublenode}
\end{figure}


\begin{figure}[htb]
  \centering
  \myincludegraphics{noimage.eps}
  \caption{The double face walkup applied to an edge}
  \label{fig:doubleface}
\end{figure}


\begin{figure}[htb]
  \centering
  \myincludegraphics{noimage.eps}
  \caption{The double edge walkup applied to a node}
  \label{fig:doubleedge}
\end{figure}


\begin{lemma}\tlabel{lemma:dwalk-planar}  
The three preceding double walkups carry plain
hypermaps into plain hypermaps.
\end{lemma}

\begin{proof} The walkups $W_n$ and $W_f$ preserve the orbit
structure of edges, except for dropping one dart.  By dropping both
darts from the same edge, one edge is lost and all others edges
remain unchanged.

For the double $W_e$, we refer to Figure~\ref{fig:doubleedge}.  
If the
original hypermap is plain, then the two darts marked 
$x$ are equal, as are
the two darts marked $y$.  
The new edge map swaps $x$ and $y$, but is 
otherwise equal to $e$ on 
the reduced set of darts $D''$.  
Swapping two darts has order two, 
as we plainly see.
\end{proof}

The following is a useful way to tell if a walkup merges.


\begin{lemma}\tlabel{lemma:ng-merge}  
Suppose, in a simple hypermap, 
that an edge $\{x,y\}$ consists of two nondegenerate darts.  
Then the walkup
 $W_f$ (resp. $W_n$) at $x$  merges.
\end{lemma}

\begin{proof} 
We have $n (f x) = e^{-1} x = e x$. So $f x$ and $e x$ are at the
same node. If they are also in the same face of a simple hypermap, 
this gives $f x = e x
= y$. So $$n y  = n f x = n f e y = y,$$ and $y$ is a fixed
point of $n$, hence degenerate, contrary to assumption.  
Thus, $f x$
and $e x$ are in different faces, and the walkup merges
by Lemma~\ref{lemma:merge-split}.  
\end{proof}




\subsection{contour}

\begin{definition}[contour~path]  A contour path is a function $p:\{0,\ldots,k\}\to D$
such that $p_{i+1} =
n^{-1} p_i$ or $f p_i$ for each $i<k$.  (That is, each
step in the path is clockwise step around a node or a
counterclockwise step around a face.)   If the contour path
is injective on $\{0,\ldots,k-1\}$
and  $p_0 = p_k$, then it is a contour loop.
\index{Index}{contour path}\index{Index}{path}
\index{Index}{contour loop}\index{Index}{loop}
\end{definition}

\begin{lemma}\tlabel{lemma:connect-contour}  If $x$ and $y$ are darts
in the same group orbit of a hypermap, then there exists
a contour path from $x$ to $y$.
\end{lemma}

\begin{proof} 
By being in the same orbit, the darts
$x$ and $y$ are joined by a path, where each step
is $z\mapsto h z$, for $h=e,n$, or $f$.  Eliminate $e$-steps
through the relation $e\circ
n\circ f = I$.   Replace each $n$-step with a sequence of
$n^{-1}$-steps.  This gives the desired path.
\end{proof}

\begin{definition}[M\"obius~contour] A M\"obius contour is an
injective contour path $p$ that satisfies
    \begin{equation}
    \tlabel{eqn:mobius}
    p_j = n p_0\quad p_k = n p_i
    \end{equation}
for some $0 < i\le j< k$ (Figure~\ref{fig:mobius}).
\end{definition}

\begin{figure}[htb]
  \centering
  \myincludegraphics{noimage.eps}
  \caption{A M\"obius contour}
  \label{fig:mobius}
\end{figure}

\begin{figure}[htb]
  \centering
  \myincludegraphics{noimage.eps}
  \caption{A M\"obius contour with three darts}
  \label{fig:3m}
\end{figure}


\begin{remark}
G. Gonthier devised the notion of M\"obius contour
as a way to prove the Four-Color theorem without appeal
to topology.  (The Appel-Haken
proof of the Four-Color theorem relies on the Jordan
curve theorem.)  A M\"obius contour is a 
combinatorial M\"obius strip that
twists to make 
its left-hand side into
its right-hand side.  A planar hypermap has no such contour.  
Figure~\ref{fig:violate-jct}
redraws a violation of the Jordan curve theorem
as a M\"obius contour.   
\end{remark}

\begin{figure}[htb]
  \centering
  \myincludegraphics{noimage.eps}
  \caption{A path that tunnels from the interior to the exterior
   of a simple closed curve
   is analogous to a M\"obius contour.}
  \label{fig:violate-jct}
\end{figure}






\begin{lemma}\tlabel{lemma:no-mobius}  
Planar hypermaps have no M\"obius contours.
\end{lemma}

\begin{proof} For a contradiction, assume that there exist planar
hypermaps with M\"obius contours.  An edge walkup carries
planar hypermaps into planar hypermaps. An edge walkup
at a dart that is not on the M\"obius contour carries the
M\"obius contour to a M\"obius contour 
and reduces the number of darts.  
In the M\"obius Condition~\ref{eqn:mobius},
an edge walkup at a dart that is not at position $0$, $i$, $j$, $k$
along the contour carries the M\"obius contour to a M\"obius contour
and reduces the number of darts. Thus, a counterexample with
the smallest possible number of darts contains no
darts except those on the M\"obius contour, and its only darts
are at positions $0$, $i=j=1$, $k=2$.

This is a three darted hypermap (Figure~\ref{fig:3m}.)  
The M\"obius condition, the
definition of contours, together with $e\circ n\circ f=I$ force
$e=n=f$, all permutations of order three. We reach the contradiction
that this hypermap is not planar:
    $$3 = \# e + \# n + \# f,\quad 5 = 3+2 = \# D + 2 \#\tangle{e,n,f}.$$
\end{proof}


\begin{lemma}\tlabel{lemma:node-nonsimple}  
Let $(D,e,n,f)$ be a hypermap with no M\"obius contours.
Let $y$ be a dart at a node $N$.
Suppose that the $f$-orbit of $y$ meets $N$ in
at least two darts.  Then any contour path from $f y$
to $f^{-1} y$ passes through the node $N$. 
\end{lemma}

\begin{proof} Let $F$ be the $f$-orbit of $y$.
Let $i>0$ be the smallest index such that $f^i y\in N$.
By assumption $f^i y \ne y$.
Let $F' = \{y,f y,\ldots f^{i-1} y\}$.
For a contradiction, assume that there is
a contour path  $(f y,\ldots,f^{-1} y)$ that avoids $N$.
Note that $f y\in F'$ and $f^{-1} y\not\in F'$.
%
The contour path has a subpath $(w,x,\ldots,z)$, where
$w\in F'$, $z\not\in F'$,  $x,\ldots\not\in F$.  
We have $x=n^{-1}w$.

We form a path $x$ to $z$ along this subpath, then
follow $f$-steps to $f^i y\in N$, passing through $y$ and
then $n x = w$ along the way.
Then take $n^{-1}$-steps around the node $N$ to $n y$.  This
path is injective.  It follows the order
   $$
   x\ldots y\ldots n x\ldots n y.
   $$
This is a M\"obius contour.
\end{proof}


\begin{definition}[interior]\label{def:interior} 
a dart $y$ lies in the {\it interior} of a contour
loop $L$ if there is a an injective contour path
$x_0,x_1,\ldots,x_k=y$ such that $x_1 = f x_0$ (if $k>0$), and
such that $x_i$ lies on the loop $L$ if and only if $i=0$.
%$P$ is called an interior path from $x_0$ to $y$.
Write $D_{int}(L)$ for the set of darts in the interior of $L$.
\end{definition}

\begin{lemma}\tlabel{lemma:contour-path-type}
Suppose that a hypermap has no M\"obius contours.
Let $L$ be a contour loop.  Let $P$ be any injective contour path
that starts and ends on $L$, but visits no other darts of $L$ inbetween.  
Then the first and last steps of $P$ are both of the same
type ($n^{-1}$ or $f$).
\end{lemma}

\begin{proof}  Suppose $P$ is $n x,f n x,\ldots,n y,y$.   Form
contour path starting at $x$, then $n^{-1}$ steps to $L$, then
follow $L$ to $y$, and on to $n x$.  Follow $P$ back to $n y$.  This
is a M\"obius contour.

Suppose $P$ is $n x,x,\ldots,f^{-1} y,y$.  Form a contour path
starting at $x$, then along $P$ to $y$, along $L$ to $n x$, and
continuing on $L$ to $n y$.  This is a M\"obius contour.
\end{proof}


%\begin{definition}[interior]  Let $L$ be a contour loop on a hypermap.  We say that a dart
%lies in the interior (resp. exterior) of the loop $L$, if there is a
%path $P$ as in the previous lemma starting and ending with $f$ steps
%(resp. $n^{-1}$ steps).
%\end{definition}

\begin{lemma}\tlabel{lemma:dart-interior}
Let $L$ be a contour loop on a plain hypermap without
M\"obius contours.  Assume a dart $x$ lies in the interior of the loop $L$. 
Then every dart in its $f$-orbit lies in
the interior of the loop.  Moreover, if the dart
$x$ does not lie on the same node as any dart in $L$, then every
dart in the $n$-orbit of $x$ lies in the interior 
of $L$.
\end{lemma}

\begin{proof} Let $P= x_0,\ldots,x$ be an injective path that certifies that $x$ lies
in the interior of $L$.  If $f x$ lies along this path already or if it lies on $L$,
then it is clearly interior.  Otherwise, $x_0,\ldots,x,f x$ is a certifying path
for $f x$.  Similarly, use the certifying path $x_0,\ldots,x,n^{-1} x$ for $n^{-1} x$.
\end{proof}


\begin{definition}[interior~face,~node]  A face or a node is interior to a
loop in a hypermap if all of its darts are interior.
\end{definition}

\begin{lemma}\tlabel{lemma:contour-interior-exterior}
Suppose that a hypermap has no M\"obius contours.  Let $L$ be a contour loop.
Let $x$ be a dart interior to $L$.  Then there does not exist an injective contour path
$$x_0,\ldots,x_k=x$$
such that $x_i$ does not lie on $L$ for all $i$ and such that $x_0$ lies at a node
visited by $L$.
\end{lemma}

\begin{proof} Assume for a contradiction that the path exists. 
We can reverse the path to get a contour path from $x$ back to a dart $x'_0$ (not on $L$ but at a node
visited by $L$).  In fact, Lemma~\ref{lemma:dart-interior}, shows that the interior points consist of $f$-orbits
and $n$-orbits (away from nodes visited by $L$).  Thus an $f$-step from $x_i$ to $x_{i+1}$ can be replaced
with a sequence of $f$ steps from $x_{i+1}$ to $x_i$.  Similarly and $n^{-1}$ step from $x_i$ to $x_{i+1}$
can be replaced with a sequence of $n^{-1}$ steps from $x_{i+1}$ to $x_i$.  When we reach a dart $x'_0$ on a node
visited by $L$ we stop.  

There is a contour path $y_0,f y_0,\ldots,x$ that certifies that $x$ is interior.  Concatenate this
with the reversed path from $x$ back to $x'_0$.  Then add a sequence of $n^{-1}$ steps to reach a dart $z$
on $L$.  Pass to a subsequence to get an injective contour path $y_0,\ldots,z$ that starts with an $f$ step
and ends with an $n^{-1}$ step.  This is prohibited by Lemma~\ref{lemma:contour-path-type} 
\end{proof}


%
%\begin{lemma}\tlabel{lemma:dart-n-fixed}
%Suppose that in a nonempty hypermap without M\"obius contours,
%there is a face that coincides with a group orbit of
%darts.  Then that face contains a dart that is fixed under $n$.
%\end{lemma}
%
%\begin{proof}  If the face contains a single dart, then it is
%obviously fixed by $n$.  Assume that the face contains at least two
%darts.  For a contradiction, assume that none of the darts is fixed
%by $n$.  Thus, every node contains at least two darts.
%
%We will use the face path $z,f z,f^2 z,\ldots$ to construct a
%M\"obius contour. Since the set of darts is finite, the face path
%must eventually revisit a node already encountered.  Thus, we can
%find a subpath $z',f z',\ldots,f^{k+1} z'$ such that the first $k$
%darts lie on distinct nodes, but $f^{k+1} z'$ and $z'$ lie in the
%same node $A$.
%
%If we had $f^{k+1} z' = z'$, then we have the full $f$-orbit in this
%subpath, and hence also the full group orbit.  We have
%$k>0$, so $f z'$ is then a dart that has no other darts in its node,
%and is hence a fixed-point.  Hence $f^{k+1} z'\ne z'$.
%
%Continue the path further, so that $f^{r+1} z'$ is at the same node
%as some $f^p z'$ (with $p < r$), at a different node than
%$z',\ldots,f^r z'$, but so that the only repeated node among
%$z',\ldots,f^r z'$ is the one at $z'$.  We have $0 < p$.
%
%If $f^{r+1} z' = f^p z'$, then $f^{r+1-p} z' = z'$ so all darts are
%in the segment $z',f z',\ldots,f^{r-p} z'$ and the only node with
%more than one dart is the one at $z'$.  Thus, we have a fixed point.
%So $f^{r+1} z'\ne f^p z'$.
%
%Let $0 < k_1 < k_2 < \cdots < k_m < r+1$ be the indices at which
%$f^{k_i} z'$ is at $A$.  Let $x = n^{-1} f^{k_m} z'$. If the segment
%$x,n^{-1} x,\ldots,z'$ contains some $f^{k_i} z'$, we obtain a
%M\"obius contour.  [X DETAIL.]  Take $n^{-1}$ steps from $x$ to
%$z'$. Then take $f$-steps to $y = f^p z'$, on to $n x = f^{k_m} z'$,
%on to $f^{r+1} z'$, then $n^{-1}$ steps to $n y$.  This is a
%M\"obius contour.
%\end{proof}
%
%\begin{lemma}[Jordan curve theorem for hypermaps]\tlabel{lemma:jct-hypermap}
%  If a plain hypermap
%has no M\"obius contours then it is planar.
%\end{lemma}
%
%\begin{proof}  Face double walkups along edges preserve the planar
%index.  By repeated application, we reduce to the case where every
%group orbit contains a single face. In
%other words, $f$ acts transitively on the darts in a given group orbit.
%
%If there are any darts that are fixed by all three maps $e,n,f$,
%then the walkup at that dart eliminates the dart while preserving
%the planar index.  Thus, we may assume there are no such darts.
%
%If there are any darts that are fixed points under $n$, then the
%dart is degenerate.  The double walkup (of any type) along the edge
%that meets that dart eliminates the dart while preserving the planar
%index.  Thus, we may assume that there are no such darts.
%
%By the previous lemma, the plane hypermap must be empty.  Thus, our
%planar-index preserving walkups have transformed an
%arbitrary plane hypermap without M\"obius contours into the empty
%hypermap, which is clearly planar.  The result follows.
%\end{proof}
%




\section{Quotient}


\begin{definition}[isomorphic] Two hypermaps $(D,e,n,f)$ and $(D',e',n',f')$ are
isomorphic, if there is a bijection $F:D\to D'$ such that
    $$h'\circ F = F\circ h$$
for $(h,h')=(e,e'), (f,f'), (n,n')$.
\end{definition}


\begin{definition}[normal~collection]
Let $H$ be a hypermap. Assume that 
there are no darts fixed by $e$ 
(so that we can distinguish $f x$
from $n^{-1} x$ at each dart). 
Let $\cal L$ be a collection of contour
loops.  We say that $\cal L$ is a normal collection if the following
conditions hold of its loops. \begin{itemize}
 \item No dart is visited by two different loops.
 \item Every loop visits at least two nodes.
 \item If a loop visits a node, then every dart at that node is
 visited by some loop.
\end{itemize}
\end{definition}

From a normal collection we can form a new hypermap.   A dart in the
new set $D'$ of darts
$D$ is a maximal sequence $[x,n^{-1} x, n^{-2} x,\ldots,n^{-k} x]$
    of $n^{-1}$ steps appearing in some loop in $\cal L$.
The map $f'$ takes the maximal sequence
    $[x,n^{-1}x,\ldots,y]$ to the maximal
   sequence (in the same contour loop) starting 
    $[f y,\ldots]$.
The map ${n'}^{-1}$ takes the maximal sequence
    $[\ldots,y]$ to the maximal sequence (in some other contour loop)
starting $[n^{-1}y,\ldots]$. Equivalently, 
$n'$ takes the maximal sequence
$[x,\ldots]$ to the maximal sequence ending $[\ldots,n x]$. The map $e$ is
defined by $e'\circ n'\circ f' = I$.  

\begin{definition}[quotient]  The hypermap constructed from the normal collection
is called the quotient of $H$ by $\cal L$, and is denoted $H/{\cal
L}$.  The hypermap $H$ is said to be a cover of $H/{\cal L}$.
\end{definition}

Intuitively, we can represent the quotient hypermap as a graph whose
cycles under $f$ 
are precisely the contour loops in the normal family (Figure~\ref{fig:quot}).

%% No sketch has been made for this.
\begin{figure}[htb]
  \centering
  \myincludegraphics{noimage.eps}
  \caption{The contour loops in a normal 
   family become faces in the
   quotient}
  \label{fig:quot}
\end{figure}


\begin{example}\label{ex:Hall} 
Assume that $H$ is a hypermap with no fixed points under $e$.
Assume that every face visits at least two nodes.
Then the set of all faces
defines a normal collection of contour loops (follow $f$ around each face:
$x,f x,\ldots$).  Each dart of the quotient is then just a singleton
set consisting of a single dart of $H$, and the quotient is
isomorphic to $H$ itself.
\end{example}

\begin{example}\label{ex:H2} 
Assume that $H$ is a hypermap with no fixed points
under $e$.  Let $F = (x,f x,\ldots)$ be a face 
that visits at least
three nodes and that meets each node in at most one dart.
Let $\cal L$ be the
collection with two contour loops:  $(x,f x,\ldots)$ and its
complement
$$[n^{-1} x,
n^{-2} x,\ldots,n x,f n x = y,n^{-1} y, n^{-2} y,\ldots, n y, f n
y,\ldots]
$$
(See Figure~\ref{fig:contour-comp}.) 
The family $\cal L$ is normal.
The quotient hypermap $H/{\cal L}$ has two faces $F$ and a
backside $F'$ of the same cardinality $k$.
\end{example}


\begin{figure}[htb]
  \centering
  \myincludegraphics{noimage.eps}
  \caption{The complementary contour loop traces the remaining darts
   at the same nodes as the origonal contour loop.}
  \label{fig:contour-comp}
\end{figure}


\begin{example}\label{ex:H2k} 
We 
construct a
hypermap $H_{2k}$, whose darts are arranged in two faces.  Each
face is $Z_n$, a cyclic group of order $n$ with generator $1$.
The face map is $x\mapsto x+1$.
The node map swaps the two faces $Z_n$.
The usual condition $e\circ n\circ f = I$ defines the edge map.
If a hypermap is isomorphic to $H_{2k}$ for
some $k$, then we say that it is {\it cyclic}.  For example,
the hypermap constructed in the previous example is cyclic.
\index{Index}{cyclic hypermap}
\end{example}

\begin{lemma}\tlabel{lemma:quotient-plain}
Let $H$ be a plain hypermap, and let $\cal L$ be a
normal family.  Then $H/{\cal L}$ is a plain hypermap.
\end{lemma}

\begin{proof} Let $e'$, $f'$, and $n'$ be the edge, face, and node maps on the
quotient hypermap.  Write $[\ldots, x]$ for the node in the quotient
ending in dart $x\in H$ and $[x,\ldots]$ for the node in the quotient
starting with dart $x\in H$.  We have $e^2 x = x$, so that for any
dart $[\ldots x]$ in the quotient:
    $$\begin{array}{lll}
    {e'}^{-2} [\ldots, x] &= n' f' n' f' [\ldots, x] = n' f' n' [f x, \ldots] \\&=
    n' f' [\ldots, n f x] = n' [f n f, \ldots] = [\ldots, n f n f x]\\ &=
    [\ldots, e^{-2} x] = [\ldots, x].
    \end{array}$$
Thus, $e'$ has order $2$ on the quotient.
\end{proof}

%\begin{lemma}\tlabel{lemma:quotient-planar}
%Let $H$ be a plain planar hypermap, and let $\cal L$
%be a normal family.  Then $H/{\cal L}$ is a plain planar hypermap.
%\end{lemma}
%
%\begin{proof} Suppose $H/{\cal L}$ is not planar.
%Let $P$ be a M\"obius contour on $H/{\cal L}$.  It lifts uniquely to
%a contour on $H$ with the property that the darts visited on $H$ are
%precisely the darts that belong to a dart in the quotient.  This is
%compatible with the node map $n$.  So the contour path lifts to a
%M\"obius contour on $H$.  Thus, $H$ is not planar.
%\end{proof}


\subsection{flag}

By the end of the chapter, we present an algorithm that
generates all plain planar hypermaps satisfying certain general
conditions.   The algorithm  proceeds by inserting one face
at a time into a hypermap.  
The algorithm  marks certain faces as `true.'
Roughly, this  means that the the face cannot be modified
at any later stage of the algorithm.   When all of its faces
are true, the hypermap stands in final form.
The function that marks each face as true or false is a
{\it flag}.


\begin{definition}[flag]  A flag on a hypermap
is a boolean function on its set of faces 
that satisfies the following two
constraints.
\begin{itemize}
    \item If darts $x,y$ belong to true faces,
    then there is a contour path from $x$ to $y$ that remains
    in true faces.
    \item Every edge of a false face is shared with a true face.
    \end{itemize}
%An isomorphism of flagged hypermaps is an isomorphism of
%hypermaps that respects the flags.
\end{definition}

\begin{example} Let $H/{\cal L}$ be a quotient hypermap with
normal family $\cal L$.  The canonical boolean function
$\phi$ on the set of faces of the quotient is the function that
is true exactly when every dart in
the face is a singleton of $H$.
\end{example}

\begin{example} The cyclic hypermap of Example~\ref{ex:H2k}, 
carries a flag
that marks one face true and the other false.
\end{example}

\begin{example} Let 
$H$ be a plain hypermap with a single group orbit, and 
such that $e$ has no fixed points,
and let $\cal L$ be the example of Example~\ref{ex:Hall}, 
then the canonical
map takes value $\op{true}$ on every face.  This is a flag.
In fact,
Lemma~\ref{lemma:connect-contour} provides the contour paths 
that are required in the definition of flag.
\end{example}

\begin{lemma}\tlabel{lemma:all-dart}  
Let $H$ be a hypermap with normal family ${\cal L}$.
If the canonical boolean function on the set of faces of
$H/{\cal L}$ has at least as many
true values as there are faces of $H$, then $\cal L$ is the normal family
in Example~\ref{ex:Hall}. In particular, $H/{\cal L}$ is isomorphic to $H$.
\end{lemma}

\begin{proof}  If a face takes value $\op{true}$ 
in $H/{\cal L}$, then its darts are
singletons, and the face of $H/{\cal L}$ is naturally identified with
 a face in $H$.  This is an injective map from the 
set of true faces of $H/{\cal L}$ to
the set of faces of $H$.  The hypothesis of the lemma implies that this
injective map is bijective.
All of the darts of $H$ are accounted for under this bijection.
Thus, the quotient has no
false faces.  The result follows.
\end{proof}


\subsection{face insertion}


In this section, we describe an inductive construction of all
plain, planar hypermaps with a single group orbit that satisfy certain constraints.
We define a set ${\cal H}$ of tuples $(H,\phi,E,N,\lambda)$.
The tuples are assumed to satisfy the following conditions:
\begin{itemize}
    \item $H$ is a hypermap.
    \item $\phi$ is a flag on $H$.
    \item $E$ is an edge of size two, with one
    dart in a true face and the other in a false face.
    \item $N$ is the size of the false face meeting $E$.
    \item $\lambda$ is a finite increasing sequence of natural numbers:
        $$
        \lambda = (\lambda_1,\lambda_2,\ldots,\lambda_r)
        $$
    such that $\lambda_1 = 1$ and $\lambda_r \le N-1$.
\end{itemize}

We define a function from $\cal H$ to the set of hypermaps with
flags.  We call it face insertion, because the function creates a
hypermap with one more true face than its source hypermap $H$.

We write $x_0$ for the dart of $E$ in the
false face and $y_0$ for the dart of $E$ in the true face.  
Write $f^i
x_0 = x_i$ for the darts in the false face.  (Set $x_N = x_0$.)

\subsection{partition transformation}

We describe an algorithm $\lambda\mapsto A(\lambda)$ that takes
as input a partition $\lambda$ (and the number $N$)
and produces as output a list of
tuples.
In the first step of the algorithm,
replace each $\lambda_j$ except for the
first occurrence of a given natural number by a dummy symbol $*$.
Insert $N$ at the end of the list. For example, if $N=7$,
    $$(1,1,1,2,3,3,3,5,5)$$ becomes
    $$(1,*,*,2,3,*,*,5,*,7).$$

Next, break the sequence at each natural number (except at the initial $1$
and terminal $N$)
creating shorter sequences, duplicating the number at the break,
so that each sequence begins and ends with a natural number,
and is padded inbetween with dummy symbols. So
    $$(1,*,*,2,3,*,*,5,*,7)$$ becomes
$$(1,*,*,2),\ (2,3),\ (3,*,*,5),\ (5,*,7).$$

Next,
delete any ordered pair of the form $(j,j+1)$. In our example, there
is one such ordered pair: $(2,3)$.  Our example becomes
  $$
  (1,*,*,2),\  (3,*,*,5),\ (5,*,7).
  $$
This list of tuples is the output from the algorithm.
In our example,
  $$A(1,1,1,2,3,3,3,5,5) = ((1,*,*,2),(3,*,*,5),(5,*,7)).$$



\subsection{new hypermap}

This output $A(\lambda)$
becomes an instruction for how to draw new edges to create a
new hypermap. A tuple $(j,*,*,\ldots,*,k)$ with $r$ dummy symbols
is an instruction to insert an edge between darts $x_j$ and $x_k$
and then to insert $r$ new nodes along this edge.
Expressed in terms of double walkups, we first apply (in
reverse) the double walkup $W_f$, and then apply $r$ times (in
reverse) the double walkup $W_n$.  Each tuple creates $r$ new
nodes, one new face, and $r+1$ new edges.  This action, 
iterated over all the tuples in $A(\lambda)$, creates a new hypermap
$H'$. 
The function that carries
$(H,\phi,E,\lambda)$ to $H'$ is called
face-insertion.  
Figure~\ref{fig:7darts} gives an example.


\begin{figure}[htb]
  \centering
  \myincludegraphics{noimage.eps}
  \caption{Face-insertion on a $7$-gon when
   $A(\lambda) = ((1,*,*,2),(3,*,*,5),(5,*,7))$.}
  \label{fig:7darts}
\end{figure}


The double node and face walkups preserve edges (except for
the one that the walkup removes).
Thus, the edge $E=\{x_0,y_0\}$
becomes an edge $\{x_0',y_0'\}$ in $H'$
(with $y_0$ corresponding with $y'_0$).

We mark the face containing the dart $x'_0$ true, and the other
faces created by face-insertion false.   All other faces are
marked true or false as in the flag on the originating hypermap $H$.  
Through this
construction, the $H'$ contains exactly one more true face
than the originating hypermap.

\begin{lemma}\tlabel{lemma:flag} 
The boolean function on faces constructed in this way is a
flag.
\end{lemma}

\begin{proof}  We show first that every edge of a false face is shared
with a true face.  If the false face is one of the newly created
faces, its edges are those along the new true face together
with some edges from the old false
face. Either way, the edges are shared with a true face.   If it is a
false face from the originating hypermap $H$, 
it shares edges with the
same true faces as before.

Next, we show that every pair of darts $x,y$ in true faces can be
joined by a contour path.  Let $H'=(D',e',n',f')$.  
If neither of $x,y$ lies in the
$f'$-orbit of $x'_0$, we can use the contour path that was used in the
originating hypermap.  If both are in the $f'$-orbit of $x'_0$, then
we can join $x$ to $y$ by a sequence of $f'$-steps.

If $x$ lies in the $f'$-orbit of $x'_0$, but $y$ does not, 
we first join $x$
to $x'_0$ by $f'$-steps, then to $n'^{-1} x'_0= f' y'_0$, then to $y'_0$,
which lies in a true face.
From $y'_0$, we follow a path $y$.
Similarly, if $y$ lies in the $f'$-orbit of $x'_0$ but $x$ does not; 
we follow a contour path from $x$ to $y'_0$, then to $n'^{-1} y'_0 = f'
x'_0$, then to $y$.
\end{proof}

\subsection{generating hypermap}

The following lemma
and the remark that follows are the principal results of the
chapter.  They present an algorithm that 
generates all nice planar hypermaps by face-insertions.
This book uses the algorithm to classify of a family of
planar hypermaps, called {\it tame hypermaps}.

\begin{theorem}  Let $H$ be a hypermap with the following properties:
    \begin{enumerate}
        \item It is plain, planar, simple, and has a single group orbit.
        \item The edge map $e$ has no fixed points.
        \item The node map $n$ has no fixed points.
        \item The size of every face is at least $3$.
        \item There are at least $2$ faces. 
        %%  (All hypoth. Needed?)
    \end{enumerate}
Let $\cal L$ be a normal family of contour paths in $H$ with
simple quotient. Assume that
the canonical boolean function $\phi$ on $H/{\cal L}$ is a flag. Let
$E$ be an edge of $H/{\cal L}$ that meets both a true face
and a false face. Let $N$ be the cardinality of that false face. Let
$M$ be a constant such that every face of $H$ has cardinality at
most $M$. Then there is exists a partition $\lambda=(\lambda_1,\ldots,\lambda_r)$ with at most
$M$ parts satisfying $$1=\lambda_1\le \cdots\lambda_r \le N-1$$
and a normal family
${\cal L}'$ with simple quotient such that $H/{\cal L}'$  such that
with its canonical boolean
function is isomorphic to the image of face-insertion on
$(H/{\cal L},\phi,E,N,\lambda)\in {\cal H}$. 
Moreover, $H/{\cal L}'$
has one more true face than $H/{\cal L}$ has.
\end{theorem}

\begin{remark}
\label{remark:nice-algorithm} 
The lemma implies that sufficiently nice hypermaps can be recovered
from their quotients by face insertions.  (Here {\it `nice'} means
that it satisfies the conditions of the lemma.)  
Every nice hypermap has a cyclic quotient 
$H_{2k}=H/{\cal L}_0$
(Examples~\ref{ex:H2} and~\ref{ex:H2k}).
Generate all face-insertions of $H/{\cal L}_0$ 
(as we run over relevant partitions).
Among the new hypermaps is some $H/{\cal L}_1$, which
has one more $\op{true}$ face than $H/{\cal L}_0$.  By 
Lemma~\ref{lemma:flag}, 
the canonical boolean function on this new quotient
is a flag.  Iterate to get a sequence
$H_{2k} = H/{\cal L}_0, H/{\cal L}_1,\ldots, H/{\cal L}_k$,
each obtained from the previous one by face-insertion.
By Lemma~\ref{lemma:all-dart}, 
as
soon as a quotient $H/{\cal L}_k$ 
has sufficiently many true faces, it is
isomorphic to the the original hypermap $H$, and the process
terminates.
Note that we choose the edge at which we apply the face
insertion, but we must run through
all partitions.  To make the entire process finite, we can bound the
number of partitions that must be considered, say by placing a
priori bounds $M$ on the cardinalities of the faces of $H$.
\end{remark}

\begin{proof}  We set some notation.
Write $f,e,n$ for the maps on $H$ and $f',e',n'$ for the maps on the
quotient hypermap.   
The dart $y'$ of $E$ that 
lies on the true face is a
singleton set, which can be identified with a dart $y$ in $H$.  Let
$y_0 = e y$, and let $y_k = {f}^k y_0$ be the darts of $H$ on the
face of $y_0$ (Figure~\ref{fig:algordart}).  
Let $r$ be the size of this face. 

We define a map $\psi$.
Let $*$ be a symbol.
A map
$\psi$ on the face of $y_0$ sends 
$y_i$ to $*$ if $y_i$ is disjoint from
the loops of ${\cal L}$ and sends $y_i$
to the dart of $H/{\cal L}$
containing $y_i$ otherwise.
As $H$ is simple, no two darts
$y_k$ map to the same dart in $H/{\cal L}$.
\begin{figure}[htb]
  \centering
  \myincludegraphics{noimage.eps}
  \caption{An example of the algorithm}
  \label{fig:algordart}
\end{figure}

We can write the map $\psi$ in terms of a map $\psi'$ on indices.
Let $x_0 = e' y'$ and let $x_k = {f'}^k x_0$.  Let $r'$ be the
cardinality of the $f'$-orbit through $x_0$.  
We claim that (1) $\psi(y_i)$ is either $*$ or one of the
darts $x_k$.  We justify this claim below.
When $\psi(y_i)$ is a dart, set $\psi(y_i) =
x_{\psi' i}$, for some $0< i \le r$, $0 < \psi' i \le r'$. 
When $\psi(y_i)$ is not a dart,
set $\psi' i = *$, 
We claim further that (2) if
$0< i< i'\le r$, and both $\psi(y_i)$ and $\psi(y_{i'})$ are darts, then
$\psi' i < \psi' i'$.  We justify this claim below.
Accepting these two claims, we complete the proof.    

We construct a partition $\lambda$.
Form a sequence
    $$(\psi' 1,\psi' 2,\ldots,\psi' r).$$
Form a partition $\lambda$ from this sequence, 
by replacing each $*$ in the
subsequence $(j,*,*,\ldots,*)$ with $j$. Thus,
    $$(1,*,*,2,*,4,*,*,5)$$
becomes
    $$(1,1,1,2,2,4,4,4,5).$$
Face-insertion on $(H/{\cal L},\phi,E,N,\lambda)$ yields a new
hypermap $H'$.  

We claim that $H'$ is isomorphic to a quotient
$H/{\cal L}'$, which is specified as follows. The collection
${\cal L}'$ contains all the loops of ${\cal L}$ 
except the loop through the $f'$-orbit of $x_0$.  The collection
${\cal L}'$ also contains
the loop corresponding through the $f$-orbit
of $y_0$ as well as the loops through the false faces on
the new hypermap.  

We describe the contour loops associated with false faces 
in more detail.
The false faces on the new hypermap give a
contour loop of ${\cal L}'$ for each subsequence
$(j,*,\ldots,*,j')$ with $j< j'$ 
and one for each subsequence $(j,j')$ with
$j+1< j'$.  
First take the case $(j,j')$, with $j+1<j'$. 
Write $(j,j') = (\psi'i,\psi' i')$.
The inequality $j+1<j'$ implies that both $n^{-1}
y_i$ and $n y_{i'}$ are in $x_j$ and $x_{j'}$ respectively.
The contour loop for $(j,j')$ runs through 
all the darts of $x_k$ for $j<k<j'$,
the sequence of darts of $x_{j'}$ ending with $n y_{i'}$,
and the sequence darts of $x_j$ starting with $n^{-1} y_i$.

We continue with our description of the contour loops associated
with false faces.
In the case $(j,*,\ldots,j')$, we form consecutive sequences of darts
starting with $n^{-1} y_i$ and another ending at 
$n y_{i'}$ as in the previous case, and
all the darts of $x_k$ for $j<k<j'$, 
Each $y_k$, for $i < k < i'$ belongs to a node $N_k$ in $H$.
We also have the loop visit the darts in $N_k\setminus\{y_k\}$.

Now that the collection ${\cal L}'$ has been described in detail,
it is a routine matter to check that the quotient $H/{\cal L}'$
has the properties asserted in the lemma.
%
%{\bf Claim 1}:  % revised 26/10/07: 
%The proof is now complete except for two earlier claims.  We claim
%that $\psi(y_i)$ is either $*$ or one of the darts $x_k$. 
%For a contradiction, 
%suppose that the dart $z$ of $H/{\cal L}$ containing
%$y_i$ is not equal to any $x_k$.
%Let $L$ be the $f'$-orbit
%of $x_0$.  This is a false face.  This implies that some
%dart of $L$ has the form $x_p=[\ldots,y_q,n^{-1} y_q,\ldots]$.
%Set $x = n^{-1} y_q$.  The face $L$ has a dart of the form 
%$[y_1,\ldots]$.
%Set $y=y_1$.  The dart $n y$ lies in a true face.
%Start at $x$ and follow the darts of $H$ around the contour $L$ 
%to $n x = y_q$, passing through $y$ along the way.  From
%$y_q$, take $f$-steps to $y_i$ in $z$, 
%then step around that node to a true face
%of $H/{\cal L}$.  From there,
%follow a path along true faces (as guaranteed by the definition
%of flag) to $n y$.  The path follows the order
%   $$x \ldots y \ldots n x \ldots n y.$$ 
%We may assume that the path from $y_q$ to $y_i$ is minimal in the
%sense that the $f$-steps between $y_q$ to $y_i$ 
%are darts that map to $*$.  
%% We can clearly cut away from the end if there is another dart
%% like $y_i$.  We can cut away from the front if there is another
%% dart that maps to some x_k.
%% Injective because the 
%With this extra condition, the contour
%is injective and hence a M\"obius contour in 
%the planar hypermap $H$.  
%This contradicts Lemma~\ref{lemma:no-mobius}.
%
%%% 
%{\bf Claim 2}: 
%We claim further that if $i<i'\le r$, and both $\psi(y_i)$ and
%$\psi(y_{i'})$ are darts, then $\psi' i < \psi' i'$.  
%For a contradiction, assume that $\psi' i' \le \psi' i$. 
%In fact, we may assume that $\psi'i' < \psi' i$, because
%each dart of the face of $y_0$ lies at a different node. 
%We may assume that $\psi' j = *$ for $i < j < i'$.  We may
%also assume that the indices $\psi' 1,\psi' 2,\ldots,\psi' i$
%(excluding those that are $*$) are increasing.  
%That is we pick $i$ and $i'$ so that we have the longest
%possible run of increasing indices, terminated by a falling
%index.
%%Form a contour loop
%%$L'$ from $y_1$ along the darts $y_{i'}$ until reaching $x_{\psi' i}$,
%%then continuing for the rest of the path of $L$ back to $y_1$.
%%Giving the identical argument as in the preceding paragraph, but
%%using $L'$ instead of $L$, we again get a M\"obius contour $H$.  
%%Again, we contradict Lemma~\ref{lemma:no-mobius}.
%Let $L$ be the $f'$-orbit of $x_0$.  The is some dart of $L$
%of the form $x_p = [\ldots,y_q,n^{-1}y_q,\ldots]$, with
%$i\le q< i'$.  Let $x = n^{-1}y_q$.  Let $y= y_1$.  The
%dart $n y$ lies in a true face.  Start at $x$ and
%follow the darts of $H$ along $L$ to $y$, then take
%$f$-steps to $y_i$, follow the darts along $L$ to $n x =y_q$.
%Then take $f$-steps again to $y_{i'}$.  Take $n^{-1}$-steps
%around that node to a dart that lies on a true face.  Follow
%true faces back to $n y$.  The path follows the order
%  $$
%  x\ldots y \ldots n x\ldots n y.
%  $$
%  With these
%extra conditions, the contour
%is injective and hence a M\"obius contour in 
%the planar hypermap $H$.  
%This contradicts Lemma~\ref{lemma:no-mobius}.

{\bf Claims 1 and 2:}\FIXX{Check once more.}
The proof is now complete except for two earlier claims.  We claim
that $\psi(y_i)$ is either $*$ or one of the darts $x_k$ and that the darts
$x_k$ have increasing indices.  If the claim is false, then
we can find a counterexample with smallest index.  That is, if we remove
from the list
   $$
   \psi(y_1),\psi(y_2),\ldots,\psi(y_{i'-1})
   $$
those that equal $*$, then we are left with an increasing sequence of 
darts $x_k$.  But no larger index $i'$ has this property.  Thus, either
$\psi(y_{i'})$ is a dart that is not of the form $x_k$ or has this form
but with a drop in index $k$.  Let $i\le i'-1$
be the largest index for which $\psi(y_i)=x_k$ for some $k$.
The dart $x_k$ has the form $[\ldots,y_i,n^{-1}y_i,\ldots]$.  Let $x=n^{-1}y_i$.
Let $y=y_1$.  The dart $n y$ lies in a true face.

Let $L$ be the $f'$-orbit of $x_0$.  This is a false face.
Start at $x$, follow a contour path along $L$ until reaching $y$, then
take $f$-steps from $y$ to $y_{i'}$, passing $y_i = n x$ along the way.
One of the loops of the collection ${\cal L}$ passes through the the node of $y_{i'}$.
Take $n^{-1}$-steps at that node until a dart in a true face is reached.  Then
follow a contour path through true faces (as guaranteed by the definition
of flag) to $n y$.  
This contour path follows the order
  $$
  x\ldots y \ldots n x\ldots n y.
  $$
Since we picked the indices $i,i'$ to give the smallest counterexample, it can be checked
that this contour path is injective and hence a M\"obius contour in 
the planar hypermap $H$.  
This contradicts Lemma~\ref{lemma:no-mobius}.
\end{proof}



\section{Connected Sum}\label{sec:csum}

A hypermap has more detail than its quotient hypermaps.
When a hypermap overwhelms with detail, a carefully chosen
quotient can hide the unwanted detail and expose its main features.
Face-insertion works in the opposite direction,
by adding structure to a hypermap.
The algorithm of Remark~\ref{remark:nice-algorithm}
builds all nice hypermaps through a series of face-insertions.

This section describes a second way to embellish
a hypermap.  It will be used to restore detail that
a quotient has hidden.  Imagine a normal family in which every
contour loop except one is a single face.  In the quotient, that
exceptional loop becomes a face, a facade hides the detail of what
the original hypermap contains.  A connected sum 
restores what the facade
hides.

%Let $H=(D,e,n,f)$ be a hypermap.  We write
%$\op{face}(x)$, $\op{edge}(x)$, and $\op{node}(x)$
%for the face, edge, and node containing the dart $x\in D$.
%We write $H/f$, $H/e$, and $H/n$ for the set of faces,
%edges, and darts.

The connected sum construction is a combinatorial analogue of
the connected sum of two surfaces.  In this topological construction,
disks are removed from both surfaces and the circular boundaries
of the two disks are glued together with a map that reverses orientation.
In a connected sum, faces are removed
from both hypermapss and the remaining darts are
are spliced together by means of the
map $\phi$ that reverses orientation on faces.  

To expand this analogy, 
we note that a quotient replaces each contour loop with a face.  This
is a combinatorial analogue of decomposing a surface by cutting
along 
embedded circles and then capping the circles with disks
(Figure~\ref{fig:connected-sum}).
% http://en.wikipedia.org/wiki/Manifold_decomposition.


Let $A$ and $B$ be hypermaps with disjoint dart sets $D_A, D_B$.
Let their 
face and node
maps be
    $f_A,f_B$, and $n_A,n_B$, respectively.
Let $\phi:F_A\to F_B$ be a bijection between a face of $A$ and a
face of $B$ such that
    $$
    \phi(f_A x) = f_B^{-1}\phi(x),\quad \forall
    x\in F_A.
    $$
Assume that the faces $F_A$ and $F_B$ have at most one dart at
every node.

\begin{definition}[connected~sum,~$\#$] Let the connected sum
$A\,\#_\phi\, B$ be the following hypermap
$(D,f,n,e)$:
    $$D = D_A \cup D_B \setminus (F_A\cup F_B)$$
    $$f x = f_A x \text{\ if }  x\in D_A,\qquad \ f_B x
    \text{\ \ otherwise}.
    $$
    $$n x = \begin{cases}
    n_A x &
        \text{if }  x\in D_A \text{ and } n_A x\not\in F_A\\
    n_B[\phi(n_A x)] &
        \text{if }  x\in D_A \text{ and } n_A x\in F_A\\
    n_B x &
        \text{if }  x\in D_B \text{ and } n_B x\not\in F_B\\
    n_A[\phi^{-1}(n_B  x)] &
        \text{if }  x\in D_B \text{ and } n_B x\in F_B\\
    \end{cases}
    $$
Let $e$ be defined by the relation $e\circ n\circ f = 1$.
\end{definition}

The connected sum leaves faces of $A$ and $B$ intact,
except for the two faces that are deleted.  The node map
leaves all the nodes of $A$ and $B$ intact, except those
at nodes of $F_A$ and $F_B$.  Along these nodes, the nodes
are joined.



\begin{figure}[htb]
  \centering
  \myincludegraphics{noimage.eps}
  \caption{The connected sum construction}
  \label{fig:connected-sum}
\end{figure}


To justify this construction, the following lemma shows that the
maps $\phi,\phi^{-1}$ are applied only to darts in their domains.

\begin{lemma}\tlabel{lemma:one-dart-one-node}
With context as above (simple hypermaps, etc.),
if $x\in F_B$ then $n_B x\not\in F_B$.  If $x\in
F_A$, then $n_A x\not\in F_A$.
\end{lemma}

\begin{proof} The faces $F_A$ and $F_B$ have at most one
dart at each node.
\end{proof}

\begin{lemma}\tlabel{lemma:sum-no-fp}
If $n_A$ and $n_B$ have no fixed points, then
    the node map on the connected sum does not either.
\end{lemma}

\begin{proof}  The node map leaves all the nodes of $A$ and $B$
intact, except those at nodes of $F_A$ and $F_B$.  Along these
nodes, the nodes are joined.  If the node $N_A$ of size $a>1$ is
joined with node $N_B$ of size $b>1$, a node of size $a+b-2>1$ is
formed.
\end{proof}

\begin{lemma}\tlabel{lemma:sum-simple} 
If $A$ and $B$ are simple, then
the connected sum is simple.
\end{lemma}

\begin{proof} The connected sum preserves the structure of faces.
The set of nodes of $A$ (resp. $B$) maps one-to-one to the 
set of nodes of the
connected sum.  Thus, in the connected sum, each face visits each
node at most once.
\end{proof}

\begin{lemma}\tlabel{lemma:sum-plain}
If $A$ and $B$ are plain, then
the connected sum is plain.
\end{lemma}

\begin{proof} The edges of $A$ and $B$ that do not have a dart
in $F_A\cup F_B$ carry over to edges of the connected sum.  For
such edges, the verification is easy.

Let $x$ be a dart in $A$ such that $e_A x \in F_A$.  Then,
     $$
     \begin{array}{lll}
     e^{-2} x &= n f n f x \\ &= n f\, (n_B \phi n_A)\, f_A x\\
               &= (n_A \phi^{-1} n_B)\, f_B\, (n_B \phi n_A)\, f_A x\\
               &= n_A\, (\phi^{-1} f_B^{-1})\, e_B^2 \phi\, n_A f_A x\\
               &= n_A (f_A\, \phi^{-1})\,\phi\, n_A f_A x\\
               &= e_A^2 x\\
               &= x.
     \end{array}
     $$
The proof for a dart $x$ in $B$ is similar.
\end{proof}

\begin{lemma}\tlabel{lemma:sum-planar}
Assume that $A$ and $B$ are plain.
Let $C$ be the connected sum of $A$ and $B$.
If two of $A$,$B$,$C$ are planar, then
    the third is planar as well.
\end{lemma}

\begin{proof}  We count the number of nodes, edges, faces, and orbits
on the connected sum.  Let $r$ be the size of $F_A$ (or $F_B$).
$$
\begin{array}{lll}
  \#n_C &= \#n_A + \#n_B - r\\
  \#e_C &= \#e_A + \#e_B - r\\
  \#f_C &= \#f_A + \#f_B - 2\\
  \#D_C &= \#D_A + \#D_B - 2 r\\ 
  \#\tangle{e_C,n_C,f_C} &= \#\tangle{e_A,n_A,f_A} + \#\tangle{e_B,n_B,f_B} - 1\\
\end{array}
$$
From this, it follows that two planar indices are zero,
then so is that of the connected sum.
\end{proof}



\subsection{connected sum and quotient}

The connected sum construction adds detail to a hypermap and
a quotient removes detail.  We formulate this in a special
case as a lemma.    The contour loop $C$ becomes a single
face $F$ in the quotient hypermap.

Let $C$ be a contour loop in a hypermap $(D,n,e,f)$.
We make the following assumption of simplicity:
$C$ is injective and visits at least two nodes.  
Moreover, if $x,y$ are
two darts in the contour path at the same node, then
$$
x, n^{-1} x,\ldots, y,\quad\text{or}\quad y, n^{-1} y,\ldots, x
$$
is a sequence of steps along the path.
From this hypothesis, it follows that there does not exist a node
at which every dart belongs to $C$, because this node would then be
the entire contour loop, contrary to the assumption of at least two nodes.
It follows that each node visited by $C$, there are two darts $x,y$
such that the darts visited by $C$ are
   $$
   x,n^{-1}x,\ldots,n y,
   $$
with remaining darts
   $$
   y, n^{-1}y,\ldots, nx.
   $$
We define the complementary contour loop $C'$ to be the loop that
passes through these remaining darts, joining the nodes together with $f$-steps:
   $$
   y_1,n^{-1}y_1,\ldots,n x_1, f n x_1 = y_2, n^{-1} y_2,\ldots.
   $$

\begin{lemma}  Let $(D,e,n,f)$ be a plain hypermap.
This sequence $C'$ is a contour loop that satisfies the assumption
of simplicity.  Moreover, it visits all the nodes that $C$ does.  
\end{lemma}

\begin{proof}  The first statement is clear.  When $C$ makes an $f$-step
from $n y_k$ to $x_{k+1}$,  the loop $C'$ makes a corresponding $f$-step
from $n x_{k+1}$ to 
$$f n x_{k+1} = f n f n y_k = f e n y_k = y_k.$$  Thus, the two loops
visit the same nodes.
\end{proof}

\begin{lemma}  
Let $(D,n,e,f)$ be a hypermap.  
Consider a family of contour loops
$\cal L$ with the property that all but one contour is
a face of the hypermap.  Let $C$ be the contour loop that
is not a face of the hypermap.  Assume that $C$ satisfies the
assumption of simpicity.
\end{lemma}

Let $(D,n,e,f)$ be a plain hypermap.  Let $C$ be a contour loop
that satisfies the simplicity assumption.  
We let $D_C$ be the set of darts $x$ such that there is an injective
contour path $x=x_0,x_1,\ldots,x_n\in C$ such that
\begin{itemize}
\item $x_i\not\in C$, for $i<n$,
\item $f x_{n-1} = x_n$, if $n\ne 0$.
\end{itemize}
By construction $D_C$ contains the darts on the loop $C$ (take $n=0$).

\begin{lemma} Let $(D,n,e,f)$ be a plain hypermap.  
Assume that $e$ has no fixed points on $D$.
Let $C$ be a contour
loop that satisfies the simplicity assumption.  Then
the sets of darts $D_C$ and $D_{C'}$ are disjoint.  Each consists
of a union of faces.  If $(D,n,e,f)$ has a single group orbit, then
$D_C \cup D_{C'}=D$.
\end{lemma}

\begin{proof}\FIXX{Insert.}
\end{proof}

Let $H=(D,e,n,f)$ be a plain hypermap.  Assume that $e$ has
no fixed points on $D$, that $C$ is a contour loop that satisfies
the simplicity assumption, and that $H$ has a single group orbit.

From the contour loop $C$ we obtain a normal family ${\cal L}$ consisting
of $C$ together with the faces of $D_{C'}$.  We obtain a second
normal family ${\cal L}'$ consisting of $C'$ and the faces of $D_C$.
Form quotients
  $$A= H/{\cal L},\quad B=H/{\cal L}'$$
The quotients have  distinguish faces $F_A,F_B$ corresponding to the
contour loops $C,C'$ respectively.  Moreover, there is a bijection $\phi$
between the darts of $F_A$ and $F_B$ according to the nodes of the hypermap $H$ that the contour loop passes through.

We are now ready to give a precise statement about how connected sums and quotients are inverse operations.

\begin{lemma}
In this context, the bijection $\phi$ satisfies 
   $$\phi(f_A x) = f_B^{-1} \phi(x),\quad x\in F_A.$$
Moreover, $A\#_\phi B$ is isomorphic to $H$.
\end{lemma}

\begin{proof}\FIXX{Insert.}
\end{proof}




%%%%%%%%%%%%%%%%%


\chapter{Fan}

A hypermap is a combinatorial object, but a sphere packing
is a geometric object.  This chapter combines the combinatorics
with the geometry, as objects called fans.  A fan is a geometric
object, yet there is a hypermap attached to it.
The hypermap encodes the combinatorial properties of the fan.

%\begin{definition}[azimuth cycle]  Let $v\in\ring{R}^3$ and $W \subset
%\ring{R}^3$.  We say that $\sigma:W\to W$ is an azimuth cycle on $W$
%coming from $v$, if there is a orthonormal $2$-frame $P=(0,e_1,e_2)$
%with $e_1 \times e_2 = v/|v|$, and a cycle is $\sigma:W\to W$ with
%respect to $P$. (By REF%%, an azimuth cycle is unique, but may not
%exist.)
%\end{definition}

%[ change the following definition, so that edges are triples
%$(v,w,u)$, where $u$ is a unit point orthogonal to $v$ and $w$.
%Assume that if $(v,w,u)\in E$ then $(w,v,-u)\in E$.  We then ask for
%an azimuth cycle on the points $u\times v$ rather than on the
%points $w$.  The third element $u$ allows for the case that the $v$
%and $w$ are antipodal, or the long end of a great circle, which is
%convenient for some of the proofs. In standard situations, we can
%just take $u$ to be the unit length point in the direction $v\times
%w$.  This change ripples through the text.  For instance, the proof
%that  $\#c = \#f$ for linear graphs reduces all the way down to the
%case of a single plane.]

If $e=\{v_1,v_2\}$ is a set of two points and $v$ is any other point,
set
  $$
  \begin{array}{lll}
  C(v,e) &= \op{aff}_+(v,e)\\
  C^0(v,e) &= \op{aff}^0_+(v,e)\\
  \end{array}
  $$
%We drop the $v$ from the notation, when it is clear from context,
%and write $C(e), C^0(e)$.
% Note that the base point is always variable $\orgn$.

\begin{definition}[fan]  Let $(\orgn,V,E)$ be a triple consisting of a point,
a set of
points, and a set of pairs of elements of $V$.  The triple is said to be
a {\it fan\/} if the following conditions hold.
    \begin{itemize}
    \item $V$ is finite and nonempty.
    \item $\orgn\not\in V$.
    %\item Each element of $E$ has two elements.
    \item For each $v\in V$, the set
        $$
        %% WW changed notation from E_v to E(v) to allow deformations E_t
        E(v) = \{w\in V\mid \{v,w\}\in E\}
        $$
        is cyclic with respect to $(\orgn,v)$.
    \item For each $e\in E$, $V\cap C^0(\orgn,e)=\emptyset$.
    \item For sets $e,e'\in E$,   we have
        $$C^0(\orgn,e) \cap C^0(\orgn,e')\ne\emptyset\ \Rightarrow (e = e').$$
    \end{itemize}
Call $C^0(\orgn,e)$ or $C(\orgn,e)$ a {\it blade\/} of the fan.
\index{Index}{blade}
\index{Index}{fan}
\end{definition}

We make a series of remarks about this definition.

\begin{remark}\tlabel{rem:fan}\rating{30}
\begin{itemize}
\item The point $\orgn$ is a base point that will be fixed throughout
the chapter.  
\item The pair $(V,E)$ is a graph with nodes $V$ and edges $E$.  The set
$E(v)$ is the set of edges around a fixed node $v$.
Note that $w\in E(v)$ if and only if $v\in E(w)$.   
%
\item The final condition implies that the sets $C^0(\orgn,e)$
do not meet.   This condition will eventually yield planar
hypermaps.
%
\item
By the condition that $E(v)$ should be cyclic,
for each $v\in V$, we have an azimuth cycle $\sigma(v):E(v)\to E(v)$.
We allow $E(v) = \{w\}$ to be a
singleton set. If so,
$\sigma(v)$ is the identity map on $E(v)$.
%
\item
Sometimes we write $\sigma(v,w)$ for $\sigma(v)(w)\in E(v)$.
%
\item 
The hypothesis of an azimuth cycle
prevents $\{\orgn,v,v'\}$ from being a collinear set, when $\{v,v'\}\in
E$.  In particular, there are no loops: $\{v,v\}\not\in E$.
%
\end{itemize}
\end{remark}



\section{Topology}\label{sec:topology}

Let $(\orgn,V,E)$ be a fan.  We define sets of darts $D_1,D_2,D$:
    $$
    \begin{array}{lll}
    D_1 &= \{(\orgn,v,w,w')\mid v\in V,\ w\in E(v),\ w' = \sigma(v,w)\}\\
    D_2 &= \{(\orgn,v) \mid v\in V,\ \ E(v) = \emptyset\},\\
    D   &= D_1\cup D_2.
    \end{array}
    $$
We call $D_1$ the reduced set of darts and $D$ the extended set of darts.
%
We define a permutation $n$ on $D_1$ by
    $$n(\orgn,v,w,w') = (\orgn,v,w',\sigma(v,w')).$$
We define a permutation $f$ on $D_1$ by
    $$
    f (\orgn,v,w,w') = (\orgn,w,\sigma(w)^{-1} v,v).
    $$
Define a permutation $e$ on $D_1$ by
    $$
    e (\orgn,v,w,w') = (\orgn,w,v,\sigma(w,v)).
    $$
(Note that the symbol $e$ has two meanings according to context, both
as the edge permutation and as an element of $E$.)
Define permutations $e,n,f$ on $D_2$ by making them degenerate on $D_2$:
    $$
    e (\orgn,v) = n(\orgn,v) = f(\orgn,v) = (\orgn,v).
    $$
Write $\op{hyp}_r(\orgn,V,E)=(D_1,e,n,f)$ and
Write $\op{hyp}(\orgn,V,E)=(D,e,n,f)$.  We call them the reduced hypermap
and the (extended) hypermap associated with $(\orgn,V,E)$.  The next
lemma justifies this terminology.



\begin{lemma}\rating{40}
Let $(\orgn,V,E)$ be a fan.  Let $D = D_1\cup D_2$
and $\op{hyp}(\orgn,V,E) = (D,e,n,f)$, as constructed above.  Then
    \begin{itemize}
    \item $\op{hyp}(\orgn,V,E)$ is a plain hypermap.
    \item  $e$ has no fixed
points in $D_1$.
    \item  $f$ has no fixed points on $D_1$.
    \item For every pair of distinct nodes, there is at most one
    edge meeting both.
    \item The two darts of an edge (of $D_1$) lie at different nodes.
    \end{itemize}
\end{lemma}

\begin{proof}  We compute
    $$e(n(f(\orgn,v,w,w'))) = e(n(\orgn,w,\sigma(w)^{-1} v,v))) =
        e(\orgn,w,v,\sigma(w, v)) = (\orgn,v,w,\sigma(w, v)) = (\orgn,v,w,w').$$
So it is a hypermap. We compute
    $$e(e(\orgn,v,w,w')) = e(\orgn,w,v,\sigma(w,v)) = (\orgn,v,w,w').$$
So it is plain. A fixed point in $D_1$ under $e$ would force $v = w\in E(v)$,
but by construction $v\not\in E(v)$.  The argument that $f$ has no
fixed points is similar.

   We show that for every pair of distinct nodes, there is at most one edge
meeting both.
That is,
        $$(n^k e x = e n^\ell x)\Rightarrow (n^\ell x = x).$$
Let $x = (\orgn,v,w,w')\in D_1$.  Let $\sigma=\sigma(v)$. Then
    $$
    \begin{array}{lllllll}
    n^\ell x &= (\orgn,v,\sigma^\ell w,\sigma^{\ell+1}w)\\
    e n^\ell x &= (\orgn,\sigma^\ell w,*,*)\\
    e x &= (\orgn,w,*,*)\\
    n^k e x &= (\orgn,w,*,*)\\
    n^k e x &= e n^\ell x &\ \Rightarrow (w = \sigma^\ell w) &\ \Rightarrow
    (n^\ell x &= (\orgn,v,w,\sigma w) = x)
    \end{array}
    $$

Finally, we show that each dart of an edge lies on a different node.
That is, $e x \ne n^k x$, for $x\in D_1$.  We have
    $$
    \begin{array}{lll}
        e(\orgn,v,w,w') &= (\orgn,w,*,*),\quad w\in E(v)\\
        n^k(\orgn,v,w,w') &= (\orgn,v,*,*),\quad v\not\in E(v).
    \end{array}
    $$
The result follows.
\end{proof}

\subsection{basics}

There is hardly any topology that comes up in this book.  Most of
what is needed appears in this chapter.  We make use of some basic
notions in topology such as continuity, connectedness, and compactness.

%\begin{remark} The term {\it connected} is now being used in
%two different senses: in the topological sense and in a combinatorial
%sense for hypermaps. 
%\end{remark}
%%  12/2008: connected now reserved for topology.  "Single group orbit"
%%  for hypermaps.





The set $\ring{R}^3$ is a metric space under the
Euclidean distance function $d(v,w) = |v-w|$.  Subsets of
$\ring{R}^3$ are a metric space under the restriction of the metric
$d$ to the subset. Subsets carry the metric space topology. 
If $Y$ is an open set in $\ring{R}^3$, write
$\comp{Y}$ for its set of connected components.
If two
points in $\ring{R}^3$ 
can be joined by a continuous path that avoids $X$,
then they lie in the same component of $Y$.
If we produce a family of nonempty connected open sets in
$Y$, whose union is all of $Y$, then
this family is $\comp{Y}$.
%embers of the family are the components of $Y$.
Let $$S^2(\orgn) = \{ v \mid | v-\orgn | = 1\}$$ be the unit sphere in
$\ring{R}^3$, centered at $\orgn$.  






\subsection{component and dart}

Let $(\orgn,V,E)$ be a fan and let $(D,e,n,f) = \op{hyp}(\orgn,V,E)$
be the associated hypermap.  Write $D = D_1\cup D_2$ as a union of
reduced darts $D_1$ and non-reduced darts $D_2$.

\begin{definition}[X,~Y]\label{def:XY}
Let $(\orgn,V,E)$ be a fan.  Let $X=X(\orgn,V,E)$ be the union of the
cones
   $$C(\orgn,e)$$
as $e$ ranges over $E$.  Let $Y=Y(\orgn,V,E)$ be the complement
$Y = \ring{R}^3\setminus X$.
\index{Index}{X}\index{Index}{Y}.
\end{definition}

%% WW Move the following remark elsewhere:
%If $e=\{v,v'\}\in E$, then $\orgn,v,v'$ are not collinear
%(Remark~\ref{rem:fan}), so that $C(\orgn,e)$
%does not lie in a line, and does not contain any
%distinct points $u,u'\in
%C(\orgn,e)\setminus\{\orgn\}$ 
%with $\op{aff}\{u,u',\orgn\}$ collinear. 
%In particular, $C(\orgn,e)$ does not contain a line through $\orgn$.

We associate a wedge $\Wdart(x)$, a subset $\Wdart(x,\epsilon)$,
and an azimuth angle $\op{azim}(x)$
with each dart $x\in D$.  If
$x=(\orgn,v,w,w')\in D_1$, set
$\Wdart(x) = W(\orgn,v,w,w')$ and $\op{azim}(x) =
\op{azim}(\orgn,v,w,w')$.   If $x=(\orgn,v)\in D_2$, then we set
$\Wdart(x) = \ring{R}^3\setminus \op{aff}\{\orgn,v\}$ and $\op{azim}(x) = 2\pi$.  For any $x = (\orgn,v,\ldots)\in D$, set
    $$
    \Wdart(x,\epsilon) = \Wdart(x) \cap \op{rcone}^0(\orgn,v,\cos\epsilon).
    $$


\begin{lemma}\tlabel{lemma:disjoint}\rating{50}
Let $(D,e,n,f)$ be the hypermap attached to a 
fan $(\orgn,V,E)$.
Let $N$ be a node of $D$.  There exists $v\in V$
such that the darts of $N$ are precisely
the darts of the form $(\orgn,v,\ldots)$.  Furthermore, there is a 
disjoint sum decomposition of $\ring{R}^3$ given by
  $$
  \ring{R}^3 = 
  \op{aff}\{\orgn,v\} \cup
  \bigcup_{x\in N} \Wdart(x)  \cup 
  \bigcup_{\{v,w\}\in E} \op{aff}_+^0(\{\orgn,v\},w).
  $$
\end{lemma}

\begin{corollary}\tlabel{cor:W}\rating{10}
Let $x = (\orgn,v,\ldots)$ be a node.
We have $\Wdart(x)\cap C(\orgn,e)=\emptyset$, for $e\in E(v)$.
\end{corollary}

\begin{proof} The decomposition of Lemma~\ref{lemma:disjoint} is
disjoint.  It follows directly from the definitions that
   $$C(\orgn,e)\subset \op{aff}_+^0(\{\orgn,v\},w) \cup 
    \op{aff}\{\orgn,v\}.$$
\end{proof}

\begin{lemma}\rating{80} 
For each $x$, and $\epsilon$ sufficiently small and positive,
$\Wdart(x,\epsilon)$ is nonempty and lies in a single connected
component of $Y(\orgn,V,E)$.
\end{lemma}

\begin{proof}  First we show that $\Wdart(x,\epsilon)$ lies in $Y$,
for $\epsilon$ small.  Let $x=(\orgn,v,w,w')\in D_1$.  
Let $S^2(\orgn)$ be the unit sphere centered at $\orgn$.
By making $\epsilon$ small enough,
the sets $\Wdart(x,\epsilon)\cap S^2(\orgn)$
avoid the compact sets $C(\orgn,e)\cap S^2(\orgn)$ when $v\not\in e$.
Thus, $\Wdart(x,\epsilon)$ also avoids $C(\orgn,e)$ when $v\not\in e$.
By Corollary~\ref{cor:W}, $\Wdart(x,\epsilon)$ avoids $C(\orgn,e)$, when $v\in e$.
Thus, $\Wdart(x,\epsilon)\subset Y$, for $\epsilon$ small.

To complete the proof, it is enough to show that each $\Wdart(x,\epsilon)$ is
connected.  
The  set
   $$
   R=\{(r,\theta,\epsilon') \in (0,\infty) \times (\theta_1,\theta_2) \times (0,\epsilon)\}
   $$
is connected.
The set $\Wdart(x,\epsilon)$  is the image of $R$
under a spherical coordinate representation (Definition~\ref{def:sph}).
It is readily verified that the polar coordinate representation is
a continuous map. As the image of a connected set under a continuous map
is connected, $\Wdart(x,\epsilon)$ is connected.
\end{proof}

\begin{definition}[leads~into] For each dart $x$, 
there is then a well-defined connected
component $U_x$ of $Y(\orgn,V,E)$ 
that contains $\Wdart(x,\epsilon)$ (for all
sufficiently small $\epsilon$). Say the dart {\it leads into}
$U_x$.
\end{definition}


\subsection{component and face}

%% WW: notation e for edge of E and for edge of the hypermap,
% and for the edge permutation on darts.

We soon prove that any two darts $x,y$  in the same
face of the hypermap $\op{hyp}(\orgn,V,E)$ lead
into the same component of $Y(\orgn,V,E)$.  The
strategy of  proof
is to construct a connected set inside $Y(\orgn,V,E)$
that joins $\Wdart(x,\epsilon)$ to $\Wdart(f x,\epsilon)$.  
In this section, we define these connected sets $\Wedge(x,\mu,\epsilon)$
and study their basic properties.

Let $(\orgn,V,E)$ be a fan.   Let
$x=(\orgn,v,w,w')$ be a dart, and  $e = \{v,w\}\in E$.
Let 
  $$C(\orgn,e,\epsilon) =
    \{ c\in C(\orgn,e) \mid c\not\in \op{rcone}(\orgn,v,\epsilon)
    \cup\op{rcone}(\orgn,w,\epsilon)\}.
  $$
Let 
  $$n = (v-\orgn)\times (w-\orgn)$$
and let
  $$
  \Wedge(x,\mu,\epsilon) = \{ x  \in \ring{R}^3 \mid
     x = c + \mu'\, n\, |c-\orgn|,\quad
     c\in C(\orgn,e,\epsilon/2),\quad
     0 < \mu' < \mu
     \}.
  $$

\begin{lemma}\rating{80}  
Let $(\orgn,V,E)$ be a fan.
Let $x=(\orgn,v,w,w')$ be a dart in the associated hypermap.
Pick $\epsilon$ so small that $\Wdart(x,\epsilon)$
and $\Wdart(f x,\epsilon)$ each lie in a single component 
of $Y(\orgn,V,E)$ and
that $\Wdart(x,\epsilon)\cap \Wdart(f x,\epsilon) =\emptyset$.
For $\mu,\epsilon > 0$, the set $\Wedge(x,\mu,\epsilon)$ is
open and connected.
\end{lemma}

\begin{proof}  Let $\orgn + y \in \Wedge(x,\mu,\epsilon)$.
Then $y$ can be written uniquely as
$$
  y = t (c - \orgn) + \mu' t n,\quad
  t > 0,\quad
  0 < \mu' < \mu,\quad
  c \in C(\orgn,e,\epsilon/2) \cap S^2(\orgn).
$$
The parameter $c$ ranges over an open arc of the circle
$$S^2(\orgn) \cap \op{aff}(\orgn,v,w).$$
The parameters $(t,\mu',c)$ give a homeomorphism of $\Wedge(x,\mu,\epsilon)$
with a product of three connected open sets.  The result follows.
\end{proof}


% XXS English to here.

\begin{lemma}\rating{100}
Let $(\orgn,V,E)$ be a fan with dart
$x=(\orgn,v,w,w')$.
For all sufficiently small $\epsilon > 0$, there exists
$\mu=\mu(\epsilon)>0$ such that
the set $\Wedge(x,\mu(\epsilon),\epsilon)$ is
contained in a single component of $Y(\orgn,V,E)$.
\end{lemma}

We write $\Wedge(x,\epsilon) = \Wedge(x,\mu(\epsilon),\epsilon)$.

\begin{proof}  Fix $\epsilon>0$ small enough that $\Wdart(x,\epsilon)$
and $\Wdart(f x,\epsilon)$ are contained in a single component.
By the previous lemma, $W=\Wedge(x,\mu,\epsilon)$ 
is connected.  
%It is enough to
%show that $W\cap S^2(\orgn)$  does not meet $X(\orgn,V,E) \cap S^2(\orgn)$.
By construction, the set $W$ does not meet the blade
$C(\orgn,\{v,w\})$.  If the lemma is false, we can find an
edge $e\ne \{v,w\}$,
and
a sequence $c_i + \mu'_i n \in \Wedge(x,\mu_i,\epsilon) 
 %\cap S^2(\orgn)
\cap C(\orgn,e)$, and $c_i\in C(\orgn,\{v,w\},\epsilon/2) 
 \cap S^2(\orgn)$. 
with $\mu_i \to 0$ and $\mu'_i\to 0$. 
By compactness, a subsequence
of  $\{c_i\}$ converges to an
element $c$ in 
  $$(C(\orgn,\{v,w\})\cap C(\orgn,e)\cap S^2)
  $$
As $(\orgn,V,E)$ is a fan, this
implies that $\{v,w\}$ meets $e$, say $v\in e$.
The elements $c_i\not\in \op{rcone}^0(\orgn,v,\epsilon/2)$, so
in the limit we have $c\not\in\op{rcone}^0(\orgn,v,\epsilon/2)$.
However, the intersection of $C(\orgn,v)$ with $C(\orgn,\{v,w\})$
lies in $\op{rcone}^0(\orgn,v,\epsilon/2)$.  This is a contradiction.
\end{proof}

\begin{lemma}\tlabel{lemma:umu}\rating{100}
Let $(\orgn,V,E)$ be a fan.
Let $x=(\orgn,v,w,w')$ be a dart in the associated hypermap.
For all $\epsilon > 0$,
the set $U=\Wedge(x,\epsilon)$ meets $W=\Wdart(x,\epsilon)$ and 
$W'=\Wdart(f x,\epsilon)$.
Furthermore, the darts $x$ and $f x$ lead into the same
component of $Y(\orgn,V,E)$.
\end{lemma}

\begin{proof}
We will show that $U$ meets $W$.    Let $e=\{v,w\}$. Pick 
 $$c\in C(\orgn,e) \cap 
      (\op{rcone}^0(\orgn,v,\epsilon)\setminus \op{rcone}(\orgn,v,\epsilon/2)).
 $$
Let $n$ be as above. 

 We describe the behavior of 
$u(\mu)=c + \mu n$ for sufficiently small $\mu>0$ 
 The point lies in  $U$.  
We show that $u(\mu)\in \Wdart(x,\epsilon)$.
By construction $u(\mu)\in \op{rcone}^0(\orgn,v,\epsilon)$.
The direction of the normal $n$ gives
   $$
   \op{azim}(\orgn,v,w,c) < \op{azim}(\orgn,v,w,u(\mu)).
   $$
Note that $\op{azim}(\orgn,v,w,c) = 0$, because 
$c\in\op{aff}_+^0(\{\orgn,v\},w)$. 
The function $\op{azim}$ is equal to $\op{dih}_V$ for
small angles (Lemma~\ref{lemma:dih-azim}).  Moreover,
the function $\dih_V$ is continuous (provided
we avoid the set along which $\{\orgn,v,w\}$ or $\{\orgn,v,c\}$ is
collinear), as can be seen
from the explicit formula 
for the function in Lemma~\ref{lemma:dihform}.
Hence, $\op{azim}(\orgn,v,w,u(\mu))$ tends to zero with $\mu$.
Thus, eventually
   $$
   0 < \op{azim}(\orgn,v,w,u(\mu)) < \op{azim}(\orgn,v,w,w').
   $$
This is precisely the condition for $u(\mu)\in \Wdart(x,\epsilon)$.
The argument that $U$ meets
$W'$ is similar.

Each of the sets $U$, $W$, $W'$ lies in a single connected
component of the complement.  Since $W$ and $U$ meet, they
lie in the same component.  Likewise, $U$ and $W'$ lie in the
same component.  By transitivity, $W$ and $W'$ lie in the same
component.
\end{proof}

\begin{lemma}\tlabel{lemma:face-component}\rating{10}
Let $(\orgn,V,E)$ be a fan.
If $x$ and $y$ are darts in the same face of the hypermap
$\op{hyp}(\orgn,V,E)$, then they lead into the same component.
\end{lemma}

\begin{proof}  This is an easy induction, based on the previous
lemma.  
\end{proof}


\begin{lemma}\tlabel{lemma:approach-Ce}\rating{35}
Let $(\orgn,V,E)$ be a fan.  Let $U\in\comp{Y(\orgn,V,E)}$.  
Let $e\in E$ and let $\{x,y\}$
be the two darts forming an edge of $\op{hyp}(\orgn,V,E)$
corresponding to $e$.  Let $\gamma:[0,1]\to U\cup C^0(\orgn,e)$
be a path that begins in $U$ and ends on $C^0(\orgn,e)$.
Then for all $\epsilon>0$ sufficiently small,
there exists  $t\in[0,1]$ such that
  $$\gamma(t)\in 
  %\Wdart(x,\epsilon) \cup 
  %\Wdart(f x,\epsilon)\cup 
  %\Wdart(y,\epsilon)\cup 
  %\Wdart(f y,\epsilon) 
  %\cup 
  \Wedge(x,\epsilon)
  \cup \Wedge(y,\epsilon).
  $$
\end{lemma}

\begin{proof}
Suppose for a contradiction, that path avoids the given union
of sets.  By going to a smaller interval $[0,t']$, $t'\le 1$,
we may assume that $\gamma(t)\in U$ for $t < t'$ and $\gamma(t')\in
C^0(\orgn,e)$.  

Pick $\epsilon>0$ so small that 
  $$\gamma(t')\not\in \op{rcone}^0(\orgn,u,\epsilon),\quad u=v,w.
  $$
%Fix $\epsilon$ small and positive.
%Let $x = (\orgn,v,w,w')$.
%If $\gamma(t')\in \op{rcone}^0(\orgn,v,\epsilon)$, then $\gamma(t)\in
%\op{rcone}^0(\orgn,v,\epsilon)$ for $t$ sufficiently close to $t'$.  Th%e
%azimuth angle $\op{azim}(\orgn,v,w,\gamma(t))$ is small but nonzero, 
%when
%$t$ is close to $t'$.  If it is small and positive, then 
%it follows that $\gamma(t)\in \Wdart(x,\epsilon)$.  If it is negative, %then
%$\gamma(t)\in \Wdart(n^{-1} x,\epsilon)$ and $n^{-1} x = f y$.
%
%The proof is similar, if $\gamma(t')\in \op{rcone}^0(\orgn,w,\epsilon)$.
%Finally, in the remaining case, 
Then $\gamma(t')\in C(\orgn,\{v,w\},\epsilon/2)$ and for $t$ near $t'$,
$$\gamma(t)\in \Wedge(x,\epsilon)\cup \Wedge(y,\epsilon).$$
\end{proof}

\begin{lemma}\tlabel{lemma:Ux}\rating{25}
Let $(\orgn,V,E)$ be a fan with component $U\in\comp{Y(\orgn,V,E)}$.
Then, there exists a
dart $x$ that leads into $U$.
\end{lemma}

\begin{proof}  Pick a curve (say an arc of a
circle centered at $\orgn$) from any $p\in U$ to a point on $X=X(\orgn,V,E)$.
By picking an appropriate circle, 
we may assume that the first point of the curve in $X$ lies
in $C^0(\orgn,e)$ for some edge $e\in E$.  
By Lemma~\ref{lemma:approach-Ce},
the component $U$ contains one of the connected
sets $\Wedge(y,\epsilon),\ldots$.
For each of these connected sets, there is a dart leading
into its component.
\end{proof}

\begin{lemma}\rating{25}
Let $(\orgn,V,E)$ be a fan and $(\orgn,V,E')$
a fan with one fewer edge:
  $$
  E' = E\setminus\{e\}, \quad X(\orgn,V,E) = X(\orgn,V,E')\cup C(\orgn,e).
  $$
The inclusion  $$Y(\orgn,V,E) \subset Y(\orgn,V,E').
   $$
gives a map
from $\comp{Y(\orgn,V,E)}$ to
$\comp{Y(\orgn,V,E')}$.
\end{lemma}

\begin{lemma}\tlabel{lemma:pre-walkup}\rating{20}
Let $(\orgn,V,E)$ be a fan and $(\orgn,V,E')$
the fan with one fewer edge $e\not\in E'$.
Then the double walkup on $e$
carries $\op{hyp}(\orgn,V,E)$ to
$\op{hyp}(\orgn,V,E')$.
\end{lemma}

\begin{lemma}\tlabel{lemma:join-comp}\rating{35}
Let $(\orgn,V,E)$ be a fan and $(\orgn,V,E')$
a fan with one fewer edge $e=\{x,y\}$.
Let $U,U'$ be different
components of $Y(\orgn,V,E)$ with the same
image in $\comp{Y(\orgn,V,E')}$.  Then 
$x$ and $y$ lead into $U$ and $U'$, respectively, suitably ordered.
\end{lemma}

\begin{proof}  Let $Y'=Y(\orgn,V,E')$.
As $U$ and $U'$ have the same image in $\comp{Y'}$,
there is a path in $U\cup U'\cup C^0(\orgn,e)$, starting in $U$
and ending in $U'$.  As $U$ and $U'$ are different components,
the path meets $C^0(\orgn,e)$.
By Lemma~\ref{lemma:approach-Ce}, the
initial part of the path in $U\cup C^0(\orgn,e)$ up to the first
intersection with $C^0(\orgn,e)$ meets $\Wedge(x)\cup \Wedge(y)$, where
  $
  \Wedge(z) = 
    %\Wdart(z,\epsilon) \cup \Wdart(f z,\epsilon) 
    %\cup 
    \Wedge(z,\epsilon).
  $
(We assume that $\epsilon$ is positive and sufficiently
small, then drop it from the notation.)
Similarly, the terminal part of the path in $U'\cup C^0(\orgn,e)$, 
from the
final intersection with $C^0(\orgn,e)$ on, meets $\Wedge(x)\cup \Wedge(y)$.
Recall that $\Wedge(x)$ and $\Wedge(y)$ are connected.  Thus,
if the initial and terminal segments of the path were to
meet the
same set (say $\Wedge(x)$), then $U$ and $U'$ would be equal components.
Thus, $U$ contains (say) $\Wedge(x)$ and $U'$ contains $\Wedge(y)$.
%The result now follows from the definition of what it means
%to lead into a component.
By Lemma~\ref{lemma:umu}, $\Wedge(x,\epsilon)$ meets $\Wdart(x,\epsilon)$,
so $x$ leads into $U$.  Similarly, $y$ leads into $U'$.
\end{proof}


\begin{lemma}\tlabel{lemma:U-bij}\rating{30}
Let $(\orgn,V,E)$ be a fan and $(\orgn,V,E')$
a fan with one fewer edge.  Let $\{x,y\}$ be the edge of the hypermap
that corresponds to the edge of $E\setminus E'$.
Let $Y=Y(\orgn,V,E)$ and $Y'=Y(\orgn,V,E')$.
Then
the map  $\comp{Y}\to \comp{Y'}$ is onto.
Furthermore, if the the double walkup
on $\{x,y\}$ splits,
%(see Lemma~\ref{lemma:pre-walkup} 
%and Definition~\ref{def:merge-split}).  
then $\comp{Y}\to\comp{Y'}$ is bijective.
\end{lemma}

\begin{proof}
First we show that the map is onto.  Let $U'$
be a component of $Y'$.  If $p\in U'$,
then a sufficiently small neighborhood of $p$ is contained
in $Y'$ and contains a point $q\not\in C(\orgn,\{x,y\})$.  The
point $q$ belongs to some component $U$ of $Y$.  The image
of $U$ is $U'$.

We show that the map is injective.  If not, two
components $U_1,U_2$ map to the same component of $Y'$.
By Lemma~\ref{lemma:join-comp}, we may assume
that $x$ leads into $U_1$ and
$y$ leads into $U_2$.
However, when splitting, $x$ and $y$ are in the same face of
the hypermap and hence lead into the
same component.  Thus, $U_1=U_2$.
\end{proof}

%\begin{lemma}\tlabel{lemma:split-component}
%Let $(\orgn,V,E)$ be a fan.  Suppose that
%deleting an edge of the graph is given combinatorially as a splitting
%double face walkup.  Then the number of components of
%$Y(\orgn,V,E)$ is preserved by the edge deletion.  All
%components are left as before, except one that differs only by the
%presence of the edge.
%\end{lemma}
%
%\begin{proof}
%The components other those along the edge are unaffected. Suppose
%that there are two components $U,U'$ in $H'$ that become
%connected after the edge deletion.  If sets $\Wedge(x,\epsilon)$ and
%$\Wedge({e x},\epsilon)$  were both to belong to the
%same component $U$, then $U\cup\{\text{the open-ended edge}\}$ WW
%and $U$ are disjoint open after the deletion, contrary to the
%assumption they become connected.  So $\Wedge(x,\epsilon) \subset U$ and
%$\Wedge({e x},\epsilon) \subset U'$.  Thus, $x$ and $e x$ lie on
%different faces of $H'$ and this implies that the double walkup
%merges, contradicting the hypothesis that it splits. Thus, the
%number of components remains constant.
%\end{proof}
%

\begin{lemma}\tlabel{lemma:U-merge}\rating{35}
Let $(\orgn,V,E)$ be a fan and $(\orgn,V,E')$
a fan with one fewer edge.  Let  $\{x,y\}$ be the set of darts
for that  edge.
Suppose that $x$ and $y$ lead into different components $U_x,U_y$
of $Y(\orgn,V,E)$.
Then $U_x$ and $U_y$ map to the same component of $Y'=Y(\orgn,V,E')$.
\end{lemma}

\begin{proof}
Let $f,n$ be the face and node maps on the hypermap.
If $y$ leads into $U_y$, then so does
$f y = n^{-1} x$.  Let $z$ be the dart in
$(D',\cdot,n',\cdot)=\op{hyp}(\orgn,V,E')$ that is the image of the dart $n^{-1}x$
in $H$. We see that
$\Wdart(x,\epsilon)$ and $\Wdart(n^{-1} x,\epsilon)$ combine into a single
region $\Wdart({n'}^{-1}z,\epsilon)$.  Thus, $U_x$ and $U_y$ map to the
same component of $Y'$.
\end{proof}

\begin{lemma}\tlabel{lemma:orbit-connect}\rating{35}
Let $(\orgn,V,E)$ be a fan with hypermap 
$(D,e,n,f)$.
Fix a union $A$ of group orbits.
Let $X_1$ be
the union of blades $C(\orgn,{\{v,w\}})$ with a dart 
$x =
(\orgn,v,w,w')$ in $A$.  Similarly, 
let $X_2$ be the union of blades with darts
not in $A$.  
Then $$X_1\cap X_2 = \{\orgn\}.$$
\end{lemma}

\begin{proof} 
If the intersection  is not $\{\orgn\}$, then by
the definition
of a fan,
there are blades of the form
  $$
  \begin{array}{lll}
   C_1 &= C(\orgn,\{u,v\})\subset X_1,\\
   C_2 &= C(\orgn,\{v,w\})\subset X_2.\\
  \end{array}
  $$
By the definition of fans, $u$ and $w$ are
in the same orbit of the azimuth cycle $\sigma(v)$ on $E(v)$.
The azimuth cycle is used to define the node map.
Hence, the darts 
   $$
   (\orgn,v,u,\sigma(v,u)) \text{ and }
   (\orgn,v,w,\sigma(v,w))
   $$
are connected by a contour path of $n^{-1}$-steps.
This gives a contour path from a dart in $A$ to a dart that
is not in $A$.
This is contrary to the assumption that these darts lie in
separate group orbits.
\end{proof}



%%% EXIST DART
%%%
%\begin{lemma}\tlabel{lemma:dart-lead}
%For each component of 
%$Y$, there exists a
%dart $x$ that leads into that component.
%\end{lemma}
%
%\begin{proof} This follows from the claim
%of the following lemma (Lemma~\ref{lemma:dart-curve}), 
%which constructs
%a continuous curve from any point in 
%$Y$ to a wedge
%$\Wdart(x,\epsilon)$.
%\end{proof}
%
%\begin{lemma}\tlabel{lemma:dart-curve}
%Let $(\orgn,V,E)$ be a fan.  Assume that $E\ne\emptyset$.
%%Let $X=X(\orgn,V,E)$.
%Let $v\in Y$.  
%Then there exists a nondegenerate dart 
%$x=(\orgn,v_1,w_1,w_2)$ of $\op{hyp}(\orgn,V,E)$ with the
%following properties:
%\begin{itemize}
%\item Let $F$ be the orthonormal frame with respect to 
%  $(\orgn,v_1,w_1)$ and let $P(F,r,\theta,\epsilon)$ be the spherical
%  coordinate representation with respect to this frame
%  (Definition~\ref{def:sph}).
%\item We have
%   $$v = P(F,r_v,\theta_v,\epsilon_v).$$
%\item 
%   For all $0 < \epsilon \le \epsilon_v$, we have
%   $P(F,r_v,\theta_v,\epsilon)$ does not meet any $C(\orgn,e)$, 
%   for $e\in E$.
%\item For all $\epsilon>0$ sufficiently small,
%   $P(F,r_v,\theta_v,\epsilon)\in \Wdart(x)$.
%\end{itemize}
%\end{lemma}
%
%\begin{proof} 
%If $v$ lies in some $\Wdart(x,\epsilon)$, the conclusion follows
%directly from the definition of $\Wdart(x,\epsilon)$.  We may
%assume that $v$ does not lie in any of the sets $\Wdart(x,\epsilon)$.
%
%Now assume that no  dart $x$ satisfying the conditions
%of the lemma exists.  
%Consider the circle
%   $$S^1 = \{u\mid |u-\orgn|=1,\quad
%     (u-\orgn)\cdot (v-\orgn) = 0\}.$$
%The set $S^1$ is a metric space in the induced metric
%from $\ring{R}^3$ and a connected topological space in the
%metric space topology.
%Let $A\subset S^1$ be defined as
%  $$
%  A = \{u \in S^1 \mid  \op{aff}\{\orgn,v,u\} \cap X(\orgn,V,E)\ne\emptyset\}.
%  $$
%
%We claim that $A$ is not empty.  Let $u'\in X(\orgn,V,E)$ be any
%element such that $u'\not\in \op{aff}\{\orgn,v\}$.  Then
%$\op{aff}\{\orgn,v,u'\} = \op{aff}\{\orgn,v,u\}$ for some $u\in S^1$.
%This element $u$ belongs to $A$.
%
%The set $X(\orgn,V,E)$ is closed, and this implies that $A\subset S^1$
%is closed.  
%
%We define a function $f:A\to E$ as follows.  Take $u\in A$.
%Write
%$u'\in\op{aff}(\orgn,v,u)$ in polar coordinates
%$(r,\theta)$ with respect to $(\orgn,v,u)$ (Definition~\ref{def:polar}).
%Let $\theta$ be the smallest polar coordinate such
%that the point $u'$ with polar coordinates $(r,\theta)$ lies
%in $X(\orgn,V,E)$.  (By a compactness argument
%on $X(\orgn,V,E)\cap S^2(\orgn)$, $\theta$ exists.)  By the
%assumption that the desired dart $x$ does not exist, we
%find that $u'$ lies in a unique $C(\orgn,e)$.  Define $f(u)=e$.
%
%In fact, the element $u'$  lies in $C^0(\orgn,e)$.  Under small
%perturbations of $u\in S^1$, the corresponding $u'$ also
%lies in $C^0(\orgn,e)$.  Thus, the preimage $f^{-1}(e)$ is an
%open set of $S^1$, for each $e\in E$.  The sets $S^1\setminus A$,
%and $f^{-1}(e)$ give a finite cover of $S^1$ by disjoint open
%sets.  By the connectedness of $S^1$, we have $f^{-1}(e)=S^1$
%for some $e\in E$.
%
%However, by Lemma~\ref{tarski:miss-plane}, there is a 
%plane $\op{aff}\{\orgn,v,u\}$ that does not meet $C(\orgn,e)$.
%This contradiction gives the desired conclusion.
%\end{proof}
%
%%::DONE:: Why not replace this with a result that takes any
%%plane that meets $X()$.  It isn't necessary to close in on
%%a corner.  The only use of generalized polar coords was this.
%

\section{Deformation}

In this section, we consider various deformations $(\orgn,V_t,E_t)$,
for $t\in I\subset \ring{R}$, of
a fan $(\orgn,V_0,E_0)$.  
We assume that $t=0\in I$ gives the initial point
of the deformation.  
In general, these deformations
are obtained by choosing replacing each vertex $w_0\in V$
with a path $w:I\to \ring{R}^3$, with value $w_t$ at $t$.

\begin{definition}[fan~deformation]
If every configuration $(\orgn,V_t,E_t)$ of the 
deformation is a fan,
if $v$ is a continuous function of $t$ for all $v_0\in V$,
if $v_0\mapsto v_t$is a bijection of $V_0$ with $V_t$, 
and if
 every deformation
preserves the edge set:
   \begin{equation}\tlabel{eqn:edge}
   E_t = \{ \{v_t,w_t\} \mid \{v_0,w_0\}\in E_0\}.
   \end{equation}
then call it a
fan deformation.
\end{definition}

\begin{lemma}\tlabel{lemma:deform-invariant}
Let $(\orgn,V_t,E_t)$ (for $0\le t\le t_{\text{max}}$) 
be a fan deformation. Then we have the following invariants
of the deformation:
\begin{itemize}
\item \rating{30} 
The bijection $V_0\to V_t$ of vertex sets
given by $v_0\mapsto v_t$ 
is equivariant for
azimuth cycles on $E_0(v_0)$ and $E_t(v_t)$.
\item \rating{30} 
There is a bijection between the darts of $\op{hyper}(\orgn,V_0,E_0)$
and those of $\op{hyper}(\orgn,V_t,E_t)$ given by
   $$(\orgn,v_0,\ldots) \mapsto (\orgn,v_t,\ldots).$$
\item \rating{30} 
The hypermaps $\op{hyper}(\orgn,V_0,E_0)$ and $\op{hyper}(\orgn,V_t,E_t)$
are isomorphic under this bijection.
\item \rating{400} 
There is a bijection $U_0\mapsto U_t$ 
between $\comp{Y(\orgn,V_0,E_0)}$
and $\comp{Y(\orgn,V_t,E_t)}$ such that whenever a
dart $x_0$ leads into
$U_0$, the dart $x_t$ leads into
$U_t$.
\end{itemize}
\end{lemma}

\begin{proof} By the Identity~\ref{eqn:edge}, the deformation is
a bijection $V_0\to V_t$ and preserves $E_t$.  There
is thus a natural bijection between $E(v_0)$ and $E_t(v_t)$.
This bijection is compatible with the azimuth cycles on both
sets.  In fact, for the azimuth cycle to change, there must
be some earlier time $0\le\tau \le t$ when two azimuth angles coincide:
$$\op{azim}(\orgn,v_\tau,w_\tau,u_\tau) = \op{azim}(\orgn,v_\tau,w_\tau,u'_\tau)$$
with $w_\tau,u_\tau,u'_\tau\in E_\tau(v_\tau)$.  However, this
implies that the intersection of two blades 
   $$
   C(\orgn,\{v_\tau,u_\tau\}) \text{ and } C(\orgn,\{v_\tau,u'_\tau\})
   $$
is larger than what is allowed by the definition of fan.  Thus,
the bijection preserves the azimuth cycle.

It is clear that there is a bijection of darts of the two
hypermaps.   The edge, node, and face maps on darts are defined
in terms of the azimuth cycle.  As the azimuth cycle is
preserved, the two hypermaps are easily seen to be isomorphic.

We pick the parameter $\epsilon$ as follows.
By a compactness argument, we can find
$\epsilon>0$ that gives
   $$
   \begin{array}{rll}
   \Wdart(x_t,\epsilon)&\subset Y(\orgn,V_t,E_t)\\
   \Wedge(x_t,\epsilon)&\subset Y(\orgn,V_t,E_t)
   \end{array}
   $$
for all darts $x_0$ and all subsequent times $0\le t\le t_{\text{max}}$.

We define a relation $R_t$ consisting of all ordered pairs $(U_0,U_t)$
such that there exists a dart $x_0$ of $(\orgn,V_0,E_0)$ and 
corresponding dart $x_t$ of $(\orgn,V_t,E_t)$ such that $x_0$
and $x_t$ lead into $U_0$ and $U_t$, respectively.  If we show
that this relation is a bijective function, then the final claim
of the lemma follows.

To show that the relation $R_t$ is a function, we must check that
if $x_0$ and $x'_0$ both lead into $U_0$, then $x_t$ and $x'_t$
both lead into the same component $U'_t$.  If $x_0$ and $x_0'$
lie in the same face, then $x_t$ and $x'_t$ lie in the same face.
In this case, Lemma~\ref{lemma:face-component} 
implies that both darts, $x_t$
and $x'_t$ lead into the same component. 
Hence, we assume that $x_0$ and $x'_0$ lie in different faces, but lead
into the same component.  

There is a smallest time
$s>0$ for which $\varphi(s)$ fails, where 
 $\varphi(t)$ is the statement that $x_t$ and $x'_t$ lead
into the same component of $Y(\orgn,V_t,E_t)$.
To prove this, note that 
if $x_t$ and $x'_t$ lead into the same component, there is
a path $\gamma$ in $Y(\orgn,V_t,E_t)$
from $\Wdart(x_t,\epsilon)$ to $\Wdart(x'_t,\epsilon)$.  This path avoids 
$X(\orgn,V_{\tau},E_{\tau})$ and connects $\Wdart(x_{\tau},\epsilon)$ to
$\Wdart(x'_{\tau},\epsilon)$ for all $\tau$ sufficiently close to $t$.
Thus, the set of times $t$ for which $\varphi(t)$ holds is an
open set containing $t=0$.  Thus, a smallest time $s$ exists.

The next paragraph produces a contradiction
by showing that $\varphi(s)$ holds.  It does this by
constructing a curve that connects $\Wdart(x_s,\epsilon)$ to $\Wdart(x'_s,\epsilon)$.
It constructs the curve by perturbing the curve that
exists at time $t<s$, for $t$ sufficient close to $s$.  

To construct
a curve at time $s$,  the main point is to construct
a tubular neighborhood of $X(\orgn,V_s,E_s)$.
Let $W_t(F,\rho)$ be the union of the sets
   $$\Wdart(y_t,\rho\epsilon) \text{ and } \Wedge(y_t,\rho\epsilon)$$
as $y_t$ runs over the darts in a face $F$, considered as a face
of $\op{hyp}(\orgn,V_t,E_t)$.
As $t$ approaches $s$ from below, the points $v_t\in V_t$
approach $v_s\in V_s$.  It follows that we can find
some $t<s$ sufficiently close to $s$ has the
following properties.
\begin{itemize}
%% WW This will take some work to prove formally.
 \item For each face $F$, the set of boundary points of $W_t(F,0.1)$ that
   also lie in $Y(\orgn,V_t,E_t)$ is a subset of $W_s(F,1)$.
 \item  $X(\orgn,V_s,E_s)$ lies in the union of $X(\orgn,V_t,E_t)$ with
  the
  sets $W_t(F,0.1)$ (as $F$ runs over all faces of the hypermap).
 \item $\exists p_0 \in (W(x_t,1)\cap W(x_s,1)) \setminus W(x_t,0.1)$.
 \item $\exists p'_0 \in (W(x'_t,1)\cap W(x'_s,1)) \setminus W(x'_t,0.1)$.
\end{itemize}
(Here $0.1$ does not have any special properties, except for being a small number.)
Let $\gamma$ be a path in $Y(\orgn,V_t,E_t)$ from $p_0$ to $p'_0$.
By using these  properties, 
we can break the path into finitely many
segements $[a_i,a_{i+1}]$ such that for each $i$ we have one
of the following two conditions
\begin{itemize} 
\item
$\gamma|_{(a_i,a_{i+1})}$ does not meet $X(\orgn,V_s,E_s)$.
\item There is a face $F$ such that
$\gamma(a_i), \gamma(a_{i+1})\in W_s(F,1)$.
\end{itemize}

To see this, for a given face $F$, we can choose the time $a$ at
which $\gamma$ first enters $W_t(F,0.1)$ and $a'$ when $\gamma$ last
exits $W_t(F,0.1)$.  Use $[a,a']$ as a segment.  Repeat over the
remaining faces.  What remains is a set of segments of the first type.
We can arrange the initial segment starting at $p_0$ and the final
segment up to $p'_0$ to have the first type.

Along each interval of the first type, the path takes values in
$Y(\orgn,V_s,E_s)$.  Along each interval of the second type,
we may use that $W_s(F,1)$ is connected to modify
the curve so that it lies in $Y(\orgn,V_s,E_s)$ and takes
the same values at the endpoints.  Modifying the curve in this way,
we obtain a path in $Y(\orgn,V_s,E_s)$
from $p_0\in\Wdart(x_s,\epsilon)$ to $p'_0\in\Wdart(x'_s,\epsilon)$.  This is the desired
contradiction.  This completes
the proof that the relation $R_t$ is a function.

The function $R_t$ is onto, because every component of $Y(\orgn,V_t,E_t)$
has a dart $x_t$ that leads into it (Lemma~\ref{lemma:Ux}).  The function is injective,
if the opposite relation $R_t^{op}$ consisting of all ordered pairs
$(U_t,U_0)$ is a function.  This is established by the same argument
that shows that $R_t$ is a function.  The lemma follows. 
\end{proof}

%

\section{Path} % Visibility % DCG Sec. 12.7

%%

%% Moved from the end of [DCG SEC 12.7]
\begin{definition}[visible]
Let $(\orgn,V,E)$ be a fan with component $U\in \comp{Y(\orgn,V,E)}$, 
and let $v,v'\in \ring{R}^3$.  We say
that $v$ is visible from $v'$ in $U$ 
if $\{v,v',\orgn\}$ is not
collinear,  and
  $$\op{aff}_+^0(\orgn,\{v,v'\})\subset U.$$
The visibility relation is clearly symmetric in $v$ and $v'$.
\end{definition}



\begin{lemma}\rating{60} 
Let $(\orgn,V,E)$ be a fan and let $v,v'\in V$.
Suppose that $v$ is visible from $v'$ in some component $U$.  
Then
$(\orgn,V,E\cup\{v,v'\})$ is a fan.
\index{Index}{visible}
\end{lemma}



\begin{lemma}\rating{50}  
Let $(\orgn,V_0,E_0)$ be a fan with cyclic hypermap.  Let $(\orgn,V_t,E_t)$
for $t\in I = [0,t']$ be a deformation of $(\orgn,V_0,E_0)$. 
%Suppose that $|v_t-v'_t|\ge 2$ for all $t$ and all $v_t,v_t'\in V_t$.  
%Suppose for two vertices $v_0,v'_0\in V_0$, such that $v'_0$
%is not visible from $v_0$ (in the component of $F$).
%Suppose that for all $t$, and $|v_t-v'_t|< L$.
%Add hypothesis to make result true.
Let $v_0,v_0'\in V_0$.  Suppose for all $t\in I$, and all $w_0\in V_0$,
$w_t\not\in \op{aff}_+^0(\orgn,\{v_t,v_t'\})$.
Then for all $t\in I$, $v'_t$ is not visible from $v_t$.
\end{lemma}

\begin{proof} For a contradiction, assume there exists some $t\in I$,
such that $v'_t$ is visible from $v_t$.  
Fix $s>0$, the infimum of $t$ for which $v'_t$ is visible
from $v_t$.   Then there is $w_s\in V_s$ such that
$w_s\in \op{aff}_+^0(\orgn,\{v_s,v_s'\})$.  This contradicts the hypothesis.
\end{proof}



\begin{definition}[convex,~concave]\label{def:concave}
A dart of of a hypermap of a fan is {\it convex\/}
if its azimuth angle is less than $\pi$, and otherwise that it
is {\it concave}.  
(A weakly convex dart has azimuth angle at most $\pi$.)
\index{Index}{convex}\index{Index}{concave}\index{Index}{weakly convex}
%
\end{definition}


%\subsection{Preservation of Minimal Distance}
%\subsection{Proof that Distances Remain at least $2$} %DCG 12.13, p140
    \label{sec:proof-2}

\begin{definition}[$S_U$]
Let $(\orgn,V,E)$ be a fan, and let $U\in [Y(\orgn,V,E)]$.  Let $D_U$ be the set of darts leading into the hypermap, and $E_{U}\subset D_U$ the subset of reduced darts.
Let $*$ be a symbol. Form the disjoint union
 $$
 S_U =  \{*\} \cup E_U \cup D_U.
$$
As the union is disjoint, we name the inclusion $\iota:D_U\to E_U$ to keep the sets separate.
\end{definition}



\begin{definition}[structured~point] A structured point $\phi$ of $U$ is a pair $(v,s)$,
where $v\in\ring{R}^3$ and $s\in S_U$ such that
\begin{enumerate}
\item If $s = \{*\}$, then $v \in U$.
\item If $s = (\orgn,w,\ldots)\in D_U$, then $v\in\op{aff}_+^0(\orgn,\{w\})$.
\item If $s=(\orgn,v,w,w')\in E_U$, then $v\in\op{aff}_+^0(\orgn,\{u,w\})$.
\end{enumerate}
\end{definition}

\begin{remark} The purpose of a structured point is to deal with point in the boundary of $U$.  The structural information  tracks how a boundary point ``degenerates'' from points of $U$.  No such degeneracy information is needed when $v\in U$.  Hence in this case, we assign it the dummy symbol $*$.  When $s\in E_U$, the degneration lands in an edge; when $s\in D_U$, the degeneration lands over a dart.
\end{remark}

\begin{lemma}\rating{30}
Let $(\orgn,V,E)$ be a fan and let $U$ be a component of $Y(\orgn,V,E)$.
The following relation $\prec$ makes $S_U$
a partially ordered set.  (Here $f$ is the face map on the hypermap.)
$$
x \prec y \Leftrightarrow   
\begin{cases}
   x \in D_U\cup E_U & y = *,\quad {\text{or}}\\
   x = \iota y, \iota (f y) & y \in E_U
\end{cases}
$$
\end{lemma}

Write $x \preceq y$ for $x \prec y$ or $x = y$.
If $\phi(v,s)$ is a structured point, we write $v(\phi)$ and $s(\phi)$ for the first and second components of $\phi$.  

If $s(\phi)\in D_U$, we say that $\phi$ is a vertex point.  If $\phi$ is a structured point and $\epsilon > 0$, we write $U(\phi,\epsilon)$ for the following set
$$
\begin{cases}
  \op{rcone}^0(\orgn,v(\phi),\epsilon)&s(\phi) = *,   \\
  \Wdart(x,\epsilon)& s(\phi) = x\in D_U,   \\
  \Wedge(x,\epsilon) &s(\phi) = x\in E_U.   \\
\end{cases}
$$

\begin{definition}[link]\label{def:link}  Two structured points $(v,s)$
and $(v',s')$ are linked if
\begin{itemize}
 \item $\{\orgn,v,v'\}$ is not collinear,
 \item $\exists s''\in S_U.\ \ {\text{ s.t.}}$
   \begin{itemize}
     \item $s \preceq s'',\quad s'\preceq s''$,
      \item for all $v''\in\op{aff}_+^0(\orgn,\{v,v'\})$,
           $(v'',s'')$ is a structured point.
   \end{itemize}
 \item For every $\epsilon > 0$, there exists $\delta>0$ such that
   $$
   \op{aff}_+^0(\orgn,\{v,v'\}) \cap \op{rcone}(v,\delta) \subset
     U((v,s),\epsilon)
   $$
   (together with the symmetrical statement with $(v,s)$ and $(v',s')$ interchanged).
\end{itemize}
\end{definition}

\begin{definition}[path] A sequence $\phi_0,\phi_1,\ldots,\phi_r$ is a path in $U$ from $\phi_0$ to $\phi_r$, if % (setting $v_i = v(\phi_i)$),
%\begin{itemize}
%\item $v_i\ne \orgn$,
%\item $v_{i+1}\not\in \op{aff}_+^0(\orgn,\{v_i\})$,
%\item $V\cap \op{aff}_+^0(\orgn,\{v(\phi_i),v(\phi_{i+1})\}) = \emptyset,$
%\item For every sufficiently small $\epsilon>0$, there exists $v_i'\in U(\phi_i,\epsilon)$ such that
%   {
%   \begin{itemize}
%     \item $\{\orgn,v_i',v_i'\}$ is not collinear,
%     \item $\op{aff}_+(\orgn,\{v_i',v_{i+1}'\})\subset U$.
%   \end{itemize}
%   }
%\end{itemize}
$\phi_i$ is linked with $\phi_{i+1}$ for all $i<r$.
\end{definition}

A path $\phi_0,\ldots,\phi_r$ is a vertex path if $s(\phi_i)\in D_U$,
for $0<i<r$.  

%\begin{remark}\label{rem:structure}
%Given a path $\phi_0,\ldots,\phi_r$, we can structure
%each point $w\in\op{aff}_+(\orgn,\{v(\phi_i),v(\phi_{i+1})\})$ as follows.  If $w\in U$, use $*$.  Elseif $w\in \op{aff}_+^0(\orgn,\{ v(\phi_j)\})$, use
%$s(\phi_j)$.  In the remaining cases, the structure of $w$ is an edge, which we determine as follows.  If $s(\phi_i)=s(\phi_{i+1})$, use $s(\phi_i)$.  
%If $\{s(\phi_i),s(\phi_{i+1})\}\cap E_U = \{s_0\}$, use $s_0$.  In the remaining case, $\{s(\phi_i),s(\phi_{i+1})\} = \{x,e x\}$, with $x\in D_U$.  Use $x\in E_U$ (considered in $E_U$ rather than $D_U$).  Although these rules are rather involved at the logical level, they express a geometrically obvious condition that the structure of $w$ should be the natural geometric extension of the structure of its endpoints $\phi_i$.  For example, if $\phi_i$ and $\phi_{i+1}$ lie on the same blade, then an intermediate point $w$ should be given the structure of that plade (expressed as an element of $E_U$).
%\end{remark}
%
%%\begin{remark}
%%\label{flexremark} In Section~\ref{remark:proof-2}, to allow for
%%more flexible deformations, we drop all constraints on the lengths
%%of (undistinguished) edges $\{v_1,v_2\}$ that cross the boundary of
%%the substandard component.  We deform in such a way that the edges $\{v_1,v_2\}$
%%will maintain a length of at least $2$.
%%\end{remark}

\begin{definition}[path~length,~step~length]
The {\it path-length} of a vertex-path is the sum of the lengths
$$
\sum_{i=0}^{r-1}\arc(\orgn,v(\phi_i),v(\phi_{i+1}))
$$
The {\it step length} is $r$.
\end{definition}

%For a fixed $(\orgn,V,E)$, there is a constant $C$ such that
%each term $\arc(\orgn,v_i,v_{i+1})$ has length greater than $C$.  It
%follows that the path-length tends to infinity as the length of
%the vertex-path tends to infinity.  It also follows that there are only
%finitely many vertex-paths that are contenders for the shortest path-length,
%and that some path attains the smallestt path-length among all vertex-paths.

%More generally, we consider paths
%$v=v_0,v_1,v_2,\ldots,v_r=v'$ such that 
%\begin{itemize}
%\item for each $i$, there exists $w_i$, $w_{i+1}$ with $\{w_i,w_{i+1}\}\in E$
%such that $C^0(v_i,v_{i+1})\subset C^0(w_i,w_{i+1})$, or
%\item $C^0(v_i,v_{i+1})\subset U$.
%\end{itemize}
%Again, we measure the path-length as the sums of the arcs.

\begin{definition}[path~angle]  Let $\phi_0,\ldots,\phi_r$ be a path (in
some component $U$ of a fan).  We define the path angle $\alpha_i$
for $0 < i < r$ as follows.  If $s(\phi_i)\in \{*\}\cup E_U$, then
we define $\alpha_i$ to be the dihedral angle 
   $$\dih_V(\{\orgn,v(\phi_i)\},\{v(\phi_{i-1}),v(\phi_{i+1})\})$$
If $s(\phi_i)=(\orgn,v,u,w)\in D_U$, order $\{w_1,w_2\} = \{v(\phi_{i-1},v(\phi_{i+1})\}$ with respect to the azimuth cycle so that
  $$\op{azim}(x) = \op{azim}(\orgn,v,u,w_1)+\op{azim}(\orgn,v,w_1,w_2)+
    \op{azim}(\orgn,v,w_2,w).$$
Then set $\alpha_i = \op{azim}(\orgn,v,w_1,w_2)$.  (In particular,
if $\alpha_i >\pi$, then $s(\phi_i)\in D_U$ is a concave dart.)
\end{definition}

\begin{lemma}\tlabel{lemma:path-case}\rating{400}
Let $(\orgn,V,E)$ be a fan.  Let $U$ be a component of $Y(\orgn,V,E)$.
Let $\phi,\phi',\phi''$ be 
a path in $U$.
Let $v,v',v''$ be the corresponding vertices in $V$.
Assume that $\alpha(\phi')\le\pi$.
Set $w(t) = (1-t) v' + t v''$.  One of the following holds.
\begin{enumerate}
 \item $\phi,\phi''$ is a path in $U$.
 \item There exists $0 < t \le 1$, $r>1$, and a path $\phi_0,\ldots,\phi_r$ in $U$ such that
   \begin{itemize}
    \item $v(\phi_0) = u$, $v(\phi_r) = w(t)$;
    \item There is a closed half plane with bounding line through $\orgn$ that contains $v(\phi_i)$;
     \item The path length is equal to the arclength from $v$ to $w(t)$;
     \item $s(\phi_i)$ is a concave dart in $D_U$ for $0 < i < r$;
     \item The path angle $\alpha(\phi_i) \ge \pi$, for $0 < i < r$;
   \end{itemize}
\end{enumerate}
\end{lemma}

\begin{proof} We assume that $\phi,\phi''$ is not a path in $U$.
To avoid trivialities,
we also assume that $(\phi_0,\phi_1,\phi_2)=(\phi,\phi',\phi'')$
does not satisfy the conclusions of the theorem already.

we can structure the point $\phi(t)=(w(t),s(t))$ as in Definition~\ref{def:link}.  For small $t\ge 0$, we consider the path
$\phi,\phi(t),\phi''$.  This is indeed a path for small $t$.

As we increase $t$, three things can go wrong.  (Assume that
we stop at the smallest $t$ for which one of the following occurs.)
(1)  $\op{aff}_+^0(\orgn,\{v',w(t)\})\cap V\ne\emptyset$.
We consider the modified path
 $$
   \phi,
   \phi_1,\phi_2,\ldots,\phi_k,\phi(t)
   $$
where the inserted terms $\phi_j$ are for points 
 $\op{aff}_+^0(\orgn,\{v_i,u_t\})\cap V$. The structure of each
inserted term is a concave dart $D_U$.
We have
   $$v(\phi_i)\in \op{aff}_+^0(\orgn,\{v',w(t)\}).$$
In particular, the points lie in a common half plane.  The other
properties are readily verified.

(2) $\{\orgn,\phi_i(t),\phi_{i-1}\}$ is collinear.  In this case,
there is a family of halfplanes passing through this line.  We can
vary this halfplane $A^0$ until $A^0\cap V\ne\emptyset$.  
We then repeat the construction of (1), with $A^0$ replacing
$\op{aff}_+^0(\orgn,\{v',w(t)\})$.

(3) We increase $t$ until $t=1$.  In this case, $\phi(t)$ coalesces
with $\phi''$ and we have a path $\phi,\phi''$.
\end{proof}

\begin{lemma}\tlabel{lemma:min-path}\rating{200}
Let $(\orgn,V,E)$ be a fan.  Let $U$ be a component of $Y(\orgn,V,E)$.
For any choice of two structured points of $U$, there exists a path
between those points of least path-length.
If a path in $U$ has least path length, then
all its intermediate points $\phi_i$ have 
$\alpha_i\ge\pi$.  There exists a path in $U$ of least path
length that has the property that every intermediate point $\phi_i$
is a concave dart.
% If alpha > pi it is necessarily a dart, by def of path angle.
% darts with angles >= pi are necessarily concave.
\end{lemma}

\begin{definition}[minimal~vertex~path] A path of least path length in $U$
is called a minimal
vertex path.
\end{definition}

\begin{proof}  
%Let $D_{cv}\subset D_U$ be the subset of concave darts.  
For
any path $\phi_0,\ldots,\phi_r$, let
   $$a = \#\{i \mid 0 < i < r,\quad (s(\phi_i)\not\in D_U \text{ or }
       \alpha_i < \pi)\}.$$
We define the complexity of a path to be the ordered pair $(a,-r)$.
There is a partial order on complexities given by the lexicograpical order.
One pair $(a,-r)$ smaller complexity than another $(a',r')$ if $a<a'$
or if $a=a'$ and $r>r'$.  If we fix the number $a$, and  $L>0$, then there is a constant $r(a,L)$ such
that every path with path-length at most $L$ and constant $a$ has step length at
most $r(a,L)$.  (Choose $C>0$ such that every
arclength between concave darts is at least $C$. The path has at most $L/C$ such steps.  Then set $r(a,L) = 2 a + L/C$.)

Thus, among paths of path-length at most $L$, from $\phi$ to $\phi'$,
there is one of least complexity. If this path
has an intermediate structured point $\phi_i$ such that 
$\alpha(\phi_i)<\pi$ or $s(\phi_i)\not\in D_U$, then we use 
Lemma~\ref{lemma:path-case} to modify the path.  The new path
may have greater step size, but it always has smaller complexity.
By the spherical triangle inequality and the conclusions of Lemma~\ref{lemma:path-case}, the modified path length is not increased.
If $\alpha(\phi_i)<\pi$, the modified path length decreases.  
Hence, if a path is minimal, then $\alpha(\phi_i)\ge\pi$, at each intermediate structured point.  This established the first
claim of the lemma.

Repeating, we may assume that $a=0$.  If such a path visits the same dart twice it creates a ``loop'' which cannot be part of a minimal  vertex path. Hence we can restrict the minimization to paths that visit each dart at most once, which are finite in number.  Thus, a vertex path exists for which all intermediate points are structured as darts.  As the path angles must satisfy $\alpha(\phi_i)\ge\pi$, they are necessarily concave.
\end{proof}


\begin{lemma}\tlabel{lemma:visr1}\rating{30}  
Let $(\orgn,V,E)$ be a fan. Let $U$ be a component of $Y(\orgn,V,E)$.  Let $\phi_0,\ldots,\phi_r$ be a minimal vertex path in $U$ from $\phi_0$ to $\phi_r$.  Let $v=v(\phi_0)$ and $v'=v(\phi_r)$, with $v,v'\in V$.  Assume that there is no edge $e\in E$ such that $\op{aff}_+(\orgn,\{v,v'\})\subset C(\orgn,e)$.
Then $v$ is visible from $v'$ in $U$ if and only if $r=1$. 
\end{lemma}

\begin{proof}
If $\{\orgn,v,v'\}$ is collinear, then $r>1$ and $v$ is not visible from $v'$.  The result holds here.  Assume non collinearity.  

Let $C^0 = \op{aff}_+^0(\orgn,\{v,v'\})$.  If $r=1$, then
$V\cap C^0 = \emptyset$.  We have
$C^0(\orgn,e)\cap C^0 = \emptyset$, for $e\in E$.  It follows that
$C^0\subset U$, so that $v$ is visible from $v'$.

Conversely, if $v$ is visible from $v'$, then $\phi_0,\phi_r$ is a vertex path from $\phi_0$ to $\phi_r$.  By the spherical triangle inequality, it has minimal path-length.  By the case of equality in the spherical triangle inequality, it is the unique minimal vertex path.  So $r=1$.
\end{proof}

\subsection{convexity}


\begin{lemma}\tlabel{lemma:convex-vis}\rating{20}
Let $(\orgn,V,E)$ be a fan.  Let $U$ be a component of $Y(\orgn,V,E)$.
Assume that every dart that leads into $U$ is convex.  Then
 for every pair of structured points $\phi,\phi'$ of $U$, 
$v(\phi)$ is visible from $v(\phi')$, or there exists an edge
$e\in E$ such that 
  $$\op{aff}_+(\orgn,\{v(\phi),v(\phi')\})\subset C(\orgn,e).$$
In particular, the minimal path-length between any
two structured points in $U$ is less than $\pi$. 
\end{lemma}

\begin{proof}  The minimal vertex path between $\phi$ and $\phi'$ has $r=1$ (Lemma~\ref{lemma:min-path}.
By Lemma~\ref{lemma:visr1}, $v(\phi)$ is visible from $v(\phi')$.  
In particular, $\{\orgn,v(\phi),v(\phi')\}$ is not collinear.
The angle between any two points that are not collinear with $\orgn$ is less than $\pi$. 
\end{proof}

\begin{lemma}\tlabel{lemma:details}\rating{20}
Let $(\orgn,V,E)$ be a fan.  Let $U$ be a component of $Y(\orgn,V,E)$.  Assume that there is a face $F$ such that every dart that leads into $U$ lies in $F$.  Assume every dart of $F$ is convex 
except possibly one $x$
at vertex $v$.  Then  every vertex $w$ of $F$ is visible from $v$, is adjacent to $v$ along an edge (meaning the vertex at $f x$ or $f^{-1} x$, for face map $f$), or is $v$ itself.
\end{lemma}

\begin{proof} Similar to the proof of Lemma~\ref{lemma:convex-vis}.
\end{proof}

%\begin{proof}  Let $w$ be a vertex that is not $v$ and is not adjacent %to $v$. 
%In the minimal vertex path from $w$ to $v$, every intermediate
%vertex is concave.  By assumption, no such vertices exist.  Therefore,
%the vertex-path runs directly from $w$ to $v$.  In particular,
%$w$ is visible from $v$.
%\end{proof}

\begin{lemma}\tlabel{lemma:path-arc}\rating{300}
Let $(\orgn,V,E)$ be a fan.  Let $U$ be a component of $Y(\orgn,V,E)$.
Assume that every dart that leads into $U$ is weakly convex.  Let
$\phi,\phi'$ be structured points of $U$.  Then the minimal path-length
from $\phi$ to $\phi'$ is equal to the arclength between $v(\phi)$ and
$v(\phi')$.
\end{lemma}

\begin{proof}  Let $\phi_0,\ldots,\phi_r$ be a minimal vertex path.
If $r=1$, the the proof proceeds as in the previous lemma. Assume that $r>1$.
At each intermediate point $\phi_i$, 
the path angle $\alpha_i$ at $\phi_i$ is at least $\pi$.  The angle of the dart is at most $\pi$, so each path angle is precisely $\pi$.  It follows that there is a plane through $\orgn$ containing all the points $v_i=v(\phi_i)$.  The azimuth angles of the darts $s(\phi_i)$ are $\pi$, and
$s(\phi_i)$ are consecutive darts along a face of the hypermap.

If we show that all the points $v(\phi_i)$ lie in the same closed half plane through $\orgn$, then the angles sum as desired to give the conclusion:
  $$
  \sum_{i=0}^{r-1}\arc(\orgn,v_i,v_{i+1}) = \arc(\orgn,v_0,v_r).
  $$

To complete the proof, we show that if the points $v_i$
do not lie in the same half plane, then the vertex path does not have minimal path length.  
Write $v=v(\phi)$ and use
similar notation for other structured points.
Pick any structured point $\phi''$ such that
$\{\orgn,v,v''\}$ 
is collinear and $\orgn$ lies between the other two.  
We have $v''\in \op{aff}_+(\orgn,\{v_i,v_{i+1}\})$ for some $i$.
Infinitely many closed half planes pass through $\orgn,v,v''$.  One such half
plane contains $v=v_0,\ldots,v_i,v''$.  Shift this half-plane slightly
into $U$ so that the angle $\alpha''$ at $\phi''$ is less than $\pi$.
In this way, we construct a new vertex path
of the same path length but with a path angle less than $\pi$.

However, no path with a path angle less than $\pi$ has minimal path length (Lemma~\ref{lemma:min-path}).
This completes the proof.
\end{proof}

\begin{lemma}\tlabel{lemma:convex-face-simple}\rating{30}
Let $(\orgn,V,E)$ be a fan.  Let $U$ be a component
of $Y(\orgn,V,E)$.  Assume that there is a face $F$ such that every
dart that leads into $U$ is convex and is an element of $F$.  Then
$F$ is a simple face.
\end{lemma}

\begin{proof} Let $x,x'$ be two darts that lead into $U$.  Let
$(v,x)$, $(v',x')$ be the corresponding structured points, with $v,v'\in V$.
Let $\phi_0,\ldots,\phi_r$ be a minimal vertex path from $(v,x)$ to 
$(v',x')$.  By Lemma~\ref{lemma:visr1}  and~\ref{lemma:convex-vis}, 
$r=1$ and $v$ is visible from $v'$.  By the definition of visible
$v\ne v'$.  Hence $x$ and $x'$ lie at different vertices and $F$ is simple.
\end{proof}

\begin{lemma}\tlabel{lemma:concave-simple}\rating{30}
Let $(\orgn,V,E)$ be a fan.  Let $U$ be a component of $Y(\orgn,V,E)$.  Assume that there is a face  $F$ that leads into $U$ in which every dart is concave.  Then $F$ is simple or there exists a plane through $\orgn$ that contains all the vertices of $F$.
\end{lemma}

\begin{proof}  Assume that $F$ is not simple.
If two darts $x,y$
of the face appear at a node, the azimuth angle constraint forces both darts to
have angle $\pi$, so that the node has degree two and lies in the same plane
as the adjacent nodes. Let the hypermap of $(\orgn,V,E)$ be $(D,n,e,f)$.  The two darts $f x$ and $f^{-1}y$ also appear in $F$.  Moreover, from $n y = x$, we derive $n (f x) = e x = e n y = f^{-1} y$.  So $f x$ and $f^{-1} y$ are at the same node.  If $f x = f^{-1} y$, then $n$ is degenerate at $f x$ and the node has size one.  Otherwise, we obtain a second pair of darts $x',y'$ in $F$ that lie at the same node in plane $A$.
Continuing in this manner, we find that $f x',f^{-1} y'$, and eventually
the vertices of all darts lie in the plane $A$.
\end{proof}

\begin{lemma}\tlabel{lemma:convex-2-dart}\rating{150}
Let $(\orgn,V,E)$ be a fan, with cyclic hypermap.  
Let $U$ be a component of $Y(\orgn,V,E)$.
Assume
that every dart of $F$ is weakly convex.  If there is at most
one convex dart in $F$, then all the darts have azimuth angle $\pi$,
and all the vertices of $F$ lie in a plane through $\orgn$.  The path length of the path following the darts of $F$ around its cycle is $2\pi$.
If there are two convex darts in $F$ with vertices $v,w$, then
$\{\orgn,v,w\}$ is collinear, all the vertices of $F$ lie on two planes through $\{\orgn,v,w\}$, and the path length around $F$ is again $2\pi$.
\end{lemma}

\begin{proof}
If every dart but at most one 
has azimuth angle exactly $\pi$, then the set $\{\orgn\}\cup V$ is
planar.  This forces the final dart to have azimuth angle $\pi$ as well. 
The sum of the angles at $\orgn$ is exactly $2\pi$.  The path length
is then also $2\pi$.

If there are only two convex darts $x,y$ (at $v,w\in V$), 
then the darts from $x$ to $y$ (stepping around the face with with face map) lie at a set of vertices of $V$ that all lie on a common plane through $\orgn$.  Similarly, the vertices encountered stepping from $y$ to $x$ (with the inverse of the face map) all lie on a common plane through $\orgn$.  Thus, $v,w$ lie on two distinct planes through $\orgn$.  This forces $\{\orgn,v,w\}$ to be collinear.
Thus, the sum of the arclengths from $x$ to $y$ is $\pi$, and the sum from $y$ back to $x$ is also $\pi$, giving path length $2\pi$ for $P$.
\end{proof}

\begin{lemma}\tlabel{lemma:convex-half-plane}\rating{150}
Let $(\orgn,V,E)$ be a fan.  
Let $U$ be a component of $Y(\orgn,V,E)$.
Let $F$ lead into $U$.
that every dart of $F$ is weakly convex. Assume that there
are at least three convex darts in $F$.  Let $x,y$
be consecutive darts on $F$ and let $u,v$ be the corresponding points of $V$.
Choose $p\in U$ that does not lie in the plane $\op{aff}\{\orgn,u,v\}$.
Then $U$
is contained in the half-space $\op{aff}_+^0(\{\orgn,u,v\},p)$.
\end{lemma}

\begin{proof} As $U$ is open, it is enough to show that it lies
in each $\op{aff}_+(\{orgn,u,v\},p)$.  For a contradiction, let
$q\in \op{aff}_-^0(\orgn,u,v\},p)\cap U$. 
 By Lemma~\ref{lemma:convex-vis},
we have $A^0=\op{aff}_+^0(\orgn,\{p,q\})\subset U$. 
By perturbing $q$, we may assume that $A^0\cap V=\emptyset$. 

If $x=(\orgn,v,\ldots)$, write $v(x)$ for its second component.
Consider the cylic order on the darts of $F$ given by the face map.
Moving from $p$ to
$q$, pick consecutive darts $x,y$ ($y = f x$) 
such that $A^0$ crosses $\op{aff}\{\orgn,v(x),v(y)\}$ first, say at point $p'$. Let $i\ge0$ be as small as possible such that $y'=f^i x$ is
convex.  Set $y'' = f y'$.  Similarly, set $x' = f^{-j} x$, $x'' = f^{-1} x'$.

Let $w\ne\orgn$ be a point on the intersection of planes $\op{aff}\{\orgn,v(x'),v(x'')\}\cap \op{aff}\{\orgn,v(y'),v(y'')\}$.
By the first crossing construction,
we have $p'\in \op{aff}_+(\{\orgn,w\},\{v(x'),v(y')\})$.  
We see from this
that $p'\in\op{aff}_+(\orgn,\{v(z),v(z')\})$,
for some $z = f^k x$, $z' = f^{k+1}x$, with $-j \le k< i$. 
As $A^0\cap V=\emptyset$,
we have more strongly that $p'\in\op{aff}_+^0(\orgn,\{u,v\})\subset X(\orgn,V,E)$.
Thus, $p'\not\in U$, which contains $A^0$.  This is a contradiction.
\end{proof}


\begin{lemma}\tlabel{lemma:convex-hyper}\rating{250}
Let $(\orgn,V,E)$ be a fan, with cyclic hypermap.  
Let $U$ be a component of $Y(\orgn,V,E)$.  Let $F$ lead
into $U$.
Assume
that every dart of $F$ is weakly convex. 
let $P$ be the vertex path that steps through the vertices of $V$ listed
in the order of the face-map $f$ on $F$.  Then the path-length of the closed
path $P$ is at most $2\pi$.
\end{lemma}

\begin{proof}  Lemma~\ref{lemma:convex-2-dart} handles the case
when there are at most two convex darts in $F$.
Assume that there are at least three convex darts.
We prove the result by induction on the number of convex darts, with base case given by two convex darts.

Pick distinct planes $\op{aff}(\orgn,\{u,v\})$ and $\op{aff}(\orgn,\{u',v'\})$ with vertices $u,v,u,v'\in V$, such that there are no points of
$V$ along the line of intersection.  We may further assume that 
the arclength from $v$ to $v'$ is less than $\pi$ and
$\op{aff}_+(\orgn,\{v,v'\})$ is the path length of the path that
steps consecutively through the darts from $v$ to $v'$\FIXX{Why?}

Pick $w\ne \orgn$ in the intersection of these two planes along
the half circle of radius $1$, centered at $\orgn$, that lies
in the half plane $\op{aff}_+(\{\orgn,u\},v)$.  Modify the fan
as follows. Add to $V$ the point $w$ and delete the vertices along
$F$ strictly between $v$ and $v'$.  Delete the edges with a vertex
at any of these deleted vertices.  Add edges $\{v,w\}$ and $\{w,v'\}$.

The resulting fan $(\orgn,V',E')$ has a corresponding face $F'$ that is
still weakly convex.  It has fewer convex darts (so that the induction
hypothesis applies to it).  By the triangle inequality, the
arclength from $\{v,w\}$ then to $\{w,v'\}$ is at least the arclength
of $v$ to $v'$.  So we increase the perimeter path length to get
a path length at most $2\pi$.
\end{proof}

%We consider a deformation $(\orgn,V_t,E)$ such that the only vertex that moves is
%$v$.  Suppose that $(u,v,w)$ are consecutive (distinct) vertices in the face  $F$ and that the dart at $w$ is convex.
%Let $v_t$ be the point in the plane of $\{\orgn,u,v\}$ at distance $1$ from $\orgn$ that
%is extended by arclength $t$ beyond the arc from $(u-\orgn)/|u-\orgn|$ to $(v-\orgn)/|v-\orgn|$.
%As $t$ increases, the angle at $w$ increases until for some value $t=s$, the angle
%at $w$ is $\pi$.  The angle at $v_t$ remains between $0$ and $\pi$.  If $t<t'$, then $U_t\subset U_{t'}$.   By the spherical-triangle inequality, it is
%also increasing in the perimeter path-length.   
%By Lemma~\ref{lemma:path-arc}, the path-length between any two structured 
%points of $U$ is the same as its arclength.
%Because of the bound on the path-length
%at most $\pi$ in $\bar U_t$, it is not possible for $C^0(\orgn,\{v_t,w\})$ to meet
%$C^0(\orgn,\{u,u'\})$ for any edge $\{u,u'\}\ne\{v_t,w\}\in E$.   
%When $t=s$, we reduce to a case known by the induction hypothesis.
%\end{proof}
%
%
%


\begin{lemma}\tlabel{lemma:7-sides}
    \oldlabel{5.1.2}
Let $(\orgn,V,E)$ be a fan with cyclic hypermap.
Let $F$ be one of the two faces of the
hypermap.  Assume that every dart of $F$ has azimuth angle at
most $\pi$.  Assume that each edge of $F$ has arclength 
greater than $\pi/4$.
Then the size of the face $F$ is at most seven.
\end{lemma}

\begin{proof} By Lemma~\ref{lemma:convex-hyper},
 its perimeter path-length is at
most $2\pi$.  If there are eight sides, the perimeter
is greater than $8 (\pi/4) =2\pi$.
\end{proof}




\subsection{triangulation}




\begin{lemma}\tlabel{lemma:exists-diagonal}\rating{150}
 Let $(\orgn,V,E)$ be a fan.  Let $U$ be a component of $Y(\orgn,V,E)$.
Assume that one of the following conditions hold:
\begin{itemize} 
  \item There are darts $x,y$ in different group orbits  of the 
   hypermap that lead into $U$, or
  \item Some face $F$ whose darts lead into $U$ 
has more than three darts, and there is a convex dart of $F$.
\end{itemize}
Then there exists $\{u,v\}$ such that $\{u,v\}\not\in E$ and such
that $u$ is visible from $v$.
\end{lemma}

\begin{proof} Suppose that there are darts $x,y$ in different group orbits
of the hypermap that lead into $U$.  Let $v,v'\in V$ be the vertices of $x,y$, respectively.
Let $(\phi=\phi_0,\ldots,\phi_r=\phi')$ be a minimal vertex path.
By Lemma~\ref{lemma:orbit-connect},  the group orbits
lie in topologically distinct sets of vertices and edges of the fan.   
Thus, for some $i$,
   $$
   \op{aff}_+^0(\orgn,\{v_i,v_{i+1}\})\subset U
   $$
and $\{v_i,v_{i+1}\}\not\in E$.  This is the desired pair.

We may now assume that the darts leading into $U$ all belong to the same
group orbit of the hypermap.  In particular, we may assume that
the darts leading into $U$ are non-degenerate.
Suppose that $F$ leads into $U$ and has at least four darts.
Let $(x,y,z)$ be consecutive darts of $F$ under the face map.  
with corresponding vertices $(u,v,w)$.  Suppose that $y$ is convex.
This implies that $\{\orgn,u,v,w\}$ is not coplanar.

Consider the minimal vertex path $\phi_0,\ldots,\phi_r$ from
$\phi_0=(u,x)$ to $\phi_r=(w,z)$.  
If any link $\phi_i,\phi_{i+1}$ is not along an edge,
then $\{v(\phi_i),v(\phi_{i+1})\}$ gives the lemma.
So assume these links are each along some edge.  In particular, we may
assume that $r>1$, because $\{u,w\}$ is not an edge.

The azimuth angle at $x$ is at most
   $$\op{azim}(\orgn,u,v,v(\phi_1))\le 
     \dih_V(\{\orgn,u\},\{v,w\}) <\pi.$$
Thus, $x$ is convex.  Similarly, $z$ is convex.  

The minimal vertex path $\phi'_0,\phi'_1,\ldots$
from $(v,y)$ to $\phi_1$ does not pass
by way of $\phi_0,\phi_r$.  Thus, $\{v,v(\phi'_1)\}$ gives the
lemma.
\end{proof}

%
%As in the proof of 
%Lemma~\ref{lemma:min-path}, we consider the blade $C^0_s= aff_+^0(\orgn,\{u,(1-t)v + t w\})$ and the
%sup $s$ of $t\le 1$ for which $C^0_t\subset U$.  If $s=1$ and $C^0_1\subset U$,
%then the assumption that that there are at least four darts in $F$ allows us
%to conclude that $\{v,w\}\not\in F$, so that $C^0_1$ is the desired blade.
%If, on the other hand, $s<1$, we find that $C^0_s$ contains points
%$v'_1,\ldots,v'_k$ of $V$.  We may order the points along $C^0_s$.  If
%$v'_1$ is not the vertex consecutive to $u$ in the order imposed by the face map on $F$,
%then $C^0(\orgn,\{u,v'_1\})$ gives the desired blade.  Hence we may assume that
%the vertices appear in the face $F$ in the order $(v'_1,u,v,w)$.  The azimuth
%angle at $u$ is less than the dihedral angle of $(\{\orgn,u\},\{v,w\})$.  In particular,
%it is a convex dart at $u$.  By a symmetrical argument, we also find a desired blade at $w$
%or prove that the azimuth angle of the dart at $w$ is convex.
%
%
%Consider the minimal vertex path from $v=v_0$ to $v'_1$.  Any intermediate
%vertices on this path are concave by Lemma~\ref{lemma:min-path}.  
%In particular, the path
%does not pass through the darts $x$ and $z$ at $u$ and $w$.  That is, the path
%jumps from $y$ past the adjacent darts $x$ and $y$.  Hence 
%$C^0(\orgn,\{v_0,v'_1\})$ gives the desired blade.
%\end{proof}
%


\subsection{planarity}

\begin{lemma}\tlabel{lemma:tri-tri}\rating{100}
Let $(\orgn,V,E)$ be a fan.  Assume
that its hypermap is a triangulation. (That is, all the darts are reduced; there is a single group orbit; the size of each face is $3$; and every dart is
convex).   
Let $\#c$ be the number
of components of $Y(\orgn,V,E)$.  We have
    \begin{itemize}
    \item The hypergraph $\op{hyp}(\orgn,V,E)$ is planar.
    \item The faces are in bijection with the components, given by which
    component each face leads into.
        \item Let $U$ be a component. Then $U \cap B(\orgn,1)$ 
       is measurable and eventually radial at $\orgn$.  Its solid angle is
        $$-\pi + \sum_{x\in F} \op{azim}(x)$$
    \end{itemize}
\end{lemma}

\begin{proof}
Each component 
of $Y(\orgn,V,E)$ has
the form $U(v_1,v_2,v_3)=\op{aff}_+^0(\orgn,\{v_1,v_2,v_3\})$.  In fact,
these sets are open, convex, and hence connected, 
and their union is all of $Y(\orgn,V,E)$; thus,
they are the components.   There is a bijection
between components and faces $U \leftrightarrow F$, which associates this component
with the three darts $(\orgn,v_i,v_j,v_k)$, $(i,j,k)$ a cyclic permutation
of $(1,2,3)$  (assuming appropriate orientation on the indices $(1,2,3)$).

Each component $U(v_1,v_2,v_3)$ is measurable and eventually radial
at $\orgn$.  By the area formula for a solid triangle, the solid angle of $U(v_1,v_2,v_3)$
is 
   $$-\pi + \sum_{x\in F} \op{azim}(x)$$
as stated in the lemma.  If we sum this relation over all components, we obtain
the solid angle of a sphere:
   \begin{equation}\tlabel{eqn:euler}
   \begin{array}{lll}
   4\pi &= -\sum_{F} \pi + \sum_{x} \op{azim}(x)\\
        &= -\#f \pi + 2 \# n \pi.
   \end{array}
   \end{equation}
In a triangulation, $\#\tangle{e,n,f}=1$, $2 \# e  = 3\#f$, $\# D = 2 \#e$.
It follows from these relations that
   $$
   \#n + \# e + \# f = 2\#\tangle{e,n,f}+\# D.
   $$
Thus, the hypermap is planar.  The lemma is now fully established.
\end{proof}


\begin{lemma}\tlabel{lemma:cx-dart}  
Let $(\orgn,V,E)$ be a fan. Assume that the hypermap contains a single
group orbit.   
Assume that no half space
through $\orgn$ contains $V$.  Then every face of the hypermap contains
a convex dart.
\end{lemma}

\begin{proof}
Suppose for a contradiction that 
some face $F$ of the resulting hypermap only has concave darts.  Then
the face is simple (Lemma~\ref{lemma:concave-simple}).  

We form a cyclic quotient of the hypermap.
There
is a normal family consisting of the concave face and the complementary path.
The quotient is is cyclic.  We delete from $V$ the vertices that are not on the face $F$ and from $E$ the edges that are not edges coming from the face $F$,
to get a fan $(\orgn,V',E')$ for the cyclic quotient.
The quotient hypermap has two faces $F$ and $F'$.  Every dart
of $F'$ is convex.  Assume that $F'$ leads into $U'\in[Y(\orgn,V',E')]$.

By Lemma~\ref{lemma:convex-half-plane}, $U'$
lies in the intersection $R$ of half-planes
$$
\op{aff}_+(\{\orgn,u,v\},p),
$$
where $\{u,v\}$ runs over adjacent vertices.  


We have $V'\subset R$.\FIXX{You could save some effort by rewording lemma:convex-half-plane so that it covers the construction in this paragraph.}  We claim, in fact, that
every vertex of $V\subset R$.
(This will contradict the assumption that $V$ does not lie in a half plane.)
If not, we use that the hypermap is has a single group orbit 
to find an edge $\{w,w'\}$ such
that one endpoint $w$ lies in $R$ and the other $w'$ does not.  
Let $\op{aff}\{\orgn,u,v\}$ be the bounding plane of $R$ that 
is crossed first (say at point $q$) by
$C(\orgn,\{w,w'\})$.
Let consecutive vertices be $u,v,v'$ on $F$.  Since
$q\in \op{aff}_+(\{\orgn,v,v'\},u)$, 
we see that $q$ has non-negative barycentric
coordinate with respect to $u$ in $\{\orgn,v,v',u\}$.  
We have $q\in \op{aff}\{\orgn,u,v\}$. Also, $q\in\op{aff}\{\orgn,u,v\}$.
Hence, $q\in \op{aff}_+(\{\orgn,v\},u)$.  Similarly,
$q\in\op{aff}_+(\{\orgn,u\},v)$.
We conclude that 
  $$q\in \op{aff}_+(\orgn,\{u,v\}).$$
By the definition of fans, the meeting of two
blades implies that $w\in\{u,v\}$. Say $v=w$.

$(\orgn,v,v',u)$ is a dart of $F$.  Thus, $u$ follows $v'$ in the azimuth cycle on $E(v)$.  Thus,
  $$0 \le \op{azim}(\orgn,v,u,w) \le \op{azim}(\orgn,v,u,u').$$
This implies that $w\in R$, for a contradiction.
\end{proof}

\begin{lemma}\tlabel{lemma:hypermap-planar}\rating{400}
Let $(\orgn,V,E)$ be a fan.    
Let $\#c$ be the number
of components of $Y(\orgn,V,E)$.  Let $t>0$. We have
    \begin{itemize}
    \item The hypermap $\op{hyp}(\orgn,V,E)$ is planar.
    \item No two faces in a given group orbit lead
    into the same component.
    \item $\#c = 1 + \#f - \#\tangle{e,n,f}$
    \item Let $U$ be a component. Let $F\mapsto U$ mean
    that the face is non-degenerate and leads into $U$.  
     Then $U \cap B(\orgn,t)$ 
       is measurable and eventually radial at $\orgn$.  Its solid angle is
        $$\op{sol}(U) = 4\pi + \sum_{F\mapsto U}(-2\pi + \sum_{x\in F}
        (\op{azim}(x)-\pi))$$
    \end{itemize}
\end{lemma}

\begin{proof}  
Add to the fan finitely many isolated
points  $V' = V\cup S$ to make $(\orgn,V',E)$ (by isolated we mean that the edge set $E$ does not change) such that no half-space
through $\orgn$ contains $V'$.    While the hypermap
has more than one group orbit, 
we may use Lemma~\ref{lemma:exists-diagonal} to find $u,v\in V'$ such
that $C^0=C^0(\orgn,\{u,v\})$ lies in a single component of $Y(\orgn,V',E)$ and
joins two group orbits.  We add the edge $\{u,v\}$ to $E$ to form
a new fan and hypermap.
We repeat this process until the hypermap has one group orbit.  In particular, this
inserts edges to each of the extra added points $S\subset V'$.


By Lemma~\ref{lemma:cx-dart},
every face has at least one convex dart.  By Lemma~\ref{lemma:exists-diagonal},
we can add an edge to every face (that is not already a triangle).  By repeatedly
inserting edges, eventually we obtain a triangulation.  It satisfies
the conclusions of Lemma~\ref{lemma:tri-tri}.

Now we reverse the process.  Starting with the triangulation, we remove the
edges that we have inserted into the preplanar graph  (in a first-in first-out order).  
Each edge removal is
described combinatorially as a double-walkup.  (Note that
in the case of an edge to an isolated vertex  $v\in S\subset V'$, the second walkup
in the double walkup
 removes the degenerate dart at $v$.  Hence in reversing the process,
the vertices $S$ may be removed.)  Double-walkups of
a planar hypermap preserve planarity by Lemma~\ref{lemma:walkup-planar}.  Hence the original
fan has a planar hypermap.  Also, by the calculations of planar indices
in Lemma~\ref{lemma:index}, every double walkup either merges or  splits
 increasing the number of components by $1$.

In the triangulation the set of components is in natural bijection with the set of faces.
Each merging double walkup is applied to an edge $\{x,y\}$ such that each endpoint $x$,
$y$ leads into a different face, hence into a different component.  By Lemma~\ref{XX},
and Lemma~\ref{lemma:U-merge}, the number of components and faces both drop by exactly one.
In particular, the components and faces remain in natural bijection after the
double walkup.  Each double walkup that splits increases the number of faces by one,
increases $\#\tangle{e,n,f}$ by one, and preserves the number of components.
The announced formula for the number of components thus holds.

Finally, the solid angle formula is easily checked to be compatible with
the deletion of the edges.  That is, when two areas are joined, we have
  $$\op{sol}(C_1) + \op{sol}(C_2) = \op{sol}(C_1\cup C_2)$$
and the combinatorial formula is similarly additive.
The lemma is now completely established.
\end{proof}



\begin{lemma}[Jordan curve theorem]\tlabel{lemma:jct}\rating{50}
Let $(\orgn,V,E)$ be a fan.
If $\op{hyp}(\orgn,V,E)$ is a combinatorial polygon (a
 hypermap with a single group orbit such that every node has cardinality two), then
$Y(\orgn,V,E)$ has exactly two components.
\end{lemma}

\begin{proof} By the preceding lemma, $\# c = \#f$, and the
hypermap is planar.  Since every node and every edge has order two,
we have $\#D = 2\#n = 2\# e = \#n +\#e$.  Also by assumption,
$\#\tangle{e,n,f} = 1$.  The Euler relation gives:
    $$
    \#c = \#f = (\#D - \#n - \#e) + 2\#\tangle{e,n,f} =2.
    $$
\end{proof}



\subsection{deleting a node}

\FIXX{Revise.}

When we delete vertices and edges from a fan, its hypermap
changes.  We show that the new hypermap is a quotient
of the one we start with.

Let $(v,V,E)$ be a fan with hypermap $\op{hyp}(v,V,E)$.
Let $V'\subset V$ be a set of nodes. 
Let $E'\subset E$ be the set of edges $e$ that meet $V'$.
We have a fan $(v,V\setminus V',E\setminus E')$.

The set $V$ is in bijection with the nodes of the hypermap.
Explicitly, $v\in V$ corresponds with the set of darts
 $\{(v_0,v,\ldots)\}$.
We define a normal family of contour loops on 
$\op{hyp}(v,V,E)$.  Let $D''$ be the set of darts that lie in the
nodes constituting $V'$.  Let $D'$ be the remaining darts.
We place each $x\in D'$ in a contour
loop by the following permutation $r$ on $D'$:
    $$r x =
    \begin{cases}
    f x, & f x \in D'\\
    n^{-1} x & f x \in D''\\
    \end{cases}
    $$
For each $x\in D'$, we obtain a contour loop $(x,r x, r^2 x,\ldots)$.
Let $\cal L$ be this family of contour loops.  It is normal.

\begin{lemma}\tlabel{lemma:quot-fan}
In this context,
$\op{hyp}(v,V,E)/{\cal L}$ is isomorphic to 
$\op{hyp}(v,V\setminus V',E\setminus E')$.
\end{lemma}

The hypermap $\op{hyp}(v,V,E)$ attached to a fan $(v,V,E)$
has two types of darts: the non-degenerate darts of the form
$(v,v_1,v_2,v_3)$ and the degenerate darts of the form $(v,w)$.
Each degenerate dart is a  fixed point for the edge, face, and node
permutations.

A special case of the construction of the previous section
is to delete a degenerate node (dart) from a fan.
Degenerate darts $x=(v,w)$ are in bijection with vertices
$w\in V$ such that $E_w = \emptyset$.
If $V'$ is a set of degenerate darts, with corresponding elements
$V'\subset V$, 
we obtain a fan
$\op{hyp}(v,V\setminus V',E)$.  There is a canonical bijection
between components of $Y(v,V,E)$ and components of $Y(v,V\setminus V',E)$.

Each degenerate dart $x=(v,w)$ leads into a uniquely determined
component of $\op{hyp}(v,V,E)$.  
Figure~\ref{fig:deg-pent-hex} %% WW not yet drawn. Two frames of DCG~page126.
shows a degenerate dart that leads into a component where
the other darts leading into the component form a pentagon
or hexagon.

\begin{figure}[htb]
  \centering
  \myincludegraphics{noimage.eps}
  \caption{Degenerate darts lead into components} % DCG examples.
  \label{fig:deg-pent-hex}
\end{figure}


\subsection{distance}


\begin{lemma}\tlabel{dist2}\rating{400}
Let $(\orgn,V,E)$ be a fan with component $U$ of
$Y(\orgn,V,E)$.  Assume the hypermap is cyclic. 
Let $F$ be the simple face of the hypermap
that leads into $U$.   Suppose that there are $v,w\in V$ such that
the following conditions hold:
\begin{itemize}
 \item $\{v,w\}\not\in E$;
 \item $2\le |v'-\orgn| \le 2.51$, for $v'\in V$,
 \item $|v'-w'|> 3.2$ if
    \begin{itemize}
       \item $v'\ne w'\in V$,
       \item $\{v',w'\}\not\in E$,
       \item $\not\exists u'.\quad \{v',u'\}, \{u',w'\}\in E$, and
       \item $\op{aff}_+^0(\orgn,\{v',w'\})\subset U$;
    \end{itemize}
 \item $|v'-w'|\ge \sqrt8$ if
    \begin{itemize}
      \item $v'\ne w'\in V$,
      \item $\{v',w'\}\not\in E$, and
      \item $\op{aff}_+^0(\orgn,\{v',w'\})\subset U$;
    \end{itemize}
 \item $|v'-w'|\le 3.2$ if $\{v',w'\}\in E$;
 \item $2\le |v'-w'|$, if $v'\ne w'$ and $\{v',w'\}\ne \{v,w\}$.
 \item $v(1-t) + w t \in U$ for all sufficiently small $t>0$.
\end{itemize}
Then $|v-w|>2$.
\end{lemma}


\begin{proof}
The proof is by contradiction.  Assume that $|v-w|\le 2$.
If $\op{aff}_+^0(\orgn,\{v,w\})\subset U$, then $|v-w|\ge\sqrt8$ and
we are done.  Assume otherwise.  Since $(1-t)v + t w\in U$ for $t>0$
small, we can find a smallest positive $t\le 1$ (say $t'$) for which
$(1-t)v + t w\not\in U$.  We have
$$p_0 = (1-t') v + t' w \in \op{aff}_+(\orgn,\{v_1,v_2\})$$
for some edge $\{v_1,v_2\}\in E$.  
In fact, $p_0\in \op{aff}_+(\orgn,\{v_1,v_2\})$.

We can structure each point of $\op{aff}^0_+(\orgn,\{v'\})$ with
$v'\in V$ with the corresponding dart $(\orgn,v',\ldots)$ of $F$.
Structure each point of $\op{aff}_+^0(\orgn,\{v',w'\})$ with
$\{v',w'\}\in E_U$, with the corresponding $(\orgn,v',u',w')\in E_U$.
Structure points of $U$ with $*$. Thus, we consider our points as
structured without further comment.


We have $|v_1-v_2|\ge2.91$ by Lemma~\ref{tarski:E:part4:10}.

Let $\phi_0,\ldots,\phi_r$ be a minimal vertex-path from $v$ to
$v_1$, and let $\phi'_0,\ldots,\phi'_s$ be a minimal vertex-path from
$v$ to $v_2$ (for some natural numbers $r,s$). 
Set $\beta_i = \arc(\orgn,\phi_i,\phi_{i+1})$ and
$\beta'_i = \arc(\orgn,\phi'_i,\phi'_{i+1})$.
By minimality, 
  $$
   \sum \beta_i + \sum \beta'_i \le 2 \arc(\orgn,v,w) + \arc(\orgn,v_1,v_2).
  $$
(Compare the minimal path with one running from $v$ to $p_0$ to $v_1$
together with one from $v$ to $p_0$ to $v_2$. Then use $\arc(\orgn,v,p_0)\le
\arc(\orgn,v,w)$.)

We say that two vertices are subadjacent (with subvertex $u$)
if they are both adjacent
to the same vertex $u$.

Note that if $r+s\le4$, then one of the arclengths on one of the two minimal paths
must be at
least $3.2$, for otherwise consecutive pair of structure points
in the path corresponds to
an 
edge of $E$ or to a subadjacent pair. 
In this case, the vertices and subvertices completely enumerate
all the vertices around $F$ and hence all of $V$.  This means that
$w$ lies on the vertex path or is one of the subvertices.  It
cannot lie on the vertex path because $w$ is outside $\op{aff}_+(\orgn,
\{v,v_1,v_2\})$ and the vertices of the vertex path inside.
It cannot be a subvertex by Lemma~\ref{tarski:subvertex}.
Hence some length is at least $3.2$.

We have $r+s\le3$ by the following calculations.  Here
$y\in[2,T]$, where $T=2.51$.
    $$5\arc(T,T,2) > \arc(2,2,3.2)+ 2 \arc(2,2,2).$$
    $$3\arc(T,T,2) + \arc(T,y,3.2) > \arc(y,2,3.2) + 2 \arc(2,2,2).$$
    $$3\arc(T,T,2) + \arc(T,y,3.2) > \arc(2,2,3.2) + 2\arc(y,2,2).$$
    $$2\arc(y,T,2) + \arc(T,T,2) + \arc(T,T,3.2) > \arc(2,2,3.2) + 2\arc(y,2,2) $$
(In every possible placement of the edge of length greater than $3.2$,
we have that $\sum \beta_i + \sum\beta'_i$ is at least the left-hand side
and the right-hand side is at least $\arc(\orgn,v_1,v_2)+2\arc(\orgn,v,w)+$,
which is contrary to minimality.)

Next, we prove the lemma in the case that the distance from
$v$ to one of the endpoints, say $v_1$, of $\{v_1,v_2\}$ is at least
$3.2$. In this case, Lemma~\ref{tarski:dcg-1220} shows
that we have an
impossible geometric configuration. The
constraints are as follows.  There are five points: $\orgn,v_1,w,v,v_2$.
The plane $\op{aff}_+\{\orgn,v_1,v_2\}$ separates the point $w$ from $v$. The
distance constraints are as follows:
    $$2\le |u-\orgn| \le T,$$
for $u=v_1,w,v,v_2$; $|v-v_1|\ge 3.2$, $|v-w|\le2$, $|v-v_2|\ge2$,
$|w-v_1|\ge2$, $|w-v_2|\ge2$, $2\le |v_1-v_2|\le 3.2$.

We may now assume that the distances from $v$ to the vertices $v_1$ and
$v_2$ are at most $3.2$.  This condition is incompatible with $r+s=2$
(and the fact that some step of a minimal path has distance greater than 
$3.2$).  

Hence, $r+s=3$.
Say $r=1$ and $s=2$. We have $\{v,v_1\}$ is
an edge of $E$ or a special pair.  Let
$\{v,u\}=\{\phi'_0,\phi'_1\}$. We
have $|u-v_1|\ge\sqrt8$ because $\{u,v_1\}$ is not in $E$,
and $\max(|u-v|,|u-v_1|)>3.2$, by our assumption that there is at least
one large step.  Lemma~\ref{tarski:dcg-p142} now shows that
there does not exist a configuration of five points
$\orgn$, $u$, $v$, $v_1$, $v_2$, with all distances at least $2$
satisfying these constraints.
\end{proof}



\subsection{containment} %DCG  12.11, p 136
    \oldlabel{4.12}

%
%Let $H =\op{hyp}(\orgn,V,E)$ be the hypermap of a fan,
%and let $U$ be a component
%of $Y(\orgn,V,E)$.
%
%We assume
%that we are working with a component
%with the following properties. If $v$ is a concave dart and $w$
%is not adjacent to $v$, and yet is visible from $v$, then
%$|v-w|\ge3.2$. If $v$ is a concave corner, then $|v-w|\ge3.07$ for
%both adjacent corners $w$. If $v$ is a concave corner and
%$|v-\orgn|\ge2.2$, then $|v-w|\ge3.2$ for both adjacent corners $w$.
%
%
%Recall from Definition~\ref{def:concave} that we call a spherical
%region {\it convex} if its interior angles are all less than
%$\pi$. 
%%The case where the substandard component is a convex triangle will be
%%treated in Section~\ref{x-5.7}. Hence, we may also assume in
%We assume
%that the component is
%not a convex triangle.
%
%In Section~\ref{sec:tcc}, a region $TC(\orgn,v,w_1,w_2,1.255,\lambda)$,
%called the
%truncated corner cell (tcc), is introduced.
%We construct a {\it truncated corner cell\/} at each corner.  It depends on 
%parameters $1.255$ and $\lambda \in [1.6,1.945]$. In all applications, we
%take
%    $\lambda = 1.945 = 3.2-1.255$, $\lambda = 1.815 = 3.07-1.255$, or
%    $\lambda = 1.6 = 3.2/2$.
%In all applications $w_1,v,w_2$ will be consecutive corners of
%a standard component.
%
%By construction tccs at adjacent
%corners of a standard component are separated by a plane . Tccs at
%nonadjacent corners do not overlap if the corners are
%$\ge2\lambda$ apart. Tccs will only be used in substandard components
%satisfying this condition. It will be shown in
%this section that tccs lie in the cone over the substandard component
%(for suitable $\lambda$).
%


\begin{lemma}\tlabel{lemma:4.12.3}
%%    \oldlabel{4.12.3}
%The truncated corner cells with parameter $1.6$ in a substandard component do
%not meet at interior points.
Let $(\orgn,V,E)$ be a fan.  Let $U$ be a component of $Y(\orgn,V,E)$.
Suppose that the set of darts leading into $U$ forms a simple face $F$.
Assume that $|v-w|\ge 3.07$ for all $v,w\in V$, distinct, such that
  \begin{itemize}
   \item $\op{aff}_+^0(\orgn,\{v,w\})\subset U$, and
   \item the dart at $v$ in $F$ is concave.
  \end{itemize}
Assume that $|v-w|\ge 3.2$ for $v,w\in V$, distinct, such that
  \begin{itemize}
   \item $\op{aff}_+^0(\orgn,\{v,w\})\subset U$, and
   \item $\{v,w\}\not\in E$.  % NB. no specials.
  \end{itemize}
Assume that $2\le |v-v_0|\le 2.51$ for all $v\in V$.
Let $x,x'$ be darts of $F$ that are not adjacent, appearing at $v,v'\in V$.  
Then
$$
V(x,\epsilon(v))\cap V(x',\epsilon(v')) = \emptyset,
$$
where $\epsilon(u) = \cosarc(|u-\orgn|,1.255,\lambda)$ and $\lambda=1.6$.
\end{lemma}

\begin{proof}
Assume for a contradiction that these sets meet at $p$.
Let $\phi=(v,x)$ and $\phi'=(v',x')$ be the structured points for
the darts $x,x'$.  Let $\phi_0,\ldots,\phi_r$ be a minimal vertex
path from $\phi$ to $\phi'$.  By minimality,
$$
\begin{array}{lll}
 2.0 &> 2\arc(2,1.255,\lambda) \ge \arc_V(\orgn,v,p) +
\arc_V(\orgn,p,v') \\
  &\ge \sum \arc_V(\orgn,v(\phi_i),v(\phi_{i+1})\ge
  r \arc(2.51,2.51,3.07)\\ 
  &> r (1.316).
\end{array}
$$
Hence, $r=1$.
Thus, $\op{aff}_+^0(v_0,\{v,v'\})\subset U$ and by tarski\tarf{tarski:cone-meet},
$$
  3.2 \le |v-v'| < 3.2.
$$
\end{proof}

%We may assume that the corners are not adjacent. If a nonadjacent
%corner $w$ is visible from $v$, then $|w-v|\ge3.2$, and an
%interior point intersection $p$ is incompatible with the triangle
%inequality: $|p-v|\le 1.6$, $|p-w|<1.6$. If $w$ is not visible, we
%have a chain $v=v_0,v_1,\ldots,v_r=w$ such that $v_{i+1}$ is
%visible from $v_i$. Imagine a taut string inside the substandard component
%extending from $v$ to $w$. The radial projections of $v_i$ are the
%corners of the string's path.   The string bends in an angle
%greater than $\pi$ at each $v_i$, so the angle at each
%intermediate $v_i$ is greater than $\pi$. That is, they are
%concave. Thus, by our deformations $|v_i-v_{i+1}|\ge3.07$. The
%string has path-length at least $r \arc(T,T,3.07)>r (1.316)$.
%But the corner cells lie in cones of arcradius
%$\arc(2,1.255,\lambda)< 1$. So $2(1.0)>r(1.316)$, or $r=1$.  Thus,
%$w$ is visible from $v$.
%\end{proof}

 
\begin{lemma}\tlabel{lemma:4.12.4}
%%   \oldlabel{4.12.4}
%The corner cell for $\lambda \le 1.815$ does not meet at an
%interior point with the $1.255$-cone wedge around another corner
%$w$.
Let $(\orgn,V,E)$ be a fan.  Let $U$ be a component of $Y(\orgn,V,E)$.
Suppose that the set of darts leading into $U$ forms a simple face $F$.
Assume that $|v-w|\ge 3.07$ for all $v,w\in V$, distinct, such that
  \begin{itemize}
   \item $\op{aff}_+^0(\orgn,\{v,w\})\subset U$, and
   \item the dart at $v$ in $F$ is concave.
  \end{itemize}
Assume that $|v-w|\ge 3.2$ for $v,w\in V$, distinct, such that
  \begin{itemize}
   \item $\op{aff}_+^0(\orgn,\{v,w\})\subset U$, and
   \item $\{v,w\}\not\in E$.  % NB. no specials.
  \end{itemize}
Assume that $2\le |v-v_0|\le 2.51$ for all $v\in V$.
Let $x,x'$ be darts of $F$ that are not adjacent, appearing at $v,v'\in V$.  
Then
$$
V(x,\epsilon(v,\lambda))\cap V(x',\epsilon(v',1.255)) = \emptyset,
$$
where $\epsilon(u,h) = \cosarc(|u-\orgn|,1.255,h)$ and $\lambda=1.815$.
\end{lemma}


\begin{proof}
As in the previous proof, if the sets meet, then
    $$\arc(2,1.255,\lambda) +\arc(2,1.255,1.255) > r \arc(2.51,2.51,3.07),$$
and again $r=1$.  So each of the two vertices in question is visible
from the other. By tarski~\tarf{tarski:cone-meet}, 
$$
3.07 \le |v-v'| < 1.815 + 1.255 = 3.07.
$$
\end{proof}

\begin{lemma}\tlabel{lemma:4.12.5}
%%    \oldlabel{4.12.5}
%The corner cell for $\lambda \le 1.945$ at a corner $v$ satisfying
%$|v-\orgn|\ge2.2$ does not meet at an interior point with the $1.255$-cone
%wedge around another corner $w$.
Let $(\orgn,V,E)$ be a fan.  Let $U$ be a component of $Y(\orgn,V,E)$.
Suppose that the set of darts leading into $U$ forms a simple face $F$.
Assume that $|v-w|\ge 3.2$ for all $v,w\in V$, distinct, such that
  \begin{itemize}
   \item $\op{aff}_+^0(\orgn,\{v,w\})\subset U$, and
   \item the dart at $v$ in $F$ is concave.
  \end{itemize}
Assume that $|v-w|\ge 3.2$ for $v,w\in V$, distinct, such that
  \begin{itemize}
   \item $\op{aff}_+^0(\orgn,\{v,w\})\subset U$, and
   \item $\{v,w\}\not\in E$.  % NB. no specials.
  \end{itemize}
Assume that $2\le |v-v_0|\le 2.51$ for all $v\in V$.
Let $x,x'$ be darts of $F$ that are not adjacent, appearing at vertices
$v,v'$.  Assume $2.2\le |v-\orgn|$.  Then
$$
V(x,\epsilon(v,\lambda))\cap V(x',\epsilon(v',1.255)) = \emptyset,
$$
where $\epsilon(u,h) = \cosarc(|u-\orgn|,1.255,h)$ and $\lambda=1.9$.
\end{lemma}

\begin{proof}
As in the previous proof, if the sets meet at $p$, then
    $$\arc(2,1.255,\lambda) +\arc(2,1.255,1.255) > r \arc(2.51,2.51,3.2),$$
and again $r=1$.  Then the desired contradiction 
follows from tarski\tarf{tarski:cone-meet}
    $$3.2\le |v-v'|<  1.945 + 1.255 = 3.2.$$
\end{proof}





