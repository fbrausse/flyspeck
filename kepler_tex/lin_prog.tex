%% June 2006.



\section{Introduction}

Until now, the most poorly redacted part of the proof of the
Kepler conjecture has been the linear programming.  The proof
involves approximately 100,000 linear programs, each involving
some 200 variables and 2000 constraints.  The data for these
linear programs fills about three gigabytes of storage.

The purpose of this document is to give a specification of the
linear programs. This specification has never been produced,
except in highly abbreviated form in the original 1998 proof. The
linear programs are left so vague that it would be extremely
difficult from that description alone to give independent
corroboration of the results.

The original proof appeared as a series of papers named {\it
Sphere Packings I} through {\it Sphere Packings VI}.  This paper
may be viewed as {\it Sphere Packings VII}.

\subsubsection{Flyspeck}

An important part of the motivation for the work comes from the
Flyspeck project, which aims to give a formal proof of the Kepler
conjecture.  The word flyspeck has the form F*P*K, which is an
acronym for the Formal Proof of the Kepler conjecture.

Steven Obua has made important progress toward the formal
verification of linear programs, as needed for the Flyspeck
project.  What he lacks to complete the project is a specification
of the linear programs in a form suitable for formal verification.
This document is the first step toward bridging that gap.

\subsubsection{Geometry}

This document avoids all geometry.  Instead we work at a purely
combinatorial level.    A hypermap (which is just a finite set
with two permutations) encodes most of the combinatorial
structure.  The rest is encoded as discrete collection of flags.
The linear programs can be specified directly from the hypermap
and flags.  There is a database of a few thousand hypermaps that
are used in the construction of the systems of inequalities.  The
flags are abstractions of various geometrical conditions, and yet
it is not necessary to understand anything about these geometrical
conditions to work with the flags.

\subsubsection{Formal Inequalities}

There is a tight relationship between the linear inequalities that
appear in the linear programs and certain nonlinear inequalities
that appear in a database.  A simple example is illustrated in
Figure~\ref{XX}.  The three angles $\alpha_i$ of the triangle
satisfy
    \begin{equation}
    \label{eqn:2pi}
    \alpha_1 + \alpha_2 + \alpha_3 = 2\pi.
    \end{equation}
This equality can be viewed two ways.  It can be viewed linearly
as the equation of an affine plane in $\ring{R}^3$.  Or, it can be
viewed nonlinearly, where each $\alpha_i$ is a nonlinear function
of the edge lengths of the triangles, and the equation asserts a
nonlinear relation among edge lengths.

For humans it is a simple matter to switch back and forth between
these points of view.  To understand the 1998 proof of the Kepler
conjecture, it is necessary to jump continually back and forth
between these two views of an equation.

For computers, it is not such a simple matter to jump from one
point of view to another.  To assist in the full automation of the
proof, we create a new object called {\it formal inequalities}
whose purpose is to provide the connection between the two types
of inequalities.  A formal inequality can be {\it deformalized}
into an ordinary linear inequality.  It can also admits a
nonlinear {\it interpretion}.  A formal inequality thus appears as
a structure that unifies a linear inequality with a nonlinear
inequality (Figure~\ref{fig:formal}). With this new structure, we
have the framework that would allow us to completely automate the
process of retrieving linear programs from a database of nonlinear
inequalities.

\begin{figure}[htb]
  \centering
  %% XX \myincludegraphics{\ps/formal.eps}
  \caption{Unification of nonlinear and linear inequalities}
  \label{fig:formal}
\end{figure}

\subsubsection{Infeasibility}

To prove the Kepler conjecture, it is enough to show that no
counterexamples exist.  In technical terms, it is enough to show
that no contravening centered packings exist (except for those
attached to the face-centered cubic and hexagonal close packings).

The strategy for proving the nonexistence of counterexamples is
remarkably naive. In simple terms, it goes as follows.  First, put
all the counterexamples in a box. Then measure the width of the
box. If the width of the box is negative, the box contains no
counterexamples.

This basic idea can be embellished in various ways.  The box can
be replaced by more general polyhedra defined by hyperplane
inequalities.  Instead of showing that the box has negative width,
we can show that the system of hyperplane inequalities admits no
feasible solutions.

A further refinement of this idea allows us to cover the set of
counterexamples with a finite collection of polyhedra and then
show that each polyhedron is empty.  We can do this in with an
iterative refinement.  If a given cover of the space of
counterexamples contains a polyhedron that is not empty, we can
replace it with two others that give a finer cover of the space of
counterexamples.  If by a series of  refinements we eventually
arrive at a cover by empty polyhedra, our goal (of eliminating all
counterexamples) is accomplished.

In the original 1998 proof, the series of refinements was guided
by detailed geometrical information about the space of centered
packings.  In this revision of the proof, the series of
refinements is purely algebraic and combinatorial.

\subsubsection{Linear Programs}

This idea of covering the space of counterexamples with a finite
collection of empty polyhedron can be expressed somewhat
differently in terms of linear programming.  In fact, it is
possible to describe the problem in such a way that no reference
is made to the space of counterexamples.

The basic aim of the linear program is to show the infeasibility
of a certain system.  However, the primary linear programs are not
infeasible in the traditional sense.  It is necessary to weaken
the definition of infeasibility to encompass systems that
eventually become infeasible after certain branch and bound
operations.  These operations are somewhat subtle, because the
underlying combinatorial structure of the linear program changes
as branching operations progress.   In Section~\ref{sec:weak}, we
will say that a system is weakly infeasible, if a combinatorial
tree of systems can be constructed starting with the given system
as the root, and infeasible systems as the outermost leaves.  The
hypermap and its flags encode the possible branchings of the tree.
The main theorem of this document will then assert that each
primary system is weakly infeasible.

This weak infeasibility result is one of the main steps of the
proof of the Kepler conjecture.  Another document will show that
if there is a counterexample to the Kepler conjecture, then it can
be used to construct a strongly feasible solution to a primary
system.  The existence of a strongly feasible solution is
incompatible with weak infeasibility.  Thus, weak infeasibility
leads to a proof that there are no counterexamples to the Kepler
conjecture.

\subsubsection{Functional Language}

Our aim has been to give a description of the linear programming
part of the Kepler conjecture in a way that is conducive to
implementation in a functional programming language.  We have
allowed ourselves to drift somewhat from conventional mathematical
syntax toward expressions that have a bit more the look and feel
of a functional programming language.  We note that there are
scarcely any theorems in this document.  This is actually a highly
desirable thing, from the point of view of formalization.

\subsubsection{Modular design}

In a long and complex proof, it is desirable to modularize and
encapsulate the various parts of the proof to the greatest extent
possible.  We have been able to make this document almost entirely
self contained: it can be read with almost no reference to other
chapters in this book.  To achieve this, we have repeated a few
results from earlier chapters, such as material on hypermaps.
There are a few outside facts that we make reference to, such as
the existence of a database of nonlinear inequalities, and the
existence of an archive of planar hypermaps.  For the purposes of
this document, is not necessary to understand the origin of this
database or this archive. They can simply be accepted as part of
the givens of our situation.

The main theorem (Theorem~\ref{thm:lpbound}) of this document is
one of the main parts of the proof of the Kepler conjecture.
Although the theorem seems at first glance to be rather simple,
this apparent simplicity is deceptive, because the definitions on
which the theorem rests, are extremely long and cumbersome.  This
was the price of encapsulation.

\subsubsection{Warning and Hope}

There are some minor incompatibilities between how things are set
up here and how things were set up in the original 1998 proof of
the Kepler conjecture.  These incompatibilities were introduced to
simplify matters.  My hope is that they have not broken anything.
However, the only way to be sure of this is to generate all the
linear programs and check that they all hold.  I have not carried
this out, so the reader is encouraged to be a bit skeptical of
everything that is written here, at least until those
verifications have been made.

As part of the Flyspeck project, it is our hope that one day there
will be a formal proof of the main theorem of this document.  S.
Obua can provide details about the current status of this project.

\section{Review of Hypermaps}

The combinatorial structure of the Kepler conjecture is expressed
through various hypermaps.   This chapter takes
the connected sums of hypermaps, as described in Section~\ref{sec:csum}.

%
%\subsection{Basic Definitions}
%
%\begin{definition}  A {\it hypermap} $H=(D,e,n,f)$ is a finite set $D$
%together with three permutations $e,n,f:D\to D$ satisfying the
%identity $e\circ n\circ f = I$.  The elements of $D$ are called
%{\it darts}.  The permutations $e,n,f$ are called the {\it edge},
%{\it node}, and {\it face} permutations, respectively.
%\end{definition}
%

\begin{definition}  A {\it face} of a hypermap $H=(D,e,n,f)$ is an orbit of $D$
under $f$.  We write $\op{face}(\alpha)$ for the face of
$\alpha\in D$.  We write $H/f$ for the set of faces.
\end{definition}

\begin{definition}  An {\it edge} of a hypermap $H=(D,e,n,f)$ is an orbit of $D$
under $e$.  We write $\op{edge}(\alpha)$ for the edge of
$\alpha\in D$.  We write $H/e$ for the set of edges.
\end{definition}

\begin{definition}  A {\it node} of a hypermap $H=(D,e,n,f)$ is an orbit of $D$
under $n$.  We write $\op{node}(\alpha)$ for the node of
$\alpha\in D$.  We write $H/n$ for the set of nodes.
\end{definition}


\section{Linear Programming}


\subsection{Preliminaries}

There is a vast mathematical literature describing algorithms to
solve the constrained optimization problem of finding a vector
$x\in\ring{R}^n$ that minimizes the objective function
    $$x\mapsto c\cdot x$$
where $c\in\ring{R}^n$ subject to a system of linear inequalities,
$A x\le b$ for some real-valued matrix $A$ and vector $b\in
\ring{R}^m$.  These algorithms have been implemented in various
software packages.

The basic linear program admits many variations.  We can maximize
the objective function
$$x \mapsto c\cdot x$$ rather than maximize it.
Any linear equation
    $a\cdot x = b$ can be written as a system of inequalities
        $$a \cdot x \le b \text{ and } -a \cdot x \le -b.$$
This allows us to reduce constrained linear optimization with
equality constraints to a constrained linear optimization with
inequality constraints.

Because of this extensive and well-developed literature on the
subject of linear programming, we will take it as given that we
can solve large-scale linear programs in a highly reliable way.

\subsection{Infeasibility}


The linear programming problems in this document are {\it
infeasibility} problems.  We assume a system of inequalities in
the variables $x$ of the form
    \begin{equation}
    \label{eqn:lpsys}
    \begin{array}{lll}
    A x &\le c\\
    a &\le x \le b,\\
    \end{array}
    \end{equation}
with explicitly given lower and upper bounds $(a\le x\le b)$ on
the variables $x$. (An inequality of vectors $a\le b$ means
component-wise inequality $a_i\le b_i$, for all $i$.) The problem
is to verify that the system is infeasible; that is, there does
not exist an $x$ satisfying the system of inequalities.

\begin{definition}
  A certificate of infeasibility for the system \ref{eqn:lpsys},
  is a pair $(u,v)$ satisfying
    $$
    \begin{array}{lll}
    0&\le u\\
    0&\le v\\
    0&\le u A + v\\
    0& > u(c-Aa) + v(b-a). \\
    \end{array}
    $$
\end{definition}

\begin{example}
If there is an index $i$ for which $b_i < a_i$, then the system is
clearly infeasible.  In this case, we have an obvious certificate
of infeasibility given by $(u,v)=(0,e_i)$, where $e_i$ is the
standard basis vector at index $i$.
\end{example}

\begin{lemma}
  If a certificate of infeasibility $(u,v)$ exists for the system
  \ref{eqn:lpsys}, then the system is infeasible.
\end{lemma}

\begin{proof}
    Suppose for a contradiction that $x$ is a feasible solution.
    Then
    $$
    \begin{array}{lll}
    0 &\le (u A + v)(x-a) \\
      &= [u (c- A a) + v (b- a)] - [u (c - A x)] - [v (b - x)]\\
      &< 0.
    \end{array}
    $$
\end{proof}

To prove a system infeasible, it is not necessary to know how a
certificate of infeasibility is produced.  It can be produced, for
instance, by an untrusted computer algorithm.  All that is needed
to be known is that it is a certificate.

If an approximation $(u',v')$ to a certificate of infeasibility is
produced (say by computer), it can often be adjusted to give a
true certificate.  Given an approximation $(u',v')$, define
$(u,v)$ by
    $$
    \begin{array}{lll}
    u &= \max(0,u')\\
    v &= \max(0,v',-u A)\\
    \end{array}
    $$
If $(u,v)$ satisfies
    $$u( c-A a) + v(b-a) <0,$$
then it is a certificate of infeasibility.

The theory of duality for linear programming insures that if a
system is infeasible, then a certificate of infeasibility exists.
Since the equations defining a certificate of infeasibility are
linear in $u$ and $v$, linear programming algorithms may be used
to produce certificates:
    $$
    \begin{array}{rll}
        \min_{u,v} &u (c-A a) + v(b-a) \text{ such that }\\
        0 &\le u\\
        0 &\le v \\
        0& \le u A + v\\
    \end{array}
    $$
This has a feasible solution $u=v=0$.  If we add the constraint
that the objective function is at least $-1$, then there exists a
bounded feasible solution.

\begin{remark}
\label{rem:bounded}
If explicit upper bounds $b_j$  are not known
for particular variables $x_j:j\in J$, then we can search for a
certificate $(u,v)$ that satisfies $v_j = 0:j\in J$.  This
eliminates the dependence on $b_j$.  If explicit lower bounds
$a_j$ are not known for particular variables $x_j:j\in J$, then we
can search for a certificate that satisfies $v_j = 0:j\in J$ and
$u A e_j = 0: j\in J$. This eliminates the dependence on the
variables $x_j:j\in J$. Again, this augmented system has a
feasible solution $u=v=0$, and we can impose a constraint to make
the objective function bounded.
\end{remark}

\subsection{The Abstract Theory}

We assume a finite indexing set $I$ that will be used  to index
all the variables for a linear program. Let $\ring{R}^I$ be the
vector space of real-valued functions on $I$. We define the
standard dot product $(\cdot)$ on $\ring{R}^I$ by
    $$v\cdot w = \sum_{i\in I} v_i w_i.$$

\begin{definition}
A formal inequality (on the indexing set $I$) is a triple
    $$(f:I\to\ring{R},s\in\{(\le),(=),(\ge)\},c\in\ring{R}).$$
The second coordinate $s$ is a binary relation on $\ring{R}$.
\end{definition}

Let $\op{FOR}_I$ be the set of formal inequalities.  We drop the
subscript $I$ when the indexing set $I$ is fixed.  There is a
function $\op{def}:\op{FOR}\to \ring{R}^I \to \bool$ (short for
deformalize) given by
    $$\op{def}:\ (f,s,c)\ v \mapsto s(f\cdot v,c).$$

\begin{remark}
We use the shorthand of writing $(f,\op{sgn},c)$ as
   $$f\,\,\op{{\bf sgn}},\, c,$$
where $\op{{\bf sign}}$ is the curly version $(\seq,\sle,\sge)$ of
the relation symbol $(=,\le,\ge)$. For example, the formal
inequality
    $(\optt{y}(\alpha),(\ge),0)$ is written simply as $\optt{y}(\alpha)\sge 0$.
The curly relation symbol reminds us that the inequality is
formal.  Without the modified relation symbol, this shorthand
would be dangerously ambiguous, because a formula such as
    $$\optt{y}(\alpha) = 0$$
could be read either as a formal identity
$(\optt{y}(\alpha),(=),0)$, or as an ordinary vector equality
between the vector $\optt{y}(\alpha)$ and the zero vector $0$.

We generalize the notation to allow statements such as
    $$v \seq w$$
for the formal inequality $(v-w,(=),0)$.
\end{remark}

%The image of $\op{def}$ on a formal inequalities is called an
%inequality (or equality if we wish to stress that it comes from an
%equality constraint). Let $\op{INEQ}$ be the set of inequalities.

\begin{definition}
A formal disjunction $A$ is a finite set of finite sets of
$\op{FOR}_I$. (Intuitively, the set is to be thought of as
representing a formula in disjunctive normal form, whose atomic
formulas are formal inequalities.) Let $\op{LFOR}_I$ be the set of
formal disjunctions. Again, we drop the subscript $I$, when the
indexing set $I$ is fixed.  We may consider a formal inequality as
a formal disjunction under the map
    $$a \mapsto \{\{a\}\}.$$
\end{definition}

\begin{remark}
The formal disjunction
    $$A = \{\{a_{11},a_{12},\ldots,\},\ldots,\{a_{k1},\ldots\}\}$$
will be written more suggestively as
    $$A = (a_{11}\wedge a_{12}\wedge\cdots) \vee \cdots \vee (a_{k1}\wedge
    \cdots)$$
as a formula in disjunctive normal form.  For example, the formal
disjunction
    $$A =  \{\{(y,(\le),0)\},\{y,(\ge),0\}\}$$
will be written
    $$A = (y\sle 0) \vee (y \sge 0).$$
If $A$ is a singleton: $A = \{\{\{a_{11},a_{12},\ldots,\}\}$ then
we write it as a conjunction:
    $$
    A = (a_{11}\wedge a_{12}\wedge\cdots)
    $$
\end{remark}

We extend $\op{def}$ to a function $\op{ldef}$ on $\op{LFOR}$ by
    $$\op{ldef}\ A \ v  =
    (a'_{11}\wedge a'_{12}\wedge\cdots) \vee \cdots \vee (a'_{k1}\wedge
    \cdots).$$
where $a'_{ij} = \op{def}(a_{ij}) v$ and $A\in \op{LFOR}$ has the
form
    $$
    A = \{\{a_{11},a_{12},\ldots,\},\ldots,\{a_{k1},\ldots\}\}$$

If $S$ is a set of formal disjunctions, let $\op{single}(S)$ be
the union of singleton sets (that is, conjunctions):
    $$
    \op{single}(S) = \bigcup \{a \mid \{a\} \in S\}
    $$

\begin{definition}
Let $S$ be a set of formal disjunctions (on an indexing set $I$).
We say that $S$ is {\it infeasible} and write
    $$\op{INFEAS}(S)$$
if $\op{single}(S)$ is an infeasible system of linear
inequalities.
\end{definition}


\subsection{Inequalities from a Database}
\label{sec:lookup}

There is a large database of nonlinear inequalities in a small
number of variables that has been proved correct by means of
interval arithmetic.  This subsection indicates how to extract
linear inequalities from that database to be used for linear
programming.

%Suppose that we are given a set of  configurations $C$ and wish to
%generate inequalities that are valid on a subset $S\subset C$.

We assume that the indexing set $I$ contains a subset $I'$ (called
the set of nonlinear indices). Assume a nonlinearization function
    $$
    \op{interp}:\ I' \to \ring{R}^{I} \to \ring{R}.
    $$
Extend it to
    $$
    \begin{array}{rll}
    \op{interp\_vec}:&\ \ring{R}^{I} \to \ring{R}^I \to
    \ring{R}\\
    u\ v \mapsto& \sum_{i\in I'} v_i (\op{interp}_i u) +
        \sum_{j\in I\setminus I'} v_j u_j.
    \end{array}
    $$
Extend it to
    $$
    \begin{array}{rll}
    \op{interp\_for}:&\ \op{FOR}\to\ring{R}^I\to\bool\\
    (f,s,c)\ u\mapsto& s(\op{interp\_vec} f u,c)\\
    \end{array}
    $$
Finally, extend it to
    $$
    \begin{array}{rll}
    \op{interp\_lfor}:\ \op{LFOR}\to&\ring{R}^I\to\bool\\
    ((a_{11}\wedge a_{12}\wedge\cdots) \vee \cdots \vee (a_{k1}\wedge
    \cdots)\ u\mapsto& (a'_{11}\wedge a'_{12}\wedge\cdots) \vee \cdots \vee (a'_{k1}\wedge
    \cdots)\\
    \end{array}
    $$
where $a'_{ij} = \op{interp\_for} (a_{ij}) u$.


Let $\op{nonlin}(C,L)$ be the formula
    $$
    \forall y\in\ring{R}^\ring{N}.\ (\forall c\in C.\
    c(y)) \Rightarrow L(y).
    $$
Suppose that we have a database $B$ of such nonlinear
inequalities.  Given a map $\psi:\ring{N}\to I$, there is a dual
map
    $$
    \pi_\psi: \ring{R}^I \to \ring{N}\to \ring{R},\quad v\ n\mapsto
    v_{\psi n}.
    $$


\begin{definition}
    We say that the formal disjunction $A$ is a {\it lookup} in
    database $B$ with domain constraints in $S$ (a set of formal disjunctions),
    if there exists
        $$
        \op{nonlin}(C,L)\in B
        $$
    and there exist $\psi:\ring{N}\to I$ and
    $r:C\to\op{FOR}_I$
    such that the following two conditions hold
    \begin{itemize}
    \item The system $S$ implies the domain restrictions of
    $\op{nonlin}(C,L)$.  That is, for every $c \in C$,
        $$
        \forall v\in\ring{R}^I.\ \op{interp\_for} (r(c)) v
        \Rightarrow (c(\pi_\psi v))
        $$
    \item The predicate $L$ comes by interpretation of $A$.  That
    is,
        $$
        \forall v\in \ring{R}^I.\ (\forall c\in C.\ \op{interp\_for}(r(c)) v)
        \Rightarrow (L(\pi_\psi v) =
        \op{interp\_lfor}(A) v).
        $$
    \end{itemize}
\end{definition}


In brief, a formal disjunction $A$ is a lookup of
$\op{nonlin}(C,L)$ if the interpretation of $A$ coincides with the
predicate $L$ on a suitably restricted domain, and if those domain
constraints are in the system $S$.

It should be possible to write a completely automated tool that
finds the lookups from a database $B$ whose domain constraints lie
in $S$.

\begin{example}  We give a somewhat contrived example of lookup.
Later, nonlinear lookup will be performed on a massive database of
inequalities.  Suppose that our database contains the nonlinear
inequality:
    $$\forall y_1,\ldots,y_6.\ (2\le y_i \le 2.1,\ i=1,\ldots
    6) \Rightarrow (\Gamma(y_1,\ldots,y_6) \le \pt),
    $$
    for some nonlinear function $\Gamma$ and constant $\pt$.
We can put this in the form $N = \op{nonlin}(C,L)$ with
    $$
    \begin{array}{llll}
    %J &=\{1,\ldots,6\}\\
    C &= \{\\
        &y&\mapsto (y_1\ge 2)\\
        &y&\mapsto (y_1\le 2.1)\\
        &&\cdots\\
        &y&\mapsto (y_6\le 2.1)\}\\
    L &= (y&\mapsto (\Gamma(y_1,\ldots,y_6)\le \pt)).
    \end{array}
    $$
Suppose, further, for the sake of an example that our indexing
sets contain elements $g\in I'$ and $i_1,\ldots,i_6\in I\setminus
I'$ and that
    $$
    \op{interp}(g) = (v \mapsto \Gamma(v_{i_1},\ldots,v_{i_6})).
    $$
Let $\optt{gamma} = e_g \in\ring{R}^{I}$. Then we see that the
formal disjunction (in this case a single formal inequality)
$\optt{gamma} \sle \pt$ is a lookup in the database $B = \{N\}$
with domain constraints in
    $$
    S = \{e_{i_1}\sge 2,\ldots,e_{i_6}\sge 2,
          e_{i_1}\sle 2.1,\ldots, e_{i_6}\sle 2.1\},
    $$
\end{example}


\section{Hypermap Systems}

The previous sections have described the general theory of linear
programming without reference to the specifics of the Kepler
conjecture.  It is now time to develop linear programming in the
specific context of  hypermaps.

\subsection{Statement of the Main LP Bound}

Let $\mathcal H$ be the set of  hypermaps that are connected,
plain, planar, and simple. For each $H\in \mathcal H$, there is a
set of combinatorial flags $\op{fl}$.  This set is defined in
Section~\ref{sec:flag}.  Section~\ref{sec:var-index} will specify
an indexing set $I$, a subset of nonlinear indices $I'\subset I$,
and an interpretation
  $$\op{interp}:\ I'\to \ring{R}^I\to\ring{R}$$
for $I'$.  The data $I$, $I'$, and $\op{interp}$ depend on the
parameters $(H,\op{fl})$: $I = I(H,\op{fl})$, etc.

\begin{definition}
A {\it hypermap system} is a triple $(H,\op{fl},S)$ where
    $H\in {\mathcal H}$,
    $\op{fl}$ is a
    combinatorial flag,
    and $S\subset \op{LFOR}_I$, with $I = I(H,\op{fl})$.
    Let $\op{HS}$ be the set of hypermap systems.
\end{definition}

Section~\ref{sec:PTHS} defines a subset $\op{PTHS}$ of $\op{HS}$
(the set of primary tame hypermap systems).  There is a database
of nonlinear inequalities (B) that is described in
Section~\ref{XX}. Section~\ref{sec:basic} will describe the
systems $S$ for $(H,\op{fl},S)\in\op{PTHS}$.  A predicate
$\op{weak\_infeas}$ (weak infeasibility) on hypermap systems
$\op{HS}$ will be described in Section~\ref{sec:weak}.

\begin{theorem}[Main LP Bound]\label{thm:lpbound} Assume the nonlinear
inequalities (B).  Then for every $\rho\in \op{PTHS}$, we have
$\op{weak\_infeas}(\rho)$.
\end{theorem}

Briefly put, every primary tame hypermap system is weakly
infeasible.  The result is obtained by linear relaxation and
linear programming.\footnote{The warning that appears at the
beginning of the document should be repeated.  There are some mild
incompatibilities with the linear programming verifications done
in 1998, and the new linear programs have not yet been executed.
Thus, it is possible that some adjustments may have to be made
before claiming the Main LP Bound as a theorem.}

\begin{remark}  Each of the nonlinear inequalities in the database
(B) was proved by interval arithmetic as part of the 1998 proof.
Thus, we can state the Main LP Bound with this hypothesis.  The
hypothesis is included simply to keep interval arithmetic out of
the proof.
\end{remark}

\subsection{Flags}
\label{sec:flag}

The second component of a  hypermap system $(H,\op{fl},S)$ is a
combinatorial flag.  This subsection describes the flag and some
auxiliary functions.

\begin{definition}
 A flag is a tuple
    $$
    \op{fl}=(\op{qrn},\op{qre},\op{hex\_loop\_diag},
    \op{typ}(4),\op{sum\_typ}(4),\ldots,
    \op{typ}(8),\op{sum\_typ}(8))
    $$
 with flag components $\op{qrn},\ldots,\op{sum\_typ}(8)$ to be listed below.
 \end{definition}

We give a brief description of each of the flag components.  The
interpretations of the flag components are irrelevant to the proof
of the Main LP Bound.  Interpretations are merely provided as a
guide to the intuition.  All that is relevant is the domain and
range of each function.  For instance, in the following
definition, the comment that $\op{qrn}$ ``detects quasi-regular
nodes'' is irrelevant.  Any function $\op{qrn}:H/n\to\bool$
qualifies as a flag component.

 \begin{definition}
 The flag components $\op{qrn}$ and $\op{qre}$ detect {\it quasi-regular
 nodes and edges}.
        $$\op{qrn}:H \to \bool$$
        $$\op{qre}:H \to \bool$$
 The function $\op{qrn}$ is constant on orbits of $n$, so that it
 descends to a function on $H/n$.  The function $\op{qre}$ is
 constant on the orbits of $e$, so that it descends to a function
 on $H/e$.
\end{definition}

\begin{definition} The function $\op{std}$ detects standard faces
in a hypermap.
    $$\op{std}:H/f \to \bool$$
$$\op{std}(F) = \forall\alpha\in F.\
\op{qrn}(\alpha)\wedge \op{qre}(\alpha)
$$
\end{definition}

\begin{notation}
We write $(H/f)_X$ to denote the subset of $H/f$ described by the
satisfying the given flags in $X$.  If the subscript is a natural
number, we let $(H/f)_n$ be the set of faces in $H$ of cardinality
$n$.  For example, $(H/f)_{4,\op{std}}$ is the set
    $$
    \{F \in H/f \mid \card(F) = 4 \wedge \op{std}(F)\}.
    $$
\end{notation}

%\begin{definition}
%The flag component $\op{pri}:(H/f)_{\op{std}}\to\bool$ tells
%whether a standard face is primary.
%\end{definition}

%\begin{definition}
%The flag component $\op{is\_refined}$ keeps tracks of changes to
%an initial configuration hypermap.
%   $$\op{is\_refined}:H/f\to \bool$$
%\end{definition}

In the following definitions, we give symbolic names to elements
of various sets.  For example, in the following definition
    $$\{\op{patchable},\op{puremix}\}$$
is to be understood as a finite set of cardinality two, with
elements $\op{patchable}$ and $\op{puremix}$.  Also, some of the
values depend on parameters.  For example, in the following
definition, if $F=\{\alpha_1,\ldots,\alpha_4\}$, then
    $$
    \op{sum\_typ}(4)(F)
    $$
can take on five distinct values:
    $$
    \op{upright}, \op{flat}(\alpha_1), \ldots,
    \op{flat}(\alpha_4).
    $$
This is indicated in abbreviated form in the following
definitions.

\begin{definition}
The flag components $\op{typ}(4)$ and $\op{sum\_typ}(4)$
classify the type of a standard quad face.
    $$
    \begin{array}{rll}
    \op{typ}(4):& F\in (H/f)_{4,\op{std}}
    \to
    \{\op{patchable},\op{puremix}\}\\
    \op{sum\_typ}(4):& F\in (H/f)_{4,\op{patchable}}
    \to
    \{\op{upright},\op{flat}(\alpha:F)\}.
    \end{array}
    $$
\end{definition}

\begin{remark}
The data for $\op{flat}(\alpha)$ and $\op{flat}(f^2\alpha)$ are
equivalence in a sense that I won't spell out in detail here.
Roughly, they both correspond to a connected sum obtained by drawing the
diagonal of a quadrilateral. In the 1998 proof, these two cases
were combined into a single proof. This creates fewer cases, but
leads to a proof obligation of proving their equivalence.  (This
is geometrically obvious, but it is not obvious in formal
verification.) Following Nipkow's work on the formalization of the
classification of tame plane graphs, it is my hunch that the
formal proof will go more smoothly if the proof obligation is
dropped, and the two are treated as separate cases.
\end{remark}

\begin{definition}
The flag components $\op{typ}(5)$ and $\op{sum\_typ}(5)$
classify the type of a standard pentagonal face.
    $$
    \begin{array}{rll}
    \op{typ}(5):& F\in (H/f)_{5,\op{std}}
    \to
    \{\op{patchable},\op{trunc},\op{aggA},\op{aggB},\op{crowd3},\op{notpri}\}\\
    \op{sum\_typ}(5):& F\in (H/f)_{5,\op{patchable}}
    \to
    \{\op{loop5},\op{flat}(\alpha:F),\op{flat2}(\alpha:F),\op{loop4}\}.
    \end{array}
    $$
\end{definition}
%% not primary is standard but occurring in a connected sum.

\begin{definition}
The flag components $\op{typ}(6)$ and $\op{sum\_typ}(6)$
classify the type of a standard hexagonal face.
    $$
    \begin{array}{rll}
    \op{typ}(6):& F\in (H/f)_{6,\op{std}}
    \to
    \{\op{patchable},\op{trunc},\op{agg},\op{crowd3},\op{crowd4},\op{loop51}\}\\
    \op{sum\_typ}(6):& F\in (H/f)_{6,\op{patchable}}
    \to
    \{\text{list connected sumes}\ldots\}.
    \end{array}
    $$
\end{definition}

%% XX list connected sumes 6,7,8

\begin{definition}
The flag components $\op{typ}(7)$ and $\op{sum\_typ}(7)$
classify the type of a standard heptagonal face.
    $$
    \begin{array}{rll}
    \op{typ}(7):& F\in (H/f)_{7,\op{std}}
    \to
    \{\op{patchable},\op{trunc},\op{crowd3},\op{crowd4}\}\\
    \op{sum\_typ}(7):& F\in (H/f)_{7,\op{patchable}}
    \to
    \{\text{list connected sumes}\ldots\}.
    \end{array}
    $$
\end{definition}

\begin{definition}
The flag components $\op{typ}(8)$ and $\op{sum\_typ}(8)$
classify the type of a standard octagonal face.
    $$
    \begin{array}{rll}
    \op{typ}(8):& F\in (H/f)_{8,\op{std}}
    \to
    \{\op{patchable},\op{trunc},\op{crowd3},\op{crowd4}\}\\
    \op{sum\_typ}(8):& F\in (H/f)_{8,\op{patchable}}
    \to
    \{\text{list connected sumes}\ldots,\op{agg51}\}.
    \end{array}
    $$
\end{definition}


\begin{definition} The function $\op{upq}$ detects upright
quarters in a hypermap.
    $$\op{upq}:\op{dart}(H) \to \bool$$
    $$\op{upq}(\beta) =  (\card(\op{face}\beta)=3)\wedge
      \forall\alpha\in \op{face}(\beta).\
         \op{qre}(\alpha)\wedge
         (\op{qrn}(\alpha)\Leftrightarrow
         (\beta\ne\alpha))$$
\end{definition}

\begin{definition}
The function $\op{flatq}$ detects flat quarters in a hypermap.
     $$\op{flatq}:H \to \bool$$
    $$\op{flatq}(\beta) = (\card(\op{face}\beta)=3)\wedge
       \forall\alpha\in \op{face}(\beta).\
         \op{qrn}(\alpha)\wedge
         (\op{qre}(\alpha) \Leftrightarrow
         (\beta\ne\alpha)).$$
\end{definition}
(Note: the dart $\beta$ on which $\op{flatq}$ is true is the one
for which $\op{edge}(\beta)$ is not quasi-regular; that is, the
diagonal.)


\begin{definition}
The function
    $$\op{inquad}:(H/n)\to \bool$$
gives whether an upright diagonal lies on upright quarter of type
$(4,0)$ inside a quad cluster.
    $$\op{inquad}(N) =
    (\card(N)=4) \wedge (\forall \alpha\in N.\
        \op{upq}(\alpha))
    $$
\end{definition}

\begin{definition}
The function
    $$\op{quad\_diag}:\op{dart}(H)_{\neg\op{qre}} \to \bool$$
tells whether an edge that is not quasi-regular is the diagonal of
a quadrilateral.  The function $\op{quad\_diag}$ is constant on
orbits of $e$, so it descends to a function on a subset of $H/e$.
$$
    \op{quad\_diag}(\alpha) = \op{flatq}(\alpha)\wedge
    \op{flatq}(e\alpha).
$$
\end{definition}

\begin{definition}
The flag component
    $$
    \op{hex\_loop\_diag}:\op{dart}(H)_{\neg\op{qre}}\to\bool
    $$
tells whether an edge that is not quasi-regular is the diagonal of
a hexagon, along a slice in a loop.  The function
$\op{hex\_loop\_diag}$ is constant on orbits of $e$, so it
descends to a function on a subset of $H/e$.
\end{definition}


\begin{definition}
Let $\op{fqex}$ be the predicate that picks out a dart in flat
quarters that are not part of a quadrilateral pair.
\begin{equation}
    \begin{array}{llll}
    \op{fqex}(\beta) &= \exists F\in(H/f)_{3,\neg\op{std}}.\ \op{flatq}(\beta) &\wedge\\
        \beta\in F &\wedge
        \ \neg\op{quad\_diag}(\beta).
    \end{array}
\end{equation}
\end{definition}




\subsection{Bifurcation}



\begin{definition}
We define a bifurcation relation $\op{bif}(\rho,\rho',\rho'')$ on
triples of  hypermap systems.  Let $\rho=(H,\op{fl},S)$, and
similarly for $\rho'$ and $\rho''$ with appropriately primed
notation.  We say that $\rho$ {\it bifurcates} into $\rho'$ and
$\rho''$, and write $\op{bif}(\rho,\rho',\rho'')$ if there exists
a formal disjunction $A\in S$, and formal disjunctions $A',A''$
such that
    $$
    \begin{array}{rlll}
    H &= H' &= H''\\
    \op{fl} &= \op{fl}' &= \op{fl}''\\
        A &= A'&\cup\ A''\\
    S'\setminus  \{A'\} &\subset S &\cup\ \op{single}(S)\\
    S''\setminus  \{A''\} &\subset S&\cup\ \op{single}(S)\\
    \end{array}
    $$
\end{definition}
The simplest way to bifurcate is to break a disjunction into two
disjuncts, putting one disjunct into each of the two branches of
the bifurcation.



\subsection{Weak infeasibility}
\label{sec:weak}

We define a notion of infeasibility of  hypermap systems that is
broad enough to encompass branch and bound.  We describe the
predicate $\op{weak\_infeas}$ on the set of hypermap systems.

We define a tree structure that keeps track of the
branch-and-bound operations used in the search for infeasible
hypermap systems.  We define a type of an $\op{tree}$ by
    $$
    \begin{array}{lll}
    |\ \op{LEAF}&\\
    |\ \op{DATA} &\text{of } \alpha * \op{tree}\\
    |\ \op{NODE} &\text{of } \op{tree} * \op{tree}\\
    \end{array}
    $$
%
We define a predicate $\op{is\_data}$ on trees which is true
exactly when it has the form $\op{DATA}(\cdot,\cdot)$. We define a
relation $\op{head}:(\op{tree},\alpha)\to \bool$ by
    $$
    \begin{array}{lll}
    |\ \op{DATA}(\rho,s),\rho &\to \true\\
    |\  \op{\_} &\to \false
    \end{array}
    $$
Section~\ref{sec:refine} defines a binary relation
$\op{ref}(\rho,\rho')$ (called refinement) on hypermap systems. We
define a recursive predicate $\op{wft}$ (well-formed trees) on the
trees by
    $$
    \begin{array}{lll}
    |\ \op{LEAF} &\to\true\\
    |\ \op{NODE}(s,t) &\to \op{wft}(s) \wedge \op{wft}(t)
    \wedge \op{is\_data}(s) \wedge \op{is\_data}(t)\\
    |\ \op{DATA}(\rho,s) &\to \op{wft}(s) \wedge\\
        &\quad \text{case}\ s\ \text{of}\\
        &\qquad|\op{LEAF}\to\true\\
        &\qquad|\op{DATA}(\rho',s')\to \ \op{ref}(\rho,\rho')\\
        &\qquad|\op{NODE}(\op{DATA}(\rho',s'),\op{DATA}(\rho'',s''))\to
        \ \op{bif}(\rho,\rho',\rho'')\\
        &\qquad|\op{\_} \to \false\\
    \end{array}
    $$
In brief, in a well-formed tree, a child must be obtained from the
parent by bifurcating or refinement.
%
By abuse of notation, we write $\op{INFEAS}(H,\op{fl},S)$ for
$\op{INFEAS}(S)$.

We define a predicate $\op{infeas}$ (infeasible trees) on trees by
    $$
    \begin{array}{lll}
    |\ \op{LEAF} &\to \false\\
    |\ \op{DATA}(\rho,\op{LEAF}) &\to \op{INFEAS}(\rho)\\
    |\ \op{NODE}(s,s') &\to \op{infeas}(s) \wedge \op{infeas}(s')\\
    |\ \op{DATA}(\rho,s) &\to \op{infeas}(s)\\
    |\ \op{\_} &\to \false\\
    \end{array}
    $$
In brief, the data closest to the leaves must be infeasible
systems of equations.

\begin{definition}
    Let $\op{weak\_infeas}$ be the predicate on hypermap systems given by
        $$
        \op{weak\_infeas}(\rho) = \exists t.\
        \op{head}(t,\rho) \wedge \op{wft}(t) \wedge
        \op{infeas}(t).
        $$
\end{definition}

In the actual linear programming, the trees are not used directly.
Rather, all the needed properties of the function
$\op{weak\_infeas}$ are given by the following three lemmas.

\begin{lemma}
If $\op{bif}(\rho,\rho',\rho'',L)$, $\op{weak\_infeas}(\rho')$,
and
    $\op{weak\_infeas}(\rho'')$ then $\op{weak\_infeas}(\rho)$.
\end{lemma}

\begin{proof} If $t'$ and $t''$ are the trees for $\rho'$ and
$\rho''$, then use the tree $\op{DATA}(\rho,\op{NODE}(t',t''))$
for $\rho$.
\end{proof}

\begin{lemma}
If $\op{INFEAS}(\rho)$, then $\op{weak\_infeas}(\rho)$.
\end{lemma}

\begin{proof} Use the tree $\op{DATA}(\rho,\op{LEAF})$.
\end{proof}

\begin{lemma}
If $\op{ref}(\rho,\rho')$ and $\op{weak\_infeas}(\rho')$, then
  $\op{weak\_infeas}(\rho)$.
\end{lemma}

\begin{proof} If $t'$ is a tree for $\rho'$, then for $\rho$ use
    $$
    \op{DATA}(\rho,t')
    $$
\end{proof}


\subsection{Variable indexing}
\label{sec:var-index}

We describe the finite set $I = I(H,\op{fl})$ of variable indices.
Extend notation by setting $I(\rho) = I(H,\op{fl})$, for $\rho =
(H,\op{fl},S)$.  If $e_i\in \ring{R}^I$ is a standard basis
vector, we let $\iota(e_i)=i\in I$ be the corresponding index.  We
generally give names to the basis vectors, and refer to the
corresponding indices by means of the function $\iota$.





\begin{definition}
Let $D$ be the set of darts of hypermap $H$.  The set $I$ (for
$(H,\op{fl})$ is defined as the disjoint union of the following
indexing sets:
    $$\begin{array}{llll}
        \text{index} &\text{basis vector name}&\text{namesake?}
        \\
        %
        \alpha\in D, &\optt{azim}(\alpha), &\text{yes}
        \\
        %
        \alpha\in D, &\optt{yn} (\alpha)
        \\
        %%
        \alpha\in D,
        &\optt{ye} (\alpha),
        \\
        %%
        F\in H/f,
        &\optt{sol} (F), &\text{yes}
        \\
        %%
        F\in H/f
        &\optt{sc} (F)
        \\
        %%
        F\in H/f
        &{\optt{tau\_sc}} (F)
        &\text{ }\\
        %%
        F\in (H/f)_{3,flq}
        &\optt{sigmahat} (F)
        &  \text{yes}\\
        %%
        F\in (H/f)_{3,flq}
        &\optt{tauhat} (F)
        &\text{yes}\\
        %%
        F\in (H/f)_{3,\neg\op{std}}
        &\optt{eta}(F)
        &\text{yes}\\
        %%
        \alpha\in H
        &\optt{eta}'(\alpha)
        &\text{yes}\\
        %%
        F\in (H/f)_{3,std}
        &\optt{sigma\_qrtet\_x} (F)
        &\text{yes} \\
        %%
        F\in (H/f)_{3,std}
        &\optt{tau\_sigma\_x} (F)
        &\text{yes} \\
        %%
        F\in (H/f)_{\ge4,\neg\op{std}}
        &\optt{vor0}(F)
        &\text{yes}\\
        %%
        F\in (H/f)_{\ge4,\neg\op{std}}
        &\optt{tau0}(F)
        &\text{yes}\\
        %%
        \alpha\in D
        &\optt{Adih}(\alpha)
        &\text{yes}\\
        %%
        \alpha\in D
        &\optt{quo}(\alpha)
        &\text{yes}\\
        %%
        \alpha\in D
        &\optt{quob}(\alpha)
        &\\
        %%
        \alpha \in \op{flatq}
        &\optt{mu}(\alpha)
        &\text{yes}\\
        %%
        \alpha\in \op{upq}
        &\optt{nu}(\alpha)
        &\text{yes}\\
        %%
        \text{[FINISH LIST]} %% XX FINISH
    \end{array}
    $$
\end{definition}

\subsection{Interpretation}

There are a number of interpretations of variables that have the
general form
    $$\op{interp}(\iota\optt{mango})\ v\mapsto
    \op{mango}(u_0\cdot v,u_2\cdot v,\ldots,u_k\cdot v)$$
for some nonlinear function $\op{mango}$, and vectors $u_i\in
\ring{R}^I$.  We typeset the name ($\optt{mango}$) of the basis
vector in $\optt{tt}$ font and typeset name of the nonlinear
interpreting function $\op{mango}$ as an operator. Greek symbols
are expanded as text to make the correspondence with function
names in the file
    {\it definitions\_kepler.ml} explicit. (Thus, the nonlinear
    function $\sigma$ appears
    as $\optt{sigma\_qrtet\_x}$ and so forth.)
When the correspondence between a basis vector and nonlinear
interpreting function occurs on a name-by-name basis in this way,
we will omit details.  We will simply say that it is a {\it
namesake} interpretation.

In namesake interpretations, the vectors $u_i$ are general tightly
related to the index $\iota\optt{func}\in I$.  Generally, if
$\alpha$ is a dart, $F=\op{face}(\alpha)$,  and $\card(F)=3$, and
if $\optt{func}(\alpha)$ has nonlinear namesake
$\op{func}(y_1\cdot v,\ldots,v_{i_6})$, then $u_i$ follow the
standard convention:
    $$
    \begin{array}{llll}
    u_i &= \optt{yn}_i(f^i\alpha), &i &=0,1,2\\
    u_i &= \optt{ye}_i(f\alpha), &i &=3,4,5
    \end{array}
    $$
This indexing convention is followed with the nonlinear functions:
    $$
    \optt{gamma},\ \optt{nu},\ \optt{mu},\ \text{list.}
    %
    $$
(N.B. This list is incomplete.) %% XX Complete and correct.


\subsection{Penalty Constants}
\label{sec:pc}


We define a function $\xi:(H/f)_{\neg\op{std}}\to\ring{R}$ as
follows.

If $\op{qrn}(\alpha)$ for all $\alpha\in F$, and $n=\card(F)\ge
4$, then we define $\xi$ as follows.  Let $k$ be the number of
edges of $F$ such that $\neg\op{qre}(\alpha)$.  Set
    $$
    \begin{array}{lll}
    \xi(F) & (n,k)\\
    6 \xi_\Gamma & (8,0)\\
    6 \xi_\Gamma & (7,1)\\
    4 \xi_\Gamma + 2 \xi_V & (6,2)\\
    2 \xi_\Gamma + 4 \xi_V & (5,3)\\
    0 & (4,4)\\
    6 \xi_\Gamma & (7,0)\\
    5 \xi_\Gamma & (6,1)\\
    3\xi_\Gamma + 2 \xi_V & (5,2)\\
    \xi_\Gamma + 4\xi_V & (4,3)\\
    2(0.008) & (6,0) \\
    0.008 & (5,1) \\
    0 & (4,2) \\
    0.008 & (5,0) \\
    0 & (4,1) \\
    0 & (4,0)
    \end{array}
    $$
%
\begin{itemize}
\item Set
    $\xi(F) = 3\xi_\Gamma$, if $F$ has the properties $\card(F)=5$
    and
    $$\forall\alpha\in F.\
    \op{qre}(\alpha)$$
    %
    $$
    \exists\beta\in F.\forall\alpha\in F.\
    \op{qrn}(\alpha)\Leftrightarrow
(\beta\ne\alpha).$$
%
%
\item Set $\xi(F) = \xi_\Gamma + 2\xi_V$, if $F$ has the
properties
    $\card(F)=4$ and
    $$\exists\beta\in F.\forall\alpha\in F.\
    \op{qre}(\alpha)\Leftrightarrow (\beta\ne\alpha)
    $$
    $$\exists\beta\in F.\ \op{qrn}(\alpha)\Leftrightarrow
    (\beta\ne\alpha).$$
\item Set
    $$
    \begin{array}{lll}
    \xi(F) & \card(F)\\
    3 \xi_\Gamma & 4\\
     \xi_\Gamma + 2\xi_V & 3\\
    \end{array}
    $$
if $F$ has the properties
    $$
    \exists\alpha\in F.\ \op{hex\_loop\_diag}(\alpha)
    $$
    $$
    \forall\alpha\in F.\ \op{qrn}(\alpha).
    $$
\item In all other cases, set $\xi(F)=0$.
\end{itemize}


\section{Inequalities}
%
\label{sec:basic}

This section gives a set $S=\op{basic}(H,\op{fl})$ of formal
disjunctions on the indexing set $I=I(H,\op{fl})$. Elements of
this set are call {\it basic formulas}.

The following are basic formulas.  The first formal inequality is
key.

\begin{equation}
    \sum_{F\in (H/f)} \optt{sc}(F) \sge 8\,\pt.
    \label{eqn:8pt}
\end{equation}

The next family of formal disjunctions allows quite arbitrary
bifurcations. For every $f\in\ring{R}^I$ and $c\in \ring{R}$, we
have basic formulas:
\begin{equation}
    (f \sle c) \vee (f \sge c).
\end{equation}

Some of the variables are constant on orbits:

\begin{equation}
    \begin{array}{lll}
  \optt{yn}(\alpha) &\seq \optt{yn}(n\alpha),\\
  \optt{ye}(\alpha) &\seq \optt{ye}(e\alpha),\\
  \optt{eta}'(\alpha) &\seq \optt{eta}'(e\alpha),\\
  \end{array}
\end{equation}


\begin{equation}
\optt{tau\_sc}(F) = \optt{sol}(F)\,\zeta\,\pt - \optt{sc}(F).
\end{equation}

Many of the basic formulas hold subject to various restrictions on
the indexing set.  These restrictions are indicated in a line
following the basic formula.  For example, the following basic
formula holds under the restriction that $F$ is a non-standard
face of cardinality at least four:
\begin{equation}
    \label{eqn:xi}
    \begin{array}{lll}
    \optt{sc}(F) \sle \optt{vor0} + \xi(F),\\
    &F\in (H/f)_{\ge4,\neg\op{std}}
    \end{array}
\end{equation}

\begin{equation}  % Type potA along a hex loop.
    \begin{array}{lll}
    \optt{sc}(F) &\sle \optt{vor0} + \xi(F),\\
    &&F\in (H/f)_{3,\exists\alpha\in F.\ \op{hex\_loop\_diag}(\alpha)}
    \end{array}
\end{equation}

\begin{equation}  % DCG 25.6.1 page 250, one flat quarter.
    \begin{array}{lll}
    \optt{sc}(F) & \sle (s_{1+\card(F)} - Z(3,1)) + \xi_\Gamma + 2 \xi_V,\\
    &&F\in (H/f)_{\ge6}\\
    &&\exists!\alpha\in F.\ \neg\op{qre}(\alpha)
    \end{array}
\end{equation}

\begin{equation}  % DCG 25.6.1 page 251, one flat quarter.
    \begin{array}{llll}
    \optt{yn}(\beta)&\sge 2.2&\vee\\
    \optt{ye}(\alpha)&\sge 2.7&\vee\\
    \optt{sc}(F) & \sle (s_{1+\card(F)} - Z(3,1)),\\
    &&&\neg\op{qre}(\alpha),\\
    &&&\beta=f^2e\alpha,\\
    &&&\alpha\in F\in (H/f)_{\ge6}\\
    &&&\exists!\alpha\in F.\ \neg\op{qre}(\alpha)
    \end{array}
\end{equation}

%% Two flat quarters, DCG 25.6.2, page 252.
\begin{equation}
    \begin{array}{lllll}
    \optt{yn}(\beta')&\sge 2.2&&\vee\\
    \optt{yn}(\beta')&\sge 2.2&&\vee\\
    \optt{ye}(\alpha')&\sge 2.7&&\vee\\
        \optt{ye}(\alpha')&\sge 2.7&&\vee\\
    (\optt{sc}(F) & \sle (s_{2+\card(F)} - 2 Z(3,1))&\wedge \\
    \optt{tau\_sc}(F) & \sge (t_{2+\card(F)} - 2 D(3,1))&\wedge \\
    \optt{sc}(F) & \sle \optt{vor0} + 3\xi_\Gamma), \\
    &&&&\neg\op{qre}(\alpha),\\
    &&&&\beta=f^2e\alpha,\ \beta'=f^2e\alpha',\\
    &&&&\alpha\ne\alpha'\in F\in (H/f)_{\ge5}\\
    &&&&\forall\gamma\in F.\ \op{qre}(\gamma)\Leftrightarrow (\gamma\ne\alpha,\alpha')\\
    \end{array}
\end{equation}

%% DCG 25.6.2, page 252, loop of type (n,k) = (4,2), or not.
%% Constants from table 25.1.
\begin{equation}
    \begin{array}{lllll}
    \optt{sc}(F) + \optt{sc}(\op{face}(\beta)) +
    \optt{sc}(\op{face}(\beta')) &\sle -0.1999&\wedge \\
    \optt{tau\_sc}(F) + \optt{tau\_sc}(\op{face}(\beta)) +
    \optt{tau\_sc}(\op{face}(\beta')) &\sge 0.5309 &\wedge \\
    \optt{sc}(F) & \sle s_{2+\card(F)} - 2 Z(3,1) + 2(\xi_\Gamma + 2\xi_V), \\
    &&\neg\op{qre}(\alpha),\\
    &&\beta=f^2e\alpha,\ \beta'=f^2e\alpha',\\
    &&\alpha\ne\alpha'\in F\in (H/f)_{\ge5}\\
    &&\forall\gamma\in F.\ \op{qre}(\gamma)\Leftrightarrow (\gamma\ne\alpha,\alpha')\\
    \end{array}
\end{equation}



% Page 239 DCG
Let $(H,\op{fl},S)\in\op{HS}$.   Let $F\in H/f$ and $N\in H/n$. We
have the following formal equalities:
\begin{equation}
    \begin{array}{lll}
    \tau_{sc}(F) &\seq \sol(F)\zeta\pt - \op{sc}(F)\\
    %%
    \sum_{\alpha\in N}\op{azim}(\alpha) &\seq
    2\pi\\
    %%
    \op{sol}(F) &\seq \sum_{\alpha\in F}\op{azim}(\alpha) - (\card(F)
    - 2)\pi\\
    %%
    \end{array}
    \label{eqn:tau-sc}
\end{equation}


\begin{remark}  For every  primary tame hypermap system $(H,\op{fl},S)$, the
hypermap $H$ is connected, plane, and planar.  Thus, the Euler
relation holds in the form:
    $$
    \card(H/f) - \card(H/e) + \card(H/n) = 2.
    $$
From this and the previous formal inequality, it follows that
    \begin{equation}
    \sum_{F\in H/f} \optt{sol}(F) \seq 4\pi.
    \end{equation}
As a linear consequence of other equations program, it is
redundant.  Thus, it does not need to be inserted explicitly into
the linear programs.
\end{remark}



%%%%%%%%%%%%%%%%%%%%%%%%%%%%%%%%%%%%%%%%%%%%%%%%%%%%%%%%%%%%%%%%%
\subsection{Variable Bounds}
%% Page 239 DCG.
\begin{equation}
    \begin{array}{lll}
        \optt{yn}(\alpha) &\sge0\\
        \optt{yn}(\alpha) &\sle 2\sqrt2\\
        \optt{yn}(\alpha) &\sle 2t_0 &\text{if }
        \op{qrn}(\alpha)\\
        \optt{ye}(\alpha)&\sge0\\
        \optt{ye}(\alpha)&\sle 2\sqrt2\\
        \optt{ye}(\alpha)&\sle 2t_0&\text{if }
        \op{qre}(\alpha)\\
        \optt{azim}(F) &\sge 0\\
        \optt{azim}(F) &\sle 2\pi\\
        \optt{sol}(F) &\sge 0\\
        \optt{sol}(F) &\sle 4\pi\\
    \end{array}
\end{equation}





\begin{remark}
The way the infeasibility certificates are set up in
Inequality~\ref{eqn:lpsys}, it is necessary to have lower and
upper bounds on all the variables, not just those shown here.  The
need for bounds on the other variables can be removed by the
comments of Remark~\ref{rem:bounded}.
\end{remark}




%%%%%%%%%%%%%%%%%%%%%%%%%%%%%%%%%%%%%%%%%%%%%%%%%%%%%%%%%%%%%%%%%
\subsection{Triangles}

%
\begin{equation}
    \optt{sc}(F) = \optt{sigma\_qrtet\_x}(F),\quad\text{if } F\in
    (H/f)_{3,\op{std}}.
\end{equation}
%
\begin{equation}
        \optt{tau\_sc}(F) =
        \optt{tau\_sigma\_x}(F),\quad\text{if } F\in
        (H/f)_{3,\op{std}}.
\end{equation}
%
\begin{equation}
    \begin{array}{lll}
    \optt{sc}(\op{face}(\alpha)) &\sle \optt{sigmahat}(\alpha),\\
    \optt{tau\_sc}(\op{face}(\alpha)) &\sge \optt{tauhat}(\alpha),\\
    &&\alpha\in \op{fqex}
    \end{array}
\end{equation}
%
\begin{equation}
    \begin{array}{lll}
    \optt{sc}(\op{face}(\alpha)) &\sle \optt{mu}(\alpha),\\
    \optt{tau\_sc}(\op{face}(\alpha)) &\sge \optt{tau\_mu}(\alpha),\\
    &&\alpha\in \op{quad\_diag}
    \end{array}
\end{equation}



%%25.7 page 252, Branching on Upright Quarters (triangles with a potC).
%%XX Put in as a condition that the long flat edge is not a hex-loop-diag.



\subsection{Quadrilaterals}

\begin{equation}
 \optt{sc}(F) = \optt{sigma\_quad}(F),\quad\text{if } F\in
    (H/f)_{4,\op{std}}.
\end{equation}
%
\begin{equation}
 \optt{tau\_sc}(F) = \optt{tau\_quad}(F),\quad\text{if } F\in
    (H/f)_{4,\op{std}}.
\end{equation}
%

There is a list of nonlinear inequalities that appears in the
database for quadrilateral regions.  The first of these reads



\begin{equation}
\label{eqn:exdih}
\begin{array}{lll} \op{let}\ \op{J\_310151857} &= (`\forall\ v_0\ v_1\ v_2\ v_3\ v_4.\
  (\op{is\_quad\_cluster\_v}\ v_0\ v_1\ v_2\ v_3\ v_4\ \ \Rightarrow\\
        &\ \ ((\op{sigma\_quad\_approx1}\ v_0\ v_1\ v_2\ v_3\ v_4) &<
        -5.7906\\
                + &&\ 4.56766*(\op{dih\_or\_v}\ v_0\ v_1\ v_2\ v_4)))`);;
\end{array}
\end{equation}

Let $C$ be a constant less than the right hand sides of all the
inequalities.  The function $\op{dih\_or\_v}$ takes non-negative
values.  So for instance, from Inequality~\ref{eqn:exdih}, we have
the constraint
    $$C < -5.7906.$$
Then let $\op{squad}(\op{sol},\op{azim}_1,\op{azim}_2)$ be the
maximum of $\op{sigma\_quad\_approx1}$ subject to the constraints
that the simplex formed by $v_0,v_1,v_2,v_3,v_4$ has solid angle
$\op{sol}$, and azimuth angles $\op{azim}_i$.  Or let
$\op{squad}=C$, if no such simplices exist. More precisely, set
    $$\begin{array}{rll}
    \op{squad}(s,d_1,d_2) =
        \sup \{\sigma&\mid \exists v_0\cdots v_4.\ \\
           \sigma &= \op{sig\_quad\_approx1}\ v_0\cdots v_4 \\
           &\quad \op{is\_quad\_cluster}\ v_0\cdots v_4\\
           s &= \op{solid\_v}\ v_0\cdots v_4\\
           d_1 &= \op{azim}\\
           d_2 &= \op{azim}
           \}\cup \{C\}
    \end{array}
    $$




With this definition, we can rewrite the quadrilateral
inequalities and then write them as formal inequalities that take
the following shape:
    \begin{equation}
    \label{eqn:quad}
    \op{QU}(\optt{squad},\optt{sol},\optt{azim}_1,\optt{azim}_2,i) :
    (\optt{squad} \sle a_i \optt{azim}_1 + b_i \optt{azim}_2 + c_i \optt{sol}
    + d_i),
    \end{equation}
for constants $a_i,b_i,c_i,d_i$, $i=1,\ldots,N$.  The formal
inequalities $\op{QU}$ (\ref{eqn:quad}) hold for all indices $i$
and the following parameter values:


\begin{equation}
  \begin{array}{rll}
    \optt{squad} &= \optt{sc}(F),\\
    \op{sol} &= \optt{sol}(F),\\
    \op{azim}_1 &= \optt{azim}(\alpha),\\
    \op{azim}_2 &= \optt{azim}(f\alpha),\\
    &&\alpha\in F\in (H/f)_{4,\op{std}}
   \end{array}
\end{equation}

\begin{equation}
\begin{array}{rll}
    \optt{squad} &=\sum_{i=0}^3\optt{sc}(\op{face}(n^i\alpha)),\\
    \op{sol} &=
          \sum_{i=0}^3\optt{sol}(\op{face}(n^i\alpha)),\\
    \op{azim}_1 &=\optt{azim}(f\alpha)+\optt{azim}(n f\alpha),\\
     \op{azim}_2 &= \optt{azim}(f^{-1}\alpha)+\optt{azim}(n^{-1}f^{-1}\alpha),
         \\
    &&\alpha\in N\in (H/n)_{\op{inquad}},\\
\end{array}
\end{equation}

\begin{equation}
\begin{array}{rll}
    \optt{squad} &=\optt{sc}(\op{face}(\alpha))+\optt{sc}(\op{face}(\beta)),\\
    \op{sol} &= \optt{sol}(\op{face}(\alpha))+\optt{sol}(\op{face}(\beta)),\\
    \op{azim}_1 &=\optt{azim}(\alpha)+\optt{azim}(f\beta),\\
     \op{azim}_2 &= \optt{azim}(f^{-1}\beta),
         \\
    &&\alpha,\beta \in (H)_{\op{quad\_diag}},\ \beta=e\alpha.\\
\end{array}
\end{equation}

\begin{equation}
\begin{array}{rll}
    \optt{squad} &=\optt{sc}(\op{face}(\alpha))+\optt{sc}(\op{face}(\beta)),\\
    \op{sol} &= \optt{sol}(\op{face}(\alpha))+\optt{sol}(\op{face}(\beta)),\\
    \op{azim}_1 &=\optt{azim}(f^{-1}\beta),\\
     \op{azim}_2 &= \optt{azim}(\beta)+\optt{azim}(f\alpha),
         \\
    &\alpha,\beta \in (H)_{\op{quad\_diag}},\ \beta=e\alpha\\
\end{array}
\end{equation}

\begin{remark}  There is a small conflict between nonlinear and
linear variables here that is easy to resolve.  The variables
$\optt{azim}$ are nonlinear and are interpreted as the azimuth
function.  However, in the quadrilateral inequalities give above,
they are independent variables to the function $\optt{squad}$, and
should be treated as linear variables.  To overcome these
conflicting constraints, we introduce two copies of each variable,
say $\optt{azim}$ and $\optt{azim}'$, related by the equation
    $$
    \optt{azim}(\alpha) = \optt{azim}(\alpha)'.
    $$
We then use the nonlinear copy everywhere but in the quadrilateral
inequalities, we switch to the linear copy.
\end{remark}

\subsubsection{Specialized Inequalities and Constraints}



\begin{equation}
    \begin{array}{lll}
        \optt{sc}(F) &\sle s_8 + \pt\\
        \optt{tau\_sc}(F) &\sge t_8\\
        &&F \in (H/f)_{5,\op{aggA}}.
    \end{array}
\end{equation}

\begin{equation}
    \begin{array}{lll}
        \optt{sc}(F) &\sle s_7\\
        \optt{tau\_sc}(F) &\sge t_7\\
        &&F \in (H/f)_{5,\op{aggB}}.
    \end{array}
\end{equation}

\begin{equation}
    \begin{array}{lllll}
        \optt{yn}(\alpha) &\sge 2.168,&\text{some }\alpha\in F
        &\vee\\
        \optt{azim}(\alpha) &\sge 2.89,&\text{some }\alpha\in
        F&\vee\\
        \optt{ye}(\alpha)+\optt{ye}(f\alpha)&\sge
        4.804,&\text{some }\alpha\in F&\vee\\
        \sum_{\alpha\in F} \optt{ye}(\alpha) &\sge
        11.4707&&\vee\\
        \optt{sc}(F) &\sle -0.2345&&\vee\\
        \optt{tau\_sc}(F) &\sge 0.6079,\\
        &&&&F\in (H/f)_{5,\op{aggB}}.
    \end{array}
\end{equation}

\begin{equation}
    \begin{array}{llll}
        \optt{sc}(F) &\sle -0.4339&\vee\\
        \optt{sc}(F) &\sge 0.5606\\
        &&&F\in (H/f)_{n,\op{crowd3}},\quad n\ge 5.
    \end{array}
\end{equation}

\begin{equation}
    \begin{array}{llll}
        \optt{sc}(F) &\sle -0.25\\
        \optt{tau\_sc}(F) &\sge 0.4\\
        &&F\in (H/f)_{n,\op{crowd4}},\quad n\ge 6.
    \end{array}
\end{equation}

\begin{equation}
    \begin{array}{lllll}
        \optt{sc}(F) &\sle -0.31547&&\vee\\
        \optt{yn}(\alpha) + \optt{yn}(f^4\alpha) &\sge 4.6 &\text{some
        }\alpha\in F,\\
        &&&F\in (H/f)_{n,\op{crowd4}},\quad n\ge 6.
    \end{array}
\end{equation}

\begin{equation}
    \begin{array}{llll}
        \optt{sc}(F) &\sle s_8\\
        \optt{tau\_sc}(F) &\sge t_8\\
        &&F\in (H/f)_{6,\op{agg}}.
    \end{array}
\end{equation}

\begin{equation}
    \begin{array}{llll}
        \optt{sc}(F) &\sle -0.37595\\
        \optt{tau\_sc}(F) &\sge 0.65995\\
        &&F\in (H/f)_{6,\op{loop51}}.
    \end{array}
\end{equation}





\begin{equation}
    \begin{array}{lllll}
        \optt{yn}(\alpha) &\sge 2.14 &\text{some }\alpha\in F&\vee\\
        \optt{ye}(\alpha) &\sge 2.77 &\text{some }\alpha\in
        F&\vee\\
        (\optt{sc}(F) &\sle \optt{gamma}(F) &\wedge\\
         \optt{yn}(\alpha) &\sle 2.14 \text{ all }\alpha\in
         F&\wedge\\
        \optt{eta}(F) &\sle \sqrt2) &&\vee\\
        (\optt{ye}(\alpha) &\sle 2.7 &\wedge\\
        \optt{yn}(\alpha) &\sle 2.14 \text{ all }\alpha\in
         F&\wedge\\
        \optt{eta}(F) &\sge \sqrt2 &\wedge\\
        \optt{sc}(F) &\sle \optt{svor0}(F)) &&\vee\\
        (\optt{ye}(\alpha) &\sge 2.7 &\wedge\\
        \optt{yn}(\alpha) &\sle 2.14 \text{ all }\alpha\in
         F&\wedge\\
        \optt{sc}(F) &\sle \optt{svor0}(F)),\\
        &&&&\alpha\in F\in (H/f)_{3,\op{fqex}}.
    \end{array}
\end{equation}

 Let $\op{potA}$ be the predicate on $(H/f)_{3,\neg\op{std}}$
that picks out potentially type A.  That is, $\op{qrn}(\alpha)$
holds for all three darts, $\op{qre}(\alpha)$ holds for exactly
one of the three darts, and no dart is in a hex diagonal.

\begin{equation}
    \begin{array}{lll}
    \op{potA}(F) &= \exists\beta\in F.\ \forall\alpha\in F. \\
        &\op{qrn}(\alpha)\\
        &\op{qre}(\alpha) \Leftrightarrow (\alpha=\beta)\\
        &\neg\op{hex\_loop\_diag}(\alpha).
    \end{array}
\end{equation}

\begin{equation}
 \begin{array}{lllll}
    \optt{yn}(\alpha) &\sge 2.77 &&\vee\\
      \optt{yn}(\beta) &\sge 2.77 &&\vee\\
    (\optt{yn}(\alpha) &\sle 2.77 &\wedge\\
      \optt{yn}(\beta) &\sle 2.77 &\wedge\\
      \optt{sc}(F) &\sle \optt{svoran}) &&\vee\\
   (\optt{yn}(\alpha) &\sle 2.77 &\wedge\\
      \optt{yn}(\beta) &\sle 2.77 &\wedge\\
      \optt{eta}(F) &\sge \sqrt2) &&\vee\\
     &&&&\alpha\ne\beta,\gamma\in F\in (H/f)_{3,potA}\\
     &&&&\neg\op{qre}(\alpha),\neg\op{qre}(\beta),\op{qre}(\gamma).
 \end{array}
\end{equation}

Let $\op{potC}$ be the set of triangles that are potentially of
type $C$. That is, for $F\in (H/f)_{3,\neg\op{std}}$:
\begin{equation}
    \begin{array}{lll}
    \op{potC}(\beta,F) &= \beta\in F \wedge\ \forall\alpha\in F. \\
        &\op{qrn}(\alpha)\Leftrightarrow (\alpha\ne\beta)&\wedge \\
        &\op{qre}(\alpha) \Leftrightarrow (\alpha\ne f\beta).
    \end{array}
\end{equation}













\section{Final Specs}


\subsection{Primary Tame Hypermap System}
\label{sec:PTHS}

This subsection gives a specification of primary tame hypermap
systems ($\op{PTHS}$).

Define $\op{type}(N):H/n\to\ring{N}^3$ by $\op{type}(N) = (p,q,r)$
if there are $p$ triangles, $q$ quadrilaterals, and $r$
exceptional faces that meet the node $N$.  (An exceptional face is
defined to be one with at least $5$ darts.)


\begin{definition}
Let $H$ be a hypermap.  A flag $\op{fl} = (\op{qrn},\ldots)$ is a
{\it primary flag} for $H$ if it satisfies the following
conditions:
    \begin{enumerate}
        \item $\op{qrn}(\alpha)$, for every dart $\alpha$.
        \item $\op{qre}(\alpha)$, for every dart $\alpha$.
        \item $\op{typ}(5)\ne\op{notpri}$.  (This is a double negation: the flag is
        {\it not not\/} primary.)
        \item $\exists\alpha\in F.\ \op{type}(\op{node}(\alpha)) \ne
    (3,0,1),\quad F\in (F/f)_{5,\op{aggA}}$.
        \item Except for the preceding two restrictions on $\op{typ}(n)$, the flags
        $\op{typ}(n)$ and $\op{sum\_typ}(n)$ are arbitrary, for $n=4,\ldots,8$.
    \end{enumerate}
\end{definition}

\begin{remark} By the first two properties, every face is
standard.  Furthermore, the domains of $\op{quad\_diag}$ and
$\op{hex\_loop\_diag}$ are empty, and so they leave nothing to
specify.
\end{remark}

\begin{definition}
A triple $\rho=(H,\op{fl},S)$ is a primary tame hypermap system if
    \begin{enumerate}
    \item $H$ is a tame hypermap in the archive.
    \item $\op{fl}$ is a primary flag for $H$.
    \item $S=\op{basic}(H,\op{fl})$ is the set of basic formulas for $I=I(H,\op{fl})$.
    \end{enumerate}
\end{definition}


\subsection{Refinement -- Lookup}
%
\label{sec:refine}

The refinement predicate $\op{ref}(\rho,\rho')$ is a disjunction
    \begin{equation}
    \op{ref}(\rho,\rho') =
    \op{lookup\_database}(\rho,\rho')\vee
    \op{sum\_hyper}(\rho,\rho').
    \end{equation}
This subsection describes the predicate $\op{lookup\_database}$,
and the following subsections  will describe the more complicated
predicate $\op{path\_hyper}$.

As the name suggests, $\op{lookup\_database}(\rho,\rho')$ holds
when $\rho'$ is obtained from $\rho=(H,\op{fl},S)$ by adding
lookups from a database.

\begin{definition}
We say that $\rho'$ is derived from $\rho=(H,\op{fl},S)$ by lookup
in the database $B$ and write
    $$\op{lookup\_database}(\rho,\rho')$$
if $\rho' = (H,\op{fl},S \cup S')$ where $S'$ is a set of lookups
in $B$ with domain constraints in $\op{single}(S)$. (The database
$B$ is fixed once and for all and hence is omitted from notation.)
\end{definition}

\subsection{Connected Sum -- Hypermaps}

The second mode of refinement is the connected sum of a hypermap.  This
subsection gives the details of this construction.

Let $\rho=(H,\op{fl},S)$ and $\rho'=(H',\op{fl}',S')$.   The rough
idea is that $\op{sum\_hyper}(\rho,\rho')$ holds if we can
obtain $H'$ from $H$ by a connected sum. That is,
    $$H' = H\,\#_\phi\, B.
    $$
The actual
construction of the predicate $\op{sum\_hyper}$ is somewhat more
involved because it involves a change in hypermap and
corresponding flags. This means that the indexing sets for the
system of formal disjunctions also change.  Part of the connected sum
operation involves relating the formal disjunctions $S$ on
indexing set $I(H,\op{fl})$ to formal disjunctions $S'$ on the
indexing set $I(H',\op{fl'})$.

\subsubsection{Sum Kit}.


A connected sum of hypermaps $H\,\#\,B$  requires an
orientation reversing bijection $\phi:F\to F_B$ between a face of
$H$ and a face of $B$.  Some of the flags have been designed to
give such bijections.  In fact,  if $n=\card(F)$, every flag
$\op{sum\_typ}(n)(F)$ of a patchable face $F\in (H/f)_{n}$
uniquely determines a hypermap $B$ and an orientation reversing
bijection $\phi:F\to F_B$, for some face $F_B$ of $F$.

The data of $B,\phi$ and some auxiliary information about flags is
what we call a {\it connected sum kit}.  The connected sum kits are described by
various diagrams.  We list the diagrams (Figure~\ref{XX}) and then
say how to interpret them.

(N.B. This section is incomplete.  It is necessary to give all the
figures, labeled according to the interpretive key.)  % XX DO THIS.

The following comments provide an interpretive key to the
diagrams.
\begin{itemize}
    \item $B$ is the connected, plain, planar
    hypermap obtained by inserting darts at each angle of the
    planar graph, in the usual way, as illustrated in
    Figure~\ref{XX}. %% XX Provide figure.
    (N.B. Figure is to be inserted later.) %% XX DO THIS
    \item The face $F_B$ is the outermost face; that is, the one
    bounding the unbounded region.
    \item The face $F_B$ has a distinguished dart $x$ marked on it,
    indicated as a small black triangle.
    \item In cases when the hypermap $B$ is not isomorphic to the
    mirror-image of $B$, there is a second hypermap that is
    relevant. In these cases, the diagram for mirror 
    connected sum is obtained as the
    mirror image of the diagram.  Thus, we draw a single diagram
    and give it a second mirror name that is used to refer to the
    mirror data.
    \item The orientation-reversing bijection $\phi:F\to F_B$ is
    determined by choosing $\alpha\in F$ and specifying that
    $\phi(\alpha) =x$.  The map $\phi$ extends by the rule
        $$f_B^i\phi(f^i\alpha ) = x.$$
    \item Each diagram has a name that we use as the name of the connected sum kit.
\end{itemize}

\begin{remark} As just mentioned, the orientation-reversing bijection $\phi:F\to F_B$
is determined by $\alpha\in F$.  The $\alpha$ and $f^i\alpha$ give
isomorphic connected sums if there is an automorphism of the hypermap
carrying $\alpha$ to $f^i\alpha$.  In this case, the infeasibility
for the parameter $f^i\alpha$ can be deduced from the
infeasibility for the parameter $\alpha$.  Thus, the two cases can
be combined into one.  From a mathematical point of view this
reduces the amount of verification to be done.  However, Nipkow's
experience with the formal verification of tame graphs suggest
that case mergings generate additional proof obligations, and
sometimes actually increase the amount of work required for a
formal proof.
\end{remark}








\subsubsection{Compatibility with Flags}
\label{sec:com}

Let $H'= H \,\#_\phi\, B$.   There is a natural inclusion of
darts.
    $$
    \iota: H\setminus F \to H'.
    $$
There is a natural inclusion of faces, which we will express by
the same symbol:
    $$
    \iota: H/f \setminus \{F\}\to H'/f.
    $$

\begin{definition}
We say that a flag $\op{fl}'$ is compatible with
$(H,\op{fl},B,\phi)$  if the following conditions hold:
    \begin{enumerate}
    \item $\op{fl}'$ is a flag on $H'= H\,\#_\phi\, B$.
    \item The domain $F$ of $\phi$ is a patchable face.  That is,
    $\op{type}(\card(F))(F) = \op{patchable}$.
    \item The connected sum kit $(B,F_B,\phi,\ldots)$ of $F$ is given by
    $\op{sum\_typ}(\card(F))(F)$.
    \item We have $\op{qrn}(\alpha) = \op{qrn}'(\iota\alpha)$.
    (We recall also $\op{qrn}'$ is required to be constant on each
    node.)
    \item The node $N$ of $H'/n'$ (there is at most one) that does not lie in the image of
    $\iota$ has
        $\op{qrn}'(\beta)$ for $\beta\in N$ exactly when that node
        is {\it not} marked $\otimes$ in the Figure~\ref{XX}.  (That is, in
        every case except Figure~\ref{XX}, darts at a new node satisfy
        $\neg\op{qrn}'(\beta)$.)
    \item We have $\op{qre}(\alpha)= \op{qre}'(\iota\alpha)$.
    (We recall that $\op{qre}'$ is required to be constant on each
    edge.)
    \item The edges $E$ of $H'/e'$ that do not lie in the image of
    $\iota$ satisfy $\op{qre}'(\beta)$ for $\beta\in E$ exactly
    when that edge is marked as a single line in Figure~\ref{XX}.
    (That is, double lines give $\neg\op{qre}'(\beta)$.)
    \item We have $\op{hex\_loop\_diag}(\alpha) =
    \op{hex\_loop\_diag}(\iota\alpha)$.
    \item The edges $E$ of $H'/e'$ that do not lie in the image of
    $\iota$ satisfy $\op{hex\_loop\_diag}(\alpha)$ for $\alpha\in E$
    exactly when the edge is explicitly marked as such in the
    figures.  (The figures also shed some light on the terminology: such
    edges always run between opposite corners of a hexagonal face
    and run alongside a loop of four triangles.)
    \item The flags $\op{typ}(i)$ and
    $\op{sum\_typ}(i)$ are preserved on faces:
        $$
        \begin{array}{lll}
        \op{typ}(\card(F))(F) &= \op{typ}'(\card(F))(\iota F),\\
        \op{sum\_typ}(\card(F))(F) &= \op{sum\_typ}'(\card(F))(\iota F),\\
        \end{array}
        $$
    \item Usually, a face that is not in the image of $\iota$ is
    not standard, and hence lies outside the domain of the flags
    $\op{typ}(i)$ and $\op{sum\_typ}(i)$.  In this case there is
    no compatibility condition.
    \item If $\card(F)=8$, $\op{typ}(8)(F)=\op{patchable}$,
    $\op{sum\_typ}(8)(F) = \op{agg51}$, then there is a
    pentagonal face $F'$ in $B$ that carries over to $H'$.  In this case, we
    give the compatibility condition
        $$\op{typ}'(5)(F') = \op{notpri}.$$
    (This is, in fact, the only way to obtain the flag value
    $\op{notpri}$.)
    \end{enumerate}
\end{definition}


\begin{remark}
We note that if $H'= H\,\#_\phi\, B$ and $\op{fl}$ are given,
then there is at most one flag $\op{fl}'$ that gives
compatibility.
\end{remark}

\begin{remark}
We also note that if $H' = H\,\#_\phi\, B$, then by the
compatibility conditions, the faces coming from $B$ never have
type $\op{patchable}$. This means that we can never apply one
connected sum upon another.
\end{remark}

\subsubsection{Indexing Sets}

We assume that $H' = H\,\#_\phi\, B$.  As in the previous
subsection (\ref{sec:com}) we write $\iota$ for the natural
inclusion  on darts and faces.

Under the natural inclusion
    $$\iota:H\setminus F \to H'$$
there is a correspondence between a subset of $I=I(H,\op{fl})$ and
$I'=I(H',\op{fl}')$.  We make this correspondence more explicit
here.



All the indexing sets for $I=I(H,\op{fl})$ run over a subset of
darts (for $\op{azim}$, $\op{yn}$, $\op{ye}$, and so forth) or
over a subset of faces (for $\op{sc}$, $\op{sol}$, and so forth).
Let $I^F\subset I$ be the subset of the indexing set that does not
involve any of the darts of $F$, and does not involve the face $F$
itself.

We then have a one-to-one map $I^F \to I'$.  This gives a
one-to-one map between the subset of the set of formal
inequalities on the indexing set $I$ (those involving only the
indices $I'$) and the set of formal inequalities on the indexing
set $I'$. Extending to formal disjunctions, we get a one-to-one
map $\op{LFOR}_{I^F}\to \op{LFOR}_{I'}$.

\begin{definition}  We say that a formal disjunction on the
indexing set $I'$ is {\it inherited} from one of $I$, if it lies
in the image of this one-to-one map with domain
    $$\op{LFOR}_{I^F}\subset \op{LFOR}_I.$$
\end{definition}



\begin{definition}
We say that $\rho'=(H',\op{fl}',S')$ is obtained by a hypermap
connected sum $\rho =(H,\op{fl},S)$  and write
$\op{sum\_hyper}(\rho,\rho')$ if there exists a patchable face
$F\in H/f$ such that
    \begin{enumerate}
    \item $(B,\phi)$ is the data determined by $\op{sum\_type}(\card(F))(F)$.
    \item $H' = H\,\#_\phi\, B$.
    \item The flag $\op{fl}'$ is compatible with $(H,\op{fl},B,\phi)$.
    \item Every formal disjunction $A'$ in $S'$ is either a basic
    formula for $(H,\op{fl})$ or is inherited from $S$.
    \end{enumerate}
\end{definition}

\subsection{Constants}

This document has used a number of constants: $\xi_\Gamma$,
$\xi_V$, $\pt$, $s_i$, $t_i$. $D(n,k)$, $Z(n,k)$, $t_0$, $\phi_0$,
$\delta_\op{oct}$.
%
These constants are all the same as in \cite{DCG}.  They are
listed in its index.


\section{Implementing the Proof of Weak Infeasibility}

By the definition of weak infeasibility, there exists a tree that
describes the branches followed in the search for infeasibility.
Ultimately the proof of the main LP bound is constructive, in the
sense that the proof gives all of the data needed to construct the
search tree.

Of course, many different infeasibility trees can be used as
certificate of weak infeasibility.  Making less than optimal
decisions about how to construct the tree may increase the amount
of computation needed, but if a reasonable strategy is followed,
the exact choices are not too important.  What we give in this
section are some search strategies that seem to be helpful, and
that were in fact helpful in the original 1998 proof of the Kepler
conjecture.  We wish to emphasize that it may very well be
possible to devise much more efficient search strategies than
those described here.

At every stage of the search there are many possible options about
how to proceed.  Some of the options are to
    \begin{itemize}
    \item Obtain a new hypermap by a connected sum of the current one.
    \item Bifurcate on a disjunctive formula.
    \item Enhancing the system of formal disjunctions by adding
    lookups.
    \end{itemize}
The simple act of bifurcating on a formula has infinitely many
possibilities.

To construct the infeasibility trees, it is necessary to be guided
by various heuristics.  We are not looking for a precise
algorithm, merely heuristics that can guide a computer program to
construct the appropriate trees. What actions should have high
priority and what actions should be used as a last resort?



Of course, it may be the the original system is infeasible.  The
first thing to try is the infeasibility of the system obtained
without including any inequalities that depend on the value of the
flags.  (It is often best to ignore the inequalities depending on
flags in the early stages.)  If this system is infeasible, it
implies in one fell swoop the weak infeasibility of all primary
tame hypermap systems $(H,\op{fl},S)$ with a given hypermap $H$.
In fact, thousands of hypermaps can be quickly eliminated by this
one simple strategy.  We now confine our remarks to primary tame
hypermap systems for which this simple strategy fails.

Sometimes the heuristic can be guided by geometry.  The nodes of
the hypermap represent the centers of a finite cluster of spheres
surrounding the origin.
 The variables $\optt{yn}(\alpha)$ represent the distances of
 these sphere centers from the origin.  The idea is move the
 system in the direction of a cluster of spheres in which all of
 the distances $\optt{yn}(\alpha)$ are constrained to be about
 2.0:
    $$2 \sle \optt{yn}(\alpha) \sle 2+\epsilon.$$
 Once these variables are heavily constrained, the focus moves to the
 variables $\optt{ye}(\alpha)$.  The strategy is then to constrain
 these variables as tightly as possible between an upper and lower
 bound.  Once strong constraints are obtained on these variables,
 the weak infeasibility often follows quite easily.

Another rough guide is to keep the number of branches under
control.  For example, we could easily get exponential growth in
the tree if we repeatedly bifurcate all active branches of the
tree.  Large exponential growths should be avoided.  Thus, it is
often good to bifurcate in a way that one of two possibilities is
infeasible.  For example, we can choose a constant $c$ (obtained
by linear programming) for which
    $$\optt{yn}(\alpha)\sge c$$ is infeasible.  Then a bifurcation
    on
$$(\optt{yn}(\alpha) \sge c) \vee (\optt{yn}(\alpha) \sle c)$$
has a single continuation.  This allows us to add
    $$\optt{yn}(\alpha)\sle c$$
to the system.  Adding such an inequality has further
consequences, because there are many nonlinear lookups whose
domain constraints take precisely this shape.

One rough guide is that the most useful inequalities come from
faces of the smallest size.  In particular, the inequalities for
triangles are by far the most useful.  Thus, one strategy is to
make triangles the focus of the bifurcation for as far into the
search tree as possible, bringing in large faces only as necessary
late in the game.

Since all triangles in a primary tame hypermap system are
standard, the opening moves should concentrate on standard
triangles.  A good opening strategy might be to narrow the
feasible range on the variables $\optt{yn}(\alpha)$.

\section{Conclusions}

This document is still sketchy in many places.  However, the
general framework of the linear programming has been placed on
solid foundation.  There are other references that may be
consulted for further details.  These references include the
published proof of the Kepler Conjecture in \cite{DCG} (Section
23, pages 236ff). There is the database of nonlinear inequalities
at Hales's web page. There is the archive of tame planar graphs.

\subsection{Mathematica}

There are Mathematica files that were used to generate the linear
programs.  The most important of these is {\it MathToCplex.m}
which appears with the linear programming materials in the
original 1998 proof.  This file is important because it generates
all of the basic linear programs.  There are also supplementary
files {\it MathToCplexExcept.m} and {\it MathToCplexPent.m} that
handle the more sophisticated branch and bound operations.  All
the linear programs were generated by these files.

Interactive Mathematica sessions were used to run and analyze the
linear programs.  The Mathematica programs generate text files in
cplex format and stored as external files. The Cplex linear
programming package was run on these files, by issuing a command
from Mathematica to the operating system.  The results from the
linear programs were stored as text files, then parsed back into
the Mathematica session for analysis.  The human controller
(Hales) made interactive decisions about how to branch the linear
programs, based on heuristics and intuition about how to establish
infeasibily in the quickest way possible.  Various Mathematica
procedures were written to automate the branch and bound process.

These interactive sessions were summarized in the form of log
files, to make it possible to reconstruct the moves that led to
infeasibility. These log files consist of notes written by Hales.
Mixed in with the notes are snippets of the Mathematica code that
was used to direct the branch and bound operations.  Thus, in
principle, the log files contain all that is necessary to
reconstruct the sequence of linear programs.  These are all
available from Hales's web page.

The files terminating in .m are Mathematica files.  Those in .exec
are files that control the execution of Cplex.  Those in .lp are
cplex format linear programs.  Those in .log are the output from
Cplex sessions that get parsed back into a Mathematica session.
Those in .html contain my notes and snippets of Mathematical code
that describe the branch and bound for the linear programs.

It would be clearly very valuable to rework this, so that the
process is automated from beginning to end.  The Flyspeck project
hopes to carry this even further, to give a formal proof of the
main theorem of this paper.

\subsection{An example of a log file}

The following is the file from Hales's webpage
(/SHORT/OCT/oct.zip/OCT/index.html) that describes how the
hypermaps containing an octahedral face were handled.  It is a
mixture of plain text and Mathematica code.  A few comments have
been added.  Otherwise, it is identical to the 1998 log.  The
hypermaps with octahedral faces would be a good place to start in
a second generation treatment of the linear programs for hypermaps
with an exceptional face.

\bigskip

{\obeylines\tt

(* How to do the octagonal cases.  This first block of code
generates all the linear programs for the octahedra.  The files
are stored in the SHORT/OCT/LP directory, and the file names all
start with cplexE.lp.  *) \

\

\

f[i\_] := (
        exceptStem = "SHORT/OCT/LP/cplexE.lp";
        Initialize[8,i];
        Array[(Print[{8,i, \#1}];
        WRITEOUTexcept[8,i, Length[GBLregions], \#1]) \& , Length[AllOct]]
        );
Map[f,locthi];

\

\

(* The following block of code generates code for an interactive
cplex session.  It tells cplex to read each of the files and run
the linear programming optimization in the file. *)

\

\

optText[n\_,i\_]:= Module[{},
Initialize[n,i];
        Array[{"read ",
            fileText["cplexE.lp",n,i,Length[GBLregions],\#], " lp$\backslash$n",
            "opt$\backslash$n"
            }\&, Length[AllOct]
            ]//StringJoin
        ];

\

\

(* This stores the cplex instructions to an external file called
cplexE.exec *)

\

\

Module[{vv,stream},
    vv = Map[optText[8,\#]\&,locthi]//StringJoin;
    stream=OpenWrite["SHORT/OCT/cplexE.exec"];
    WriteString[stream,vv];
    Close[stream]
    ];

\

\


(* After running the cplex program, the results are parsed back
into Mathematica as an array called values.  The following command
finds that there is just one case that didn't pass on this first
attempt.  The command is everything up to the symbol ==.
Everything following the symbol "==" is the output from the
command.  The constant 0.4429 is approximately 8 pt.  *)

\

\

hh= Select[values,\#[[1]]>0.4429\&]== {{0.459297, 8, 14, 17, 21}};

\

\

(* The rest gives a few hints about how to eliminate this final
case.  It is based on the inequalities from DCG page 252, section
25.6.2. We are able to get a smaller value of the penalty from the
final lines of DCG section 25.6.2.  What follows is a mixture of
plain text and Mathematica.  *)

\

\

LPmax[bf,"-sigma{17}"] == 0.299186; if there is a Z(4,2):   s8 +
ZLP(4,2)-Z(4,2) = -0.31398. So no (4,2)s.  Since the two flats are
opposite, the only penalties
    are from (4,1)s that mask a flat, so pen17 < 2(xiG+2xiV) = 0.045304.
    Edit the file so pen17 = 0.045304.
bf = "SHORT/OCT/LP/cplexE.lp8.14.F17.C21pen"

\

\

(* With this smaller penalty in place, we branch and bound on
eight of the triangular faces, with the usual 6.25 cutoff, as
described in DCG, Sec 24.1, page 241. We find that in every case
the bound comes out less than 8 pt. Again, everything before the
"==" is the Mathematica command, and everything following it is
the output from the interactive session.  Since this constant is
less than 8 pt, we are done. *)

\

\

ww=( Initialize[8,14]; Array[
LPdisplay["SHORT/OCT/LP/cplexE.lp8.14.F17.C21pen",
    Branch[{6,7,9,10,11,12,13,14},\#] ]\&,2\^8])//Max ==  0.441675

\

\

OCTs are done.

}




\subsubsection{Second Generation LP}

A couple of years after the proof was completed, an independent
check was made of the linear programs for tame plane graphs that
contain only triangles and quadrilaterals (2000, unpublished, but
available on Hales's web page). This is an entirely new and
independent implementation of the linear programming, but it was
not done in sufficient generality to handle hypermaps with
exceptional faces. This code is written in Java.  The results of
the linear programs are checked with a Java-based implementation
of interval arithmetic. This code has the advantage that the
branch and bound operations are entirely automated.  This code is
available from Hales's website.

The reason that this code is unable to treat hypermaps with
exceptional faces is that no implementation has been given of the
connected sum operation (called faced refinements in the original 1998
proof).


\subsection {What is still missing from this document}

This document is starting to get quite long, and it still is far
from complete.  Based on what is here, it now looks as if a
detailed specification of the linear programming part of Kepler
may run to some 100 pages. My hope at this point is that it
contains enough about the structure of the task, to make the other
sources of information comprehensible.

This final subsection has been written as a work in progress,
giving lists of topics that must eventually be carefully
specified, but that have not yet been.

\subsubsection{Inequalities}

This document does not contain a complete list of the inequalities
that are used in the linear programming infeasibility.  Consult
the Mathematica files and \cite{DCG} for the list.  There are
various important inequalities that need to be added.

\begin{itemize}
    \item There is an important identity for $\optt{vor0}$ in terms of the
    variables $\optt{quo}$, $\optt{Adih}$, etc.
    \begin{equation}
    \optt{vor0}(F) = \phi_0 \optt{sol}(F) +\sum_{\alpha\in F}
    (\optt{Adih}(\alpha) - 4\delta_\op{oct}(\optt{quo}(\alpha)
    +\optt{quob}(\alpha))).
    \end{equation}
    (Compare \cite{DCG}, Sec 25.12, page 257.)
%
    \item There are various inequalities for $\optt{quo}$ obtained
    by nonlinear lookup.
%
    \item We have
        \begin{equation}
        \optt{quob}(\alpha) = \optt{quo}(e\alpha),\quad
        \forall\alpha.
        \end{equation}
    The nonlinear interpretation of $\optt{quo}(\alpha)$ is
    the function $\op{quo}(v,w)$ of \cite{DCG}, and that of
    $\optt{quob}(\alpha)$ is $\op{quo}(w,v)$.
%
    \item There are various inequalities for $\optt{Adih}(\alpha)$
    obtained by nonlinear lookup.  These are rather
    unconventional, because they require the input of lower and
    upper bounds on the variable $\optt{azim}(\alpha)$.  There is
    a note about this at the end of Sec 25.12 of \cite{DCG}, page 257.
%
    \item When $\optt{eta}'(\alpha)\sge t_0$, then $\optt{quo}(\alpha)
    = \optt{quob}(\alpha)=0$.
    \item There are lower and upper bounds on $\optt{azim}(\alpha)$ for all
    darts $\alpha$, depending on the structure of the face to
    which $\alpha$ belongs.  These are given in a table in
    \cite[VI,p.54]{KEP98}.  The same bounds appear in the database as
    inequality 853728973.
    \item There are the hexagonal inequalities of \cite[25.13]{DCG}.
  There are some problems -- still unresolved by this document --
  about the interpretation of the variables $\sigma_R$,
  $\sigma_R^+$, $\tau_R$, $\tau_R^+$ of that paper in terms of the variables
  of this document.
  \item There is the ``$3.0$'' quadrilateral inequality of
  \cite[Sec.25.10]{DCG}.
  \item There are the $t(n)$ and $s(n)$ inequalities when there
  are no aggregates, and the weaker version of the same
  inequalities that hold over aggregates.
\end{itemize}

There are several important inequalities that hold over groups of
faces.

\begin{itemize}
  \item There are the ``$0.55$'' and ``$0.48$'' bounds found in
  \cite[Lemma 10.6,p.101]{DCG}.
  \item There are the ``$1.4$'' and ``$1.5$'' bounds found in
  \cite[Lemma~22.12,p.233]{DCG}.  There is a remark at the end of
  Section~4.10 of \cite[VI,p.37]{KEP98} that shows how these
  inequalities can sometimes be improved by removing certain faces
  from the sums of those inequalities.
\end{itemize}


There are special inequalities on simplices that are potentially
of type C or C'.  See the note below in
Subsection~\ref{sec:wrong}.





\subsubsection{Connected Sums}

This document does not yet draw all of the connected sum possibilities.
Again, this is found in \cite[Sec.25.5--25.6,pp.249--251]{DCG}.
When drawing these connected sum kits, it is necessary to add a mirror
case to each one that does not have reflectional symmetry.  When
drawing these connected sum kits, it is necessary to add markings as
described in the interpretive key in this document.

The penalties from \cite{DCG} have been changed in this document
so that they adhere to a specific face, rather than to the entire
connected sum kit.  Doing so, cleans up the scoring rule of
Equation~\ref{eqn:xi}.




\subsubsection{Lookup}

Detailed information about which inequalities were pulled from the
nonlinear database by lookup can be determined by consulting the
Mathematica files.





\subsubsection{Variables}

Complete the list of variable names.  Check the names of variables
on such things as $\optt{sigmaflat}$ $\optt{tauflat}$,
$\optt{svoran}$, $\optt{nu}$, $\optt{vorx}$.

\subsubsection{Extensions}

There are other linear programs, besides the Main LP Bound that
arise in the proof of the Kepler conjecture.  These are the {\it
small} linear programs, so-called because they involve a small
number of variables and constraints.  An example of small linear
programs are those appearing in the proof of Lemma~10.5 in
\cite{DCG}.  There are similar linear programs scattered
throughout the paper.

The linear programs for the proof of the dodecahedral conjecture
should be developed along similar lines as this.

This document separates the linear programming infeasibility
results from the geometry.  To build a bridge between the Main LP
Bound and the proof of the Kepler conjecture, it is necessary to
prove that any counterexample to the Kepler conjecture leads to a
strong solution to a tame hypermap system.  The basic ingredients
to prove this are already in \cite{DCG}, but it will be necessary
to write this up, to show exactly how it follows from results in
those papers.




\subsubsection{Things that may be wrong or incomplete}
\label{sec:wrong}

The treatment of linear programming variables related to the score
is very muddy in \cite{DCG}.   I am not aware of any explicit
errors, but it is extremely difficult to piece together precise
linear programming inequalities, based on what is written there.
Section 25.12 of \cite{DCG} is particularly confusing, and should
be avoided whenever possible.  The description of penalties in
Section~\ref{sec:pc} and Equation~\ref{eqn:xi} are intended to be
clarified versions of \cite{DCG}.

There is one important case of the scoring function that has not
been entirely specified in this document.  This concerns the
inequalities for $\optt{sc}(F)$ for triangular faces $F$ with one
edge that is not quasi-regular and one node that is not
quasi-regular.  These include the cases of simplices of type C and
type C' from \cite[Sec.9.4,pp.94--99]{DCG}, which have special
scoring rules.  On such simplices there will be an inequality that
takes the general form
    $$
    \optt{sc}(F)\sle \optt{vorx}(F).
    $$
There are a number of lookup inequalities for $\optt{vorx}$.  What
I still need to confirm is that nothing funny is going on with the
switches to type C, C', etc.

All the linear programs use weak inequalities $(\le),(\ge)$.  When
we branch over a disjunction
    \begin{equation}
    \label{eqn:dis}
    (\optt{A}\sle C)\vee (\optt{A}\sge C),
    \end{equation}
there is overlap between the two cases when equality occurs.
Normally, this is not a problem; but in a couple of special
situations it creates a problem.  (In the 1998 proof this was
handled carefully at a conceptual level, but it does not find its
way into the code.  For the fully automated proof, it has to find
its way into the code.)

Even though this issue may seem minor, I think that it may be
necessary to add some components to the flags to resolve the
problems this creates.  What I am hoping to avoid is a
modification of the flag that makes it so that in a connected sum, there
can be more than one compatible flag $\op{fl}'$.  I prefer to
maintain uniqueness for $\op{fl}'$.

One place where it creates a problem is when there is explicit
branches between scoring systems.   For example, sometimes it is
necessary to consider two cases of scoring (compressed and 
decompressed) on quarters.  Since the scoring type cannot be
determined by weak inequalities (in the ambiguous case when a face
circumradius is exactly $\sqrt2$), weak inequalities are
inadequate in such cases.  The branching should really run over a
flag giving the scoring type, rather than over weak inequalities
for the circumradius of a face.


There is a trick that can often be used to avoid the case of
equality in the Formula~\ref{eqn:dis}.   One helpful strategy is
to transform disjunctions $A\vee (\optt{x} < B)$ into implications
    \begin{equation}
    (\optt{x} \ge B) \Rightarrow A.
    \end{equation}
(where there is no weakening), rather than into weak inequalities
    \begin{equation}
    A \vee (\optt{x}\le B)
    \end{equation}
Implications can be handled with nonlinear lookups,
$\op{nonlin}(C,L)$, where the antecedent is placed in $C$ and the
consequent is placed in $L$.

\begin{example}  For example, consider the branching over edge
lengths at the cutoff $2.45$ as discussed in \cite[Sec.25.7]{DCG}.
The inequalities $\optt{ye}(\alpha)\sle 2.45$ should transformed
as antecedents in the nonlinear lookup.  %% XX Check this. Does it really work?
\end{example}
