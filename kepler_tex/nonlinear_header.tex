%%
% Author Thomas C. Hales
% LaTeX Format

%!TEX TS-program = latex    
%% This line is for TexShop. 

\documentclass[spanningrule]{cambridge7A}
%% Cambridge University Press Macros from
%% https://authornet.cambridge.org/information/productionguide/laTex_files/

% required by CUP
\usepackage[numbers]{natbib}
\usepackage{rotating}
\usepackage{floatpag}
 \rotfloatpagestyle{empty}
\usepackage{amsthm}
\usepackage{graphicx}
\usepackage{multind}\ProvidesPackage{multind}
\usepackage{times}

% my additions
\usepackage{verbatim}
\usepackage{latexsym}
\usepackage{amsfonts}
\usepackage{amsmath}
\usepackage{crop}
\usepackage{txfonts}
\usepackage[hyphens]{url}
\usepackage{setspace}
%\setstretch{2}  % for double spacing

% fonts
\usepackage[mathscr]{euscript} % powerset.
\usepackage{pifont} %ding
\usepackage[displaymath]{lineno}

%% hyperref interferes with the \part macro of cambridge
\usepackage[letterpaper,colorlinks=true,%
  citecolor=red,%
  %breaklinks=true,%
  %pdftex,
  ps2pdf,%
  hyperindex=true]{hyperref}

% Also interferes with \part macro of CUP.
% http://tug.ctan.org/tex-archive/macros/latex/contrib/pdfcomment/doc/pdfcomment.pdf
\usepackage{pdfcomment}

%%%%%%%%%%%%%%%%%%%%%%%%%%%%%%%%%%%%%%%%%

% This file contains local settings and system dependencies



% Auxiliary directories
\def\dsp{/Users/thomashales/Pictures/mathFigures/DenseSpherePackings}  % flypaper graphics
%\def\pdf{/Users/thomashales/Pictures/collect_geom} % tarski graphics
\def\pdfp{/Users/thomashales/Pictures/mathFigures/collect_geom} % kepler graphics

\def\showgraphics{t}  
% t: display graphics (there are none to show yet)
% f (default): print a "no graphics logo" where graphics would normally go.


\def\displayallproof{t} 
% t (default): display all proofs.
% f: print documents without the proofs-- theorem statements only

\def\displayrating{f}
% t (default): display all ratings (verbose is also true)
% f : don't show them.

\def\verbose{t}
% f (default): do not display debugging information,
% t : display debug information and information about the formalization.

     
%-%
% --Repository--
%-%
% generate revision number by
% svn propset svn:keywords "LastChangedRevision" kepmacros.tex
\def\svninfo{%
  Document TeXed on \today. \hfill\break
  Repository Root: https://flyspeck.googlecode.com/svn \hfill\break
  SVN $LastChangedRevision$
  }

%-%
% --Fonts--
%-%
\font\twrm=cmr8

%-%
% --Graphics--
%-%
%set \showgraphics option in flag.tex
% flypaper graphics
 \def\szincludegraphics[#1]#2{%
      \if\showgraphics t{\includegraphics[#1]{#2}}%
      \else{\includegraphics{noimage.eps}}\fi}

% % kepler graphics
% \def\pdffigtemplatex[#1]#2#3#4{%   
% % usage: \pdffigtemplatex[width=80mm]{file.eps}{labelname}{caption}
% \begin{figure}[htb]%
%   \centering
%  \szincludegraphics[#1]{\pdfp/#2}
%  \caption{#4}
%  \label{fig:#3}%
% \end{figure}%
% }

\def\tikzfig#1#2#3{%
\begin{figure}[htb]%
  \centering
\begin{tikzpicture}#3
\end{tikzpicture}
  \caption{#2}
  \label{fig:#1}%
\end{figure}%
}

\def\tikzwrap#1#2#3#4{%
\begin{wrapfigure}{r}{#4\textwidth}
  \begin{center}
\begin{tikzpicture}#3
\end{tikzpicture}
\end{center}
  \caption{#2}
  \label{fig:#1}%
\end{wrapfigure}%
}


%\def\pdfg#1#2#3#4{\if\showgraphics t{\pdffigtemplatex[#1]{#2}{#3}{#4}}\else{}\fi}
%\def\myincludegraphics#1{%
%      \if\showgraphics t{\includegraphics{#1}}%
%      \else{\includegraphics{noimage.eps}}\fi}


%-%
% --Footnotes and Endnotes--
%-%
% http://help-csli.stanford.edu/tex/latex-footnotes.shtml
%\long\def\symbolfootnote[#1]#2{\begingroup%
%\def\thefootnote{\ensuremath{\fnsymbol{footnote}}}\footnote[#1]{#2}\endgroup}

%-%
% --Special Formatting--
%-%
% http://en.wikibooks.org/wiki/LaTeX/Formatting#List_Structures
%\renewcommand{\labelitemii}{$\star$}
\renewcommand{\labelitemii}{$\circ$}
\renewcommand{\labelenumii}{\alph{enumii}}
\newenvironment{summary}
  {\begingroup\bigskip\narrower\noindent{\bf Summary.~}\it}
%  {~\ding{98}\par\phantom{!}\endgroup\bigskip}
  {~\par\phantom{!}\endgroup\bigskip}
\newenvironment{tidbit}{\smallskip\begingroup}{\endgroup\smallskip}
%\newenvironment{enumerate}
%  {\renewcommand{\labelitemi}{}\begin{itemize}}
%  {\end{itemize}}
\def\wasitemize{\relax}
\def\uncase#1{{\sc #1}}
\def\case#1{{\sc (#1)}}
\def\claim#1{{\it  #1}}
\def\calcentry#1#2#3#4{{\smallskip{\bf #1}\quad{\tt [#2]}\quad{(#3)}\quad {#4}}} % computer calc entry
\def\id#1{\ensuremath{\text{\tt #1}}}



%-%
% --Indexing, References, Citations--
%-%
\def\indy#1#2{\index{index/#1}{#2}}
%\def\eqn#1{{\bf (\ref{#1})}}   % deprecated, use \eqref.
\def\newterm#1{\indy{Index}{#1}{\it #1}\relax}
\def\oldterm#1{\indy{Index}{#1}{#1}\relax}
\def\cc#1#2{%
  \indy{Index}{computer calculation!{#1}}{\it computer calculation}%
  \ifverbose{\footnote{\guid{#1}  #2}}~\cite{website:FlyspeckProject}} % arg dropped.


%-%
% --Endnotes
%-%
%\renewcommand{\maketextnotes}{\global\textnotesontrue
%  \newwrite\textnotes
%  \immediate\openout\textnotes=\jobname.ent
% \literaltextnote{
%\notesheadername={\the\textnotesheadername}
%%\pagestyle{endnotesstyle}
%\mark{3}
%\label{textualnotes}
%\normalfont \backmattertextfont}
%}
%\newcommand{\shipnotes}{
%   \iftextnoteson
%   \theendnotes
%   \immediate\closeout\textnotes
%   \input \jobname.ent
%   \else
%   \relax
%   \fi
%}

%-%
% --Proof Display--
%-%
% set with \displayallproof in flag_fly. If f, then proofs are swallowed.
%% "proved" environment. toggle with \displayallproof
%
\def\hide#1{}
\def\swallowed{\relax}
\def\swallow#1\swallowed{}
\newenvironment{iproved}{}{}
\newenvironment{proved}{\resetproved\begin{iproved}}{\end{iproved}}
\def\hideproof{\renewenvironment{iproved}{%
   \centerline{\it -- Proof Proofed --}
  \renewenvironment{itemize}{}{}
  \renewenvironment{enumerate}{}{}
  \def\item{\relax}
  \catcode13=12
  \swallow
}{}}
\def\showproof{\renewenvironment{iproved}{\begin{proof}}{\end{proof}}}
\def\resetproved{\if\displayallproof t\showproof\else\hideproof\fi}



%-%
% --Debugging Information--
%-%
%% verbose:
\def\rating#1{\if\displayrating t%
  {{\textsc {[rating={\ensuremath {#1}}].\ }}}\else{}\fi}
\def\rz#1{\rating{#1}}
\def\cutrate{}
\def\oldrating#1{\if\displayrating t%
  {{\textsc {[former rating={\ensuremath {#1}}].\ }}}\else{}\fi}

\def\formalauthor#1{\if\verbose t{{\tt [formal proof by #1].\ }}\else{}\fi}
\def\dcg#1#2{{\if\verbose t%
  {{\tt{[DCG-#1]}}\indy{References}{ZC{#2 #1}@{DCG-#1}|page{#2}}}\else{}\fi}}
\def\tlabel#1{\label{#1}\if\verbose t{{\tt [#1].\ }%
   \indy{References}{#1|itt}}\else{}\fi}
\def\ifcverbose#1#2{\if\verbose t{{#1}}\else{#2}\fi}
\def\ifverbose#1{\ifcverbose{#1}{}}  %\verbose t{{#1}}\else{}\fi}
%\def\formal#1{\ifverbose{[{#1}]}}
\def\formal#1{\relax }
\def\formaldef#1#2{\ifverbose{\texttt{[{#1} $\leftrightsquigarrow$ {#2}]}}}
\def\footformal#1{\if\verbose t{\footnote{#1}}\else{}\fi}
\def\guid#1{{\tt[#1].\ }\indy{References}{ZA{#1}@{#1}|itt}}
\def\guid#1{\ifverbose{{\tt [#1]}}}
\def\guid#1{{{\tt [#1]}}}
\def\ineq#1{{{\tt  [#1]}}}
%\def\guid#1{{{\tt [#1]}}}

%\def\calc#1{{\textsc{calc-#1}}\indy{Interval}{{#1}@{#1}}}
%\def\xfootnote#1{\if\verbose t{\endnote{#1}}\else{\footnote{#1}}\fi}
%\def\xfootnote#1{\footnote{#1}}
%\def\xendnote#1{\if\verbose t{\endnote{#1}}\else{}\fi}


% margin notes
\setlength{\marginparwidth}{1.2in}
\def\mar#1{}
 %\ifverbose{\marginpar{\text{\raggedright\footnotesize #1}}}}
%\def\hypermark[#1]#2{\ifcverbose{\hyperref[#1]{#2}}{#2}}


%-%
%--Formatting--
%-%
\def\dfrac#1#2{\frac{\displaystyle #1}{\displaystyle #2}}
\def\textand{\text{ \ and \ }}  % for math eqns.

%-%
%--Redefining--
%-%
\def\emptyset{\varnothing}
\def\ups{\upsilonup} % Needs txfonts; else use \upsilon


%-%
% --Symbols--
%-%
% norm and brackets
\def\|{{\hskip0.1em|\hskip-0.15em|\hskip0.1em}}
\def\mid{\ :\ }
\def\tc{\hbox{:}}
\def\cooln{:\hskip-0.02em:}
\def\norm#1#2{\hbox{\ensuremath{\|#1\unskip-\unskip{#2}\|}}}
\def\normo#1{{\|#1\|}}
\def\sland{\ \land\ }
\def\abs#1{|#1|}
% brackets
\def\leftopen{(}
\def\leftclosed{[}
\def\rightopen{)}
\def\rightclosed{]}
\def\lp#1{{\llbracket{#1}\rrbracket}} 
%\def\comp#1{\llbracket #1 \rrbracket}
\def\comp#1{[#1]}
\def\tangle#1{\langle #1\rangle}
\def\ceil#1{\lceil #1\rceil}
\def\floor#1{\lfloor #1\rfloor}
%accents:
\def\=#1{\accent"16 #1}
\def\ast{\ensuremath{{}^*}}

% mathcal
\def\CalV{{\mathcal V}}
\def\CalL{{\mathcal L}}
\def\BB{{\mathcal B}}
\def\powerset{{\mathscr P}}

% mathbb
\newcommand{\ring}[1]{\mathbb{#1}}
%\def\N{{\mathbb N}}
%\def\Rp{\ring{R}^{3\,\prime}}
%\def\A{{\mathbf A}}
\def\F{{\mathbf F}} % map on faces H to H/{cal L}

% vector notation
\def\v{{\mathbf v}}
\def\u{{\mathbf u}}
\def\w{{\mathbf w}}
\def\e{{\mathbf e} }  
\def\p{{\mathbf p}}
\def\q{{\mathbf q}}

% operatorname
\def\op#1{{\operatorname{#1}}}
\def\optt#1{{\operatorname{{\texttt{#1}}}}}

%\def\opat{{\op{@}}}
\def\atn{\op{arctan\ensuremath{_2}}}
\def\azim{\op{azim}}
\def\nd{\op{node}}
\def\sol{\operatorname{sol}}
\def\vol{\op{vol}}
\def\dih{\operatorname{dih}}
\def\Adih{\operatorname{Adih}}
\def\arc{\operatorname{arc}}
\def\rad{\operatorname{rad}}
\def\bool{\operatorname{bool}}
\def\true{\op{true}}
\def\false{\op{false}}


%\def\orz{\varthetaup} % center of packing
\def\orz{{\mathbf 0}} % center of packing
\def\Wdarto{W^0_{\text{dart}}}
\def\Wdart{W_{\text{dart}}}
%\def\Wedge{W_{\text{edge}}}
\def\cell{\operatorname{cell}}
\def\dimaff{\operatorname{dim\,aff}}
\def\aff{\operatorname{aff}}
\def\card{\op{card}}

\def\del{\partial}
\def\doct{\delta_{oct}}
\def\dtet{\delta_{tet}}
\def\hm{{h_0}} % 1.26
\def\stab{c_{{\scriptstyle \text{stab}}}} % 3.01
\def\tgt{\operatorname{\it{target}}}
\def\pqr#1{#1} % marks type (p,q,r).
\def\trunc#1#2{#1\hbox{\ensuremath{[:\hskip-0.25em plus 0em minus 0em{#2}]}}}
\def\trunc#1#2{#1[\text{:}\hskip0em plus 0em minus 0em{#2}]}
\def\trunc#1#2{d_{#2}{#1}}
%\def\trunc#1#2{#1{[\le\hskip-0.25em{ #2}]}}

%% HYPERMAP macros:
% avoid e for both hypermap edge and edge {v,w}
\def\ee{\varepsilonup}
\def\ocirc{}
\def\wild{{*}}  % wildcard char.

%% PACKNG macros:
%\def\lam{\lambda}
%\def\Lam{\Lambda}
\def\bu{{\underline{\u}}}
\def\bv{{\underline{\v}}}
\def\bw{{\underline{\w}}}
\def\bV{{\underline{V}}}
%\def\arcs#1#2#3{{\arcV(#1,\{#2,#3\})}}

%% LOCAL FAN macros:
\def\smain{S_{\scriptstyle\text{main}}} 
 
 

% line numbers
\def\lll{\resetlinenumber[1]}
\def\linenumberfont{\normalfont\small\sffamily}

\def\tocpart#1{
  \addcontentsline{toc}{part}{\Large{#1}}}
% (or even \LARGE)
  
\crop
%\makeindex
\makeindex{index/Notation}
\makeindex{index/Index}

\def\linput#1{\lll\input{#1}}


\raggedbottom  % for now.
%\raggedright  % don't worry for now.

%%% end notes -kill this.
%%\maketextnotes

%%%%%%%%%%%%%%%%%%%%%%%%%%%%%%%%%%%%%%%%

% new theorems

\theoremstyle{plain}
\newtheorem{theorem}[equation]{Theorem}
\newtheorem{lemma}[equation]{Lemma}

\newtheorem{background}[equation]{Background}
\newtheorem{corollary}[equation]{Corollary}
\newtheorem{example}[equation]{Example}
\newtheorem{assumption}[equation]{Assumption}
\newtheorem{interpretation}[equation]{Interpretation}
\newtheorem{conjecture}[equation]{Conjecture}

\theoremstyle{definition}
\newtheorem{definition}[equation]{Definition}

\theoremstyle{remark}
\newtheorem{remark}[equation]{Remark}
\newtheorem{notation}[equation]{Notation}
%\newtheorem{note}[equation]{Author's Note}
\newtheorem*{note}{Author's Note.}
\newtheorem{calculation}[equation]{Calculation}
\newtheorem{assertion}[equation]{Assertion}
\newtheorem{exer}[equation]{Exercise}

%%%%%%%%%%%%%%%%%%%%%%%%%%%%%%%%%%%%%%%%%

\begin{document}
\title[for {\it Dense Sphere Packings}]{A Guide to Computer Calculations}
\author{Thomas C. Hales}
    
%%%%%%%%%%%%%%%%%%%%%%%%%%%%%%%%%%%%%%%%%
%%% FRONT
    \frontmatter
    \maketitle
    \tableofcontents

%%%%%%%%%%%%%%%%%%%%%%%%%%%%%%%%%%%%%%%%%

   \newpage
 
%%%%%%%%%%%%%%%%%%%%%%%%%%%%%%%%%%%%%%%%%
  \mainmatter
  \begin{runninglinenumbers*}
 
\begin{note}%XX
  This document is not intended for publication.  It contains some
  further notes about the proof that may be of interest to a few
  experts.
\end{note}

\begin{note}% 
This document gives a summary of the nonlinear inequalities that have
been cited in the chapter on packings.  Information about the computer
verifications can be found at \cite{hales:2009:nonlinear}.  Some
of these interval arithmetic calculations are still in the process
of being verified.  The approach to the proof of the Kepler
conjecture described here is still work in progress.  A description
of the inequalities and their current status can be found
at~\cite{hales:2009:nonlinear}.
\end{note}


\begin{note}
The following table gives a summary of the proofs in the book where
computer calculations have been used.  The table lists the lemma
number, gives a tracking number, indicates whether the calculation
involves nonlinear optimization (NL), linear programming (LP), or
graph generation (GR).  The table also gives a brief description of
the nature of the calculation.  The tracking number can be used to
search the internet for further details of the computer calculation.
\end{note}

\chapter{Nonlinear Inequalities}

\chapter{Linear Programming}

\chapter{Graph Generator}


\chapter{Packing}


\section{Packings, General Inequalities}


(Note that the following is an inequality in at most six variables; the most
difficult case to prove is that of a $4$-cell.)  Formulas for the
volumes and solid angles appear in Chapter~\ref{chapter:volume}.  An
explicit formula for the dihedral angle appears in
Chapter~\ref{part:trig}.


\begin{calculation}\label{calc:marchal}\guid{HJKDESR}\guid{1025009205}\guid{3564312720}\cutrate{}
% old \guid{WJDLOCM}
% 
%% cc:mar are the k-cell estimates for noncell clusters.
Define the function $M$ by equations (\ref{eqn:M}) and
(\ref{eqn:m-def}).  Define the function $\gamma(X,M)$ by equation
(\ref{eqn:gamma-def}).  If $X$ is a $0$, $1$, $2$, $3$, or $4$-cell,
then
\[ 
\gamma(X,M)\ge 0.
\] 
\end{calculation}

\begin{calculation}\label{calc:cc:qtr}\guid{GLFVCVK}\guid{4869905472}\guid{2477216213}\guid{8328676778}\cutrate{}
Let $\gamma_L$ be given by Definition~\ref{def:gammaL}, $\op{wt}$ by
Definition~\ref{def:wt}, and $\beta$ by Definition~\ref{def:beta}.
If $X$ is any $k$-cell that is not a quarter with $k\in\{2,3,4\}$,
then % gammaL is nonneg on quarters. cc:qtr
\[ 
\gamma_L(X) \op{wt}(X) + \beta(e,X)\ge 0.
\]  
\end{calculation}

\begin{calculation}\label{calc:cc:2bl}\guid{FHBVYXZ}\guid{1118115412}\cutrate{}
Let $\gamma_L$ be given by Definition~\ref{def:gammaL}.  Let $X$ be
any quarter.  Let $Y$ be a $3$-cell that flanks it.  Then
\[ 
\gamma_L(X)+\gamma_L(Y)\ge 0,
\] 
% 2-leaf calculation, gammaL(fourcell)+gammaL(threecell) >=0. % cc:2bl:
\end{calculation}

\begin{calculation}\label{calc:cc:5bl}\guid{ZTGIJCF}\cutrate{}
Let
\[ 
a= 0.0560305, \textand   b= -0.0445813.
\] 
\begin{enumerate}\wasitemize 
\item \case{1821661595} A $4$-cell $X$ along a spine $e$ satisfies
\[ 
\gamma_L(X)\op{wt}(X) + \beta(e,X) \ge a + b\,\op{azim}(X),
\] 
\item \case{7907792228} The $2$-cell $X_2$ and two $3$-cells $X_1,X_3$
that flank it along a spine $e$ satisfy
\[ 
\sum_{i=1}^3 \left(\gamma_L(X_i)\op{wt}(X_i) + \beta(e,X_i)\right)\ge a + b\,\sum_{i=1}^3\op{azim}(X_i).
\] 
\end{enumerate}\wasitemize 
\end{calculation}

\begin{calculation}\label{calc:cc:disks}\guid{4198769118}\cutrate{} 
% was guid 8550443271
Let
\[ 
g(h) = \arccos(h/2) - \pi/6.
\] 
If $h_1,h_2\in [1,\hm]$, then
\[ 
\op{arc}(2h_1,2h_2,2) - g(h_1) - g(h_2)\ge 0.
\] 
\end{calculation}

\begin{calculation}\label{calc:cc:alin}\guid{7991525482}\cutrate{}
Let $L$ be given by Definition~\ref{def:L}.
Let
\[ 
g(h) = \arccos(h/2) - \pi/6.
\] 
Let
\[ 
\op{reg}(a,k) = 2\pi - 2 k (\arcsin(\cos(a)\sin(\pi/k))).
\] 
Then
\[ 
\op{reg}(g(h),k) \ge c_0 + c_1 k + c_2 L(h),\quad
k = 3,4,\ldots,\quad 1\le h\le \hm,
\] 
where
\[ c_0 = 0.6327,\quad c_1 = -0.0333,\quad c_2 =
0.4754.\] 
\end{calculation}

\begin{calculation}\label{calc:cc:alin2}\guid{8540377696}\cutrate{}
Let $L$ be given by Definition~\ref{def:L}.
Let
\[ 
g(h) = \arccos(h/2) - \pi/6.
\] 
Let
\[ 
\op{reg}(a,k) = 2\pi - 2 k (\arcsin(\cos(a)\sin(\pi/k))).
\] 
Let
\[ a'=\arc(2,2,2\hm)-g(\hm) \approx
0.797.\]  Then for $k=3,~4$,\dots
\[ \op{reg}(a',k) \ge c_0 + c_1 k + c_2 L(1) +
c_3\] 
where 
\[ c_0 = 0.6327,\quad c_1 = -0.0333,\quad c_2 =
0.4754,\quad c_3 = 0.85.\] 
\end{calculation}

\begin{calculation}\label{calc:shorts}\cutrate{}
The following calculations involve many cases that are enumerated by
computer code.
\begin{enumerate}\wasitemize 
\item \case{BIXPCGW} Let $Z$ be any cell-cluster along a spine $e$
with three leaves.  Then
\[ 
\Gamma(Z)> 0.
\] 
\item \case{QITNPEA} Let $Z$ be any cell-cluster along a spine $e$
with four leaves.  Then
\[ 
\Gamma(Z)> 0.
\] 
\end{enumerate}\wasitemize 
\end{calculation}



\chapter{Cluster Inequality}

Turn to the proof\footnote{This section is not to be included
in the published version of the book.  It gives additional details about the
main cluster inequality.} of Lemma~\ref{lemma:cluster}.  Let $Z$ be a cell
cluster and let $\ee$ be the critical edge shared by the cells in the
cluster.  Call this critical edge the \newterm{spine}.  Consider all
the faces along the spine $\{ \u_0, \u_1\}$, consisting of three
points $\{ \u_0, \u_1, \u_2\}$ in the packing with circumradius less
than $\sqrt2$.  Call such a face a \newterm{leaf}. 
\indy{Index}{edge!spine}%
\indy{Index}{edge!critical}%
\indy{Index}{leaf}%

Consecutive cells around the spine are \newterm{adjacent}.  A cell
adjacent to a $4$-cell is a $3$-cell or another $4$-cell.  A cell
adjacent to a $2$-cell is a $3$-cell.  There are no $0$- or $1$-cells
along a critical edge.  \indy{Index}{adjacent}%
\indy{Index}{quarter}%

Call a $4$-cell a \newterm{quarter}, when it has exactly one critical
edge and all other edges of the simplex have length at most $2 h_-$.
The weight of any quarter is $1$.


\section{Two Leaves}

The proof is  divided into cases, according to the number of
leaves along the critical edge.

\claim{It suffices to prove Lemma~\ref{lemma:cluster} when the cluster
contains at least one quarter and at least two leaves.} Indeed, by
Calculation~\ref{calc:cc:qtr}, if $X$ is any cell,
then % gammaL is nonneg on quarters. cc:qtr ~[GLFVCVK]
\[  
\gamma(X,L) \op{wt}(X) + \beta(\ee,X)\ge 0
\]  
when $X$ is not a quarter.  This is the desired inequality.  Also, a
quarter is flanked by two leaves.

Consider a cluster with exactly two leaves.
Without loss of generality, the two leaves flank a quarter
$X$. 
The azimuth angle of what remains outside the quarter
is greater than $\pi$.  Thus, there must be a $3$-cell
along each leaf.  Let $Y$ be one of these $3$-cells.
The cell $Y$ has weight $1$.
Then 
\begin{equation}\label{eqn:cc2bl} 
\Gamma(Z)\ge \gamma(X,L)+\gamma(Y,L)\ge 0,
\end{equation}
by Calculation~\ref{calc:cc:2bl}.
%a calculation~\cite[FHBVYXZ]
% 2-leaf calculation, gammaL(fourcell)+gammaL(threecell) >=0. % cc:2bl:
\indy{Index}{azimuth}%
\indy{Index}{angle!azimuth}%

\section{Five Or More Leaves}

Let $B$ be the set of cells in the cluster that lie between any two
consecutive leaves.  $B$ is either a singleton set containing a
$4$-cell, or a set of three cells: a $2$-cell and two adjacent
$3$-cells.  Write $\op{azim}(B)$ for the azimuth angle formed by the
two leaves.  By Calculation~\ref{calc:cc:5bl}, the cells between two
consecutive leaves satisfy an
inequality: % five or more leaves along a spine are found
% at~[ZTGIJCF].  % cc:5bl:

\[  
\sum_{X\in B} \gamma(X,L)\op{wt}(X) + \beta(\ee,X) \ge a + b\,\op{azim}(B),
\] 
where
\[  
a= 0.0560305, \textand   b= -0.0445813.
\] 
It follows that
\[  
\Gamma(Z) \ge 5 a + b\, (2\pi) > 0.
\] 

\section{Three Or Four Leaves}\label{sec:3or4}

In the case of three and four leaves, the proof relies to an even
greater extent on computer calculations.  
Lemma~\ref{lemma:cluster} asserts a nonlinear inequality.  The inequality
asserts that some continuous function $f$ is positive on a compact
domain $D$ in Euclidean space.  Rigorous methods of global nonlinear
optimization can be used to show that the function $f$ is indeed
positive.

Details about global nonlinear optimization appear in the appendix.
The method is linear relaxation, which can be briefly described as
follows.\footnote{Linear relaxations appear again later in the
solution of significantly more difficult nonlinear optimization
problems.}  The domain $D$ is partitioned into finitely many subsets
$D_1,\ldots, D_r$.  The positivity of $f$ is established on each
subset $D_i$.  The graph $\{(f(x), x)\mid x\in D_i\}$ of $f$ on $D_i$
is a subset of a polyhedron $P_i$.  The inequalities defining the
polyhedron can be determined explicitly.  A linear program computes
the minimum of the first coordinate over $P_i$.  This minimum is
positive.  Since the graph is contained in $P_i$, the values of $f$ on
$D_i$ must also be positive.

This strategy has been implemented and gives the desired lower bound.
The linear programming has been implemented as a \newterm{MathProg} model
in Calculation~\ref{calc:shorts}.\footnote{MathProg implements a
subset of the \newterm{AMPL} modeling language.  The computer code
giving the model appears in shorts.mod and the computer code
controlling the branching appears in shorts.ml.}  This model
contains the number of leaves, variables representing the edges and
the azimuth angles between consecutive leaves.

The proof that each graph is contained in a explicit polyhedron $P_i$
relies on nonlinear inequalities.  In fact, the proof requires over
one hundred nonlinear inequalities that have been established by
computer.

A function on the set $D$ with finite range partitions $D$ into
finitely many subsets, according to the image of a point.  Several
simple boolean functions and functions with finite range were used to
partition $D$ into subsets $D_i$: Are there three leaves or four?  Is
there a $2$-cell in the cluster?  Is the $4$-cell a quarter?  Is the
azimuth angle of a given cell greater than $2.3$?  Is the value of
$\gamma(\wild,L)$ on a given quarter negative? Does a particular leaf have a
nonspinal edge of length greater than $2h_-$?  What is the weight of a
given $4$-cell?  The inequalities that define the polyhedra $P_i$ have
been designed specifically for the subdomains $D_i$, according to the
answers to these questions.

These linear relaxations and are sufficient to prove the bound in all
but one case.  This is the case of four leaves, three quarters, and
one $4$-cell that has two critical edges: the spine and the edge
opposite the spine.  The other four edges of the $4$-cell have length
at most $2h_-$.  In this case as well, the inequality is established
by linear relaxation, but things are more involved.

The method to prove this case by computer is as follows.  Number the
four simplices $j=1,2,3,4$, with $j=1$ representing the $4$-cell of
weight $1/2$.  Write $\gamma^j$ in abbreviation of
$\gamma(X,L)\op{wt}(X) + \beta(\ee,X)$.  The desired inequality is
\begin{equation}\label{eqn:gpos} 
\sum_{j=1}^4 \gamma^j > 0.
\end{equation}
The domain is a set of ordered four-tuples of simplices
$X_1,\ldots,X_4$ that fit together into an octahedron (the dihedral
angles along the spine sum to $2\pi$, the four simplices all have the
same spline length $y_1(X_j)$, and the shared edges are the same
length for the two simplices sharing the edge).  Instead of the single
inequality asserted by Lemma~\ref{lemma:cluster}, many inequalities of
the following form are established:
\begin{equation}\label{eqn:gpart} 
\gamma^j + a_i \dih^j + b_i^j y_1 + c_i^j (y_2+y_3+y_5+y_6) + d_i^j > 0, 
\end{equation}
for all $X \in I_i^j$, \quad $i \in I$, and $j\in \{1,2,3,4\}$.  Here,
$\gamma^j$, $\dih^j$, and $y_i$ are all functions of $X$.  The simplex
$X$ has spine of length $y_1$ and other edges with lengths $y_k$.
Also, $\dih^j$ is the dihedral angle of $X$ along the spine.  $I$ is a
finite indexing set.  Each domain $I_i^j$ is a product of intervals in
$\ring{R}^6$, under the parameterization of a simplex by the lengths
of its six edges.  The union over $i$ of the sets
\[  
\{(X_1,\ldots,X_4)\mid~ X_j \in I_i^j,\text{ for } j=1,2,3,4.\}
\] 
covers the entire domain of the desired inequality~\eqn{eqn:gpos}.   

\claim{A subset of this union already covers the domain.}  Indeed, the
dihedral angles of the four simplices along the spine sum to $2\pi$:
\[  
\sum_{j=1}^4 \gamma^j = 2\pi.
\] 
Furthermore, the spine lengths of the simplices must agree: $y_1(X_j)
= y_1(X_k)$ for all $j,k\in\{1,2,3,4\}$.  Finally, each leaf flanks
two different simplices; the edge lengths of the leaf must agree.
This gives a collection of inequalities of the form $y_i(X_j) =
y_{i'}(X_{j'})$.  These are the subset relations.

The coefficients of the inequalities~\eqn{eqn:gpart}
are chosen so that for each $i$, the following inequality holds:
\begin{equation}\label{eqn:glin} 
0 > a_i 2\pi + 
\sum_{j=1}^4 (b_i^j y_1 +  c_i^j (y_2(X_j)+y_3(X_j)+y_5(X_j)+y_6(X_j)) + d_i^j).
\end{equation}
(Notice that this inequality is linear in the variables $y_k(X_j)$ and
the domain is a product of intervals.  Hence, this inequality is
particularly easy to check.)  The sum of Inequality~\eqn{eqn:gpart} over $j$, Inequality~\ref{eqn:glin},
and the subset relation yield the desired inequality~\ref{eqn:gpos}.

A four-leafed cluster is an octahedron.  The set of octahedra is a
$13$-dimensional object, parameterized by twelve external edges and one
diagonal (the spine).  By contrast, a simplex in $\ring{R}^3$ is only
a $6$-dimensional object, parameterized by six edges.  The four-leafed
cluster inequality is an inequality in thirteen dimensions.  The preceding
arguments reduce the $13$-dimensional inequality into a series of
$6$-dimensional inequalities.  The six-dimensional inequalities are
within within the reach of a computer.

The only question is how the magical coefficients
$a_i,b_i^j,c_i^j,d_i^j$ are obtained.  Clearly, the
inequalities~\eqn{eqn:glin} and \eqn{eqn:gpart} are linear in these
coefficients.  Thus, a they can be found by linear programming.  It is
somewhat troubling that there are infinitely many constraints, as each
point $X_j$ in the domain gives one constraint.  In practice, the
infinite number of constraints can be replaced by a finite collection.
Some information is lost in approximating the system with a finite
number of constraints.  However, the coefficients, once they are
guessed by approximations, can be checked independently by computer.
(An exact fit it not necessary; any coefficients that work will do.)
It was necessary to partition the domain into smaller pieces to
produce coefficients that work.  In other words, the indexing set $I$
contains more than one element.  After some experimentation, we found
a set $I$ of cardinality twenty-three that works.  Through these methods,
coefficients were found.



\chapter{Local Fan}

\begin{calculation}\guid{2065952723}\cutrate{}\label{calc:Lexell}
%See Mathematica numerical calculation.
Let
\[ 
g(s;a,b,c,e_1,e_2,e_3) = \sum_{i=1}^3 \dih_i(2,2,2,a+s,b,c) e_i,
\] 
where $\dih_i$ is given by Definition~\ref{def:tau}.
Let $\Delta = \Delta(4,4,4,a^2,b^2,c^2)$.
Let primes denote derivatives with respect to the variable $s$.
Assume that
$e_i\in\leftclosed1,1+\sol_0/\pi\rightclosed$,  that
$a,b,c\in\leftclosed2/\hm,4\rightclosed$.
Then
\begin{equation}\label{eqn:calc:Lexell}
  \Delta (g'(0;a,b,c,e_1,e_2,e_3))^2 
- 0.01\Delta^{3/2}g''(0;a,b,c,e_1,e_2,e_3) > 0.
\end{equation}
(The factors of $\Delta$ clear the denominator in
(\ref{eqn:calc:Lexell}) to simplify the inequality to be proved.)
\end{calculation}

\begin{calculation}\guid{2158872499}\cutrate{}\label{calc:2der}
%% checked in Mathematica NMaximize
Let $y_1,y_2\in \leftclosed 2,2\hm\rightclosed$.  
\begin{enumerate}\wasitemize 
\item 
Let $g(t) = \arc(y_1,y_2+t,2)$.  Then $g''(0) < 0$.
Explicitly,
\[ 
  g''(0) = \dfrac{
    -64 + 48y_1^2 - 12 y_1^4 + y_1^6 
  + 80 y_2^2 - 8 y_1^2 y_2^2 - 3 y_1^4 y_2^2
    - 12 y_2 ^4 + 3 y_1^2 y_2^4 - y_2^6
  }{y_2^2 \sqrt{\ups(y_1^2,y_2^2,4)}^3}
\] 
and the polynomial in the numerator takes negative values on the given
domain.
\item
Let $g(t) = \arc(y_1+t,y_2-t,2)$.  Then $g''(0) < 0$.
Explicitly,
\[ 
  g''(0) = \dfrac{\sqrt{\ups(y_1^2,y_2^2,4)} \left(
      -4 y_1^2 + y_1^4 - 4y_1^3 y_2 - 4y_2^2 
   + 6 y_1^2 y_2^2 - 4 y_1 y_2^3 +y_2^4
    \right)}{y_1^2 y_2^2 (2+y_1-y_2)^2 (2+y_2-y_1)^2}
\] 
and the polynomial in the numerator takes negative values on the given
domain.
\end{enumerate}\wasitemize 
\end{calculation}

\begin{calculation}\guid{2986512815}\cutrate{}\label{calc:cc:qua}  %m11
Let $y_1y_2,y_3,y_7\in \leftclosed 2,2\hm\rightclosed$,
$y_5,y_8,y_9\in \{2,2\hm\}$, $y_4,y_6\ge 2\hm$.
Let $x_i = y_i^2$.
Assume that
\[ 
\Delta(x_1,x_2,x_3,x_4,x_5,x_6)>0,\quad{ and }
\Delta(x_3,x_2,x_7,x_9,x_8,x_4)>0.
\] 
Assume that
\[ 
\dih(y_3,y_1,y_2,y_6,y_4,y_5)+\dih(y_3,y_2,y_7,y_9,y_8,y_4) < \pi
\] 
and\footnote{If $\{\v_1,\ldots,\v_4\}$ is a set of vectors such that
$y_i = \normo{\v_i}$ and $y_{ij} = \norm{\v_i}{\v_j}$, then
$\op{cross}(y_4,\ldots,y_{12}) = \norm{\v_2}{\v_4}$.}
\[ 
\op{cross}(y_1,y_2,y_3,y_4,\ldots,y_9) \ge y_4.
\] 
Let 
\[ g(t;y_1,\ldots,y_9) =
  \tau_{tri}(y_1,y_2,y_3,y_4+t,y_5,y_6)+\tau_{tri}(y_3,y_2,y_7,y_9,y_8,y_4+t).
\] 
Then \[ g'(0)^2 - 0.01 g''(0) > 0.\] 
\end{calculation}


\begin{calculation}\guid{EFJSUSK}\cutrate{}\label{calc:irred} %cc:tau
%%cc:par
Let $(V,E,F,G)$ be an irreducible minimal fan with parameters
$(r,s)$.  Then $\tau(V,E,F) \ge d(r,s)$.  (A separate calculation
has been made for each of the cases in the list given above.)
% Interval arithmetic
% calculations~ %% cc:par partition cases for tau[r,s].
\end{calculation}










%\begin{calculation}\guid{5779862781}\cutrate{}\label{calc:cc:d2a}
%  Let $y_5,y_6\in \{ 2,2\hm\}$, $y_1,y_2,y_3\in \leftclosed
%  2,2\hm\rightclosed$, and $y_4\in \leftclosed 2,4\hm\rightclosed$.
%  Let $g(t) = \tau_{tri}(y_1+t,y_2,y_3,y_4,y_5,y_6)$.  If
%  $\Delta(y_1^2,y_2^2,\ldots,y_6^2)> 0$ and if $ g'(0)=0, $ then
%  $g''(0)<0$.\footnote{The function $g(t)$ may fail to be
%    differentiable at $t=0$ for parameters $y_1,\ldots,y_6$ for which
%    $\Delta(y_1^2,\ldots,y_6^2)=0$.  Thus, it is necessary to work on
%    the noncompact domain $\Delta>0$ for this inequality.}
%%  (The proof is an interval arithmetic calculation over a
%%  four-dimensional space~cc:d2a.  %%cc:d2a
%%  The calculation verifies that the second derivative is negative
%%  whenever the derivative is zero.)
%\end{calculation} 
%

%\begin{calculation}\guid{6645853705}\cutrate{}\label{calc:cc:d2b}
%  Let $y_5\in \{2,2\hm\}$, $y_1,y_2,y_3\in \leftclosed
%  2,2\hm\rightclosed$, and $y_4\in \leftclosed 2,4\hm\rightclosed$.
%  Let $g(t) = \tau_{tri}(y_1+t,y_2,y_3,y_4,y_5,y_6)$.  If
%  $\Delta(y_1^2,y_2^2,\ldots,y_6^2)> 0$, if
%\[ 
%\arc(y_1,2\hm,2) + \arc(y_2,2\hm,2) 
%\le y_6 \le \arc(y_1,2,2\hm)+\arc(y_2,2,2\hm)
%\]  
%and if
%$
%g'(0)=0,
%$
%then $g''(0)<0$.
%\end{calculation}
%
%\begin{calculation}\guid{5606476569}\cutrate{}\label{calc:cc:qua}
%Let $y_{12},y_{23},y_{34},y_2,y_3\in\{2,2\hm\}$.
%Let $y_1,y_4\in \leftclosed 2,2\hm\rightclosed$.
%Let $y_{14}\in\leftclosed 2\hm,y_1+y_4\rightclosed$.
%Let $y_{13}\in\leftclosed 2\hm,y_1+y_3\rightclosed$.
%Let $g(t) = \tau_{tri}(y_1+t,y_2,y_3,y_4,y_5,y_6)$.
%If 
%\[ 
%\Delta(y_1^2,y_2^2,y_3^2,y_{23}^2,y_{13}^2,y_{12}^2)> 0 \textand 
%\Delta(y_1^2,y_4^2,y_3^2,y_{43}^2,y_{13}^2,y_{14}^2)> 0,
%\]  
%if\footnote{If $\{\v_1,\ldots,\v_4\}$ is a set of vectors such that
%  $y_i = \normo{\v_i}$ and $y_{ij} = \norm{\v_i}{\v_j}$, then
%  $\op{cross}(y_4,\ldots,y_{12}) = \norm{\v_2}{\v_4}$.}
%\[ 
%\op{cross}(y_4,y_1,y_3,y_{13},y_{34},y_{14},y_2,y_{23},y_{12})\ge y_{13},
%\] 
%and if
%$
%g'(0)=0,
%$
%then $g''(0)<0$.
%\end{calculation}
%




\chapter{Tame Hypermap}

\chapter{Further Results}

\section{Strong Dodecahedral Conjecture}


\begin{calculation}[local inequality]\guid{PWVDMPT}
The local inequality  has separate cases for $D_k$-cells for
$k=1,2,3,$.
\end{calculation}

The $D_1$-cell inequality, after dividing through by the solid angle, becomes
the easy inequality:
\[
r^2 + 3 a_D \approx 0.257 \ge 0,\quad r = \sqrt2.
\]

The $D_2$-cell inequality, after dividing through by the dihedral angle, becomes
\[
 (r^2-h^2)/2 + 3 a_D (1-h/r)  + 3 b_D L(h) \ge 0, \quad h\in [1,r],\quad r = \sqrt2.
\]
This holds even if $r$ is as small as $1.34077$.

\begin{calculation}[local inequality]\guid{9627800748}
\end{calculation}

\begin{proof}[discussion.]
In the $D_4$-cell, the domain is
$y_i\in[2,\sqrt8]$, but we can restrict to $y_i\in[2,2.75]$ as follows.
If some edge has length greater than $2.75$, the simplex has
circumradius at least $1.375$.  Truncate at $1.375$, breaking the
simplex into $1$-, $2$-, and $3$-cells.
\end{proof}

\begin{calculation}[local inequality]\guid{6938212390}
\end{calculation}

\begin{proof}[discussion.]
The $D_3$-cell  has domain $y_i\in[2,\sqrt8]$ and
truncation $r=\sqrt2$, but the result holds for much smaller
truncation.  We can restrict to $y_i\in [2,2.75]$ by decreasing the
truncation to $1.375$ and breaking the $D_3$-cell into $1$-, $2$-cells.

If the circumradius is bigger than $1.35$, we can truncate at $1.35$, breaking
it into $1$- and $2$-cells.  
\end{proof}


The circumradius constraint in the $D_4$-cell inequality is not necessary.




\section{Fejes T\'oth's Contact}

\chapter{Linear Programming}

% Chqpter: Packing

\chapter{Misc Calculations}

\calcentry{Lemma~\ref{lemma:MI}}{XHVKJFS}{NL}{This is Marchal's
  inequality for $k$-cells. It separates into cases according to
  $k\in\{0,1,2,3,4\}$.}

\calcentry{Lemma~\ref{lemma:cluster}}{SQMANBP}{NL}{This is the cell
  cluster estimate for clusters.  It separates into cases according to
  the number of blades.}

\calcentry{%Lemma~\ref{lemma:13-14},~
  Inequality~\eqn{eqn:disks}}{8550443271}{NL}{The inequality asserts
  the disjointness of disks on the unit sphere of variable radius
  $g(h)$.}

\calcentry{Inequality~\eqn{eqn:alin}}{7991525482}{NL}{This is a linear
  lower approximation to the area of a regular spherical $n$-gon.}

\calcentry{Inequality~\eqn{eqn:alin2}}{8540377696}{NL}{This is a
  linear lower approximation to the area of a regular spherical
  $n$-gon.}

% Chapter: Local Fan:

\calcentry{Inequality~\eqn{eqn:g''}}{2065952723}{NL}{This is a
  calculation of the sign of a second derivative to show that the
  function $\tau$ does not have a local minimum as a function of the
  edge lengths.  It initially appears to depend on six variables, but
  the dependence on three of the variables is linear and is
  extremal at the endpoints.}

\calcentry{Inequality~\eqn{eqn:arc''}}{2158872499}{NL}{This is a
  calculation of the sign of a second derivative of $\arc$.  There are
  two cases.  It is an optimization problem in two variables.}

\calcentry{Lemma~\ref{lemma:triangle-free}}{MPWFVCK}{NL}{The main
  estimate holds for triangles.  There are four cases depending on the
  cardinality of $S$.  It is an optimization problem in three
  variables.}

\calcentry{Lemma~\ref{lemma:flat-exists}}{2986512815}{NL}{The function
  $\tau$ does not have a local minimum along the indicated curve.  It
  is an optimization problem in four variables.}

\calcentry{Lemma~\ref{lemma:quadrilateral-free}}{FVPVRZP}{NL}{The main
  estimate holds for quadrilaterals.  There are two cases according to
  whether there is a flat node.  When there is no flat node, it is an
  optimization in five variables.  When there is a flat node, there
  are further subcases $0,1,2$ according to the cardinality of $S$.
  Each subcase is an optimization problem in four variables.}

\calcentry{Lemma~\ref{lemma:pentagon-free}}{DAKGBMR}{NL}{The main
  estimate holds for pentagons.  There are two flat nodes, giving two
  cases according to whether the flat nodes are adjacent.  Each case
  has two two subcases $0,1$ according to the cardinality of $S$.
  Each subcase is an optimization problem in as many as five
  variables.}

\calcentry{Lemma~\ref{lemma:hexagon-free}}{INRCFAH}{NL}{The main estimate
  holds for pentagons.  There are several cases according to the
  configuration of flat nodes.  Depending on the case, there may be as
  many as seven variables.}




\calcentry{Lemma~\ref{lemma:face-size}}{UTIICTO}{NL}{A linear lower
  approximation to the function $\arc$ is computed.}

\calcentry{Lemma~\ref{lemma:D-local}}{XRULQNO}{NL}{This is the main cell inequality that arises in the
proof of the strong dodecahedral conjecture.  It breaks into cases according to the type of cell
$k=2,3,4$.}


%\begin{tabular}{llll}
%Lemma~\ref{lemma:face-size}&VEXKBSL&NL&A linear lower approximation to the function $\arc$ is computed.\\
%\end{tabular}


    
    \end{runninglinenumbers*}

\end{document}
%%%%%%%%%%%%%%%%%%%%%%%%%%%%%%%%%%%%%%%%%%

