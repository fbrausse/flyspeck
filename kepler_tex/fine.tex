\chapter{Geometric Detail}%DCG "The S-System" Sec.9, p85
    \label{sec:fine}
    \oldlabel{2}

The previous chapter gives  constructions unaided
needed to describe the top-level proof.  
Those constructions do not take us far.   This
chapter gives details needed to carry the proof to completion. 
We have discussed many of these constructions before, such
as V-cells, the Q-system, fans, and fitted crowns.  In this
chapter we look at how these structures are interrelated.

The top-level proof calls for
lower bounds on the volume of a $V$-cell.  
The strategy is to dissect the $V$-cell into small disjoint
pieces, each with an easily computed volume.  The volume
bound comes by adding these separate pieces of volume.
This chapter describes these pieces and shows that
they are disjoint subsets of the $V$-cell.  The next chapter
begins to add the contributions together.

If two pieces of the dissection lie in different connected components of
the complement of a fan, then they are disjoint.  This, in fact,
gives a useful way to  prove that pieces are disjoint. 

\section{Interaction}


%Let $\CalQ=\CalQ(\Lambda)$ be the $Q$-system.  
%For $v\in\Lambda$, let $\CalQ(\Lambda,v)$ be the subset
%of those with a vertex at $v$.\index{Qv@$\CalQ(\Lambda,v)$}
%% Defined already. def:q-system.


The exact volume of a general $V$-cell is never calculated.  
Lower bounds suffice.  L. Fejes T\'oth esimated
the volume of a Voronoi cells by intersecting it with a ball
$B(v_0,t)$ of some radius $t$.  This intersection is the
{\it truncated} Voronoi cell.  Truncation  also works for $V$-cells.
Often the truncated $V$-cell turns out to be almost identical to the
truncated Voronoi cell, which is much easier to estimate.
The chapter begins with some lemmas that compare the truncated $V$-cell with
the truncated Voronoi cell.


\begin{lemma} \tlabel{lemma:voronoi-truncation-over-Q}\rating{100}
Let $(\Lambda,v_0)$ be a centered packing.  Let $Z$ be the union
of the sets $\op{aff}_+(v_0,\{v_1,v_2\})$ as $\{v_0,v_1,v_2\}$
runs over the barriers with a vertex at $v_0$.  Let $X$ be the
union of the following tips:
   $$
   \op{aff}_+^0(v_0,\{v_1,v_2,v_3\}) \cap \op{aff}_-
   (\{v_1,v_2,v_3\},v_0)\cap \Omega_2(\Lambda,v_0).
   $$
as $\{v_0,v_1,v_2,v_3\}$ ranges over $\CalQ(\Lambda,v_0)$.
Then 
$$\Omega_2(\Lambda,v_0)
 \subset Z\cup X\cup \op{VC}(\Lambda,v_0).$$
%If $x$ lies in the \index{Voronoi cell} Voronoi cell at $v_0$, 
%but not in the $V$-cell at $v_0$, then there exists a
%simplex $Q\in\CalQ(\Lambda,v_0)$, such that $x$ lies in the cone (at $v_0$)
%over $Q$. Moreover, $x$ does not lie in  $\op{conv}^0(Q)$.
(That is, up to the null set $Z$, the $V$-cell contains
the truncated tip-reduced Voronoi cell.)
\end{lemma}

\begin{proof}  Let 
 $$x \in \Omega_2(\Lambda,v_0)
 \setminus (Z\cup\op{VC}(\Lambda,v_0)).
 $$
By
the definition of $V$-cell, there is a barrier $\{v_1,v_2,v_3\}$
such that
  $$\op{conv}\{v_1,v_2,v_3\}\cap \op{conv}^0\{v_0,x\}\ne\emptyset.$$ 
We have $v_0\not\in\{v_1,v_2,v_3\}$, for otherwise $x\in Z$, 
which is contrary to assumption. 
By tarski\tarf{tarski:vor-bar-tet}\tarf{tarski:vor-bar-quad},
the simplex $Q=\{v_0,v_1,v_2,v_3\}$ is a quasi-regular tetrahedron
or a flat quarter.  If the barrier is the face of a flat quarter,
then $Q$ too is in
the $Q$-system.  By tarski\tarf{tarski:pass-cone},  $x\in\op{aff}_+(v_0,\{v_1,v_2,v_3\})$.  
We get $x\in\op{aff}_+^0(v_0,\{v_1,v_2,v_3\})$ from $x\not\in Z$.

We have $x\in\op{aff}_-
   (\{v_1,v_2,v_3\},v_0)$. Otherwise, $\op{conv}^0\{v_0,x\}$ does
not meet $\op{conv}\{v_1,v_2,v_3\}$.
%We have $x\in\op{conv}(Q)\setminus Z$. Otherwise, 
%the barrier contains $x$, and the plane through the barrier contains
%$v_0$, so that $\{v_0,v_1,v_2,v_3\}$ is planar.  This is contrary
%to tarski\tarf{tarski:flat-Q}. 
The rest is clear.
\end{proof}

Although different estimates require different
truncation parameters, the truncation parameter $t_0$ that appears in
the next lemma is the most widely used.  A smaller truncation parameter never
appears.  A larger parameter is used only when the parameter $t_0$
has been tried and found wanting.  With tight truncation at $t_0$,
 the $V$-cell and Voronoi cell become
indistinguishable.

\begin{lemma}\tlabel{lemma:VC-Omega}\rating{100}
Let $(\Lambda,v_0)$ be a centered packing.
Inside the ball of radius $t_0$ at $v_0$, the $V$-cell and
Voronoi cell coincide up to a null set:
   $$B(v_0,t_0)\cap \op{VC}(\Lambda,v_0) \equiv B(v_0,t_0)\cap \Omega(\Lambda,v_0).$$
\end{lemma}

\begin{proof} Let $Z$ be the (finite) union of planes equidistant from $v_0$ and $w\in \Lambda(v_0,2t_0)$.  $Z$ is a null set.  Let $Y$ be the (finite) union of $\op{aff}_+(w,\{w_1,w_2\})$, for $w\in \Lambda(v_0,2t_0)$ and $\{w,w_1,w_2\}$ a barrier.  $Y$ is a null set.
The Voronoi cells $\Omega(\Lambda,w)$, $w\in\Lambda(v_0,2t_0)$ cover
$B(v_0,t_0)$ except for the null set $Z$.  
Suppose that some point  $x\not\in Z\cup Y$
lies in $$(B(v_0,\Lambda)\cap\op{VC}(\Lambda,v_0) )\setminus \Omega(\Lambda,v_0).$$
Then $x\in \Omega(\Lambda,w)$, where
$w\ne v_0$.  
% By Lemma~\ref{tarski:unobstr-t0}, 
From $x\not\in Y$, it follows that the point $v_0$ is
unobstructed  at $x$ by tarski\tarf{tarski:unobstr-t0}.  
Thus, $|x-w|< |x-v_0|\le t_0$.  By
tarski\tarf{tarski:unobstr-t0} again, $w$ is unobstructed at $x$, so
that $x\in \op{VC}(\Lambda,w)$, contrary to the assumption
$x\in\op{VC}(\Lambda,v_0)$.  Thus $B(v,t_0)\cap\op{VC}(\Lambda,v_0)\subset\Omega(\Lambda,v_0)$.

Similarly, if $x\in B(v_0,t_0)\cap \Omega(\Lambda,v_0)$ and $x\not\in Y$,
then $x$ is
unobstructed at $v_0$, and $x\in \op{VC}(\Lambda,v_0)$.
\end{proof}

\bigskip

%\begin{remark} The next lemma helps to determine which $V$-cell
%a given point $x$ belongs to.  If $x$ lies in the open cone over a
%simplex $Q_0$ in $\CalQ$, then Lemma~\ref{lemma:Q-divide}
%describes the $V$-cell decomposition inside $Q$;  beyond $Q$ the
%point $v_0$ is obstructed by a face of $Q$, so that such $x$ do not lie
%in the $V$-cell at $v_0$. 
%%If $x$ does not lie in the open cone over
%%a simplex in $\CalQ$, but lies in the open cone over a standard
%component $R$, then Lemma~\ref{lemma:V-cell-local} describes the
%$V$-cell.  
%It states in particular, that for unobstructed $x$, it
%can be determined whether $x$ belongs to the $V$-cell at the
%point $v_0$ by considering only the vertices $w$ that lie in the closed
%cone over $R$ (the standard component containing the radial
%projection of $x$). In this sense, the intersection of a $V$-cell
%with the open cone over $R$ is {\it local\/} to the cone over $R$.
%\end{remark}
%


A few situations call into play a set slightly larger than the set of barriers.
Let $\CalB'(\Lambda,v_0)$ be the set of triangles $T\subset \Lambda$ such that at least one
of the following holds:
\begin{itemize}
    \item $T$ is a barrier at $v_0$, or
    \item $T=\{v_0,v,w\}$ consists of a diagonal of a quarter in the
    $Q$-system together with one of its anchors.
\end{itemize}


A $V$-cell  can be dissected by choosing a fan
$(v_0,V,E)$ and intersecting it with the various connected components
$U\in [Y(v_0,V,E)]$.  To describe the intersection $A$, it
helps to know that the number and location of
most elements of $\Lambda$ have no effect on the shape of this intersection.
In the best of worlds, the intersection is determined solely 
by the vertices of $V$
at darts that lead into $U$.  When this occurs, the intersection $A$ 
{\it decouples} from the surrounding vertices. Decoupling  can
produce
a drastic drop in the complexity and dimension of the problem.
The following key lemma provides this decoupling.

%DCG Lemma 5.29, page 50.
\begin{lemma} [Decoupling Lemma]\tlabel{lemma:V-cell-local}\rating{100}
%Let $x\in I_0$, the cube of side $4$ centered at $v_0$
%parallel to coordinate axes. 
Let $(\Lambda,v_0)$ be a centered packing.  Let $\{v_0,v_1,v_2,w\}$
be a set of four vertices of $\Lambda$. 
Assume that the closed segment
$\op{conv}\{x,w\}$ intersects $\op{aff}_+(v_0,\{v_1,v_2\})$, where
$F = \{v_0,v_1,v_2\}\in \CalB'(\Lambda,v_0)$. Assume that $v_0$ 
is not obstructed at $x$. Assume that $x$ is closer to $v_0$ 
than to both $v_1$ and $v_2$. Then $x\not\in\op{VC}(\Lambda,w)$.
\end{lemma}
\index{decoupling lemma}

%\begin{remark}  The Decoupling Lemma is a crucial result.  It
%permits estimates of the scoring function in
%Chapter~\ref{sec:scoring} to be made separately for each standard
%component.  The estimates for separate standard components are far
%easier to come by than estimates for the score of the full
%centered packing.  
%\end{remark}

\begin{proof}
Assume for a contradiction that $x$ lies in $\op{VC}(\Lambda,w)$. In
particular, we assume that $w$ is not obstructed at $x$.  Since
$v_0$ is not obstructed at $x$, $w$ must be closer to $x$
than $x$ is to $v_0$.

By tarski\tarf{tarski:decouple}, 
   $|w-v_0|,|w-v_1|,|w-v_2|\le 2t_0$.  Thus, $Q=\{v_0,w,v_1,v_2\}$ is
a quarter or a quasi-regular tetrahedron.  By the definition of
$\CalB'(\Lambda,v_0)$, the face $F$ must lie in the $Q$-system.  Thus,
$F$ is a barrier.  Again, by tarski\tarf{tarski:decouple},
$\op{conv}(F)$ meets the segment from $x$ to $w$, so $x$ is obstructed
at $w$.  Thus, $x$ does not lie in $\op{VC}(\Lambda,w)$.
\end{proof}

%\section{Local Optimality}%DCG Sec. 8, p72  
%\tlabel{sec:local-opt}
%Moved after the main estimate.




\section{Fitted Crown}%DCG 9.1, p95
    \label{sec:fine-overview}
    \oldlabel{2.1}

Whenever possible, the truncated $V$-cell, defined as the intersection
of the $V$-cell with a ball $B(v_0,t)$,  replaces the 
$V$-cell.  A tight truncation parameter $t=t_0$ cuts away the peculiarities
of the $V$-cell and leaves something reassuringly similar to a Voronoi cell.
When a tight truncation parameter fails to give the desired estimates,
looser parameters, described in Section~\ref{sec:alphabet}, step in.  
These looser truncation parameters are only allowed in carefully controlled contexts (called the alphabet simplices and pure quad clusters).

In a few rare cases, even the looser truncation parameters fail to give the necessary bounds.  This is where fitted crowns enter.  They are the stopgap 
solution when all else fails.    A fitted crown is a small piece of a $V$-cell
that lies entirely outside the ball $B(v_0,t_0)$ used for tight truncation.
(The name {\it fitted crown} comes from its shape, it can resemble a small
flat-topped crown sitting on top of the round ``head'' $B(v_0,t_0)$.)

Designed as a stopgap, the fitted crowns are awkward to work with.
It is not at all evident that they really are subsets of the $V$-cell or even
that they are disjoint from one another.  The volume of a crown depends
continuously on its parameters, but not differentiably.  The fitted crown
is unique in this regard.
Consequently, the interval arithmetic calculations that estimate volumes
of fitted crowns are some of the slowest calculations of the entire lot.

In fact, 
Section~\ref{sec:anc} introduces fitted crowns and gives their volume. 
That section also introduces  
various auxiliary sets $\FC$, $\FCinner$, $\rogFC$ in
Definition~\ref{def:fitted-crown}.  The reader is advised to review those
definitions before continuing.  The fitted crown is a subset of $\FC$.
However, $\FC$ is easier to work with than the fitted crown itself.  
To show that fitted crowns are disjoint, it is enough to show that
the sets $\FC$ are disjoint.  To show that fitted crowns lie in the $V$-cell,
again it suffices to show it for $\FC$.  
This section continues to develop the
properties of fitted crowns, now
in the context of $V$-cells and the $Q$-system
of a sphere packing.  


\subsection{crown tuple}%DCG 9.2, p86
    \label{sec:deltaP}
    \oldlabel{2.3}


A fitted crown in a centered packing depends on a tuple of
four vertices $(v_0,v_1,w_1,w_2)$.  Definition~\ref{def:crown-tuple} codifies
the precise conditions on four-tuples
that carry a fitted crown.

\begin{definition}[$\eta_{V0}$]\label{def:eta0}
Set $\eta_{V0}(v,w) = \eta(|v-w|,2,2t_0)$. 
\index{zzeta@$\eta_{V0}$}
\end{definition}

%By tarski\tarf{tarski:1453}, 
%if $h\le\sqrt2$, then $\eta(2h,2,2t_0)\le \eta(2,2t_0,\sqrt2) <
%1.453$.\index{ZZZZ1.453@1.453}


\begin{definition}[crown~tuple]\label{def:crown-tuple}
Let $\Lambda$ be a packing.
We say that a four-tuple $(v_0,v,u,w)$ of vertices in $\Lambda$   
is a crown tuple if
\begin{itemize}
  \item $2t_0 < |v-v_0| <\sqrt8$.
  \item $\{v_0,v\}$ is an upright diagonal of the $Q$-system.
  \item $u$ and $w$ are anchors of $\{v_0,v\}$.
  \item $w$ is the successor of $u$ in the azimuth cycle with respect to
   $(v_0,v)$ on the
   anchors of $\{v_0,v\}$.
  \item Either
\begin{enumerate}  
    \item  $\op{azim}(v_0,v,u,w)\ge\pi$, or
    \item $\op{azim}(v_0,v,u,w) <\pi$, 
 $|u-w|\ge 2.77$,
    $\rad_V(v_0,v,u,w)\ge\eta_{V0}(v,v_0)$, and the max of 
    $\eta_V(v_0,u,w)$ and $\eta_V(v,u,w)$ is
    $\ge\sqrt2$.
    \label{enum:wedge2}
\end{enumerate}
\end{itemize}
\index{crown tuple}
We say that $(v_0,v,u,w,\epsilon)$, with $\epsilon\in\{\pm1\}$ is
a signed crown tuple if $\epsilon=1$ and $(v_0,v,u,w)$ is a crown tuple
or if $\epsilon=-1$ and $(v_0,v,w,u)$ is a crown tuple.
\index{signed crown tuple}
\end{definition}

The signed crown tuple carries no more information than the crown
tuple it represents.  A signed crown tuple is a bit of
syntactic sugar to swap the third and fourth entries
of the crown tuple by flipping a sign.  The fitted crown is
a composite region and  two of its primitive constituents are
Rogers simplices.   A flip in sign toggles between these two Rogers
simplices.  As this is done frequently, the syntax accommodates it.

The construction of fitted crown in Section~\ref{sec:anc} 
relies upon a condition: Assumption~\ref{eqn:q1q2}.  
The assumption is always satisfied for a crown tuple.

\begin{lemma}\tlabel{lemma:fitted-q1q2}\rating{100}
Let $\Lambda$ be a sphere packing with crown tuple
$(v_0,v,w_1,w_2)$.  Let $c_1=c_2=\eta_{V0}(v_0,v)$. 
Then Assumption~\ref{eqn:q1q2} holds for $(v_0,v,w_1,w_2,c_1,c_2)$.
\end{lemma}

\begin{proof} Assume not.  
We use the notation of Section~\ref{sec:anc}.
We can decrease $\op{azim}(v_0,v,w_1,w_2)$
until $\op{azim}<\pi$.  Let $S=\{v_0,v,w_1,w_2\}$.
The inequality $2.77 \le |w_1-w_2|$ implies that the simplex $S$
has positive orientation at $w_1$ and $w_2$ (by tarski\tarf{tarski:neg-orient-quad}). 
Let $p$ be the circumcenter of $S$.
If $\rad_V\{v_0,v,w_1,w_2\}\ge\eta_{V0}(v_0,v)$, then
  $$
  \begin{array}{lll}
  \op{azim}(v_0,v,w_1,q_1)&\le \op{azim}(v_0,v,w_1,p)\\
  \op{azim}(v_0,v,q_2,w_2)&\le \op{azim}(v_0,v,p,w_2).
  \end{array}
  $$
Then 
  $$
  \begin{array}{lll}
  \op{azim}(v_0,v,w_1,q_1) &\le \op{azim}(v_0,v,w_1,p)\\
    &= \op{azim}(v_0,v,w_1,w_2) - \op{azim}(v_0,v,p,w_2)\\
    &\le \op{azim}(v_0,v,w_1,w_2) - \op{azim}(v_0,v,q_2,w_2)\\
    &= \op{azim}(v_0,v,w_1,q_2).
  \end{array}
  $$
This is Assumption~\ref{eqn:q1q2}. 
\end{proof}



%Fix $i,j$, with $j\equiv i+1\mod k$. If $W = W_i$ is a crown tuple, 
%let $\{v_0,v_i,v\}^\perp$ be the plane through $v_0$ and
%the circumcenter of $\{v_0,v_i,v\}$, perpendicular to $\{v_0,v_i,v\}$.
%Skip the following step if the circumradius of $\{v_0,v_i,v\}$ is
%greater than $\eta_{V0}(v,v_0)$, but if the circumradius is at most
%this bound, the plane
%of $\{v_0,v_i,v\}^\perp$
%intersects the right circular cone boundary of $D_0$ along two rays
%emanating from $v_0$.  Let $c_i$ be a point on the ray (selected on
%the $W$-side of $\{v_0,v_i,v\}$).  Simlarly, we construct the point
%$c_i'$ for $\{v_0,v,v_j\}$ (again on the $W$-side).
%
%Define $\theta=\theta(v)$ by $\cos\theta = |v-v_0|/(2\eta_{V0}(v,v_0))$.
%If Condition~2 holds, we let $c$ be the 
%circumcenter of $\{v_0,v_i,v_j,v\}$.  The angle
%at $v_0$ between $c$ and $v$ is
%$\theta'$, where
%$$\cos\theta' = |v-v_0|/(2\rad)\le |v-v_0|/(2\eta_{V0}) = \cos\theta.$$
%We conclude that $\theta'\ge\theta$ and $c$ does not lie in $D_0$.
%Thus, the half-planes
%   $$
%   A_i,\quad B_i=\op{aff}_+(\{v_0,v\},c_i),\quad 
%   B'_i = \op{aff}_+(\{v_0,v\},c'_i), \quad
%   A_j
%   $$
%are ordered cyclically around $\{v_0,v\}$. (Set $B_i=A_i$
%or $B'_i=A_j$, if the corresponding circumradius is greater than
%$\eta_{V0}(v,v_0)$.)
%Let $W'=W'_i$ be the open wedge of $D_0$ between $B_i$ and $B'_i$.
%Let
%    $$E_w = \{x : 2 x\cdot w \le w\cdot w\},$$
%for $w = v,v_i,v_j$. These are half-spaces bounding the Voronoi
%cell. Set $E_\ell = E_{v_\ell}$.
%

%
%\begin{definition}[FC'] \label{def:delta-e}
%In both cases (Conditions~1 and~2), let $W$ be the wedge between
%$v_i$ and $v_j$ along $(v_0,v)$, $W'$ the smaller wedge, 
%and let $c=\eta_{V0}(v,v_0)$ in
%    $$
%    \begin{array}{lll}
%      \FC'(v,W) &= [E_v\cap W'\cap D_0]\cup \op{rog}^0(v_0,v,v_i,p_i,c)
%      \cup \op{rog}^0(v_0,v,v_j,p_j,c)
%      \\
%    \FC(v,W) &= \FC'(v,W) \cup 
%    \op{rog}^0(v_0,v_i,v,p_i,c)
%    \cup \op{rog}^0(v_0,v_j,v,p_j,c)\\
%    \bigd(v,W) &= \{x\in\FC(v,W)\mid |x-v_0|>t_0\},
%    \end{array}
%    $$
%where $p_i,p_j$ are selected so that the simplices $\op{rog}^0$ lie
%in the wedge $W$.
%\index{zzdelta@$\bigd(v,W)$}
%\index{zzDelta@$\FC(v,W)$}
%\end{definition}
%

%\begin{remark} We note that the union is actually a disjoint union,
%and that each of the pieces is one of the primitive regions, so
%the volume of $\FC$ is immediate.
%\end{remark}


%\begin{remark}
By definition, a Rogers simplex (Definition~\ref{def:rog}) is  the
empty set 
if the corresponding parameters are not coherent.  It is to
be understood that all discussion regarding this set
should be disregarded in the case that it is empty.
%\end{remark}



\subsection{normal tuple}


Fitted crowns are disjoint from one another and are
subsets of the $V$-cell.  The proofs of these facts 
require various preliminaries.  To prepare for the proofs,
we give various lemmas showing that fitted crowns are
disjoint from blades $\op{aff}_+(v_0,\{u_1,u_2\})$ formed from
barriers $\{v_0,u_1,u_2\}$.  The notion of {\it normal tuple} of vertices
collects together the various types of barriers.  
The following definition serves this technical short-lived purpose.  Its scope
is limited to the early sections of this chapter.


\begin{definition}[normal]  Let $(v_0,v,u_1,u_2)$ be a tuple of vertices
in a packing $\Lambda\subset\ring{R}^3$.  We say that it is {\it normal} if $2t_0<|v-v_0|<\sqrt8$
and one of the
following holds:
\begin{enumerate}
  \item (qrt) $\{v_0,u_1,u_2\}$ is a quasi-regular triangle.
  \item (upright) $\{v_0,u_1,u_2\}$ is an upright triangle; that is, say,
    $2t_0 < |u_1-v_0| < \sqrt8$, and $u_2$ is an anchor of $\{v_0,u_1\}$.
    (or vice-versa, swapping $u_1$ with $u_2$).
  \item (flat)
   $2t_0<|u_1-u_2|<\sqrt8$; and if $u_1,u_2$ are both anchors of
   $\{v_0,v\}$, 
   then
    no further anchor of $\{v_0,v\}$
   lies in the lune $\op{aff}^0_+(\{v_0,v\},\{u_1,u_2\})$.
\end{enumerate}
\end{definition}



The next lemma motivates the
definition, by showing that normal tuples detects every intersection
of a fitted crown with a barrier blade.
In this and following lemmas, we adopt a uniform notation.  Let $(\Lambda,v_0)$ be a centered packing, upright diagonal 
$\{v_0,v\}$, signed crown tuple
$(v_0,v,w,w',\epsilon)$.
%Set $R_w=\rogFC(v_0,v,w,w',\epsilon)$.
%R_w=\op{rog}^0(v_0,w,v,p,\eta_{V0}(v,v_0))$,
%where $p$ is selected so that $R_w$ lies in $W(v_0,v,w,w')$.

\begin{lemma}\tlabel{lemma:meet-normal}\rating{100}
Let $(\Lambda,v_0)$ be a centered packing.  Let $\{v_0,u_1,u_2\}$
be a barrier.  Let $(v_0,v,w_1,w_2)$ be a crown tuple.  If
$\FC(v_0,v,w_1,w_2)$ meets $F=\op{aff}_+(v_0,\{u_1,u_2\})$, then
$(v_0,v,u_1,u_2)$ is normal.
\end{lemma}

\begin{proof} For barriers of qrt and upright type, there is nothing to show.  Consider
a barrier of flat type.  Thus we may assume that $u_1$ and $u_2$ are both anchors of
$\{v_0,v\}$, and need to show that $u_1,u_2$ are consecutive anchors.

By definition, $w_1,w_2$ are consecutive anchors around $\{v_0,v\}$.
We have $\FC\subset \op{aff}_+^0(\{v_0,v\},\{w_1,w_2\}) = A_w$
and $F\subset \op{aff}_+(\{v_0,v\},\{u_1,u_2\}) = A_u$.
These lunes are disjoint unless $A_w \subset A_u$.  Assume this.

If $\{u_1,u_2\}=\{w_1,w_2\}$, we are done, because $w_1,w_2$ are consecutive.
Say $w_2\ne u_1,u_2$.  If $\op{aff}_+(v_0,\{v,w_2\})$ meets $F$, then by tarski\tarf{tarski:pass-makes-quarter},
$\{v_0,v,w_2,u\}$ for $u=u_1$ or $u=u_2$ is an upright quarter along $\{v_0,v\}$.  It
is in the $Q$-system.  As $\{v_0,u_1,u_2\}$ is the face of a quarter in the $Q$-system,
two quarters in the $Q$-system overlap. 
This is contrary to Lemma~\ref{thm:nonoverlap}.

So $w_2\in\op{aff}_+(v_0,\{v,u_1,u_2\})$.  By tarski,
 $\{v_0,v,w_2,u_i\}$ is a quarter, for
$i=1,2$.  Also, $w_2,u_i$ are consecutive.  So $w_1\in\{u_1,u_2\}$ and $|w_1-w_2|\le 2.51$,
which is contrary to the definition of a crown tuple.
\end{proof}

Our aim is to confine fitted crowns, eventually showing that they are separate
from one another 
and that they cannot escape the confines of the $V$-cell (that is, separate from
all other $V$-cells).  To set up such separation results, we must 
identify intervening obstacles separating various objects that appear to
be close.
The following lemma has just this form.  The hypothesis
is a closeness condition (a certain circumradius is small) and the conclusion is the
existence of an intervening obstacle: an anchor $w'$.

\begin{lemma}\tlabel{lemma:new-anchor}\rating{80}
Let $(\Lambda,v_0)$ be a centered packing
  Let $S=\{v_0,v,w,u\}$ be a simplex in $\Lambda$.  Assume that $\{v_0,v\}$ is an
upright diagonal, that $w$ and $u$
are anchors of $\{v_0,v\}$, and that $\rad_V(S)< \eta_{V0}(v,v_0)$.
Assume there is a crown tuple $(v_0,v,w,w')$
extending the triple $(v_0,v,w)$
(on the same side of the face as $u$).  
Then the anchor $w'$ of $\{v_0,v\}$ lies between $u$ and $w$
(that is,  $w'\in\op{aff}_+^0(\{v_0,v\},\{u,w\})$) with
    $|w'-w|\ge2.77$ and
    $\rad\{v_0,w,w',v\} \ge \eta_{V0}(v,v_0)$.
Furthermore, $\{v_0,v,w',u\}$ is an upright quarter in the $Q$-system.
\end{lemma}

\begin{proof} The conditions on $(v_0,v,w,u)$ are incompatible with the
conditions of Definition~\ref{def:crown-tuple} defining crown tuples.
Therefore, $w$ and $u$ cannot be consecutive anchors around
$\{v_0,v\}$.  Let $w'$ be the anchor such that $(v_0,v,w,w')$
is a crown tuple.  Since $w$ and $w'$ are consecutive anchors,
we see that $w'$ must be between $u$ and $w$.  It must have the
second type in Definition~\ref{def:crown-tuple}.  

We have $|w'-u|\le 2t_0$ by Lemma~\ref{tarski:pass-makes-quarter}.  
Thus, $\{v_0,v,w',u\}$
is an upright quarter.  By the definition of crown anchors, $\{v_0,v\}$
is an upright diagonal of the $Q$-system, so that $\{v_0,v,w',u\}$ is in
the $Q$-system.
The conclusion follows.
\end{proof}

We are finally ready to establish that the fitted crown does not meet 
barrier blades.  That is the content of the following series of lemmas.
Lemma~\ref{lemma:delta-tri} combines these individual lemmas into the statement that each
fitted crown is contained in a single connected component of $Y(v_0,V,E)$, where
$(v_0,V,E)$ is a fan formed from barriers.

\begin{lemma}\tlabel{lemma:FC-}\rating{80}  
Let $S=(v_0,v,u_1,u_2)$ be normal and
let $S'=(v_0,v,w_1,w_2)$ be a crown tuple in a packing $\Lambda$.
Let $a = |v-v_0|/(2\eta_{V0}(v_0,v))$.  
Then
$$
 \op{rcone}^0(v_0,v,|v-v_0|/(2\eta_{V0}(v_0,v))) \cap
  \op{aff}^0_+(\{v_0,v\},\{w_1,w_2\})
$$
does not meet $F=\op{cone}(v_0,\{u_1,u_2\})$.
In particular, the subset
$\FCinner(S')$ does not meet $F$.
\end{lemma}

\begin{proof}  Assume for a contradiction that the sets meet.
Tarski\tarf{tarski:eps-bigd-}
implies that $S$ is normal of the flat or qrt variety, and 
that $u_1$ and $u_2$ are anchors of $\{v_0,v\}$.  
The result also gives that 
$\rad_V(S)<\eta_{V0}(v,v_0)$.    
In the qrt case, $S$ forms
an upright quarter.  By tarski\tarf{tarski:consec-anchors}, $u_1$ and $u_2$
are consecutive anchors. In the flat case as well, 
$u_1$ and $u_2$ are consective
anchors around $\{v_0,v\}$.

The set $\{u_1,u_2\}$ is not $\{w_1,w_2\}$ because $(v_0,v,u_1,u_2)$
does not satisfy the properties of a crown tuple (assuming $u_1,u_2$
indexed so that $\op{azim}(v_0,v,u_1,u_2) = \dih_V(\{v_0,v\},\{u_1,u_2\})$).  The anchors $u_1,u_2$ are consecutive, as are $w_1,w_2$.
The set $\FCinner(S')$ lies in the lune $\op{aff}_+^0(\{v_0,v\},\{w_1,w_2\})$ and $F$ lies in the lune $\op{aff}_+(\{v_0,v\},\{u_1,u_2\})$.
These lunes are disjoint.
%
%If the crown tuple $S'$ is not in $W(v_0,v,u_1,u_2)$, 
%then $\FC$ lies outside the 
%lune $\op{aff}_+^0(\{v_0,v_0\},\{u_1,u_2\})$, but $F$ lies in it.  Thus,
%the crown tuple must run between $u_1$ and $u_2$.   This contradicts 
%the rules for forming crown tuples.  (Lemma~\ref{lemma:new-anchor} 
%implies that
%anchors $u_1$ and $u_2$ cannot be consecutive.)
\end{proof}

\begin{lemma}\tlabel{lemma:fine-Rw}\rating{50}
Let $S=(v_0,v,w,u_1)$ be normal and $(v_0,v,w,w',\epsilon)$ a signed crown tuple
of a packing $\Lambda$.  
Then
$\rogFC(v_0,v,w,w',\epsilon)$
does not meet $F=\op{cone}(v_0,\{w,u_1\})$.
\end{lemma}

\begin{proof}  We can use the same proof as in Lemma~\ref{lemma:FC-},
except with one tarski\tarf{tarski:eps:fine:Rw} calculation 
substituted for 
another\tarf{tarski:eps-bigd-}.
\end{proof}

%% This has been replaced with tarski:fine:Rw:u.
%% In fact the proofs are almost the same.
%% It can be deleted.

%\begin{lemma}\tlabel{lemma:fine:barrier}  
%Let $S=\{v_0,v,w,u\}$ be a set of four distinct vertices
%in a packing $\Lambda$.  
%Assume that $\{v_0,v\}$ is an upright diagonal of
%a quarter in the $Q$-system and that $w$ is an anchor of $\{v_0,v\}$.
%Let $R_w$ be as above; that is, a Rogers simplex attached to a crown tuple 
%extending $(v_0,v)$.
%Assume $x\in R_w$ satisfies $\epsilon_0(x,\{v,w,u\})= u$.
%Then there exists a vertex $w'$ that is an anchor of $\{v_0,v\}$ such
%$b=\{v,w',v_0\}$ is a barrier and $x$ is obstructed from $u$
%by $b$.
%\end{lemma}
%
%\begin{proof} Assume for a contradiction that $\epsilon_0(x)=u$.
%We separate the proof into two cases, depending on whether
%$x$ and $u$ lie on the same side of $A=\op{aff}(v,w,v_0)$.
%
%Assume that $x$ and $u$ lie on opposite sides of $A$.
%For any nonzero vertices $v',v''$,
%Let $L(v',v'')$ be the line of points equidistant from $\{v_0,v',v''\}$.
%The three lines $L(u,v)$, $L(v,w)$, $L(u,w)$ meet at the circumcenter
%$c$ of $S$.  The rays $L^+(u,v)$, $L^+(v,w)$, $L^+(u,w)$ demarcate
%the regions between $\epsilon_0=w,u,v$.  (Pick the direction
%of ray so that it runs through the circumcenter of the face $\{v_0,v',v''\}$
%if the remaining vertex has positive orientation, and so that it
%runs in the opposite direction otherwise.)
%
%If $u$ has positive orientation in $S$, then $L^+(v,w)$ runs through
%the circumcenter of $\{v_0,v,w\}$ and along an edge of $R_w$.
%The point $w/2$ in the closure of  $R_w$ also has $\epsilon_0=w$.
%It follows that $\epsilon_0$ has value $w$ on $R_w$, which is contrary
%to our assumption.
%
%Thus, $u$ has negative orientation in $S$.  
%This implies that $|u-v|,|u-w|,|u-v_0|\le 2.51$.  In particular,
%$S$ is a quarter.  Since it has the same diagonal as a quarter in
%the $Q$-system, we have that $S$ is in the $Q$-system, so that
%$\{v_0,v,w\}$ is a barrier.  By tarski\tarf{tarski:tip-cone}, we have
%that $x\in \op{cone}(u,\{v,w,v_0\})$.  In particular, $x$ is obstructed
%from $u$ by the barrier $\{v,w,v_0\}$.  Take $w'=w$ in this case.
%
%Now assume that $x$ and $u$ lie on the same side of $A$.
%First consider the special case where
%we also have that $\rad_V(S)\ge \eta_{V0}(v,v_0)$ and that that
%the orientation of $u$ is non-positive in $S$.  In this case, it
%follows that $S$ is a quarter in the $Q$-system.  By the rule
%for constructing crown tuples, there is no $R_w$ along $\{v_0,v,w\}$
%in this case.  Next consider the case where
%$\rad_V(S)\ge \eta_{V0}(v,v_0)$ and the orientation of $u$ is positive
%in $S$.  In this case, the ray $L^+(v,w)$ runs along the edge of 
%$R_w$ as before, and we see that $\epsilon_0=w$ on $R_w$.
%
%Finally consider the case where $\rad_V(S)<\eta_{V0}(v,v_0)$.
%It follows that $u$ is an anchor of $\{v_0,v\}$.
%By Lemma~\ref{lemma:new-anchor}, there exists a further anchor
%$w'$ between $u$ and $w$.  By tarski\tarf{tarski:prev},
%$\op{conv}\{x,u\}$ meets $\op{conv}\{v_0,v,w'\}$, and
% $|u-w'|\le 2.51$.  In particular $S'=\{v_0,v,w',u\}$ is
%an upright quarter and $\{v_0,v,w'\}$ is a barrier.
%\end{proof}



\begin{lemma}\tlabel{lemma:fine-Rw:5}\rating{300}
Let $\{v_0,v,w,u_1,u_2\}$ be a set of five distinct vertices in a
centered packing $(\Lambda,v_0)$.
packing.  Assume that $\{v_0,v\}$ is an upright diagonal of a quarter
in the $Q$-system and that $w$ is an anchor of $\{v_0,v\}$.
Assume that $S=(v_0,v,u_1,u_2)$ be normal.
Let $\rogFC(v_0,v,w,\ldots)$ be 
a Rogers simplex attached to a signed crown tuple $(v_0,v,\ldots)$.
Then $\rogFC$ does not meet $F=\op{cone}(v_0,\{u_1,u_2\})$.
\end{lemma}



\begin{proof}
For a contradiction, assume these sets meet at $x\in \rogFC(v_0,v,w,\ldots)\cap F$.
%We have $\epsilon_0=\epsilon_0(x,\{w,u_1,u_2\})\in\{w,u_1,u_2\}$.  We consider
%two cases depending on whether $\epsilon_0=w$.
Tarski\tarf{tarski:fine-Rw-split} enumerates six different possible
configurations of points $\{v_0,v,w,u_1,u_2\}$.  We discuss each
in turn.

In the first three cases, 
%
%Assume that $\epsilon_0=w$.  By tarski\tarf{tarski:fine:Rw:5},
we have $|w-u_1|\le 2.51$ and $|w-u_2|\le 2.51$.  The first is that
for either $u=u_1$ or $u=u_2$, we have that $(v_0,v,w,u)$ is normal
and that $\rogFC(v_0,v,w,\ldots)$ meets $\op{cone}(v_0,\{w,u\})$.  This is contrary to
Lemma~\ref{lemma:fine-Rw:5}.

The second possibility is that
$2.51<|u_1-u_2|<\sqrt8$ and that $v\in\op{cone}^0(v_0,\{u_1,u_2,w\})$
with $u_1,u_2,w$ all anchors of $\{v_0,v\}$.  By the normality
hypothesis, these are the only anchors of $\{v_0,v\}$.  None of the
corresponding tuples are crown tuples extending
$(v_0,v)$.  Thus, this case does not
occur. 

The third possibility is that $w\in \op{aff}_+(\{v_0,v\},\{u_1,u_2\})$
and $2.51<|u_1-u_2|$.  This is contrary to the normality
condition on $S$.

%In the remaining case, we have $\epsilon_0=u\in\{u_1,u_2\}$.  
%Let $S'=\{v_0,v,w,u\}$.  
In the remaining cases, there is a 
relabeling of vertices $\{u,u'\}=\{u_1,u_2\}$.
In the fourth case, we have that $\{v_0,v,u,w\}$ is an upright
quarter and $u'\in \op{aff}_+^0(\{v_0,u\},\{v,w\})$.  Inspecting
the various possibilities for $u'$, we find that
$\op{conv}\{u,u'\}$ meets $\{v_0,v,w\}$.  This forces $u'$ to be
an anchor of $\{v_0,v\}$ and $2.51 < |u-u'|$.  By normality
$u,u'$ are consecutive anchors.  But this is false, because the
anchor  $w$ is between them.
%By Lemma~\ref{lemma:fine:barrier}, there
%exist a barrier $b=\{v_0,v,w'\}$ such that $x$ is obstructed from
%$u$ by $b$.  However, if $x\in\op{cone}(v_0,\{u_1,u_2\})$, then
%there exists no such obstruction.  Thus, the intersection
%is empty.

In the fifth case, $\{v_0,v,u,w\}$ is an upright quarter and
$\rogFC(v_0,v,w,\ldots)\subset \op{aff}_+(\{v_0,v,w\},u)$.  
%By Lemma~\ref{lemma:new-anchor},
%there is an anchor $w'$ of $\{v_0,v\}$ between $u$ and $w$.
%This is not possible for geometric reasons.
In a quarter, its two anchors are consecutive, by Lemma~\ref{XX}.  Thus,
$(v_0,v,w,u)$ is a crown tuple.  This is absurd because it implies that
  $$
   2.77 \le |w-u| \le 2.51.
  $$

In the sixth case, $u_1$ and $u_2$ are anchors of $\{v_0,v\}$.
They are not consecutive and $2.51 < |u_1-u_2|$.  This contradicts
the definition of normal.
We have examined all six cases.  In each case, the assumption
that $F$ and $R$ meet leads to a contradiction.  The result follows.
\end{proof}

\subsection{fitted crown and fan}

The set $\FC$ is a composite, formed from $\FCinner$ and $\rogFC$.  The
preceding series of lemmas establishes that each of the pieces of the composite
avoides the blades formed from barriers.  We are now ready to show that $\FC$
itself lies in a connected component of $Y(v_0,V,E)$, where $(v_0,V,E)$ is a
fan formed from barriers.  Since the set $\FC$ itself is not connected, we
work with a slighly larger connected set (called $B$ in the proof) that is connected
and lies in $Y(v_0,V,E)$.  With this small adjustment, the proof is
straightforward.


\begin{lemma}\tlabel{lemma:delta-tri}\rating{200}
%\tlabel{lemma:delta-upright}
%\tlabel{lemma:delta-flat}
Let $(\Lambda,v_0)$ be a centered packing.
Let $\{v_0,v\}$ be
an upright diagonal of a quarter in the $Q$-system.
Assume that $(v_0,v,u_1,u_2)$ is normal.
Let $E$ be the set of all $\{u,v\}$ such that $\{v_0,u,v\}$
is a barrier.  Set
   $$
   V = \{v \mid \exists u.\quad \{u,v\}\in E\}.
   $$
Assume $(v_0,V,E)$ a fan.  Then, for each crown tuple
$(v_0,v,w_1,w_2)$, there is a component $U\in Y(v_0,V,E)$ such
that $\FC(v_0,v,w_1,w_2)\subset U$.  In particular, $\FC$ does
not meet any blades of the fan.
%Let $F=\{v_0,u_1,u_2\}$ be a quasi-regular triangle.   
%Assume that there
%exists a quarter in the $Q$-system along $\{v_0,v\}$.  
%Let $(v_0,v,w_1,w_2)$ be a crown tuple.
%Then
%$\FC(v_0,v,w_1,w_2)$ does not meet $A=\op{aff}_+(v_0,\{u_1,u_2\})$.
\end{lemma}

By Lemma~\ref{XX},  $(v_0,V,E)$ is a fan.  
Rather than digress to give
a proof of this fact here, we simply include it as an additional hypothesis that
can later be dropped.

\begin{proof}
The set $\FC$ is contained in the union of three open sets:
$\rogFC(v_0,v,w_1,w_2,1)$, $\rogFC(v_0,v,w_2,w_1,-1)$, and
  $$
  C=\op{rcone}^0(v_0,v,b)\cap \op{aff}_+^0(\{v_0,v\},\{w_1,w_2\}).
  $$
Each of these three sets is connected.  Moroever, $C$ meets
the two sets  $\rogFC(\cdots,\pm1)$.  Therefore the union $B$ of
these three sets is connected and open.  

We show that $B$ is contained in $Y(v_0,V,E)$.  That is,
it is disjoint from $X(v_0,V,E)$.
The set $X(v_0,V,E)$ is a union of blades $F=\op{aff}_+(v_0,\{u_1,u_2\})$
from barriers $\{v_0,u_1,u_2\}$.  By Lemma~\ref{lemma:meet-normal},
the tuple $(v_0,v,u_1,u_2)$ is normal.
By Lemma~\ref{lemma:FC-}, the set $C$ does not meet any sets $F$
of this form.  By Lemma~\ref{lemma:fine-Rw} and Lemma~\ref{lemma:fine-Rw:5}, the sets $\rogFC$ do not meet any sets $F$ of this form.

Thus, $B$ is contained in a single connected component $U$ of $Y(v_0,V,E)$.  As $B$ contains $\FC$, the result follows.
\end{proof}


\subsection{containment in V-cell}

%
%\begin{lemma}\tlabel{lemma:delta-flat}
%Let $(\Lambda,v_0)$ be a centered packing.
%Let $F=\{v_0,u_1,u_2\}$ be a triangle.  
%Assume that $|u_1-v_0|\le 2t_0$,
%$|u_2-v_0|\le 2t_0$, and $2t_0\le|u_1-u_2|\le\sqrt8$.  Let $\{v_0,v\}$
%be the diagonal of an upright quarter in the $Q$-system.  Assume
%that if $u_1$ and $u_2$ are both anchors of $v$, then they are
%consecutive anchors around $v$. Under these conditions, the set
%$\FC(v,W)$ does not overlap the cone at $v_0$ over the triangle
%$F$.
%\end{lemma}
%
%\begin{proof} The proof is identical to that of
%Lemma~\ref{lemma:delta-tri}. 
%\end{proof}
%
%\begin{lemma}\tlabel{lemma:delta-upright}
%Let $F=\{v_0,u_1,u_2\}$ be a triangle.  Assume that $2t_0\le|u_1-v_0|\le
%\sqrt8$, $2\le|u_2-v_0|\le 2t_0$, and $2\le|u_1-u_2|\le2t_0$.  Let
%$\{v_0,v\}$ be the diagonal of an upright quarter in the $Q$-system.
%Under these conditions, the set $\FC(v,W)$ does not overlap
%the cone at $v_0$ over the triangle $F$.
%\end{lemma}
%
%\begin{proof}
%The proof is identical to that of Lemma~\ref{lemma:delta-tri}.
%\end{proof}
%




\begin{lemma}\tlabel{lemma:FC-unobstr}\dcg{Lemma~9.12}{92}\rating{400}
Let $(\Lambda,v_0)$ be a centered packing.  Let $(v_0,v,w_1,w_2)$
be a crown tuple.
Assume $\{v_0,v\}$ is an upright diagonal of a quarter in the
$Q$-system.   
Then there exists a null set $Z$ such that if $x$ lies in %  the interior of  WW: FC is open
$\FC(v_0,v,w_1,w_2)\setminus Z$,
then $x$ is unobstructed at $v_0$.
\end{lemma}

\begin{proof}\FIXX{The argument is correct, but still uses geometry.} 
For a contradiction, assume that $x$ is obstructed
at $v_0$ by barrier $T =\{u_1,u_2,u_3\}$.
If $v_0\in T$, we have disjointness by Lemma~\ref{lemma:delta-tri}. Assume $v_0\not\in T$.
Let $y\in \op{conv}^0\{v_0,x\}\cap \op{conv}(T)\cap \FC$.  

%The convex hull of $T$ can be partitioned into three sets $T(i)$
%depending on which vertex of $T$ is closest to a given point in
%the convex hull. (Ties can be resolved in any consistent manner.)
%Let $y\in \FC$ be the point in the convex hull of $T$ on
%the segment from $v_0$ to $x$.  Fix $i$ so that $y\in T(i)$. If
%$v=u_i$, then each point $y$ of $T(i)$ is closer to $v$ than to
%$v_0$.  But each point of $\FC$ is closer to $v_0$ than to
%$v$.  So $x$ is not obstructed by $T$ at $v_0$.

%We may now assume that $v\ne u_i$.


Consider the special case that $S=\{v_0,u_1,u_2,u_3\}\in\CalQ(\Lambda,v_0)$.  
Let 
$U=\op{aff}_+^0(v_0,T)$.  
Let $(v_0,V,E)$ be the fan of Lemma~\ref{lemma:delta-tri}.
We claim $U$ is a component of
$Y(v_0,V,E)$.  It is open, convex and hence connected.  Its
complement in $Y$ is open.  Hence $U$ is a component.
Also, $y\in U\cap \FC$.  By Lemma~\ref{lemma:delta-tri}, $\FC\subset U$.  Then
$v\in \op{aff}_+(v_0,T)$.  From this, we get $v\in T$.  Say $v=u_1$.
It follows that $S$ is an upright quarter in the $Q$-system.
Then $\FC\subset \op{aff}_+^0(\{v_0,v\},\{w_1,w_2\})$, which is disjoint from
$\op{aff}_+(\{v_0,v\},\{u_1,u_2\})$.  So $y$ cannot be in both sets.  This
completes the proof in this special case.

By picking $Z$ suitably, we may assume that $y$ is not equidistant from
two points in $T$.  
Pick
a vertex of $T$ (say $u_1$)  closer to $y$ as any other
$u\in T$.  We claim that $v\ne u_1$.  Otherwise, 
since $|v_0-v|>2t_0$, the vertex $v_0$ has positive
orientation in $T\cup\{v_0\}$.  Then
  $$
  |y - u_1| < |y - v_0| \le |y - v|.
  $$
So $v\ne u_1$.


Let $T'= \{v_0,v,u_1\}$.
Partition the complement of a null set $Z$ in $\ring{R}^3$ 
into three sets $\Omega(T',u)$, for $u\in T'$.  We may assume $y\not\in Z$.
%into three sets $V(u_i)$,
%$V(v_0)$, $V(v)$ according to which of $\{u_i,v_0,v\}$ a point
%$z\in\ring{R}^3$ is closest to.  (Again resolve ties in any
%consistent manner.)
We have  that $\max_j |u_j-v_0| \ge 2t_0$, for otherwise
$S\in\CalQ(\Lambda)$, which has already been treated. 
Each point of $\op{conv}(T)$ is closer to some $u_i\in T$ than to $v_0$.
This implies that $y\in \Omega(T',v) \cup \Omega(T',u_1)$.  

We claim on the contrary, that $y\in \FC\subset \Omega(T',v_0)$, which is disjoint
from $\Omega(T',v)\cup\Omega(T',u_1)$.
The relation $\FCinner\subset\Omega(T',v_0)$ comes directly from the
definition of $\FCinner$.  It was designed with this very property in mind.
Consider the case $y\in\rogFC(v_0,v,w_1,w_2,\epsilon)$.  By construction
$y\in \Omega(T',v_0)\cup\Omega(T',u_1)$.  Note also that $|y-v_0|<|y-w_1|$,
so that we may assume that $u_1\ne w_1$.  Assume $|y-u_1|<|y-v_0|$.
We have two cases depending on whether $\op{aff}\{v_0,v,w_1\}$ separates
$u_1$ and $y$.  

If the plane separates $u_1$ from $y$, then
$\{v_0,v,u_1,w_1\}$ must be an upright quarter, the orientation of this
upright quarter at $u_1$ must be negative, and $u_1\ne w_2$.
%and the plane $\op{aff}\{v_0,v,w_1\}$
%must separate $u_1$ from $y$.  
By tarski\tarf{tarski:tip-cone},
we have that $y\in\op{aff}_+(u_1,\{v_0,v,w_1\})$.  This implies that
the segment $\op{conv}\{y,u_1\}$ meets the barrier $\op{conv}\{v_0,v,w_1\}$
at some $z$.  By enlarging $Z$ if necessary, we may assume that 
$z\in\op{conv}^0\{v_0,v,w_1\}$.
However, $z$ lies in the barrier $\op{conv}\{u_1,u_2,u_3\}$, because
$y$ and $u_1$ do,
which is contrary to the non meeting of barriers (Lemma~\ref{lemma:barrier-no-overlap}).

If the plane does not separate $u_1$ from $y$, then $u_1\ne w_2$ and 
$S = \{v_0,v,w_1,u_1\}$ satisfies the hypotheses of Lemma~\ref{lemma:new-anchor} and $w'=w_2$ is the anchor produced in the conclusion of that lemma.
That lemma also gives that $\{v_0,v,w_2,u_1\}$ is an upright quarter in
the $Q$-system, so that $\{v_0,v,w_2\}$ is a barrier.
We have $|y-u_1|<|y-v_0|$ and $|y-v_0|<|y-v|,|y-w_2|$.
By tarski\tarf{tarski:tip-cone}, the segment $\op{conv}\{u_1,y\}$
meets the barrier $\op{conv}\{v_0,v,w_2\}$.  Arguing as in the previous
case, we find that two barriers meet, contrary to Lemma~\ref{lemma:barrier-no-overlap}.
%
%On the other hand, we have by
%construction that $y\in \FC \subset \Omega(T',v_0)$  
%(There are
%two cases involved in this conclusion, depending on whether $u_i$
%is an anchor of $\{v_0,v\}$.)  However, the sets $\Omega(T',\cdot)$ are
%disjoint; and we reach a contradiction.  Thus, 
%$x$ is unobstructed at $v_0$.
%
%Next assume that $\max_j |u_j| < 2t_0$.  Let $S=\{v_0,u_1,u_2,u_3\}$.
%Since $T$ is a barrier, $S\in\CalQ(v_0)$.  By assumption, $\{v_0,v\}$
%is a diagonal of an upright quarter in $\CalQ(v_0)$.  By the fact
%that the interiors of quarters in $\CalQ(v_0)$ do not meet, we see
%that $v$ is not enclosed over $S$.  The set $\FC$ has
%a star convexity with respect to the ray from $v_0$ through $v$.
%Thus, if $\FC$ intersects the convex hull of
%$T$ at $y$, then $\FC$ intersects the cone over a face
%$\{v_0,u_1,u_2\}$ of $S$ at $y'$. (For simplicity, take $v_0=0$.)
%We can take $y'/|y'|$ to lie on
%the cone generated by the arc running from $v/|v|$ to $y/|y|$.
%This is impossible by Lemma~\ref{lemma:delta-tri}. 
% and \ref{lemma:delta-flat}.
\end{proof}

\begin{lemma}\tlabel{lemma:FC-VC}\rating{200}  
Let $(\Lambda,v_0)$ be a centered packing.
Let $(v_0,v,w_1,w_2)$ be a crown tuple of $\Lambda$. Assume that
 $\{v_0,v\}$ is the upright diagonal of a quarter
in the $Q$-system.  Then there is a null set $Z$ such
that $\FC(v_0,v,w_1,w_2)$ is
a subset of $\op{VC}(\Lambda,v_0)\cup Z$.
\end{lemma}

\begin{proof}
We recall that there is a truncation at distance $2$ in the definition
of $V$-cell so that $\op{VC}(\Lambda,v_0)\subset B(v_0,2)$.
The extreme point of $\FC(v_0,v,w_1,w_2)$ has distance from the origin of
 $$\eta_{V0}(v_0,v) = \eta(|v_0-v|,2,2.51)\le \eta(\sqrt8,2,2.51) < 2.$$
Thus, the truncation condition is satisfied.

We show that $\FCinner(v_0,v,w_1,w_2)\subset\op{VC}(\Lambda,v_0)$.
Suppose to the contrary, that a point 
  $$x\in \FCinner\setminus \op{VC}(\Lambda,v_0)$$
%$ in %the interior of
%$\FCinner$ lies in $\op{VC}(\Lambda,w)$, with $w\ne0$.  
By Lemma~\ref{lemma:FC-unobstr},  $x$ is at least as close
to some $w\in\Lambda$ as to  $v_0$.  
Then, $\eta_V(v_0,v,w)\le\eta_{V0}(v,v_0)$, and $w$
is an anchor of $\{v_0,v\}$.  The construction of
$\FCinner$ prevents this from happening.

Consider a point $x$ of $\rogFC(v_0,v,w_1,w_w,\epsilon)$.
By avoiding a null set $Z$ we may assume that $x$ lies in
$\op{VC}(\Lambda,u)$, with $u\ne v_0$.  
By dismissing a trivial case, we may
assume that $w_1\ne u$.

Assume that the orientation of $S=\{v_0,v,w_1,u\}$ is negative at $u$.  
Then $S$ must be an upright quarter.  By
the construction of crown tuples, we have that $\rogFC(v_0,v,w_1,\cdot)$ must
lie on the opposite side of the plane $\{v_0,v,w_1\}$ from $u$ (for
there is no crown tuple between the anchors of an upright quarter).  The
result now follows from tarski\tarf{tarski:back}.

If $\rad_V(S) <\eta_{V0}(v,v_0)$, then $u$ and $w_1$ are anchors.  In
this case, the result follows from tarski\tarf{tarski:prev}.\FIXX{Check reference}

Finally if the orientation is positive and if $\rad_V(S)\ge
\eta_{V0}(v,v_0)$, then a point of $\rogFC(v_0,v,w_1,w_2,\epsilon)$ cannot be closer to $u$ than
to $v_0$.
\end{proof}


\subsection{overlap}%DCG 9.3, p93
    \label{sec:overlap}
    \oldlabel{2.4}


\begin{lemma}\tlabel{lemma:FC-no-over}\rating{400}
Let $(\Lambda,v_0)$ be a centered packing.  Let $(v_0,u,\ldots)$
and $(v_0,v,\ldots)$ be distinct (signed) crown tuples.
Then $\FC(v_0,u,\ldots)$ does not meet $\FC(v_0,v,\ldots)$.
\end{lemma}

\begin{proof}
This is clear for  $u=v$.  Assume $u\ne v$.  

To prove $\FCinner(v_0,u,\ldots)$ and $\FCinner(v_0,v,\ldots)$
disjoint, we
may contract $\{u,v\}$ until $|u-v|=2$. The circumcenter $c$ of 
$\{v_0,u,v\}$ lies
in its convex hull.  We have $\eta_V(v_0,u,v)\ge
\eta_{V0}(v,v_0)$ and $\eta_V(v_0,u,v)\ge\eta_{V0}(u,v_0)$.  So the plane
through $\{v_0,c\}$ perpendicular to the plane $\{v_0,u,v\}$ separates
$\FCinner(v_0,u,\ldots)$ from $\FCinner(v_0,v,\ldots)$.

Next we separate points in $\FCinner(v_0,u,\ldots)$ from points of
$\rogFC(v_0,v,w,\cdot)$, where $w$ is an anchor of $v$ and $u\ne v$.  Let
$S=\{v_0,u,v,w\}$. The orientation of $S$ at $u$  is
positive.  The circumradius of $S$ satisfies
    $$
    \rad_V(S) \ge \eta_V(v_0,u,v)>\eta_{V0}(v,v_0).
    $$
By tarski\tarf{tarski:eps-inner}, 
%Thus, $\epsilon_0(S,\cdot)$ takes different values on
%$\FCinner(v_0,u,\ldots)$ and $\rogFC(v_0,v,w,\cdot)$, so that 
the sets are disjoint.

Next we separate points of $\rogFC(v_0,v,w,\cdot)$ from 
$\rogFC(v_0,u,w,\cdot)$.  (Notice
that we assume that the anchor is the same for the two upright diagonals.)
Let $S=\{v_0,u,v,w\}$.   As above, we have
    $$
    \rad_V(S) \ge \eta_{V0}(v,v_0), \quad \eta_{V0}(u,v_0).
    $$
The simplex $S$ has positive orientation at $u$ and $v$.  
Let $c_u$ be the circumcenter of
$\{v_0,u,w\}$, let $c_v$ be the circumcenter of $\{v_0,v,w\}$, and let
$c$ be the circumcenter of $S$.  Then $\rogFC(v_0,v,w,\cdot)$ lies in the
convex hull of $\{v_0,w,c_v,c\}$, but $\rogFC(v_0,u,w,\cdot)$ lies in the convex
hull of $\{v_0,w,c_u,c\}$.  Thus, the sets are disjoint.



Finally, we separate points of $\rogFC(v_0,u,w,\cdot)$ from points of
$\rogFC(v_0,v,w',\cdot)$,  where $w\ne w'$. 
By tarski\tarf{tarski:eps-outer}, if these sets meet the points fall into
one of the following three configurations, after relabelling $(v,w)\leftrightarrow(u,w')$
if necessary.  In all three configurations, $w'$ is an anchor of $v$.

In the first configuration, $Q=\{v_0,v,w,w'\}$ is an upright quarter and
$\rogFC(v_0,u,w',\ldots)$ meets $F = \op{aff}_+(v_0,\{v,w\})$.  If $Q$ is an upright
quarter, then $\{v_0,v,w\}$ is a barrier.  This is contrary to Lemma~\ref{lemma:fine-Rw:5}.

In the second configuration, $Q=\{v_0,v,w,w'\}$ is an upright quarter, and the signed crown
tuple is $(v_0,v,w,w',\epsilon)$, for some sign $\epsilon$.  However, for a crown
tuple $2.77\le |w-w'|$, but for an upright quarter $|w-w'|\le 2.51$, so these conditions
are inconsistent.

In the third configuration, $|w-w'| > 2.51$, $\op{rad}_V\{v_0,v,w,w'\} < \eta_{V0}(v_0,v)$,
and the hypotheses of Lemma~\ref{lemma:new-anchor} are met.  That is, there is an anchor
$w''$ of $\{v_0,v\}$ in $\op{aff}_+(\{v_0,v\},\{w,w'\})$ with $|w''-w|\ge 2.77$
and $\rad_V\{v_0,w,w',v\})$.  Moreover, in this third configuration, the hypotheses of
tarski\tarf{tarski:prev} are met.  This gives that $Q=\{v_0,v,w',w''\}$ is an upright
quarter (in the $Q$-system).  This implies that $\{v_0,v,w''\}$ is a barrier.  The lemma
also gives a point $x \in \rogFC(v_0,w',\ldots)$ such that $\op{conv}(x,w')$ meets
$F=\op{aff}_+(v_0,\{v,w''\})$.  This in turn implies that $\rogFC(v_0,w\,\ldots)$ meets $F$,
which is contrary to Lemma~\ref{lemma:fine-Rw:5}.
%
% If the function
%$\epsilon_0(\{v_0,w,w'\},\cdot)$ separates the sets, we are done.
%Otherwise, we may assume say that $\epsilon_0(\{v_0,w,w'\},x) = w'$
%from some $x\in \rogFC(v_0,u,w,\cdot)$.  Let $S=\{v_0,u,w,w'\}$.
%
%If $w'$ is not an anchor of $u$, then $\rad_V(S) \ge\eta_{V0}(u,v_0)$
%and the orientation of $S$ along $\{v_0,w,u\}$ is positive.  In this
%case, we have $\epsilon_0 = w$ on $\rogFC(v_0,u,w,\cdot)$, which is contrary
%to assumption. Thus, we may assume that $w'$ is an anchor of $u$.
%
%If the orientation of $\{v_0,u,w,w'\}$ is negative along $\{v_0,w,u\}$,
%then $\{v_0,u,w,w'\}$ is a quarter, 
%contrary to the existence of a crown tuple.  
%So the orientation is positive.  If $\rad_V\{v_0,u,w,w'\} <
%\eta_{V0}(u,v_0)$, then tarski\tarf{tarski:prev} implies that each point
%of $\rogFC(v_0,u,w,\cdot)$ is obstructed from $w'$.  
%But no point of $\rogFC(v_0,v,w',\cdot)$ is
%obstructed from $w$. (In fact, a barrier that crosses
%$\FC(v_0,v,\ldots)$ is inconsistent with Lemma~\ref{lemma:delta-tri}.)
%%,\ref{lemma:delta-flat}, \ref{lemma:delta-upright}.) 
%So
%$\rad_V\{v_0,u,w,w'\} \ge \eta_{V0}(u,v_0)$.  This is contrary to
%$\epsilon_0(\{v_0,w,w'\},x) = w'$ from some $x\in \rogFC(v_0,u,w,\cdot)$.
\end{proof}



\section{Alphabet Simplex}%DCG 9.4, p94
\label{sec:alphabet}
    \oldlabel{2.5}

We consider various types of simplices, formed by vertices in $\Lambda$.  
Because of their names $\SA$, $\SB$, $\SC$, $\SD$, $\SE$ we call them
{\it alphabet simplices}.
  The edge lengths of
these simplices are less than $2\sqrt{2}$.

$\SA$.  This family consists of simplices $S(y_1,\ldots,y_6)$ whose
edge lengths satisfy
    $$
    y_1,y_2,y_3\in[2,2t_0],\quad
    y_4,y_5\in[2t_0,2.77],
    \quad
    y_6\in[2,2t_0],\quad \text{and }
    \eta(y_4,y_5,y_6)<\sqrt{2}.
    $$
(These conditions imply $y_4,y_5<2.697$, because
$\eta(2.697,2t_0,2)>\sqrt2$.)

$\SB$.  This family consists of certain flat quarters that are
part of an isolated pair of flat quarters. It consists of those
satisfying $y_2,y_3\le 2.23$, $y_4\in[2t_0,2\sqrt{2}]$.

$\SC$.  This family consists of certain simplices
$S(y_1,\ldots,y_6)$ with edge lengths satisfying
    $y_1,y_4\in[2t_0,2\sqrt{2}]$, $y_2,y_3,y_5,y_6\in[2,2t_0]$.
We impose the condition that the first edge is the diagonal of
some upright quarter in the $Q$-system, and that the upper
endpoints of the second and third edges (that is, the second and
third vertices of the simplex) are consecutive anchors of this
diagonal. We also assume that $y_4< 2.77$, or that both face
circumradii of $S$ along the fourth edge are less than $\sqrt{2}$.

$\SD$.  This is
a small variation on simplices of type $\SC$.  
We define a simplex $\{v_0,v,v_1,v_2\}$ of type $\SD$
to be one satisfying the following conditions.
    \begin{enumerate}
    \item The edge $\{v_0,v\}$ is an upright diagonal of an upright quarter
        in the $Q$-system.
    \item $|v_2-v_0|\in[2.45,2t_0]$.
    \item $v_1$ and $v_2$ are anchors of $v$.
    \item $|v-v_2|\in [2.45,2t_0]$.
    \item The edge $\{v_1,v_2\}$
    is a diagonal of a flat quarter with face $\{v_0,v_1,v_2\}$.
    \item The simplex is not type $\SC$.
    \end{enumerate}


%$\SEE$.  This family consists of simplices $\{v_0,v_1,v_2,v_3\}$ such
%that 
%the edge lengths $y_{ij} = |v_i-v_j|$ satisfy
%   $$
%   y_{01},y_{02}\in[2t_0,2\sqrt{2}],\quad
%   y_{ij}\in [2,2t_0], 
%   $$
%for all other $ij$, and such that
%there is another simplex $\{w_0,w_1,w_2,w_3\}$ satisfying the same
%constraints with $y_{ij} = |w_i-w_j|$ and  such that $(w_0,w_1,w_2)=(v_0,v_1,v_2)$
%and $|w_3-v_3|>\sqrt8$.

$\SE$.  These are flat quarters that are not strict: the diagonal is precisely
$2\sqrt2$.

\begin{lemma}\tlabel{lemma:2.77}\rating{200}
If a vertex $w$ is enclosed over a simplex $S$ of type $A$, $\SB$,
or $\SC$, then its height is greater than $2.77$.  Also, $\{v_0,w\}$
is not the diagonal of an upright quarter in the $Q$-system.
More generally, the same conclusions hold if $S$ is any simplex
$S\{v_0,\ldots,v_3\}$, $y_{ij}=|v_i-v_j|$, with $y_{01},y_{23}\in(2t_0,2\sqrt{2}]$,
$y_{ij}\in[2,2t_0]$ for $\{i,j\}\ne\{0,1\},\{2,3\}$.
\end{lemma}

%% WW Is the "more generally" clause needed?

\begin{proof}
In case $A$, $\eta(y_4,y_5,y_6)<\sqrt{2}$, so an enclosed vertex
must have height greater than $2\sqrt{2}$.  It is too long to be
the diagonal of a quarter.

In case $\SB$, we use the fact that the isolated quarter does not
meet in the interior with any quarter in the $Q$-system. 
By tarski\tarf{tarski:enclosed-v}, an
enclosed vertex has length at least $2.77$.
By the symmetry of isolated quarters, this means that the diagonal
of a flat quarter must also be at least $2.77$.

In case $\SC$, the same calculation gives that the enclosed vertex
$w$ has height at least $2.77$.  Let the simplex $S$ be given by
$\{v_0,v,v_1,v_2\}$, where $\{v_0,v\}$ is the upright diagonal. By
tarski\tarf{tarski:pass-anchor}, $v_1$ and $v_2$ are anchors of
$\{v_0,w\}$. If $u$ is an anchor of $\{v_0,w\}$, then $\op{aff}_+^0(v_0,\{w,u\})$
does not meet $\op{aff}_+^0(v_0,\{v,v_i\})$ by
tarski\tarf{tarski:2t0-doesnt-pass-through}. 
The distance between $w$ and $v$ is at most
$2t_0$ by tarski\tarf{tarski:double-face}. If $\{v_0,w\}$ is the
diagonal of an upright quarter, the quarter takes the form
$\{v_0,w,v_1,v_3\}$, or $\{v_0,w,v_2,v_3\}$ for some $v_3$, by
tarski\tarf{tarski:double-face}. If both of these are quarters, then
the diagonal $\{v_1,v_2\}$ has four anchors $v$, $w$, $v_0$, and
$v_3$. The selection rules for the $Q$-system place the quarters
around this diagonal in the $Q$-system. So if bother are quarters, neither $\{v_0,w,v_1,v_3\}$
nor $\{v_0,w,v_2,v_3\}$ is in the $Q$-system. Suppose that
$\{v_0,w,v_1,v_3\}$ is a quarter, but that $\{v_0,w,v_2,v_3\}$ is not.
Then $\{v_0,w,v_1,v_3\}$ forms an isolated pair with $\{v_1,v_2,v,w\}$.
In either case, the quarters along $\{v_0,w\}$ are not in the
$Q$-system.
\end{proof}

%\begin{remark}  The proof of this lemma does not make use of all the hypotheses
%on $\SC$.  The conclusion holds for any simplex%
%
%\end{remark}

\subsection{disjointness}%DCG 9.5, p95
    \oldlabel{2.6}

Let $S=\{v_0,v_1,v_2,v_3\}$ be a simplex of type $A$, $\SB$, or
$\SC$. An edge $\{v_4,v_5\}$ of length at most $2\sqrt{2}$ such
that $|v_4-v_0|,|v_5-v_0|\le 2t_0$ cannot cross two of the edges
$\{v_i,v_j\}$ of $S$.  In fact, it cannot cross any edge $\{v_i,v_j\}$
with $|v_i-v_0|,|v_j-v_0|\le 2t_0$ by tarski\tarf{tarski:skew-quad}.  The
only possibility is that the edge $\{v_4,v_5\}$ crosses the two
edges with endpoint $v_1$, with $|v_1-v_0|\ge2t_0$ in case $\SC$.  But
this too is impossible by tarski\tarf{tarski:double-face}.

Similar arguments show that the same conclusion holds for an edge
$\{v_4,v_5\}$ of length at most $2t_0$ such that $|v_4-v_0|\le2t_0$,
$v_5\le2\sqrt{2}$.  The only additional fact that is needed is
that $\{v_4,v_5\}$ cannot cross the edge between the vertex $v$ of
an upright diagonal $\{v_0,v\}$ and an anchor
(tarski\tarf{tarski:2t0-doesnt-pass-through}).





\begin{lemma}\tlabel{lemma:no-overlap}\rating{200}
    Consider two simplices $S$, $S'$, each of  type $A$, $\SB$, $\SC$,
or a quarter in the $Q$-system.\FIXX{I'm not sure if this hypothesis is needed.  If so, the lemma
    has to be moved after standard components are introduced:
    Assume that $S$ and $S'$ do not lie
    in the cone over a quadrilateral component.}
    Then 
    $\op{conv}^0(S)$ does not meet $\op{conv}(S')$.
\end{lemma}

\begin{proof}
\FIXX{I'm not sure these next two lines are needed...; see above.
By hypothesis, the standard component is not a quadrilateral, and we
thus exclude the case of conflicting diagonals in a quad cluster.}
We claim that no vertex $w$ of $S$ is enclosed over $S'$.
Otherwise, $w$ must have height at least $2t_0$, so that $\{v_0,w\}$
is the diagonal of an upright in the $Q$-system, and this is
contrary to Lemma~\ref{lemma:2.77}. Similarly, no vertex of $S'$
is enclosed over $S$.

Let $\{v_1,v_2\}$ be an edge of $S$ crossing an edge $\{v_3,v_4\}$ of
$S'$. By the preceding remarks, neither of these edges can cross
two edges of the other simplex. The endpoints of the edges are not
enclosed over the other simplex. This means that one endpoint of
each edge $\{v_1,v_2\}$ and $\{v_3,v_4\}$ is a vertex of the other
simplex.  This forces $S$ and $S'$ to have three vertices in
common, say $v_0$, $v_2$, and $v_3$.  We have $S=\{v_0,v_1,v_3,v_2\}$
and $S'=\{v_0,v_3,v_2,v_4\}$. If
    $|v_2-v_0|\in[2t_0,2\sqrt{2}]$,
then we see that the anchors $v_3$, $v_4$ of $\{v_0,v_2\}$ are not
consecutive.  This is impossible for simplices of type $\SC$ and
upright quarters.  Thus, $v_2$ and $v_3$ have height at most
$2t_0$.  We conclude, without loss of generality, that
    $|v_4-v_0|\in[2t_0,2\sqrt{2}]$
and $|v_1-v_2|\ge 2t_0$.

The heights of the vertices of $S$ are at most $2t_0$, so it has
type $A$ or $\SB$, or it is a flat quarter in the $Q$-system. If
$S'$ is an upright quarter in the $Q$-system, then it does not
overlap an isolated quarter or a flat quarter in the $Q$-system,
so $S$ has type $\SA$. By tarski\tarf{tarski:277}, we have
$|v_1-v_2|>2.77$.  This imposes the contradictory constraints
on $\SA$
    $$
    2.77\ge |v_1-v_2|>2.77.
    $$
Thus $S'$ has type $\SC$.  This forces $S$ to have type $\SA$.  We
reach the same contradiction  $2.77 > 2.77$.
\end{proof}

\subsection{type A}%DCG 9.6, p96
    \label{sec:separation}
    \oldlabel{2.7}

Let $S = \{v_0,v_1,v_2,v_3\}$.
Let $\op{cone}^0(S) = \op{cone}^0(v_0,\{v_1,v_2,v_3\}$.
Let $V_S = \op{VC}(\Lambda,v_0)\cap \op{cone}^0(S)$, for a simplex $S$ of type $\SA$,
$\SB$, or $\SC$. 
We truncate $V_S$ to $V_S(t_S)$ by intersecting
$V_S$ with a ball of radius $t_S$.  The parameters $t_S$ depend on
the type of $S$.

If $S$ has type $\SA$, we use $t_S=+\infty$ (no truncation).

\begin{lemma}\tlabel{lemma:typeA-Omega}\rating{100}
Let $S=\{v_0,v_1,v_2,v_3\}$ be a simplex of type $\SA$.
There is a null set $Z$, such that
we have  $ \Omega(v_0,S) \cap \op{cone}^0(S) \subset V_S \cup Z$.
\end{lemma}

\begin{proof} 
We use the fact that if $b$ is a barrier, then $\op{conv}$ does
not meet $\op{conv}^0(S)$ by Lemma~\ref{XX}.  


Excluding a null set, we may assume 
for a contradiction that
$x\in \Omega(v_0,S) \cap \op{cone}^0(S) \cap \op{VC}(\Lambda,v)$,
for some $v\ne v_0$.  

% ...
By tarski\tarf{tarski:vor-bar-sqrt2}, $x$ and $v_0$ lie on the
same side of $\op{aff}\{v_1,v_2,v_3\}$.  Thus, $x$ is in
$\op{conv}^0(S)$.  
Thus, every vertex of $S$ is unobstructed at $x$.  Thus, $x$
is closer to $v$ than to any vertex of $S$.

By tarski\tarf{tarski:vor-bar-sqrt2}, $\op{conv}\{v_1,v_2,v_3\}$ 
separates
$\Omega(v_0,S)\cap \op{cone}^0(S)$ from $\Omega(v,\{v,v_1,v_2,v_3\})$ when
$v$ is enclosed over $S=\{v_0,v_1,v_2,v_3\}$.  This is contrary
to the assumption that $x$ lies in the intersection of these
two sets.

If $\Omega(v,\{v_0,v_1,v_2\})$ meets $\op{conv}^0(S)$, then
$S'=\{v,v_0,v_1,v_2\}$ must be a quarter or quasi-regular tetrahedron.
If $x$ is a barrier, then $x\not\in\op{VC}(\Lambda,v)$.  This implies
that $S'$ is a quarter that is not in the $Q$-system.
It
cannot be an isolated quarter because of the edge length
constraint $2.77$ on simplices of type $\SA$.
There must be a
conflicting diagonal $\{v_0,w\}$, where $w$ is enclosed over $Q$. ($w$
cannot be enclosed over $S$ by results of
Lemma~\ref{lemma:no-overlap}.) This shields the $V$-cell at $v$
from $\op{cone}^0(S)$ by the two barriers $\{v_0,w,v_1\}$ and $\{v_0,w,v_2\}$ of
quarters in the $Q$-system.
\end{proof}

\begin{lemma}\tlabel{lemma:typeA-crown}\rating{50}
Let $\Lambda$ be a packing with crown tuple $(v_0,v,w_1,w_2)$.
Let $S=\{v_0,v_1,v_2,v_3\}$ be a simplex of type $A$.
  $V_S$ is disjoint from all of the set $\FC(v_0,v,w_1,w_2)$.
\end{lemma}

\begin{proof}
This is evident from
Lemma~\ref{lemma:delta-tri}. % and \ref{lemma:delta-flat}.
\end{proof}


Our justification that $V_S(t_S)$ can be treated as an
independently scored entity is now complete.

\subsection{type B}%DCG 9.7, p96
    \oldlabel{2.8}

If $S(y_1,\ldots,y_6)$ has type $\SB$, we label vertices so that
the diagonal is the fourth edge, with length $y_4$. We set
$t_S=1.385$. The calculation in Lemma~\ref{lemma:2.77}
shows that any enclosed vertex over $S$ has height at least
$2.77=2t_S$.

\begin{lemma}\tlabel{lemma:typeB-Omega}\rating{100} 
Let $S=\{v_0,v_1,v_2,v_3\}$ be a simplex of type $\SB$ in a centered packing $(\Lambda,v_0)$.
There is a null set $Z$, such that
we have  $ \Omega(v_0,S) \cap \op{cone}^0(S) \cap B(v_0,1.385) 
\subset V_S \cup Z$.
\end{lemma}

\begin{proof}  As above, assume for a contradiction that there
is a point in 
 $$\Omega(v_0,S)\cap \op{cone}^0(S) \cap B(v_0,1.385)\cap \op{VC}(\Lambda,v'),$$
with $v'\ne v_0$.
Vertices outside $\op{cone}^0(S)$ cannot reach inside $S$ this way.  In
fact, such a vertex $v'$ would have to form a quarter or
quasi-regular tetrahedron with a face of $S$.  The $V$-cell at
$v'$ cannot meet $\op{cone}^0(S)$ unless it is a quarter that is not in the
$Q$-system. But by definition, an isolated quarter is not adjacent
(along a face along the diagonal) to any other quarters.
\end{proof}

%To separate the scoring of $V_S(t_S)$ from the rest of the
%standard cluster, we also show that the terms of
%Formula~\ref{eqn:3.5}  for $V_S(t_S)$ are represented
%geometrically by solids that lie in the cone $\op{cone}^0(S)$.   This
%is the purpose of the following lemma.

\begin{lemma}\tlabel{lemma:typeB-rcone}\rating{20}
 Let $S=\{v_0,v_1,v_2,v_3\}$ be a simplex of type $\SB$ in a centered packing $(\Lambda,v_0)$.
Then  $\op{rcone}^0(v_0,v_1,|v-v_0|/2.77)$ does not meet the
$\op{cone}(v_0,\{v_2,v_3\})$.
\end{lemma}

\begin{proof} This is tarski\tarf{tarski:beta:B}.
\end{proof}

\begin{lemma}\tlabel{lemma:typeB-crown}\rating{50} 
Let $(\Lambda,v_0)$ be a centered packing with crown tuple $(v_0,v,w_1,w_2)$.
Let $S=\{v_0,\ldots\}$ be a simplex of type $\SB$. Then
$$\Omega(v_0,S) \cap \op{cone}^0(S) \cap B(v_0,1.385)$$
does not meet $\FC(v_0,v,w_1,w_2)$.
\end{lemma}

\begin{proof}
The reasons given in Section~\ref{sec:separation} for the
disjointness of $\FC$ and $V_S(t_S)$ apply to this
situation as well.
\end{proof}


This completes the justification that
$V_S(t_S)$ is an object that can be treated in separation from the
rest of the local $V$-cell.

\subsection{type C}%DCG 9.8, p97
    \oldlabel{2.9}

If $S(y_1,\ldots,y_6)$ is of type $\SC$, we label vertices so that
the upright diagonal is the first edge.  We use $t_S =+\infty$ (no
truncation).   

\begin{lemma}\tlabel{lemma:typeC-Omega}\rating{100}
Let $S=\{v_0,v_1,v_2,v_3\}$ be a simplex of type $\SC$ in a centered packing
$(\Lambda,v_0)$.
There is a null set $Z$, such that
we have  $ \Omega(v_0,S) \cap \op{cone}^0(S) \subset V_S \cup Z$.
\end{lemma}

\begin{proof}  %% WW Rewrite this proof.  
Vertices outside $S$ cannot affect the shape of $V_S(t_S)$.  Any
vertex $v'$ would have to form a quarter along a face of $S$.  If
the shared face lies along the first edge, it is a quarter $Q$ in
the $Q$-system, because one and hence all quarters along this edge
are in the $Q$-system.  The faces of this quarter are then
barriers. If the shared face lies along the fourth edge, then its
length is at most $2.77$, so that the quarter cannot be part of an
isolated pair. If it is not in the $Q$-system, there must be a
conflicting diagonal. The two faces along this conflicting
diagonal of the adjacent pair in the $Q$-system (that is, the pair
taking precedence over $Q$ in the $Q$-system) are barriers that
shield the $V$-cell at $v'$ from $S$.
\end{proof}


\begin{lemma}\tlabel{lemma:typeC-crown}\rating{50}
Let $(\Lambda,v_0)$ be a centered packing with crown tuple $(v_0,v,w_1,w_2)$.
Let $S=\{v_0,\ldots\}$ be a simplex of type $\SC$. Then
$$\Omega(v_0,S) \cap \op{cone}^0(S)$$
does not meet $\FCR(v_0,v,w_1,w_2)$.
\end{lemma}

\begin{proof}
The reasons given in Section~\ref{sec:separation} for the
disjointness of $\FCR$ and $V_S(t_S)$ apply to simplices of
type $\SC$ as well. 
\end{proof}


This completes the justification that
$V_S(t_S)$ is an object that can be treated in separation from the
rest of the local $V$-cell.

\subsection{type D}%DCG 9.9, p97
    \oldlabel{2.10}



\begin{lemma}\tlabel{lemma:typeD-anchor}\rating{50}
Let $\{v_0,v,v_2,v_3\}$ be a simplex of type $\SD$ 
in a centered packing $(\Lambda,v_0)$,
with $2t_0 < |v_0-v|$.
Then $v_1$ and $v_2$ are consecutive anchors of
$\{v_0,v\}$.
\end{lemma}

On simplices $S$ of type $\SD$, we label vertices so that the
upright diagonal is the first edge.  We use $t_S=+\infty$ (no
truncation).  

Simplices of type $\SD $ are separated from quarters in the
$Q$-system and simplices of types $\SA$ and $\SB$ by procedures
similar to those described for type $\SC$.  The following lemma is
helpful in this regard.


\begin{lemma}\tlabel{lemma:C'Q}\rating{50}
The flat quarter along the face $\{v_0,v_1,v_2\}$ is
in the $Q$-system.
\end{lemma}

\begin{proof}
By tarski\tarf{tarski:245}, there cannot be an enclosed vertex
of height at most $\sqrt2$. 
So nothing is enclosed over the flat quarter.
By tarski\tarf{tarski:245bis}, there cannot be an edge of length
at most $2\sqrt2$ that crosses inside the simplex of type $\SD$.
This implies that the flat quarter does not have
a conflicting diagonal and is not part of an isolated pair.
\end{proof}


\begin{lemma}\tlabel{lemma:VC-no-conv}\rating{50}
Let $(\Lambda,v_0)$ be a centered packing.  Let $\{v_0,v_1,v_2,v_3\}$ be a simplex
of type $\SD$. 
Suppose that $v'\in\op{aff}_+^0(v_0,\{v_1,v_2,v_3\})$.  Then $\op{VC}(\Lambda,v')$ does
not meet $\op{conv}^0(S)$.
\end{lemma}

\begin{proof} If there is a point $x$ of intersection, then
$x$ is closer to $v'$ than to any point of $S$. 
By tarski\tarf{tarski:vor-bar-quad}, this implies that 
$S'=\{v',v,v_1,v_2\}$ is a quarter.  By Lemma~\ref{lemma:C'Q},
$S'$ is in the $Q$-system.  Thus, $\{v,v_1,v_2\}$ is a barrier,
and $x$ is obstructed from $v'$.
\end{proof}


Unlike the other cases, there can in fact be overlap between 
$FxfCR(v_0,\ldots)$ for a crown tuple $(v_0,v_1,v_2,v_3)$ and simplex $\{v_0,v_1,v_2,v_3\}$ 
of type $\SD$.  The
conditions defining a crown tuple are not incompatible with
the conditions defining type $\SD$.  Nevertheless, except in this
obvious case where the simplex of type $\SD$ and the crown tuple are both
constructed from $\{v_0,v_1,v_2,v_3\}$, there
can be no overlap of a $\FC(v,W)$ with a simplex of type
$\SD$.

\begin{lemma}\tlabel{lemma:typeD}\rating{50}
Let $(\Lambda,v_0)$ be a centered packing, $(v_0,v_1,v_2,v_3)$ a crown tuple and $S=\{v_0,\ldots\}$ a
simplex of type $\SD$.  If $\FCR(v_0,v_1,v_2,v_3)$ meets $V_S$, then
$S = \{v_0,v_1,v_2,v_3\}$.
\end{lemma}



\begin{lemma}\tlabel{lemma:fine-2}\rating{50}
Let $(\Lambda,v_0)$ be a centered packing.  Let $S=\{v_0,\ldots\}$ be a simplex of type
$\SA$, $\SB$, $\SC$, or $\SD$.  Then $V_S(t_S)\subset B(v_0,2)$.
\end{lemma}

\begin{proof}  For $\SA$, the truncation is at $\sqrt2 < 2$.  For type $\SB$, the truncation
is at $1.385 < 2$.  For types $\SC$ and $\SD$, each vertex of $S$ has positive orientation in
$S$.  Thus,  $V_S(t_S) \subset \op{conv}(S)$.  By tarski\tarf{tarski:convex-ball}, 
it is enough to show that each vertex
of $S$ lies in $B(v_0,2)$.  This is immediate, since by construction $|v_0-v|<\sqrt2 < 2$, for $v\in S$.
\end{proof}

\begin{lemma}\tlabel{lemma:crown-2}\rating{50}
Let $(\Lambda,v_0)$ be a centered packing with crown tuple $(v_0,v_1,w_1,w_2)$.   Then
$\FC(v_0,v_1,w_1,w_2)\subset B(v_0,2)$.
\end{lemma}

\begin{proof}
By construction $\FC\subset B(v_0,\eta_{V0}(v_0,v_1))$.
We have
   $$\eta_{V0}(v_0,v_1) = \eta(2,2.51,|v_0-v_1|) \le \eta(2,2.51,\sqrt8) < 2.$$
The result follows.
\end{proof}




%\chapter{Sphere Packing,  Hypermap, Fan}
%\chapter{Basic Properties of Standard Components}%DCG Sec.10, p99
%    \label{sec:intro}
%    \oldlabel{1}
%\label{chapter:VQ}



\section{Fan}

There are many different fans $(v,V,E)$ that
can be associated with a centered packing $(\Lambda,v)$.
As we will see, different selections of fans
will lead to different approximations to the function $\sigma(\Lambda,v)$.
It will be important for us to have many different approximations
at our disposal.  For that reason, we consider a number of
fans.

\begin{definition}[compatible]\label{def:compatible}
Let $(\Lambda,v_0)$ be a centered packing.
Let $(v_0,V,E)$ be a fan.  Let $\CalR$ be a set of simplices of $\Lambda$.  
We say that the fan is
$(\Lambda,\CalR)$-{\it compatible} if
if 
\begin{itemize}
\item Every simplex of $\CalR$ has a vertex at $v_0$.
\item
 $V\subset \Lambda$, 
\item For every $\{v_0,v_1,v_2,v_3\}\in\CalR$, we have $\{v_1,v_2\}\in E$, or $C^0(v_0,\{v_1,v_2\})\subset U$.
\item $\op{cone}^0(S)\cap\op{cone}^0(S')\ne\emptyset$ implies $S=S'$, for every $S,S'\in \CalR$.
\end{itemize}
\end{definition}

\begin{lemma}\tlabel{lemma:R-subset}\rating{20}
Let $(\Lambda,v_0)$ be a centered packing, $(v_0,V,E)$ a fan, and $\CalR$ a set of simplices of $\Lambda$.
If the fan is $(\Lambda,\CalR)$-compatible and if
 $\CalR'\subset\CalR$, then the fan is also $(\Lambda,\CalR')$-compatible.
\end{lemma}

\subsection{standard fan}

Let $(\Lambda,v_0)$ be a centered packing.  
Let $V=\Lambda(v_0,2t_0)$.
Let 
$$E = E(\Lambda,v_)) = \{(u,u')\in \Lambda(v,2t_0)^2 \mid |u-u'|\le 2t_0\}.
$$


\begin{lemma}\tlabel{lemma:pack-fan}\rating{20}
Let $(\Lambda,v_0)$ be a centered packing.  Let $V,E$ be as given.
Then $(v,V,E)$ is a fan.
\end{lemma}

\begin{proof}
Use tarski\tarf{tarski:2t0-doesnt-pass-through}.
\end{proof}

\begin{definition}[standard~fan,~-component,~-hypermap]\tlabel{def:standard-fan}  
We call $(v_0,V,E)$ the standard
fan of the centered packing $(\Lambda,v_0)$.   We call the
components of $Y(v_0,V,E)$ the standard components of the centered
packing $(\Lambda,v_0)$.  We call the hypermap $\op{hyp}(v_0,V,E)$
the standard hypermap of the centered packing.  
\end{definition}

\begin{lemma}\tlabel{lemma:Q-in-region}\rating{60}
The $(\Lambda,v_0)$ be a centered packing.  Let $(v_0,V,E)$ be the
standard fan.  Then the standard fan of $(\Lambda,v_0)$ is
$(\Lambda,\CalQ(\Lambda,v_0))$-compatible. Moreover,
for every $\{v_0,v_1,v_2,v_3\}\in\CalQ(\Lambda)$, we have
$$
 \op{aff}_+^0(v_0,\{v_1,v_2,v_3\})\subset U,
$$
for some $U\in[Y(v_0,V,E)$.
\end{lemma}

%\begin{lemma}\tlabel{lemma:graph-compat}
%Let $(\Lambda,v)$ be a centered packing.  The standard graph
%of $(\Lambda,v)$ is $(\Lambda,v)$-compatible.
%\end{lemma}

\begin{proof}  Assume for a contradiction that $Q=\{v_0,v_1,v_2,v_3\}$
with $v_1$ in the open cone over $U_1$ and with $v_2$ in the open
cone over $R_2$.  Then $\{v_0,v_1,v_2\}$ and $\{v_0,w_1,w_2\}$ (a wall
between $R_1$ and $R_2$) cross;  this is contrary to
Lemma~\ref{lemma:barrier-no-overlap}.
\end{proof}


We consider the standard graph  to be the
primary fan attached to a centered packing.  All other
standard graphs are in some sense derivative, either by deleting
certain edges from the standard graphs, or by adding additional
edges.  There are many variations.  We can form a fan from all
the barriers with a vertex at $v_0$.  We can add in edges for
each face of alphabet simplices.  We can even insert additional
edges $\{v_1,v_2\}$ into a fan $(v_0,V,E)$ when $v_1,v_2\in V$ and
$$\op{aff}_+^0(v_0,\{v_1,v_2\})\subset U\in [Y(v_0,V,E)].$$

This flexibility in constructing fans (and the corresponding
hypermaps is intentional).  
We can view it as a sort of mesh-refinement.
When we want more accurate estimate of the function $\sigma$, we
use a more richly structured fan.  When rough estimates of $\sigma$
suffice, we move to a coarsely structured fan.

For this to succeed, we need to be able to switch from one fan
to another with little effort.\FIXX{Add.}




\begin{definition}[tag,~exceptional]\tlabel{def:except} 
We say a component $U$ of $Y(v,V,E)$ is
tagged by an
$n$-gon (triangle, quadrilateral, pentagon, and so forth) if
the set of darts of $\op{hyp}(v,V,E)$ that lead into $R$ is
a single face of the hypermap and if that face is simple and has
size $n$.   We say that
a component of $Y(v,V,E)$ is exceptional, if it is not tagged by a
triangle or quadrilateral.
\index{triangle}\index{quadrilateral}\index{exceptional}
\end{definition}


\begin{lemma}\tlabel{lemma:FC-standard}
Each $\FC(v_0,v,ldots)$ lies entirely in the cone over the standard
component that contains $\{v_0,v\}$.
\end{lemma}

\begin{proof}
By Lemma~\ref{lemma:delta-tri},
the cone over a standard component is bounded by the cones  over the
quasi-regular triangles.
\end{proof}

\subsection{quad}
\label{sec:quad-class}

\begin{definition}[quad~cluster]\tlabel{def:quad-cluster}
Let $(\Lambda,v_0)$ be a centered packing.  Let $U$ be a component
of $Y(v_0,V,E)$ for some $(\Lambda,\CalQ)$-compatible fan
$(v_0,V,E)$ (not necessarily the standard fan).
Suppose that there exists a unique face of the standard hypermap
that leads into $U$ and that face is a quadrilateral $F$.  
The 
four darts have the form $(v_0,v_i,v_{i+1},v_{i-1})$ for some $v_i\in V$,
$i=1,2,3,4$.  The  four vertices $v_i$ are indexed in the order given
by the face permutation on $F$.  Suppose that each $\{v_i,v_{i+1}\}$
is an edge in the {\it standard} fan.  (By construction, it is already a
edge in the fan $(v_0,V,E)$.)  In this setting, we call 
$C=(v_0,(v_1,v_2,v_3,v_4))$ a quad cluster of the fan.  We write
$U = U_C$ and $F= F_C$ for the corresponding component and face.
\index{quad cluster}
\end{definition}



\begin{definition}[standard~component] \tlabel{def:standard-component}
A {\it standard component\/} is a pair $(U,D)$ where $D=(\Lambda,v)$ is a
centered packing and $U$ is one of its standard components.\FIXX{Update.}
\end{definition}
%
 \index{cluster!standard}
 \index{cluster!quad}

\begin{lemma}\tlabel{lemma:quad-classify}
Let $C=(v_0,(v_1,v_2,v_3,v_4))$ be a quad cluster in a fan
of a centered packing $(\Lambda,v_0)$.  Then it has
exactly one of the following forms.
\begin{itemize}
  \item (flat) $\{v_0,v_1,v_2,v_3\}$ and $\{v_0,v_1,v_3,v_4\}$ are flat
    quarters in the $Q$-system.
   \item (flat) $\{v_0,v_2,v_4,v_1\}$ and $\{v_0,v_2,v_4,v_3\}$ are flat
     quarters in the $Q$-system.
     \item (octahedron) There exists an enclosed vertex $w$, such that
       $(v_0,w,v_1,v_2,v_3,v_4)$ is a quartered octahedron with
       four upright quarters in the $Q$-system.
   \item (pure) We have $|v_1-v_3|>\sqrt8$, $|v_2-v_4|>\sqrt8$ and
     there is no enclosed vertex $w$ with $|w-v_0|<\sqrt8$.
    \item (mixed) We have $|v_1-v_3|>\sqrt8$, $|v_2-v_4|>\sqrt8$, there
      is no enclosed vertex forming a quartered octahedron, but
      there exists some enclosed vertex $w$ with $|w-v_0|<\sqrt8$.
      \index{pure}\index{mixed}
\end{itemize}
\end{lemma}

\begin{proof}
By tarski~\tarf{tarski:XX}, if some diagonal $|v_i-v_{i+2}|<\sqrt8$,
then $\op{aff}_+^0(v_0,\{v_i,v_{i+2}\})\subset U_C$. That is, the
diagonal separates the component in two.

If the quad cluster has a diagonal of length less than $\sqr8$
between two corners, there are three possible decompositions. (1)
The two quarters formed by the diagonal lie in the $Q$-system so
that the scoring rules for the $Q$-system are used.  (2) There is
a second diagonal of length at most $\sqr8$, and we use the two
quarters from the second diagonal for the scoring. (3) There is an
enclosed vertex that makes the quad cluster into a quartered
octahedron and the four upright quarters are in the $Q$-system.

Now suppose that neither diagonal is less than $\sqr8$ and the
quad cluster is not a quartered octahedron. If there is no
enclosed vertex of length at most $\sqr8$, the quad cluster
contains no quarters. This is pure.
%An upper bound on the score of the quad
%cluster $(P,D)$ is $\op{svR}(D,P,\sqr2)$. 
The remaining cases are mixed.
%called {\it mixed\/} quad clusters. Mixed quad clusters enclose a
%vertex of height at most $\sqr8$ and do not contain flat quarters.
\end{proof}





\begin{definition}[quad~structured~fan]\tlabel{def:quad-fan}
Let $(\Lambda,v_0)$ be a centered packing.   Let $V_1 = \Lambda(v_0,2t_0)$
and let $V_2$ be the set of vertices $w\in  \Lambda$ such that there
exists a quartered octahedron $(v_0,w,v_1,v_2,v_3,v_4)$.  Let $E_1$
be the set of edges in the standard fan.  Let $E_2$ be the set of
edges $(w,v_i)$, $i=1,2,3,4$ for each quartered octahedron $(v_0,w,v_1,\ldots,v_4)$ as above.  Let $E_3$ be the set of diagonals of flat quarters in
the $Q$-system that lie in a quad cluster.  
We call $(v_0,V_1\cup V_2,E_1\cup E_2 \cup E_3)$ the quad structured fan of the centered packing.
\end{definition}


\begin{lemma}\tlabel{lemma:quad-fan}\rating{60}\usage{Assembly pent-prism}
Let $(\Lambda,v_0)$ be a centered packing.  Then the quad structured fan
of $(\Lambda,v_0)$ is indeed a fan.  
It is $(\Lambda,\CalQ(\Lambda,v_0))$-compatible.
\end{lemma}

\begin{lemma}\tlabel{lemma:pure-sqrt2}
Let $(\Lambda,v_0)$ be a centered packing with pure quad-cluster $C=(v_0,w_1,w_2,w_3,w_4)$.  Let $U_C$ be the corresponding component of the complement of the standard fan.  Then
  $$
  \Omega_{\sqrt2}(\{w_1,\ldots,w_4,v_0\},v_0) \cap U \subset 
  VC(\Lambda,v_0).
  $$
\end{lemma}

\subsection{VCM}
\label{sec:VCM}

We are ready to give 
the construction of the promised measurable subset $\op{VCM}(\Lambda,v_0)$ of the $V$-cell.


\begin{lemma}\tlabel{lemma:Omega-U-meas}\rating{80}
Let $(\Lambda,v_0)$ be a centered packing.  Let $(v_0,V,E)$ be a fan
that is $(\Lambda,\emptyset)$-compatible.  For every $t>0$ and
$U\in[Y(v_0,V,E)]$, the set
  $$
  \Omega(\Lambda,v_0)\cap U \cap B(v_0,t)
  $$
is measurable.
\end{lemma}

\begin{proof}  Lemma~\ref{lemma:hypermap-planar} 
gives that $U\cap B(v_0,t)$ is measurable.
Also,
  $$
  \begin{array}{lll}
  \Omega(\Lambda,v_0)\cap B(v_0,t) &= \Omega(\Lambda(v_0,2t),v_0) 
    \cap B(v_0,t) \\
    &= \bigcap _{v\in\Lambda(v_0,2t)} \Omega(\{v_0,v\},v_0) \cap B(v_0,t),
  \end{array}
  $$
%where $\Omega(\{v_0,v\},v_0)$ is the half-space containing $v_0$ bounded by the
%perpendicular bisector of $\{v_0,v\}$.  
This expression is a finite
intersection of measurable sets, hence measurable.  Thus, $\Omega\cap U\cap B$ is also a measurable set.
\end{proof}

Let $(\Lambda,v_0)$ be a centered packing.  Let $\CalR$ be the union of 
\begin{itemize}
  \item $\CalQ(\Lambda,v_0)$, and
  \item all alphabet simplices of types $\SA$, $\SB$, $\SC$, or $\SD$ with a vertex at $v_0$.  
\end{itemize}
Form a fan $(v_0,V,E)$ with vertices $V$ the union of
\begin{itemize}
  \item $\Lambda(v_0,2t_0)$,
  \item $\{v \mid \{v_0,v\}$  diagonal of a quarter in the  $Q$-system.
\end{itemize}
and with edges $E$ the following
\begin{itemize}
  \item $\{v_1,v_2\}$ such that $\{v_0,v_1,v_2\}$ is a quasi-regular triangle in $\Lambda$.
  \item $\{v_1,v_2\}$ such that there exists $v_3$ such that $\{v_0,v_1,v_2,v_3\}\in\CalR$.
\end{itemize}

\begin{lemma}\tlabel{lemma:VCM-fan}\rating{60}
Let $(\Lambda,v_0)$, $\CalR$, $(v_0,V,E)$ be as given. Then $(v_0,V,E)$ is indeed a fan.
The fan is $(\Lambda,\CalR)$-compatible.  
\end{lemma}




\begin{lemma}\tlabel{lemma:VCM-alphabet}\rating{60}
Let $(\Lambda,v_0)$, $\CalR$, $(v_0,V,E)$ be as given.  There is an injection from $\CalR$
to $[Y(v_0,V,E)]$ given by 
  $$
  \{v_0,v_1,v_2,v_3\}\mapsto \op{aff}_+^0(v_0,\{v_1,v_2,v_3\}).
  $$
\end{lemma}

\begin{definition}[VCM]\tlabel{def:VCMD}
Let $(\Lambda,v_0)$ be  a centered packing.   
Let $\CalD$ be a set of simplices of type $\SD$ having a vertex at $v_0$.  Let $\CalC$ be the
set of crown tuples $(v_0,v_1,w_1,w_2)$ such that $\{v_0,v_1,w_1,w_2\}\not\in\SD$. 
Let $\CalR'$ be the image
of $\CalR$ in $[Y(v_0,V,E)]$ under the injection of Lemma~\ref{lemma:VCM-alphabet}.
Define $\op{VCM}_{\CalD}(\Lambda,v_0)$ to be the union of the following pieces:
    \begin{itemize}
    \item $\FCR(v_0,v_1,w_1,w_2)$, for each crown tuple $(v_0,v_1,w_1,w_2)\in\CalC$.
    \item truncations of Voronoi pieces $V_S(t_S)$ for simplices of type
        $\SA$, $\SB$,  $\SC$, $\SD$.  (For type $\SD$, we require $S\in \CalD$.)
    \item $\Omega_{t_0}(\Lambda,v_0)\cap U$ for each $U\in[Y(v_0,V,E)]\setminus \CalR'$.
    \item $\Omega(Q,v_0)$ for each quarter $Q\in\CalQ(\Lambda,v_0)$.
    \item $\Omega_{\sqrt2}(\{v_0,w_1,w_2,w_3,w_4\},v_0)\cap U_C$ 
     for each pure quad
     cluster $C=(v_0,(w_1,w_2,w_3,w_4))$.
    \end{itemize}
\end{definition}

\begin{lemma}\tlabel{lemma:Omega-fan-measure}\rating{100}
Let $(\Lambda,v_0)$ be a centered packing.  Choose $\CalD$ as in Definition~\ref{def:VCMD}.  
The individual pieces $\FCR$, $\Omega(\ldots)$, $\ldots$ 
defining $\op{VCM}_{\CalD}(\Lambda,v_0)$ are measurable.  Their union is as well.
\end{lemma}

\begin{lemma}\tlabel{lemma:VCM-VC}\rating{40}
Let $(\Lambda,v_0)$ be a centered packing.
For any choice $\CalD$ as above,  $\op{VCM}_{\CalD}(\Lambda,v_0)$ is a subset of $\op{VC}(\Lambda,v_0)$.
\end{lemma}

\begin{proof}  Each of the individual pieces of $\op{VCM}_{\CalD}$ is a subset of $\op{VC}$.
\end{proof}





\subsection{plate}
\subsection{truncated corner cell}

A section is to be inserted here giving the relation between
plates, components of fans, and $V$-cells.  There will also
be a discussion of truncated corner cells and the relation to
components of fans and $V$-cells.%
\FIXX{Add this section}


\section{Special Structure}



\subsection{sundry}



\begin{definition}[central~vertex]
Define the {\it central vertex\/} $v$ of a flat quarter to be the
vertex for which $\{v_0,v\}$ is the edge opposite the diagonal.
\end{definition}

%\subsection{types}\label{sec:types}%DCG 10.2, p100

\begin{definition}[type]
Let $v$ be a vertex of height at most $2t_0$.  We say that $v$ has
{\it type\/} $(p,q)$ if every standard component with a vertex at $\bar
v$ (the radial projection of $v$) is a triangle or a quadrilateral,
and if there are exactly $p$ triangular faces and $q$ quadrilateral
faces that meet at $\bar v$.  We write $(p_v,q_v)$ for the type of
$v$.
\index{type}
\end{definition}

%\subsection{contexts} %DCG 11.2, p.113
%    \oldlabel{3.3}

The context $\x(3,0)$ is to be regarded as two
quasi-regular tetrahedra sharing a face rather than as three
quarters along a diagonal.  In particular, by
Definition~\ref{def:q-system}, the upright quarters do not belong
to the $Q$-system.

%\subsection{slices} %DCG 11.5,p.115
%    \oldlabel{3.6}
\label{sec:slice}  % was sec:anchored-simplex

\begin{definition}[slice,~loop,~gap]
Let $\{v_0,v\}$ be an upright diagonal, and let
$v_1,v_2,\ldots,v_k=v_1$ be its anchors, ordered cyclically around
$\{v_0,v\}$.  This cyclic order gives dihedral angles between
consecutive anchors around the upright diagonal. We define the
dihedral angles so that their sum is $2\pi$, even though this will
lead us to depart from our usual conventions by assigning a
dihedral angle greater than $\pi$ when all the anchors are
concentrated in some half-space bounded by a plane through
$\{v_0,v\}$. When the dihedral angle of $S=\{v_0,v,v_i,v_{i+1}\}$ is at
most $\pi$, we say that $S$ is a {\it slice\/} if
$|v_i-v_{i+1}|\le3.2$. (The constant $3.2$ appears throughout this
chapter.) All upright quarters are slices. If an upright
diagonal is completely surrounded by slices, the
upright diagonal is sometimes called a {\it loop}. If
$|v_i-v_{i+1}|>3.2$ and the angle is less than $\pi$, we say there
is a {\it gap\/} around $\{v_0,v\}$ between $v_i$ and $v_{i+1}$.
\index{slice}\index{loop}\index{gap}
\end{definition}

\begin{definition}[unconfined,~crowded]\index{unconfined@3-unconfined}
 \index{crowded4@3-crowded}\index{crowded3@4-crowded}
% Def'n copied from linprog.tex
Consider an upright diagonal that is not a loop. Let $R$ be the
standard component that contains the upright diagonal and its
surrounding quarters.  Assume we are in the context $(4,1)$ or
$(5,1)$.  In the context $(4,1)$, suppose that there does not exist
a plane through the upright diagonal such that all three quarters
lie in the same half-space bounded by the plane. Then we say that
the context is {\it $3$-unconfined}. If such a plane exists, we say
that the context is $3$-crowded. We call the context $(5,1)$ a
$4$-crowded upright diagonal. Sections~\ref{x-3.4} and \ref{x-3.5}
reduce everything to contexts with four or five anchors around each
vertex.  If there are $5$ darts, 
Remark~\ref{rem:5dart} shows that we can assume at most one
gap. This gives contexts $(5,0)$ and $(5,1)$.  If there are four
anchors, then Lemma~\ref{x-3.9.1} will dismiss all contexts except
$(4,0)$ and $(4,1)$. Thus, every upright diagonal is exactly one of
the following: a loop, $3$-unconfined, $3$-crowded, or $4$-crowded.
%\def\Sfour{{{\cal\mathbf S}_4^+}}  --> $4$-crowded upright diagonal
%\def\Sminus{{{\cal\mathbf S}_3^-}} --> $3$-crowded upright diagonal
%\def\Splus{{{\cal\mathbf S}_3^+}}  --> $3$-unconfined upright diagonal
\end{definition}


This lemma is a consequence of the two others that follow
(Lemma~\ref{lemma:slice-quarter}, Lemma~\ref{lemma:3-crowded}).
\FIXX{Extract the geometry as a lemma that is independent.
The
context of the lemma is the set of slices that have
not been erased by previous reductions.}

\begin{lemma}\tlabel{lemma:anchor-no-overlap}
The interiors of slices do not meet.
\end{lemma}

\begin{proof}
The remaining contexts have four or  five anchors. Let $w$ and the
slice $S=\{v_0,v,v_1,v_2\}$ be as in Section~\ref{sec:slice}.
Our object is to describe the local geometry when an upright
diagonal is enclosed over a slice. If $|v_1-v_2|\le
2\sqrt{2}$, we have seen in tarski\tarf{tarski:double-face} that
there can be no enclosed upright diagonal with $\ge 4$ anchors
over the slice $S$.

Assume  $|v_1-v_2|>2\sqrt{2}$. Let $w_1,\ldots,w_k$, $k\ge4$, be the
anchors of $\{v_0,w\}$, indexed consecutively. The anchors of $\{v_0,w\}$ do not
lie in $C(S)$, and the triangles $\{v_0,w,w_i\}$ and $\{v_0,v,v_j\}$ do not
overlap. Thus, the plane $\{v_0,v_1,v_2\}$ separates $w$ from
$\{w_1,\ldots,w_k\}$. Set $S_i=\{v_0,w,w_i,w_{i+1}\}$.
By a calculation\footnote{\calc{83777706}} %A8
%$\A_8$,
    $$\pi\ge \dih(S_1)+\cdots+\dih(S_{k-1})\ge (k-1)0.956.$$

Thus, $k=4$. The common upright diagonal  of the three simplices
$\{S_i\}$ is {\it $3$-crowded}.  We claim that
$\{v_1,v_2\}=\{w_1,w_4\}$. Suppose to the contrary that, after
reindexing as necessary, $S_0=\{v_0,w,w_1,v_1\}$ is a simplex, with
$v_1\ne w_1$, that does not overlap $S_1,\ldots,S_3$. Then $\pi\ge
\dih(S_0)+\cdots+\dih(S_3)$. So
    $0.28\ge \pi-3(0.956)\ge \dih(S_0)$.
A calculation\footnote{\calc{83777706}} %A8
now implies that $|w-v_1|\ge 2\sqrt{2}$.

By tarski\tarf{tarski:336}, the four vertices
$\{v_0,w,v_1,v_2\}$ cannot be coplanar.
We have that $2\sqrt{2}\ge|w-v_0|$ and by tarski\tarf{tarski:E:part4:1},
we also have $|w-v_0|>2\sqrt2$.
This contradiction establishes that $v_1=w_1$.
\end{proof}


\begin{definition}[unconfined]
Let a $3$-unconfined node be a node that is an upright diagonal 
with four darts and one gap in a situation where none of
the quarters along this upright diagonal masks a flat quarter.
\end{definition}








\begin{definition}[tag]\tlabel{def:leads-to}
Let $(\Lambda,v_0)$ be a centered packing.  Let $(v_0,V,E)$ be a fan
and $\CalR$ a set of simplices such that the fan is
that is $(\Lambda,\CalR)$-compatible.  A face $F$ of the hypermap is said
to {\it tag} $\{v_0,v_1,v_2,v_3\}\in \CalR$ if 
\begin{itemize}
 \item $F$ is a triangle,
 \item  $\op{aff}_+(v_0,\{v_1,v_2,v_3\})$ is a component of
  $Y(v_0,V,E)$.
 \item $F$ leads into that component.
\end{itemize}
\end{definition}


\section{Score}





\subsection{Voronoi cell and simplex}


Let $S$ be a set of four vertices and let $v$ be a vertex of that simplex. Let
$\Omega(v,S)$ be the subset of $\op{conv}^0(S)$ consisting of points closer
to $v$ than to any other vertex of $S$. By
Lemma~\ref{lemma:Q-divide}, if $S\in\CalQ(\Lambda,v)$, then
$$\Omega(v,S) = \Omega(\Lambda,v)\cap\op{conv}^0(S).$$
Under the assumption that $S$ contains its circumcenter and that
every one of its faces contains its circumcenter, an explicit
formula for the volume $\op{vol}(\Omega(v,S))$ has been
calculated.  %% \cite[Section~8.6.3]{part1}. 

\begin{lemma}\FIXX{Move to volumes}  Let $S = \{v_0,v_1,v_2,v_3\}$ be a set of four points in $\ring{R}^3$.
Assume that the circumcenter of $S$ is contained in $\op{conv}^0(S)$.  Assume
that each face of $S$ is acute.
Then 
  $$
  \op{vol}\,\Omega(v_1,S) = f(x_{ij}),
  $$ 
where $x_{ij} = |v_i-v_j|^2$, and
$$
   f(x_{ij}) = \sum_{(i,j,k)\in \Pi} \frac{x_{0i} (x_{0j}+x_{ij}-x_{0i})\chi(x_{jk},x_{ik},x_{0k},x_{0i},x_{0j},x_{ij})}
   {48\ups(x_{0i},x_{0j},x_{ij})\Delta(x_{01},x_{02},x_{03},x_{23},x_{13},x_{12})^{1/2}}
$$
and $\Pi = \{(1,2,3),\ldots\}$ is the set of the six permutations of $(1,2,3)$.
(The function $\chi$ is defined in Definition~\ref{def:chi}.)
\end{lemma}

\begin{proof} By tarski\tarf{tarski:XX}, the set $\Omega(v_1,S)$ is a union of
six Rogers simplices, up to a null set.  The volume of a Rogers simplex appears
as Lemma~\ref{XX}.  Summing the contributions from the six simplices, we obtain
the given formula.
\end{proof}

The function $f(x_{ij})$ makes perfect sense for any $S$ for which $\ups(x_{0i},x_{0j},x_{ik},\Delta(x_{ij})\ne 0$,
even though the lemma is valid only under special constraints.
For any simplex $S=\{v_0,v_1,v_2,v_3\}$ such that the denominator of $f$ is nonzero, we define
$$
\op{volan}(v_0,S) = f(x_{ij}),\quad x_{ij} = |v_i-v_j|^2.
$$  
(The denominator is always nonzero when $S$ is not collinear by tarski\tarf{tarski:XX}{~}\tarf{tarski:XX}.)
The following appeared as Claim~\ref{claim:volan}.

\begin{lemma}\tlabel{lemma:volan}  %%Cf. claim:volan
Let $S=\{v_1,v_2,v_3,v_4\}$ be in the $\CalQ$-system. Then
    $$
    \sum_{i=1}^4 \op{volan}(v_i,S) = \sum_{i=1}^4
    \op{vol}(\Omega(v_i,S)) = \op{vol}(\op{conv}^0(S)).
    $$
\end{lemma}

\begin{proof} As we have explicit formulas for everything involved,
it is just a matter of plugging in the formulas and checking the sum.
(Even without a calculation, the lemma follows by the principle of analytic continuation;
but in the interest of proving everything from first principles, we
check the sum.)
\end{proof}

\subsection{truncation}

Let $(\Lambda,v)$ be a centered packing and let $(v,V,E)$ be a compatible
fan.
If $R$ is a component of $Y(v,V,E)$, we write 
  $$
  V_R(v,t) = \op{VC}(\Lambda,v_0)\cap C(R)\cap B(v,t)
  $$
We often
take $t=t_0$.

%If $\{v_0,v\}$, of length between $2t_0$ and
%$2\sqrt{2}$, is not the diagonal of an upright quarter in the
%$Q$-system, then $v$ does not affect the truncated cell $V_R(t_0)$
%and may be disregarded. For this reason we confine our attention
%to upright diagonals that lie along an upright quarter in the
%$Q$-system.


 Let $S=\{v_0,v_1,v_2,v_3\}$ be a simplex. Fix $t$ in the range
$t_0\le t\le\sqrt2$.  Assume that $t$ is at most the circumradius
of $S$. Assume that it is at least the circumradius of each of the
faces of $S$.  Let $\op{VC}_t(\Lambda,v_0,S)$ be the intersection of
$\op{VC}(\Lambda,v_0,S)$ with the ball $B(v_0,t)$. Under the assumption
that $S$ contains its circumcenter and that every one of its faces
contains it circumcenter, an explicit formula for the volume
$$\op{vol}(\op{VC}_t(\Lambda,v_0,S))$$ is calculated by means of
Lemma~\ref{XX} from the six quoins that form $\op{VC}(\Lambda,v_0,S)$.
This leads to the
following formula. Let $h_i = |v_i-v_0|/2$ and
$\eta_{ij}=\eta_V(v_0,v_i,v_j)$, and let $S_3$ be the group of
permutations of $\{1,2,3\}$ in
\begin{equation}
   \op{vol}\,\op{VC}_t(\Lambda,v_0,S) =
   \sol(S)/3 - \sum_{i=1}^3 \frac{\dih(v_i,S)}{2\pi}\op{vol}\,\op{cap_i}
   +\sum_{(i,j,k)\in S_3} \quo(R(h_i,\eta_{ij},t)).
   \tlabel{eqn:vol-theta-0}
\end{equation}


We extend Formula~\ref{eqn:vol-theta-0} by setting
    $$\quo(R(a,b,c)) = 0,$$
if the constraint $a < b < c$ fails to hold.  Similarly, set
$\op{vol}\,\op{cap}_i=0$ if $|v_i-v_0|\ge 2t$.  With these
conventions,  Formula~\ref{eqn:vol-theta-0} extends to all
simplices.  We write the extension of $\op{vol}\,\op{VC}_t(\Lambda,v,S)$
as
$$\op{vol}\,{\op{VC}^+_t}(\Lambda,v,S).$$


\subsection{score}
\tlabel{sec:ssc}

We show that the function $\sigma$ can be expressed as a sum over
terms attached to each of the standard components.



%Recall $|S|$ is the convex hull of a set $S\subset
%\ring{R}^3$.

We break $\sigma$ into a sum
   \begin{equation}
   \sigma(\Lambda,v) = \sum_R\,\sigma_R(\Lambda,v),
   \end{equation}
indexed by the standard clusters $(R,D)$.  Let
   $$
   \op{VC}_R(\Lambda,v) = \op{VC}(\Lambda,v)\cap \op{cone}(R),
   $$
whenever $R$ is a measurable subset of the unit sphere.  Let
   $$
   \CalQ(\Lambda,v,R) = \{Q\in \CalQ(\Lambda,v) : \op{conv}^0(Q)\subset \op{cone}(R)\}.
   $$
By Lemma~\ref{lemma:Q-in-region},
 each $Q$ is entirely contained in the cone over a single
standard component.

\begin{definition}[$\op{svR}$,~$\sigma$] \tlabel{def:score-std-region}
If $(v,V,E)$ is compatible with $\Lambda$, then each $Q\in\CalQ(\Lambda,v)$
lies in a uniquely determined component $R$ of $Y(v,V,E)$.
Let $(v,V,E)$ be a compatible fan and let $R$ be a
component of $Y(v,V,E)$.  Set
      $$
      \op{svR}(v,R,\Lambda,\lambda) =
      \op{sovo}(v,\op{VC}_R(\Lambda,v),\lambda).
      $$
Set
      $$
      \sigma(\Lambda,v,R) = \op{svR}(v,R,\Lambda,\lambda_{oct}) 
      + \lambda_{oct,v}
         \sum_{Q\in\CalQ(v,R,D)} A_1(Q,c(Q,D),v_0).
      $$
\index{vzorR@$\op{svR}$} \index{zzsigmaR@$\sigma$}
\end{definition}

\begin{lemma}\tlabel{lemma:sigma-sum}
Let $(v,V,E)$ be a compatible fan.
$\sigma(\Lambda,v) = \sum_R\sigma_R(\Lambda,v,R)$, where the sum runs
over all components of $Y(v,V,E)$.
\end{lemma}

\begin{proof}
   $$
   \begin{array}{lll}
      \sigma(\Lambda,v)
      &= \lambda_{oct,v} (\op{vol}\,\Omega(\Lambda,v) + A_0(\Lambda,v))+16\pi/3\\
      &= \lambda_{oct,v} (\op{vol}\,\op{VC}(\Lambda,v)+\sum_{Q\in\CalQ(\Lambda,v)}
         A_1(Q,c(Q,D),v_0)) + (4) (4\pi/3)\\
      &= \sum_R \left (\lambda_{oct,v} \op{vol}\,\op{VC}_R(\Lambda,v) 
         +\lambda_{oct,v}
         \sum_{Q\in\CalQ(R,D)} A_1(Q,c(Q,D),v_0) +
         \lambda_{oct,s}\sol(v,R)\right).
   \end{array}
   $$
\end{proof}

Also, we have
    \begin{equation}
    \op{svR}(\Lambda,v)=\sum_{R\subset Y(v,V,E)}
    \op{svR}(\Lambda,v,R).
    \tlabel{eqn:vorD}
    \end{equation}

\begin{lemma}\tlabel{lemma:R'}
Let $(v,V,E)$ be a compatible fan.  If $R\subset Y(v,V,E)$
is a component that is disjoint from $\op{conv}^0(Q)$ for all 
$Q\in\CalQ(\Lambda,v)$, then
   $$
   \sigma(\Lambda,v,R) = \op{svR}(\Lambda,v,R).
   $$
If, on the other hand, $R = \op{aff}_+^0(v,\{v_1,v_2,v_3\})$ for
some $\{v,v_1,v_2,v_3\}\in\CalQ(\Lambda,v)$, then
   $$  
   \sigma(\Lambda,v,R) =  \sigma(v,Q,c(Q,D)).
   $$
\end{lemma}

\begin{proof} Substitute the definition of $A_1$
(Equation~\ref{eqn:a1-sigma}) into the definition of
$\sigma_R(\Lambda,v)$, noting that $\op{VC}(\Lambda,v,Q) 
= \op{VC}_{R}(\Lambda,v)$,
\end{proof}

\begin{remark}   Lemma~\ref{lemma:R'} explains why we have chosen
the same symbol $\sigma$ for the functions $\sigma(\Lambda,v,R)$ and
$\sigma(Q,c,v)$.  We can view Lemma~\ref{lemma:R'} as asserting a
linear relation in the functions $\sigma$:
   $$\sigma(\Lambda,v,R) = \sigma(\Lambda,v,R') + \sum \sigma(v,Q,c).$$
The sum runs over $Q\in\CalQ(\Lambda,v)$ that lie in the cone over $R$.
\end{remark}

%\subsection{score}

%% 
By the results of Sections~\ref{x-2.7}, \ref{x-2.8}, \ref{x-2.9},
$\sigma(\Lambda,v)$ can be broken into a corresponding sum,
    $$
    \begin{array}{lll}
    \sigma_R(\Lambda,v) &\le \sum_Q \sigma(Q) + \sigma(V_P),
                \hbox{ for quarters $Q$ in the $Q$-system, where}\\
    \sigma(V_P) &= \op{sovo}(\tildeV_P(t_0),\lambda_{oct})+  \sum_{\SA,\SB,\SC} \op{sovo}(V_S(t_S),\lambda_{oct})
        +\op{sovo}(\FCR(\cdots),\lambda_{oct}).\\
    \end{array}
    $$
Because of the separation
results of Sections~\ref{x-2.7}--\ref{x-2.8},  we may score
$\tildeV_P(t_0)$ by Formula~\ref{eqn:3.7}. Bounds on the score of
simplices of type $\SB$ appear in \calc{193836552}.



\subsection{truncated tetrahedron}


We set
    \begin{equation}
    \begin{array}{lll}
    \op{sv}(S,t) &=
    \sol(S)\phi(t,t,\lambda_{oct})
    +\sum_{i=1,\ h_i\le t}^3 d_i (1-h_i/t) (\phi(h_i,t,\lambda_{oct})-
    \phi(t,t,\lambda_{oct})) \\
    &+\sum_{(i,j,k)\in S_3}
    \lambda_{oct,v}
    \quo(R(h_i,\eta(y_i,y_j,y_{k+3}),t)).
    \tlabel{eqn:3.5}
    \end{array}
    \end{equation}
In the definition, we adopt the convention that $\quo(R)=0$, if
$R=R(a,b,c)$ does not exist (that is, if the condition
    $0< a\le b\le c$
is violated). In the second sum, $S_3$ is the set of permutations
on three letters. This definition is compatible with
Definition~\ref{def:svor}.

We have
    \begin{equation}
    \begin{array}{lll}
    \op{svR}(v,P,\Lambda,t) &=
    \sol(P)\phi(t,t)
    +\sum_{|v_i-v_0|\le 2t} d_i (1-|v_i-v_0|/(2t)) (\phi(|v_i-v_0|/2,t)-
    \phi(t,t)) \\
    &-\sum_{R} 4\doct \quo(R).
    \tlabel{eqn:3.7}
    \end{array}
    \end{equation}
The first sum runs over vertices in $P$ of height at most $2t$.
The second sum runs over Rogers simplices $R(|v_i-v_0|/2,\eta(F),t)$
in $P$, where $F=\{v_0,v_1,v_2\}$ is a face of circumradius
$\eta(F)$ at most $t$, formed by vertices in $P$.  The constant
$d_i$ is the total dihedral angle along $\{v_0,v_i\}$ of the
standard cluster. The truncations $t=t_0=1.255$ and $t=\sqrt2$
will be of particular importance.
    Set $A(h) = (1-h/t_0) (\phi(h,t_0)-\phi(t_0,t_0))$.\index{A}

\begin{remark}  We have introduced both untruncated and truncated
versions of functions $\op{svR}$ and $\sigma$.  The truncated versions
are used to give upper bounds on the untruncated versions.  For
example,  in the function $\sigma(\Lambda,v)$, the $V$-cell contributes
through its volume.  The volume
appears with a negative coefficient 
$\lambda_v=-4\doct$.  Thus, we obtain an
upper bound on $\sigma(\Lambda,v)$ by discarding bits of volume from the
$V$-cell.   This suggests that we might try to give upper bounds
on the score $\sigma(\Lambda,v)$ by truncating the $V$-cell in various
ways. This is the reason for the truncated versions of these
functions.
\end{remark}




\subsection{squander}% DCG 10.1, p99
    %\heads{3. Functions}


% Plus Formula 7 on scores.

We consider the functions
    $\sigma_R(\Lambda,v)-\lambda\zeta\sol(R)\,\pt$,
for $\lambda=0$, $1$, or $3.2$, where $R$ is a standard cluster.
%The constant $3.2$ was determined experimentally.
We write
    $$
    \tau_R(\Lambda,v) = \sol(R)\zeta\,\pt -
    \sigma_R(\Lambda,v).
    $$
We will see that $\tau_R(\Lambda,v)$ has a simple interpretation.  If $(\Lambda,v)$
is a centered packing with standard clusters $\{R\}$, set $\tau(\Lambda,v)
= \sum_{R}\tau_R(\Lambda,v)$.
\smallskip



\begin{lemma}\tlabel{lemma:sigma-tau}
    %\proclaim{Lemma 3.2}
    $$\sigma(\Lambda,v) = {4\pi \zeta\,\pt} - \tau(\Lambda,v).$$
\end{lemma}

\begin{proof} Let $\{R\}$ be the standard clusters in $(\Lambda,v)$. Then
    $$
    \sigma(\Lambda,v) = \sum_R\sigma_{R_i}(\Lambda,v) +
        (4\pi-\sum_R\sol(R_i))\zeta\,\pt = 4\pi \zeta\,\pt - \sum_R\tau_{R_i}(\Lambda,v).
    $$
\end{proof}


\begin{lemma}\tlabel{lemma:squander-contravene}
If there are standard clusters $R_1,\ldots,R_k$ such that
$$\sum_{i=1}^k \tau_{R_i}(\Lambda,v)> \squander,$$
then $(\Lambda,v)$ does not contravene.
\end{lemma}

\begin{proof}
$$\sigma(\Lambda,v) = 4\pi\zeta,\pt -\sum_R{R_i}(\Lambda,v) < 8\,\pt.$$
\end{proof}


The function $\tau_R(\Lambda,v)$ gives the amount {\it squandered\/} by a
particular standard cluster $R$.  If nothing is squandered, then
$\tau_{R_i}(\Lambda,v)=0$ for every standard cluster, and the upper bound
on $\sigma(\Lambda,v)$ is
    $4\pi\zeta\,\pt\approx 22.8\,\pt$.
We note that $14.8\,\pt > \squander+\epsilon_0$.  We sometimes use this
approximation.  Set $\trgt = \squander+\epsilon_0$ (the squander target).



%% WW Deleting mentioned \calc{629256313},
%% \calc{917032944}, \calc{738318844}, and \calc{587618947}.
%% Perhaps they are no longer needed in the proof!

%% Major Deletion: SVN:16 has proof of local optimality. Gone in SVN:23.






\subsection{misplaced}

%% WAS IN FORMULATION. DOESN'T BELONG IN TARSKI.

We conclude with the proof of the main theorem of the chapter.

\begin{proof} {\bf (Theorem~\ref{thm:nonoverlap})}
The rules defining the $Q$-system specify a uniquely determined
set of simplices.  The proof that their interiors are pairwise
disjoint is established by the preceding series of lemmas.
Lemma~\ref{tarski:qrtet-over} shows that the interiors of
quasi-regular tetrahedra do not meet the interiors of other
simplices in the $Q$-system. Lemma~\ref{tarski:oct-over} shows that
the quarters in quartered octahedra are well-behaved.
Lemma~\ref{tarski:adj-over} shows that the interiors of other
quarters in adjacent pairs are disjoint from the interiors of
other simplices in the $Q$-system. Finally, 
tarski\tarf{tarski:iso-over} treats isolated
quarters. These cases cover all
possibilities since every simplex in the $Q$-system is a
quasi-regular tetrahedron or strict quarter, and every strict
quarter is either part of an adjacent pair or isolated.
\end{proof}


\begin{definition}[height] \label{def:height}  Let $\Lambda$ be a
packing.  Assume that the coordinate system is fixed in
such a way that the origin is a vertex of the packing.  The {\it
height\/} of a vertex is its distance from the origin.
%
 \index{height}
\end{definition}

\begin{definition}[enclosed] \label{def:enclosed}\index{enclosed}
We say that a vertex is {\it enclosed\/} over a figure if it lies
in the interior of the cone at the origin generated by the figure.
%
 \index{vertex!enclosed}\index{enclosed}
\end{definition}

%\begin{definition}[dihedral~angle]\label{def:dih}
%In general, let $\dih(S)$ be the dihedral angle of a simplex $S$
%along its first edge. When we write a simplex in terms of its
%vertices $(w_1,w_2,w_3,w_4)$, then $\{w_1,w_2\}$ is understood to
%be the first edge.
%%
% \index{dih (dihedral angle)}
%\end{definition}


Our simplices are generally assumed to come labeled with a
distinguished vertex, fixed  at the origin. (The origin will
always be at a vertex of the packing.) We number the edges of each
simplex $1,\ldots,6$, so that edges $1$, $2$, and $3$ meet at the
origin, and the edges $i$ and $i+3$ are opposite, for $i=1,2,3$.
$S(y_1,y_2,\ldots,y_6)$ denotes a simplex whose edges have lengths
$y_i$, indexed in this way. We refer to the endpoints away from
the origin of the first, second, and third edges as the first,
second, and third vertices.
%
 \index{labels!edge}
 \index{first!edge}



