\chapter{Geometric Detail}%DCG "The S-System" Sec.9, p85
    \label{sec:fine}
    \oldlabel{2}

The previous chapter gives the constructions
needed in the top-level description of the proof.  
These constructions do not take us far.   This
chapter gives the details of geometric constructions that
will be used to carry the proof to completion. 
We have discussed many of these constructions before, such
as V-cells, the Q-system, fans, and fitted crowns.  In this
chapter we look at how these structures are interrelated.

\section{Interaction}


Let $\CalQ=\CalQ(\Lambda)$ be the $Q$-system.  
For $v\in\Lambda$, let $\CalQ(\Lambda,v)$ be the subset
of those with a vertex at $v$.\index{Qv@$\CalQ(\Lambda,v)$}



\begin{lemma} \tlabel{lemma:voronoi-truncation-over-Q}  
Let $(\Lambda,v_0)$ be a centered packing.  Let $X$ be the union
of the sets $\op{aff}_+(v_0,\{v_1,v_2\})$ as $\{v_0,v_1,v_2\}$
runs over the barriers with a vertex at $v_0$.  Let $Z$ be the
union of the following tips:
   $$
   \op{aff}_+^0(v_0,\{v_1,v_2,v_3\}) \cap \op{aff}_-
   (\{v_1,v_2,v_3\},v_0)\cap \Omega_2(\Lambda,v_0).
   $$
as $\{v_0,v_1,v_2,v_3\}$ ranges over $\CalQ(\Lambda,v_0)$.
Then 
$$\Omega_2(\Lambda,v_0)
 \subset X\cup Z\cup \op{VC}(\Lambda,v_0).$$
%If $x$ lies in the \index{Voronoi cell} Voronoi cell at $v_0$, 
%but not in the $V$-cell at $v_0$, then there exists a
%simplex $Q\in\CalQ(\Lambda,v_0)$, such that $x$ lies in the cone (at $v_0$)
%over $Q$. Moreover, $x$ does not lie in  $\op{conv}^0(Q)$.
(That is, up to the null set $X$, the $V$-cell contains
the truncated tip-reduced Voronoi cell.)
\end{lemma}

\begin{proof}  Let 
 $$x \in \Omega_2(\Lambda,v_0)
 \setminus (X\cup\op{VC}(\Lambda,v_0)).
 $$
By
the definition of $V$-cell, there is a barrier $\{v_1,v_2,v_3\}$
such that
  $$\op{conv}\{v_1,v_2,v_3\}\cap \op{conv}^0(v_0,x)\ne\emptyset.$$ 
We have $v_0\not\in\{v_1,v_2,v_3\}$, for otherwise $x\in X$, 
which is contrary to assumption. 
By tarski\tarf{tarski:vor-bar-tet}\tarf{tarski:vor-bar-quad},
the simplex $Q=\{v_0,v_1,v_2,v_3\}$ is a quasi-regular tetrahedron
or a flat quarter.  If the barrier is the face of a flat quarter,
then $Q$ too is in
the $Q$-system.  By tarski\tarf{tarski:pass-cone},  $x\in\op{aff}_+(v_0,\{v_1,v_2,v_3\})$.  
We get $x\in\op{aff}_+^0(v_0,\{v_1,v_2,v_3\})$ from $x\not\in X$.

We have $x\in\op{aff}_-
   (\{v_1,v_2,v_3\},v_0)$. Otherwise, $\op{conv}^0(v_0,x)$ does
not meet $\op{conv}\{v_1,v_2,v_3\}$.
%We have $x\in\op{conv}(Q)\setminus X$. Otherwise, 
%the barrier contains $x$, and the plane through the barrier contains
%$v_0$, so that $\{v_0,v_1,v_2,v_3\}$ is planar.  This is contrary
%to tarski\tarf{tarski:flat-Q}. 
The rest is clear.
\end{proof}



\begin{lemma}\tlabel{lemma:VC-Omega}
Let $(\Lambda,v_0)$ be a centered packing.
Inside the ball of radius $t_0$ at $v_0$, the $V$-cell and
Voronoi cell coincide up to a null set:
   $$B(v_0,t_0)\cap \op{VC}(\Lambda,v_0) \equiv B(v_0,t_0)\cap \Omega(\Lambda,v_0).$$
\end{lemma}

\begin{proof} Let $X$ be the (finite) union of planes equidistant from $v_0$ and $w\in \Lambda(v_0,2t_0)$.  $X$ is a null set.  Let $Y$ be the (finite) union of $\op{aff}_+(w,\{w_1,w_2\})$, for $w\in \Lambda(v_0,2t_0)$ and $\{w,w_1,w_2\}$ a barrier.  $Y$ is a null set.
The Voronoi cells $\Omega(\Lambda,w)$, $w\in\Lambda(v_0,2t_0)$ cover
$B(v_0,t_0)$ except for the null set $X$.  
Suppose that some point  $x\not\in X\cup Y$
lies in $$(B(v_0,\Lambda)\cap\op{VC}(\Lambda,v_0) )\setminus \Omega(\Lambda,v_0).$$
Then $x\in \Omega(\Lambda,w)$, where
$w\ne v_0$.  
% By Lemma~\ref{tarski:unobstr-t0}, 
From $x\not\in Y$, it follows that the point $v_0$ is
unobstructed  at $x$ by tarski\tarf{tarski:unobstr-t0}.  
Thus, $|x-w|< |x-v_0|\le t_0$.  By
tarski\tarf{tarski:unobstr-t0} again, $w$ is unobstructed at $x$, so
that $x\in \op{VC}(\Lambda,w)$, contrary to the assumption
$x\in\op{VC}(\Lambda,v_0)$.  Thus $B(v,t_0)\cap\op{VC}(\Lambda,v_0)\subset\Omega(\Lambda,v_0)$.

Similarly, if $x\in B(v_0,t_0)\cap \Omega(\Lambda,v_0)$ and $x\not\in Y$,
then $x$ is
unobstructed at $v_0$, and $x\in \op{VC}(\Lambda,v_0)$.
\end{proof}

\bigskip

%\begin{remark} The next lemma helps to determine which $V$-cell
%a given point $x$ belongs to.  If $x$ lies in the open cone over a
%simplex $Q_0$ in $\CalQ$, then Lemma~\ref{lemma:Q-divide}
%describes the $V$-cell decomposition inside $Q$;  beyond $Q$ the
%point $v_0$ is obstructed by a face of $Q$, so that such $x$ do not lie
%in the $V$-cell at $v_0$. 
%%If $x$ does not lie in the open cone over
%%a simplex in $\CalQ$, but lies in the open cone over a standard
%region $R$, then Lemma~\ref{lemma:V-cell-local} describes the
%$V$-cell.  
%It states in particular, that for unobstructed $x$, it
%can be determined whether $x$ belongs to the $V$-cell at the
%point $v_0$ by considering only the vertices $w$ that lie in the closed
%cone over $R$ (the standard region containing the radial
%projection of $x$). In this sense, the intersection of a $V$-cell
%with the open cone over $R$ is {\it local\/} to the cone over $R$.
%\end{remark}
%


Let $\CalB_0'$ be the set of triangles $T$ such that at least one
of the following holds:
\begin{itemize}
    \item $T$ is a barrier at $v_0$, or
    \item $T=\{v_0,v,w\}$ consists of a diagonal of a quarter in the
    $Q$-system together with one of its anchors.
\end{itemize}

%DCG Lemma 5.29, page 50.
\begin{lemma} [Decoupling Lemma]\tlabel{lemma:V-cell-local}
%Let $x\in I_0$, the cube of side $4$ centered at $v_0$
%parallel to coordinate axes. 
Let $(\Lambda,v_0)$ be a centered packing.  Let $\{v_0,v_1,v_2,w\}$
be a set of four vertices of $\Lambda$. 
Assume that the closed segment
$\op{conv}\{x,w\}$ intersects $\op{aff}_+(v_0,\{v_1,v_2\})$, where
$F = \{v_0,v_1,v_2\}\in \CalB'_0$. Assume that $v_0$ 
is not obstructed at $x$. Assume that $x$ is closer to $v_0$ 
than to both $v_1$ and $v_2$. Then $x\not\in\op{VC}(\Lambda,w)$.
\end{lemma}
\index{decoupling lemma}

%\begin{remark}  The Decoupling Lemma is a crucial result.  It
%permits estimates of the scoring function in
%\Chap~\ref{sec:scoring} to be made separately for each standard
%region.  The estimates for separate standard regions are far
%easier to come by than estimates for the score of the full
%centered packing.  
%\end{remark}

\begin{proof}
Assume for a contradiction that $x$ lies in $\op{VC}(\Lambda,w)$. In
particular, we assume that $w$ is not obstructed at $x$.  Since
$v_0$ is not obstructed at $x$, $w$ must be closer to $x$
than $x$ is to $v_0$.

By tarski\tarf{tarski:decouple}, 
   $|w-v_0|,|w-v_1|,|w-v_2|\le 2t_0$.  Thus, $Q=\{v_0,w,v_1,v_2\}$ is
a quarter or a quasi-regular tetrahedron.  By the definition of
$\CalB'_0$, the face $F$ must lie in the $Q$-system.  Thus,
$F$ is a barrier.  Again, by tarski\tarf{tarski:decouple},
$\op{conv}(F)$ meets the segment from $x$ to $w$, so $x$ is obstructed
at $w$.  Thus, $x$ does not lie in $\op{VC}(\Lambda,w)$.
\end{proof}

%\section{Local Optimality}%DCG Sec. 8, p72  
%\tlabel{sec:local-opt}
%Moved after the main estimate.




\section{Fitted Crown}%DCG 9.1, p95
    \label{sec:fine-overview}
    \oldlabel{2.1}


%
%In this \chap, we define a decomposition of a $V$-cell. Let
%$\op{VC}$ be the $V$-cell at $v_0$.  For any $t > 0$, let
%$V(t)$ be the intersection of $\op{VC}$ with the ball $B(v_0,t)$ at
%$v_0$ of radius $t$. We write $\op{VC}$ as the disjoint union
%of $V(t_0)$ and its complement $\delta$.
%
%Assume that there is an upright 
%diagonal $\{v_0,v\}$. 
%We will define $\delta_i(v)\subset\delta$, where $i$ runs
%over some finite indexing set.
%The sets $\delta_i(v)$ will be defined so as not to overlap
%one another. 
%
%We will define a set $\CalS$ of simplices, each having a vertex at
%$v_0$. (The letter `$\CalS$' is for simplex.)
%The vertices of the simplices will be vertices of the
%packing, and their edges will have length at most $2\sqrt{2}$. The
%sets $\op{cone}^0(v_0,S)$, for distinct $S\in\CalS$, will not overlap. Over a
%simplex $S\in\CalS$, the $V$-cell will be truncated at a radius
%$t_S\ge t_0$. After defining the constants $t_S$, we will set
%    $$V_S(t_S) = \op{cone}^0(v_0,S) \cap V(t_S) =\op{cone}^0(S)\cap B(t_S)\cap \op{VC}(v_0).$$
%That is, $V_S(t_S)$ is the part of the $V$-cell at $v_0$,
%contained in the cone over $S$ and in the ball of radius $t_S$.
%If
%    $\op{VC}(v_0) \cap \op{cone}^0(S)\subset B(t_S)\subset B(t'_S)$,
%then
%    $V_S(t_S)=V_S(t'_S)$.
%
%Since $t_S\ge t_0$, the sets $V_S(t_S)$ and $\delta$ may meet at
%interior points. Nevertheless, we will show that $V_S(t_S)$ does
%not meet the interior of any $\delta(v)$.  Let $\tildeV(t_0)$ be
%the set of points in $V(t_0)$ that do not lie in $\op{cone}^0(S)$,
%$S\in\CalS$. We will derive an explicit formula for the volume of
%$\tildeV(t_0)$.
%
%In $\op{VC}(v_0)$, there are nonoverlapping sets
%$$\delta(v),\quad   V_S(t_S),\quad \tildeV(t_0).$$
%Let $\delta'$ be the complement in $\op{VC}(v_0)$ of the union of
%these sets. These sets give a decomposition of $\op{VC}(v_0)$.
%Corresponding to this decomposition is a formula for $\sigma(\Lambda,v)$
%of the form
%    $$
%    \sigma(\Lambda,v) =
%        \op{sovo}(\tildeV(t_0),\lambda_{oct})
%        + \sum_{\CalS}\op{sovo}(V_S(t_S),\lambda_{oct})
%        -\sum_{v,i} 4\doct\op{vol}(\delta_i(v))
%        -4\doct\op{vol}(\delta').
%    $$
%Since $\op{vol}(\delta')\ge0$, we obtain an upper bound on
%$\sigma(\Lambda,v)$ by dropping the rightmost term.



\subsection{crown tuple}%DCG 9.2, p86
    \label{sec:deltaP}
    \oldlabel{2.3}

Section~\ref{sec:anc} defines a region called fitted crowns.  We study
fitted crowns further in the context of $V$-cells and the $Q$-system
of a sphere packing.  Fitted crowns $FCR$ are defined at the same time
as various auxiliary sets $FC$, $\FCinner$, $\rogFC$ in
Definition~\ref{def:fitted-crown}.  The reader is advised to review those
definitions before continuing.

\begin{definition}\label{def:eta0}
Set $\eta_{V0}(v,w) = \eta(|v-w|,2,2t_0)$. 
\index{zzeta@$\eta_{V0}$}
\end{definition}

%By tarski\tarf{tarski:1453}, 
%if $h\le\sqrt2$, then $\eta(2h,2,2t_0)\le \eta(2,2t_0,\sqrt2) <
%1.453$.\index{ZZZZ1.453@1.453}



\begin{definition}\label{def:crown-tuple}
Let $\Lambda$ be a packing.
%Let $D_0 = \op{rcone}^0(v_0,v,|v-v_0|/(2\eta_{V0}(v,v_0)))$.
%Let $v_1,\ldots,v_k$ be the anchors around $\{v_0,v\}$ indexed
%cyclically. The half planes $A_i=\op{aff}_+(\{v_0,v\},v_i)$
%slice $\ring{R}^3$ into $k$ open wedges
%$W_i$, between
%    $A_i$ and $A_j$,
%where $j\equiv i+1\mod k$, so that
%    $\ring{R}^3\setminus (A_1\cup\cdots\cup A_k) =\cup W_i$.
We say that a four-tuple $(v_0,v,u,w)$ of vertices in $\Lambda$   
is a crown tuple if
\begin{itemize}
  \item $2t_0 < |v-v_0| <\sqrt8$.
  \item $u$ and $w$ are anchors of $\{v_0,v\}$.
  \item $w$ is the successor of $u$ in the azimuth cycle with respect to
   $(v_0,v)$ on the
   anchors of $\{v_0,v\}$.
\begin{enumerate}  
    \item Either $\op{azim}(v_0,v,u,w)\ge\pi$, or
    \item $\op{azim}(v_0,v,u,w) <\pi$, 
 $|u-w|\ge 2.77$,
    $\rad_V(v_0,v,u,w)\ge\eta_{V0}(v,v_0)$, and the max of 
    $\eta_V(v_0,u,w)$ and $\eta_V(v,u,w)$ is
    $\ge\sqrt2$.
    \label{enum:wedge2}
\end{enumerate}
\end{itemize}
\index{crown tuple}
\end{definition}

The definition of fitted crown relies upon Assumption~\ref{eqn:q1q2}.

\begin{lemma}\tlabel{lemma:fitted-q1q2}  
Let $\Lambda$ be a sphere packing with crown tuple
$(v_0,v,w_1,w_2)$.  Let $c_1=c_2=\eta_{V0}(v_0,v)$. 
Then Assumption~\ref{eqn:q1q2} holds for $(v_0,v,w_1,w_2,c_1,c_2)$.
\end{lemma}

\begin{proof} Assume not.  
We use the notation of Section~\ref{sec:anc}.
We can decrease $\op{azim}(v_0,v,w_1,w_2)$
until $\op{azim}<\pi$.  Let $S=\{v_0,v,w_1,w_2\}$.
$2.77 \le |w_1-w_2|$ implies that the simplex $S$
has positive orientation at $w_1$ and $w_2$ (Lemma~\ref{XX}). 
The circumcenter $p$ can be written, 
   $$p + p_1 + s_1 u_1 = p_2 + s_2 u_2,\quad s_1,s_2 > 0.$$
If Assumption~\ref{eqn:q1q2} fails,
   $$\op{azim}(v_0,v,w_1,q_2) < \op{azim}(v_0,v,w_1,q_1).$$
Then the circumradius $|p-v_0|$ is
   $$\rad_V(v_0,v,w_1,w_2) < \eta_{V0}(v_0,v).$$
This contradicts the definition of crown tuple.
\end{proof}



%Fix $i,j$, with $j\equiv i+1\mod k$. If $W = W_i$ is a crown tuple, 
%let $\{v_0,v_i,v\}^\perp$ be the plane through $v_0$ and
%the circumcenter of $\{v_0,v_i,v\}$, perpendicular to $\{v_0,v_i,v\}$.
%Skip the following step if the circumradius of $\{v_0,v_i,v\}$ is
%greater than $\eta_{V0}(v,v_0)$, but if the circumradius is at most
%this bound, the plane
%of $\{v_0,v_i,v\}^\perp$
%intersects the right circular cone boundary of $D_0$ along two rays
%emanating from $v_0$.  Let $c_i$ be a point on the ray (selected on
%the $W$-side of $\{v_0,v_i,v\}$).  Simlarly, we construct the point
%$c_i'$ for $\{v_0,v,v_j\}$ (again on the $W$-side).
%
%Define $\theta=\theta(v)$ by $\cos\theta = |v-v_0|/(2\eta_{V0}(v,v_0))$.
%If Condition~2 holds, we let $c$ be the 
%circumcenter of $\{v_0,v_i,v_j,v\}$.  The angle
%at $v_0$ between $c$ and $v$ is
%$\theta'$, where
%$$\cos\theta' = |v-v_0|/(2\rad)\le |v-v_0|/(2\eta_{V0}) = \cos\theta.$$
%We conclude that $\theta'\ge\theta$ and $c$ does not lie in $D_0$.
%Thus, the half-planes
%   $$
%   A_i,\quad B_i=\op{aff}_+(\{v_0,v\},c_i),\quad 
%   B'_i = \op{aff}_+(\{v_0,v\},c'_i), \quad
%   A_j
%   $$
%are ordered cyclically around $\{v_0,v\}$. (Set $B_i=A_i$
%or $B'_i=A_j$, if the corresponding circumradius is greater than
%$\eta_{V0}(v,v_0)$.)
%Let $W'=W'_i$ be the open wedge of $D_0$ between $B_i$ and $B'_i$.
%Let
%    $$E_w = \{x : 2 x\cdot w \le w\cdot w\},$$
%for $w = v,v_i,v_j$. These are half-spaces bounding the Voronoi
%cell. Set $E_\ell = E_{v_\ell}$.
%

%
%\begin{definition} \label{def:delta-e}
%In both cases (Conditions~1 and~2), let $W$ be the wedge between
%$v_i$ and $v_j$ along $(v_0,v)$, $W'$ the smaller wedge, 
%and let $c=\eta_{V0}(v,v_0)$ in
%    $$
%    \begin{array}{lll}
%      \FC'(v,W) &= [E_v\cap W'\cap D_0]\cup \op{rog}^0(v_0,v,v_i,p_i,c)
%      \cup \op{rog}^0(v_0,v,v_j,p_j,c)
%      \\
%    \FC(v,W) &= \FC'(v,W) \cup 
%    \op{rog}^0(v_0,v_i,v,p_i,c)
%    \cup \op{rog}^0(v_0,v_j,v,p_j,c)\\
%    \bigd(v,W) &= \{x\in\FC(v,W)\mid |x-v_0|>t_0\},
%    \end{array}
%    $$
%where $p_i,p_j$ are selected so that the simplices $\op{rog}^0$ lie
%in the wedge $W$.
%\index{zzdelta@$\bigd(v,W)$}
%\index{zzDelta@$\FC(v,W)$}
%\end{definition}
%

%\begin{remark} We note that the union is actually a disjoint union,
%and that each of the pieces is one of the primitive regions, so
%the volume of $\FC$ is immediate.
%\end{remark}

\begin{figure}[htb]
  \centering
  \myincludegraphics{\ps/diag46.ps}
  \caption{XX Move to volume.tex
    $\FC$ lies in a cone. The intersection of
    that cone with the unit sphere is the shaded region.}
  \label{fig:anchor-quarter}
\end{figure}

\begin{remark}
Recall that a Rogers simplex (Definition~\ref{def:rog}) is defined to be the
empty set 
if the corresponding parameters are not coherent.  It is to
be understood that the all discussion regarding this set
can (and should) be disregarded in the case that it is empty.
\end{remark}



\subsection{normal tuple}


The following definition will be used briefly, then discarded.
Its purpose is to group together a few cases in the next
few lemmas.

\begin{definition}  Let $(v_0,v,u_1,u_2)$ be a tuple of vertices
in a packing $\Lambda\subset\ring{R}^3$.  We say that it is {\it normal} if $2t_0<|v-v_0|<\sqrt8$
and one of the
following holds:
\begin{enumerate}
  \item (qrt) $\{v_0,u_1,u_2\}$ is a quasi-regular triangle.
  \item (upright) $\{v_0,u_1,u_2\}$ is an upright triangle; that is, say,
    $2t_0 < |u_1-v_0| < \sqrt8$, and $u_2$ is an anchor of $\{v_0,u_1\}$.
    (or vice-versa, swapping $u_1$ with $u_2$).
  \item (flat)
   $2t_0<|u_1-u_2|<\sqrt8$; and if $u_1,u_2$ are both anchors of
   $\{v_0,v\}$, 
   then
    no further anchor of $\{v_0,v\}$
   lies in the lune $\op{aff}^0_+(\{v_0,v\},\{u_1,u_2\})$.
\end{enumerate}
\end{definition}



In the following lemmas, we adopt a uniform notation.  Let $(\Lambda,v_0)$ be a centered packing.
$\{v_0,v\}$ is an upright diagonal: $|v-v_0|<\sqrt8$.
$(v_0,v,w,w')$ is  a crown tuple.  
Set $R_w=\rogFC(v_0,w,v,w')$.
%R_w=\op{rog}^0(v_0,w,v,p,\eta_{V0}(v,v_0))$,
%where $p$ is selected so that $R_w$ lies in $W(v_0,v,w,w')$.

\begin{lemma}\tlabel{lemma:meet-normal}
Let $(\Lambda,v_0)$ be a centered packing.  Let $\{v_0,u_1,u_2\}$
be a barrier.  Let $(v_0,v,w_1,w_2)$ be a normal tuple.  If
$\FC(v_0,v,w_1,w_2)$ meets $F=\op{aff}_+(v_0,\{u_1,u_2\})$, then
$(v_0,v,u_1,u_2)$ is normal.
\end{lemma}

\begin{proof} We have $\FC\subset \op{aff}_+^0(\{v_0,v\},\{u_1,u_2\}$
and $F\subset \op{aff}_+(\{v_0,v\},\{u_1,u_2\})$.
Continue. XX
\end{proof}

\begin{lemma}\tlabel{lemma:new-anchor}
Let $(\Lambda,v_0)$ be a centered packing
  Let $S=\{v_0,v,w,u\}$ be a simplex in $\Lambda$.  Assume that $\{v_0,v\}$ is an
upright diagonal, that $w$ and $u$
are anchors of $\{v_0,v\}$, and that $\rad_V(S)< \eta_{V0}(v,v_0)$.
Assume there is a crown tuple $(v_0,v,w,w')$
extending the triple $(v_0,v,w)$
(on the same side of the face as $u$).  
Then the anchor $w'$ of $\{v_0,v\}$ lies between $u$ and $w$
(that is,  $w'\in\op{aff}_+^0(\{v_0,v\},\{u,w\})$) with
    $|w'-w|\ge2.77$ and
    $\rad\{v_0,w,w',v\} \ge \eta_{V0}(v,v_0)$.
\end{lemma}

\begin{proof} The conditions on $(v_0,v,w,u)$ are incompatible with the
conditions of Definition~\ref{def:crown-tuple} defining crown tuples.
Therefore, $w$ and $u$ cannot be consecutive anchors around
$\{v_0,v\}$.  Let $w'$ be the anchor such that $(v_0,v,w,w')$
is a crown tuple.  Since $w$ and $w'$ are consecutive anchors,
we see that $w'$ must be between $u$ and $w$.  It must have the
second type in Definition~\ref{def:crown-tuple}.  The conclusion follows.
\end{proof}



\begin{lemma}\tlabel{lemma:FC-}  Let $S=(v_0,v,u_1,u_2)$ be normal and
let $S'=(v_0,v,w_1,w_2)$ be a crown tuple in a packing $\Lambda$.
Let $a = |v-v_0|/(2\eta_{V0}(v_0,v))$.  
Then
$$
 \op{rcone}^0(v_0,v,|v-v_0|/(2\eta_{V0}(v_0,v))) \cap
  \op{aff}^0_+(\{v_0,v\},\{w_1,w_2\})
$$
does not meet $F=\op{cone}(v_0,\{u_1,u_2\})$.
In particular, the subset
$\FCinner(S')$ does not meet $F$.
\end{lemma}

\begin{proof}  Assume for a contradiction that the sets meet.
Tarski\tarf{tarski:eps-bigd-}
implies that $S$ is normal of the flat or qrt variety, and 
that $u_1$ and $u_2$ are anchors of $\{v_0,v\}$.  
The result also gives that 
$\rad_V(S)<\eta_{V0}(v,v_0)$.    
In the qrt case, $S$ forms
an upright quarter.  By tarski\tarf{tarski:consec-anchors}, $u_1$ and $u_2$
are consecutive anchors. In the flat case as well, 
$u_1$ and $u_2$ are consective
anchors around $\{v_0,v\}$.

The set $\{u_1,u_2\}$ is not $\{w_1,w_2\}$ because $(v_0,v,u_1,u_2)$
does not satisfy the properties of a crown tuple (assuming $u_1,u_2$
indexed so that $\op{azim}(v_0,v,u_1,u_2) = \dih_V(\{v_0,v\},\{u_1,u_2\})$).  The anchors $u_1,u_2$ are consecutive, as are $w_1,w_2$.
The set $\FCinner(S')$ lies in the lune $\op{aff}_+^0(\{v_0,v\},\{w_1,w_2\})$ and $F$ lies in the lune $\op{aff}_+(\{v_0,v\},\{u_1,u_2\})$.
These lunes are disjoint.
%
%If the crown tuple $S'$ is not in $W(v_0,v,u_1,u_2)$, 
%then $\FC$ lies outside the 
%lune $\op{aff}_+^0(\{v_0,v_0\},\{u_1,u_2\})$, but $F$ lies in it.  Thus,
%the crown tuple must run between $u_1$ and $u_2$.   This contradicts 
%the rules for forming crown tuples.  (Lemma~\ref{lemma:new-anchor} 
%implies that
%anchors $u_1$ and $u_2$ cannot be consecutive.)
\end{proof}

\begin{lemma}\tlabel{lemma:fine-Rw}
Let $S=(v_0,v,w,u_1)$ be normal and $(v_0,v,w,w')$ a crown tuple
of a packing $\Lambda$.  
Then
$\rogFC(v_0,w,v,w')$
does not meet $F=\op{cone}(v_0,\{w,u_1\})$.
\end{lemma}

\begin{proof}  We can use the same proof as in Lemma~\ref{lemma:FC-},
except with one tarski\tarf{tarski:eps:fine:Rw} calculation 
substituted for 
another\tarf{tarski:eps-bigd-}.
\end{proof}

%% This has been replaced with tarski:fine:Rw:u.
%% In fact the proofs are almost the same.
%% It can be deleted.

%\begin{lemma}\tlabel{lemma:fine:barrier}  
%Let $S=\{v_0,v,w,u\}$ be a set of four distinct vertices
%in a packing $\Lambda$.  
%Assume that $\{v_0,v\}$ is an upright diagonal of
%a quarter in the $Q$-system and that $w$ is an anchor of $\{v_0,v\}$.
%Let $R_w$ be as above; that is, a Rogers simplex attached to a crown tuple 
%extending $(v_0,v)$.
%Assume $x\in R_w$ satisfies $\epsilon_0(x,\{v,w,u\})= u$.
%Then there exists a vertex $w'$ that is an anchor of $\{v_0,v\}$ such
%$b=\{v,w',v_0\}$ is a barrier and $x$ is obstructed from $u$
%by $b$.
%\end{lemma}
%
%\begin{proof} Assume for a contradiction that $\epsilon_0(x)=u$.
%We separate the proof into two cases, depending on whether
%$x$ and $u$ lie on the same side of $A=\op{aff}(v,w,v_0)$.
%
%Assume that $x$ and $u$ lie on opposite sides of $A$.
%For any nonzero vertices $v',v''$,
%Let $L(v',v'')$ be the line of points equidistant from $\{v_0,v',v''\}$.
%The three lines $L(u,v)$, $L(v,w)$, $L(u,w)$ meet at the circumcenter
%$c$ of $S$.  The rays $L^+(u,v)$, $L^+(v,w)$, $L^+(u,w)$ demarcate
%the regions between $\epsilon_0=w,u,v$.  (Pick the direction
%of ray so that it runs through the circumcenter of the face $\{v_0,v',v''\}$
%if the remaining vertex has positive orientation, and so that it
%runs in the opposite direction otherwise.)
%
%If $u$ has positive orientation in $S$, then $L^+(v,w)$ runs through
%the circumcenter of $\{v_0,v,w\}$ and along an edge of $R_w$.
%The point $w/2$ in the closure of  $R_w$ also has $\epsilon_0=w$.
%It follows that $\epsilon_0$ has value $w$ on $R_w$, which is contrary
%to our assumption.
%
%Thus, $u$ has negative orientation in $S$.  
%This implies that $|u-v|,|u-w|,|u-v_0|\le 2.51$.  In particular,
%$S$ is a quarter.  Since it has the same diagonal as a quarter in
%the $Q$-system, we have that $S$ is in the $Q$-system, so that
%$\{v_0,v,w\}$ is a barrier.  By tarski\tarf{tarski:tip-cone}, we have
%that $x\in \op{cone}(u,\{v,w,v_0\})$.  In particular, $x$ is obstructed
%from $u$ by the barrier $\{v,w,v_0\}$.  Take $w'=w$ in this case.
%
%Now assume that $x$ and $u$ lie on the same side of $A$.
%First consider the special case where
%we also have that $\rad_V(S)\ge \eta_{V0}(v,v_0)$ and that that
%the orientation of $u$ is non-positive in $S$.  In this case, it
%follows that $S$ is a quarter in the $Q$-system.  By the rule
%for constructing crown tuples, there is no $R_w$ along $\{v_0,v,w\}$
%in this case.  Next consider the case where
%$\rad_V(S)\ge \eta_{V0}(v,v_0)$ and the orientation of $u$ is positive
%in $S$.  In this case, the ray $L^+(v,w)$ runs along the edge of 
%$R_w$ as before, and we see that $\epsilon_0=w$ on $R_w$.
%
%Finally consider the case where $\rad_V(S)<\eta_{V0}(v,v_0)$.
%It follows that $u$ is an anchor of $\{v_0,v\}$.
%By Lemma~\ref{lemma:new-anchor}, there exists a further anchor
%$w'$ between $u$ and $w$.  By tarski\tarf{tarski:prev},
%$\op{conv}\{x,u\}$ meets $\op{conv}\{v_0,v,w'\}$, and
% $|u-w'|\le 2.51$.  In particular $S'=\{v_0,v,w',u\}$ is
%an upright quarter and $\{v_0,v,w'\}$ is a barrier.
%\end{proof}



\begin{lemma}\tlabel{lemma:fine-Rw:5}
Let $\{v_0,v,w,u_1,u_2\}$ be a set of five distinct vertices in a
centered packing $(\Lambda,v_0)$.
packing.  Assume that $\{v_0,v\}$ is an upright diagonal of a quarter
in the $Q$-system and that $w$ is an anchor of $\{v_0,v\}$.
Assume that $S=(v_0,v,u_1,u_2)$ be normal.
Let $R_w$ be as above; that is, a Rogers simplex attached to a crown tuple $(v_0,v,\ldots)$.
Then $R_w$ does not meet $F=\op{cone}(v_0,\{u_1,u_2\})$.
\end{lemma}

\begin{proof}
For a contradiction, assume these sets meet at $x\in R_w\cap F$.
%We have $\epsilon_0=\epsilon_0(x,\{w,u_1,u_2\})\in\{w,u_1,u_2\}$.  We consider
%two cases depending on whether $\epsilon_0=w$.
Tarski\tarf{tarski:fine-Rw-split} enumerates six different possible
configurations of points $\{v_0,v_1,w,u_1,u_2\}$.  We discuss each
in turn.

In the first three cases, 
%
%Assume that $\epsilon_0=w$.  By tarski\tarf{tarski:fine:Rw:5},
we have $|w-u_1|\le 2.51$ and $|w-u_2|\le 2.51$.  The first is that
for either $u=u_1$ or $u=u_2$, we have that $(v_0,v,w,u)$ is normal
and that $R_w$ meets $\op{cone}(v_0,\{w,u\})$.  This is contrary to
Lemma~\ref{lemma:fine-Rw:5}.

The second possibility is that
$2.51<|u_1-u_2|<\sqrt8$ and that $v\in\op{cone}^0(v_0,\{u_1,u_2,w\})$
with $u_1,u_2,w$ all anchors of $\{v_0,v\}$.  By the normality
hypothesis, these are the only anchors of $\{v_0,v\}$.  None of the
corresponding tuples are crown tuples extending
$(v_0,v)$.  Thus, this case does not
occur. 

The third possibility is that $w\in \op{aff}_+(\{v_0,v\},\{u_1,u_2\})$
and $2.51<|u_1-u_2|$.  This is contrary to the normality
condition on $S$.

%In the remaining case, we have $\epsilon_0=u\in\{u_1,u_2\}$.  
%Let $S'=\{v_0,v,w,u\}$.  
In the remaining cases, there is a 
relabeling of vertices $\{u,u'\}=\{u_1,u_2\}$.
In the fourth case, we have that $\{v_0,v_1,u,w\}$ is an upright
quarter and $u'\in \op{aff}_+^0(\{v_0,u\},\{v_1,w\})$.  Inspecting
the various possibilities for $u'$, we find that
$\op{conv}\{u,u'\}$ meets $\{v_0,v_1,w\}$.  This forces $u'$ to be
an anchor of $\{v_0,v_1\}$ and $2.51 < |u-u'|$.  By normality
$u,u'$ are consecutive anchors.  But this if false, because the
anchor  $w$ is between them.
%By Lemma~\ref{lemma:fine:barrier}, there
%exist a barrier $b=\{v_0,v,w'\}$ such that $x$ is obstructed from
%$u$ by $b$.  However, if $x\in\op{cone}(v_0,\{u_1,u_2\})$, then
%there exists no such obstruction.  Thus, the intersection
%is empty.

In the fifth case, $\{v_0,v_1,u,w\}$ is an upright quarter and
$R_w\subset \op{aff}_+(\{v_0,v_1,w\},u)$.  By Lemma~\ref{lemma:new-anchor},
there is an anchor $w'$ of $\{v_0,v_1\}$ between $u$ and $w$.
This is not possible for geometric reasons. XX detail.

In the sixth case, $u_1$ and $u_2$ are anchors of $\{v_0,v_1\}$.
They are not consecutive and $2.51 < |u_1-u_2|$.  This contradicts
the definition of normal.
We have examined all six cases.  In each case, the assumption
that $F$ and $R$ meet leads to a contradiction.  The result follows.
\end{proof}

\subsection{fitted crown and fan}

\begin{lemma}\tlabel{lemma:delta-tri}
%\tlabel{lemma:delta-upright}
%\tlabel{lemma:delta-flat}
Let $(\Lambda,v_0)$ be a centered packing.
Let $\{v_0,v\}$ be
an upright diagonal of a quarter in the $Q$-system.
Assume that $(v_0,v,u_1,u_2)$ is normal.
Let $E$ be the set of all $\{u,v\}$ such that $\{v_0,u,v\}$
is a barrier.  Set
   $$
   V = \{v \mid \exists u.\quad \{u,v\}\in E\}.
   $$
Then $(v_0,V,E)$ is a fan.  Moreover, for each crown tuple
$(v_0,v,w_1,w_2)$, there is a component $U\in Y(v_0,V,E)$ such
that $\FC(v_0,v,w_1,w_2)\subset U$.
%Let $F=\{v_0,u_1,u_2\}$ be a quasi-regular triangle.   
%Assume that there
%exists a quarter in the $Q$-system along $\{v_0,v\}$.  
%Let $(v_0,v,w_1,w_2)$ be a crown tuple.
%Then
%$\FC(v_0,v,w_1,w_2)$ does not meet $A=\op{aff}_+(v_0,\{u_1,u_2\})$.
\end{lemma}

\begin{proof}
We prove that $(v_0,V,E)$ is a fan in a section devoted to fans XX.

The set $\FC$ is contained in the union of the open sets
$\rogFC(v_0,w_1,v,w_2)$, $\rogFC(v_0,w_2,v,w_1)$, and
  $$
  C=\op{rcone}^0(v_0,v,b)\cap \op{aff}_+(\{v_0,v\},\{w_1,w_2\}).
  $$
Each of these three sets is connected.  Moroever, $C$ meets
the two sets of the form $\rogFC$.  Therefore the union $B$ of
these three sets is connected and open.  

We show that $B$ is contained in $Y(v_0,V,E)$.  That is,
it is disjoint from $X(v_0,V,E)$.
The set $X(v_0,V,E)$ is a union of sets $F=\op{aff}_+(v_0,\{u_1,u_2\})$
for barriers $\{v_0,u_1,u_2\}$.  By Lemma~\ref{lemma:meet-normal},
the tuple $(v_0,v,u_1,u_2)$ is normal.
By Lemma~\ref{lemma:FC-}, the set $C$ does not meet any sets $F$
of this form.  By Lemma~\ref{lemma:fine-Rw} and Lemma~\ref{lemma:fine-Rw:5}, the sets $\rogFC$ do not meet any sets $F$ of this form.

Thus, $B$ is contained in a single connected component $U$ of $Y(v_0,V,E)$.  As $B$ contains $\FC$, the result follows.
\end{proof}


\subsection{containment in V-cells}

%
%\begin{lemma}\tlabel{lemma:delta-flat}
%Let $(\Lambda,v_0)$ be a centered packing.
%Let $F=\{v_0,u_1,u_2\}$ be a triangle.  
%Assume that $|u_1-v_0|\le 2t_0$,
%$|u_2-v_0|\le 2t_0$, and $2t_0\le|u_1-u_2|\le\sqrt8$.  Let $\{v_0,v\}$
%be the diagonal of an upright quarter in the $Q$-system.  Assume
%that if $u_1$ and $u_2$ are both anchors of $v$, then they are
%consecutive anchors around $v$. Under these conditions, the set
%$\FC(v,W)$ does not overlap the cone at $v_0$ over the triangle
%$F$.
%\end{lemma}
%
%\begin{proof} The proof is identical to that of
%Lemma~\ref{lemma:delta-tri}. 
%\end{proof}
%
%\begin{lemma}\tlabel{lemma:delta-upright}
%Let $F=\{v_0,u_1,u_2\}$ be a triangle.  Assume that $2t_0\le|u_1-v_0|\le
%\sqrt8$, $2\le|u_2-v_0|\le 2t_0$, and $2\le|u_1-u_2|\le2t_0$.  Let
%$\{v_0,v\}$ be the diagonal of an upright quarter in the $Q$-system.
%Under these conditions, the set $\FC(v,W)$ does not overlap
%the cone at $v_0$ over the triangle $F$.
%\end{lemma}
%
%\begin{proof}
%The proof is identical to that of Lemma~\ref{lemma:delta-tri}.
%\end{proof}
%

\begin{lemma}\tlabel{lemma:FC-unobstr}
Let $(\Lambda,v_0)$ be a centered packing.  Let $(v_0,v,w_1,w_2)$
be a crown tuple.
Assume $\{v_0,v\}$ is an upright diagonal of a quarter in the
$Q$-system.   If $x$ lies in %  the interior of  WW: FC is open
$\FC(v_0,v,w_1,w_2)$,
then $x$ is unobstructed at $v_0$.
\end{lemma}

\begin{proof} For a contradiction, assume that $x$ is obstructed
at $v_0$ by barrier $T =\{u_1,u_2,u_3\}$.
If $v_0\in T$, we have disjointness by Lemma~\ref{XX}. Assume $v_0\not\in T$.

The convex hull of $T$ can be partitioned into three sets $T(i)$
depending on which vertex of $T$ is closest to a given point in
the convex hull. (Ties can be resolved in any consistent manner.)
Let $y\in \FC$ be the point in the convex hull of $T$ on
the segment from $v_0$ to $x$.  Fix $i$ so that $y\in T(i)$. If
$v=u_i$, then each point $y$ of $T(i)$ is closer to $v$ than to
$v_0$.  But each point of $\FC$ is closer to $v_0$ than to
$v$.  So $x$ is not obstructed by $T$ at $v_0$.

We may now assume that $v\ne u_i$.

Partition $\ring{R}^3$ geometrically into three sets $V(u_i)$,
$V(v_0)$, $V(v)$ according to which of $\{u_i,v_0,v\}$ a point
$z\in\ring{R}^3$ is closest to.  (Again resolve ties in any
consistent manner.)

Assume further that $\max_j u_j \ge 2t_0$. This implies that $y\in
T(i) \subset V(v) \cup V(u_i)$.  On the other hand, we have by
construction that $y\in \FC \subset V(v_0)$.  (There are
two cases involved in this conclusion, depending on whether $u_i$
is an anchor of $\{v_0,v\}$.)  However, the sets $V(\cdot)$ are
disjoint; and we reach a contradiction.  Thus, under these
assumptions, $x$ is unobstructed at $v_0$.

Next assume that $\max_j u_j < 2t_0$.  Let $S=\{v_0,u_1,u_2,u_3\}$.
Since $T$ is a barrier, $S\in\CalQ(v_0)$.  By assumption, $\{v_0,v\}$
is a diagonal of an upright quarter in $\CalQ(v_0)$.  By the fact
that the interiors of quarters in $\CalQ(v_0)$ do not meet, we see
that $v$ is not enclosed over $S$.  The set $\FC$ has
a star convexity with respect to the ray from $v_0$ through $v$.
Thus, if $\FC$ intersects the convex hull of
$T$ at $y$, then $\FC$ intersects the cone over a face
$\{v_0,u_1,u_2\}$ of $S$ at $y'$. (For simplicity, take $v_0=0$.)
We can take $y'/|y'|$ to lie on
the cone generated by the arc running from $v/|v|$ to $y/|y|$.
This is impossible by Lemma~\ref{lemma:delta-tri}. 
% and \ref{lemma:delta-flat}.
\end{proof}

\begin{lemma}  Let $(\Lambda,v_0)$ be a centered packing.
Let $(v_0,v,w_1,w_2)$ be a crown tuple of $\Lambda$. Assume that
 $\{v_0,v\}$ is the upright diagonal of a quarter
in the $Q$-system.  Then there is a null set $Z$ such
that $\FC(v_0,v,w_1,w_2)$ is
a subset of $\op{VC}(\Lambda,v_0)\cup Z$.
\end{lemma}

\begin{proof}
We recall that there is a truncation at distance $2$ in the definition
of $V$-cell so that $\op{VC}(\Lambda,v_0)\subset B(v_0,2)$.
The extreme point of $\FC(v_0,v,w_1,w_2)$ has distance from the origin of
 $$\eta_{V0}(v_0,v) = \eta(|v_0-v|,2,2.51)\le \eta(\sqrt8,2,2.51) < 2.$$
Thus, the truncation condition is satisfied.

We show that $\FCinner(v_0,v,w_1,w_2)\subset\op{VC}(\Lambda,v_0)$.
Suppose to the contrary, that a point 
  $$x\in \FCinner\setminus \op{VC}(\Lambda,v_0)$$
%$ in %the interior of
%$\FCinner$ lies in $\op{VC}(\Lambda,w)$, with $w\ne0$.  
Then $x$ is at least as close
to some $w\in\Lambda$ as to  $v_0$.  
Then, $\eta_V(v_0,v,w)\le\eta_{V0}(v,v_0)$, and $w$
is an anchor of $\{v_0,v\}$.  The construction of
$\FCinner$ prevents this from happening.

Let $w=w_1$ or $w_2$, and
consider a point $x$ of $\rogFC(v_0,w,v,\cdot)$.
By avoiding a null set $Z$ we may assume that $x$ lies in
$\op{VC}(\Lambda,u)$, with $u\ne v_0$.  
By dismissing a trivial case, we may
assume that $w\ne u$.

Assume that the orientation of $S=\{v_0,v,w,u\}$ is negative at $u$.  
Then $S$ must be an upright quarter.  By
the construction of crown tuples, we have that $\rogFC(v_0,w,v,\cdot)$ must
lie on the opposite side of the plane $\{v_0,v,w\}$ from $u$ (for
there is no crown tuple between the anchors of an upright quarter).  The
result now follows from tarski\tarf{tarski:back}.

If $\rad_V(S) <\eta_{V0}(v,v_0)$, then $u$ and $w$ are anchors.  In
this case, the result follows from tarski\tarf{tarski:prev}.

Finally if the orientation is positive and if $\rad_V(S)\ge
\eta(|v-v_0|/2)$, then a point of $\rogFC(v_0,w,v,\cdot)$ cannot be closer to $u$ than
to $v_0$.
\end{proof}


\subsection{overlap}%DCG 9.3, p93
    \label{sec:overlap}
    \oldlabel{2.4}

[XX Eliminate the function $\epsilon_0$.  ]

\begin{lemma}\tlabel{lemma:FC-no-over}
Let $(\Lambda,v_0)$ be a centered packing.  Let $(v_0,u,\ldots)$
and $(v_0,v,\ldots)$ be distinct crown tuples.
Then $\FC(v_0,u,\ldots)$ does not meet $\FC(v_0,v,\ldots)$.
\end{lemma}

\begin{proof}
This is clear for  $u=v$.  Assume $u\ne v$.  

To treat the points in $\FCinner(v_0,u,\ldots)$ and $\FCinner(v_0,v,\ldots)$, we
may contract $\{u,v\}$ until $|u-v|=2$.  By the constraints on the
edges of $\{v_0,u,v\}$, the circumcenter $c$ of this triangle lies
in the convex hull of the triangle.  We have $\eta_V(v_0,u,v)\ge
\eta_{V0}(v,v_0)$ and $\eta_V(v_0,u,v)\ge\eta_{V0}(u,v_0)$.  So the plane
through $\{v_0,c\}$ perpendicular to the plane $\{v_0,u,v\}$ separates
$\FCinner(v_0,u,\ldots)$ from $\FCinner(v_0,v,\ldots)$.

Next we separate points in $\FCinner(v_0,u,\ldots)$ from points of
$\rogFC(v_0,w,v,\cdot)$, where $w$ is an anchor of $v$ and $u\ne v$.  Let
$S=\{v_0,u,v,w\}$. The orientation of $S$ along $\{v_0,v,w\}$ is
positive.  The circumradius of $S$ satisfies
    $$
    \rad_V(S) \ge \eta_V(v_0,u,v)>\eta_{V0}(v,v_0).
    $$
Thus, $\epsilon_0(S,\cdot)$ takes different values on
$\FCinner(v_0,u,\ldots)$ and $\rogFC(v_0,w,v,\cdot)$, so that the sets are disjoint.

Next we separate points of $\rogFC(v_0,w,v,\cdot)$ from 
$\rogFC(v_0,w,u,\cdot)$.  (Notice
that we assume that the anchor is the same for the two upright diagonals.)
Let $S=\{v_0,u,v,w\}$.   As above, we have
    $$
    \rad_V(S) \ge \eta_{V0}(v,v_0), \quad \eta_{V0}(w,v_0).
    $$
The simplex $S$ has positive orientation along the faces
$\{v_0,u,w\}$ and $\{v_0,v,w\}$.  Let $c_u$ be the circumcenter of
$\{v_0,u,w\}$, let $c_v$ be the circumcenter of $\{v_0,v,w\}$, and let
$c$ be the circumcenter of $S$.  Then $\rogFC(v_0,w,v,\cdot)$ lies in the
convex hull of $\{v_0,w,c_v,c\}$, but $\rogFC(v_0,w,u,\cdot)$ lies in the convex
hull of $\{v_0,w,c_u,c\}$.  Thus, the sets are disjoint.

Finally, we separate points of $\rogFC(v_0,w,u,\cdot)$ from points of
$\rogFC(v_0,w',v,\cdot)$,  where $w\ne w'$ and $u\ne v$.  If the function
$\epsilon_0(\{v_0,w,w'\},\cdot)$ separates the sets, we are done.
Otherwise, we may assume say that $\epsilon_0(\{v_0,w,w'\},x) = w'$
from some $x\in \rogFC(v_0,w,u,\cdot)$.  Let $S=\{v_0,u,w,w'\}$.

If $w'$ is not an anchor of $u$, then $\rad_V(S) \ge\eta_{V0}(u,v_0)$
and the orientation of $S$ along $\{v_0,w,u\}$ is positive.  In this
case, we have $\epsilon_0 = w$ on $\rogFC(v_0,w,u,\cdot)$, which is contrary
to assumption. Thus, we may assume that $w'$ is an anchor of $u$.

If the orientation of $\{v_0,u,w,w'\}$ is negative along $\{v_0,w,u\}$,
then $\{v_0,u,w,w'\}$ is a quarter, 
contrary to the existence of a crown tuple.  
So the orientation is positive.  If $\rad_V(\{v_0,u,w,w'\}) <
\eta_{V0}(u,v_0)$, then tarski\tarf{tarski:prev} implies that each point
of $\rogFC(v_0,w,u,\cdot)$ is obstructed from $w'$.  
But no point of $\rogFC(v_0,w',v,\cdot)$ is
obstructed from $w$. (In fact, a barrier that crosses
$\FC(v_0,v,\ldots)$ is inconsistent with Lemma~\ref{lemma:delta-tri}.)
%,\ref{lemma:delta-flat}, \ref{lemma:delta-upright}.) 
So
$\rad_V(\{v_0,u,w,w'\}) \ge \eta_{V0}(u,v_0)$.  This is contrary to
$\epsilon_0(\{v_0,w,w'\},x) = w'$ from some $x\in \rogFC(v_0,w,u,\cdot)$.
\end{proof}



\section{Alphabet Simplex}%DCG 9.4, p94
    \oldlabel{2.5}

We consider various types of simplices, formed by vertices in $\Lambda$.  Because of their names $\SA$, $\SB$, $\SC$, $\SD$, we call them
{\it alphabet simplices}.
  The edge lengths of
these simplices are less than $2\sqrt{2}$.

$\SA$.  This family consists of simplices $S(y_1,\ldots,y_6)$ whose
edge lengths satisfy
    $$
    y_1,y_2,y_3\in[2,2t_0],\quad
    y_4,y_5\in[2t_0,2.77],
    \quad
    y_6\in[2,2t_0],\quad \text{and }
    \eta(y_4,y_5,y_6)<\sqrt{2}.
    $$
(These conditions imply $y_4,y_5<2.697$, because
$\eta(2.697,2t_0,2)>\sqrt2$.)

$\SB$.  This family consists of certain flat quarters that are
part of an isolated pair of flat quarters. It consists of those
satisfying $y_2,y_3\le 2.23$, $y_4\in[2t_0,2\sqrt{2}]$.

$\SC$.  This family consists of certain simplices
$S(y_1,\ldots,y_6)$ with edge lengths satisfying
    $y_1,y_4\in[2t_0,2\sqrt{2}]$, $y_2,y_3,y_5,y_6\in[2,2t_0]$.
We impose the condition that the first edge is the diagonal of
some upright quarter in the $Q$-system, and that the upper
endpoints of the second and third edges (that is, the second and
third vertices of the simplex) are consecutive anchors of this
diagonal. We also assume that $y_4< 2.77$, or that both face
circumradii of $S$ along the fourth edge are less than $\sqrt{2}$.

$\SD$.  XX Insert $\SD$

%$E$.  This family consists of simplices $\{v_0,v_1,v_2,v_3\}$ such
%that 
%the edge lengths $y_{ij} = |v_i-v_j|$ satisfy
%   $$
%   y_{01},y_{02}\in[2t_0,2\sqrt{2}],\quad
%   y_{ij}\in [2,2t_0], 
%   $$
%for all other $ij$, and such that
%there is another simplex $\{w_0,w_1,w_2,w_3\}$ satisfying the same
%constraints with $y_{ij} = |w_i-w_j|$ and  such that $(w_0,w_1,w_2)=(v_0,v_1,v_2)$
%and $|w_3-v_3|>\sqrt8$.

\begin{lemma}\tlabel{lemma:2.77}
If a vertex $w$ is enclosed over a simplex $S$ of type $A$, $\SB$,
or $\SC$, then its height is greater than $2.77$.  Also, $\{v_0,w\}$
is not the diagonal of an upright quarter in the $Q$-system.
\end{lemma}

\begin{proof}
In case $A$, $\eta(y_4,y_5,y_6)<\sqrt{2}$, so an enclosed vertex
must have height greater than $2\sqrt{2}$.  It is too long to be
the diagonal of a quarter.

In case $\SB$, we use the fact that the isolated quarter does not
meet in the interior with any quarter in the $Q$-system. 
By tarski\tarf{tarski:enclosed-v}, an
enclosed vertex has length at least $2.77$.
By the symmetry of isolated quarters, this means that the diagonal
of a flat quarter must also be at least $2.77$.

In case $\SC$, the same calculation gives that the enclosed vertex
$w$ has height at least $2.77$.  Let the simplex $S$ be given by
$\{v_0,v,v_1,v_2\}$, where $\{v_0,v\}$ is the upright diagonal. By
tarski\tarf{tarski:pass-anchor}, $v_1$ and $v_2$ are anchors of
$\{v_0,w\}$. The edge between $w$ and its anchor cannot cross
$\{v,v_i\}$ by tarski\tarf{tarski:2t0-doesnt-pass-through}. (Recall
that two sets are said to {\it cross\/} if their radial
projections overlap.) The distance between $w$ and $v$ is at most
$2t_0$ by tarski\tarf{tarski:double-face}. If $\{v_0,w\}$ is the
diagonal of an upright quarter, the quarter takes the form
$\{v_0,w,v_1,v_3\}$, or $\{v_0,w,v_2,v_3\}$ for some $v_3$, by
tarski\tarf{tarski:double-face}. If both of these are quarters, then
the diagonal $\{v_1,v_2\}$ has four anchors $v$, $w$, $v_0$, and
$v_3$. The selection rules for the $Q$-system place the quarters
around this diagonal in the $Q$-system. So neither $\{v_0,w,v_1,v_3\}$
nor $\{v_0,w,v_2,v_3\}$ is in the $Q$-system. Suppose that
$\{v_0,w,v_1,v_3\}$ is a quarter, but that $\{v_0,w,v_2,v_3\}$ is not.
Then $\{v_0,w,v_1,v_3\}$ forms an isolated pair with $\{v_1,v_2,v,w\}$.
In either case, the quarters along $\{v_0,w\}$ are not in the
$Q$-system.
\end{proof}

\begin{remark}  The proof of this lemma does not make use of all the hypotheses
on $\SC$.  The conclusion holds for any simplex
$S(y_1,\ldots,y_6)$, with $y_1,y_4\in[2t_0,2\sqrt{2}]$,
$y_2,y_3,y_5,y_6\in[2,2t_0]$.
\end{remark}

\subsection{disjointness}%DCG 9.5, p95
    \oldlabel{2.6}

Let $S=\{v_0,v_1,v_2,v_3\}$ be a simplex of type $A$, $\SB$, or
$\SC$. An edge $\{v_4,v_5\}$ of length at most $2\sqrt{2}$ such
that $|v_4-v_0|,|v_5-v_0|\le 2t_0$ cannot cross two of the edges
$\{v_i,v_j\}$ of $S$.  In fact, it cannot cross any edge $\{v_i,v_j\}$
with $|v_i-v_0|,|v_j-v_0|\le 2t_0$ by tarski\tarf{tarski:skew-quad}.  The
only possibility is that the edge $\{v_4,v_5\}$ crosses the two
edges with endpoint $v_1$, with $|v_1-v_0|\ge2t_0$ in case $\SC$.  But
this too is impossible by tarski\tarf{tarski:double-face}.

Similar arguments show that the same conclusion holds for an edge
$\{v_4,v_5\}$ of length at most $2t_0$ such that $|v_4-v_0|\le2t_0$,
$v_5\le2\sqrt{2}$.  The only additional fact that is needed is
that $\{v_4,v_5\}$ cannot cross the edge between the vertex $v$ of
an upright diagonal $\{v_0,v\}$ and an anchor
(tarski\tarf{tarski:2t0-doesnt-pass-through}).





\begin{lemma}\tlabel{lemma:no-overlap}
    Consider two simplices $S$, $S'$, each of  type $A$, $\SB$, $\SC$,
or a quarter in the $Q$-system.
    %% XX I'm not sure if this hypothesis is needed.  If so, the lemma
    %% has to be moved after standard regions are introduced:
    %Assume that $S$ and $S'$ do not lie
    %in the cone over a quadrilateral region.  
    Then 
    $\op{conv}^0(S)$ does not meet $\op{conv}(S')$.
\end{lemma}

\begin{proof}
%% XX I'm not sure these next two lines are needed...; see above.
%By hypothesis, the standard region is not a quadrilateral, and we
%thus exclude the case of conflicting diagonals in a quad cluster.
We claim that no vertex $w$ of $S$ is enclosed over $S'$.
Otherwise, $w$ must have height at least $2t_0$, so that $\{v_0,w\}$
is the diagonal of an upright in the $Q$-system, and this is
contrary to Lemma~\ref{lemma:2.77}. Similarly, no vertex of $S'$
is enclosed over $S$.

Let $\{v_1,v_2\}$ be an edge of $S$ crossing an edge $\{v_3,v_4\}$ of
$S'$. By the preceding remarks, neither of these edges can cross
two edges of the other simplex. The endpoints of the edges are not
enclosed over the other simplex. This means that one endpoint of
each edge $\{v_1,v_2\}$ and $\{v_3,v_4\}$ is a vertex of the other
simplex.  This forces $S$ and $S'$ to have three vertices in
common, say $v_0$, $v_2$, and $v_3$.  We have $S=\{v_0,v_1,v_3,v_2\}$
and $S'=\{v_0,v_3,v_2,v_4\}$. If
    $|v_2-v_0|\in[2t_0,2\sqrt{2}]$,
then we see that the anchors $v_3$, $v_4$ of $\{v_0,v_2\}$ are not
consecutive.  This is impossible for simplices of type $\SC$ and
upright quarters.  Thus, $v_2$ and $v_3$ have height at most
$2t_0$.  We conclude, without loss of generality, that
    $|v_4-v_0|\in[2t_0,2\sqrt{2}]$
and $|v_1-v_2|\ge 2t_0$.

The heights of the vertices of $S$ are at most $2t_0$, so it has
type $A$ or $\SB$, or it is a flat quarter in the $Q$-system. If
$S'$ is an upright quarter in the $Q$-system, then it does not
overlap an isolated quarter or a flat quarter in the $Q$-system,
so $S$ has type $\SA$. By tarski\tarf{tarski:277}, we have
$|v_1-v_2|>2.77$.  This imposes the contradictory constraints
on $\SA$
    $$
    2.77\ge |v_1-v_2|>2.77.
    $$
Thus $S'$ has type $\SC$.  This forces $S$ to have type $\SA$.  We
reach the same contradiction  $2.77 > 2.77$.
\end{proof}

\subsection{type A}%DCG 9.6, p96
    \label{sec:separation}
    \oldlabel{2.7}

Let $S = \{v_0,v_1,v_2,v_3\}$.
Let $\op{cone}^0(S) = \op{cone}^0(v_0,\{v_1,v_2,v_3\}$.
Let $V_S = \op{VC}(\Lambda,v_0)\cap \op{cone}^0(S)$, for a simplex $S$ of type $\SA$,
$\SB$, or $\SC$. 
We truncate $V_S$ to $V_S(t_S)$ by intersecting
$V_S$ with a ball of radius $t_S$.  The parameters $t_S$ depend on
the type of $S$.

If $S$ has type $\SA$, we use $t_S=+\infty$ (no truncation).

\begin{lemma} Let $S=\{v_0,v_1,v_2,v_3\}$ be a simplex of type $\SA$.
There is a null set $E$, such that
we have  $ \Omega(v_0,S) \cap \op{cone}^0(S) \subset V_S \cup E$.
\end{lemma}

\begin{proof} 
We use the fact that if $b$ is a barrier, then $\op{conv}$ does
not meet $\op{conv}^0(S)$ by Lemma~\ref{XX}.  


Excluding a null set, we may assume 
for a contradiction that
$x\in \Omega(v_0,S) \cap \op{cone}^0(S) \cap \op{VC}(\Lambda,v)$,
for some $v\ne v_0$.  

% ...
By tarski\tarf{tarski:vor-bar-sqrt2}, $x$ and $v_0$ lie on the
same side of $\op{aff}\{v_1,v_2,v_3\}$.  Thus, $x$ is in
$\op{conv}^0(S)$.  
Thus, every vertex of $S$ is unobstructed at $x$.  Thus, $x$
is closer to $v$ than to any vertex of $S$.

By tarski\tarf{tarski:vor-bar-sqrt2}, $\op{conv}\{v_1,v_2,v_3\}$ 
separates
$\Omega(v_0,S)\cap \op{cone}^0(S)$ from $\Omega(v,\{v,v_1,v_2,v_3\})$ when
$v$ is enclosed over $S=\{v_0,v_1,v_2,v_3\}$.  This is contrary
to the assumption that $x$ lies in the intersection of these
two sets.

If $\Omega(v,\{v_0,v_1,v_2\})$ meets $\op{conv}^0(S)$, then
$S'=\{v,v_0,v_1,v_2\}$ must be a quarter or quasi-regular tetrahedron.
If $x$ is a barrier, then $x\not\in\op{VC}(\Lambda,v)$.  This implies
that $S'$ is a quarter that is not in the $Q$-system.
It
cannot be an isolated quarter because of the edge length
constraint $2.77$ on simplices of type $\SA$.
There must be a
conflicting diagonal $\{v_0,w\}$, where $w$ is enclosed over $Q$. ($w$
cannot be enclosed over $S$ by results of
Lemma~\ref{lemma:no-overlap}.) This shields the $V$-cell at $v$
from $\op{cone}^0(S)$ by the two barriers $\{v_0,w,v_1\}$ and $\{v_0,w,v_2\}$ of
quarters in the $Q$-system.
\end{proof}

\begin{lemma} Let $S=\{v_0,v_1,v_2,v_3\}$ be a simplex of type $A$.
  $V_S$ is disjoint from all of the set $\FC(v,W)$.
\end{lemma}

\begin{proof}
This is evident from
Lemma~\ref{lemma:delta-tri}. % and \ref{lemma:delta-flat}.
\end{proof}


Our justification that $V_S(t_S)$ can be treated as an
independently scored entity is now complete.

\subsection{type B}%DCG 9.7, p96
    \oldlabel{2.8}

If $S(y_1,\ldots,y_6)$ has type $\SB$, we label vertices so that
the diagonal is the fourth edge, with length $y_4$. We set
$t_S=1.385$. The calculation in Lemma~\ref{lemma:2.77}
shows that any enclosed vertex over $S$ has height at least
$2.77=2t_S$.

\begin{lemma} Let $S=\{v_0,v_1,v_2,v_3\}$ be a simplex of type $\SB$.
There is a null set $E$, such that
we have  $ \Omega(v_0,S) \cap \op{cone}^0(S) \cap B(v_0,1.385) 
\subset V_S \cup E$.
\end{lemma}

\begin{proof}  As above, assume for a contradiction that there
is a point in 
 $$\Omega(v_0,S)\cap \op{cone}^0(S) \cap B(v_0,1.385)\cap \op{VC}(\Lambda,v'),$$
with $v'\ne v_0$.
Vertices outside $\op{cone}^0(S)$ cannot reach inside $S$ this way.  In
fact, such a vertex $v'$ would have to form a quarter or
quasi-regular tetrahedron with a face of $S$.  The $V$-cell at
$v'$ cannot meet $\op{cone}^0(S)$ unless it is a quarter that is not in the
$Q$-system. But by definition, an isolated quarter is not adjacent
(along a face along the diagonal) to any other quarters.
\end{proof}

%To separate the scoring of $V_S(t_S)$ from the rest of the
%standard cluster, we also show that the terms of
%Formula~\ref{eqn:3.5}  for $V_S(t_S)$ are represented
%geometrically by solids that lie in the cone $\op{cone}^0(S)$.   This
%is the purpose of the following lemma.

\begin{lemma} Let $S=\{v_0,v_1,v_2,v_3\}$ be a simplex of type $\SB$.
Then  $\op{rcone}^0(v_0,v_1,|v-v_0|/2.77)$ does not meet the
$\op{cone}(v_0,\{v_2,v_3\})$.
\end{lemma}

\begin{proof} This is tarski\tarf{tarski:beta:B}.
\end{proof}

\begin{lemma} $\Omega(v_0,S) \cap \op{cone}^0(S) \cap B(v_0,1.385)$
does not meet the sets $\delta(v)$.
\end{lemma}

\begin{proof}
The reasons given in Section~\ref{sec:separation} for the
disjointness of $\delta(v)$ and $V_S(t_S)$ apply to this
situation as well.
\end{proof}


This completes the justification that
$V_S(t_S)$ is an object that can be treated in separation from the
rest of the local $V$-cell.

\subsection{type C}%DCG 9.8, p97
    \oldlabel{2.9}

If $S(y_1,\ldots,y_6)$ is of type $\SC$, we label vertices so that
the upright diagonal is the first edge.  We use $t_S =+\infty$ (no
truncation).   

\begin{lemma} Let $S=\{v_0,v_1,v_2,v_3\}$ be a simplex of type $C$.
There is a null set $E$, such that
we have  $ \Omega(v_0,S) \cap \op{cone}^0(S) \subset V_S \cup E$.
\end{lemma}

\begin{proof}  %% XX Rewrite this proof.
Vertices outside $S$ cannot affect the shape of $V_S(t_S)$.  Any
vertex $v'$ would have to form a quarter along a face of $S$.  If
the shared face lies along the first edge, it is a quarter $Q$ in
the $Q$-system, because one and hence all quarters along this edge
are in the $Q$-system.  The faces of this quarter are then
barriers. If the shared face lies along the fourth edge, then its
length is at most $2.77$, so that the quarter cannot be part of an
isolated pair. If it is not in the $Q$-system, there must be a
conflicting diagonal. The two faces along this conflicting
diagonal of the adjacent pair in the $Q$-system (that is, the pair
taking precedence over $Q$ in the $Q$-system) are barriers that
shield the $V$-cell at $v'$ from $S$.
\end{proof}

The reasons given in Section~\ref{sec:separation} for the
disjointness of $\delta(v)$ and $V_S(t_S)$ apply to simplices of
type $\SC$ as well. This completes the justification that
$V_S(t_S)$ is an object that can be treated in separation from the
rest of the local $V$-cell.

\subsection{type D}%DCG 9.9, p97
    \oldlabel{2.10}

We introduce a small variation on simplices of type $\SC$, called
type $\SD $.  We define a simplex $\{v_0,v,v_1,v_2\}$ of type $\SD$
to be one satisfying the following conditions.
    \begin{enumerate}
    \item The edge $\{v_0,v\}$ is an upright diagonal of an upright quarter
        in the $Q$-system.
    \item $|v_2-v_0|\in[2.45,2t_0]$.
    \item $v_1$ and $v_2$ are anchors of $v$.
    \item $|v-v_2|\in [2.45,2t_0]$.
    \item The edge $\{v_1,v_2\}$
    is a diagonal of a flat quarter with face $\{v_0,v_1,v_2\}$.
    \end{enumerate}

It follows that $v_1$ and $v_2$ are consecutive anchors of
$\{v_0,v\}$.

On simplices $S$ of type $\SD $, we label vertices so that the
upright diagonal is the first edge.  We use $t_S=+\infty$ (no
truncation).  

Simplices of type $\SD $ are separated from quarters in the
$Q$-system and simplices of types $\SA$ and $\SB$ by procedures
similar to those described for type $\SC$.  The following lemma is
helpful in this regard.


\begin{lemma}\tlabel{lemma:C'Q}
 The flat quarter along the face $\{v_0,v_1,v_2\}$ is
in the $Q$-system.
\end{lemma}

\begin{proof}
By tarski\tarf{tarski:245}, there cannot be an enclosed vertex
of height at most $\sqrt2$. 
So nothing is enclosed over the flat quarter.
By tarski\tarf{tarski:245bis}, there cannot be an edge of length
at most $2\sqrt2$ that crosses inside the slice.
(XX slice has not yet been defined.) 
This implies that the flat quarter does not have
a conflicting diagonal and is not part of an isolated pair.
\end{proof}


\begin{lemma}\tlabel{lemma:VC-no-conv}
Suppose that $v'$ is enclosed over $S$.  Then $\op{VC}(\Lambda,v')$ does
not meet $\op{conv}^0(S)$.
\end{lemma}

\begin{proof} If there is a point $x$ of intersection, then
$x$ is closer to $v'$ than to any point of $S$. 
By tarski\tarf{tarski:vor-bar-quad}, this implies that 
$S'=\{v',v,v_1,v_2\}$ is a quarter.  By Lemma~\ref{lemma:C'Q},
$S'$ is in the $Q$-system.  Thus, $\{v,v_1,v_2\}$ is a barrier,
and $x$ is obstructed from $v'$.
\end{proof}


Unlike the other cases, there can in fact be overlap between
$\FC(v,W)$ and simplex of type $\SD$, when the upright
diagonal of the simplex is $\{v_0,v\}$.  This is because the
conditions defining a crown tuple are not incompatible with
the conditions defining type $\SD$.  Nevertheless, except in the
obvious case where the simplex of type $\SD$ and the crown tuple are both
constructed between the same consecutive anchors of $\{v_0,v\}$, there
can be no overlap of a $\FC(v,W)$ with a simplex of type
$\SD$.


\subsection{summary}


The construction of the decomposition of the $V$-cell $\op{VC}(\Lambda,v_0)$
is now complete. It consists of the pieces

    \begin{itemize}
    \item $\delta(v)=\delta_i(v)$,
         for each diagonal $\{v_0,v\}$ of an upright quarter
        in the $Q$-system, and $i$ as in Definition~\ref{def:crown-tuple}.
    \item truncations of Voronoi pieces $V_S(t_S)$ for simplices of type
        $\SA$, $\SB$, or $\SC$ (and on rare occasion $\SD$),
    \item $\tildeV(t_0)$, the truncation at $t_0$ of all parts of
        $\op{VC}(\Lambda,v_0)$ that do not lie in any of the cones $\op{cone}^0(S)$ over
        simplices
        of type $\SA$, $\SB$ or $\SC$,
    \item $\delta'$, the part not lying in any of the preceding.
    \end{itemize}





%\chapter{Sphere Packing,  Hypermap, Fan}
%\chapter{Basic Properties of Standard Regions}%DCG Sec.10, p99
%    \label{sec:intro}
%    \oldlabel{1}
%\label{chapter:VQ}

\section{Fan}

There are many different fans $(v,V,E)$ that
can be associated with a centered packing $(\Lambda,v)$.
As we will see, different selections of fans
will lead to different approximations to the function $\sigma(\Lambda,v)$.
It will be important for us to have many different approximations
at our disposal.  For that reason, we consider a number of
fans.

\begin{definition}\label{def:compatible}
Let $(v,V,E)$ be a fan.  We say that it is
{\it compatible} with a centered packing $(\Lambda,w)$ 
if 
\begin{itemize}
\item 
$v=w\in\Lambda$
\item
 $V\subset \Lambda$, 
\item  $Q\in \CalQ(\Lambda,v)$ implies
$\op{conv}^0(Q)\cap X(v,V,E)=\emptyset$,
\item  If $e\in E$, $(v,v_1,v_2,v_3)\in \CalQ(\Lambda,v)$,
and $C(v,e)\cap C(v,\{v_1,v_2\})\ne \emptyset$, then
   $e = \{v_1,v_2\}$.
\end{itemize}
\end{definition}

\subsection{standard graph}

Let $(\Lambda,v)$ be a centered packing.  The set $U=U(\Lambda,v)$
is defined in Definition~\ref{def:U}.  
Let 
$$E = E(\Lambda,v) = \{(u,u')\in U(\Lambda,v) \mid |u-u'|\le 2t_0\}.
$$

\begin{lemma}\tlabel{lemma:Q-in-region} 
Each simplex in the $Q$-system with a
vertex at $v_0$ lies entirely in the closed cone over some
standard region $R$.
\end{lemma}

\begin{proof}  Assume for a contradiction that $Q=\{v_0,v_1,v_2,v_3\}$
with $v_1$ in the open cone over $R_1$ and with $v_2$ in the open
cone over $R_2$.  Then $\{v_0,v_1,v_2\}$ and $\{v_0,w_1,w_2\}$ (a wall
between $R_1$ and $R_2$) cross;  this is contrary to
Lemma~\ref{lemma:barrier-no-overlap}.
\end{proof}

\begin{lemma}\tlabel{lemma:pack-fan}
Let $(\Lambda,v)$ be a centered packing.  Then
$(v,U,E)$ is a fan.
\end{lemma}

\begin{proof}
Use tarski\tarf{tarski:2t0-doesnt-pass-through}.
\end{proof}

\begin{definition}  We call $(v,U,E)$ the standard
graph of the centered packing $(\Lambda,v)$.   We call the
components of $Y(v,U,E)$ the standard components of the centered
packing $(\Lambda,v)$.  We call the hypermap $\op{hyp}(v,U,E)$
the standard hypermap of the centered packing.  
\end{definition}

We consider the standard graph  to be the
primary fan attached to a centered packing.  All other
standard graphs are in some sense derivative, either by deleting
certain edges from the standard graphs, or by adding additional
edges.


\begin{lemma}\tlabel{lemma:graph-compat}
Let $(\Lambda,v)$ be a centered packing.  The standard graph
of $(\Lambda,v)$ is $(\Lambda,v)$-compatible.
\end{lemma}

\begin{definition}  We call a component $R$ of $Y(v,V,E)$ an
$n$-gon (triangle, quadrilateral, pentagon, and so forth) if
the set of darts of $\op{hyp}(v,V,E)$ that lead into $R$ is
a single face of the hypermap and if that face is an orbit of
cardinality $n$.   We say that
a component of $Y(v,U,E)$ is exceptional, if it is not a
triangle or quadrilateral.
\end{definition}


\begin{lemma}\tlabel{lemma:FC-standard}
Each $\FC(v_0,v,ldots)$ lies entirely in the cone over the standard
region that contains $\{v_0,v\}$.
\end{lemma}

\begin{proof}
By Lemma~\ref{lemma:delta-tri},
the cone over a standard region is bounded by the cones  over the
quasi-regular triangles.
\end{proof}

\subsection{deleting a node}

 XX Move to hypermap and geometry.

Here we describe a construction of deleting nodes from a 
fan.  


Let $(v,V,E)$ be a fan with hypermap $\op{hyp}(v,V,E)$.
Let $V'\subset V$ be a set of nodes. 
Let $E'\subset E$ be the set of edges $e$ that meet $V'$.
We have a fan $(v,V\setminus V',E\setminus E')$.

We can identify $V'$ with a subset of $V$, where
the node containing darts $(v,w,\ldots)$ is identified with $w\in V$.
We define a normal family of contour loops on 
$\op{hyp}(v,V,E)$.  Let $D''$ be the set of darts that lie in the
nodes constituting $V'$.  Let $D'$ be the remaining darts.
We place each $x\in D'$ in a contour
loop by the following permutation $r$ on $D'$:
    $$r x =
    \begin{cases}
    f x, & f x \in D'\\
    n^{-1} x & f x \in D''\\
    \end{cases}
    $$
For each $x\in D'$, we obtain a contour loop $(x,r x, r^2 x,\ldots)$.
Let $\cal L$ be this family of contour loops.  It is normal.

\begin{lemma}\tlabel{lemma:quot-fan}
In this context,
$\op{hyp}(v,V,E)/{\cal L}$ is isomorphic to 
$\op{hyp}(v,V\setminus V',E\setminus E')$.
\end{lemma}



\subsection{degenerate darts}

 XX Move to hypermap and geometry??

The hypermap $\op{hyp}(v,V,E)$ attached to a fan $(v,V,E)$
has two types of darts: the non-degenerate darts of the form
$(v,v_1,v_2,v_3)$ and the degenerate darts of the form $(v,w)$.
Each degenerate dart is a  fixed point for the edge, face, and node
permutations.

A special case of the construction of the previous section
is to delete a degenerate node (dart) from a fan.
Each degenerate dart $x=(v,w)$ uniquely determines element
$w\in V$.  This element $w$ has no edges: $E_w = \emptyset$.
If $V'$ is a set of degenerate darts, with corresponding elements
$V'\subset V$, 
we obtain a fan
$\op{hyp}(v,V\setminus V',E)$.  There is a canonical bijection
between components of $Y(v,V,E)$ and components of $Y(v,V\setminus V',E)$.

Each degenerate dart $x=(v,w)$ leads into a uniquely determined
component of $\op{hyp}(v,V,E)$.  
Figure~\ref{fig:deg-pent-hex} %% WW not yet drawn. Two frames of DCG~page126.
shows a degenerate dart that leads into a component where
the other darts leading into the component form a pentagon
or hexagon.

\begin{figure}[htb]
  \centering
  \myincludegraphics{noimage.eps}
  \caption{Degenerate darts lead into components} % DCG examples.
  \label{fig:deg-pent-hex}
\end{figure}


\subsection{aggregated graphs}
%\section{Contravening Plane Graphs defined}
\label{sec:stargraph}


Aggregated graphs are fans that are obtained from
a fan by deleting various vertices.  

Aggregates will be formed only when the standard hypermap has
very specific combinatorial structures.  We describe those
combinatorial structures here.




\begin{definition}
Let $(v,V,E)$ be a fan such that $V\subset U(\Lambda,v)$.
An aggregating contour loop in $\op{hyp}(v,V,E)$ 
is any of the following specific
types of contour loops satisfying the 
two additional conditions that follow.
See Figure~\ref{fig:agg} %% WW not yet drawn. DCG~page126, p.158,
%% First three are DCG~page 126.
for the exact combinatorial structures.
\begin{itemize}
\item A contour loop with six darts tracing through five nodes. 
\item A contour loop with seven darts tracing through five nodes (with three
  darts at one node).
   The darts in this contour loop
   meet a triangle face and an octahedral face (Figure~\ref{fig:tri-pent}). 
   %pent-tri
\item A contour loop with seven darts tracing through six nodes.
\item A contour loop with six darts tracing through four nodes. 
\item A contour loop with 14 darts tracing through eight nodes
   (Figure~\ref{fig:degree6}). %oct-agg
\item A contour loop with 11 darts tracing through six nodes %hex-agg.
   (Figure~\ref{fig:degree6}).
\end{itemize}
\index{aggregating}
To qualify as an aggregating 
contour loop we impose the following two additional conditions.
\begin{enumerate}
\item Each dart $(v,v_1,\ldots)$ in a face visited by the contour 
loop satisfies $v_1\in U(\Lambda,v)$.
\item Each face step $(v,v_1,\ldots)\to (v,w_1,\ldots)$ in the contour
loop satisfies $|v_1-w_1|\le 2t_0$.
\end{enumerate}
\end{definition}

Note that each aggregating contour loop has the property that
if it meets a dart of a face at a node, then it meets every dart
of that face at that node.

\begin{definition}  Let $(\Lambda,v_0)$ be a centered packing with
standard fan $(v_0,V,E)$ and hypermap $H$.  For
each such collection of data, choose once and for all
a maximal set of disjoint aggregating
contour loops in $H$.  This extends to a normal family by adding
contour loops for each face of $H$ that is disjoint from the aggregating
loops.    We call the quotient the aggregating hypermap of $(\Lambda,v_0)$.
\index{aggregating hypermap}
\end{definition}

\begin{lemma}\tlabel{lemma:agg-hyper}
The aggregating hypermap is connected, planar, plane,
and has no degenerate darts.   Every face of the aggregating hypermap
is simple. 
\end{lemma}

As in Lemma~\ref{XX}, aggregating hypermap is the hypermap
obtained from a fan (obtained by deleting edges and
vertices from the standard fan).

\begin{figure}[htb]
  \centering
  \myincludegraphics{noimage.eps}
  \caption{Some aggregating contour loops} % DCG examples.
  \label{fig:agg}
\end{figure}

\begin{figure}[htb]
  \centering
  \myincludegraphics{\ps/tripent.eps}
  \caption{An aggregating contour loop forming a pentagon in the quotient}
  \label{fig:tri-pent}
\end{figure}

\begin{figure}[htb]
  \centering
  \myincludegraphics{\ps/degree6.eps}
  \caption{Further aggregating loops}
  \label{fig:degree6}
\end{figure}



It will be shown in
Lemma~\ref{lemma:deg5} that there is at most one aggregating
contour loop in the standard hypermap of a contravening centered packing,
so we will not worry here about how to select one.




\subsection{connected sums and aggregating}

A special case frequently occurs.  The family of contour loops
$\cal L$ often has the property that all but one contour is
a face of the hypermap.  Let $C$ be the contour loop that
is not a face of the hypermap.  The process of deleting an
edge can then be viewed as removing the internal structure
of the contour loop $C$.  The contour loop $C$ becomes a single
face $F$ in the quotient hypermap.

Connected sums, which were described in Section~\ref{XX} allow
us to reverse the process by inserting internal structure to
a face $F$.  This allows us to retrieve the original hypermap
$\op{hyp}(v,V,E)$
from the quotient
$\op{hyp}(v,V,E)/{\cal L}$.  Specifically, by connected sums, we can recover
the original hypermaps from the quotients described by
Figures~\ref{fig:agg}, \ref{fig:tri-pent}, \ref{fig:degree6}.


XX add masked quarter, enhanced graph, maximal graph,


\section{Special Structure}

\subsection{quad}
\label{sec:quad-class}


\begin{lemma}\tlabel{lemma:quad-classify}
Let $C=(v_0,(v_1,v_2,v_3,v_4))$ be a quad cluster.  Then it has
exactly one of the following forms.
\begin{itemize}
  \item (flat) $\{v_0,v_1,v_2,v_3\}$ and $\{v_0,v_1,v_3,v_4\}$ are flat
    quarters in the $Q$-system.
   \item (flat) $\{v_0,v_2,v_4,v_1\}$ and $\{v_0,v_2,v_4,v_3\}$ are flat
     quarters in the $Q$-system.
     \item (octahedron) There exists an enclosed vertex $w$, such that
       $(v_0,w,v_1,v_2,v_3,v_4)$ is a quartered octahedron with
       four upright quarters in the $Q$-system.
   \item (pure) We have $|v_1-v_3|>\sqrt8$, $|v_2-v_4|>\sqrt8$ and
     there is no enclosed vertex $w$ with $|w-v_0|<\sqrt8$.
    \item (mixed) We have $|v_1-v_3|>\sqrt8$, $|v_2-v_4|>\sqrt8$, there
      is no enclosed vertex forming a quartered octahedron, but
      there exists some enclosed vertex $w$ with $|w-v_0|<\sqrt8$.
      \index{pure}\index{mixed}
\end{itemize}
\end{lemma}

\begin{proof}
If the quad cluster has a diagonal of length at most $\sqr8$
between two corners, there are three possible decompositions. (1)
The two quarters formed by the diagonal lie in the $Q$-system so
that the scoring rules for the $Q$-system are used.  (2) There is
a second diagonal of length at most $\sqr8$, and we use the two
quarters from the second diagonal for the scoring. (3) There is an
enclosed vertex that makes the quad cluster into a quartered
octahedron and the four upright quarters are in the $Q$-system.

Now suppose that neither diagonal is less than $\sqr8$ and the
quad cluster is not a quartered octahedron. If there is no
enclosed vertex of length at most $\sqr8$, the quad cluster
contains no quarters. An upper bound on the score of the quad
cluster $(P,D)$ is $\op{svR}(D,P,\sqr2)$. The remaining cases are
called {\it mixed\/} quad clusters. Mixed quad clusters enclose a
vertex of height at most $\sqr8$ and do not contain flat quarters.
\end{proof}


\subsection{sundry shapes}



\begin{definition}
Define the {\it central vertex\/} $v$ of a flat quarter to be the
vertex for which $\{v_0,v\}$ is the edge opposite the diagonal.
\end{definition}

%\subsection{types}\label{sec:types}%DCG 10.2, p100

\begin{definition}
Let $v$ be a vertex of height at most $2t_0$.  We say that $v$ has
{\it type\/} $(p,q)$ if every standard region with a vertex at $\bar
v$ (the radial projection of $v$) is a triangle or a quadrilateral,
and if there are exactly $p$ triangular faces and $q$ quadrilateral
faces that meet at $\bar v$.  We write $(p_v,q_v)$ for the type of
$v$.
\index{type}
\end{definition}

%\subsection{contexts} %DCG 11.2, p.113
%    \oldlabel{3.3}

The context $\x(3,0)$ is to be regarded as two
quasi-regular tetrahedra sharing a face rather than as three
quarters along a diagonal.  In particular, by
Definition~\ref{def:q-system}, the upright quarters do not belong
to the $Q$-system.

%\subsection{slices} %DCG 11.5,p.115
%    \oldlabel{3.6}
%    \label{sec:slice}  % was sec:anchored-simplex

\begin{definition}
Let $\{v_0,v\}$ be an upright diagonal, and let
$v_1,v_2,\ldots,v_k=v_1$ be its anchors, ordered cyclically around
$\{v_0,v\}$.  This cyclic order gives dihedral angles between
consecutive anchors around the upright diagonal. We define the
dihedral angles so that their sum is $2\pi$, even though this will
lead us to depart from our usual conventions by assigning a
dihedral angle greater than $\pi$ when all the anchors are
concentrated in some half-space bounded by a plane through
$\{v_0,v\}$. When the dihedral angle of $S=\{v_0,v,v_i,v_{i+1}\}$ is at
most $\pi$, we say that $S$ is a {\it slice\/} if
$|v_i-v_{i+1}|\le3.2$. (The constant $3.2$ appears throughout this
\chap.) All upright quarters are slices. If an upright
diagonal is completely surrounded by slices, the
upright diagonal is sometimes called a {\it loop}. If
$|v_i-v_{i+1}|>3.2$ and the angle is less than $\pi$, we say there
is a {\it gap\/} around $\{v_0,v\}$ between $v_i$ and $v_{i+1}$.
\index{slice}\index{loop}\index{gap}
\end{definition}

\begin{definition}\index{unconfined@3-unconfined}
 \index{crowded4@3-crowded}\index{crowded3@4-crowded}
% Def'n copied from linprog.tex
Consider an upright diagonal that is not a loop. Let $R$ be the
standard region that contains the upright diagonal and its
surrounding quarters.  Assume we are in the context $(4,1)$ or
$(5,1)$.  In the context $(4,1)$, suppose that there does not exist
a plane through the upright diagonal such that all three quarters
lie in the same half-space bounded by the plane. Then we say that
the context is {\it $3$-unconfined}. If such a plane exists, we say
that the context is $3$-crowded. We call the context $(5,1)$ a
$4$-crowded upright diagonal. Sections~\ref{x-3.4} and \ref{x-3.5}
reduce everything to contexts with four or five anchors around each
vertex.  If there are $5$ darts, 
Remark~\ref{rem:5dart} shows that we can assume at most one
gap. This gives contexts $(5,0)$ and $(5,1)$.  If there are four
anchors, then Lemma~\ref{x-3.9.1} will dismiss all contexts except
$(4,0)$ and $(4,1)$. Thus, every upright diagonal is exactly one of
the following: a loop, $3$-unconfined, $3$-crowded, or $4$-crowded.
%\def\Sfour{{{\cal\mathbf S}_4^+}}  --> $4$-crowded upright diagonal
%\def\Sminus{{{\cal\mathbf S}_3^-}} --> $3$-crowded upright diagonal
%\def\Splus{{{\cal\mathbf S}_3^+}}  --> $3$-unconfined upright diagonal
\end{definition}

XX
This lemma is a consequence of the two others that follow
(Lemma~\ref{lemma:slice-quarter}, Lemma~\ref{lemma:3-crowded}).
Extract the geometry as a lemma that is independent.
The
context of the lemma is the set of slices that have
not been erased by previous reductions.

\begin{lemma}\tlabel{lemma:anchor-no-overlap}
The interiors of slices do not meet.
\end{lemma}

\begin{proof}
The remaining contexts have four or  five anchors. Let $w$ and the
slice $S=\{v_0,v,v_1,v_2\}$ be as in Section~\ref{x-3.6}.
Our object is to describe the local geometry when an upright
diagonal is enclosed over a slice. If $|v_1-v_2|\le
2\sqrt{2}$, we have seen in tarski\tarf{tarski:double-face} that
there can be no enclosed upright diagonal with $\ge 4$ anchors
over the slice $S$.

Assume  $|v_1-v_2|>2\sqrt{2}$. Let $w_1,\ldots,w_k$, $k\ge4$, be the
anchors of $\{v_0,w\}$, indexed consecutively. The anchors of $\{v_0,w\}$ do not
lie in $C(S)$, and the triangles $\{v_0,w,w_i\}$ and $\{v_0,v,v_j\}$ do not
overlap. Thus, the plane $\{v_0,v_1,v_2\}$ separates $w$ from
$\{w_1,\ldots,w_k\}$. Set $S_i=\{v_0,w,w_i,w_{i+1}\}$.
By a calculation\footnote{\calc{83777706}} %A8
%$\A_8$,
    $$\pi\ge \dih(S_1)+\cdots+\dih(S_{k-1})\ge (k-1)0.956.$$

Thus, $k=4$. The common upright diagonal  of the three simplices
$\{S_i\}$ is {\it $3$-crowded}.  We claim that
$\{v_1,v_2\}=\{w_1,w_4\}$. Suppose to the contrary that, after
reindexing as necessary, $S_0=\{v_0,w,w_1,v_1\}$ is a simplex, with
$v_1\ne w_1$, that does not overlap $S_1,\ldots,S_3$. Then $\pi\ge
\dih(S_0)+\cdots+\dih(S_3)$. So
    $0.28\ge \pi-3(0.956)\ge \dih(S_0)$.
A calculation\footnote{\calc{83777706}} %A8
now implies that $|w-v_1|\ge 2\sqrt{2}$.

By tarski\tarf{tarski:336}, the four vertices
$\{v_0,w,v_1,v_2\}$ cannot be coplanar.
We have that $2\sqrt{2}\ge|w-v_0|$ and by tarski\tarf{tarski:E:part4:1},
we also have $|w-v_0|>2\sqrt2$.
This contradiction establishes that $v_1=w_1$.
\end{proof}


\begin{definition}
Let a $3$-unconfined node be a node that is an upright diagonal 
with four darts and one gap in a situation where none of
the quarters along this upright diagonal masks a flat quarter.
\end{definition}







\section{Score}





\subsection{Voronoi cell and simplices}


Let $S$ be a set of four vertices and let $v$ be a vertex of that simplex. Let
$\Omega(v,S)$ be the subset of $\op{conv}^0(S)$ consisting of points closer
to $v$ than to any other vertex of $S$. By
Lemma~\ref{lemma:Q-divide}, if $S\in\CalQ(\Lambda,v)$, then
$$\Omega(v,S) = \Omega(\Lambda,v)\cap\op{conv}^0(S).$$
Under the assumption that $S$ contains its circumcenter and that
every one of its faces contains its circumcenter, an explicit
formula for the volume $\op{vol}(\Omega(v,S))$ has been
calculated.  %% \cite[Section~8.6.3]{part1}. 

XX MOVE TO VOLUMES:

\begin{lemma}  Let $S = \{v_0,v_1,v_2,v_3\}$ be a set of four points in $\ring{R}^3$.
Assume that the circumcenter of $S$ is contained in $\op{conv}^0(S)$.  Assume
that each face of $S$ is acute.
Then 
  $$
  \op{vol}\,\Omega(v_1,S) = f(x_{ij}),
  $$ 
where $x_{ij} = |v_i-v_j|^2$, and
$$
   f(x_{ij}) = \sum_{(i,j,k)\in \Pi} \frac{x_{0i} (x_{0j}+x_{ij}-x_{0i})\chi(x_{jk},x_{ik},x_{0k},x_{0i},x_{0j},x_{ij})}
   {48\ups(x_{0i},x_{0j},x_{ij})\Delta(x_{01},x_{02},x_{03},x_{23},x_{13},x_{12})^{1/2}}
$$
and $\Pi = \{(1,2,3),\ldots\}$ is the set of the six permutations of $(1,2,3)$.
(The function $\chi$ is defined in Definition~\ref{def:chi}.)
\end{lemma}

\begin{proof} By tarski\tarf{tarski:XX}, the set $\Omega(v_1,S)$ is a union of
six Rogers simplices, up to a null set.  The volume of a Rogers simplex appears
as Lemma~\ref{XX}.  Summing the contributions from the six simplices, we obtain
the given formula.
\end{proof}

The function $f(x_{ij})$ makes perfect sense for any $S$ for which $\ups(x_{0i},x_{0j},x_{ik},\Delta(x_{ij})\ne 0$,
even though the lemma is valid only under special constraints.
For any simplex $S=\{v_0,v_1,v_2,v_3\}$ such that the denominator of $f$ is nonzero, we define
$$
\op{volan}(v_0,S) = f(x_{ij}),\quad x_{ij} = |v_i-v_j|^2.
$$  
(The denominator is always nonzero when $S$ is not collinear by tarski\tarf{tarski:XX}{~}\tarf{tarski:XX}.)
The following appeared as Claim~\ref{claim:volan}.

\begin{lemma}\tlabel{lemma:volan}  %%Cf. claim:volan
Let $S=\{v_1,v_2,v_3,v_4\}$ be in the $\CalQ$-system. Then
    $$
    \sum_{i=1}^4 \op{volan}(v_i,S) = \sum_{i=1}^4
    \op{vol}(\Omega(v_i,S)) = \op{vol}(\op{conv}^0(S)).
    $$
\end{lemma}

\begin{proof} As we have explicit formulas for everything involved,
it is just a matter of plugging in the formulas and checking the sum.
(Even without a calculation, the lemma follows by the principle of analytic continuation;
but in the interest of proving everything from first principles, we
check the sum.)
\end{proof}

\subsection{truncation}

Let $(\Lambda,v)$ be a centered packing and let $(v,V,E)$ be a compatible
fan.
If $R$ is a component of $Y(v,V,E)$, we write 
  $$
  V_R(v,t) = \op{VC}(\Lambda,v_0)\cap C(R)\cap B(v,t)
  $$
We often
take $t=t_0$.

%If $\{v_0,v\}$, of length between $2t_0$ and
%$2\sqrt{2}$, is not the diagonal of an upright quarter in the
%$Q$-system, then $v$ does not affect the truncated cell $V_R(t_0)$
%and may be disregarded. For this reason we confine our attention
%to upright diagonals that lie along an upright quarter in the
%$Q$-system.


 Let $S=\{v_0,v_1,v_2,v_3\}$ be a simplex. Fix $t$ in the range
$t_0\le t\le\sqrt2$.  Assume that $t$ is at most the circumradius
of $S$. Assume that it is at least the circumradius of each of the
faces of $S$.  Let $\op{VC}_t(\Lambda,v_0,S)$ be the intersection of
$\op{VC}(\Lambda,v_0,S)$ with the ball $B(v_0,t)$. Under the assumption
that $S$ contains its circumcenter and that every one of its faces
contains it circumcenter, an explicit formula for the volume
$$\op{vol}(\op{VC}_t(\Lambda,v_0,S))$$ is calculated by means of
Lemma~\ref{XX} from the six quoins that form $\op{VC}(\Lambda,v_0,S)$.
This leads to the
following formula. Let $h_i = |v_i-v_0|/2$ and
$\eta_{ij}=\eta_V(v_0,v_i,v_j)$, and let $S_3$ be the group of
permutations of $\{1,2,3\}$ in
\begin{equation}
   \op{vol}\,\op{VC}_t(\Lambda,v_0,S) =
   \sol(S)/3 - \sum_{i=1}^3 \frac{\dih(v_i,S)}{2\pi}\op{vol}\,\op{cap_i}
   +\sum_{(i,j,k)\in S_3} \quo(R(h_i,\eta_{ij},t)).
   \tlabel{eqn:vol-theta-0}
\end{equation}


We extend Formula~\ref{eqn:vol-theta-0} by setting
    $$\quo(R(a,b,c)) = 0,$$
if the constraint $a < b < c$ fails to hold.  Similarly, set
$\op{vol}\,\op{cap}_i=0$ if $|v_i-v_0|\ge 2t$.  With these
conventions,  Formula~\ref{eqn:vol-theta-0} extends to all
simplices.  We write the extension of $\op{vol}\,\op{VC}_t(\Lambda,v,S)$
as
$$\op{vol}\,{\op{VC}^+_t}(\Lambda,v,S).$$


\subsection{score}
\tlabel{sec:ssc}

We show that the function $\sigma$ can be expressed as a sum over
terms attached to each of the standard regions.

\begin{definition} \tlabel{def:standard-cluster}
A {\it standard cluster\/} is a pair $(R,D)$ where $(\Lambda,v)$ is a
centered packing and $R$ is one of its standard regions.  A {\it
quad cluster\/} is the standard cluster obtained when the standard
region is a quadrilateral.
\end{definition}
%
 \index{cluster!standard}
 \index{cluster!quad}
 \index{quad cluster}

%Recall $|S|$ is the convex hull of a set $S\subset
%\ring{R}^3$.

We break $\sigma$ into a sum
   \begin{equation}
   \sigma(\Lambda,v) = \sum_R\,\sigma_R(\Lambda,v),
   \end{equation}
indexed by the standard clusters $(R,D)$.  Let
   $$
   \op{VC}_R(\Lambda,v) = \op{VC}(\Lambda,v)\cap \op{cone}(R),
   $$
whenever $R$ is a measurable subset of the unit sphere.  Let
   $$
   \CalQ(\Lambda,v,R) = \{Q\in \CalQ(\Lambda,v) : \op{conv}^0(Q)\subset \op{cone}(R)\}.
   $$
By Lemma~\ref{lemma:Q-in-region},
 each $Q$ is entirely contained in the cone over a single
standard region.

\begin{definition} \tlabel{def:score-std-region}
If $(v,V,E)$ is compatible with $\Lambda$, then each $Q\in\CalQ(\Lambda,v)$
lies in a uniquely determined component $R$ of $Y(v,V,E)$.
Let $(v,V,E)$ be a compatible fan and let $R$ be a
component of $Y(v,V,E)$.  Set
      $$
      \op{svR}(v,R,\Lambda,\lambda) =
      \op{sovo}(v,\op{VC}_R(\Lambda,v),\lambda).
      $$
Set
      $$
      \sigma(\Lambda,v,R) = \op{svR}(v,R,\Lambda,\lambda_{oct}) 
      + \lambda_{oct,v}
         \sum_{Q\in\CalQ(v,R,D)} A_1(Q,c(Q,D),v_0).
      $$
\index{vzorR@$\op{svR}$} \index{zzsigmaR@$\sigma$}
\end{definition}

\begin{lemma}\tlabel{lemma:sigma-sum}
Let $(v,V,E)$ be a compatible fan.
$\sigma(\Lambda,v) = \sum_R\sigma_R(\Lambda,v,R)$, where the sum runs
over all components of $Y(v,V,E)$.
\end{lemma}

\begin{proof}
   $$
   \begin{array}{lll}
      \sigma(\Lambda,v)
      &= \lambda_{oct,v} (\op{vol}\,\Omega(\Lambda,v) + A_0(\Lambda,v))+16\pi/3\\
      &= \lambda_{oct,v} (\op{vol}\,\op{VC}(\Lambda,v)+\sum_{Q\in\CalQ(\Lambda,v)}
         A_1(Q,c(Q,D),v_0)) + (4) (4\pi/3)\\
      &= \sum_R \left (\lambda_{oct,v} \op{vol}\,\op{VC}_R(\Lambda,v) 
         +\lambda_{oct,v}
         \sum_{Q\in\CalQ(R,D)} A_1(Q,c(Q,D),v_0) +
         \lambda_{oct,s}\sol(v,R)\right).
   \end{array}
   $$
\end{proof}

Also, we have
    \begin{equation}
    \op{svR}(\Lambda,v)=\sum_{R\subset Y(v,V,E)}
    \op{svR}(\Lambda,v,R).
    \tlabel{eqn:vorD}
    \end{equation}

\begin{lemma}\tlabel{lemma:R'}
Let $(v,V,E)$ be a compatible fan.  If $R\subset Y(v,V,E)$
is a component that is disjoint from $\op{conv}^0(Q)$ for all 
$Q\in\CalQ(\Lambda,v)$, then
   $$
   \sigma(\Lambda,v,R) = \op{svR}(\Lambda,v,R).
   $$
If, on the other hand, $R = \op{aff}_+^0(v,\{v_1,v_2,v_3\})$ for
some $\{v,v_1,v_2,v_3\}\in\CalQ(\Lambda,v)$, then
   $$  
   \sigma(\Lambda,v,R) =  \sigma(v,Q,c(Q,D)).
   $$
\end{lemma}

\begin{proof} Substitute the definition of $A_1$
(Equation~\ref{eqn:a1-sigma}) into the definition of
$\sigma_R(\Lambda,v)$, noting that $\op{VC}(\Lambda,v,Q) 
= \op{VC}_{R}(\Lambda,v)$,
\end{proof}

\begin{remark}   Lemma~\ref{lemma:R'} explains why we have chosen
the same symbol $\sigma$ for the functions $\sigma(\Lambda,v,R)$ and
$\sigma(Q,c,v)$.  We can view Lemma~\ref{lemma:R'} as asserting a
linear relation in the functions $\sigma$:
   $$\sigma(\Lambda,v,R) = \sigma(\Lambda,v,R') + \sum \sigma(v,Q,c).$$
The sum runs over $Q\in\CalQ(\Lambda,v)$ that lie in the cone over $R$.
\end{remark}

%\subsection{score}

By the results of Sections~\ref{x-2.7}, \ref{x-2.8}, \ref{x-2.9},
$\sigma(\Lambda,v)$ can be broken into a corresponding sum,
    $$
    \begin{array}{lll}
    \sigma_R(\Lambda,v) &= \sum_Q \sigma(Q) + \sigma(V_P),
                \hbox{ for quarters $Q$ in the $Q$-system, where}\\
    \sigma(V_P) &= \op{sovo}(\tildeV_P(t_0),\lambda_{oct})+  \sum_{\SA,\SB,\SC} \op{sovo}(V_S(t_S),\lambda_{oct})
        - \sum_v 4\doct\op{vol}(\delta_P(v)) - 4\doct\op{vol}(\delta'_P).\\
    \end{array}
    $$

By dropping the final term, $4\doct\op{vol}(\delta'_P)$, we obtain
an upper bound on $\sigma(V_P)$.  Because of the separation
results of Sections~\ref{x-2.7}--\ref{x-2.8},  we may score
$\tildeV_P(t_0)$ by Formula~\ref{eqn:3.7}. Bounds on the score of
simplices of type $\SB$ appear in \calc{193836552}.



\subsection{truncated tetrahedron}


We set
    \begin{equation}
    \begin{array}{lll}
    \op{sv}(S,t) &=
    \sol(S)\phi(t,t,\lambda_{oct})
    +\sum_{i=1,\ h_i\le t}^3 d_i (1-h_i/t) (\phi(h_i,t,\lambda_{oct})-
    \phi(t,t,\lambda_{oct})) \\
    &+\sum_{(i,j,k)\in S_3}
    \lambda_{oct,v}
    \quo(R(h_i,\eta(y_i,y_j,y_{k+3}),t)).
    \tlabel{eqn:3.5}
    \end{array}
    \end{equation}
In the definition, we adopt the convention that $\quo(R)=0$, if
$R=R(a,b,c)$ does not exist (that is, if the condition
    $0< a\le b\le c$
is violated). In the second sum, $S_3$ is the set of permutations
on three letters. This definition is compatible with
Definition~\ref{def:svor}.

We have
    \begin{equation}
    \begin{array}{lll}
    \op{svR}(v,P,\Lambda,t) &=
    \sol(P)\phi(t,t)
    +\sum_{|v_i-v_0|\le 2t} d_i (1-|v_i-v_0|/(2t)) (\phi(|v_i-v_0|/2,t)-
    \phi(t,t)) \\
    &-\sum_{R} 4\doct \quo(R).
    \tlabel{eqn:3.7}
    \end{array}
    \end{equation}
The first sum runs over vertices in $P$ of height at most $2t$.
The second sum runs over Rogers simplices $R(|v_i-v_0|/2,\eta(F),t)$
in $P$, where $F=\{v_0,v_1,v_2\}$ is a face of circumradius
$\eta(F)$ at most $t$, formed by vertices in $P$.  The constant
$d_i$ is the total dihedral angle along $\{v_0,v_i\}$ of the
standard cluster. The truncations $t=t_0=1.255$ and $t=\sqrt2$
will be of particular importance.
    Set $A(h) = (1-h/t_0) (\phi(h,t_0)-\phi(t_0,t_0))$.\index{A}

\begin{remark}  We have introduced both untruncated and truncated
versions of functions $\op{svR}$ and $\sigma$.  The truncated versions
are used to give upper bounds on the untruncated versions.  For
example,  in the function $\sigma(\Lambda,v)$, the $V$-cell contributes
through its volume.  The volume
appears with a negative coefficient 
$\lambda_v=-4\doct$.  Thus, we obtain an
upper bound on $\sigma(\Lambda,v)$ by discarding bits of volume from the
$V$-cell.   This suggests that we might try to give upper bounds
on the score $\sigma(\Lambda,v)$ by truncating the $V$-cell in various
ways. This is the reason for the truncated versions of these
functions.
\end{remark}




\subsection{squander}% DCG 10.1, p99
    %\heads{3. Functions}

%Set $\zeta^{-1}:=\sol(S(2,2,2,2,2,2))=2\arctan(\sqrt{2}/5)$. The
%constant $\zeta$ is related to the other fundamental constants by the
%relations $\pt= 2/\zeta-\pi/3$ and $\doct=(\pi-2/\zeta)/\sqrt{8}$.
%Rogers's bound is $\sqrt{2}/\zeta\approx 0.7796$.

% Plus Formula 7 on scores.

We consider the functions
    $\sigma_R(\Lambda,v)-\lambda\zeta\sol(R)\,\pt$,
for $\lambda=0$, $1$, or $3.2$, where $R$ is a standard cluster.
%The constant $3.2$ was determined experimentally.
We write
    $$
    \tau_R(\Lambda,v) = \sol(R)\zeta\,\pt -
    \sigma_R(\Lambda,v).
    $$
We will see that $\tau_R(\Lambda,v)$ has a simple interpretation.  If $(\Lambda,v)$
is a centered packing with standard clusters $\{R\}$, set $\tau(\Lambda,v)
= \sum_{R}\tau_R(\Lambda,v)$.
\smallskip



\begin{lemma}\tlabel{lemma:sigma-tau}
    %\proclaim{Lemma 3.2}
    $$\sigma(\Lambda,v) = {4\pi \zeta\,\pt} - \tau(\Lambda,v).$$
\end{lemma}

\begin{proof} Let $\{R\}$ be the standard clusters in $(\Lambda,v)$. Then
    $$
    \sigma(\Lambda,v) = \sum_R\sigma_{R_i}(\Lambda,v) +
        (4\pi-\sum_R\sol(R_i))\zeta\,\pt = 4\pi \zeta\,\pt - \sum_R\tau_{R_i}(\Lambda,v).
    $$
\end{proof}


\begin{lemma}\tlabel{lemma:squander-contravene}
If there are standard clusters $R_1,\ldots,R_k$ such that
$$\sum_{i=1}^k \tau_{R_i}(\Lambda,v)> \squander,$$
then $(\Lambda,v)$ does not contravene.
\end{lemma}

\begin{proof}
$$\sigma(\Lambda,v) = 4\pi\zeta,\pt -\sum_R{R_i}(\Lambda,v) < 8\,\pt.$$
\end{proof}

We note that $14.8\,\pt > \squander$.  We sometimes use this
approximation.


The function $\tau_R(\Lambda,v)$ gives the amount {\it squandered\/} by a
particular standard cluster $R$.  If nothing is squandered, then
$\tau_{R_i}(\Lambda,v)=0$ for every standard cluster, and the upper bound
on $\sigma(\Lambda,v)$ is
    $4\pi\zeta\,\pt\approx 22.8\,\pt$.



%% XX Deleting mentioned \calc{629256313},
%% \calc{917032944}, \calc{738318844}, and \calc{587618947}.
%% Perhaps they are no longer needed in the proof!

%% Major Deletion: SVN:16 has proof of local optimality. Gone in SVN:23.






\subsection{misplaced}

%% WAS IN FORMULATION. DOESN'T BELONG IN TARSKI.

We conclude with the proof of the main theorem of the \chap.

\begin{proof} {\bf (Theorem~\ref{thm:nonoverlap})}
The rules defining the $Q$-system specify a uniquely determined
set of simplices.  The proof that their interiors are pairwise
disjoint is established by the preceding series of lemmas.
Lemma~\ref{tarski:qrtet-over} shows that the interiors of
quasi-regular tetrahedra do not meet the interiors of other
simplices in the $Q$-system. Lemma~\ref{tarski:oct-over} shows that
the quarters in quartered octahedra are well-behaved.
Lemma~\ref{tarski:adj-over} shows that the interiors of other
quarters in adjacent pairs are disjoint from the interiors of
other simplices in the $Q$-system. Finally, we treat isolated
quarters in Lemma~\ref{lemma:iso-over}. These cases cover all
possibilities since every simplex in the $Q$-system is a
quasi-regular tetrahedron or strict quarter, and every strict
quarter is either part of an adjacent pair or isolated.
\end{proof}


\begin{definition} \label{def:height}  Let $\Lambda$ be a
packing.  Assume that the coordinate system is fixed in
such a way that the origin is a vertex of the packing.  The {\it
height\/} of a vertex is its distance from the origin.
%
 \index{height}
\end{definition}

\begin{definition} \label{def:enclosed}\index{enclosed}
We say that a vertex is {\it enclosed\/} over a figure if it lies
in the interior of the cone at the origin generated by the figure.
%
 \index{vertex!enclosed}\index{enclosed}
\end{definition}

\begin{definition}\label{def:dih}
In general, let $\dih(S)$ be the dihedral angle of a simplex $S$
along its first edge. When we write a simplex in terms of its
vertices $(w_1,w_2,w_3,w_4)$, then $\{w_1,w_2\}$ is understood to
be the first edge.
%
 \index{dih (dihedral angle)}
\end{definition}


Our simplices are generally assumed to come labeled with a
distinguished vertex, fixed  at the origin. (The origin will
always be at a vertex of the packing.) We number the edges of each
simplex $1,\ldots,6$, so that edges $1$, $2$, and $3$ meet at the
origin, and the edges $i$ and $i+3$ are opposite, for $i=1,2,3$.
$S(y_1,y_2,\ldots,y_6)$ denotes a simplex whose edges have lengths
$y_i$, indexed in this way. We refer to the endpoints away from
the origin of the first, second, and third edges as the first,
second, and third vertices.
%
 \index{labels!edge}
 \index{first!edge}



