\chapter{Partitioning Cells}%DCG "The S-System" Sec.9, p85
    \label{sec:fine}
    \oldlabel{2}

\section{Interaction of cells with the Q-system}

We study the structure of the $V$-cell of a centered packing $(v_0,\Lambda)$.
Let $\CalQ$ be the set of simplices
in the $Q$-system.  For $v\in\Lambda$, let $\CalQ(v,\Lambda)$ be the subset
of those with a vertex at $v$.\index{Qv@$\CalQ(v,\Lambda)$}



\begin{lemma} \label{lemma:voronoi-truncation-over-Q}
If $x$ lies in the \index{Voronoi cell} Voronoi cell at $v_0$, 
but not in the $V$-cell at $v_0$, then there exists a
simplex $Q\in\CalQ(v,\Lambda)$, such that $x$ lies in the cone (at $v$)
over $Q$. Moreover, $x$ does not lie in the interior of $Q$.
\end{lemma}

\begin{proof}
By
the definition of $V$-cell, there is a barrier $\{v_1,v_2,v_3\}$
that meets $\op{conv}^0(v,x)$.  
By Lemma~\ref{tarski:vor-bar-tet}
and Lemma~\ref{tarski:vor-bar-quad},
the simplex $Q=\{v,v_1,v_2,v_3\}$ is a quasi-regular tetrahedron
or a flat quarter.  If it is a flat quarter then it shares a diagonal
with a barrier coming from the $Q$-system.  Thus, $Q$ is in
the $Q$-system too.  By Lemma~\ref{tarski:pass-cone}, the
point $x$ lies in the cone over $Q$.

If $x\in\op{conv}^0(Q)$, then the barrier, which is a face
of $Q$, does not separate
$x$ from $v$.  The rest is clear.
\end{proof}



\begin{lemma}\label{lemma:VC-Omega}
Inside the ball of radius $t_0$ at $v_0$, the $V$-cell and
Voronoi cell coincide:
   $$B(v,t_0)\cap \op{VC}(v) \equiv B(v,t_0)\cap \Omega(v_0).$$
That is, they are equal up to a null set.
\end{lemma}

\begin{proof} Let $x\in B(v,t_0)\cap \op{VC}(v)\cap\Omega(w)$, where
$w\ne v$.  
% By Lemma~\ref{lemma:unobstr-t0}, 
The point $v_0$ is
unobstructed  at $x$ by Lemma~\ref{lemma:unobstr-t0}.  
Thus, $|x-w|< |x-v|\le t_0$.  By
Lemma~\ref{lemma:unobstr-t0} again, $w$ is unobstructed at $x$, so
that $x\in \op{VC}(w)$, contrary to the assumption
$x\in\op{VC}(v)$.  Thus $B(v,t_0)\cap\op{VC}(v)\subset\Omega(v)$.
Similarly, if $x\in B(v,t_0)\cap \Omega(v)$, then $x$ is
unobstructed at $v_0$, and $x\in \op{VC}(v)$.
\end{proof}

\bigskip

%\begin{remark} The next lemma helps to determine which $V$-cell
%a given point $x$ belongs to.  If $x$ lies in the open cone over a
%simplex $Q_0$ in $\CalQ$, then Lemma~\ref{lemma:Q-divide}
%describes the $V$-cell decomposition inside $Q$;  beyond $Q$ the
%point $v_0$ is obstructed by a face of $Q$, so that such $x$ do not lie
%in the $V$-cell at $v_0$. 
%%If $x$ does not lie in the open cone over
%%a simplex in $\CalQ$, but lies in the open cone over a standard
%region $R$, then Lemma~\ref{lemma:V-cell-local} describes the
%$V$-cell.  
%It states in particular, that for unobstructed $x$, it
%can be determined whether $x$ belongs to the $V$-cell at the
%point $v_0$ by considering only the vertices $w$ that lie in the closed
%cone over $R$ (the standard region containing the radial
%projection of $x$). In this sense, the intersection of a $V$-cell
%with the open cone over $R$ is {\it local\/} to the cone over $R$.
%\end{remark}
%


Let $\CalB_0'$ be the set of triangles $T$ such that at least one
of the following holds:
\begin{itemize}
    \item $T$ is a barrier at $v_0$, or
    \item $T=\{v_0,v,w\}$ consists of a diagonal of a quarter in the
    $Q$-system together with one of its anchors.
\end{itemize}

%DCG Lemma 5.29, page 50.
\begin{lemma} [Decoupling Lemma]\label{lemma:V-cell-local}
%Let $x\in I_0$, the cube of side $4$ centered at $v_0$
%parallel to coordinate axes.  
Assume that the closed segment
$\{x,w\}$ intersects the closed $2$-dimensional cone with center
$v_0$ over $F=\{v_0,v_1,v_2\}$, where $F\in\CalB'_0$. Assume that $v_0$ 
is not obstructed at $x$. Assume that $x$ is closer to $v_0$ 
than to both $v_1$ and $v_2$. Then $x\not\in\op{VC}(w)$.
\end{lemma}
\index{decoupling lemma}

%\begin{remark}  The Decoupling Lemma is a crucial result.  It
%permits estimates of the scoring function in
%\Chap~\ref{sec:scoring} to be made separately for each standard
%region.  The estimates for separate standard regions are far
%easier to come by than estimates for the score of the full
%centered packing.  Eventually, the separate estimates for each
%standard will be reassembled with linear programming techniques in
%\Chap~\ref{XX}.
%\end{remark}

\begin{proof}
Assume for a contradiction that $x$ lies in $\op{VC}(w)$. In
particular, we assume that $w$ is not obstructed at $x$.  Since
$v_0$ is not obstructed at $x$, $w$ must be closer to $x$
than $x$ is to $v_0$.

By Lemma~\ref{tarski:decouple}, 
   $|w|,|w-v_1|,|w-v_2|\le 2t_0$.  Thus, $Q=\{v_0,w,v_1,v_2\}$ is
a quarter or a quasi-regular tetrahedron.  By the definition of
$\CalB'_0$, the face $F$ must lie in the $Q$-system.  Thus,
$F$ is a barrier.  Again, by Lemma~\ref{tarski:decouple},
$\op{conv}(F)$ meets the segment from $x$ to $w$, so $x$ is obstructed
at $w$.  Thus, $x$ does not lie in $\op{VC}(w)$.
\end{proof}

%\section{Local Optimality}%DCG Sec. 8, p72  
%\label{sec:local-opt}
%Moved after the main estimate.




\section{Fitted Crowns}%DCG 9.1, p95
    \label{sec:fine-overview}
    \oldlabel{2.1}



In this \chap, we define a decomposition of a $V$-cell. Let
$\op{VC}$ be the $V$-cell at $v_0$.  For any $t > 0$, let
$V(t)$ be the intersection of $\op{VC}$ with the ball $B(v_0,t)$ at
$v_0$ of radius $t$. We write $\op{VC}$ as the disjoint union
of $V(t_0)$ and its complement $\delta$.

Assume that there is an upright 
diagonal $\{v_0,v\}$. 
We will define $\delta_i(v)\subset\delta$, where $i$ runs
over some finite indexing set.
The sets $\delta_i(v)$ will be defined so as not to overlap
one another. 

We will define a set $\CalS$ of simplices, each having a vertex at
$v_0$. (The letter `$\CalS$' is for simplex.)
The vertices of the simplices will be vertices of the
packing, and their edges will have length at most $2\sqrt{2}$. The
sets $\op{cone}^0(v_0,S)$, for distinct $S\in\CalS$, will not overlap. Over a
simplex $S\in\CalS$, the $V$-cell will be truncated at a radius
$t_S\ge t_0$. After defining the constants $t_S$, we will set
    $$V_S(t_S) = \op{cone}^0(v_0,S) \cap V(t_S) =\op{cone}^0(S)\cap B(t_S)\cap \op{VC}(v_0).$$
That is, $V_S(t_S)$ is the part of the $V$-cell at $v_0$,
contained in the cone over $S$ and in the ball of radius $t_S$.
If
    $\op{VC}(v_0) \cap \op{cone}^0(S)\subset B(t_S)\subset B(t'_S)$,
then
    $V_S(t_S)=V_S(t'_S)$.

Since $t_S\ge t_0$, the sets $V_S(t_S)$ and $\delta$ may meet at
interior points. Nevertheless, we will show that $V_S(t_S)$ does
not meet the interior of any $\delta(v)$.  Let $\tildeV(t_0)$ be
the set of points in $V(t_0)$ that do not lie in $\op{cone}^0(S)$,
$S\in\CalS$. We will derive an explicit formula for the volume of
$\tildeV(t_0)$.

In $\op{VC}(v_0)$, there are nonoverlapping sets
$$\delta(v),\quad   V_S(t_S),\quad \tildeV(t_0).$$
Let $\delta'$ be the complement in $\op{VC}(v_0)$ of the union of
these sets. These sets give a decomposition of $\op{VC}(v_0)$.
Corresponding to this decomposition is a formula for $\sigma(v,\Lambda)$
of the form
    $$
    \sigma(v,\Lambda) =
        \op{sovo}(\tildeV(t_0),\lambda_{oct})
        + \sum_{\CalS}\op{sovo}(V_S(t_S),\lambda_{oct})
        -\sum_{v,i} 4\doct\op{vol}(\delta_i(v))
        -4\doct\op{vol}(\delta').
    $$
Since $\op{vol}(\delta')\ge0$, we obtain an upper bound on
$\sigma(v,\Lambda)$ by dropping the rightmost term.



\subsection{definition}%DCG 9.2, p86
    \label{sec:deltaP}
    \oldlabel{2.3}



\begin{definition}\label{def:eta0}
Set $\eta_0(h)=\eta(2h,2,2t_0)$.
\index{zzeta@$\eta_0$}
\end{definition}

%By Lemma~\ref{tarski:1453}, 
%if $h\le\sqrt2$, then $\eta_0(h)\le \eta_0(\sqrt2) <
%1.453$.\index{ZZZZ1.453@1.453}



Let $v$ be a vertex with $2t_0 < |v| <\sqrt8$.
Let $D_0 = \op{rcone}^0(v_0,v,|v|/(2\eta_0(|v|/2)))$.
Let $v_1,\ldots,v_k$ be the anchors around $\{v_0,v\}$ indexed
cyclically. The half planes $A_i=\op{aff}_+(\{v_0,v\},v_i)$
slice $\ring{R}^3$ into $k$ open wedges
$W_i$, between
    $A_i$ and $A_j$,
where $j\equiv i+1\mod k$, so that
    $D\ring{R}^3\setminus (A_1\cup\cdots\cup A_k) =\cup W_i$.

\begin{definition}\label{def:wedge}
Let $\CalW=\CalW(v_0,v)$ be the set of wedges $W=W_i$ along $(v_0,v)$ 
such that either
\begin{enumerate}
    \item The azimuth angle of $W$ (along $\{v_0,v\}$) is at least $\pi$.
    \item The azimuth angle of $W$ is less than $\pi$, 
 $|v_i-v_j|\ge 2.77$,
    $\rad(v_0,v,v_i,v_j)\ge\eta_0(|v|/2)$, and the
    circumradius of $\{v_0,v_i,v_j\}$ or $\{v,v_i,v_j\}$ is
    $\ge\sqrt2$.
    \label{enum:wedge2}
\end{enumerate}
\index{wedge}
\index{W@$\CalW$}
\end{definition}

(If $(v_0,v)$ has only one or two anchors, then it is understood
 that $W$ is all of $\ring{R}^3$ or all of $\ring{R}^3$, except a half-plane.)
Fix $i,j$, with $j\equiv i+1\mod k$. If $W = W_i$ is a wedge in
$\CalW$, let $\{v_0,v_i,v\}^\perp$ be the plane through $v_0$ and
the circumcenter of $\{v_0,v_i,v\}$, perpendicular to $\{v_0,v_i,v\}$.
Skip the following step if the circumradius of $\{v_0,v_i,v\}$ is
greater than $\eta_0(|v|/2)$, but if the circumradius is at most
this bound, the plane
of $\{v_0,v_i,v\}^\perp$
intersects the right circular cone boundary of $D_0$ along two rays
emanating from $v_0$.  Let $c_i$ be a point on the ray (selected on
the $W$-side of $\{v_0,v_i,v\}$).  Simlarly, we construct the point
$c_i'$ for $\{v_0,v,v_j\}$ (again on the $W$-side).

Define $\theta=\theta(v)$ by $\cos\theta = |v|/(2\eta_0(|v|/2))$.
If Condition~2 holds, we let $c$ be the 
circumcenter of $\{v_0,v_i,v_j,v\}$.  The angle
at $v_0$ between $c$ and $v$ is
$\theta'$, where
$$\cos\theta' = |v|/(2\rad)\le |v|/(2\eta_0) = \cos\theta.$$
We conclude that $\theta'\ge\theta$ and $c$ does not lie in $D_0$.
Thus, the half-planes
   $$
   A_i,\quad B_i=\op{aff}_+(\{v_0,v\},c_i),\quad 
   B'_i = \op{aff}_+(\{v_0,v\},c'_i), \quad
   A_j
   $$
are ordered cyclically around $\{v_0,v\}$. (Set $B_i=A_i$
or $B'_i=A_j$, if the corresponding circumradius is greater than
$\eta_0(|v|/2)$.)
Let $W'=W'_i$ be the open wedge of $D_0$ between $B_i$ and $B'_i$.
Let
    $$E_w = \{x : 2 x\cdot w \le w\cdot w\},$$
for $w = v,v_i,v_j$. These are half-spaces bounding the Voronoi
cell. Set $E_\ell = E_{v_\ell}$.

\begin{definition} \label{def:delta-e}
In both cases (Conditions~1 and~2), let $W$ be the wedge between
$v_i$ and $v_j$ along $(v_0,v)$, $W'$ the smaller wedge, 
and let $c=\eta_0(|v|/2)$ in
    $$
    \begin{array}{lll}
      \BigD'(v,W) &= [E_v\cap W'\cap D_0]\cup \op{rog}^0(v_0,v,v_i,p_i,c)
      \cup \op{rog}^0(v_0,v,v_j,p_j,c)
      \\
    \BigD(v,W) &= \BigD'(v,W) \cup 
    \op{rog}^0(v_0,v_i,v,p_i,c)
    \cup \op{rog}^0(v_0,v_j,v,p_j,c)\\
    \bigd(v,W) &= \{x\in\BigD(v,W)\mid |x|>t_0\},
    \end{array}
    $$
where $p_i,p_j$ are selected so that the simplices $\op{rog}^0$ lie
in the wedge $W$.
\index{zzdelta@$\bigd(v,W)$}
\index{zzDelta@$\BigD(v,W)$}
\end{definition}

\begin{remark} We note that the union is actually a disjoint union,
and that each of the pieces is one of the primitive regions, so
the volume of $\BigD$ is immediate.
\end{remark}

\begin{figure}[htb]
  \centering
  \myincludegraphics{\ps/diag46.ps}
  \caption{$\BigD(v,W)$ lies in a cone. The intersection of
    that cone with the unit sphere is the shaded region.}
  \label{fig:anchor-quarter}
\end{figure}

\begin{remark}
Recall that in the definition of Rogers simplex (Definition~\ref{def:rog}, 
it is defined to be the
empty set 
if the corresponding parameters are not coherent.  It is to
be understood that the all discussion regarding this set
can (and should) be disregarded in the case that it is empty.
\end{remark}

We present a series of lemmas that explore the geometry of the sets
$\BigD(v,W)$.



\begin{lemma}\label{lemma:new-anchor}
  Let $S=\{v_0,v,w,u\}$ be a simplex.  Assume that $\{v_0,v\}$ is an
upright diagonal, that $w$ and $u$
are anchors of $\{v_0,v\}$, and that $\rad(S)< \eta_0(|v|/2)$.
Assume there is a wedge $W$ of $\CalW$ along the face $\{v_0,v,w\}$
(on the same side of the face as $u$).  
Then there exists an anchor $w'$ of $\{v_0,v\}$ between $u$ and $w$
(that is, in $\op{aff}_+^0(\{v_0,v\},\{u,w\})$) with
    $|w'-w|\ge2.77$ and
    $\rad\{v_0,w,w',v\} \ge \eta_0(|v|/2)$.
\end{lemma}

\begin{proof} The conditions on $S$ are incompatible with the
conditions of Definition~\ref{def:wedge} defining wedges.
Therefore, $w$ and $u$ cannot be consecutive anchors around
$\{v_0,v\}$.  Let $w'$ be the anchor such that $S'=\{v_0,v,w,w'\}$
determine the wedge $W$.  Since $w$ and $w'$ are consecutive anchors,
we see that $w'$ must be between $u$ and $w$.  It must have the
second type in Definition~\ref{def:wedge}.  The conclusion follows.
\end{proof}






In the following lemmas, we adopt a uniform notation.
$\{v_0,v\}$ is always be an upright diagonal: $|v|<\sqrt8$.
$W$ is  a wedge along $(v_0,v)$
between two anchors $w$ and $w'$.  Set $R_w=\op{rog}^0(v_0,w,v,p,\eta_0(|v|/2))$,
where $p$ is selected so that $R_w$ lies in $W$.


The following definition will be used briefly, then discarded.
Its purpose is to group together a few separate cases in the next
few lemmas.

\begin{definition}  Let $(v_0,v,u_1,u_2)$ be a four-tuple of vertices
in $\ring{R}^3$.  We say that it is {\it normal} if $2t_0<|v|<\sqrt8$
and one of the
following holds:
\begin{enumerate}
  \item (qrt) $\{v_0,u_1,u_2\}$ is a quasi-regular triangle.
  \item (upright) $\{v_0,u_1,u_2\}$ is an upright triangle; that is, say,
    $2t_0 < |u_1| < \sqrt8$, and $u_2$ is an anchor of $u_1$.
  \item (flat)
   $2t_0<|u_1-u_2|<\sqrt8$; and if $u_1,u_2$ are both anchors $(v_0,v)$, 
   then
    no further anchor of $\{v_0,v\}$
   lies in the lune $\op{aff}^0_+(\{v_0,v\},\{u_1,u_2\})$.
\end{enumerate}
\end{definition}

\begin{lemma}\label{lemma:BigD-}  Let $S=(v_0,v,u_1,u_2)$ be normal.
Then
$\BigD'(v,W)$ does not meet $F=\op{cone}(v_0,\{u_1,u_2\})$.
\end{lemma}

\begin{proof}  By elementary geometry\tarf{tarski:eps-bigd-}, 
the hypotheses imply
that $S$ is normal of the flat or qrt variety, and 
that $u_1$ and $u_2$ are anchors.  The lemma also
gives that 
$\rad(S)<\eta_0(|v|/2)$.    
In the qrt case, $S$ forms
an upright quarter.  By elementary geometry\tarf{tarski:consec-anchors}, $u_1$ and $u_2$
are consecutive anchors. In the flat case as well, 
$u_1$ and $u_2$ are consective
anchors around $\{v_0,v\}$.

If the wedge $W$ is not between
the anchors $u_1$ and $u_2$, then $\BigD$ lies outside the 
lune $\op{aff}_+^0(\{v_0,v_0\},\{u_1,u_2\})$, but $F$ lies inside it.  Thus,
the wedge must run between $u_1$ and $u_2$.   This contradicts 
the rules for forming wedges.  (See
Lemma~\ref{lemma:new-anchor}, which implies that
anchors $u_1$ and $u_2$ cannot be consecutive.)
\end{proof}

\begin{lemma}\label{lemma:fine-Rw}
Let $S=(v_0,v,w,u_1)$ be normal.
Suppose that $w$ is an anchor of $\{v_0,v\}$.
Then
$R_w$
does not meet $F=\op{cone}(v_0,\{u_1,u_2\})$.
\end{lemma}

\begin{proof}  We can use the same proof as in Lemma~\ref{lemma:BigD-},
except with one lemma\tarf{tarski:eps:fine:Rw} substituted for 
another\tarf{tarski:eps-bigd-}.
\end{proof}

\begin{lemma}\label{lemma:fine:barrier}  
Let $S=\{v_0,v,w,u\}$ be a set of four distinct vertices
in the packing.  Assume that $\{v_0,v\}$ is an upright diagonal of
a quarter in the $Q$-system and that $w$ is an anchor of $\{v_0,v\}$.
Let $R_w$ be as above; that is, a Rogers simplex attached to a wedge
$W\in \CalW$ around the diagonal $\{v_0,v\}$.
Assume $x\in R_w$ satisfies $\epsilon_0(x,\{v,w,u\})= u$.
Then there exists a vertex $w'$ that is an anchor of $\{v_0,v\}$ such
$b=\{v,w',v_0\}$ is a barrier and $x$ is obstructed from $u$
by $b$.
\end{lemma}

\begin{proof} Assume for a contradiction that $\epsilon_0(x)=u$.
We separate the proof into two cases, depending on whether
$x$ and $u$ lie on the same side of $A=\op{aff}(v,w,v_0)$.

Assume that $x$ and $u$ lie on opposite sides of $A$.
For any nonzero vertices $v',v''$,
Let $L(v',v'')$ be the line of points equidistant from $\{v_0,v',v''\}$.
The three lines $L(u,v)$, $L(v,w)$, $L(u,w)$ meet at the circumcenter
$c$ of $S$.  The rays $L^+(u,v)$, $L^+(v,w)$, $L^+(u,w)$ demarcate
the regions between $\epsilon_0=w,u,v$.  (Pick the direction
of ray so that it runs through the circumcenter of the face $\{v_0,v',v''\}$
if the remaining vertex has positive orientation, and so that it
runs in the opposite direction otherwise.)

If $u$ has positive orientation in $S$, then $L^+(v,w)$ runs through
the circumcenter of $\{v_0,v,w\}$ and along an edge of $R_w$.
The point $w/2$ in the closure of  $R_w$ also has $\epsilon_0=w$.
It follows that $\epsilon_0$ has value $w$ on $R_w$, which is contrary
to our assumption.

Thus, $u$ has negative orientation in $S$.  
This implies that $|u-v|,|u-w|,|u|\le 2.51$.  In particular,
$S$ is a quarter.  Since it has the same diagonal as a quarter in
the $Q$-system, we have that $S$ is in the $Q$-system, so that
$\{v_0,v,w\}$ is a barrier.  By Lemma~\ref{tarski:tip-cone}, we have
that $x\in \op{cone}(u,\{v,w,v_0\})$.  In particular, $x$ is obstructed
from $u$ by the barrier $\{v,w,v_0\}$.  Take $w'=w$ in this case.

Now assume that $x$ and $u$ lie on the same side of $A$.
First consider the special case where
we also have that $\rad_V(S)\ge \eta_0(|v|/2)$ and that that
the orientation of $u$ is non-positive in $S$.  In this case, it
follows that $S$ is a quarter in the $Q$-system.  By the rule
for constructing wedges $W\in\CalW$, there is no $R_w$ along $\{v_0,v,w\}$
in this case.  Next consider the case where
$\rad_V(S)\ge \eta_0(|v|/2)$ and the orientation of $u$ is positive
in $S$.  In this case, the ray $L^+(v,w)$ runs along the edge of 
$R_w$ as before, and we see that $\epsilon_0=w$ on $R_w$.

Finally consider the case where $\rad_V(S)<\eta_0(|v|/2)$.
It follows that $u$ is an anchor of $S$.
By Lemma~\ref{lemma:new-anchor}, there exists a further anchor
$w'$ between $u$ and $w$.  We are in the situation of Lemma~\ref{lemma:prev}.
We have that $\op{conv}\{x,u\}$ meets $\op{conv}\{v_0,v,w'\}$ and
that $|u-w'|\le 2.51$.  In particular $S'=\{v_0,v,w',u\}$ is
an upright quarter and $\{v_0,v,w'\}$ is a barrier.
\end{proof}



\begin{lemma}\label{lemma:fine-Rw:5}
Let $\{v_0,v,w,u_1,u_2\}$ be a set of five distinct vertices in the
packing.  Assume that $\{v_0,v\}$ is an upright diagonal of a quarter
in the $Q$-system and that $w$ is an anchor of $\{v_0,v\}$.
Assume that $S=(v_0,v,u_1,u_2)$ be normal.
Let $R_w$ be as above; that is, a Rogers simplex attached to a wedge
$W\in \CalW$ around the diagonal $\{v_0,v\}$.
Then $R_w$ does not meet $F=\op{cone}(v_0,\{u_1,u_2\})$.
\end{lemma}

\begin{proof}
For a contradiction, assume these sets meet at $x\in R_w\cap F$.
We have $\epsilon_0=\epsilon_0(x,\{w,u_1,u_2\})\in\{w,u_1,u_2\}$.  We consider
two cases depending on whether $\epsilon_0=w$.

Assume that $\epsilon_0=w$.  By Lemma~\ref{tarski:fine:Rw:5},
we have $|w-u_1|\le 2.51$ and $|w-u_2|\le 2.51$.  By the same
lemma, 
there are three possibilities.  The first is that
for either $u=u_1$ or $u=u_2$, we have that $(v_0,v,w,u)$ is normal
and that $R_w$ meets $\op{cone}(v_0,\{w,u\})$.  This is contrary to
Lemma~\ref{lemma:fine-Rw:5}.
The second possibility is that
$2.51<|u_1-u_2|<\sqrt8$ and that $v\in\op{cone}^0(v_0,\{u_1,u_2,w\})$
with $u_1,u_2,w$ all anchors of $\{v_0,v\}$.  By the normality
hypothesis, these are the only anchors of $\{v_0,v\}$.  None of the
corresponding wedges $W$ satisfy the conditions to belong to
a wedge of $\CalW$ around $\{v_0,v\}$.  Thus, this case does not
occur. The third possibility is that $w\in \op{aff}_+(\{v_0,v\},\{u_1,u_2\})$
and $2.51<|u_1-u_2|$.  This is contrary to the normality
condition on $S$.

In the remaining case, we have $\epsilon_0=u\in\{u_1,u_2\}$.  
Let $S'=\{v_0,v,w,u\}$.  By Lemma~\ref{lemma:fine:barrier}, there
exist a barrier $b=\{v_0,v,w'\}$ such that $x$ is obstructed from
$u$ by $b$.  However, if $x\in\op{cone}(v_0,\{u_1,u_2\})$, then
there exists no such obstruction.  Thus, the intersection
is empty.
\end{proof}

\begin{lemma}\label{lemma:delta-tri}
Let $F=\{v_0,u_1,u_2\}$ be a quasi-regular triangle.  Let $\{v_0,v\}$ be
an diagonal. Assume that there
exists a quarter in the $Q$-system along $\{v_0,v\}$.  Then
$\BigD(v,W)$ does not meet $C=\op{cone}(v_0,\{u_1,u_2\})$.
\end{lemma}

\begin{proof}
We have separated $\BigD^-(v,W)$ and $R_w$ from $C$.
Together they form $\BigD(v,W)$.
\end{proof}






\begin{lemma}\label{lemma:delta-flat}
Let $F=\{v_0,u_1,u_2\}$ be a triangle.  Assume that $|u_1|\le 2t_0$,
$|u_2|\le 2t_0$, and $2t_0\le|u_1-u_2|\le\sqrt8$.  Let $\{v_0,v\}$
be the diagonal of an upright quarter in the $Q$-system.  Assume
that if $u_1$ and $u_2$ are both anchors of $v$, then they are
consecutive anchors around $v$. Under these conditions, the set
$\BigD(v,W)$ does not overlap the cone at $v_0$ over the triangle
$F$.
\end{lemma}

\begin{proof} The proof is identical to that of
Lemma~\ref{lemma:delta-tri}. 
\end{proof}

\begin{lemma}\label{lemma:delta-upright}
Let $F=\{v_0,u_1,u_2\}$ be a triangle.  Assume that $2t_0\le|u_1|\le
\sqrt8$, $2\le|u_2|\le 2t_0$, and $2\le|u_1-u_2|\le2t_0$.  Let
$\{v_0,v\}$ be the diagonal of an upright quarter in the $Q$-system.
Under these conditions, the set $\BigD(v,W)$ does not overlap
the cone at $v_0$ over the triangle $F$.
\end{lemma}

\begin{proof}
The proof is identical to that of Lemma~\ref{lemma:delta-tri}.
\end{proof}


\begin{lemma}
Let $\{v_0,v\}$ be an upright diagonal of a quarter in the
$Q$-system.   If $x$ lies in the interior of $\BigD(v,W)$,
then $x$ is unobstructed at $v_0$.
\end{lemma}

\begin{proof} For a contradiction, assume that $x$ is obstructed
at $v_0$ by barrier $T =\{u_1,u_2,u_3\}$.


The convex hull of $T$ can be partitioned into three sets $T(i)$
depending on which vertex of $T$ is closest to a given point in
the convex hull. (Ties can be resolved in any consistent manner.)
Let $y\in \BigD(v,W)$ be the point in the convex hull of $T$ on
the segment from $v_0$ to $x$.  Fix $i$ so that $y\in T(i)$. If
$v=u_i$, then each point $y$ of $T(i)$ is closer to $v$ than to
$v_0$.  But each point of $\BigD(v,W)$ is closer to $v_0$ than to
$v$.  So $x$ is not obstructed by $T$ at $v_0$.

We may now assume that $v\ne u_i$.

Partition $\ring{R}^3$ geometrically into three sets $V(u_i)$,
$V(v_0)$, $V(v)$ according to which of $\{u_i,v_0,v\}$ a point
$z\in\ring{R}^3$ is closest to.  (Again resolve ties in any
consistent manner.)

Assume further that $\max_j u_j \ge 2t_0$. This implies that $y\in
T(i) \subset V(v) \cup V(u_i)$.  On the other hand, we have by
construction that $y\in \BigD(v,W) \subset V(v_0)$.  (There are
two cases involved in this conclusion, depending on whether $u_i$
is an anchor of $\{v_0,v\}$.)  However, the sets $V(\cdot)$ are
disjoint; and we reach a contradiction.  Thus, under these
assumptions, $x$ is unobstructed at $v_0$.

Next assume that $\max_j u_j < 2t_0$.  Let $S=\{v_0,u_1,u_2,u_3\}$.
Since $T$ is a barrier, $S\in\CalQ(v_0)$.  By assumption, $\{v_0,v\}$
is a diagonal of an upright quarter in $\CalQ(v_0)$.  By the fact
that the interiors of quarters in $\CalQ(v_0)$ do not meet, we see
that $v$ is not enclosed over $S$.  The set $\BigD(v,W)$ has
a star convexity with respect to the ray from $v_0$ through $v$.
Thus, if $\BigD(v,W)$ intersects the convex hull of
$T$ at $y$, then $\BigD(v,W)$ intersects the cone over a face
$\{v_0,u_1,u_2\}$ of $S$ at $y'$. (We can take $y'/|y'|$ to lie on
the cone generated by the arc running from $v/|v|$ to $y/|y|$.
This is impossible by Lemmas~\ref{lemma:delta-tri} and
\ref{lemma:delta-flat}.
\end{proof}

\begin{lemma}  Let $\{v_0,v\}$ be the upright diagonal of a quarter
in the $\CalQ(v_0)$-system.  Then $\BigD(v,W)$ is
a subset of $\op{VC}(v_0)$.
\end{lemma}

\begin{proof}
We begin by showing that $\BigD^-(v,W)\subset\op{VC}(v_0)$.
Suppose to the contrary, that a point $x$ in the interior of
$\BigD^-$ lies in $\op{VC}(w)$, with $w\ne0$.  Then $x$ is closer
to $w$ than to  $v_0$.  Thus, $\eta(v_0,v,w)<\eta_0(|v|/2)$, and $w$
is an anchor of $\{v_0,v\}$.  The face $E_w$ in the construction
$\BigD^-(v,W)$ prevents this from happening.

Now consider a point $x$ of $R_w$, which we assume to lie in
$\op{VC}(u)$, with $u\ne0$.  To avoid a trivial case, we may
assume that $w\ne u$.

Assume that the orientation of $S=\{v_0,v,w,u\}$ is negative along
the face $\{v_0,v,w\}$.  Then $S$ must be an upright quarter.  By
the construction of wedges $W\in\CalW$, we have that $R_w$ must
lie on the opposite side of the plane $\{v_0,v,w\}$ from $u$ (for
there is no wedge between the anchors of an upright quarter).  The
result now follows from Lemma~\ref{lemma:back}.

If $\rad(S) <\eta_0(|v|/2)$, then $u$ and $w$ are anchors.  In
this case, the result follows from Lemma~\ref{lemma:prev}.

Finally if the orientation is positive and if $\rad(S)\ge
\eta(|v|/2)$, then a point of $R_w$ cannot be closer to $u$ than
to $v_0$.
\end{proof}


\subsection{overlap}%DCG 9.3, p93
    \label{sec:overlap}
    \oldlabel{2.4}


\begin{lemma}  The sets $\BigD(v,W)$ do not overlap one another.
\end{lemma}

\begin{proof}
This is clear for two sets around the same vertex $v$.  Consider
the sets $\BigD(u,W(u))$ and $\BigD(v,W(v))$ at $u$ and $v$.

To treat the points in $\BigD^-(u,W(u))$ and $\BigD^-(v,W(v))$, we
may contract $\{u,v\}$ until $|u-v|=2$.  By the constraints on the
edges of $\{v_0,u,v\}$, the circumcenter $c$ of this triangle lies
in the convex hull of the triangle.  We have $\eta(v_0,u,v)\ge
\eta_0(|v|/2)$ and $\eta(v_0,u,v)\ge\eta_0(|u|/2)$.  So the plane
through $\{v_0,c\}$ perpendicular to the plane $\{v_0,u,v\}$ separates
$\BigD^-(u,W(u))$ from $\BigD^-(v,W(v))$.

Next we separate points in $\BigD^-(u,W(u))$ from points of
$R_w^{(v)}$, where $w$ is an anchor of $v$ and $u\ne v$.  Let
$S=\{v_0,u,v,w\}$. The orientation of $S$ along $\{v_0,v,w\}$ is
positive.  The circumradius of $S$ satisfies
    $$
    \rad(S) \ge \eta(v_0,u,v)>\eta_0(|v|/2).
    $$
Thus, $\epsilon_0(S,\cdot)$ takes different values on
$\BigD^-(u,W(u))$ and $R_w^{(v)}$, so that the sets are disjoint.

Next we separate points of $R_w^{(v)}$ from $R_w^{(u)}$.  (Notice
that we assume that the anchor is the same for the two Rogers
simplices.) Let $S=\{v_0,u,v,w\}$.   As above, we have
    $$
    \rad(S) \ge \eta_0(|v|/2), \quad \eta_0(|w|/2).
    $$
The simplex $S$ has positive orientation along the faces
$\{v_0,u,w\}$ and $\{v_0,v,w\}$.  Let $c_u$ be the circumcenter of
$\{v_0,u,w\}$, let $c_v$ be the circumcenter of $\{v_0,v,w\}$, and let
$c$ be the circumcenter of $S$.  Then $R_w^{(v)}$ lies in the
convex hull of $\{v_0,w,c_v,c\}$, but $R_w^{(u)}$ lies in the convex
hull of $\{v_0,w,c_u,c\}$.  Thus, the sets are disjoint.

Finally, we separate points of $R_w^{(u)}$ from points of
$R_{w'}^{(v)}$, where $w\ne w'$ and $u\ne v$.  If the function
$\epsilon_0(\{v_0,w,w'\},\cdot)$ separates the sets, we are done.
Otherwise, we may assume say that $\epsilon_0(\{v_0,w,w'\},x) = w'$
from some $x\in R_w^{(u)}$.  Let $S=\{v_0,u,w,w'\}$.

If $w'$ is not an anchor of $u$, then $\rad(S) \ge\eta_0(|u|/2)$
and the orientation of $S$ along $\{v_0,w,u\}$ is positive.  In this
case, we have $\epsilon_0 = w$ on $R_w^{(u)}$, which is contrary
to assumption. Thus, we may assume that $w'$ is an anchor of $u$.

If the orientation of $\{v_0,u,w,w'\}$ is negative along $\{v_0,w,u\}$,
then $\{v_0,u,w,w'\}$ is a quarter, contrary to the existence of $W\in
\CalW$.  So the orientation is positive.  If $\rad(\{v_0,u,w,w'\}) <
\eta_0(|u|/2)$, then Lemma~\ref{lemma:prev} implies that each point
of $R_w$ is obstructed from $w'$.  But no point of $R_{w'}^{(v)}$ is
obstructed from $w$. (In fact, a barrier that crosses
$\BigD(v,W(u))$ is inconsistent with Lemmas~\ref{lemma:delta-tri},
\ref{lemma:delta-flat}, \ref{lemma:delta-upright}.) So
$\rad(\{v_0,u,w,w'\}) \ge \eta_0(|u|/2)$.  This is contrary to
$\epsilon_0(\{v_0,w,w'\},x) = w'$ from some $x\in R_w^{(u)}$.
\end{proof}



\section{Simplex Types}%DCG 9.4, p94
    \oldlabel{2.5}

We consider various types of simplices $A$, $B$, $C$, $D$, $E$.  Each type has
its vertices at vertices of the packing.  The edge lengths of
these simplices are at most $2\sqrt{2}$.

$A$.  This family consists of simplices $S(y_1,\ldots,y_6)$ whose
edge lengths satisfy
    $$
    y_1,y_2,y_3\in[2,2t_0],\quad
    y_4,y_5\in[2t_0,2.77],
    \quad
    y_6\in[2,2t_0],\quad \text{and }
    \eta(y_4,y_5,y_6)<\sqrt{2}.
    $$
(These conditions imply $y_4,y_5<2.697$, because
$\eta(2.697,2t_0,2)>\sqrt2$.)

$\SB$.  This family consists of certain flat quarters that are
part of an isolated pair of flat quarters. It consists of those
satisfying $y_2,y_3\le 2.23$, $y_4\in[2t_0,2\sqrt{2}]$.

$\SC$.  This family consists of certain simplices
$S(y_1,\ldots,y_6)$ with edge lengths satisfying
    $y_1,y_4\in[2t_0,2\sqrt{2}]$, $y_2,y_3,y_5,y_6\in[2,2t_0]$.
We impose the condition that the first edge is the diagonal of
some upright quarter in the $Q$-system, and that the upper
endpoints of the second and third edges (that is, the second and
third vertices of the simplex) are consecutive anchors of this
diagonal. We also assume that $y_4< 2.77$, or that both face
circumradii of $S$ along the fourth edge are less than $\sqrt{2}$.

% XX Insert SD

$E$.  This family consists of simplices $\{v_0,v_1,v_2,v_3\}$ such
that 
the edge lengths $y_{ij} = |v_i-v_j|$ satisfy
   $$
   y_{01},y_{02}\in[2t_0,2\sqrt{2}],\quad
   y_{ij}\in [2,2t_0], 
   $$
for all other $ij$, and such that
there is another simplex $\{w_0,w_1,w_2,w_3\}$ satisfying the same
constraints with $y_{ij} = |w_i-w_j|$ and  such that $(w_0,w_1,w_2)=(v_0,v_1,v_2)$
and $|w_3-v_3|>\sqrt8$.

\begin{lemma}
    \label{lemma:2.77}
If a vertex $w$ is enclosed over a simplex $S$ of type $A$, $\SB$,
or $\SC$, then its height is greater than $2.77$.  Also, $\{v_0,w\}$
is not the diagonal of an upright quarter in the $Q$-system.
\end{lemma}

\begin{proof}
In case $A$, $\eta(y_4,y_5,y_6)<\sqrt{2}$, so an enclosed vertex
must have height greater than $2\sqrt{2}$.  It is too long to be
the diagonal of a quarter.

In case $\SB$, we use the fact that the isolated quarter does not
meet in the interior with any quarter in the $Q$-system. 
By Lemma~\ref{tarski:enclosed-v}, an
enclosed vertex has length at least $2.77$.
By the symmetry of isolated quarters, this means that the diagonal
of a flat quarter must also be at least $2.77$.

In case $\SC$, the same calculation gives that the enclosed vertex
$w$ has height at least $2.77$.  Let the simplex $S$ be given by
$\{v_0,v,v_1,v_2\}$, where $\{v_0,v\}$ is the upright diagonal. By
Lemma~\ref{lemma:pass-anchor}, $v_1$ and $v_2$ are anchors of
$\{v_0,w\}$. The edge between $w$ and its anchor cannot cross
$\{v,v_i\}$ by Lemma~\ref{lemma:2t0-doesnt-pass-through}. (Recall
that two sets are said to {\it cross\/} if their radial
projections overlap.) The distance between $w$ and $v$ is at most
$2t_0$ by Lemma~\ref{lemma:double-face}. If $\{v_0,w\}$ is the
diagonal of an upright quarter, the quarter takes the form
$\{v_0,w,v_1,v_3\}$, or $\{v_0,w,v_2,v_3\}$ for some $v_3$, by
Lemma~\ref{lemma:double-face}. If both of these are quarters, then
the diagonal $\{v_1,v_2\}$ has four anchors $v$, $w$, $v_0$, and
$v_3$. The selection rules for the $Q$-system place the quarters
around this diagonal in the $Q$-system. So neither $\{v_0,w,v_1,v_3\}$
nor $\{v_0,w,v_2,v_3\}$ is in the $Q$-system. Suppose that
$\{v_0,w,v_1,v_3\}$ is a quarter, but that $\{v_0,w,v_2,v_3\}$ is not.
Then $\{v_0,w,v_1,v_3\}$ forms an isolated pair with $\{v_1,v_2,v,w\}$.
In either case, the quarters along $\{v_0,w\}$ are not in the
$Q$-system.
\end{proof}

\begin{remark}  The proof of this lemma does not make use of all the hypotheses
on $\SC$.  The conclusion holds for any simplex
$S(y_1,\ldots,y_6)$, with $y_1,y_4\in[2t_0,2\sqrt{2}]$,
$y_2,y_3,y_5,y_6\in[2,2t_0]$.
\end{remark}

\subsection{disjointness}%DCG 9.5, p95
    \oldlabel{2.6}

Let $S=\{v_0,v_1,v_2,v_3\}$ be a simplex of type $A$, $\SB$, or
$\SC$. An edge $\{v_4,v_5\}$ of length at most $2\sqrt{2}$ such
that $|v_4|,|v_5|\le 2t_0$ cannot cross two of the edges
$\{v_i,v_j\}$ of $S$.  In fact, it cannot cross any edge $\{v_i,v_j\}$
with $|v_i|,|v_j|\le 2t_0$ by Lemma~\ref{lemma:skew-quad}.  The
only possibility is that the edge $\{v_4,v_5\}$ crosses the two
edges with endpoint $v_1$, with $|v_1|\ge2t_0$ in case $\SC$.  But
this too is impossible by Lemma~\ref{lemma:double-face}.

Similar arguments show that the same conclusion holds for an edge
$\{v_4,v_5\}$ of length at most $2t_0$ such that $|v_4|\le2t_0$,
$v_5\le2\sqrt{2}$.  The only additional fact that is needed is
that $\{v_4,v_5\}$ cannot cross the edge between the vertex $v$ of
an upright diagonal $\{v_0,v\}$ and an anchor
(Lemma~\ref{lemma:2t0-doesnt-pass-through}).





\begin{lemma}
    \label{lemma:no-overlap}
    Consider two simplices $S$, $S'$, each of  type $A$, $\SB$, $\SC$,
or a quarter in the $Q$-system.
    %% XX I'm not sure if this hypothesis is needed.  If so, the lemma
    %% has to be moved after standard regions are introduced:
    %Assume that $S$ and $S'$ do not lie
    %in the cone over a quadrilateral region.  
    Then the interiors
    of
    $S$ and $S'$ do not meet.
\end{lemma}

\begin{proof}
%% XX I'm not sure these next two lines are needed...; see above.
%By hypothesis, the standard region is not a quadrilateral, and we
%thus exclude the case of conflicting diagonals in a quad cluster.
We claim that no vertex $w$ of $S$ is enclosed over $S'$.
Otherwise, $w$ must have height at least $2t_0$, so that $\{v_0,w\}$
is the diagonal of an upright in the $Q$-system, and this is
contrary to Lemma~\ref{lemma:2.77}. Similarly, no vertex of $S'$
is enclosed over $S$.

Let $\{v_1,v_2\}$ be an edge of $S$ crossing an edge $\{v_3,v_4\}$ of
$S'$. By the preceding remarks, neither of these edges can cross
two edges of the other simplex. The endpoints of the edges are not
enclosed over the other simplex. This means that one endpoint of
each edge $\{v_1,v_2\}$ and $\{v_3,v_4\}$ is a vertex of the other
simplex.  This forces $S$ and $S'$ to have three vertices in
common, say $v_0$, $v_2$, and $v_3$.  We have $S=\{v_0,v_1,v_3,v_2\}$
and $S'=\{v_0,v_3,v_2,v_4\}$. If
    $|v_2|\in[2t_0,2\sqrt{2}]$,
then we see that the anchors $v_3$, $v_4$ of $\{v_0,v_2\}$ are not
consecutive.  This is impossible for simplices of type $\SC$ and
upright quarters.  Thus, $v_2$ and $v_3$ have height at most
$2t_0$.  We conclude, without loss of generality, that
    $|v_4|\in[2t_0,2\sqrt{2}]$
and $|v_1-v_2|\ge 2t_0$.

The heights of the vertices of $S$ are at most $2t_0$, so it has
type $A$ or $\SB$, or it is a flat quarter in the $Q$-system. If
$S'$ is an upright quarter in the $Q$-system, then it does not
overlap an isolated quarter or a flat quarter in the $Q$-system,
so $S$ has type $\SA$. By Lemma~\ref{tarski:277}, we have
$|v_1-v_2|>2.77$.  This imposes the contradictory constraints
on $\SA$
    $$
    2.77\ge |v_1-v_2|>2.77.
    $$
Thus $S'$ has type $\SC$.  This forces $S$ to have type $\SA$.  We
reach the same contradiction  $2.77 > 2.77$.
\end{proof}

\subsection{type A}%DCG 9.6, p96
    \label{sec:separation}
    \oldlabel{2.7}

Let $S = \{v_0,v_1,v_2,v_3\}$.
Let $\op{cone}^0(S) = \op{cone}^0(v_0,\{v_1,v_2,v_3\}$.
Let $V_S = \op{VC}(v_0)\cap \op{cone}^0(S)$, for a simplex $S$ of type $\SA$,
$\SB$, or $\SC$. 
We truncate $V_S$ to $V_S(t_S)$ by intersecting
$V_S$ with a ball of radius $t_S$.  The parameters $t_S$ depend on
the type of $S$.

If $S$ has type $\SA$, we use $t_S=+\infty$ (no truncation).

\begin{lemma} Let $S=\{v_0,v_1,v_2,v_3\}$ be a simplex of type $\SA$.
There is a null set $E$, such that
we have  $ \Omega(v_0,S) \cap \op{cone}^0(S) \subset V_S \cup E$.
\end{lemma}

\begin{proof} 
We use the fact that if $b$ is a barrier, then $\op{conv}$ does
not meet $\op{conv}^0(S)$ by Lemma~\ref{XX}.  


Excluding a null set, we may assume 
for a contradiction that
$x\in \Omega(v_0,S) \cap \op{cone}^0(S) \cap \op{VC}(v)$,
for some $v\ne v_0$.  

% ...
By Lemma~\ref{tarski:vor-bar-sqrt2}, $x$ and $v_0$ lie on the
same side of $\op{aff}\{v_1,v_2,v_3\}$.  Thus, $x$ is in
$\op{conv}^0(S)$.  
Thus, every vertex of $S$ is unobstructed at $x$.  Thus, $x$
is closer to $v$ than to any vertex of $S$.

By Lemma~\ref{tarski:vor-bar-sqrt2}, $\op{conv}\{v_1,v_2,v_3\}$ 
separates
$\Omega(v_0,S)\cap \op{cone}^0(S)$ from $\Omega(v,\{v,v_1,v_2,v_3\})$ when
$v$ is enclosed over $S=\{v_0,v_1,v_2,v_3\}$.  This is contrary
to the assumption that $x$ lies in the intersection of these
two sets.

If $\Omega(v,\{v_0,v_1,v_2\})$ meets $\op{conv}^0(S)$, then
$S'=\{v,v_0,v_1,v_2\}$ must be a quarter or quasi-regular tetrahedron.
If $x$ is a barrier, then $x\not\in\op{VC}(v)$.  This implies
that $S'$ is a quarter that is not in the $Q$-system.
It
cannot be an isolated quarter because of the edge length
constraint $2.77$ on simplices of type $\SA$.
There must be a
conflicting diagonal $\{v_0,w\}$, where $w$ is enclosed over $Q$. ($w$
cannot be enclosed over $S$ by results of
Lemma~\ref{lemma:no-overlap}.) This shields the $V$-cell at $v$
from $\op{cone}^0(S)$ by the two barriers $\{v_0,w,v_1\}$ and $\{v_0,w,v_2\}$ of
quarters in the $Q$-system.
\end{proof}

\begin{lemma} Let $S=\{v_0,v_1,v_2,v_3\}$ be a simplex of type $A$.
  $V_S$ is disjoint from all of the set $\BigD(v,W)$.
\end{lemma}

\begin{proof}
This is evident from
Lemmas~\ref{lemma:delta-tri} and \ref{lemma:delta-flat}.
\end{proof}


Our justification that $V_S(t_S)$ can be treated as an
independently scored entity is now complete.

\subsection{type B}%DCG 9.7, p96
    \oldlabel{2.8}

If $S(y_1,\ldots,y_6)$ has type $\SB$, we label vertices so that
the diagonal is the fourth edge, with length $y_4$. We set
$t_S=1.385$. The calculation in Lemma~\ref{lemma:2.77}
shows that any enclosed vertex over $S$ has height at least
$2.77=2t_S$.

\begin{lemma} Let $S=\{v_0,v_1,v_2,v_3\}$ be a simplex of type $\SB$.
There is a null set $E$, such that
we have  $ \Omega(v_0,S) \cap \op{cone}^0(S) \cap B(v_0,1.385) 
\subset V_S \cup E$.
\end{lemma}

\begin{proof}  As above, assume for a contradiction that there
is a point in 
 $$\Omega(v_0,S)\cap \op{cone}^0(S) \cap B(v_0,1.385)\cap \op{VC}(v'),$$
with $v'\ne v_0$.
Vertices outside $\op{cone}^0(S)$ cannot reach inside $S$ this way.  In
fact, such a vertex $v'$ would have to form a quarter or
quasi-regular tetrahedron with a face of $S$.  The $V$-cell at
$v'$ cannot meet $\op{cone}^0(S)$ unless it is a quarter that is not in the
$Q$-system. But by definition, an isolated quarter is not adjacent
(along a face along the diagonal) to any other quarters.
\end{proof}

%To separate the scoring of $V_S(t_S)$ from the rest of the
%standard cluster, we also show that the terms of
%Formula~\ref{eqn:3.5}  for $V_S(t_S)$ are represented
%geometrically by solids that lie in the cone $\op{cone}^0(S)$.   This
%is the purpose of the following lemma.

\begin{lemma} Let $S=\{v_0,v_1,v_2,v_3\}$ be a simplex of type $\SB$.
The cone $\op{rcone}^0(v_0,v_1,|v|/2.77)$ does not meet the
cone $\op{cone}(v_0,\{v_2,v_3\}$.
\end{lemma}

\begin{proof} This is Lemma~\ref{tarski:beta:B}.
\end{proof}

\begin{lemma} $\Omega(v_0,S) \cap \op{cone}^0(S) \cap B(v_0,1.385)$
does not meet the sets $\delta(v)$.
\end{lemma}

\begin{proof}
The reasons given in Section~\ref{sec:separation} for the
disjointness of $\delta(v)$ and $V_S(t_S)$ apply to this
situation as well.
\end{proof}


This completes the justification that
$V_S(t_S)$ is an object that can be treated in separation from the
rest of the local $V$-cell.

\subsection{type C}%DCG 9.8, p97
    \oldlabel{2.9}

If $S(y_1,\ldots,y_6)$ is of type $\SC$, we label vertices so that
the upright diagonal is the first edge.  We use $t_S =+\infty$ (no
truncation).   

\begin{lemma} Let $S=\{v_0,v_1,v_2,v_3\}$ be a simplex of type $C$.
There is a null set $E$, such that
we have  $ \Omega(v_0,S) \cap \op{cone}^0(S) \subset V_S \cup E$.
\end{lemma}

\begin{proof}  %% XX Rewrite this proof.
Vertices outside $S$ cannot affect the shape of $V_S(t_S)$.  Any
vertex $v'$ would have to form a quarter along a face of $S$.  If
the shared face lies along the first edge, it is a quarter $Q$ in
the $Q$-system, because one and hence all quarters along this edge
are in the $Q$-system.  The faces of this quarter are then
barriers. If the shared face lies along the fourth edge, then its
length is at most $2.77$, so that the quarter cannot be part of an
isolated pair. If it is not in the $Q$-system, there must be a
conflicting diagonal. The two faces along this conflicting
diagonal of the adjacent pair in the $Q$-system (that is, the pair
taking precedence over $Q$ in the $Q$-system) are barriers that
shield the $V$-cell at $v'$ from $S$.
\end{proof}

The reasons given in Section~\ref{sec:separation} for the
disjointness of $\delta(v)$ and $V_S(t_S)$ apply to simplices of
type $\SC$ as well. This completes the justification that
$V_S(t_S)$ is an object that can be treated in separation from the
rest of the local $V$-cell.

\subsection{type D}%DCG 9.9, p97
    \oldlabel{2.10}

We introduce a small variation on simplices of type $\SC$, called
type $\SCp $.  We define a simplex $\{v_0,v,v_1,v_2\}$ of type $\SCp $
to be one satisfying the following conditions.
    \begin{enumerate}
    \item The edge $\{v_0,v\}$ is an upright diagonal of an upright quarter
        in the $Q$-system.
    \item $|v_2|\in[2.45,2t_0]$.
    \item $v_1$ and $v_2$ are anchors of $v$.
    \item $|v-v_2|\in [2.45,2t_0]$.
    \item The edge $\{v_1,v_2\}$
    is a diagonal of a flat quarter with face $\{v_0,v_1,v_2\}$.
    \end{enumerate}

It follows that $v_1$ and $v_2$ are consecutive anchors of
$\{v_0,v\}$.

On simplices $S$ of type $\SCp $, we label vertices so that the
upright diagonal is the first edge.  We use $t_S=+\infty$ (no
truncation).  

Simplices of type $\SCp $ are separated from quarters in the
$Q$-system and simplices of types $\SA$ and $\SB$ by procedures
similar to those described for type $\SC$.  The following lemma is
helpful in this regard.


\begin{lemma}\label{lemma:C'Q}
 The flat quarter along the face $\{v_0,v_1,v_2\}$ is
in the $Q$-system.
\end{lemma}

\begin{proof}
By Lemma~\ref{tarski:245}, there cannot be an enclosed vertex
of height at most $\sqrt2$. 
So nothing is enclosed over the flat quarter.
By Lemma~\ref{tarski:245bis}, there cannot be an edge of length
at most $2\sqrt2$ that crosses inside the slice.
(XX slice has not yet been defined.) 
This implies that the flat quarter does not have
a conflicting diagonal and is not part of an isolated pair.
\end{proof}


\begin{lemma}
Suppose that $v'$ is enclosed over $S$.  Then $\op{VC}(v')$ does
not meet $\op{conv}^0(S)$.
\end{lemma}

\begin{proof} If there is a point $x$ of intersection, then
$x$ is closer to $v'$ than to any point of $S$. 
By Lemma~\ref{tarski:vor-bar-quad}, this implies that 
$S'=\{v',v,v_1,v_2\}$ is a quarter.  By Lemma~\ref{lemma:C'Q},
$S'$ is in the $Q$-system.  Thus, $\{v,v_1,v_2\}$ is a barrier,
and $x$ is obstructed from $v'$.
\end{proof}


Unlike the other cases, there can in fact be overlap between
$\BigD(v,W)$ and simplex of type $\SCp$, when the upright
diagonal of the simplex is $\{v_0,v\}$.  This is because the
conditions defining a wedge $W\in\CalW$ are not incompatible with
the conditions defining type $\SCp$.  Nevertheless, except in the
obvious case where the simplex of type $\SCp$ and the wedge are both
constructed between the same consecutive anchors of $\{v_0,v\}$, there
can be no overlap of a $\BigD(v,W)$ with a simplex of type
$\SCp$.

\subsection{type D}

XX insert

\subsection{type E}

XX insert


\subsection{summary}


The construction of the decomposition of the $V$-cell $\op{VC}(v_0)$
is now complete. It consists of the pieces

    \begin{itemize}
    \item $\delta(v)=\delta_i(v)$,
         for each diagonal $\{v_0,v\}$ of an upright quarter
        in the $Q$-system, and $i$ as in Definition~\ref{def:wedge}.
    \item truncations of Voronoi pieces $V_S(t_S)$ for simplices of type
        $\SA$, $\SB$, or $\SC$ (and on rare occasion $\SCp$),
    \item $\tildeV(t_0)$, the truncation at $t_0$ of all parts of
        $\op{VC}(v_0)$ that do not lie in any of the cones $\op{cone}^0(S)$ over
        simplices
        of type $\SA$, $\SB$ or $\SC$,
    \item $\delta'$, the part not lying in any of the preceding.
    \end{itemize}





\chapter{From Sphere Packings to Hypermaps}
%\chapter{Basic Properties of Standard Regions}%DCG Sec.10, p99
    \label{sec:intro}
    \oldlabel{1}
\label{chapter:VQ}

\section{Fans}

There are many different fans $(v,V,E)$ that
can be associated with a centered packing $(v,\Lambda)$.
As we will see, different selections of fans
will lead to different approximations to the function $\sigma(v,\Lambda)$.
It will be important for us to have many different approximations
at our disposal.  For that reason, we consider a number of
fans.

\begin{definition}\label{def:compatible}
Let $(v,V,E)$ be a fan.  We say that it is
{\it compatible} with a centered packing $(w,\Lambda)$ 
if 
\begin{itemize}
\item 
$v=w\in\Lambda$
\item
 $V\subset \Lambda$, 
\item  $Q\in \CalQ(v,\Lambda)$ implies
$\op{conv}^0(Q)\cap X(v,V,E)=\emptyset$,
\item  If $e\in E$, $(v,v_1,v_2,v_3)\in \CalQ(v,\Lambda)$,
and $C(v,e)\cap C(v,\{v_1,v_2\})\ne \emptyset$, then
   $e = \{v_1,v_2\}$.
\end{itemize}
\end{definition}

\subsection{standard graph}

Let $(v,\Lambda)$ be a centered packing.  The set $U=U(v,\Lambda)$
is defined in Definition~\ref{def:U}.  
Let 
$$E = E(v,\Lambda) = \{(u,u')\in U(v,\Lambda) \mid |u-u'|\le 2t_0\}.
$$

\begin{lemma}
\label{lemma:Q-in-region} Each simplex in the $Q$-system with a
vertex at $v_0$ lies entirely in the closed cone over some
standard region $R$.
\end{lemma}

\begin{proof}  Assume for a contradiction that $Q=\{v_0,v_1,v_2,v_3\}$
with $v_1$ in the open cone over $R_1$ and with $v_2$ in the open
cone over $R_2$.  Then $\{v_0,v_1,v_2\}$ and $\{v_0,w_1,w_2\}$ (a wall
between $R_1$ and $R_2$) cross;  this is contrary to
Lemma~\ref{lemma:barrier-no-overlap}.
\end{proof}

\begin{lemma}
Let $(v,\Lambda)$ be a centered packing.  Then
$(v,U,E)$ is a fan.
\end{lemma}

\begin{proof}
Use Lemma~\ref{lemma:2t0-doesnt-pass-through}.
\end{proof}

\begin{definition}  We call $(v,U,E)$ the standard
graph of the centered packing $(v,\Lambda)$.   We call the
components of $Y(v,U,E)$ the standard components of the centered
packing $(v,\Lambda)$.  We call the hypermap $\op{hyp}(v,U,E)$
the standard hypermap of the centered packing.  
\end{definition}

We consider the standard graph  to be the
primary fan attached to a centered packing.  All other
standard graphs are in some sense derivative, either by deleting
certain edges from the standard graphs, or by adding additional
edges.


\begin{lemma}
Let $(v,\Lambda)$ be a centered packing.  The standard graph
of $(v,\Lambda)$ is $(v,\Lambda)$-compatible.
\end{lemma}

\begin{definition}  We call a component $R$ of $Y(v,V,E)$ an
$n$-gon (triangle, quadrilateral, pentagon, and so forth) if
the set of darts of $\op{hyp}(v,V,E)$ that lead into $R$ is
a single face of the hypermap and if that face is an orbit of
cardinality $n$.   We say that
a component of $Y(v,U,E)$ is exceptional, if it is not a
triangle or quadrilateral.
\end{definition}


\begin{lemma}
Each $\BigD(v,W)$ lies entirely in the cone over the standard
region that contains $\{v_0,v\}$.
\end{lemma}

\begin{proof}
By Lemma~\ref{lemma:delta-tri},
the cone over a standard region is bounded by the cones  over the
quasi-regular triangles.
\end{proof}

\subsection{deleting a Node}

%% XX Move to hypermap and geometry.

Here we describe a construction of deleting nodes from a 
fan.  


Let $(v,V,E)$ be a fan with hypermap $\op{hyp}(v,V,E)$.
Let $V'\subset V$ be a set of nodes. 
Let $E'\subset E$ be the set of edges $e$ that meet $V'$.
We have a fan $(v,V\setminus V',E\setminus E')$.

We can identify $V'$ with a subset of $V$, where
the node containing darts $(v,w,\ldots)$ is identified with $w\in V$.
We define a normal family of contour loops on 
$\op{hyp}(v,V,E)$.  Let $D''$ be the set of darts that lie in the
nodes constituting $V'$.  Let $D'$ be the remaining darts.
We place each $x\in D'$ in a contour
loop by the following permutation $r$ on $D'$:
    $$r x =
    \begin{cases}
    f x, & f x \in D'\\
    n^{-1} x & f x \in D''\\
    \end{cases}
    $$
For each $x\in D'$, we obtain a contour loop $(x,r x, r^2 x,\ldots)$.
Let $\cal L$ be this family of contour loops.  It is normal.

\begin{lemma} In this context,
$\op{hyp}(v,V,E)/{\cal L}$ is isomorphic to 
$\op{hyp}(v,V\setminus V',E\setminus E')$.
\end{lemma}



\subsection{degenerate Darts}

%% XX Move to hypermap and geometry??

The hypermap $\op{hyp}(v,V,E)$ attached to a fan $(v,V,E)$
has two types of darts: the non-degenerate darts of the form
$(v,v_1,v_2,v_3)$ and the degenerate darts of the form $(v,w)$.
Each degenerate dart is a  fixed point for the edge, face, and node
permutations.

A special case of the construction of the previous section
is to delete a degenerate node (dart) from a fan.
Each degenerate dart $x=(v,w)$ uniquely determines element
$w\in V$.  This element $w$ has no edges: $E_w = \emptyset$.
If $V'$ is a set of degenerate darts, with corresponding elements
$V'\subset V$, 
we obtain a fan
$\op{hyp}(v,V\setminus V',E)$.  There is a canonical bijection
between components of $Y(v,V,E)$ and components of $Y(v,V\setminus V',E)$.

Each degenerate dart $x=(v,w)$ leads into a uniquely determined
component of $\op{hyp}(v,V,E)$.  
Figure~\ref{fig:deg-pent-hex} %% WW not yet drawn. Two frames of DCG~page126.
shows a degenerate dart that leads into a component where
the other darts leading into the component form a pentagon
or hexagon.


\subsection{aggregated graphs}
%\section{Contravening Plane Graphs defined}
\label{sec:stargraph}


Aggregated graphs are fans that are obtained from
a fan by deleting various vertices.  

Aggregates will be formed only when the standard hypermap has
very specific combinatorial structures.  We describe those
combinatorial structures here.

\begin{definition}
Let $(v,V,E)$ be a fan such that $V\subset U(v,\Lambda)$.
An aggregating contour loop in $\op{hyp}(v,V,E)$ 
is any of the following specific
types of contour loops satisfying the 
two additional conditions that follow.
See Figure~\ref{fig:agg} %% WW not yet drawn. DCG~page126, p.158,
%% First three are DCG~page 126.
for the exact combinatorial structures.
\begin{itemize}
\item A contour loop with six darts tracing through five nodes. 
\item A contour loop with seven darts tracing through five nodes (with three
  darts at one node).
   The darts in this contour loop
   meet a triangle face and an octahedral face (Figure~\ref{fig:tri-pent}). 
   %pent-tri
\item A contour loop with seven darts tracing through six nodes.
\item A contour loop with six darts tracing through four nodes. 
\item A contour loop with 14 darts tracing through eight nodes
   (Figure~\ref{fig:degree6}). %oct-agg
\item A contour loop with 11 darts tracing through six nodes %hex-agg.
   (Figure~\ref{fig:degree6}).
\end{itemize}
\index{aggregating}
To qualify as an aggregating 
contour loop we impose the following two additional conditions.
\begin{enumerate}
\item Each dart $(v,v_1,\ldots)$ in a face visited by the contour 
loop satisfies $v_1\in U(v,\Lambda)$.
\item Each face step $(v,v_1,\ldots)\to (v,w_1,\ldots)$ in the contour
loop satisfies $|v_1-w_1|\le 2t_0$.
\end{enumerate}
\end{definition}

Note that each aggregating contour loop has the property that
if it meets a dart of a face at a node, then it meets every dart
of that face at that node.

\begin{definition}  Let $(v_0,\Lambda)$ be a centered packing with
standard fan $(v_0,V,E)$ and hypermap $H$.  For
each such collection of data, choose once and for all
a maximal set of disjoint aggregating
contour loops in $H$.  This extends to a normal family by adding
contour loops for each face of $H$ that is disjoint from the aggregating
loops.    We call the quotient the aggregating hypermap of $(v_0,\Lambda)$.
\index{aggregating hypermap}
\end{definition}

\begin{lemma} The aggregating hypermap is connected, planar, plane,
and has no degenerate darts.   Every face of the aggregating hypermap
is simple. 
\end{lemma}

As in Lemma~\ref{XX}, aggregating hypermap is the hypermap
obtained from a fan (obtained by deleting edges and
vertices from the standard fan).

\begin{figure}[htb]
  \centering
  \myincludegraphics{noimage.eps}
  \caption{Some aggregating contour loops} % DCG examples.
  \label{fig:agg}
\end{figure}

\begin{figure}[htb]
  \centering
  \myincludegraphics{\ps/tripent.eps}
  \caption{An aggregating contour loop forming a pentagon in the quotient}
  \label{fig:tri-pent}
\end{figure}

\begin{figure}[htb]
  \centering
  \myincludegraphics{\ps/degree6.eps}
  \caption{Further aggregating loops}
  \label{fig:degree6}
\end{figure}



It will be shown in
Lemma~\ref{lemma:deg5} that there is at most one aggregating
contour loop in the standard hypermap of a contravening centered packing,
so we will not worry here about how to select one.




\subsection{connected sums and aggregating}

A special case frequently occurs.  The family of contour loops
$\cal L$ often has the property that all but one contour is
a face of the hypermap.  Let $C$ be the contour loop that
is not a face of the hypermap.  The process of deleting an
edge can then be viewed as removing the internal structure
of the contour loop $C$.  The contour loop $C$ becomes a single
face $F$ in the quotient hypermap.

Connected sums, which were described in Section~\ref{XX} allow
us to reverse the process by inserting internal structure to
a face $F$.  This allows us to retrieve the original hypermap
$\op{hyp}(v,V,E)$
from the quotient
$\op{hyp}(v,V,E)/{\cal L}$.  Specifically, by connected sums, we can recover
the original hypermaps from the quotients described by
Figures~\ref{fig:agg}, \ref{fig:tri-pent}, \ref{fig:degree6}.

\subsection{masked quarters}



\subsection{enhanced graph}
\subsection{maximal graph}

\section{Scoring}





\subsection{Voronoi cells and simplices}


Let $S$ be a set of four vertices and let $v$ be a vertex of that simplex. Let
$\Omega(v,S)$ be the subset of $\op{conv}^0(S)$ consisting of points closer
to $v$ than to any other vertex of $S$. By
Lemma~\ref{lemma:Q-divide}, if $S\in\CalQ(v,\Lambda)$, then
$$\Omega(v,S) = \Omega(v,\Lambda)\cap\op{conv}^0(S).$$
Under the assumption that $S$ contains its circumcenter and that
every one of its faces contains its circumcenter, an explicit
formula for the volume $\op{vol}(\Omega(v,S))$ has been
calculated.  %% \cite[Section~8.6.3]{part1}. 

%% XX MOVE TO VOLUMES:

\begin{lemma}  Let $S = \{v_0,v_1,v_2,v_3\}$ be a set of four points in $\ring{R}^3$.
Assume that the circumcenter of $S$ is contained in $\op{conv}^0(S)$.  Assume
that each face of $S$ is acute.
Then 
  $$
  \op{vol}\,\Omega(v_1,S) = f(x_{ij}),
  $$ 
where $x_{ij} = |v_i-v_j|^2$, and
$$
   f(x_{ij}) = \sum_{(i,j,k)\in \Pi} \frac{x_{0i} (x_{0j}+x_{ij}-x_{0i})\chi(x_{jk},x_{ik},x_{0k},x_{0i},x_{0j},x_{ij})}
   {48\ups(x_{0i},x_{0j},x_{ij})\Delta(x_{01},x_{02},x_{03},x_{23},x_{13},x_{12})^{1/2}}
$$
and $\Pi = \{(1,2,3),\ldots\}$ is the set of the six permutations of $(1,2,3)$.
(The function $\chi$ is defined in Definition~\ref{tarski:XX}.)
\end{lemma}

\begin{proof} By Lemma~\ref{tarski:XX}, the set $\Omega(v_1,S)$ is a union of
six Rogers simplices, up to a null set.  The volume of a Rogers simplex appears
as Lemma~\ref{XX}.  Summing the contributions from the six simplices, we obtain
the given formula.
\end{proof}

The function $f(x_{ij})$ makes perfect sense for any $S$ for which $\ups(x_{0i},x_{0j},x_{ik},\Delta(x_{ij})\ne 0$,
even though the lemma is valid only under special constraints.
For any simplex $S=\{v_0,v_1,v_2,v_3\}$ such that the denominator of $f$ is nonzero, we define
$$
\op{volan}(v_0,S) = f(x_{ij}),\quad x_{ij} = |v_i-v_j|^2.
$$  
(The denominator is always nonzero when $S$ is not collinear by Lemmas~\ref{tarski:XX} and \ref{tarski:XX}.)
The following appeared as Claim~\ref{claim:volan}.

\begin{lemma}\tlabel{lemma:volan}  %%Cf. claim:volan
Let $S=\{v_1,v_2,v_3,v_4\}$ be in the $\CalQ$-system. Then
    $$
    \sum_{i=1}^4 \op{volan}(v_i,S) = \sum_{i=1}^4
    \op{vol}(\Omega(v_i,S)) = \op{vol}(\op{conv}^0(S)).
    $$
\end{lemma}

\begin{proof} As we have explicit formulas for everything involved,
it is just a matter of plugging in the formulas and checking the sum.
(Even without a calculation, the lemma follows by the principle of analytic continuation;
but in the interest of proving everything from first principles, we
check the sum.)
\end{proof}

\subsection{truncation}

Let $(v,\Lambda)$ be a centered packing and let $(v,V,E)$ be a compatible
fan.
If $R$ is a component of $Y(v,V,E)$, we write 
  $$
  V_R(v,t) = \op{VC}(v_0)\cap C(R)\cap B(v,t)
  $$
We often
take $t=t_0$.

%If $\{v_0,v\}$, of length between $2t_0$ and
%$2\sqrt{2}$, is not the diagonal of an upright quarter in the
%$Q$-system, then $v$ does not affect the truncated cell $V_R(t_0)$
%and may be disregarded. For this reason we confine our attention
%to upright diagonals that lie along an upright quarter in the
%$Q$-system.


 Let $S=\{v_0,v_1,v_2,v_3\}$ be a simplex. Fix $t$ in the range
$t_0\le t\le\sqrt2$.  Assume that $t$ is at most the circumradius
of $S$. Assume that it is at least the circumradius of each of the
faces of $S$.  Let $\op{VC}_t(v_0,S)$ be the intersection of
$\op{VC}(v_0,S)$ with the ball $B(v_0,t)$. Under the assumption
that $S$ contains its circumcenter and that every one of its faces
contains it circumcenter, an explicit formula for the volume
$$\op{vol}(\op{VC}_t(v_0,S))$$ is calculated by means of
Lemma~\ref{XX} from the six quoins that form $\op{VC}(v_0,S)$.
This leads to the
following formula. Let $h_i = |v_i|/2$ and
$\eta_{ij}=\eta(v_0,v_i,v_j)$, and let $S_3$ be the group of
permutations of $\{1,2,3\}$ in
\begin{equation}
   \op{vol}\,\op{VC}_t(v_0,S) =
   \sol(S)/3 - \sum_{i=1}^3 \frac{\dih(v_i,S)}{2\pi}\op{vol}\,\op{cap_i}
   +\sum_{(i,j,k)\in S_3} \quo(R(h_i,\eta_{ij},t)).
   \tlabel{eqn:vol-theta-0}
\end{equation}


We extend Formula~\ref{eqn:vol-theta-0} by setting
    $$\quo(R(a,b,c)) = 0,$$
if the constraint $a < b < c$ fails to hold.  Similarly, set
$\op{vol}\,\op{cap}_i=0$ if $|v_i|\ge 2t$.  With these
conventions,  Formula~\ref{eqn:vol-theta-0} extends to all
simplices.  We write the extension of $\op{vol}\,\op{VC}_t(v,S)$
as
$$\op{vol}\,{\op{VC}^+_t}(v,S).$$


\subsection{scores}
\tlabel{sec:ssc}

We show that the function $\sigma$ can be expressed as a sum over
terms attached to each of the standard regions.

\begin{definition} \tlabel{def:standard-cluster}
A {\it standard cluster\/} is a pair $(R,D)$ where $(v,\Lambda)$ is a
centered packing and $R$ is one of its standard regions.  A {\it
quad cluster\/} is the standard cluster obtained when the standard
region is a quadrilateral.
\end{definition}
%
 \index{cluster!standard}
 \index{cluster!quad}
 \index{quad cluster}

%Recall $|S|$ is the convex hull of a set $S\subset
%\ring{R}^3$.

We break $\sigma$ into a sum
   \begin{equation}
   \sigma(v,\Lambda) = \sum_R\,\sigma_R(v,\Lambda),
   \end{equation}
indexed by the standard clusters $(R,D)$.  Let
   $$
   \op{VC}_R(v,\Lambda) = \op{VC}(v,\Lambda)\cap \op{cone}(R),
   $$
whenever $R$ is a measurable subset of the unit sphere.  Let
   $$
   \CalQ(v,R,\Lambda) = \{Q\in \CalQ(v,\Lambda) : \op{conv}^0(Q)\subset \op{cone}(R)\}.
   $$
By Lemma~\ref{lemma:Q-in-region},
 each $Q$ is entirely contained in the cone over a single
standard region.

\begin{definition} \tlabel{def:score-std-region}
If $(v,V,E)$ is compatible with $\Lambda$, then each $Q\in\CalQ(v,\Lambda)$
lies in a uniquely determined component $R$ of $Y(v,V,E)$.
Let $(v,V,E)$ be a compatible fan and let $R$ be a
component of $Y(v,V,E)$.  Set
      $$
      \op{svR}(v,R,\Lambda,\lambda) =
      \op{sovo}(v,\op{VC}_R(v,\Lambda),\lambda).
      $$
Set
      $$
      \sigma(v,R,\Lambda) = \op{svR}(v,R,\Lambda,\lambda_{oct}) 
      + \lambda_{oct,v}
         \sum_{Q\in\CalQ(v,R,D)} A_1(Q,c(Q,D),v_0).
      $$
\index{vzorR@$\op{svR}$} \index{zzsigmaR@$\sigma$}
\end{definition}

\begin{lemma} 
Let $(v,V,E)$ be a compatible fan.
$\sigma(v,\Lambda) = \sum_R\sigma_R(v,R,\Lambda)$, where the sum runs
over all components of $Y(v,V,E)$.
\end{lemma}

\begin{proof}
   $$
   \begin{array}{lll}
      \sigma(v,\Lambda)
      &= \lambda_{oct,v} (\op{vol}\,\Omega(v,\Lambda) + A_0(v,\Lambda))+16\pi/3\\
      &= \lambda_{oct,v} (\op{vol}\,\op{VC}(v,\Lambda)+\sum_{Q\in\CalQ(v,\Lambda)}
         A_1(Q,c(Q,D),v_0)) + (4) (4\pi/3)\\
      &= \sum_R \left (\lambda_{oct,v} \op{vol}\,\op{VC}_R(v,\Lambda) 
         +\lambda_{oct,v}
         \sum_{Q\in\CalQ(R,D)} A_1(Q,c(Q,D),v_0) +
         \lambda_{oct,s}\sol(v,R)\right).
   \end{array}
   $$
\end{proof}

Also, we have
    \begin{equation}
    \op{svR}(v,\Lambda)=\sum_{R\subset Y(v,V,E)}
    \op{svR}(v,R,\Lambda).
    \tlabel{eqn:vorD}
    \end{equation}

\begin{lemma}\tlabel{lemma:R'}
Let $(v,V,E)$ be a compatible fan.  If $R\subset Y(v,V,E)$
is a component that is disjoint from $\op{conv}^0(Q)$ for all 
$Q\in\CalQ(v,\Lambda)$, then
   $$
   \sigma(v,R,\Lambda) = \op{svR}(v,R,\Lambda).
   $$
If, on the other hand, $R = \op{aff}_+^0(v,\{v_1,v_2,v_3\})$ for
some $\{v,v_1,v_2,v_3\}\in\CalQ(v,\Lambda)$, then
   $$  
   \sigma(v,R,\Lambda) =  \sigma(v,Q,c(Q,D)).
   $$
\end{lemma}

\begin{proof} Substitute the definition of $A_1$
(Equation~\ref{eqn:a1-sigma}) into the definition of
$\sigma_R(v,\Lambda)$, noting that $\op{VC}(v,Q) 
= \op{VC}_{R}(v,\Lambda)$,
\end{proof}

\begin{remark}   Lemma~\ref{lemma:R'} explains why we have chosen
the same symbol $\sigma$ for the functions $\sigma(v,R,\Lambda)$ and
$\sigma(Q,c,v)$.  We can view Lemma~\ref{lemma:R'} as asserting a
linear relation in the functions $\sigma$:
   $$\sigma(v,R,\Lambda) = \sigma(v,R',\Lambda) + \sum \sigma(v,Q,c).$$
The sum runs over $Q\in\CalQ(v,\Lambda)$ that lie in the cone over $R$.
\end{remark}

\subsection{scoring}

By the results of Sections~\ref{x-2.7}, \ref{x-2.8}, \ref{x-2.9},
$\sigma(v,\Lambda)$ can be broken into a corresponding sum,
    $$
    \begin{array}{lll}
    \sigma_R(v,\Lambda) &= \sum_Q \sigma(Q) + \sigma(V_P),
                \hbox{ for quarters $Q$ in the $Q$-system, where}\\
    \sigma(V_P) &= \op{sovo}(\tildeV_P(t_0),\lambda_{oct})+  \sum_{\SA,\SB,\SC} \op{sovo}(V_S(t_S),\lambda_{oct})
        - \sum_v 4\doct\op{vol}(\delta_P(v)) - 4\doct\op{vol}(\delta'_P).\\
    \end{array}
    $$

By dropping the final term, $4\doct\op{vol}(\delta'_P)$, we obtain
an upper bound on $\sigma(V_P)$.  Because of the separation
results of Sections~\ref{x-2.7}--\ref{x-2.8},  we may score
$\tildeV_P(t_0)$ by Formula~\ref{eqn:3.7}. Bounds on the score of
simplices of type $\SB$ appear in \calc{193836552}.



\subsection{truncated tetrahedron}


We set
    \begin{equation}
    \begin{array}{lll}
    \op{sv}(S,t) &=
    \sol(S)\phi(t,t,\lambda_{oct})
    +\sum_{i=1,\ h_i\le t}^3 d_i (1-h_i/t) (\phi(h_i,t,\lambda_{oct})-
    \phi(t,t,\lambda_{oct})) \\
    &+\sum_{(i,j,k)\in S_3}
    \lambda_{oct,v}
    \quo(R(h_i,\eta(y_i,y_j,y_{k+3}),t)).
    \tlabel{eqn:3.5}
    \end{array}
    \end{equation}
In the definition, we adopt the convention that $\quo(R)=0$, if
$R=R(a,b,c)$ does not exist (that is, if the condition
    $0< a\le b\le c$
is violated). In the second sum, $S_3$ is the set of permutations
on three letters. This definition is compatible with
Definition~\ref{def:svor}.

We have
    \begin{equation}
    \begin{array}{lll}
    \op{svR}(v,P,\Lambda,t) &=
    \sol(P)\phi(t,t)
    +\sum_{|v_i|\le 2t} d_i (1-|v_i|/(2t)) (\phi(|v_i|/2,t)-
    \phi(t,t)) \\
    &-\sum_{R} 4\doct \quo(R).
    \tlabel{eqn:3.7}
    \end{array}
    \end{equation}
The first sum runs over vertices in $P$ of height at most $2t$.
The second sum runs over Rogers simplices $R(|v_i|/2,\eta(F),t)$
in $P$, where $F=\{v_0,v_1,v_2\}$ is a face of circumradius
$\eta(F)$ at most $t$, formed by vertices in $P$.  The constant
$d_i$ is the total dihedral angle along $\{v_0,v_i\}$ of the
standard cluster. The truncations $t=t_0=1.255$ and $t=\sqrt2$
will be of particular importance.
    Set $A(h) = (1-h/t_0) (\phi(h,t_0)-\phi(t_0,t_0))$.\index{A}

\begin{remark}  We have introduced both untruncated and truncated
versions of functions $\op{svR}$ and $\sigma$.  The truncated versions
are used to give upper bounds on the untruncated versions.  For
example,  in the function $\sigma(v,\Lambda)$, the $V$-cell contributes
through its volume.  The volume
appears with a negative coefficient 
$\lambda_v=-4\doct$.  Thus, we obtain an
upper bound on $\sigma(v,\Lambda)$ by discarding bits of volume from the
$V$-cell.   This suggests that we might try to give upper bounds
on the score $\sigma(v,\Lambda)$ by truncating the $V$-cell in various
ways. This is the reason for the truncated versions of these
functions.
\end{remark}




\subsection{squander}% DCG 10.1, p99
    %\heads{3. Functions}

%Set $\zeta^{-1}:=\sol(S(2,2,2,2,2,2))=2\arctan(\sqrt{2}/5)$. The
%constant $\zeta$ is related to the other fundamental constants by the
%relations $\pt= 2/\zeta-\pi/3$ and $\doct=(\pi-2/\zeta)/\sqrt{8}$.
%Rogers's bound is $\sqrt{2}/\zeta\approx 0.7796$.

% Plus Formula 7 on scores.

We consider the functions
    $\sigma_R(v,\Lambda)-\lambda\zeta\sol(R)\,\pt$,
for $\lambda=0$, $1$, or $3.2$, where $R$ is a standard cluster.
%The constant $3.2$ was determined experimentally.
We write
    $$
    \tau_R(v,\Lambda) = \sol(R)\zeta\,\pt -
    \sigma_R(v,\Lambda).
    $$
We will see that $\tau_R(v,\Lambda)$ has a simple interpretation.  If $(v,\Lambda)$
is a centered packing with standard clusters $\{R\}$, set $\tau(v,\Lambda)
= \sum_{R}\tau_R(v,\Lambda)$.
\smallskip



\begin{lemma}\label{lemma:sigma-tau}
    %\proclaim{Lemma 3.2}
    $$\sigma(v,\Lambda) = {4\pi \zeta\,\pt} - \tau(v,\Lambda).$$
\end{lemma}

\begin{proof} Let $\{R\}$ be the standard clusters in $(v,\Lambda)$. Then
    $$
    \sigma(v,\Lambda) = \sum_R\sigma_{R_i}(v,\Lambda) +
        (4\pi-\sum_R\sol(R_i))\zeta\,\pt = 4\pi \zeta\,\pt - \sum_R\tau_{R_i}(v,\Lambda).
    $$
\end{proof}


\begin{lemma}
If there are standard clusters $R_1,\ldots,R_k$ such that
$$\sum_{i=1}^k \tau_{R_i}(v,\Lambda)> \squander,$$
then $(v,\Lambda)$ does not contravene.
\end{lemma}

\begin{proof}
$$\sigma(v,\Lambda) = 4\pi\zeta,\pt -\sum_R{R_i}(v,\Lambda) < 8\,\pt.$$
\end{proof}

We note that $14.8\,\pt > \squander$.  We sometimes use this
approximation.


The function $\tau_R(v,\Lambda)$ gives the amount {\it squandered\/} by a
particular standard cluster $R$.  If nothing is squandered, then
$\tau_{R_i}(v,\Lambda)=0$ for every standard cluster, and the upper bound
on $\sigma(v,\Lambda)$ is
    $4\pi\zeta\,\pt\approx 22.8\,\pt$.



%% XX Deleting mentioned \calc{629256313},
%% \calc{917032944}, \calc{738318844}, and \calc{587618947}.
%% Perhaps they are no longer needed in the proof!

%% Major Deletion: SVN:16 has proof of local optimality. Gone in SVN:23.




\section{Special Structures}

\subsection{Classification of Quadrilateral Regions}
\label{sec:quad-class}


\begin{lemma}\label{quad:classify}
Let $C=(v_0,(v_1,v_2,v_3,v_4))$ be a quad cluster.  Then it has
exactly one of the following forms.
\begin{itemize}
  \item (flat) $\{v_0,v_1,v_2,v_3\}$ and $\{v_0,v_1,v_3,v_4\}$ are flat
    quarters in the $Q$-system.
   \item (flat) $\{v_0,v_2,v_4,v_1\}$ and $\{v_0,v_2,v_4,v_3\}$ are flat
     quarters in the $Q$-system.
     \item (octahedron) There exists an enclosed vertex $w$, such that
       $(v_0,w,v_1,v_2,v_3,v_4)$ is a quartered octahedron with
       four upright quarters in the $Q$-system.
   \item (pure) We have $|v_1-v_3|>\sqrt8$, $|v_2-v_4|>\sqrt8$ and
     there is no enclosed vertex $w$ with $|w|<\sqrt8$.
    \item (mixed) We have $|v_1-v_3|>\sqrt8$, $|v_2-v_4|>\sqrt8$, there
      is no enclosed vertex forming a quartered octahedron, but
      there exists some enclosed vertex $w$ with $|w|<\sqrt8$.
      \index{pure}\index{mixed}
\end{itemize}
\end{lemma}

\begin{proof}
If the quad cluster has a diagonal of length at most $\sqr8$
between two corners, there are three possible decompositions. (1)
The two quarters formed by the diagonal lie in the $Q$-system so
that the scoring rules for the $Q$-system are used.  (2) There is
a second diagonal of length at most $\sqr8$, and we use the two
quarters from the second diagonal for the scoring. (3) There is an
enclosed vertex that makes the quad cluster into a quartered
octahedron and the four upright quarters are in the $Q$-system.

Now suppose that neither diagonal is less than $\sqr8$ and the
quad cluster is not a quartered octahedron. If there is no
enclosed vertex of length at most $\sqr8$, the quad cluster
contains no quarters. An upper bound on the score of the quad
cluster $(P,D)$ is $\op{svR}(D,P,\sqr2)$. The remaining cases are
called {\it mixed\/} quad clusters. Mixed quad clusters enclose a
vertex of height at most $\sqr8$ and do not contain flat quarters.
\end{proof}



\begin{definition}
Define the {\it central vertex\/} $v$ of a flat quarter to be the
vertex for which $\{v_0,v\}$ is the edge opposite the diagonal.
\end{definition}






%%%%%%%%%%%%%%%%  LOCAL BOUNDS %%%%%%%%%%%%%%%%


\chapter{Local Bounds}



\section{A Sample Proof}

A large number of bounds on the function
$\sigma$ are required in this book.  In fact, the main
point of the book is to prove the bound
  $$
  \sigma(v,\Lambda)\le 8\,\pt,
  $$
for any centered packing $(v,\Lambda)$.
Hundreds of pages of texts in \cite{DCG} are devoted to
proving various bounds on $\sigma$.  A careful analysis of
this text reveals that these arguments are largely repetitive.
Through further examination of this text, we can replace these
long arguments with various structures that permit the automation
of these proofs.  This process may seem artificial at first.
However, the savings will be significant.  We will succeed in
trivializing over a hundred of pages of estimates from \cite{DCG}.

To ease our way into this new way of presenting estimates, 
we make a detailed analysis of
one proof from \cite{DCG}.  The following lemma and its proof are
based on \cite{DCG}[Lemma~12.4].  The proof will quote results that
have not yet been proved.  We will accept these results 
as established facts for the time being, 
so that we can concentrate on the structure
of the proof as presented.


\begin{lemma}\label{lemma:nonagon}  
Let $(v_0,\Lambda)$ be a centered packing.
Let $(v_0,V,E)$ be the standard fan associated with $(v_0,\Lambda)$,
and let $H=\op{hyp}(v_0,V,E)$ be the associated hypermap.  
Suppose that there is a contour loop $C$ in $H$ that visits
at most nine nodes.  Assume that the contour loop does not make
repeat visits to the same node.
Let $\{V_1,\ldots,V_r\}$ be the components of $Y(v_0,V,E)$ that
are interior to the contour loop $C$.  (XX explain terminology.)
Set 
   $$
   \begin{array}{lll}
   \sigma_C &= \sum_{i=1}^r \sigma(v_0,V_i,\Lambda),\\
   \tau_C &= \sum_{i=1}^r \tau(v_0,V_i,\Lambda).\\
   \end{array}
   $$
Assume that
    $$
    \sigma_C \le s_9,\quad \text{ and } \quad \sigma_C \ge t_9,
    $$
where 
    $$
    s_9 = -0.1972 \quad \text{ and } t_9 = 0.6978.
    $$
Then, $(v_0,\Lambda)$ does not contravene.  That is,
$$
\sigma(v_0,\Lambda) \le 8\,\pt - \epsilon_0.
$$
\end{lemma}

First we will give the proof in the same form as \cite{DCG}.
After presenting the proof, we will add further commentary and analysis.


\begin{proof}  
By Lemma~\ref{XX},
  $$
  \sigma(v_0,\Lambda) = \sum_{R}\sigma(v_0,R,\Lambda),
  $$
as $R$ runs over the components of $Y(v_0,V,\Lambda)$.
We wish to show that
  $$
  \sum_{R}\sigma(v_0,R,\Lambda)\le 8\,\pt -\epsilon_0,
  $$
By Lemma~\ref{XX}, this is equivalent to
  \begin{equation}
  \sum_{R} \tau(v_0,R,\Lambda)\ge (4\pi\zeta-8)\,\pt-\epsilon_0.
  \label{eqn:targeted}
  \end{equation}
By Lemma~\ref{XX}, we have $\tau(v_0,R,\Lambda)\ge0$, for
each component $R$.  Thus, it is enough to show that
Inequality~\ref{eqn:targeted} holds when the sum is restricted
to some subset of the components.

The given bounds on $\sigma_C$ and $\tau_C$ are
already sufficient to conclude that all other components of
$Y(v_0,V,E)$ are triangles:
   $$
   t_9 + t_4 > (4\pi\zeta -8)\,\pt + \epsilon_0.
   $$
(The lower bound $t_4$ on a component that is not a triangle apepars
in Theorem~\ref{thm:the-main-theorem}.)
The hypermap has no nodes of type $(4,0)$ or $(6,0)$:
$$
   t_9 + 4.14\,\pt > (4\pi\zeta-8)\,\pt + \epsilon_0.
$$
(The lower bound $4.14\,\pt$ on the $p$ triangles at a vertex
of type $(p,0)$ appears in Table~\ref{eqn:old5.1}.)

Suppose that there are $\ell$ nodes of type $(5,0)$.  The
contour cycle has visits  $m\le 9$ nodes and does
not make repeat visits to the same node.  By Lemma~\ref{XX}, 
%% Lemma XX not yet typed.
there are $m-2 + 2\ell\le 7+2\ell$ 
triangular components $U_1,\ldots,U_{m-2+2\ell}$
disjoint from $V_1,\ldots,V_r$, such that
   $$
   \{U_1,\ldots,U_{m-2+2\ell},V_1,\ldots,V_r\}
   $$
are all the components of $Y(v_0,V,E)$.

If $\ell\le 3$, then the result follows by the estimate
   $$
   s_9 + [(7+2\ell)\,\pt - 0.48\,\ell\,\pt] < 8\,\pt - \epsilon_0.
   $$
(The upper bound in square brackets follows from Lemma~\ref{lemma:0.55}.)
If, on the other hand, $\ell\ge 4$, then the result follows from
the estimate
   $$
   t_9 + 4 (0.55)\,\pt > (4\pi\zeta -8)\,\pt + \epsilon_0.
   $$
(The upper bound $4(0.55)\,\pt$ is found in Lemma~\ref{lemma:0.55}.)
This completes the proof.
\end{proof}

\subsection{initial analysis}

We turn to an analysis of the lemma and proof.  This type of result is what
we will call an ``assembly problem.''  This type of result
assembles estimates that have been established elsewhere, without
any attempt to improve the estimates.  The proof
does not require any deep geometrical understanding.  The geometry
has been entirely encapsulated in the basic framework of
fans, the hypermap, and the components
of $Y(v_0,V,E)$.  The one combinatorial fact that is needed
(the number of triangles is $m-2+2\ell$) 
has been established elsewhere.  It is not necessary
to understand anything about the nonlinear functions $\sigma(v_0,R,\Lambda)$
and $\tau(v_0,R,\Lambda)$, other than the cited estimates.   
The displayed inequalities are inequalities of real numbers,
which are easily checked by computer
or even a hand-held calculator.  There is no arithmetic beyond
the displayed equations.
In summary,
the proof is nearly void of any substance.   This emptiness
works to our advantage.

At the same time, some intelligence is required to discover the
correct way to assemble the estimates into a proof.   There are
many ways that we might attempt to assembly the proof from estimates
established elsewhere, and various blind attempts at assembly will
fail.  

In the example we provide the search space for the proof
branches in two ways.  First of all, there is the usual branching
into various cases.  Either, there is a component other than
$R_1,\ldots,R_r$ with more than three sides, or there isn't.
Either there is a node of type $(p,0)$, with $p\ne 5$; or there isn't.
Either the number of nodes of type $(5,0)$ is less than four; or it isn't.

The second type of variation among possible proofs comes from
the grouping of standard components and the choice of which estimate
to apply to each group.  In the case, $t_9+t_4$ of the proof, the
standard components are combined into three groups:
the group surrounded by the contour loop $C$ in the hypermap
(leading to the term $t_9$), the single component that is not triangular
(leading to the term $t_4$), and all other components (which have
the unmentioned lower bound of $0$).  The term
   $$
   t_9 + 4(0.55)pt 
   $$
comes from a combination of three groups:
the components surrounded by the contour loop $C$, the triangular
regions around four of the nodes of type $(5,0)$, and all other
components (which have the unmentioned lower bound of $0$).
As we will see, the choice of whether to use a bound based on
the function $\sigma$, or a bound based on $\tau$ can be
viewed as a variation of this second type.

\subsection{list form of the proof}

The main outline of this proof can be put into a list:
\bigskip

\centerline{\it centered packings whose standard hypermap has nonagonal face\hss}

\begin{itemize}
\item {\tt if:} there is a non-triangular face:
\begin{itemize}
\item $-t_9$,
\item $-t_4$,
\item $4\pi\zeta\,\pt$.
\end{itemize}
\item {\tt else if:} there is a node of type $(p,0)$, with $p\ne 5$.
\begin{itemize}
\item $-4.14\,\pt$,
\item $-t_9$,
\item $4\pi\zeta\,\pt$.
\end{itemize}
\item {\tt else if:} the number $\ell$ of type $(5,0)$ nodes is at most $3$.
\begin{itemize}
\item $s_9$
\item $[(7+2\ell)\,\pt - 0.48\,\ell\,\pt]$
\end{itemize}
\item {\tt else:} the number of type $(5,0)$ nodes is at least $4$.
\begin{itemize}
\item $-t_9$,
\item $-4(0.55)\,\pt$,
\item $4\pi\zeta\,\pt$.
\end{itemize}
\end{itemize}

Note that we have made use of the identity
$\sigma = 4\pi\zeta\,\pt - \tau$
explicitly by inserting the constant $4\pi\zeta\,\pt$ wherever
appropriate, and negating the $\tau$-terms.

The two types of variation in the proof can be seen by the way
we have presented the list.  The branching according to cases
is labeled with {\tt if} and {\tt else} keywords.  The variation
that comes by the grouping of terms is found in the lists without
these keywords.  

The entire list can be {\it evaluated}.  The value of
a sublist marked with $\star$ is the sum of the items in the list.
For example, the value of 
\begin{itemize}\item[]
\begin{itemize}
\item $-t_9$,
\item $-4(0.55)\,\pt$,
\item $4\pi\zeta\,\pt$.
\end{itemize}
\end{itemize}
is just $-t_9 - 4(0.55)\,\pt + 4\pi\zeta\,\pt$.
The value of a list with bulleted items
is the maximum of the items in the list.
The main conclusion of the lemma is that the maximum value from all
the various cases at the outermost level is at most $8\,\pt -\epsilon_0$.


\subsection{assembly proofs as rooted trees}

We continue with an informal analysis of the proof of Lemma~\ref{lemma:nonagon}.
A well-known construction allows us to associate a rooted
tree with our nested list.  We simply create a node in the tree
for every item that appears at any level of the nested list.  We
draw an edge from each item to each item nested beneath it.
The root of the tree is the outermost entry.  In our example, the
outermost entry is the centered packing with a nonagonal face.
The leaves (sinks) sinks in the tree correspond to innermost list
items, that is, those with no sublists.  In our example, the graph
appears as Figure~\ref{fig:rooted-tree}. 

%% WW not yet done.
\begin{figure}[htb]
  \centering
  \myincludegraphics{noimage.eps}
  \caption{A proof represented as a rooted tree}
  \label{fig:rooted-tree}
\end{figure}

Up to this point, we have clarified the structure of the proof in
our example.  However, we wish to carry this analysis much further.
We wish give a careful specification of the preceding construction
so that a large number of proofs 
can be represented as rooted trees, then
expressed as a piece of computer code, and finally automatically evaluated.
With this in mind, we give a careful definition of the rooted
trees and their nodes.

%% http://en.wikipedia.org/wiki/Tree_%28graph_theory%29 

\begin{definition}
A tree is a graph in which any two vertices are joined by a unique path.
A directed tree is a directed graph in whose underlying graph is a tree.
A source in a directed graph is a node that is not the terminal node
of any directed edge.  A sink is a node that is not the initial node
of any directed edge.  We use the term leaf as a synonym of sink.
A directed tree is rooted if there is a unique node (called
the root) that is a source.
A labeling of a graph $G$ from a set $X$ is a map from $X$ to the set
of vertices of $G$.
\end{definition}

Any tree with a distinguished node can be made into a directed
tree in a unique way that makes the distinguished node the root. 
For that reason, we write {\it rooted tree} rather than
{\it rooted directed tree}.

The labels on the nodes much be a data structure that is sufficiently
rich to keep track of the {\it state} of whatever estimates are
being made.
We will consider finite labeled rooted trees, whose label set
is a product $LA=\{{\tt sum, max}\} \times {\cal C}$.   The first
component of the label identifies its {\it type} as either a sum
node or product node. 
The second component of the label is what we call a {\it cluster}.
The set of clusters is defined as a disjoint union of three different sets
of clusters:  vertex clusters, LP clusters, and constant clusters.
Constant clusters are the easiest to define: a constant cluster
is a real number $x\in\ring{R}$.  

\begin{definition}
A vertex
cluster is a tuple
    $$
    (v,V,E,\Lambda,{\cal R},{\cal Q},\ell,x,b),
    $$
where $(v,V,E)$ is a fan that is compatible with
the centered packing $(v,\Lambda)$.  
The set ${\cal R}$ is a subset of the set of components of $Y(v,V,E)$,
${\cal Q}$ is a set of quarters and quasi-regular tetrahedra,
$\ell\in I$ is a discrete index from some set $I$ (which we can
take to be $\ring{Z}$, for example), 
$x$ is a real number, and
$b$ is either true or false.
\end{definition}

Intuitively, we consider $\Lambda$ as the sphere packing under
consideration, with center $v$, and fan $(v,V,E)$.
The purpose of the set ${\cal R}$ is to specify which components
enter into the approximation.  The purpose of $\ell$ is to specify
which (of finitely many) approximation functions to use on the
regions ${\cal R}$.  The parameter $x$ is a constant that in \cite{DCG}
is called a {\it penalty}.  The cluster approximation is to be
thought of as $x$ plus the value of the function $\ell$ on ${\cal R}$.
Finally, the boolean parameter $b$ keeps track of the branching into
cases.

We delay the definition of LP clusters.  They are approximations
obtained through linear programs.



There is to be an evaluation function $\op{val}$, which assigns a value
to each cluster.  The value of a constant cluster $x$ is defined to be $x$.
The evaluation function will eventually encompass a large number of
separate cases, distinguished by the index $\ell$.  
In fact, virtually every approximation
to any of the functions $\sigma$ and $\tau$ that appears anywhere
in this book appears as case of the evaluation function for some index
$\ell$.

The set of values of the evaluation function is 
$\ring{R}^\star = \ring{R}\cup\{-\infty\}$.
Addition, $\max$, and inequalities are to be extended from
$\ring{R}$ to $\ring{R}^\star$ in the usual way.  For example,
  $$
  \begin{array}{lll}
    -\infty + x &= -\infty,\\
    -\infty &< & x,\\
    \max(-\infty,x) &= x,\\
    \end{array}
  $$

The purpose of the value $-\infty$ is the following.  Various
clusters may carry the {\it false} boolean flag $b$.  Or the
LP clusters may represent an infeasible linear program.  When
this happens, the most convenient way to handle evaluation is
to assign the value $-\infty$.

\begin{definition}
Let $\op{val}$
be a function $LA\to \ring{R}^\star$.  We say that
a rooted tree $T$ is compatible with $\op{val}$ if for every
node $t$ in the tree, we have
   $$
   \begin{array}{lll}
     \end{array}
   $$
\end{definition}

Either max or sum condition depending on the type: XX.

\section{Triangles}


\subsection{positivity}%DCG 9.10, p98 %% Delta(v) stuff
    \oldlabel{2.11}
    \label{sec:pos}

The function $\op{sovo}$ depends on a parameter $\lambda$.
A frequently used choice of parameter $\lambda$ is
$$
 %\lambda_{oct}=(\lambda_v,\lambda_s)=(-4\doct,1/3).
 \lambda_{sq} =(\lambda_v,\lambda_s)=(4\doct,\zeta\,\pt-1/3).
$$
\index{sovo}\index{ZZlambda@$\lambda$}
%\index{ZZlambdaoct@$\lambda_{oct}$}
\index{ZZlambdasq@$\lambda_{sq}$}


\begin{lemma}\label{lemma:rog-squ0}
Let $R=\op{rog}^0(v_0,v_1,v_2,v_3,t_0)$ be a Rogers simplex
with $|v_i-v_j|\ge 2$ for $i,j=0,1,2$, $i\ne j$.
Then $\op{sovo}(v_0,R,\lambda_{sq})\ge 0$.
\end{lemma}

\begin{proof}
The $abc$-parameters are $1 \le |v_i-v_j|/2 = a$,
$2/\sqrt{3}\le b = \eta_V(v_0,v_1,v_2)$ (Lemma~\ref{tarski:XX}),
and $c = t_0 > \sqrt6/2$.  By Lemma~\ref{lemma:rog-tet},
$$
0 = \op{sovoR}(1,2/\sqrt{3},\sqrt{6}/2,\lambda_{sq}) 
   \le \op{sovoR}(a,b,c,\lambda_{sq}).
$$
\end{proof}

XX Move following to primitive vol

\begin{lemma}\label{lemma:phi-sq0}
For $1\le h\le t_0$, we have
$$
\phi(h,t_0,\lambda_{sq})\ge 0.
$$
\end{lemma}

\begin{proof}
This is a polynomial in $h$, which is easy to evaluate over the
given range.
\end{proof}

%% XX Move to primitive volume.

\begin{lemma}\label{lemma:pl-sq0}
Let $PL(v_0,v_1,w_1,w_2,t_0)$ be a plate with
vertices $v_i,w_i$ in a packing $\Lambda$ and $|w_1-w_2|\ge 2 t_0$.
Then
$\op{sovo}(v_0,PL(v_0,v_1,w_1,w_2,t_0),\lambda_{sq}) \ge 0$. 
\end{lemma}

\begin{proof} In the notation of Definition~\ref{def:plate},
a plate is a composite of Rogers simplices and
a frustum $W''\cap FR(v_0,v_1,h,h/t)$,
where $2h = |v_0-v_1|$.  (The assumption $|w_1-w_2|\ge 2 t_0$
is needed for the Assumption~\ref{eqn:q1q2}.)
It is enough to show non-negativity for each piece in the composite.
The non-negativity of the Rogers simplex follows from Lemma~\ref{lemma:rog-squ0}.
For the frustum, we have by Lemma~\ref{lemma:sovoFR},
$$
\begin{array}{lll}
\op{sovo}(v_0,W''\cap FR(v_0,v_1,h,h/t_0),\lambda_{sq}) &=
  \sol(W''\cap FR(v_0,v_1,h,h/t_0) \phi(t_0,t_0,\lambda_{sq})\\
\end{array}
$$
Hence, the result follows from Lemma~\ref{lemma:phi-sq0}.
\end{proof}

%% XX Might want to generalize to more than standard components.

\begin{lemma}
    \label{lemma:tau-positive}
    Let $R$ be a standard component that is not a triangle in a
    centered packing $(v_0,\Lambda)$.
    $\tau_{0}(v_0,R,\Lambda)\ge 0$.
\end{lemma}

\begin{proof}
The function $\tau_0$ is expressed on a standard component as a sum
of terms
  $$
  \op{sovo}(v_0,PL,\lambda_{sq}),
  $$
for various plates $PL$ and
   $$
   \op{sovo}(v_0,B,\lambda_{sq}),
   $$
for some $t_0$-radial measurable set.  %% XX Reference this decomposition.
The plates give a non-negative contribution by Lemma~\ref{XX},
and the $t_0$-radial measurable set gives a contribution
   $$
   \sol(v_0,B)\phi(t_0,t_0,\lambda_{sq}),
   $$
which is positive by Lemma~\ref{lemma:phi-sq0}.
\end{proof}


\begin{lemma}\label{lemma:roger0}
    %proclaim{Lemma 3.1}
    %\oldlabel{part3.3.1}
    $\tau(v,R,\Lambda)\ge 0$, for all standard components $R$.
\end{lemma}

XX Doesn't this proof need quarter estimates and exceptional region
estimates?

\begin{proof}
If $R$ is not a quasi-regular tetrahedron, then $\sigma(v,R,\Lambda)\le0$
by Theorem~\ref{lemma:quad0} and $\sol(R)> 0$, so that the result
is immediate. If $R$ is a quasi-regular tetrahedron, the result
appears in the archive of inequalities \calc{53415898}.
\end{proof}



\begin{lemma}
        \label{lemma:no-enclosed-tri}
        A triangular standard region does not contain any enclosed
        vertices.
\end{lemma}

\begin{proof}
    This fact is proved in Lemma~\ref{lemma:2t0-doesnt-pass-through}.
\end{proof}





\subsection{types}\label{sec:types}%DCG 10.2, p100

Let $v$ be a vertex of height at most $2t_0$.  We say that $v$ has
{\it type\/} $(p,q)$ if every standard region with a vertex at $\bar
v$ (the radial projection of $v$) is a triangle or a quadrilateral,
and if there are exactly $p$ triangular faces and $q$ quadrilateral
faces that meet at $\bar v$.  We write $(p_v,q_v)$ for the type of
$v$.









\section{Bounds in Quadrilateral Regions}%DCG 10.4, p104
    \label{sec:bounds}



\subsection{pure bound}%DCG 8.2, p73



\begin{lemma} \label{lemma:wedge} Consider the wedge of a cone
    $$
    W =W(\alpha,z_0) =
    \{ t\, x : 0\le t \le 1, x\in P(\alpha,z_0)\}\subset\ring{R}^3,
    $$
where $P(\alpha,z_0)$ has the form
    $$
    P = \{(x_1,x_2,x_3) :
    x_3 = z_0,\   x_1^2+x_2^2+x_3^2\le 2,\ 0\le x_2\le \alpha x_1\},
    $$
with $z_0\ge1$.  Let $A$ be the volume of the intersection of the
wedge with $B(0,1)$. Then
    $$A\le\doct\,\op{vol}(W).$$
Equality is attained if and only if $W$ has zero volume.\index{cone}
\end{lemma}

\begin{proof} This is calculated in \cite[Sec. 4]{part2}.  See the
second frame of Figure~\ref{fig:doct}.
\end{proof}

\begin{lemma} \label{lemma:cone}
Let $C$ be the cone at $v_0$ over a set $P$, where $P$ is
measurable and every point of $P$ has distance at least $1.18$
from $v_0$.  Let $A$ be the volume of the intersection of $C$
with $B(v_0,1)$. Then
    $$A\le\doct\,\op{vol}(C).$$
Equality is attained if and only if $C$ has zero volume.
\end{lemma}

\begin{proof} The ratio $A/\op{vol}(C)$ is at most $1/1.18^3 < \doct$.   See the
first frame of Figure~\ref{fig:doct}.
\end{proof}

\begin{figure}[htb]
  \centering
  \myincludegraphics{\ps/haII42.ps}
  \caption{Some sets of low density.}
  \label{fig:doct}
\end{figure}

\begin{lemma}\label{lemma:pure0}
Let $(R,D)$ be a pure quad cluster.  Then
  $\sigma_R(v,\Lambda)\le 0$.
\end{lemma}

\begin{proof}  The pure quad cluster breaks into the types
of regions of low density described by Lemmas~\ref{lemma:cone},
\ref{lemma:wedge}, and \ref{lemma:rog-doct}.  XX FILL IN DETAILS XX.
\end{proof}




\subsection{quad cluster bound}

\begin{lemma} \label{lemma:quarter0}\dcg{Lemma 8.12}
Let $Q$ be a quarter in the $Q$-system (either flat or upright).
Then $\sigma(Q)\le 0$. 
\end{lemma}

\begin{proof} This is an assembly problem.\footnote{JPOMPNK}
\end{proof}


The following theorem is also one of the main results of this
\chap. It is a key part of the proof of local optimality.


\begin{theorem}\label{lemma:quad0} Let $(R,D)$ be a quad cluster.
Then $\sigma_R(v,\Lambda)\le 0$.
\end{theorem}\index{cluster!quad}

\begin{proof}
The types of quad clusers have been classified in Lemma~\ref{lemma:quad-class}.
We prove the bound for each type.
If it consists of two flat quarters, then the result is
Lemma~\ref{lemma:quarter0}.  If it is a quartered octahedron with
four upright quarters, then the result again follows from
Lemma~\ref{lemma:quarter0}.  If it is a mixed quad cluster,
the result follows from Lemma~\ref{lemma:1.04}.  Finally,
if it is a pure quad cluster, then an upper bound on the score
is given by $\sigma_R(D,\sqrt2)$ by Lemma~\ref{lemma:pure0}.  
\end{proof}







\subsection{local optimality}%DCG 8.1, p72
\label{sec:local-opt}

\begin{lemma}  %=claim\label{claim-F}
Contravening centered packings $(v,\Lambda)$ exist such that
$\sigma(v,\Lambda)=8\pt$. If $(v,\Lambda)$ is a contravening centered packing, and
if the hypermap of $(v,\Lambda)$ is isomorphic to $G_{fcc}$ or $G_{hcp}$,
then $\sigma(v,\Lambda) \le 8\,\pt$.
\end{lemma} %\label{lemma:local-optimality} in local_opt.tex

\begin{proof}
In each of these two hypermaps there are $8$ triangles and
$6$ quadrilaterals.  In the corresponding centered packings,
there are  eight quasi-regular tetrahedra and six quad clusters.
In each triangular region $\sigma_R(v,\Lambda)\le 1,\pt$ by Lemma~\ref{lemma:1pt}.
In each quad cluser $\sigma_R(v,\Lambda)\le 0$ by Lemma~\ref{lemma:quad0}.  
Thus, the total is
at most $8\,\pt$.
\end{proof}











\subsection{a mixed quad bound}%DCG 10.5, p107

In Definition~\ref{def:delta-e}, we found a region $\delta(v)$
that lies outside the ball of radius $t_0$ at $0$ but inside
$\op{VC}(v_0)$.  A formula for its volume is developed
in Section~\ref{sec:anc}.  It introduces two functions
$\cro$ and $\anc$.


\smallskip
If $(P,D)$ is a mixed quad cluster, let $(P,D')$ be the new quad
cluster obtained by removing all the enclosed vertices.  We define
a $V$-cell $V(P,D')$ of $(P,D')$ and the truncation of $V(P,D')$
at $t_0$. We take its score $\op{vor}_{0,P}(D')$  as we do for
standard clusters.  $(P,D')$ does not contain any quarters.

\begin{lemma} \label{lemma:mixed-vor0}
%\proclaim{Proposition 4.7}
If $(P,D)$ is a mixed quad cluster, $\sigma_P(D') <
\op{svR}(v,P,\Lambda,t_0)$.  
% Moreover, we can erase any number of the enclosed 
% vertices over the mixed quad cluster.
\end{lemma}

\begin{proof}
%
Suppose there exists an enclosed vertex that has context
$\x(2,1)$; that is, there is a single upright quarter
$Q=S(y_1,y_2,\ldots,y_6)$ and no additional anchors.  In this
context $\sigma(Q)=\mu(Q)$. Let $v$ be the enclosed vertex.  To
compare $\sigma_P(v,\Lambda)$ with $\op{svR}(P,D')$, consider the $V$-cell
near $Q$. The quarter $Q$ cuts a wedge of angle $\dih(Q)$ from the
crown at $v$. There is an anchor term for the two anchors of $v$
along the faces of $Q$. Let $V_P^v$ be the truncation at height
$t_0$ of $V_P$ near $v$ and under the four Rogers simplices
stemming from the two anchors.
(Figure~\ref{fig:anchor-quarter:bis} shades the truncated parts of
the quad cluster.) As a consequence
\smallskip
    \begin{equation}
        \op{sovo}(V_P,\lambda_{oct}) <(1-\dih(Q)/(2\pi))\cro(y_1/2)+\anc(y_1,y_2,y_6)
        +\anc(y_1,y_3,y_5) +\op{sovo}(V_P^v,\lambda_{oct}).
    \label{eqn:4.8}
    \end{equation}
Combining this inequality with Lemma~\ref{a:contex21}, we get the
result.

Now suppose there is an enclosed vertex $v$ with context
$\x(3,1)$. Let the quad cluster have corners $v_1$, $v_2$, $v_3$,
$v_4$, ordered consecutively.  Suppose the two quarters along $v$
are $Q_1=\{v_0,v,v_1,v_2\}$ and $Q_2=\{v_0,v,v_2,v_3\}$.  We consider
two cases.

\noindent Case 1:  $\dih(Q_1)+\dih(Q_2)<\pi$ or
$\rad(v_0,v,v_1,v_3)\ge\eta(|v|,2,2t_0)$. In this case, the use of
correction terms to the crown are legitimate as in
Definition~\ref{def:wedge}. Proceeding as in context $\x(2,1)$, we
find that
\smallskip
    \begin{equation}
    \op{sovo}(V_P,\lambda_{oct}) < (1-(\dih(Q_1)+\dih(Q_2))/(2\pi))\cro(|v|/2)
    +\anc(F_1) +\anc(F_2) +\op{sovo}(V_P^v,\lambda_{oct}).
    \label{eqn:4.10}
    \end{equation}
Here $V_P^v$ is defined by the truncation at height $t_0$ under the
$V$-face determined by $v$ and under the Rogers simplices stemming
from the side of $F_i$ that occur in the definition of $\anc$. Also,
$\anc(F_i)=\anc(y_i,y_j,y_k)$ for a face $F_i$ with edges $y_i$
along an upright quarter. By a
calculation\footnote{\calc{554253147}} applied to both $Q_1$ and
$Q_2$, we have
    \begin{equation}
    \op{sovo}(V_P,\lambda_{oct}) +\sum_{i=1}^2\sigma(Q_i)
    < \op{sovo}(V_P^v,\lambda_{oct}) + \sum_{i=1}^2 \op{sv}_0(Q_i).
    \label{eqn:4.11}
    \end{equation}
That is, by truncating near $v$, and changing the scoring of the
quarters to $\op{sv}_0$, we obtain an upper bound on the score.

\noindent Case 2:  $\dih(Q_1)+\dih(Q_2)\ge\pi$ and
    $\rad(v_0,v,v_1,v_3)\le \eta_0(|v|/2)$.
 In the mixed case,
$\sqr8<|v_1-v_3|$, so
$$\sqr2<{\frac{1}{2}}|v_1-v_3|\le\rad \le \eta_0(|v|/2),$$
and this implies $|v|\ge 2.696$. We
have\footnote{\calc{855677395}}
$$\sum_{i=1}^2 \sigma(Q_i) < \sum_{i=1}^2 \op{sv}_0(Q_i) +
\sum_{i=1}^2 0.01(\pi/2-\dih(Q_i))< \sum_{i=1}^2 \op{sv}_0(Q_i).$$
Inequality~\ref{eqn:4.11} holds, for $V_P^v=V_P$.

In the general case, we run over all enclosed vertices $v$ and
truncate around each vertex.  For each vertex we obtain
Inequality~\ref{a:context21} or \ref{eqn:4.11}. These inequalities can
be coherently combined over multiple enclosed vertices because the
$V$-faces were associated with different vertices $v$ and none of
the Rogers simplices used in the terms $\anc()$ overlap. More
precisely, if $Z$ is a set of enclosed vertices, set $V_P^Z =
\cap_{v\in Z} V_P^v$, and $V_P^{v,Z} = V_P^Z\cap V_P^v$. Coherence
means that we obtain valid inequalities by adding the superscript
$Z$ to $V_P$ and $V_P^v$ in Inequalities~\ref{context21} and
\ref{eqn:4.11}, if $v\not\in Z$. In sum,
    $\sigma_P(v,\Lambda) < \op{svR}(v,P,\Lambda,t_0)$.
%
\end{proof}


%% INTRO SPIV

\section{Quarters} %DCG 11.
    \label{sec:upright}
    \oldlabel{3}


%\section{Erasing Upright Quarters} %DCG 11.1,p.112 (deleted)
    \oldlabel{3.1}

% Fix an exceptional cluster $R$. 

%\section{Truncation} (deleted)
    \oldlabel{3.2}

%% XX Clean this up. What if something is masked?
%% In fact, just get rid of "erasing" It is so messy.


\subsection{contexts} %DCG 11.2, p.113
    \oldlabel{3.3}

The context $\x(3,0)$ is to be regarded as two
quasi-regular tetrahedra sharing a face rather than as three
quarters along a diagonal.  In particular, by
Definition~\ref{def:q-system}, the upright quarters do not belong
to the $Q$-system.

\subsection{slices} %DCG 11.5,p.115
    \oldlabel{3.6}
    \label{sec:slice}  % was sec:anchored-simplex

Let $\{v_0,v\}$ be an upright diagonal, and let
$v_1,v_2,\ldots,v_k=v_1$ be its anchors, ordered cyclically around
$\{v_0,v\}$.  This cyclic order gives dihedral angles between
consecutive anchors around the upright diagonal. We define the
dihedral angles so that their sum is $2\pi$, even though this will
lead us to depart from our usual conventions by assigning a
dihedral angle greater than $\pi$ when all the anchors are
concentrated in some half-space bounded by a plane through
$\{v_0,v\}$. When the dihedral angle of $S=\{v_0,v,v_i,v_{i+1}\}$ is at
most $\pi$, we say that $S$ is a {\it slice\/} if
$|v_i-v_{i+1}|\le3.2$. (The constant $3.2$ appears throughout this
\chap.) All upright quarters are slices. If an upright
diagonal is completely surrounded by slices, the
upright diagonal is sometimes called a {\it loop}. If
$|v_i-v_{i+1}|>3.2$ and the angle is less than $\pi$, we say there
is a {\it gap\/} around $\{v_0,v\}$ between $v_i$ and $v_{i+1}$.

To understand how the interiors of slices meet, we
need a bound satisfied by vertices enclosed over a slice.


\begin{lemma}
    \label{lemma:anc-simplex-not-enc}
A vertex $w$ of height between 2 and $2\sqrt{2}$, enclosed in the cone
over a slice $\{v_0,v,v_1,v_2\}$ with diagonal $\{v_0,v\}$ satisfies
$|w-v|\le 2t_0$. In particular, if $|w|\le 2t_0$, then $w$ is an anchor.
\end{lemma}

\begin{proof}
This appears as Lemma~\ref{tarski:anc-simplex-not-enc}.
\end{proof}


\begin{corollary}
A vertex of height at most $2t_0$ is never enclosed over a slice.
\end{corollary}

\begin{proof}  If so, it would be an anchor to the upright diagonal, contrary to
the assumption that the slice is formed by consecutive
anchors.
\end{proof}


\subsection{slices do not overlap} %DCG 11.6, p.116
    \oldlabel{3.7}



\begin{definition}\index{unconfined@3-unconfined}
 \index{crowded4@3-crowded}\index{crowded3@4-crowded}
% Def'n copied from linprog.tex
Consider an upright diagonal that is not a loop. Let $R$ be the
standard region that contains the upright diagonal and its
surrounding quarters.  Assume we are in the context $(4,1)$ or
$(5,1)$.  In the context $(4,1)$, suppose that there does not exist
a plane through the upright diagonal such that all three quarters
lie in the same half-space bounded by the plane. Then we say that
the context is {\it $3$-unconfined}. If such a plane exists, we say
that the context is $3$-crowded. We call the context $(5,1)$ a
$4$-crowded upright diagonal. Sections~\ref{x-3.4} and \ref{x-3.5}
reduce everything to contexts with four or five anchors around each
vertex.  If there are $5$ darts, 
Remark~\ref{rem:5dart} shows that we can assume at most one
gap. This gives contexts $(5,0)$ and $(5,1)$.  If there are four
anchors, then Lemma~\ref{x-3.9.1} will dismiss all contexts except
$(4,0)$ and $(4,1)$. Thus, every upright diagonal is exactly one of
the following: a loop, $3$-unconfined, $3$-crowded, or $4$-crowded.
%\def\Sfour{{{\cal\mathbf S}_4^+}}  --> $4$-crowded upright diagonal
%\def\Sminus{{{\cal\mathbf S}_3^-}} --> $3$-crowded upright diagonal
%\def\Splus{{{\cal\mathbf S}_3^+}}  --> $3$-unconfined upright diagonal
\end{definition}


This lemma is a consequence of the two others that follow. The
context of the lemma is the set of slices that have
not been erased by previous reductions.

\begin{lemma}
    \label{lemma:anchor-no-overlap}
The interiors of slices do not meet.
\end{lemma}

The remaining contexts have four or  five anchors. Let $w$ and the
slice $S=\{v_0,v,v_1,v_2\}$ be as in Section~\ref{x-3.6}.
Our object is to describe the local geometry when an upright
diagonal is enclosed over a slice. If $|v_1-v_2|\le
2\sqrt{2}$, we have seen in Lemma~\ref{lemma:double-face} that
there can be no enclosed upright diagonal with $\ge 4$ anchors
over the slice $S$.

Assume  $|v_1-v_2|>2\sqrt{2}$. Let $w_1,\ldots,w_k$, $k\ge4$, be the
anchors of $\{v_0,w\}$, indexed consecutively. The anchors of $\{v_0,w\}$ do not
lie in $C(S)$, and the triangles $\{v_0,w,w_i\}$ and $\{v_0,v,v_j\}$ do not
overlap. Thus, the plane $\{v_0,v_1,v_2\}$ separates $w$ from
$\{w_1,\ldots,w_k\}$. Set $S_i=\{v_0,w,w_i,w_{i+1}\}$.
By a calculation\footnote{\calc{83777706}} %A8
%$\A_8$,
    $$\pi\ge \dih(S_1)+\cdots+\dih(S_{k-1})\ge (k-1)0.956.$$

Thus, $k=4$. The common upright diagonal  of the three simplices
$\{S_i\}$ is {\it $3$-crowded}.  We claim that
$\{v_1,v_2\}=\{w_1,w_4\}$. Suppose to the contrary that, after
reindexing as necessary, $S_0=\{v_0,w,w_1,v_1\}$ is a simplex, with
$v_1\ne w_1$, that does not overlap $S_1,\ldots,S_3$. Then $\pi\ge
\dih(S_0)+\cdots+\dih(S_3)$. So
    $0.28\ge \pi-3(0.956)\ge \dih(S_0)$.
A calculation\footnote{\calc{83777706}} %A8
now implies that $|w-v_1|\ge 2\sqrt{2}$.

By Lemma~\ref{tarski:336}, the four vertices
$\{v_0,w,v_1,v_2\}$ cannot be coplanar.
We have that $2\sqrt{2}\ge|w|$ and by Lemma~\ref{tarski:E:part4:1},
we also have $|w|>2\sqrt2$.
This contradiction establishes that $v_1=w_1$.

\begin{lemma}
Around a $3$-crowded upright diagonal, all of the slices
are quarters.
\end{lemma}

\begin{proof}  The proof makes use of constants and inequalities from
several different calculations.\footnote{\calc{815492935}} %A2
\footnote{\calc{83777706}} %A8
\footnote{\calc{855294746}} %A12
%$\A_2$, $\A_8$, and $\A_{12}$.
The dihedral angles are at most $\pi-
2(0.956) < 1.23$. This forces $y_4\le 2t_0$, for each simplex $S$.
So they are all quarters.
\end{proof}

\begin{lemma}
    \oldlabel{3.7.1}\label{lemma:3-crowded}
If there is $3$-crowded upright diagonal, then the three 
slices squander more than $0.5606$ and score at most $-0.4339$.
\end{lemma}


\begin{proof}  The proof makes use of constants and inequalities from
several different calculations.\footnote{\calc{815492935}} %A2
\footnote{\calc{83777706}} %A8
\footnote{\calc{855294746}} %A12
%$\A_2$, $\A_8$, and $\A_{12}$.
The three slices squander at
least
    $$
    3 (1.01104) - \pi (0.78701) > 0.5606.
    $$
The bound on score follows similarly from $\nu<-0.9871+0.80449\dih$.
\end{proof}

\begin{lemma}
    \oldlabel{3.7.2}
If a simplex at a $3$-crowded upright diagonal meets at an
interior point with a slice, the centered packing does
not contravene.
\end{lemma}

\begin{proof}
Suppose that $\{v_0,v,v_1,v_2\}$ is a slice that another
slice overlaps, with $\{v_0,v\}$ the upright diagonal.  Let
$\{v_0,w\}$ be a $3$-crowded upright diagonal. We score the two
simplices $S'_i = \{v_0,v,w,v_i\}$ by truncation at $\sqrt{2}$.
Truncation at $\sqrt{2}$ is justified by Lemma~\ref{tarski:old372}.
A calculation\footnote{\calc{855294746}} %A12
gives
%$\A_{12}$,
    $$\tau_V(S'_1,\sqrt{2})+\tau_V(S'_2,\sqrt{2})\ge 2(0.13) +
        0.2(\dih(S'_1)+\dih(S'_2)-\pi) > 0.26.
    $$
Together with the three simplices around the $3$-crowded upright
diagonal that squander at least $0.5606$, we obtain the stated
bound.
\end{proof}



\subsection{four and five darts} %DCG 11.7, p.118 % Was "Five Anchors"
    \oldlabel{3.8}
    \label{sec:five-anchors}
    %\section{Four darts} %DCG 11.8, p. 120 % Was "Four anchors"
    \oldlabel{3.9}



\begin{remark}\label{rem:5dart}
The situation of five darts at an upright diagonal is
described in Section~\ref{sec:5updart}.
%    \oldlabel{3.8.1}
\end{remark}

\begin{definition}
Let a $3$-unconfined node be a node that is an upright diagonal 
with four darts and one gap in a situation where none of
the quarters along this upright diagonal masks a flat quarter.
\end{definition}

\begin{lemma}\dcg{Cor~11.25}{122}
If there are four anchors and if the upright diagonal is enclosed over a
flat quarter, then there are four slices and at least three
quarters around the upright diagonal.
\end{lemma}

\begin{proof}
This follows by Lemma~\ref{tarski:dcg-p122}.
\end{proof}


\subsection{penalty constants}

\begin{definition}\index{zzxiG@$\xiG$}\index{zzxiV@$\xiV$}
We set $\xiG = 0.01561$, $\xiV = 0.003521$, $\xiG'=0.00935$,
$\xik=-0.029$, $\xikG = \xik+\xiG = -0.01339$.
\end{definition}

The first two constants appear in calculations%
%$\A_{10}$ and $\A_{11}$ as
\footnote{\calc{73974037}} %A10
\footnote{\calc{764978100}} %A11
as penalties for erasing upright quarters that are compressed, and
decompressed, respectively. $\xiG'$ is an improved bound on the
penalty for erasing when the upright diagonal is at least $2.57$.
Also, $\xik$ is an upper bound\footnote{\calc{618205535}} %A9
 on $\kappa$, when the
upright diagonal is at most $2.57$.  If the upright diagonal is at
least $2.57$, then we still obtain the bound%
\footnote{\calc{618205535}} %A9
$\xikG =-0.02274+\xiG'$ on the sum of $\kappa$ with the
penalty from erasing an upright quarter.

Recall that $\xiV=0.003521$, $\xiG=0.01561$, $\xiG'=0.00935$. They are
the penalties that result from erasing a 
decompressed upright quarter, a comprssed upright quarter, 
and a comprssed upright quarter
with diagonal $\ge2.57$. (See calculations.%
\footnote{\calc{73974037}} %A10
\footnote{\calc{764978100}} %A11)


%
%
%\section{Summary} %DCG 11.9 p 122
%    \oldlabel{3.10}
%    \label{sec:upright-summary}
%
%The following index summarizes the cases of upright quarters that have
%been treated in Section~\ref{sec:upright}. If the number of anchors is
%the number of slices (no gaps), the results appear in
%Section~\ref{x-5.11}. Every other possibility has been treated.
%
%    \begin{itemize}
%    \item 0,1,2 anchors\hfill Sec.~\ref{x-3.3}
%    \item $3$ anchors \hfill Sec.~\ref{x-3.4}
%        \begin{itemize}
%        \item context $\x(3,0)$
%        \item context $\x(3,1)$
%        \item context $\x(3,2)$
%        \item context $\x(3,3)$
%        \end{itemize}
%    \item $4$ anchors \hfill Sec.~\ref{x-3.9}
%        \begin{itemize}
%        \item $0$ gaps (Section~\ref{x-5.11})
%        \item $1$ gap
%        \item $2$ or more gaps
%        \end{itemize}
%    \item $5$ anchors \hfill Sec.~\ref{x-3.8}
%        \begin{itemize}
%        \item $0$ gaps (Section~\ref{x-5.11})
%        \item $1$ gap ($4$-crowded)
%        \item $2$ or more gaps
%        \end{itemize}
%    \item $6$ or more anchors \hfill Sec.~\ref{x-3.5}
%    \end{itemize}
%
%
%\smallskip
%By truncation and various comparison lemmas, we have entirely eliminated
%upright diagonals except when there are between three and five anchors.
%We may assume that there is at most one gap around the upright
%diagonal.
%
%\smallskip
%1.  Consider a slice $Q$ around a remaining upright
%diagonal. The score of is $\nu(Q)$ if $Q$ is a quarter, the
%analytic function $\op{svan}(Q)$ if the simplex is of type $\SC$
%(Section~\ref{x-2.5}), and the truncated function $\op{sv}_0(Q)$
%otherwise.
%
%\smallskip
%2.  Consider a flat quarter $Q$ in an exceptional cluster. An
%upper bound on the score is obtained by taking the maximum of all
%of the following functions that satisfy the stated conditions on
%$Q$.  Let $y_4$ denote the length of the diagonal and $y_1$ be the
%length of the opposite edge.
%
%(a)  The function $\mu(Q)$.
%
%(b)  $\op{sv}_0(Q) - 0.0063$, if $y_4\ge 2.6$ and $y_1\ge
%2.2$.\hfill
%    (Lemma~\ref{lemma:0.008})
%
%(c)  $\op{sv}_0(Q) - 0.0114$, if $y_4\ge 2.7$ and $y_1\le 2.2$.
%    \hfill (Lemma~\ref{lemma:0.008})
%
%(d)  $\nu(Q_1)+\nu(Q_2)+\op{sv}_x(S)$, if there is an enclosed
%vertex
%    $v$ over $Q$ of height between $2t_0$ and $2\sqrt{2}$ that
%    partitions the convex hull of $(Q,v)$ into two upright quarters
%    $Q_1$, $Q_2$ and a third simplex $S$. Here $\op{sv}_x=\op{svan}$
%    if $S$ is of type $\SC$, and $\op{sv}_x=\op{sv}_0$ otherwise.
%    \hfill (Lemma~\ref{lemma:unerased})
%
%(e)  $\op{sv}(Q,1.385)$ if the simplex is of type $\SB$
%(Section~\ref{x-2.5}).
%
%(f) $\op{sv}_0(Q)$ if the simplex is an isolated quarter with
%    $\max(y_2,y_3)\ge2.23$, $y_4\ge2.77$,
%    and $\eta_{456}\ge\sqrt2$.
%
%\smallskip
%3.   If $S$ is a simplex is of type $\SA$, its score is
%$\op{svan}(S)$. (Section~\ref{x-2.5}.)
%
%\smallskip
%
%    Formula~\ref{eqn:3.7} is used on these remaining pieces.
%    On top of what is obtained for the standard cluster by summing all
%these terms, there is a penalty $\pi_0=0.008$ each time a
%$3$-unconfined upright diagonal is erased.
%
%\smallskip
%5.  The remaining upright diagonals that are not completely
%surrounded by slices are $3$-unconfined, $3$-crowded,
%or $4$-crowded from Section~\ref{x-3.7}, \ref{x-3.8},  and
%\ref{x-3.9}.
%


\subsection{flat quarters} %DCG 11.10, p124
    \oldlabel{3.11}
    \label{sec:some-flat}




In the next lemma, we score a flat quarter by any of the functions
on the given domains
     $$\hat\sigma=
        \begin{cases}
            \Gamma,& \eta_{234},\eta_{456}\le\sqrt2,\\
             \op{svan}, &\eta_{234}\ge\sqrt2,\\
            \op{sv}_0, & y_4\ge 2.6, y_1\ge2.2,\\
            \op{sv}_0, & y_4\ge 2.7,\\
            \op{sv}_0,& \eta_{456}\ge\sqrt2.
        \end{cases}
    $$

\begin{lemma}
    \oldlabel{3.11.1}
    \label{lemma:hatsigma}
$\hat\sigma$ is an upper bound on the functions in
Section~\ref{x-3.10}.2(a)--(f). That is, each function in
Section~\ref{x-3.10}.2 is dominated by some choice of $\hat\sigma$.
\end{lemma}

\begin{proof}  The only case in doubt is the function of 3.10(d):
$$\nu(Q_1)+\nu(Q_2)+\op{sv}_x(S).$$ This is established by the
following lemma.
\end{proof}


We consider the context $\x(3,1)$ that occurs when two upright
quarters in the $Q$-system lie over a flat quarter. Let $\{v_0,v\}$ be
the upright diagonal, and assume that $\{v_0,v_1,v_2,v_3\}$ is the
flat quarter, with diagonal $\{v_2,v_3\}$. Let $\sigma$ denote the
score of the upright quarters and other slice lying
over the flat quarter.

\begin{lemma}\label{lemma:min0-svor}
    \oldlabel{3.11.2}
    $\sigma\le \min(0,\op{sv}_0)$.
\end{lemma}

\begin{proof}
The bound of $0$ is established in Theorem~\ref{lemma:quad0}.
The bound of $\op{sv}_0$ is established in Lemma~\ref{a:min0-svor}.
\end{proof}



%\chapter{Further Bounds}%DCG Sec. 14, p. 157
    \oldlabel{5.12}
    \label{sec:fb}




\section{Miscellaneous Bounds}
\subsection{small dihedral angles} %DCG 14.1, p157
\label{sec:small-dih}

Recall that Section~\ref{sec:the-main-theorem} defines an integer $n(R)$
that is equal to the number of sides if the region is a polygon.  Recall
that if the dihedral angle along an edge of a standard cluster is at
most $1.32$, then there is a flat quarter along that edge
(Lemma~\ref{x-3.11.4}).

\begin{lemma}
    \oldlabel{5.12.1}\dcg{Lemma~14.1}{157}
Let $R$ be an exceptional cluster with a dihedral angle
$\le1.32$ at a vertex $v$. Then $R$ squanders $>t_n+1.47\,\pt$, where
$n=n(R)$.
\end{lemma}

\begin{proof}
In most cases we establish the stronger bound $t_n+1.5\,\pt$. In the
proof of Theorem~\ref{thm:the-main-theorem}, we erase all upright
diagonals, except those completely surrounded by slices. The
contribution to $t_n$ from the flat quarter $Q$ at $v$ in that proof is
$D(3,1)$ (Sections~\ref{x-4.5} and Inequalities~\ref{eqn:tau>D(n,k)}).
Note that $\epsilon_\tau(Q)=0$ here because there are no deformations.
If we replace $D(3,1)$ with $3.07\,\pt$ from Lemma~\ref{x-3.11.4}, then
we obtain the bound. Now suppose the upright diagonal is completely
surrounded by slices. Analyzing the constants of
Section~\ref{x-5.11}, we see that $\DLP(n,k)-D(n,k)>1.5\,\pt$ except
when $(n,k)=(4,1)$.

Here we have four slices around an upright diagonal. Three
of them are quarters.  We erase and take a penalty. Two possibilities
arise.  If the upright diagonal is enclosed over the flat quarter, its
height is $\ge2.6$ by Lemma~\ref{tarski:last:E} and the top face of the
flat quarter has circumradius at least $\sqrt2$.  The penalty is
$2\xiG'+\xiV$, so the bound holds by the last statement of
Lemma~\ref{x-3.11.4}.

If, on the other hand, the upright diagonal is not enclosed over the
flat diagonal, the penalty is $\xiG+2\xiV$.  In this case, we obtain the
weaker bound $1.47\,\pt+t_n$:
    $$3.07\,\pt > D(3,1) + 1.47\,\pt +\xiG+2\xiV.$$
\end{proof}

\begin{remark} \label{remark:1.47}
If there are $r$ nonadjacent vertices with dihedral angles
$\le1.32$, we find that $R$ squanders $t_n+r(1.47)\,\pt$.
\end{remark}

In fact, in the proof of the lemma, each $D(3,1)$ is replaced with
$3.07\,\pt$ from Lemma~\ref{x-3.11.4}.  The only questionable case
occurs when two or more of the vertices are anchors of the same upright
diagonal (a loop). Referring to Section~\ref{x-5.11}, we have the
following observations about various contexts.

\begin{itemize}
    \item $(4,1)$ can mask only one flat quarter and it is treated in the
lemma.
    \item $(4,2)$ can mask only one flat quarter and
    $\DLP(4,2)-D(4,2)>1.47\,\pt$.
    \item $(5,0)$ can mask two flat quarters.  Erase the five upright quarters,
        and take a penalty $4\xiV+\xiG$.  We get
    $$D(3,2)+2(3.07)\,\pt > t_5+4\xiV+\xiG+2(1.47)\,\pt.$$
    \item $(5,1)$ can mask two flat quarters, and $\DLP(5,1)-D(5,1)>2(1.47)\,\pt$.
\end{itemize}




\begin{lemma}\dcg{Lemma~14.6}{165}\label{lemma:excess-1}
Let $R$ be an exceptional standard region.  Let $V$
be a set of vertices of $R$.  If $v\in V$, let $p_v$ be the number
of triangular regions at $v$ and let $q_v$ be the number of
quadrilateral regions at $v$.  Assume that $V$ has the following
properties:
    \begin{enumerate}
        \item No two
        vertices in $V$ are adjacent.
        \item No two vertices
        in $V$ lie on a common quadrilateral.
        \item If $v\in V$, then there are five standard regions at
        $v$.
        \item If $v\in V$, then the corner over $v$ is a central
        vertex of a flat quarter in the cone over $R$.
        \item If $v\in V$, then $p_v\ge 3$.  That is, at least
        three of the five standard regions at $v$ are triangular.
        \item If $R'\ne R$ is an exceptional region at $v$, and if $R$
        has interior angle at least $1.32$ at $v$, then $R'$ also has interior
        angle at least $1.32$ at $v$.
        \item If $(p_v,q_v)=(3,1)$, then the internal angle at $v$ of the exceptional
        region is at most $1.32$.
    \end{enumerate}
  Define $a:\N\to \R$ by
  $$a(n) = \begin{cases}
    14.8 &n=0,1,2,\\
    1.4 & n=3,\\
    1.5 & n=4,\\
    0 & \text{otherwise.}
  \end{cases}
  \index{aZ@$a(n)$}
  $$
Let $\{F\}$ be the union of $\{R\}$ with the set of triangular and
quadrilateral regions that have a vertex at some $v\in V$. Then
    $$\sum_F\tau_F(v,\Lambda) > \sum_{v\in V} (p_v d(3) + q_v d(4) + a
    (p_v))\,\pt.$$
\end{lemma}

\begin{proof}   We erase all upright diagonals in the
$Q$-system, except for those that carry a penalty: loops,
$3$-unconfined, $3$-crowded, and $4$-crowded diagonals.

We assume that if $(p_v,q_v)=(3,1)$, then the internal angle is at
most $1.32$. Because of this, if we score the flat quarter by
$\op{svR}_0$, then the flat quarter $Q$ satisfies
(Lemma~\ref{lemma:1.32})
   \begin{equation}
   \op{svR}_0(Q) > 3.07\,\pt > 1.4\,\pt + D(3,1) + 2\xiV + \xiG.
   \label{eqn:307}
   \end{equation}



Every flat quarter that is masked by a remaining upright quarter
in the $Q$-system has $y_4\ge2.6$.  Moreover, $y_1\ge2.2$ or
$y_4\ge2.7$.  Let $\pi_v = 2\xiV + \xiG$ if the flat quarter is
masked, and $\pi_v = 0$ otherwise.

We claim that the flat quarter (scored by $\op{svR}_0$) together with
the triangles and quadrilaterals at a given vertex $v$ squander at
least
   \begin{equation}
   (p_v d(3) + q_v d(4) + a(p_v))\,\pt + D(3,1) + \pi_v
   \label{eqn:one-v}
   \end{equation}
If $p_v=4$, this is \calc{314974315}.  If $p_v=3$, we may assume
by the preceding remarks that there are two exceptional regions at
$v$.  If the internal angle of $R$ at $v$ is at most $1.32$, then
we use Inequality~\ref{eqn:307}.  If the angle is at least $1.32$,
then by hypothesis, the angle $R'$ at $v$ is at least $1.32$.  We
then appeal to the calculations \calc{675785884} and
\calc{193592217}.

To complete the proof of the lemma, it is enough to show that we
can erase the upright quarters masking a flat quarter at $v$
without incurring a penalty greater than $\pi_v$.  For then, by
summing the Inequality~\ref{eqn:one-v} over $v$, we obtain the
result.

If the upright diagonal is enclosed over the masked flat quarter,
then the upright quarters can be erased with penalty at most
$\xiV$ (by Remark~\ref{remark:3rd-quarter}). Assume the upright
diagonal is not enclosed over the masked flat quarter.

If there are at most three upright quarters, the penalty is at
most $2\xiV + \xiG$.  Assume four or more upright quarters.  If
the upright diagonal is not a loop, then it must be $4$-crowded.
This can be erased with penalty
   $$2\xiV + 2\xiG - \kappa < 2\xiV + \xiG.$$

Finally, assume that the upright quarter is a loop with four or
more upright quarters.  Lemma~\ref{lemma:loop} limits the
possibilities to parameters $(5,0)$ or $(5,1)$.  In the case of a
loop $(5,1)$, there is no need to erase because $|V|\le3$ and by
Lemma~\ref{lemma:loop}, the hexagonal standard region squanders at
least
   $$t_6 + 3 a(p_v)\,\pt$$
as required by the lemma.  In the case of a loop $(5,0)$ in a
pentagonal region, if $|V|=1$ then there is no need to erase
(again we appeal to Lemma~\ref{lemma:loop}).  If $|V| =2$, then
the two vertices share a penalty of $4\xiV + \xiG$, with each
receiving
   $$2\xiV + \xiG/2 < 2\xiV +\xiG.$$
\end{proof}
