%------------------------------------------------------------
% Author: Thomas C. Hales
% Format: LaTeX
% Book Chapter: Dense Sphere Packings
%------------------------------------------------------------

\chapter*{Preface}


%{
%
%\narrower
%
%{\it ``A personality which has been veiled by a formal method
%  throughout many chapters is suddenly seen face to face in the
%  Preface.'' }
%% Introductory note, p3, Famous Prefaces, Harvard Classics, Vol
%% 39. P.F. Collier & Son, 1910.
%
%}

%\bigskip

%\centerline{\it ``Those who justify themselves do not convince.''
%  --Lao-Tzu}
%% quoted in Watts's essay in "Modern Buddhism" ed. Donald Lopez,
%% page 160


{

\narrower

{\it

  ``I think there's a revolution in mathematics around the corner. I
  think that~\dots{}   %in later times
  people will look back on the fin-de-si\`ecle of the twentieth century
  and say `Then is when it happened' (just like we look back at the
  Greeks for inventing the concept of proof and at the nineteenth
  century for making analysis rigorous). I really believe that. And it
  amazes me that no one seems to notice.

  ``Never before have the Platonic mathematical world and the physical
  world been this similar, this close. Is it strange that I expect
  leakage between these two worlds? That I think the proof strings
  will find their way to the computer memories?\dots 

  ``What I expect is that some kind of computer system will be
  created, a proof checker, that all mathematicians will start using
  to check their work, their proofs, their mathematics. I have no idea
  what shape such a system will take. But I expect some
  system to come into being that is past some threshold so that it is
  practical enough for real work, and then quite suddenly some kind of
  `phase transition' will occur and everyone will be using that
  system.''

{\hfill\rm--Freek Wiedijk~\cite{FWR}} % http://www.cs.ru.nl/~freek/jordan/index.html

}

}

\newpage

{

\narrower\parindent=0pt
\parskip=0.4\baselineskip

\hbox{}
\vskip1truein

{\it

Alecos: Christos has a problem with the `foundational quest'!

Christos:  Wrong!  I have two problems with your  {\rm {version}} of it!  One, it
didn't fail and, two, it wasn't a tragedy!  Granted, there are some tragic
parts!  But the ending is happy, as in the `Oresteia'!  

Apostolos:  Happy for whom?  Cantor, going insane?  G\"odel starving himself to
death out of paranoia? Hilbert or Russell and their psychotic sons? Or Frege with--

Christos: `The meaning
is in the ending!' you said so yourself!  So, follow the quest for ten more years and
you get a brand-new triumphant finale with the creation of the computer, which is the
quest's real hero!   Your problem is, simply, that you see it as a story of people!

Apostolos: Well, stories do tend to be about people!

Christos:  So, choose the right people!  And show what they really did!  All we we learn
of the great von Neumann is he said `It's over'  when he heard G\"odel!

Alecos: But it was over in a sense, wasn't it?  Pop went Hilbert's `no ignorabimus'!

Christos:  But then came the quest's jeune premier, its parsifal~\dots{}  Alan Turing!
He said `Ok, we can't prove everything! So, let's see what we can prove!' and to define
proof, he invented, in 1936, a theoretical machine which contains all the ideas of the
computer!\dots{}  which, after the war, he and von Neumann, the quest's proudest sons,
brought to full life!

{\hfill\rm--Doxiadis and Papadimitriou, ~~Logicomix~\cite{Logicomix}} % page 303.

}

}




\newpage

{

\narrower\parindent=0pt
\parskip=0.4\baselineskip

\hbox{}
\vskip1truein

{\it

  ``Despite the unusual nature of the proof, the editors of the Annals
  of Mathematics agreed to publish it, provided it was accepted by a
  panel of twelve referees. In 2003, after four years of work, the
  head of the referee's panel G\'abor Fejes T\'oth (son of L\'aszl\'o
  Fejes T\'oth) reported that the panel were `99\% certain' of the
  correctness of the proof.''

{\hfill\rm -- Wikipedia entry on the Kepler Conjecture}

}

}

%%DD Figure of cannonballs.

\bigskip

{

\narrower\parindent=0pt
\parskip=0.4\baselineskip

{\it

``Sometimes fixing a 1 percent defect takes 500 percent effort.''

{\hfill\rm-- Joel Spolsky, Joel on Software~\cite{Spolsky}} % page 122

}

}

\bigskip

{

\narrower\parindent=0pt
\parskip=0.4\baselineskip

{\it

``Every one fully persuaded is a fool.''

{\hfill\rm-- Barthasar Graci\'an, the Art of Worldly Wisdom~\cite{Gracian}} % p110

}

{

\it

%Believe not everything, but only what is proven: the former is foolish, 
%the latter the act of a sensible man. 

%{\hfill -- Democritus}

}

}



\newpage

\section*{The Kepler Conjecture}

In 1611, Johannes Kepler wrote a booklet in which he asserted that the familiar
cannonball arrangement of congruent balls in space achieves the
highest possible density.  This assertion has become known as the
Kepler conjecture.  This book presents a proof.

As early as 1831, Gauss established a special case of the conjecture,
by proving that the cannonball arrangement is optimal among all
\newterm{lattices}~\cite{Gau31}.  Later in the nineteenth century,
Thue solved the corresponding problem in two dimensions, showing that
the hexagonal arrangement of disks in the plane achieves optimal
density~\cite{Thu92} and \cite{Thu10}.  Hilbert, in his famous list of
mathematical problems, made the Kepler conjecture part of his eighteenth
problem.  In 1953, Fejes T\'oth formulated a general strategy to
confirm the Kepler conjecture, but lacked the computational resources
to carry it out~\cite{Fej53}.  The conjecture was finally resolved in
1998, even though the full proof was not published until
2006~\cite{Hales:2006:DCG}.  Section~\ref{sec:history} gives
additional historical background.

The Kepler conjecture has become a test of the capability of computers to
deliver a reliable mathematical proof.  The original proof 
involved many long computer calculations that
led a team of referees to exhaustion.  This book has
 redesigned the proof in a way that makes the correctness of
the computer proof as transparent as possible.


\section*{Formal Proofs}

After all is said and done, a proof is only as reliable as the
processes that are used to verify its correctness.  The ultimate
standard of proof is a formal proof, which is nothing other
than an unbroken chain of logical inferences from an explicit set of
axioms.  While this may be the mathematical ideal of proof, actual
mathematical practice generally deviates significantly from the ideal.

In recent years, as part of this project, I have been increasingly preoccupied by the
processes that mathematicians rely on to ensure the correctness of complex
proofs. A century ago, Russell's paradox and other antinomies threatened
 set theory  with fires of destruction.
Researchers from Frege to G\"odel solved the problem of
rigor in mathematics and found a theoretical solution but did not
extinguish the fire at the foundations of mathematics
because they omitted the practical implementation. Some, such as
Bourbaki, have even gone so far as to claim that ``formalized
mathematics cannot in practice be written down in full'' and call
such a project
``absolutely unrealizable'' \cite[pp.~10--11]{Bour:68:Sets}. % Theory of
                                                          % Sets, page
                                                          % 10,11.

While it is true that formal proofs may be too long to print,
computers -- which do not have the same limitations as paper -- have
become the natural host of formal mathematics. In recent decades,
logicians and computer scientists have reworked the foundations of
mathematics, putting them in an efficient form designed for real use
on real computers.

For the first time in history, it is possible to generate and verify
every single logical inference of major mathematical theorems.  This
has now been done for many theorems, including the four-color theorem, the prime number
theorem, the Jordan curve theorem, the Brouwer fixed point theorem,
and the fundamental theorem of calculus.  Freek Wiedijk
reports that 87\% of a list of one hundred famous theorems have now been
checked formally \cite{wiedijk:100}.  The list of remaining
theorems contains two particular challenges: the independence of the
Continuum hypothesis and Fermat's Last theorem.
% list as of 5/5/2010.
% 87 as of April 2012.

Some mathematicians remain skeptical of the process because computers
have been used to generate and verify the logical inferences.
Computers are notoriously imperfect, with flaws ranging from software
bugs to defective chips.  Even if a computer verifies the inferences,
who  verifies the verifier, or then verifies the verifier of the
verifier?  Indeed, it would be unscientific of us to place an
unmerited trust in computers.

The choice comes down to two competing verification processes.  The
first is the traditional process of referees, which depends largely on
the luck of the draw -- some referees are meticulous, while others are
careless.  The second process is formal computer verification, which
is less dependent on the whims of a particular referee.  In my view,
the choice between the conventional  process by human referee and computer
verification is as evident as the choice between a sundial and an atomic
clock in science.

The standard of proof I have adopted is the highest scientific
standard available by current technology.  
%That standard, which
%continues to evolve with the advancement of technology, is formal
%verification by computer.
%The boundary that separates an easy proof from a difficult
%one shifts with current technology.  
The introduction of steel in
architecture is not a mere reinforcement of wood and stone; it changes
the world of structural possibilities.  There is no longer any
reason to limit proofs to ten thousand pages when our
technology supports a million pages.

The style of formal proofs is different from that of conventional
ones.  It is easier to formalize several short snappy proofs
than a few intricate ones.  Humans enjoy surprising new perspectives,
but computers benefit from repetition and standardization.  Despite
these differences, I have sought  proofs that might bring
pleasure to the human reader while providing precise instructions for
the implementation in silicon.

\section*{Conventions}

To make formalization proceed more smoothly, long proofs have been
broken into a sequence of smaller claims.  Each claim starts a new paragraph
and is set in italics.  The second sentence of the paragraph begins with
the word {\it indeed} when the proof of the claim is direct and with
the word {\it otherwise} when the proof is indirect by contradiction.
\indy{Index}{claim (italic format of small claims)}%

Lemmas and theorems that are marked with an asterisk appear out of
the natural logical sequence.  Care should be taken to avoid logical gaps
when they are cited.

The pronoun \fullterm{we}{we (use of personal pronoun)} is used
inclusively for the author and reader as we work our way through the
proofs in this book.  The pronoun \fullterm{I}{I (use of personal
  pronoun)} refers to the author alone.

The asterisk $\wild$ is used as a wildcard symbol in patterns.  It
replaces a term in contexts where the name of the term is not
relevant.  It can also denote a bound variable.  For example, the
function $f(\wild,y)$ of a single variable is obtained from $f$ by
evaluating the second argument at a fixed value $y$.
\indy{Notation}{4@$\wild$ (wildcard symbol)}

The union of the family $X$ of sets is written as $\bigcup\ X$ or as
$\bigcup_{x\in X} x$ without any difference in meaning.  The first form
is preferred because of its economy.  We also use both expressions
$\bigcap\ X$ and $\bigcap_{x\in X} x$ for the intersection of a family of sets.


The documentation of the computer calculations for the Kepler
conjecture has evolved over time.  The 1998 preprint version of the
proof of the Kepler conjecture contains long appendices that list
hundreds of calculations that enter into the proof.  These appendices
were cut from the published version of the proof because it is more
useful to store the computer part of the proof at a computer code
repository that is permanent, versioned, and freely available.  The
computer code and documentation are housed at \newterm{Google Code
  project hosting}.  Separate documentation, which is available at the
project site, describes the computer calculations that appear in this
book.  When this book uses an external calculation, it is marked in
italic font as a \cc{notation}{This explains notation.}.

\section*{A Blueprint}

The book is a blueprint for formal proofs because it gives the design
of the formal proof to be constructed.  
The parts of this book that cover the text portions of the proof of
the Kepler conjecture are being formally verified in the proof
assistant HOL Light.  I dream of a fully formally verified solution to
the proof that includes the computer portions of the proof
as well.  Details about and credits for this large team effort appear in
Appendix~\ref{sec:credit}.  

Decisions about what to
include in this book have been shaped by the list of theorems already
available in the library of the proof assistant \newterm{HOL Light}.
For example, this book accepts basic point-set topology and measure
theory because they have been formalized by Harrison~\cite{HOLL}.


The book is divided into four parts, the first of which
 describes the major ideas, methods, and
organization of the proof.  

The part on foundations provides background material about
constructions in discrete geometry.  The first of these chapters
covers trigonometric identities and basic vector geometry.  The second
treats volume from an elementary point of view.  The third chapter
covers planar graph theory from a purely combinatorial perspective.
The fourth chapter continues with planar graphs, now from a geometric
perspective.

The next part of the book gives the solution to the packing problem.
The first chapter  gives a top-level overview of the major
steps of the proof,   describing how the problem can be reduced from
a problem with infinitely many variables to one in finitely many
variables.  The remaining chapters in this part flesh out the proof.

The final section of the book views dense sphere packings from a larger perspective.
It  resolves  another longstanding conjecture in
discrete geometry: Bezdek's strong dodecahedral conjecture.

\section*{Simplifications}

Many simplifications of the original proof have been found over the
past several years.  These simplifications are published here for the
first time.  Gonthier has reworked the proof of the four-color
theorem to avoid the use of the Jordan curve theorem, using instead
the much simpler notion of M\"obius contour from the theory of
hypermaps.  I have followed Gonthier's lead.

The optimality of the face-centered cubic packing is an assertion
about infinite space-filling packings.  For computational purposes, it is
useful to reduce the sphere packing problem to finite packings.  A
\newterm{correction term} is associated with each different reduction from
infinite packings to finite packings.  Ferguson and I worked together to
produce the original proof of the Kepler conjecture.  The two of us considered a
large number of different correction terms, seeking one that
would simplify the computations as much as possible.  In a discussion
of the solution of the packing problem, I wrote that ``correction
terms are extremely flexible and easy to construct, and soon Samuel
Ferguson and I realized that every time we encountered difficulties in
solving the minimization problem, we could adjust $f$ [the correction
term] to skirt the difficulty\dots.  If I were to revise the proof
to produce a simpler one, the first thing I would do would be to
change the correction term once again.  It is the key to a simpler
proof''~\cite{Hales:2000:cannonballs}.  Marchal has recently found a simple 
correction term, giving a new way  to  reduce from infinite packings
to finite packings~\cite{marchal:2009}.  This book implements his reduction step.

There are many other improvements of the proof that are not visible in
the book because they are implemented in computer code, including a
reduction of the number of lines of computer code from over 187,000 to
about 10,000.  Needless to say, the quickest way to be sure that a
block of computer code will not execute a bug is to delete the code
altogether.




\bigskip
\hbox{}



\bigskip
\hbox{}

{
\parindent=0pt
\obeylines

Thomas C. Hales
Pittsburgh, PA
%Jan 1, 2011

}







