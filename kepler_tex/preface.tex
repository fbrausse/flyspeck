%------------------------------------------------------------
% Author: Thomas C. Hales
% Format: LaTeX
% Book Chapter: Dense Sphere Packings
%------------------------------------------------------------

\chapter*{Preface}

% Believe not everything, but only what is proven: the former is foolish, the latter the act of a sensible man. -- Democritus.

%{
%
%\narrower
%
%{\it ``A personality which has been veiled by a formal method
%  throughout many chapters is suddenly seen face to face in the
%  Preface.'' }
%% Introductory note, p3, Famous Prefaces, Harvard Classics, Vol
%% 39. P.F. Collier & Son, 1910.
%
%}

%\bigskip

%\centerline{\it ``Those who justify themselves do not convince.''
%  --Lao-Tzu}
%% quoted in A. Watts's essay in "Modern Buddhism" ed. Donald Lopez,
%% page 160


{

\narrower

{\it

  ``I think there's a revolution in mathematics around the corner. I
  think that $\ldots$ %in later times
  people will look back on the fin-de-siecle of the twentieth century
  and say `then is when it happened' (just like we look back at the
  Greeks for inventing the concept of proof and at the nineteenth
  century for making analysis rigorous). I really believe that. And it
  amazes me that no one seems to notice $\ldots$

  ``Never before have the platonic mathematical world and the physical
  world been this similar, this close. Is it strange that I expect
  leakage between these two worlds? That I think the proof strings
  will find their way to the computer memories?$\ldots$

  ``What I expect is that some kind of computer system will be
  created, a proof checker, that all mathematicians will start using
  to check their work, their proofs, their mathematics. I have no idea
  what shape such a system will $\ldots$ take. But I expect some
  system to come into being that is past some threshold so that it is
  practical enough for real work, and then quite suddenly some kind of
  `phase transition' will occur and everyone will be using that
  system.''

{\hfill--Freek Wiedijk \cite{FWR}} % http://www.cs.ru.nl/~freek/jordan/index.html

}

}

\newpage

{

\narrower\parindent=0pt
\parskip=0.4\baselineskip

{\it

Alecos: Christos has a problem with the `foundational quest'!

Christos:  Wrong!  I have two problems with your  {\rm {version}} of it!  One, it
didn't fail and, two, it wasn't a tragedy!  Granted, there are some tragic
parts!  But the ending is happy, as in the `Oresteia'!  

Apostolos:  Happy for whom?  Cantor, going insane?  G\"odel starving himself to
death out of paranoia? Hilbert or Russell and their psychotic sons? Or Frege with--

Christos: `The meaning
is in the ending!' you said so yourself!  So, follow the quest for ten more years and
you get a brand-new triumphant finale with the creation of the computer, which is the
quest's real hero!   Your problem is, simply, that you see it as a story of people!

Apostolos: Well, stories do tend to be about people!

Christos:  So, choose the right people!  And show what they really did!  All we we learn
of the great von Neumann is he said `It's over'  when he heard G\"odel!

Alecos: But it was over in a sense, wasn't it?  Pop went Hilbert's `no ignorabimus'!

Christos:  But then came the quest's jeune premier, its parsifal $\ldots$ Alan Turing!
He said `Ok, we can't prove everything! So, let's see what we can prove!' and to define
proof, he invented, in 1936, a theoretical machine which contains all the ideas of the
computer! $\ldots$ which, after the war, he and von Neumann, the quest's proudest sons,
brought to full life!

{\hfill--Logicomix} % page 303.

}

}




\newpage

{

\narrower\parindent=0pt
\parskip=0.4\baselineskip

{\it

``Ever hear of the Kepler Conjecture?''

``Nope.''

I laid the notebook on the table and flipped through the pages. ``It was first stated
in 1611 by Johannes Kepler,'' I said.  ``Kepler becomae interested in the problem while
he was corresponding with an Englishman named Thomas Harriot, who was trying
to help his friend Sir Walter Raleigh figure out the best way to stack cannonballs on ship
decks.  The goal was to find the densest possible spherical arrangements, $\ldots$ basically,
the way grocers stack oranges''

``Okay,'' he said, nodding.

``Kepler's conjecture {\rm{seems}} perfectly sound,'' I said.

``That is does,'' Ben said.

``But here's the thing. $\ldots$  I looked it up and discovered that, in 1998, a proof
had finally been put forward by an American mathematician named Thomas Hales.  In 2003,
a committee that had been assigned to verify Hales's work confirmed that they were ninety-nine
percent certain of the proof's correctness.  But that one percent was key.  The mathematical
world is still waiting for the publication of the data that will prove the Kepler Conjecture definitively.''

``Sucks for Thomas Hales,'' Ben said.

``I agree.  But it makes sense that they have to be certain, doesn't it?''


{\hfill--Michelle Richmond, No One You Know} % page 177.

}

}

%%DD Figure of cannonballs.

\bigskip

{

\narrower\parindent=0pt
\parskip=0.4\baselineskip

{\it

Sometimes fixing a $1$ percent defect takes $500$ percent effort.

{\hfill-- Joel Spolsky, Joel on Software} % page 122

}

}

\bigskip

{

\narrower\parindent=0pt
\parskip=0.4\baselineskip

{\it

Every one fully persuaded is a fool.

{\hfill-- Balthasar Graci\'an, the Art of Worldly Wisdon} % p110

}

}



\newpage

\section*{Blueprint for Formal Proofs}

In 1611, Kepler wrote a booklet in which he asserted that the familiar
cannonball arrangement of congruent balls in space achieves the
highest possible density.  No other arrangement fills a larger
fraction of space.  This assertion is the Kepler conjecture.  In 1900,
Hilbert made this conjecture part of his eighteenth problem.  This
book presents a proof of this assertion.

This assertion has become a test of the capability of
computers to deliver a reliable mathematical proof.  The original
proof by Sam Ferguson and me involved many long computer calculations
that exhausted the efforts of a team of referees.  This book represents my efforts
to redesign the proof in a way that makes the correctness of the computer proof as transparent
as possible.

After all is said and done, no proof is more reliable than the
reliability of the processes that are used to verify its
correctness.   The ultimate standard of proof is a formal proof.  A formal proof
is nothing other than an unbroken chain of logical inferences from an explicit set
of axioms.  While this may be the mathematical ideal of proof, actual mathematical
practice generally deviates significantly from the ideal.



%More than ten years have passed since a proof was first
%obtained. Why give a new presentation of the proof?
%
%The original proof was not widely understood.  The complexity was not
%because of conceptual challenges.  In fact, the proof makes only
%modest demands on the theoretical training of the reader.  It is
%possible to read and understand the proof with a knowledge of a
%limited body of mathematics, such as basic calculus and elementary
%Euclidean geometry.
%
%Nevertheless, the proof involves many long calculations. Even worse,
%it it relies on computer calculations.  An error in any calculation or
%a bug in the computer code has the potential to topple the entire
%proof.
%
%The referees were conscientious and checked many of the calculations.
%However, for the most part, the computer code lay beyond the scope of
%referee review, and even careful quality control can let a
%occasional bug slip through undetected.
%
%After all is said and done, no proof is more reliable than the
%reliability of the processes that are used to verify its
%correctness.  These processes include the checking that the author
%makes before releasing the proof for public scrutiny, the checking
%of the referees, and the checking done by readers after publication.

In recent years, as part of this project, I have been increasingly preoccupied by the
processes that mathematicians rely on to insure the correctness of complex
proofs. Researchers from Frege to G\"odel, who solved a problem of
rigor in mathematics, found a theoretical solution but did not
extinguish the burning fire at the foundations of mathematics,
because they omitted the practical implementation. Some, such as
Bourbaki, have even gone so far as to claim that ``formalized
mathematics cannot in practice be written down in full'' and call
such a project
``absolutely unrealizable'' \cite[p 10,11]{Bour:68:Sets}. % Theory of
                                                          % Sets, page
                                                          % 10,11.

While it is true that formal proofs may be too long to print,
computers -- which do not have the same limitations as paper -- have
become the natural host of formal mathematics. In recent decades,
logicians and computer scientists have reworked the foundations of
mathematics, putting them in an efficient form designed for real use
on real computers.

For the first time in history, it is possible to generate and verify
every single logical inference of major mathematical theorems.  This
has now been done for the four-color theorem, the prime number
theorem, the Jordan curve theorem, the Brouwer fixed point theorem,
the fundamental theorem of calculus, and many other theorems.  Freek
Wiedijk reports that 82\% of a list of 100 famous theorems have now
been checked formally \cite{wiedijk:100}.  The list of 18 remaining
theorems contains two particular challenges: the independence of the
Continuum Hypothesis and Fermat's Last theorem.

Some mathematicians remain skeptical of the process because
computers have been used to generate and verify the logical
inferences.  Computers are notoriously imperfect, with flaws ranging
from software bugs to defective chips.  Even if a computer verifies
the inferences, who will verify the verifier, or then verify the
verifier of the verifier?  
Indeed, it would be unscientific of us to
place an unmerited trust in computers.

The choice comes down to two competing verification processes.  The
first is the traditional process of referees, which depends largely on
the luck of the draw -- some referees are meticulous, others are
careless.  The second process is formal computer verification. In this
case, the process is less dependent on the whims of a particular
referee.  In my view, the choice between the conventional referee
process and computer verification is as clear as the choice between
a sundial and an atomic clock in contemporary science.  

The boundary that separates an ``easy'' proof from a ``difficult''
proof shifts with current technology.  The introduction of steel in
architecture is not a mere reinforcement of wood and stone, it changes
the architect's world of possibilities.  There will no longer be any
reason to limit ourselves to ten-thousand-page proofs when our
technology supports million page proofs.

The standard of proof I have adopted is the highest scientific standard
available by current technology.  That 
standard is formal verification by computer.  This standard
continues to evolve with the advancement of technology.

My dream is to have some day a fully formally verified solution to the
packing problem.  This project is still unfinished, but significant progress is
being made.  This book is an attempt to rearrange the proof in such a
way to make the formal verification easier.  The book is called {\it a
  blueprint for formal proofs} because it is the text that gives the
design of the formal proof to be constructed.  My decisions about what
to include in this book has been shaped by the list of theorems
already available in the library of proof assistants such as {\tt HOL
  Light}.  For example, book assumes basic point-set topology and
measure theory, because these topics have been formalized by John
Harrison~\cite{unknown}.

The style of formal proofs is different from that of
conventional proofs.  It is better to have a large number of short
snappy proofs, rather than a few ingenious ones.  Humans enjoy
surprising new perspectives, but computers prefer standardization.
Despite these differences, I have worked to make proofs that
will bring pleasure  to the human reader while providing precise instructions
for the implementation in silicon.



\section*{Structure of this Book}

The book is divided into parts.
The introductory part describe the major ideas, methods, and
organization of the proof.  

%There is an essay on each major computer
%component of the proof. The purpose is to provide a panoramic view of
%proof, to provide intuition about proof strategies.  After reading
%this part of the book, the reading should understand what the proof is
%all about, without yet dipping into technical details.
The part on foundations provides background material about
constructions in discrete geometry.    The first
of these chapters
covers trigonometric identities and basic vector geometry.  The second
treats volume from an elementary point of view.  The third chapter
covers planar graph theory from a purely combinatorial point of view.
The fourth chapter continues with planar graphs, now from a 
geometric point of view.

The next part of the book gives the solution to the packing problem.
The first chapter in this part gives a top-level overview of the major
steps of the proof.  It describes how the problem can be reduced from
a problem in infinitely many variables to a problem in finitely many
variables.  The remaining chapters in this part flesh out that
skeleton.

The final part of the book resolves some other longstanding conjectures in
discrete geometry: K. Bezdek's strong dodecahedral conjecture and Fejes
T\'oth's full contact conjecture.

Many simplifications of the original proof have been found over the past
several years.  The simplified proof is published here for the first time.
G. Gonthier expresses his formal proof of the four-color
theorem in terms of hypermaps.  He reworks the proofs of the
four-color theorem to avoid the use of the Jordan curve theorem, using
instead the much simpler notion of M\"obius contour.  I have followed
Gonthier's lead in these respects and also avoid the use of the Jordan curve theorem.

The packing problem gives an assertion about packings of spheres in
all of $\ring{R}^3$.  There are many ways to introduce {\it correction
  terms} that reduce the packing problem to a minimization problem in
a finite number of variables.  In a discussion of the solution of the
packing problem, I wrote that ``correction terms are extremely
flexible and easy to construct, and soon Samuel Ferguson and I
realized that every time we encountered difficulties in solving the
minimization problem, we could adjust $f$ [the correction term] to
skirt the difficulty. $\ldots$ The correction function did not become
fixed until it came time for Ferguson to defend his thesis, and we
finally felt obligated to stop tampering with it.  However, if I were
to revise the proof to produce a simpler one, the first thing I would
do would be to change the correction function once again.  It is the
key to a simpler proof.''  C. Marchal has recently fulfilled my
longstanding dream of finding a vastly simpler correction term.~\cite{unknown}.  His
construction has indeed become the key to the simpler proof presented
in this book.  I gratefully acknowledge his contributions to this
project.





\bigskip
\hbox{}



\bigskip
\hbox{}

{
\parindent=0pt
\obeylines

Thomas C. Hales
Pittsburgh, PA
May 2010

}







