


\def\odpcvgh{
\tikzfig{trig}{\guid{ODPCVGH} Trigonometric and inverse trigonometric
functions}
%
%arctangent function on the domain \leftopen -4,4\rightopen\ 
%and the $\arccos$ function on $\leftclosed-1,1\rightclosed$.}}
{
[scale=0.5]
\draw (-2*1.57,0) sin (-1.57,-1) cos (0,0) sin (1.57,1) cos (3.14,0) sin (3*1.57,-1);
\draw   (-2*1.57,-1) cos (-1.57,0) sin  (0,1) cos (1.57,0) sin (2*1.57,-1) cos (3*1.57,0); 
\draw[help lines,<->] (-3.3,0) -- (3*1.57 + 0.2,0);
\draw[help lines,<->] (0,-1) -- (0,2.0);
\draw plot[smooth] file {tikz/tan.table};
\node at (-0.5,-1.8) {$\tan$};
\node at (2,0.5) {$\sin$};
\node at (1.1,-0.3) {$\cos$};
% GG need axis labels and ticks, base points of labels should be precisely aligned.
\begin{scope}[xshift=10cm]
\draw plot[smooth] file {tikz/arctan.table} node[above] {$\arctan$};
\draw plot[smooth] file {tikz/arccos.table} node[right] {$\arccos$};
%\draw[gray,->,very thin] (-1.2,0) -- (1.4,0);
\draw[help lines,<->] (0,-1.6) -- (0,3.2);
\draw[help lines,<->] (-4,0) -- (4,0);
\end{scope}
}}

%
\tikzfig{circle}{\guid{HGMTQFG} Lemma~\ref{lemma:circle} as a special case of the Pythagorean theorem}
{
[scale=0.1]
\draw (0,0)  --(12,0) --  (12,5) --  cycle;
\draw[very thin] (11,0) -- (11,1) -- (12.0,1);
\path (5,-1.5) node {$\cos x$};
\path (16,2.5) node {$\sin x$};
\path (6,5)  node {$1$};
}

%
\tikzfig{tan}{\guid{GQQAKYI} 
The tangent function on $\leftopen-\pi/2,\pi/2\rightopen$}
{
[scale=0.2]
\draw plot[smooth] file {tikz/tan.table};
\draw[help lines,->] (-1.57,0) -- (1.57,0);
\draw[help lines,->] (0,-6.0) -- (0,6.0);
}

\tikzfig{arctrig}{\guid{RUJPPWJ} 
The arctangent function on the domain \leftopen -4,4\rightopen\ 
and the $\arccos$ function on $\leftclosed-1,1\rightclosed$.}
{
[scale=0.4]
\draw plot[smooth] file {tikz/arctan.table};
\draw plot[smooth] file {tikz/arccos.table};
%\draw[gray,->,very thin] (-1.2,0) -- (1.4,0);
\draw[help lines,->] (0,-0.2) -- (0,3.2);
\draw[help lines,->] (-4,0) -- (4,0);
}

\tikzfig{atn-polar}{\guid{YOXQFUB} 
The function $\atn$ gives the polar angle $\theta$ of $(x,y)$.}
{
[scale=0.15]
\draw[gray,->,very thin] (-4,0) -- (14,0);
\draw[gray,->,very thin] (0,-2) -- (0,5);
\draw (0,0)  --(12,0) --  (12,5) --  cycle;
\draw[very thin] (11,0) -- (11,1) -- (12.0,1);
\path (6,-1.5) node {$x$};
\draw[very thin] (4,0) arc (0:22.62:4);
\path (14,2.5) node {$y$};
\path (2,3) node {$\theta$};
}

\def\tuligly{
\tikzfig{M}{\guid{TULIGLY} The quartic polynomial $M$}
{
[scale=2.0]
\draw[help lines,<->] (1.0,0) -- (1.5,0);
\draw[help lines,<->] (1,-0.2) -- (1,1.2);
\draw plot[smooth] file {tikz/TULIGLY.table};
}}

\def\bjliekb{
\tikzfig{L}{\guid{BJLIEKB} Detail of the quartic $M$ and piecewise linear function $L$ on
the domain $\leftclosed1.2,1.35\rightclosed$}
{
[scale=12.0]
\draw plot[smooth] file {tikz/BJLIEKB.table};
\draw[help lines,<->] (1.18,0) -- (1.37,0);
\draw[help lines,<->] (1.2,-0.01) -- (1.2,0.25);
\draw[help lines] (1.23175,0.25) -- (1.23175,-0.01) node[anchor=north,black] {$h_-$};
\draw[help lines] (1.26,0.25) -- (1.26,-0.01) node[anchor=north,black] {$~~\hm$};
\draw[help lines] (1.3254,0.25) -- (1.3254,-0.01) node[anchor=north,black] {$h_+$};
\draw (1.2,0.230769 ) -- (1.26,0);  %
\draw (1.26,0) -- (1.35,0);
}}

%



