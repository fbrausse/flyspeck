% global graphics definitions.


% - Tikz.execute() now causes a stack overflow (in ML but not with HOL-Light loaded).

% keywords: shorten (shorten lines), pos (placing nodes on a curve)
%  positioning (for anchors)
% z= set z coordinate.
% .. controls +(up:2cm) and +(left:2cm) .. (1,3) 
% barycentric:

% watch out for variable overwrite with macros, as in setellipseplane.  
% It overwrites \vx,\vy, creates havoc, etc.

% BUG in TIKZ.
% arc: The start angle is not the angle on the ellipse, but rather the angle
% on the corresponding circle.

% TIKZ annoyance.
% Things are converted to points at an indefinite time.
% If I set coordinates in cm then extract again, they will be in points?

%%%%%%%%%%%%%%%%%%%%%%%%%%%%%%%%%%%%%%%%%%%%%%%%%%%%%%%%%%%%%%%%%%%%%%%%%%%%%%%%
%% INITIALIZATION
%%%%%%%%%%%%%%%%%%%%%%%%%%%%%%%%%%%%%%%%%%%%%%%%%%%%%%%%%%%%%%%%%%%%%%%%%%%%%%%%

% init.tex has now been merged with this file.
% DEPRECATED: \tikzset{help lines/.style={very thin,gray}}
% CUP complained of "very thin" lines.  Change to thin.
\tikzset{help lines/.style={gray}}

% Remove background for final version.
% \tikzset{background rectangle/.style={fill=gray!7,rounded corners=1ex}} 
% was blue!7, now bw
%\tikzset{every picture/.style={show background rectangle}}
%\tikzset{background rectangle/.style={fill=blue!20,rounded corners=1ex}}

% autoGUID functions are contained in this file.
%\input /tmp/x.txt  % generate by tikz.ml : Tikz.execute.


%\input tikz/tikz_auto.txt  % copy of /tmp/x.txt.
\input x.txt

%These have been merged with x.txt: \input /tmp/y.txt
%\input /tmp/z.txt

\def\smalldot#1{\draw[fill=black] (#1) node [inner sep=1.3pt,shape=circle,fill=black] {}}
\def\graydot#1{\draw[fill=gray] (#1) node [inner sep=1.3pt,shape=circle,fill=gray] {}}
\def\whitedot#1{\draw[fill=gray] (#1) node [inner sep=1.3pt,shape=circle,fill=white,draw=black] {}}
\tikzset{dartstyle/.style={fill=black,rotate=-90,inner sep=0.7pt,dart,shape border uses incircle}}
\tikzset{grayfatpath/.style={line width=1ex,line cap=round,line join=round,draw=gray}}

\pgfmathsetmacro\cmofpt{(2.54/72.0)}

%\def\dart#1{{\node[fill=black,rotate=-90,inner sep=1.3pt,dart,shape border uses incircle] at (#1) {$a$}}}

% EXTRACTING COORDINATES.
% NOT RECOMMENDED: it converts everything to points; units get messed up.

% search pdf manual for "Extracting coordinates" to get x and y coords of a point.

%\draw
%  let
%    \p1=($(a.north)!0.5!(b.south)$),
%    \p2=(current bounding box.west),
%    \p3=(current bounding box.east)
%  in
%    (\x2,\y1) -- (\x3, \y1);

%\newdimen\Fx
%\newdimen\Fy
%\newdimen\Gx
%\newdimen\Gy
%\def\extractF#1{\pgfextractx\Fx{\pgfpointanchor{#1}{center}};\pgfextracty\Fy{\pgfpointanchor{#1}{center};}}
%\def\extractG#1{\pgfextractx\Gx{\pgfpointanchor{#1}{center}};\pgfextracty\Gy{\pgfpointanchor{#1}{center};}}

%%%%%%%%%%%%%%%%%%%%%%%%%%%%%%%%%%%%%%%%%%%%%%%%%%%%%%%%%%%%%%%%%%%%%%%%%%%%%%%%
%% FCC CLOSE PACKING CHAPTER
%%%%%%%%%%%%%%%%%%%%%%%%%%%%%%%%%%%%%%%%%%%%%%%%%%%%%%%%%%%%%%%%%%%%%%%%%%%%%%%%


\def\figSEYIMIE{
\tikzfig{fcc-fun-domain}{\guid{SEYIMIE} 
The fundamental domain of the FCC lattice can be partitioned into
two regular tetrahedra and a regular octahedron.  The fundamental domain tiles space.  Tetrahedra
and octahedra tile space in the ratio 2:1.
}
{
[scale=2.0]
\autoSEYIMIE
\coordinate (U) at (0,0);
\draw[thick,fill=gray!20] (U)--(w1)--(w13)--(w3)--cycle;
\draw[thick,fill=gray!35] (w1)--(w12)--(w123)--(w13)--cycle;
\draw[thick,fill=gray!5] (w3)--(w13)--(w123)--(w23)--cycle;
\draw[black!70] (w1)--(w3);
\draw[black!70] (w13)--(w23);
\draw[black!70] (w13)--(w12);
\draw[black!70] (w23)--(w12)--(w2)--cycle;
\draw[black!70] (w1)--(w2)--(w3);
\draw[black!70] (w2)--(U);
\foreach \i in {U,w1,w2,w3,w12,w13,w23,w123} {\smalldot {\i}; }
}
}

\def\figPQJIJGE{
\tikzfig{rhombic-dodec}{\guid{PQJIJGE} 
The Voronoi cell of the  FCC packing is a rhombic dodecahedron.
It can be constructed by placing a square-based pyramid along each facet
of an inscribed cube.  
}
{
[scale=1.0]
\pgfmathsetmacro\r{sqrt(2.0)}
\begin{scope}
\autoPQJIJGE
\draw[black,line join=bevel] (wtop)--(w23)--(wleft)--(w2)--cycle;
\draw[black,line join=bevel] (wbottom)--(w3)--(wleft);
\draw[black,line join=bevel] (w123)--(wback)--(w23);
\draw[black,line join=bevel] (w13)--(wback)--(w123);
\draw[black,line join=bevel] (wback)--(w3);
\draw[ball color=gray!10,draw=gray] (wcenter) circle (\r);
\draw[thick,fill=gray!40,nearly transparent,line join=bevel] (wbottom)--(w1)--(wfront)--(w0)--cycle;
\draw[thick,fill=gray!5,nearly transparent,line join=bevel] (wfront)--(w2)--(wleft)--(w0)--cycle;
\draw[thick,fill=gray!30,nearly transparent,line join=bevel] (wfront)--(w1)--(wright)--(w12)--cycle;
\draw[thick,fill=gray!10,nearly transparent,line join=bevel] (wfront)--(w12)--(wtop)--(w2)--cycle;
\draw[thick,fill=gray!20,nearly transparent,line join=bevel] (wright)--(w123)--(wtop)--(w12)--cycle;
\draw[thick,fill=gray!70,nearly transparent,line join=bevel] (wbottom)--(w1)--(wright)--(w13)--cycle;
\end{scope}
\begin{scope}[shift={(5,0)}]
\autoPQJIJGE
\draw[thick,draw=black,fill=gray!40,line join=bevel] (wbottom)--(w1)--(wfront)--(w0)--cycle;
\draw[thick,draw=black,fill=gray!5,line join=bevel] (wfront)--(w2)--(wleft)--(w0)--cycle;
\draw[thick,draw=black,fill=gray!30,line join=bevel] (wfront)--(w1)--(wright)--(w12)--cycle;
\draw[thick,draw=black,fill=gray!10,line join=bevel] (wfront)--(w12)--(wtop)--(w2)--cycle;
\draw[thick,draw=black,fill=gray!20,line join=bevel] (wright)--(w123)--(wtop)--(w12)--cycle;
\draw[thick,draw=black,fill=gray!70,line join=bevel] (wbottom)--(w1)--(wright)--(w13)--cycle;
\draw[gray,line join=bevel] (wtop)--(w23)--(wleft)--(w2)--cycle;
%\draw[gray] (wbottom)--(w3)--(wleft);
%\draw[gray] (w123)--(wback)--(w23);
%\draw[gray] (w13)--(wback)--(w123);
%\draw[gray] (wback)--(w3);
\draw[very thick,black,line join=bevel] (w0)--(w1)--(w12)--(w2)--cycle;
\draw[very thick,black,line join=bevel] (w123)--(w12)--(w1)--(w13);
\draw[very thick,black,line join=bevel] (w2)--(w12)--(w123);
\draw[gray,line join=bevel] (w2)--(w23)--(w123)--(w13);
\end{scope}
}
}

\def\figSGIWBEN{
\tikzfig{fcc-hcp-pattern}{\guid{SGIWBEN} 
The patterns of twelve neighboring points in the FCC and HCP packings.
In both cases, the convex hull of the twelve points
is a polyhedron with six squares and eight triangles, but the top layer of the
HCP pattern is rotated 60 degrees with respect to the FCC pattern. 
The FCC pattern is a cuboctahedron.  In the HCP
pattern,
there is a uniquely determined plane of reflectional symmetry, 
containing six of the
twelve points.  
}
{
[scale=1.8]
\begin{scope}
\autoSGIWBEN
% top half
\draw[fill=gray!20] (w8)--(w5)--(w6)--(w10)--(w11)
 -- (w1)--(w2)--(w3)--cycle;
\draw[gray] (w5)--(w4)--(w2);
\draw[gray] (w10)--(w5);
\draw[gray] (w12)--(w2) (w12)--(w4);
\draw[fill=gray!70] (w1)--(w2)--(w4)--(w5)--(w6)--(w7)--cycle;
\draw (w4)--(w8);
\draw[] (w1)--(w2)--(w3)--cycle;
\draw[] (w1)--(w3)--(w9)--(w7)--cycle;
\draw[] (w7)--(w9)--(w6)--cycle;
\draw[] (w6)--(w9)--(w8)--(w5)--cycle;
\draw[fill=gray] (w3)--(w8)--(w9)--cycle;
\node at ($(w10)+(-0.3,0)$) {$B$};
\node at ($(w5)+(-0.3,0)$) {$A$};
\node at ($(w8)+(-0.3,0)$) {$C$};
% bottom half
\draw[fill=gray] (w10)--(w11)--(w12)--cycle;
\draw[] (w1)--(w7)--(w11)--cycle;
\draw[] (w7)--(w6)--(w10)--(w11)--cycle;
\draw[very thick] (w2)-- (w1)--(w7)--(w6)--(w5);
\node at (-90:1.2) {FCC};
\foreach\i in {w2,w1,w7,w6,w5,w3,w9,w8,w10,w11} {
  \smalldot{\i}; }
%\foreach\i in {w2,w1,w7,w6,w5,w3,w9,w8,w10,w11} {
%  \node[shape=circle,fill=white] at (\i) {\i}; }
\foreach\i in {w12,w4} {
  \graydot{\i}; }
\whitedot{w0};
\end{scope}
\begin{scope}[shift={(3.5,0)}]
\autoSGIWBEN
% top half
\draw[fill=gray!20] (w13)--(w5)--(w6)--(w10)--(w11)
 -- (w1)--(w2)--(w15)--cycle;
\draw[gray] (w10)--(w5);
\draw[gray] (w12)--(w2) (w12)--(w4);
\draw[fill=gray!70] (w1)--(w2)--(w4)--(w5)--(w6)--(w7)--cycle;
\draw[] (w2)--(w15)--(w14)--(w1)--cycle;
\draw[] (w1)--(w14)--(w7)--cycle;
\draw[] (w7)--(w14)--(w13)--(w6)--cycle;
\draw[] (w6)--(w13)--(w5)--cycle;
\draw[gray] (w4)--(w15);
\draw[fill=gray] (w13)--(w14)--(w15)--cycle;
\node at ($(w10)+(-0.3,0)$) {$B$};
\node at ($(w5)+(-0.3,0)$) {$A$};
\node at ($(w13)+(-0.3,0)$) {$B$};
% bottom half
\draw[fill=gray] (w10)--(w11)--(w12)--cycle;
\draw[] (w1)--(w7)--(w11)--cycle;
\draw[] (w7)--(w6)--(w10)--(w11)--cycle;
\draw[very thick] (w2)-- (w1)--(w7)--(w6)--(w5);
\draw[gray] (w5)--(w4)--(w2);
\node at (-90:1.2) {HCP};
\foreach\i in {w2,w1,w7,w6,w5,w13,w14,w15,w10,w11} {
  \smalldot{\i}; }
\foreach\i in {w12,w4} {
  \graydot{\i}; }
\whitedot{w0};
%\foreach\i in {w2,w1,w7,w6,w5,w13,w14,w15,w10,w11} {
%  \node[shape=circle,fill=white] at (\i) {\i}; }
\end{scope}
}
}


\def\figDHQRILO{
\tikzfig{fcc-packing}{\guid{DHQRILO} 
The face-centered cubic (FCC) packing.
}
{
[scale=0.9]
\begin{scope}
\autoDHQRILO
\coordinate (w32) at ($(w3)-(w2)$);
\coordinate (w13) at ($(w1)-(w3)$);
\coordinate (p) at ($-4 *(w32) -4 *(w13)$);
\foreach \i/\j in 
  {0/0,1/1,1/0,2/2,2/1,2/0,3/3,3/2,3/1,3/0,4/4,4/3,4/2,4/1,4/0} {
\coordinate (V\i\j)  at ($(p)+ \i*(w32) + \j*(w13)$);
\draw[ball color=gray!10,draw=gray] (V\i\j) circle (0.5);
}
\foreach \i/\j in 
{4/0,4/1,4/2,4/3,4/4,3/0,3/1,3/2,3/3,2/0,2/1,2/2,1/0,1/1,0/0} {
\coordinate (V\i\j) at ($ - \i*(w1) + \j *(w3)$);
\draw[ball color=gray!10,draw=gray] (V\i\j) circle (0.5);
}
\end{scope}
}
}

\def\figKSOEMIZ{
\tikzfig{sanskrit}{\guid{KSOEMIZ} 
  Derivation of Sanskrit formula \eqref{eqn:sanskrit}.  A cannonball packing can be
  converted to unit cubes  in a staircase aligned along the rear column.
  Six staircase shapes fill an $(n+1)^3$ cube without its diagonal of $n+1$ 
 unit cubes, or a
  rectangle of dimensions $n$ by $n+1$ by $n+2$.   
}
{
[scale=0.8]
\begin{scope}
\autoDHQRILO
\coordinate (w32) at ($(w3)-(w2)$);
\coordinate (w13) at ($(w1)-(w3)$);
\coordinate (p) at ($-2 *(w32) -2 *(w13)$);
\foreach \i/\j in 
  {0/0,1/1,1/0,2/2,2/1,2/0} {
\coordinate (V\i\j)  at ($(p)+ \i*(w32) + \j*(w13)$);
\draw[ball color=gray!10,draw=gray] (V\i\j) circle (0.5);
}
\foreach \i/\j in 
{2/0,2/1,2/2,1/0,1/1,0/0} {
\coordinate (V\i\j) at ($ - \i*(w1) + \j *(w3)$);
\draw[ball color=gray!10,draw=gray] (V\i\j) circle (0.5);
}
\end{scope}
\foreach \i/\j/\k in {
   0/2/0,0/1/0,1/1/0,0/0/0,1/0/0,2/0/0,
  0/2/1,0/1/1,1/1/1,0/2/2}
{
\coordinate (f3) at ($1.3*(e3)$);
\coordinate (C\i\j\k) at ($(2,-0.5)+\i*(e1) + \j*(e2)+\k *(f3)$);
\draw[fill=gray!25] (C\i\j\k)-- +(e1)-- +(e13)-- +(e3)--cycle;
\draw[fill=gray!50] ($(C\i\j\k)+(e1)$)-- +(e2)-- +(e23)-- +(e3)--cycle;
\draw[fill=gray!7] ($(C\i\j\k)+(e3)$)-- +(e1)-- +(e12) -- +(e2)--cycle;
}
\foreach \i/\j/\k in {
   0/2/0,0/1/0,1/1/0,0/0/0,1/0/0,2/0/0,
  0/2/1,0/1/1,1/1/1,0/2/2}
{
\coordinate (C\i\j\k) at ($(6,-0.5)+\i*(e1) + \j*(e2)+\k *(e3)$);
\draw[fill=gray!25] (C\i\j\k)-- +(e1)-- +(e13)-- +(e3)--cycle;
\draw[fill=gray!50] ($(C\i\j\k)+(e1)$)-- +(e2)-- +(e23)-- +(e3)--cycle;
\draw[fill=gray!7] ($(C\i\j\k)+(e3)$)-- +(e1)-- +(e12) -- +(e2)--cycle;
}
\draw[thick] ($(C022) + 0.5 *(e3) + 0.5*(e123)$) -- +(90:1);
\begin{scope}[shift={(-2,-6.8)}]
\autoDHQRILO
\coordinate (m2) at ($-1.0*(e2)$);
\coordinate (m1) at ($-1.0*(e1)$);
\coordinate (m3) at ($-1.0*(e3)$);
\draw[thick,fill=gray!25] (w0)-- +($4*(e1)$)-- +($4*(e13)$)-- +($4*(e3)$)--cycle;
\draw[thick,fill=gray!50] ($4*(e1)$)-- +($4*(e2)$)-- +($4*(e23)$)-- +($4*(e3)$)--cycle;
\draw[thick,fill=gray!7] ($4*(e3)$)-- +($4*(e1)$)-- +($4*(e12)$) -- +($4*(e2)$)--cycle;
\draw (e1)-- ++(e3)-- ++(e1)-- ++(e3)-- ++(e1) 
 -- ++($2*(e3)$) -- ++($4*(e2)$);
\draw ($4*(e1)+4*(e2)+(e3)$)-- ++(m2)-- ++(e3)-- ++(m2)-- ++(e3)
 -- ++($2*(m2)$) -- ++($4*(m1)$);
\draw ($4*(e1)+(e2)$)-- ++($4*(e3)$)-- ++($2*(m1)$)-- ++(e2)-- ++(m1)--
  ++(e2)-- ++(m1);
\draw[thick,fill=gray!25] ($4*(e1)+(e2)+3*(e3)$)-- ++(e3)-- ++(m1)-- ++(m3)--cycle;
\draw[thick,fill=gray!50] ($3*(e1)+3*(e3)$)-- ++(e2)-- ++(e3)-- ++(m2)--cycle;
\draw[thick,fill=gray!7] ($3*(e1)+3*(e3)$)-- ++(e1)-- ++(e2)-- ++(m1)--cycle;
\draw[thick] ($0.5*(e1)+0.5*(e3)$)-- ++(m2);
\draw[thick] ($1.5*(e1)+0.5*(e3)$)-- ++(m2);
\draw[thick] ($4*(e1)+3.5*(e2)+0.5*(e3)$)-- ++(e1);
\draw[thick] ($4*(e1)+3.5*(e2)+1.5*(e3)$)-- ++(e1);
\draw[thick] ($4*(e3)+3.5*(e2)+0.5*(e1)$)-- ++(e3);
\draw[thick] ($4*(e3)+2.5*(e2)+0.5*(e1)$)-- ++(e3);
\end{scope}
\begin{scope}[shift={(5,-6.8)}]
\autoDHQRILO
\coordinate (m2) at ($-1.0*(e2)$);
\coordinate (m1) at ($-1.0*(e1)$);
\coordinate (m3) at ($-1.0*(e3)$);
\draw[thick,fill=gray!25] (w0)-- ++($3*(e1)$)-- ++($4*(e3)$)-- ++($3*(m1)$)--cycle;
\draw[thick,fill=gray!50] ($3*(e1)$)-- ++($5*(e2)$)-- ++($4*(e3)$)-- ++($5*(m2)$)--cycle;
\draw[thick,fill=gray!7] ($4*(e3)$)-- ++($3*(e1)$)-- ++($5*(e2)$) -- ++($3*(m1)$)--cycle;
\draw (e3)-- ++(e1)-- ++(e3)-- ++(e1)-- ++(e3) -- ++(e1)--
  ++($5*(e2)$);
\draw ($3*(e1)+4*(e2)$)-- ++(e3)-- ++(m2)-- ++(e3)-- ++(m2)
 -- ++($2*(e3)$) -- ++(m1)-- ++(e2)-- ++(m1)-- ++(e2)-- ++(m1);
\draw ($4*(e3)+(e2)$)-- ++($3*(e1)$)-- ++($4*(m3)$);
\draw[thick] ($0.5*(e1)+0.5*(e3)$)-- ++(m2);
\draw[thick] ($0.5*(e1)+1.5*(e3)$)-- ++(m2);
\draw[thick] ($3*(e1)+3.5*(e2)+0.5*(e3)$)-- ++(e1);
\draw[thick] ($3*(e1)+4.5*(e2)+0.5*(e3)$)-- ++(e1);
\draw[thick] ($4*(e3)+3.5*(e2)+0.5*(e1)$)-- ++(e3);
\draw[thick] ($4*(e3)+4.5*(e2)+0.5*(e1)$)-- ++(e3);
\end{scope}
}
}

\def\figBDCABIA{
\tikzfig{pascal}{\guid{BDCABIA} 
The binomial coefficient $\binom{d+n}{n}$ 
gives the general formula 
for the number of balls in a $d$-dimensional pyramid of side $n+1$.
As Harriot observed, the recursion of Pascal's triangle 
$
\binom{d+n}{n}=\binom{d+(n-1)}{n-1}+\binom{(d-1)+n}{n}
$
can be interpreted as a partition of a pyramid of side $n+1$ into a pyramid
of side $n$ resting on a  pyramidal base of side $n+1$ in dimension $d-1$.
}
{
[scale=0.9]
\autoBDCABIA
\pgfmathsetmacro\r{0.3};
\pgfmathsetmacro\s{\r*0.5};
\pgfmathsetmacro\sp{1.85};
\coordinate (u1) at (240:\sp);
\coordinate (u2) at (-60:\sp);
\foreach \i in {0,1,2,3,4} {
  \pgfmathsetmacro\a{5-\i};
  \draw[gray] ($\i*(u1)+(u2)+(-0.5,0)$)-- ++($\a*(u2)$);
  \node at ($\i*(u1)+(u2)+(-0.5,0.8)$) {$\i$-D};
}
\begin{scope} %0D
\foreach \i in {1,2,3,4,5,6} {
\coordinate (V\i) at ($(w0)+\i*(u2)$);
\node at ($(V\i)+(-0.2,0)$) {1};
\draw[ball color=gray!10,draw=gray] ($(V\i)+(\s,0)$) circle (\s);
}
\end{scope}
\begin{scope} %1D
\foreach \i in {1,2,3,4,5} {
 \coordinate (V\i) at ($(u1)+\i*(u2) $);
\node at ($(V\i)+(-0.2,0)$) {\i};
\foreach \j in {1,...,\i} {
\pgfmathsetmacro\t{\r*\j};
\coordinate (V\i\j) at ($(V\i)+\t*(w1)$);
\draw[ball color=gray!10,draw=gray] (V\i\j) circle (\s);
}
}
\end{scope}
\begin{scope} %2D
\foreach \i/\c in {1/1,2/3,3/6,4/10} {
 \coordinate (V\i) at ($2*(u1)+\i*(u2) $);
\node at ($(V\i)+(-0.2,0)$) {\c};
\foreach \k in {1,...,\i} {
\foreach \j in {1,...,\k} {
\pgfmathsetmacro\t{\r*\j};
\pgfmathsetmacro\v{\r*\i - \r*\k};
\coordinate (V\i\j\k) at ($(V\i)+\t*(w1) + \v*(w2)$);
\draw[ball color=gray!10,draw=gray] (V\i\j\k) circle (\s);
}
}
}
\end{scope}
\begin{scope} %3D
\coordinate (w31) at ($(w3)-(w1)$);
\coordinate (w21) at ($(w2)-(w1)$);
\foreach \i/\c in {1/1,2/4,3/10} {
 \coordinate (V\i) at ($3*(u1)+\i*(u2) $);
\node at ($(V\i)+(-0.2,0)$) {\c};
\foreach \k in {1,...,\i} {
\foreach \j in {1,...,\k} {
\pgfmathsetmacro\u{\r*\i};
\pgfmathsetmacro\t{\r*\k-\r*\j};
\pgfmathsetmacro\v{\r*\i - \r*\k};
\coordinate (V\i\j\k) at ($(V\i)+\u*(w1)+\t*(w31) + \v*(w21)$);
\draw[ball color=gray!10,draw=gray] (V\i\j\k) circle (\s);
}
}
\foreach \k in {1,...,\i} {
\foreach \j in {1,...,\k} {
\pgfmathsetmacro\t{\r*\j};
\pgfmathsetmacro\v{\r*\i - \r*\k};
\coordinate (V\i\j\k) at ($(V\i)+\t*(w1) + \v*(w3)$);
\draw[ball color=gray!10,draw=gray] (V\i\j\k) circle (\s);
}
}
}
\end{scope}
\begin{scope} %4D
\foreach \i/\c in {1/1,2/5} {
 \coordinate (V\i) at ($4*(u1)+\i*(u2) $);
\node at ($(V\i)+(-0.2,0)$) {\c};
}
\draw[ball color=gray!10,draw=gray] ($(V1)+(\s,0)$) circle (\s);
\node at ($(V2)+(\s,0)$) {$*$};
\end{scope}
\begin{scope} %5D
\foreach \i/\c in {1/1} {
 \coordinate (V\i) at ($5*(u1)+\i*(u2) $);
\node at ($(V\i)+(-0.2,0)$) {\c};
}
\draw[ball color=gray!10,draw=gray] ($(V1)+(\s,0)$) circle (\s);
\end{scope}
}
}


\def\figNTNKMGO{
\tikzfig{tri-square}{\guid{NTNKMGO} 
The pyramid on a square base is the same lattice packing
as the pyramid on a triangular base.  The only differences are the
orientation of the lattice in space and the exposed facets of the lattice.
Their orientation and exposed facets are matched as shown.
}
{
[scale=2.3]
\autoNTNKMGO
\pgfmathsetmacro\s{0.5};
\pgfmathsetmacro\d{sqrt(0.5)};
\coordinate (u) at ($0.5*(w1)+0.5*(w3)+\d*(w2)$);
%\foreach \h/\i/\j in {3/0/1,3/1/1,4/0/0,3/0/0,3/1/0,2/0/0,2/1/0,2/2/0} {
% \coordinate (V\i\j\h) at ($\i*(w1)+\j*(w3)+\h*(u)$);
% \draw[ball color=gray!10,draw=gray] (V\i\j\h) circle (\s);
%}
\draw[gray] (w0)--(w3)--(u)--cycle;
\draw (w0)--(w1)--(u)--cycle;
\draw (w1)-- ++(w3) --(u)--cycle;
\draw[gray] (w3)-- ++(w1);
\coordinate (ss) at (w1);
\begin{scope}[shift={(0.18,0.46)}]
\autoNTNKMGO
\coordinate (u) at ($0.5*(w1)+0.5*(w3)+\d*(w2)$);
\draw (w1)--(u)-- ++(w1)--cycle;
\draw (w1)-- ++(w3)-- ($(w1)+(u)$)--cycle;
\draw[gray] (u)--  ($(w1)+(w3)$);
\end{scope}
\begin{scope}[shift={(ss)}]
\autoNTNKMGO
\coordinate (u) at ($0.5*(w1)+0.5*(w3)+\d*(w2)$);
\draw[gray] (w0)--(w3)--(u)--cycle;
\draw (w0)--(w1)--(u)--cycle;
\draw (w1)-- ++(w3) --(u)--cycle;
\draw[gray] (w3)-- ++(w1);
\end{scope}
}
}

\def\drawball#1#2{\shade[ball color=gray!20,draw=gray](#1,#2) circle (1);  }

\def\figPTFTWZM{
\tikzfig{musin}
{\guid{PTFTWZM}
  Newton's claim -- twelve is the maximum number of congruent balls
  that can be tangent to a given congruent ball-- was confirmed in the 1953.
  Musin and Tarasov only recently proved that the arrangement shown here is the
  unique  arrangement of thirteen congruent
  balls that shrinks the thirteen by the least possible amount to
  permit tangency~\cite{Musin-Tarasov}.  Each node of the graph represents one
  of the thirteen balls and each edge represents a pair of touching
  balls.  The node at the center of the graph corresponds to the
  uppermost ball in the second frame. The other twelve balls are perturbations
  of the FCC tangent arrangement.}
{
\begin{scope}[scale=0.004]
%Set the coordinates of the points.
%\tikzstyle{every node}=[draw,shape=circle];
\path (45:400) coordinate (P0) ;
\path (135:400)  coordinate (P1) ;
\path (225:400) coordinate (P2) ;
\path (315:400) coordinate (P3) ;
\path (0:200) coordinate (P4) ;
\path (90:200) coordinate (P5) ;
\path (180:200) coordinate (P6) ;
\path (270:200) coordinate (P7) ;
\path(45:150) coordinate (P8) ;
\path (135:150) coordinate (P9) ;
\path (225:150) coordinate (P10) ;
\path (315:150) coordinate (P11) ; 
\path (0,0) coordinate (P12) ;
\foreach \i in {0,...,12}
{
  \fill (P\i) circle (15);
}
%Draw edges.
\draw
  (P12) -- (P8)
  (P12) -- (P9)
  (P12) -- (P10)
  (P12) -- (P11)
  (P8) -- (P4)
  (P4) -- (P11)
  (P11) -- (P7)
  (P7) -- (P10)
  (P10) -- (P6)
  (P6) -- (P9)
  (P9) -- (P5)
  (P5) -- (P8)
%
  (P0) -- (P1)
  (P1) -- (P2)
  (P2) -- (P3)
  (P3) -- (P0)
%
  (P0) -- (P5)
  (P5) -- (P1)
  (P1) -- (P6)
  (P6) -- (P2)
  (P2) -- (P7)
  (P7) -- (P3)
  (P3) -- (P4)
  (P4) -- (P0);
\end{scope}
%
\begin{scope}[scale=0.5,xshift=8cm]
\drawball{-0.504725}{0.79793}
\drawball{0.987379}{-0.530059}
\drawball{-0.406371}{-1.76776}
\drawball{-1.8337}{-0.370827}
\drawball{1.68242}{1.01951}
\drawball{0.}{2.0538}
\drawball{1.35457}{-1.58937}
\drawball{0.}{0.}
\drawball{-1.68242}{1.20943}
\drawball{-1.35457}{-1.43645}
\drawball{1.8337}{0.000711695}
\drawball{0.504725}{1.431}
\drawball{0.406371}{-1.25805}
\drawball{-0.987379}{0.159943}
\end{scope}
}}


\def\figAZGXQWC{
\tikzfig{tet-oct-ratio}{\guid{AZGXQWC} 
A regular tetrahedron whose edge is two units
can be partitioned into four  unit-edge tetrahedra and one
unit-edge octahedron at its center.  Similarly, a regular octahedron whose
edge is two units can be partitioned into six unit-edge octahedra
and eight unit-edge tetrahedra.
}
{
[scale=2.5]
\autoAZGXQWC
\draw[thick,fill=gray!5,line join=bevel] (w0)--(w2)--(w3)--cycle;
\draw[thick,fill=gray!15,line join=bevel] (w1)--(w2)--(w3)--cycle;
\draw[thick,fill=gray!35,line join=bevel] (w2)--(w0)--(w1)--cycle;
\draw[black!70,line join=bevel] (w01)--(w12)--(w13)--cycle; %1
\draw[black!70,line join=bevel] (w01)--(w02)--(w03)--cycle; %0
\draw[black!70,line join=bevel] (w12)--(w02)--(w23)--cycle; %2
\draw[black!70,line join=bevel] (w03)--(w13) -- (w23)--cycle; %3
\begin{scope}[shift={(2,0.06)}]  %3
\autoAZGXQWC
\draw[thick,fill=gray!5,line join=bevel] (w03)--(w23)--(w3)--cycle;
\draw[thick,fill=gray!15,line join=bevel] (w13)--(w23)--(w3)--cycle;
\draw[gray,fill=gray!10,line join=bevel] (w03)--(w13) -- (w23)--cycle; %3
\end{scope}
\begin{scope}[shift={(2,-0.06)}]  %1
\autoAZGXQWC
\draw[thick,fill=gray!15,line join=bevel] (w1)--(w12)--(w13)--cycle;
\draw[thick,fill=gray!35,line join=bevel] (w12)--(w01)--(w1)--cycle;
\draw[gray,fill=gray!10,line join=bevel] (w01)--(w12)--(w13)--cycle; %1
\end{scope}
\begin{scope}[shift={(1.95,-0.05)}]  %0
\autoAZGXQWC
\draw[thick,fill=gray!5,line join=bevel] (w0)--(w02)--(w03)--cycle;
\draw[thick,fill=gray!35,line join=bevel] (w02)--(w0)--(w01)--cycle;
\draw[gray,fill=gray!10,line join=bevel] (w01)--(w02)--(w03)--cycle; %0
\end{scope}
\begin{scope}[shift={(2,0.0)}]  %oct
\autoAZGXQWC
\draw[gray,fill=gray!40,line join=bevel] (w12)--(w23)--(w13)--cycle; %1
\draw[gray,fill=gray!40,line join=bevel] (w02)--(w23)--(w03)--cycle; %0
\draw[gray,fill=gray!30,line join=bevel] (w01)--(w12) -- (w02)--cycle; %3
\draw[gray,fill=gray!20,line join=bevel] (w12)--(w02)--(w23)--cycle; %2
\end{scope}
\begin{scope}[shift={(2.3,0.05)}]  %2
\autoAZGXQWC
\draw[thick,fill=gray!5,line join=bevel] (w02)--(w2)--(w23)--cycle;
\draw[thick,fill=gray!15,line join=bevel] (w12)--(w2)--(w23)--cycle;
\draw[thick,fill=gray!35,line join=bevel] (w2)--(w02)--(w12)--cycle;
\draw[gray] (w12)--(w02)--(w23)--cycle; %2
\end{scope}
}
}

\def\figCCQCYWU{
\tikzfig{hex-layers}{\guid{CCQCYWU} 
If the centers in one hexagonal layer of a close packing are placed
at the sites marked $A$, then the hexagonal layer above it will
occupy all the sites marked $B$, or all of the sites marked $C$.
In general, each hexagon layer of a close packing is determined by
its label $A$, $B$, or $C$, which must always differ from the label of the
layer below.
The HCP packing is $\ldots ABABAB\ldots$.  The FCC packing is
$\ldots ABCABCABC\ldots$.  There are infinitely many other close packings,
consisting of hexagonal layers, corresponding to sequences of $A$, $B$,
$C$.
}
{
[scale=0.9]
\coordinate (v1) at (2,0);
\coordinate (v2) at (60:2);
\coordinate (A) at (0,0);
\coordinate (B) at ($0.333*(v1)+0.333*(v2)$);
\coordinate (C) at ($0.666*(v2)-0.333*(v1)$);
\clip (-1,0)--(9,0)--(9,4)--(-1,4)--cycle;
\foreach \i in { -2,-1,0,1,2,3,4 } {
  \foreach \j in {0,1,2,3,4 } {
    \coordinate (A\i\j) at ($(A)+\i*(v1) + \j*(v2)$);
    \node[shape=circle,draw,fill=gray!5] at (A\i\j) {A};
}
}
\foreach \i in { -2,-1,0,1,2,3,4 } {
  \foreach \j in {0,1,2,3,4 } {
    \coordinate (B\i\j) at ($(B)+\i*(v1) + \j*(v2)$);
    \node[shape=circle,draw,fill=gray!15] at (B\i\j) {B};
}
}
\foreach \i in { -2,-1,0,1,2,3,4 } {
  \foreach \j in {0,1,2,3,4 } {
    \coordinate (C\i\j) at ($(C)+\i*(v1) + \j*(v2)$);
    \node[shape=circle,draw,fill=gray!25] at (C\i\j) {C};
}
}
}}

\def\figTCFVGTS{
\tikzfig{face-centered-cubic}{\guid{TCFVGTS} 
The intersection of the FCC packing with a cube of side $\sqrt8$.
The name {\it face-centered cubic} comes from this depiction.  The cube
has volume $\sqrt8^3$ and
contains a total of four balls 
(eight eighths from the corners and six halves from
the facets), 
giving density $4(4\pi/3)/\nsqrt8^3 = \pi/\nsqrt{18}$.
}
{
[scale=0.95]
\autoTCFVGTS
\draw[thick,fill=gray!70] (w0)--(w1)--(w12)--(w2)--cycle;
\draw[thick,fill=gray!90] (w1)--(w13)--(w123)--(w12)--cycle;
\draw[thick,fill=gray!80] (w2)--(w12)--(w123)--(w23)--cycle;
\draw[fill=gray!10] (w0)-- plot[smooth] coordinates {\tcfa} --cycle;
\draw[fill=gray!10] (w1)--plot[smooth] coordinates {\tcfb} --cycle;
\draw[fill=gray!10] (w12)--plot[smooth] coordinates {\tcfc} --cycle;
\draw[fill=gray!10] (w2)--plot[smooth] coordinates {\tcfd} --cycle;
\draw[fill=gray!10] plot[smooth] coordinates {\tcfe} 
  --plot[smooth] coordinates {\tcff}
  --plot[smooth] coordinates {\tcfg}
  --plot[smooth] coordinates {\tcfh}
   --cycle;
\draw[fill=gray!10] (w1)--plot[smooth] coordinates {\tcfi} --cycle;
\draw[fill=gray!10] (w13)--plot[smooth] coordinates {\tcfj} --cycle;
\draw[fill=gray!10] (w123)--plot[smooth] coordinates {\tcfk} --cycle;
\draw[fill=gray!10] (w12)--plot[smooth] coordinates {\tcfl} --cycle;
\draw[fill=gray!10] plot[smooth] coordinates {\tcfm} 
  --plot[smooth] coordinates {\tcfn}
  --plot[smooth] coordinates {\tcfo}
  --plot[smooth] coordinates {\tcfp}
   --cycle;
\draw[fill=gray!10] (w2)--plot[smooth] coordinates {\tcfq} --cycle;
\draw[fill=gray!10] (w12)--plot[smooth] coordinates {\tcfr} --cycle;
\draw[fill=gray!10] (w123)--plot[smooth] coordinates {\tcfs} --cycle;
\draw[fill=gray!10] (w23)--plot[smooth] coordinates {\tcft} --cycle;
\draw[fill=gray!10] plot[smooth] coordinates {\tcfu} 
  --plot[smooth] coordinates {\tcfv}
  --plot[smooth] coordinates {\tcfw}
  --plot[smooth] coordinates {\tcfx}
   --cycle;
\draw[thick] (w0)--(w1)--(w12)--(w2)--cycle;
\draw[thick] (w1)--(w13)--(w123)--(w12)--cycle;
\draw[thick] (w2)--(w12)--(w123)--(w23)--cycle;
\foreach\i in {w0,w1,w2,w23,w12,w13,w123,wfront,wtop,wright} {
  \smalldot{\i};
}
}
}

\def\figAFRJFRK{
\tikzfig{rhombus}{\guid{AFRJFRK} 
At one stage of the proof that the FCC lattice
is the optimal packing among lattices, we show that an optimal lattice
consists of parallel sheets of a rhombic tiling, and that a ball from one
sheet rests on three balls centered at $\v_1,\v_2,\v_3$ in the layer below.
}
{
[scale=1.0]
\coordinate (U) at (0,0);
\coordinate (V) at (1,0);
\coordinate (W) at (70:1);
\foreach \i in {0,1,2,3} {\draw[thick,black] ($\i*(W)$)-- ++ (8,0); }
\foreach \i in {0,1,2,3,4,5,6,7,8} {\draw ($\i*(V)$)-- ++ ($3*(W)$); }
\foreach \i in {0,1,2,3} {
  \foreach \j in {0,1,2,3,4,5,6,7,8} {
    \coordinate (U\i\j) at ($\i*(W) + \j*(V)$);
    \smalldot {U\i\j};
}
}
\node[anchor=south east] at ($3*(V)+(W)$) {$\v_1$};
\node[anchor=south east] at ($4*(V)+(W)$) {$\v_2$};
\node[anchor=south east] at ($3*(V)+2*(W)$) {$\v_3$};
}
}

\def\figOCULYIA{
\tikzfig{2D-hex}{\guid{OCULYIA} The optimal packing in two dimensions.
}
{
[scale=1.0]
\coordinate (U) at (0,0);
\coordinate (V) at (1,0);
\coordinate (W) at (60:1);
\foreach \i in {0,1,2,3} {\draw[thick,black] ($\i*(W)$)-- ++ (8,0); }
\foreach \i in {0,1,2,3,4,5,6,7,8} {\draw ($\i*(V)$)-- ++ ($3*(W)$); }
\foreach \i in {0,1,2,3} {
  \foreach \j in {0,1,2,3,4,5,6,7,8} {
    \coordinate (U\i\j) at ($\i*(W) + \j*(V)$);
    \smalldot {U\i\j};
    \draw (U\i\j) circle (0.5);
}
}
\begin{scope}
\clip (0,0) -- (U30)--(U38)--(U08)--cycle;
\foreach \i in {0,1,2,3,4,5,6,7,8,9,10,11}
  {\draw ($\i*(V)$) -- ++ ($3*(W)- 3*(V)$); }
\end{scope}
}
}

\def\figSENQMWT{
\tikzfig{2D-proof}{\guid{SENQMWT} 
This partition of the plane gives a proof of Thue's theorem.
The disks have radius $2/\nsqrt{3}$.
Each shaded sector and each triangle in an arbitrary packing
has density at most $\pi/\nsqrt{12}$.
}
{
[scale=2.7]
\autoSENQMWT
}
}

\def\figEVIAIQP{
\tikzfig{voronoi}{\guid{EVIAIQP} 
Voronoi cells of a two-dimensional sphere packing.
}
{
[scale=2.7]
\autoEVIAIQP
}
}

\def\figEVIAIQPx{
\tikzfig{voronoix}{\guid{EVIAIQP} 
Voronoi cells of a two-dimensional sphere packing.
}
{
[scale=2.7]
\autoEVIAIQP
}
}


\def\figANNTKZP{
\tikzfig{delaunay}{\guid{ANNTKZP} 
Delaunay triangles of a two-dimensional sphere packing.
}
{
[scale=2.7]
\autoANNTKZP
}
}

\def\figORQISJR{ \tikzfig{rogers-intro}{\guid{ORQISJR} Rogers
    simplices of a two-dimensional sphere packing.  Heavy edges are
    facets of Voronoi cells.  The Rogers simplices that are not right
    triangles are shaded.  } 
{ 
[scale=2.7] 
\autoORQISJR
} }


\def\figODGBUWK{ \tikzfig{marchal-intro}{\guid{ODGBUWK} 
    Marchal cells of a two-dimensional sphere packing. } 
{ 
[scale=2.7] 
\autoBWEYURN
} }


\def\figFIFJALK{ \tikzfig{ferguson-hales}{\guid{FIFJALK}  
The hybrid partition of space that was used to prove the Kepler conjecture
in 1998.
 For simplicity, we illustrate the two-dimensional analogue of that
partition.
}
{ 
[scale=2.7] 
\autoFIFJALK
} }

\def\figCCKQLLH{ \tikzfig{delaunay-proof}{\guid{CCKQLLH}  
The horizontal segment is a fixed edge of length $2$ 
of a Delaunay triangle.  The shaded region constrains the position
of the third vertex of the Delaunay triangle.  The white dots indicate
the three positions of the third vertex that minimize the area of the
Delaunay triangle.
}
{ 
[scale=0.5] 
\coordinate (U) at (0,0);
\coordinate (V) at (2,0);
\coordinate (W) at (60:2);
\coordinate (W1) at (120:2);
\coordinate (W2) at ($2*(V) + (W1)$);
\draw (U) circle (2);
\draw (V) circle (2);
\draw (W) circle (2);
\draw[thick] (U)--(V);
\smalldot{U};
\smalldot{V};
\begin{scope}
%\clip (W) circle (2);
\clip (V) arc (0:180:2.0cm) -- ++ (0,4) -- ++ (8,0) --cycle;
\clip ($2 *(V)$)  arc (0:180:2.0cm) -- ++(-2,0) -- ++ (0,4) -- ++ (8,0) --cycle;
\draw[fill=gray!30] (W) circle (2);
\end{scope}
\whitedot{W};
\whitedot{W1};
\whitedot{W2};
\node[anchor=north east] at (U) {$\v_1$};
\node[anchor=north west] at (V) {$\v_2$};
} }




%%%%%%%%%%%%%%%%%%%%%%%%%%%%%%%%%%%%%%%%%%%%%%%%%%%%%%%%%%%%%%%%%%%%%%%%%%%%%%%%
%% TRIG CHAPTER
%%%%%%%%%%%%%%%%%%%%%%%%%%%%%%%%%%%%%%%%%%%%%%%%%%%%%%%%%%%%%%%%%%%%%%%%%%%%%%%%

\def\figYOXQFUB{
\tikzfig{atn-polar}{\guid{YOXQFUB} 
The function $\atn$ gives the polar angle $\theta$ of $(x,y)$.}
{
[scale=0.20]
\draw[gray,->,thin] (-4,0) -- (14,0);
\draw[gray,->,thin] (0,-2) -- (0,5);
\draw (0,0)  --(12,0) --  (12,5) --  cycle;
\draw[thin] (11,0) -- (11,1) -- (12.0,1);
\path (6,-1.5) node {$x$};
\draw[thin] (4,0) arc (0:22.62:4);
\path (14,2.5) node {$y$};
\path (5.5,1.35) node {$\theta$};
}
}

\def\figODPCVGH{
\tikzfig{trig}{\guid{ODPCVGH} Trigonometric and inverse trigonometric
functions.}
%
%arctangent function on the domain \leftopen -4,4\rightopen\ 
%and the $\arccos$ function on $\leftclosed-1,1\rightclosed$.}}
{
[scale=0.5]
\draw[gray] (-2*1.57,0) sin (-1.57,-1) cos (0,0) sin (1.57,1) cos (3.14,0) sin (3*1.57,-1);
\draw   (-2*1.57,-1) cos (-1.57,0) sin  (0,1) cos (1.57,0) sin (2*1.57,-1) cos (3*1.57,0); 
\draw[help lines,<->] (-3.3,0) -- (3*1.57 + 0.2,0);
\draw[help lines,<->] (0,-1) -- (0,2.0);
%omit tan
%\draw plot[smooth] file {tikz/tan.table};
%\node at (-0.5,-1.8) {$\tan$};
\node at (3.0,0.9) {$\sin$};
\node at (2,-1.3) {$\cos$};
% GG need axis labels and ticks, base points of labels should be precisely aligned.
\begin{scope}[xshift=10cm]
\draw plot[smooth] file {tikz/arctan.table};
\node at (2.5,1.8)  {$\arctan$};
%omit arccos.
%\draw plot[smooth] file {tikz/arccos.table};
%\node at (-1.4,1.8) {$\arccos$};
\draw[help lines,<->] (0,-1.6) -- (0,2.0);
\draw[help lines,<->] (-4,0) -- (4,0);
\end{scope}
}}

\def\figRUESSGQ{
\tikzfig{cosadd}{\guid{RUESSGQ} A geometric derivation of the cosine addition law comes
by writing the rectangle area $(s_x+c_y)(s_y+c_x)$ as the sum of the shaded rhombus
$c_{x+y}$ and unshaded pieces
$2 s_x s_y + c_y s_y + c_x s_x$, where $s_x =\sin x$, $c_y = \cos y$, etc. The
derivation in the text is less intuitive, but avoids measure.
}
{
[scale=2.8]
\coordinate (R11) at (1.50881, 1.26604);
\coordinate (R11x) at (1.50881, 0);
\coordinate (R11y) at (0, 1.26604);
\coordinate (Rh1) at (0.866025, 0.5);
\coordinate (Rh2) at (0.642788, 0.766044);
\coordinate (Rh1x) at (0.866025, 0.0);
\coordinate (Rh1y) at (1.50881,0.5);
\coordinate (Rh2x) at (0.642788,1.26604);
\coordinate (Rh2y) at (0,0.766044);
\draw (0,0) rectangle (R11);
\draw[fill=black!40] (0,0) --(Rh1)--(R11)--(Rh2)--cycle;
%\draw[fill=black!40] (Rh1x)--(R11x)--(Rh1y)--(Rh1)--cycle;
%\draw[fill=black!40] (Rh2)--(Rh2x)--(R11y)--(Rh2y)--cycle;
\draw (Rh1x)--(Rh1)--(Rh1y);
\draw (Rh2x)--(Rh2)--(Rh2y);
\node at (0.26,0.06) {$y$};
\node at (0.07,0.2) {$x$};
\node at (1.43,1.1)  {$x$};
\node at (1.3,1.2) {$y$};
\node at (-0.1,0.37) {$c_x$};
\node at (-0.1,1.0) {$s_y$};
\node at (0.32   ,1.34) {$s_x$};
\node at (1.0,1.34) {$c_y$};
}}


\def\figITGCYIF{
\tikzfig{affset}{\guid{ITGCYIF}  When $\card(V)+\card(V')-1\in\{1,2\}$,
the set
$\op{aff}_+(V,V')$ 
 is a segment, ray, or line; simplex, blade, half-plane, or plane.}
{
[scale=3.0]
\begin{scope}[shift={(0,0)}]
\coordinate (P0) at (0,0);
\coordinate (P1) at (1,0);
\draw (P0)--(P1);
\smalldot{P0};\smalldot{P1};
\end{scope}
\begin{scope}[shift={(1.2,0)}]
\coordinate (P0) at (0,0);
\coordinate (P1) at (1,0);
\draw[->] (P0)--(P1);
\smalldot{P0};
\end{scope}
\begin{scope}[shift={(2.4,0)}]
\coordinate (P0) at (0,0);
\coordinate (P1) at (1,0);
\draw[<->] (P0)--(P1);
\end{scope}
\begin{scope}[shift={(0.1,-0.6)},scale=0.3]
\coordinate (P0) at (0,0);
\coordinate (P1) at (1,0);
\coordinate (P2) at (60:1);
\draw[fill=gray!40] (P0)--(P1)--(P2)--cycle;
\smalldot{P0};\smalldot{P1};\smalldot{P2};
\end{scope}
\begin{scope}[shift={(0.8,-0.6)},scale=0.3]
\coordinate (P0) at (0,0);
\coordinate (P1) at (1,0);
\coordinate (P2) at (60:1);
\begin{scope}
\clip (P0) -- (0:1.4) arc (0:60:1.4)--cycle;
\shade[inner color=gray!50,shading=radial] (P0) circle (1.4); 
\end{scope}
\smalldot{P0};\smalldot{P1};\smalldot{P2};
\draw[->] (P0) -- (P1) -- +(0.3,0);
\draw[->] (P0) -- (P2) -- +(60:0.3);
\end{scope}
\begin{scope}[shift={(1.7,-0.6)},scale=0.3]
\coordinate (P0) at (0,0);
\coordinate (P1) at (1,0);
\coordinate (P2) at (60:1);
\begin{scope}
\clip (P0) -- (0:1.75) arc (0:180:1.25)--cycle;
\shade[inner color=gray!50,outer color=white,shading=radial] (0.5,0) circle (1.5);
\end{scope}
\smalldot{P0};\smalldot{P1};\smalldot{P2};
\draw[<->] (-1,0,0) -- (2,0);
\end{scope}
\begin{scope}[shift={(2.8,-0.6)},scale=0.3]
\coordinate (P0) at (0,0);
\coordinate (P1) at (1,0);
\coordinate (P2) at (60:1);
\shade[inner color=gray!60,outer color=white,shading=radial] (0.5,0) circle (1.5);
\smalldot{P0};\smalldot{P1};\smalldot{P2};
\end{scope}
}
}


\def\figLIKEURF{
\tikzfig{arcV}{\guid{LIKEURF} The angle $\theta =\op{arc}_V(\u,\{\v,\w\})=\op{arc}(a,b,c)$.
}
{
[scale=1.8]
\coordinate (P0) at (0,0);
\coordinate (P1) at (1,0);
\coordinate (P2) at (60:1);
\begin{scope}
\clip (P0)--(0:1.4) arc (0:60:1.4)--cycle;
\shade[inner color=gray!60,outer color=white,shading=radial] (P0) circle (1.4);
\end{scope}
%\draw[fill=gray!20,draw=gray!20] (P0) -- (0:1.3) arc (0:60:1.3)--cycle;
\draw[fill=gray!20,draw=gray!20] (P0) -- (P1)--(P2)--cycle;
\smalldot{P0} node[anchor=north] {$\u$} node[anchor=south west] {~~$\theta$};
\smalldot{P1} node[anchor=north] {$\v$};
\smalldot{P2} node[anchor=south east] {$\w$};
\draw (P1)-- node[anchor= north east,fill=gray!20] {$c$} (P2) ;
\draw (P0)-- node[anchor= north] {$b$} (P1) ;
\draw (P0)-- node[anchor= south east] {$a$} (P2) ;
\draw[->] (P0) -- (P1) -- +(0.3,0);
\draw[->] (P0) -- (P2) -- +(60:0.3);
}}

\def\figGJRSLPT{ 
  \tikzfig{dih}{\guid{GJRSLPT} The dihedral angle
    $\theta=\dih_V(\{\v_0,\v_1\},\{\v_2,\v_3\})$  is
    calculated by projection of $\v_2$ and $\v_3$
    to a plane $P$ with normal $\v_1-\v_0$.
    The azimuth angle  (Definition~\ref{def:azim}) is closely
    related to the dihedral angle, but depends on the ordering
    $(\v_0,\v_1,\v_2,\v_3)$ and takes values between $0$ and $2\pi$,
    unlike the dihedral angle, which takes values between $0$ and
    $\pi$.} 
{ 
[scale=1.2] 
\pgfmathsetmacro\x{-sqrt(0.5)*0.6}
\pgfmathsetmacro\y{-sqrt(0.5)*2}
\coordinate (v0) at (4,0);
\coordinate (v1) at (2,0);
\coordinate (v2) at (3,1);
\coordinate (v3) at ($(1,0)+(\x,\y)$);
\coordinate (v5) at (5,0);
\coordinate (v2x) at (5,1);
\coordinate (v3x) at ($(5,0)+(\x,\y)$);
\draw[gray] (v2)--(v2x);
\draw[fill=gray!10] ($(3,0)+(0.3,0)$) arc[start angle=0,end angle=360,x radius =0.3,y radius = 1.0];
\draw[fill=gray!10] ($(1,0)+(0.6,0)$) arc[start angle=0,end angle=360,x radius =0.6,y radius = 2.0];
\draw[fill=gray!20] ($(5,0)+(0.6,0)$) arc[start angle=0,end angle=360,x radius =0.6,y radius = 2.0];
%\draw ($(5,0)+(0.3,0)$) arc[start angle=0,end angle=360,x radius =0.3,y radius = 1.0];
%\draw[thick] (5,0)--(6,0);
\draw[thick] (1,0)--($(3,0)+(-0.3,0)$) (3,0)--($(5,0)+(-0.6,0)$);
\draw[thick] (3,0)--(v2);
\draw[thick] (1,0)--(v3);
\draw[thick] (5,0)--(v2x);
\draw[thick] (5,0)--(v3x);
\draw[gray] (v3)--(v3x);
\foreach \i in {v0,v1,v2,v3} {
\smalldot{\i}; }
\node[anchor=south] at (v0) {$\v_0$};
\node[anchor=south] at (v1) {$\v_1$};
\node[anchor=south] at (v2) {$\v_2$};
\node[anchor=east] at (v3) {$\v_3$};
\node[anchor=south] at (5,2) {$P$};
\node[anchor=east] at ($(5,0)+(-4pt,0)$) {$\theta$};
\graydot{v2x};
\graydot{v3x};
\coordinate (vr) at (5,0);
\graydot {vr};
\coordinate (vl) at (1,0);
\graydot {vl};
\coordinate (vm) at (3,0);
\graydot {vm};
}
}

\def\figGJRSLPTdeprecated{ 
  \tikzfig{dih}{\guid{GJRSLPT} The dihedral angle
    $\theta=\dih_V(\{\v_0,\v_1\},\{\v_2,\v_3\})$ between two planes is
    calculated by projection to a plane $P$ with normal $\v_1-\v_0$.
    The azimuth angle $\theta$ (Definition~\ref{def:azim}) is closely
    related to the dihedral angle, but depends on the ordering
    $(\v_0,\v_1,\v_2,\v_3)$ and takes values between $0$ and $2\pi$,
    unlike the dihedral angle, which takes values between $0$ and
    $\pi$.} 
{ 
[scale=2] 
\tikzset{sample plane/.estyle={cm={0,1,-0.5,0.5,(0,0)}}}
\begin{scope}
[sample plane,z={(-1cm,2cm)}]
\coordinate (P3) at (0,0);
\coordinate (v3) at (0,0,2) ;
\coordinate (P2) at (1.0cm,0.2cm);
\coordinate (v2) at (1,0.2,1.5);
\coordinate (P01) at (0.5,0.7);
\coordinate (v0) at (0.5,0.7,1);
\coordinate (v1)  at (0.5,0.7,2.3) ;
\node[anchor=east] at (v1) {$\v_1$};
\node[anchor=north] at (v3) {$\v_2$};
\node[anchor=south] at (v2) {$\v_3$};
\node[anchor=west] at (P01) {$\theta$};
\draw[thin,draw=white,fill=gray!30] (v1)-- (v3)-- (P3) -- (P01) -- cycle;
\draw[thin,draw=white,fill=gray!80] (v1)-- (v2)-- (P2) -- (P01) -- cycle;
\begin{scope}
\clip (P01) -- (P2) -- (v2) -- (v1) -- (v3) -- (P3) -- cycle;
\foreach \x in {0.1,0.2,...,1.0}
  {
  \draw[gray!0.2] ($(P01)!\x!(v1)$) -- ++ ($(P2)-(P01)$);
  \draw[gray!0.2] ($(P01)!\x!(v1)$) -- ++ ($(P3)-(P01)$);
  \draw[gray!0.2] ($(P01)!\x!(P2)$) -- ++ ($(v1) - (P01)$); 
  \draw[gray!0.2] ($(P01)!\x!(P3)$) -- ++ ($(v1) - (P01)$); 
  }
\end{scope}
\draw[thick] (P3) -- (P2) -- (P01) -- cycle;
\draw[thick] (P2)--(v2) -- (v1) -- (v3) -- (P3);
\draw[thick] (P01) -- (v1);
\node[anchor=north] at (v0) {$\v_0$};
\foreach \x in {{v0},{v1},{v2},{v3},{P01},{P2},{P3}} \smalldot{\x};
\draw[->,thick] (P01) -- ($(P01)!0.5!(v0)$);
\end{scope}
}}

\def\setellipseplane#1#2#3#4#5{
 % calculate great circle through p,q lying on sphere of radius R.
 % project to ellipse in xy-plane.
 % SAMPLE CALL.
 % \setellipseplane{\R}{0.6}{-0.6}{0.2}{0.8}
 % \clip (0,0) -- ($(O)!3!(A)$) -- ($(O)!3!(B)$) -- cycle;
 % \draw[ellipse plane] (0,0) circle (1);
 \pgfmathsetmacro\r{#1} % radius of sphere, centered at origin.
 \pgfmathsetmacro\px{#2} % p = point on sphere of radius R.
 \pgfmathsetmacro\py{#3}
 \pgfmathsetmacro\pz{sqrt(\r*\r - (\px*\px + \py*\py))}
 \pgfmathsetmacro\qx{#4} % q = second point on sphere of radius R.
 \pgfmathsetmacro\qy{#5}
 \pgfmathsetmacro\qz{sqrt(\r*\r - (\qx*\qx + \qy*\qy))}
 \pgfmathsetmacro\rx{\py*\qz -\pz*\qy} % r = p x q.
 \pgfmathsetmacro\ry{\pz*\qx -\px*\qz}
 \pgfmathsetmacro\rz{\px*\qy -\py*\qx}
 \pgfmathsetmacro\nr{sqrt(\rx*\rx+\ry*\ry+\rz*\rz)} % n = unit vector direction p x q.
 \pgfmathsetmacro\nx{\rx/\nr}
 \pgfmathsetmacro\ny{\ry/\nr}
 \pgfmathsetmacro\nz{\rz/\nr}
 \pgfmathsetmacro\mx{\r*\ny/veclen(\nx,\ny)} % m= vector orth. to n in plane z=0.
 \pgfmathsetmacro\my{-\r*\nx/veclen(\nx,\ny)} % m= major axis of ellipse.
 \pgfmathsetmacro\mz{0}
 \pgfmathsetmacro\vx{\my*\nz-\mz*\ny} % v = m x n= projects to minor axis of ellipse.
 \pgfmathsetmacro\vy{\mz*\nx-\mx*\nz}
 \pgfmathsetmacro\ellrotate{atan2(\vx,\vy)}
 \pgfmathsetmacro\ellx{veclen(\vx,\vy)}
 \pgfmathsetmacro\elly{veclen(\mx,\my)}
 \tikzset{ellipse plane/.estyle={cm={\mx,\my,\vx,\vy,(0,0)}}}
}

\def\figNUPFYMD{
\tikzfig{sloc}{\guid{NUPFYMD} The spherical law of cosines
gives the angle $\gamma$ of a spherical triangle in terms of
its edge lengths $a$, $b$, and $c$.  The polar form of the spherical
law of cosines gives the side $c$ in terms of the angles $\alpha$, $\beta$, and $\gamma$.}
{
[scale=1]
\def\R{1.5}
\draw[ball color=gray!10,draw=gray] (0,0) circle (\R);
\def\r{veclen(1,2)}
\coordinate (O) at (0,0);
\coordinate (A) at (0.6,-0.6);
\coordinate (B) at (0.2,0.8);
\coordinate (C) at (-0.6,-0.4);
\node[anchor=south west] at ($(C) + (1.1ex,-0.3ex)$) {$\gamma$};
\node[anchor=south east] at ($(A) + (-0.5ex,0.2ex)$) {$\alpha$};
\node[anchor=north] at ($(B) + (-0.3ex,-0.6ex)$) {$\beta$};
\path (A) -- node[anchor=north] {$b$} (C);
\path (B) -- node[anchor=east] {$a~$} (C);
\path (B) -- node[anchor=west] {$~c$} (A);
\smalldot{A};
\smalldot{B};
\smalldot{C};
\begin{scope}
\setellipseplane{\R}{0.6}{-0.6}{0.2}{0.8}
\clip (0,0) -- ($(O)!3!(A)$) -- ($(O)!3!(B)$) -- cycle;
\draw[ellipse plane] (0,0) circle (1);
\end{scope}
\begin{scope}
\setellipseplane{\R}{0.2}{0.8}{-0.6}{-0.4}
\clip (0,0) -- ($(O)!3!(B)$) -- ($(O)!3!(C)$) -- cycle;
\draw[ellipse plane] (0,0) circle (1);
\end{scope}
\begin{scope}
\setellipseplane{\R}{0.6}{-0.6}{-0.6}{-0.4}
\clip (0,0) -- ($(O)!3!(A)$) -- ($(O)!3!(C)$) -- cycle;
\draw[ellipse plane] (0,0) circle (1);
\end{scope}
}
}


\def\figROHSJRP{
\tikzfig{polarcycle}{\guid{ROHSJRP} The polar cycle is the cyclic permutation of a finite
set of nonzero points in the plane in a counterclockwise direction.
}
{
[scale=1]
\coordinate (O) at (0,0);
\node[anchor=north] at (O) {$\orz$};
\smalldot{O};
\pgfmathsetseed{1800};
\foreach \p in {0,30,...,330}
 {
 \pgfmathrandominteger{\a}{-15}{15};
 \pgfmathrandominteger{\b}{4}{10};
 \coordinate (V\p) at (\p+\a:0.2*\b);
 \smalldot{V\p};
 }
\coordinate (V360) at (V0);
\foreach \p in {0,30,...,330}
  \pgfmathsetmacro\q{\p+30}
  \draw[->,shorten >= 2.5pt] (V\p) -- (V\q);
}
}

\def\figHOUNZSY{
\tikzfig{azimuthcycle}{\guid{HOUNZSY} The azimuth cycle is a cyclic permutation of a finite
set $V$ of points in $\ring{R}^3$ that projects orthogonally to the polar cycle in the plane.
}
{
[scale=1]
\coordinate (O) at (0,0);
%\node[anchor=north] at (O) {$\orz$};
\smalldot{O};
\pgfmathsetseed{3500};  %1800
\foreach \t in {0,60,...,300}
 {
 \pgfmathrandominteger{\a}{-15}{15};
 \pgfmathrandominteger{\b}{4}{9.8};
 \coordinate (V\t) at (\t+\a:0.15*\b);
 \smalldot{V\t};
 \pgfmathrandominteger{\c}{18}{28};
 \path 
  let \p1 = (V\t) in 
   coordinate (W\t) at (\cmofpt*\x1,\cmofpt*\y1,0.2*\c);
 %
 \smalldot{W\t};
 \draw[gray] (V\t)--(W\t);
 }
\begin{scope}
\clip (V0) -- (V60) -- (V120) -- (V180) -- (V240) -- (V300) -- cycle;
\draw[help lines,step=0.2] (-2,-2) grid (2,2);
\end{scope}
\coordinate (V360) at (V0);
\coordinate (W360) at (W0);
\draw[help lines] (0,0,0) -- (0,0,0.2*28);
\foreach \t in {0,60,...,300}
  {
  \pgfmathsetmacro\s{\t+60}
  \draw[->,shorten >= 2.5pt] (V\t) -- (V\s);
  \draw[->,shorten >= 2.5pt,black,thick] (W\t) -- (W\s);
  }
\node[anchor=west] at (W300) {$V$};
}
}

%%%%%%%%%%%%%%%%%%%%%%%%%%%%%%%%%%%%%%%%%%%%%%%%%%%%%%%%%%%%%%%%%%%%%%%%%%%%%%%%
%% VOLUME CHAPTER
%%%%%%%%%%%%%%%%%%%%%%%%%%%%%%%%%%%%%%%%%%%%%%%%%%%%%%%%%%%%%%%%%%%%%%%%%%%%%%%%


\def\figWQBMWZO{
\tikzfig{tarskislice}{\guid{WQBMWZO} To first order approximation in $\delta$,
the surface area $2\pi y\, \ell$ of rotation of a
slice of width $\delta$ 
of a unit sphere equals the area $2\pi y\, (\delta\sec\alpha)=2\pi\,\delta$
of the surface area of the cone tangent to the sphere, which is independent
of $y$ and $\alpha$.  It follows that the surface
area of the part of a unit sphere between two parallel planes depends only on the separation
of the two planes.}
{
[scale=1]
\pgfmathsetmacro\r{2.5};
\pgfmathsetmacro\alphav{40};
\pgfmathsetmacro\beta{90-\alphav};
\pgfmathsetmacro\deltav{0.2*\r};
\coordinate (A) at ({\r/sin(\alphav)},0);
\coordinate (B) at (\beta:\r);
\coordinate (C) at ({\r*cos(\beta)},0);
\coordinate (D) at ($(C)+(\deltav,0)$);
\smalldot{0,0};
\draw (0,0) -- node[anchor=east] {$1$} (B) 
  -- node[anchor=east] {$y$} (C);
\node[anchor=north east] at ($(B) + (0ex,-2ex)$) {$\alpha$} ;
\path [name path=lineAB] (A)--(B);
\path [name path=lineDvert] (D)-- ++(0,2);
\draw [fill=black!20,name intersections={of=lineAB and lineDvert, by=E}] 
  (C) -- node[anchor=north] {$\delta$} (D) -- (E)  -- node[anchor=south west] {$\ell$} (B) -- cycle;
\draw (\r,0) arc (0:180:\r);
\draw (E)-- ($(E)!3!(B)$);
%
}
}


\def\figLWQUMHN{
\tikzfig{abcmpi}{\guid{WQUMHN} The area 
of a spherical triangle $T$ is calculated by inclusion-exclusion:  six
lunes with areas $2\alpha,2\alpha,2\beta,2\beta,2\gamma,2\gamma$ cover both $T$ and the congruent
 antipodal
triangle three times and the rest of the unit sphere once.  This gives the
equation 
$6\,\op{area}(T)+(4\pi-2\,\op{area}(T)) = 4\alpha+4\beta+4\gamma$, or
 $\op{area}(T) = \alpha+\beta+\gamma-\pi$.}
{
[scale=1]
\def\R{1.5}
\draw[ball color=gray!10,draw=gray] (0,0) circle (\R);
\def\r{veclen(1,2)}
\coordinate (O) at (0,0);
\coordinate (A) at (0.6,-0.6);
\coordinate (B) at (0.2,0.8);
\coordinate (C) at (-0.6,-0.4);
\node[anchor=south west] at ($(C) + (1.1ex,-0.3ex)$) {$\gamma$};
\node[anchor=south east] at ($(A) + (-0.5ex,0.2ex)$) {$\alpha$};
\node[anchor=north] at ($(B) + (-0.3ex,-0.6ex)$) {$\beta$};
\node[anchor=north east] at ($(C) + (-1.0ex,-0.3ex)$) {$\gamma$};
\node[anchor=north west] at ($(A) + (0.0ex,0.0ex)$) {$\alpha$};
\node[anchor=south] at ($(B) + (0.4ex,0.6ex)$) {$\beta$};
\foreach \i/\j in {I/30,II/80,III/140,IV/200,V/265,VI/315} \node at (\j:1.25*\R) {$\i$};
\smalldot{A};
\smalldot{B};
\smalldot{C};
\begin{scope}
\setellipseplane{\R}{0.6}{-0.6}{0.2}{0.8}
\draw (-\mx,-\my) arc [start angle=-90,end angle=90,x radius=\ellx,y radius=\elly,rotate=\ellrotate];
\end{scope}
\begin{scope}
\setellipseplane{\R}{0.2}{0.8}{-0.6}{-0.4}
\draw (-\mx,-\my) arc [start angle=-90,end angle=90,x radius=\ellx,y radius=\elly,rotate=\ellrotate];
\end{scope}
\begin{scope}
\setellipseplane{\R}{0.6}{-0.6}{-0.6}{-0.4};
\draw (\mx,\my) arc [start angle=-90,end angle=90,x radius=\ellx,y radius=\elly,rotate=\ellrotate];
\end{scope}
}
}

\def\figHFHVHSV{
\tikzfig{primitive}{\guid{HFHVHSV}  All solids in this book can be constructed from
the rectangle, tetrahedron, solid spherical triangle, 
and wedges of a frustum, conic cap, and ball.
}
{
[scale=1]
\pgfmathsetmacro\boxsize{2.3}
\def\R{0.90}
\def\RR{1.1*\R}
\begin{scope} %% solid spherical triangle.
[shift={(2*\boxsize,0)}]
\coordinate (O) at (0,0);
\pgfmathsetmacro\Ax{-\R*0.2};
\pgfmathsetmacro\Ay{\R*0.6};
\pgfmathsetmacro\Bx{-\R*0.8};
\pgfmathsetmacro\By{\R*0.2};
\pgfmathsetmacro\Cx{-\R*0.3};
\pgfmathsetmacro\Cy{-\R*0.3};
%\pgfmathsetmacro\argA{atan2(\Ax,\Ay)};
%\pgfmathsetmacro\argB{atan2(\Bx,\By)};
%\pgfmathsetmacro\argC{atan2(\Cx,\Cy)};
\coordinate (A) at (\Ax,\Ay);
\coordinate (B) at (\Bx,\By);
\coordinate (C) at (\Cx,\Cy);
%\smalldot{A};
%\smalldot{B};
%\smalldot{C};
\smalldot{O};
\draw[ball color=gray!50,draw=gray] (0,0) circle (\R);
%\draw (0,0) circle (\R);
\draw[nearly transparent,help lines] (O)--(A) -- (B) --(C) --cycle;
\draw[help lines] (O)--(B);
\begin{scope}
%\path
%\clip (0,0) circle (\R);
\setellipseplane{\R}{\Ax}{\Ay}{\Bx}{\By};
\clip (-\mx,-\my) arc [start angle={-90},end angle={90},x radius=\ellx,y radius=\elly,rotate=\ellrotate] -- (0,-\RR) -- cycle;
\setellipseplane{\R}{\Bx}{\By}{\Cx}{\Cy};
\clip (-\mx,-\my) arc [start angle={-90},end angle={90},x radius=\ellx,y radius=\elly,rotate=\ellrotate] --  (\RR,\RR) -- cycle;
\setellipseplane{\R}{\Cx}{\Cy}{\Ax}{\Ay};
\clip (\mx,\my) arc [start angle={-90},end angle={90},x radius=\ellx,y radius=\elly,rotate=\ellrotate] -- (\RR,-\RR)  -- (-\RR,-\RR) -- (-\RR,\RR) --  cycle;
\draw[ball color=gray,draw=gray] (0,0) circle (\R);
% \draw (B) arc [start angle={\argB-\ellrotate},end angle={\argC-\ellrotate},x radius=\ellx,y radius=\elly,rotate=\ellrotate];
\setellipseplane{\R}{\Ax}{\Ay}{\Bx}{\By};
\draw (-\mx,-\my) arc [start angle={-90},end angle={90},x radius=\ellx,y radius=\elly,rotate=\ellrotate];
\setellipseplane{\R}{\Bx}{\By}{\Cx}{\Cy};
\draw (-\mx,-\my) arc [start angle={-90},end angle={90},x radius=\ellx,y radius=\elly,rotate=\ellrotate];
\setellipseplane{\R}{\Cx}{\Cy}{\Ax}{\Ay};
\draw (\mx,\my) arc [start angle={-90},end angle={90},x radius=\ellx,y radius=\elly,rotate=\ellrotate];
\end{scope}
\node at (-\RR,\RR) {TRI};
\end{scope} %% end solid spherical triangle
\begin{scope} %% conic cap
[shift={(\boxsize,-\boxsize)}]
\coordinate (O) at (0,0);
\draw[ball color=gray!50,draw=gray] (0,0) circle (\R);
\pgfmathsetmacro\ry{0.6*\R}
%\pgfmathsetmacro\cx{0.55*\R}
\pgfmathsetmacro\cx{0.62*\R}
\pgfmathsetmacro\z{sqrt(\R*\R - \ry*\ry - \cx*\cx)} % (\cx,\ry,\z) is a point on sphere.
\pgfmathsetmacro\rx{\ry*\z/(sqrt(\cx*\cx + \z*\z))} % computed minor axis of elliptic projection.
%\pgfmathsetmacro\rx{0.2*\R}
\pgfmathsetmacro\xc{\cx - \rx*\rx/\cx}
\pgfmathsetmacro\yc{\ry * sqrt(1.0 - \rx*\rx/(\cx*\cx))}
\draw[semitransparent,help lines] (-\xc,\yc) --(O) -- (-\xc,-\yc)--cycle;
\begin{scope}
\path[clip,draw] (180:\cx) circle[x radius = \rx,y radius = \ry];
\draw[ball color=gray] (0,0) circle (\R);
\end{scope}
\path [draw,help lines] (180:\cx) circle[x radius = \rx,y radius = \ry];
\draw[help lines] (O)-- (-\xc,\yc);
\draw[help lines] (O)-- (-\xc,-\yc);
\node at (-\RR,\RR) {CAP};
\end{scope}
\begin{scope} %% ball
[shift={(2*\boxsize,-\boxsize)}]
\draw[ball color=gray!50,draw=gray] (0,0) circle (\R);
\node at (-\RR,\RR) {B};
\end{scope}
\begin{scope} %% rectangle
[shift={(0,0)}]
%\def\zx{-0.3*\R}
%\def\zy{-0.4*\R}
\draw[help lines,->] (0,0) -- (1.5*\R,0);
\draw[help lines,->] (0,0) -- (0,\R);
\draw[help lines,->] (0,0,0) --  (0,0,2*\R);
\draw[fill=gray!50,nearly transparent] (0,0,\R)-- ++ (\R,0,0) -- ++(0,0,-\R) -- ++ (0,0.5*\R,0) -- ++ (-\R,0,0) -- ++ (0,0,\R)
 -- cycle;
\draw (0,0.5*\R,\R) -- ++ (\R,0,0) -- ++ (0,0,-\R);
\draw (\R,0,\R) -- ++ (0,0.5*\R,0);
\node at (-\RR,\RR) {RECT};
\end{scope}
\begin{scope} %% tetrahedron
[shift={(\boxsize,0)}]
\pgfmathsetmacro\r{0.7*\R};
\coordinate (A) at (-0.7*\r,-0.2*\r,\r);
\coordinate (B) at (1.2*\r,-0.2*\r,1.5*\r);
\coordinate (C) at (0.2*\r,\r,-0.8*\r);
\coordinate (D) at (1.2*\r,-0.5*\r,-0.5*\r);
\draw[fill=gray!50,nearly transparent,help lines] (A)--(B)--(D)--cycle;
\draw[fill=gray!30,nearly transparent] (A)--(B)--(C)--cycle;
\draw[fill=gray!90,nearly transparent] (B)--(D)--(C)--cycle;
\coordinate (zABC) at (barycentric cs:A=0.54,B=0.28,C=0.16);
\coordinate (xBCD) at (barycentric cs:B=0.29,C=0.275,D=0.433);
\coordinate (yABC) at (barycentric cs:A=0.32,B=0.09,C=0.58);
\draw[help lines] (0,0,0) --  (zABC);
\draw[->]  (zABC) -- (0,0,2*\R);
\draw[help lines] (0,0) -- (xBCD);
\draw[->] (xBCD) -- (1.2*\R,0);
\draw[help lines] (0,0) -- (yABC);
\draw[->] (yABC)-- (0,\R);
%\node at (A) {$A$};
%\node at (C) {$C$};
%\node at (B) {$B$};
%\node at (D) {$D$};
\draw[help lines] (A) -- (D);
\node at (-\RR,\RR) {TET};
\end{scope}
\begin{scope} %% frustum
[shift={(0,-\boxsize+0.8*\R)}]
\coordinate (origin) (0,0);
\pgfmathsetmacro\a{0.8*\R};
\pgfmathsetmacro\b{1.6*\R};
\pgfmathsetmacro\ra{0.1*\R};
\pgfmathsetmacro\sa{0.4*\R};
\pgfmathsetmacro\sb{\sa*\b/\a};
\pgfmathsetmacro\rb{\ra*\b/\a};
\pgfmathsetmacro\tanAx{\sa* sqrt(1 - \ra*\ra/(\a* \a))};
\pgfmathsetmacro\tanAy{\a - \ra*\ra/ \a};
\coordinate (tanA1) at (-\tanAx ,-\tanAy);
\pgfmathsetmacro\ang{atan2(\tanAx,\tanAy-\a)}; 
\coordinate (tanA2) at ( {\sa* sqrt(1 - \ra*\ra/(\a* \a))},-{\a - \ra*\ra/ \a});
\coordinate (tanB1) at (- {\sb* sqrt(1 - \rb*\rb/(\b* \b))},-{\b - \rb*\rb/ \b});
\coordinate (tanB2) at ( {\sb* sqrt(1 - \rb*\rb/(\b* \b))},-{\b - \rb*\rb/ \b});
%\node at (origin) {$0$};
%\node at (0,-\a) {$a$};
%\node at (0,-\b) {$b$};
\draw[help lines] (origin)-- (tanA1);
\draw[help lines] (origin)-- (tanA2);
\draw (tanA1)-- (tanB1);
\draw (tanA2)-- (tanB2);
\draw[fill=gray!30] (0,-\a) circle[x radius=\sa,y radius = \ra];
%\draw[help lines] (0,-\b) circle[x radius=\sb,y radius = \rb];
%\begin{scope}
%\clip (tanB1)--(tanB2)-- ++(0,-0.5*\R) -- ++(-2*\R,0) -- cycle;
%\draw (0,-\b) circle[x radius=\sb,y radius = \rb];
%\end{scope}
\begin{scope}
\draw (tanB1) arc[x radius=\sb,y radius=\rb,start angle={180+\ang},end angle={360-\ang}];
\clip (tanB1) arc[x radius=\sb,y radius=\rb,start angle={180+\ang},end angle={360-\ang}]
 -- (tanA2)
 arc[x radius=\sa,y radius=\ra,start angle={-\ang},delta angle={-(180+2*\ang)}] -- cycle;
%\shade [upper left=white,lower left=gray!20,lower right=black,upper right=gray]
% (-\R,-2*\R) rectangle (R,-0.5*\R);
\shade [upper left=black!30,lower left=black!60,upper right=gray!40,lower right=white]
 (-\R,-2*\R) rectangle (0,-0.5*\R);
\shade [lower left=white,upper left=gray!40,lower right=black,upper right=gray]
 (0,-2*\R) rectangle (\R,-0.5*\R);
\draw[help lines] (tanB2) arc[x radius=\sb,y radius=\rb,end angle={180+\ang},start angle={-\ang}];
%\draw (0,-\b) circle[x radius=\sb,y radius = \rb];
%\node at (origin) {$\ang$};
\end{scope}
\node at (-\RR,\RR-0.8*\R) {FR};
\end{scope}
}
}



%%%%%%%%%%%%%%%%%%%%%%%%%%%%%%%%%%%%%%%%%%%%%%%%%%%%%%%%%%%%%%%%%%%%%%%%%%%%%%%%
%% HYPERMAP CHAPTER
%%%%%%%%%%%%%%%%%%%%%%%%%%%%%%%%%%%%%%%%%%%%%%%%%%%%%%%%%%%%%%%%%%%%%%%%%%%%%%%%

\tikzset{darrow/.style={->,shorten >=0.5ex,shorten <= 0.5ex,thick}}
\tikzset{rarrow/.style={<-,shorten >=0.5ex,shorten <= 0.5ex,thick}}
\tikzset{barrow/.style={<->,shorten >=0.5ex,shorten <= 0.5ex,thick}}

\def\figZGPXAWJ{
\tikzfig{hypermap_ex}{\guid{ZGPXAWJ} A plane graph is given by the gray edges
and circular nodes as shown.  Twelve darts mark the angles of the plane graph.  Darts may be
  permuted about faces ($f$), nodes ($n$), and edges ($e$) of the plane graph to form a hypermap.
}
{
[scale=1.4] % was 1
\def\planargraph{%
%\coordinate (A) at (0,0);
%\coordinate (B) at 
\def\r{1.1}
\coordinate (A) at (0,0);
\coordinate (B) at ($(A)+(30:0.4*\r)$);
\coordinate (C) at ($(B)+(0:\r)$);
\coordinate (D) at ($(C)+(-30:0.4*\r)$); % (2,0);
\coordinate (E) at ($(B)+(60:\r)$); % (1.0,1.4);
\coordinate (F) at ($(E)+(90:0.4*\r)$); % (1.0,2.0);
%\node (A) at (0,0) {};
%\node (B) at (1,0.5) {};
%\node at (C) {$C$};
%\node at (D) {$D$};
%\node at (E) {$E$};
%\node at (F) {$F$};
\draw[help lines] (A)--(B)--(C)--(D)--(C)--(E)--(F)--(E)--(B);
\smalldot{A}; \smalldot{B}; \smalldot{C}; \smalldot{D}; \smalldot{E}; \smalldot{F};
}
%\tikzstyle{every node}=[]
\def\dartlist{
\pgfmathsetmacro\sep{1.7};
\node[dartstyle]  (A1) at ($(B)!1.4!(A)$) {};
\node[dartstyle]  (B1) at ($(B)+(270:\sep ex)$) {};
\node[dartstyle]  (B2) at ($(B)+(150:\sep ex)$) {};
\node[dartstyle]  (B3) at ($(B)+(30:\sep ex)$) {};
\node[dartstyle]  (C1) at ($(C)+(270:\sep ex)$) {};
\node[dartstyle]  (C2) at ($(C)+(30:\sep ex)$) {};
\node[dartstyle]  (C3) at ($(C)+(150:\sep ex)$) {};
\node[dartstyle]  (E1) at ($(E)+(150:\sep ex)$) {};
\node[dartstyle]  (E2) at ($(E)+(30:\sep ex)$) {};
\node[dartstyle]  (E3) at ($(E)+(270:\sep ex)$) {};
\node[dartstyle]  (D1) at ($(C)!1.4!(D)$) {};
\node[dartstyle]  (F1) at ($(E)!1.4!(F)$) {};
}
%darts
%\dart{(0,0.2)};
\def\boxsize{3.0}
\def\boxA{2.4}
\begin{scope}
[shift={(0,0)}]
\planargraph
\end{scope}
\begin{scope} %% f
[shift={(\boxsize+0*\boxA,0)}]
\planargraph;
\dartlist;
\draw[darrow] (A1) .. controls +(90:2ex) .. (B2);
\draw[darrow] (B2) -- (E1);
\draw[rarrow] (F1) .. controls +(210:2ex).. (E1);
\draw[darrow] (F1) .. controls +(330:2ex) .. (E2);
\draw[darrow] (E2) -- (C2);
\draw[rarrow] (D1) .. controls +(90:2ex).. (C2);
\draw[darrow] (D1) .. controls +(210:2ex).. (C1);
\draw[darrow] (C1) -- (B1);
\draw[rarrow] (A1) .. controls  +(330:2ex).. (B1);
\draw[darrow] (B3) -- (C3);
\draw[darrow] (C3) -- (E3);
\draw[darrow] (E3) -- (B3);
\node at ($(A)!.5!(D) +(-90:2ex)$) {$f$};
\end{scope}
\begin{scope} %% n
[shift={(0.0*\boxA+0.0*\boxsize,-\boxsize)}]
\planargraph;
\dartlist;
\draw[barrow] (A1) .. controls ($(A1)+(180:4ex)$) and ($(A1)+(240:4ex)$) .. (A1);
\draw[darrow] (B1) .. controls ($(B)+(-30:2ex)$).. (B3);
\draw[darrow] (B3) .. controls ($(B)+(90:2ex)$).. (B2);
\draw[darrow] (B2) .. controls ($(B)+(210:2ex)$).. (B1);
\draw[barrow] (F1) .. controls ($(F1)+(60:4ex)$) and ($(F1)+(120:4ex)$).. (F1);
\draw[darrow] (E1) .. controls ($(E)+(210:2ex)$).. (E3);
\draw[darrow] (E3) .. controls ($(E)+(-30:2ex)$).. (E2);
\draw[darrow] (E2) .. controls ($(E)+(90:2ex)$).. (E1);
\draw[darrow] (C1) .. controls ($(C)+(-30:2ex)$).. (C2);
\draw[darrow] (C2) .. controls ($(C)+(90:2ex)$).. (C3);
\draw[darrow] (C3) .. controls ($(C)+(210:2ex)$).. (C1);
\draw[barrow] (D1) .. controls ($(D1)+(300:4ex)$) and ($(D1)+(360:4ex)$).. (D1);
\node at ($(A)!.5!(D) +(-90:2ex)$) {$n$};
\end{scope}
\begin{scope} %% e
[shift={(0*\boxA+\boxsize,-\boxsize)}]
\planargraph;
\dartlist;
\draw[barrow] (A1) .. controls  +(330:2ex).. (B1);
\draw[barrow] (B3) -- (C1);
\draw[barrow] (D1) .. controls +(90:2ex).. (C2);
\draw[barrow] (B2) -- (E3);
\draw[barrow] (C3) -- (E2);
\draw[barrow] (F1) .. controls +(210:2ex).. (E1);
\node at ($(A)!.5!(D) +(-90:2ex)$) {$e$};
\end{scope}
}
}

%% baselines don't align here.
\def\figLMGQYKG {
\tikzfig{dart+}{\guid{LMGQYKG} A fragment of a hypermap showing a dart $x$ and its entourage.
 The node map is given by the horizontal arrows, the edge map by arrows descending towards the
  lower left, and the face map by arrows rising towards the upper left.}
{
[scale=1.0]
\def\r{1.1}
\node (X) at (0,0) {$x$};
\node (NX) at (0:\r) {$n\,x$};
\node (FMX) at (-60:\r) {$f^{-1}\,x$};
\node (EX) at (-120:\r) {$e\,x$};
\node (NMX) at (180:\r) {$n^{-1}\,x$};
\node (FX) at (120:\r) {$f\,x$};
\node (EMX) at (60:\r) {$e^{-1}\,x$};
\draw[darrow] (X) --  % node[anchor=south,gray] {$\scriptsize{n}$} 
  (NX);
\draw[darrow] (NX)-- % node[anchor=west] {$e$} 
(FMX);
\draw[darrow] (FMX)-- % node[anchor=east] {$f$}
 (X);
\draw[darrow] (X)-- % node[anchor=west] {$e$} 
(EX);
\draw[darrow] (EX)-- % node[anchor=east] {$f$} 
 (NMX);
\draw[darrow] (NMX)-- %node % [anchor=south] {$n$}
  (X);
\draw[darrow] (X)-- %node % [anchor=east] {$f$}
  (FX);
\draw[darrow] (FX)-- %node % [anchor=south] {$n$}
  (EMX);
\draw[darrow] (EMX)-- %node % [anchor=west] {$e$}
  (X);
}
}


\def\figANDKKER{
\tikzfig{dart-fix}{\guid{ANDKKER} A fragment of a hypermap showing a dart $x$ and its entourage
when $x$ is fixed by the permutation $f$.  The vertical
arrow on the left has type $f$.
The two arrows pointing to the right along the top of the diagram have type $n$.  The
two arrows pointing to the left along the bottom of the diagram have type $e$.
}
{
[scale=1.0]
\def\r{1.0}
\node (X) at (0,0) {$x$};
\node (NX) at (0:1.1*\r) {$n\,x$}; % = e^{-1}\,x$};
\node (EX) at (-140:\r) {$e\,x$};
\node (NMX) at (140:\r) {$n^{-1}\,x$};
\draw[darrow] (X) -- (EX);
\draw[darrow] (EX)-- (NMX);
\draw[darrow] (NMX) -- (X);
\draw[darrow] (X) .. controls +(1.1*\r/2,3ex).. (NX);
\draw[darrow] (NX).. controls +(-1.1*\r/2,-3ex).. (X);
}
}


\def\figGFJIBZV{
\tikzfig{walk}{\guid{GFJIBZV} A fragment of a hypermap in its original state,
and three modified fragments after applying a face walkup $W_f$, edge walkup $W_e$, 
and node walkup $W_n$ at the dart $x$ at the center.
Each walkup eliminates the dart $x$.  The other darts 
are the same as in the original, but undergo modified permutations $e$, $n$, and $f$.}
{
[scale=0.95]
\def\r{1.1}
\begin{scope}
\node[dartstyle] (X) at (0,0) {};
\node[anchor=south] at ($(X)+(90:1.3ex)$) {$x$};
\node[dartstyle] (NX) at (0:\r) {};
\node[dartstyle] (FMX) at (-60:\r) {};
\node[dartstyle] (EX) at (120:\r) {};
\node[dartstyle] (NMX) at (60:\r) {};
\node[dartstyle] (FX) at (-120:\r) {};
\node[dartstyle] (EMX) at (180:\r) {};
\draw[darrow] (X) --  node[anchor=south] {$n$} 
  (NX);
\draw[darrow] (NX)--  node[anchor=west] {$e$} 
(FMX);
\draw[darrow] (FMX)--  node[anchor=west] {$f$}
 (X);
\draw[darrow] (X)--  node[anchor=east] {$e$} 
(EX);
\draw[darrow] (EX)--   node[anchor=south] {$f$}  (NMX);
\draw[darrow] (NMX)-- node  [anchor=west] {$n$}
  (X);
\draw[darrow] (X)-- node  [anchor=west] {$f$}
  (FX);
\draw[darrow] (FX)-- node  [anchor=east] {$n$}
  (EMX);
\draw[darrow] (EMX)-- node  [anchor=south] {$e$}
  (X);
\draw[thick,gray] (-\r,-1.4*\r)--(\r,-1.4*\r);
\end{scope}
\def\boxA{3.0}
\begin{scope}
[shift={(\boxA,0)}]
\node[dartstyle,draw=gray,fill=white] (X) at (0,0) {};
\node[anchor=south] at ($(X)+(90:1.3ex)$) {};
\node[dartstyle] (NX) at (0:\r) {};
\node[dartstyle] (FMX) at (-60:\r) {};
\node[dartstyle] (EX) at (120:\r) {};
\node[dartstyle] (NMX) at (60:\r) {};
\node[dartstyle] (FX) at (-120:\r) {};
\node[dartstyle] (EMX) at (180:\r) {};
\node at ($(FX)!.5!(FMX) +(-90:1.5ex)$) {$W_f$};
\draw[darrow] (NX)--  node[anchor=west] {$e$} 
(FMX);
\draw[darrow] (FMX)--  node[anchor=east] {$f$}
 (NMX);
\draw[darrow] (EX)--  node[anchor=west] {$f$} 
 (FX);
\draw[darrow] (NMX)-- node  [anchor=west] {$n$}
  (NX);
\draw[darrow] (FX)-- node  [anchor=east] {$n$}
  (EMX);
\draw[darrow] (EMX)-- node  [anchor=east] {$e$}
  (EX);
\draw[thick,gray] (-\r,-1.4*\r)--(\r,-1.4*\r);
\end{scope}
\begin{scope}
[shift={(2*\boxA,0)}]
\node[dartstyle,draw=gray,fill=white] (X) at (0,0) {};
\node[anchor=south] at ($(X)+(90:1.3ex)$) {};
\node[dartstyle] (NX) at (0:\r) {};
\node[dartstyle] (FMX) at (-60:\r) {};
\node[dartstyle] (EX) at (120:\r) {};
\node[dartstyle] (NMX) at (60:\r) {};
\node[dartstyle] (FX) at (-120:\r) {};
\node[dartstyle] (EMX) at (180:\r) {};
\node at ($(NX)!.5!(FMX)$) {$W_e$};
\draw[darrow] (NX)--  node[anchor=north] {$e$} 
(EX);
\draw[darrow] (FMX)--  node[anchor=north] {$f$}
 (FX);
\draw[darrow] (EX)--  node[anchor=south] {$f$} 
 (NMX);
\draw[darrow] (NMX)-- node  [anchor=west] {$n$}
  (NX);
\draw[darrow] (FX)-- node  [anchor=east] {$n$}
  (EMX);
\draw[darrow] (EMX)-- node  [anchor=south] {$e$}
  (FMX);
\draw[thick,gray] (-\r,-1.4*\r)--(\r,-1.4*\r);
\end{scope}
\begin{scope}
[shift={(3*\boxA,0)}]
\node[dartstyle,draw=gray,fill=white] (X) at (0,0) {};
\node[anchor=south] at ($(X)+(90:1.3ex)$) {};
\node[dartstyle] (NX) at (0:\r) {};
\node[dartstyle] (FMX) at (-60:\r) {};
\node[dartstyle] (EX) at (120:\r) {};
\node[dartstyle] (NMX) at (60:\r) {};
\node[dartstyle] (FX) at (-120:\r) {};
\node[dartstyle] (EMX) at (180:\r) {};
\node at ($(FX)!.5!(EMX)$) {$W_n$};
\draw[darrow] (NX)--  node[anchor=west] {$e$} 
(FMX);
\draw[darrow] (FMX)--  node[anchor=north] {$f$}
 (FX);
\draw[darrow] (EX)--  node[anchor=south] {$f$} 
 (NMX);
\draw[darrow] (NMX)-- node  [anchor=north] {$n$}
  (EMX);
\draw[darrow] (FX)-- node  [anchor=south] {$n$}
  (NX);
\draw[darrow] (EMX)-- node  [anchor=east] {$e$}
  (EX);
\draw[thick,gray] (-\r,-1.4*\r)--(\r,-1.4*\r);
\end{scope}
}
}



\def\figLNIZBPW{
\tikzfig{walkdegen}{\guid{LNIZBPW} All three walkup transformations have
the same effect at a dart $x$ that is degenerate, regardless of whether
the degeneracy is a face-, edge-, or node-degeneracy.  The three different kinds
of degenerate darts are shown in the hypermap fragments on the outside of the figure, and
the hypermap fragment after applying a walkup is shown at the center of the diagram.}
{
[scale=1.0]
\def\r{0.75}
\def\R{1.8}
\begin{scope}
[shift={(0,0)}]
\node[dartstyle] (NT) at (0:\r) {};
\node[dartstyle] (ET) at (240:\r) {};
\node[dartstyle] (FT) at (120:\r) {};
\node[dartstyle,draw=gray,fill=white] (X) at (0,0) {};
\draw[darrow] (FT)-- node[anchor=south] {$n$} (NT);
\draw[darrow] (NT)-- node[anchor=north] {$e$} (ET);
\draw[darrow] (ET)-- node[anchor=east] {$f$} (FT);
\node (T) at (0:2.4*\r) {post-walkup};
\end{scope}
\begin{scope}
[shift={(180:\R)}]
\node[dartstyle] (EX) at (-120:\r) {};
\node[dartstyle] (NMX) at (120:\r) {};
\node[dartstyle] (NX) at (0:\r) {};
\node[dartstyle] (X) at (0,0) {};
\node (X1) at (-1.1ex,0ex) {$x$};
\draw[darrow] (X)-- node[pos=0.2,anchor=north] {$\,\,e$} (EX);
\draw[darrow] (EX)-- node[anchor=east] {$f$} (NMX);
\draw[darrow] (NMX)-- node[anchor=south] {$~~~n$} (X);
\draw[darrow] (X) ..controls +(\r/2,2ex).. node[anchor=south] {$n$} (NX);
\draw[darrow] (NX) ..controls +(-\r/2,-2ex).. node[anchor=north] {$e$} (X);
\node (T) at (-90:1.4*\r) {$f\,x =x$};
\end{scope}
\begin{scope}
[shift={(60:\R)}]
\node[dartstyle] (EX) at (-120:\r) {};
\node[dartstyle] (FX) at (120:\r) {};
\node[dartstyle] (EMX) at (0:\r) {};
\node[dartstyle] (X) at (0,0) {};
\node (X1) at (0.5ex,1.08ex) {$x$};
\draw[darrow] (X) ..controls +(-90:1.3*\r/2) .. node[pos=0.4,anchor=north west] {$e$} (EX);
\draw[darrow] (EX) ..controls +(90:1.3*\r/2).. node[pos=0.4,anchor=east] {$f$} (X);
\draw[darrow] (EMX)-- node[anchor=north] {$e$} (X);
\draw[darrow] (X) -- node[pos=0.2,anchor=east] {$f\,$} (FX);
\draw[darrow] (FX) -- node[anchor=south] {$n$} (EMX);
\node (T) at (-30:1.6*\r) {$n\,x=x$};
\end{scope}
\begin{scope}
[shift={(-60:\R)}]
\node[dartstyle] (FMX) at (-120:\r) {};
\node[dartstyle] (FX) at (120:\r) {};
\node[dartstyle] (NX) at (0:\r) {};
\node[dartstyle] (X) at (0,0) {};
\node (X1) at (0.7ex,-1.0ex) {$x$};
\draw[rarrow] (X) ..controls +(90:1.3*\r/2) .. node[anchor=west] {$n$} (FX);
\draw[rarrow] (FX) ..controls +(-90:1.3*\r/2).. node[anchor=east] {$f$} (X);
\draw[darrow] (X) -- node[anchor=south] {$n$} (NX);
\draw[darrow] (NX)-- node[anchor=north] {$e$} (FMX);
\draw[darrow] (FMX) -- node[anchor=east] {$f$} (X);
\node (T) at (-20:1.6*\r) {$e\,x=x$};
\end{scope}
}
}


\def\figILWHXTE{
\tikzfig{split}{\guid{ILWHXTE} When darts $x$ and $e\,x$ lie in the same orbit of $f$ (upper left),
the face walkup $W_f$ at $x$ splits the orbit into two (lower left).  
But when the 
darts $x$ and $e\,x$ lie in different orbits of $f$ (upper right), the edge walkup $W_f$ at $x$
merges the two orbits into one (lower right).}
{
[scale=1.0]
\def\r{1.1}
\tikzset{ellipsis/.style={
{decoration={
   markings,
   mark=between positions 0.43 and 0.57 step 0.07
   with
   {
    \draw[fill=white,draw=white]  (0,0) circle (0.8ex) {};
    \node[dartstyle] at (0,0) {};
     }
   }
  }
}}
\begin{scope}
[shift={(0,0)}]
\node[dartstyle] (X) at (0,0) {};
\node[anchor=south] at ($(X)+(90:1.3ex)$) {$x$};
\node[dartstyle] (NX) at (0:\r) {};
\node[dartstyle] (FMX) at (-60:\r) {};
\node[dartstyle] (EX) at (120:\r) {};
\node[anchor=south east] at ($(EX)+(180:0ex)$) {$e x$};
\node[dartstyle] (NMX) at (60:\r) {};
\node[dartstyle] (FX) at (-120:\r) {};
\node[dartstyle] (EMX) at (180:\r) {};
\draw[darrow,gray] (X) --  node[anchor=south,gray] {$n$}   (NX);
\draw[darrow,gray] (NX)--  node[anchor=west,gray] {$e$} (FMX);
\draw[darrow,thick] (FMX)--  node[anchor=west] {$f$}  (X);
\draw[darrow,gray] (X)--  node[anchor=east] {$e$} (EX);
\draw[darrow,thick] (EX)--   node[anchor=south] {$f$}  (NMX);
\draw[darrow,gray] (NMX)-- node  [anchor=west] {$n$}   (X);
\draw[darrow,thick] (X)-- node  [anchor=west] {$f$}   (FX);
\draw[darrow,gray] (FX)-- node  [anchor=east] {$n$}   (EMX);
\draw[darrow,gray] (EMX)-- node  [anchor=south] {$e$}   (X);
\draw[darrow,ellipsis,postaction={decorate}] (NMX) ..controls (0:2*\r).. node[anchor=west] {$f$} (FMX);
\draw[darrow,ellipsis,postaction={decorate}] (FX) .. controls (180:2*\r).. node[anchor=east] {$f$} (EX);
\end{scope}
\def\boxA{3.5}
\begin{scope}
[shift={(0,-\boxA)}]
\node[dartstyle,draw=gray,fill=white] (X) at (0,0) {};
\node[anchor=south] at ($(X)+(90:1.3ex)$) {};
\node[dartstyle] (NX) at (0:\r) {};
\node[dartstyle] (FMX) at (-60:\r) {};
\node[dartstyle] (EX) at (120:\r) {};
\node[dartstyle] (NMX) at (60:\r) {};
\node[dartstyle] (FX) at (-120:\r) {};
\node[dartstyle] (EMX) at (180:\r) {};
\node at ($(FX)!.5!(FMX) +(-90:1.5ex)$) {$W_f$};
\draw[darrow,gray] (NX)--  node[anchor=west] {$e$} (FMX);
\draw[darrow,thick] (FMX)--  node[anchor=east] {$f$}  (NMX);
\draw[darrow,thick] (EX)--  node[anchor=west] {$f$}  (FX);
\draw[darrow,gray] (NMX)-- node  [anchor=west] {$n$}   (NX);
\draw[darrow,gray] (FX)-- node  [anchor=east] {$n$}   (EMX);
\draw[darrow,gray] (EMX)-- node  [anchor=east] {$e$}  (EX);
\draw[darrow,ellipsis,postaction={decorate}] (NMX) ..controls (0:2*\r).. node[anchor=west] {$f$} (FMX);
\draw[darrow,ellipsis,postaction={decorate}] (FX) .. controls (180:2*\r).. node[anchor=east] {$f$} (EX);
\end{scope}
%% Right column of diagram
\tikzset{ellipsisB/.style={
{decoration={
   markings,
   mark=between positions 0.4 and 0.6 step 0.1
   with
   {
    \draw[fill=white,draw=white]  (0,0) circle (0.67ex) {};
    \node[dartstyle] at (0,0) {};
     }
   }
  }
}}
\begin{scope}
[shift={(1.4*\boxA,0)}]
\node[dartstyle] (X) at (0,0) {};
\node[anchor=south] at ($(X)+(90:1.3ex)$) {$x$};
\node[dartstyle] (NX) at (0:\r) {};
\node[dartstyle] (FMX) at (-60:\r) {};
\node[dartstyle] (EX) at (120:\r) {};
\node[anchor=south east] at ($(EX)+(180:0ex)$) {$e x$};
\node[dartstyle] (NMX) at (60:\r) {};
\node[dartstyle] (FX) at (-120:\r) {};
\node[dartstyle] (EMX) at (180:\r) {};
\draw[darrow,gray] (X) --  node[anchor=south,gray] {$n$}   (NX);
\draw[darrow,gray] (NX)--  node[anchor=west,gray] {$e$} (FMX);
\draw[darrow,thick] (FMX)--  node[anchor=west] {$f$}  (X);
\draw[darrow,gray] (X)--  node[anchor=east] {$e$} (EX);
\draw[darrow,thick] (EX)--   node[anchor=south] {$f$}  (NMX);
\draw[darrow,gray] (NMX)-- node  [anchor=west] {$n$}   (X);
\draw[darrow,thick] (X)-- node  [anchor=west] {$f$}   (FX);
\draw[darrow,gray] (FX)-- node  [anchor=east] {$n$}   (EMX);
\draw[darrow,gray] (EMX)-- node  [anchor=south] {$e$}   (X);
\draw[darrow,ellipsisB,postaction={decorate}] (NMX) ..controls +(120:\r).. node[anchor=south] {$f$} (EX);
\draw[darrow,ellipsisB,postaction={decorate}] (FX) .. controls +(-60:\r).. node[anchor=north] {$f$} (FMX);
\end{scope}
\begin{scope}
[shift={(1.7*\boxA,-\boxA)}]
\node[dartstyle,draw=gray,fill=white] (X) at (0,0) {};
\node[anchor=south] at ($(X)+(90:1.3ex)$) {};
\node[dartstyle] (NX) at (0:\r) {};
\node[dartstyle] (FMX) at (-60:\r) {};
\node[dartstyle] (EX) at (120:\r) {};
\node[dartstyle] (NMX) at (60:\r) {};
\node[dartstyle] (FX) at (-120:\r) {};
\node[dartstyle] (EMX) at (180:\r) {};
\node at ($(FX)!.5!(FMX) +(-90:1.5ex)$) {$W_f$};
\draw[darrow,gray] (NX)--  node[anchor=west] {$e$} (FMX);
\draw[darrow,thick] (FMX)--  node[anchor=east] {$f$}  (NMX);
\draw[darrow,thick] (EX)--  node[anchor=west] {$f$}  (FX);
\draw[darrow,gray] (NMX)-- node  [anchor=west] {$n$}   (NX);
\draw[darrow,gray] (FX)-- node  [anchor=east] {$n$}   (EMX);
\draw[darrow,gray] (EMX)-- node  [anchor=east] {$e$}  (EX);
\draw[darrow,ellipsisB,postaction={decorate}] (NMX) ..controls +(120:\r).. node[anchor=south] {$f$} (EX);
\draw[darrow,ellipsisB,postaction={decorate}] (FX) .. controls +(-60:\r).. node[anchor=north] {$f$} (FMX);
\end{scope}
}
}

\def\figYKJFZEB{
\tikzfig{doublenode}{\guid{YKJFZEB} When a hypermap comes from a plane graph, the double
{\it node} walkup at the two darts of an edge contracts the
corresponding edge of the graph.
}
{
[scale=1.0]
\def\r{1.0}
\def\planargraph{%
\coordinate (A) at (0,0);
\coordinate (B) at ($(A)+(90:\r)$);
\coordinate (C) at ($(B)+(30:\r)$);
\coordinate (D) at ($(A)+(-30:\r)$); 
\coordinate (E) at ($(A)+(-150:\r)$); 
\coordinate (F) at ($(B)+(150:\r)$); 
\draw[help lines] (E)--(A)--(D);
\draw[help lines] (F)--(B)--(C);
\draw[help lines] (A)--(B);
\draw[help lines] (A)--(B);
\foreach \i in {C,F} {
 \draw[help lines] (\i)-- +(90:2.5ex);
 \draw[help lines] (\i)-- +(-30:2.5ex);
 \draw[help lines] (\i)-- +(-150:2.5ex);
}
\foreach \i in {D,E} {
 \draw[help lines] (\i)-- +(-90:2.5ex);
 \draw[help lines] (\i)-- +(30:2.5ex);
 \draw[help lines] (\i)-- +(150:2.5ex);
}
\foreach \i in {A,B,C,D,E,F} \graydot{\i};
}
\def\dartlist{
\pgfmathsetmacro\sep{1.7};
\foreach \i in {F,C,A} {
 \node[dartstyle] at ($(\i)+(-90:\sep ex)$) {};
 \node[dartstyle] at ($(\i)+(30:\sep ex)$) {};
 \node[dartstyle] at ($(\i)+(150:\sep ex)$) {};
}
\foreach \i in {B,D,E} {
 \node[dartstyle] at ($(\i)+(90:\sep ex)$) {};
 \node[dartstyle] at ($(\i)+(-30:\sep ex)$) {};
 \node[dartstyle] at ($(\i)+(-150:\sep ex)$) {};
}
}
\planargraph
\dartlist
\draw ($(A)+(150:\sep ex)$) circle (0.9ex);
\draw ($(B)+(-30:\sep ex)$) circle (0.9ex);
%\node at ($(E)+(-90:4ex)$) {};
\begin{scope}
[shift={(4,0)}]
\def\planargraphB{%
\coordinate (A) at (0,0);
\coordinate (B) at ($(A)+(90:\r)$);
\coordinate (C) at ($(B)+(30:\r)$);
\coordinate (D) at ($(A)+(-30:\r)$); 
\coordinate (E) at ($(A)+(-150:\r)$); 
\coordinate (F) at ($(B)+(150:\r)$); 
\coordinate (AB) at ($(A)!0.5!(B)$);
\draw[help lines] (E)--(AB)--(D);
\draw[help lines] (F)--(AB)--(C);
%\draw[help lines] (A)--(B);
\foreach \i/\j in {C/-30,F/-150} {
 \draw[help lines] (\i)-- +(90:2.5ex);
 \draw[help lines] (\i)-- +(\j:2.5ex);
}
\foreach \i/\j in {D/30,E/150} {
 \draw[help lines] (\i)-- +(-90:2.5ex);
 \draw[help lines] (\i)-- +(\j:2.5ex);
}
\foreach \i in {AB,C,D,E,F} \graydot{\i};
%\graydot{A}; \graydot{B}; \graydot{C}; \smalldot{D}; \smalldot{E}; \smalldot{F};
}
\def\dartlistB{
\pgfmathsetmacro\sep{1.7};
\foreach \i in {F,C} {
 \node[dartstyle] at ($(\i)+(-90:\sep ex)$) {};
 \node[dartstyle] at ($(\i)+(30:\sep ex)$) {};
 \node[dartstyle] at ($(\i)+(150:\sep ex)$) {};
}
\foreach \i in {D,E} {
 \node[dartstyle] at ($(\i)+(90:\sep ex)$) {};
 \node[dartstyle] at ($(\i)+(-30:\sep ex)$) {};
 \node[dartstyle] at ($(\i)+(-150:\sep ex)$) {};
}
\foreach \j in {0,90,180,270} {
 \node[dartstyle] at ($(AB)+(\j:\sep ex)$) {};
}
}
\planargraphB
\dartlistB
\end{scope}
}
}

\def\figRVNJBTK{
\tikzfig{doubleface}{\guid{RVNJBTK} When a hypermap comes from a plane graph, the double
{\it face} walkup at the two darts of an edge deletes the
corresponding edge from the graph.
}
{
[scale=1.0]
\def\r{1.0}
\def\planargraph{%
\coordinate (A) at (0,0);
\coordinate (B) at ($(A)+(90:\r)$);
\coordinate (C) at ($(B)+(30:\r)$);
\coordinate (D) at ($(A)+(-30:\r)$); 
\coordinate (E) at ($(A)+(-150:\r)$); 
\coordinate (F) at ($(B)+(150:\r)$); 
\draw[help lines] (E)--(A)--(D);
\draw[help lines] (F)--(B)--(C);
\foreach \i in {C,F} {
 \draw[help lines] (\i)-- +(90:2.5ex);
 \draw[help lines] (\i)-- +(-30:2.5ex);
 \draw[help lines] (\i)-- +(-150:2.5ex);
}
\foreach \i in {D,E} {
 \draw[help lines] (\i)-- +(-90:2.5ex);
 \draw[help lines] (\i)-- +(30:2.5ex);
 \draw[help lines] (\i)-- +(150:2.5ex);
}
\foreach \i in {A,B,C,D,E,F} \graydot{\i};
}
\def\dartlist{
\pgfmathsetmacro\sep{1.7};
\foreach \i in {F,C,A} {
 \node[dartstyle] at ($(\i)+(-90:\sep ex)$) {};
 \node[dartstyle] at ($(\i)+(30:\sep ex)$) {};
 \node[dartstyle] at ($(\i)+(150:\sep ex)$) {};
}
\foreach \i in {B,D,E} {
 \node[dartstyle] at ($(\i)+(90:\sep ex)$) {};
 \node[dartstyle] at ($(\i)+(-30:\sep ex)$) {};
 \node[dartstyle] at ($(\i)+(-150:\sep ex)$) {};
}
}
\planargraph
\draw[help lines] (A)--(B);
\dartlist
\draw ($(A)+(150:\sep ex)$) circle (0.9ex);
\draw ($(B)+(-30:\sep ex)$) circle (0.9ex);
%\node at ($(E)+(-90:4ex)$) {};
\begin{scope}
[shift={(4,0)}]
\def\dartlistB{
\pgfmathsetmacro\sep{1.7};
\foreach \i in {F,C} {
 \node[dartstyle] at ($(\i)+(-90:\sep ex)$) {};
 \node[dartstyle] at ($(\i)+(30:\sep ex)$) {};
 \node[dartstyle] at ($(\i)+(150:\sep ex)$) {};
}
\foreach \i in {D,E} {
 \node[dartstyle] at ($(\i)+(90:\sep ex)$) {};
 \node[dartstyle] at ($(\i)+(-30:\sep ex)$) {};
 \node[dartstyle] at ($(\i)+(-150:\sep ex)$) {};
}
\foreach \i in {A,B} {
 \node[dartstyle] at ($(\i)+(90:\sep ex)$) {};
 \node[dartstyle] at ($(\i)+(-90:\sep ex)$) {};
}
}
\planargraph
\dartlistB
\end{scope}
}
}

\def\figCBQQAKM{
\tikzfig{doubleedge}{\guid{CBQQAKM} When a hypermap comes from a plane graph, the double
{\it edge} walkup at the two darts of a node deletes the
corresponding node from the graph.
}
{
[scale=1.0]
\def\r{1.0}
\def\planargraph{%
\coordinate (A) at (0,0);
\coordinate (B) at ($(A)+(-10:\r)$);
\coordinate (C) at ($(B)+(10:\r)$);
\draw[help lines] (A)--(B)--(C);
\draw[help lines] (C)-- +(60:2.5ex);
\draw[help lines] (C)-- +(-60:2.5ex);
\draw[help lines] (A)-- +(120:2.5ex);
\draw[help lines] (A)-- +(-120:2.5ex);
\foreach \i in {A,B,C} \graydot{\i};
}
\def\dartlist{
\pgfmathsetmacro\sep{1.7};
\foreach \i/\j/\k/\l in {A/30/150/270,C/0/120/240} {
 \node[dartstyle] at ($(\i)+(\j:\sep ex)$) {};
 \node[dartstyle] at ($(\i)+(\k:\sep ex)$) {};
 \node[dartstyle] at ($(\i)+(\l:\sep ex)$) {};
}
 \node[dartstyle] at ($(B)+(90:\sep ex)$) {};
 \node[dartstyle] at ($(B)+(-90:\sep ex)$) {};
}
\planargraph
\dartlist
\draw ($(B)+(90:\sep ex)$) circle (0.9ex);
\draw ($(B)+(-90:\sep ex)$) circle (0.9ex);
\begin{scope}
[shift={(4,0)}]
\def\planargraphB{%
\coordinate (A) at (0,0);
\coordinate (B) at ($(A)+(-10:\r)$);
\coordinate (C) at ($(B)+(10:\r)$);
\draw[help lines] (A)--(C);
\draw[help lines] (C)-- +(60:2.5ex);
\draw[help lines] (C)-- +(-60:2.5ex);
\draw[help lines] (A)-- +(120:2.5ex);
\draw[help lines] (A)-- +(-120:2.5ex);
\foreach \i in {A,C} \graydot{\i};
}
\def\dartlistB{
\pgfmathsetmacro\sep{1.7};
\foreach \i/\j/\k/\l in {A/30/150/270,C/0/120/240} {
 \node[dartstyle] at ($(\i)+(\j:\sep ex)$) {};
 \node[dartstyle] at ($(\i)+(\k:\sep ex)$) {};
 \node[dartstyle] at ($(\i)+(\l:\sep ex)$) {};
}
}
\planargraphB
\dartlistB
\end{scope}
}
}

\def\figKLANQLT{
\tikzfig{doubleplain}{\guid{KLANQLT} A double edge walkup of a hypermap at
the two darts of a node preserves plainness, depicted as bidirectional $e$-arrows.  
The three frames show 
a fragment of a plain hypermap, the fragment after the first edge walkup at $x$,
and after the second edge walkup at $y$.
}
{
[scale=1.0]
{
\def\r{1.4}
\begin{scope}
\node[dartstyle] (A) at (0,0)  {};
\node[dartstyle] (B) at (\r,0) {};
\node[dartstyle] (C) at ($(B)+(\r,0)$) {};
\node[dartstyle] (D) at ($(A)+(60:\r)$) {};
\node[dartstyle] (E) at ($(B)+(60:\r)$) {};
\node[dartstyle] (F) at ($(C)+(60:\r)$) {};
\node[anchor=south] at (E) {$x$};
\node[anchor=north] at (B) {$y$};
\foreach \i/\j/\d/\s in {C/B/north/f,B/A/north/f,A/D/east/n,D/E/south/f,F/C/east/n,E/F/south/f} {
 \draw[darrow] (\i)-- node[anchor=\d] {$\s$} (\j);
}
\foreach \i/\j/\d/\s in {B/D/east/e,B/E/east/n,C/E/east/e}
 { \draw[barrow] (\i)-- node[anchor=\d] {$\s$} (\j);
}
\end{scope}
\begin{scope} % second frame.
[shift={(3.7,0)}]
\node[dartstyle] (A) at (0,0)  {};
\node[dartstyle] (B) at (\r,0) {};
\node[dartstyle] (C) at ($(B)+(\r,0)$) {};
\node[dartstyle] (D) at ($(A)+(60:\r)$) {};
\node (E) at ($(B)+(60:\r)$) {};
\node[dartstyle] (F) at ($(C)+(60:\r)$) {};
\node[anchor=north] at (B) {$y$};
\foreach \i/\j/\d/\s in {B/A/north/f,A/D/east/n,D/F/south/f,F/C/east/n,
  D/B/east/e,C/D/south/e} {
 \draw[darrow] (\i)-- node[anchor=\d] {$\s$} (\j);
}
\draw[darrow] (B)..controls +(30:\r/2) and ($(C)+ (180:\r/3)$) .. node[anchor=north] {$e$} (C);
\draw[darrow] (C)..controls +(210:\r/2).. node[anchor=north] {$f$} (B);
\end{scope}
\begin{scope} % third frame.
[shift={(7.4,0)}]
\node[dartstyle] (A) at (0,0)  {};
\node (B) at (\r,0) {};
\node[dartstyle] (C) at ($(B)+(\r,0)$) {};
\node[dartstyle] (D) at ($(A)+(60:\r)$) {};
\node (E) at ($(B)+(60:\r)$) {};
\node[dartstyle] (F) at ($(C)+(60:\r)$) {};
\foreach \i/\j/\d/\s in {C/A/north/f,A/D/east/n,D/F/south/f,F/C/east/n} {
 \draw[darrow] (\i)-- node[anchor=\d] {$\s$} (\j);
}
\draw[barrow] (C)-- node[anchor=south] {$e$} (D);
\end{scope}
}
}
}

\def\figTUPLKAJ{
\tikzfig{shade-dart}{\guid{TUPLKAJ} When a hypermap is constructed from a plane
graph, a contour path can be depicted by a sequence of arrows between darts that follow alongside
the edges of the graph.  The contour path can be reconstructed from a shaded path
alongside the edges of the graph.  The direction of the path is such that the edges
of the graph remain to the right of motion.
}
{
[scale=1.0]
\def\r{1.5}
\pgfmathsetmacro\sep{3.2};
\def\planegraph{
\node (A) at (0,0) {};
\node (C) at ($(A)+(120:\r)$) {};
\node (B) at ($(C)+(-120:\r)$) {};
\node (D) at ($(C)+(60:\r)$) {};
\node (F) at ($(C)+(120:\r)$) {};
\node (E) at ($(D)+(120:\r)$) {};
\node (G) at ($(C)+(180:\r)$) {};
\foreach \i in {A,B,C,D,E,F,G} { \graydot{\i}; };
\draw[help lines] (A)--(C)--(D)--(E)--(F)--(G)--(B)--(C)--(F);
\draw[help lines] (C)--(G);
}
\planegraph
\foreach \i/\j/\d in {B/B'/90,C/C'/-150,C/C''/150,C/C'''/90,D/D'/180,E/E'/-90,F/F'/0,F/F''/-90} {
  \node[dartstyle] (\j) at ($(\i)+(\d:\sep ex)$) {}; }
\foreach \i/\d in {A/-60,C/0,C/-90,B/-90,D/0,E/90,F/135,G/180,G/30,G/-30} {
  \node[dartstyle] at ($(\i)+(\d:\sep ex)$) {};
}
\foreach \i/\j in {B'/C',C'/C'',C''/C''',C'''/D',D'/E',E'/F',F'/F''}
 { \draw[darrow] (\i)--(\j); }
\begin{scope}
[shift={(3*\r,0)}]
\begin{scope}
\planegraph
\foreach \i/\j/\d in {B/B'/90,D/D'/180,E/E'/-90,F/F'/0,F/F''/-90} {
  \node (\j) at ($(\i)+(\d:\sep ex)$) {}; }
\node (D'') at ($(D')+(0:1ex)$) {};
% grayfatpath.
\draw [grayfatpath,line join=round] (B')--($(D')+(0:1ex)$)--($(E)+(-90:3.2ex)$)--(F'');
%\planegraph
\end{scope}
\end{scope}
}
}

\def\figJMTLBJN{
\tikzfig{contour-comp}{\guid{JMTLBJN} The complement contour traces the remaining darts
at the same nodes as the original contour loop.  The shaded  path uses the conventions
of Figure~\ref{fig:shade-dart}.
}
{
[scale=1.0]
\def\r{1.5}
\pgfmathsetmacro\sep{3.2};
\def\planegraph{
\node (A) at (0,0) {};
\node (B) at ($(A)+(30:\r)$) {};
\node (C) at ($(A)+(90:\r)$) {};
\node (D) at ($(A)+(150:\r)$) {};
\node (A') at ($(A)+(-90:4ex)$) {};
\node (B') at ($(B)+(0:4ex)$) {};
\node (C') at ($(C)+(90:4ex)$) {};
\node (D') at ($(D)+(180:4ex)$) {};
\foreach \i in {A,B,C,D} { \graydot{\i}; };
\draw[help lines] (A)--(B)--(C)--(D)--(A);
\draw[help lines] (A)--(C);
\draw[help lines] (A)--(A');
\draw[help lines] (B)--(B');
\draw[help lines] (C)--(C');
\draw[help lines] (D)--(D');
}
\planegraph
\draw[grayfatpath] 
  ($(A)+(90:1.5ex)$)--
  ($(B)+(180:2ex)$)--
  ($(C)+(-90:1.5ex)$)--
  ($(D)+(0:2ex)$)--cycle;
\begin{scope}
[shift={(4,0)}]
\planegraph
\draw[grayfatpath] 
  ($(A)+(-90:2ex)$)--
  ($(B)+(0:2.5ex)$)--
  ($(C)+(90:2ex)$)--
  ($(D)+(180:2.5ex)$)--cycle;
\end{scope}
}
}

\def\figMSWJHND{
\tikzfig{mobius}{\guid{MSWJHND} The horizontal contour path is a M\"obius contour.
}
{
[scale=1.0]
\def\r{1.5}
\draw[line width=2ex,line cap=round,line join=round,draw=gray!40] (0,0)--(5*\r,0);
\foreach \i/\j in {A/0,B/1,C/2,D/3,E/4,F/5}
 { \node[dartstyle] (\i) at (\r*\j,0) {}; }
\foreach \i/\j in {A/B,B/C,C/D,D/E,E/F}
 { \draw[darrow] (\i)--(\j); }
\draw[darrow] (A)..controls +(30:\r) and ($(D)+(150:\r)$).. node[anchor=south] {$n$} (D);
\draw[darrow] (C)..controls +(-30:\r) and ($(F)+(-150:\r)$).. node[anchor=north] {$n$} (F);
}
}

\def\figIWKICBI{
\tikzfig{3m}{\guid{IWKICBI} For every set $D$ of cardinality three and cyclic permutation
of that set, there is a hypermap with dart set $D$ and $e=f=n$ all equal to the given
cyclic permutation.  This hypermap has a M\"obius contour $[x;f x;f^2x]$, for each $x\in D$.
}
{
[scale=1.0]
\def\r{1.5}
\node[dartstyle] (A) at (0,0) {};
\node[dartstyle] (B) at (0:\r) {};
\node[dartstyle] (C) at (60:\r) {};
%\draw[line width=2ex,line cap=round,line join=round,draw=gray!40] (A)--(B)--(C)--(A);
\foreach \i/\j in {A/B,B/C,C/A}
 { \draw[darrow] (\i)--(\j); }
}
}

\def\figOOXPORQ{
\tikzfig{mobius-strip}{\guid{OOXPORQ} A M\"obius contour of a hypermap embedded in a M\"obius strip.  
}
{
[scale=1.0]
\def\r{2.5}
\coordinate (U) at (0,0) ;
\coordinate (A) at (90:\r) ;
\coordinate (B) at (30:\r) ;
\coordinate (C) at (-30:\r) ;
\coordinate (D) at (-90:\r) ;
\coordinate (E) at (-150:\r) ;
\coordinate (F) at (150:\r) ;
\coordinate (E') at  ($(E)+(45:2ex)$);
\coordinate (B') at  ($(B)+(180:2ex)$);
%\coordinate 
\draw[rounded corners=4ex,fill=gray!20] (A)--(B)--(C)--(D)--(E)--(F)--cycle;
%\draw (E') ..controls +(80:0.5*\r) and ($(B')+(180:0.3*\r)$).. (B');
\begin{scope}
% clip  A
\clip[rounded corners=4ex] (D)--(E) 
   ..controls +(60:0.4*\r) and ($(B')+(180:0.3*\r)$).. 
   (B')--(B)--(C)--cycle;
\shade [upper left=black!60,lower left=black!20,
 upper right=gray,lower right=white]
 (-\r,-1*\r) rectangle (\r,2*\r);
\draw[rounded corners=4ex] (A)--(B)--(C)--(D)--(E)--(F)--cycle;
% clip B
\clip[rounded corners=4ex] (E)--(D)..controls +(60:0.4*\r).. (A)--(F)--cycle;
\shade [upper left=black!90,lower left=black!40,
 upper right=black!40,lower right=gray!30]
 (-\r,-\r) rectangle (\r,2*\r);
\draw[rounded corners=4ex,fill=black!60,draw=black!60] 
  ($(F)!.10!(E)$)--($(C)!0.10!(D)$)--(B)--(A)--cycle;
\end{scope}
\node[dartstyle] (G) at (140:0.4*\r) {};
\node[dartstyle] (H) at ($(G)+(30:0.3*\r)$) {};
\node[dartstyle] (I) at ($(H)+(120:0.3*\r)$) {};
\node[dartstyle] (J) at ($(G)+(120:0.3*\r)$) {};
\coordinate (J1) at ($(F)!.33!(E)$);
\coordinate (G1) at ($(F)!.66!(E)$);
\coordinate (I1) at ($(A)!.33!(B)$);
\coordinate (H1) at ($(A)!.66!(B)$);
\draw[darrow]   (J1) ..controls +(80:0.1*\r).. node[anchor=south] {$f$} (J);
\draw[darrow]   (J)-- node[anchor=south] {$f$} (I);
\draw[thick] (I1)..controls +(150:0.1*\r).. node[anchor=south,pos=0.9] {$f$} (I);
\draw[darrow] (G1)..controls +(80:0.1*\r).. node[anchor=north] {$f$} (G);
\draw[darrow] (G)-- node[anchor=north] {$f$} (H);
\draw[thick] (H1)..controls +(150:0.1*\r)..node[anchor=north,pos=0.8] {$f$} (H);
\draw[barrow] (G)-- node[anchor=east] {$n$} (J) ;
\draw[barrow] (I)-- node[anchor=west] {$n$} (H);
\coordinate (C1) at ($(C)!0.33!(D)$);
\coordinate (D1) at ($(C)!0.66!(D)$);
\begin{scope}
\clip[rounded corners=4ex] (D)--(E) 
   ..controls +(60:0.4*\r) and ($(B')+(180:0.3*\r)$).. 
   (B')--(B)--(C)--cycle;
\draw (D1)..controls +(60:0.35*\r)..(I1);
\draw (C1)..controls +(60:0.30*\r)..(H1);
\clip[rounded corners=4ex] (E)--(D)..controls +(60:0.4*\r).. (A)--(F)--cycle;
\draw[black!80] (D1)--(G1);
\draw[black!80] (C1)--(J1);
\end{scope}
}
}

\def\figWMWCTGH{
\tikzfig{quot}{\guid{WMWCTGH} When the hypermap $H$ comes from a plane graph,
we may depict the normal family $\cal L$ as a collection of
 shaded loops alongside the edges of
the graph (Figure~\ref{fig:shade-dart}).  
The subquotient is the hypermap associated with the traversed
edges of the plane graph.
}
{
[scale=1.0,rotate=90]
\def\r{1.3}
\pgfmathsetmacro\sep{3.2};
\def\planegraph{
\node (A) at (0,0) {};
\node (C) at ($(A)+(120:\r)$) {};
\node (B) at ($(C)+(-120:\r)$) {};
\node (D) at ($(C)+(60:\r)$) {};
\node (F) at ($(C)+(120:\r)$) {};
\node (E) at ($(D)+(120:\r)$) {};
\node (G) at ($(C)+(180:\r)$) {};
\node (H) at ($(G)+(-120:\r)$) {};
}
\planegraph
\foreach \i in {A,B,C,D,E,F,G,H} { \graydot{\i}; };
\draw[help lines] (A)--(C)--(D)--(E)--(F)--(G)--(B)--(C)--(F);
\draw[help lines] (C)--(G);
\draw[help lines] (G)--(H)--(A);
\foreach \i/\j/\d in {B/B'/90,D/D'/180,E/E'/-90,F/F'/0,F/F''/-90} {
  \node (\j) at ($(\i)+(\d:\sep ex)$) {}; }
\draw [grayfatpath] ($(B)+(90:2.5ex)$)--($(D)+(180:1.5ex)$)--
   ($(E)+(270:2.5ex)$)--($(G)+(0:1.5ex)$)--cycle;
\draw [grayfatpath] ($(E)+(90:2.5ex)$)--($(G)+(180:1.5ex)$)--
  ($(B)+(240:1.5ex)$)--
  ($(A)+(-30:2.5ex)$)--
  ($(C)+(0:1.5ex)$)--($(D)+(0:1.5ex)$)--cycle;
\draw[grayfatpath] ($(B)+(30:2.5ex)$)--($(A)+(150:2.5ex)$)--($(C)+(270:2.5ex)$)--cycle;
\node at (-1.5*\r,-\r) {$H/{\cal L}$};
\node at (-1.5*\r,3*\r) {$H$,~~${\cal L}$};
\begin{scope}
[shift={(0,-3.6*\r)}]
\begin{scope}
\planegraph
\foreach \i in {A,B,C,D,E,F,G} { \graydot{\i}; };
\draw[help lines] (B)--(C)--(D)--(E)--(F)--(G)--(B);
\draw[gray] (A)--(B)--(C)--(A);
\end{scope}
\end{scope}
}
}



\def\figALMINNP{
\tikzfig{graph-gen}{\guid{ALMINNP} 
This marked hypermap is described in (Example~\ref{ex:graph-gen}).  As usual, the angles
of a plane graph are the darts of the hypermap, and the permutations $e,n,f$ are
derived from the structure of the graph.  The constants $(m,p,q)$ are $(3,1,5)$,
because $y=f^{m+1}x=f^{3+1}x$, $z=f^{p+1}y=f^{1+1}y$, and the contour path from $x$ to $z$ makes
$q+1={5+1}$ steps of type $f$.
}
{
[scale=1.0,rotate=90]
\def\r{2.5}
\def\s{0.8*\r}
\def\t{0.4*\r}
\node (U) at (0,0) {};
\node (A1) at (360/7:\r) {};
\node (A2) at (2*360/7:\r) {};
\node (A3) at (3*360/7:\r) {};
\node (A4) at (4*360/7:\r) {};
\node (A5) at (5*360/7:\r) {};
\node (A6) at (6*360/7:\r) {};
\node (A7) at (7*360/7:\r) {};
\node (B1) at (26+360/7:\s) {};
\node (B2) at (26+2*360/7:\s) {};
\node (B3) at (26+3*360/7:\s) {};
\node (B4) at (26+4*360/7:\s) {};
\node (B5) at (26+5*360/7:\s) {};
\node (B6) at (26+6*360/7:\s) {};
\node (B7) at (26+7*360/7:\s) {};
\node (C1) at (1*360/7:\t) {};
%\node (C2) at (2*360/7:\t) {};
\node (C3) at (46+2*360/7:\t) {};
\node (C4) at (110+2*360/7:\t) {};
\node (C5) at (26+5*360/7:\t) {};
\draw[grayfatpath,line width = 3ex,draw=gray!30] (A1.center)--(A2.center)--(A3.center)--(A4.center)--(A5.center)--(A6.center)--(A7.center)--cycle;
\draw[fill=gray!30] (A1.center)--(A2.center)--(A3.center)--(A4.center)--(A5.center)--(A6.center)--(A7.center)--cycle;
\draw[fill=white] (B1.center)--(B2.center)--(B3.center)--(B4.center)--(B5.center)--(B6.center)--(B7.center)--cycle;
\foreach\i in {A1,A2,A3,A4,A5,A6,A7,B1,B2,B3,B4,B5,B6,B6,B7,C1,C3,C4,C5} 
 { \smalldot {\i}; };
\draw[black] (A1)--(A2)--(A3)--(A4)--(A5)--(A6)--(A7)--(A1);
\draw[black] (B1)--(B2)--(B3)--(B4)--(B5)--(B6)--(B7)--(B1);
\draw[black] (A1)--(B1)--(A2)--(B2)--(A3)--(B3)--(A4)--(B4)--(A5)--(B5)--(A6)--(B6)--(A7)--(B7)--(A1);
\draw[black] (B7)--(C1)--(B1)--(B2)--(C3)--(C4)--(B4)--(C5)--(B6);
\draw[black] (B5)--(C5);
\draw[black] (B3)--(C4);
\draw[black] (B3)--(C3);
%\draw[black] (C1)--(C2);
\draw[grayfatpath,draw=gray!30] ($(B2)+(-90:2.8ex)$)--
   ($(B3)+(0:2ex)$)--
   ($(B4)+(90:2.8ex)$)--
   ($(C4)+(180:1.5ex)$)--
   ($(C3)+(180:1.5ex)$)--cycle;
\draw[grayfatpath,draw=gray!30]
  ($(B1)+(-100:1.5ex)$)--
  ($(B2)+(-30:2.ex)$)--
  ($(C3)+(0:1.5ex)$)--
  ($(C4)+(0:1.5ex)$)--
  ($(B4)+(40:2.ex)$)--
  ($(B5)+(110:1.8ex)$)--
  ($(B6)+(135:1.8ex)$)--
  ($(B7)+(-150:1.5ex)$)--
  cycle;
\node at ($(B1)+(-100:4ex)$) {$x$};
\node at ($(B2)+(-40:5.3ex)$) {$f x$};
\node at ($(C3)+(0:3.5ex)$) {$f^2 x$};
\node at ($(C4)+(0:3.5ex)$) {$f^3 x$};
\node at ($(B4)+(35:5ex)$) {$y$};
\node at ($(C5)+(80:2ex)$) {$f y$};
\node at ($(B6)+(145:4ex)$) {$z$};
}}



\def\figKCSQIOY{
\tikzfig{transform}{\guid{KCSQIOY} 
Starting in the first frame with the marked hypermap described in (Example~\ref{ex:graph-gen}),
we take its transform to obtain the second frame, and the transform again to obtain the third frame. 
Each transform replaces the contour loop through the dart $x$ with two contour loops.
A shaded face indicates each contour loop that is confined to a face. A gray loop indicates
a multi-face contour loop.
}
{
[scale=1.0]
\def\r{1.8}
\def\s{0.8*\r}
\def\t{0.4*\r}
\def\basicframe{
\node (U) at (0,0) {};
\node (A1) at (360/7:\r) {};
\node (A2) at (2*360/7:\r) {};
\node (A3) at (3*360/7:\r) {};
\node (A4) at (4*360/7:\r) {};
\node (A5) at (5*360/7:\r) {};
\node (A6) at (6*360/7:\r) {};
\node (A7) at (7*360/7:\r) {};
\node (B1) at (26+360/7:\s) {};
\node (B2) at (26+2*360/7:\s) {};
\node (B3) at (26+3*360/7:\s) {};
\node (B4) at (26+4*360/7:\s) {};
\node (B5) at (26+5*360/7:\s) {};
\node (B6) at (26+6*360/7:\s) {};
\node (B7) at (26+7*360/7:\s) {};
\node (C1) at (1*360/7:\t) {};
%\node (C2) at (2*360/7:\t) {};
\node (C3) at (46+2*360/7:\t) {};
\node (C4) at (110+2*360/7:\t) {};
\node (C5) at (26+5*360/7:\t) {};
\draw[grayfatpath,line width = 2ex,draw=gray!30] (A1.center)--(A2.center)--(A3.center)--(A4.center)--(A5.center)--(A6.center)--(A7.center)--cycle;
\draw[fill=gray!30] (A1.center)--(A2.center)--(A3.center)--(A4.center)--(A5.center)--(A6.center)--(A7.center)--cycle;
\draw[fill=white] (B1.center)--(B2.center)--(B3.center)--(B4.center)--(B5.center)--(B6.center)--(B7.center)--cycle;
\foreach\i in {A1,A2,A3,A4,A5,A6,A7,B1,B2,B3,B4,B5,B6,B6,B7,C1,C3,C4,C5} 
 { \smalldot {\i}; };
\draw[black] (A1)--(A2)--(A3)--(A4)--(A5)--(A6)--(A7)--(A1);
\draw[black] (B1)--(B2)--(B3)--(B4)--(B5)--(B6)--(B7)--(B1);
\draw[black] (A1)--(B1)--(A2)--(B2)--(A3)--(B3)--(A4)--(B4)--(A5)--(B5)--(A6)--(B6)--(A7)--(B7)--(A1);
\draw[black] (B7)--(C1)--(B1)--(B2)--(C3)--(C4)--(B4)--(C5)--(B6);
\draw[black] (B5)--(C5);
\draw[black] (B3)--(C4);
\draw[black] (B3)--(C3);
%\draw[black] (C1)--(C2);
\draw[grayfatpath,draw=gray!30] ($(B2)+(-90:2.8ex)$)--
   ($(B3)+(0:2ex)$)--
   ($(B4)+(90:2.8ex)$)--
   ($(C4)+(180:1.5ex)$)--
   ($(C3)+(180:1.5ex)$)--cycle;
}
\begin{scope}
[shift={(0,0)},rotate=90]
\basicframe
\draw[grayfatpath,draw=gray!30]
  ($(B1)+(-100:1.5ex)$)--
  ($(B2)+(-30:2.ex)$)--
  ($(C3)+(0:1.5ex)$)--
  ($(C4)+(0:1.5ex)$)--
  ($(B4)+(40:2.ex)$)--
  ($(B5)+(110:1.8ex)$)--
  ($(B6)+(135:1.8ex)$)--
  ($(B7)+(-150:1.5ex)$)--
  cycle;
\node at ($(B1)+(-100:3ex)$) {$x$};
\node at ($(B4)+(35:5ex)$) {$y$};
\node at ($(B6)+(145:4ex)$) {$z$};
\end{scope}
\begin{scope}
[shift={(2.15*\r,0)},rotate=90]
\basicframe
\draw[grayfatpath,draw=gray!30]
  ($(B1)+(-100:1.5ex)$)--
  ($(B2)+(-30:2.ex)$)--
  ($(C3)+(0:1.5ex)$)--
  ($(C4)+(0:1.5ex)$)--
  ($(B4)+(45:2.2ex)$)--
  ($(C5)+(110:1.8ex)$)--
  ($(B6)+(135:1.8ex)$)--
  ($(B7)+(-150:1.5ex)$)--
  cycle;
\draw[grayfatpath,draw=gray!30]
  ($(B5)+(110:1.8ex)$)--
  ($(B6)+(200:3.5ex)$)--
  ($(C5)+(290:1.3ex)$)--
  ($(B4)+(5:3.5ex)$)--
  cycle;
\node at ($(B1)+(-110:3ex)$) {$x$};
\end{scope}
\begin{scope}
[shift={(4.3*\r,0)},rotate=90]
\basicframe
%\draw[fill=gray!30] (A1.center)--(A2.center)--(A3.center)--(A4.center)--(A5.center)--(A6.center)--(A7.center)--cycle;
%\draw[grayfatpath,line width = 2ex,draw=gray!30] (A1.center)--(A2.center)--(A3.center)--(A4.center)--(A5.center)--(A6.center)--(A7.center)--cycle;
\draw[fill=gray!30] (B1.center)--(B2.center)--(C3.center)--(C4.center)--
  (B4.center)--(C5.center)--(B6.center)--(B7.center)--cycle;
\foreach\i in {A1,A2,A3,A4,A5,A6,A7,B1,B2,B3,B4,B5,B6,B6,B7,C1,C3,C4,C5} 
 { \smalldot {\i}; };
%\draw[black] (A1)--(A2)--(A3)--(A4)--(A5)--(A6)--(A7)--(A1);
\draw[black] (B1)--(B2)--(B3)--(B4)--(B5)--(B6)--(B7)--(B1);
%\draw[black] (A1)--(B1)--(A2)--(B2)--(A3)--(B3)--(A4)--(B4)--(A5)--(B5)--(A6)--(B6)--(A7)--(B7)--(A1);
\draw[black] (B7)--(C1)--(B1)--(B2)--(C3)--(C4)--(B4)--(C5)--(B6);
\draw[black] (B5)--(C5);
%\draw[black] (B3)--(C4);
%\draw[black] (B3)--(C3);
\draw[grayfatpath,draw=gray!30]
  ($(B5)+(110:1.8ex)$)--
  ($(B6)+(200:3.5ex)$)--
  ($(C5)+(290:1.3ex)$)--
  ($(B4)+(5:3.5ex)$)--
  cycle;
\node at ($(B1)+(-110:3ex)$) {$x$};
\end{scope}
}}

%%%%%%%%%%%%%%%%%%%%%%%%%%%%%%%%%%%%%%%%%%%%%%%%%%%%%%%%%%%%%%%%%%%%%%%%%%%%%%%%
%% FAN CHAPTER
%%%%%%%%%%%%%%%%%%%%%%%%%%%%%%%%%%%%%%%%%%%%%%%%%%%%%%%%%%%%%%%%%%%%%%%%%%%%%%%%

\def\figIIAHJXI{
\tikzfig{fan}{\guid{IIAHJXI} A fan with six nodes and five edges.  An unbounded blade is
associated with each edge.}
{
[scale=1.0,rotate=-10]
%\draw[white] (0,0)--(1,0);
\def\r{1.8}
\coordinate (U) at (0,0);
\coordinate (A) at (-\r,\r);
\coordinate (B) at ($(A)+(-10:0.7*\r)$);
\coordinate (C) at ($(B)+(135:0.4*\r)$);
\coordinate (D) at ($(C)+(-10:0.7*\r)$);
\coordinate (E) at ($(D)+(125:0.4*\r)$);
\coordinate (F) at ($(E)+(-20:0.7*\r)$);
\draw[fill=black!45,draw=black!45] (U)--(E)--(F)--cycle;
\draw[fill=black!70,draw=black!70] (U)--(D)--(E)--cycle;
\draw[fill=black!25,draw=black!30] (U)--(C)--(D)--cycle;
\draw[fill=black!50,draw=black!50] (U)--(B)--(C)--cycle;
\draw[fill=black!10,draw=black!20] (U)--(A)--(B)--cycle;
\foreach \i \in in {A,B,C,D,E,F} { \smalldot{\i}; }
\draw[thick] (A)--(B)--(C)--(D)--(E)--(F);
\node[anchor=north] {$\orz$};
}}



\def\figSCDMRGM{
\tikzfig{Wdart}{
  \guid{SCDMRGM} The set $W=W(x,\ee)$ at the dart $x=(\v,\w)$ is the
  cone over a wedge of a disk on the unit sphere.  The geodesic radius of
  the disk is $\ee$.  If $\normo{\v}=\normo{\w}=1$, then the disk is
  centered at $\v$, and  the wedge extends counterclockwise from the
  geodesic arc of $(\v,\w)$ until it meets the next blade of the
  fan through $\v$.  The angle of this wedge at $\v$ equals $\op{azim}(x)$.
}
{
[scale=1.0]
[shift={(0,0)}]
\def\R{1.8}
\coordinate (O) at (0,0);
\pgfmathsetmacro\y{0.5*\R}
%\pgfmathsetmacro\x{0.55*\R}
\pgfmathsetmacro\x{0.50*\R}
\pgfmathsetmacro\z{sqrt(\R*\R - \y*\y - \x*\x)} % (\x,\y,\z) is a point on sphere.
\pgfmathsetmacro\rx{\y*\z/(sqrt(\x*\x + \z*\z))} % computed minor axis of elliptic projection.
%\pgfmathsetmacro\rx{0.2*\R}
%\pgfmathsetmacro\xc{\x - \rx*\rx/\x}  % tangent line from
%\pgfmathsetmacro\yc{\y * sqrt(1.0 - \rx*\rx/(\x*\x))}
%\draw[semitransparent,help lines] (-\xc,\yc) --(O) -- (-\xc,-\yc)--cycle;
\pgfmathsetmacro\Vx{\R*\x/(sqrt(\x*\x+\z*\z))} % (\Vx,\vy,\vz)=(\Vx,0,?) center of ellipse on unit sphere.
\coordinate (V) at (180:\Vx);
\coordinate (W) at (60:\Vx);
\coordinate (U) at (-60:\Vx);
\begin{scope}
\draw[ball color=gray!50,draw=gray] (0,0) circle (\R);
\path[clip] (180:\x) circle[x radius = \rx,y radius = \y]; % 180 turn.
\draw[ball color=gray] (0,0) circle (\R);
\end{scope}
\begin{scope}
\setellipseplane{\R}{-\Vx}{0}{\Vx*cos(60)}{\Vx*sin(60)}
\clip[ellipse plane] (0,0) circle (1);
\setellipseplane{\R}{\Vx*cos(60)}{-\Vx*sin(60)}{-\Vx}{0}
\clip[ellipse plane] (0,0) circle (1);
\draw[ball color=gray!50,draw=gray] (0,0) circle (\R);
\end{scope}
%\path [draw,help lines] (180:\x) circle[x radius = \rx,y radius = \y];
\foreach \i in {V,W,U} { \smalldot{\i}; }
\begin{scope}
\setellipseplane{\R}{-\Vx}{0}{\Vx*cos(60)}{\Vx*sin(60)}
\clip (0,0) --(V)--(120:\R)--(W)--cycle;
\draw[ellipse plane] (0,0) circle (1);
\end{scope}
\begin{scope}
\setellipseplane{\R}{\Vx*cos(60)}{-\Vx*sin(60)}{-\Vx}{0}
\clip (0,0)--(U)--(-120:\R)--(V)--cycle;
\draw[ellipse plane] (0,0) circle (1);
\end{scope}
\begin{scope}
\clip (0,0)--(W)--(0:\R)--(U)--cycle;
\setellipseplane{\R}{\Vx*cos(60)}{-\Vx*sin(60)}{\Vx*cos(60)}{\Vx*sin(60)}
%\draw[ellipse plane] (0,0) circle (1);
\end{scope}
\node[anchor=north] at (V) {$\v$};
\node[anchor=south west] at (W) {$\w$};
\node at ($(V)+(90:3ex)$) {$W$};
\draw[decorate,decoration={brace,mirror}] (V)-- node[below=0.35ex] {$~\,\ee$} ($(V)!0.5!(W)!0.04!(120:\R)$);
}}

\def\figJCAEBKL{
\tikzfig{vt}{\guid{JCAEBKL} As a point $\w(t)$ slides from $\v_0$ to $\v_1$, the blade $C^0\{\v,\w(t)\}$
avoids the blades of the fan $(V,E)$, when $0< t<1$. }
{
[scale=1.0,rotate=20]
\def\r{1.5}
\coordinate (U) at (0,0);
\coordinate (V) at (70:1*\r);
\coordinate (V1) at ($(V)+(100:\r)$);
\coordinate (V0) at ($(V)+(20:\r)$);
\coordinate (Wt) at ($(V0)!0.3!(V1)$);
\draw[fill=black!60] (U)--(V0)--(V1)--cycle;
\begin{scope}
\clip (U)--(V0)--(V1)--cycle;
\shade [upper left=black!20,lower left=black!70,upper right=black!30,lower right=black!50]
 (V1) rectangle (V -| V0);
\end{scope}
\begin{scope}
\clip (U)--(Wt)--(V)--cycle;
\shade [upper left=black!100,lower left=black!90,upper right=black!80,lower right=black!50]
 (V) rectangle (Wt);
\end{scope}
\begin{scope}
\clip (U)--(V)--(V0)--cycle;
\shade [upper left=black!30,lower left=black!30,upper right=black!20,lower right=black!100]
 (U) rectangle (V0);
\end{scope}
%\draw[fill=black!80] (U)--(Wt)--(V)--cycle;
\draw[fill=black!10] (U)--(V)--(V1)--cycle;
%\draw[fill=black!25] (U)--(V)--(V0)--cycle;
\draw[thick] (V)--(Wt);
\foreach \i in {V,V1,V0} { \smalldot {\i}; }
\draw[thick] (V)--(V0)--(V1);
\draw[fill=white,draw=black] (Wt) circle (1.5pt);
\node[anchor=north] at (U) {$\orz$};
\node[anchor=east] at (V1) {$t=1,~\v_1$};
\node[anchor=west] at (V0) {$\v_0,~t=0$};
\node[anchor=south] at (Wt) {$\w(t)$};
\node[anchor=east] at (V) {$\v$};
}
}

\def\figRNMBPJD{
\tikzfig{reduction}{\guid{RNMBPJD} To prove results about fans by induction, we add a blade $C^0\{\v,\w\}$
between $\v$ and $\w$ to a fan to form a new fan with smaller invariant $N$.  This partitions one of the
 topological components $U$ into $U(x')$, $U(y')$, and the new blade
$C^0\{\v,\w\}$.  We have darts $x'$ and $y'$ of the new hypermap, leading into $U(x')$ and $U(y')$, respectively.
The face containing the dart $y'$ in the new hypermap is a triangle.}
{
[scale=1.0]
\def\r{2.8}
\coordinate (U) at (0,0);
\coordinate (V1) at (115:\r);
\coordinate (V2) at (100:1.3*\r);
\coordinate (V3) at (80:1.4*\r);
\coordinate (V4) at (55:1.2*\r);
\coordinate (V5) at (65:0.7*\r);
\draw[fill=black!30] (U)--(V1)--(V2)--cycle;
\draw[fill=black!50] (U)--(V2)--(V3)--cycle;
\draw[fill=black!40] (U)--(V3)--(V4)--cycle;
\draw[fill=black!20] (U)--(V5)--(V1)--cycle;
\draw[fill=gray!60,ultra nearly transparent,thin,dotted] (V5)--(V1)--(V4)--cycle;
\begin{scope}
\clip (V1)--(V5)--(V4)--($(V5)+(90:0.2*\r)$)--cycle;
\shade [upper left=black!20,lower left=black!60,upper right=gray!60,lower right=black!70]
 (V1 |- V5) rectangle (V4);
\end{scope}
\draw[fill=black!40] (U)--(V5)--(V4)--cycle;
\draw[thick] (V1)--(V2)--(V3)--(V4)--(V5)--cycle;
\foreach \i in {V1,V2,V3,V4,V5} { \smalldot{\i}; }
\node[anchor=north] at (U) {$\orz$};
\node[anchor=east] at (V1) {$\v$};
\node[anchor=west] at (V4) {$\w$};
\node at ($(V4)+(-150:3.5ex)$) {$y'$};
\node at ($(V1)+(33:2.7ex)$) {$x'$};
}
}

\def\tttLSG#1#2#3{
 \draw[darrow] (#1)--(#2);
 \draw[darrow] (#2)--(#3);
 \draw[darrow] (#3)--(#1);
}

\def\qqqLSG#1#2#3#4{
 \draw[darrow] (#1)--(#2);
 \draw[darrow] (#2)--(#3);
 \draw[darrow] (#3)--(#4);
 \draw[darrow] (#4)--(#1);
}


\def\figLSGYGPA{
  \tikzfig{polyfan}{\guid{LSGYGPA} A bounded polyhedron $P$, its fan
    $(V_P,E_P)$, and the face permutation of the front half of its
    hypermap $\op{hyp}(V_P,E_P)$.  The set of facets of $P$ is in
    bijection with the set of topological components of $Y(V_P,E_P)$
    and with the set of faces of the hypermap.  The set of edges of a
    single facet $F$ of $P$ is in bijection with (the set of darts in)
    the corresponding face of the hypermap.}
{
[scale=1.0,z={(-0.2,-0.2)}]
\def\r{1.4}
\def\coords{
\pgfmathsetmacro\z{sqrt(1/12)*\r}
\pgfmathsetmacro\y{sqrt(2/3)*\r}
\coordinate (U) at (0,0,0);
\coordinate (H1) at (\r,0,0);
\coordinate (H2) at (0.5*\r,0,{sqrt(3/4) * \r});
\coordinate (H3) at (-0.5*\r,0,{sqrt(3/4)*\r});
\coordinate (H4) at (-\r,0,0);
\coordinate (H5) at (-0.5*\r,0,{-sqrt(3/4) * \r});
\coordinate (H6) at (0.5*\r,0,{-sqrt(3/4)*\r});
\coordinate (B1) at (0,\y,{sqrt(1/3)*\r});
\coordinate (B2) at (0.5*\r,\y,-\z);
\coordinate (B3) at (-0.5*\r,\y,-\z);
\coordinate (C2) at (-0.5*\r,{-\y},{\z});
\coordinate (C3) at (0.5*\r,{-\y},{\z});
}
%
\begin{scope}
\coords
\draw[fill=black!60] (H1)--(H2)--(C3)--cycle;
\draw[fill=black!40] (H3)--(C2)--(H4)--cycle;
\draw[fill=black!50] (H2)--(H3)--(C2)--(C3)--cycle;
\draw[fill=black!30] (B1)--(H2)--(H3)--cycle;
\draw[fill=black!10] (B1)--(H3)--(H4)--(B3)--cycle;
\draw[fill=black!40] (B2)--(H1)--(H2)--(B1)--cycle;
\draw[fill=black!20] (B3)--(B2)--(B1)--cycle;
\end{scope}
\begin{scope}
[shift={(2.5*\r,0)}]
\coords
\tikzset{plane3/.style={fill=black!70,draw=black!70}}
\tikzset{plane2/.style={fill=black!45,draw=black!50}}
\draw[fill=black!10,thick] (B3)--(B2)--(H1)--(C3)--(C2)--(H4)--cycle; % background plane
\draw[plane2] (U)--(C2)--(H3)--cycle; % lower plane 2
\draw[plane3] (U)--(H2)--(C3)--cycle; % lower 3.
\draw[fill=black!30,draw=black!30] (H1)--(H2)--(H3)--(H4)--cycle; % H plane
\draw[plane3] (U)--(B3)--(B1)--cycle; % upper 3
\draw[plane2] (B2)--(B1)--(H3)--(U)--cycle; % upper plane 2
\draw[plane3] (U)--(B1)--(H2)--cycle; % mid 3.
\foreach \i in {H1,H2,H3,H4,B1,B2,B3,C2,C3} { \smalldot {\i}; }
\draw[thick] (B3)--(B1)--(H2)--(C3);
\draw[thick] (B2)--(B1)--(H3)--(C2);
\draw[thick] (H1)--(H2)--(H3)--(H4);
\end{scope}
\begin{scope}
[shift={(5.0*\r,0)}]
\coords
%\draw[help lines] (B3)--(B2)--(H1)--(C3)--(C2)--(H4)--cycle; % background plane
%\foreach \i in {H1,H2,H3,H4,B1,B2,B3,C2,C3} { \graydot {\i}; }
\foreach \name/\base/\theta in { aH2/H2/-30, aC3/C3/-30, aH1/H1/-30,
   bC2/C2/-45, bC3/C3/-135, bH2/H2/-135, bH3/H3/-45,
   cC2/C2/-130, cH3/H3/-130, cH4/H4/-130,
   dH1/H1/80, dH2/H2/80, dB1/B1/-20, dB2/B2/-20,
   eH2/H2/150, eH3/H3/30, eB1/B1/-90,
   fH3/H3/100, fB1/B1/-170, fB3/B3/-170, fH4/H4/100,
   gB1/B1/70, gB2/B2/70, gB3/B3/70 }
   \node[dartstyle] (\name) at ($(\base)+(\theta:1.3ex)$) {};
\tttLSG{aH2}{aC3}{aH1}
\tttLSG{gB1}{gB2}{gB3}
\tttLSG{eH2}{eB1}{eH3}
\tttLSG{cH3}{cH4}{cC2}
\qqqLSG{bC3}{bH2}{bH3}{bC2}
\qqqLSG{dH1}{dB2}{dB1}{dH2}
\qqqLSG{fB1}{fB3}{fH4}{fH3}
\end{scope}
}}

%%%%%%%%%%%%%%%%%%%%%%%%%%%%%%%%%%%%%%%%%%%%%%%%%%%%%%%%%%%%%%%%%%%%%%%%%%%%%%%%
%% PACKING CHAPTER
%%%%%%%%%%%%%%%%%%%%%%%%%%%%%%%%%%%%%%%%%%%%%%%%%%%%%%%%%%%%%%%%%%%%%%%%%%%%%%%%

%%%%%%%%%%%%%%%%%%%%%%%%%%%%%%%%%%%%%%%%%%%%%%%%%%%%%%%%%%%%%%%%%%%%%%%%%%%%%%%%
%% PACKING CHAPTER  FIGURES
%%%%%%%%%%%%%%%%%%%%%%%%%%%%%%%%%%%%%%%%%%%%%%%%%%%%%%%%%%%%%%%%%%%%%%%%%%%%%%%%



\def\figTULIGLY{
\tikzfig{M}{\guid{TULIGLY} The quartic polynomial $M$.}
{
[xscale=10.0,yscale=2.0]
\draw[help lines,->] (1.0,0) -- (1.5,0);
\draw[help lines,->] (1,0) -- (1,1.2);
\draw[help lines] (1.414,0)-- (1.414,1);
\draw[help lines] (1,1)--(1.414,1);
\draw[help lines] (1.3254,0) node [black,anchor=north] {$~~h_+\phantom{\sqrt2}$} --(1.3254,1);
\node[anchor=north] (C) at (1,0) {$1$};
\node[anchor=east] (O) at (1,0) {$0$};
\node[anchor=east] (A) at (1,1) {$1$};
\node[anchor=south] at (1.15,0.4) {$M$};
\node[anchor=north] (B) at (1.41,0) {$\sqrt2$};
\draw plot[smooth] file {tikz/TULIGLY.table};
}}

\def\figBJLIEKB{
\tikzfig{L}{\guid{BJLIEKB} Detail of the quartic $M$ and piecewise linear function $L$ on
the domain $\leftclosed1.2,1.35\rightclosed$.}
{
[scale=12.0]
\draw plot[smooth] file {tikz/BJLIEKB.table};
\draw[help lines,<->] (1.18,0) -- (1.37,0);
\draw[help lines,<->] (1.2,-0.01) -- (1.2,0.25);
\node[anchor=east] at (1.2,0.18) {$M$};
\node[anchor=east] at (1.2,0.23) {$L$};
\draw[help lines] (1.23175,0.25) -- (1.23175,-0.01) node[anchor=north,black] {$h_-$};
\draw[help lines] (1.26,0.25) -- (1.26,-0.01) node[anchor=north,black] {$~~\hm$};
\draw[help lines] (1.3254,0.25) -- (1.3254,-0.01) node[anchor=north,black] {$h_+$};
\draw (1.2,0.230769 ) -- (1.26,0);  %
\draw (1.26,0) -- (1.35,0);
}}
%


\def\figJXEHXQY{
\tikzfig{fg}{\guid{JXEHXQY} The functions $g$ takes negative values,
but the function $f$ remains positive, as predicted by the cell cluster
inequality.  The nondifferentiability at $2h_0$ is inherited from the
nondifferentiability of $L$.}
{
[xscale=12.0,yscale=100.0]
\draw plot[smooth] file {tikz/jxehxqy1a.table};
\draw plot[smooth] file {tikz/jxehxqy1b.table} node[anchor=west,black] {$f$};
\draw plot[smooth] file {tikz/jxehxqy2a.table};
\draw plot[smooth] file {tikz/jxehxqy2b.table} node[anchor=west,black]{$g$};
\draw[help lines,<->] (2.46,0) -- (2.66,0);
%\draw[help lines] (2.4635,-0.004) -- (2.4635,0.01);
\draw[help lines] (2.4635,0.01) -- (2.4635,-0.004) node[anchor=north,black] {$2h_-$};
\draw[help lines] (2.6508,0.01) -- (2.4635,0.01) node[anchor=east,black] {$0.01$};
\draw[help lines] (2.6508,0.00) -- (2.4635,0.00) node[anchor=east,black] {$0.00$};
\draw[help lines] (2.52,0.01) -- (2.52,-0.004) node[anchor=north,black] {$2\hm$};
\draw[help lines] (2.6508,0.01) -- (2.6508,-0.004) node[anchor=north,black] {$2h_+$};
}}
%



\def\figPQFEXQN{
\tikzfig{fg1}{\guid{PQFEXQN} The functions $\beta_0$.}
{
[xscale=20.0,yscale=150.0]
\draw plot[smooth] file {tikz/pqfexqn.table};
\draw[help lines,<->] (1.23,0) -- (1.33,0);
%\draw[help lines] (1.2318,-0.004) -- (1.2318,0.01);
\draw[help lines] (1.2318,0.005) -- (1.2318,-0.002) node[anchor=north,black] {$h_-$};
\draw[help lines] (1.3254,0.005) -- (1.2318,0.005) node[anchor=east,black] {$0.005$};
\draw[help lines] (1.3254,0.00) -- (1.2318,0.00) node[anchor=east,black] {$0.000$};
\draw[help lines] (1.26,0.005) -- (1.26,-0.002) node[anchor=north,black] {$\hm$};
\draw[help lines] (1.3254,0.005) -- (1.3254,-0.002) node[anchor=north,black] {$h_+$};
}}
%

\def\figZXEVDCAdeprecated{
\tikzfig{marchal-polyhedron}{ jjj
}
{
%[scale=1.0,z={(0.0,-0.7)},y={(0.0,0.7)},x={(1,0.0)}] %  polyhedron of a regular dodec.
[scale=1.0,z={(0,0)}]
\def\an{10}
\begin{scope}
%[cm={cos(\an),sin(\an),sin(-\an),cos(\an),(0,0)}]
\pgfmathsetmacro\d{2.10292} % 20 Solid[2,2,2,d,d,d]  = 4Pi.
\pgfmathsetmacro\theta{1.107} % theta = arc(2,2,d)
\coordinate (U) at (0,0,0);
\coordinate (V0) at (1.73205, 0, 0); % (sqrt3,0,0); % northpole
% first ring.
\coordinate (V1) at (0.774597, 0.478727, 1.47337);
\coordinate (V2) at (0.774597, -1.25332, 0.910593);
\coordinate (V3) at (0.774597, -1.25332, -0.910593);
\coordinate (V4) at (0.774597, 0.478727, -1.47337);
\coordinate (V5) at (0.774597, 1.54919, 0);
% bottom
\coordinate (W0) at (-1.73205, 0, 0); % -(sqrt3,0,0);
\coordinate (W1) at (-0.774597, -0.478727, -1.47337);
\coordinate (W2) at (-0.774597, 1.25332, -0.910593);
\coordinate (W3) at (-0.774597, 1.25332, 0.910593);
\coordinate (W4) at (-0.774597, -0.478727, 1.47337);
\coordinate (W5) at (-0.774597, -1.54919, 0);
% circumcenters
\coordinate (c015) at (0.866025, 0.535233, 0.38887);
\coordinate (c012) at (0.866025, -0.204441, 0.629204);
\coordinate (c023) at (0.866025, -0.661585, 0.);
\coordinate (c034) at (0.866025, -0.204441, -0.629204);
\coordinate (c045) at (0.866025, 0.535233, -0.38887);
%
\coordinate (c12) at (0.204441, -0.330792, 1.01807);
\coordinate (c23) at (0.20444086553483118,-1.0704662693192695,0);
\coordinate (c34) at (0.204441, -0.330792, -1.01807);
\coordinate (c45) at (0.204441, 0.866025, -0.629204);
\coordinate (c15) at (0.204441, 0.866025, 0.629204);
%
\coordinate (d015) at ($(U)-(c015)$);
\coordinate (d012) at ($(U)-(c012)$);
\coordinate (d023) at ($(U)-(c023)$);
\coordinate (d034) at ($(U)-(c034)$);
\coordinate (d045) at ($(U)-(c045)$);
%
\coordinate (d12) at ($(U)-(c12)$);
\coordinate (d23) at ($(U)-(c23)$);
\coordinate (d34) at ($(U)-(c34)$);
\coordinate (d45) at ($(U)-(c45)$);
\coordinate (d15) at ($(U)-(c15)$);
%
\foreach \i in {V0,V1,V2,V3,V4,V5,W0,W1,W2,W3,W4,W5,
  c015,c012,c023,c034,c045,c12,c23,c34,c45,c15,d015,d012,d023,d034,d045,d12,d23,d34,d45,d15}
  {\smalldot{\i}; \node[anchor=south] at (\i) {\i}; }
\foreach \i/\j in { V0/V1,V0/V2,V0/V3,V0/V4,V0/V5,V1/V2,V2/V3,V3/V4,V4/V5,V5/V1,V2/W5,V3/W1,V4/W2,V5/W3,V1/W4,
   W0/W1,W0/W2,W0/W3,W0/W4,W0/W5,W1/W2,W2/W3,W3/W4,W4/W5,W5/W1,W1/V4,W2/V5,W3/V1,W4/V2,W5/V3}
  { \draw[thick] (\i)--(\j); }
%\draw (0,0)--(1,0);
%\draw[ball color=gray!10,shading=ball,semitransparent] (0,0) circle (1);
%%\draw[fill=gray,thick] (V1)--(V5)--(V0)--cycle;
%%\draw[fill=gray,thick] (V1)--(V2)--(V0)--cycle;
%\draw[fill=gray,thick] (V2)--(V3)--(V0)--cycle;
%%\draw[fill=gray,thick] (V2)--(W4)--(V1)--cycle;
%\draw[fill=gray,thick] (V2)--(W4)--(W5)--cycle;
%%\draw[fill=gray,thick] (V1)--(W4)--(W3)--cycle;
%\draw[fill=gray,thick] (W4)--(W3)--(W0)--cycle;
%%\draw[fill=gray,thick] (V1)--(W3)--(V5)--cycle;
%\draw[fill=gray,thick] (W3)--(V5)--(W2)--cycle;
%%\draw[fill=gray,thick,transparent] (W3)--(W4)--(W0)--cycle;
%%\draw[fill=gray,thick] (W3)--(W0)--(W2)--cycle;
%%\draw[fill=gray,thick] (W3)--(W2)--(V5)--cycle;
%%\draw[fill=gray,thick] (W2)--(V5)--(V4)--cycle;
%%\draw[fill=gray,thick] (V5)--(V4)--(V0)--cycle;
 \draw[ball color=gray!10,shading=ball,semitransparent] (0,0) circle (1); % ball
 % \draw[fill=white,transparent] (c15) --(d23) --(c45)--(c045)--(c015)--cycle; % hidden 
%\draw[fill=white,transparent] (d23)--(c45)--(d12)--(d012)--(d023)--cycle; % hidden top
\draw[fill=white,transparent] (d23)--(c15)--(d34)--(d034)--(d023)--cycle; % upper left
\draw[fill=white,transparent] (c15)--(d34)--(c12)--(c012)--(c015)--cycle; % upper right
\draw[fill=white,transparent] (d34)--(d034)--(d045)--(d45)--(c12)--cycle; % lower left
\draw[fill=white,transparent] (c012)--(c12)--(d45)--(c23)--(c023)--cycle; % lower right
%\draw[fill=white,transparent] (c012)--(c023)--(c034)--(c045)--(c015)--cycle; % right
\foreach \i in {V0,V1,V2,V3,V4,V5,W0,W1,W2,W3,W4,W5}
%    c015,c012,c023,c034,c045,c12,c23,c34,c45,c15,d015,d012,d023,d034,d045,d12,d23,d34,d45,d15}
  {\smalldot{\i}; \node[anchor=south] at (\i) {\i}; }
%\foreach \i in {V0,V1,V2,V3,V4,V5,W0,W1,W2,W3,W4,W5,
%    c015,c012,c023,c034,c045,c12,c23,c34,c45,c15,d015,d012,d023,d034,d045,d12,d23,d34,d45,d15}
%  {\smalldot{\i}; }
\end{scope}
}
}

\def\figDEQCVQL{
\tikzfig{saturated}{\guid{DEQCVQL} A random packing of disks in $\ring{R}^2$
and a saturated extension.
}
{
[scale=1.0,z={(0,0)}]
\autoDEQCVQL
}
}

\def\figXOHAZWO{
\tikzfig{voronoi}{\guid{XOHAZWO} Voronoi cells of a packing in
$\ring{R}^2$.
}
{
[scale=1.3,z={(0,0)}]
\autoXOHAZWO
}
}


\def\figKFETCJS{
\tikzfig{vset}{\guid{KFETCJS}   We use lists $[\u_0;\cdots;\u_k]$ of points in a packing to
select a Voronoi cell (shaded), one of its edges (thick segment), 
and one of its extreme points (white dot).
 For simplicity, we illustrate the two-dimensional analogue.
}
{
[scale=1.0,z={(0,0)}]
\def\ff{
\pgfmathsetmacro\r{0.55};
\pgfmathsetmacro\sq{1.732*\r};
\coordinate (A) at (0:\r);
\coordinate (B) at (60:\r);
\coordinate (C) at (120:\r);
\coordinate (D) at (180:\r);
\coordinate (E) at (240:\r);
\coordinate (F) at (300:\r);
\coordinate (U0) at (0,0);
\coordinate (U2) at (30:\sq);
\coordinate (U1) at (90:\sq);
\draw[gray] (A)--(B)--(C)--(D)--(E)--(F)--cycle;
\smalldot {U0};
\smalldot {U1};
\smalldot {U2};
\foreach \i in {150,210,270,330} 
  {\coordinate (V\i) at (\i:\sq); \smalldot {V\i}; };
}
\begin{scope}[xshift=0,yshift=0]
\ff
\fill[gray,draw=black] (A)--(B) -- (C) --(D)--(E)--(F)--cycle;
\smalldot {U0};
\node[anchor=north] at (U0) {$\u_0$};
\node[anchor=north] at (270:\sq) {$\Omega(V,[\u_0])$};
\end{scope}
\begin{scope}[xshift=3cm,yshift=0]
\ff
\draw[ultra thick,black] (B)-- (C);
\node[anchor=north] at (270:\sq) {$\Omega(V,[\u_0;\u_1])$};
\node[anchor=north] at (U0) {$\u_0$};
\node[anchor=south] at (U1) {$\u_1$};
\end{scope}
\begin{scope}[xshift=6cm,yshift=0]
\ff
\whitedot{B};
\node[anchor=north] at (U0) {$\u_0$};
\node[anchor=south] at (U1) {$\u_1$};
\node[anchor=south west] at (U2) {$\u_2$};
\node[anchor=north] at (270:\sq) {$\Omega(V,[\u_0;\u_1;\u_2])$};
\end{scope}
}
}



\def\figHFFTUNW{
\tikzfig{rogers-omega}{\guid{HFFTUNW} The points $\omega_j$ are constructed
in a nested sequence of faces of a Voronoi cell.
 For simplicity, we illustrate the two-dimensional analogue.  The hexagon
is a Voronoi cell about $\u_0$ and $\omega_j=\omega_j(V,[\u_0;\u_1;\u_2])$.
}
{
[scale=1.0,z={(0,0)}]
\pgfmathsetmacro\r{1.2};
\pgfmathsetmacro\sq{1.732*\r};
\coordinate (A) at (0:\r);
\coordinate (B) at (60:\r);
\coordinate (C) at (120:\r);
\coordinate (D) at (180:\r);
\coordinate (E) at (240:\r);
\coordinate (F) at (300:\r);
\coordinate (U0) at (0,0);
\coordinate (U2) at (30:\sq);
\coordinate (U1) at (90:\sq);
\coordinate (W1) at (90:\sq*0.5);
\draw[gray] (A)--(B)--(C)--(D)--(E)--(F)--cycle;
\smalldot {U0};
\smalldot {U1};
\smalldot {U2};
\node[anchor=north] at (U0) {~~~~$\omega_0\!=\!\u_0$};
\node[anchor=south] at (U1) {$\u_1$};
\node[anchor=south west] at (U2) {$\u_2$};
\whitedot {U0};
\whitedot {B};
\whitedot {W1};
\node[anchor=south] at (W1) {$\omega_1$};
\node[anchor=south west] at (B) {$\omega_2$};
}
}

\def\figBUGZBTW{
\tikzfig{rogers-random}{\guid{BUGZBTW} The Rogers partition of a packing.
 For simplicity, we illustrate the two-dimensional analogue.  Heavy
edges are facets of Voronoi cells.  The Rogers simplices that are not right triangles
are shaded.
}
{
[scale=2.5,z={(0,0)}]
\autoBUGZBTW
}
}


\def\figNOCHOTB{
\tikzfig{rogers-sqrt2}{\guid{NOCHOTB} 
There are eight two-dimensional Rogers simplices
of diameter $\sqrt2$
that meet at the center of a square.
Whenever the diameter of the Rogers simplex
is less than $\sqrt2$, the simplex is one of exactly 
six that meet at an extreme point of a Voronoi cell.
 Heavy edges are facets of Voronoi cells.  
}
{
[scale=1.0,z={(0,0)}]
\pgfmathsetmacro\r{1.414};
\pgfmathsetmacro\s{1.1547};
\draw (45:\r)--(135:\r)--(225:\r)--(315:\r)--cycle;
\foreach \i in {0,90,180,270} {\draw[very thick] (0,0) -- (\i:\r); }
\foreach \i in {45,135,225,315} {
  \coordinate (V\i) at (\i:\r);
  \draw (0,0) -- (V\i); 
  \smalldot{V\i};
  }
\node at (35:\r * 0.48) {$\sqrt2$} ;
\begin{scope}
[xshift=3.0cm]
\draw (-30:\s)--(90:\s)--(210:\s)--cycle;
\foreach \i in {30,150,270} {\draw[very thick] (0,0) -- (\i:1.3); }
\foreach \i in {-30,90,210} {
  \coordinate (V\i) at (\i:\s);
  \draw (0,0) -- (V\i); 
  \smalldot{V\i};
  }
\end{scope}
}
}

\def\figKVIVUOT{
\tikzfig{marchal-variety}{\guid{KVIVUOT} 
A cell
can be visualized by intersecting it with a Rogers simplex.
The Rogers simplex is drawn as an orthosimplex formed from
four extreme points (white dots) of a rectangle. 
The shape of the intersection $\cell(\bu,k)\cap R(\bu)$  (dark gray) 
depends on the relationship
between $h_i=h(d_i\bu)$ and $\sqrt2$.  The constants $h_1$, $h_2$,
and $h_3$ are the distances from the lower front left of the rectangle to
the upper front left, upper front right, and upper back right, respectively.
Each column features a particular
Rogers simplex, and the cells in each column partition
the Rogers simplex.  The empty cells are not illustrated.
}
{
[scale=0.92,z={(0,0)}]
%  \pgfmathsetmacro\rho{0.75};
  \pgfmathsetmacro\r{1.414};
\def\simplexA{
  \pgfmathsetmacro\a{1.5};
  \pgfmathsetmacro\b{2.0};
  \pgfmathsetmacro\bb{sqrt(\b*\b-\a*\a)};
  \pgfmathsetmacro\c{\b+0.2};
  \pgfmathsetmacro\cc{sqrt(\c*\c-\b*\b)};
  \coordinate (rho) at (0.33,0.66);
  \coordinate (U) at (0,0);
  \coordinate (omega1) at (90:\a);
  \coordinate (omega2) at (\bb,\a);
  \coordinate (omega3) at ($(omega2)+ \cc*(rho)$);
  \coordinate (omega4) at ($(omega1)+ \cc*(rho)$);
  \coordinate (p2) at (\bb,0);
  \coordinate (p3) at ($(p2)+\cc*(rho)$);
  \coordinate (p4) at ($\cc*(rho)$);
}
\def\simplexB{
  \pgfmathsetmacro\a{1};
  \pgfmathsetmacro\b{2.0};
  \pgfmathsetmacro\bb{sqrt(\b*\b-\a*\a)};
  \pgfmathsetmacro\c{\b+0.2};
  \pgfmathsetmacro\cc{sqrt(\c*\c-\b*\b)};
  \coordinate (rho) at (0.33,0.66);
  \coordinate (U) at (0,0);
  \coordinate (omega1) at (90:\a);
  \coordinate (omega2) at (\bb,\a);
  \coordinate (omega3) at ($(omega2)+ \cc*(rho)$);
  \coordinate (omega4) at ($(omega1)+ \cc*(rho)$);
  \coordinate (p2) at (\bb,0);
  \coordinate (p3) at ($(p2)+\cc*(rho)$);
  \coordinate (p4) at ($\cc*(rho)$);
}
\def\simplexC{
  \pgfmathsetmacro\a{1};
  \pgfmathsetmacro\b{1.35};
  \pgfmathsetmacro\bb{sqrt(\b*\b-\a*\a)};
  \pgfmathsetmacro\c{2.0};
  \pgfmathsetmacro\cc{sqrt(\c*\c-\b*\b)};
  \coordinate (rho) at (0.33,0.66);
  \coordinate (U) at (0,0);
  \coordinate (omega1) at (90:\a);
  \coordinate (omega2) at (\bb,\a);
  \coordinate (omega3) at ($(omega2)+ \cc*(rho)$);
  \coordinate (omega4) at ($(omega1)+ \cc*(rho)$);
  \coordinate (p2) at (\bb,0);
  \coordinate (p3) at ($(p2)+\cc*(rho)$);
  \coordinate (p4) at ($\cc*(rho)$);
}
\def\simplexD{
  \pgfmathsetmacro\a{1};
  \pgfmathsetmacro\b{1.2};
  \pgfmathsetmacro\bb{sqrt(\b*\b-\a*\a)};
  \pgfmathsetmacro\c{1.4};
  \pgfmathsetmacro\cc{sqrt(\c*\c-\b*\b)};
  \coordinate (rho) at (0.33,0.66);
  \coordinate (U) at (0,0);
  \coordinate (omega1) at (90:\a);
  \coordinate (omega2) at (\bb,\a);
  \coordinate (omega3) at ($(omega2)+ \cc*(rho)$);
  \coordinate (omega4) at ($(omega1)+ \cc*(rho)$);
  \coordinate (p2) at (\bb,0);
  \coordinate (p3) at ($(p2)+\cc*(rho)$);
  \coordinate (p4) at ($\cc*(rho)$);
}
\def\drawB{
\draw[fill=gray!25,draw=gray] (U)--(p2)--(omega2)--cycle;
\draw[fill=gray!30,draw=gray] (p2)--(p3)--(omega3)--(omega2)--cycle;
\draw[fill=gray!20,draw=gray] (omega1)--(omega3)--(omega4)--cycle;
\draw[thick] (U)--(omega3);
\draw[fill=gray!35,thick] (U)--(omega1)--(omega3)--cycle;
\draw[fill=gray!40,thick] (U)--(omega3)--(omega2)--cycle;
}
\autoKVIVUOT
\begin{scope} % 1st column, 0-cell
\node at (75:2.0*\r) {$\sqrt2\le h_1$;};
\simplexA
\drawB
\draw[thick,fill=gray] (omega3) --(omega1)--plot[smooth] coordinates{\kvoneab}--(omega2)--cycle;
\draw[thick] (omega1)--(omega2);
\whitedot{U};
\whitedot{omega1};
\whitedot{omega2};
\whitedot{omega3};
\node at (140:\r) {$0$-cell:};
\end{scope}
% 1st column, 1-cell
\begin{scope}[shift={(0,-2.5)}]
\simplexA
\drawB
\draw[thick,fill=gray] (U)--plot[smooth] coordinates{\kvoneac}
 --plot[smooth] coordinates {\kvonecb}--cycle;
\draw[thick] (U)-- (omega1)--(omega3)--(omega2)--cycle;
\draw (omega1)--(omega2);
\draw[thick] plot[smooth] coordinates{\kvoneab};
\whitedot{U};
\whitedot{omega1};
\whitedot{omega2};
\whitedot{omega3};
\node at (140:\r) {$1$-cell:};
\end{scope}
\begin{scope}[shift={(0,-5.0)}]
\node at (140:\r) {$2$-cell:};
\end{scope}
\begin{scope}[shift={(0,-7.5)}]
\node at (140:\r) {$3$-cell:};
\end{scope}
\begin{scope}[shift={(0,-10.0)}]
\node at (140:\r) {$4$-cell:};
\end{scope}
\begin{scope}[shift={(2.5,0)}] % 2st column, 0-cell
\node at (75:2.0*\r) {$h_1\! <\!\!\sqrt2\le h_2$;};
\simplexB
\drawB
\draw[thick,fill=gray!80] (omega3) --(omega2)
  --plot[smooth] coordinates{\kvBda}--cycle;
\draw[] (omega1)--(omega2);
\draw[thick,fill=gray] (omega2)
  --plot[smooth] coordinates{\kvBcd} --cycle;
\whitedot{U};
\whitedot{omega1};
\whitedot{omega2};
\whitedot{omega3};
\end{scope}
\begin{scope}[shift={(2.5,-2.5)}] % 2st column, 1-cell
\simplexB
\drawB
\draw[thick,fill=gray!80] (U)
  --plot[smooth] coordinates{\kvBab}
  --plot[smooth] coordinates{\kvBbc}--cycle;
\draw[thick,fill=gray] (U)
  --plot[smooth] coordinates{\kvBcd}--cycle;
\draw[thick,fill=gray] (U)
  --plot[smooth] coordinates{\kvBda}--cycle;
\draw[] (omega1)--(omega2);
\whitedot{U};
\whitedot{omega1};
\whitedot{omega2};
\whitedot{omega3};
\end{scope}
\begin{scope}[shift={(2.5,-5.0)}] % 2nd column, 2-cell
\simplexB
\drawB
\draw[thick,fill=gray] (U)
  --plot[smooth] coordinates{\kvBda}--(omega1)--cycle;
\draw[thick,fill=gray!80] (omega1)
  --plot[smooth] coordinates{\kvBda}--cycle;
\draw[] (omega1)--(omega2);
\whitedot{U};
\whitedot{omega1};
\whitedot{omega2};
\whitedot{omega3};
\end{scope}
\begin{scope}[shift={(5.0,0)}] % 3rd column, 0-cell
\node at (75:2.0*\r) {$h_2\! <\!\!\sqrt2\le h_3$;};
\simplexC
\drawB
\draw[thick,fill=gray!80] (omega3) 
  --plot[smooth] coordinates{\kvCca}--cycle;
\draw[] (omega1)--(omega2);
\whitedot{U};
\whitedot{omega1};
\whitedot{omega2};
\whitedot{omega3};
\end{scope}
\begin{scope}[shift={(5.0,-2.5)}] % 3rd column, 1-cell
\simplexC
\drawB
\draw[thick,fill=gray!80] (U) 
  --plot[smooth] coordinates{\kvCab}
  --plot[smooth] coordinates{\kvCbc}
  --cycle;
\draw[thick,fill=gray] (U) 
  --plot[smooth] coordinates{\kvCca}
  --cycle;
\draw[] (omega1)--(omega2);
\whitedot{U};
\whitedot{omega1};
\whitedot{omega2};
\whitedot{omega3};
\end{scope}
\begin{scope}[shift={(5.0,-5.0)}] % 3rd column, 2-cell
\simplexC
\drawB
\draw[thick,fill=gray] (U) 
  --plot[smooth] coordinates{\kvCca}--(omega1)--cycle;
\draw[thick,fill=gray!80] (omega1) 
  --plot[smooth] coordinates{\kvCca}
  --cycle;
\draw[] (omega1)--(omega2);
\whitedot{U};
\whitedot{omega1};
\whitedot{omega2};
\whitedot{omega3};
\end{scope}
\begin{scope}[shift={(5.0,-7.5)}] % 3rd column, 3-cell
\simplexC
\drawB
\draw[thick,fill=gray] (U) --(omega2)--(1.045949,1.278063)--
  (omega1)--cycle;
\draw[thick] (omega1)--(omega2);
\whitedot{U};
\whitedot{omega1};
\whitedot{omega2};
\whitedot{omega3};
\end{scope}
\begin{scope}[shift={(7.5,0)}] % 4th column, 4-cell
\node at (75:2.0*\r) {$h_3\!\! <\!\!\sqrt2$.};
\end{scope}
\begin{scope}[shift={(7.5,-9.50)}] % 4th column, 4-cell
\simplexD
\drawB
\draw[thick,fill=gray] (U) --(omega1)--(omega3)--(omega2)--cycle;
\draw[thick] (omega1)--(omega2);
\whitedot{U};
\whitedot{omega1};
\whitedot{omega2};
\whitedot{omega3};
\end{scope}
}
}



\def\figELMXAFH{
\tikzfig{rogers-ill-defined}{\guid{ELMXAFH} 
Two different parameters $\bu,\bv\in\bV(k)$ can determine
the same Rogers simplex $R(\bu)=R(\bv)$.  In this example,
$\bu=[\u_0;\u_1;\u_2]$ and $\bv=[\u_0;\u_1;\u_2']$.
}
{
[scale=1.0,z={(0,0)}]
\pgfmathsetmacro\r{0.85};
\coordinate (U) at (0,0);
\node[anchor=north] at (U) {~~~~$\omega_0\!=\!\u_0$};
%\node[anchor=north] at (U) {$\omega_0$};
\coordinate (omega1) at (90:\r);
\node[anchor=south east] at (omega1) {$\omega_1$};
\coordinate (omega2) at (\r,\r);
\node[anchor=south east] at (omega2) {$\omega_2$};
\coordinate (U1) at (90:2*\r);
\node[anchor=south] at (U1) {$\u_1$};
\smalldot{U1};
\coordinate (U2) at (0:2*\r);
\node[anchor=south] at (U2) {$\u_2$};
\smalldot{U2};
\coordinate (U3) at (2.0*\r,2.0*\r);
\node[anchor=south] at (U3) {$\u'_2$};
\smalldot{U3};
\draw[fill=gray!30] (U)--(omega1)--(omega2)-- cycle;
\draw[very thick] (\r,0)--(\r,2*\r);
\draw[very thick] (0,\r)--(2*\r,\r);
\whitedot{U};
\whitedot{omega1};
\whitedot{omega2};
\node [anchor=east] at (0,\r*0.5) {$R$};
}
}

\def\figYAJOTSL{
\tikzfig{rogers-fact}{\guid{YAJOTSL} In dimension $k$, the union of $(k+1)!$ 
nondegenerate Rogers simplices is again a simplex.
}
{
[scale=1.3,z={(0,0)}]
\pgfmathsetmacro\s{1.1547};
\pgfmathsetmacro\y{-0.8};
\begin{scope}
\smalldot{0,0};
\node[anchor=north] at (0,\y) {$1!$};
\end{scope}
\begin{scope}[xshift=1.8cm]
\smalldot{-1,0};
\smalldot{1,0};
\draw (-1,0) -- (1,0);
\draw[very thick] (0,-0.1)--(0,0.1);
\node[anchor=north] at (0,\y) {$2!$};
\end{scope}
\begin{scope}
[xshift=4.2cm]
\draw (-30:\s)--(90:\s)--(210:\s)--cycle;
\foreach \i in {30,150,270} {\draw[very thick] (0,0) -- (\i:0.7); }
\foreach \i in {-30,90,210} {
  \coordinate (V\i) at (\i:\s);
  \draw (0,0) -- (V\i); 
  \smalldot{V\i};
  }
\node[anchor=north] at (0,\y) {$3!$};
\end{scope}
\begin{scope}
[xshift=6.9cm]
\autoYAJOTSL
\node[anchor=north] at (0,\y) {$4!$};
\end{scope}
}
}

\def\figBWEYURN{
\tikzfig{marchal-2d}{\guid{BWEYURN} The Marchal partition.
For simplicity, we illustrate the two-dimensional analogue of the partition.
The cells are shaded according to level: $0$, $1$, $2$, and $3$. The $3$-cells are the darkest.
Each disk has radius $\sqrt2$.
}
{
[scale=2.9]
\autoBWEYURN
}
}


\def\figYAHDBVO{
\tikzfig{reference-w}{\guid{YAHDBVO} A set of reference points $\w_i$
can be constructed outside an open disk of radius $r$ and inside 
a given polyhedron $P$ in $\ring{R}^2$.
}
{
[scale=1.0,z={(0,0)}]
\pgfmathsetmacro\r{1.0}
\coordinate (V0) at (-1.40573615605411817: 2.43152527011496122);
\coordinate (V2) at (62.2310998421659392: 1.50101500229540341);
\coordinate (V4) at (166.419445012932584, 1.80579527399678152);
\coordinate (V6) at (206.27916875919712, 1.73501080990908485);
\coordinate (V8) at (293.766402309343221, 1.59228179599956721);
\coordinate (W0) at (-1.40573615605411817: 1);
\coordinate (W2) at (62.2310998421659392: 1.00);
\coordinate (W4) at (166.419445012932584: 1.);
\coordinate (W6) at (206.27916875919712: 1.);
\coordinate (W8) at (293.766402309343221: 1.0);
\coordinate (W1) at (30.4126818430559105: 1.17685374876640436);
 \coordinate (W3) at   (114.325272427549265: 1.62769569077693288);
\coordinate (W5) at    (186.349306886064852: 1.06370463494847955);
\coordinate (W7) at    (250.022785534270156: 1.38419608377047099);
\coordinate (W9) at    (326.180333076644558: 1.18455676175278368);
\coordinate (V1) at (2.44232788210935681, 0.410380646773918112);
\coordinate (V3) at   (-1.28455749338839698, 2.37275986207847556);
\coordinate (V5) at   (-1.88311718784593563, -0.105068476930449517); 
\coordinate (V7) at     (-0.883749792700191605, -2.12898695386381265);
\coordinate (V9) at   (2.4156664831038106, -0.676081495229782448);
\draw[fill=gray!20] (V1)--(V3)--(V5)--(V7)--(V9)--cycle;
\draw[fill=gray!10] (0,0) circle (1.0);
\foreach \i in {W0,W1,W2,W3,W4,W5,W6,W7,W8,W9} \smalldot{\i};
\draw[gray] (W0) node[black,anchor=west] {$\w_0$} --(W1) node[black,anchor=west] {$\w_1$} 
 --(W2) node[black,anchor=south] {$\w_2$} 
 --(W3) node[black,anchor=south] {$\w_3$}
 --(W4) node[black,anchor=east] {$\w_4$}
 --(W5) node[black,anchor=east] {$\w_5$}
 --(W6) node[black,anchor=east] {$\w_6$} 
 --(W7) node[black,anchor=north west] {$\w_7$}
 --(W8) node[black,anchor=north west] {$\w_8$}
 --(W9) node[black,anchor=north west] {$\w_9$}
 --cycle;
\node[anchor=east] at (V0) {$P$};
}}

\def\figZXEVDCA{
\tikzfig{marchal-polyhedron}{\guid{ZXEVDCA} A polyhedron is
constructed by extending planes through
the circular boundaries of
disks $D_i$ on the unit sphere.
}
{
[scale=1.5,z={(0,0)}]
 \draw[ball color=gray!10,shading=ball,semitransparent] (0,0) circle (1); % ball
%
% AUTO GEN START
\autoZXEVDCA
}}


%%%%%%%%%%%%%%%%%%%%%%%%%%%%%%%%%%%%%%%%%%%%%%%%%%%%%%%%%%%%%%%%%%%%%%%%%%%%%%%%
%% LOCAL FAN CHAPTER
%%%%%%%%%%%%%%%%%%%%%%%%%%%%%%%%%%%%%%%%%%%%%%%%%%%%%%%%%%%%%%%%%%%%%%%%%%%%%%%%

\def\figFWYNFVS{
\tikzfig{reflex}{\guid{FWYNFVS} To represent a local fan, each blade
is intersected with the unit sphere to give a spherical polygon. 
The spherical polygon of a nonreflexive local fan has no reflex angles.}
{
[scale=1.2]
\pgfmathsetmacro\Ax{0.4};
\pgfmathsetmacro\Ay{0.6};
\pgfmathsetmacro\Bx{-0.4};
\pgfmathsetmacro\By{0.6};
\coordinate (A') at (\Ax,\Ay);
\coordinate (B') at (\Bx,\By);
\coordinate (A) at ($(0,0)!2!(A')$);
\coordinate (B) at ($(0,0)!2!(B')$);
\draw[ball color=gray!10,shading=ball] (0,0) circle (1); % ball
\draw[fill=gray!90,semitransparent] (A)--(0,0)--(B)--cycle;
\begin{scope}
  \setellipseplane{1.0}{\Ax}{\Ay}{\Bx}{\By};
  \clip (-\mx,-\my) arc 
  [start angle={-90},end angle={90},x radius=\ellx,y radius=\elly,rotate=\ellrotate] 
  -- (0,-1.0) -- cycle;
  \clip (A)--(0,0)--(B)--cycle;
\draw[ball color=gray!10,shading=ball] (0,0) circle (1); % ball
\end{scope}
\begin{scope}
  \clip (A)--(0,0)--(B)--cycle;
 \setellipseplane{1.0}{\Ax}{\Ay}{\Bx}{\By};
  \draw[thick] (-\mx,-\my) arc 
  [start angle={-90},end angle={90},x radius=\ellx,y radius=\elly,rotate=\ellrotate] 
  -- (0,-1.0) -- cycle;
\end{scope}
\smalldot{A};
\smalldot{B};
\whitedot{A'};
\whitedot{B'};
\begin{scope}[xshift=3.0cm]  % nonreflexiv quad
\pgfmathsetmacro\Cx{-0.4};
\pgfmathsetmacro\Cy{-0.6};
\pgfmathsetmacro\Dx{0.4};
\pgfmathsetmacro\Dy{-0.6};
%\pgfmathsetmacro\argA{atan2(\Ax,\Ay)};
%\pgfmathsetmacro\argB{atan2(\Bx,\By)};
%\pgfmathsetmacro\argC{atan2(\Cx,\Cy)};
\coordinate (A') at (\Ax,\Ay);
\coordinate (B') at (\Bx,\By);
\coordinate (C') at (\Cx,\Cy);
\coordinate (D') at (\Dx,\Dy);
\draw[ball color=gray!10,shading=ball] (0,0) circle (1); % ball
\begin{scope}
\clip (A')--(B')-- ++ (120:1) -- cycle;
 \setellipseplane{1.0}{\Ax}{\Ay}{\Bx}{\By};
\draw[ellipse plane] (0,0) circle (1);
\end{scope}
\begin{scope}
\clip (B')--(C')-- ++ (200:1) -- cycle;
 \setellipseplane{1.0}{\Bx}{\By}{\Cx}{\Cy};
\draw[ellipse plane] (0,0) circle (1);
\end{scope}
\begin{scope}
\clip (C')--(D')-- ++ (240:1) -- cycle;
 \setellipseplane{1.0}{\Cx}{\Cy}{\Dx}{\Dy};
\draw[ellipse plane] (0,0) circle (1);
\end{scope}
\begin{scope}
\clip (D')--(A')-- ++ (-60:1) -- cycle;
 \setellipseplane{1.0}{\Dx}{\Dy}{\Ax}{\Ay};
\draw[ellipse plane] (0,0) circle (1);
\end{scope}
\whitedot{A'};
\whitedot{B'};
\whitedot{C'};
\whitedot{D'};
\node at (-90:1.5) {nonreflexive};
%\draw (A')--(B')--(C')--(D')--cycle;
\end{scope}
\begin{scope}[xshift=6.0cm]  % nonreflexiv quad
\pgfmathsetmacro\Cx{-0.4};
\pgfmathsetmacro\Cy{-0.6};
\pgfmathsetmacro\Dx{-0.2};
\pgfmathsetmacro\Dy{0.3};
%\pgfmathsetmacro\argA{atan2(\Ax,\Ay)};
%\pgfmathsetmacro\argB{atan2(\Bx,\By)};
%\pgfmathsetmacro\argC{atan2(\Cx,\Cy)};
\coordinate (A') at (\Ax,\Ay);
\coordinate (B') at (\Bx,\By);
\coordinate (C') at (\Cx,\Cy);
\coordinate (D') at (\Dx,\Dy);
\draw[ball color=gray!10,shading=ball] (0,0) circle (1); % ball
\begin{scope}
\clip (A')--(B')-- ++ (120:1) -- cycle;
 \setellipseplane{1.0}{\Ax}{\Ay}{\Bx}{\By};
\draw[ellipse plane] (0,0) circle (1);
\end{scope}
\begin{scope}
\clip (B')--(C')-- ++ (200:1) -- cycle;
 \setellipseplane{1.0}{\Bx}{\By}{\Cx}{\Cy};
\draw[ellipse plane] (0,0) circle (1);
\end{scope}
\begin{scope}
\clip (C')--(D')-- ++ (240:1) -- cycle;
 \setellipseplane{1.0}{\Cx}{\Cy}{\Dx}{\Dy};
\draw[ellipse plane] (0,0) circle (1);
\end{scope}
\begin{scope}
\clip (D')--(A')-- ++ (-240:1) -- cycle;
 \setellipseplane{1.0}{\Dx}{\Dy}{\Ax}{\Ay};
\draw[ellipse plane] (0,0) circle (1);
\end{scope}
\whitedot{A'};
\whitedot{B'};
\whitedot{C'};
\whitedot{D'};
\node[anchor=north west] at (D') {$\p$};
%\draw (A')--(B')--(C')--(D')--cycle;
\node at (-90:1.5) {reflex angle at $\p$};
\end{scope}
}
}

\def\figQTCGYTB{
\tikzfig{fan-type}{\guid{QTCGYTB} A nonreflexive
local fan is generic, lunar, or circular.  The figures follow the conventions
of Remark~\ref{rem:visual}.}
{
[scale=1.0]
\begin{scope}[xshift=0.0cm]  % generic quad
\pgfmathsetmacro\Ax{0.4};
\pgfmathsetmacro\Ay{0.6};
\pgfmathsetmacro\Bx{-0.4};
\pgfmathsetmacro\By{0.6};
\pgfmathsetmacro\Cx{-0.4};
\pgfmathsetmacro\Cy{-0.6};
\pgfmathsetmacro\Dx{0.4};
\pgfmathsetmacro\Dy{-0.6};
\coordinate (A') at (\Ax,\Ay);
\coordinate (B') at (\Bx,\By);
\coordinate (C') at (\Cx,\Cy);
\coordinate (D') at (\Dx,\Dy);
\draw[ball color=gray!10,shading=ball] (0,0) circle (1); % ball
\begin{scope}
\clip (A')--(B')-- ++ (120:1) -- cycle;
 \setellipseplane{1.0}{\Ax}{\Ay}{\Bx}{\By};
\draw[ellipse plane] (0,0) circle (1);
\end{scope}
\begin{scope}
\clip (B')--(C')-- ++ (200:1) -- cycle;
 \setellipseplane{1.0}{\Bx}{\By}{\Cx}{\Cy};
\draw[ellipse plane] (0,0) circle (1);
\end{scope}
\begin{scope}
\clip (C')--(D')-- ++ (240:1) -- cycle;
 \setellipseplane{1.0}{\Cx}{\Cy}{\Dx}{\Dy};
\draw[ellipse plane] (0,0) circle (1);
\end{scope}
\begin{scope}
\clip (D')--(A')-- ++ (-60:1) -- cycle;
 \setellipseplane{1.0}{\Dx}{\Dy}{\Ax}{\Ay};
\draw[ellipse plane] (0,0) circle (1);
\end{scope}
\whitedot{A'};
\whitedot{B'};
\whitedot{C'};
\whitedot{D'};
\node at (-90:1.5) {generic};
%\draw (A')--(B')--(C')--(D')--cycle;
\end{scope}
\begin{scope}[xshift=3.0cm]  % lunar 
\pgfmathsetmacro\Ax{0.1};
\pgfmathsetmacro\Ay{0.6};
\pgfmathsetmacro\Px{-0.1};
\pgfmathsetmacro\Py{-0.6};
\pgfmathsetmacro\Bx{cos(-60)};
\pgfmathsetmacro\By{sin(-60)};
\pgfmathsetmacro\Cx{cos(-125)};
\pgfmathsetmacro\Cy{sin(-125)};
\coordinate (A) at (\Ax,\Ay);
\coordinate (P) at (\Px,\Py);
\coordinate (B) at (\Bx,\By);
\coordinate (C) at (\Cx,\Cy);
\draw[ball color=gray!10,shading=ball] (0,0) circle (1); % ball
\begin{scope}
\clip (A)--(B)-- ++ (0:1) -- ++(90:1.3) --cycle;
 \setellipseplane{1.0}{\Ax}{\Ay}{\Bx}{\By};
\draw[ellipse plane] (0,0) circle (1);
\end{scope}
\begin{scope}
\clip (A)--(C)-- ++ (180:1) -- ++(90:1.3)-- cycle;
 \setellipseplane{1.0}{\Ax}{\Ay}{\Cx}{\Cy};
\draw[ellipse plane] (0,0) circle (1);
\end{scope}
\begin{scope}
\clip (P)--(C)-- ++ (-45:1) -- ++(45:1.3)-- cycle;
 \setellipseplane{1.0}{\Px}{\Py}{\Cx}{\Cy};
\draw[ellipse plane,black!70] (0,0) circle (1);
\end{scope}
\begin{scope}
\clip (P)--(B)-- ++ (-135:1) -- ++(135:1.3)-- cycle;
 \setellipseplane{1.0}{\Px}{\Py}{\Bx}{\By};
\draw[ellipse plane,black!70] (0,0) circle (1);
\end{scope}
\whitedot{A};
\whitedot{B};
\whitedot{C};
\node at (-90:1.5) {lunar};
\graydot{P};
\end{scope}
\begin{scope}[xshift=6.0cm]  % circular
\pgfmathsetmacro\Ax{0.1};
\pgfmathsetmacro\Ay{0.6};
\pgfmathsetmacro\Px{-0.1};
\pgfmathsetmacro\Py{-0.6};
\pgfmathsetmacro\Bx{cos(-60)};
\pgfmathsetmacro\By{sin(-60)};
\pgfmathsetmacro\Cx{-\Bx};
\pgfmathsetmacro\Cy{-\By};
\coordinate (A) at (\Ax,\Ay);
\coordinate (P) at (\Px,\Py);
\coordinate (B) at (\Bx,\By);
\coordinate (C) at (\Cx,\Cy);
\draw[ball color=gray!10,shading=ball] (0,0) circle (1); % ball
\begin{scope}
\setellipseplane{1.0}{\Px}{\Py}{\Bx}{\By};
\draw[ellipse plane,black!50] (0,0) circle (1);
\end{scope}
\begin{scope}
\clip (B)--(C)-- ++ (45:0.5) -- ++(-45:2.1) --cycle;
 \setellipseplane{1.0}{\Ax}{\Ay}{\Bx}{\By};
\draw[ellipse plane] (0,0) circle (1);
\end{scope}
\whitedot{A};
\whitedot{B};
\node at (-90:1.5) {circular};
\end{scope}
}
}

\def\figGJBTZJI{
\tikzfig{lateral-motion}{\guid{GJBTZJI} A lateral motion of $\w_i$
follows a circular path at fixed distance from $\orz$ and $\w_j$, for some
pair of adjacent indices $i$ and $j$.}
{
[scale=1.4,rotate=90]
\pgfmathsetmacro\xr{0.3};
\pgfmathsetmacro\yr{1.5};
\pgfmathsetmacro\cx{-0.20};
\pgfmathsetmacro\cy{\yr* sqrt(1.0 - \cx*\cx/ (\xr *\xr)};
\pgfmathsetmacro\dx{-0.27};
\pgfmathsetmacro\dy{\yr* sqrt(1.0 - \dx*\dx/ (\xr *\xr)};
\coordinate (A) at (-1,0);
\coordinate (B) at (1,0);
\coordinate (C) at (\cx,\cy);
\coordinate (D) at (\dx,\dy);
\coordinate (U) at (0,0);
\coordinate (orz) at (-120:1.5);
\draw[gray] (0.3,0) arc[x radius=\xr,y radius=\yr,start angle=0,rotate=0,end angle=360] ;
\smalldot{A};
\smalldot{B};
\smalldot{C};
\draw (A)--(B);
\graydot{U};
\draw[->,thick,black] (C)--(D);
\node[anchor=west] at (A) {$\orz$};
\node[anchor=west] at (B) {$\w_j$};
\node[anchor=north] at (C) {$\w_i$};
}
}

\def\figMVFCDJQ{
\tikzfig{radial-motion}{\guid{MVFCDJQ} A radial motion of $\w_i$ towards or away from $\orz$
follows a circular path at fixed distance from $\w_{i-1}$ and $\w_{i+1}$.}
{
[scale=1.4]
\pgfmathsetmacro\xr{0.3};
\pgfmathsetmacro\yr{1.5};
\pgfmathsetmacro\cx{-0.20};
\pgfmathsetmacro\cy{\yr* sqrt(1.0 - \cx*\cx/ (\xr *\xr)};
\pgfmathsetmacro\dx{-0.27};
\pgfmathsetmacro\dy{\yr* sqrt(1.0 - \dx*\dx/ (\xr *\xr)};
\coordinate (A) at (-1,0);
\coordinate (B) at (1,0);
\coordinate (C) at (\cx,\cy);
\coordinate (D) at (\dx,\dy);
\coordinate (U) at (0,0);
\coordinate (orz) at (-120:1.5);
\draw[gray] (0.3,0) arc[x radius=\xr,y radius=\yr,start angle=0,rotate=0,end angle=360] ;
\smalldot{A};
\smalldot{B};
\smalldot{C};
\smalldot{orz};
\draw (A)--(B);
\graydot{U};
\draw[->,thick,black] (C)--(D);
\node[anchor=north] at (A) {$\w_{i-1}$};
\node[anchor=north] at (B) {$\w_{i+1}$};
\node[anchor=east] at (C) {$\w_i$};
\node[anchor=north] at (orz) {$\orz$};
}
}

\def\figWKUYEXM{
\tikzfig{slice-torsor}{\guid{WKUYEXM} Given $p,q\in I$, the slice $I[p,q]$ follows
the cyclic order through $I$ from $p$ to $q$, then returns directly from $q$ to $p$.}
{
[scale=1.0]
\foreach \i/\j in {0/p,60/1+p,120/2+p,180/3+p,240/4+p,300/5+p}
 {
   \coordinate (V\i) at (\i:1.0);
   \smalldot{V\i};
   \node at (\i:1.5) {$\j$};
   \draw[->,shorten >= 3pt,shorten <=3pt] (\i:1.0)--(60+\i:1.0);
}
\node at (270:1.8) {$I$};
\begin{scope} [xshift=5.0cm]
\foreach \i/\j/\k in {0/p/60,60/1+p/120,120/2+p/180,180/3+p/240,240/4+p/0}
 {
   \coordinate (V\i) at (\i:1.0);
   \smalldot{V\i};
   \node at (\i:1.5) {$\j$};
   \draw[->,shorten >= 3pt,shorten <=3pt] (\i:1.0)--(\k:1.0);
}
\node at (270:1.8) {~~~~~~~$I[p,4+p]$};
\end{scope}
%\node at (V0) {$
%\foreach \i/\j in {0/60,60/120,120/180,180/240,240/300,300/0} 
%  {
%   draw[->] (V\i)--(V\j);
%   }
}
}

\def\figLZBQINL{
\tikzfig{main-pent}{\guid{LZBQINL} A pentagon is triangulated by two diagonals
of length at most $1+\sqrt5$ as shown.  The 
peripheral nodes $\w_1$ and $\w_4$
are rigidly determined up to finite ambiguity
by the simplex with extreme points
$\orz$, $\w_0$, $\w_2$, and $\w_3$.}
{
[scale=1.0]
\foreach \i in {0,1,2,3,4}
 {
 \coordinate (V\i) at ({\i * 72}:1.0);
 \smalldot{V\i};
 \node at ($(0,0)!1.3!(V\i)$) {$\w_{\i}$};
 }
\draw (V0)--(V1)--(V2)--(V3)--(V4)--cycle;
\draw (V2)--(V0)--(V3);
}
}


\def\figARTLHOI{
\tikzfig{main-hex}{\guid{ARTLHOI} A hexagon is triangulated as shown.  The 
peripheral nodes $\w_1$, $\w_3$, and $\w_5$ 
are rigidly determined up to finite ambiguity
by the simplex with extreme points
$\orz$, $\w_0$, $\w_2$, and $\w_4$.}
{
[scale=1.0]
\foreach \i in {0,1,2,3,4,5}
 {
 \coordinate (V\i) at ({\i * 60}:1.0);
 \smalldot{V\i};
 \node at ($(0,0)!1.3!(V\i)$) {$\w_{\i}$};
 }
\draw (V0)--(V1)--(V2)--(V3)--(V4)--(V5)--cycle;
\draw (V0)--(V2)--(V4)--cycle;
}
}

\def\figDEJKNQK{
\tikzfig{dk-poly}{\guid{DEJKNQK} The fans associated with these
spherical polygons attain values of $\tau$
close to the lower bounds $d(k)$.  The choice of constants $d(k)$ was
based on these examples.  The points lie on a sphere of
radius $2$.  The polygon is triangulated with diagonals of Euclidean
length $2h_0$.}
{
[scale=1.3]
\autoDEJKNQK
\begin{scope}[shift={(6,0)}]
\autocDEJKNQK
\draw[ball color=gray!10,shading=ball] (0,0) circle (1); % ball
\draw[thick] (v1) --plot[smooth] coordinates {\dejka}
  --plot[smooth] coordinates{\dejkb}
  --plot[smooth] coordinates{\dejkc}
  --plot[smooth] coordinates{\dejkd}
  --plot[smooth] coordinates{\dejke}
  --plot[smooth] coordinates{\dejkf};
\draw[thick,gray] plot[smooth] coordinates {\dejkg};
\draw[thick,gray] plot[smooth] coordinates {\dejkh};
\draw[thick,gray] plot[smooth] coordinates {\dejki};
\foreach \i in {v1,v2,v3,v4,v5,w6} { \smalldot{\i}; };
\node at (-45:1.3) {$d(6)$};
\end{scope}
\begin{scope}[shift={(3,0)}]
\autocDEJKNQK
\draw[ball color=gray!10,shading=ball] (0,0) circle (1); % ball
\draw[thick] plot[smooth] coordinates {\dejka}
  --plot[smooth] coordinates{\dejkb}
  --plot[smooth] coordinates{\dejkj}
  --plot[smooth] coordinates{\dejkk}
  --plot[smooth] coordinates{\dejkf};
\draw[thick,gray] plot[smooth] coordinates {\dejkg};
\draw[thick,gray] plot[smooth] coordinates {\dejkh};
\foreach \i in {v1,v2,v3,v4,w5} { \smalldot{\i}; };
\node at (-45:1.3) {$d(5)$};
\end{scope}
\begin{scope}[shift={(0,0)}]
\autocDEJKNQK
\draw[ball color=gray!10,shading=ball] (0,0) circle (1); % ball
\draw[thick] plot[smooth] coordinates {\dejka}
  --plot[smooth] coordinates{\dejkl}
  --plot[smooth] coordinates{\dejkm}
  --plot[smooth] coordinates{\dejkf};
\draw[thick,gray] plot[smooth] coordinates {\dejkg};
\foreach \i in {v1,v2,v3,w4} { \smalldot{\i}; };
\node at (-45:1.3) {$d(4)$};
\end{scope}
}
}

\def\figQTICQYN{
\tikzfig{bk-mono}{\guid{QTICQYN} The angles $\beta(k)$ are increasing
in $k$ on a nonreflexive local fan.  The three arrows mark the angles
$\beta(2)$, $\beta(3)$, and $\beta(4)$.}
{
[scale=1.7]
\autoQTICQYN
\draw[ball color=gray!10,shading=ball] (0,0) circle (1); % ball
\draw[thick] (v1) --plot[smooth] coordinates {\qta}
  --plot[smooth] coordinates{\qtb}
  --plot[smooth] coordinates{\qtc}
  --plot[smooth] coordinates{\qtd}
  --plot[smooth] coordinates{\qte};
\draw[thick,gray] plot[smooth] coordinates {\qtf};
\draw[thick,gray] plot[smooth] coordinates {\qtg};
\foreach \i in {v1,v2,v3,v4,v5} {\smalldot{\i}; }
\draw[very thick,->] ($(v3)!0.35!(v1)$) % node[anchor=west] {$~~\beta(1)$}
 -- ++ (-0.09,-0.02) --
   ($(v3)!0.3!(v2)+(0.0,0.08)$);
\draw[very thick,->] ($(v3)!0.5!(v1)$) % node[anchor=west] {$~~\beta(1)$}
  .. controls ++(180:0.3) and ++ (120:0.45) ..  ($(v3)!0.45!(v4)+(0,-0.1)$);
\draw[very thick,->] ($(v3)!0.7!(v1)$) % node[anchor=west] {$~~\beta(1)$}
  .. controls ++(180:0.6) and ++ (200:1.0) ..  ($(v3)!0.7!(v5)$);
\node[fill=gray!30] at (0.8,-0.55) {$\beta(4)$};
\node[fill=gray!20] at (0.4,-0.3) {$\beta(3)$};
\node[fill=gray!10] at (0.3,0.05) {$\beta(2)$};
}}

\def\figSXYGYPC{
\tikzfig{fan-slice}{\guid{SXYGYPC} A slice of a fan along $(\v,\w)$.}
{
[scale=1.7]
\begin{scope}
\autoQTICQYN
\draw[ball color=gray!5,shading=ball] (0,0) circle (1); % ball
\node[anchor=east,fill=gray!35,shape=circle] at (v3) {$\v$};
\node[anchor=south,fill=gray!45,shape=circle] at ($(v4)+(0.05,0.05)$) {$\w$};
\draw[thick] (v1) --plot[smooth] coordinates {\qta}
  --plot[smooth] coordinates{\qtb}
  --plot[smooth] coordinates{\qtc}
  --plot[smooth] coordinates{\qtd}
  --plot[smooth] coordinates{\qte};
\draw[thick,gray] plot[smooth] coordinates {\qtg};
\foreach \i in {v1,v2,v3,v4,v5} {\smalldot{\i}; }
\node at (-90:1.3) {$(V,E,F)$};
\end{scope}
\begin{scope}[shift={(3,0)}]
\autoQTICQYN
\draw[ball color=gray!5,shading=ball] (0,0) circle (1); % ball
\node[anchor=east,fill=gray!35,shape=circle] at (v3) {$\v$};
\node[anchor=south,fill=gray!45,shape=circle] at ($(v4)+(0.05,0.05)$) {$\w$};
\draw[thick] (v1) --plot[smooth] coordinates {\qta}
  --plot[smooth] coordinates{\qtb}
  --plot[smooth] coordinates{\qtg}
 --plot[smooth] coordinates{\qte};
\foreach \i in {v1,v2,v3,v4} {\smalldot{\i}; }
\node at (-90:1.3) {$(V[\v,\w],E[\v,\w],F[\v,\w])$};
\end{scope}
}}

\def\figHEABLRG{
\tikzfig{polar}{\guid{HEABLRG} A quadrilateral and its polar.}
{
[scale=1.7]
\autoHEABLRG
\draw[ball color=gray!10,shading=ball] (0,0) circle (1); % ball
\draw[thick] (v1) --plot[smooth] coordinates {\qta}
  --plot[smooth] coordinates{\qtb}
  --plot[smooth] coordinates{\qtc}
  --plot[smooth] coordinates{\qtd};
\draw[thick,gray] (w1) --plot[smooth] coordinates {\qte}
  --plot[smooth] coordinates{\qtf}
  --plot[smooth] coordinates{\qtg}
  --plot[smooth] coordinates{\qth};
\foreach \i in {v1,v2,v3,v4} {\smalldot {\i}; }
\foreach \i in {w1,w2,w3,w4} {\whitedot {\i}; }
}
}

\def\figHBMVLMY{
\tikzfig{tame-graph}{\guid{HBMVLMY} Here are some of the 
tens of thousands of planar graphs
whose hypermaps are tame. The ones depicted here are the ones
that are the most difficult to eliminate through linear programming.}
{
[scale=1.2]
\begin{scope}
%tikz format auto generated by GentikzA.java
%format for fig.tex in flypaper.
%in LaTeX file, enclose code in tikzpicture environment
% invariant: 161847242261
%Set the coordinates of the points.
\coordinate  (P0) at (1.000000,0.000000);\coordinate  (P1) at (-0.500001,0.866025);\coordinate  (P2) at (-0.499997,-0.866027);\coordinate  (P3) at (0.432205,-0.235924);\coordinate  (P4) at (-0.416334,-0.648872);\coordinate  (P5) at (-0.391194,0.457910);\coordinate  (P6) at (-0.057662,0.526278);\coordinate  (P7) at (0.161902,0.296009);\coordinate  (P8) at (0.427308,0.018805);\coordinate  (P9) at (-0.079539,-0.334912);\coordinate  (P10) at (-0.285954,-0.040786);\coordinate  (P11) at (-0.099166,0.160803);\coordinate  (P12) at (0.117432,-0.069114);%Draw edges.
\draw
  (P0) -- (P2)
  (P0) -- (P1)
  (P0) -- (P6)
  (P0) -- (P7)
  (P0) -- (P8)
  (P0) -- (P3)
  (P1) -- (P2)
  (P1) -- (P4)
  (P1) -- (P5)
  (P1) -- (P6)
  (P2) -- (P3)
  (P2) -- (P4)
  (P3) -- (P8)
  (P3) -- (P9)
  (P3) -- (P4)
  (P4) -- (P9)
  (P4) -- (P10)
  (P4) -- (P5)
  (P5) -- (P10)
  (P5) -- (P11)
  (P5) -- (P6)
  (P6) -- (P11)
  (P6) -- (P7)
  (P7) -- (P11)
  (P7) -- (P12)
  (P7) -- (P8)
  (P8) -- (P12)
  (P8) -- (P9)
  (P9) -- (P12)
  (P9) -- (P10)
  (P10) -- (P12)
  (P10) -- (P11)
  (P11) -- (P12)
;
\smalldot {P0};
\smalldot {P1};
\smalldot {P2};
\smalldot {P3};
\smalldot {P4};
\smalldot {P5};
\smalldot {P6};
\smalldot {P7};
\smalldot {P8};
\smalldot {P9};
\smalldot {P10};
\smalldot {P11};
\smalldot {P12};
\end{scope}
\begin{scope}[shift={(2.5,0)}]
%tikz format auto generated by GentikzA.java
%format for fig.tex in flypaper.
%in LaTeX file, enclose code in tikzpicture environment
% invariant: 223336279535
%Set the coordinates of the points.
\coordinate  (P0) at (1.000000,0.000000);\coordinate  (P1) at (-0.000001,1.000000);\coordinate  (P2) at (-1.000000,-0.000002);\coordinate  (P3) at (0.000004,-1.000000);\coordinate  (P4) at (0.075646,-0.707517);\coordinate  (P5) at (-0.621959,-0.182765);\coordinate  (P6) at (-0.385435,0.404165);\coordinate  (P7) at (-0.116041,0.202692);\coordinate  (P8) at (0.207638,0.593419);\coordinate  (P9) at (0.656366,-0.024110);\coordinate  (P10) at (0.060331,-0.379672);\coordinate  (P11) at (-0.248291,-0.122670);\coordinate  (P12) at (0.270863,0.113577);%Draw edges.
\draw
  (P0) -- (P3)
  (P0) -- (P1)
  (P0) -- (P8)
  (P0) -- (P9)
  (P0) -- (P4)
  (P1) -- (P2)
  (P1) -- (P6)
  (P1) -- (P7)
  (P1) -- (P8)
  (P2) -- (P3)
  (P2) -- (P5)
  (P2) -- (P6)
  (P3) -- (P4)
  (P3) -- (P5)
  (P4) -- (P9)
  (P4) -- (P10)
  (P4) -- (P5)
  (P5) -- (P10)
  (P5) -- (P11)
  (P5) -- (P6)
  (P6) -- (P11)
  (P6) -- (P7)
  (P7) -- (P11)
  (P7) -- (P12)
  (P7) -- (P8)
  (P8) -- (P12)
  (P8) -- (P9)
  (P9) -- (P12)
  (P9) -- (P10)
  (P10) -- (P12)
  (P10) -- (P11)
  (P11) -- (P12)
;
\smalldot {P0};
\smalldot {P1};
\smalldot {P2};
\smalldot {P3};
\smalldot {P4};
\smalldot {P5};
\smalldot {P6};
\smalldot {P7};
\smalldot {P8};
\smalldot {P9};
\smalldot {P10};
\smalldot {P11};
\smalldot {P12};
\end{scope}
\begin{scope}[shift={(5,0)}]
%tikz format auto generated by GentikzA.java
%format for fig.tex in flypaper.
%in LaTeX file, enclose code in tikzpicture environment
% invariant: 86506100695
%Set the coordinates of the points.
\coordinate  (P0) at (1.000000,0.000000);\coordinate  (P1) at (-0.000001,1.000000);\coordinate  (P2) at (-1.000000,-0.000002);\coordinate  (P3) at (0.000004,-1.000000);\coordinate  (P4) at (0.058274,-0.623308);\coordinate  (P5) at (0.586244,-0.039936);\coordinate  (P6) at (-0.397562,-0.431544);\coordinate  (P7) at (-0.658013,0.135059);\coordinate  (P8) at (-0.136514,0.457208);\coordinate  (P9) at (0.288729,0.480077);\coordinate  (P10) at (0.143226,-0.045224);\coordinate  (P11) at (-0.011925,-0.372743);\coordinate  (P12) at (-0.289428,0.033574);%Draw edges.
\draw
  (P0) -- (P3)
  (P0) -- (P1)
  (P0) -- (P9)
  (P0) -- (P5)
  (P1) -- (P2)
  (P1) -- (P7)
  (P1) -- (P8)
  (P1) -- (P9)
  (P2) -- (P3)
  (P2) -- (P6)
  (P2) -- (P7)
  (P3) -- (P4)
  (P3) -- (P6)
  (P4) -- (P5)
  (P4) -- (P11)
  (P4) -- (P6)
  (P5) -- (P9)
  (P5) -- (P10)
  (P5) -- (P11)
  (P6) -- (P11)
  (P6) -- (P12)
  (P6) -- (P7)
  (P7) -- (P12)
  (P7) -- (P8)
  (P8) -- (P12)
  (P8) -- (P10)
  (P8) -- (P9)
  (P9) -- (P10)
  (P10) -- (P12)
  (P10) -- (P11)
  (P11) -- (P12)
;
\smalldot {P0};
\smalldot {P1};
\smalldot {P2};
\smalldot {P3};
\smalldot {P4};
\smalldot {P5};
\smalldot {P6};
\smalldot {P7};
\smalldot {P8};
\smalldot {P9};
\smalldot {P10};
\smalldot {P11};
\smalldot {P12};
\end{scope}
\begin{scope}[shift={(0,-2.5)}]
%tikz format auto generated by GentikzA.java
%format for fig.tex in flypaper.
%in LaTeX file, enclose code in tikzpicture environment
% invariant: 179189825656
%Set the coordinates of the points.
\coordinate  (P0) at (1.000000,0.000000);\coordinate  (P1) at (-0.000001,1.000000);\coordinate  (P2) at (-1.000000,-0.000002);\coordinate  (P3) at (0.000004,-1.000000);\coordinate  (P4) at (0.001837,-0.775263);\coordinate  (P5) at (-0.675910,-0.020307);\coordinate  (P6) at (-0.144568,0.651636);\coordinate  (P7) at (0.276419,0.326985);\coordinate  (P8) at (0.607919,0.211628);\coordinate  (P9) at (0.426484,-0.262255);\coordinate  (P10) at (-0.136907,-0.316383);\coordinate  (P11) at (-0.209000,0.207046);\coordinate  (P12) at (0.178864,-0.027711);%Draw edges.
\draw
  (P0) -- (P3)
  (P0) -- (P1)
  (P0) -- (P8)
  (P0) -- (P9)
  (P0) -- (P4)
  (P1) -- (P2)
  (P1) -- (P6)
  (P1) -- (P7)
  (P1) -- (P8)
  (P2) -- (P3)
  (P2) -- (P4)
  (P2) -- (P5)
  (P2) -- (P6)
  (P3) -- (P4)
  (P4) -- (P9)
  (P4) -- (P10)
  (P4) -- (P5)
  (P5) -- (P10)
  (P5) -- (P11)
  (P5) -- (P6)
  (P6) -- (P11)
  (P6) -- (P7)
  (P7) -- (P11)
  (P7) -- (P12)
  (P7) -- (P8)
  (P8) -- (P12)
  (P8) -- (P9)
  (P9) -- (P12)
  (P9) -- (P10)
  (P10) -- (P12)
  (P10) -- (P11)
  (P11) -- (P12)
;
\smalldot {P0};
\smalldot {P1};
\smalldot {P2};
\smalldot {P3};
\smalldot {P4};
\smalldot {P5};
\smalldot {P6};
\smalldot {P7};
\smalldot {P8};
\smalldot {P9};
\smalldot {P10};
\smalldot {P11};
\smalldot {P12};
\end{scope}
\begin{scope}[shift={(2.5,-2.5)}]
%tikz format auto generated by GentikzA.java
%format for fig.tex in flypaper.
%in LaTeX file, enclose code in tikzpicture environment
% invariant: 154005963125
%Set the coordinates of the points.
\coordinate  (P0) at (1.000000,0.000000);\coordinate  (P1) at (-0.000001,1.000000);\coordinate  (P2) at (-1.000000,-0.000002);\coordinate  (P3) at (0.000004,-1.000000);\coordinate  (P4) at (-0.062775,-0.739232);\coordinate  (P5) at (0.631843,-0.074689);\coordinate  (P6) at (-0.296051,-0.363530);\coordinate  (P7) at (-0.647257,0.147055);\coordinate  (P8) at (-0.110245,0.511533);\coordinate  (P9) at (0.374160,0.428751);\coordinate  (P10) at (0.132938,0.143982);\coordinate  (P11) at (0.216828,-0.206516);\coordinate  (P12) at (-0.294541,-0.005965);%Draw edges.
\draw
  (P0) -- (P3)
  (P0) -- (P1)
  (P0) -- (P9)
  (P0) -- (P5)
  (P1) -- (P2)
  (P1) -- (P7)
  (P1) -- (P8)
  (P1) -- (P9)
  (P2) -- (P3)
  (P2) -- (P4)
  (P2) -- (P6)
  (P2) -- (P7)
  (P3) -- (P4)
  (P4) -- (P5)
  (P4) -- (P11)
  (P4) -- (P6)
  (P5) -- (P9)
  (P5) -- (P10)
  (P5) -- (P11)
  (P6) -- (P11)
  (P6) -- (P12)
  (P6) -- (P7)
  (P7) -- (P12)
  (P7) -- (P8)
  (P8) -- (P12)
  (P8) -- (P10)
  (P8) -- (P9)
  (P9) -- (P10)
  (P10) -- (P12)
  (P10) -- (P11)
  (P11) -- (P12)
;
\smalldot {P0};
\smalldot {P1};
\smalldot {P2};
\smalldot {P3};
\smalldot {P4};
\smalldot {P5};
\smalldot {P6};
\smalldot {P7};
\smalldot {P8};
\smalldot {P9};
\smalldot {P10};
\smalldot {P11};
\smalldot {P12};
\end{scope}
\begin{scope}[shift={(5,-2.5)}]
%tikz format auto generated by GentikzA.java
%format for fig.tex in flypaper.
%in LaTeX file, enclose code in tikzpicture environment
% invariant: 65974205892
%Set the coordinates of the points.
\coordinate  (P0) at (1.000000,0.000000);\coordinate  (P1) at (-0.000001,1.000000);\coordinate  (P2) at (-1.000000,-0.000002);\coordinate  (P3) at (0.000004,-1.000000);\coordinate  (P4) at (0.154120,-0.177621);\coordinate  (P5) at (0.408791,0.153633);\coordinate  (P6) at (-0.141558,-0.708599);\coordinate  (P7) at (-0.555582,-0.185895);\coordinate  (P8) at (-0.597146,0.237445);\coordinate  (P9) at (-0.051260,0.626435);\coordinate  (P10) at (0.031693,0.133714);\coordinate  (P11) at (-0.208512,-0.324435);\coordinate  (P12) at (-0.259209,0.172156);%Draw edges.
\draw
  (P0) -- (P3)
  (P0) -- (P1)
  (P0) -- (P5)
  (P1) -- (P2)
  (P1) -- (P8)
  (P1) -- (P9)
  (P2) -- (P3)
  (P2) -- (P6)
  (P2) -- (P7)
  (P2) -- (P8)
  (P3) -- (P4)
  (P3) -- (P6)
  (P4) -- (P5)
  (P4) -- (P10)
  (P4) -- (P11)
  (P4) -- (P6)
  (P5) -- (P9)
  (P5) -- (P10)
  (P6) -- (P11)
  (P6) -- (P7)
  (P7) -- (P11)
  (P7) -- (P12)
  (P7) -- (P8)
  (P8) -- (P12)
  (P8) -- (P9)
  (P9) -- (P12)
  (P9) -- (P10)
  (P10) -- (P12)
  (P10) -- (P11)
  (P11) -- (P12)
;
\smalldot {P0};
\smalldot {P1};
\smalldot {P2};
\smalldot {P3};
\smalldot {P4};
\smalldot {P5};
\smalldot {P6};
\smalldot {P7};
\smalldot {P8};
\smalldot {P9};
\smalldot {P10};
\smalldot {P11};
\smalldot {P12};
\end{scope}
\begin{scope}[shift={(0,-5)}]
%tikz format auto generated by GentikzA.java
%format for fig.tex in flypaper.
%in LaTeX file, enclose code in tikzpicture environment
% invariant: 50803004532
%Set the coordinates of the points.
\coordinate  (P0) at (1.000000,0.000000);\coordinate  (P1) at (-0.000001,1.000000);\coordinate  (P2) at (-1.000000,-0.000002);\coordinate  (P3) at (0.000004,-1.000000);\coordinate  (P4) at (0.501942,-0.287024);\coordinate  (P5) at (-0.101608,-0.693944);\coordinate  (P6) at (-0.439645,-0.202605);\coordinate  (P7) at (-0.546087,0.250258);\coordinate  (P8) at (0.002162,0.621158);\coordinate  (P9) at (0.532364,0.288416);\coordinate  (P10) at (0.262791,0.104950);\coordinate  (P11) at (0.041436,-0.288401);\coordinate  (P12) at (-0.130503,0.158699);%Draw edges.
\draw
  (P0) -- (P3)
  (P0) -- (P1)
  (P0) -- (P9)
  (P0) -- (P4)
  (P1) -- (P2)
  (P1) -- (P7)
  (P1) -- (P8)
  (P1) -- (P9)
  (P2) -- (P3)
  (P2) -- (P5)
  (P2) -- (P6)
  (P2) -- (P7)
  (P3) -- (P4)
  (P3) -- (P5)
  (P4) -- (P10)
  (P4) -- (P11)
  (P4) -- (P5)
  (P5) -- (P11)
  (P5) -- (P6)
  (P6) -- (P11)
  (P6) -- (P12)
  (P6) -- (P7)
  (P7) -- (P12)
  (P7) -- (P8)
  (P8) -- (P12)
  (P8) -- (P10)
  (P8) -- (P9)
  (P9) -- (P10)
  (P10) -- (P12)
  (P10) -- (P11)
  (P11) -- (P12)
;
\smalldot {P0};
\smalldot {P1};
\smalldot {P2};
\smalldot {P3};
\smalldot {P4};
\smalldot {P5};
\smalldot {P6};
\smalldot {P7};
\smalldot {P8};
\smalldot {P9};
\smalldot {P10};
\smalldot {P11};
\smalldot {P12};
\end{scope}
\begin{scope}[shift={(2.5,-5)}]
%tikz format auto generated by GentikzA.java
%format for fig.tex in flypaper.
%in LaTeX file, enclose code in tikzpicture environment
% invariant: 39599353438
%Set the coordinates of the points.
\coordinate  (P0) at (1.000000,0.000000);\coordinate  (P1) at (-0.000001,1.000000);\coordinate  (P2) at (-1.000000,-0.000002);\coordinate  (P3) at (0.000004,-1.000000);\coordinate  (P4) at (-0.061825,-0.735713);\coordinate  (P5) at (0.590894,-0.095214);\coordinate  (P6) at (-0.258210,-0.374742);\coordinate  (P7) at (-0.690985,0.120301);\coordinate  (P8) at (-0.153037,0.558301);\coordinate  (P9) at (0.227224,0.258380);\coordinate  (P10) at (-0.095601,0.198875);\coordinate  (P11) at (0.202521,-0.179627);\coordinate  (P12) at (-0.318323,0.032374);%Draw edges.
\draw
  (P0) -- (P3)
  (P0) -- (P1)
  (P0) -- (P9)
  (P0) -- (P5)
  (P1) -- (P2)
  (P1) -- (P7)
  (P1) -- (P8)
  (P2) -- (P3)
  (P2) -- (P4)
  (P2) -- (P6)
  (P2) -- (P7)
  (P3) -- (P4)
  (P4) -- (P5)
  (P4) -- (P11)
  (P4) -- (P6)
  (P5) -- (P10)
  (P5) -- (P11)
  (P6) -- (P11)
  (P6) -- (P12)
  (P6) -- (P7)
  (P7) -- (P12)
  (P7) -- (P8)
  (P8) -- (P12)
  (P8) -- (P10)
  (P8) -- (P9)
  (P9) -- (P10)
  (P10) -- (P12)
  (P10) -- (P11)
  (P11) -- (P12)
;
\smalldot {P0};
\smalldot {P1};
\smalldot {P2};
\smalldot {P3};
\smalldot {P4};
\smalldot {P5};
\smalldot {P6};
\smalldot {P7};
\smalldot {P8};
\smalldot {P9};
\smalldot {P10};
\smalldot {P11};
\smalldot {P12};
\end{scope}
\begin{scope}[shift={(5,-5)}]
%tikz format auto generated by GentikzA.java
%format for fig.tex in flypaper.
%in LaTeX file, enclose code in tikzpicture environment
% invariant: 242652038506
%Set the coordinates of the points.
\coordinate  (P0) at (1.000000,0.000000);\coordinate  (P1) at (-0.000001,1.000000);\coordinate  (P2) at (-1.000000,-0.000002);\coordinate  (P3) at (0.000004,-1.000000);\coordinate  (P4) at (0.078211,-0.679962);\coordinate  (P5) at (-0.610774,-0.147903);\coordinate  (P6) at (-0.438406,0.374203);\coordinate  (P7) at (0.020099,0.613686);\coordinate  (P8) at (0.543809,0.256725);\coordinate  (P9) at (0.445403,-0.175374);\coordinate  (P10) at (0.072477,-0.235810);\coordinate  (P11) at (-0.284755,0.025574);\coordinate  (P12) at (0.083288,0.232480);%Draw edges.
\draw
  (P0) -- (P3)
  (P0) -- (P1)
  (P0) -- (P8)
  (P0) -- (P9)
  (P0) -- (P4)
  (P1) -- (P2)
  (P1) -- (P6)
  (P1) -- (P7)
  (P1) -- (P8)
  (P2) -- (P3)
  (P2) -- (P5)
  (P2) -- (P6)
  (P3) -- (P4)
  (P3) -- (P5)
  (P4) -- (P9)
  (P4) -- (P10)
  (P4) -- (P5)
  (P5) -- (P11)
  (P5) -- (P6)
  (P6) -- (P11)
  (P6) -- (P7)
  (P7) -- (P11)
  (P7) -- (P12)
  (P7) -- (P8)
  (P8) -- (P12)
  (P8) -- (P9)
  (P9) -- (P12)
  (P9) -- (P10)
  (P10) -- (P12)
  (P10) -- (P11)
  (P11) -- (P12)
;
\smalldot {P0};
\smalldot {P1};
\smalldot {P2};
\smalldot {P3};
\smalldot {P4};
\smalldot {P5};
\smalldot {P6};
\smalldot {P7};
\smalldot {P8};
\smalldot {P9};
\smalldot {P10};
\smalldot {P11};
\smalldot {P12};
\end{scope}
\begin{scope}[shift={(0,-7.5)}]
%tikz format auto generated by GentikzA.java
%format for fig.tex in flypaper.
%in LaTeX file, enclose code in tikzpicture environment
% invariant: 88089363170
%Set the coordinates of the points.
\coordinate  (P0) at (1.000000,0.000000);\coordinate  (P1) at (-0.000001,1.000000);\coordinate  (P2) at (-1.000000,-0.000002);\coordinate  (P3) at (0.000004,-1.000000);\coordinate  (P4) at (-0.155080,-0.658553);\coordinate  (P5) at (0.282365,-0.247170);\coordinate  (P6) at (-0.581501,-0.121043);\coordinate  (P7) at (-0.507495,0.316925);\coordinate  (P8) at (-0.009450,0.601710);\coordinate  (P9) at (0.495379,0.119018);\coordinate  (P10) at (0.163993,0.121155);\coordinate  (P11) at (-0.173735,-0.246943);\coordinate  (P12) at (-0.186995,0.151025);%Draw edges.
\draw
  (P0) -- (P3)
  (P0) -- (P1)
  (P0) -- (P9)
  (P0) -- (P5)
  (P1) -- (P2)
  (P1) -- (P7)
  (P1) -- (P8)
  (P2) -- (P3)
  (P2) -- (P4)
  (P2) -- (P6)
  (P2) -- (P7)
  (P3) -- (P4)
  (P4) -- (P5)
  (P4) -- (P11)
  (P4) -- (P6)
  (P5) -- (P9)
  (P5) -- (P10)
  (P5) -- (P11)
  (P6) -- (P11)
  (P6) -- (P12)
  (P6) -- (P7)
  (P7) -- (P12)
  (P7) -- (P8)
  (P8) -- (P12)
  (P8) -- (P10)
  (P8) -- (P9)
  (P9) -- (P10)
  (P10) -- (P12)
  (P10) -- (P11)
  (P11) -- (P12)
;
\smalldot {P0};
\smalldot {P1};
\smalldot {P2};
\smalldot {P3};
\smalldot {P4};
\smalldot {P5};
\smalldot {P6};
\smalldot {P7};
\smalldot {P8};
\smalldot {P9};
\smalldot {P10};
\smalldot {P11};
\smalldot {P12};
\end{scope}
\begin{scope}[shift={(2.5,-7.5)}]
%tikz format auto generated by GentikzA.java
%format for fig.tex in flypaper.
%in LaTeX file, enclose code in tikzpicture environment
% invariant: 75641658977
%Set the coordinates of the points.
\coordinate  (P0) at (1.000000,0.000000);\coordinate  (P1) at (0.309016,0.951057);\coordinate  (P2) at (-0.809018,0.587784);\coordinate  (P3) at (-0.809015,-0.587788);\coordinate  (P4) at (0.309021,-0.951055);\coordinate  (P5) at (0.187792,-0.702174);\coordinate  (P6) at (0.680658,0.064146);\coordinate  (P7) at (-0.062189,-0.439606);\coordinate  (P8) at (-0.617596,-0.230966);\coordinate  (P9) at (-0.433066,0.481096);\coordinate  (P10) at (0.175409,0.478259);\coordinate  (P11) at (0.266124,-0.065383);\coordinate  (P12) at (-0.243552,0.035498);%Draw edges.
\draw
  (P0) -- (P4)
  (P0) -- (P1)
  (P0) -- (P6)
  (P1) -- (P2)
  (P1) -- (P9)
  (P1) -- (P10)
  (P1) -- (P6)
  (P2) -- (P3)
  (P2) -- (P8)
  (P2) -- (P9)
  (P3) -- (P4)
  (P3) -- (P5)
  (P3) -- (P7)
  (P3) -- (P8)
  (P4) -- (P5)
  (P5) -- (P6)
  (P5) -- (P11)
  (P5) -- (P7)
  (P6) -- (P10)
  (P6) -- (P11)
  (P7) -- (P11)
  (P7) -- (P12)
  (P7) -- (P8)
  (P8) -- (P12)
  (P8) -- (P9)
  (P9) -- (P12)
  (P9) -- (P10)
  (P10) -- (P12)
  (P10) -- (P11)
  (P11) -- (P12)
;
\smalldot {P0};
\smalldot {P1};
\smalldot {P2};
\smalldot {P3};
\smalldot {P4};
\smalldot {P5};
\smalldot {P6};
\smalldot {P7};
\smalldot {P8};
\smalldot {P9};
\smalldot {P10};
\smalldot {P11};
\smalldot {P12};
\end{scope}
\begin{scope}[shift={(5,-7.5)}]
%tikz format auto generated by GentikzA.java
%format for fig.tex in flypaper.
%in LaTeX file, enclose code in tikzpicture environment
% invariant: 34970074286
%Set the coordinates of the points.
\coordinate  (P0) at (1.000000,0.000000);\coordinate  (P1) at (0.309016,0.951057);\coordinate  (P2) at (-0.809018,0.587784);\coordinate  (P3) at (-0.809015,-0.587788);\coordinate  (P4) at (0.309021,-0.951055);\coordinate  (P5) at (-0.031138,-0.619586);\coordinate  (P6) at (0.346777,-0.171245);\coordinate  (P7) at (-0.618072,-0.273500);\coordinate  (P8) at (-0.465434,0.287031);\coordinate  (P9) at (-0.032547,0.645875);\coordinate  (P10) at (0.547836,0.313561);\coordinate  (P11) at (0.107152,0.233092);\coordinate  (P12) at (-0.182322,-0.141405);%Draw edges.
\draw
  (P0) -- (P4)
  (P0) -- (P1)
  (P0) -- (P10)
  (P0) -- (P6)
  (P1) -- (P2)
  (P1) -- (P9)
  (P1) -- (P10)
  (P2) -- (P3)
  (P2) -- (P7)
  (P2) -- (P8)
  (P2) -- (P9)
  (P3) -- (P4)
  (P3) -- (P5)
  (P3) -- (P7)
  (P4) -- (P5)
  (P5) -- (P6)
  (P5) -- (P12)
  (P5) -- (P7)
  (P6) -- (P10)
  (P6) -- (P11)
  (P6) -- (P12)
  (P7) -- (P12)
  (P7) -- (P8)
  (P8) -- (P12)
  (P8) -- (P11)
  (P8) -- (P9)
  (P9) -- (P11)
  (P9) -- (P10)
  (P10) -- (P11)
  (P11) -- (P12)
;
\smalldot {P0};
\smalldot {P1};
\smalldot {P2};
\smalldot {P3};
\smalldot {P4};
\smalldot {P5};
\smalldot {P6};
\smalldot {P7};
\smalldot {P8};
\smalldot {P9};
\smalldot {P10};
\smalldot {P11};
\smalldot {P12};
\end{scope}
\begin{scope}[shift={(0,-10)}]
%tikz format auto generated by GentikzA.java
%format for fig.tex in flypaper.
%in LaTeX file, enclose code in tikzpicture environment
% invariant: 164470574315
%Set the coordinates of the points.
\coordinate  (P0) at (1.000000,0.000000);\coordinate  (P1) at (0.309016,0.951057);\coordinate  (P2) at (-0.809018,0.587784);\coordinate  (P3) at (-0.809015,-0.587788);\coordinate  (P4) at (0.309021,-0.951055);\coordinate  (P5) at (0.301829,-0.600457);\coordinate  (P6) at (-0.505993,-0.509200);\coordinate  (P7) at (-0.609018,0.297959);\coordinate  (P8) at (0.045883,0.622448);\coordinate  (P9) at (0.543510,0.228507);\coordinate  (P10) at (0.255666,-0.208506);\coordinate  (P11) at (-0.275237,-0.280677);\coordinate  (P12) at (-0.107787,0.177349);%Draw edges:
\draw
  (P0) -- (P4)
  (P0) -- (P1)
  (P0) -- (P9)
  (P0) -- (P10)
  (P0) -- (P5)
  (P1) -- (P2)
  (P1) -- (P8)
  (P1) -- (P9)
  (P2) -- (P3)
  (P2) -- (P7)
  (P2) -- (P8)
  (P3) -- (P4)
  (P3) -- (P6)
  (P3) -- (P7)
  (P4) -- (P5)
  (P4) -- (P6)
  (P5) -- (P10)
  (P5) -- (P11)
  (P5) -- (P6)
  (P6) -- (P11)
  (P6) -- (P7)
  (P7) -- (P11)
  (P7) -- (P12)
  (P7) -- (P8)
  (P8) -- (P12)
  (P8) -- (P9)
  (P9) -- (P12)
  (P9) -- (P10)
  (P10) -- (P12)
  (P10) -- (P11)
  (P11) -- (P12)
;
\smalldot {P0};
\smalldot {P1};
\smalldot {P2};
\smalldot {P3};
\smalldot {P4};
\smalldot {P5};
\smalldot {P6};
\smalldot {P7};
\smalldot {P8};
\smalldot {P9};
\smalldot {P10};
\smalldot {P11};
\smalldot {P12};
\end{scope}
\begin{scope}[shift={(2.5,-10)}]
%tikz format auto generated by GentikzA.java
%format for fig.tex in flypaper.
%in LaTeX file, enclose code in tikzpicture environment
% invariant: 100126458338
%Set the coordinates of the points:
\coordinate  (P0) at (1.000000,0.000000);\coordinate  (P1) at (0.309016,0.951057);\coordinate  (P2) at (-0.809018,0.587784);\coordinate  (P3) at (-0.809015,-0.587788);\coordinate  (P4) at (0.309021,-0.951055);\coordinate  (P5) at (-0.255660,-0.569981);\coordinate  (P6) at (-0.131571,-0.213048);\coordinate  (P7) at (0.474711,0.048551);\coordinate  (P8) at (-0.623853,-0.105106);\coordinate  (P9) at (-0.355461,0.246241);\coordinate  (P10) at (-0.103076,0.620920);\coordinate  (P11) at (0.464454,0.459073);\coordinate  (P12) at (0.123396,0.198054);%Draw edges:
\draw
  (P0) -- (P4)
  (P0) -- (P1)
  (P0) -- (P11)
  (P0) -- (P7)
  (P1) -- (P2)
  (P1) -- (P10)
  (P1) -- (P11)
  (P2) -- (P3)
  (P2) -- (P8)
  (P2) -- (P9)
  (P2) -- (P10)
  (P3) -- (P4)
  (P3) -- (P5)
  (P3) -- (P8)
  (P4) -- (P5)
  (P5) -- (P6)
  (P5) -- (P8)
  (P6) -- (P7)
  (P6) -- (P12)
  (P6) -- (P9)
  (P6) -- (P8)
  (P7) -- (P11)
  (P7) -- (P12)
  (P8) -- (P9)
  (P9) -- (P12)
  (P9) -- (P10)
  (P10) -- (P12)
  (P10) -- (P11)
  (P11) -- (P12)
;
\smalldot {P0};
\smalldot {P1};
\smalldot {P2};
\smalldot {P3};
\smalldot {P4};
\smalldot {P5};
\smalldot {P6};
\smalldot {P7};
\smalldot {P8};
\smalldot {P9};
\smalldot {P10};
\smalldot {P11};
\smalldot {P12};
\end{scope}
\begin{scope}[shift={(5,-10)}]
%tikz format auto generated by GentikzA.java
%format for fig.tex in flypaper.
%in LaTeX file, enclose code in tikzpicture environment
% invariant: 215863975889
%Set the coordinates of the points:
\coordinate  (P0) at (1.000000,0.000000);\coordinate  (P1) at (0.309016,0.951057);\coordinate  (P2) at (-0.809018,0.587784);\coordinate  (P3) at (-0.809015,-0.587788);\coordinate  (P4) at (0.309021,-0.951055);\coordinate  (P5) at (0.220027,-0.705352);\coordinate  (P6) at (-0.593564,-0.346666);\coordinate  (P7) at (-0.396140,0.121754);\coordinate  (P8) at (-0.196369,0.551778);\coordinate  (P9) at (0.377559,0.476590);\coordinate  (P10) at (0.557184,-0.120630);\coordinate  (P11) at (-0.045615,-0.332596);\coordinate  (P12) at (0.152204,0.051028);%Draw edges:
\draw
  (P0) -- (P4)
  (P0) -- (P1)
  (P0) -- (P9)
  (P0) -- (P10)
  (P0) -- (P5)
  (P1) -- (P2)
  (P1) -- (P8)
  (P1) -- (P9)
  (P2) -- (P3)
  (P2) -- (P6)
  (P2) -- (P7)
  (P2) -- (P8)
  (P3) -- (P4)
  (P3) -- (P5)
  (P3) -- (P6)
  (P4) -- (P5)
  (P5) -- (P10)
  (P5) -- (P11)
  (P5) -- (P6)
  (P6) -- (P11)
  (P6) -- (P7)
  (P7) -- (P11)
  (P7) -- (P12)
  (P7) -- (P8)
  (P8) -- (P12)
  (P8) -- (P9)
  (P9) -- (P12)
  (P9) -- (P10)
  (P10) -- (P12)
  (P10) -- (P11)
  (P11) -- (P12)
;
\smalldot {P0};
\smalldot {P1};
\smalldot {P2};
\smalldot {P3};
\smalldot {P4};
\smalldot {P5};
\smalldot {P6};
\smalldot {P7};
\smalldot {P8};
\smalldot {P9};
\smalldot {P10};
\smalldot {P11};
\smalldot {P12};
\end{scope}
}
}

\def\figCXFENOK{
\tikzfig{fthex}{\guid{CXFENOK} This hypermap is not a contact fan.}
{
[scale=0.004]
% java render/Gentikz "125913905253 14 6 0 1 2 3 4 5 3 0 5 6 3 6 5 7 3 7 5 4 3 7 4 8 3 8 4 3 3 8 3 9 3 9 3 2 3 9 2 10 3 10 2 1 3 10 1 11 3 11 1 0 3 11 0 6 6 6 7 8 9 10 11 "
%tikz format auto generated by Gentikz, then hand edited.
%in LaTeX file, enclose code in tikzpicture environment
% invariant: 125913905253
%Set the coordinates of the points:
%\tikzstyle{every node}=[draw,shape=circle];
\path ( 400,0) node (P0) {};
\path (60:400)  node (P1) {};
\path (120:400) node (P2) {};
\path ( -400,0) node (P3) {};
\path ( -200,-346) node (P4) {};
\path ( 200,-346) node (P5) {};
\path (330:220) node (P6) {};
\path (270:220) node (P7) {};
\path(210:220) node (P8) {};
\path (30:220) node (P11) {};
\path (150:220) node (P9) {};
\path (90:220) node (P10) {}; 
%Draw edges:
\draw
  (P0) -- (P5)
  (P0) -- (P1)
  (P0) -- (P11)
  (P0) -- (P6)
  (P1) -- (P2)
  (P1) -- (P10)
  (P1) -- (P11)
  (P2) -- (P3)
  (P2) -- (P9)
  (P2) -- (P10)
  (P3) -- (P4)
  (P3) -- (P8)
  (P3) -- (P9)
  (P4) -- (P5)
  (P4) -- (P7)
  (P4) -- (P8)
  (P5) -- (P6)
  (P5) -- (P7)
  (P6) -- (P11)
  (P6) -- (P7)
  (P7) -- (P8)
  (P8) -- (P9)
  (P9) -- (P10)
  (P10) -- (P11)
;
\foreach \i in {P0,P1,P2,P3,P4,P5,P6,P7,P8,P9,P10,P11} {
  \smalldot {\i};
}
}}



%%%%%%%%%%%%%%%%%%%%%%%%%%%%%%%%%%%%%%%%%%%%%%%%%%%%%%%%%%%%%%%%%%%%%%%%%%%%%%%%
%% SAMPLE AND DEPRECATED FIGURES
%%%%%%%%%%%%%%%%%%%%%%%%%%%%%%%%%%%%%%%%%%%%%%%%%%%%%%%%%%%%%%%%%%%%%%%%%%%%%%%%


%
%\tikzfig{circle}{\guid{HGMTQFG} Lemma~\ref{lemma:circle} as a special case of the Pythagorean theorem}
%{
%[scale=0.1]
%\draw (0,0)  --(12,0) --  (12,5) --  cycle;
%\draw[thin] (11,0) -- (11,1) -- (12.0,1);
%\path (5,-1.5) node {$\cos x$};
%\path (16,2.5) node {$\sin x$};
%\path (6,5)  node {$1$};
%}


\def\sampleArc{
\tikzfig{sample-arc}{A sample arc.}
{
[scale=1.0]
\draw (0,0) arc[x radius=0.2,y radius=2.0,start angle=30,rotate=0,end angle=270] ;
\draw (0,0) -- (0:2);
\draw (0,0) -- (30:2);
\draw (0,0) -- (60:2);
\draw (0,0) -- (90:2);
\draw (-1,-1)--(2,-1);
\draw (-1,0)--(2,0);
\draw (-1,1)--(2,1);
}
}

%
\def\sampleA{
\tikzfig{circle}{\guid{HGMTQFG} Lemma~\ref{lemma:circle} as a special case of the Pythagorean theorem.}
{
[scale=0.2]
\draw (0,0)  --(12,0) --  (12,5) --  cycle;
\draw[thin] (11,0) -- (11,1) -- (12.0,1);
\path (5,-1.5) node {$\cos x$};
\path (16,2.5) node {$\sin x$};
\path (6,5)  node {$1$};
}
}

%
\def\sampleB{
\tikzfig{tan}{\guid{GQQAKYI} 
The tangent function on $\leftopen-\pi/2,\pi/2\rightopen$.}
{
[scale=0.2]
\draw plot[smooth] file {tikz/tan.table};
\draw[help lines,->] (-1.57,0) -- (1.57,0);
\draw[help lines,->] (0,-6.0) -- (0,6.0);
}
}

\def\sampleC{
\tikzfig{arctrig}{\guid{RUJPPWJ} 
The arctangent function on the domain \leftopen -4,4\rightopen\ 
and the $\arccos$ function on $\leftclosed-1,1\rightclosed$.}
{
[scale=0.4]
\draw plot[smooth] file {tikz/arctan.table};
\draw plot[smooth] file {tikz/arccos.table};
\draw[help lines,->] (0,-0.2) -- (0,3.2);
\draw[help lines,->] (-4,0) -- (4,0);
}
}

\def\sampleD{
\tikzfig{atn-polar}{\guid{YOXQFUB} 
The function $\atn$ gives the polar angle $\theta$ of $(x,y)$.}
{
[scale=0.25]
\draw[gray,->,thin] (-4,0) -- (14,0);
\draw[gray,->,thin] (0,-2) -- (0,5);
\draw (0,0)  --(12,0) --  (12,5) --  cycle;
\draw[thin] (11,0) -- (11,1) -- (12.0,1);
\path (6,-1.5) node {$x$};
\draw[thin] (4,0) arc (0:22.62:4);
\path (14,2.5) node {$y$};
\path (6,2) node {$\theta$};
}
}

\def\rrkrgpvjw#1#2{\shade[ball color=gray](#1,#2) circle (1);  }

\def\figKRGPVJW{
\tikzfig{svdw}
{\guid{KRGPVJW} The Sch\"utte-van der Waerden contact graph and packing.  
Four edges that
belong to the standard graph but not the contact graph are shown in gray.  Twelve
balls in the packing are centered near the centers of the edges of a cube.}
{
{
\begin{scope}[scale=0.004]
%Set the coordinates of the points:
%\tikzstyle{every node}=[draw,shape=circle];
\path (45:400) coordinate (P0) ;
\path (135:400)  coordinate (P1) ;
\path (225:400) coordinate (P2) ;
\path (315:400) coordinate (P3) ;
\path (0:200) coordinate (P4) ;
\path (90:200) coordinate (P5) ;
\path (180:200) coordinate (P6) ;
\path (270:200) coordinate (P7) ;
\path(45:150) coordinate (P8) ;
\path (135:150) coordinate (P9) ;
\path (225:150) coordinate (P10) ;
\path (315:150) coordinate (P11) ; 
\path (0,0) coordinate (P12) ;
\foreach \i in {0,...,12}
{
  \fill (P\i) circle (15);
}
%Draw edges:
\draw
  (P12) -- (P8)
  (P12) -- (P9)
  (P12) -- (P10)
  (P12) -- (P11)
  (P8) -- (P4)
  (P4) -- (P11)
  (P11) -- (P7)
  (P7) -- (P10)
  (P10) -- (P6)
  (P6) -- (P9)
  (P9) -- (P5)
  (P5) -- (P8)
%
  (P0) -- (P1)
  (P1) -- (P2)
  (P2) -- (P3)
  (P3) -- (P0)
%
  (P0) -- (P5)
  (P5) -- (P1)
  (P1) -- (P6)
  (P6) -- (P2)
  (P2) -- (P7)
  (P7) -- (P3)
  (P3) -- (P4)
  (P4) -- (P0);
\draw[gray,thin]
  (P8) -- (P9)
  (P9) -- (P10)
  (P10)--(P11)
  (P11)--(P8);
\end{scope}
%
\begin{scope}[scale=0.5,xshift=8cm]
\def\rr{\rrkrgpvjw}
\rr{-0.504725}{0.79793}
\rr{0.987379}{-0.530059}
\rr{-0.406371}{-1.76776}
\rr{-1.8337}{-0.370827}
\rr{1.68242}{1.01951}
\rr{0.}{2.0538}
\rr{1.35457}{-1.58937}
\rr{0.}{0.}
\rr{-1.68242}{1.20943}
\rr{-1.35457}{-1.43645}
\rr{1.8337}{0.000711695}
\rr{0.504725}{1.431}
\rr{0.406371}{-1.25805}
\rr{-0.987379}{0.159943}
\end{scope}
%\shade[ball color=blue] (2,2) circle (1); % color = gray
%\shade[ball color=blue] (2.5,2) circle (1); % color = gray
}
}}

