% ------------------------------------------------------------ 
% Author: Thomas C. Hales 
% Format: LaTeX Book Section: Dense Sphere Packings
% ------------------------------------------------------------

%\chapter{Further Results}



\section{Strong Dodecahedral Theorem}
\label{sec:further}

The same methods that have been used to prove the Kepler conjecture
can be used to prove some other longstanding conjectures in discrete
geometry.  This section gives a proof of the strong dodecahedral
conjecture.

Earlier sections are written in a formal blueprint style.
Complete proofs are provided, even for statements that might be
viewed as geometrically obvious.  In this section, we relax our
standards of proof just a bit.  What we write is still a proof by
traditional mathematical standards, but not as detailed as earlier
chapters.


Bezdek has conjectured that the Voronoi cell of smallest surface is
the regular dodecahedron with unit inradius.  This is the
\newterm{strong dodecahedral conjecture}~\cite{Bezdek00}, \cite{Bezdek05}.  
%
\indy{Index}{Bezdek, K.}%
\indy{Index}{decomposition!Voronoi}%


\begin{theorem}[strong dodecahedral conjecture]\guid{HKJSPQG}\hspace{-4pt}
  The surface area of a Voronoi cell in a packing is at least the
  surface area of the regular dodecahedron with unit inradius.
\end{theorem}
\indy{Index}{dodecahedral conjecture}%


This section gives a proof of this theorem.  We
begin with some simple observations.

\begin{remark}
  If a packing is  saturated, then the surface area of a Voronoi
  cell is finite.  \claim{We may assume without loss of 
    generality that the packing is saturated.}
  Indeed, consider a new facet $F$ that is created on a Voronoi cell
  by the addition of a new point to the packing $V$.  Let $X$ be the
  polygonal boundary of the new facet.  The area minimizing surface
  that has $X$ as a boundary is the facet $F$.  Thus the new facet
  replaces a surface of larger area with a surface of smaller area.
  That is, by saturating a packing, the surface area of a Voronoi cell
  can only decrease.
\end{remark}

\begin{remark}
As in the proof of the Kepler conjecture, the truncation of a Voronoi cell is easier
to study than the Voronoi cell itself.  In order to obtain sharp bounds, the truncation
must have no effect on the optimal Voronoi cell.  This constraint forces the
 truncation parameter to be at least
the circumradius $\sqrt{3}\tan{\pi/5}\approx 1.258$ of the regular dodecahedron.   The truncation parameter $\sqrt{2}$
that we use in the proof of the Kepler conjecture satisfies this constraint and is
therefore  well-suited for the strong dodecahedral
conjecture.
\end{remark}

\begin{lemma}[]\guid{JXVEXYV} Let $r_0\ge0$.  % X->Y
  The surface area of the Voronoi cell $\Omega(V, \u_0)$ is at least
  that of $\Omega(V, \u_0)\cap B(r_0)$, where $B(r_0)$ is the ball of radius
  $r_0$ centered at $ \u_0$.  
\end{lemma}
\indy{Index}{surface area}%


\begin{proof} The surface element for a parameterized surface
  $r(\theta,\phi)$ in spherical coordinates is
\[
  % ff = {r[theta, phi] Cos[theta]Sin[phi], r[theta, phi]
  %   Sin[theta]Sin[phi], r[theta, phi] Cos[phi]}; n = Cross[D[ff,
  % theta], D[ff, phi]]; n.n // Simplify
%
  r \nsqrt{r_\theta^2 + (r^2 + r_\phi^2)\sin^2\phi } \,\,d\theta\,d\phi,
\]
which is at least the surface element $r_0^2 \sin\phi\, d\theta\,d\phi$
of a sphere of radius $r_0$, provided
$r(\theta,\phi)\ge r_0$.   Hence, projection of a surface outside
sphere onto the sphere is area decreasing.
\end{proof}


Fejes T\'oth's classical dodecahedral conjecture is the corresponding conjecture
about volumes rather than surface areas, asserting that  the
Voronoi cell of smallest volume is the regular dodecahedron of
unit inradius.  \indy{Index}{fejestoth@Fejes T\'oth, L.}%
The strong dodecahedral conjecture yields the dodecahedral conjecture
as a corollary.

\begin{lemma}[]\guid{QRBKJAW}
  If the surface area of a Voronoi cell is at least the surface area
  of a regular dodecahedron with unit inradius, then its volume is also
   at least that  of a regular dodecahedron.
\end{lemma}

\begin{proof} Let $A_1,\ldots,A_n$ be the areas of the facets of a
  Voronoi cell.  Let $h_1,\ldots,h_n$ be the distances from the affine
  hulls of the facets to the center of the Voronoi cell.  Then $h_i\ge
  1$.  Assume that $\sum A_i \ge A_D$, where $A_D$ is the surface area
  of a regular dodecahedron.  Then its volume is
\[
\op{vol} = \sum A_i h_i/3 \ge \sum A_i/3 \ge A_D/3 = \op{vol}_D,
\]
where $\op{vol}_D$ is the volume of the regular dodecahedron.
\end{proof}
\indy{Index}{regular dodecahedron!volume}%
\indy{Notation}{A@$A$ (Voronoi cell face area)}%
\indy{Notation}{volD@$\op{vol}_D$ (volume of dodecahedron)}%


\subsection{$D$-cells}



The notation follows Section~\ref{sec:rogers}.  Let $V$ be a saturated
packing. 
%If $\bu =[\u_0;\u_1;\u_2;\u_3]\in \bV(3)$, then let $b(\bu) = h([
%\u_0; \u_1; \u_2])$.  
Let $\Omega( V,\u_0)$ be a Voronoi cell with  Rogers's partition 
\[
\Omega(V, \u_0) = \bigcup \ \{ R( \bu) \mid { \bu\in  \bV(3),\quad \trunc{\bu}{0}= [\u_0] }\}.
\]
\indy{Notation}{zzZ@$\Omega$ (Voronoi cell)}%
\indy{Index}{decomposition!Rogers}%

\begin{definition}[$B$,~$D_k$-cell]
  Let $B$ be the ball of radius $\sqrt2$ centered at
  $\u_0$.  We define $D_k$-cells for $k=1,2,3,4$ for each $
  \bu=[\u_0;\ldots;\u_3]\in \bV(3)$ by
\[
D_k(\bu) = \Omega(V,\u_0)\cap \cell(\bu,k),
\]
where $\cell(\bu,k)$ is the  Marchal $k$-cell of $\bu$.
\end{definition}
%
\indy{Notation}{D@$D_k$ ($D$-cell)}%
\indy{Index}{D-cell}%


A $D_k$-cell, which is a subset of
$\Omega(V,\u_0)\cap B$, is the adaptation of a $k$-cell to the
geometry of the strong dodecahedral conjecture.  

\begin{lemma}[]\guid{ZERRZRM}
  Let $V$ be a saturated packing and let $\u_0\in V$.  If the intersection
of a $D_i$-cell with a $D_j$-cell is not a null set, then $i=j$ and the two
cells are equal.   The union
  of all the $D_k$-cells at $\u_0$ is $\Omega(V, \u_0)\cap B$.
\end{lemma}
\indy{Index}{null set}%

\begin{proof} This follows from the corresponding facts for cells in
  Lemma~\ref{lemma:marchal-partition}.  Each null set is in fact a subset
  of a plane.
\end{proof}


\subsection{surface area and dihedral angle}

Every cell $D_k(\bu)$ is eventually radial at $ \u_0$ and has a
solid angle $\sol(\u_0,D_k(\bu))$.  Every cell $D_k(\bu)$ has
an \newterm{exposed} surface area $\op{surf}(D_k(\u))$, the area of
the intersection of $D_k(\bu)$ and the boundary of $\Omega(V,
\u_0)\cap B$.  It consists of the sum of the areas of the analytic
facets (linear or spherical surfaces) that do not meet the point $
\u_0$.  The total surface area of $\Omega(V,\u_0) \cap B$ is the sum of the
exposed surface areas $\op{surf}(D_k(\u))$.  \indy{Index}{exposed}%

We use the functions $\dih_i$ in
\eqref{eqn:dihi} to introduce a function of six variables $y=(y_1,y_2,\ldots,y_6)$:
\[
\op{soly}(y)=
\dih_1(y)+\dih_2(y)+\dih_3(y)-\pi.
\]
By Girard's formula for the solid angle of a simplex 
(Lemma~\ref{lemma:prim-volume}),
\[
\op{soly}(y_1,y_2,y_3,y_4,y_5,y_6)=
\sol(\u_0,\op{conv}\{\u_0,\u_1,\u_2,\u_3\})
\]
when
\[
y_i = \begin{cases}
\norm{\u_0}{\u_i}, & i\in\{1,2,3\},\\
\norm{\u_j}{\u_k}, & i\in\{4,5,6\}\textand\{i-3,j,k\} = \{1,2,3\}.\\
\end{cases}
\]
\indy{Index}{surface area!exposed}%
\indy{Notation}{surf@$\op{surf}$ (surface area)}%
\indy{Notation}{solid@$\sol$ (solid angle)}%
\indy{Notation}{solid@$\op{soly}$ (solid angle as a function of edges)}%


Every cell $D_k(\bu)$ has a set $E(k,\bu)$ of distinguished edges
(that is, the edges of the cell that extend from $\u_0$ to a midpoint $(\u_0+\u_i)/2$) and a
dihedral angle $\dih(e)$ for $e\in E(k,\bu)$.  Each edge has a length
$h(e)\in\leftclosed1,\sqrt2\rightclosed$.  \indy{Index}{angle!dihedral}%
\indy{Notation}{dih}%
\indy{Index}{edge!length}%
%\indy{Notation}{E3a@$E(k,\bu)$ (distinguished edge set of a $D$-cell)}% % doesn't parse

\subsection{local inequality}

We reduce the strong dodecahedral conjecture to an estimate from an earlier
chapter \eqref{conj:L12} and a
local inequality.  



\begin{definition}[$a_D$,~$b_D$,~$y_D$,~$\nu_D$,~$f$]\guid{TCOSFNQ}
  Define constants $a_D$, $b_D$, $y_D$, and functions $\nu_D$, $f$ as
  follows.  Let $y_D\approx 2.1029$ be defined by the condition
\[
\op{soly}(2,2,2,y_D,y_D,y_D) = \pi/5.
\]
For any $\u_i\in\ring{R}^3$, let $g(\u_0,\u_1,\u_2,\u_3)$ be the
volume of the intersection of the convex hull of
$S=\{\u_0,\ldots,\u_3\}$ with set of points closer to $ \u_0$ than to
any other point in $S$.  When
\begin{equation}\label{eqn:uy}
  \norm{\u_0}{\u_i}=2\textand \norm{ \u_i}{ \u_j} = y\text{ for }i,j\ge 1,
\end{equation} 
this volume depends only on $y$. Write $\nu(y) = g(\u_0,\ldots,\u_3)$.
Set
% Sep 20, 2011
% 3 b , correct for the sum over 3 terms in eqn. below.
% The 3 is not for volume, it is for the 3 radial edges.
% compare ineq "9627800748 a".
\[
  f(y;a,b) = \nu(y) +  a\, \op{soly}(2,2,2,y,y,y) + 3 b \dih(2,2,2,y,y,y).
\]
The linear system
\begin{equation}\label{eqn:fyD}
f(y_D;a,b) = 0,\quad \dfrac{\partial f}{\partial y}(y_D;a,b) = 0
\end{equation}
has a unique solution in $a,b$ with values $a=a_D\approx -0.581$,
$b=b_D\approx 0.0232$.
\end{definition}
\indy{Notation}{am@$a_D$ (dodecahedral parameter)}%
\indy{Notation}{b@$b_D$ (dodecahedral parameter)}%
\indy{Notation}{yd@$y_D$ (dodecahedral parameter)}%
\indy{Notation}{zzn@$\nu_D$ (dodecahedral function)}%
\indy{Index}{convex hull}%
\indy{Index}{regular dodecahedron!volume}%

Note that the regular dodecahedron has volume $20 \nu(y_D)$ and surface
area $60 \nu(y_D)$.  Also,
\begin{equation}
  2\,\op{soly}(2,2,2,y_D,y_D,y_D) =\dih(2,2,2,y_D,y_D,y_D)=2\pi/5.
\end{equation}
\indy{Index}{regular dodecahedron}%
\indy{Index}{regular dodecahedron!surface area}%

\begin{lemma}[local inequality]\guid{PWVDMPT}\label{lemma:D-local}
For any cell $D_k(\bu)$
\[
  \op{surf}(D_k(\bu)) + 3 a_D \sol(D_k(\bu)) 
+ 3 b_D \sum_{e\in E(k,\bu)} L(h(e)) \dih(e) \ge 0,
\]
where $L$ is the function of Definition~\ref{def:L}.  Equality holds
precisely when the cell is a null set or a $4$-cell with edges
$(2,2,2,y_D,y_D,y_D)$.
\end{lemma}
\indy{Index}{local inequality}%



For a cell $D_4(\bu)$ with parameters of the form
  \eqref{eqn:uy}, the local inequality reduces to the inequality
  $f(y;a_D,b_D)\ge 0$.  The constants $a_D$ and $b_D$ are chosen so that
  $y=y_D$ is a critical point of $f$ with value $f(y_D;a_D,b_D)=0$.  In
  particular, the local inequality asserts that $f$ has a local
  minimum at $y=y_D$.


\begin{proof} 
  The dimension of the set of all $D_k$-cells up to rigid motion 
is $\tbinom{k}{2}$, which is at most six.  
This cell inequality is a nonlinear inequality in a small number of variables
and is verified by a 
\cc{TNVWUGK}{}.  % The $D_4$-inequality is 9627800748 and $D_3$ is 6938212390.
\end{proof}

\begin{lemma}[]\guid{OIEKCEZ}
  The local inequality and the estimate \eqref{conj:L12}
\[
\sum L(h) \le 12
\]
imply the strong dodecahedral conjecture.
\end{lemma}

\begin{proof} 
  Sum the local inequality over all the $D_k$-cells in a Voronoi cell.
  The solid angles sum to $4\pi$ and the dihedral angles around each
  edge sum to $2\pi$:
\begin{align*}
  \op{surf}(\Omega) &\ge \op{surf}(\Omega\cap B)\\
  &=\sum_{k,D_k\text{-cell}} \op{surf}(D_k(\bu))\\
  &\ge -12\pi a_D - 6\pi\, b_D  \sum L(h)\\
  &\ge -12\pi a_D - 72\pi\, b_D\\
  &= -60 \op{soly}(2,2,2,y_D,y_D,y_D) a_D - 180 \dih(2,2,2,y_D,y_D,y_D) b_D\\
  &= 60 (\nu(y_D) - f(y_D;a_D,b_D))\\
  &= 60 \nu(y_D).\\
\end{align*}
The final term is the surface area of a regular dodecahedron.
\end{proof}

The case of equality occurs only for the regular dodecahedron.
We note that the strong dodecahedral conjecture follows from the same
estimate~\eqref{eqn:L12} that is used to prove the Kepler conjecture.


\begin{exer}
  Recently, Musin and Tarasov solved the Tammes problem for $k=13$
  points on a sphere in dimension $n=2$~\cite{Musin-Tarasov}.  In
  parting, we leave it as a challenging problem to adapt their
  solution to the framework of this book to obtain an independent
  solution to the Tammes problem.
\end{exer}
