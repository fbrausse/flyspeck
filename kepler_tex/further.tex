% ------------------------------------------------------------ 
% Author:
% Thomas C. Hales 
% Format: LaTeX Book Chapter: Dense Sphere Packings
% ------------------------------------------------------------

\chapter{Two Further Results}


The same methods that have been used to prove the Kepler conjecture can
be used to prove some other longstanding conjectures in discrete geometry.
This chapter sketches proofs of the strong dodecahedral conjecture and Fejes T\'oth's
full contact conjecture.

Earlier chapters were written in a ``formal blueprint'' style.   Complete proofs
were provided, even for statements that might be viewed as geometrically obvious.  In
this chapter, we relax our standards of proof just a bit.  What we write will still be a
proof by traditional mathematical standards, but not as detailed as earlier chapters.

\section{Strong Dodecahedral Theorem}

K. Bezdek has conjectured that the Voronoi cell of smallest surface is
the regular dodecahedron with inradius $1$.  This is known as the
strong dodecahedral conjecture.  \indy{Index}{Bezdek, K.}%
\indy{Index}{Voronoi cell}%


\begin{theorem}[strong dodecahedral conjecture]\guid{HKJSPQG}
  The surface area of a Voronoi cell in a packing is at least the
  surface area of the regular dodecahedron with inradius $1$.
\end{theorem}
\indy{Index}{dodecahedral conjecture}
\indy{Index}{dodecahedral conjecture!strong}

This section sketches a proof of the strong dodecahedral theorem.  We
begin with some simple observations.

\begin{remark}
  If a packing is not saturated, then the surface area of a Voronoi
  cell may be infinite.  \claim{In fact, there is no loss of
    generality in restricting our attention to saturated packings.}
  Indeed, consider a new facet $F$ that is created on a Voronoi cell
  by the addition of a new point to the packing $V$.  Let $X$ be the
  polygonal boundary of the new facet.  The area minimizing surface
  that has $X$ as a boundary is the facet $F$.  Thus the new facet
  replaces a surface of larger area with a surface of smaller area.
  That is, by saturating a packing, the surface area of a Voronoi cell
  can only decrease.
\end{remark}


\begin{lemma}[]\guid{JXVEXXV}
  The surface area of the Voronoi cell $\Omega(V, \u_0)$ is at least
  that of $\Omega(V, \u_0)\cap B$, where $B$ is the ball of radius
  $1.26$ centered at $ \u_0$.
\end{lemma}
\indy{Index}{surface area}%
\indy{Index}{Dodecahedral Conjecture}%

\begin{proof} The surface element for a parametrized surface
  $r(\theta,\phi)$ in spherical coordinates is
\begin{displaymath}
  % ff = {r[theta, phi] Cos[theta]Sin[phi], r[theta, phi]
  %   Sin[theta]Sin[phi], r[theta, phi] Cos[phi]}; n = Cross[D[ff,
  % theta], D[ff, phi]]; n.n // Simplify
%
  r \sqrt{r_\theta^2 + (r^2 + r_\phi^2)\sin^2\phi } \,\,d\theta\,d\phi 
\ge r_0^2 \sin\phi\, d\theta\,d\phi,
\end{displaymath}
when $r(\theta,\phi)\ge r_0$.  The right-hand side is the surface
element of a sphere of radius $r_0$.  Hence projection of a surface
outside sphere onto the sphere is area decreasing.
\end{proof}


L. Fejes T\'oth's classical dodecahedral conjecture asserts that the
Voronoi cell of smallest volume is the regular dodecahedron of
inradius $1$.  \indy{Index}{fejestoth@Fejes T\'oth, L.}%
\indy{Index}{Dodecahedral Conjecture}%
The strong dodecahedral conjecture yields the dodecahedral conjecture
as a corollary.

\begin{lemma}[]\guid{QRBKJAW}
  If the surface area of a Voronoi cell is at least the surface area
  of a regular dodecahedron with inradius $1$, then the volume of a
  Voronoi cell is at least the volume of a regular dodecahedron.
\end{lemma}

\begin{proof} Let $A_1,\ldots,A_n$ be the areas of the faces of a
  Voronoi cell.  Let $h_1,\ldots,h_n$ be the distances from the affine hulls of the faces
  to the center of the Voronoi cell.  Then $h_i\ge 1$.  Assume that
  $\sum A_i \ge A_D$, where $A_D$ is the surface area of a dodecahedron.
  Then its volume is
\begin{displaymath}
\op{vol} = \sum A_i h_i/3 \ge \sum A_i/3 = \op{vol}_D,
\end{displaymath}
where $\op{vol}_D$ is the volume of the regular dodecahedron.
\end{proof}
\indy{Index}{regular dodecahedron!volume}%
\indy{Notation}{A@$A$ (Voronoi cell face area)}%
\indy{Notation}{V@$\op{vol}_D$ (volume of dodecahedron)}%


\subsection{$D$-cells}



The notaton follows Section~\ref{sec:rogers}.  Let $V$ be a saturated
packing.

Let $\Omega( V,\u_0)$ be a Voronoi cell.  Rogers's partition of the cell
is
\begin{displaymath}
\Omega(V, \u_0) = \bigcup_{ \bu\in  \bV(3), \trunc{\bu}{0}= [\u_0] } R( \bu).
\end{displaymath}
\indy{Notation}{ZZZomega@$\Omega$ (Voronoi cell)}%
\indy{Index}{Rogers's partition}%


Define $D_k$-cells for $k=2,3,4$, for each $ \bu=[\u_0;\ldots;\u_3]\in
\bV(3)$.  Recall that $B$ is the ball of radius $1.26$ centered at $
\u_0$.  If $\bu =[\u_0;\u_1;\u_2;\u_3]\in \bV(3)$, let $b(\bu) = h([
\u_0; \u_1; \u_2])$.  \indy{Notation}{D@$D_k$}%
\indy{Index}{D-cell}%

\begin{definition}[$D_4$]
  Define a $D_4$-cell to be empty unless $h(\bu)<1.3$.  If
  $h(\bu)<1.3$ define it to be
\begin{displaymath}
  D_4(\bu)=\bigcup_{\omega( \bu)
=\omega(\bv),  \v\in  \bV(3),\ \trunc{\v}{0}= \trunc{\u}{0}}  R(\v)\cap B.
\end{displaymath}
\indy{Index}{edge!length}%
\end{definition}

The $D_4$-cell is equal to
\begin{displaymath}
  \op{conv}\{ \u_0, \u_1, \u_2, \u_3\} \cap \Omega(V, \u_0)\cap B.
\end{displaymath}
The circumradius function ($\rad$) is monotonic in the edge lengths
when $h(\bu)<1.3$.  If any edge $\norm{ \u_i}{ \u_j}$ is greater than
$2.52$, then
\begin{displaymath}
h(\bu) > \rad(2,2,2,2,2,2.52) > 1.3.
\end{displaymath}  
Hence all edges of $\op{conv}\{\u_0,\ldots,\u_3\}$ have length at most
$2.52$.

\begin{definition}[$D_3$]
  Define the $D_3$-cell to be empty unless $h(\bu)\ge 1.3$ and
  $b(\bu)< 1.26$.  If $h(\bu) \ge 1.3$ and $b(\bu)< 1.26$, there is a
  unique point $\xi$ on the segment from $\omega( \trunc{\bu}{2})$ to
  $\omega(\bu)$ at distance $1.3$ from $ \u_0$.  Let
\begin{displaymath}
  R_1( \u) = \op{conv}(\{ \u_0,\omega( \trunc{\bu}{1}),\omega( \trunc{\bu}{2}),\xi\} ),\ 
  R_2( \u) = \op{conv}(\{ \u_0,\omega( \trunc{\bu}{1}),\xi,\omega( \u)\} ).
\end{displaymath}
Let $ \bu' = [ \u_0; \u_2; \u_1; \u_3]$.
Define the $D_3$-cell to be
\begin{displaymath}
D_3(\bu) = (R_1( \bu) \cup R_1( \bu'))\cap B.
\end{displaymath}
\end{definition}

The condition $b(\bu)< 1.26$ implies the constraint $\norm{ \u_i}{
  \u_j}< 2.62$, for $i,j\in\{0,1,2\}$.

\begin{definition}[$D_2$]
  If the $D_4$ cell and $D_3$ cell of $\bu$ are empty (that is, if
  $b(\bu)\ge 1.26$), then let the $D_2$ cell be $D_2(\bu)=R(\bu)\cap
  B$.  Otherwise $b(\bu)< 1.26$ and $h(\bu) \ge 1.3$.  In this case,
  let the $D_2$ cell be the remainder: $D_2(\bu)=R_2(\bu)\cap B$.
\end{definition}

\begin{lemma}[]\guid{ZERRZRM}
  Let $V$ be a saturated packing and let $\u_0\in V$.  The
  intersection of two distinct $D_k$-cells is a null set.  The union
  of all the $D_k$-cells at $\u_0$ is $\Omega(V, \u_0)\cap B$.
\end{lemma}
\indy{Index}{null set}%

\begin{proof} This is clear from the construction.  Each null set is
  in fact a subset of a plane.
\end{proof}

Every cell $D_k(\bu)$ is eventually radial at $ \u_0$ and has a
well-defined solid angle $\sol(D_k(\bu))$.  Every cell $D_k(\bu)$ has
an {\it exposed} surface area $\op{surf}(D_k(\u))$.  This is the area
of the intersection of the boundary of $\Omega(V, \u_0)\cap B$ with
$D_k(\bu)$.  It consists of the sum of the areas of the analytic
facets (linear or spherical surfaces) that do not meet the point $
\u_0$.  The total surface area of $\Omega(V,\u_0)$ is the sum of the
exposed surface areas $\op{surf}(D_k(\u))$.  \indy{Index}{exposed}%
\indy{Index}{surface area!exposed}%
\indy{Notation}{surf@$\op{surf}$}%
\indy{Notation}{solidangle@$\sol$ (solid angle)}%


Every cell $D_k(\bu)$ has a set $E(k,\bu)$ of distinguished edges
(that is, the edges of the cell that have an endpoint at $\u_0$) and a
dihedral angle $\dih(e)$ for $e\in E(k,\bu)$.  Each edge has a length
$h(e)\in\leftclosed1,1.26\rightclosed$.  \indy{Index}{angle!dihedral}%
\indy{Notation}{dih}%
\indy{Index}{edge!length}%
\indy{Notation}{E@$E$ (edge)}%

\subsection{local inequality}

The strong dodecahedral conjecture reduces to a kissing estimate and a
local inequality.  \indy{Index}{kissing estimate}%


\begin{definition}[$a_D$,~$b_D$,~$y_D$,~$v_D$,~$f$]\guid{TCOSFNQ}
  Define constants $a_D$, $b_D$, $y_D$, and functions $v_D$, $f$ as
  follows.  Let $y_D$ be defined by the condition
\begin{displaymath}
\sol(2,2,2,y_D,y_D,y_D) = \pi/5.
\end{displaymath}
For any $\u_i\in\ring{R}^3$, let $g(\u_0,\u_1,\u_2,\u_3)$ be the
volume of the intersection of the convex hull of
$S=\{\u_0,\ldots,\u_3\}$ with set of points closer to $ \u_0$ than to
any other point in $S$.  When
\begin{equation}\label{eqn:uy}
  \norm{\u_0}{\u_i}=2\text{ and }\norm{ \u_i}{ \u_j} = y\text{ for }i,j\ge 1,
\end{equation} 
this volume depends only on $y$. Write $v(y) = g(\u_0,\ldots,\u_3)$.
Set
\begin{displaymath}
  f(y) = v(y) + a \sol(2,2,2,y,y,y) + 3 b \dih(2,2,2,y,y,y).
\end{displaymath}
The linear system
\begin{equation}\label{eqn:fyD}
f(y_D) = 0,\quad f'(y_D) = 0
\end{equation}
has a unique solution in $a,b$ with values $a=a_D=-0.581\ldots$,
$b=b_D=0.0232\ldots$.
\end{definition}
\indy{Notation}{a@$a_D$ (dodecahedral parameter)}%
\indy{Notation}{b@$b_D$ (dodecahedral parameter)}%
\indy{Notation}{yd@$y_D$ (dodecahedral parameter)}%
\indy{Notation}{yd@$v_D$ (dodecahedral function)}%
\indy{Index}{convex hull}%
\indy{Index}{regular dodecahedron!volume}%

Note that the regular dodecahedron has volume $20 v(y_D)$ and surface
area $60 v(y_D)$.  Also,
\begin{equation}
  2\sol(2,2,2,y_D,y_D,y_D) =\dih(2,2,2,y_D,y_D,y_D)=2\pi/5.
\end{equation}
\indy{Index}{regular dodecahedron}%
\indy{Index}{regular dodecahedron!surface area}%

\begin{lemma}[local inequality]\guid{PWVDMPT}\label{lemma:D-local}
For any cell $D_k(\bu)$
\begin{displaymath}
  \op{surf}(D_k(\bu)) + 3 a_D \sol(D_k(\bu)) 
+ 3 b_D \sum_{e\in E(k,\bu)} L(h(e)) \dih(e) \ge 0,
\end{displaymath}
where $L$ is the function of Definition~\ref{def:L}.  Equality holds
precisely when the cell is a $4$-cell with edges
$(2,2,2,y_D,y_D,y_D)$.
\end{lemma}
\indy{Index}{local inequality}%

\begin{remark}
  The dimension of a $k$-cell is at most $6$.  The local cell
  inequality is within reach of interval arithmetic computer
  calculations.  For a cell $D_4(\bu)$ whose parameters have the form
  \eqn{eqn:uy}, the local inequality reduces to the inequality
  $f(y)\ge 0$.  The constants $a_D$ and $b_D$ were chosen so that
  $y=y_D$ is a critical point of $f$ with value $f(y_D)=0$.  In
  particular, the local inequality asserts that $f$ has a local
  minimum at $y=y_D$.
\end{remark}

\begin{lemma}[]\guid{OIEKCEZ}
  The local inequality and the kissing number estimate \eqn{conj:L12}
\begin{displaymath}
\sum L(h) \le 12
\end{displaymath}
imply the strong dodecahedral conjecture.
\end{lemma}

\begin{proof} 
  Sum the local inequality over all the $D_k$-cells in a Voronoi cell.
  The solid angles sum to $4\pi$ and the dihedral angles around each
  edge sum to $2\pi$:
\begin{eqnarray*}
  \op{surf}(\Omega) &\ge& \op{surf}(\Omega\cap B)\\
  &=&\sum_{k,\bu} \op{surf}(D_k(\bu))\\
  &\ge& -12\pi a_D - 6\pi\, b_D  \sum L(h)\\
  &\ge& -12\pi a_D - 72\pi\, b_D\\
  &=& -60 \sol(2,2,2,y_D,y_D,y_D) a_D - 180 \dih(2,2,2,y_D,y_D,y_D) b_D\\
  &=& 60 (v(y_D) - f(y_D))\\
  &=& 60 v(y_D).\\
\end{eqnarray*}
The final term is the surface area of a regular dodecahedron.
\end{proof}

Thus, the strong dodecahedral conjecture follows from the same kissing
number estimate that is used to prove the Kepler conjecture.  The case
of equality in these inequalities occurs only for the regular
dodecahedron.

\newpage
\section{Fejes T\'oth's Full Contact Conjecture}



Call a nonempty packing $ V$ in $\ring{R}^3$ in which every point has
distance $2$ from $12$ other points a {\it packing with full
  contact}. L Fejes T\'oth has conjectured the following.
\indy{Index}{packing!full contact}%
\indy{Index}{full contact}%
\indy{Notation}{V@$V$ (packing)}%


\begin{theorem}[Packings with Full Contact]\guid{BDEDUTL}\label{thm:fc} 
  Let $ V$ be a packing with full contact.  Then for every point $
  \u\in V$, the arrangement of $12$ around that point is the kissing
  configuration of the face-centered cubic or hexagonal-close packing.
\end{theorem}
\indy{Index}{packing!hexagonal}%
\indy{Index}{packing!face-centered cubic}%
\indy{Index}{kissing configuration}%

This section  sketches a proof of this result.  


\begin{lemma}[]\guid{LIHVTRE} \label{lemma:gap}
  Let $V$ be any packing with full contact, and let $\u\ne\v\in V$.
  Then $\norm{\u}{\v}=2$ or $\norm{\u}{\v} \ge 2.52$.
\end{lemma}
\indy{Index}{packing!full contact}%
\indy{Index}{full contact}%

\begin{proof} Let $ \u_1,\ldots, \u_{12}$ be the twelve kissing points
  around $\u$.  Assume that $\u\ne \u_i$.  By
  Assertion~\ref{conj:L12},
\begin{displaymath}
  12 + L(h( \u, \v)) 
  = \sum_{i=1}^{12} L(h( \u, \u_i)) + L(h( \u, \v)) \le 12.
\end{displaymath}
This imples that $L(h( \u, \v))\le 0$, so $\norm{ \u}{ \v}\ge 2.52$.
\end{proof}

A packing $V$ may always be translated so that $\orz\in V$.  We study
the structure of a kissing configuration centered at $\orz$ whose
points have the separation property of Lemma~\ref{lemma:gap}.

\begin{definition}[$S^2(2)$,~$\CalV$]
  Let $S^2(2)$ be the sphere of radius $2$, centered at $\orz$.  Let
  $\CalV$ be the set of packings $V\subset \ring{R}^3$ such that
\begin{itemize}
\item $\card(V) = 12$,
\item $V\subset S^2(2)$.
\item $\norm{\u}{\v} \in \{0,2\}\cup
  \leftclosed2.52,\infty\rightopen$, for all $\u,\v\in V$.
\end{itemize}
\indy{Notation}{V@$\CalV$ (twelve sphere configurations)}%
\indy{Notation}{S@$S^2(2)$ (sphere of radius $2$)}%
\end{definition}

The strategy of the proof will be to classify the hypermaps of the
contact fan $(V,E_{ctc})$, for $V\in \CalV$.  The classification will
contain two hypermaps, the contact hypermap of the face-centered cubic
(FCC) and the contact hypermap of the hexagonal close packing (HCP).
From this, the proof of Fejes T\'oth's conjecture will follow.

The classification result is analogous to the classification result
that we have already obtained for tame hypermaps.  This suggests that
we should develop a proof along exactly the same lines as earlier
chapters.  We will define a new collection of hypermaps whose
properties are analogous to those defining a tame hypermap.  We will
define \newterm{hypermaps with tame contact} to be hypermaps in this
new collection.  A computer generated classification of these
hypermaps will show that there are only a few possibilities.  Those
other than the FCC and HCP hypermaps will be eliminated by linear
programming methods.

\subsection{main estimate}




We use the function $\tau(V,E,F)$ from Definition~\ref{def:tau}.  When
$V$ is a packing, $(V,E)$ is a \newterm{biconnected} graph, and $V\subset
S^2(2)$, the function $\tau(V,E,F)$ takes the following form:
\begin{displaymath}
\tau(V,E,F) = \sol(U_F) + (2-k(F)) \sol_0,
\end{displaymath}
% \indy{Notation}{ZZtauf@$\tau(V,E,F)$ (contact fans)}%
% \indy{Index}{kissing configuration}%

\begin{remark}[Lexell's theorem]
  It is a consequence of Lexell's theorem that the area of a spherical
  triangle, viewed as a function of its edge lengths, does not have a
  interior point local minimum when the the edge lengths are
  constrained to lie in given intervals.  The minimum occurs at a
  point where each edge length lies at an endpoint of its interval.
  Lexell's theorem is a considerable aid in finding the minimum of
  $\sol(U_F)$ or $\tau$.
\end{remark}

The next theorem is the analogue for packings with full contact of the
main estimate (Lemma~\ref{lemma:empty-d}).  It is similar to the main
estimate in Leech's proof of the problem of $13$-spheres
\cite{Leech:1956:MG}.


\begin{theorem}\guid{VGJDQJV}\label{lemma:main-estimate-12}
  Let $V$ be a packing in $S^2(2)$, $E$ a set of edges, and $F$ a face
  of $\op{hyp}(V,E)$ such that $(V,E,F)$ is a local fan.  Assume that
  $(V,E)$ is a biconnected graph.  Assume that $F$ has at least three
  darts Assume that every edge in $E$ has length at most $3.2$.  Let
  $S$ be a subset of $E$ such that the length of every edge in $S$ is
  at least $2.52$.  Let $U=U_F$ be the topological component of
  $Y(V,E)$ corresponding to $F$.  Assume that if $\{\u,\v\}\subset V$
  with $C^0\{\u,\v\}\subset U$ and $2\le\norm{\u}{\v} \le 2.52$, then
  $\{\u,\v\}\in E$.  Then
\begin{displaymath}\tau(V,E,F) \ge \min(d(r,s),\op{tgt})\end{displaymath}
where
\begin{displaymath}
d(r,s) = 0.103 (2-s) + 0.277 (r+2s-4),
\end{displaymath}
$\op{tgt}=1.541$, 
$s = \card(S)$, and $r = \card(E)-\card(S)$.
%\indy{Notation}{d@$d$ (contact weight constant)}
\indy{Notation}{tgt@$\op{tgt}=1.541$}
\end{theorem}

\begin{proof} This proof imitates the proof of the main estimate from
  \cite{Hales:2006:DCG}.  Here is a review of the method.  (Compare
  Chapter~\ref{sec:local}.)

  For a contradiction, let the packing $(V,E,F,S)$ violate the given
  inequality.  Among all counterexamples to the theorem, we may assume
  that $(V,E,F,S)$ is a counterexample that minimizes $k=r+s$.  Let
  $k_{min}$ be the smallest value attained.



  \claim{There exists a counterexample that minimizes
\begin{equation}\label{eqn:td}
\tau(V,E,F)-\min(d(r,s),\op{tgt})
\end{equation}
among all (local) counterexamples that have parameters that satisfy
$k_{min}=k$.}  Indeed, a compactness argument shows that a sequence
tending to the minimum value has a convergent subsequence.  (Compare
Lemma~\ref{lemma:compact-fan} and Lemma~\ref{lemma:c-bound}.)

We may assume that the counterexample $(V,E,F,S)$ is local and minimal
in this sense.

\claim{In a minimal counterexample, all edges of length at least
  $2.52$ belong to $S$.}  Indeed, $d(r-1,s+1)>d(r,s)$.

\claim{A minimal counterexample $(V,E,F,S)$ does not have any edges
  $\ee=\{\u,\v\}\subset V$, satisfying $C^0(\ee)\subset U$ and
  $\norm{\u}{\v}\le 3.2$.}  Otherwise, $\ee$ may be added to the edge
set of the fan.  The face $F$ splits into two faces $F_1$ and $F_2$.
By the additivity of the constants $d(r,s)$ under splitting (analogous
to Equation~\ref{eqn:drs}), one of the two faces $F_1$ or $F_2$ is a
counterexample as well.  This is contrary to the assumed minimality of
$F$: $k(F_1),k(F_2)<k_{min}=k(F)$.

% \claim{In a minimal counterexample $(V,E,F,S)$ every edge is
%   extremal.  That is every edge in $S$ has length $2.52$ or $3.2$
%   and every edge in $E\setminus S$ has length $2$.}  Indeed, this
% follows by Lexell's theorem.

Call a node $\v\in V$ \newterm{concave} or \newterm{convex}, according
to whether $\op{azim}(x)\ge\pi$ or $\op{azim}(x)<\pi$, where $x$ is
the dart of $F$ at $\v$.  \indy{Index}{concave node}%
\indy{Index}{convex node}%

\claim{In a minimal counterexample $(V,E,F)$, if $\v\in V$ is a
  concave node, then both edges at $\v$ have length $3.2$.}  Indeed,
if both edges at $\v$ have length less than $3.2$, there exists a
deformation of the local fan $(V,E,F)$ that fixes every node except
$\v$, and decreases the solid angle $U_F$.  This cannot be a
minimizing counterexample.  Now assume that one of the edges
$\{\v,\u\}$ has length $3.2$.  Let $\{\v,\w\}\in E$ be the other edge.
By minimality it has extremal length $2,2.52$, or $3.2$.  Let
$y=\norm{\u}{\w}$.  We check that
\[
\dfrac{d\sol(2,2,2,y,3.2,t)}{dt} >0,
\]
when $t\in\{2.52,3.2\}$.  Hence the triangle of largest area occurs
when $\norm{\v}{\w}=3.2$.  (The triangle has largest area when $U_F$
has smallest area because we are at a concave node.)
%% CALC XX  Calculation.

\claim{In a minimal counterexample, some node is concave.}  Otherwise,
every node is convex.  Every edge has arclength at least
$\arc(2,2,2)=\pi/3$.  By Lemma~\ref{lemma:convex-hyp}, the cardinality
$k$ of $E$ satisfies $(\pi/3)k < 2\pi$, so $k<6$.  (The inequality is
strict for a generic fan.)  By Lexell's theorem, the five edges have
extremal lengths.  That is every edge in $S$ has length $2.52$ or
$3.2$ and every edge in $E\setminus S$ has length $2$.  The only
remaining degrees of freedom are the lengths of diagonals.  As the
polygon is at most a pentagon, the proof has now been reduced to a
finite number of interval arithmetic verifications of dimension at
most $2$.
%% CALC XX Calculation.

\claim{In a minimal counterexample, $k>6$.}  Otherwise, there is a
minimal counterexample with $k\le 6$ and some concave node $\v$. One
half of $\op{rcone}(\orz,\v,\cos\theta)$, where
$\theta=\arc(2,2,3.2)=1.854\ldots$ fits inside the region.  The
half-cone has solid angle
\begin{displaymath}
s=\pi(1-\cos\theta)=4.02\ldots.
\end{displaymath}
This gives
\begin{displaymath}
\tau(V,E,F) > s + (2-k)\sol_0 > \op{tgt}.
\end{displaymath}
\indy{Index}{arcradius}%


\claim{In a minimal counterexample, there is exactly one concave
  node.}  Otherwise some minimal counterexample has $12\ge k>6$ and
two or more concave nodes.  Let $k'$ be the number of concave  nodes.  
Calculations~\cite[cc:lft]{hales:2009:nonlinear}%
%% CALC XX Calculation 
show that the azimuth angle of the dart in $F$ at each convex node is
at least $\alpha=1.73$.  Place a wedge of angle $\alpha$ of
$\op{rcone}(\orz,\u,\cos(\pi/6))$ and a half
$\op{rcone}(\orz,\u,\cos\theta)$ at the convex and concave nodes
respectively, where $\theta=\arc(2,2,3.2)-\pi/6$.  These regions are
disjoint and their combined solid angle is less than $\sol(F)$.  Hence
\begin{eqnarray*}
\tau(V,E,F)&=&\sol(F)+(2-k)\sol_0 \\
&\ge& \pi k' (1-\cos\theta)
+ \alpha (k-k') (1-\cos(\pi/6)) + (2-k)\sol_0\\
&\ge&\op{tgt}.\\
\end{eqnarray*}


\claim{In a minimal counterexample, there cannot be exactly one
  concave node.} Otherwise, $U_F$ is star convex about the concave
node.  This allows us to deform without obstruction from other nodes.
The node adjacent to the concave node $\v$ has interior angle greater
than $\pi/2$ and can be deformed by decreasing the distance between it
and $\v$, to decrease the solid angle of $U$.  This shows that the
function $\tau$ has no local minimum among such arrangements.
\indy{Index}{star convex}%
\indy{Index}{convex!star}%

The various claims cover all possible cases.  This completes the
proof.
\end{proof}

\subsection{biconnected fans}

We may create  fans that are biconnected graphs in the same way as in
\cite{Hales:2006:DCG}.  Here is a review
of the construction.



\begin{lemma}\guid{NJFWRPQ}\label{lemma:V'-bi} 
Let $V\in \CalV$.  Then there exists $V'\in \CalV$ with
  the following properties.
\begin{itemize}
\item There is a bijection $\phi:V'\mapsto V$ that induces a bijection
  of contact graphs:
\[
\phi_*:(V,E_{ctc}) \cong (V',E'_{ctc}).
\]
\item Let $E'$ be the set of all pairs $\{\u,\v\}\subset V'$
  such that $2.52\le\norm{\u}{\v} <\sqrt8$.  Set $E =
  E'_{ctc}\cup E'$.  Then $(V',E)$ is a fan.
\item The graph $(V',E)$ is biconnected.
\end{itemize}
\end{lemma}

\begin{proof}
  Begin with the contact fan $(V,E_{ctc})$.  Let $E'$ be the set
  of all pairs $\{\u,\v\}\subset V$ such that
  $2.52<\norm{\u}{\v}<\sqrt8$.

  \claim{$(V,E_{ctc}\cup E')$ is a fan.} Indeed, it is checked by
  \cite[Lemma~4.30]{Hales:2006:DCG} that the blades satisfy the
  intersection property of fans, except possibly when two new blades
  are the diagonals of a quadrilateral face in $(V,E_{ctc})$.  (The
  cited lemma uses the constant $2.51$ instead of $2.52$, but this
  does not affect the outcome.)  The diagonals of a quadrilateral face
  in $(V,E_{ctc})$ is a spherical rhombus and one of its diagonals is
  necessarily at least $\sqrt8$ (with extreme case a square of side
  $2$).  The other fan properties are easily checked.

  If the hypermap $\op{hyp}(V,E_{ctc}\cup E')$ is not connected,
  the set of nodes $V_1\subset V$ in one combinatorial component can
  be moved closer to another combinatorial component until a new edge
  is formed.  This can be done in a way that the deformation of $V$
  remains in $\CalV$ and no new edges of length at most $2.52$ are formed.
  Continuing in this fashion, a connected hypermap is obtained.

  Further deformations within a connected hypermap produce a
  biconnected hypermap with the given properties.
\end{proof}


\begin{lemma}\guid{CTJYKFZ}\label{lemma:dj}
Let $H=(D,e,n,f)$ be a connected hypermap with more than
two darts.  Assume that the hypermap has no loops or double joins. Then
every face has at least three darts.
\end{lemma}

\begin{proof}
Otherwise,
if some face has only two darts, then because of the no double join
condition, the two edges meeting the face, $\{x, e x\}$ and $\{ e^{-1} f^{-1} x, f^{-1}
x\}$, must actually be equal.  That is $ x = f e x = n^{-1} x$, so that
$x$ is a fixed point of the node map.  Similarly, $f x$ is a fixed
point of the node map.  Then $\{x, f x\}$ is a combinatorial
component, which is contrary to the assumption that the hypermap is
connected with more than two darts.
\end{proof}


\begin{definition}[$D_U$,~$m_U$]
Let $V\in \CalV$.  Let $U$ be a topological component of
$Y(V,E_{ctc})$, and let $D_U$ be the set of all darts that lead into
$U$.  It is a union of faces of $\op{hyp}(V,E_{ctc})$.  For each $x\in
D_U$, let $m(x) >0$ be the smallest natural number such that $f^m
x$ and $x$ lie at the same node.   Let $m_U$ be the maximum of $m(x)$, as
$x$ runs over $D_U$.  The constant $m_U$ can be viewed as a
{\it simplified face size}. 
\indy{Index}{contact!full}%
\indy{Index}{fan}%
\indy{Notation}{m@$m$ (simplified face size)}%
\indy{Notation}{DU@$D_U$ (the set of darts leading into $U$)}
\end{definition}

\begin{lemma}[biconnected]\guid{BTZPFMU}\label{lemma:biconnected}
  Let $V\in \CalV$.  Then $\op{hyp}(V,E_{ctc})$ is biconnected.
\end{lemma}

\begin{proof}
  By Lemma~\ref{lemma:V'-bi}, we may replace $V$ with a new set in
  $\CalV$ if necessary so that $(V,E)$ is a biconnected fan, where
  $E\setminus E_{ctc}$ is the set of pairs $\{\u,\v\}\subset V'$ such
  that $2.52\le\norm{\u}{\v} <\sqrt8$.  We will show that the smaller
  fan $(V,E_{ctc})$ is also biconnected.

Let $U'$ be a topological component of $Y(V,E_{ctc})$.  Let $D'=D_{U'}$.
Let $m'=m(U')$, $r' = \card(D')$, and
let $s'$ be twice the number of combinatorial components of $\op{hyp}(V,E_{ctc})$
that meet $D'$.  Let $k'=r'+s'$.  Set $\tau(U') = \sol(U') + (2-k')\sol_0$.

\claim{We claim that $\tau(U')\ge \min(d(r',s'),\op{tgt})$.}
Indeed, up to a null set (given by the finite
union of blades $C^0(\ee)$ for $\ee\in E\setminus E_{ctc}$), 
$U'$ is the union of topological components $U_F$ of $Y(V,E)$, which are in bijection
with the faces $F$ of $\op{hyp}(V,E)$.  The function $\tau(U')$ is additive:
\[
\tau(U') = \sum_{U_F\subset U'} \tau(V,E,F).
\]
By the biconnectedness of $(V,E)$, each value $\tau(V,E,F)$ is the same before and after
localization.  Lemma~\ref{lemma:main-estimate-12} gives a lower bound on the
constants $\tau(V,E,F)$.  The constants $d(r',s')$ are also additive:
\[
d(r',s') = \sum_{U_F\subset U'} d(r(F)),s(F)),
\]
where $s(F)$ is the cardinality of the set of edges of $E\setminus E_{ctc}$ that meet $F$,
and $r(F) = \card(F)-s(F)$.
Thus, the claimed lower bound on $\tau(U')$ follows from the main estimate (Lemma~\ref{lemma:main-estimate-12}).


\claim{If $m'\le 5$, then  $D'$ is a simple face.} Otherwise,
either the face is not
  simple or there is more than one face that leads into $U$.  Then
  there is some node $\v$ interior to the $m'$-gon.  The azimuth
  angles at $\v$ are each less than $2\pi/5$. They cannot sum to
  $2\pi$ as required.

\claim{$D'$ is a  simple face.}  Otherwise, assume for a contradiction
that $D'$ is not simple,  $m'\ge 6$, and $d(r',s')<\op{tgt}$.
  From the classification of \cite[p.~126,~Fig.~12.1]{Hales:2006:DCG},
  and the inequalities $d(9,0) > \op{tgt}$, $d(6,2) > \op{tgt}$, it
  follows that the set $D'$ has cardinality $8$ and $m'=6$.

This arrangement involves $7$ nodes: the six
  nodes counted by $m'$ and the node in the center of the hexagonal
  arrangement.  As there are $12$ nodes in all, there are five
  additional nodes.  Each of these five nodes meets a non-triangular
  topological componenet of $Y(V,E_{ctc})$.  By \eqn{eqn:delta0}, 
\begin{displaymath}
  \sum_{U'\subset Y(V,E_{ctc})} \tau(U') =\sum_F \tau(V,E,F) = (4\pi - 20\sol_0) < \op{tgt}.
\end{displaymath}
However, $\sum_{U'} \tau(U')$ is at least
\begin{eqnarray*}
d(8,0) + d(5,0) &>& \op{tgt}, \text{ or }\\
d(8,0) + 2 d(4,0) &>& \op{tgt}.
\end{eqnarray*}
Thus, $D'$ is a simple face.
\indy{Index}{weight!total}%
\indy{Index}{weight}%

\claim{The hypermap is biconnected.}  Otherwise,
  if the hypermap is not connected, then we can find two faces of the hypermap that
  lead into the same topological component of $Y(V,E_{ctc})$.  If the
  hypermap is connected but not biconnected, then some face of the
  hypermap is not simple.  Both possibilities contradict the claim that $D'$
is a simple face.
\end{proof}



%  
%
%\subsection{odds and ends}
%
%\begin{remark}\label{rem:local-same} %XX move
%\end{remark}
%
%In summary, we may work without generality with packings in the
%following collection.
%
%\begin{definition}
%  If $V\in\CalV$, with contact fan $(V,E_{ctc})$, then we say that $E$
%  is \newterm{admissible} if
%\begin{itemize}
%\item $(V,E)$ is a fan,
%\item $E_{ctc}\subset E$, and
%\item if $\{\u,\v\}\in E\setminus E_{ctc}$, then
%  $2.52\le\norm{\u}{\v}\le 3.2$.
%\end{itemize}
%\indy{Index}{fan}%
%\indy{Index}{fan!aggregate}%
%\end{definition}
%
%\begin{definition}[$\CalV_1$]
%  Let $\CalV_{agg}$ be the set of packings $V$ in $\CalV$ such that
%  $H=\op{hyp}(V,E_{agg})$ has the following properties:
%\begin{itemize}
%\item $H$ is  biconnected.
%\item $H$ is plain, has no loops or double joins.
%\item The edge map $e$ of $H$ has no fixed points.
%\item Every face of $H$ has at least three darts.
%\item Every face of $H$ is simple.
%\item $H$ is planar.
%\end{itemize}
%\end{definition}
%
%
%It is better to work with the packing $V'$ and fan $(V',E'_{agg})$
%rather than $V$.  For simplicity, we now drop the primes from the
%notation.  Call $(V,E_{agg})$ the \newterm{aggregate fan},
%$\op{hyp}(V,E_{agg})$ the \newterm{aggregate hypermap}, and so forth.
%By Lemma~\ref{lemma:fan-plain}, the hypermap is plain, has no loops or
%double joins.  The edge permutation $e$ of the hypermap has no fixed
%points.  Each face has at least two darts.  In a biconnected planar
%graph, each face is a simple polygon.  The blades of the fan satisfy
%the \case{intersection} property, and this implies that the hypermap
%is planar.
%
%
%




\subsection{tame contact}

This subsection modifies the notion of tameness to cover hypermaps
that arise as the contact fan of a packing with full contact.  In the
definition of tame hypermap, two functions $b$ and $d$ are used.  In
this section we define two new functions that are used to define tame
contact.  They will also be called $b$ and $d$ because we have no
further use for the functions in Chapter~\ref{sec:tame}.  
Recall that $\op{tgt}=1.541$.
\indy{Index}{tame}%
\indy{Index}{hypermap!tame}%

\begin{definition}[b]\guid{IHRZTPV}
  Define $b:\ring{N}\times \ring{N}\to \ring{R}$ by
  $b\pqr{(p,q)}=\op{tgt}$, except for the following values:
\begin{displaymath}
b(0,3)=b(1,3)=0.618,\quad b(2,2)=0.412.
\end{displaymath}
\end{definition}
\indy{Notation}{b@$b$ (contact weight parameter)}%

\begin{definition}[d]\guid{VUJQZCG}
Define $d:\ring{N}\to \ring{R}$ by
\begin{displaymath}d(k) = \begin{cases}
0 & k\le 3, \\
0.206 + 0.277 (k-4),& k=4,\ldots,8,\\
%0.206 & k=4, \\
%0.483 & k=5, \\
%0.760 & k=6, \\
%1.037 & k=7, \\
%1.314 & k=8,\\
\op{tgt},& k>8.\\
\end{cases}
\end{displaymath}
%(In particular, $d(k) = 
\end{definition}
\indy{Notation}{d@$d$ (contact weight parameter)}%

The function $d$ is related to the two-variable function in Lemma~\ref{lemma:main-estimate-12}: $d(k) = d(k,0)$, when $4\le k\le 8$.

\begin{definition}[weight~assignment]\guid{GLIQSFM}
%
  Recall that a {\it weight assignment\/} on a hypermap $H$ is a
  function $\tau$ on the set of faces of $H$, taking values in the set
  of non-negative real numbers. A weight assignment $\tau$
is a {\it contact}
  weight assignment if the following two properties hold:
%
  \indy{Index}{weight assignment}%
  \indy{Index}{contact!weight assignment}%
  \indy{Notation}{ZZtau@$\tau$ (weight assignment)}%
\begin{enumerate}
\item If the face $F$ has cardinality $k$, then
$\tau(F) \ge d(k)$
\item If a node $\v$ has type $(p,q,0)$, then
  \begin{displaymath}\sum_{F:\,\v\cap F\ne\emptyset} \tau(F) \ge
    b{\pqr{(p,q)}}.\end{displaymath}
\end{enumerate}
The sum $\sum_F \tau(F)$ is called the {\it total weight} of $\tau$.
\indy{Index}{total weight}%
\end{definition}


\begin{definition}[tame contact]\guid{XJPQTIV}
  A hypermap has {\it tame contact\/} if it satisfies the following
  conditions.
%
\indy{Index}{tame contact}%
\indy{Index}{contact!tame}%
\indy{Index}{planar}%
\indy{Index}{biconnected}%
\indy{Index}{nondegenerate}%
\indy{Index}{loop}%
\indy{Index}{double join}%
%
\begin{description}
%\label{definition:tame}
%1
\item \case{planar} The hypermap is plain and planar.
\item \case{biconnected} The hypermap is biconnected.  In particular,
  every face meets every node in at most one dart.
\item \case{nondegenerate} The edge map $e$ has no fixed points.
\item \case{no loops} The two darts of each edge lie in different
  nodes.
\item \case{no double join} At most one edge meets any two given
  nodes.
%The set of edges meeting any two given
%  nodes has cardinality at most $1$.
%\label{definition:tame:40}
%      \item\case{blank}
%      \item \case{triangles} If $L$ is a contour loop with $3$ face
%        steps, and if there exists a node in the exterior of $L$,
%        then $L$ is a face of the hypermap.
%\item \case{blank}
% \item \case{quadrilaterals} If $L$ is a $4$-step contour loop, and
%   there is at least one node in the exterior of $L$, then the
%   interior of $L$ takes one of the forms illustrated in Figure
%   \ref{fig:fourcircuit-FT}.
%    %\label{definition:tame:4-circuit-FT}
%    \begin{figure}[htb]
%      \centering \myincludegraphics{\pdfp/fourcircuitFT.eps}
%        \caption{Tame $4$-circuits}
%        \label{fig:fourcircuit-FT}
%    \end{figure}
\item \case{face count} There are at least two faces.
\item \case{face size} The cardinality of each face is at least $3$
  and at most $8$.
%\label{definition:tame:length}
\item \case{node count} There are $12$ nodes.
\item \case{node size} The cardinality of every node is at least $2$
  and at most $4$.
%\label{definition:tame:degree}
%    \item \case{node} {\tt NO CONDITION}
%\label{definition:tame:degreeE}
\item \case{weights} There exists a contact weight assignment of total
  weight less than $\op{tgt}$.
%\label{definition:tame:squander}
\end{description}
%
\end{definition}


%\subsection{tame contact}

\begin{theorem}\guid{ZXZSVPH} The contact hypermap of a 
  packing $V\in \CalV$ is a hypermap with tame contact.
\end{theorem}
\indy{Index}{hypermap!tame}%
\indy{Index}{hypermap!contact}%
\indy{Index}{hypermap}%
\indy{Index}{contact!full}%

\begin{proof} It is enough to go through the list of properties that
  define a tame contact hypermap and to verify that the contact
  hypermap satisfies each one.  We use the weight assignment $F\mapsto
  \tau(V,E_{ctc},F)$.

\begin{description}
\item \case{planar} The contact hypermap is plain and planar by the
  general properties of fans.\footnote{Earlier chapters give a long discussion of planarity.
In this chapter, we are not attempting to give a formalizable blueprint, so we relax
our standards and regard
planarity as an ``obvious'' feature of fans.}
\item \case{biconnected} The hypermap is biconnected because of
  Lemma~\ref{lemma:biconnected}.
\item \case{nondegenerate}, \case{no loops}, \case{no double join} The
  edge map has no fixed points by the general properties of fans.
  There are no loops or multiple joins by the general properties of
  fans.
\item \case{face count} Each node has at least two darts by
  biconnectness. Each face is simple; so the two darts at a node lie
  in different faces.  Thus, there are at least two faces.
\item \case{face size} The cardinality of each face is at least three
  because there are no loops or double joins (Lemma~\ref{lemma:dj}).  The cardinality of a
  face is at most $8$ because of the estimate $d(9,0)>\op{tgt}$.
\item \case{node count} There are twelve nodes by the definition of a
  packing with full contact.
\item \case{node size} It is already established that the cardinality
  of each node is at least two.  The proof that the cardinality is
  never five or greater appears in Lemma~\ref{lemma:no-5}.
\item \case{weights} The inequality $\tau(V,E_{ctc},F)\ge d(k)$ is
  Theorem~\ref{lemma:main-estimate-12}.
  \indy{Notation}{ZZtauf@$\tau(V,E_{ctc},F)$}%
  The total weight of the weight assignment is given by
  equation~\eqn{eqn:delta0}:
\begin{displaymath}
  \sum_F \tau(V,E_{ctc},F) = (4\pi - 20\sol_0) < \op{tgt}.
\end{displaymath}
\indy{Index}{weight!total}%
Let $\v$ be a node of type $(p,q,0)$.  
%Let $A$ be the set of faces
%that meet the node $\v$. 
Then
\begin{displaymath}
\sum_{F\mid F\cap \v\ne\emptyset}\tau(V,E_{ctc},F) > d(4)~q.
\end{displaymath}
This gives the nonzero entries in the table of bounds $b(p,q)$.  The
remaining entries follow from Lemma~\ref{lemma:no-5}.
\end{description}
\end{proof}




\begin{lemma}[]\guid{CQRHDZE}\label{lemma:no-5} 
  Let $V\in \CalV$.  Every node of $(V,E_{ctc})$ has degree at most
  four.  Furthermore, suppose the type of a node is $(p,q,0)$.  Then
  $(p,q)$ must be
\begin{displaymath}
(0,3),~(1,3),~\text{ or}~~(2,2).
\end{displaymath}
\end{lemma}

\begin{proof} The interior angles of a spherical polygon in the
  contact graph have the following lower $\alpha_k$ and upper bounds
  $\beta_k$, as a function of the number of sides $k$.
\[
\begin{array}{llll}
  \phantom{\ge}k~~&\alpha_k & \beta_k\\
  \phantom{\ge}3~~&\op{azim}(2,2,2,2,2,2)  &\op{azim}(2,2,2,2,2,2)\\
  \phantom{\ge}4~~&\op{azim}(2,2,2,2.52,2,2) &2\,\op{azim}(2,2,2,2,2.52,2)\\
  {\ge}5~~& \op{azim}(2,2,2,2.52,2,2) ~~~& 2\pi.
\end{array}
\]
Thus,
\begin{displaymath}
  p\,\alpha_3 + q\,\alpha_4 +r\, \alpha_5 
\le 2\pi \le p\,\beta_3 + q\,\beta_4 + r \,\beta_5.
\end{displaymath}
There is no solution for
$(p,q,r)$ in natural numbers when $p+q+r\ge 5$.
The only three solutions in $(p,q,r)$ with $r=0$
are those given. 
\end{proof}



\subsection{classification}


\begin{lemma}[tame hypermap classification]\guid{AZYOJBE}\rating{ZZ}
  \label{lemma:contact-classification} Every hypermap with tame
  contact is isomorphic to a hypermap in the following list of eight
  hypermaps, or is isomorphic to the opposite of a hypermap in the
  list.  \indy{Index}{isomorphic}%
\end{lemma}

\begin{proof}
  The set of all hypermaps has been classified by the same computer
  algorithm described in Section~\ref{sec:proof-classification}.
  \indy{Index}{contact!tame}%
  \indy{Index}{hypermap}%
  \indy{Index}{hypermap!tame}%
\end{proof}



\begin{lemma}[]\guid{MWWSZTX}\label{lemma:fcc-ft} Let $V\in \CalV$.
  Suppose that $H=\op{hyp}(V,E_{ctc})$ is a hypermap with tame
  contact.  Then $H$ is the FCC or HCP contact hypermap.
\end{lemma}

\begin{proof} According to the classification of hypermaps with tame
  contact, there are eight hypermaps.  Two are the hypermaps of the
  FCC and HCP.  The remaining six must be eliminated.

There are some linear
  programming constraints that are immediately available to us:
\begin{itemize}
\item The angles around each node sum to $2\pi$.
\item Each angle of a triangle is $\alpha_3$.
\item Each angle of each rhombus lies between $\alpha_4$ and $\beta_4$.
\item The opposite angles of each rhombus are equal.
\item The sum of two adjacent angles of a rhombus lies between
\begin{displaymath}
  \alpha_4 + \beta_4 \text{~~and~~} 2\,\op{azim}(2,2,2,\sqrt8,2,2).
\end{displaymath}
\end{itemize}
These linear programs eliminate all but the FCC and HCP hypermaps.
\end{proof}
%% CALC XX Calculation

%\begin{theorem}[packings with full contact]\guid{ANSXBOJ}  
%  Fejes T\'oth's conjecture on packings with full contact holds.
%\end{theorem}
%\indy{Index}{packing}%

\begin{lemma}[]\guid{YRTPQXK}\label{lemma:kiss-fcc}
  Let $V\in \CalV$ be a packing such that $\op{hyp}(V,E_{ctc})$ is
  isomorphic to the FCC or HCP contact hypermap.  Then $V$ is
  congruent to the FCC or HCP configuration in $S^2(2)$.
\end{lemma}
\indy{Index}{packing!hexagonal}%
\indy{Index}{packing!face-centered cubic}%
\indy{Index}{kissing configuration}%
\indy{Index}{contact!full}%
\indy{Index}{hypermap}%

\begin{proof} Every face of the hypermap of $(V,E_{ctc})$ is a
  triangle or quadrilateral.  The eight triangles in the FCC or HCP
  contact hypermap determine eight equilateral triangles in $V$ of
  edge length $2$.  The eight triangles rigidly determine $V$ up to
  congruence.
\end{proof}

\begin{proof}[Proof of Theorem~\ref{thm:fc}]  %[Packings with Full
  % Contact]
  The contact hypermap of a packing with full contact has tame
  contact.  By Theorem~\ref{lemma:fcc-ft}, this hypermap is that of
  the FCC or HCP.  By Lemma~\ref{lemma:kiss-fcc}, the kissing
  configuration of the packing is congruent to the FCC or HCP.  As the
  center of the packing may be chosen at an arbitrary point in the
  packing, every point in the packing is congruent to one of these two
  arrangements.  The result follows.
\end{proof}


