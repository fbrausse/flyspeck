%------------------------------------------------------------
% Author: Thomas C. Hales
% Format: LaTeX
% Book Chapter: Dense Sphere Packings
%------------------------------------------------------------

\chapter{Further Results}

\begin{note}%XX 
  The results sketched in this chapter are still preliminary.  A
  number of the estimates that have been stated have not yet been
  rigorously proved by computer.
\end{note}

\section{Strong Dodecahedral Theorem}

K. Bezdek has conjectured that the Voronoi cell of smallest surface is
the regular dodecahedron with inradius $1$.  This is known as the
strong dodecahedral conjecture.  L. Fejes T\'oth's classical
dodecahedral conjecture asserts that the Voronoi cell of smallest
volume is the regular dodecahedron of inradius $1$.
\indy{Index}{Bezdek, K.}%
\indy{Index}{Voronoi cell}%
\indy{Index}{fejestoth@Fejes T\'oth, L.}%
\indy{Index}{Dodecahedral Conjecture}%


\begin{lemma}[]\guid{QRBKJAW}
The strong dodecahedral conjecture implies the
  dodecahedral conjecture.
\end{lemma}

\begin{proof} Let $A_1,\ldots,A_n$ be the areas of the faces of a
  Voronoi cell.  Let $h_1,\ldots,h_n$ be the distances of the faces
  from the center of the Voronoi cell.  Then $h_i\ge 1$.  Assume that
  $\sum A_i \ge A_D$, where $A$ is the surface area of a dodecahedron.
  Then its volume is
\begin{displaymath}
\op{vol} = \sum A_i h_i/3 \ge \sum A_i/3 = \op{vol}_D,
\end{displaymath}
where $\op{vol}_D$ is the volume of the regular dodecahedron.
\end{proof}
\indy{Index}{regular dodecahedron!volume}%
\indy{Notation}{A@$A$ (Voronoi cell face area)}%
\indy{Notation}{V@$\op{vol}_D$ (volume of dodecahedron)}%

This section sketches a proof of the strong dodecahedral theorem.  The
notaton follows Section~\ref{sec:rogers}.

Let $\Omega( \u_0)$ be a Voronoi cell.  Rogers's partition of the cell
is
\begin{displaymath}
\Omega( \u_0) = \bigcup_{ \u\in  V(3), \u[0]= \u_0 } R( \u).
\end{displaymath}
Let 
\begin{displaymath}
a = \norm{ \v( \u_0, \u_1)}{   \u_0},\quad
b = \norm{( \u_0, \u_1, \u_2)}{  \u_0},\quad
c = \norm{ \v( \u_0, \u_1, \u_2, \u_3)}{   \u_0}.\quad
\end{displaymath}
Let $B$ be the ball of radius $1.26$ centered at $ \u_0$.
\indy{Notation}{ZZZomega@$\Omega$ (Voronoi cell)}%
\indy{Index}{Rogers's partition}%


\subsection{$D$-cells}

Define $D_k$-cells for $k=2,3,4$, for each $ \u\in V(3)$.
\indy{Notation}{D@$D_k$}%
\indy{Index}{D-cell}%

\case{$D_4$} Define a $D_4$-cell to be empty unless $c<1.3$.  If
$c<1.3$ define it to be
\begin{displaymath}
\bigcup_{\v( \u)=\v(\v),  \v\in  V(3),\ \v[0]= \u[0]}  R(\v)\cap B.
\end{displaymath}
This is equal to
\begin{displaymath}
\op{conv}\{ \u_0, \u_1, \u_2, \u_3\} \cap \Omega( \u_0)\cap B.
\end{displaymath}
(The circumradius is monotonic in the edge lengths when $c<1.3$.  If
any edge $\norm{ \u_i}{ \u_j}$ is greater than $2.52$, then $c >
\op{rad}(2,2,2,2,2,2.52) > 1.3$.  Hence all edges of the simplex have
length at most $2.52$.)  \indy{Index}{edge!length}%

\case{$D_3$} Define the $D_3$-cell to be empty unless $c\ge 1.3$ and
$b< 1.26$.  If $c \ge 1.3$ and $b< 1.26$, there is a unique point $p$
on the segment from $\v( \u[2])$ to $\v( \u)$ at distance $1.3$ from $
\u_0$.  Let
\begin{displaymath}
R_1( \u) = \op{conv}(\{ \u_0,\v( \u[1]),\v( \u[2]),p\} ),\quad
R_2( \u) = \op{conv}(\{ \u_0,\v( \u[1]),p,\v( \u)\} ),\quad
\end{displaymath}
Let $ \u' = ( \u_0, \u_2, \u_1, \u_3)$.
Define the $D_3$-cell to be
\begin{displaymath}
(R_1( \u) \cup R_1( \u'))\cap B.
\end{displaymath}
(The condition $b< 1.26$ implies the constraint $\norm{ \u_i}{ \u_j}<
2.62$, for $i,j\in\{0,1,2\}$.)

\case{$D_2$} Define the $D_3$-cell to be empty unless $c\ge 1.3$.
Set $R'( \u) = R_2( \u)$ if $b< 1.26$ an $R'( \u) = R( \u)$
otherwise.  Define the $D_3$-cell to be
\begin{displaymath}
R'( \u) \cap B.
\end{displaymath}

\begin{lemma}[]\guid{ZERRZRM}
Two $D_k$-cells either intersect in a set of measure
  zero or coincide.  The union of the $D_k$-cells is $\Omega(
  \u_0)\cap B$.
\end{lemma}
\indy{Index}{measure!measure zero}%

Every $D_k$-cell $X$ is eventually radial at $ \u_0$ and has
a well-defined solid angle $\sol(X)$.
\indy{Notation}{solidangle@$\sol$ (solid angle)}%

Every $D_k$-cell $X$ has an {\it exposed} surface area $\op{surf}(X)$.
This is the area of the intersection of the boundary of 
$\Omega( \u_0)\cap B$ with $X$.  It consists of the sum of the
areas of the analytic faces that do not meet the point $ \u_0$.
\indy{Index}{exposed}%
\indy{Index}{surface area!exposed}%
\indy{Notation}{surf@$\op{surf}$}%

Every $D_k$-cell has a set $E(X)$ of distinguished edges and a
dihedral angle $\dih(e)$ for $e\in E(X)$.  Each edge has a length
$h(e)$.  \indy{Index}{angle!dihedral}%
\indy{Notation}{dih}%
\indy{Index}{edge!length}%
\indy{Notation}{E@$E$ (edge)}%

\subsection{kissing estimates}

The strong dodecahedral conjecture reduces to a kissing estimate.
\indy{Index}{kissing estimate}%

\begin{lemma}[]\guid{JXVEXXV}
The surface area of $\Omega( \u_0)$ is at least that
of $\Omega( \u_0)\cap B$.
\end{lemma}
\indy{Index}{surface area}%
\indy{Index}{Dodecahedral Conjecture}%

\begin{definition}[$a$,~$b$,~$y_D$]\guid{TCOSFNQ}
Define constants $a$ and $b$ as follows.  Let $y_D$ be defined
by the condition
\begin{displaymath}
\sol(2,2,2,y_D,y_D,y_D) = \pi/5.
\end{displaymath}
Let $S=\{ \u_0, \u_1, \u_2, \u_3\}$ be given.  Let $\op{vol}(S)$ be
the volume of the intersection of the convex hull of $S$ with set of
points closer to $ \u_0$ than to any other point in $S$.  When $\norm{
  \u_0}{ \u_i} = 2$ and $\norm{ \u_i}{ \u_j} = y$ for $i,j\ge 1$, this
volume depends only on $y$. Write $v(y) = \op{vol}(S)$.  Set
\begin{displaymath}
f(y) = v(y) + a \sol(2,2,2,y,y,y) + 3 b \dih(2,2,2,y,y,y).
\end{displaymath}
The linear system
\begin{displaymath}
f(y_D) = 0,\quad f'(y_D) = 0
\end{displaymath}
has a unique solution in $a,b$ with values $a=-0.581\ldots$,
$b=0.0232\ldots$.
\end{definition}
\indy{Notation}{a@$a$ (dodecahedral parameter)}%
\indy{Notation}{b@$b$ (dodecahedral parameter)}%
\indy{Notation}{yd@$y_D$ (dodecahedral parameter)}%
\indy{Index}{convex hull}%
\indy{Index}{regular dodecahedron!volume}%

Note that the regular dodecahedron has volume $20 v(y_D)$ and surface
area $60 v(y_D)$.  Also,
\begin{displaymath}
2\sol(2,2,2,y_D,y_D,y_D) =\dih(2,2,2,y_D,y_D,y_D)=2\pi/5.
\end{displaymath}
\indy{Index}{regular dodecahedron}%
\indy{Index}{regular dodecahedron!surface area}%

\begin{conjecture}[local inequality]  For any $D_k$-cell $X$
\begin{displaymath}
\op{surf}(X)/3 + a \sol(X) + b \sum_{e\in E(X)} L(h(e)) \dih(e) \ge 0.
\end{displaymath}
Equality should hold precisely when $X$ is a $4$-cell with edges
$(2,2,2,y_D,y_D,y_D)$.
\end{conjecture}
\indy{Index}{local inequality}%

\begin{note} %%
  The dimension of a $k$-cell is at most $6$.  Thus, it should be
  possible to verify this inequality directly by interval arithmetic.
\end{note}

\begin{lemma}[]\guid{OIEKCEZ}
  The local inequality and the kissing number estimate
\begin{displaymath}
\sum L(h) \le 12
\end{displaymath}
imply the strong dodecahedral conjecture.
\end{lemma}

\begin{proof} 
  Sum the local inequality over all the $D_k$-cells in a Voronoi cell.
  The solid angles sum to $4\pi$ and the dihedral angles around each
  edge sum to $2\pi$:
\begin{displaymath}
\begin{array}{lll}
\op{surf}(\Omega) &\ge \op{surf}(\Omega\cap B)\\
&=\sum_X \op{surf}(X)\\
&\ge -12\pi a - 6\pi\, b  \sum L(h)\\
&\ge -12\pi a - 72\pi\, b\\
&= 60 (v(y_D) - f(y_D))\\
&= 60 v(y_D).\\
\end{array}
\end{displaymath}
The final term is the surface area of a regular dodecahedron.
\end{proof}

Thus, the strong dodecahedral conjecture follows from the same kissing
number estimate that is used to prove the Kepler conjecture.  The case
of equality in these inequalities occurs only for the regular
dodecahedron.

\section{Packings with Full Contact}



Call a nonempty packing $ V$ in $\ring{R}^3$ in which every point has
distance $2$ from $12$ other points a {\it packing with full
  contact}. L Fejes T\'oth has made the following conjecture.
\indy{Index}{packing!full contact}%
\indy{Index}{full contact}%
\indy{Notation}{V@$V$ (packing)}%

\begin{conjecture}\guid{BDEDUTL} Let $ V$ be a packing with full
  contact.  Then for every point $ \u\in V$, the arrangement of $12$
  around that point is the kissing configuration of the face-centered
  cubic or hexagonal-close packing.
\end{conjecture}
\indy{Index}{packing!hexagonal}%
\indy{Index}{packing!face-centered cubic}%
\indy{Index}{kissing configuration}%

\begin{lemma}[]\guid{LIHVTRE} Conjecture~\ref{conj:L12} implies that for
  every point $ \u\in V$ in a packing with full contact, $\norm{ \u'}{
    \u}\ge 2.52$, whenever $\norm{ \u'}{ \u}> 2$.
\end{lemma}
\indy{Index}{packing!full contact}%
\indy{Index}{full contact}%

\begin{proof} Let $ \u_1,\ldots, \u_{12}$ be the twelve kissing
  points.  The conjecture gives
\begin{displaymath}
  12 + L(h( \u, \u')) 
= \sum_{i=1}^{12} L(h( \u, \u_i)) + L(h( \u, \u')) \le 12.
\end{displaymath}
This imples that $L(h( \u, \u'))\le 0$, so $\norm{ \u}{ \u'}\ge 2.52$.
\end{proof}

Thus, the contact fan is the same as the standard fan (using a cutoff
for edges that is strictly less than $2.52$) for any point in a
packing with full contact.  Consider the hypermap attached to the fan.
\indy{Index}{hypermap}%
\indy{Index}{fan}%


\subsection{hypermaps with tame contact}

The notion of tameness is modified to cover hypermaps that arise as
the standard fan of a packing with full contact.
\indy{Index}{tame}%
\indy{Index}{hypermap!tame}%

\begin{definition}[b]\guid{IHRZTPV}
  Define $b:\ring{N}\times \ring{N}\to \ring{R}$ by
  $b\pqr{(p,q,0)}=1.541$, except for the following values:
\begin{displaymath}
b(0,3,0)=b(1,3,0)=0.618,\quad b(2,2,0)=0.412.
\end{displaymath}
\end{definition}
\indy{Notation}{b@$b$ (contact weight parameter)}%

\begin{definition}[d]\guid{VUJQZCG}
Define $d:\ring{N}\to \ring{R}$ by
\begin{displaymath}d(k) = \begin{cases}
0 & k=3, \\
0.206 & k=4, \\
0.483 & k=5, \\
0.760 & k=6, \\
1.037 & k=7, \\
1.314 & k=8,\\
\op{tgt}=1.541 & \text{otherwise}.
\end{cases}
\end{displaymath}
(In particular, $d(k) = 0.206 + 0.277 (k-4)$, for $k=4,\ldots,8$.)
\end{definition}
\indy{Notation}{d@$d$ (contact weight parameter)}%

\begin{definition}[weight~assignment]\guid{GLIQSFM}
%
  A {\it weight assignment\/} of a hypermap $H$ is a function $\tau$
  on the set of faces of $H$, taking values in the set of non-negative
  real numbers. A weight assignment is a {\it contact} weight
  assignment if the following properties hold:
%
\indy{Index}{weight assignment}%
\indy{Index}{contact!weight assignment}%
\indy{Notation}{ZZtau@$\tau$ (weight assignment)}%
\begin{enumerate}
\item If the face $F$ has cardinality $k$, then
$\tau(F) \ge d(k)$
\item If a node $v$ has type $(p,q,0)$, then
  \begin{displaymath}\sum_{F:\,v\cap F\ne\emptyset} \tau(F) \ge
    b{\pqr{(p,q,0)}}.\end{displaymath}
\end{enumerate}
The sum $\sum_F \tau(F)$ is called the {\it total weight} of $\tau$.
\indy{Index}{total weight}%
\end{definition}



A hypermap has {\it tame contact\/} if it satisfies the following
conditions.
%
\indy{Index}{tame contact}%
\indy{Index}{contact!tame}%
\indy{Index}{planar}%
\indy{Index}{biconnected}%
\indy{Index}{nondegenerate}%
\indy{Index}{loop}%
\indy{Index}{double join}%
%
\begin{nomerate}
%\label{definition:tame}
%1
\item \case{planar} The hypermap is plain, planar.
\item \case{biconnected} The hypermap is connected and biconnected.
  In particular, every face meets every node in at most one dart.
\item \case{nondegenerate} The edge map $e$ has no fixed points.
\item \case{no loops} The two darts of each edge lie in different nodes.
\item \case{no double join} The set of edges meeting any two given
  nodes has cardinality at most $1$.
%\label{definition:tame:40}
%      \item\case{blank}
%      \item \case{triangles} If $L$ is a contour loop with $3$ face
%        steps, and if there exists a node in the exterior of $L$,
%        then $L$ is a face of the hypermap.
\item \case{blank}
  % \item \case{quadrilaterals} If $L$ is a $4$-step contour loop, and
  %   there is at least one node in the exterior of $L$, then the
  %   interior of $L$ takes one of the forms illustrated in Figure
  %   \ref{fig:fourcircuit-FT}.
%    %\label{definition:tame:4-circuit-FT}
%    \begin{figure}[htb]
%        \centering
%        \myincludegraphics{\pdfp/fourcircuitFT.eps}
%        \caption{Tame $4$-circuits}
%        \label{fig:fourcircuit-FT}
%    \end{figure}
\item \case{face count} There are at least two faces.
\item \case{face size} The cardinality of each face is at least $3$
  and at most $8$.
%\label{definition:tame:length}
\item \case{node count} There are $12$ nodes.
\item \case{node size} The cardinality of every node is at least $2$
  and at most $4$.
%\label{definition:tame:degree}
%    \item \case{node} {\tt NO CONDITION}
%\label{definition:tame:degreeE}
\item \case{weights} There exists a contact weight assignment
of total weight less than the target, $\op{tgt}=1.541$.
%\label{definition:tame:squander}
\end{nomerate}
%

The set of all hypermaps with tame contact have been classified (up to
isomorphism, possibly reversing orientation).  There are $8$ such
hypermaps.  They have been classified by the same process described in
Section~\ref{sec:proof-classification}.  \indy{Index}{contact!tame}%
\indy{Index}{hypermap}%
\indy{Index}{hypermap!tame}%


\subsection{aggregate fans}

One may create aggregated fans in the same way as in
\cite{Hales:2006:DCG} so that each face is simple.  Here is a review
of the construction.  \indy{Index}{fan}%
\indy{Index}{fan!aggregate}%

Extend the length of the edges in the fan to anything less than
$\sqrt8$ to create a new fan.  The blades of the new fan do not meet
and give a new planar hypermap.  Call the newly added edges {\it cut
  edges}.  After moving different connected components closer together
(without creating any new edges to the stadard fan), the new hypermap
may be assumed to be connected.  Similarly, it is biconnected, without
loss of generality.  Call this the {\it aggregate fan, aggregate
  hypermap}, and so on.  Each face is now simple with a certain number
$r$ of standard edges (length exactly $2$) and a number $s$ of cut
edges (length at least $2.52$ and less than $\sqrt8$), for a total of
$r+s$ edges.  The parameters satisfy $0\le s$ and $3-s \le r$.
\indy{Index}{cut edges}%
\indy{Index}{hypermap!aggregate}%


\subsubsection{main estimate}

The function $\tau(F)$, when restricted to kissing configurations,
takes the following form:
\begin{displaymath}
\tau(F) = \sol(F) + (2-k(F)) \sol_0.
\end{displaymath}
\indy{Notation}{ZZtauf@$\tau(F)$ (aggregate fans)}%
\indy{Index}{kissing configuration}%
The following is the analogue for packings with full contact of the
main estimate:

\begin{theorem}\guid{VGJDQJV}\label{lemma:main-estimate-12} If a face
  $F$ of the aggregate hypermap has $r$ standard edges and $s$ cut
  edges, then
\begin{displaymath}\tau(F) \ge \min(d(r,s),\op{tgt})\end{displaymath}
where 
\begin{displaymath}
d(r,s) = 0.103 (2-s) + 0.277 (r+2s-4).
\end{displaymath}
\indy{Notation}{d@$d$ (contact weight constant)}
\end{theorem}

\begin{proof} This proof imitates the proof of the main estimate from
  \cite{Hales:2006:DCG}.  Here is a review of the method.

  It is enough to consider a single simple face of the aggregate
  hypermap, so without generality, assume that the set $V$ consists of
  nodes that meet the face.  Let $U\subset Y(V,E)$ be the connected
  component corresponding to the face $F$.

  Additional internal blades $C^0(\cdot)\subset U$ may be added to the
  fan of length at most $3.2$ as long as the blades do not cross.  By
  the additivity of the constants $d(r,s)$ (compare
  Equation~\ref{eqn:drs}), there is a counterexample that minimizes
  $r+s$.  Such a counterexample will not have any blades of length at
  most $3.2$ and none will be created as the example is deformed to
  decrease $\tau(F)$.  When $r$ and $s$ are fixed, deformations that
  decrease the solid angle $\sol(F)$, decrease $\tau(F)$.
  \indy{Index}{fan!blade}%

  Call a dart $x\in F$ concave or convex, according to whether
  $\op{azim}(x)\ge\pi$ or $\le\pi$.  Edges may be stretched (to
  decrease solid angle) at a concave dart until both edges at that
  node have length $3.2$. Assume this.  \indy{Index}{concave dart}%
  \indy{Index}{convex dart}%
  \indy{Index}{dart}%

\claim{[Assume $r+s\le6$ and there is a concave dart.]}  In this case, a
  half disk of arcradius $\arc(2,2,3.2)=1.854\ldots$ fits inside the
  region.  It has area
\begin{displaymath}
\pi(1-\cos(\arc(2,2,3.2)))=4.02\ldots.
\end{displaymath}
This gives
\begin{displaymath}
\tau(F) > 4.02 + (2-k)\sol_0 \ge 4.02 -4 \sol_0 > \op{tgt}.
\end{displaymath}
\indy{Index}{arcradius}%
\indy{Notation}{ZZtauf@$\tau(F)$ (aggregate fans)}%

\claim{[Every dart is convex.]}  In this case, every edge has
arclength at least $\arc(2,2,2)=\pi/3$.  By
Lemma~\ref{lemma:convex-hyp}, the number of sides $r+s$ satisfies
$(\pi/3)(r+s) < 2\pi$, so $r+s\le5$.  Lexell's theorem reduces the
argument to situations where every edge is as long or as short as
possible.  For cut edges, this means the length is $2.52$ or $3.2$,
and for uncut edges, this means the length is $2$ or $3.2$.  (If the
Lexell deformations produce a concave dart, then it falls back into
case 1.)  The only remaining degrees of freedom are the lengths of
diagonals.  As the polygon is at most a pentagon, the proof has now
been reduced to a finite number of interval arithmetic verifications
of dimension at most $2$.  \indy{Index}{Lexell's Theorem}%

\claim{[$r+s>6$ and there is a concave dart.]} In this case,
let $v,x$ be the number of concave and convex darts respectively.
Calculations~\cite[cc:lft]{hales:2009:nonlinear} show that the azimuth
angle at each convex darts is at least $1.73$.  Place a (wedge if
angle $1.73$ of a) disk at each convex dart of arcradius $\pi/6$, and
a half disk at each concave dart of arcradius $\arc(2,2,3.2)-\pi/6$.
These regions are disjoint and their combined area is less than
$\sol(F)$.  Hence
\begin{displaymath}
\begin{array}{lll}
\tau(F)&=\sol(F)+(2-v-x)\sol_0 \\
&\ge \pi v (1-\cos(\arc(2,2,3.2)-\pi/6)) \\
&\qquad+ 1.73 x (1-\cos(\pi/6)) + (2-v-x)\sol_0.
\end{array}
\end{displaymath}
The total number of darts is at most the total number of nodes, which
is at most $12$.  The rightmost term is at least $\op{tgt}$ if $v>
1$. Thus, $v=1$.  \indy{Index}{angle!azimuth}%
\indy{Index}{azimuth}%

As in~\cite{Hales:2006:DCG}, $v=1$ implies that the region is star
convex about the concave node.  This allows us to deform without
obstruction from other nodes.  The dart $y$ adjacent to the concave
dart $z$ can be deformed by decreasing the distance between it and
$z$, to decrease the solid angle of $U$.  In turn, the concave dart
can be deformed further to increase the distance between it and $y$,
to decrease the solid angle of $U$.  This shows that the function
$\tau$ has no local minimum among such arrangements, and the proof
necessarily reduces to a case previously considered.  This completes
the proof.
\end{proof}
\indy{Index}{star convex}%
\indy{Index}{convex!star}%

\subsubsection{no aggregates}

Let $(V,E)$ be the standard fan of a packing $V\subset \BB$ with full
contact.  Let $U$ be a connected component of $Y(V,E)$ and let $D'$ be
the set of all darts that lead into $U$.  For each $x\in D'$, let $j =
j(x) >0$ be the smallest natural number such that $f^j x$ and $x$ lie
at the same node.  Pick $x'\in D'$ that maximizes $x\mapsto j(x)$.
\indy{Index}{contact!full}%
\indy{Index}{fan}%

\begin{lemma}[]\guid{VDUBAWF}\label{lemma:DU} If $j(x')\le 5$, then the
  darts of $D'$ all belong to the same simple face $F$.
\end{lemma}
\indy{Index}{face!simple}%

\begin{proof} Assume to the contrary that either the face is not
  simple or there is more than one face that leads into $U$.  Then
  there is some node $v$ interior to the $j(x')$-gon.  The azimuth
  angles at $v$ are each less than $2\pi/5$. They cannot sum to $2\pi$
  as required.
\end{proof}

Consider possible aggregates with $j(x')\ge 6$ and $d(r,s)<\op{tgt}$.
From the classification of \cite[p.~126,~Fig.~12.1]{Hales:2006:DCG},
and the inequalities $d(9,0) > \op{tgt}$, $d(6,2) > \op{tgt}$, it
follows that the set $D'$ is either simple with at most $8$ darts, or
a nonsimple face $F$ with $8$ darts and $j(x')=6$.

\begin{lemma}[]\guid{BTZPFMU}\label{lemma:simple} The set of darts of
  the standard hypermap that lead into each connected component $U$ is
  a simple face $F$.
\end{lemma}
\indy{Index}{face!simple}%
\indy{Index}{hypermap}%

\begin{proof} One case remains: $8$ darts and $j(x')=6$.  Suppose this
  occurs in a contravening fan.  This arrangement involves $7$ nodes:
  the six nodes counted by $j(x')$ and the node in the center of the
  hexagonal arrangement.  As there are $12$ nodes in all, there are
  five additional nodes.  Each of these five nodes meets a
  non-triangular region.  The total weight is then at least
\begin{displaymath}
\begin{array}{lll}
d(8,0) + d(5,0) &> \op{tgt}\hbox{ or }\\
d(8,0) + 2 d(4,0) &> \op{tgt}\\
\end{array}
\end{displaymath}
\end{proof}
\indy{Index}{weight!total}%
\indy{Index}{weight}%








\subsection{contravention gives tame contact}

\begin{theorem}\guid{ZXZSVPH} The standard hypermap of a 
  packing $V\subset \BB$ with full contact is a tame contact hypermap.
\end{theorem}
\indy{Index}{hypermap!tame}%
\indy{Index}{hypermap!contact}%
\indy{Index}{hypermap}%
\indy{Index}{contact!full}%

\begin{proof} It is enough to go through the list of properties that
  define a tame contact hypermap and to verify that the standard
  hypermap satisfies each one.

\begin{nomerate}
\item \case{planar} The standard hypermap is plain and planar by the
  general properties of fans.
\item \case{biconnected} The hypermap is connected because of
  Lemma~\ref{lemma:DU} which establishes that the faces are in
  bijection with the connected components of $Y(V,E)$.  It is
  biconnected, because every face is simple by
  Lemma~\ref{lemma:simple}.  \indy{Index}{biconnected}%
\item \case{nondegenerate}, \case{no loops}, \case{no double join} The
  edge map has no fixed points by the general properties of fans.
  There are no loops or multiple joins by the general properties of
  fans.
\item \case{face count} Each node has at least two darts by
  biconnectness. Each face is simple; so the two darts at a node lie
  in different faces.  Thus, there are at least two faces.
\item \case{face size} The cardinality of each face is at least three
  because there are no loops or multiple joins.  The cardinality of a
  face is at most $8$ because of the estimate $d(9,0)>\op{tgt}$.
\item \case{node count} There are twelve nodes by the definition of a
   packing with full contact.
\item \case{node size} It is already established that the cardinality
  of each node is at least two.  The proof that the cardinality is
  never five or greater appears in Lemma~\ref{lemma:no-5}.
\item \case{weights} The inequality $\tau(F)\ge d(k)$ is
  Theorem~\ref{lemma:main-estimate-12}.
  \indy{Notation}{ZZtauf@$\tau(F)$}%
  The total weight of the weight assignment is given by
  equation~\eqn{eqn:delta0}:
\begin{displaymath}
\sum_F \tau(F) = (4\pi - 20\sol_0) < \op{tgt}.
\end{displaymath}
\indy{Index}{weight!total}%
Let $v$ be a node of type $(p,q,0)$.  Let $A$ be the set of faces that
meet the node $v$. Then
\begin{displaymath}
\tau_{F\in A}\tau(F) > d(4)~q.
\end{displaymath}
This gives the nonzero entries in the table of bounds $b(p,q,0)$.  The
remaining entries follow from Lemma~\ref{lemma:no-5}.
\end{nomerate}
\end{proof}




\begin{lemma}[]\guid{CQRHDZE}\label{lemma:no-5} 
  Every node has degree at most four.  Furthermore, suppose the
  hypermap of a  packing with full contact has node of type
  $(p,q,0)$.  Then $(p,q)$ must be one of the following values:
\begin{displaymath}
(0,3)~(1,3)~(2,2).
\end{displaymath}
\end{lemma}

\begin{proof} Let $\alpha_0 = \op{azim}(2,2,2,2,2,2)$.  The azimuth
  angle of a rhombus lies between $\beta_0 =
  \op{azim}(2,2,2,2.52,2,2)$ and $\beta_1 =
  2\op{azim}(2,2,2,2,2.52,2)$.  That of an region with at least $5$
  sides is at least $\beta_0$ and at most $2\pi$.  Thus,
\begin{displaymath}
p\alpha_0 + (q+r) \beta_0 \le 2\pi \le p\alpha_0 + q\beta_1 + r 2\pi.
\end{displaymath}
The only three solutions to these inequalities among the natural
numbers $(p,q)$ with $r=0$ are those given. There is no solution for
$(p,q,r)$ in natural numbers, if $p+q+r\ge 5$.
\end{proof}





\subsection{linear programs and conclusion}

\begin{lemma}[]\guid{YRTPQXK}\label{lemma:kiss-fcc} Let $H$ be the
  hypermap of the face-centered cubic or hexagonal-close packing.
  Assume that it occurs as the standard hypermap of a  packing $V\subset \BB$
  with full contact.  Then the kissing configuration of the 
  packing is congruent to that of the face-centered cubic or hexagonal
  close packing.
\end{lemma}
\indy{Index}{packing!hexagonal}%
\indy{Index}{packing!face-centered cubic}%
\indy{Index}{kissing configuration}%
\indy{Index}{contact!full}%
\indy{Index}{hypermap}%

\begin{proof} Every face of the hypermap is a triangle or
  quadrilateral.  The hypermap is the same as the contact hypermap.
  The contact hypermap of the face-centered cubic and hexagonal-close
  packings fixes the eight regular triangles in the kissing
  arrangement.  The eight regular triangles fix the kissing
  arrangement up to congruence.
\end{proof}

\begin{lemma}[]\guid{MWWSZTX}\label{lemma:fcc-ft} Let $H$ be a hypermap
  with tame contact.  Assume that it occurs as the aggregate fan of a
   packing $V\subset\BB$  with full contact.  Then $H$ is the contact
  hypermap of the face-centered cubic or hexagonal-close packing.
\end{lemma}

\begin{proof} According to the classification of hypermaps with tame
  contact, there are eight hypermaps.  Two are the hypermaps of the
  fcc and hcp.  The remaining six must be eliminated.
\end{proof}

\begin{note}%%XX
  I have not eliminated the other six, but it seems fairly trivial in
  comparison with the linear programming that is required for the
  proof of the Kepler Conjecture.  There are some obvious linear
  programming constraints:
\begin{itemize}
\item the angles around each node sum to $2\pi$.
\item each angle of a triangle is $\alpha_0$.
\item each angle of each rhombus is between $\beta_0$ and $\beta_1$.
\item the opposite angles of each rhombus are equal.
\item the sum of two adjacent angles of a rhombus are between
\begin{displaymath}
\beta_0 + \beta_1 \hbox{ and } \op{azim}(2,2,2,\sqrt8,2,2)2.
\end{displaymath}
\end{itemize}
I suspect that these inequalities together possibly with equally
trivial inequalities for a pentagon will show that all but the hcp and
fcc are not feasible linear programs.
\end{note}

\begin{theorem}[packings with full contact]\guid{ANSXBOJ}  
Fejes T\'oth's conjecture on packings with full contact holds.
\end{theorem}
\indy{Index}{packing}%

\begin{proof} The standard hypermap of a  packing with full
  contact has tame contact.  By Theorem~\ref{lemma:fcc-ft}, this
  hypermap is that of the fcc or hcp.  By Lemma~\ref{lemma:kiss-fcc},
  the kissing configuration of the  packing is congruent to
  the fcc or hcp.  As the center of the packing may be chosen at an
  arbitrary point in the packing, every point in the packing is
  congruent to one of these two arrangements.  The result follows.
\end{proof}
\indy{Index}{hypermap}%
\indy{Index}{tame}%

