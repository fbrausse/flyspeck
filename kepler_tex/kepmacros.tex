 
%-%
% --Repository--
%-%
% generate revision number by
% svn propset svn:keywords "LastChangedRevision" kepmacros.tex
\def\svninfo{%
  TeXed on \today; \hfill\break
  Repository Root: https://flyspeck.googlecode.com/svn \hfill\break
  SVN $LastChangedRevision$
  }

%-%
% --Fonts--
%-%
\font\twrm=cmr8

%-%
% --Graphics--
%-%
%set \showgraphics option in flag_fly.tex
% flypaper graphics
\def\myincludegraphics#1{%
      \if\showgraphics t{\includegraphics{#1}}%
      \else{\includegraphics{noimage.eps}}\fi}
\def\szincludegraphics[#1]#2{%
      \if\showgraphics t{\includegraphics{#2}}%
      \else{\includegraphics{noimage.eps}}\fi}

% kepler graphics
\def\pdffigtemplatex#1#2#3{%
\begin{figure}[htb]%
  \centering
  \myincludegraphics{\pdfp/#1}
  \caption{#3}
  \label{fig:#2}%
\end{figure}%
}
\def\pdfg#1#2#3{\if\showgraphics t{\pdffigtemplatex{#1}{#2}{#3}}\else{}\fi}


% tarski graphics
\def\pdffigtemplate#1#2#3{%
\begin{figure}[htb]%
  \centering
  \myincludegraphics{\pdf/#1}
  \caption{#3}
  \label{tarski:fig:#2}%
\end{figure}%
}
\def\pdffig#1#2#3{\if\showgraphics t{\pdffigtemplate{#1}{#2}{#3}}\else{}\fi}

%-%
% --Footnotes--
%-%
% http://help-csli.stanford.edu/tex/latex-footnotes.shtml
\long\def\symbolfootnote[#1]#2{\begingroup%
\def\thefootnote{\ensuremath{\fnsymbol{footnote}}}\footnote[#1]{#2}\endgroup}

%-%
% --Special Formatting--
%-%
% http://en.wikibooks.org/wiki/LaTeX/Formatting#List_Structures
%\renewcommand{\labelitemii}{$\star$}
\renewcommand{\labelitemii}{$\circ$}
\newenvironment{summary}
  {\begingroup\bigskip\narrower\noindent{\bf Summary.~}\it}
  {~\ding{98}\par\phantom{!}\endgroup\bigskip}
\newenvironment{nomerate}
  {\renewcommand{\labelitemi}{}\begin{itemize}}
  {\end{itemize}}
\def\uncase#1{{\sc #1}}
\def\case#1{{\sc (#1)}}
\def\claim#1{{\it  #1}}




%-%
% --Indexing, References, Citations--
%-%
\def\indy#1#2{\index{index/#1}{#2}}
\def\eqn#1{{\bf (\ref{#1})}}
\def\newterm#1{\indy{Index}{#1}{\it #1}\relax}

%-%
% --Proof Display--
%-%
% set with \displayallproof in flag_fly. If f, then proofs are swallowed.
%% "proved" environment. toggle with \displayallproof
%
\def\hide#1{}
\def\swallowed{\relax}
\def\swallow#1\swallowed{}
\newenvironment{iproved}{}{}
\newenvironment{proved}{\resetproved\begin{iproved}}{\end{iproved}}
\def\hideproof{\renewenvironment{iproved}{%
   \centerline{\it -- Proof Proofed --}
  \renewenvironment{itemize}{}{}
  \renewenvironment{enumerate}{}{}
  \def\item{\relax}
  \catcode13=12
  \swallow
}{}}
\def\showproof{\renewenvironment{iproved}{\begin{proof}}{\end{proof}}}
\def\resetproved{\if\displayallproof t\showproof\else\hideproof\fi}



%-%
% --Debugging Information--
%-%
%% verbose:
\def\rating#1{\if\displayrating t{{\textsc {[rating={\ensuremath {#1}}].\ }}}\else{}\fi}
\def\oldrating#1{\if\displayrating t{{\textsc {[former rating={\ensuremath {#1}}].\ }}}\else{}\fi}
\def\formal#1{\if\verbose t{{\tt [formal: #1].\ }}\else{}\fi}
\def\formalauthor#1{\if\verbose t{{\tt [formal proof by: #1].\ }}\else{}\fi}
\def\footformal#1{\if\verbose t{\footnote{\formal{#1}}}\else{}\fi}
\def\dcg#1#2{{\if\verbose t{{\tt{[DCG-#1]}}\indy{References}{ZC{#2 #1}@{DCG-#1}|page{#2}}}\else{}\fi}}
\def\tlabel#1{\label{#1}\if\verbose t{{\tt [#1].\ }%
   \indy{References}{#1|itt}}\else{}\fi}
\def\ifverbose#1{\if\verbose t{{#1}}\else{}\fi}
\def\guid#1{{\tt[#1].\ }\indy{References}{ZA{#1}@{#1}|itt}}
\def\calc#1{{\textsc{calc-#1}}\indy{Interval}{{#1}@{#1}}}
\def\guid#1{{\tt [#1]}}


%-%
%--Formatting--
%-%
\def\dfrac#1#2{\frac{\displaystyle #1}{\displaystyle #2}}

%-%
%--Redefining--
%-%
\def\emptyset{\varnothing}


%-%
% --Symbols--
%-%
% norm
\def\|{{\hskip0.1em|\hskip-0.15em|\hskip0.1em}}
\def\mid{\ :\ }
\def\norm#1#2{\|#1 - #2\|}
\def\normo#1{{\|#1\|}}
\def\sland{\ \land\ }


% mathcal
\def\CalV{{\mathcal V}}
\def\CalL{{\mathcal L}}
\def\BB{{\mathcal B}}
\def\powerset{{\mathscr P}}

% brackets
\def\leftopen{(}
\def\leftclosed{[}
\def\rightopen{)}
\def\rightclosed{]}

% mathbb
\def\R{{\mathbb R}}
\def\N{{\mathbb N}}
\newcommand{\ring}[1]{\mathbb{#1}}
\def\A{{\mathbf A}}
\def\Rp{\ring{R}^{3\,\prime}}

% vector notation
\def\v{{\mathbf v}}
\def\u{{\mathbf u}}
\def\w{{\mathbf w}}
\def\e{{\mathbf e} }  
\def\p{{\mathbf p}}
\def\q{{\mathbf q}}

% operatorname
\def\op#1{{\operatorname{#1}}}
\def\optt#1{{\operatorname{{\texttt{#1}}}}}

\def\opat{{\op{@}}}
\def\atn{\op{arctan\ensuremath{_2}}}
\def\azim{\op{azim}}
\def\sol{\operatorname{sol}}
\def\vol{\op{vol}}
\def\dih{\operatorname{dih}}
\def\Adih{\operatorname{Adih}}
\def\arc{\operatorname{arc}}
\def\rad{\operatorname{rad}}
\def\bool{\operatorname{bool}}
\def\true{\op{true}}
\def\false{\op{false}}
\def\tangle#1{\langle #1\rangle}
\def\ceil#1{\lceil #1\rceil}
\def\floor#1{\lfloor #1\rfloor}
\def\ups{\upsilonup} % Needs txfonts; else use \upsilon
%\def\orz{\varthetaup} % center of packing
\def\orz{{\mathbf 0}} % center of packing
%\def\comp#1{\llbracket #1 \rrbracket}
\def\comp#1{[#1]}
\def\Wdart{W^0_{\text{dart}}}
\def\bWdart{W_{\text{dart}}}
%\def\Wedge{W_{\text{edge}}}
\def\cell{\operatorname{cell}}
\def\dimaff{\operatorname{dim\,aff}}
\def\card{\op{card}}

\def\del{\partial}
\def\doct{\delta_{oct}}
\def\dtet{\delta_{tet}}
\def\hm{{h_0}} % 1.26
\def\tgt{\operatorname{\it{target}}}
\def\pqr#1{#1} % marks type (p,q,r).

%% HYPERMAP macros:
% avoid e for both hypermap edge and edge {v,w}
\def\ee{\varepsilonup}
\def\ocirc{}

%% FORMULATION macros:
%\def\lam{\lambda}
%\def\Lam{\Lambda}
\def\bu{{\underline{\u}}}
\def\bv{{\underline{\v}}}
%\def\arcs#1#2#3{{\arcV(#1,\{#2,#3\})}}