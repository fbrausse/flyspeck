% This is the first chapter.
% Author: Thomas Hales, copyright 2009.

\chapter{Trigonometry}\label{part:trig}
\indy{Index}{trigonometry}

The first part of this book consists of a series of essays
on packings.  The second part of this book presents
a systematic development of foundational material on
trigonometry, volume, hypermap, and fan.  There is a separate
chapter on each of these topics.  The purpose of the
foundational material is to build a bridge between
the foundations of mathematics, as presented in formal
theorem proving systems such as {\tt HOL Light} and the
solution to the packing problem.  
This chapter is the first of the four foundational chapters.


In this chapter, trigonometry is developed analytically.  The basic
trigonometric functions are defined by their power series
representations, and calculus of a single real variable is used to
develop the basic properties of these functions.  We give a brief
review of the definitions and properties.


\subsection{sine and cosine}

The cosine and sine functions are defined by their infinite series:%
\footformal{sin,\ cos,\ SIN\_0\, COS\_0}%
\footformal{DIFF\_SIN\, DIFF\_COS}
    \begin{equation}\label{eqn:cos-def}\cos(x) = 1 - x^2/2! + x^4/4! \cdots,\qquad
  \sin(x) = x - x^3/3! + x^5/5! \cdots.
    \indy{Index}{cos}
    \indy{Index}{cosine}
    \indy{Index}{sin}
    \indy{Index}{sine}
    \indy{Index}{cosine!series definition}
    \indy{Index}{sine!series definition}
    \end{equation}
Convergence is absolute for every real number $x$.
The series can be evaluated at $0$:
    \begin{equation}\label{eqn:cos0}
    \cos(0) = 1,\qquad \sin(0) = 0.
    \end{equation}

These series may be differentiated term-by-term to establish the identities:
\indy{Index}{cosine!derivative}
    \begin{equation}\label{eqn:cos'}
    \frac{d \cos(x)}{dx} = -\sin(x),\qquad \frac{ \sin(x)}{dx} = \cos(x).
    \end{equation}

The powers $(\cos(x))^n$ and $(\sin(x))^n$ are conventionally written
$\cos^n(x)$ and $\sin^n(x)$.

Trigonometric identities follow easily from these definitions.  We
give a couple of examples.
\indy{Index}{trigonometric identities}

\begin{lemma}\guid{WPMXVYZ}\label{lemma:circle}\formal{SIN\_CIRCLE} 
   $$\sin^2(x) + \cos^2(x) = 1.$$
\end{lemma}
\indy{Index}{trigonometric identities!circle identity}

\begin{proved}
By basic properties of derivatives and Equation~\ref{eqn:cos'},
the derivative of the function $f(x) = \cos^2(x) +\sin^2(x)$ is
identically zero.   So the function is constant.  From
Equation~\ref{eqn:cos0}, it follows that $f(x)=f(0)=1$.
\swallowed\end{proved}


\begin{lemma}\guid{WNYVJPE}\label{lemma:sin-add}\formal{SIN\_ADD,\ COS\_ADD}
  $$\begin{array}{lll}
  \sin(x+y) &= \sin(x)\cos(y) + \cos(x)\sin(y)\\
  \cos(x+y)  &= \cos(x)\cos(y) - \sin(x)\sin(y).
  \end{array}$$
\end{lemma}
\indy{Index}{trigonometric identities!addition}

\begin{proved}
Fix $y$.  Let
    $$\begin{array}{lll}
    f(x) = &(\cos(x+y) - \cos(x)\cos(y) +
    \sin(x)\sin(y))^2 +\\ & (\sin(x+y) -\sin(x)\cos(y) -
    \cos(x)\sin(y))^2.
    \end{array}$$
The derivative of $f$ is identically zero.  The function is
therefore constant. Also, $f(0)=0$.  Thus, $f$ is identically zero.
If a sum of real squares is zero, the individual terms are zero. The
identities follow.
\swallowed\end{proved}

\begin{lemma}\guid{KGLLRQT}\label{lemma:cos-neg}\formal{COS\_NEG,\ SIN\_NEG}
The cosine is an even function.  The sine is an odd
function.
    $$\cos(-x) = \cos(x),\quad\sin(-x) = -\sin(x).$$
\end{lemma}
\indy{Index}{function}
\indy{Index}{function!even}
\indy{Index}{function!odd}


\begin{proved}
The derivative of
    $$(\cos(-x) - \cos(x))^2 + (\sin(-x) +\sin(x))^2$$
is identically zero.  Argue as in the previous lemmas.
\swallowed\end{proved}

\subsection{periodicity}
\label{sec:pi}
\indy{Index}{periodicity}

It is known that the cosine function has a unique root between $0$
and $2$. The constant $\pi$ is defined as twice that root.  Thus, by
definition $\cos(\pi/2) = 0$, and the cosine has no root greater
than $0$ and less than $\pi/2$. The $\cos$ function is in fact
positive on the interval $\leftclosed 0,\pi/2\rightopen$.
\indy{Index}{cosine!roots}

\begin{lemma}\guid{CPIREMF}\label{lemma:sin-pi2}\formal{SIN\_PI2}
    $\sin (\pi/2) = 1.$
\end{lemma}

\begin{proved}
The derivative of $\sin$ is non-negative between $0$ and $\pi/2$.
Its value at $0$ is $0$.  Thus, $\sin$ is non-negative on
$[0,\pi/2]$.  We can check the identity of the lemma by checking
that the squares of the two sides are equal. Then $\sin^2(\pi/2) =
{1-\cos^2(\pi/2)} = 1$.
\swallowed\end{proved}

\begin{lemma}\guid{SCEZKRH}\rating{1}\label{lemma:cos-sin}
    $$\cos(x) = \sin(\pi/2 - x).$$
\end{lemma}

\begin{proved}
    Apply the addition law for the sine function,
    $$\sin(\pi/2 - x) = \sin(\pi/2)\cos(-x) + \cos(\pi/2)\sin(-x)$$
    and use
    $\sin(\pi/2) = 1$, $\cos(\pi/2) = 0$.  Then use that $\cos$ is
    an even function.
\swallowed\end{proved}

Similarly,~%
\footformal{SIN\_COS,\ SIN\_PERIODIC\_PI,\ COS\_PERIODIC\_PI, 
SIN\_PERIODIC,\ COS\_PERIODIC}%
$\cos(\pi/2 - x) = \sin(x)$ or $\cos(\pi/2 + x) =
-\sin(x)$, $\sin(\pi/2 + x) = \cos(x)$.  Further,
    $$\begin{array}{lll}
      \sin(\pi + x) &= \phantom{-}\cos(\pi/2 + x) &= -\sin(x),\\
      \cos(\pi + x) &= -\sin(\pi/2 + x) &= -\cos(x),\\
      \sin(2\pi + x) &= -\sin(\pi + x) &= \phantom{-}\sin(x),\\
      \cos(2\pi + x) &= -\cos(\pi + x) &= \phantom{-}\cos(x).
      \end{array}$$
\indy{Index}{trigonometric functions!periodicity}
\indy{Index}{function!periodic}
\indy{Index}{periodicity}


\subsection{tangent}
\label{sec:tangent}

\begin{definition}[tangent]\label{def:tan}
Let $\tan(x) = \sin(x)/\cos(x)$, defined when $\cos(x)\ne0$.
\indy{Index}{tangent}
\end{definition}


\begin{lemma}\guid{KWYPRWZ}\label{lemma:tan-add}\formal{TAN\_ADD}
    If $\cos(x)\ne 0$, $\cos(y)\ne 0$, and $\cos(x+y)\ne0$ then
    $$\tan(x+y) = \frac{\tan(x) + \tan(y) }{ 1 - \tan(x)\tan(y)}$$
\end{lemma}
\indy{Index}{trigonometric identities!tangent}

\begin{proved}
  Divide the first line of Lemma~\ref{lemma:sin-add} by the second
  line of the same lemma.
\swallowed\end{proved}

\begin{lemma}\guid{KSQDZSF}\label{lemma:tan-pi4}\formal{TAN\_PI4}
    $$\tan(\pi/4) = 1.$$
\end{lemma}

\begin{proved}
    $$\tan(\pi/4) = \sin(\pi/2-\pi/4)/\cos(\pi/4) =
    \cos(\pi/4)/\cos(\pi/4) = 1.$$
\swallowed\end{proved}


\subsection{arctangent}

We review the properties of the arctangent function.

\begin{definition}[arctangent]\label{def:arctan}\formal{atn,\ ATN,\ ATN\_TAN,\ ATN\_BOUNDS,\ TAN\_ATN}
There is a unique function $\arctan:\ring{R}\to\ring{R}$ with
image $(-\pi/2,\pi/2)$ such that
    $$\tan(\arctan x) =x.$$
\indy{Index}{arctangent}
\end{definition}

If $-\pi/2 < x < \pi/2$, then also $\arctan(\tan(x)) = x$. In
particular,\footformal{ATN\_1}
    $$\arctan(1) = \arctan(\tan(\pi/4)) = \pi/4.$$


The function $\arctan$ is differentiable with derivative%
\footformal{DIFF\_ATN,\ ATN\_MONO\_LT,\ ATN\_MONO\_LT\_EQ}
    $$\frac{d \arctan(x)}{dx} = \frac{1}{1 + x^2}.$$
The derivative is everywhere positive, and the function $\arctan$ is
strictly increasing.
\indy{Index}{arctangent!derivative}


It will often be necessary to use $\arctan(y/x)$ when $x$ is near $0$.
For that reason we introduce the following variant of $\arctan$:
$$
\atn: \ring{R}^2\setminus\{(0,0)\} \to (-\pi,\pi].
$$
$$
\atn(x,y) = \begin{cases}
   \arctan(y/x) & x > 0\\
   \pi/2- \arctan(x/y) & y > 0 \\
   \pi + \arctan(y/x) & x< 0,\  y\ge 0\\
   -\pi/2- \arctan(x/y) & y< 0 \\
\end{cases}
\indy{Index}{arctan}
$$
There is some overlap between cases. Nevertheless, 
this function is well-defined.  
This function is the {\it arctan2} function of ANSI C.  It gives the
angular argument of $(x,y)$ (with the branch cut along the negative axis).
That is, $x = r\cos\phi$, $y=r\sin\phi$, with $\phi=\atn(x,y)$.
Note that some implementations of this function reverse the order of the two arguments: $(y,x)$.
\indy{Index}{arctangent!near 0}


\subsection{inverse trig}
\indy{Index}{function!inverse trigonometric}
Other inverse trigonometric functions will generally be reduced to
the arctangent.  In this section, we define the $\arccos$ function and show how it can be expressed in terms of $\atn$.

\begin{definition}[arccos]\label{def:arccos}\formal{acs,\ ACS\_COS,\ COS\_ACS}
$\arccos y$ is the function on the interval $[-1,1]$
taking values in $[0,\pi]$ that is the inverse function of $\cos$:
    $$\begin{array}{lll}
        y\in [-1,1] &\Rightarrow \cos(\arccos y) = y\\
        x\in[0,\pi] &\Rightarrow \arccos(\cos x) = x
    \end{array}$$
\indy{Index}{arccos}
\end{definition}

\begin{lemma}\guid{FMGMALU}\rating{2}\label{lemma:sin-arccos}\formal{sin\_acs\_t} 
If $y\in[-1,1]$, then
    $$\sin(\arccos(y)) = \sqrt{1-y^2}.$$
\end{lemma}

\begin{proved}
    The range of $\arccos(y)$ is $[0,\pi]$.  On this interval, $\sin$
    is nonnegative.  Thus, we can check the identity by squaring
    both sides.  It then follows from the circle identity
    (Lemma~\ref{lemma:circle}).
\swallowed\end{proved}


Our preference is to remove the $\arccos$ function whenever
possible, by replacing it with the $\atn$ function through the
following identity.  


\begin{lemma}\guid{OUIJTWY}\rating{1}\label{lemma:arccos-arctan}\formal{acs\_atn2\_t}  
If $y\in [-1,1]$, then
    $$\arccos(y) = \pi/2 - \atn({ \sqrt{1-y^2}},{y}).$$
\end{lemma}
\indy{Index}{trigonometric identities!arccos}
\indy{Index}{trigonometric identities!arctan}

\begin{proved}
The endpoints $y=\pm1$ can be checked directly.
If $y\in (-1,1)$,  $x = \arccos(y)$, 
$z = \arctan(y/\sqrt{1-y^2})$, then
    $-\pi/2 < \pi/2 - x < \pi/2$, and $-\pi/2 < z < \pi/2$.  It is
    therefore enough to check that
        $\tan(\pi/2 - x) = \tan(z)$.
    But
        $$\tan(\pi/2-x) = \frac{\cos(x)}{\sin(x)} = \frac{y}{
        \sin(\arccos(y))} = \frac{y}{ \sqrt{1-y^2}} = \tan(z).$$
\swallowed\end{proved}

\begin{definition}[arcsine]\label{def:arcsin}\formal{asn}
$\arcsin y$ is the function on the interval $\leftclosed
-1,1\rightclosed$ taking values in $[-\pi/2,\pi/2]$ that is the
inverse function of $\sin$:
    $$\begin{array}{lll}
        y\in [-1,1] &\Rightarrow \sin(\arcsin y) = y\\
        x\in[-\pi/2,\pi/2] &\Rightarrow \arcsin(\sin x) = x
    \end{array}$$
\indy{Index}{arcsine}
\end{definition}

As we do not have need for the arcsine function, we do not develop
any of its properties.





\section{Vector Geometry}

The previous section defines basic trigonometic and
inverse trigonometric functions.  This section turns
to vector geometry.  It describes the basic vector
space operations in $\ring{R}^N$, the dot product, and
norm.  It develops basic properties of the norm such
as the Cauchy-Schwartz and triangle inequalities.
It presents basic definitions such as the affine span and
convex hull.

For any finite set, $\ring{R}^N$ is defined as the set of functions
$f:N\to\ring{R}$. We write $v_i$ for the value of the function $v$ at $i\in N$.
\indy{Index}{real numbers}
We call elements of $\ring{R}^N$ {\it vectors}.
\indy{Index}{vector}
  There is a zero vector $0$, defined as the function that is identically zero.
\indy{Index}{vector!zero}
If $n\in\ring{N}$, we write $\ring{R}^n$ as an abbreviation of $\ring{R}^N$,
where $N=\{0,\ldots,n-1\}$.
\indy{Notation}{RN@$\ring(R)^n$}

There is an addition and scalar multiplication on $\ring{R}^N$
defined by addition and scalar multiplication of functions.
    $$\begin{array}{lll}
    (u + v)_i &= u_i + v_i.\\
    (s u)_i & s u_i,\quad s\in\ring{R}.\\
    \end{array}
    $$
\indy{Index}{function!addition}
\indy{Index}{function!multiplication}
The addition is commutative and associative, and gives
$\ring{R}^N$ the properties of a vector space.
\indy{Index}{vector space} 
We define the difference of two vectors as $u - v = u + (-1) v$.
\indy{Index}{vector!difference}

There is a dot product
$$(\cdot):\ring{R}^N\to\ring{R}^N\to\ring{R}$$ defined by
    $$u\cdot v = \sum_{i\in N} u_i v_i.$$
\indy{Index}{vector!dot product}
The dot product satisfies the following
properties:
    $$\begin{array}{lll}
        u \cdot (v + w) &= u \cdot v + u \cdot w\\
        (u + v)\cdot w &= u \cdot w + v \cdot w\\
        (s u)\cdot w &= s(u \cdot w) = u \cdot (s w)\\
        0 &\le u\cdot u\\
    \end{array}$$


There is a norm\indy{Index}{norm}
$$\normo{u} = \sqrt{u\cdot u}.$$
It satisfies $\normo{ s u } = \normo{s} \, \normo{u}$.  We have $\normo{u}=0$  if and
only if $u=0$, the zero function.
\indy{Index}{vector!norm}

\begin{lemma}\guid{JJKJALK}[Cauchy-Schwarz inequality]\formal{Jordan/metric\_spaces.ml:cauchy\_schwartz}
    $$|u \cdot v| \le \normo{u}\,\normo{v}.$$
Equality holds exactly when $\normo{v}u = \pm \normo{u}v$.  Furthermore,
$u\cdot v = \normo{u}\,\normo{v}$ holds exactly when $\normo{v} u = \normo{u} v$.
\end{lemma}
\indy{Index}{Cauchy-Schwarz Inequality}

\begin{proved}
   Let $w = \normo{v} u \pm \normo{u} v$.  Then expanding $w\cdot w$ we get
    $$0\le w\cdot w = 2\normo{u}^2\normo{v}^2 \pm 2\normo{u}\, \normo{v} (u\cdot v) =
    2\normo{u}\,
    \normo{v} (\normo{u}\, \normo{v} \pm (u \cdot v)).$$
    If $2\normo{u} \,\normo{v} = 0$, then $u$ or $v$ is zero, and the result
    easily follows.  Otherwise divide both sides of the inequality
    by the positive quantity $2 \normo{u} \,\normo{v}$ to get the result.
\swallowed\end{proved}

\begin{lemma}\guid{OIPLPTM}[triangle inequality]\formal{Jordan/metric\_spaces.ml:norm\_triangle}
  $$
  \normo{u + v} \le \normo{u} + \normo{v }.
  $$
Equality holds exactly when $\normo{v}u = \normo{u}v$.
\end{lemma}
\indy{Index}{Triangle Inequality}

\begin{proved}
Both sides are nonnegative; it is enough to compare the squares of
both sides.  By the Cauchy-Schwartz inequality,
    $$\normo{u + v}^2 = u\cdot u + 2 u\cdot v + v\cdot v \le
      u\cdot u + 2 \normo{ u}\,\normo{v} + v\cdot v = (\normo{u}+\normo{v})^2.
    $$
The case of equality follows from the case of equality in the
Cauchy-Schwartz inequality.
\swallowed\end{proved}

There is a distance function $d(u,v) = \norm{ u }{ v}$ that makes
$\ring{R}^N$ into a metric space.  
\indy{Index}{metric space}
\indy{Index}{function!distance}






\subsection{parallelepiped}
\indy{Index}{parallelepiped}

Let $\{v_1,v_2,v_3,v_4\}$ be a set of vectors of cardinality four.
There are $6$ (four choose two) different pairs of vectors. We let
$y_{ij} = \norm{v_i}{v_j}$ and $x_{ij}=y_{ij}^2$.
\indy{Notation}{yij@$y_(ij)$}

The following polynomial appears in many different functions related to the geometry of three dimensions.  The formula following the definition shows that it is closely related to the square of the volume of a parallelepiped.  The interpretation as  volume is not relevant until the next chapter.  Its non negativity is immediately relevant. 
\indy{Index}{parallelepiped!volume}

%% WW repeated DEF.
\begin{definition}[$\Delta$]\label{def:delta}\formal{definitions\_kepler.ml:delta\_x}  Let 
$$
\begin{array}{lll}
\Delta(x_1,\ldots,x_6) &= x_1 x_4 (- x_1+x_2+x_3- x_4+x_5+x_6)+\\&
            x_2 x_5 (x_1- x_2+x_3+x_4- x_5+x_6)
            +x_3 x_6 (x_1+x_2- x_3+x_4+x_5- x_6)
            - \\&x_2 x_3 x_4- x_1 x_3 x_5- x_1 x_2 x_6- x_4 x_5 x_6
\end{array}
$$
\end{definition}
\indy{Notation}{ZZDelta@$\Delta$}

%We write $\Delta_j$ for the $j$th partial derivative of $\Delta$. 
%Let $D = \det(v_2-v_1,v_3-v_1,v_4-v_1)$.
By Tarski arithmetic, %\tarf{cayley-menger-pos}, 
%  $$
%  4 D^2 = \Delta(x_{12},x_{13},x_{14},x_{34},x_{24},x_{23}).
%  $$
%In particular, 
the polynomial $\Delta$ is positive on values $x_{ij}=\norm{v_i}{v_j}^2$
that come from affinely independent vectors $\{v_1,\ldots,v_4\}$.
\indy{Index}{determinant}





\subsection{affine geometry}


%% WW This section is repeated verbatim in Tarski collection

Most of the following definitions apply to $n$-dimensional
Euclidean space; however, this book uses them only in
three dimensions.  The first definition 
gives the affine span of a finite set.  For example,
the affine span of two distinct points is a line;
the affine spane of three independent points is a plane.
By placing additional
positivity constraints on the linear combinations, the definitions
extend to a large assortment of other geometric objects
such as rays, half-planes, convex hulls, and cones.  
Each of these comes in two versions: an open version
defined by strict inequality and a closed version defined
by weak inequality.  For example, the closed half-plane,
includes a bounding line and the open half-plane does
not.  In this chapter, there is no need
to attach a topological meaning to `open' and `closed',
other than being defined by strict and weak inequality.


\begin{definition}[affine]\label{def:aff} 
 If $S = \{v_1,v_2,\ldots,v_k\}$ 
and $S'=\{v_{k+1},\ldots,v_n\}$ are  finite sets, then
set
	$$\begin{array}{lll}
      \op{aff}\, S &= \{t_1 v_1 +\cdots t_k v_k \mid
	t_1 +\cdots+t_k = 1\}.\\
        \op{aff}_{\pm} (S,S') &= \{t_1 v_1 +\cdots t_n v_n \mid
	t_1 +\cdots+t_n = 1, \pm t_j \ge 0, \text{ for } j>k.\}.\\
        \op{aff}^0_{\pm} (S,S') &= \{t_1 v_1 +\cdots t_n v_n \mid
	t_1 +\cdots+t_n = 1, \pm t_j > 0, \text{ for } j>k.\}.\\
		\end{array}
        $$
\indy{Index}{aff}
\indy{Index}{affine}
\end{definition}


In the next definition $\op{conv}\,S$ agrees with the usual
notion of the convex hull of a finite set of points.
In general, $\op{conv}^0\,S$ is a smaller set that excludes
the points in $S$ itself.  


\begin{definition}[convex hull]  If $S = \{v_1,v_2,\ldots,v_n\}$ is a finite set
of points in $\ring{R}^3$, then
set
	$$
        \begin{array}{lll}
          \op{conv}\, S &= \op{aff}_+\, (\emptyset,S)\\
	   \op{conv}^0 S &= \op{aff}^0_+\, (\emptyset,S).\\
           \end{array}
        $$
\indy{Index}{conv}
\indy{Index}{convex hull}
\end{definition}

In the following definition of a cone, the point $v$ serves
as apex, and $S$ is a generating set for the positive directions.
In the special case that $S$ is a singleton $\{w\}$, 
the cone gives
a ray originating at $v$ and passing through $w$.  Later
chapters call sets of the form $\op{cone}(v,\{u_1,u_2\})$ blades.
Blades are planar sets bounded by two rays
in the plane originating at $v$.

\begin{definition}[cone]
Let $S=\{v_1,\ldots,v_n\}$ be a finite set of points in 
$\ring{R}^3$.  Let $v\in\ring{R}^3$. Set
  $$\begin{array}{lll}
  \op{cone}(v,S) &= \op{aff}_+(\{v\},S)\\
  \op{cone}^0(v,S) &= \op{aff}^0_+(\{v\},S)\\
  \end{array}
  $$
\indy{Index}{cone}
\end{definition}

%The Voronoi cell is one of the fundamental geometric objects in this book.  Earlier chapters have already discussed it at great length.  Some authors use a weak inequality in the definition, others strict.  The definition takes strict inequalities.


%\begin{definition}[Voronoi cell $\Omega$] 
%Let $S$ be a finite set of points in 
%$\ring{R}^3$.  Let $v\in\ring{R}^3$. Set
 % $$
  %\Omega(v,S) = \{x \mid \norm{v}{x} \le \norm{w}{x}, \forall w\in S\setminus\{v\}\}
  %$$
%\end{definition}

%% Changed to weak inequality May 14, 2009. -tchales.
	
\begin{definition}[line]	
Any set of the form $\op{aff}\{v,w\}$ for some $v\ne w$ is called a 
 {\it line.}
\end{definition}
\indy{Index}{line}

\begin{definition}[collinear]  A set $S$ is collinear if there exists
a line that contains every point of $S$.
\indy{Index}{collinear}
\end{definition}

\begin{definition}[plane, half plane]	
If $A=\op{aff}\{u,v,w\}$ for some set $\{u,v,w\}$ that is not collinear,
then $A$ is a {\it plane.}  Sets $\op{aff}(\{u,v\},\{w\})$, with
$\{u,v,w\}$ not collinear, are called half-planes.
\end{definition}
\indy{Index}{plane}
\indy{Index}{half-plane}

\begin{definition}[coplanar] A set $S$ is  coplanar if there exists
a plane that contains every point of $S$.
\indy{Index}{coplanar}
\end{definition}

\begin{definition}[half space] A set $\op{aff}_{\pm}(\{u,v,w\},\{v'\})$,
with $\{u,v,w,v'\}$ not coplanar, is called a half-space.  If
we replace $\op{aff}_{\pm}$ with $\op{aff}_{\pm}^0$, the set is an
{\it open half-space}.
\end{definition}
\indy{Index}{half space}
\indy{Index}{half space!open}

\section{Angle}\label{sec:angle}

Returning from a brief digression into vector geometry, we
are ready to give geometric content to trigonometric functions.
This section covers fundamental identities
in both Euclidean and spherical trig, including the law
of cosines, the law of sines, the spherical law of cosines, 
and a beautiful formula due to Euler for
the area of a spherical triangle.

If $u,v$ are given vectors with $u$ and $v$ nonzero, then by the
Cauchy-Schwarz inequality,
    $$-1 \le \frac{u\cdot v}{\normo{u}\,\normo{v}} \le 1.$$
\indy{Index}{Cauchy-Schwarz Inequality}

\begin{definition}[angle,\ arclength]\label{def:angle}
Let $u,v,w$ be vectors with $u\ne v,w$.
Define
    $$
    \arc_V(u,\{v,w\}) = \arccos\left(\frac{(u-v)\cdot (u-w)}{\norm{u}{v}\,\norm{u}{w}}\right).
    $$
It is the {\it angle} at $u$ formed by $v$ and $w$.
\indy{Index}{angle}
\indy{Index}{arc}
\indy{Index}{arclength}
\end{definition}

The name $\arc_V$ derives from its interpretation as the
length of a geodesic arc on a unit sphere
centered at $u$ from point $v$ to $w$.
\indy{Index}{arc!geodesic}
As a matter of notation, the subscript $V$ is a reminder that
the arguments to the function are vectors.  Later, the function
$\arc$, without the subscript, will give the angle as a function
of the three edge lengths of a triangle.
\indy{Index}{function!vector}

%\begin{definition}[arc length]  The arclength of a geodesic arc on a unit sphere
%centered at $v_0$ from point $v_1$ to $v_2$ is the angle formed by
%$v_1$ and $v_2$ at $v_0$.
%\end{definition}



By the relation between $\arccos$ and $\atn$
(Lemma~\ref{lemma:arccos-arctan}), %if $|u\cdot v|\ne \normo{u}\,\normo{v}$,
%then 
    \begin{equation}\label{eqn:angle}
    \arc_V(u,\{v,w\}) = \frac{\pi}2 - \atn\left ({\sqrt{(\normo{u}^2\normo{v}^2 -
    (u\cdot v)^2)}}, {u\cdot v}\right).
    \end{equation}
\indy{Index}{arclength}

\begin{definition}[arc]
When $a,b>0$, $c\ge0$,  $a + b \ge c$, $b + c \ge a$, 
and $c+a \ge b$, set 
 $$\arc(a,b,c) = \arccos(\frac{a^2 + b^2 - c^2}{2 a b}).$$
\indy{Index}{arc}
\end{definition}


Let $\ups$ (the symbol is a greek upsilon, which is written with a
wider stroke than a roman vee) be the polynomial
    $$\ups(x,y,z) = -x^2 - y^2 - z^2 + 2 x y + 2 y z + 2 z x.$$
\indy{Notation}{ZZups@$\ups$}
%% WW Repeated def (tarski.tex)
This polynomial factors
    $$\ups(a^2,b^2,c^2) = 16 s (s-a) (s-b) (s-c),$$
where $s = (a+b+c)/2$.  Heron's formula for the area of 
a triangle with
sides $a,b,c$ is $$4\sqrt{\ups(a^2,b^2,c^2)}.$$  This
book never uses Heron's formula for area, but $\ups$
appears in many other useful formulas.
\indy{Index}{Heron's Formula}
  The chapter on
Tarski arithmetic shows that the polynomial
$\ups$, like $\Delta$, 
is the square of a Cayley-Menger determinant.\footnote{
\indy{Index}{determinant!Cayley-Menger}
\indy{Index}{Cayley}
\indy{Index}{Menger}
Cayley and Menger compute the determinant $D$ of the
matrix with rows $v_1-v_0$, $\ldots,$ $v_n-v_0$, 
for vectors $v_i\in\ring{R}^n$.   Let
$x_{ij} = \norm{v_i}{v_j}^2$, entries of a matrix $[x_{ij}]$.
Write $\underbar 1$ for a row vector of length $n$ 
whose entries all equal $1\in\ring{R}$.
Then 
$$
D^2 = \frac{(-1)^{n-1}}{2^n}
    \left|\begin{matrix}[x_{ij}]& {}^t{\underbar 1}\\ {\underbar 1}& 0
        \end{matrix}\right|.
$$
The right-hand side is a polynomial in the squares of the edge lengths.
For $n=2,3$, we get $\ups$  and $\Delta$ up to a positive constant.  Since
the left-hand side is a square, these polynomials must
be non-negative, whenever the variables $x_{ij}$ can be interpreted
as the squares of edge lengths.}
\indy{Index}{edge length}
\indy{Notation}{ZZups@$\ups$}
\indy{Notation}{ZZdelta@$\delta$}


If there are vectors $v_a$, $v_b$, and $v_c$ such that $a = \norm{v_b
}{ v_c}$, $b = \norm{v_a }{ v_c}$, $c = \norm{v_a }{ v_b}$, then $0\le
\ups(a^2,b^2,c^2)$ follows from the triangle inequality applied to
each factor in the factorization of $\ups$: $2(s-a) = (b+c-a)$, and
so forth.  By the case of equality in Cauchy-Schwartz
inequality, the polynomial is strictly positive unless $v_a$,
$v_b$, and $v_c$ are collinear.
\indy{Index}{Cauchy-Schwarz Inequality}
\indy{Index}{Triangle Inequality}

By the relation between $\arccos$ and $\atn$, we have
  $$
  \arc(a,b,c) = 
    \pi/2 - \atn({\sqrt{\ups(a^2,b^2,c^2)}},{ a^2 + b^2 - c^2}).
    $$



\begin{lemma}\guid{HQTBPCM}[law of cosines]\rating{5}
Let $v_a,v_b,v_c$ be vectors with $v_a\ne v_c$, $v_b\ne v_c$.
    Let $\gamma$ be the angle formed by vectors $v_a$ and $v_b$ at $v_c$.  Let $a
    = \norm{v_b }{ v_c}$, $b = \norm{v_a }{ v_c}$, and $c = \norm{v_a }{ v_b}$.  Then
        $$c^2 = a^2 + b^2 - 2 a b \cos\gamma.$$
Equivalently, $\gamma = \arc(a,b,c)$.
%if $v_a$, $v_b$, and $v_c$ are not collinear then

\end{lemma}
\indy{Notation}{ZZgamma@$\gamma$}
\indy{Index}{arc}
\indy{Index}{law of cosines} 
\indy{Index}{trigonometric identities!law of cosines}
\indy{Index}{cosine!law of}
\begin{proved}
    $$\begin{array}{lll}c^2 &= \norm{v_b }{ v_a}^2 = ((v_b - v_c) - (v_a - v_c))\cdot ((v_b - v_c) - (v_a -
    v_c)) \\ &= a^2 + b^2 - 2 (v_b - v_c)\cdot (v_a - v_c) = a^2 +b ^2 - 2 a b
    \cos\gamma.
    \end{array}$$
    Solve this equation for $\gamma$ to get the equivalent expression. 
\swallowed\end{proved}

Although area does not matter to us, we remark
that $(a b \sin\gamma)/2$ is the area of the
triangle with vertices $v_a$, $v_b$, and $v_c$.  The following
lemma then
computes the area of a triangle.  This is Heron's formula
once again.

\begin{lemma}\guid{UKBAHKV}[law of sines]\rating{20}
Assume that $a,b\ne 0$ and $a+b\ge c$, $b+c\ge a$, and $c+a\ge b$.
Let $\gamma=arc(a,b,c)$.  Then
        $$2 a b \sin\gamma = \sqrt{\ups(a^2,b^2,c^2)}.$$
\end{lemma}
\indy{Index}{trigonometric identities!law of sines}
\indy{Index}{law of sines}
\indy{Index}{sine!law of sines}
\begin{proved}
Both sides are non-negative.  It is enough to check
that their squares are equal.  By the definition of $\arc$:
      $$4 a^2 b^2 \sin^2\gamma = 4 a^2 b^2 (1-\cos^2\gamma) = (4 a^2 b^2 - (a^2 + b^2 -
      c^2)^2) = \ups(a^2,b^2,c^2).$$
      % checked 4/4/2008
\swallowed\end{proved}




\subsection{cross product}% uses angle.

This book makes infrequent use of the cross product.
A definition and the most basic properties will suffice.

\begin{definition}[cross product]   Let $u =(x,y,z)$ and $u' = (x',y',z')$.  
Let the cross product be defined
by
    $$
    u \times u' = (y z' - y' z, z x' - x z', x y' - y x').
    $$
\indy{Index}{cross product}
\indy{Index}{vector!cross product}
\end{definition}

\begin{lemma}\guid{KVVWPNA}\rating{20}  
For any vectors $u,v$, we have 
    $$\normo{u \times v} = \normo{u}\,\normo{v}\sin\gamma,$$
where $\gamma=\arc_V(0,\{u,v\})$.
\end{lemma}

\begin{proved}
   Both sides are non-negative, so it is enough to compare the
   squares of both sides.  The square of the left-hand side is
   $$
   \begin{array}{lll}
   &(y z'- y'z)^2 + (z x' - x z')^2 + (x y' - y x')^2 \\
    &\qquad\qquad=
   (x^2 + y^2 + z^2)(x'^2 + y'^2 + z'^2) - (x x' + y y' + z z')^2
   \\&\qquad\qquad= \normo{u}^2\normo{v}^2 - (u\cdot v)^2 = \normo{u}^2\normo{v}^2 \sin^2\gamma.
   \end{array}
   $$
\swallowed\end{proved}


\begin{lemma}\guid{BKMUSOX}\rating{20}
    $$
    u\times v = -v\times u,\quad
    (u\times v)\cdot w = (v\times w)\cdot u.
    $$
\end{lemma}

\begin{proved}
These are elementary calculations.
\swallowed\end{proved}



\subsection{dihedral angle}

In a tetrahedron, the dihedral angle is
the angle formed by two faces.  In general,
the dihedral angle refers to the angle formed by two half-planes
meeting at a line.  The dihedral angle is determined
by two points $\{w_0,w_1\}$ on the line of intersection
and a further point $w_2$ and $w_3$ on each of the two half-planes.
We write the dihedral angle in the form $\dih_V(\{w_0,w_1\},\{w_2,w_3\})$ to emphasize that the dihedral angle depends only
on the line through $\{w_0,w_1\}$, never on their order.
\indy{Index}{dihedral}
\indy{Index}{angle!dihedral}
\indy{Index}{tetrahedron}
\indy{Index}{function!vector}

Similarly, the order in which we list 
the two half-planes does not affect
the angle.  The definition projects
to a plane orthogonal to the line, carrying
the half-planes to rays meeting at a point, and
the identifies the dihedral angle with the angle
between the rays.
\indy{Index}{vector!projection}
\indy{Index}{orthogonality}  

The subscript $V$ is a reminder 
that the dihedral angle takes vector arguments.
Later, a second version, without the subscript, will
compute the angle as a function of the lengths of edges of a 
tetrahedron.
\indy{Index}{edge length}


\begin{definition}[dihedral angle]\label{def:dih} Assume that $w_0\ne w_1$.
We write $\dih_V(\{w_0,w_1\},\{w_2,w_3\})$ for the angle $\gamma\in[0,\pi]$
formed
by 
    $$
    v'_a = (v_c\cdot v_c) v_a - (v_c\cdot v_a) v_c\quad\text{and }v'_b =
            (v_c\cdot v_c) v_b - (v_c\cdot v_b) v_c,
    $$
where $v_a = w_2-w_0$, $v_b=w_3-w_0$ and $v_c=w_1-w_0$.  We call it
the dihedral angle formed by $w_2$ and $w_3$ along $\{w_0,w_1\}$.
\indy{Index}{dih}
\indy{Index}{dihedral angle}
\indy{Index}{angle!dihedral}
\end{definition}

Up to positive scalars, $v'_a$ and $v'_b$ are the projections of
$v_a$ and $v_b$ to a plane orthogonal to $v_c$.  The
dihedral angle is the angle between the projections.

The dihedral angle is unchanged if $v_c$ is replaced by $s v_c$ with
$s\ne0$. The dihedral angle is unchanged if $v_a$ is replaced with
$s v_a + t v_c$ with $0 < s$ and $t$ arbitrary.  It is unchanged if
$v_b$ is replaced with $s v_b + t v_c$ with $0 < s$ and $t$
arbitrary.  In particular, the dihedral angle formed by $v_a$ and
$v_b$ along $v_c$ is the same as that formed by $v_a/\normo{v_a}$ and
$v_b/\normo{v_b}$ along $v_c/\normo{v_c}$.

The dihedral angle is degenerate and will not be used when $v_c =
0$, $v'_a = 0$, or $v'_b = 0$.




\begin{lemma}\guid{RLXWSTK}[spherical law of cosines]\label{lemma:sloc}\oldrating{100}
\rating{0}
\formalauthor{Nguyen Quang Truong}
Let $\gamma$ be the dihedral angle formed by $v_a$ and $v_b$ along
$\{v_0,v_c\}$.  Let $a$, $b$, and $c$ be the
angle between $v_b$ and $v_c$, $v_a$ and $v_c$, and $v_a$ and
$v_b$, respectively. %Assume $v_c\ne v_0$.  
Assume that $\{v_0,v_a,v_c\}$ is not
collinear. Assume that $\{v_0,v_b,v_c\}$ is not collinear.
Then
    $$\cos\gamma = \frac{\cos c - \cos a \cos b}{\sin a\sin b}.$$
\end{lemma}
\indy{Index}{cosine!spherical law of cosines}
\indy{Index}{spherical law of cosines}
\indy{Index}{trigonometric identities!spherical}

The spherical law of cosines is the most fundamental identity
of spherical trigonometry.
Although we have not yet discussed spherical
triangles, we remark that $a$, $b$, and $c$ may be interpreted as
the arclengths of the sides of a spherical triangle with
vertices $v'_a/\normo{v'_a}$, $v'_b/\normo{v'_b}$, and $v'_c/\normo{v'_c}$,
where $v'_i = (v_i-v_0)$.  Also,
$\gamma$ measures the angle of the spherical triangle opposite the
side $c$.
\indy{Index}{geometry!spherical}
\indy{Index}{triangle!spherical}


\begin{proof}  Let $v'_i = v_i - v_0$, for $i=a,b,c$.  Having
made this change, we drop the primes from the notation.
By examining the formula giving the dihedral angle, we see that the
dihedral angle is unchanged if $v_a$, $v_b$, and $v_c$ are replaced
by $v_a/\normo{v_a}$, $v_b/\normo{v_b}$, $v_c/\normo{v_c}$, respectively.  Hence
may now assume that $\normo{v_a}=\normo{v_b}=\normo{v_c}=1$.

Let $v'_a$ and $v'_b$ be the vectors in Definition~\ref{def:dih}.
By the law of cosines, we have
        $$\cos\gamma = \frac{v'_a\cdot v'_b}{\normo{v'_a}\,\normo{v'_b}}.$$
We compute the length of $v'_a$ (keeping in mind that $v_a$, $v_b$,
and $v_c$ all have unit length):
        $$
        \normo{v'_a}^2 = v'_a\cdot v'_a =
        (v_a - (v_c\cdot v_a)v_c)\cdot (v_a - (v_c\cdot v_a) v_c) =
        1 - (v_c\cdot v_a)^2 = \sin^2 b.
        $$
So $\normo{v'_a} =\sin b$. Similarly, $\normo{v'_b} = \sin a$.  This gives
the denominator in the spherical law of cosines. To compute the
numerator, we expand the dot product:
    $$
    v'_a\cdot v'_b = (v_a - (v_c\cdot v_a) v_c)\cdot (v_b - (v_c\cdot v_b) v_c)
    = (v_a\cdot v_b) - (v_c\cdot v_a) (v_c\cdot v_b) = \cos c - \cos
    a \cos b.
    $$
The identity follows.
\end{proof}

The spherical law of cosines gives the angles of a spherical
triangle as a function of its sides.  In spherical geometry,
there is a duality between angles and sides of a triangle.
This book does not go into details about this duality.%
\footnote{In three dimensionsal Eulidean space, the orthogonal
complement of a plane through the origin is a line through
the origin.  This gives a duality between planes and lines
through the origin.  The intersection of each plane and line with
a unit sphere at the origin yields a duality between great
circles and antipodal
pairs of points (the poles of the great circle).  The three edges
of a spherical triangle $ABC$ lie on three great circles which
determine three
antipodal pairs of points.  From each of the three antipodal pair of points, 
a coherent
choice can be made of one of the two poles (choose the pole
closer to the opposite vertex of $ABC$).  
These three poles are the vertices
of the dual triangle $A'B'C'$.  Each statement about the triangle $ABC$
can be dualized to a statement about $A'B'C'$.
In particular, the edges $a,b,c$ and angles $\alpha,\beta,\gamma$ of $ABC$ 
are related to those $a',b',\ldots$ of $A'B'C'$ by
   $$
   a + \alpha' = \pi,\quad a' + \alpha= \pi,
   $$
and so forth.}  
One
consequence of this duality is a formula for the edges of
a triangle as a function of its edges.  Up to signs,
it has the same formula as the law of cosines.
\indy{Index}{great circle}

\begin{lemma}\guid{NLVWBBW}[spherical law of cosines - second form]\oldrating{80}
\rating{0}
\formalauthor{Nguyen Quang Truong}
Let $v_a,v_b,v_c$ be three points in $\ring{R}^3$.
Let $\alpha,\beta,\gamma$ be the dihedral angles: 
   $$
   \begin{array}{lll}
     \alpha &= \dih_V(\{v_0,v_a\},\{v_b,v_c\})\\
     \beta &= \dih_V(\{v_0,v_b\},\{v_a,v_c\})\\
     \gamma&= \dih_V(\{v_0,v_c\},\{v_b,v_a\})\\
     \end{array}
   $$
Let $c$ be the
angle between $v_a$ and $v_b$ at $v_0$. 
Assume that $\{v_0,v_a,v_c\}$ is not
collinear. Assume that $\{v_0,v_b,v_c\}$ is not collinear.
Then
    $$
    \cos c = \frac{\cos \gamma + \cos \alpha \cos \beta}
     {\sin \alpha\sin \beta}.
    $$
\end{lemma}
\indy{Index}{cosine!spherical law of cosines}
\indy{Index}{spherical law of cosines}

\begin{proof}  
Here is a direct
computational proof that avoids duality.
Let $a$ be the angle between $v_b$ and $v_c$, and let $b$ be the angle
between $v_a$ and $v_c$ at $v_0$.
Let $A=\cos a$, $B=\cos b$, $C=\cos c$,
$A'=\sin a$, $B'=\sin b$, $C'=\sin c$.  By the
spherical law of cosines, we have
   $$\sin^2\beta = 1-\left(\frac{B-A C}{A' C'}\right)^2
     = \frac{p}{A'^2 C'^2},$$
where $p=1-A^2 - B^2 - C^2 + 2 A B C$.
In particular, $p\ge 0$.
Computing $\sin^2\alpha$ and the remaining terms in the same way, we get
   $$
   \begin{array}{lll}
     \sin\alpha\sin\beta &= \frac{p}{A' B' C'^2}\\ 
      \\
     \cos\gamma + \cos\alpha \cos\beta &=
         \frac{C - A B}{A' B'} + \frac{A - B C}{B' C'} \frac{B - A C}{A' C'}
         = \frac{p C}{A' B' C'^2}.\\
    % C &= \frac{\cos\gamma + \cos\alpha \cos\beta}{\sin\alpha \sin\beta}\\
   \end{array}
   $$
The result follows.
\end{proof}

The following lemma gives a formula for the dihedral angle
of a tetrahedron in terms of its edge lengths.  The
familiar polynomials $\ups$ and $\Delta$ appear once again.
\indy{Index}{ZZups@$\ups$}
\indy{Index}{ZZdelta@$\delta$}


\begin{lemma}\guid{OJEKOJF} \label{lemma:dihform}\oldrating{80}
\rating{0}
\formalauthor{Nguyen Quang Truong}
Let $v_0,v_1,v_2,v_3$ 
be vectors with $\{v_0,v_1,v_2\}$ not collinear, 
and $\{v_0,v_1,v_3\}$ not
collinear. 
Let $\gamma$ be the dihedral angle formed
by $v_2$ and $v_3$ along $\{v_0,v_1\}$. Let
    $$(x_1,\ldots,x_6) = 
    (x_{01},x_{02},x_{03},x_{23},x_{13},x_{12}),
    \text{ where } x_{ij}=\norm{v_i}{v_j}^2.$$
Let $\Delta_4$ be the partial derivative of $\Delta(x_1,\ldots,x_6)$ with
respect to $x_4$.
The dihedral angle $\gamma=\dih_V(\{v_0,v_1\},\{v_2,v_3\}$
is given by
    $$
    \gamma=\arccos(\frac{\Delta_4(x_1,\ldots,x_6)}{\sqrt{
    \ups(x_1,x_2,x_6)\ups(x_1,x_3,x_5)}}).
    $$
%Assuming that $\gamma\ne 0,\pi$, 
It is also given by
    $$
    \gamma=\frac{\pi}{2} - \atn
     ({\sqrt{4 x_1 \Delta(x_1,\ldots,x_6)}},{\Delta_4(x_1,\ldots,x_6)}).
    $$
\end{lemma}
%% pi/ 2. -  arctan(  deltax4/ (sqrt (4. * x1 * delta)))
\indy{Index}{angle!dihedral}
\indy{Index}{dihedral}
\indy{Index}{dihform}


\begin{proof}
Let $\beta = \arc_V(v_0,\{v_1,v_2\})$.
We have $v'_2\ne 0$, and $v'_3 \ne 0$ (where $v_2'$ and
$v_3'$ are the projections computed as in Definition~\ref{def:dih}).  
    By expanding definitions and dot products
    $$
    v'_2\cdot v'_2 = (v_1\cdot v_1) ((v_1\cdot v_1)(v_2\cdot v_2) -
    (v_1\cdot v_2)^2) =  x_1^2 x_2 \sin^2 \beta = \frac{1}{4}
    x_1
    \ups(x_1,x_2,x_6).
    $$
    Similarly,
    $$v'_3 \cdot v'_3 = \frac{1}{4} x_1 \ups(x_1,x_3,x_5).$$

Let $y_i = \sqrt{x_i}$. Then
    $$\begin{array}{lll}
    v'_2\cdot v'_3 &= (v_1\cdot v_1)((v_1\cdot v_1)(v_2\cdot v_3) -
    (v_1\cdot v_2)(v_1\cdot v_3) ) \vspace{6pt} \\  &
    = x_1 \left(\frac{ x_1 (x_2 + x_3 -
    x_1)}{2} - \frac{(x_1 + x_2 - x_6)(x_1 + x_3 -
    x_5)}{4} \right)\vspace{6pt}\\&
    %= \frac{x_1}{4} (2 x_1 (x_2+x_3-x_4) -
    %(x_1+x_2-x_6)(x_1+x_3-x_5)) \vspace{6pt}\\&
    = {x_1\Delta_4(x_1,\ldots,x_6)}/{4}.
    \end{array}
    $$
The result follows in terms of $\arccos$.

To translate to the $\atn$, 
we use the $\arccos$-$\atn$ identity
and the following polynomial identity
    $$
    \frac{16}{x_1^2}(\normo{v'_2}^2 \normo{v'_3}^2 - (v'_2\cdot v'_3)^2) =
    \ups(x_1,x_2,x_6)\ups(x_1,x_3,x_5) - \Delta_4(x_1,\ldots,x_6)^2
    = 4 x_1 \Delta(x_1,\ldots,x_6).
    $$
\end{proof}






\subsection{Euler angle sum}

The expression $\alpha_1+\alpha_2+\alpha_3-\pi$ is Girard's
formula (known first to T. Harriot) 
\indy{Index}{Girard's formula}
\indy{Index}{geometry!spherical}
\indy{Index}{triangle!spherical}
for the area of a spherical triangle with angles
$\alpha_1$, $\alpha_2$, $\alpha_3$.  The following lemma
gives a slick formula for the area, discovered by
Euler and Lagrange.
\indy{Index}{Girard}
% Albert Girard's Book on trigonometry was published in 1626. Harriot lived 1560 - 1621.

\begin{lemma}\guid{JLPSDHF}[Euler triangle]\label{lemma:euler}\rating{600}
Let $v_0,v_1,v_2,v_3$ be points in $\ring{R}^3$. 
Let 
  $$(y_1,\ldots,y_6) =(y_{01},y_{02},y_{03},y_{23},y_{13},y_{12}),
   \text{ where } y_{ij}=\norm{v_i}{v_j}.$$
Set
$x_i = y_i^2$.   
and
    $$
    p = y_1 y_2 y_3 + y_1 (w_2\cdot w_3) + y_2 (w_1\cdot w_3) + y_3
    (w_1\cdot w_2).
    $$
where $w_i = v_i- v_0$.
Let $$\alpha_i =\dih_V(\{v_0,v_i\},\{v_j,v_k\})$$
where $\{i,j,k\}=\{1,2,3\}$.
Assume that $\Delta(x_1,\ldots,x_6)>0$. 
Then
    $$
    \alpha_1+\alpha_2+\alpha_3 - \pi
     = {\pi} - 2\atn({\Delta(x_1,\ldots,x_6)^{1/2}},{2 p}),
    $$
for three given dihedral angles $\alpha_i$.
\end{lemma}
\indy{Index}{Euler}
\indy{Index}{triangle!Euler}

\begin{proof}
%% I checked all the details of this proof in 
%% Math'ca on May12,2007
The angles are unchanged if the vectors $w_i$ are rescaled so that
$\normo{w_i}=1$.  The given formula is also unchanged under rescalings:
the factor $a$ is homogeneous of degree $3$ under a change $w_i
\mapsto t w_i$ for $t>0$, and so is $\sqrt{\Delta}$ by the
formula for $\Delta$.  Thus, we may assume that $\normo{w_i}=1$, for
$i=1,2,3$. We have $y_1=y_2=y_3=1$.  It is convenient to use
different notation $a=x_4$, $b=x_5$, $c=x_6$ for the other
variables. Expanding the dot products in $p$ by the law of cosines
we have
    $$2 p = 8 - (a+b+c).$$

We have $$\Delta(x_1,\ldots,x_6) = \Delta(1,1,1,a,b,c) =
    \ups(a,b,c) - a b c.$$
Since $0 <\Delta$, the arctangent formula
in Lemma~\ref{lemma:dihform} 
applies for the dihedral angles $\alpha_i$. Making
this substitution and clearing the $3$ from the denominator, the
desired identity now takes the form $f(a,b,c)=0$, where
    $$
    f(a,b,c)= -\pi/2 - \sum_{i=1}^3\arctan(u_i/\sqrt{\Delta}) +
    2\arctan(2 p/\sqrt{\Delta}),
    $$
for some rational functions $u_i$ of $a,b,c$.  We prove that this
trig identity holds whenever $\Delta>0$.

Fix $b,c$ and differentiate $f$ with respect
to $a$.  The partial derivative $\partial f/\partial a$ has the form
$g(a,b,c)/\sqrt{\Delta}$ for some rational function of $a,b,c$ (with
nonzero denominator).  By brute force, we see that $g(a,b,c)$ is
identically $0$.  (Euler himself did not shun brute force.  See
\cite{Euler}.)

By symmetry, the partial derivatives with respect to $b$ and $c$ are
also identically zero.  The function $f(a,b,c)$  is continuously
differentiable whenever $\Delta>0$.  Thus, its directional
derivatives are zero.  Thus, $f(a,b,c)$ is constant on connected
sets of the domain of $\Delta>0$.  Without loss of generality,
assume that $a\le b\le c$.  We have, for $a\le t\le b$ that
$d\Delta(1,1,1,t,b,c)/dt\ge 0$. Hence $f(a,b,c)=f(b,b,c)$. 
Similarly,
 $f(b,b,c)=f(c,c,c)$. 
To complete the lemma, we evaluate the constant $f(c,c,c)$
by taking $c$ small.
$\Delta=3c^2-c^3$, $2p= 8-3c$,  $u_1=u_2=u_3 = c -c^2/2$. With this,
we compute
    $$f(a,b,c)= f(c,c,c) = \lim_{c\mapsto0} f(c,c,c) = 
    3 \arctan(1/\sqrt3)-\pi/2 =0.$$
\end{proof}
\indy{Notation}{ZZdelta@$\Delta$}






\subsection{spherical triangle inequality}
\indy{Index}{triangle!spherical}
\indy{Index}{spherical triangle inequality}

The geodesic length between two points
$u,v$ on a unit sphere centered at $p$ is $\arc_V(p,\{u,v\})$.
The following lemma is part of the verification that
the function $d(u,v) = \arc_V(p,\{u,v\})$ is a metric
on the unit sphere.  We exclude the degenerate case when
points on the sphere are antipodal.

\begin{lemma}\guid{KEITDWB}\label{lemma:sph-tri-ineq}
\oldrating{80}
\rating{0}
\formalauthor{Nguyen Quang Truong}
Let $\{p,u,v,w\}$ be a set of four points in $\ring{R}^3$.
Assume that $p$ is not collinear with any of the other two points.
Then
   $$
   \arc_V(p,\{u,w\}) \le \arc_V(p,\{u,v\}) + \arc_V(p,\{v,w\}).
   $$
Equality occurs if and only if $v\in\op{aff}_+(p,\{u,w\})$.
\end{lemma}

\begin{proof} Let $v'$ be the projection of $v$ to the plane
$\op{aff}\{p,u,w\}$.  
By the spherical law of cosines, for a right angle
   $$
   \cos\psi = \cos\beta\cos\alpha \le \cos\beta,
   $$
where $\psi = \arc_V(p,\{u,v\})$, $\beta = \arc_V(p,\{u,v'\})$, $\alpha=\arc_V(p,\{v,v'\})$.
Thus, $\arc_V(p,\{u,v'\})=\beta\le \psi=\arc_V(p,\{u,v\})$.
Similarly, $\arc_V(p,\{v',w\}) \le \arc_V(p,\{v,w\})$.
Thus, it is enough to show that 
  $$
  \arc_V(p,\{u,w\}) \le \arc_V(p,\{u,v'\}) + \arc_V(p,\{v',w\}).
  $$
The points $p,u,w,v'$ are coplanar.
By the additivity of planar angle (Lemma~\ref{lemma:polar-sum}), if 
$v'\in \op{aff}_+(p,\{u,w\})$, then
   $$
   \arc_V(p,\{u,w\}) = \arc_V(p,\{u,v'\}) + \arc_V(p,\{v',w\}),   
   $$
and otherwise,
   $$
   \arc_V(p,\{u,w\}) = \norm{\arc_V(p,\{u,v'\}) }{ \arc_V(p,\{v',w\})}.
   $$
The inequality follows.

Tracing this argument, we see that equality occurs exactly when
$\alpha=0$ and $v'\in \op{aff}_+(p,\{u,w\})$.  Equivalently,
$v'=v\in\op{aff}_+(p,\{u,w\})$.
\end{proof}

\begin{lemma}\guid{FGNMPAV}\oldrating{40}
\rating{0}
\formalauthor{Nguyen Quang Truong}
\label{lemma:sph-tri-multi}
Let $\{p,u_0,u_1,u_2,\ldots,u_r\}$ be a set of points in $\ring{R}^3$.
Assume that no triple $\{p,u_i,u_{i+1}\}$ is collinear.  Assume
that $\{p,u_0,u_r\}$ is not collinear.  Then
$$
  \arc_V(p,\{u_0,u_r\}) \le \sum_{i=0}^{r-1} \arc_V(p,\{u_i,u_{i+1}\}).
$$
\end{lemma}

\begin{proof} This is an easy induction on $r$ with base
case given by Lemma~\ref{lemma:sph-tri-ineq}.
\end{proof}

\subsection{Lexell's theorem}

\begin{lemma}\guid{UWIPRDV}[Lexell]\rating{500}
% was 1000 with old proof including lemma ZHH
Fix two points on a unit sphere $v_1,v_2$ that
are not antipodal.
Let $u,u'$ be an two other points the sphere in the same open hemisphere determined by the great circle through $v_1,v_2$.  Then the two spherical triangles $\{v_1,v_2,u\}$ and $\{v_1,v_2,u'\}$ have the same area if and only if
the four points $u$, $u'$, $v^*_1$, $v^*_2$ are concircular, where $v^*_i$ is the points antipodal to $v_i$.
\end{lemma}
\indy{Index}{Lexell}
\indy{Index}{Lexell's Theorem}

\begin{proof}  We pick coordinates so that the Lexell circle (through $u,v^*_1,v^*_2$) has constant zenith angle $\phi$.  Without loss of generality, we can find a coordinate system such that 
$$
\begin{array}{lll}
v_1 &= \{\cos\theta\sin\phi,+\sin\theta\sin\phi,-\cos\phi\}\\
v_2 &= \{\cos\theta\sin\phi,-\sin\theta\sin\phi,-\cos\phi\}\\
u &= \{\cos\alpha\sin\phi,\sin\alpha\sin\phi,\cos\phi\}\\
\end{array}
$$
By Euler's formula for solid angle (Lemma~\ref{lemma:euler}), the locus of points along which the solid angle is constant is given by
\begin{equation}\label{eqn:Deltap}
p/\sqrt{\Delta} = \hbox{constant}.
\end{equation}
We have $\Delta>0$ when the points $0,v_1,v_2,u$ are not coplanar.
If we set $a = v_1\cdot v_2$, $b = u\cdot v_1$, $c= u\cdot v_2$, we have
$$
\Delta(1,1,1,2-2a,2-2b,2-2c) = 4 (1-a^2-b^2-c^2+2 a b c)
$$ 
and
$p = 1+a+b+c$.    In terms of $\phi,\alpha,\theta$, we calculate
$$
\sqrt{\Delta}/p = 2\cos\phi \tan\theta.
$$
This is independent of $\alpha$, proving that every point on the Lexell circle (except for the degenerate points $u= -v_1,u=-v_2$ with $\Delta=0$) gives the same solid angle.

To check that points on different Lexell circles give different solid angles, we may pick any convenient point on the circle.  For example, there is an isosceles triangle $b=c$.  Under this specialization, the left-hand-side of equation (\ref{eqn:Deltap}) has derivative
$$
\Delta^{3/2} \frac{\partial (p/\sqrt{\Delta})}{\partial c} = 8 (1-a^2)(1+c) > 0.
$$
As its derivative is positive, the function is increasing.  Different Lexell circles give different values.
\end{proof}
\indy{Index}{Euler}
\indy{Notation}{ZZdelta@$\Delta$}

\begin{proof}  By the duality of triangles mentioned above, it is enough to prove the dual statement, which follows.  By Girard's formula, two triangles have the same area if and only if the angle sums on the two triangles are equal.
Under duality, we consider triangles with the same perimeter.
\end{proof}
\indy{Index}{Girard}
\indy{Index}{Girard's formula}

\begin{lemma}\guid{ZHHSGTF}\rating{0}  Fix one point $v$ on the unit sphere, with antipodal point $v^*$.  Consider two half-great-circles $R_1$ and $R_2$ between $v$ and $v^*$ that are not coplanar.  Two great circles $G$ and $G'$ cut triangles with vertex $v$ along $R_1$ and $R_2$, having the same perimeter if and only if the great circles $R_1$, $R_2$, $G$, and $G'$ are tangent to a common circle $C$.
\end{lemma}
\indy{Index}{great circle}

\begin{proof} Note that the two tangents to a circle through a given point have the same length.  If $C$ exists, then this fact implies that a great circle $G$ that is tangent to $C$ cuts a triangle with vertex $v$ along $R_1$ and $R_2$ whose perimeter is equal to the sum of the distances from $v$ to the two points of tangency $C\cap R_1$ and $C\cap R_2$.  This is independent of $G$.

Conversely, for $G$ any great circle there is a unique $C$ that is inscribing circle of the great circles $R_1$, $R_2$, and $G$.  The perimeter of the triangle is the sum of the distances from $v$ to the points $C\cap R_1$ and $C\cap R_2$.
If a second $G$ gives a triangle with the same perimeter, its circle $C'$
must satisfy $C'\cap R_i = C\cap R_i$.  This forces $C=C'$.
\end{proof}

\subsection{regular polygons}
\indy{Index}{polygon}
\indy{Index}{polygon!regular}

\begin{lemma}\guid{GOTCJAH}\rating{300}\label{lemma:ngon}
Let $C$ be a circle on the unit sphere with arcradius $a<\pi/2$.  Among all spherical $n$-gons that
contain $C$  (that is among all $n$-fold intersection of hemispheres containing $C$), that of smallest area is the regular $n$-gon.  
\end{lemma}

In other words, the extremal case consists of the intersection of $n$-hemispheres whose bounding great circles are tangent to the circle $C$ at $n$-equally spaced bounds around $C$.

\begin{proof} 
Consider a spherical triangle with sides $a,b,c$ and opposite angles $\alpha,\beta,\gamma$.  If $\gamma=\pi/2$, then by Girard's formula, the area
of the triangle is
$$
\alpha+\beta-\pi/2,
$$
and by the law of cosines 
$$
\cos(\alpha) =\sin(\beta)\cos(a).
$$
This determines the area $A(a,\beta)$ of the triangle 
as a function of $a$ and $\beta$.
\indy{Index}{Girard}
\indy{Index}{Girard's formula}

In the smallest intersection of hemispheres,  every bounding great circle will be tangent to $C$.
The area of the regular $n$-gon circumscribing $C$ is
$$
\sum_{i=1}^n A(a,\beta_i) 2,
$$
with angle sum
$$
\sum_{i=1}^n \beta_i = 2\pi.
$$
Fixing $a$, we compute the second partial of $A$ with respect to $\beta$:
$$
\frac{\partial^2 A(a,\beta)}{\partial \beta^2} = \frac{\cos(a)\sin^2(a)\sin(\beta)}{\sin^2(\alpha)} > 0.
$$
The function is convex.
By convexity, the minimum area occurs when all angles are equal
$\beta_i = \pi/n$.
\end{proof}
\indy{Index}{convex}

\begin{lemma}\guid{BBEVFIC}\rating{100}\label{lemma:ngon-area}
The minimum area of an intersection of $n$-hemispheres containing a circle $C$ 
of arcradius $a<\pi/2$ is
$$
2\pi - 2 n (\arcsin(\cos(a)\sin(\beta)),
$$
where $\beta = \pi/n$.
\end{lemma}

\begin{proof} This is equal to the area formula
$$
2 n A(a,\beta)
$$
in the previous lemma.
Alternatively, the polygon breaks into $2n$ triangles, each computed by Girard's
formula to have area
$$
\beta - (\pi/2 - \alpha)  = \pi/n - \arcsin(\cos(\alpha)) = 
\pi/n - \arcsin(\cos(a)\sin(\beta)).
$$
\end{proof}


\indy{Index}{Girard}
\indy{Index}{Girard's formula}
\indy{Index}{polygon}


\section{Coordinate Systems}

This final section of the chapter establishes the existence
and basic properties of the standard coordinate systems
(polar coordinates, spherical coordinates, and cylindrical
coordinates).  The azimuth (or longitudinal) angle of the
spherical coordinate system determines a cyclic permutation,
called the azimuth cycle,
on a finite set of points in $\ring{R}^3$, ordered according
to increasing angle.  This section also describes the basic
properties of that permutation.  This section confines its
scope to two and three dimensions.
\indy{Index}{azimuth}
\indy{Index}{angle!azimuth}
\indy{Index}{azimuth!azimuth cycle}
\indy{Index}{cyclic permutation}
\indy{Index}{angle}
\indy{Index}{coordinate systems}

\subsection{polar coordinates}
\label{sec:polar}
\indy{Index}{polar coordinates}
\indy{Index}{coordinate systems!polar coordinates}


For every pair of real numbers $x$ and $y$,  there are real numbers
$r$ and $\theta$ such that
    \begin{equation}\label{eqn:polar}
    x = r\cos\theta,\quad y = r\sin\theta.
    \end{equation}
If $x$ and $y$ are both zero, then we take $r=0$, and the
Equations~\ref{eqn:polar} hold for all choices of $\theta$. If $x$
and $y$ are not both zero, then we can take $0<r$, and $\theta$ is
uniquely determined (up to multiples of $2\pi$).  We can choose
$0\le\theta < 2\pi$.

Let $W=\{p_1,\ldots,p_k\}$ be a finite set of
nonzero points in the plane, with
polar coordinates $p_i = (r_i\cos\theta_i,r_i\sin\theta_i)$.
We wish to order the set of points according to increasing angle.
To deal with degenerate cases when some points have exactly
the same angle,
we order the points with the lexicographic order on their
polar coordinates.  Write $p_i \prec p_j$ if
$\theta_i < \theta_j$ or ($\theta_i=\theta_j$ and $r_i<r_j$).
This is a total order on the points.  (We will not ever
actually need the degenerate cases when two angles are equal,
but by defining a total order, we never need to think about
it again.)
\indy{Index}{total order}
We have a cyclic permutation $\sigma:W\to W$ which sends
$p\in W$ to the next larger element with respect to this order,
or back to the first element if $p$ is the largest.
We will call $\sigma$ the {\it polar cycle}
of the set $W$.
\indy{Index}{lexicographic order}
\indy{Index}{polar cycle}
\indy{Index}{permutation}
\indy{Index}{cyclic permutation}




For $\psi\in\ring{R}$, let $T:\ring{R}^2\to\ring{R}^2$ be the
rotation of the plane:
   \begin{equation}
   \label{eqn:rotate}
   (x,y) \mapsto  (x\cos\psi + y\sin\psi,-x\sin\psi+y\cos\psi).
   \indy{Index}{rotation}
   \end{equation}
Let $\sigma'$ be the polar cycle for $T(W)$.  Then it is easily
checked that
$$
   \sigma'(T p) = T (\sigma p),\quad \text{ for } p\in W. 
$$

\begin{lemma}\guid{PDPFQUK}\label{lemma:polar2}
\oldrating{50}
\rating{0}
\formalauthor{Nguyen Quang Truong}
\formal{thetaij\_t}
Let $\theta_i$ be real numbers such that $0\le \theta_i < 2\pi$, for $i=1,2$.
Let $$\theta_{ji} = \theta_i - \theta_j + 2\pi k_{ji},$$
where we pick integers $k_{ij}$ so that $0\le \theta_{ji}< 2\pi$.
Then 
$$
  \theta_{12} + \theta_{21} = \begin{cases}
    2\pi, & \text{ if }\theta_i\ne\theta_j\\
    0,    & \text{ if }\theta_i=\theta_j.
    \end{cases}
$$
\end{lemma}

\begin{proof} This is elementary.
\end{proof}

The next lemma gives a precise form to the observation
that given finitely many rays eminating from the origin
in the plane, the sum of the included angles is $2\pi$.
To state it precisely, we use the polar cycle to place
a cyclic order on the rays.  There is a degenerate case,
when there is at most one ray.


\begin{lemma}\guid{ISRTTNZ}\label{lemma:polar-sum}\rating{100}\formal{thetapq\_wind\_t}
Let $W\subset\ring{R}^2$ be a finite set,
of cardinality $n$. Suppose each point in $W$ is nonzero.
Let $\sigma$ be the polar cycle on $W$.  
Let $p=(r(p)\cos\theta(p),r(p)\sin\theta(p))$, for $p\in W$, with
$0\le\theta(p)<2\pi$.
Write
   $$
   \theta(p,q) = \theta(q) - \theta(p) + 2\pi k_{pq},
   $$
where we choose integers $k_{pq}$ so that $0\le \theta(p,q) < 2\pi$.
Then have for all $p\in W$,
and all $0\le i \le j < n$,
   $$
   \theta(p,\sigma^i(p)) +\theta(\sigma^i(p),\sigma^j(p)) =
   \theta(p,\sigma^j(p)).
   $$
Moreover, if there exists $p,q\in W$ such that $\theta(p)\ne\theta(q)$,
we have 
  $$
  \sum_{i=0}^{n-1} \theta(\sigma^{i}p,\sigma^{i+1} p) = 2\pi.
  $$
(If $\theta(p)=\theta(q)$ for all $p,q\in W$, then all the
summands are zero.)
\end{lemma}

\begin{proof}
Fix $p\in W$.
For $0\le i<n$, define $\theta_i$ by
   $\theta_0=\theta(p)$ and 
   $$\theta_i = \theta(\sigma^i(p)) + 2\pi \ell_i,$$
where we choose $\ell_i$ so that $\theta_0\le \theta_i < \theta_0+2\pi$.
It follows from the definition of the polar cycle that
$\theta_i \le \theta_j$ for $0\le i\le j < n$. We find that
$\theta(\sigma^i p ,\sigma^j p) = \theta_j - \theta_i$.
The first identity becomes
  $$
  (\theta_i-\theta_0) + (\theta_j-\theta_i) = (\theta_j-\theta_0),
  $$
which is certainly true.
The second identity becomes
  $$
  \sum_{i=0}^{n-2} (\theta_{i+1}-\theta_i) + \theta(\sigma^{n-1}p,p)
  = \theta(p,\sigma^{n-1}p) + \theta(\sigma^{n-1}p,p).
  $$
This is $0$ or $2\pi$, by the previous lemma.
\end{proof}

\subsection{spherical coordinates}
\label{sec:spherical}



\begin{definition}[spherical coordinates]
Let $x,y,z$ be any real numbers.  A
triple $(r,\theta,\phi)$ such that
    \begin{equation}
    \label{eqn:spherical}
    x = r\cos\theta\sin\phi,\quad y = r\sin\theta\sin\phi,\quad
    z = r\cos\phi
    \end{equation}
with $0\le r$, $0\le\theta<2\pi$, and $0\le\phi\le\pi$ are called
spherical coordinates of $(x,y,z)$.  (We follow the variable
naming conventions of American calculus textbooks, which differ
from the international scientific notation.)
\end{definition}
\indy{Index}{spherical coordinates}
\indy{Index}{coordinate systems!spherical coordinates}

Spherical coordinates of any $(x,y,z)$ exist.
We have $r = \sqrt{x^2+y^2+z^2}$.  In the degenerate case $r=0$,
the Equations~\ref{eqn:spherical} become independent of $\theta$
and $\phi$. In the degenerate case when $\phi = 0$ or $\phi =
\pi$, the equations become independent of $\theta$. If $0<r =
\sqrt{x^2+y^2+z^2}$, and $\phi\ne 0,\pi$,  then $\phi$ is uniquely
determined by $x,y,z$. Also, $\theta$ is uniquely determined.


%\begin{definition}[azimuth]\label{def:azimuth}
We call $\theta$ the {\it azimuth angle\/} and $\phi$ the {\it
zenith angle\/} of $(x,y,z)$.  The azimuth angle is also known as
the longitude.  The zenith angle is also known as the latitude. The
azimuth angle is a polar coordinate of $(x,y)$:
    $$
    (x,y) = (r'\cos\theta,r'\sin\theta), \quad r' = r\sin\phi.
    $$
\indy{Index}{azimuth}
\indy{Index}{azim}
\indy{Index}{zenith}
\indy{Index}{latitude}
\indy{Index}{longitude}
\indy{Index}{angle!azimuth}
\indy{Index}{angle!zenith}


\subsection{general coordinates}

\indy{Index}{coordinate systems}
We generalize zenith angle, azimuth angle,
spherical coordinates, and cylindrical coordinates 
to a general orthonormal frame.

We begin with the zenith angle in a general orthogonal frame.
The following lemma identifies the zenith angle $\phi$ with respect to
a general orthonormal coordinate frame with $e_3$ as the third
vector in that frame.  As it is easily expressed in terms
of the more basic function $\arc_V$, 
we will have little need to refer directly to the zenith angle.
\indy{Index}{frame}
\indy{Index}{orthogonal frame}

\begin{lemma}\guid{QAFHJNM}[zenith]
\oldrating{50}
\rating{0}
\formalauthor{Nguyen Quang Truong}
Let $(v,w)$ be an ordered pair of points in $\ring{R}^3$, with $v\ne w$.
Let $u\ne v$.  Set $\phi = \arc_V(v,\{u,w\})\in[0,\pi]$.
Let $e_3$ be the unit vector $(w-v)/\norm{w}{v}$.  Let $r = \norm{u}{v}$.
Then $u$
can be expressed in the form
   $$
   u = v + u' +
   r\cos\phi\, e_3,
   $$
where $u'\cdot e_3 = 0$.
\indy{Index}{zenith}
\indy{Index}{orthonormal}
\indy{Index}{angle!zenith}
\end{lemma}

\begin{proof} This follows directly from the definition of $\arc_V$:
  $$(u-v)\cdot e_3 = r\cos\phi.$$
\end{proof}



\begin{lemma}\guid{EYFCXPP}[cylindrical coordinates]\oldrating{80}
\rating{0}
\formalauthor{Nguyen Quang Truong}
Let $v$ and $w$ be distinct points in 
$\ring{R}^3$.  Let $(e_1,e_2,e_3)$ be positively oriented\footnote{That is, they are mutually orthogonal unit vectors such that $e_3 = e_1 \times e_2$.} orthonormal
vectors such that $e_3 = (w-v)/\norm{w}{v}$.
Then every
$u\in\ring{R}^3$ that is not in the line $\op{aff}(v,w)$
can be uniquely expressed in the form
   $$
   u = v + r\cos\psi\, e_1 + r\sin\psi\, e_2 + h (w-v),
   $$
for some $0< r$, $0\le \psi < 2\pi$, $h\in\ring{R}$.
Furthermore,
assume that $w_1$ and $w_2$ do
not lie in the line $\op{aff}(v,w)$.
Then there exist unique $\psi,\theta,r_1,r_2,h_1,h_2$
 such
that $0\le\psi<2\pi$, $0\le\theta < 2\pi$, $0 < r_1$, $0 < r_2$, and
  $$
  \begin{array}{lll}
    w_1 &= v + r_1\cos\psi\, e_1 + r_1\sin\psi\, e_2 + h_1(w-v),\\
    w_2 &= v + r_2\cos(\psi+\theta)\, e_1 + r_2\sin(\psi+\theta)\, e_2 
     + h_2(w-v),\\
\end{array}
  $$
Finally, the angle $\theta$ is independent of the choice of $e_1,e_2$
satisfying the given properties.
\end{lemma}
\indy{Index}{coordinate systems!cylindrical coordinates}
\indy{Index}{cylindrical coordinates}

\begin{definition}[azim] 
We define $\op{azim}(v,w,w_1,w_2)$, the azimuth angle, 
to be the uniquely determined
angle $\theta$, where $\theta$ is the angle given by the previous
lemma.
\indy{Index}{azim}
\indy{Index}{azimuth}
\indy{Index}{angle!azimuth}
\end{definition}

\begin{lemma}\guid{XPHCPNY}[spherical coordinates]\label{lemma:sph}\rating{60}\formal{spherical\_coord\_t}
Let $\{v,w,u\}$ be a set of three points in $\ring{R}^3$.
Assume that the set is not collinear.  
Construct orthonormal vectors  $e_1,e_2,e_3$ by setting
$e_3 = (v-w)/\norm{v}{w}$; 
$e_1\in\op{aff}_+^0(\{v,w\},u)$ orthonormal with $e_3$;
$e_2 = e_3\times e_1$.
Let $u'$ be a point that does not
lie in the line $\op{aff}\{v,w\}$.
Set $r = \norm{v }{ u'}$. Let
$\phi$ be the zenith angle of $u'$ with respect ot $(v,w)$, and let
$\theta$ be the azimuth angle $\op{azim}(v,w,u,u')$.  Then
   \begin{equation}
   u' = v + r \cos\theta \sin\phi\, e_1 + r \sin\theta\sin\phi\, e_2 +
   r\cos\phi\,e_3.
   \label{eqn:sph}
   \end{equation}
\end{lemma}
\indy{Index}{coordinate systems}
\indy{Index}{coordinate systems!spherical coordinates}
\indy{Index}{angle!zenith}
\indy{Index}{angle!azimuth}
\indy{Index}{spherical coordinates}

\begin{definition}[frame]\label{def:sph}
Equation~\ref{eqn:sph} is called the spherical coordinate representation of
$u'$ with respect to $(v,w,u)$.  The set $E=\{e_1,e_2,e_3\}$ is called
the orthonormal frame for $(v,w,u)$.  
%We write 
%  $$u' = P(E,r,\theta,\phi)$$
%for Equation~\ref{eqn:sph}.
\end{definition}
\indy{Index}{coordinate systems}
\indy{Index}{coordinate systems!spherical coordinates}
\indy{Index}{spherical coordinates}

The following gives the existence of polar coordinates on any oriented
plane in three dimensions, with a general point $v$ on the plane
serving as the origin.  We use a normal vector $n$ to orient the plane,
then obtain polar coordinates as the restriction of the
spherical coordinates $(r,\theta,\phi)$ to the plane.
The following lemma shows that the value of $\phi$ is fixed, so that
it may be dropped from the notation.
\indy{Index}{polar coordinates}
\indy{Index}{spherical coordinates}

\begin{lemma}\guid{YBXRVTS}\label{lemma:polar-gen}\oldrating{60}
\rating{0}
\formalauthor{Nguyen Quang Truong}
Let $\{v,w,u\}$ be a set of three points in $\ring{R}^3$
that is not collinear.
Let $n = (w-v) \times (u-v)$.
Then the zenith angle of any $u'\ne v$ in the plane $\op{aff}\{v,w,u\}$,
computed with respect to $(v,v+n)$,
is $\pi/2$.
\end{lemma}
\indy{Index}{zenith}
\indy{Index}{angle!zenith}

\begin{definition}[polar coordinate]\label{def:polar}
We call  the two remaining coordinates, $(r,\theta)$, 
the polar coordinates of $u'\in\op{aff}\{v,w,u\}$ with
respect to $(v,w,u)$.
\end{definition}
\indy{Index}{coordinate systems}
\indy{Index}{coordinate systems!polar coordinates}
\indy{Index}{polar coordinates}

In the special case that $\op{aff}\{v,w,u\}=\ring{R}^2\subset \ring{R}^3$, this
construction agrees with the previously defined polar coordinates of a point in
the plane.

\subsection{azimuth cycle}

We have already defined the polar cycle, a cyclic permutation on a set
of vectors in the plane, that traverses them in order of increasing
angle.  We now make the corresponding construction in three dimensional
space.  There is  a cyclic permutation on a set of vectors in space
that traverses them in order of increasing azimuth angle.  We call it
the azimuth cycle.  Most of the work has already been done, because we
can simply project the vectors to a plane and take the corresponding
polar cycle on the set of projections.  However, we must impose
a nondegeneracy condition on the set we start with, to insure that
the projection to the plane is one-to-one.  The following
definition captures this nondegeneracy condition.
\indy{Index}{azimuth cycle}
\indy{Index}{azimuth}
\indy{Index}{cyclic permutation}
\indy{Index}{vector!projections}


\begin{definition}[cyclic set] Let $(v,w)$ be an ordered pair of points in
$\ring{R}^3$, with $v\ne w$.
Let $W$ be a finite set of points in $\ring{R}^3$.
We say that $W$ is cyclic with respect to $(v,w)$ if
the following two conditions hold.
First, $p = q + h (w-v)$, with $p,q\in W$ and $h\in \ring{R}$
implies that $p=q$.  Second, $W$ does not meet the line
through $v$ and $w$.
\end{definition}
\indy{Index}{cyclic set}

Cyclicity is precisely the condition we need for the
set $W$ to map injectively to the nonzero points of the 
plane under the projection
onto the first two cylindrical coordinates.  We now define
the azimuth cycle as the polar cycle on the projection of $W$.
\indy{Index}{cyclic}


\begin{definition}[azimuth cycle]
Let $v$ and $w$ be distinct points in
$\ring{R}^3$, with $v\ne w$.
Let $W$ be a finite set of points in $\ring{R}^3$ that is
cyclic  with respect to $(v,w)$.
Pick a positively oriented orthonormal frame $\{e_1,e_2,e_3\}$
with $e_3\cdot (w-v) > 0$.
Let $f$ be the projection map:
   $$v + x\, e_1 + y\, e_2 + z\, e_3 \mapsto
     (x,y).$$
Let $\sigma'$ be the polar cycle on $f(W)$. Define 
$\sigma:W\to W$ by $f\sigma(p) =\sigma'f(p)$.
Call $\sigma$ the azimuth cycle
on $W$ with respect to $(v,w)$.
\indy{Index}{azimuth cycle} 
\indy{Index}{orthonormal frame}
\indy{Index}{polar cycle}
\end{definition}

Facts about the polar cycle lift to facts about the azimuth cycle.
The next few lemmas are easy consequences of this sort.


\begin{lemma}\guid{NLOFMTR}\rating{80} The azimuth cycle $\sigma:W\to W$ on
a cyclic set $W$ with respect to $(v,w)$ does not depend
on the choice of $e_1$ (and $e_2$ is determined by $e_1$).
\end{lemma}
\indy{Index}{azimuth cycle}
\indy{Index}{cyclic set}

\begin{proof} This follows from independence of $\sigma'$ from
rotations in the $\{e_1,e_2\}$ plane  (Equation~\ref{eqn:rotate}).
\end{proof}


\begin{lemma}\guid{YVREJIS}\oldrating{40} 
\rating{0}
\formalauthor{Nguyen Quang Truong}
Let $(v,w)$ be an ordered pair of points in $\ring{R}^3$,
with $v\ne w$.  Assume that $\{w_1,w_2\}$ is cyclic
with respect to $(v,w)$.  Then
  $$
  \op{azim}(v,w,w_1,w_2) + \op{azim}(v,w,w_2,w_1) 
  = \begin{cases} 2\pi, & \text{if }\op{azim}(v,w,w_1,w_2)\ne 0,\\
    0, & \text{if }\op{azim}(v,w,w_1,w_2)=0.
    \end{cases}
    $$
\end{lemma}
\indy{Index}{cyclic}

\begin{proof} This follows immediately from Lemma~\ref{lemma:polar2}.
\end{proof}

\begin{lemma}\guid{ULEKUUB}\rating{60} \label{lemma:2pi-sum}
Let $(v,w)$ be an ordered pair of points in $\ring{R}^3$,
with $v\ne w$.  Let $W$ be a finite set in $\ring{R}^3$ of
cardinality $n$ that
is cyclic with respect to $(v,w)$,
with azimuth cycle $\sigma$.
Then have for all $p\in W$,
and all $0\le i \le j < n$,
   $$
   \op{azim}(v,w,p,\sigma^i(p)) +
    \op{azim}(v,w,\sigma^i(p),\sigma^j(p)) =
   \op{azim}(v,w,p,\sigma^j(p)).
   $$
Moreover, if there exists $q\in W$ such that 
$\op{azim}(v,w,p,q)\ne0$,
we have 
  $$
  \sum_{i=0}^{n-1} \op{azim}(v,w,\sigma^ip,\sigma^{i+1}p) = 2\pi.
  $$
(If $\op{azim}(v,w,p,q)=0$ for all $q\in W$, then all the
summands are zero.)
\end{lemma}
\indy{Index}{azim}
\indy{Index}{azimuth}
\indy{Index}{azimuth cycle}
\indy{Notation}{ZZsigma@$\sigma$}

\begin{proof} This follows immediately from 
Lemma~\ref{lemma:polar-sum}.
\end{proof}


The azimuth and dihedral angles are closely related.   By construction,
the azimuth angle takes values between $0$ and $2\pi$, but the dihedral
angle is never greater than $\pi$.  The following lemma reveals that
the azimuth angle is an oriented extension of the dihedral angle, always
equal to $\dih$ or $2\pi - \dih$.
\indy{Index}{angle!azimuth}
\indy{Index}{angle!dihedral}
\indy{Index}{dih}
\indy{Index}{azimuth}


\begin{lemma}\guid{QQZKTXU}\label{lemma:dih-azim}
\oldrating{100}
\rating{0}
\formalauthor{Nguyen Quang Truong}
Let $w\ne v$ be a nonzero vectors in $\ring{R}^3$.
  Assume that $v_1$ and $v_2$ do not lie in the line $\op{aff}(v,w)$.
Let
  $$\gamma = \dih_V(\{v,w\},\{v_1,v_2\}).$$
  Then
    $$
    \cos(\op{azim}(v,w,v_1,v_2)) = \cos\gamma.
    $$
\end{lemma}

\begin{proof}  For simplicity,
we will take our base point $v=0$.
Let $v_i' = (w\cdot w) v_i - (w\cdot v_i) w$.  
We have $v_1'\ne 0$.  Set $e_1 = v_1'/\normo{v'_1}$.  Choose a unit vector
$e_2$ so that $\det(e_1,e_2,w)>0$ and $e_1\cdot e_2 = w\cdot e_2=0$.
Write $v_i$ in cylindrical coordinates as 
   $$
   \begin{array}{lllll}
     v_1 &= r_1 e_1 &    &+h_1 w\\
     v_2 &= r_2 \cos\theta\, e_1 &+ r_2 \sin\theta\, e_2 &+ h_2 w.
    \end{array}
   $$
By the definition of $\op{azim}$, we have $\op{azim}(v,w,v_1,v_2)=\theta$.  
By definition, $\cos\gamma$ is the angle between $v_1'$ and $v_2'$.
We compute
   $$
   \begin{array}{lll}
     v_1' &= \normo{v'_1} e_1 \\
     v_2' &= (w\cdot w) r_2 \cos\theta\, e_1 
       &+ (w\cdot w) r_2 \sin\theta\, e_2 \\
     \end{array}
   $$
The result $\cos\theta=\cos\gamma$ 
is now a result of the definition of angle 
(Definition~\ref{def:angle}).
\end{proof}





