% ------------------------------------------------------------ 
% Author: Thomas C. Hales 
% Format: LaTeX Book Chapter: Dense Sphere Packings
% ------------------------------------------------------------

\chapter{Credits}\label{sec:credit}

This book is just a blueprint, which gives instructions about how to
construct the formal proof.
\newterm{Flyspeck} is the name of an ongoing project to construct a
formal proof of the Kepler conjecture in the HOL Light proof
assistant, along the lines described in this book.  The eventual aim
of the project is to give a formal verification of the computer
portions of the proof as well as the standard text portions of the
proof.  The project is about 80\% complete as of May 2012.  
 The source code for the project and
information about the current project status are available at
\cite{website:FlyspeckProject}.



Here is the fine print about the current project status. (I hope that
this status report is out of date by the time this book is printed.)
There are four components to the formalization project (text, hypermaps, linear programs,
and nonlinear inequalities), at various stages of completion.
\begin{enumerate}
\item
The English mathematical text in this book has been fully
  formalized, with the following explicit omissions.
\begin{enumerate}
\item The formal proof does not include the parts of the text (such
  as remarks, introductory passages, and auxiliary results in 
  Section~\ref{sec:further}) that are
  not strictly a part of the proof of the Kepler conjecture.
\item There are three specific sections of this book that describe
  the relationship between the text and the computer portions of
  the proof: Section~\ref{sec:hypermap-algorithm} (Hypermap Algorithm), 
  Section~\ref{sec:weight} (Main Estimate), and Section~\ref{sec:tsp} (Linear Programs).
  They have not yet been formalized.
\item There are still a few lemmas scattered throughout in the book that remain to be formalized.
These lemmas are listed at~\cite{website:FlyspeckProject}.
Combined, these scattered lemmas amount to fewer than 350 lines of English proof text.
\end{enumerate}
\item Formal verification of the hypermap generation program is complete~\cite{Nipkow:2005:Tame}.
\item Formal verification of linear programs is near 
completion~\cite{Obua:2005:Thesis} and~\cite{Solovyev:LP}.
 The technology to verify linear programs in HOL Light has been developed and 
  an efficient implementation has been made, but the actual formal verifications of the
  long list of linear programs remains to be made.
\item The technology to make formal verifications of nonlinear inequalities over the
field of real numbers has been designed and implemented.   These are slow verifications,
and the software must be further optimized, before the verification of large collections
of inequalities can be considered practical.  This research will appear in a forthcoming doctoral 
thesis by A. Solovyev
at the University of Pittsburgh.  Earlier work on formal proofs of nonlinear inequalities
is mentioned in~\cite{HHMNOZ}.
\end{enumerate}


The Flyspeck project has been a large team effort over a period of years, 
and my contributions have
been just a fraction of the whole.   This
appendix cites their contributions. The principal formalizer of each
chapter is indicated with an asterisk.

\def\x#1{\text{#1}}

\[
\begin{array}{lll}
\x{\bf Text~Chapter} & \x{\bf Author~of~formalization~work}
\vspace{4pt}\\
\x{Trigonometry} & \x{Nguyen Quang Truong\ast; Rute, Jason;}\\
  &\x{Harrison, John; Vu Khac Ky}\\
\x{Volume} & \x{Harrison, John\ast; Nguyen Tat Thang}\\
\x{Hypermap} &  \x{Tran Nam Trung}\ast\\
\x{Fan} & \x{Hoang Le Truong\ast; Harrison, John}\\
\x{Packing} & \x{Solovyev, Alexey\ast; Vu Khac Ky\ast;}\\
&\x{Nguyen Tat Thang; Hales, Thomas}\\
\x{Local Fan} & \x{Nguyen Quang Truong\ast}; \x{Hoang Le Truong\ast}\\
\x{Tame Hypermap} & \x{Solovyev, Alexey\ast; Dat Tat Dang; Trieu Thi Diep;}\\
&\x{Vu Quang Thanh;  Vuong Anh Quyen}\vspace{8pt}\\
\x{\bf Code-Verification} & \x{\bf Author~of~formalization~work}
\vspace{4pt}\\
\x{Hypermap Generation} & \x{Nipkow, Tobias\ast; Bauer, Gertrud\ast}\\
\x{Linear Programs} & \x{Obua, Stephen\ast; Solovyev, Alexey\ast}\\
\x{Nonlinear Inequalities} &  \x{Solovyev, Alexey\ast}\\
\end{array}
\]


\[
\begin{array}{lll}

\end{array}
\]


\newpage