%% WAS IN FORMULATION. DOESN'T BELONG IN TARSKI.

We conclude with the proof of the main theorem of the \chap.

\begin{proof} {\bf (Theorem~\ref{thm:nonoverlap})}
The rules defining the $Q$-system specify a uniquely determined
set of simplices.  The proof that their interiors are pairwise
disjoint is established by the preceding series of lemmas.
Lemma~\ref{lemma:qrtet-over} shows that the interiors of
quasi-regular tetrahedra do not meet the interiors of other
simplices in the $Q$-system. Lemma~\ref{lemma:oct-over} shows that
the quarters in quartered octahedra are well-behaved.
Lemma~\ref{lemma:adj-over} shows that the interiors of other
quarters in adjacent pairs are disjoint from the interiors of
other simplices in the $Q$-system. Finally, we treat isolated
quarters in Lemma~\ref{lemma:iso-over}. These cases cover all
possibilities since every simplex in the $Q$-system is a
quasi-regular tetrahedron or strict quarter, and every strict
quarter is either part of an adjacent pair or isolated.
\end{proof}


\begin{definition} \label{def:height}  Let $\Lambda$ be a
saturated packing.  Assume that the coordinate system is fixed in
such a way that the origin is a vertex of the packing.  The {\it
height\/} of a vertex is its distance from the origin.
%
 \index{height}
\end{definition}

\begin{definition} \label{def:enclosed}\index{enclosed}
We say that a vertex is {\it enclosed\/} over a figure if it lies
in the interior of the cone at the origin generated by the figure.
%
 \index{vertex!enclosed}\index{enclosed}
\end{definition}

\begin{definition}\label{def:dih}
In general, let $\dih(S)$ be the dihedral angle of a simplex $S$
along its first edge. When we write a simplex in terms of its
vertices $(w_1,w_2,w_3,w_4)$, then $\{w_1,w_2\}$ is understood to
be the first edge.
%
 \index{dih (dihedral angle)}
\end{definition}


Our simplices are generally assumed to come labeled with a
distinguished vertex, fixed  at the origin. (The origin will
always be at a vertex of the packing.) We number the edges of each
simplex $1,\ldots,6$, so that edges $1$, $2$, and $3$ meet at the
origin, and the edges $i$ and $i+3$ are opposite, for $i=1,2,3$.
$S(y_1,y_2,\ldots,y_6)$ denotes a simplex whose edges have lengths
$y_i$, indexed in this way. We refer to the endpoints away from
the origin of the first, second, and third edges as the first,
second, and third vertices.
%
 \index{labels!edge}
 \index{first!edge}



