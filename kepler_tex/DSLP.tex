\section{Centered Packings and Linear Programming}

We have intentionally avoided all discussion of centered packings
and geometry in our treatment of linear programs and in the
statement of the Main LP bound (\ref{XX}).

This section is not needed in the proof of the Main LP bound.
Rather, it describes how it is to be used.  To make use of the
Main LP bound in the Kepler conjecture, we need to relate the data
of the Main LP back to centered packings and the nonlinear
inequality that arises there.  It is the purpose of this section
to establish that link.

To accomplish this, we define a relation
    $$\pi:\op{DECOMP}\times \op{TLS}\times \op{ISOH}
    \to\bool$$
where $\op{ISOH}$ is the set of isomorphisms between hypermaps.
Roughly, $\pi(D,(H,\op{fl},S),\phi)$ is true exactly when the
centered packing $D$ has hypermap $\op{hyp}(D) =\phi(H)$ and the
centered packing $D$ is related to the data $(H,\op{fl},S)$ in a
suitable way.

\begin{theorem}  For every contravending centered packing $D$
whose hypermap is not isomorphic to $H_{fcc}$ or $H_{hcp}$, there
exist $\rho\in\op{PTLS}$ and $\phi$ such that
    $$\pi(D,\rho,\phi)$$
\end{theorem}

\begin{theorem}
Assume $D$ is a contravening centered packing and
    $$
    \pi(D,\rho,\phi).
    $$
Assume further that if $\op{trans}(\rho,\rho')$, then

    If
    $$\pi(D,\rho,\phi),$$
    then
    $$\sigma(D) \le \op{interp\_lin}(\op{sc})(\rho).$$
\end{theorem}

\begin{corollary}
Assume the hypotheses (A) and (B) of Theorem XX are met.  Let $D$
be a centered packing whose hypermap is not isomorphic to
$H_{fcc}$ or $H_{fcc}$.  Then
    $$\sigma(D) < 8\,\pt.$$
\end{corollary}

\begin{proof}  Let $\rho,\phi$ be the
primary tame linear system and isomorphism $\phi$ given by Theorem
XX, associated with $D$. By hypothesis, we may apply Theorem XX
and Theorem XX, which assert that
    $$\sigma(D) \le \op{interp\_lin}(\op{sc})(\rho) < 8\,\pt.$$
\end{proof}

\subsection{Definition of $R$}

The relation $R$ is defined as the conjunction of several
subordinate relations
    $$R(D,\rho,\phi) = \bigwedge_K R_k (D,\rho,\phi).$$
The subordinate relations are defined in this subsection. The
indexing set is
    $$
    K = \{H,\op{qrn},\op{qre},\op{std\_quad},\ldots\}.
    $$


\begin{definition}
    $$R_H(D,(H,{\op{fl}}),\phi) \Leftrightarrow (\phi(H)=\op{hyp}(D))$$
\end{definition}

\begin{definition}
The function $R_{V}$ tests compatibility with vectors.
    $$R_V(D,(H,{\op{fl}}),\phi) \Leftrightarrow
        (\forall \alpha\in \op{dart}(H).\ v(\alpha) =
        v(D,\phi\alpha))$$
\end{definition}

\begin{definition}
 The functions $\op{qrn}$ and $\op{qre}$ detect {\it quasi-regular
 nodes and edges}.
        $$R_{\op{qrn}}(D,(H,\op{fl}),\phi) \Leftrightarrow
        \forall N\in H/n.\ \op{qrn}(N) = \op{qrn}(D,\phi(N)).
        $$
        $$R_{\op{qre}}(D,(H,\op{fl}),\phi) \Leftrightarrow
        \forall E\in H/e.\ \op{qre}(E) = \op{qre}(D,\phi(E)).
        $$
\end{definition}

%\begin{definition}
%The flag component $\op{is\_refined}$ keeps tracks of changes to
%an primary tame linear system.
%    $$\op{is\_refined}:H/f\to \bool$$
%    $$R_{\op{is\_refined}}(D,(H,\op{fl}),\phi) =
%      \begin{cases} \true &  \forall F\in H/f.\
%      \op{is\_refined}(F)\\
%                    \false & \text{otherwise}.
%      \end{cases}
%      $$
%\end{definition}


[FILL IN]  Relate tame linear systems back to centered packings.

[FILL IN] Define $v(D,\alpha)$, etc. on centered packings.

[FILL IN] Move relations back up to where the subscript is
defined.

\subsection{Satisfaction of Axioms (A)}

[FILL IN]  Use the definition of $R$ to check the axioms (A).



This final lemma in this subsection is not needed for the linear
programs. It will be useful in relating the Main LP Bound back to
the Kepler conjecture.

\begin{lemma}  Let $X$ be a set.  Let $R$ be a relation on
$X\times \op{HS}$ such that
    \begin{itemize}
    \item $\op{bif}(\rho,\rho',\rho'',L) \wedge R(x,\rho)\Rightarrow
      R(x,\rho') \vee R(x,\rho'')$,
    \item $R(x,\rho) \Rightarrow \neg \op{INFEAS}(\rho)$,
    \item $R(x,\rho) \wedge \op{ref}(\rho,\rho') \Rightarrow
    R(x,\rho')$.
    \end{itemize}
Then for all $x\in X$ and $\rho\in\op{PTHS}$ such that
$R(x,\rho)$, we have $\neg \op{weak\_infeas}(\rho)$.
\end{lemma}

\begin{proof} By structural induction over trees.
\end{proof}
