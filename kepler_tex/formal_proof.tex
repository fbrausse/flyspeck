% Branched made from latex/notices... on 5/5/2010.

\chapter{Formal Proof}

{\narrower\it 

  ``There remains but one course for the recovery of a sound and healthy
  condition -- namely, that the entire work of the understanding be
  commenced afresh, and the mind itself be from the very outset not
  left to take its own course, but guided at every step; and the
  business be done as if by machinery.'' --  F. Bacon, 1620, Novum Organum
% preface.

}

\bigskip


%%%%%%%%%%%%%%%%%%%%%%

\bigskip

\section{Bugs}

Daily,\footnote{This chapter was first published in the Notices of the AMS, December
  2008.} we confront the errors of computers.  They crash, hang,
succumb to viruses, run buggy software, and harbor spyware.  Our
tabloids report bizarre computer glitches: the library patron who is
fined \$40 trillion for an overdue book because a barcode is scanned
as the size of the fine; or the dentist in San Diego who was delivered
over sixteen thousand tax forms to his doorstep when he abbreviated
``suite'' in his address as ``su.''

% http://amartester.blogspot.com/2007/04/bugs-per-lines-of-code.html
  On average,
a programmer introduces 1.5 bugs per line while typing.
% The "private bug rate" of 1.5 per line refers to 
% "Beizer 1984" referred to on page 23 of "Testing Computer software" by Kaner et al.
Most are typing errors that are spotted at once.
About one bug per hundred lines of computer code ships  to
market without detection.  Bugs are an accepted
part of programming culture.
The book that describes itself as the ``bestselling software testing
book of all time'' states that ``testers shouldn't want to verify
that a program runs correctly''~\cite{KFN}.
Another book on software
testing states, ``Don't insist that every bug be fixed \dots.
When the programmer fixes a minor bug, he might create
a more serious one.''  Corporations may keep critical bugs
off the books to
limit legal liability.
 Only those bugs should be corrected
that affect  profit.
The tools designed to root out bugs are themselves
full of bugs. ``Indeed, test tools are often buggier than
comparable (but cheaper) development tools''
\cite{KBP}.
% Lessons Learned in Software Testing, Kaner, Bach, Pettichord 2001.
As for hardware reliability, former 
Intel President Andy Grove himself said, 
``I have come to the conclusion that no microprocessor is ever
perfect; they just come closer to perfection \dots.''
\cite[p.~221]{Mac}.
% quoted p221. MacKenzie, Mechanizing Proof.
%(at the time of the famous pentium bug) 


Bugs can be far-reaching.
The bug causing the 
explosion of the Ariane 5 rocket cost hundreds of millions
of dollars.  As long ago as 1854, Thoreau wrote that 
``by the error of some calculator
the vessel often splits upon a rock that should have reached
a friendly pier.''  % Walden, Economy, page 13.
In 2007, the {\it New York Times\/} reported Shamir's warning that
even a small math error in a widely used computer chip could 
be exploited to defeat cryptography and would
place
``the security of the global electronic commerce system at risk
\dots{}''~\cite{NYT}.
% Adding Math to List of Security Threats, New York Times, November 17, 2007.





\section{Mathematical Certainty}

By contrast, philosophers tell us that
mathematics consists of analytic truths,
free of all imperfection.  We prove that $1+1=2$ by
recalling the definition of $1$ as the successor of $0$,
$2$ as the successor of $1$, and then invoking twice the recursive
definition
of addition: 
  $$1+1 = 1 + S(0) = S(1 + 0) = S(1) = 2.$$

If only all proofs were so simple.  
Mathematical error is as old as mathematics itself.
Euclid's very first proposition asks, ``On a given straight line
to construct an equilateral triangle.''  Euclid's construction
makes the implicit assumption -- not justified by the axioms -- that
two circles, each passing through the other's center, must intersect.
We revere Euclid, not because he got everything right, but because
he set us on the right path.

We have entered an era of proofs of extraordinary complexity.
Take, for example, Almgren's masterpiece in geometric measure
theory, called appropriately enough the {\it ``Big Paper.'' }
%http://www.worldscibooks.com/mathematics/4253.html
The preprint is 1728 pages long. Each line is a chore. He spent over a
decade writing it in the 1970s and early 80s.  It was not published
until 2000.  Yet the theorem is fundamental.  It establishes the
regularity of minimizing rectifiable currents, up to codimension two;
in basic terms, it shows that higher dimensional soap bubbles are
smooth rather than jagged -- just as one would naturally expect.  How
am I to develop enough confidence in the proof that I am willing to
cite it in my own research?  Do the stellar reputations of the author
and editors suffice, or should I try to understand the details of the
proof?  I would consider myself very fortunate if I could work through
the proof in a year.

Computer proofs, which are sprouting up in many fields of mathematics,
compound the complexity: 
the nonexistence of a projective plane of order ten,
the proof that the Lorenz equations have a strange attractor,
the double-bubble problem for minimizing soap bubbles enclosing
two equal volumes, the optimality of the Leech lattice among
24-dimensional lattice packings, hyperbolic three-manifolds,
and the one that got it all started: the four-color theorem.
% W. Tucker solved Smale's 14th problem by computer, establishing
%that the Lorenz equations have a strange attractor. 
%A. Kumar and H. Cohn
%The Search for a Finite Projective Plane of Order 10
%http://www.cecm.sfu.ca/organics/papers/lam/paper/html/node5.html
%http://arxiv.org/abs/math/9609207 Homotopy 3-manifolds.
What assurance of correctness do complex computer proofs provide?



\section{Formal Proof}

Traditional mathematical proofs are written in a way to make them
easily understood by mathematicians. Routine logical steps are
omitted. An enormous amount of context is assumed on the part of the
reader. Proofs, especially in topology and geometry, rely on intuitive
arguments that a trained mathematician would be capable
of translating into a more rigorous
argument.


A formal proof is a proof in which every logical inference has been
checked all the way back to the fundamental axioms of mathematics.
All the intermediate logical steps are supplied, without exception. No
appeal is made to intuition, even if the translation from intuition to
logic is routine. Thus, a formal proof is less intuitive, and yet less
susceptible to logical errors.

There is a wide gulf that separates traditional proof from formal proof.
For example, Bourbaki's Theory of Sets was designed as a purely theoretical
edifice that was never intended to be used in the proof of actual theorems.
Indeed, Bourbaki declares that ``formalized mathematics cannot
in practice be written down in full'' and calls such a project
``absolutely unrealizable.''  % Bourbaki, Elements of Sets, Addison-Wesley, Reading, MA, 1968,  pp.10,11.
The basic trouble with various foundational systems is that meta-mathematical arguments (for
example, abbreviations that are external to the system or
inductions over the syntactical form of an expression) 
are usually introduced early on, and without these simplifying meta-arguments,
the vehicle stalls, never making it up the steep incline from primitive notions to 
high-level concepts.   The gulf can be extreme.  In fact, Matthias has calculated
that to expand the definition of the number $1$ fully in terms of Bourbaki primitives requires
over four trillion symbols.
% A Term of Length 4,523,659,424,929 Synthese 133 (2002) 75--86.
In Bourbaki's view, 
the foundations of mathematics are roped-off museum pieces
to be silently appreciated, 
but not handled directly.

There is an opposing view that regards the 
foundational enterprise
as unfinished until it is realized in practice and written down in full.
This chapter sketches the current state of this endeavor.
It has been necessary to commence afresh and to retool the foundations
of mathematics for practical efficiency, while preserving
its reliability and austere beauty.  For anything beyond a trivial
proof, the number of logical inferences is so large that a computer is
used to ensure that no steps are omitted.   This endeavor raises basic questions
about trust in computers.  This chapter also places formal proofs within
a broader context of automating more general mathematical tasks.

As the art is currently practiced, each formal proof starts with a traditional
mathematical proof, which is rewritten in a greatly expanded form, in which all the
assumptions are made explicit and all cases are treated in full.
For example, a traditional mathematical proof might show that a graph is
planar by drawing the graph on a sheet of paper.  The expanded form of
the proof  replaces the picture by careful argument.  From the
expanded text, a computer script is prepared, which generates all
the logical inferences of the proof.  The transcription of a single traditional
proof into a formal proof is a major undertaking.

\bigskip
\noindent
\framebox{\parbox{4.2in}{
\smallskip
\centerline{\it Three Early Milestones}
\smallskip

1954 -- Davis programs the Presburger
algorithm for additive arithmetic into the
``Johniac'' computer at the Institute
for Advanced Study.  
Johniac proves that the sum of two
even numbers is even, to usher in the era of computer proof.

\smallskip

1956 -- The automation of Russell and Whitehead's
{\it Principia Mathematica\/} begins~\cite{NSS}.
By the end of 1959, Wang's procedure has generated proofs
of every theorem of the Principia in the predicate calculus~\cite{Wang}.

\smallskip

1968 -- De Bruijn designs
the first computer program to check the validity of general mathematical
proofs.  His program Automath eventually checks
every proposition in a primer that Landau writes for his
daughter on the construction
of real numbers as Dedekind cuts.  
% daughter: see http://www.cs.ru.nl/~freek/aut/aut-4.1-manual.pdf
% Landau as google book: http://books.google.com/books?id=U9B5FKvx3pYC
% Landau as pdf: http://www.cs.ru.nl/~freek/aut/
% The culminating result was the proof that $ii = - 1$.
\smallskip

}}
\bigskip

\bigskip
\noindent
\framebox{\parbox{4.2in}{ \smallskip \centerline{\it N. G. de Bruijn}

%% Mar 23, 2010. removed.
%    \smallskip On April 24, 2008, F. Wiedijk and I visited N. G. de
%    Bruijn at his home in Nuenen, shortly before his ninetieth
%    birthday. (Nuenen is the Dutch town where Vincent van Gogh lived
%    when he painted the {\it Potato Eaters}.)  We discussed Automath,
%    Brouwer, Heyting, and some of his coauthors (Knuth and Erd\"os).
%    De Bruijn has contributed to many fields of mathematics, including
%    analytic number theory, Penrose tilings, quasicrystals, and
%    optimal control.


\smallskip
De Bruijn indices give a notation that
eliminates all dummy  variables from formulas with
quantifiers: $\forall\,x.~P(x)$ becomes
$(\forall~P~1)$.  This notation solves the problem of
free variable capture.

\smallskip
De Bruijn observed that the ratio of lengths of a formal proof to
the corresponding conventional proof is remarkably
stable across different proofs.  The ratio, called the de Bruijn factor,
has become the standard benchmark to measure the overhead of a formal proof.



\smallskip
}}
\bigskip

\subsection{examples}

Computer proof assistants have been under development for decades (see
Box~``Early Milestones''), 
but only recently has it become practical to prove major
theorems formally.  The most spectacular example is Gonthier's formal
proof of the four-color theorem.  His starting point is the
second-generation proof by Robertson et al.  Although both the traditional
proof and Gonthier use a computer, the two computer
processes differ from one another in the same way that 
adding $1+1=2$ on a calculator differs from the mathematical
justification of $1+1=2$ by definitions, recursion, and a rigorous
construction of the natural numbers.  In short, a large logical gulf
separates them.  As a result of Gonthier's formalization, the proof of
the four-color theorem has become one of the most meticulously
verified proofs in history.

In recent years, several other significant theorems have been formally
verified. (See Table~\ref{table}.)  The table lists the theorems, which
proof assistant was used (there are many to choose from), the person
who produced a formal proof, and the mathematicians who produced the
original proof.  The Prime Number Theorem, which asserts that the number
of primes less than $n$ is asymptotic to $n/\log\,n$, has two
essentially different proofs: the elementary proof of Selberg and
Erd\"os and the analytic proof of Hadamard and de la Vall\'ee Poussin.
Formal versions of both proofs have been produced.  More ambitious
projects are in store: Gonthier's team is now formalizing the
Feit-Thompson odd order theorem, and the leading problem of the
document {\it Ten Challenging Research Problems for Computer Science}
is the formalization of the proof of Fermat's Last
Theorem~\cite{Berg}.
% J. Bergstra, 5 July 2005.






\smallskip

\begin{table}[ht]
\caption{Examples of Formal Proofs}
\centering
\begin{tabular}{l l l l l}
\hline
Year\hspace{0.5em} &Theorem\hspace{8em} &Proof System\hspace{2em}  &Formalizer\hspace{3em} &Traditional Proof\\ [0.5ex]
\hline \\
1986 &First Incompleteness &Boyer-Moore   &Shankar &G\"odel \\
1990 &Quadratic Reciprocity&Boyer-Moore &Russinoff &Eisenstein\\
1996 &Fundamental - of Calculus &HOL Light &Harrison &Henstock\\
2000 &Fundamental - of Algebra &Mizar &Milewski    &Brynski\\ % email tchales@gamil.com, Aug 17 from Milewski mentions  Brynski.
2000 &Fundamental - of Algebra &Coq &Geuvers et al.   &Kneser\\
2004 &Four Color &Coq &Gonthier &Robertson et al.\\
2004 &Prime Number &Isabelle &Avigad et al. &Selberg-Erd\"os\\
2005 &Jordan Curve  &HOL Light &Hales &Thomassen \\
2005 &Brouwer Fixed Point &HOL Light &Harrison &Kuhn \\
2006 &Flyspeck I &Isabelle &Bauer-Nipkow &Hales \\
2007 &Cauchy Residue &HOL Light &Harrison &classical \\
2008 &Prime Number &HOL Light &Harrison &analytic proof \\
% 2008 &Flyspeck II &Isabelle &Obua &Hales \\
 [1ex]
\hline
\end{tabular}
\label{table}
\end{table}
% Shankar 1986 dissertation:
% 1986 asserted in Shankar's book, Metamathematics, Machines and G�del's Proof (available as Google book, page xi.

% Quadratic Reciprocity: David M. Russinoff, A Mechanical Proof of Quadratic Reciprocity. J. Autom. Reasoning 8(1): 3-21 (1992).
% http://www.russinoff.com/papers/gauss.pdf
% 

% Fund. Th. of Alg: http://www.cs.ru.nl/~freek/100/
% Aug 21, 2000 for Mizar: http://mizar.uwb.edu.pl/JFM/pdf/polynom5.pdf
% Isabelle: 2008; http://isabelle.in.tum.de/library/HOL/HOL-Complex/Fundamental_Theorem_Algebra.html
% 
%
% A Constructive Proof of the Fundamental Theorem of Algebra without Using the Rationals
% See google book: http://books.google.com/books?id=wbHgOnNZNdYC
%
%Source	Lecture Notes In Computer Science; Vol. 2277 archive
%Selected papers from the International Workshop on Types for Proofs and Programs table of contents
%Pages: 96 - 111  
%Year of Publication: 2000
%ISBN:3-540-43287-6
%Authors	
%Herman Geuvers	
%Freek Wiedijk	
%Jan Zwanenburg	
%Publisher	
%Springer-Verlag  London, UK


The box~``Formal Jordan Curve Theorem'' 
displays the statement of the Jordan Curve theorem in computer
readable form as it appears in the formal proof.  The complete
specification of the theorem should also list all definitions, all the
way back to the primitives.  Without giving the detailed definitions
here, we note that {\it top2} refers to the standard topology on the
plane; {\it top2}~$A$ indicates that $A$ is an open set in the plane;
$\hbox{\it euclid}\,\,2$ is the Euclidean plane; and {\it connected
  top2}~$A$ means that $A$ is a connected set in the plane.

\bigskip
\noindent
\framebox{\parbox{4.2in}{
\parindent=0pt
\smallskip

\centerline{\it Formal Jordan Curve Theorem}
\smallskip

\vbox{
\def\hb{\hfill\break}
\def\h{\hbox{}}
\def\s#1{\hskip#1}
\def\w{\hskip0.65em}

\obeylines

  {\tt %let JORDAN\_CURVE\_THEOREM = prove\_by\_refinement(
  \h~~~$\forall C.\s0.4em \hbox{\it simple\_closed\_curve}\s0.3em\hbox{\it top2} \s0.3em C\w\Rightarrow$\hb
  \h~~~~~( $\exists A\, B.\s0.4em\hbox  {\it top2}\s0.3em A\w\wedge\w{\it top2}\s0.3em B\w\wedge$\hb
  \h~~~~~~~$\hbox{\it connected}\s0.3em\hbox{\it top2}\s0.3em A\w\wedge\w\hbox{\it connected top2}\s0.3em B\w\wedge$\hb
  \h~~~~~~~$A \ne \emptyset\w\wedge\w B \ne \emptyset\w\wedge$\hb
  \h~~~~~~~$A \cap B = \emptyset\w\wedge\w A \cap C = \emptyset\w \wedge\w B \cap C = \emptyset\w \wedge$\hb
  \h~~~~~~~$A \cup B \cup C =\hbox{\it euclid}\hskip0.3em 2$ )
  \h~~~%$\cdots$);;


}
\smallskip
}


}}
\medskip




A large library is maintained of all previously established proofs in
the system, and anyone may use any result that has been previously
established.  Although every step of every proof is always checked, as
researchers contribute to the system, interaction with the system
gradually moves away from the primitive foundations towards something
more closely resembling the high-level practice of mathematicians.
The hope is that proof assistants will eventually become sufficiently user-friendly to
become a familiar part of the mathematical workplace, much as email,
\TeX\relax, computer algebra systems, and web browsers are today.






\section{HOL Light}


This section gives a brief introduction to one foundational system
designed for doing mathematical proofs on a computer.  The system is
called \newterm{HOL Light}, an acronym for a lightweight implementation of
Higher Order Logic.  I have singled it out because of its simple
design and because it is the system that I understand the best.  Some
understanding of the design of a simple system is helpful before
turning to questions of soundness in the next section.  HOL Light by
itself is only a small part of the overall formal-theorem-proving
landscape.  There are several competing systems to choose from, built
on various logical foundations with their own powerful features.
People argue about the relative merits of the different systems much
in the same way that people argue about the relative merits of
operating systems, political loyalties, or programming languages.  To
some extent, preferences show a geographical bias: HOL in the UK,
Mizar in Poland, Coq in France, and Isabelle in Germany and the UK.


The basic components of the HOL Light system are its types, terms,
theorems, rules of inference, and axioms.  Each is briefly described
in turn.  The Box~``The HOL Light System'' % The HOL Light System
gives a summary of the entire system.

\subsection{types}

Much day-to-day mathematics is written at a level of abstraction that
is indifferent to its exact representation as sets.  For example, it
does not matter how an ordered pair is encoded as a set, as long as
the ordered pair has the characteristic property
 $$
 (x,y) = (x',y') \quad \Leftrightarrow\quad  x = x' \hbox{ and } y=y'.
 $$
 It is bad style to break the abstraction to write $2\in(0,1)$.  This
 layer of abstraction is good news because it allows us to shift from
 Zermelo-Fraenkel-Choice (ZFC) set theory to a different foundational
 system with equanimity and ease.

Many proof assistants are based on types.  Types are familiar to
computer programmers.  In a typed computer language, $3$ is an integer
and $[1.0;2.0;3.0]$ is an array of floating point numbers. An attempt
to add $3$ to this array results in a type mismatch error, and the
computer program will not compile.  The type checking mechanism of
programming languages conveniently detects many bugs at the time of
compilation.


ZFC set theory has no such type checking mechanism.  As de Bruijn
puts it,
``Theoretically, it seems perfectly legitimate
to ask whether the union of the cosine function
and the number $e$ (the basis of natural
logarithms) contains a finite geometry''~\cite{dbXY}.
% - N. G. de Bruijn, Types in Mathematics, page 29
Mathematicians have the good sense not to ask such questions.
However, when moving mathematics to a computer, which is lacking in
common sense, it is useful to introduce types into the foundations to
prevent this kind of nonsense.  By convention, a colon is written
before the name of a type.  For instance, we write the type of the
real number $e$ as $\tc\ring{R}$, or simply $e:\ring{R}$, to indicate
that $e$ is a real number.  The cosine function has a different type
$\tc\ring{R}\to\ring{R}$, or $\cos:\ring{R}\to\ring{R}$.  The type of
the union operator forces its two arguments to have the same type, so
that an attempt to take the union of the cosine function with $e$ is
then flat out rejected.

HOL Light is a new axiomatic foundation with types, different from the
usual ZFC.  The types are presented in
Box~``The HOL Light System.'' % The HOL Light System.
There are only two primitive types: the boolean type {\it \tc bool}
and an infinite type {\it \tc ind}. The rest are formed with type
variables joined by arrows.  A mechanism is also provided for creating
a new type that is in bijection with a nonempty subset of an existing
type, allowing the system to be extended with types for ordered pairs,
integers, rational numbers, real numbers, and so forth.

\subsection{terms}

Terms are the basic mathematical objects of the HOL Light system.  The syntax is based
on Church's $\lambda$-calculus, which uses the notation
   $$
   \lambda x.\ f (x)
   $$
   to represent the function that takes $x$ to $f(x)$, which a
   mathematician would write as $f:\ring{N}\to\ring{N}$, $x\mapsto
   f(x)$.  The name $\lambda$-calculus is derived from the use of the
   letter $\lambda$ to mark function arguments.
   The Box~``The HOL Light System''
   lays out the construction of terms.



In ZFC set theory, there is a bijection of sets
  $$
  Z^{X \times Y} \simeq (Z^Y)^X.
  $$
  In other words, a function $(x,y)\mapsto f(x,y)$ from the Cartesian
  product $X \times Y$ to $Z$ can be viewed as a function on $X$ that
  maps $x$ to a function $f(x,\cdot):Y\to Z$.  The right-hand side of
  this bijection is called the \newterm{curried form} of the function (named
  after the logician Haskell Curry).  In typed systems, the curried
  form of multivariate functions is generally preferred.  Treating
  $X,Y,Z$ as types, we write the type of the curried function as
  $f:X\to (Y \to Z)$ or simply $f:X\to Y\to Z$.

The system has only two primitive constants.  One of them\footnote{The
  second constant is the Hilbert choice operator $(\varepsilon)$,
  discussed below.  Recall that every term that is not a variable, a
  function application, or $\lambda$-abstraction is a constant.
  {\it Constancy'\/} is thus a broader notion here than in first-order logic
  and includes terms such as equality that take arguments.
  Parentheses are drawn around the equality symbol $( = )$ to denote
  the prefixed curried form, with $( = )\, x\, x$ as an alternative
  syntax for $x = x$.} is the equality symbol $( = )$ of type $\tc
A\to A\to bool$.  That is, equality is a curried function that takes
two arguments of the same type and returns the boolean type.




\bigskip
\noindent
\framebox{\parbox{4.2in}{ 
\parindent=0pt 
\smallskip 
\centerline{\it
      The HOL Light
      System%\footnote{\tt Note to editor: typeset the entire HOL Light system on a single page.}
    } % EDITOR
   %

    \smallskip 
{\bf HOL Light} (Lightweight Higher Order Logic) is a
    foundational system designed for doing mathematical proofs on a
    computer.  The notation is based on a typed $\lambda$-calculus.

   %
    \bigskip {\bf 1. Types:} The collection of types is freely
    generated from {\it type variables} $\tc A, \tc B,\ldots$ and {\it
      type constants} $\tc bool$ (boolean), $\tc ind$ (infinite type),
    joined by {\it arrows} $( \to )$.  The colon is used as a
    notational device to indicate a type.  For example, $\tc bool$,
    $\tc bool\to A$, and $\tc(bool\to A)\to (ind \to B)$ are types.

   %
    \bigskip {\bf 2. Terms:} The collection of terms is freely
    generated from {\it variables} $x,y$,\dots{} and {\it constants}
    $0,$\dots, using {\it abstraction} ($\lambda x. t$ where $x$ is a
    variable and $t$ a term) and {\it application} ($f(x)$ for
    compatibly typed terms $x$ and $f$).  Each term has a type.  The
    notation $x\tc A$ indicates that the type of term $x$ is $\tc A$.
    Variables and constants are assigned a type at the moment of
    creation; the types of abstractions and applications are defined
    recursively: the type of $\lambda x. t$ is $\tc A\to B$ when $x\tc
    A$ and $t\tc B$; the type of $f(x)$ is $\tc B$ if $f\tc A\to B$
    and $x\tc A$.  

    %
   \bigskip {\bf 3. Theorems:} A theorem is a {\it
      sequent} $\{p_1,\ldots,p_k\} \vdash q$, where $p_1,\ldots,p_k,q$
    are terms of type $\tc bool$.  The terms $p_1,\ldots,p_k$ are
    called the assumptions and $q$ is called the conclusion of the
    sequent.  The design of the system prevents the construction of
    theorems except through inferences from existing theorems, new
    definitions, and axioms.

}}

\bigskip
\noindent
\framebox{\parbox{4.2in}{ \parindent=0pt 

  %
 \smallskip
 {\bf 4. Inference
      Rules:} The system has ten inference rules and a mechanism for
    defining new constants and types. Each inference rule is depicted
    as a fraction; the inputs to the rule are listed in the numerator,
    and the output in the denominator.  The inputs to the rules may be
    terms or other theorems.  In the following rules, we assume that
    $p$ and $p'$ are equal, up to a renaming of bound variables, and
    similarly for $b$ and $b'$.  (Such terms are called
    $\alpha$-equivalent.)  

\quad On first reading, ignore the
    assumption lists $\Gamma$ and $\Delta$. They propagate silently
    through the inference rules but are really not what the rules are
    about.  When taking the union $\Gamma\cup\Delta$,
    $\alpha$-equivalent assumptions should be considered as equal.

    \smallskip 
%\protect\twocolumn %\framebox{ %\vbox{ 

\smallskip 

Equality is reflexive:  %\footnote{\tt Note to editor: Typeset rules in a double column format, four per column. } % EDITOR
    $$ \frac{a}{\vdash a=a} $$ 

Equality is transitive: 
$$ \frac{\Gamma
      \vdash a=b;~~~\Delta\vdash b'=c} {\Gamma\cup\Delta \vdash
      a=c} $$ 

Equal functions applied to equals are equal: $$
    \frac{\Gamma\vdash f=g;~~~\Delta\vdash a=b}
    {\Gamma\cup\Delta\vdash f\hskip0.1em a = g\hskip0.1em b} $$ 

The
    rule of abstraction holds. Equal terms give equal functions: $$
    \frac{x;~~~\Gamma\vdash a=b} {\Gamma \vdash \lambda x.\ a~=\lambda
      x.\ b} ~\hbox{\ (if $x$ is not free in $\Gamma$)} $$ 

The
    application of the function $x\mapsto a$ to $x$ gives $a$: $$
    \frac{(\lambda x.~a)\, x} {\vdash (\lambda x.\ a)\, x = a}
    $$ 

%}} %\pagebreak %\framebox{\vbox{ %%ASSUME 

Assume $p$, then conclude $p$: $$ \frac{p\tc bool} {p \vdash p} $$ 

An equality-based rule of modus ponens holds: %% EQ_MP 
$$ \frac{\Gamma\vdash p;~~~\Delta \vdash p'=q} {\Gamma\cup \Delta \vdash q} $$ 
% equivalent Harrison's but order reversed.  

If the assumption $q$ gives conclusion $p$ and the assumption $p$ gives $q$, then they are equivalent: $$ \frac{\Gamma \vdash p;~~~\Delta\vdash q} {(\Gamma\setminus q)\cup (\Delta\setminus p) \vdash p=q} $$ 

Type variable substitution holds.  If arbitrary types are substituted in parallel for type variables in a sequent, a theorem results.  

Term variable substitution holds.  If arbitrary terms are substituted in parallel for term variables in a sequent, a theorem results.  %}} %\protect\onecolumn 

}} % FRAMEBOX

\bigskip
\noindent
\framebox{\parbox{4.2in}{
\parindent=0pt


\bigskip {\bf 5. Mathematical Axioms:} There are only three mathematical axioms.  
$$\begin{array}{lll}
    \hbox{Axiom of Extensionality:} &\quad\forall f.\hskip1em(\lambda x.\, f\, x) = f.\\
    \hbox{Axiom of Infinity:} &\quad\exists f\tc ind\to ind.~~(\op{ONE\_ONE}\,f) \land \neg(\op{ONTO}\, f).\\
    \hbox{Axiom of Choice:}&\quad
    \forall P\,x.\hskip1em P x \Rightarrow  P(\varepsilon P).\\
  \end{array} $$ 

Extensionality asserts that every function is
  determined by its input-output relation. Dedekind's axiom of
  infinity asserts the existence of a function that is one-to-one but
  not onto.  The Hilbert choice operator $\varepsilon$ applied to a
  predicate $P$ chooses a term that satisfies the predicate, provided
  the predicate is satisfiable.



}} % FRAMEBOX
\bigskip



\subsection{axioms, inference, and theorems}


There are three mathematical axioms: an axiom of extensionality that
asserts that a function is determined by the values that it takes on
all inputs, an axiom of infinity that asserts that the type $\tc ind$
is not finite, and an axiom of choice.  The system has ten rules of
inference, as described in Box~``The HOL Light System.'' % The HOL Light System
For example, the first two state that equality is reflexive and
transitive.  The final two rules of inference allow one to substitute
new terms for the free variables in a theorem and allow one to
substitute new types for the type variables in a theorem.  Beyond
these ten rules of inference are mechanisms for defining new constants
and new types.  A theorem is expressed in {\it sequent} form; that is,
as a set of assumptions, followed by a conclusion.


\subsection{extending the primitive system}

This primitive system lacks the customary logical operators.  There
are no symbols for ``and'', ``or'', ``not'', and ``implies.''  There are no
universal or existential quantifiers.  The set membership operator is
absent.  It is remarkable none of this is needed to express the rules
of inference.

Logical operators are defined later.  For example, the boolean
constant $T$ (true) can be defined as the conclusion of any theorem
that has no assumptions.  The most accessible yet jarringly iconoclast
theorem comes from the reflexive law applied to equality itself:
$$
\vdash ( = ) ( = ) ( = ).
$$
Each new definition becomes a theorem.
So then $\vdash T = ((=) (=) (=))$.  Conjunction $( \land )$
is roundaboutly defined as the
curried function
 that on boolean inputs $p$ and $q$
returns $(\lambda f.\ f\, p\, q) = (\lambda f.\ f\, T\, T)$; that is,
conjunction yields  true exactly when no curried function $f$ is able to
distinguish $(p,q)$ from
$(T,T)$.
The other logical operations are built with similar tricks.

The inference rules and axioms
become bits of data that are processed by other computer procedures.
For example, to give a formal proof that 
$$
%% CHECKED May 31, 2008; Sep2.
2682440^4 + 15365639^4 + 18796760^4 = 20615673^4
$$
a human is not required to type each primitive inference.  An
automated procedure takes any arithmetic identity as input, generates
the inferences, and produces the theorem as output.  A large number of
such small decision procedures have been programmed into the system to
handle routine tasks such as polynomial simplification, basic
tautologies in logic, and decidable fragments of arithmetic.
Procedures that automatically search for steps in a proof are also
programmed into the computer.  New procedures may be contributed by
any user at any time to automate further tasks.  The design of the
kernel of the system prevents a rogue user from writing computer code
that could compromise the soundness of the system.


All the basic theorems of mathematics up through the Fundamental
Theorem of Calculus are proved from scratch on the user's laptop in
about two minutes every time the system loads, so that the casual user
does not need to be concerned with the low-level details.  Basic facts
of logic and elementary mathematics are simply there in the system to
be used as needed.


\section{Soundness}

HOL Light is both an axiomatic system for doing mathematics and a
computer program that implements the system.  How trustworthy is it?

If the computer is set aside for a moment, and the axiomatic system
alone analyzed, it is known to be consistent relative to ZFC.  That
is, an inconsistency in the HOL Light system would imply the
inconsistency of ZFC.



\subsection{computer implementation}

{\narrower\it  

  You've got to prove the theorem-proving program correct. You're in a
  regression aren't you?  --A. Robinson~\cite[p.~288]{Mac}.
% page 288, Chapter 8, MacKenzie.

}

\smallskip

The more pressing question is the soundness and reliability of the
computer program that implements the logic.  An earlier section
reported that a typical software program has approximately one bug per
100 lines of computer code.  The most reliable software ever created,
for example mission-critical software written for the space shuttle,
has fewer than one bug per 10,000 lines of computer code.  Various
proof assistants vary widely in reliability, ranging from some of the
world's most carefully crafted code at the upper end, to rubbish at
the lower end.  I confine my attention to the upper-end.
% http://amartester.blogspot.com/2007/04/bugs-per-lines-of-code.html



The computer code that implements the axioms and rules of inference is
referred to as the kernel of the system.  It takes fewer than 500
lines of computer code to implement the kernel of HOL Light.  (By
contrast, a Linux distribution contains approximately 283 million
lines of computer code.)
% http://en.wikipedia.org/wiki/Linux (Code size) In a later study, the
% same analysis was performed for Debian GNU/Linux version 4.0.[55]
% This distribution contained over 283 million source lines of code.
A bug anywhere in the kernel of this system might have fatal consequences.  For example,
if one of the axioms  is incorrectly typed, it might lead to an inconsistent system.




Yes, it is a regress, but a rather manageable one.  The kernel is
a tiny amount of computer code, but it verifies hundreds of thousands of lines of
code.  Eventually, it may verify
millions.  The same
kernel verifies everything from the prime number theorem to the
correctness of hardware designs.

Since the kernel is so small, it can be checked on many different
levels.  The code has been written in a friendly programming style for
the benefit of a human audience.  The source code is available for
public scrutiny.  Indeed, the code has been studied by eminent
logicians.  By design, the mathematical system is spartan and clean.
The computer code has these same attributes.  A powerful type-checking
mechanism within the programming language prevents a user from
creating a theorem by any means except through this small fixed
kernel.  Through type-checking, soundness is ensured, even after a
large community of users contributes further theorems and computer
code.  I imagine a poster\footnote{A T-shirt has already been
  made!}
% T-shirt Wiedijk email April 16, 2008 tchales@gmail.com
of the lines of the kernel,  taught in undergraduate courses, and
published throughout the world, as the bedrock of mathematics.  It is
math commenced afresh as executable code.

Experience from other top-tier theorem-proving systems has been that
about three to five bugs are found in each system over a period
of 15-20 years of use.  After decades of use on many different
systems, to my knowledge, only one proof has ever had to be retracted
as a result of bug in a theorem-proving system, and this in a system
that I do not rank in the top-tier: in 1995 a heap overflow error led
to the false claim that the theorem-prover REVEAL had solved the
Robbins conjecture. %% page 289, MacKenzie.
We can assert with utmost confidence that the error rates of top-tier
theorem-proving systems are orders of magnitude lower than error rates
in the most prestigious mathematical journals.  Indeed, since a formal
proof starts with a traditional proof, then does more
checking even at the human level, it would be hard for the outcome to
be otherwise.

As an extra check, Harrison gave what can almost be described as a
formal proof in HOL Light of its own soundness~\cite{HaSelf}.  To get
around the self-referential limitations imposed by G\"odel, he gave
two separate proofs.  In the first proof, a weakened version of HOL
Light is created, without the axiom of infinity.  The standard version
is used to give a formal proof of the soundness of the weakened
version.  In the second proof, a strengthened version of HOL Light is
created, with an additional axiom giving a large cardinal.  The
strengthened version then proves the standard version sound.  These
proofs go beyond traditional relative consistency proofs in logic in
two respects.  First, they are formal proofs, rather than
conventional proofs.  Second, the proofs establishthe
soundness not only of the logic, but also the underlying soundness of the
computer code implementing the logic.\footnote{The soundness of the
  computer code is considered relative to a semantic model of the
  underlying programming language.  This model may differ from the
  real-world behavior of the programming language, a reminder that the
  task of verification is never complete.}

\subsection{export}

In the past few years, a number of programs have been written to
automatically translate a proof written in one system into a proof in
another system.  If a proof in one system is incorrect because of an
underlying flaw in the theorem-proving program itself, then the export
to a different system fails, and the underlying flaw is exposed.
(Except of course, unless the second theorem-proving program also has
a bug that is perfectly aligned with the bug in the first system.
Since these systems are largely independently designed and
implemented, the events of failure in different systems are treated as
nearly independent, so that the probability of a perfect alignment of
failures across $n$ systems, goes to zero roughly as $p^n$, where $p$
is the individual failure rate.)

The soundness of HOL Light has been
exported by Adams~\cite{Adams}.  The
export is a formal proof within a second theorem-prover that
the HOL Light logic and implementation are sound.  It will soon be
within reach for several systems to give proofs of one another's
soundness.  When this is achieved, the probability of a false
certification of a pseudo-proof is pushed an order of magnitude closer
to zero.  With a computer -- indeed with any physical artifact,
whether a codex, transistor, or a flash drive made of proteins from
salt-marsh bacteria --
% bug proteins 
% http://www.getusb.info/50-terabyte-flash-drive-made-of-bug-protein/
% http://www.tomshardware.com/news/bacteria-drives-store-terabytes,3125.html
it is never a matter of achieving philosophical certainty.  It is a
scientific knowledge of the regularity of nature and human technology,
akin to the scientific evidence that Planck's constant $\hbar$ lies
reliably within its experimental range.  Technology can push the
probability of a false certification ever closer to zero: $10^{-6}$,
$10^{-9}$, $10^{-12}$\dots. The intent is that one day a system will
store a million proofs without so much as a misplaced semicolon.

A bug in the compiler, operating system, or underlying hardware has
the potential to compromise a formal proof.  To minimize such bugs,
formal proofs can be made about the correctness of the ambient
computational environment.  Indeed, verification of hardware design,
compilers, and computer languages has long been one of the principal
aims of formal methods.  HOL itself was initially created for hardware
verification.  As early as 1989, a simple computer system from
high-level language down to microprocessor was ``formally specified
and mechanically verified''~\cite{BHMY}.
%% quoted in MacKenzie page 243, ref 77.
Today, the semantics of various high-level programming languages have
been defined with complete mathematical rigor~\cite{Harper}.  In
recent work that is nothing short of spectacular, Leroy has
developed a formally verified compiler for the C programming
language~\cite{CC}.  (When the target of a formal verification is a
piece of computer code rather than a standard mathematical text, the
formalization checks that the computer code conforms to a precise
specification of the algorithm, certifying that the computer code is
bug free.)


\section{Full Automation}

Formal proofs are part of a larger project of automating all
mechanizable mathematical tasks, from conjecture making to concept
formation.  This section touches on the problem of fully automated
proofs -- the discovery of proofs entirely by computer without any
human intervention.  The next section briefly describes the ultimate
challenge of producing an automated mathematician.  Progress has been
gradual.  Fifty years ago, it was famously predicted that within a
decade ``a digital computer will discover and prove an important new
mathematical theorem.''
%  MacKenzie, page 89.  H. Simon and A. Newell 
This did not happen as scheduled.


Most success has been with the development of algorithms to solve
special classes of problems.  The WZ algorithm gives automated proofs
of identities of hypergeometric sums.  Gr\"obner basis methods solve
ideal membership problems.  Wu's geometry algorithm proves theorems
such as Pappus' theorem and Pascal's theorem on the ellipse.
% See Chou's article "Proving Elementary Geometry Theorems Using Wu's Algorithm" in *25 years.*
Tarski's algorithm  solves
problems that can be formulated in the first-order language of the real numbers.
The list of specialized algorithms is in fact enormous.

The most widely acclaimed example of a fully automated computer proof
is the solution of the Robbins conjecture in 1996.  The conjecture
asserts that an alternative definition is equivalent to the usual
definition of a Boolean algebra.  Remarkably, the
solution does not involve any human assistance, specialized
algorithms, or software designed with this particular problem in mind.
Just type the problem into W. McCune's general purpose theorem prover
{\it EQN}, hit return, and wait eight days for the solution to
appear~\cite{Mc1}, \cite{Mc2}.

Yet the story is only a qualified success.  It has remained almost an
isolated example, rather than the first in a torrent of results.  The
conjecture itself has the rather special form of a word problem in an
abstractly defined algebraic system -- a type of problem particularly
suited for computer search.  The proof that was found by computer can
be expressed as a short yet nonobvious sequence of
substitutions. (See box.) % EDITOR: Full Automation of the Robbins..


\bigskip
\noindent
\framebox{\parbox{4.2in}{ \smallskip \centerline{\it Full Automation
      of the \newterm{Robbins Conjecture}} \smallskip Let $S$ be a nonempty set
    with an associative commutative binary operation $(x,y)\mapsto xy$
    and a unary operation $x\mapsto[x]$ which, for convenience, we
    write synonymously as $x\mapsto \bar x$.  The Robbins conjecture
    (in Winker form) asserts that the general Robbins identity
   $$
   [[ab][a\bar b]] = a
   $$
   implies the existence of $c,d\in S$ such that $[cd]=\bar c$.  Here
   is the original proof that EQN discovered, as reconstructed
   in~\cite{fit}.
\begin{proof}  A solution is $c=x^3u$, $d=x u$, where $u=[x\bar x]$ and $x$ is arbitrary.
Abbreviate $j=[cd]$,  $e=u[x^2]\bar c$.  Over the equality sign, 
a prime indicates a direct application of the Robbins identity; a superscript
indicates a substitution of the numbered line; no superscript indicates a rewriting of abbreviations $c,d,e,j,u$.
$$
\begin{array}{lll}
%zero
 0: [u [x^2]] &= [[x\bar x][xx]] =' x.\\
%two
 1: [x u [x u [x^2]\bar c]] &=' [   [[x u x^2] [x u [x^2]]]  [x u [x^2]\bar c]] = 
       [  [\bar c [x u [x^2]]]   [\bar c x u [x^2]] ] =' \bar c.\\ 
%four
 2: [u\bar c]&= [u[x^2 u x]]=^0 [u[x^2 u[u[x^2]]]] =' [ [[u x^2][u [x^2]]] [x^2 u [u [x^2]]] ] 
  \\&='
   [u [x^2]] =^0 x.\\ 
%seven:
 3: [j u]&= [[xcu]u]=' [[xcu][[uc][u\bar c]]] =^2 [[xcu][x[cu]]] =' x\\
%eight
 4: [x[x[x^2]u\bar c]] &=' [  [[x[u\bar c]][x u \bar c]] [x [x^2]u \bar c]] =^2 [ [[x^2][x u \bar c]] [[x^2] x u\bar c]] =' [x^2]\\
%ten:
5: [x\bar c] &=^1 [x [x u [x u [x^2]\bar c]]] =^0  [[u [x^2] ] [x u [x u [x^2]\bar c]]]  
  \\&= [[u [x^2]] [u x[ x e]]] =^4 [  [u[x[xe]]][u x[xe]] ] ='     u\\
%thirteen:
6: [j x]&=' [j[[xc][x\bar c]]] =^{5} [j[[xc]u]] =[[uxc][u[xc]]]=' u\\
[1ex]
%
7:  [cd]&= j =' [[j[x\bar c]][jx\bar c]] =^{5} [[ju][jx\bar c]] =^3 [x[j x \bar c]] 
  =^2[[\bar c u][\bar c j x]]
  \\& =^{6} [ [\bar c [jx]][\bar c j x]] =' \bar c.
 \end{array}
$$
\end{proof}
}} 
\bigskip


Overall, the level today of fully automated computer proof (lying
outside special purpose algorithms) remains that of undergraduate
homework exercises: a group in which every element has order two is
necessarily abelian; a set is not in
bijection with its powerset (Cantor's theorem);
%
% JSTOR: 2004 Annual Meeting of the Association for Symbolic Logic
% E-mail: cebrown(andrew. cmu. edu. The Theorem Proving System TPS can
% be ... TPS can prove automatically are: THM 1 5B: If some iterate of
% function f has a ...  links.jstor.org/
% sici?sici=1079-8986(200503)11%3A1%3C92%3A2AMOTA%3E2.0.CO%3B2-V - Similar pages
if some iterate of a function has a unique fixed point, then the
function has a fixed point; the base $e$ for natural logarithms is
irrational~\cite{TPS},~\cite{Bee}.  Because of current limitations,
fully automated proof tools generally serve to fill in intermediate
steps of a larger formal proof.  They are not ready to take on the
Riemann hypothesis.


\section{Automated Discovery}

What happens if one sets aside rigor and lets a computer explore?  A
groundbreaking project was D. Lenat's 1976 Stanford thesis.  His
computer program \newterm{AM}, which is short for Automated Mathematician, was designed to
discover new mathematical concepts.  When AM was set loose to explore
in the wild, it discovered the concepts of natural number, addition,
multiplication, prime numbers, Pythagorean triples, and even the
fundamental theorem of arithmetic.  The thesis touched off a firestorm
of criticism and praise.

To put AM in context, consider a hypothetical program that is
instructed to discover new concepts by deleting conditions from the
list of axioms defining a finite abelian group.  The computer would
then immediately discover the concepts of infinite group, nonabelian
group, monoid, and so forth because these concepts all arise as
subsets of the axioms.  We might hear exaggerated claims that
%These discoveries could be sensationalized:
a program in Artificial Intelligence has made the ultimate leap
  from the finite to infinite, and from the abelian to the nonabelian,
  rediscovering fundamental concepts in seconds that mathematicians
  have grappled with for centuries.  There are nagging questions
about the emptiness of AM's discoveries; a suggestive representation
of the problem gives the answer away.

More recent projects stir the imagination, even if the field is still
young.  Computer programs have generated over one thousand conjectures
in graph theory, expressing numerical relationships between different
graph invariants.  One open conjecture is described in
Box~``An Open Computer-Generated Conjecture.'' % An Open Computer-Generated Conjecture
No technological barriers prevent us from unleashing conjecturing
machines in all branches of mathematics to see what moonshine they
reveal.


\bigskip
\noindent
\framebox{\parbox{4.2in}{ \smallskip \centerline{\it An Open
      Computer-Generated Conjecture} \smallskip Let $G$ be a finite
    graph with the following properties: \begin{enumerate} \item It
      has at least two vertices.  \item The graph is simple; that is,
      it has no loops or double joins.  \item It is regular; that
      is, every vertex has the same degree.  \item The graph is
      connected.  \end{enumerate} For example, the complete graph (the
    graph with an edge between every two vertices) on $n$ vertices has
    these properties, when $n\ge 2$.  Define the {\it total domination
      number} of $G$ to be the size of the smallest subset of vertices
    such that every vertex of $G$ is adjacent to some vertex in the
    subset.  The {\it path covering number} is the size of the
    smallest partition of the vertices into subsets, such that there
    exists a path confined to each subset $S$ that steps through each
    vertex of S exactly once, that is, the induced graph on $S$ has a
    Hamiltonian path.  \smallskip The computer program Graffiti.pc
    conjectures that {\it the total domination number of $G$ is at
      least twice the path covering number of $G$}.  For example, the
    complete graph on $n$ vertices has path covering number one
  because it has a Hamiltonian path.  Its total domination number is
    two (take any two vertices).  The conjecture is sharp in this case
    by these direct observations~\cite{DLPWW}.

\smallskip
}} 


\section{Flyspeck}

My interest in formal proofs grows out of a practical desire for a
thorough verification of my own research that goes beyond what the
traditional peer review process has been able to provide.  A few years
ago, I launched a project called \newterm{Flyspeck} to give a formal proof
of the Kepler conjecture, asserting that no packing of congruent balls
in three-dimensional Euclidean space can have density greater than the
density of the FCC packing (also known as the
cannonball arrangement).  The name Flyspeck, which quite appropriately
can mean to scrutinize, is derived from the acronym FPK for the
Formal Proof of the Kepler conjecture.


The original proof of this theorem was unusually difficult to check.
In a letter of qualified acceptance for publication in the {\it Annals
  of Mathematics}, an editor described the process, ``The referees put
a level of energy into this that is, in my experience,
unprecedented. They ran a seminar on it for a long time. A number of
people were involved, and they worked hard. They checked many local
statements in the proof, and each time they found that what you
claimed was in fact correct. Some of these local checks were highly
nonobvious at first, and required weeks to see that they worked
out\dots.{} They have not been able to certify the correctness of the
proof, and will not be able to certify it in the future, because they
have run out of energy to devote to the problem.''  In addition to a
three hundred page text, the proof relies on about forty thousand lines of
custom computer code.  To the best of my knowledge, the computer code
was never carefully examined by the referees.  The policy of the {\it
  Annals of Mathematics} states, ``The human part of the proof, which
reduces the original mathematical problem to one tractable by the
computer, will be refereed for correctness in the traditional
manner. The computer part may not be checked line-by-line, but will be
examined for the methods by which the authors have eliminated or
minimized possible sources of error\dots.''

Ultimately, the mathematical corpus is no more reliable than the
processes that assure its quality.  A formal proof attains a much
higher level of quality control than can be achieved by ``local
checks'' and an ``examination of methods.''


Flyspeck may take as many as twenty work-years to complete. Bauer,
Obua, and Zumkeller have already defended Ph.D. theses on the
project~\cite{Bauer:2006:Thesis}, \cite{obua:phd},
\cite{Zumkeller:2008:Thesis}.  Together with the work of their advisor
Nipkow, who is one of the principal architects of the Isabelle proof
assistant, nearly half of the computer code used in the proof of the
Kepler conjecture is now certified.

%% XX Put material on the status of the Flyspeck project here.

\section{Quod Erat Demonstrandum}


The Flyspeck project is a minute speck in the overarching QED project,
an anonymous manifesto declaring that all significant mathematical
results should be preserved in a vast library of formal proofs.  The
labor required to realize such a library would be staggering.  In
1991, de Bruijn proposed an assembly line to turn mathematical ideas
into formally verified proofs~\cite{dB91}.  The standard benchmark for
the human labor to transcribe one printed page of textbook mathematics
into machine verified formal text is one week.  To undertake the
formalization of just one hundred thousand pages of core mathematics
would be one of the most ambitious collaborative projects ever
undertaken in pure mathematics, the sequencing of a mathematical
genome.  One might imagine a massive wiki collaboration that settles
the text of the most significant theorems in contemporary mathematics
from Poincar\'e to Sato-Tate.

Outsourcing is the brute force solution to the QED manifesto.  Most
researchers, however, prefer beauty over brute force; we may hope for
advances in our understanding that will permit us someday to convert a
printed page of textbook mathematics into machine verified formal text
in a matter of hours rather than after a full week's labor.  As long
as transcription from traditional proof into formal proof is based on
human labor rather than automation, formalization remains an art
rather than a science.  Until that day of automation, we fall short of
a practical understanding of the foundations of mathematics.



\section{Recommended Reading and Software}

By far the best overview of the subject is the book {\it Mechanizing
  Proof,} winner of the 2003 Merton Book Award of the American
Sociological Association~\cite{Mac}.
% Review by Hayes: http://www.americanscientist.org/template/BookReviewTypeDetail/assetid/12866.
The QED manifesto can be found at~\cite{QED}.
Historical surveys include~\cite{Bled},
%``Automated Theorem Proving after 25 Years,'' 
\cite{Ha07},~\cite{Gor}, and
% A short survey of automated reasoning.
\cite{Mu}.
% Present State of Mechanical Deduction
For something more comprehensive, see~\cite{Ha09}.
% Harrison's book


Several theorem proving systems are extensively documented and are available for download,
including HOL Light~\cite{HOLL}, Isabelle~\cite{Isa}, Coq~\cite{COQ},
 Mizar~\cite{Mizar}, TPS, 
PVS,
ACL2, 
NuPRL, and MetaPRL.
A web-browser version
of Coq allows one to experiment with a proof assistant without
downloading any software~\cite{PW}.

