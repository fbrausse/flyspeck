
\chapter{Fan}\label{sec:fan}




\begin{summary}
This chapter is the final foundational chapter.  The main concept is that of a fan.
A fan is a geometric object that bridges the expanse between sphere packings and hypermaps.  A fan determines a set $V$ of points in $\ring{R}^3$.  Later chapters will interpret the set $V$ as the set of centers of a packing of congruent balls.   The same set $V$ can be interpreted as the set of nodes of a hypermap, or as he set of nodes of a graph.  In fact, a fan might be viewed as  a particular geometric realization of a hypermap.    The main result  is an Euler formula for fans.  This formula  implies that the hypermap of a fan is planar.  The proofs have been carefully organized to avoid any use of the Jordan curve theorem.


A bridge  also connects polyhedra with fans.  This chapter associates a fan with every bounded polyhedron in $\ring{R}^3$ that contains the origin in its interior.  A consequence of this association is the Euler formula for polyhedra.
\end{summary}


\indy{Index}{fan}%
\indy{Index}{hypermap}%
\indy{Index}{planar}%
\indy{Index}{packing}%
\indy{Index}{Jordan curve theorem}%

\section{Definitions}



\begin{remark}
Some may have already encountered other mathematical objects that go by the name of fan.
The definition of fan here is unrelated to its definitions in other mathematical contexts.   In particular, a fan in this book is not a fan from the theory of toric varieties.
\end{remark}

%In this chapter,  a choice of base point $\orz\in\ring{R}^3$ serves as the origin.  It does no harm to assume, in fact, that $\orz =0$.


If $S$ is a set of points,
abbreviate
  $$
  \begin{array}{lll}
  C_\pm(S) &= \op{aff}_\pm(\orz,S),\\
  C^0_\pm(S) &= \op{aff}^0_\pm(\orz,S).\\
  \end{array}
  $$
When the subscript is absent, the subscript $+$ is implied: $C_+(S) = C(S)$, and
so forth.  The parentheses around $S$ are frequently omitted: 
$$C^0\{\v,\w\}=C^0_+(\{\v,\w\}) = \op{aff}^0_+(\{\orz\},\{\v,\w\}).$$

\begin{definition}[fan,~blade]  
Let $(V,E)$ be a pair consisting of a set $V\subset \ring{R}^3$ and a set of pairs of elements of $V$.  The pair is said to be
a {\it fan\/} if the following properties hold.
    \begin{nomerate}
    \item \case{V} $V$ is finite and  nonempty,
    \item \case{origin} $\orz\not\in V$.
    \item \case{independence} If $\{\v,\w\} \in E$, then $\v$ and $\w$ are independent.
    \item \case{intersection}
    For all $\ee ,\ee '\in E \cup \{\{\v\}\mid \v\in V\}$, 
 $$C(\ee )\cap C(\ee ') = C(\ee \cap \ee ').$$
    \end{nomerate}
When $\ee\in E$, call $C^0(\ee)$ or $C(\ee)$ a {\it blade\/} of the fan.
\end{definition}
\indy{Index}{fan}%
\indy{Notation}{V@$( V, E ) $ (fan)}%
\indy{Index}{V@\uncase{V}}%
\indy{Index}{origin@\uncase{origin}}%
\indy{Index}{indepen@\uncase{independence}}%
\indy{Index}{intersection@\uncase{intersection}}%
\bigskip\hbox{~}\bigskip


%Repeat an older version of the definition (in use between May 15 2009 and June 18 2009).
%\begin{definition}[retro-fan]  Let $(\orgz,V,E)$ be a triple consisting of a point,
%a set of
%points, and a set of pairs of elements of $V$.  The triple is said to be
%a {\it retro-fan\/} if the following conditions hold.
%    \begin{itemize}
%    \item $V$ is finite and nonempty.
%    \item $\orz\not\in V$.
%    %\item Each element of $E$ has two elements.
%    \item For each $\v\in V$, the set
%        $$
%        %% WW changed notation from E_\v to E(\v) to allow deformations E_t
%        E(\v) = \{\w\in V\mid \{\v,\w\}\in E\}
%        $$
%        is cyclic with respect to $(\orz,\v)$.
%    \item For each $\ee\in E$, $V\cap C^0(\orz,\ee)=\emptyset$.
%    \item  Sets $\ee,\ee'\in E$ give
%        $$C^0(\orz,\ee) \cap C^0(\orz,\ee')\ne\emptyset\ \Rightarrow (\ee = \ee').$$
%    \item Sets $\v,\v'\in V$ give
%     $$\op{aff}^0_+(\orz,\v) = \op{aff}^0_+(\orz,\v')\ \Rightarrow (\v=\v').$$
%      %% Added condition May 15, 2009./ killed June 18
%    \end{itemize}
%\end{definition}

\begin{lemma}\guid{CTVTAQA}\rating{40}\label{lemma:subset-fan}
If $(V,E)$ is a fan, then for every $E'\subset E$,  $(V,E')$ is also a fan.
\end{lemma}

\begin{proof} This is immediate.
\end{proof}

\begin{lemma}\guid{XOHLED}\rating{ZZ}  Let $(V,E)$ be a fan.
For each $\v\in V$, the set
        $$
        E(\v) = \{\w\in V\mid \{\v,\w\}\in E\}
        $$
        is cyclic with respect to $(\orz,\v)$.
\end{lemma}
\indy{Notation}{E@$E(\v)$ (edge set)}%

\begin{proof}  If $\w\in E(\v)$, then $\v$ and $\w$ are independent.
Also, if $\w\ne \w'\in E(\v)$, then
$$
C\{\v,\w\}\cap C\{\v,\w'\} = C\{\v\}.
$$
This implies that $E(\v)$ is cyclic.
\end{proof}

\begin{remark}\tlabel{rem:fan}\rating{30}
\begin{itemize}
\item The pair $(V,E)$ is a graph with nodes $V$ and edges $E$.  The set
$$\{\{\v,\w\}\mid \w\in E(\v)\}$$ is the set of edges at node $\v$.
There is an evident symmetry:  $\w\in E(\v)$ if and only if $\v\in E(\w)$.   
%
\item
Since $E(\v)$ is cyclic,
each $\v\in V$ has an azimuth cycle $\sigma(\v):E(\v)\to E(\v)$.
It is allowed that $E(\v) = \{\w\}$,  a
\indy{Index}{azimuth cycle}%
singleton set. If so,
$\sigma(\v)$ is the identity map on $E(\v)$.
%
To make the notation less cumbersome, $\sigma(\v,\w)$ denotes the value of the map $\sigma(\v)$ at $\w$.
%
\item 
The property \case{independence} implies that the graph has no loops: $\{\v,\v\}\not\in E$.
%
\item The property \case{intersection} implies that distinct sets $C^0(\ee)$
do not meet.   This property of fans will eventually be related to  the property of planarity of hypermaps.
\indy{Index}{hypermap}%
\indy{Index}{planar}%
%
\end{itemize}
\end{remark}

\begin{remark}[verifying the fan properties]\label{remark:fan-verify}  
We will often be presented with a pair $(V,E)$ and asked to verify that it is a fan.  Here are a few tips about how to verify the fan properties in practice.
\begin{nomerate} 
\item \case{V}  If $V$ is defined as a subset of or as the image of a finite set, then it is evidently finite.   Also, if $V$ is a bounded subset of $\ring{R}^3$ and if there is a minimum distance between distinct points of $V$, then $V$ is finite.  Lemma~\ref{lemma:V-finite} gives the finiteness result when the minimum distance is $2$.
\item \case{origin}  If $(V,E)$ is a fan, then any subset of $V$ inherits the property $\orz\not\in V$ from $V$.  
\item \case{independence} \claim{If the property \case{intersection} is known, then to prove the indepence of $\u$ and $\v$, it is enough to show the strict form of the triangle inequality:}
\begin{equation}\label{eqn:strict-triangle}
\norm{\u}{\v} < \normo{\u} + \normo{\v}.
\end{equation}
Indeed, the strict form of the triangle inequality implies that $\orz\not\in \op{conv}\{\u,\v\}$.  Also,
 the intersection property implies 
 that $C^0\{\v\} \cap C^0\{\u\} = \emptyset$.  These conditions imply linear independence.
The inequality~(\ref{eqn:strict-triangle}) is equivalent to
$$
\arc_V(\orz,\{\u,\v\} < \pi.
$$
\item \case{intersection} The intersection property is generally the most difficult to verify in practice.  If $(V,E)$ is a fan, then the intersection property is inherited by subsets of $V$ and $E$.  Also, note  that
$$
C(\ee\cap \ee') \subset C(\ee) \cap C(\ee')
$$
always holds by elementary geometry.  Hence it is enough to check the reverse inclusion.  Furthermore, if $\ee = \ee'$, then the intersection property is a triviality.  There verification comes down to checking two cases:
$$
C(\ee) \cap C(\ee') = \{\orz\},
$$
when $\ee\cap \ee' = \emptyset$, and
$$
C(\ee)\cap C(\ee') = C\{\v\},
$$
when $\ee\cap \ee' = \{\v\}$.
\end{nomerate}
\end{remark}


Let $(V,E)$ be a fan.  Define a set of darts $D$ to be the union of
two subsets $D_1,D_2$:
    $$
    \begin{array}{lll}
    D_1 &= \{(\v,\w)\mid \{\v,\w\}\in E\}\\
    D_2 &= \{\v \mid \v\in V,\ \ E(\v) = \emptyset\},\\
    D   &= D_1\cup D_2.
    \end{array}
    $$
Darts in $D_2$ are said to be {\it isolated}; and darts in $D_1$ are {\it non-isolated}.
%
\indy{Index}{dart}%
\indy{Index}{dart!isolated}%
\indy{Index}{dart!non-isolated}%
\indy{Index}{isolated}%
\indy{Index}{non-isolated}%
\indy{Notation}{D@$D$ (dart)}%

Define a permutation $n$ on $D_1$ by
    $$n(\v,\w) = (\v,\sigma(\v,\w)).$$
Define a permutation $f$ on $D_1$ by
    $$
    f (\v,\w) = (\w,\sigma(\w)^{-1} \v).
    $$
Define a permutation $e$ on $D_1$ by
    $$
    e (\v,\w) = (\w,\v).
    $$
Define permutations $e,n,f$ on $D_2$ by making them degenerate on $D_2$:
    $$
    e (\v) = n(\v) = f(\v) = \v.
    $$
Set %$\op{hyp}_r(V,E)=(D_1,e,n,f)$ and 
$\op{hyp}(V,E)=(D,e,n,f)$. %for %  them the non-isolated hypermap
%the hypermap associated with $(V,E)$.  
The next lemma shows that $(D,e,n,f)$ is indeed a hypermap.



\begin{lemma}\guid{AAUHTVE}\rating{70}
Let $(V,E)$ be a fan.  Let $D = D_1\cup D_2$
and $\op{hyp}(V,E) = (D,e,n,f)$, as constructed above.  Then
    \begin{itemize}
    \item $\op{hyp}(V,E)$ is a plain hypermap.
    \item  $e$ has no fixed
points in $D_1$.
    \item  $f$ has no fixed points in $D_1$.
    \item For every pair of distinct nodes, there is at most one
    edge meeting both.
    \item The two darts of an edge of $D_1$ lie at different nodes.
    \end{itemize}
\indy{Notation}{hyp@$\op{hyp}$ (hypermap)}%
\end{lemma}

\begin{proof}  Plainness is an elementary calculation:
    $$
\begin{array}{lll}
e(n(f(\v,\w))) &= e(n(\w,\sigma(\w)^{-1} \v))) &=
        e(\w,\v)\\ 
&= (\v,\w).
\end{array}
$$
So $\op{hyp}(V,E)$ is a hypermap. Compute
    $$e(e(\v,\w)) = e(\w,\v) = (\v,\w).$$
A fixed point in $D_1$ under $e$ would force $\v = \w\in E(\v)$,
but by construction $\v\not\in E(\v)$.  The argument that $f$ has no
fixed points is similar.

The next step is to show that for every pair of distinct nodes, there is at most one edge
meeting both.
That is,
        $$(n^k e x = e n^\ell x)\Rightarrow (n^\ell x = x).$$
Let $x = (\v,\w)\in D_1$.  Let $\sigma=\sigma(\v)$. Then
    $$
    \begin{array}{lllllll}
    n^\ell x &= (\v,\sigma^\ell \w)\\
    e n^\ell x &= (\sigma^\ell \w,*)\\
    e x &= (\w,*)\\
    n^k e x &= (\w,*)\\
    n^k e x &= e n^\ell x \\&\ \Rightarrow (\w = \sigma^\ell \w) \\&\ \Rightarrow
    (n^\ell x = (\v,\w) = x)
    \end{array}
    $$

Finally,  each dart of an edge lies on a different node.
That is, $e x \ne n^k x$, for $x\in D_1$.   In detail:
    $$
    \begin{array}{lll}
        e(\v,\w) &= (\w,*),\quad \w\in E(\v)\\
        n^k(\v,\w) &= (\v,*),\quad \v\not\in E(\v).
    \end{array}
    $$
The result follows.
\end{proof}

\section{Topology}\label{sec:topology}

\subsection{basics}

%There is hardly any topology that comes up in this book.  Most of
%what is needed appears in this chapter.  
This chapter uses some basic
notions in topology such as continuity, connectedness, and compactness.

\begin{remark} The term {\it ``connected''} is now being used in
two different senses: in the topological sense and in a combinatorial
sense for hypermaps.    To reduce the confusion, this book calls the connected components
of a topological space {\it ``topological components''} and the connected
components of a hypermap {\it ``combinatorial components''}.
\end{remark}
\indy{Index}{connected}%
\indy{Index}{connected!topological component}%
\indy{Index}{connected!combinatorial component}%


Basic facts about the topology of Euclidean space will be assumed.  In particular,
the set $\ring{R}^3$ is a metric space under the
Euclidean distance function $d(\v,\w) = \norm{\v}{\w}$.  Every subset of
$\ring{R}^3$ is a metric space under the restriction of the metric
$d$ to the subset.  A subset carries the metric space topology.
In particular, $$S^2 = \{ \v \mid \normo{ \v} = 1\},$$ the unit sphere in
$\ring{R}^3$ centered at $\orz$, is a metric space and a topological space.
\indy{Index}{metric space} %

If $Y$ is an open set in $\ring{R}^3$, write
$\comp{Y}$ for its set of topological components.
\indy{Notation}{1@$\comp{Y}$ (topological components)}%
The family of topological components of $Y$ has the following properties:
the members are pairwise disjoint, nonempty, connected open sets; and the
union of the family is all of $Y$.
Conversely, any family with these properties must be the
family of topological components of $Y$.
If two
points in $\ring{R}^3$ 
can be joined by a continuous path in $Y$,
then the two points lie in the same topological component of $Y$.
\indy{Index}{path connected}%
\indy{Index}{connected!topological component}%
\indy{Notation}{1@$\comp{Y}$ (topological components)}%
 






\subsection{topological component and dart}

Let $(V,E)$ be a fan and let $(D,e,n,f) = \op{hyp}(V,E)$
be the associated hypermap.  
\indy{Index}{fan}%
\indy{Index}{dart}%
\indy{Index}{hypermap}%

\begin{definition}[X,~Y]\label{def:XY}
Let $(V,E)$ be a fan.  Let $X=X(V,E)$ be the union of the
blades
   $$C(\ee)$$
as $\ee$ ranges over $E$.  Let $Y=Y(V,E)$ be the complement
$Y = \ring{R}^3\setminus X$.
\indy{Notation}{X@$X$ (fan)}%
\indy{Notation}{Y@$Y$ (fan)}%
\end{definition}


A wedge $\Wdart(x)$, a subset $\Wdart(x,\epsilon)$,
and an azimuth angle $\op{azim}(x)$ are associated with
with each dart $x=(\v,\w)\in D$.  Define 
\indy{Index}{wedge}%
\indy{Index}{angle!azimuth}%
\indy{Index}{azimuth}%
\indy{Index}{dart}%
\indy{Notation}{Wdart@$\Wdart$ (wedge)}%
\indy{Notation}{azim}%
$$
\Wdart(x)=
\begin{cases} 
W^0(\orz,\v,\w,\sigma(\v,\w)),&\text{if }\card(E(\v))>1,\\
\ring{R}^3\setminus \op{aff}_+(\{\orz,\v\},\w),&\text{if } E(\v) = \{\w\},\\
\ring{R}^3\setminus \op{aff}\{\orz,\v\},&\text{if } E(\v) = \emptyset.\\
\end{cases}
$$
Define
$$
\bWdart(x) = 
\begin{cases} 
W(\orz,\v,\w,\sigma(\v,\w)),&\text{if }\card(E(\v))>1,\\
\ring{R}^3,&\text{otherwise }.\\
\end{cases}
%\{ \u\mid 0\le \op{azim}(\orz,\v,\w,\u)\le \op{azim}(\orz,\v,\w,\sigma(\v,\w)) \}.
$$
\indy{Notation}{Wdart@$\bWdart$ (closure of $\Wdart$)}%
($\bWdart(x)$ is the closure of $\Wdart(x)$.)


Define $\op{azim}(x)$ as the azimuth angle of $\Wdart(x)$:
\indy{Notation}{azim}%
$$
\op{azim}(x)=\begin{cases}
\op{azim}(\orz,\v,\w,\sigma(\v,\w)), &\text{if } \card(E(\v)) > 1,\\
2\pi, & \text{otherwise.}\\
\end{cases}
$$
For any $x = (\v,\ldots)\in D$, set
    $$
    \Wdart(x,\epsilon) = \Wdart(x) \cap \op{rcone}^0(\orz,\v,\cos\epsilon).
    $$

%All the hypermaps in this book are connected. $D_2$ is not needed.


\begin{lemma}\guid{VBTIKLP}\tlabel{lemma:disjoint}\rating{120}
Let $(V,E)$ be a fan with hypermap $(D,e,n,f)$.  Let $N$ be a node of the hypermap.  There exists $\v\in V$
such that the darts of $N$ are precisely
the darts of the form $(\v,\ldots)$.  Furthermore, there is a 
disjoint sum decomposition of $\ring{R}^3$ given by
  $$
  \ring{R}^3 = 
  \op{aff}\{\orz,\v\} \cup
  \bigcup_{x\in N} \Wdart(x)  \cup 
  \bigcup_{\{\v,\w\}\in E} \op{aff}_+^0(\{\orz,\v\},\w).
  $$
\end{lemma}
\indy{Index}{disjoint sum decomposition}%
\indy{Index}{hypermap}%
\indy{Index}{dart}%

\begin{definition}
Under the identification of nodes of $D$ with $V$,
write $\v(x)\in V$ for the vertex corresponding to a dart $x\in D$. 
\end{definition}

\begin{proof}
The proof begins with the existence of the disjoint sum decomposition.
First of all, $\ring{R}^3$ is the disjoint union of $\op{aff}\{\orz,\v\}$
and its complement.

The case when $\card(E(\v))\le 1$ follows immediately from the definitions.  
Therefore assume  that $\card(E(\v)) >1$.
Fix $\u$ such that $\{\v,\u\}\in E$, and let $\sigma$ be the azimuth
cycle on $E(\v)$.  Let $\alpha(i)=\op{azim}(\orz,\v,\sigma^i \u,\sigma^{i+1}\u)$.   By Lemma~\ref{lemma:2pi-sum}, the sum of the angles $\alpha(i)$ is $2\pi$.  Every $\p\in\ring{R}^3\setminus\op{aff}\{\orz,\v\}$ satisfies
$$
\sum_{i=0}^j \alpha(i) <
\op{azim}(\orz,\v,\u,\p) < \sum_{i=0}^{j+1} \alpha(i).
$$
or 
$$
\sum_{i=0}^j \alpha(i) = \op{azim}(\orz,\v,\u,\p)
$$
for a unique $0 \le j < n$, where $n$ is the cardinality of $E(\v)$. 
These conditions are exactly the membership conditions for the sets
$
\Wdart(\v,\sigma^j \u)
$
and $\op{aff}_+^0(\{\orz,\v\},\sigma^j \u)$, respectively.
The result follows.
\end{proof}

\begin{corollary}\guid{IBZWFFH}\tlabel{cor:W}\rating{40}
Let $x = (\v,\ldots)$ be a dart and $\w\in E(\v)$.
Then $\Wdart(x)\cap C\{\v,\w\}=\emptyset$.
\end{corollary}

\begin{proof} The decomposition of Lemma~\ref{lemma:disjoint} is
disjoint.  It follows directly from the definitions that
   $$C\{\v,\w\}\subset \op{aff}_+^0(\{\orz,\v\},\w) \cup 
    \op{aff}\{\orz,\v\}.$$
\end{proof}

\begin{lemma}\guid{JGIYDLE}\rating{120} 
For each dart $x$, and $\epsilon$ sufficiently small and positive,
$\Wdart(x,\epsilon)$ is nonempty and lies in a single 
topological component of $Y(V,E)$.
\end{lemma}
\indy{Index}{connected!topological component}%
\indy{Notation}{Wdart@$\Wdart$ (dart)}%

\begin{proof}  (Beware of the notational subtleties: $\epsilon\in\ring{R}$ is not $\ee\in E$.)  The proof first shows that $\Wdart(x,\epsilon)$ lies in $Y$,
for $\epsilon$ small.  Let $x=(\v,\w)\in D_1$.  
Let $S^2$ be the unit sphere centered at $\orz$.
By making $\epsilon$ small enough,
the sets $\Wdart(x,\epsilon)\cap S^2$
avoid the compact sets $C(\ee)\cap S^2$ when $\v\not\in \ee$.
Thus, $\Wdart(x,\epsilon)$ also avoids $C(\ee)$ when $\v\not\in \ee$.
By Corollary~\ref{cor:W}, $\Wdart(x,\epsilon)$ avoids $C(\ee)$, when $\v\in \ee$.
Thus, $\Wdart(x,\epsilon)\subset Y$, for $\epsilon$ small.

To complete the proof, it is enough to show that each $\Wdart(x,\epsilon)$ is
connected.  
The  set
   $$
    (0,\infty) \times (\theta_1,\theta_2) \times (0,\epsilon)
   $$
is connected.
The set $\Wdart(x,\epsilon)$  is the image of this product
under a spherical coordinate representation (Definition~\ref{def:sph}).
\indy{Index}{spherical coordinates}%
It is readily verified that the spherical to Cartesian coordinate transformation is
a continuous map. As the image of a connected set under a continuous map, $\Wdart(x,\epsilon)$ is connected.
\end{proof}
\indy{Index}{connected}%
\indy{Index}{polar coordinates}%

\begin{definition}[leads~into] For each dart $x$, 
there is then a well-defined topological
component $U_x$ of $Y(V,E)$ 
that contains $\Wdart(x,\epsilon)$ (for all
sufficiently small positive $\epsilon$). Say the dart {\it leads into}
$U_x$.
\end{definition}
\indy{Index}{connected}%
\indy{Notation}{U@$U_x$ (connected component)} %


\section{Planarity}


\subsection{face attributes}


\begin{lemma}[sweep]\guid{DHVFGBC}\rating{400}\label{lemma:sweep}  
Let $(V,E)$ be a fan with hypermap $(D,e,n,f)$.  
Suppose that $\op{azim}(x)<\pi$
for all darts $x\in D$.  Fix a dart $x\in D$.
Let $\v = \v(x)$, $\v_0 = \v(f x)$,
and $\v_1 = \v(f^2 x)$.  Let $\w(t) = (1-t) \v_0 + t \v_1$, for
$0\le t\le 1$.  Then
\begin{itemize}
\item $\v$ and $\w(t)$ are independent for each $t\in[0,1]$.
\item If $0 < t \le 1$, and $C^0(t)$ meets $X$, then $t=1$ and $\{\v,\v_1\}\in E$.
%\item For $0< t < 1$, it follows $C^0(t)\subset Y$.
%item If $\{\v,\v_1\}\in E$ (that is, if the face of $x$ is a triangle), 
%then $C^0(1)$ is the blade $C^0\{\v,\v_1\}$ of the fan.
%\item If $\{\v,\v_1\}\not\in E$, then $C^0(1)\subset Y$.
\end{itemize}
\end{lemma}


\begin{figure}[htb]
  \centering
  \szincludegraphics[width=50mm]{\pdfp/vt.eps}
  \caption{Adding an edge to the fan.}
  \label{fig:vt}
\end{figure}


\begin{proof}
Abbreviate $C^0(t) = C^0\{\v,\w(t)\}$.
Let $Y = Y(V,E)$ and $X = X(V,E)$.
It follows from the definition of a fan that $\{\v,\v_0\}\in E$ and
that $\v$ and $\v_0$ are independent.  By continuity, $\v$ and $\w(t)$
are independent when $t$ is positive and sufficiently small.  
%If it exists, set $t'$ as
%the smallest $t>0$ for which $\{\orz,\v,\w(t)\}$ is a collinear set.  If it exists, let $I=\{t\mid 0< t < t'\}$, otherwise let $I=\{t\mid 0 < t \le 1\}$.  For each $t\in I$, the blade $C^0(t)$ is not a collinear set and is contained in a unique plane $A(t)$ through $\orz$ and $\v$.
Let $I\subset\leftopen0,1\rightclosed$ be an interval that contains $(0,\epsilon)$ for some sufficiently small positive $\epsilon$, and such that $\v$ and $\w(t)$ are independent for all $t\in I$.

\claim{If $t\in I$ and $C^0(t)$ meets $X$, then $t=1$ and $\{\v,\v_1\}\in E$.} 
Indeed,  a consideration of possible intersections with vertices $\w\in V$ and blades
$C^0(\ee)\subset X$ shows that for $t>0$ sufficiently small,
$C^0(t)$ does not meet $X$, hence $C^0(t)\subset Y$.  Assuming 
that $C^0(t)$ meets $X$ for some $t\in I$, let $a$
be the smallest such $t\in I$.
$C^0(a)$ cannot meet $X$ at a vertex $\w\in V$, because $\op{azim}(x)<\pi$ whenever $\w=\v(x)$, 
which means that 
%for any $t$ for which  $C^0(t)$ meets $\w$ 
there is a smaller $t<a$ for which $C^0(t)$ meets a blade at $\w$.
Thus, $C^0(a)$ first meets $X$ along a blade $C^0(\ee)$. If
the intersection with this blade is transversal, again one can find a smaller $t$ that
gives an intersection with the blade.  Hence, 
$C^0(a)$ and $C^0(\ee)$ are coplanar.  From the disjointness
properties of blades of a fan, it follows that $\ee = \{\v,\v_1\}\in E$,
that $a=1$, and that $C^0(1)$ is a blade of the fan.  The claim follows.

Now assume for a contradiction that $\v$ and $\w(t)$ are dependent for some $t\in\leftopen0,1\rightclosed$.  Let $b\in I$ be the least such constant.  
Pick $0<a<b$.  Then
$\{\orz,\w(a),\w(b),\v\}$ lie in a unique plane $A$.  Since all $\w(t)$
lie in a line,  $\w(t)\in A$ all $t\in I$.  Then $\v_0\in C^0(a)\cap X$,
contradicting the established disjointness of $X$ from $C^0(a)$.  Thus, $b$ does not exist.  This proves the first conclusion of the lemma.  

Now $I= \{t\mid 0 < t \le 1\}$ and the second conclusion of the lemma follow immediately from the claim.
\end{proof}

\begin{lemma}\guid{RWXUYZZ}\rating{100} \label{lemma:UF}
Let $(V,E)$ be a fan with hypermap $(D,e,n,f)$.   Assume that $\op{azim}(x)<\pi$ for all darts $x\in D$. Then for every face $F$ of the hypermap, there exists a topological component $U$ of $Y(V,E)$ such that for every $x\in F$, the dart $x$ leads into $U$. 
\end{lemma}
\indy{Index}{connected!topological component}%
\indy{Index}{hypermap}%
\indy{Index}{dart}%

Write $F\mapsto U_F$ for this map from faces to topological components.
\indy{Notation}{UF@$U_F$}%

\begin{proof}  Fix any dart $x\in F$ and construct the set $C^0(t)$ as
in the previous lemma.  
For all $\epsilon>0$
sufficiently small, there exists $\delta>0$ such that set $C^0(t)$ meets
both $\Wdart(x,\epsilon)$ and $\Wdart(f x,\epsilon)$ for all $0<t<\delta$.  
By the previous lemma, the set $C^0(t)$ lies in a single
component $U$ when $t$ is positive and sufficiently small.  
Thus,
$x$ and $f x$ lead into the same component $U$.  By induction, for all
$y\in F$, the dart $y$ leads into $U$.
\end{proof}

\begin{lemma}\guid{DWWUTKW}\rating{ZZ}\label{lemma:add-edge}
Let $(V,E)$ be a fan and let $\v,\w\in V$ be independent.   Suppose that
$C^0\{\v,\w\}\subset U_F$ for some face $F$.  Let $E' = E\cup\{\{\v,\w\}\}$.  Then
$(V,E')$ is a fan.
\end{lemma}

\begin{proof} Each of the defining properties of a fan will be established in turn.
The vertex set is unchanged.  It remains finite and nonempty, and it does not contain $\orz$.  The property \case{independence} of $E'$ follow from the corresponding property of $E$
and the assumed independence for $\{\v,\w\}$.  

In the verification of the intersection property:
$$
C(\ee)\cap C(\ee') = C(\ee \cap \ee'),
$$
it is enough to consider the case $\ee = \{\v,\w\}$ and $\ee' \ne \ee$, the other cases being trivial.  Then from the known facts $C\{\v,\w\} = C^0\{\v,\w\}\cup C\{\v\}\cup C\{\w\}$, $C^0\{\v,\w\}\subset U_F$,   $C(\ee')\subset X(V,E)$, and $X(V,E)\cap U_F=\emptyset$, it follows that
$$
\begin{array}{rll}
C\{\v,\w\} \cap C(\ee')  &= (C\{\v\} \cup C\{\w\}) \cap C(\ee') \\
  &= C(\{\v\}\cap \ee') \cap C(\{\w\}\cap \ee')\\
  &= C(\{\v,\w\}\cap \ee').
\end{array}
$$
(The last equality uses the observation that at most one of the two intersections $\cdot\cap\ee'$ in the penultimate line is nonzero.)
Thus, $(V,E')$ is a fan.
\end{proof}


\begin{lemma}\guid{JUTSTKG}\rating{120}\label{lemma:lead-exists}
Let $(V,E)$ be a fan with hypermap $(D,e,n,f)$. Let $Y=Y(V,E)$. Assume that $\op{azim}(x)<\pi$ for all darts $x$.  For every topological component $U$ of $Y$, there is a dart $x\in D$ that leads into $U$.
\end{lemma}
\indy{Index}{connected!topological component}%
\indy{Index}{fan}%
\indy{Index}{hypermap}%

\begin{proof}  
In the notation of Lemma~\ref{lemma:sweep}, to show the dependence of the sets $C^0(t)$ on the initial dart $x$, write $C^0(t,x)$.

Let $\p\in U$.  Choose a continous path $\varphi:[0,1]\to \ring{R}^3\setminus\{0\}$
such that $\varphi(t)\in U$ for $t<1$ and $\varphi(1)\not\in U$.  Then
$\q=\varphi(1)\in X$.  If $\q\in C^0\{\v\}$ for some $\v\in V$,
then there exists a dart $x$ with node $\v = \v(x)$ such that for all sufficiently small positive $\epsilon$, there exists some $0\le t < 1$ such that  $\varphi(t)\in \Wdart(x,\epsilon)\subset U_x$.  
Thus, $x$ leads into $U$.
\indy{Notation}{ZZddgamma@$\varphi$ (path)}%

The other possibility is that
$\q\in C^0\{\v,\w\}$ for some $\{\v,\w\}\in E$.  In this case, there is a unique
edge $\{x,y\}$ of the hypermap such that $\v=\v(x)$ and $\w=\v(y)$.   (That is, $x=(\v,\w)$ and $y=(\w,\v)$.)
There
is also a small neighborhood of $\q$ such that every point $\q'$ in that neighborhood
takes one of the following forms:
\begin{itemize} \item $\q'\in C^0\{\v,\w\}$;
\item $\q'\in C^0(s,x)\subset U_x$ for some $0<s<1$;
\item $\q'\in C^0(s,y)\subset U_y$ for some $0<s<1$.
\end{itemize}
Points of the first form do not meet $Y$.  Thus,  $\varphi(t)\in U_x$ or $\varphi(t)\in U_y$.  
\end{proof}

\begin{lemma}[triangle~attributes]\guid{KVQWYDL}\rating{200} \label{lemma:triangle}
Let $(V,E)$ be a fan with hypermap $(D,e,n,f)$. 
Let $Y=Y(V,E)$.
Assume that $\op{azim}(x)<\pi$
for all darts $x\in D$.  Fix a face $F$ of cardinality three, fix
$x_0\in F$, and set $x_i = f^i x_0$. Then
\indy{Index}{triangle attributes}%
\begin{itemize}  
\item $U_F$ is equal to the intersection of the three half-spaces:
$$A^0_+(i)=\op{aff}_+^0(\{\orz,\v(x_{i+1}),\v(x_{i+2})\},\v(x_i)),\quad i=0,1,2.$$
\item If a dart $y$ leads into $U_F$, then $y\in F$.
\end{itemize}
\end{lemma}
\indy{Index}{half-space}%

\begin{proof} The intersection of two half-spaces, $A^0_+(1)\cap A^0_+(2)$ is
the wedge $\Wdart(x_0)$.   The sets $C^0(t,x_0)\subset\Wdart(x_0)$ sweep out precisely
the intersection of $\Wdart(x_0)$ with $A^0_+(0)$.  The sets $C^0(t,x_0)$ belong to
$U_F$.  Hence the intersection $U'$ of the three half-spaces is a subset of $U_F$.

Suppose for a contradiction 
that $\p$ is a point of $U_F$ that does not belong to $U'$.  Choose a continuous path $\varphi:[0,1]\to U_F$ with $\varphi(0)\in U'$ and $\varphi(1)=\p$.  Let $t>0$ be the first time such that $\varphi(t)\not\in U'$.  Then $\q=\varphi(t)$ lies in the set consisting of the closed intersection of half-spaces $A_+(i)$ corresponding to $A^0_+(i)$ and lies
in one of the bounding planes.  Let 
$$
X' = \bigcup C(i),\quad\text{ where } C(i)=C\{\v(x_i),\v(x_{i+1})\}.
$$
Then $\q\in X'\subset X$.  This yields the impossibility:
$\q\in X\cap Y = \emptyset$.   Thus, $U'=U_F$.

Let $y$ be any dart that leads into $U_F$ at vertex $\v(y)$.  Then
$\Wdart(y,\epsilon)$ meets $U_F$ for all $\epsilon>0$ sufficiently small.
This implies that $\v(y)$ lies in the intersection of the closed half-spaces $A_+(i)$.  As previously established, this intersection is the disjoint union of $U_F$ and
$X'$.  As $\v(y)\in X$, which does not meet $U_F$, it follows that $\v(y)\in X'$.
The set $X'$ is the disjoint union of the rays $C\{\v(x_i)\}$ and
the three blades $C^0(i)$.  These blades do not meet $V$, hence
$\v(y)=\v(x_i)$ for some $i$.  Thus, $y$ and $x_i$ belong to the same
node.  The sets $\Wdart(y)$ and $\Wdart( x_i)$ are disjoint for distinct darts at the same
node, and this implies that $y=x_i\in F$.
\end{proof}

\begin{corollary}\guid{MOZNWEH}\rating{60}\label{lemma:girard-component}
Let $F$ be a face of size $3$ in the context of Lemma~\ref{lemma:triangle}.  Then for $r>0$, $U_F \cap B(\orz,r)$ is measurable and $r$-radial at $\orz$.
The solid angle of $U_F$ is given by the formula
$$
\sol(U_F) = -\pi + \sum_{x\in F}\op{azim}(x),
$$
\end{corollary}
\indy{Index}{angle}%
\indy{Notation}{F@$F$ (face)}%
\indy{Notation}{U@$U_F$ (component)}%

\begin{proof} An intersection of half-spaces through the origin 
with $B(\orz,r)$ is measurable and
$r$-radial.  The solid angle is given by Girard's formula for
a spherical triangle (Lemma~\ref{lemma:prim-volume}).
\end{proof}
\indy{Index}{Girard's formula}%

\begin{lemma}[face attributes]\guid{PIIJBJK}\rating{1000}\label{lemma:face}
Let $(V,E)$ be a fan with hypermap $(D,e,n,f)$. 
Assume that $\op{azim}(x)<\pi$
for all darts $x\in D$.  Then
\begin{nomerate}
\item \case{bijection} The map $F\mapsto U_F$ is a bijection between faces of the hypermap
and topological components of $Y$.
\item \case{half-space} The topological component $U_F$ is the intersection of the open
half-spaces $\op{aff}_+^0(\{\orz,\v(x),\v({f x})\},\v(f^2 x))$, as $x$ runs
over $F$.
\item \case{solid angle} For every $F$, the intersection $B(\orz,r)\cap U_F$ is measurable and
eventually radial at $\orz$.  The solid angle of $U_F$ is given by the
formula
$$
\sol(U_F) = 2\pi + \sum_{x\in F}(\op{azim}(x)-\pi).
$$
\item \case{diagonal}  If $x,y\in F$ are distinct, with corresponding vertices $\v(x),\v(y)\in V$, then
$\v(x)$ and $\v(y)$ are independent.
Furthermore, 
either $x,y$ are adjacent under the face map, or $C^0\{\v(x),\v(y)\}\subset U_F$.  {\it That is, the ``diagonals'' of the polygon $U_F$ are all interior.}
%%\item  {\bf [triangulation]~} Triangulations of $U_F$ exist.  More precisely,
%there is a fan $(V,E')$ such that $E\subset E'$ and such that every face of
%$\op{hyp}(V,E')$ has cardinality $3$.
\end{nomerate}
\end{lemma}

This lemma has several significant corollaries.  The corollaries all hold under the assumptions of the lemma: $(V,E)$ is a fan such that $\op{azim}(x)<\pi$ for all darts $x$ in its hypermap.  The proof of Lemma~\ref{lemma:face} appears after the corollaries.

\begin{corollary}\guid{GINGUAP}\rating{40}
Each topological component $U_F$ is convex.
\end{corollary}
\indy{Index}{component!topological}%
\indy{Index}{convex}%

\begin{proof} It is the intersection of half-spaces.
\end{proof}

\begin{corollary}\guid{SRPRNPL}\rating{60}  
The hypermap of the fan $(V,E)$ is simple.
\end{corollary}
\indy{Index}{hypermap!simple}%

\begin{proof}  Let $x\in F$.  By the intersection of half-spaces property, $U_F$ is contained in the wedge $\Wdart(x)$ at $x$.  If there is a second dart $y$ at the same node in $F$, then $U_F$ is also contained in $\Wdart(y)$. However, by Lemma~\ref{lemma:disjoint}, the wedges at a given node are disjoint.
\end{proof}

\begin{corollary}\guid{WGVWSKE}\rating{150}  
The hypermap of the fan $(V,E)$ is connected.
\end{corollary}
\indy{Index}{connected}%
\indy{Index}{hypermap}%
\indy{Index}{component!combinatorial}%

%\begin{proof} Let $x,y$ be any two darts.  After replacing $x$ with $f x$ if necessary (which does not change the combinatorial component)
%assume that $\{\orz,\v(x),\v(y)\}$ is not a collinear set. 
%For each blade $C^0(\ee)$ of the fan that meets $C=C^0\{\v(x),\v(y)\}$
%pick one of the two endpoints of $\ee$.  This gives a sequence
%$$
%\v(x)=\v_0,\v_1,\ldots,\v_k=\v(y)
%$$
%such that $C^0(\{\v_i,\v_{i+1}\})$ lies in a single topological component $U_i$.  Each $U_i$ has the form $U_F$ for some face $F=F_i$ of the hypermap.
%Thus,  a combinatorial path is constructed from $x$ to $y$ by moving by the face map from dart to dart within each $F_i$ and by the node map from dart to dart around a given node $\v_j$.
%\end{proof}

\begin{proof}  Let $[D]$ denote the set of combinatorial components of $D$.
There is a well-defined continuous (in fact, locally constant) function from $Y$ onto $[D]$ given as follows.  For $\p\in Y$, choose $F$ such that $\p\in U_F$ and send $\p$ to the class of $F$ in $[D]$.  By Lemma~\ref{lemma:face} \case{bijection}, this map is well-defined. This map extends continuously to $C^0(\ee)$, for $\ee\in E$, by sending $\p\in C^0(\ee)$ to the the combinatorial component of $D$ containing the edge $\{x,y\}$ of the hypermap that corresponds with $\ee$.  The domain
$$
Y\cup \bigcup C^0(\ee)
$$
is connected.  The continuous map from this connected set onto the discrete set is necessarily constant.  As the map is onto, the set $[D]$ reduces to a singleton.
\end{proof}

\begin{corollary}\guid{GGRLKHP}\rating{100}  
The hypermap of the fan $(V,E)$ is planar.
\end{corollary}
\indy{Index}{planar}%
\indy{Index}{hypermap}%

\begin{proof}  The solid angle of a sphere is $4\pi$.  The set $X(V,E)$
has measure zero, so that
\begin{equation}\label{eqn:solid-sum}
4\pi = \sol(Y)= \sum_F \sol(U_F) = 
\sum_F ( 2\pi + \sum_{x\in F} (\op{azim}(x)-\pi) ).
\end{equation}
The double sum over faces and darts in a face can be replaced by
a single sum over all darts.  
The sum of the azimuth angles of all darts at a node is $2\pi$. Thus,
all the azimuth angle terms give $2\pi\,\#n$.
Thus, the formula~(\ref{eqn:solid-sum}) becomes
$$
4\pi = 2\pi\, \#f +2\pi\,\#n - \pi\, \#D.
$$
In a plain hypermap in which the edge map has no fixed points, $\#D = 2\,\#e$.
The relation becomes
$$
2 + \#D = \#f + \#e + \#n.
$$
This is the condition of planarity for a connected hypermap.
\end{proof}
\indy{Index}{hypermap!connected}%

\subsection{proof of face attributes}

Now turn to the proof of the face-attribute lemma (Lemma~\ref{lemma:face}).  The proof breaks into a series of small lemmas.  The primary proof method is an induction on the following invariant of a fan $(V,E)$.  If $(V,E)$ is a fan,  let $N(V,E)$ be the natural number
$$
\sum_F (k_F - 3),
$$
where the sum runs over faces of the  hypermap, and $k_F$ is the cardinality of the face $F$.
%We prove the conclusion of the lemma, together with the additional
%conclusion:
%\begin{itemize}
%\item If $C^0(\ee)$ is any diagonal of $U_F$ (with $\ee\not\in E$), then the %fan $(V,E'')$, where $E'' = E\cup\{\ee\}$, satisfies
%$N(V,E'')+1 = N(V,E)$.
%\end{itemize}

\begin{lemma}\guid{DWFBRQY}\rating{ZZ} Lemma~\ref{lemma:face} holds under the additional assumption that $N(V,E) = 0$.
\end{lemma}
\indy{Index}{face!attribute}%
\indy{Notation}{N@$N$ (face)}%
\indy{Notation}{kfz@$k_F$ (cardinality of face)}%

\begin{proof}
If $N(V,E)=0$, then the hypermap is a triangulation.  By Lemmas~\ref{lemma:UF} and \ref{lemma:lead-exists}, every topological component of $Y$ has
the form $U_F$ for some face $F$.  By Lemma~\ref{lemma:triangle}, $U$ uniquely determines the face $F$.  Thus, there is a bijection between faces of the hypermap and topological components.  By Lemma~\ref{lemma:triangle}, the topological component $U_F$ is the intersection of open half-spaces as asserted.  The solid angle formula is given by Corollary~\ref{lemma:girard-component}.  The assertion of the lemma about diagonals 
%and triangulations 
is trivial for a hypermap that is already a triangulation. This completes the proof in the base case $N(V,E)=0$.
\end{proof}
%\indy{Index}{triangulation}%

When $N(V,E)>0$,  there exists a face $F$ of the hypermap that is not a triangle.  By Lemma~\ref{lemma:sweep}, there are $x,y\in F$ such that $C^0\{\v,\w\}\subset U_F$, where $\v = \v(x)$ and $\w = \v(y)$. Form a new fan $(V,E')$ on the same vertex set with
$E' = E\cup \{\{\v,\w\}\}$.   (See Lemma~\ref{lemma:add-edge}.)  The construction of $E'$ depends on choices: $F$, $x\in F$, and $y\in F$.   The following notation is useful.  Add primes to symbols denoting objects related to $(V,E')$.  
%Write
%$\v=\v(x)$ and $\w=\v(y)$.  
Let $x',y'\in D'$ be the darts such that $\{x',y'\}$ is the edge in the hypermap corresponding to $\{\v,\w\}\in E'$.  In other words, $x'=(\v,\w)$ and $y'=(\w,\v)$, $x=(\v,\cdot)$ and $y=(\w,\cdot)$. The darts $x',y'$ lead into topological components $U(x')$ and $U(y')$ of $Y'=Y(V,E')$  and belong to faces $F(x')$, $F(y')$ of $H'=\op{hyp}(V,E')$.

\begin{lemma}\guid{ZSZIUQE}\rating{ZZ} 
Assume that $N(V,E)>0$.  Let $E'=E\cup \{\{\v,\w\}\}$, as above.  Assume that $F(x')$  is a triangle. Then $N(V,E')<N(V,E)$.
\end{lemma}


\begin{proof}    The hypermap $H=\op{hyp}(V,E)$ is obtained from $H'=\op{hyp}(V,E')$ by a double walkup transformation on the edge $\{x',y'\}$.    The faces $F(x')$ and $F(y')$ are distinct (by Lemma~\ref{lemma:triangle}, which asserts that $y'$ does not lead into $U(x')$).  Thus, the walkup transformation merges two faces.   Then 
$$N(V,E) - N(V,E') = ((k+1)-3) ~~-~~ ((k-3) + (3-3)) = 1 >0,$$
where $k$ is the cardinality of $F(y')$.
\end{proof}

In the proof of Lemma~\ref{lemma:face},
assume for a contradiction that there exists a fan $(V,E)$ 
satisfying the assumptions of the lemma, but not the conclusion.
Among all such counterexamples with fixed vertex set $V$,  pick
$E$ to minimize  $N(V,E)$.  Refer to this fan as a minimal counterexample (to
Lemma~\ref{lemma:face}).  

By the assumed minimality of $N(V,E)$, the conclusions of Lemma~\ref{lemma:face} holds for the
modified fan $(V,E')$.  The strategy of the proof of Lemma~\ref{lemma:face} will
be to use the modified fan $(V,E')$ to show that the conclusions of the Lemma~\ref{lemma:face} hold for $(V,E)$ as well.   These conclusions will be established for $(V,E)$ through a series of lemmas.  The result will contradict the assumption that
$(V,E)$ is a counterexample.

\begin{lemma}\guid{OBHTHCD}\rating{ZZ}
The conclusion \case{bijection} of Lemma~\ref{lemma:face}  holds for any minimal counterexample:
$F\mapsto U_F$ is a bijection
\end{lemma}

\begin{proof}
Assume that $F(x')$ is a triangle.
The two faces $F(x')$ and $F(y')$ merge into a single face $F$ of $\op{hyp}(V,E)$.
Then 
\begin{equation}\label{eqn:U}
U= U(x')\cup U(y')\cup C^0(\ee)
\end{equation} 
is a connected open set in $Y$.
If $F'\ne F(x'),F(y')$ is any other face in $H'$, then $U_{F'}$ is
a connected open set in $Y$.  Moreover, the set $U$ and sets $U_{F'}$
are pairwise disjoint and exhaust $Y$, so that they are precisely the topological
components of $Y$.  Some dart of $F$ leads into $U$, so $U=U_F$.  It follows
that the number of faces is equal to the number of topological components, so that the map $F\mapsto U_F$ is a bijection.
\end{proof}


\begin{lemma}\guid{TXFBALB}\rating{ZZ}  The conclusion \case{solid angle} of Lemma~\ref{lemma:face} holds for every minimal counterexample $(V,E)$.
\end{lemma}

\begin{proof} Every topological component of $Y$ 
except $U$ is already a topological component of $Y'$ and the conclusion
holds for components of $Y'$.  The topological component $U$ is a disjoint union of two
components of $Y'$ and a set $C^0(\ee)$ of measure zero.  Thus, it
is also measurable and eventually radial.  The solid angle formula is
additive over the disjoint union in~(\ref{eqn:U}), so the formula holds for $U$.
\end{proof}
\indy{Index}{component!topological}%


\begin{lemma}\guid{GGZWYRM}\rating{ZZ}  Let $(V,E)$ be a minimal counterexample.  For any dart $x$ on any face $F$ and dart $z\in F$ that is not adjacent to $x$ under the face map, 
$$
C^0(\{\v(x),\v(z)\} \subset U_F.
$$
\end{lemma}

\begin{proof}  By excluding trivial cases of the proof,  the set $F$ and the dart $x\in F$ can be the choices used to construct $E'$.
We may assume that $F(x')$ is a triangle.
If $z=f^2x$, then the diagonal is precisely $C^0\{\v,\w\}$,
for which the conclusion has already been established.  Otherwise, $z$ can be identified with
a dart $z'\in F(y')$.  Then by minimality,
$$
C^0\{\v(x),\v(z)\} = C^0\{\v(x'),\v(z')\} \subset U(y') \subset U_F.
$$
\end{proof}

%Next show the extra conclusion added at the beginning of the lemma.
%If the added diagonal is $\{\v,\w\}$, then $E''=E'$ and  have already
%checked that $N(V,E')+1=N(V,E)$.  Since $x$ was arbitrary,  may assume
%that $\v(x)\in e$.  Let $E''' = E \cup \{e\, \{\v,\w\}\}$.  By the minimality,
%$N(V,E''') + 1 = N(V,E')$.  Also, $E''' = E''\cup \{\{\v,\w\}\}$.  The
%same argument that shows $N(V,E')+1=N(V,E)$ shows $N(V,E''')+1 = N(V,E'')$.
%Hence,
%$$
%N(V,E'')+1 = N(V,E''')+2 = N(V,E')+1 = N(V,E).
%$$

\begin{lemma}\guid{HYUAZSE}\rating{ZZ}  The property \case{half-space} of Lemma~\ref{lemma:face} holds for for every minimal counterexample.
\end{lemma}
\indy{Index}{component!topological}%

\begin{proof}
By the minimality of the counterexample $(V,E)$, it is enough to consider
the face $F$ used in the construction of $E'$.  (Indeed, the other faces of $\op{hyp}(V,E)$ can be identified with faces of $\op{hyp}(V,E')$ and these cases are easily treated.) Take $F(x')$ to be a triangle.
It will first be shown that the intersection $U_1$ of half-spaces lies in $U=U_F$.
Every point in $\ring{R}^3$ lies in the plane
$$
A=\op{aff}\{\orz,\v,\w\}
$$
or in one of the two open half-spaces bounded by this plane.  These half-spaces, $A(x')$ and $A(y')$, 
contain $U(x')$ and $U(y')$ respectively, by the minimality of $(V,E)$.  Also,
$$
A(x')\cap U_1 \subset U(x')\subset U.
$$
Similarly, $A(y')\cap U_1 \subset U$.  Also,
$$
A \cap U_1 \subset A \cap \Wdart(x) \cap \Wdart(y) \subset C^0\{\v,\w\} \subset U.
$$
Thus,
$U_1\subset U$.

\claim{For any dart $z$ of $F$, the set $U$ is a subset of the half-space with bounding
plane $\{\orz,\v(z),\v(f z)\}$.} Indeed, without loss of generality, assume that $z=x$, 
as $x$ can be chosen to be an
arbitrary dart of $F$.  By the minimality of $(V,E)$,  
the partition \eqn{eqn:U} of $U$ gives three pieces
contained respectively in the three parts:
$$
\Wdart(x') \cup C^0\{\v,\w\} \cup \Wdart(x''),
$$
where $x',x''\in D'$ correspond to the single dart $x$ in $D$:
$$
\op{azim}(x') + \op{azim}(x'') = \op{azim}(x) < \pi.
$$
Thus, $U$ itself is contained in the lune
$$
W^0(\{\orz,\v(x)\},\{\v(f x),\v(f^{-1} x)\}),
$$
which is contained in the desired half-space.  This proves the claim.

The reverse inclusion $U\subset U_1$ follows immediately from the claim.
\end{proof}

%\begin{lemma}\guid{YQWIVOS}\rating{ZZ}
%Let $(V,E)$ be a minimal counterexample.  Then a triangulation of $(V,E)$ exists.
%\end{lemma}
%
%\begin{proof}  A triangulation of $(V,E')$ exists by the minimality of $(V,E)$.  A triangulation
%of $(V,E')$ is also a triangulation of $(V,E)$.
%\end{proof}

The preceding lemmas show that a minimal counterexample satisfies all of the conclusions of Lemma~\ref{lemma:face}.  Hence it is not a countexample at all, and the conclusion of Lemma~\ref{lemma:face} holds.  This completes the proof.



\section{Polyhedron}

This section shows that a polyhedron determines a fan.  It begins by recalling basic terminology about affine and convex sets.

\begin{definition}[affine~set,~affine~hull,~affine~dimension]
An  {\it affine set} $A\subset\ring{R}^n$ is a set of the form
$$
A=\op{aff}(S)
$$
The
affine hull of $P\subset\ring{R}^n$ is the smallest affine set containing $P$.  The affine dimension of $P$ (written $\dimaff(P)$) is $\card(S)-1$, where $S$ is a set of smallest cardinality such that
$$
P \subset \op{aff}(S).
$$
\end{definition}
In particular, the affine dimension of the empty set is $-1$.
\indy{Index}{affine}%
\indy{Index}{dimension}%
\indy{Index}{vector space!affine}%
\indy{Index}{affine hull}%
\indy{Notation}{dimaff@$\dimaff$}%

\begin{definition}[relative interior,~closure,~relative boundary] Let $A$ be the affine hull of a set $P\subset\ring{R}^n$.    An interior point $\p$ of $P$ is a point that contains an open ball $B(\p,r)$ entirely contained in $P$.  A point $\p$ of $P$ belongs to the relative interior of $P$ if there is an open ball such that $B(\p,r) \cap A\subset P$.  Let $\op{ri}(P)$ be the set of relative interior points.  The closure of $P$ (denoted $\op{cl}(P)$)  is the set
$$
\op{cl}(P) = \{\p \mid \forall\,r >0.~ B(\p,r) \cap P \ne \emptyset\}.
$$
The complement $\op{cl}(P)\setminus \op{ri}(P)$ is the {\it relative boundary} of $P$.
\end{definition}
\indy{Notation}{ri (relative interior)}%
\indy{Notation}{cl (closure)}%
\indy{Index}{convex}%
\indy{Index}{affine hull}%
\indy{Index}{closure}%
\indy{Index}{interior!relative}%
\indy{Index}{relative boundary}%

\begin{definition}[face,~facet,~edge,~extreme~point]
Let $P$ be a convex set.  A {\it face} of $P$ is a convex set $F$ such that the conditions
$$
\v,\w\in P,\quad s \v + t \w \in F,\quad s>0,\quad t>0,\quad s+t = 1,
$$ 
imply that $\v,\w\in F$.  A face $F$ is {\it proper} if $F\ne \emptyset,P$.    An extreme point (resp. edge) is a face of $P$ of affine dimension $0$ ($1$, respectively).   A facet of $P$ is a proper face of affine dimension $\dimaff(P)-1$.
\indy{Index}{face}%
\indy{Index}{proper}%
\indy{Index}{facet}%
\indy{Index}{edge}%
\indy{Index}{vertex}%
\indy{Index}{dimension} %
% A point $\u\in P$ is an extreme point of $P$, if for every $\v,\w\in P\setminus\{\u\}$, the point $\v$ is not of the form $t \v + (1-t) \w$, with $0\le t\le 1$.
\end{definition}

\begin{remark} The term {\it face} occurs in this book with two meanings: the face of a hypermap and the face of a convex set.  The two contexts are sufficiently different that we hope to avoid misunderstanding.
\end{remark}

\begin{lemma}[Krein-Milman]\guid{MUGGQUF}\rating{0} Every compact convex set $P\subset\ring{R}^n$ is the convex hull of its set of extreme points.
\end{lemma}

\begin{proof}  See \cite[Theorem~2.6.16]{webster:1994}.
\end{proof}


\begin{definition}[polyhedron,~vertex]  A {\it polyhedron} is the intersection of
a finite number of closed half-spaces in $\ring{R}^n$.  An extreme point of a polyhedron is also called a {\it vertex}.
\end{definition}
\indy{Index}{polyhedron}%

\begin{lemma}\guid{LTHQIAA}\rating{ZZ}\label{lemma:aff-poly}
An affine set in $\ring{R}^n$ is a polyhedron.
\end{lemma}

\begin{proof} See \cite[Cor~1.4.2]{webster:1994}.
\end{proof}

A polyhedron is closed and convex.  An intersection of faces of a convex set is again a face.  A bounded polyhedron falls within the scope of the Krein-Milman theorem.  In particular, a bounded polyhedron is the convex hull of its set of vertices.
\indy{Index}{closed}%

Let $P\subset\ring{R}^n$ be a bounded polyhedron with affine hull $A$. Write 
$$
P = A \cap A^+_1 \cap \cdots \cap A^+_r,
$$
where
$A^+_i = \{\p\mid \u_i\cdot \p \le a_i\}$ with bounding hyperplane
$A_i=\{\p\mid \u_i \cdot \p = a_i\}$, for some $\u_i\in \ring{R}^n$ and $a_i\in\ring{R}$. 
Assume that this representation is minimal in the sense that none of the factors $A^+_i$ may be omitted.  Let $F_i = A_i\cap P$.
\indy{Notation}{P@$P$ (polyhedron)}%


\begin{lemma}\guid{CZZHBLI}\rating{ZZ}\label{lemma:webster}  
Let $P\subset\ring{R}^n$ be a bounded polyhedron.  Then
\indy{Index}{polyhedron}%
\begin{enumerate}
\item The facets of $P$ are $F_i$, $i=1,\ldots,r$.
\item The relative boundary of $P$ is $F_1\cup\cdots \cup F_r$.
\item Every proper face is the intersection of the facets that contain it.
\item Every face of a face of $P$ is a face of $P$.
\item The vertices of a face $F$ are precisely the vertices of $P$ that are contained in $F$.
\item If $F$ and $F'$ are two faces of $P$ whose relative interiors meet, then
$F=F'$.
\end{enumerate}
\end{lemma}
\indy{Index}{facet}%
\indy{Index}{relative boundary}%
\indy{Index}{face}%

\begin{proof} See \cite[Thm~3.2.1]{webster:1994} for the first three conclusions.
See \cite[Th~2.6.5]{webster:1994} for the proof of the next conclusion,

Turn to the fifth conclusion.  A vertex of $F$ is a face of $P$ (of dimension $0$) by the fourth conclusion.  Hence every vertex of $F$ is a vertex of $P$.  Conversely, a face of $P$ contained in $F$ is a fortiori a face of $F$.

See \cite[Cor~2.6.7]{webster:1994} for the final conclusion.
\end{proof}

\begin{corollary}\guid{QOEPBJD}\rating{ZZ}
A face of a polyhedron is a polyhedron.  
\end{corollary}

\begin{proof} By Lemma~\ref{lemma:webster}, each facet is defined by a system of linear inequalities.  (See Lemma~\ref{lemma:aff-poly}.)  A proper face is an intersection of finitely many facets, and is therefore given by the conjunction of the inequalities defining the various facets.
\end{proof}


\begin{lemma}\guid{NEHRQPR}\rating{ZZ} \label{lemma:scale} 
Let $P$ be a bounded polyhedron with $\orz$ as an interior point.  Suppose that there are proper faces $F,F'$ of $P$, points $\p\in F$, $\q\in F'$, and positive scalars $s,t >0$ such that $s \p = t \q$.  Then $s=t$.
\end{lemma}

\begin{proof}  The faces $F,F'$ are subsets of facets of $P$.  It does no harm to assume that $F$ and $F'$ are themselves facets.   Without loss of generality, assume for a contradiction that $s=1$ and $t>1$.  By Lemma~\ref{lemma:webster}, 
 $F = A_i \cap P$, where $A_i = \{\q \mid \q\cdot \u = a\}$ for some $\u$ and $a$.  Also, 
$$
C\{\p\}\cap P = \op{conv}\{\orz,\p\} = \{\q \mid \orz\le \q\cdot \u \le a\}.
$$
In particular, $t \p\not\in P$, for $t>1$.  This contradiction gives $s = t$.
\end{proof}





\begin{definition} Let $P$ be a bounded polyhedron.
Let $V_P$ be the set of vertices of $P$.  Let $E_P$ be the set of pairs $\{\v,\w\}$ of vertices such that $\op{conv}\{\v,\w\}$ is an edge of $P$.
\end{definition}
\indy{Index}{polyhedron}%
\indy{Notation}{E@$E_P$ (edge)}%

\begin{lemma}\guid{JLIGZGS}\label{lemma:polyhedron}% \rating{800} including azim<pi.
Let $P$ be a bounded polyhedron in $\ring{R}^3$ with the interior point $\orz$.
Then $(V_P,E_P)$ is a fan.
\end{lemma}
\indy{Index}{fan}%

\begin{proof} The properties of a fan can be checked one by one.
By the Krein-Milman lemma, the set of vertices is non-empty.  By Lemma~\ref{lemma:webster}, there are finitely many faces, so that $V_P$ is finite.  Since $\orz$ is an interior point, it does not meet any face.  In particular, $\orz\not\in V_P$.   In particular,
for all $\ee\in E_P$, 
$\orz\not\in \op{conv}(\ee)$.

Suppose for a contradiction that $\{\v,\w\}\in E_P$ and that $\v$ and $\w$ are dependent.  As $\orz\not\in \op{conv}(\ee)$, some dependency has the form $s \v = t \w$, for some $s, t>0$.  By Lemma~\ref{lemma:scale}, $s=t$ and $\v=\w$, which is contrary to the definition of edge as a face of dimension $1$.

Finally, the intersection property $C(\ee)\cap C(\ee') = C(\ee \cap \ee')$ can be checked.
By Lemma~\ref{lemma:scale},
$$
C(\ee) \cap C(\ee') = \{t \p \mid \p\in \op{conv}(\ee) \cap \op{conv}(\ee') \text{ and } t \ge 0\}.
$$
$\op{conv}(\ee)$ and $\op{conv}(\ee')$ are both faces of $P$.  The intersection is again a face of $P$.  The intersection is the convex hull of its vertices, that is, the convex hull of $\ee \cap \ee'$.  Thus,
$$
C(\ee)\cap C(\ee') = \{t \p \mid \p\in \op{conv}(\ee\cap \ee')\} = C(\ee \cap \ee').
$$
Thus, all the defining conditions of a fan are satisfied.
\end{proof}


\begin{lemma}\guid{AMHFNXP}\rating{ZZ}\label{lemma:WF} 
Let $P$ be a bounded polyhedron in $\ring{R}^3$ with $\orz$ as an interior point.  Let $(V_P,E_P)$ be the associated fan.  There is a bijection between the facets of $P$ and the topological components of $Y(V_P,E_P)$ given by 
$$
F \mapsto W_F = \{t \p \mid \p\in \op{ri}(F),~t >0\}.
$$
\end{lemma}
\indy{Index}{component!topological}%

\begin{proof} It is enough to check that the following four claims about  $W_F$.

\claim { $W_F$ is connected.} Indeed, by Lemma~\ref{lemma:webster}, the relative interior of a convex polyhedron is the intersection of an affine set with open half-spaces, which is the intersection of convex sets, and is therefore convex. The set $\op{ri}(F)$ is convex, hence connected.    The positive half-line $I=\{t\mid t>0\}$ is connected.  The continuous image of the connected set $\op{ri}(F)\times I$ of these two sets is $W_F$.  Hence $W_F$ is connected.

\claim { $W_F$ is  open.}  Indeed, this is a standard $\epsilon$-argument.  Let $A$ be the affine hull of $F$.  For any $\p\in \op{ri}(F)$, pick $r>0$ such that $B(\p,r)\cap A\subset \op{ri}(F)$.  Pick $r'>0$ and $0<\epsilon<1$ such that for all $\q\in B(\p,r')$, there exists $t$ such that $|t|<\epsilon$ and $(1+t)\q\in A$.  After shrinking $r'$,  if $\q\in B(\p,r')$, then $(1+t)\q \in B(\p,r)\cap A \subset \op{ri}(F)$.   That is, $B(\p,r')\subset W_F$.  Hence $W_F$ is open.

\claim { The sets $W_F$ are pairwise disjoint, and the map $F\mapsto W_F$ is one-to-one.}  Indeed, select any two facets $F,F'$ for which $W_F\cap W_{F'}\ne \emptyset$.  That is, there exist $\p\in \op{ri}(F)$, $\q\in \op{ri}(F')$, and $s,t>0$ such that
$s \p = t \q$.  By Lemma~\ref{lemma:scale}, $s=t$ and $\p=\q\in \op{ri}(F)\cap \op{ri}(F')$.  By the final statement of Lemma~\ref{lemma:webster}, this implies that $F=F'$.

\claim { The union of the sets $W_F$ is $Y(V_P,E_P)$.}  Indeed, select any $\p\in Y(V_P,E_P)$.  As $\orz$ lies in the interior of the bounded polyhedron, we may rescale $\p$ by a positive scalar $t$ so that $t \p$ lies in the boundary of $P$, and hence (by Lemma~\ref{lemma:webster}) in a facet $F$.  If $t \p \in \op{ri}(F)$, then $\p\in W_F$, as desired.  Otherwise, $t \p$ lies in the relative boundary of $F$.  The facets of a three dimensional polyhedron have dimension $2$; and the facets forming the relative boundary of $F$ have dimension $1$.  These faces are edges of $P$.  Thus, $t \p$ lies in an edge of $P$, so that $\p\in X(V_P,E_P)$, which is contrary to the assumption that $\p\in Y(V_P,E_P)$.  The conclusion follows.

From these claims, it follows that the sets $W_F$ are the topological components of $Y(V_P,E_P)$.
\end{proof}
\indy{Index}{connected}%

\begin{lemma}\guid{WBLARHH}\rating{ZZ}\label{lemma:facet-bi}
Let $P\subset\ring{R}^3$ be a bounded polyhedron with interior point $\orz$.  The
facets of $P$ are in bijection with the faces of $\op{hyp}(V_P,E_P)$, under the
correspondence
$$
F\leftrightarrow F' \text{ if and only if } W_F = U_{F'}.
$$
\end{lemma}

\begin{proof}  The faces of the hypermap are in bijection with the set $[Y]$ of topological components of $Y(V_P,E_P)$.  The facets of $P$ are also in bijection with $[Y]$.
\end{proof}

\begin{lemma}\guid{BSXAQBQ}\rating{ZZ} Let $P$ be a bounded polyhedron with $\orz$ as an interior point.  Then $\op{azim}(x) < \pi$  for every dart $x$ in the hypermap $\op{hyp}(V_P,E_P)$.
\end{lemma}
\indy{Index}{polyhedron}%
\indy{Index}{interior!point}%
\indy{Index}{hypermap}%
\indy{Index}{dart}%

\begin{proof}   The dart $x=(\v,\w)$ leads into some topological component $U_x$, which is equal to $W_F$, for some facet $F$ of $P$.  The relative interior of $F$ is contained in $W_F\subset \Wdart(x)$.  The facet $F$ is contained in $\bWdart(x)$. Now
$$
\bWdart(x) = \Wdart(x)\cup A,
$$
where $A$ is the union of the two bounding half-planes.   The relative interior of $F$ does not meet $A$.   

By Lemma~\ref{lemma:scale}, the vertex $\v$ is a face of $F$.  The vertex $\v$ is the intersection of edges (that is, facets) of $F$ containing $\v$.  These edges are contained in $A$.  Each of the two half-planes forming $A$ contains such an edge. (In fact, each contains exactly one edge.)
\indy{Index}{component!topological}%
\indy{Index}{dart}%
\indy{Index}{wedge}%
\indy{Index}{vertex}%
\indy{Index}{interior!relative}%

If $\op{azim}(x)>\pi$, the segment between points chosen on the relative interiors of edges (of $F$ meeting $\v$) in the two different bounding half-planes is not contained in $\bWdart(x)$.  This is contrary to the convexity of $F$.
\indy{Index}{convex}%
\indy{Notation}{azim}%

If $\op{azim}(x)=\pi$, then $A$ is a plane.  It is the bounding plane of a half-space containing $F$.  Directly from the definition of face, it can be checked that the intersection $A\cap F$ is a proper face of $F$.  This face contains an edge (which is a facet of $F$) and must therefore be a single edge.  This contradicts the observation that $A\cap F$ contains at least two edges.  Hence $\op{azim}(x)<\pi$.
\end{proof}


