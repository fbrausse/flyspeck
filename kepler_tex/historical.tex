% Author Thomas C. Hales
% Copyright 2003, Thomas C. Hales
% AMS-TEX Format
% Revised Oct 4, 2003. Added material on fcc and hcp in Sec. 1

% Reformatted into LaTex. Feb 17, 2004.


%\documentstyle{amsppt}
%\topmatter
%\loadmsbm
%\UseAMSsymbols

%\hoffset=0.75truein
%\voffset=0.5truein


% Intro Historical

% Completed Name indexing...



This book gives a solution to the sphere packing
problem,  which asserts that the density of a packing of
congruent spheres in three dimensions is never greater than
$\pi/\sqrt{18}\approx 0.74048\ldots$. This is the oldest problem
in discrete geometry and is an important part of Hilbert's 18th
problem. An example of a packing achieving this density is the
face-centered cubic packing.\index{Names}{Hilbert}


\section{History}
\label{sec:intro-review}

\subsection{face-centered cubic packing}

A packing of spheres is an arrangement of
nonoverlapping spheres of radius 1 in Euclidean space.
Each sphere is determined by its center, so equivalently it is a collection
of points in Euclidean space separated by distances of at least 2.
The density of a packing is defined as the $\limsup$ of
the densities of the partial packings formed by spheres inside
a ball with fixed center of radius $R$.
(By taking the $\limsup$,
rather than $\liminf$ as the density, we solve the sphere packing problem
in the strongest possible sense.)
Defined as a limit, the density
is insensitive to changes in the packing in any bounded region.
For example, a finite number of spheres can be removed from the
face-centered cubic packing without affecting its density.

Consequently, it is not possible to hope for any strong uniqueness
results for packings of optimal density.  The uniqueness
established by this work is as strong as can be hoped for. It
shows that certain local structures (centered packings) attached
to the face-centered cubic (fcc) and hexagonal-close packings
(hcp) are the only structures that maximize a local density
function.

Although we do not pursue this point, Conway and Sloane develop
a theory of tight packings that is more restrictive than having the
greatest possible density \cite{CoSl95}.\index{Names}{Conway, J.}\index{Names}{Sloane, N.J.A}
An open problem is to prove that their
list of tight packings in three dimensions is complete.

The face-centered cubic packing appears in
Diagram~\ref{fig:fcc-pack}.

\begin{figure}[htb]
  \centering
  \myincludegraphics{\ps/fcc_small.eps}
  \caption{The face-centered-cubic packing.}
  \label{fig:fcc-pack}
\end{figure}


\subsection{early work}
\label{sec:early}

The study of the mathematical properties of the face-centered cubic
packing can be traced back to a Sanskrit work composed around 499 CE.
I quote an extensive passage from the commentary that K. Plofker\index{Names}{Plofker, K.}
has made about the formula
for the number of balls in triangular piles\cite{Plo00}:
\medskip

{
\narrower
\font\ninerm=cmr9
\ninerm
\def\=#1{\accent"16 #1}

 The excerpt below is taken
 from a Sanskrit work composed around 499 CE, the \=Aryabha\d t\={\i}ya
 of \=Aryabha\d ta, and the commentary on it written in 629 CE
 by Bh\=askara (I).\index{Names}{Bh\=askara}  
  The work is a compendium of various rules in
 mathematics and mathematical astronomy, and the results are probably
 not due the \=Aryabha\d ta \index{Names}{\=Aryabha\d ta}
  himself but derived from an earlier source:
 however, this is the oldest source extant for them.  (My translation's
 from the edition by K. S. Shukla,\index{Names}{Shukla, K. S.} 
  {\it The \=Aryabha\d t\={\i}ya of
 \=Aryabha\d ta with the Commentary of Bh\=askara I and Some\'svara},
 New Delhi: Indian National Science Academy 1976; my inclusions are
 in square brackets. There is a corresponding English translation
 by Shukla and K. V. Sarma, {\it The \=Aryabha\d t\={\i}ya
 of \=Aryabha\d ta}, New Delhi: Indian National Science Academy 1976.
 It might be easier to get hold of the earlier English translation by
 W. E. Clark,\index{Names}{Clark, W. E.} 
 {\it The \=Aryabha\d t\={\i}ya of \=Aryabha\d ta},
 Chicago: University of Chicago Press, 1930.)

      Basically, the rule considers the series in arithmetic
 progression
 $
 S_i = 1 + 2 + 3 + \ldots + i
 $
 (for whose sum the formula is known) as the number of objects
 in the $i$th layer of a pile with a total of $n$ layers, and
 specifies the following two equivalent formulas for the ``accumulation
 of the pile'' or $ \sum_{i=1}^n S_i $:

 $$
 \sum_{i=1}^n S_i = \frac{{n(n+1)(n+2)}}{6},
 $$

 $$
 \sum_{i=1}^n S_i = \frac{{(n+1)^3 - (n+1)}}{6}.
 $$

 What he says is this:

 {\it \=Aryabha\d t\={\i}ya}, Ga\d nitap\=ada 21:

 {\narrower
    For a series [lit. ``heap''] with a common difference and first term
 of 1, the product of three [terms successively] increased by 1 from
 the total, or else the cube of [the total] plus 1 diminished by [its]
 root, divided by 6, is the total of the pile [lit. ``solid heap''].
 }

 Bh\=askara's commentary on this verse:

 {\narrower
    [This] heap [or] series is specified as having one for its common
 difference and initial term. This same series with one for its common
 difference and initial term is said [to be] ``heaped up.'' ``The
 product of three [terms successively] increased by one from the total''
 of this so-called heaped-up ``series with one for its common
 difference and initial term'': i.e., the product of three terms, starting
 from the total and increasing by one. Namely, the total, that plus one,
 and [that] plus one again. That [can] be stated [as follows]: the total,
 that plus one, and that total plus two. The product of those three
 divided by 6 is the ``solid heap,'' the accumulation of the series.
 Now another method: The cube of the root equal to that [total] plus
 one is diminished by its root, and divided by 6: thus it follows.
 ``Or else'': [i.e.], the cube of that root plus one, diminished by
 its own root, divided by 6, is the ``solid heap.''
    Example: Series with 5, 8, and 14 respectively for their total layers:
 tell me [their] triangular-shaped piles.
    In order, the totals are 5, 8, 14.
    Procedure: Total 5. This plus one: 6. This plus one again: 7. Product
 of those three: 210. This divided by 6 is the accumulation of the series:
 35.
    [He goes on to give the answers for the second two cases, but you
 doubtless get the picture.]   --  K. Plofker
 }


}



The modern mathematical study of spheres and their close packings can be
traced to T. Hariot.\index{Names}{Hariot, Thomas.}  
Hariot's work -- unpublished, unedited,
and largely undated -- shows a preoccupation with sphere packings.
He seems to have first taken an interest in packings at the
prompting of Sir Walter Raleigh.\index{Names}{Raleigh, Walter}  
At the time, Hariot was Raleigh's
mathematical assistant,  and Raleigh gave him the problem of determining
formulas for the number of cannonballs in regularly stacked piles.
In 1591 he prepared a chart of triangular numbers for Raleigh.
Shirley,\index{Names}{Shirley} Hariot's biographer, writes,

{
\narrower
\font\ninerm=cmr9
\ninerm

    Obviously, this is a quick reference chart prepared for Ralegh to give information on the ground space required for the storage of cannon
    balls in connection with the stacking of armaments for his marauding vessels. The chart is ingeniously arranged so that it is possible to
    read directly the number of cannon balls on the ground or in a pyramid pile with triangular, square, or oblong base. All of this Harriot had
    worked out by the laws of mathematical progression (not as Miss Rukeyser suggests by experiment), as the rough calculations
    accompanying the chart make clear. It is interesting to note that on adjacent sheets, Harriot moved, as a mathematician naturally would,
    into the theory of the sums of the squares, and attempted to determine graphically all the possible configurations that discrete particles
    could assume -- a study which led him inevitably to the corpuscular or atomic theory of matter originally deriving from Lucretius\index{Names}{Lucretius} and
    Epicurus.\index{Names}{Epicurus} \cite[p.242]{Shi83}

}

\smallskip
Hariot connected sphere packings to Pascal's triangle long before
Pascal introduced the triangle. See Diagram~\ref{fig:pascal}.\index{Names}{Pascal}  Each entry in the triangle represents the sphere packing obtained
by stacking the two piles directly above it upon one another.  Each
diagonal of the sphere packing gives packings in a fixed dimension.

When he reached the diagonal representing four dimensions, he drew a
rough sketch of a four dimensional sphere packing.
To my knowledge,
this is the first recorded visual representation of the fourth dimension.
(I am intent on showing that 
it all started with sphere packings.)
T. Banchoff's historical notes on the fourth dimension
indicate that mathematicians before Hariot generally
regarded the fourth dimension
as a ``Monster of Nature, less possible than a Chimaera or Centaure''
\cite{Ban}.  This particular phrase is due to Wallis, who was born 
56 years after Hariot.
\index{Names}{Wallis}

\begin{figure}[htb]
  \centering
  \myincludegraphics{\ps/diag21.ps}
  \caption{Hariot's view of Pascal's triangle.}
  \label{fig:pascal}
\end{figure}

Hariot was the first to distinguish between the face-centered
cubic and hexagonal close packings \cite[p.52]{Mas66}.

Kepler\index{Names}{Kepler, J.} 
became involved in sphere packings through his correspondence
with Hariot in the early years of the 17th century.
Kargon\index{Names}{Kargon} writes, in his history of atomism in England,


{
\narrower
\font\ninerm=cmr9
\ninerm

    Hariot's theory of matter appears to have been virtually that of Democritus, \index{Names}{Democritus} Hero of Alexandria, \index{Names}{Hero of Alexandria} and, in a large measure, that of Epicurus
    and Lucretius. According to Hariot the universe is composed of atoms with void space interposed. The atoms themselves are eternal and
    continuous. Physical properties result from the magnitude, shape, and motion of these atoms, or corpuscles compounded from them$\ldots$.

    Probably the most interesting application of Hariot's atomic theory was in the field of optics. In a letter to Kepler on 2 December 1606
    Hariot outlined his views. Why, he asked, when a light ray falls upon the surface of a transparent medium, is it partially reflected and
    partially refracted? Since by the principle of uniformity, a single point cannot both reflect and transmit light, the answer must lie in the
    supposition that the ray is resisted by some points and not others.

    ``A dense diaphanous body, therefore, which to the sense appears to be continuous in all parts, is not actually continuous. But it has
    corporeal parts which resist the rays, and incorporeal parts vacua which the rays penetrate$\ldots$''

    It was here that Hariot advised Kepler to abstract himself mathematically into an atom in order to enter `Nature's house'. In his reply of 2
    August 1607, Kepler declined to follow Harriot, ad atomos et vacua. Kepler preferred to think of the reflection-refraction problem in terms
    of the union of two opposing qualities --
    transparence and opacity. Hariot was surprised. ``If those assumptions and reasons satisfy you, I
    am amazed.'' \cite[p.26]{Kar66}

}

\smallskip
Despite Kepler's initial reluctance to adopt an atomic theory, he
was eventually swayed, and in 1611 he published an essay that
explores the consequences of a theory of matter composed of small
spherical particles.  Kepler's essay was the ``first recorded step
towards a mathematical theory of the genesis of inorganic or
organic form'' \cite[p.v]{Why66}.

Kepler's essay describes
the face-centered cubic packing and asserts that ``the packing will
be the tightest possible, so that in no other arrangement  could more
pellets be stuffed into the same container.''  This assertion has
come to be known as the sphere packing problem, or the Kepler conjecture.   
The purpose of this book is to give a solution to this problem.

%\section{History}
\label{sec:history}

The next episode in the history of this problem is a debate between
Isaac Newton\index{Names}{Newton, Isaac} and David Gregory.\index{Names}{Gregory, David}  Newton and Gregory discussed the
question of how many spheres of equal radius can be arranged to
touch a given sphere.  This is the three-dimensional analogue of the
simple fact that in two dimensions six pennies, but no more, can be
arranged to touch a central penny.  This is the kissing-number
problem in $n$-dimensions. In three dimensions, Newton said that the
maximum was twelve spheres, but Gregory claimed that thirteen might
be possible.

Newton was correct.
In the 19th century, the first papers claiming a proof of the
kissing-number problem appeared
in \cite{Ben74}, \cite{Gun75}, \cite{Hop74}.
Although some writers cite these papers
as a proof, they are hardly rigorous by today's standards.
Another incorrect
proof appears in \cite{Boe52}.
  The first proper proof was obtained
by B. L. van der Waerden \index{Names}{van der Waerden, B. L.}
and Sch\"utte \index{Names}{Sch\"utte} in 1953 \cite{Sch53}.
An elementary proof appears in Leech \cite{Lee56}.\index{Names}{Leech, J.}
The influence of van der Waerden, Sch\"utte, and Leech upon the
papers in this collection is readily apparent.  Although the
connection between the Newton-Gregory problem and the sphere packing problem
is not obvious, L. Fejes T\'oth \index{Names}{Fejes T\'oth, L.}
in 1953, in the first
work describing a strategy to solve the sphere packing problem, made
a quantitative version of the Gregory-Newton problem the first step
\cite{Fej53}.


The two-dimensional analogue of the sphere packing problem is to show
that the honeycomb packing in two dimensions gives the highest
density.  This result was established in 1892 by Thue, \index{Names}{Thue}
with a second
proof appearing in 1910 (\cite{Thu92}, \cite{Thu10}). G. Szpiro's
\index{Names}{Szpiro, G.}
book on the Kepler conjecture calls Thue's proofs into question
(\cite{Szp02}).  C. Siegel 
\index{Names}{Siegel, C.}
said that Thue's original proof is
``reasonable, but full of holes'' (\cite{Szp02}). A number of other
proofs have appeared since then. Three are particularly notable.
Rogers's proof generalizes to give a bound on the density of
packings in any dimension \cite{Rog58}. A proof by L. Fejes T\'oth
extends to give bounds on the density of packings of convex disks
\cite{Fej50}. A third proof, also by L. Fejes T\'oth, extends to
non-Euclidean geometries \cite{Fej53}. Another early proof appears
in \cite{SeM44}.

In 1900, Hilbert made the sphere packing problem of his 18th
problem \cite{hilbert}. Milnor, 
\index{Names}{Milnor, J.} in his review of Hilbert's 18th
problem, breaks the problem into three parts \cite{Mil76}.
\index{Names}{Hilbert, D.}

{
\narrower
\font\ninerm=cmr9
\ninerm

1.  Is there in $n$-dimensional Euclidean Space $\ldots$ only a finite
number of essentially different kinds of groups of motions with a
[compact] fundamental region?

2.  Whether polyhedra also exist which do not appear as fundamental
    regions of groups of motions, by means of which nevertheless
    by a suitable juxtaposition of congruent copies a complete filling
    up of all [Euclidean] space is possible?

3.  How can one arrange most densely in space an infinite number
    of equal solids of given form, e.g. spheres with given radii $\ldots$,
    that is, how can one so fit them together that the ratio of the
    filled to the unfilled space may be as great as possible?

}

\smallskip
Writing of the third part, Milnor states,

{
\narrower
\font\ninerm=cmr9
\ninerm

For $2$-dimensional disks this problem has been solved by Thue and
Fejes T\'oth, who showed that the expected hexagonal (or honeycomb)
packing of circular disks in the plane is the densest possible.
However, the corresponding problem in $3$ dimensions remains
unsolved. This is a scandalous situation since the (presumably)
correct answer has been known since the time of Gauss. (Compare
Hilbert and Cohn-Vossen.)  All that is missing is a proof.
\index{Names}{Gauss}
}

\subsection{recent literature}

Past progress on the sphere packing problem can be arranged into
four categories:
\begin{itemize}
    \item bounds on the density,
    \item descriptions of classes of packings for
which the bound of $\pi/\sqrt{18}$ is known,
    \item convex bodies other
than spheres for which the packing density can be determined
precisely,
    \item strategies of proof.
\end{itemize}

%\subhead 4.1. Bounds\endsubhead
\subsubsection{bound}

Various upper bounds have been established on the density of
packings.
\smallskip

{\obeylines

 \ 0.884\ \ (Blichfeldt) \cite{Bli19},
 \ 0.835\ \ (Blichfeldt) \cite{Bli29},
 \ 0.828\ \ (Rankin) \cite{Ran47},
 \ 0.7797\ (Rogers) \cite{Rog58},
 \ 0.77844 (Lindsey) \cite{Lin86},
 \ 0.77836 (Muder)\cite{Mud88},
 \ 0.7731\ (Muder) \cite{Mud93}.

}

\smallskip
Rogers's is a particularly natural bound.
  As the dates indicate, it remained the best available
bound for many years.  His monotonicity lemma and his
decomposition of Voronoi cells into simplices have become important
elements in the solution to the sphere packing problem.
We give a new proof of Rogers's bound
in ``Sphere Packings III.''  A function $\tau$,
used throughout this
collection, measures the departure of various objects from
Rogers's bound.

Muder's bounds, although they appear to be rather small
improvements of Rogers's bound, are the first to make use of the
full Voronoi cell in the determination of densities. As such, they
mark a transition to a greater level of sophistication and
difficulty.  Muder's influence on the work in this collection is
also apparent.\index{Names}{Muder, D.}\index{Names}{Rogers, C. A.}

A sphere packing admits a Voronoi decomposition: around
every sphere take the convex region consisting of points closer to that sphere
center than to any other sphere center.   L. Fejes T\'oth's
dodecahedral
conjecture asserts that the Voronoi cell of smallest volume is
a regular dodecahedron with inradius 1 \cite{Fej42}.
The dodecahedral conjecture implies a bound of 0.755 on sphere
packings.  L. Fejes T\'oth actually gave a complete proof except
for one estimate. A footnote in his paper documents the gap, ``In the
proof, we have relied to some extent solely on intuitive
observation [Anschauung].''
 As L. Fejes T\'oth pointed out, that estimate is extraordinarily
difficult, and the dodecahedral conjecture has resisted all efforts
until now \cite{McL98}.\index{Names}{Fejes T\'oth, L.}

The missing estimate in L. Fejes T\'oth's paper is an explicit form
of the Newton-Gregory problem.  What is needed is an explicit bound
on how close the 13th sphere can come to touching the central
sphere.  Or more generally, minimize the sum of the distances
of the 13 spheres from the central sphere.
No satisfactory bounds are known.  Boerdijk has a conjecture for the arrangement
that minimizes the average distance of the 13 spheres from the
central sphere.\index{Names}{Boerdijk}
Van der Waerden\index{Names}{van der Waerden, B.L.}
has a conjecture for the closest arrangement of 13 spheres in which
all spheres have the same distance from the central sphere.
Bezdek has shown that the dodecahedral conjecture would follow from
weaker bounds than those originally proposed by L. Fejes T\'oth
\index{Names}{Bezdek, K.}
\cite{Bez97}.

A proof of the dodecahedral conjecture has traditionally been
viewed as the first step toward a solution to the sphere packing problem,
and if little progress has been made until now toward a complete
solution, the difficulty of the dodecahedral
conjecture is certainly responsible to a large degree.

\subsubsection{packing type}

If the infinite dimensional space of all packings is too unwieldy,
we can ask if it is possible to establish the bound $\pi/\sqrt{18}$
for packings with special structures.

If we restrict the problem
to packings whose sphere centers are the points of a lattice, the
 packings are described by a finite number of parameters, and the
problem becomes much more accessible.  Lagrange proved that the
densest lattice packing in two dimensions is the familiar honeycomb
arrangement \cite{Lag73}. Gauss proved that the densest lattice
packing in three dimensions is the face-centered cubic \cite{Gau31}.
In dimensions 4--8, the optimal lattices are described by their root
systems, $A_2$, $A_3$, $D_4$, $D_5$, $E_6$, $E_7$, and $E_8$. A.
Korkine and G. Zolotareff showed that $D_4$ and $D_5$ are the
densest lattice packings in dimensions 4 and 5 (\cite{KoZ73},
\index{Names}{Korkine}\index{Names}{Zolotareff, G.}
\cite{KoZ77}). Blichfeldt determined the densest lattice packings in
dimensions 6--8 \cite{Bli35}. \index{Names}{Blichfeldt}
Cohn and Kumar solved the problem in
dimension 24 \cite{CoKu}.  With the exception of dimension $24$,
beyond dimension $8$, there are no proofs of optimality, and yet
there are many excellent candidates for the densest lattice
packings.  For a proof of the existence of optimal lattices, see
\cite{Oes90}.\index{Names}{Cohn, H}\index{Names}{Kumar, A.}


Although lattice packings are of particular interest because they
relate to so many different branches of mathematics, Rogers has
conjectured that in sufficiently high dimensions, the densest
packings are not lattice packings \cite{Rog64}.   In fact, the
densest known packings in various dimensions are not lattice
packings.  The third edition of \cite{CS} gives several examples
of nonlattice packings that are denser than any known lattice
packings (dimensions 10, 11, 13, 18, 20, 22). The densest packings
of typical convex sets in the plane, in the sense of Baire
categories, are not lattice packings \cite{Fej95}.
\index{Names}{Rogers, C.A.}

Gauss's theorem on lattice densities has been generalized by
A. Bezdek, W. Kuperberg, and E. Makai, Jr. \cite{BKM91}.
They showed that packings of parallel
strings of spheres never have density greater than $\pi/\sqrt{18}$.
\index{Names}{Bezdek, A.}\index{Names}{Kuperberg, W.}\index{Names}{Makai, E. Jr.}

\subsubsection{convex body}

If the optimal sphere packings are too difficult to determine,
we might ask whether
the problem can be solved for other convex bodies.
To avoid trivialities, we restrict our attention to convex bodies
whose packing density is strictly less than 1.

  The first convex body in Euclidean 3-space that does not tile
for which the packing density was explicitly determined is
an infinite cylinder \cite{Bez90}.
Here A. Bezdek and W. Kuperberg prove
that the
optimal density is obtained by arranging the cylinders in
parallel columns in the honeycomb arrangement.

In 1993, J. Pach exposed the humbling depth of our ignorance when he issued
the challenge to determine the packing density for some bounded convex
body that does not tile space \cite{MP93}.
 This question was answered by
A. Bezdek \cite{Bez94}, who determined the packing density of a rhombic
dodecahedron that has one corner clipped so that it no longer tiles.
The packing density equals the ratio of the
volume of the clipped
rhombic dodecahedron to the volume of the unclipped rhombic dodecahedron.
\index{Names}{Pach, J.}

\subsubsection{proof strategy}

In 1953, L. Fejes T\'oth proposed a program to solve the
sphere packing problem \cite{Fej53}.
A single Voronoi cell cannot lead to a bound better
than the dodecahedral conjecture.   L. Fejes T\'oth considered
weighted averages of the volumes of collections of Voronoi cells.
 These weighted
averages involve up to 13 Voronoi cells.  He showed that if a particular
weighted average of volumes is greater than the volume of the
rhombic dodecahedron, then the sphere packing problem follows.
The sphere packing problem is an optimization problem in an infinite
number of variables.  L. Fejes T\'oth's weighted-average argument
was the first indication that it might be possible to reduce
the sphere packing problem to a problem in a finite number of variables.
Needless to say, calculations involving the weighted averages of the
volumes of
several Voronoi cells will be significantly more difficult than those
involved in establishing the dodecahedral conjecture.
\index{Names}{Fejes T\'oth, L.}

To justify his approach, which limits the number of Voronoi cells
to 13, Fejes T\'oth needs a preliminary estimate of how close
a 13th sphere can come to a central sphere.  It is at this point
in his formulation of the sphere packing problem that an explicit
version of the Newton-Gregory problem is required.  How
close can 13 spheres come to a central sphere, as measured by
the sum of their distances from the central sphere?

%Strictly speaking, neither L. Fejes T\'oth's program nor my own program
%educes the sphere packing
%problem to a finite number of variables, because if
%it turned out that one of
%the optimization problems in finitely many
%variables had an unexpected global maximum, the program would
%fail, but the sphere packing problem would remain intact.
%In fact, the failure of a program has
%no implications for the sphere packing problem.  The proof that the
%sphere packing problem  reduces to a finite number of variables comes only as
%corollary to the full proof of the sphere packing problem.

L. Fejes T\'oth made another significant suggestion in \cite{Fej64}.
He was the first to suggest the use of computers in the sphere packing problem.
After describing his program, he writes,

{
\narrower
\font\ninerm=cmr9
\ninerm

Thus it seems that the problem can be reduced to the determination
of the minimum of a function of a finite number of variables,
providing a programme realizable in principle.  In view of the
intricacy of this function we are far from attempting to
determine the exact minimum.  But, mindful of the rapid development
of our computers, it is imaginable that the minimum may
be approximated with great exactitude.

}

\smallskip
The most widely publicized attempt to prove the sphere packing problem
was that of Wu-Yi Hsiang \cite{Hsi93}.  (See also \cite{Hsi93a},
\cite{Hsi93b}, \cite{Hsi02}.)  Hsiang's approach can be viewed as
a continuation and extension of L. Fejes T\'oth's program.
Hsiang's paper contains major gaps and errors \cite{CoHMS94}.
  The mathematical arguments against his argument appear
in my
debate with him in the {\it Mathematical Intelligencer}
(\cite{Hal94}, \cite{Hsi95}).
There are now many published sources that agree with the central
claims of \cite{Hal94} against Hsiang.
Conway and Sloane report that the paper ``contains serious flaws.''
G. Fejes T\'oth feels that ``the greater part of the work has yet
to be done'' \cite{Fej95}.   K. Bezdek concluded,
after an extensive study of Hsiang's work, ``his work is far from being
complete and correct in all details'' \cite{Bez97}.
 D. Muder writes, ``the
community has reached a consensus on it: no one buys it'' \cite{Mud97}.
\index{Names}{Hsiang, W.-Y.}



