%% Quadratic Volumes
%% File Created 3/22/07.

\section{Properties of Measure}

Nowhere in the proof of the Kepler conjecture do we need a notion
of integration.  Measure alone suffices.  (However, there are a few
volumes described below that I do not see how to calculate without
first writing them as an integral.)

We need a concepts of null set, measurable set, and volume in
three dimensions.  For the purposes of this paper, we can take the
the three dimensional Lebesgue measure.   
The null sets can be defined
to be the sets of zero Lebesgue measure. The measurable sets can
be defined as the bounded Lebesgue measurable sets.  The volume of
a measurable set can be defined as its Lebesgue measure.

As we will see in a moment, we need considerably less than this.
We could, for instance,
take measurable to mean Jordan measurable.   We could define
measure to mean Jordan measure.  We could define null sets to be
finite unions of planes, spheres, and cones.



\subsection{properties of null sets}

We assume that have the following
properties.

\begin{enumerate}%[Null Set]
\item A finite union of null sets is a null set.\\
 \item A plane is a null set.\\
 \item A sphere is a null set.\\
 \item A circular cone is a null set; that is, a union of all
  lines through a fixed point $P$ and forming fixed
 forming fixed angle with a line through $P$.
\tlabel{enum:null}
\end{enumerate}

We write $A\equiv B$ if sets $A$ and $B$ are equal up to a null set.
That is, there exists a null set $E$ such that
   $(A\setminus B) \cup (B\setminus A) \subset E$.
\index{null set}\index{ZZZequiv@$\equiv$}

\subsection{properties of measurability}

\begin{enumerate}%[Measurable set]
 \item The union of two measurable sets is measurable.\\
 \item The intersection of two measurable sets is measurable.\\
 \item The difference of two measurable sets is measurable.
\tlabel{enum:measure}
\end{enumerate}

\subsection{properties of volume}

\begin{enumerate}%[Volume]
 \item The volume is defined for every measurable set.  It is
    a non-negative real number.
 \item If $X$ and $Y$ are  measurable, and if
 the symmetric difference of
 $X$ and $Y$ is contained in a null set, then 
    $X$ and $Y$ have the same volume.\\
 \item If $X$ and $Y$ are measurable sets, and if $X\cap
 Y$ is contained in a null set, then
    $$
    \op{vol}(X\cup Y) = \op{vol}(X) + \op{vol}(Y).
    $$
  \item (linear stretch) If $X\subset \ring{R}^3$, $t\in\ring{R}$, 
    $i=1,2,3$, and $e_i\in\ring{R}^3$ is the $i$th standard basis vector,
    set 
      $$T_i(X,t) = \{ u + (t-1) u_i e_i \mid u\in X\}.
      $$
    If $X$ is measurable, then $X'=T_i(X,t)$ is as well,
    and $\op{vol}(X') = |t|\op{vol}(X)$.
  \item (translation) If $X\subset \ring{R}^3$ and $v\in\ring{R}^3$, then let
    $X+v = \{x + v\mid x\in X\}$.  If $X$ is measurable, then $X+v$ is
    as well, and $\op{vol}(X) = \op{vol}(X+v)$.
\tlabel{enum:volume}
\end{enumerate}

In particular, if $X$ is contained in a null set, we may take
$X=Y$ in the preceding to deduce that $\op{vol}(X)=0$.

In addition to these properties, we will also need specific
volume calculations of primitive regions as described in
Lemmas~\ref{lemma:prim-volume} and~\ref{lemma:wedge-vol}.
%% NB Don't need lemma:wedge-sol because solid of a FR is a SC.
%% Don't need lemma:prim-sol, because solds are SC or ST.

\subsection{radial sets and solid angle}\label{sec:solid}

As mentioned in the introduction, it is possible to eliminate all
surface integrals by replacing them by volumes.  The most
important case of this is the surface area that goes by the name
`solid angle.'  This gives a definition of solid angle as a
volume.


\begin{definition}
    A set $C$ is $r$-radial at center $x$ if  $C\subset B(x,r)$
    and if
        $x + u \in C$ implies
        $x + t u \in C$ for all $t$ satisfying $0\le |u| t < r$.
A set $C$ is eventually radial at center $x$ if $C\cap B(x,r)$ is
$r$-radial at center $x$, for some $r>0$.
\end{definition}

\begin{lemma}\tlabel{lemma:r-r'}
Assume that $C$ is measurable and $r$-radial at $x$.  Let $0\le r'<r$,
then $C\cap B(x,r')$ is measurable and
$\op{vol}(C\cap B(x,r')) = \op{vol}(C) (r'/r)^3$.
\end{lemma}

\begin{proof}  We can transform $C$ into $C\cap B(x,r')$ by
a series of translations and stretch transformations.
\end{proof}


\begin{definition}\tlabel{def:sol}
If $C$ is measurable and eventually radial at center $x$, then we
define the solid angle of $C$ at $x$ to be
    $$
    \op{sol}(x,C) = 3 \op{vol}(C\cap B(x,r))/r^3,
    $$
where $r$ is as in the definition of eventually radial. 
By Lemma~\ref{lemma:r-r'}, this
definition is independent of any such $r$.  When the center $x$ is
clear from the context, we write $\op{sol}(C)$ for
$\op{sol}(x,C)$.
\end{definition}



The following properties follow immediately from the definitions.
If $C$ is $r$-radial for some $r>0$ then it is eventually radial.
If $C$ is measurable and $r$-radial, then the volume of $C$
satisfies
    $$
    \op{vol}(C) = \op{sol}(C) r^3/3.
    $$
If $C$ is bounded away from $x$, then $C$ is eventually radial at
$x$, and $\op{sol}(C) = 0$.








\section{Primitive Volumes}

There are only a few primitive volumes that will be needed.  All
other volumes that we need will be obtained from these with
Hilbert style scissor and congruence operations.  In this sense,
our discussion is hardly about measure at all.  The focus is rather on
the geometry of the various regions and how to decompose them into
primitives.

We prefer to take the volume of open sets whenever that can be
arranged.  We begin with a description of some of the primitive
regions.






\subsection{ball}

\begin{definition}  The open ball $B(x,r)$ with center $x$ and
radius $r$ is the set
    $$
    \{ y\in\ring{R}^3 \mid |x-y| < r.\}
    $$
\end{definition}


\subsection{lune}

The set $\op{aff}^0_+(\{v_0,v_1\},\{v_2,v_3\})$ was defined
in Definition~\ref{def:aff}.  We call it a lune.  It is the intersection
of two open half-spaces
    $$
    \op{aff}^0_+(\{v_0,v_1\},\{v_2,v_3\})
    =\op{aff}^0_+(\{v_0,v_1,v_2\},v_3)\cap
    \op{aff}^0_+(\{v_0,v_1,v_3\},v_2)
    $$

\subsection{wedge}

A lune has a dihedral angle $\dih(\{v_0,v_1\},\{v_2,v_3\})$ between
$0$ and $\pi$.   For larger angles than this, we use a wedge
$W(v_0,v_1,w_1,w_2)$.  Assume that $v_0\ne v_1$ and that
$w_1$ and $w_2$ do not lie on
the line $\op{aff}\{v_0,v_1\}$.  Set

$$
W(v_0,v_1,w_1,w_2) = 
\{x\not\in\op{aff}\{v_0,v_1\} \mid 0< \op{azim}(v_0,v_1,w_1,x) < \op{azim}(v_0,v_1,w_1,w_2)\}.
$$


\subsection{solid triangle}

\begin{definition} The solid triangle $ST(v_0,\{v_1,v_2,v_3\},r)$ is
specified by four points $v_i\in\ring{R}^3$, and a radius $r\ge0$. 
    $$
    ST(v_0,\{v_1,v_2,v_3\},r) = 
    B(v_0,r)\cap \op{cone}(v_0,\{v_1,v_2,v_3\}).
    $$
\end{definition}



\subsection{conic cap}

% renamed from spherical cap.

\begin{definition}
The conic cap $SC(v_0,v_1,r,a)$ is specified by an apex
$v_0\in\ring{R}^3$, a radius $r\ge0$, a non-zero vector $v_1-v_0$ giving
direction, and constant $a$.  The conic cap is the intersection of
the ball $B(v_0,r)$ with a solid right-circular cone:
    $$
    SC(v_0,v_1,r,a)=\{y \in B(v_0,r) \mid (y-v_0)\cdot (v_1-v_0) > |y-v_0|\, |v_1-v_0|\, a\}.
    $$
\end{definition}

\subsection{frustum}

\begin{definition} The frustum
$FR(v_0,v_1,h',h,a)$ is specified by an apex $v_0\in\ring{R}^3$, heights
$0\le h'\le h$, a vector $v_1-v_0$ giving its direction, and
$a\in[0,1]$. The set $FR$ is given as
    $$
    \{ y \mid (y-v_0)\cdot (v_1-v_0) > |y-v_0| |v_1-v_0|  a \ \ \land\ \
       h'|v_1-v_0| < (y-v_0) \cdot (v_1-v_0) < h|v_1-v_0| \}.
    $$
\end{definition}

When $h'=0$, the frustum extends to the apex, and
we write $FR(v_0,v_1,h,a)=FR(v_0,v_1,h',h,a)$.

\subsection{tetrahedron}

\begin{definition} A tetrahedron is a set of the form
$$\op{conv}^0\{v_1,v_2,v_3,v_4\}.$$
\end{definition}

By Lemma~\ref{tarski:hedra-tope}, this set can also be described
as the intersection of four open half-spaces, with each bounding
plane defined by three of the four points.
Taking this into account,
these sets can all be defined by linear and quadratic
constraints.

\subsection{primitive}

\begin{definition} A primitive region is any of the following.

\begin{enumerate}%%[Primitive Volumes]
 \item A solid triangle $ST$.
 \item A tetrahedron $S$.
 \item A wedge of a frustum (with $h'=0$); 
that is, the intersection of a frustum with
 a lune:
    $$
     FR(v_0,v_1,h,a) \cap \op{aff}^0_+(\{v_0,v_1\},\{v_2,v_3\}).
    $$
\item A wedge of a conic cap; that is, the intersection of a conic cap
with
    a lune:
    $$
    SC(v_0,v_1,r,c) \cap\op{aff}^0_+(\{v_0,v_1\},\{v_2,v_3\}).
    $$
\tlabel{enum:volume-prim}
\end{enumerate}

\end{definition}

\subsection{primitive volume calculation}

\begin{lemma}\tlabel{lemma:prim-volume} 
\begin{enumerate} 
 \item Let $v_1,v_2,v_3$ be unit vectors.
   A solid triangle $ST(v_0,\{v_1,v_2,v_3\},r)$ has volume
   $$
   (\alpha_{123}+\alpha_{231}+\alpha_{312}-\pi)r^3/3,
   $$
   where $\alpha_{ijk} = \dih_V(\{v_0,v_i\},\{v_j,v_k\})$.
  \item The conic cap $SC(v_0,v_1,r,a)$ has volume:
   $$
    2\pi(1-a) r^3/3,
   $$
 \item A frustum $FR(v_0,v_1,h,a)$ has volume:
   $$
   \pi (t^2-h^2) h/3,\quad h = t a.
   $$
 \item A tetrahedron $\op{conv}^0(\{v_1,v_2,v_3,v_4\})$ has volume:
   $$
   \sqrt{\Delta(x_{12},x_{13},x_{14},x_{34},x_{24},x_{23})}/12,
   $$
   where $x_{ij} = |v_i-v_j|^2$.
\end{enumerate}
\end{lemma}

Euler formula (Lemma~\ref{lemma:euler}) gives an
equivalent expression for $(\alpha_{123}+\alpha_{231}+\alpha_{312}-\pi)$.
Euler's formula will often be used instead of this formula.

\begin{proof}
The formula for the volume of a solid triangle is $r^3/3$ times
its solid angle.  The formula 
   $$\alpha_{123}+\alpha_{231}+\alpha_{312}-\pi$$
for the area of a spherical triangle is classical.    
The conic cap volume is
$r^3/3$ times its solid angle.  
The volume of a right-circular cone is $1/3$ its base times height.
The volume of a tetrahedron is
   $$|\det(v_2-v_1,v_3-v_1,v_4-v_1)|/6.$$
By Lemma~\ref{tarski:cm4}, 
the square of this determinant is given by a formula
$\op{CM}_4(x_{ij})$, which by Lemma~\ref{tarski:cm4} is
$\Delta/4$, with $\Delta\ge0$.  The result follows.
\end{proof}



\subsection{wedge}

If the region is realized by revolution along an axis $\op{aff}\{v_0,v_1\}$, 
then
we can also give the volume of the intersection of the region
with a wedge $W=W(v_0,v_1,v_2,v_3)$.
  In the following
let $\theta = \azim(v_0,v_1,v_2,v_3)$.


\begin{lemma}\tlabel{lemma:wedge-vol}  Let $C$ be either $SC(v_0,v_1,r,a)$ or
   $FR(v_0,v_1,h,a)$.  Let $m$ be the volume of $C$.  
   Then $C\cap W$ has volume $m\,\theta/(2\pi)$.   
\end{lemma}

\begin{lemma}\tlabel{lemma:wedge-sol}  Let $C$ be either $SC(v_0,v_1,r,a)$ or
   $FR(v_0,v_1,h,a)$.  Then $C$ and $C\cap A$ are eventually 
radial at $v_0$. Furthermore,
    $C\cap W$ solid angle 
  $s\,\theta/(2\pi)$, where $s$ is the solid angle of $C$.
\end{lemma}


\begin{proof}
These are elementary integrals.
\end{proof}


\subsection{solid angle of primitives}

All of the primitive sets are eventually radial, so we may take their
solid angle.  By Lemma~\ref{lemma:wedge-sol}, it is enough to compute
the solid angle before intersecting with a lune.

\begin{lemma} \tlabel{lemma:prim-sol}
\begin{enumerate}
    \item  $ST(v_0,\{v_1,v_2,v_3\})$ is eventually radial at $v_0$
     with solid angle 
     $$
     (\alpha_{123}+\alpha_{231}+\alpha_{312}-\pi),\quad
     \alpha_{ijk}=\dih_V(\{v_0,v_i\},\{v_j,v_k\}).
     $$
    \item $\op{conv}^0\{v_0,v_1,v_2,v_3\}$ is eventually radial at $v_0$
      with solid angle
           $$
     (\alpha_{123}+\alpha_{231}+\alpha_{312}-\pi),\quad
     \alpha_{ijk}=\dih_V(\{v_0,v_i\},\{v_j,v_k\}).
     $$
    \item $SC(v_0,v_1,r,a)$ is eventually radial at $v_0$ with solid
      angle 
      $2\pi(1-a)$.
    \item $FR(v_0,v_1,h,a)$ is eventually radial at $v_0$ with solid
      angle
        $$
        2\pi (1-a).
        $$
\end{enumerate}
\end{lemma}

\begin{proof} In every case, the intersection of 
  the region with $B(v_0,r')$, for $r'>0$ sufficiently small, is
  a conic cap or a solid triangle.  These two volumes have
  already been calculated.  This gives the results as stated.
\end{proof}

\subsection{combining solid angle and volume}

It is often convenient to consider various linear combinations
of the solid angle and volume of eventually radial sets.  

\begin{definition}\label{def:sovo}
With
that in mind, we define the function
  $$
  \op{sovo}(v_0,V,\lambda) = \lambda_v \op{vol}(X) + \lambda_s
  \op{sol}(v_0,V),
  $$
where $V$ is a measurable set that is eventually radial at $v_0$
and $\lambda=(\lambda_v,\lambda_s)$ is a pair of real numbers
determining the linear combination.
Two particularly important choice of parameter $\lambda$ are
$$
 \begin{array}{lll}
 \lambda_{oct}&=(\lambda_v,\lambda_s)=(-4\doct,1/3)\\
 \lambda_{sq}& =(\lambda_v,\lambda_s)=(4\doct,\zeta\,\pt-1/3)\\
 \end{array}
$$
\index{sovo}\index{ZZlambda@$\lambda$}
\index{ZZlambdaoct@$\lambda_{oct}$}
\index{ZZlambdasq@$\lambda_{sq}$}
\end{definition}

We define some auxiliary functions that will help us express
the value of $\op{sovo}$ on primitive regions.

\begin{definition}\tlabel{def:A}\tlabel{def:phi}
Define the function
 $$
 \phi(h,t,\lambda)=
   \lambda_v  t h (t+h)/6 + \lambda_s 
 $$
Define the function
 $$A(h,t,\lambda) = (1-h/t) (\phi(h,t,\lambda) - \phi(t,t,\lambda)).$$
\index{A}
\index{phi}
\end{definition}

\begin{lemma} If $V$ is measurable and $t$-radial at $v_0$,
then $$\op{sovo}(v_0,V,\lambda) = \op{sol}(v_0,V)\phi(t,t,\lambda).$$
\end{lemma}

\begin{proof} We have $$\vol(V) = \op{sol}(v_0,V)t^3/3,$$
so $$\op{sovo}(v_0,V,\lambda) = 
  \op{sol}(v_0,V)(\lambda_v t^3/3 + \lambda_s) = 
   \op{sol}(v_0,V)\phi(t,t,\lambda).$$
\end{proof}

\begin{lemma}\tlabel{lemma:sovoFR} 
  We have
  $$
  \begin{array}{lll}
  \op{sovo}(v_0,FR(v_0,v_1,h,h/t),\lambda) 
   &= \op{sol}(v_0,FR(v_0,v_1,h,h/t))\\
  \phi(h,t,\lambda) 
   &= 2\pi A(h,t,\lambda) + \sol(v_0,FR(v_0,v_1,h,h/t))\phi(t,t,\lambda).
  \end{array}
  $$
\end{lemma}

\begin{proof}
We have 
  $$
  \begin{array}{lll}
  \op{sol}(v_0,FR(v_0,v_1,h,h/t)) &= 2\pi(t-h)/t,\\
  \op{vol}(FR(v_0,v_1,h,h/t)) &= \pi(t^2-h^2)h/3\\
      &= \sol\cdot t h (t+h)/6,\\
  \op{sovo}(v_0,FR(v_0,v_1,h,h/t)) &= 
     \lambda_v \sol\cdot t h (t+h)/6 + \lambda_s \sol\\
   &= \sol\cdot \phi(h,t,\lambda).
  \end{array}
  $$
Also,
  $$
  \begin{array}{lll}
  \sol\cdot\phi(h,t,\lambda) &= 2\pi (1-h/t)\phi(h,t,\lambda)\\
  &= 2\pi A(h,t,\lambda) + 2\pi (1-h/t)\phi(t,t,\lambda)\\
  &= 2\pi A(h,t,\lambda) + \sol\cdot \phi(t,t,\lambda).
  \end{array}
  $$
\end{proof}

\section{Scissors and Volumes}
\tlabel{sec:measure-second}

There are many other volumes that can be computed from the
primitive ones enumerated in Definition~\ref{enum:volume-prim}.

\subsection{lune}  

To give a simple example of a derived volume, we consider the
lune $A=\op{aff}_+^0(\{v_0,v_1\},\{v_2,v_3\})$.  It is eventually
radial at $v_0$, so we may compute its solid angle.

\begin{lemma}  $\sol(v_0,A) = 2\dih_V(\{v_0,v_1\},\{v_2,v_3\})$.
\end{lemma}

\begin{proof}
Let $B_{\pm} = \op{aff}_\pm(\{v_0,v_2,v_3\},v_1)$.  $B_- \cap B_+$
is a null set.  The intersections $A\cap B_{\pm}\cap B(v_0,r)$ 
are solid triangles.  This gives the solid angle of $A$ as
follows:
   $$\begin{array}{lll}
   \sol(v_0,A) &= \sol(v_0,A\cap B_+)+\sol(v_0,A\cap B_-) \\
   &= 
   \sol(ST(v_0,\{v_1,v_2,v_3\})) + \sol(ST(v_0,\{v_1',v_2,v_3\})) \\
   &=
   2\dih_V(\{v_0,v_1\},\{v_2,v_3\}).
   \end{array}
   $$
Here, we have used $v_1'= 2 v_0 - v_1$, the reflection of $v_1$
through $v_0$.  The usual calculation of the volume of a solid triangle
inverts this proof, 
and derives the volume from the solid angle of a lune.
\end{proof}



\begin{lemma}\tlabel{lemma:wedge:sol} 
Assume that the sets $\{v_0,v_1,w_1\}$ and
$\{v_0,v_1,w_w\}$ are not collinear. 
$$\sol(v_0,W(v_0,v_1,w_1,w_2)) = 2\op{azim}(v_0,v_1,w_1,w_2).$$
\end{lemma}    

\begin{proof} Every wedge is a union of two lunes, up to a null set.
\end{proof}

\begin{lemma}  
   $$
   \begin{array}{lll}
    \op{sol}(v_0,B(v_0,r)) &= 4\pi\\
    \op{vol}(v_0,B(v_0,r)) &= 4\pi r^3/3\\
   \end{array}
   $$
\end{lemma}

\begin{proof}
The ball is $r$-radial at $v_0$, so the volume is given by
Definition~\ref{def:sol} in terms of solid angle.  It is enough
to check that a hemisphere has solid angle $2\pi$.  This follows
from Lemma~\ref{lemma:wedge:sol}.
\end{proof}  




\subsection{Rogers simplex}

\begin{definition} \tlabel{def:ortho}
An {\it orthosimplex} is a tetrahedron
    $$\op{conv}^0(x,x+v_1,x+v_1+v_2,x+v_1+v_2+v_3),$$
where $v_i\cdot v_j=0$, for $1\le i<j\le 3$.   We write
$\op{orth}^0(x,v_1,v_2,v_3)$ for this orthosimplex.
 \index{orthosimplex}
\end{definition}

\begin{figure}[htb]
  \centering
  \myincludegraphics{\ps/rogers.eps}
  \caption{The Rogers simplex is an orthosimplex.}
  \tlabel{fig:rogers}
\end{figure}


\begin{definition} \tlabel{def:rog}
Let $\{v_0,v_1,v_2,v_3\}$ be a set of four points in $\ring{R}^3$.
Assume that they are not coplanar.  Let $p$ be the circumcenter
of $\{v_0,v_1,v_2\}$ and $r$ its circumradius (see Definition~\ref{def:circumrad2}).  Let $c\ge r$.
By Lemma~\ref{tarski:rog-exist}, there exists a unique
point $p'$ in $A=\op{aff}_+(\{v_0,v_1,v_2\},v_3)$ at equal distance $c$
from $v_0,v_1,v_2$.
Let $$
    \op{rog}^0(v_0,v_1,v_2,v_3,c) = 
    \op{ortho}^0(v_0,w_1,w_2,w_3),
    \quad w_1=(v_0+v_1)/2,\quad w_1+w_2=p,\quad w_1+w_2+w_3=p'.
    $$
(We define $\op{rog}(v_0,\ldots,v_3,c)$ similarly, where we use
$\op{conv}$ instead of $\op{conv}^0$.)
We take $\op{rog}^0$ to be the empty set, if $c< r$.
 \index{rogers simplex}
\end{definition}



\begin{lemma} The vectors $w_1,w_2,w_3$ of Definition~\ref{def:ortho}
are indeed mutually orthogonal.
\end{lemma}

\begin{proof} The orthogonality of $w_1$ and $w_2$ is found in
Lemma~\ref{tarski:eta-ortho}.  The orthogonality of $w_3$ with the
others is found in Lemma~\ref{tarski:rog-ortho}.
\end{proof}

\begin{definition}\tlabel{def:abc}
We associate with $\op{rog}^0(v_0,v_1,v_2,v_3,c)$ the constants
$a=|v_1-v_0|/2$, $b=\eta(v_0,v_1,v_2)$, and $c$.
We call these the $abc$ parameters of $\op{rog}^0$. 
\end{definition}

The Rogers simplex is a tetrahedron.  Hence it is one of our
primitive regions.  It is eventually radial at $v_0$, hence
it has a solid angle at $v_0$.  When we mention its dihedral
angle, it is understood that it refers to 
   $$
   \dih_V(\{v_0,v_1\},\{v_2,p'\})=\dih_V(\{v_0,(v_0+v_1)/2\},\{p,p'\}),
   $$
where $p$ and $p'$ are the points 
constructed in Definition~\ref{def:rog}.

The squares of the edge lengths of the tetrahedron are
   $$
   (a^2,b^2,c^2,c^2-b^2,c^2-a^2,b^2-a^2).
   $$
Define the two functions
   $$
   \begin{array}{lll}
     \op{volR}(a,b,c) &= \begin{cases}
       a\sqrt{(b^2-a^2)(c^2-b^2)}/6& 0 < a < b < c,\\
       0&,\text{otherwise}
       \end{cases}\\
     \op{solR}(a,b,c) &= \begin{cases}
      2 \arctan\left(\sqrt{\frac{(b-a) (c-b)}{(a+b)
   (b+c)}}\right).& 0 < a < b < c,\\
      0&,\text{otherwise}
     \end{cases}
     \end{array}
   $$
Specializing the formulas for volume and solid angle to this
setting we get the following expressions for volume and solid angle.
(The calculation of the
 solid angle formula is based on Euler's formula in 
Lemma~\ref{lemma:euler}.)

\begin{lemma} Let $R=\op{rog}^0(v_0,v_1,v_2,v_3,c)$ and let $a$, $b$,
$c$ be the $abc$-parameters of $R$.  Then
$$
\begin{array}{lll}
\op{vol}(R) &= \op{volR}(a,b,c)\\
\op{sol}(v_0,R) &= \op{solR}(a,b,c)\\
\end{array}
$$
\end{lemma}





\begin{remark}
The volume of a unit cube aligned along the coordinate axes is $1$.  
If we want to insist on deriving all volumes from the primitive
volumes, then we can derive the volume of the cube by partitioning
it into six Rogers simplices,
each of volume $\op{vorR}(1,\sqrt2,\sqrt3) = 1/6$, for a total
of $1$, as desired.  Figure~\ref{fig:rogers} shows one of the six
Rogers simplices.
\end{remark}



\subsection{Rogers's lemma}


The following lemma is the key step in the proof of Rogers's
bound on the density of sphere packings \cite{Rog58}.

\begin{lemma} \tlabel{lemma:rogers}
Suppose that $a,b,c$ and $a',b',c'$
are real numbers that satisfy $0 <a \le b \le c$, $0 \le a'\le b'\le c'$,
$a \le a'$, $b \le b'$, $c \le c'$. Then
  $$
  \op{solR}(a',b',c')\op{volR}(a,b,c) \le \op{solR}(a,b,c)\op{volR}(a',b',c').
  $$
\end{lemma}

\begin{proof} If any of the equalities hold: $a=b$, $b=c$, $a'=b'$,
$b'=c'$, then both sides are zero.  We assume $a<b<c$ and $a'<b'<c'$.
Let $w_1=a\,e_1,w_2=\sqrt{b^2-a^2}\, e_2,w_3=\sqrt{c^2-b^2}\, e_3,$
for the standard basis $e_i$.  Each point of the orthosimplex
$S = \op{ortho}^0(0,w_1,w_2,w_3)$ has
the form
   $$s(t_1,t_2,t_3) = t_1 w_1 + t_2 w_2 + t_3 w_3$$
where $t_i>0$ and $t_1+t_2+t_3< 1$.  Similarly,
we define $w'_i$, $S'$, and $s'(t_1,t_2,t_3)$ for the primed objects.

The inequality of the lemma is equivalent to
  $$
  \frac{\sol(0,S')}{\op{vol}(S')} \le \frac{\sol(0,S)}{\op{vol}(S)}.
  $$
By the scaling properties of the measure $\op{solR}$ and $\op{volR}$
scale by the same factor under linear stretching along coordinate axes.
By such a transformation, $S'$ can be transformed to $S$.  The
transformation $T$ is given by 
   $$
   T s'(t_1,t_2,t_3) = s(t_1,t_2,t_3).
   $$
Under this transformation $T$, the volumes become equal.
The desired inequality follows from
   $$
   \sol(0,T S') \le \sol(0,S).
   $$
This follows if the transformation $T$ satisfies
   $T(S'\cap B(0,r))\subset S\cap (0,r)$.
By Lemma~\ref{tarski:rog-lemma}, we have that 
   $$|s(t_1,t_2,t_3)|\le |s'(t_1,t_2,t_3)|.$$
This means that $T$ carries each point of $S'$ to a point closer to
the origin.  In particular,
  $T(S'\cap B(0,r))\subset S\cap B(0,r)$.
\end{proof}

%% WW Repeated parts of def.
\begin{definition}  Define 
  $$
  \begin{array}{lll}
  \dtet &= \sqrt8 \arctan(\sqrt2/5)\\
  \doct &= \pi/\sqrt8 - \sqrt2 \arctan(\sqrt2/5)\\
  \delta(a,b,c)&= \op{solR}(a,b,c)/(3\op{volR}(a,b,c)),
  \end{array}
  $$
for $a<b<c$.  
\index{ZZdeltatet@$\dtet$}
\index{ZZdeltaoct@$\doct$}
\index{ZZdelta@$\delta$}
\end{definition}

\begin{lemma}\tlabel{lemma:doct-calc}
  $\delta(1,2/\sqrt{3},\sqrt2)=\doct$.
\end{lemma}

\begin{proof}  In this calculation, we do not use Euler's formula
for the solid angle (Lemma~\ref{lemma:euler}).  
Use the dihedral angle formula instead.
A calculation gives
  $$
  \doct(1,2/\sqrt{3},\sqrt2)=
  3 \sqrt{2} \left(\frac{\pi }{4}-\arctan
   \left(\frac{1}{\sqrt{2}}\right)\right).$$
To complete the proof, we need the trig identity
  $$\arctan(\sqrt2/5)  = 3\arctan(\sqrt2/2)-\pi/2.$$
Both sides are between $0$ and $\pi/2$.  Thus, we can prove this
by taking the tangent of both sides. By the addition formula
(Lemma~\ref{lemma:tan-add}),
if $x=\arctan(\sqrt2/2)$, then
   $$\tan(3 x) = \frac{\tan^3(x) - 3\tan(x)}{1-3 \tan^2(x)} = -5/\sqrt2.$$
The result follows.
\end{proof}

\begin{lemma}\tlabel{lemma:dtet-cal}
  $\delta(1,2/\sqrt3,\sqrt6/2)=\dtet$.
\end{lemma}

\begin{proof} Calculating as in Lemma~\ref{lemma:doct-calc}, and using
the same trig identity, we get
$$\begin{array}{lll}
  \delta &=
2\sqrt{2} \left(3\arctan\left(\sqrt2/2\right) - \pi/2)\right),\\
  &=2\sqrt{2}\left(\arctan\left(\sqrt2/5\right)\right),\\
  &=\dtet
\end{array}
$$
\end{proof}

\begin{lemma}\tlabel{lemma:rog-doct}
Suppose $1\le a\le  b\le c$,  $2/\sqrt{3}\le b$, and $\sqrt2\le c$.  Then
$$
\delta(a,b,c) \le \doct.
$$
\end{lemma}

\begin{proof} This follows from Lemma~\ref{lemma:rogers} and
Lemma~\ref{lemma:doct-calc}.
\end{proof}

\begin{lemma}\tlabel{lemma:rog-tet}
Let $1\le a \le b \le c$, $2/\sqrt{3}\le b$ and $\sqrt6/2\le c$.
Then $\delta(a,b,c) \le  \dtet$.
\end{lemma}

\begin{proof}  This follows from Lemma~\ref{lemma:rogers} and
Lemma~\ref{lemma:dtet-cal}.
\end{proof}

By Lemma~\ref{tarski:eta-root3}, the circumradius of a triangle
with sides at least $2$ is always at least $\eta(2,2,2)=2/\sqrt3$.



\subsection{quoin}

Define the function $\op{quovol}$ when $0<a<b<c$ by
    \begin{equation}
    \begin{array}{lll}
    6\op{quovol}(a,b,c) &= (a+2c)  %
    % -(a^2+ac-2c^2)
    (c-a)^2\arctan(e)
        +a(b^2-a^2)e\\&-4c^3\arctan(e(b-a)/(b+c)),
    \tlabel{eqn:3.3}
    \end{array}
    \end{equation}
where $e\ge0$ is given by $e^2(b^2-a^2)=(c^2-b^2)$.
Extend the function by $0$, when the condition $0<a<b<c$ fails.


\begin{definition}\label{def:quoin}
Let $\{v_0,v_1,v_2,v_3\}$ be a set of four points in $\ring{R}^3$.
Let $c>  \eta(v_0,v_1,v_2)$.  Let $p$ be the circumcenter
of $\{v_0,v_1,v_2\}$.  Let $p'$ be the point 
in $\op{aff}_+(\{v_0,v_1,v_2\},v_3)$ at equal distance $c$
from $v_0,v_1,v_2$ (given by Lemma~\ref{tarski:mk-point}).
We define
$\op{quo}(v_0,v_1,v_2,v_3,c)$ to be the following set 
(Figure~\ref{fig:quoin}):
   $$
   B(v_0,c) \cap \op{aff}_+^0(\{v_0,v_1,v_2\},v_3)
   \cap \op{aff}_-^0(\{v_0,p,p'\},v_1) \cap
   \op{aff}_-^0(\{v_1,p,p'\},v_0).
   $$
We associate with this set the $abc$-parameters, defined
by $a = |v_0-v_1|/2$, $b=|p-v_0|$, $c$ (as given).
\end{definition}

\begin{figure}[htb]
  \centering
  \myincludegraphics{\ps/quoin.eps}
  \caption{The quoin above a Rogers simplex is the part of the
  shaded solid outside
   the illustrated box.  It is bounded by the two
  shaded planes, the plane through
   the front face of the box, and a sphere
   centered at the origin passing through the opposite corner of the box.}
  \tlabel{fig:quoin}
\end{figure}



\begin{lemma}\tlabel{lemma:quo-vol}
Let $Q=\op{quo}(v_0,v_1,v_2,v_3,c)$. Let $(a,b,c)$ be the
$abc$-parameters of $Q$.  
Then $$\op{vol}(Q) = \op{quovol}(a,b,c).$$
%
 \index{quoin}
\end{lemma}

\begin{proof} We give the proof in some detail, because it illustrates
our method of calculating derived volumes from primitive volumes.
The calculations are essentially formal.

Let $\chi_X$ be the characteristic
function of $X$.  If $P = \chi_X$, write $\bar P$ for the characteristic
function of the complement of $X$.  We consider characteristic functions
only up to a null set, and this means that we can ignore issues such
as whether we take open half-spaces or closed half-spaces and so forth.

Set
$$
\begin{array}{llll}
  A &= \chi_X,\quad X = B(v_0,c)&\text{(ball)}\\
  B &= \chi_X,\quad X = \op{aff}_+^0(\{v_0,v_1,v_2\},v_3)&\text{(front)}\\
  C &= \chi_X,\quad X = \op{aff}_-^0(\{(v_0+v_1)/2,p,p'\},v_0)
    &\text{(top)}\\
  D &= \chi_X,\quad X = \op{aff}_-^0(\{v_0,p,p'\},v_1)&\text{(diag)}\\
  E &= \chi_X,\quad X = \op{aff}_+^0(\{v_0,v_1,p'\},p)&\text{(diag)}\\
  F &= \chi_X,\quad X = \op{rcone}^0(v_0,v_1,a/c)\\
\end{array}
$$

We see from Figure~\ref{fig:quoin} that we have the following implications
$(f(x)=1)\implies (g(x)=1)$ when $f,g$ are any of the following characteristic
functions
  $$
   (f,g) = (B\bar C \bar D E,A),\quad
   (B\bar C E F,A),\quad (A B C D, E),\quad (A B C E,F).
  $$
(These implications are justified without pictures in Lemmas~\ref{tarski:BCDE},\ref{tarski:BCEF},\ref{tarski:ABCD}, and~\ref{tarski:ABCE}.)
We recognize $ABEF$ as the characteristic function $[SC]$ of a
conic cap, $B\bar C E F$ as the characteristic function $[WFR]$ of a
wedge of a frustum,  $AB\bar D E$ as the characteristic function $[ST]$
of a solid triangle, $B\bar C\bar D E$ as the characteristic function $[R]$
of a Rogers simplex.  The characteristic function $[Q]$
of the quoin is given
by $A B C D$.  We let $[X]$ be the characteristic function of
$A B C \bar D E$.

We then have formally that
$$
\begin{array}{lllll} \,[SC] &=  A B E F \\
     &= A B \bar C E F &+ A B C E F\\
     &= B \bar C E F &+ A B C D E F &+ A B C \bar D E F\\
     &= [WFR] &+ A B C D &+ A B C \bar D E\\
     &= [WFR] &+ [Q] &+ [X]\\ 
     \,[ST] &= A B \bar D E\\
     &= A B C \bar D E &+ A B \bar C \bar D E\\
     &= [X] &+ B \bar C \bar D E\\
     &= [X] &+ [R].
\end{array}
$$
Solving for $[Q]$, we get
\begin{equation}\tlabel{eqn:qr}
  [Q] = [SC] - [WFR] + [R] - [ST].
\end{equation}
Thus, the volume of a quoin is expressed in terms of primitive volumes.
Substituting the given formulas for the volumes of primitives, we obtain
the result.  (It is necessary to use the Euler formula for solid
angle.)
\end{proof}

\begin{remark}  This type of analysis can be turned into an algorithm
for computing regions described by quadratic constraints in terms
of primitive volumes \cite{quad}.  All the volumes that arise in the
proof of the Kepler conjecture can be computed by this algorithm.
In particular, the proof of Lemma~\ref{lemma:quo-vol}.
\end{remark}

\begin{remark}\tlabel{rem:RQ}  
We can rewrite Equation~\ref{eqn:qr} as
$$
  [ST]-[R] = ([SC]-[WFR]) - [Q].
$$
That is, as we can see from Figure~\ref{fig:quoin}, the region
in a ball above and outside a Rogers simplex is the same as the
region in a ball above and outside a frustum and outside the quoin.
\end{remark}

\begin{lemma}\tlabel{lemma:solquo}  The quoin
$\quo(v_0,v_1,v_2,v_3,c)$ is eventually radial at $v_0$ and
has solid angle $0$.
\end{lemma}


\begin{proof}  This is trivial, because the quoin is bounded
away from $v_0$.
\end{proof} 

Consider the function
$\op{sovo}(v_0,\quo(v_0,v_1,v_2,v_3,c),\lambda)$, expressed
as a function of the $abc$-parameters of the quoin.  By
Lemma~\ref{lemma:solquo}, the contribution from the solid angle
is zero, so that
$$
\op{sovo}(v_0,\quo(v_0,v_1,v_2,v_3,c),\lambda) = 
 \lambda_v \op{quovol}(a,b,c).
$$


\begin{lemma}\tlabel{lemma:sovo:rog}  Let $R = \op{rog}^0(v_0,
v_1,v_2,v_3,c)$.  Let $(a,b,c)$ be the $abc$-parameters of $R$.
Let $Q=\quo(v_0,v_1,v_2,v_3,c)$.  Let $\op{dih}(R)$ be the
dihedral angle of $R$ along $\{v_0,v_1\}$.
Then
  $$
  \begin{array}{lll}
  \op{sovo}(v_0,R,\lambda) = \sol(v_0,R)\phi(c,c,\lambda) +
  \op{sovo}(v_0,Q) + \op{dih}(R)\cdot A(a,c,\lambda).
  \end{array}
  $$
\end{lemma}

\begin{proof}
By the identity of Remark~\ref{rem:RQ}, we have
  $$
  \op{sovo}(v_0,R,\lambda) - \op{sovo}(v_0,Q) = 
  \op{sovo}(v_0,ST,\lambda) - \op{sovo}(v_0,SC,\lambda) + 
  \op{sovo}(v_0,WFR,\lambda),
  $$
for regions $ST$, $SC$, $WFR$ described in that remark.
We have
  $$
  \begin{array}{lll}
  \op{sovo}(v_0,ST,\lambda) &= \sol(v_0,ST)\phi(c,c,\lambda) &= 
  \sol(v_0,R)\phi(c,c,\lambda)\\
  \op{sovo}(v_0,WFR,\lambda) &= \op{sol}(v_0,FR)\cdot\frac{\dih(R)}{2\pi}
   \phi(a,c,\lambda)\\
  \op{sovo}(v_0,SC,\lambda) &= \op{sol}(v_0,FR)\cdot\frac{\dih(R)}{2\pi}
   \phi(a,c,\lambda)\\ 
  \op{sol}(v_0,FR) (\phi(a,c,\lambda) -\phi(c,c,\lambda) &=
   2\pi A(a,c,\lambda).
  \end{array}
  $$
The result follows from the equations.
\end{proof}



%\subsection{caps}
%
%\begin{lemma}\tlabel{lemma:cap-rogers}
%Let $B(0,t)$ be a ball of radius $t$ centered at the origin.  Let
%$v_1$ and $v_2$ be vertices.  Assume that $|v_1|< 2t$ and $|v_2|<2
%t$.  Truncate the ball by cutting away the caps
%   $$\op{cap}_i = \{x\in B(0,t) :  |x- v_i| < |x|\}.$$
%Assume that the circumradius of the triangle $\{0,v_1,v_2\}$ is
%less than $t$. Then the intersection of the caps, $\op{cap}_1\cap
%\op{cap}_2$, is the union of four quoins.
%\end{lemma}
%
%\begin{proof} This is true by inspection.  See Figure~\ref{fig:capriquoin}.
%Slice the intersection $\op{cap}_1\cap\op{cap}_2$ into four pieces
%by two perpendicular planes: the plane through $\{0,v_1,v_2\}$,
%and the plane perpendicular to the first and passing through $0$
%and the circumcenter of $\{0,v_1,v_2\}$.  Each of the four pieces
%is a quoin.
%\end{proof}
%
%\begin{figure}[htb]
%  \centering
%  \myincludegraphics{\ps/capriquoin.eps}
%  \caption{The intersection of two caps on the unit ball can
%   be partitioned into four quoins (shaded).}
%  \tlabel{fig:capriquoin}
%\end{figure}
%

\subsection{truncating Rogers}

\begin{lemma}\tlabel{lemma:sovo:truncRog}
Let $R'=\op{rog}^0(v_0,v_1,v_2,v_3,c)$.  Let $r$ be the
circumradius of $\{v_0,v_1,v_2\}$.  Assume that $r$, $t$,
and $c$ are real numbers that satisfy $r\le t\le c$.
Let $d$ be the dihedral angle of $R'$ along the edge $\{v_0,v_1\}$.
Then
   $$
   \op{sovo}(v_0,R'\cap B(v_0,t),\lambda) = 
   d [(1-\cos\psi)\phi(t,t,\lambda)+A(h,t,\lambda)]
   -s \phi(t,t,\lambda) + \lambda_v \op{quovol}(a,b,t),
   $$
where $s = d (1-\cos\psi) - \sol(v_0,R')$ and $\cos\psi = y/(2c)$.
\end{lemma}

\begin{proof}  Let $R = \op{rog}^0(v_0,v_1,v_2,v_3,t)$.   Let
$q$ (resp. $q'$) be the point at equidistance $t$ 
(resp. $c$) from $v_0,v_1,v_2$ in
$\op{aff}_+(\{v_0,v_1,v_2\},v_3)$.
The point $q$ (resp. $q'$) is an extreme point of $R$ (resp. $R'$).
Let 
  $$
  \begin{array}{lll}
  W_1 = W(v_0,v_1,v_2,q), \quad W_2 = W(v_0,v_1,q,q').\\
  A = \{x \mid (x-v_0)\cdot (v_1-v_0) > |x-v_0| |v_1-v_0| (y/2t)\},\\
  \bar A = \{x \mid (x-v_0)\cdot (v_1-v_0) < |x-v_0| |v_1-v_0| (y/2t)\},\\
  \end{array}
  $$
We have the following relations by Lemma~\ref{tarski:XX}:
XX Put these in Tarski in the Volume section.
  $$
  \begin{array}{lll}
  W_1 \cap R' = R,\quad R\cap B = R\\
  R' \cap (W_1\cup W_2) \equiv R',\quad
  W_1\cap W_2 = \emptyset\\
  W_2\cap R'\cap B\cap A = FR(v_0,v_1,h,y/(2t))\cap W_2\\
  \end{array}
  $$
Also, $W_2\cap R'\cap B\cap\bar A$ is measurable and $t$-radial
at $v_0$.
For simplicity, surpress the parameters $v_0$ and $\lambda$ from
$\op{sovo}$.  It follows that
  $$
  \begin{array}{lll}
  \op{sovo}(R'\cap B) &= \op{sovo}(R'\cap B\cap W_1) + 
  \op{sovo}(R'\cap B\cap W_2)\\
  &= \op{sovo}(R)+ \op{sovo}(R'\cap B\cap W_2\cap A) +
  \op{sovo}(R'\cap B\cap W_2\cap \bar A).\\
  \op{sovo}(R) &=\sol(R)\phi(t,t,\lambda) + \op{sovo}(Q) +
     \dih(R) A(h,t,\lambda).\\
  \op{sovo}(R'\cap B\cap W_2\cap A) &= \op{sovo}(FR(v_0,v_1,h,y/(2t))\cap W_2)\\
   &= \dih(R'') A(h,t,\lambda) + \sol(W_2\cap R'\cap B\cap A)\phi(t,t,\lambda).\\
   \op{sovo}(R'\cap B\cap W_2\cap \bar A) &= 
        \sol(R'\cap B\cap W_2\cap \bar A) \phi(t,t,\lambda)\\
   &= (\sol(R)-\sol(R'\cap B\cap W_2\cap A) -\sol(R')\phi(t,t,\lambda).
  \end{array}
  $$
Combining these equations, we get the result.
XX ADD external references for intermediate results, such as lemma:sovo:rog.
XX Clean up notation, h vs. y, etc.
\end{proof}

\begin{lemma}\label{lemma:truncRog:noQ}
Let $R=\op{ortho}^0(w_0,v_1,v_2,v_3)$ and $B=B(w_0,t)$.
Assume that $|v_1|< t < |v_1+v_2|$.  Then
$$
  \begin{array}{lll}
  \op{sovo}(w_0,B\cap R,\lambda) &=
  \sol(R)\phi(t,t,\lambda) + \dih(R)(1-y/(2t))(\phi(y/2,t,\lambda)
  -(\phi(t,t,\lambda)))\\
  &=\sol(R)\phi(t,t,\lambda) + \dih(R) A(y/2,t,\lambda)\\
  \end{array}
$$
\end{lemma}

We note that this formula is that obtained from Lemma~\ref{lemma:sovo:truncRog} by setting the quoin term to zero.

\begin{proof}
Set
  $$
  \begin{array}{lll}
  A &= \{x \mid (x-v_0)\cdot (v_1-v_0) > |x-v_0| |v_1-v_0| (y/2t)\},\\
  \bar A &= \{x \mid (x-v_0)\cdot (v_1-v_0) < |x-v_0| |v_1-v_0| (y/2t)\},\\
  W &= W(w_0,w_0+v_1,w_0+v_1+v_2,w+0+v_1+v_2+v_3\\ 
  \end{array}
  $$
We have that $R\cap (A\cap \bar A)\equiv R$ and $A\cap \bar A=\emptyset$.
Also, by Lemma~\ref{tarski:XX},
XX Put these in Tarski in the Volume section.
  $$
  \begin{array}{lll}
  R\cap B\cap A &= FR(v_0,v_1,h,y/(2t))\cap W\\
  \end{array}
  $$
Also, $R\cap B\cap\bar A$ is measurable and $t$-radial
at $v_0$.

We have
  $$
  \begin{array}{lll}
  \op{sovo}(B\cap R) &= \op{sovo}(A\cap B\cap R) + \op{sovo}(\bar A\cap B\cap R) \\
   \op{sovo}(\bar A\cap B\cap R) &= \sol(\bar A\cap B\cap R) \phi(t,t,\lambda)\\
    &= (\sol(B\cap R)-\sol(A\cap B\cap R))\phi(t,t,\lambda)\\
    &= (\sol(B\cap R)-\sol(FR\cap W))\phi(t,t,\lambda)\\
  \op{sovo}(A\cap B\cap R) &= \op{sovo}(FR\cap W)\\
     &=\sol(FR\cap W)\phi(y/2,t,\lambda)\\
   \sol(FR\cap W) &= \dih(R)\sol(FR)/(2\pi)\\
  \sol(FR) = 2\pi (1-y/(2t))\\
  \end{array}
  $$
These identities combine to give the proof.
\end{proof}

\section{Aggregate Regions}\tlabel{sec:tcc}
    %\oldlabel{4.10}

We will consider in this section several different types
of regions that are aggregates of pieces that have already been
considered.  The various regions share certain features.

The constructions will depend on points
 $v_0,v_1,w_1,w_2\in\ring{R}^3$. Assume that $\{v_0,v_1,w_1,w_2\}$
is not coplanar.  We also choose constants $c_1,c_2>0$.


Let
$W=W(v_0,v_1,w_1,w_2)$ be the corresponding wedge.  
Let $p_i$ be the circumcenter of
$\{v_0,v_1,w_i\}$ and $b_i$ its circumradius.
Let $u_i$ be the normal to $\{v_0,v_1,w_i\}$, directed so that
$(-1)^i\azim(v_0,v_1,w_i,w_i+u_i) < 0$.  That is, $u_i$ points
from the half-plane $\op{aff}_+^0(\{v_0,v_1\},w_i)$ into the wedge $W$.

For $c_i > b_i$, let $q_i = q(c_i) = p_i + s_i u_i$ (with $s_i>0$) be the
unique point that is equidistant $c_i$ from $v_0,v_1,w_i$.
(The unique existence of $q_i$ is established by Lemma~\showref{tarski:XX}.)
If $c_i\le b_i$, we set $c_i=w_i$.

By the choice of normals $u_i$, we have that
 $$
 0\le \op{azim}(v_0,v_1,w_1,q_1)\quad
 \op{azim}(v_0,v_1,w_1,q_2) \le \op{azim}(v_0,v_1,w_1,w_2).
 $$
Equality occurs exactly when $c_1\le b_1$ (resp. $c_2\le b_2$).

We make the assumption that
\begin{equation}\tlabel{eqn:q1q2}
\op{azim}(v_0,v_1,w_1,q_1) \le \op{azim}(v_0,v_1,w_1,q_2).
\end{equation}
(This assumption will be briefly lifted in Section~\showref{XX}.)

We define the wedges 
$$
   \begin{array}{lll}
   W &= W(v_0,v_1,w_1,w_2)\\
   W_1 &= W(v_0,v_1,w_1,q_1)\\
   W_2 &= W(v_0,v_1,q_2,w_2)\\
   W'' &= W(v_0,v_1,q_1,q_2)\\
   \end{array}
$$
Under Assumption~\ref{eqn:q1q2}, we have that
$W_1,W_2,W''$ are mutually disjoint, and that
   $$
   W \equiv W_1 \cup W_2 \cup W''.
   $$

Define Rogers simpices for $i=1,2$:
  $$
  \begin{array}{lll}
  R_i &= \op{ortho}^0(v_0,v_1,w_i,w_i+u_i,c_i)\\
  R_i'&= \op{ortho}^0(v_0,w_i,v_1,v_1+u_i,c_i)\\
  \end{array}
  $$
The $abc$-parameters of $R_i$ are $a=|v_1-v_0|/2$, $b_i$, $c_i$.
Those of $R_i$ are $a'_i=|w_i-v_0|/2$, $b_i$, $c_i$.
The dihedral angle of $R_i$ along the edge $\{v_0,v_1\}$ is
equal to the dihedral angle (and azimuth angle) of $W_i$.
We have $R_i,R_i'\subset W_i$.

The aggregates that we consider will be constructed in various
ways from $R_i,R'_i$ and regions $W''\cap X$, for various regions
$X$ that are rotationally symmetric along the line $\op{aff}\{v_0,v_1\}$.


In the following subsections, we specialize this general
context to various specific aggregates.

\subsection{plates}

The first aggregate that we consider will be called a plate.

\begin{definition}\tlabel{def:plate}
Let $v_0,v_1,w_1,w_2$ be points in $\ring{R}^3$ that are not
coplanar.  Let $t > 0$.  Choose $c_1=c_2=t$ and construct
the regions $R_1,R_2,W'',\ldots$ for the parameters 
$v_0,v_1,w_1,w_2,c_1,c_2$.
Define the plate in terms of the regions $R_1,R_2,W''$ as follows
  $$
  PL(v_0,v_1,w_1,w_2,t) = 
  R_1 \cup (W''\cap FR(v_0,v_1,h,h/t))\cup R_2,
  $$
where $h = |v_1-v_0|/2$.
\end{definition}

Under Assumption~\ref{eqn:q1q2}, the regions $R_1,R_2,(W''\cap FR)$
are disjoint and have the appearance of Figure~\ref{fig:plate}.
According to construction, if $t$ is less than the circumradius
of $\{v_0,v_1,w_i\}$, then the corresponding Rogers simplex $R_i$
is empty, the wedge $W_i$ is empty, and that piece can be dropped
from the description of the plate.

% Not yet sketched.
\begin{figure}[htb]
  \centering
  \myincludegraphics{noimage.eps}
  \caption{A plate}
  \tlabel{fig:plate}
\end{figure}


\begin{lemma}
Let $v_0,v_1,w_1,w_2$, $h,t$, be given as in the definition of
the plate. Let $q_i$ be the point constructed above. 
Assume that $h\le t$.  
Let $PL=PL(v_0,v_1,w_1,w_2,t)$.
Let $Q_i=\quo^0(v_0,v_1,w_i,q_i,t)$.
Then for all $\lambda$, we have
  $$
  \op{sovo}(v_0,PL,\lambda) = 
  \sol(v_0,PL)\phi(t,t,\lambda) + \op{sovo}(v_0,Q_1,\lambda) +
  \op{sovo}(v_0,Q_2,\lambda) +
  \op{azim}(v_0,v_1,w_1,w_2) A(h,t,\lambda).
  $$
\end{lemma}

\begin{proof} We have already calculated the function $\op{sovo}$
on the individual pieces $R_i$ and $W''\cap FR$.  The result
follows by assembling the pieces into the aggregate and 
Lemmas~\ref{lemma:sovo:rog} and \ref{lemma:sovoFR}:
  $$
  \begin{array}{lll}
  \op{sovo}(PL) &= \op{sovo}(R_1) + \op{sovo}(R_2) + \op{sovo}(W''\cap FR) \\
  \op{sovo}(R_i) &= \op{sol}(R_i) \phi(t,t,\lambda) +\op{sovo}(Q)
   + \dih(R_i) A(h,t,\lambda)\\
  \op{sovo}(W''\cap FR) &= \op{sol}(W''\cap FR)\phi(t,t,\lambda) +
   \op{azim}(W'') A(h,t,\lambda)\\
  \sol(PL) &=\sol(R_1)+\sol(R_2)+\sol(W''\cap FR)\\
  \op{azim}(v_0,v_1,w_1,w_2) &= \dih(R_1)+\dih(R_2)+\op{azim}(W'').
  \end{array}
  $$
The result follows immediately from these equations.
\end{proof}



\subsection{corner cells}

Let $\{v_0,v_1,w_1,w_2\}$ be a set of four points in $\ring{R}^3$.  
Assume that $\{v_0,v_1,w_1,w_2\}$ is not coplanar.
We
attach a {\it corner cell} $CC(v_0,v_1,w_1,w_2,t,\mu)$
to these four points and positive
real parameters $t,\mu$.  Let $y=|v_0-v_1|$, 
 $b=\eta(y,t,\mu)$, and $\psi=\arc(y,t,\mu)$.

Construct the cone $C=\op{rcone}^0(v_0,v_1,\cos\psi)$.
Let $P$ be the half-space containing $v_0$ bounded by
the perpendicular bisector of  $\{v_0,v_1\}$.
Let $CC_1 = C\cap P \cap B(v_0,t)$.
Let $W=W(v_0,v_1,w_1,w_2)$.
Let $CC(v_0,v_1,w_1,w_2,t,\mu) = CC_1 \cap W$.

\begin{lemma} \label{lemma:sovo:CC} 
Suppose that $y/(2t) \ge \cos\psi$.  Let the other notation
be as above.   We have
  $$
  \op{sovo}(v_0,CC(v_0,v_1,w_1,w_2,t,\mu),\lambda)=
  \op{azim}(v_0,v_1,w_1,w_2) \left((1-\cos\psi)\phi(t,t,\lambda)+
    A(y/2,t,\lambda)\right).
  $$
\end{lemma}

\begin{proof}
Let
$$
 \begin{array}{lll}
 A &= \{x \mid (x-v_0)\cdot (v_1-v_0) > |x-v_0| |v_1-v_0| (y/2t)\},\\
  \bar A &= \{x \mid (x-v_0)\cdot (v_1-v_0) < |x-v_0| |v_1-v_0| (y/2t)\},\\
 W &= W(v_0,v_1,w_1,w_2)\\
 \end{array}
$$
Let $h = y/2$.
XX NEW TARSKI RELATIONS HERE??
We have $A\cap \bar A = \emptyset$ and $A\cup \bar A \equiv \ring{R}^3$.
$A\cap CC = W \cap FR(v_0,v_1,h,h/t)$.  Also,  $\bar A \cap C$
is $t$-radial and measurable.
It follows from Lemma~\ref{lemma:sovoFR} that
$$
  \begin{array}{lll}
  \op{sovo}(CC) &= \op{sovo}(A\cap CC) + \op{sovo}(\bar A\cap CC)\\
  \op{sovo}(A\cap CC) &= \op{sovo}(W\cap FR) \\
     &= \op{azim}(W) A(h,t,\lambda) + \sol(W\cap FR(v_0,v_1,h,h/t))\phi(t,t,\lambda)\\
  &=\op{azim}(W) A(h,t,\lambda) + (\sol(CC)-\sol(\bar A \cap CC))\phi(t,t,\lambda)\\
  \op{sovo}(\bar A \cap CC) &= \sol(\bar A \cap CC)\phi(t,t,\lambda)\\
  \sol(CC) &= \op{azim}(W) \sol(SC(v_0,v_1,t,\cos\psi))/(2\pi) \\
         &=\op{azim}(W) (1-h/t)\\
  \end{array}
$$
The result follows immediately from these equations.
\end{proof}

\subsection{truncated corner cells}
%\subsection{Formulas for Truncated corner cells}
\tlabel{sec:ftcc}
    %\oldlabel{4.11}



Starting from the corner cell $CC(v_0,v_1,w_1,w_2,t,\mu)$, 
we define a subset $TCC(v_0,v_1,w_1,w_2,t,\mu)$ called
the {\it truncated corner cell}.  
We use the construction of $b_i$, $u_i$, $p_i$, $q_i$, $R_i$,
$W_1$, $W_2$, $W''$, $\ldots$ from Section~\ref{sec:tcc},
associated with the parameters $v_0,v_1,w_1,w_2$ and
$c_1=c_2 = y/(2\cos\psi)$.  Assumption~\ref{eqn:q1q2} remains
in force.  Let $B  = B(v_0,t)$.

\begin{definition} We define the truncated corner cell
to be
$$
TCC(v_0,v_1,w_1,w_2,t,\mu) =
(R_1\cap B)\cup (R_2\cap B) \cup (W''\cap CC(v_0,v_1,w_1,w_2,t,\mu)).
$$
\end{definition}

Let $y=|v_1-v_0|$ and $h=y/2$.
The points $p_i$ and
 $q_i$ are construction in Section~\ref{sec:tcc}.
We also need the Rogers simplices 
$R''_i= \op{rog}^0(v_0,v_1,w_i,q_i,t)$ based on the parameter
$t$, rather than $c_i$.  The Rogers simplex $R''_i$
has $abc$-parameters
$(h,b_i,t)$.  

Let $s(y_1,y_2,y_3,t,\lambda)$ be given by the following formula.
Let $h = y_1/2$, $b = \eta(y_1,y_2,y_3)$, $\psi = \arc(y_1,t,\lambda)$,
and $c = y_1/(2\cos\psi)$ in
  $$
  s(y_1,y_2,y_3) = \op{dihR}(h,b,c) (1-\cos\psi) - \op{solR}(h,b,c).
  $$


\begin{lemma}\label{lemma:tcc}  
Let $TCC=TCC(v_0,v_1,w_1,w_2,t,\mu)$ be as constructed.
Let $p_i$ be the circumcenter constructed in Section~\ref{sec:tcc}.
Let $q_i(t)$
Assume~\ref{eqn:q1q2}.  Let $y = |v_1-v_0|$.  
Let $s_i = s(y,|w_i-v_0|,|v_1-w_i|,t,\lambda)$, for $i=1,2$.
Assume that
$y/(2t) \ge \cos\psi$.  
Let $Q_i = \op{quo}^0(v_0,v_1,w_i,q_i,t)$.  Then
  $$
  \begin{array}{lll}
  \op{sovo}(v_0,TCC,\lambda) &= 
  \op{azim}(v_0,v_1,w_1,w_2) \left((1-\cos\psi)\phi(t,t,\lambda)+
    A(y/2,t,\lambda)\right) \\
    &\quad + \op{sovo}(v_0,Q_1,\lambda) - s_1\phi(t,t,\lambda) \\
    &\quad + \op{sovo}(v_0,Q_2,\lambda) - s_2\phi(t,t,\lambda) \\
  \end{array}
  $$
\end{lemma}

\begin{proof}  The result follows by combining the function
$\op{sovo}$ on each of the pieces of the aggregate $TCC$.
By Lemmas~\ref{lemma:sovo:CC}, \ref{lemma:sovo:truncRog}, and
\ref{lemma:truncRog:noQ}, we have
$$
\begin{array}{lll}
  \op{sovo}(TCC) &= \op{sovo}(W''\cap TCC) + \op{sovo}(W_1\cap TCC)
  + \op{sovo}(W_2)\cap TCC)\\
  &=\op{sovo}(W''\cap TCC) + \op{sovo}(R_1\cap B)
  + \op{sovo}(R_2\cap B)\\
  &= \op{azim}(W'') \left((1-\cos\psi)\phi(t,t,\lambda)+
    A(y/2,t,\lambda)\right) 
   + \op{sovo}(R_1\cap B)
  + \op{sovo}(R_2\cap B)\\
  \op{azim}(W'') &= \op{azim}(v_0,v_1,w_1,w_2)-\dih(R_1)-\dih(R_2)\\
  \op{sovo}(R_i\cap B) &= \dih(R_i) [(1-\cos\psi)\phi(t,t,\lambda)+A(y/2,t,\lambda)]\\
   &\quad -(\dih(R_i) (1-\cos\psi) - \sol(R_i)) \phi(t,t,\lambda) + 
   \lambda_v \op{quovol}(a,b,t),
\end{array}
$$
The result follows from these equations and the definition of
the function $s$.
\end{proof}

The dependence of $\op{sovo}$ on the azimuth angle
is linear with coefficient $(1-\cos\psi)\phi(t,t,\lambda)$.
For fixed, $\psi$, $t$, and $\lambda$, this coefficient
has fixed sign.  Also, if $\op{azim}<\pi$, the azimuth
angle depends monotonically on $|w_1-w_2|$.  We thus get
an (upper) bound on $\op{sovo}$ (for parameter $\lambda=\lambda_{oct}$)
when $|w_1-w_2|$ is chosen to be as small as possible.

\medskip

We give a second expession for the truncated corner cell.

Let $A_i^\pm$ be the half-spaces
$$
  A_i^\pm = \op{aff}_\pm^0(\{v_0,p_i,q_i\},v_0).
$$
Let $L_i^\pm = CC(v_0,v_1,w_1,w_2,t,\lambda)\cap A_i^\pm$.

\begin{lemma}\label{lemma:LL}
We have
  $$
  TCC(v_0,v_1,w_1,w_2,t,\lambda) = L_1^+\cap L_2^+.
  $$
Moreover, 
$L_1^-\cap L_2^- =\emptyset$.
\end{lemma}

\begin{proof}
The second statement follows from the containment
   $L_i^- \subset W_i$, because, as we have seen,
$W_1$
from $W_2$.

Now consider the first statement.  
We have
 $$
 W'' \cap TCC = W'' \cap CC = W'' \cap L_1^+ \cap L_1^-.
 $$
The first equality holds by construction of the truncated corner
cell.  The second equality holds because the parts $L_i^-$ excised
from $CC$ to form $TCC$ are contained in $W_i$.

We also have
  $$
  W_i \cap TCC = R_i \cap B(v_0,t) = W_i \cap L_i^+.
  $$
%% IS THIS A NEW TARSKI??
The result follows.
\end{proof}




\subsection{inverted truncated corner cells}
%\subsection{Analytic continuation} %DCG 13.3, p144
\oldlabel{5.3}

In this section, we develop a formula for $\op{sovo}$ on 
a truncated corner cell that remains valid if the 
Assumption~\ref{eqn:q1q2} is not in force.



When Assumption~\ref{eqn:q1q2} does not necessarily hold, we
define $TCC(v_0,v_1,w_1,w_2,t,\lambda) = L_1^+ \cap L_2^+$.
By Lemma~\ref{lemma:LL}, 
this is a compatible extension of the definition
of truncated corner cells, up to a null set.
The proof that $L_1^+\cap L_2^-$ relies on Assumption~\ref{eqn:q1q2}.

We have by inclusion-exclusion, the formula
$$
\begin{array}{lll}
\op{sovo}(v_0,TCC,\lambda) &=
\op{sovo}(v_0,CC,\lambda) -
\op{sovo}(v_0,CC\cap L_1^-,\lambda) -
\op{sovo}(v_0,CC\cap L_2^-,\lambda) \\
 &\quad +
\op{sovo}(v_0,CC\cap L_1^- \cap L_2^-,\lambda).
\end{array}
$$

If we compare this formula with the calculations for 
$\op{sovo}(v_0,TCC,\lambda)$ in Lemma~\ref{lemma:tcc}, 
we find that, in the notation of that lemma:
$$
\op{sovo}(v_0,CC\cap L_i^-) = \op{sovo}(v_0,Q_i,\lambda) - s_i.
$$

Thus, the formula for $\op{sovo}(TCC)$ in the general case,
differs from the formula under Assumption~\ref{eqn:q1q2} through
the term $\op{sovo}(v_0,CC\cap L_1^-\cap L_2^-,\lambda)$.

We note that $CC\cap L_1^-\cap L_2^-$ is $t$-radial.  Thus,
$$
\op{sovo}(v_0,CC\cap L_1^-\cap L_2^-,\lambda) =
\sol(v_0,CC\cap L_1^-\cap L_2^-)\phi(t,t,\lambda).
$$
In particular, if $phi(t,t,\lambda)<0$, then
$\op{sovo}(v_0,TCC,\lambda)$ is at most the right-hand side
of the identity in Lemma~\ref{lemma:tcc}.  


\begin{figure}[htb]
  \centering
  \myincludegraphics{\ps/samfigA54.eps}
  \caption{Different forms of truncated corner cells are shown.  The
  structure
  shown in the middle frame cannot occur.}
  \tlabel{fig:chi-anal-vs-geom}
\end{figure}

XX THIS is the justification that the given set is $t$-radial.
Move it to the appropriate place:


We let $c_0$ be the point of height $t_0$ on the intersection of the
planes $\{0,v,v_1\}^\perp$ and $\{0,v,v_2\}^\perp$. We claim that $c_0$ lies
over the truncated spherical region of the tcc, rather than the wedges
of $t_0$-cones or the Rogers simplices along the faces $\{0,v,v_1\}$ and
$\{0,v,v_2\}$.  (This implies that $c_0$ cannot protrude beyond the corner
cell as depicted in the second frame of the figure.) To see the claim,
consider the tcc as a function of $y_4=|v_1-v_2|$. When $y_4$ is
sufficiently large the claim is certainly true.  Contract $y_4$ until
$c_0=c_0(y_4)$ meets the perpendicular bisector of $\{0,v\}$. Then $c_0$
is equidistant from $0,v,v_1$ and $v_2$ so it is the circumcenter of
$\{0,v,v_1,v_2\}$. It has distance $t_0$ from the origin, so the
circumradius is $t_0$. This implies that $y_4\le 2t_0$.


\subsection{truncated corner cell calculations}
%\subsection{More on Truncated Corner Cells}

In this section, we calculate several different bounds
on the function $\op{sovo}$ on truncated corner cells for
various choices of the parameters $t$, $\mu$, and $\lambda$.

\begin{lemma}\tlabel{lemma:tcc-est} 
Let $TCC=TCC(v_0,v_1,w_1,w_2,t,\mu)$ 
be a truncated corner cell with
parameters $t=1.255$, $\mu=1.6$ and azimuth angle at least $\pi$. 
Assume that $|v_1-w_i|\ge 3.07$ for $i=1,2$.
Let $y=|v_1-v_0|$.  Assume that $2\le y\le 2t$.
Assume that $2\le |v_1-w_i|\le 2t$, for $i=1,2$.
Then $\op{sovo}(v_0,TCC,\lambda_{sq}) > 0.297$.
\end{lemma}

\begin{proof}
Let $z_i = |v_1-w_i|$, for $i=1,2$.
Our estimate is based on Lemma~\ref{lemma:tcc}.  From that lemma,
we have
  $$
  \begin{array}{lll}
  \op{sovo}(v_0,TCC,\lambda) &= T_0 + T_1 + T_2 + T_3 \\
  T_0 &= \op{sovo}(v_0,Q_1,\lambda) +  \op{sovo}(v_0,Q_2,\lambda)\\
  T_1  &=  - s_1\phi(t,t,\lambda) \\
  T_2  &= - s_2\phi(t,t,\lambda) \\
  T_3 &= 
  \op{azim}(v_0,v_1,w_1,w_2) \left((1-\cos\psi)\phi(t,t,\lambda)+
    A(y/2,t,\lambda)\right) \\
  \end{array}
  $$
The condition $z_i\ge3.07$
forces the circumradius of $\{v_0,v_1,w_i\}$ to be greater
than $t$.  This implies that $Q_i=\emptyset$.  The corresponding
term $T_0$ is zero.

The coefficient of $\op{azim}$ in the term $T_3$
is an explicit function of a single variable $y\in[2,2t]$.
It is minimized and takes a positive value when $y=2t$.
In particular,
the coefficient of $\op{azim}$ is
positive. We obtain a lower bound on the $T_3$ by taking
$\op{azim}(v_0,v_1,w_1,w_2)\ge \pi$ and $y=2t$.
This gives $T_3 > 0.32$.

For the given choices of $t,\lambda$, we have
$0 < \phi(t,t,\lambda) < 0.6671$.  The term $s_i$ is
maximized when $y_3=2t_0$, $y_5=3.07$,
so that $s_i < 0.017$.  (This was checked with interval arithmetic in
Mathematica.) Thus,
    $$\op{sovo}(v_0,TCC,\lambda)  = T_0 + T_1 + T_2 + T_3
   0 - 2 (0.0017)(0.6671) +0.32 > 0.297.$$
\end{proof}



\begin{lemma}\label{lemma:CC815}  
Let $CC=CC(v_0,v_1,w_1,w_2,t,\mu)$
be an untruncated corner
cell with parameters $\mu=1.815$, $t=1.255$,
$y=|v_1-v_0|$, with $2\le y\le 2.2$, 
and azimuth angle at least $\pi$.  Then
 $$\op{sovo}(v_0,CC,\lambda_{sq}) > \squander +\maxpi.$$
\end{lemma}

XX Replace $\squander+\maxpi$ with a decimal constant.
There is a problem in the original proof, because the constant
$2.51$ rather than $2.2$ appears.  I must check that this is just
a typo.

\begin{proof}  According to Lemma~\ref{lemma:sovo:CC},
the function $\op{sovo}(CC)$ has the form
$\op{azim}(v_0,v_1,w_1,w_2) f(y)$, for
some explicit rational function $f$ of the
variable $y$. 
The minimum, which occurs at $y=2.2$, is positive.
A lower bound is then $\pi f(2.2)$. 
We evaluate this constant to get the result.
\end{proof}

Before we move to the next estimate on $\op{sovo}$, we
need a disjointness result that is provided by the following
lemma.

\begin{lemma}\label{lemma:2CCrad}
Let $\{v_0,v_1,v_1'\}$ be a set of three points in $\ring{R}^3$.
Let $t=1.255$ and 
$\mu = 3.2 - t = 1.945$.  Let $y = |v_1-v_0|$ and $y'=|v_1'-v_0|$.
Set $\psi = \arc(y,t,\mu)$.
Assume that $|v_1-v'_1|\ge 3.2$ and that $y,y'\le 2t$.
Then 
   $$
   \op{rcone}(v_0,v_1,y/(2t))\cap \op{rcone}(v_0,v_1',\cos\psi)
   =\emptyset.
   $$
\end{lemma}

XX Move to the Tarski chapter.

\begin{proof}
Let $p$ be a point in the intersection.  Both sets are cones
that are symmetrical through the plane $A=\op{aff}(v_0,v_1,v_1')$.
Thus, the reflection $p'$ of $p$ through $A$ is also
in the intersection.  By the convexity of the two sets so is
the midpoint $(p+p')/2$.  Hence, we may assume without loss of
generality that $p\in A$.

Since both sets are cones,
we can scale $p$ to get another point in the intersection  
that belongs to the bisector of $\{v_0,v_1\}$.  We may
move along the bisector towards $\op{aff}(v_0,v_1')$ until
the boundary of $\op{rcone}(v_0,v_1,y/(2t))$ is reached.
Changing notation,
we assume that $p$ is this point on the bisector.
We then have $|p-v_1|=|p-v_0|=t$, $|p-v_0|< \mu$.
This violates the triangle inequality:
  $$
  3.2\le |v_1-v_1'| \le |v_1-p| + |p-v_1'| < t + \mu = 3.2. 
  $$
\end{proof}


\begin{lemma}\tlabel{lemma:2tcc} 
Let $$CC=CC(v_0,v_1,w_1,w_2,t,\mu)$$ and 
$$CC'=CC(v_0,v_1',w_1',w_2',t,\mu)$$ 
be untruncated corner cells, both
with azimuth angle at least $\pi$ and parameters $t=1.255$ and 
$\mu=1.945$.
Assume $|v_1-v_1'|\ge 3.2$.  Let $y=|v_1-v_0|$ and
$y'=|v_1'-v_0|$.  Suppose that
$2\le y\le 2t$ and $2\le y'\le 2t$.  Then
$\op{sovo}(v_0,CC\cup CC',\lambda_{sq}) > \squander + \maxpi$.
\end{lemma}

XX Replace the right-hand side with a decimal constant to avoid the definitions in this
chapter.

\begin{proof}
Suppose first that $CC$ and $CC'$ are disjoint.
Following the proof of Lemma~\ref{lemma:CC815}, 
but with parameter $\mu=1.945$, a lower bound on
$\op{sovo}$ is obtained when  $y=2t$,
$\op{azim}=\pi$. The explicit formulas give 
  $$\op{sovo}(v_0,C,\lambda) > 0.734$$
for $C=CC,CC'$.
The result follows in this case.

Suppose that $CC$ meets $CC'$. 
We have
 $$
 \op{sovo}(CC\cup CC')=
 \op{sovo}(CC)+\op{sovo}(CC')-\op{sovo}(CC\cap CC').
 $$
It follows from Lemma~\ref{lemma:2CCrad} that
$CC\cap CC'$ is $t$-radial at $v_0$.  Thus,
$$\op{sovo}(v_0,CC\cap CC',\lambda) =
  \sol(v_0,CC\cap CC')\phi(t,t,\lambda).$$
The constant $\phi(t,t,\lambda)$ is positive.  Thus, we
minimize $\op{sovo}(CC\cup CC')$ by taking 
the intersection $CC\cap CC'$ to be as large as possible.
This happens when $v_1$ is as close to $v_1'$ as possible:
$|v_1-v_1'|=\ell=3.2$.  Assume this.

Let $q$ and $q'$ be the two points defined by distances
$t$ from $v_0$, $\mu$ from $v_1$, and $\mu$ from $v_1'$.
The existence of such points is given by Lemma~\showref{tarski:XX}.
XX New Tarski?
Let $A=\op{aff}_+(v_0,\{q,v_1,v_1'\})$ and
$A'=\op{aff}_+(v_0,\{q',v_1,v_1'\})$.
Let $FR = FR(v_0,v_1,y/2,y/(2\cos\psi))$ and
$FR' = FR(v_0,v'_1,y'/2,y/(2\cos\psi))$.
We have the relations
$$
\begin{array}{lll}
CC\cap CC' &\equiv (A\cap CC\cap CC') \cup (A'\cap CC\cap CC').\\
\sol(v_0,A\cap CC\cap CC') &= \sol(v_0,A\cap CC) + \sol(v_0,A\cap CC')
  -\sol(v_0,A)\\
    &\le \sol(v_0,A\cap FR) + \sol(v_0,A\cap FR') - \sol(v_0,A)
\end{array}
$$
This final expression is now easily estimated in terms of primitive
regions.  Adding the similar
term for $A'$, we get
a function $f(y,y')$ that gives a lower bound on 
$\op{sovo}(CC\cup CC')$.  Recall that $\lambda=\lambda_{sq}$.
    $$
    \begin{array}{lll}
    \alpha_1 &= \dih(y_1,t,y_2,\mu,\ell,\mu),\\
    \alpha_2 &= \dih(y_2,t,y_1,\mu,\ell,\mu),\\
    \sol &= \sol(y_2,t,y_1,\mu,\ell,\mu),\\
    \phi_i &= \phi(y_i/2,t,\lambda),\quad i=1,2,\\
    f(y_1,y_2)&=
    2\phi(t,t,\lambda_{sq})\sol+
    2\sum_1^2 \alpha_i(1-y_i/(2t))(\phi(t,t,\lambda)-\phi_i)\\
        &\quad +
       \sum_1^2 \op{sovo}(v_0,CC(y_i\pi-2\alpha_i,t,\mu),\lambda).
    \end{array}
    $$
Here $CC(y,\beta,t,\mu)$ is any corner cell
$$CC(v_0,v_1,w_1,w_2,t,\mu)$$ with $|v_1-v_0|=y$ and
$\op{azim}(v_0,v_1,w_1,w_2)=\beta$.
An interval calculation\footnote{\calc{984628285}} %A14
gives $f(y_1,y_2)>\squander+\maxpi$, for $y_1,y_2\in[2,2t]$.
\end{proof}




%\section{Scores of Simplices and Cones}


%\begin{remark}\tlabel{remark:vor}\index{vor}\index{c-vor}\index{score}
% Deleted function {c-vor} that has been replaced by sovo

%\label{eqn:3.2} deleted. It should be replaced by sovo(FR) formula ref.

%\section{The Function K}
%\tlabel{sec:K} %DCG p105-106.
%% No longer used.  Proof of -1.04 lemma was rewritten.
%%
%
%We define a function $K(S)$ on
%certain simplices $S$ with circumradius at least $\sqr2$. Let
%$S=S(y_1,y_2,\ldots,y_6)$.  Let $R(a,b,c)$ denote a Rogers
%simplex. Set
%    \begin{equation}
%    K(S) = K_0(y_1,y_2,y_6)+K_0(y_1,y_3,y_5)
%    + \dih(S)(1-y_1/\sqr8) \phi(y_1/2,\sqr2),
%    \tlabel{eqn:KS}
%    \end{equation}
%where
%    $$
%    $$
%(If the given Rogers simplices do not exist because the condition
%$0<a<b<c$ is violated, we set the corresponding terms in these
%expressions to 0.) The function $K(S)$ represents the part of the
%score coming from the four Rogers simplices along two of the faces
%of $S$, and the conic region extending out to $\sqr2$ between the
%two Rogers simplices along the edge $y_1$ (Figure~\ref{fig:KS}).
%This region is closely related to the regions $\BigD(v,W)$ of
%Definition~\ref{def:delta-e}, with the difference that the regions
%$\BigD$ lie in a ball of radius $\eta_0(|v|/2)$, but the regions
%here are truncated at $\sqrt2$.
%
%\begin{figure}[htb]
%  \centering
%  \myincludegraphics{\ps/diag43.ps}
%  \caption{The set measured by the function $K(S)$.}
%  \tlabel{fig:KS}
%\end{figure}
%
%

\subsection{crowns}
%\section{The Function anc}
\tlabel{sec:anc} %DCG p 107.

This subsection considers
one final aggregate region.

Let $\eta(x,y,z)$ be the circumradius of a triangle with
sides $x,y,z$ and let $\eta_0(h,t) = \eta(2,2h,2t)$.

We return to the contex established at the beginning
of Section~\ref{sec:tcc}.  We use the parameters $v_0,v_1,w_1,w_2$
and $c=c_1=c_2=\eta_0(h,t)$, where $h =|v_0-v_1|/2 \le t$. 
Assumption~\ref{eqn:q1q2} remains in force.
Let $W_1,W_2,W'',W,R_i,R_i',p_i,q_i$ be as given at the
beginning of Section~\ref{sec:tcc} for these parameters.

Let 
  $$\bar B = \bar B(v_0,t) = \{x \mid |x-v_0| > t\},$$
the complement of a closed ball of radius $t$ at $v_0$.

We define the fitted crown to be
$$
FCR(v_0,v_1,w_1,w_2,t) =
  (W'' \cap \bar B \cap FR(v_0,v_1,h,h/c)) \cup
  (R_1\cap \bar B) \cup (R_2 \cap \bar B) \cup
  (R_1'\cap \bar B) \cup (R_2' \cap \bar B).
$$
As described at the beginning of the section, the 
sets $R_i\cap \bar B$ and $R_i'\cap \bar B$ are empty
(and also $W_i=\emptyset$) when the circumradius of 
$\{v_0,v_1,w_i\}$ is greater than $c=c_i$.

We define some functions that will be used in a formula
for the value of $\op{sovo}$ on a fitted crown.
Define
\begin{equation}\cro(h,t,\lambda) =
2\pi(1-h/\eta_0(h,t))(\phi(h,\eta_0(h,t),\lambda)-\phi(t,t,\lambda). 
\end{equation} 

\begin{lemma} Let $t$ and $h$ be real numbers satisfying 
$0 < t \le h$.
Let $b=h/\eta_0(h,t)$.
Let $CR=FR(v_0,v_1,h,b) \setminus B(v_0,t)$.
  Then
$$\op{sovo}(v_0,CR,\lambda) = \cro(h,t,\lambda).$$
\end{lemma}

\begin{proof}  Let $RC=\op{rcone}^0(v_0,v_1,h,b)$.
Then $B' = B(v_0,t)\cap RC = FR\cap B(v_0,t)$.
Furthermore, $B'$ is $t$-radial with solid angle equal to that
of $FR$.  Thus, by Lemmas~\showref{XX},
$$
\begin{array}{lll}
\op{sovo}(CR) &= \op{sovo}(FR) - \op{sovo}(B')\\
 &= \op{sol}(FR) (\phi(h,b,\lambda) - \phi(t,t,\lambda))\\
 &= \cro(h,t,\lambda).
\end{array}
$$
\end{proof}

Similarly, if $WCR = W(v_0,v_1,w_1,w_2) \cap CR$, then
$$
\op{sovo}(v_0,WCR,\lambda) = \op{azim}(v_0,v_1,w_1,w_2)\cro(h,t,\lambda)/(2\pi),
$$
because $WCR$ 
is rotationally symmetric about the axis $\op{aff}\{v_0,v_1\}$.


%\begin{figure}[htb]
%  \centering
%  \myincludegraphics{\ps/diag44.ps}
%  \caption{An illustration of the terms $\anc$.}
%  \tlabel{fig:anchor}
%\end{figure}

Let $\op{dihR}(a,b,c)$ be the dihedral angle along the edge
$\{v_0,v_1\}$ of a
Rogers simplex $\op{rog}^0(v_0,v_1,v_2,v_3,c)$ with $abc$-parameters
$(a,b,c)$.  Similarly, let $\op{solR}(a,b,c)$ (resp. $\op{sovoR}(a,b,c,\lambda)$)
be the solid angle (resp. value of $\op{sovo}$)
at $v_0$ of such a Rogers simplex.  By Lemmas~\showref{XX} and \showref{XX},
these values depend only on the $abc$-parameters.

Set
    \begin{equation}
    \begin{array}{lll}
    \anc(y_1,y_2,y_6,t,\lambda) &= 
     -\op{dihR}_1\cro(y_1/2,t,\lambda)/(2\pi)
        -\op{solR}_1\phi(t,t,\lambda)+\op{sovoR}_1\\
    &-\op{dihR}_2(1-y_2/(2t))(\phi(y_2/2,t,\lambda)-\phi(t,t,\lambda))
        -\op{solR}_2\phi(t,t,\lambda) + \op{sovoR}_2,
    \tlabel{eqn:4.5}
    \end{array}
    \end{equation}\index{anc@$\anc$}
where $\op{dihR}_i$, $\op{solR}_i$, $\op{sovoR}_i$ are the values
of $\op{dihR}$, $\op{solR}$, and $\op{sovoR}(\cdot,\lambda)$
at $(a_i,b,c) = (y_i/2,\eta(y_1,y_2,y_6),\eta_0(y_1/2))$, for $i=1,2$.
Recall that the terms are defined as zero if the inequalities
$0 < a_i \le b\le c$ are violated.  Hence, the function $\anc$ is
zero, except at points in a certain domain.

%% Moved from DCG 11.2 Contexts.
Set
    $$\kappa(y_1,y_2,y_3,y_5,y_6,\alpha,t,\lambda) =
   \alpha\,\cro(y_1/2,t,\lambda)/(2\pi) +
        \anc(y_1,y_2,y_6,t,\lambda)+\anc(y_1,y_3,y_5,t,\lambda).
    $$
    \index{zzkappa@$\kappa$}

We are finally ready to state the main result about fitted crowns.
Assumption~\ref{eqn:q1q2} remains in force.

\begin{lemma}\label{lemma:sovo:FCR}
Let $\{v_0,v_1,w_1,w_2\}$ be a set of four points in $\ring{R}^3$.
Assume the set is not planar.
Let $0 < t < h$, where $h = |v_1-v_0|/2$.
Set $\alpha = \op{azim}(v_0,v_1,w_1,w_2)$.
Let 
 $$(y_1,y_2,z_2,z_1) =
   (|w_1-v_0|,|w_2-v_0|,|w_2-v_1|,|w_1-v_1|).
 $$
Then
$$
\op{sovo}(v_0,FCR(v_0,v_1,w_1,w_2,t),\lambda) =
 \kappa(2h,y_1,y_2,z_2,z_1,\alpha,t,\lambda).
$$
\end{lemma}

\begin{proof}
Let $B = B(v_0,t)$.  We have
$$R\equiv (\bar B\cap R) \cup (B\cap R),\quad B\cap \bar B = \emptyset,
$$
for $R=R_i,R'_i$.  Moreover, $B\cap R_i$ is $t$-radial at $v_0$.
Thus, 
 $$
\begin{array}{lll}
 \op{sovo}(\bar B\cap R_i) &= \op{sovo}(R) - \op{sovo}(B\cap R_i)\\
 &= \op{sovo}(R) - \sol(R_i)\phi(t,t,\lambda) \\
 &= \op{anc}(2h,y_i,z_i) - \dih(W_i)\cro(h,t,\lambda)/(2\pi)\\
  &\quad +\op{sovo}(v_0,R_i',\lambda)\\
  &\quad +\sol(R_i')\phi(t,t,\lambda)+\dih(R_i')A(y_i/2,t,\lambda).\\ 
\end{array}
 $$

We have by Lemma~\ref{lemma:truncRog:noQ}, 
$$
\begin{array}{lll}
\op{sovo}(\bar B\cap R_i') &= \op{sovo}(R'_i) - \op{sovo}(B\cap R'_i)\\
\op{sovo}(B\cap R'_i) &= \sol(R'_i)\phi(t,t,\lambda) + \dih(R'_i) A(y_i/2,t,\lambda)\\
\end{array}
$$

We have by Lemmas~\showref{XX} and \showref{XX}, that
$$
\begin{array}{lll}
\op{sovo}(FCR) &= \op{sovo}(W''\cap FCR) + \op{sovo}(W_1\cap FCR)
 +\op{sovo}(W_2\cap FCR).\\
\op{sovo}(W_i\cap FCR) &= \op{sovo}(\bar B\cap R_i) + \op{sovo}(\bar B\cap R'_i).\\
 \op{sovo}(W''\cap FCR) &= \op{azim}(W'')\cro(h,t,\lambda)/(2\pi),\\
\op{azim}(v_0,v_1,w_1,w_2)&= \op{azim}(W'')+\dih(R_1)+\dih(R_2).\\
\end{array}
$$
These equations give the lemma.
\end{proof}

\begin{lemma}  The solid angle of $FCR(v_0,v_1,w_1,w_2,t,\lambda)$
is zero at $v_0$.
\end{lemma}

\begin{proof}  The region $FCR$ is contained in the complement
of the ball of radius $t>0$ at $v_0$.  It is bounded away from
zero.  Hence, its solid angle is zero.
\end{proof}

It follows that $\op{sovo}(v_0,FCR,\lambda) = \lambda_v\op{vol}(FCR)$,
where $\lambda=(\lambda_v,\lambda_s)$.



\section{Finiteness and Volume}

\begin{lemma}\tlabel{lemma:Zcount}
    For all $p\in\ring{R}^3$ and all $r\ge 0$, the set
    $\ring{Z}^3\cap B(p,r)$ is finite of cardinality at most
    $4\pi (r+\sqrt3)^3/3$.
\end{lemma}

\begin{proof}  If $v\in\ring{Z}^3\cap B(p,r)$, then the ith
coordinate $v_i$ of $v$ must lie in the finite range
    $$
    p_i - r \le v_i \le p_i + r.
    $$
Hence there are only finitely many possibilities for $v$.


Place an open unit cube at each point of $\ring{Z}^3\cap B(p,r)$.
The cubes are measurable, disjoint, and contained in
$B(p,r+\sqrt3)$.  Thus, the combined volume of the cubes, which is
$|\ring{Z}^3\cap B(p,r)|$,  is no greater than the volume of the
containing ball.  The result follows.
\end{proof}

\begin{lemma}\tlabel{lemma:Zlow-count}
  For all $p\in\ring{R}^3$ and all $r\ge\sqrt3$, the set
    $\ring{Z}^3\cap B(p,r)$ is finite of cardinality at least
    $4\pi (r-\sqrt3)^3/3$.
\end{lemma}

\begin{proof} We have already established finiteness in
Lemma~\ref{lemma:Zcount}.  Place a closed unit cube at each point
of $\ring{Z}^3\cap B(p,r)$.  The cubes are measurable and cover
$B(p,r-\sqrt3)$.  Thus, the combined volume of the cubes is at
least the volume of the covered ball.  The result follows.
\end{proof}

\begin{lemma}\tlabel{lemma:Zr2}
For all $p\in\ring{R}^3$, and $k,k'>0$, there exists a $C$ such
that for all $r\ge k'$, we have
    $$
    \ring{Z}^3 \cap (B(p,r+k) \setminus B(p,r-k')) \le C r^2.
    $$
\end{lemma}

\begin{proof}  When $r \ge k'+\sqrt3$, the previous two lemmas show
that the cardinality is at most $4\pi/3$ times
    $$(r + \sqrt3)^3 - (r - \sqrt3)^3 \le C' r^2$$
for some $C'$.  Similarly, if $k'\le r\le k'+\sqrt3$, the
cardinality is at most some fixed constant $C''$.  The result
easily follows.
\end{proof}

