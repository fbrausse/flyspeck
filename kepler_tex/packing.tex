% file started March 22, 2009

\chapter{Packing}

\begin{summary}
This chapter comprises much of the core material of the book.  This is a book
about dense sphere packings, and at last we take up the topic of dense sphere
packings.
 Associated with a sphere packing $V$ in $\ring{R}^3$
are various subsidiary decompositions of space.
This chapter focuses on three such decompositions: the Voronoi decomposition into
polyhedra,  the Rogers decomposition into simplices, and the Marchal decomposition
into cells.  Each of these decompositions leads to a bound on the density of sphere
packings.  The bounds in the first two cases are not sharp.  The Marchal decomposition
leads to a sharp bound $\pi/\sqrt{18}$ on the density of sphere packing in three
dimensions.  The final sections of this chapter launch a detailed study of the properties
of the Marchal cell decomposition.
\end{summary}

\section{The Primitive State of Our Subject Revealed}


\subsection{definition}



Informally, a \newterm{packing} is an arrangement of congruent
balls in Euclidean three space that are nonoverlapping in the sense
that the interiors of the balls are pairwise disjoint.  By convention,
we take the radius of the congruent balls to be $1$.
%the scale
%invariance of density, without loss of generality, units can be chosen
%so that each ball has radius $1$. 
Let $ V$ be the set of centers of the balls in a
packing. The choice of unit radius for the
balls implies that any two points in $ V$ have distance at
least $2$ from each other. 
 Formally, the packing is identified
with the set of centers $V$.
\indy{Notation}{V@$ V$ (packing)}%

%
%The density of a packing does not decrease when balls are added to the
%packing. Thus, to construct packings of maximal density, one may add
%nonoverlapping balls until there is no room to add further balls.  A
A packing in which no further balls can be added is said to be {\it
saturated}.

\begin{definition}[saturated,~packing]\guid{XASMJUK} 
\formaldef{packing}{packing}
\formaldef{saturated}{saturated}
A \newterm{packing} $ V\subset \ring{R}^3$ is a set such that
\begin{displaymath}
\forall  \u,~\v\in  V.~  \norm{ \u}{\v} < 2 \Rightarrow ( \u=\v).
\end{displaymath} 
A set is \newterm{saturated} if for every $\p\in\ring{R}^3$, there
exists $ \u\in V$ such that $\norm{ \u}{\p}< 2$.
\end{definition}
\indy{Index}{saturated}%
\indy{Index}{packing}%



Let $B(\p,r)$ denote the open ball in
Euclidean three-space at center $\p$ and radius $r$.  The open ball
is measurable with measure $4\pi r^3/3$.
 Set $ V(\p,r) = V \cap
B(\p,r)$. %and $ V^*(\p,r) = V(\p,r)\setminus \{\p\}$.
\formaldef{ball}{ball}
\formaldef{$V(\p,r)$}{V INTER ball(p,r)}
\indy{Index}{measure}%
\indy{Notation}{B@$B(\p,r)$}%

\begin{lemma}[]\guid{KIUMVTC}\rz{0}
\oldrating{80}
\formalauthor{Nguyen Tat Thang}
\label{lemma:V-finite}
Let $ V$ be a packing and let $\p\in\ring{R}^3$.
Then the set $ V(\p,r)$ is finite.
\end{lemma}

\begin{proof}  Let $\p = (p_1,p_2,p_3)$. The floor function gives the map
\begin{displaymath}\v=(v_1,v_2,v_3)\mapsto (\lfloor 2(v_1-p_1)
  \rfloor, \lfloor 2(v_2-p_2) \rfloor, \lfloor 2(v_3-p_3) \rfloor).
\end{displaymath}
It is a one-to-one map from $ V(\p,r)$ into the set $\ring{Z}^3\cap B(\orz,2r + 1)$.  
By Lemma~\ref{lemma:Zcount} the range of this one-to-one map is finite. 
Hence the domain $ V(\p,r)$ of the map is also finite.%
\footnote{An alternative proof uses the open cover of the compact ball 
$\bar B(\p,r)$ by the sets $\bar B(\p,r)\setminus V$ and $B(\v,1)$ for $\v\in V$. By compactness, the cover is necessarily finite.}
\end{proof}
\indy{Notation}{1@$\lfloor\cdot\rfloor$ (floor)}%






\subsection{Voronoi cell}

Geometric decompositions of space give a way to estimate the density of sphere packings.
One of the most popular decompositions of space is the Voronoi cell decomposition.

\begin{definition}[Voronoi~cell,~$\Omega$]\guid{YGFWXEH}\label{def:voronoi} 
\formaldef{$\Omega$}{voronoi\_closed}
%\indy{Index}{Voronoi cell}% 
Let $V\subset\ring{R}^3$ and $\v\in V$.
The \newterm{Voronoi cell} 
$\Omega(V,\v)$
is the set of points at least as close to $\v$ as to
any other point in $V$. 
% Let $\Omega_t( V,\v) = \Omega( V,\v)
%\cap B(\v,t)$ be the truncated Voronoi cell at radius $t$.
\end{definition}
\indy{Notation}{ZZZomega@$\Omega$ (Voronoi cell)} %

\begin{lemma}[Voronoi partition]\guid{TIWWFYQ}\rz{100}
If $V$ is a saturated packing, then 
\begin{equation}\label{eqn:vor-rn} 
\ring{R}^3 = \bigcup \{\Omega(V,\v)\mid \v \in V\}.
\end{equation}
\end{lemma}

\begin{proof}
  If $V$ is a saturated packing, then every point $\p$ has distance
  less than $2$ from some point of $V$.  The set $V(\p,2)$ is finite
  by Lemma~\ref{lemma:V-finite}.  Hence, $\p$ is at least as close to
  some $\v\in V$ as it is to any other $\w\in V$.  This means that
  $\p\in\Omega(V,\v)$.  
\end{proof}

We use half-spaces to separate one Voronoi cell from another.

\begin{definition}[half-space]\guid{BGXHPKY}
\formaldef{$A$}{bis}
\formaldef{$A_+$}{bis\_le}
\begin{align*} 
A(\u,\v) &= \{\p\in\ring{R}^3
\mid 2(\v-\u)\cdot \p = \normo{\v}^2 - \normo{\u}^2 \},\\
A_+(\u,\v) &= \{\p\in\ring{R}^3
\mid 2(\v-\u)\cdot \p \le \normo{\v}^2 - \normo{\u}^2 \},
\end{align*}
when $\u,\v\in\ring{R}^3$.  The plane $A(\u,\v)$ is the \newterm{bisector} of
$\{\u,\v\}$ and $A_+(\u,\v)$ is the half-space of points at least as
close to $\u$ as to $\v$.  
\end{definition}

Each Voronoi cell is a bounded polyhedron.

\begin{lemma}[Voronoi polyhedron]\guid{RHWVGNP}\rz{200}\label{lemma:V4} 
  Let $V\subset\ring{R}^3$ be a saturated packing.  Then
  $\Omega(V,\v)\subset B(\v,2)$.  Also, $\Omega(V,\v)$ is a polyhedron
  defined by the intersection of the finitely many half-spaces
  $A_+(\v,\u)$ for $\u\in V(\v,4)\setminus\{v\}$.
\end{lemma}

\begin{proof} 
The Voronoi cell $\Omega(V,\v)$ is the
intersection of the half-spaces $A_+(\v,\u)$ as $\u$ runs over
$V\setminus \{\v\}$.

Let $\p\not\in B(\v,2)$.  
By saturation, there exists $\u\in V$ such that $\norm{\p}{\u}<2$.
Then 
\begin{displaymath} 
\norm{\p}{\u} < 2 \le \norm{\p}{\v}.
\end{displaymath}
Hence, $\p\not\in\Omega(V,\v)$.  This proves the first conclusion.


Let $\Omega'$ be the intersection of the half-spaces $A_+(\v,\u)$ as
$\u$ runs over $V(\v,4)$.  Clearly, $\Omega(V,\v)\subset \Omega'$.
Assume for a contradiction that $\p\in \Omega'\setminus\Omega(V,\v)$.
The intersection of the ray $\op{aff}_+\{\v,\{\p\}\}$ with
$\Omega(V,\v)$ is a closed and bounded convex subset of the line.  By
general principles of convex sets, this intersection is an interval
$\op{conv}\{\v,\p'\}$ for some $\p'\in\Omega(V,\v)\subset B(\v,2)$.
For some small $t>0$, the point lies beyond the interval but remains
within the ball:
\begin{displaymath} 
\q = (1+t)\p' -t \v\in (B(\v,2)\cap \Omega')\setminus \Omega(V,\v).
\end{displaymath}
Pick $\u\in V\setminus V(\v,4)$ such that $\q\in A_+(\u,\v)$.  By the
triangle inequality,
\begin{displaymath} 
\norm{\u}{\v} \le \norm{\u}{\q} + \norm{\v}{\q} \le 2\norm{\v}{\q} < 4.
\end{displaymath}
This contradicts the assumption $\u\not\in V(\v,4)$.

The number of half-spaces $A_+(\v,\u)$ for $\u\in V(\v,4)$ is finite by
Lemma~\ref{lemma:V-finite}.  A set defined by the intersection of a finite number
of closed half-spaces is a polyhedron.
\end{proof}

\begin{lemma}[Voronoi compact]\guid{DRUQUFE}\rz{0}
\formalauthor{Nguyen Tat Thang}
\oldrating{80} 
Let $ V$ be a saturated packing.  For every $\v\in  V$, 
the Voronoi cell $\Omega( V,\v)$,  is
compact, convex, and measurable.
\end{lemma}

\begin{proof}  By the previous lemma, it is a bounded polyhedron.  Every bounded
polyhedron is compact, convex, and measurable.
\end{proof}




\subsection{reduction to a finite packing}

We finally state the main result of this book, the Kepler conjecture.
The proof fills most of this book. This section describes the
outline of the proof and gives references to the sources of the
details of the proof.


\begin{theorem}[Kepler's conjecture on Dense Packings]\guid{IJEKNGA}\rz{0} 
\label{theorem:kepler}   
\formaldef{Kepler conjecture}{kepler\_conjecture}%
No packing of congruent balls in Euclidean three space has density
greater than that of the face-centered cubic (FCC) packing.
\end{theorem}
\indy{Index}{face-centered cubic}
\indy{Index}{FCC}
\indy{index}{HCP}
\indy{Index}{Kepler conjecture}

\begin{remark}\guid{LLFORJR}\rz{0}
This density is $\pi/\sqrt{18}\approx 0.74.$  There are other
packings, such as the HCP or the FCC
packing with finitely many balls removed, that attain this
same density.
\end{remark}

The Kepler conjecture is a statement about space-filling packings.  A
space-filling packing is specified by a countable number of real coordinates,
three for the position of each of countably many balls.  The first
task in resolving the conjecture is to reduce the problem to one
involving only a finite number of balls.  This is accomplished by
Lemma~\ref{lemma:deltabound}.

The relevant concepts are \newterm{negligibility} and {\it
  FCC-compatibility}, given as follows.  Fcc-compatibility means that
the Voronoi cells on average have volume at least that of those in the
FCC-packing.  Negligibility means that the error term is insignificant.


\begin{definition}[negligible,~FCC-compatible]\guid{ZREKEVW}\label{def:negligible}
\formaldef{FCC compatible}{fcc\_compatible}
\formaldef{negligible}{negligible\_fun\_0}
A function $G:V\to \ring{R}$ on a set $V\subset\ring{R}^3$
is \newterm{negligible\/}
if there is a constant $c_1$ such that for all $r\ge1$,
% and all $\p\in\ring{R}^3$,
\begin{displaymath}\sum_{\v\in V(\orz,r)} G(\v) \le c_1
r^2.\end{displaymath}
A function $G: V\to\ring{R}$ is
\newterm{FCC-compatible\/}
if for all $\v\in V$, 
\begin{displaymath}\sqrt{32}\le \op{vol}(\Omega(V,\v)) +
G(\v).\end{displaymath}
\indy{Index}{negligible}%
\indy{Index}{FCC-compatible}%
\indy{Notation}{G (negligible function)}%
\end{definition}


\begin{remark}\guid{RTMZJVG}\rz{0}
  The value $\op{vol}(\Omega(V,\v)) + G(\v)$ may be interpreted as an
  \newterm{adjusted\/} volume of the Voronoi cell. The constant
  $\sqrt{32}$ that appears in the definition of FCC-compatibility is
  the volume of the Voronoi cell in the HCP and FCC packings.  The
  corrected volume is at least the volume of these Voronoi cells when
  the correction term $G$ is FCC-compatible.  
%
\indy{Index}{corrected    volume}%
\end{remark}

% \begin{remark}\guid{GRKKKGN}\rz{0} In \cite{Hales:2006:DCG}, the full Voronoi cell
%   $\Omega(V,\v)$ is used, rather than $\Omega(V,\v)$.  The truncation at
%   radius $2$ is just a matter of convenience to guarantee the
%   boundedness and hence the finite volume of the (truncated) Voronoi
%   cell.  In \cite{Hales:2006:DCG}, the same effect was achieved by
%   requiring all packing%s
%   to be saturated.  Drop the assumption of saturation on $ V$.
%\end{remark}



The density $\delta( V,\p,r)$ of a packing $ V$ within a bounded
region of space is defined as a ratio. The numerator is volume of
$B(V,\p,r)$, defined as the intersection with $B(\p,r)$ of the union
of all balls in the packing.  The denominator is the volume of
$B(\p,r)$. 
%An ordered pair $( V,\v)$ with $\v\in V$ is called a \newterm{centered
%packing}.  \indy{Index}{packing!centered}%
\indy{Notation}{ZZdelta@$\delta( V,\p,r)$}%
\indy{Notation}{V@$ V$ (packing)}%
%\indy{Notation}{V@$ V^*$(packing)}%


\begin{lemma}[reduction to finite dimensions]\guid{JGXZYGW}\rz{0} 
\oldrating{150}
\formalauthor{Nguyen Tat Thang}
\label{lemma:deltabound} % 
If there exists a \newterm{negligible} \newterm{FCC-compatible}
function $G: V\to\ring{R}$ for a saturated packing $ V$, then there
exists a constant $c$ such that for all $r\ge1$,
% and all $\p\in\ring{R}^3$, %%  removed on  Jan 16, 2009 after formalization.
\begin{displaymath} 
\delta( V,\orz,r)
\le \pi/\sqrt{18} + c/r.
\end{displaymath}
%The constant $c$ depends on $ V$ only through the constant
%$c_1$ of Definition~\ref{def:negligible}.
\end{lemma}

%\begin{remark}\label{conj:fcc-neg} 
%For every saturated packing $ V$, there exists a negligible
%FCC-compatible function $G: V\to R$.
%\end{remark}



\begin{proof} 
The volume of $B( V,\orz,r)$ is at most the product of the volume
$4\pi/3$ of a single ball $4\pi/3$ with the number of centers in
$B(\orz,r+1)$.  Hence
\begin{equation} 
\op{vol}\, B( V,\orz,r) \le \card( V(\orz,r+1)) 4\pi/3.
\label{eqn:Abound}
\end{equation}

%In a %saturated packing 
Each truncated Voronoi cell is contained in a ball of
radius $2$ that is concentric with the unit ball in that cell.  The volume
of the large ball $B(\orz,r+3)$ is at least the combined volume of 
all truncated Voronoi
cells centered in $B(\orz,r+1)$. This observation,
combined with FCC-compatibility and negligibility, gives
\begin{equation} 
\begin{split} 
\sqrt{32}\,\,\card( V(\orz,r+1))
&\le \sum_{\v\in V(\orz,r+1)} (G(\v) +
\op{vol}(\Omega(V,\v))) \\
&\le c_1 (r+1)^2 + \op{vol}\,B(\orz,r+3) \\
&\le c_1 (r+1)^2 + (1+3/r)^3 \op{vol}\,B(\orz,r)
\label{eqn:Bbound}
\end{split}.
\end{equation}
\indy{Index}{FCC-compatible}%
Recall that $\delta( V,\orz,r)=
\op{vol}\,B( V,\orz,r)/\op{vol}\,B(\orz,r)$. Divide Inequality
\ref{eqn:Abound} through by $\op{vol}\,B(\orz,r)$.  Use
Inequality~\ref{eqn:Bbound} to eliminate $\card( V(\orz,r+1))$ from the
resulting inequality.  This gives
\begin{displaymath}\delta( V,\orz,r)
\le \frac{\pi}{\sqrt{18}} (1+3/r)^3 + c_1 \frac{(r+1)^2}{r^3\sqrt{32}}.
\end{displaymath}
The result follows for an appropriately chosen constant $c$
(depending on $c_1$).
\end{proof}

\begin{remark}[Kepler conjecture in precise terms]\guid{ZHIQGGN}\rz{0}
\label{remark:precise} 
The precise meaning of the \newterm{sphere packing problem} or the
\newterm{Kepler conjecture} is to prove the bound bound $\delta(
V,\orz,r) \le \pi/\sqrt{18} + c/r$ for every saturated packing $ V$.
The error term $c/r$ comes from the boundary effects of a bounded
container holding the balls.  The error tends to zero as the radius
$r$ of the container tends to infinity.  Thus, by the preceding lemma,
the existence of a negligible FCC-compatible function provides the
solution to the packing problem.  The strategy is to define a
negligible function and then to solve an optimization problem in
finitely many variables to establish that the function is also
FCC-compatible.
\end{remark}





\section{Rogers Simplex}\label{sec:rogers}



% Think of $ V$ as the set of centers of a packing of congruent balls
% of radius $1$. To be saturated means that there is no room for
% further balls to be added to the packing. There is no loss in
% generality in assuming that the packing is saturated, when searching
% for the greatest possible density of a packing.

Rogers gave a bound on the density of sphere packings in Euclidean
space of arbitrary dimension~\cite{Rogers:1958:Packing}.  His bound
states that the density of a packing in $n$-dimensions cannot exceed
the ratio of the volume of $A \cap T$ to the volume of $T$, where $T$
is a regular tetrahedron of side length $2$ and $A$ is the set of
$n+1$ balls of radius $1$ placed at the extreme points of $T$.  In two
dimensions, the Rogers's bound is sharp and gives a solution to the
sphere packing problem.  In three dimensions the bound is
approximately $0.7797$, which differs significantly from the optimal
value $0.74$\dots.  Rogers bound is the unattainable density that
would result if regular tetrahedra could tile
space.\footnote{Aristotle erroneously believed that regular tetrahedra
  tile space: ``It is agreed that there are only three plane figures
  which can fill a space, the triangle, the square, and the hexagon,
  and only two solids, the pyramid and the cube.''~\cite{Aristotle}.}

To prove his bound, Rogers gives a partition of Euclidean space into
simplices with extreme points in a packing $V$.   This section develops
the basis properties of Rogers simplices.  In the next section, we
 show how to modify the simplices to obtain a sharp bound on the
density of packings.




\subsection{faces}

Rogers partition is a refinement of the Voronoi cell decomposiiton.
In preparation for this decomposition, this subsection goes into
greater detail about the structure of the faces of a Voronoi cell.  We
parametrize various faces of the Voronoi cell by lists of points in a
saturated packing $V$.

\begin{definition}[$\Omega$ reprise]\guid{BBDTRGC} 
\formaldef{$\Omega(V,W)$}{voronoi\_set}
\formaldef{$\Omega(V,\bu)$}{voronoi\_list}
Let $V$ be a saturated packing.
The notation $\Omega(V,*)$ can be \newterm{overloaded} to denote intersections
of Voronoi cells, when the second argument is a set or list of points.
If $W\subset V$, %(or if $W$ is an ordered tuple of elements of $ V$), 
then the intersection of the family of Voronoi cells is  $\Omega(V,W)$:
\begin{displaymath}\Omega(V,W) = \bigcap \{\Omega(V, \u)\mid \u\in W
\}.\end{displaymath}
Define $\Omega$ on lists % by recursion:
%\begin{displaymath} 
%\begin{align} 
%\Omega(V,[]) &= \ring{R}^3,\\
%\Omega(V,\u_0\cooln[\u_1;\ldots;\u_k]) &= \Omega(V,\u_0)\cap \Omega(V,[\u_1;\ldots;\u_k]).\\
%\end{align}
%\end{displaymath}
to be the same as its value on point sets: 
\begin{align*} 
%\Omega(V,[]) &= \ring{R}^3,\\
%\Omega(V,\u_0\cooln\bu) &= \Omega(V,\u_0)\cap \Omega(V,\bu).\\
\Omega(V,[\u_0;\ldots;\u_k]) = \Omega(V,\{\u_0;\ldots;\u_k\}).
\end{align*}
%Call $\Omega(V,W)$ the $W$-face of $\Omega(V, \u)$ when $ \u\in W$.
\end{definition}

An intersection of Voronoi cells can be written in many equivalent forms:
\begin{displaymath} 
  \Omega(V,\v)\cap \Omega(V,\u) =\Omega(V,\{\u,\v\})= \Omega(V,\v)\cap A_+(\u,\v) 
  = \Omega(V,\v)\cap A(\u,\v) =  \cdots
\end{displaymath}





\begin{definition}[$\bV$]\guid{NOPZSEH} 
  \formaldef{$\bV(k)$}{barV} Let $V$ be a saturated packing.
  When $k=0,1,2,3$, let $ \bV(k)$ be the set of lists
  $\bu=[\u_0;\ldots;\u_k]$ of length $k+1$, with $ \u_i\in V$, such
  that
\begin{equation}\label{eqn:omega-dim} 
\dimaff(\Omega(V,[\u_0;\ldots;\u_j])) = 3-j,
\end{equation}
for all $0<j\le k$.  (Recall that $\dimaff(X)$ is the affine dimension
of $X$ from Definition~\ref{def:affine}.)  Set $\bV(k)=\emptyset$ for
$k>3$.  \indy{Notation}{dimaff@$\dimaff$}%
\end{definition}

In particular, $V$ can be identified with $\bV(0)$ under the natural
bijection $\v\mapsto[\v]$, and $\bV(1)$ is the set of lists $[\u;\v]$
of distinct elements such that the Voronoi cells at $ \u$ and $\v$
have a common facet (Lemma~\ref{lemma:omega-facet}).  

\begin{notation}[underscore]
  % Given a saturated packing $V$, and natural number $k$, the
  % definition gives $\bV(k)$ as a certain set of lists of points in
  % $V$.
  We hope that the use of underscores does not lead to confusion.  The
  underscore in $\bV(k)$ is a function
\begin{displaymath} 
\underline{\phantom V}:\{V \mid \text{$V$ saturated packing} \}
\times \ring{N} \to \ldots
\end{displaymath}
Contrast this with the use of the underscore in $\bu$.  Here the
underscore is not a function, but part of its name, following a
general typographic convention to mark lists of points. The two uses
are coherent in the sense that $\bu\in\bV(k)$.
\end{notation}

\begin{notation}[$\trunc{\bu}{j}$]\guid{JNRJQSM}\rz{0}
%\indy{Notation}{u@$\bu$ (list of points)}% index misformats
\indy{Notation}{d@$d_j$ (truncation of lists)}%
\formaldef{$\trunc{\bu}{j}$} {truncate\_simplex}
When $\bu=[\u_0;\ldots;\u_k]$ and $j\le k$, write
%\footnote{The  notation follows the syntax of Python slices.} 
$\trunc{\bu}{j} = 
[\u_0;\ldots;\u_j]$ for the truncation of the list.  
\indy{Notation}{1@$\trunc{*}{:*}$}%
\end{notation}

Truncation $\bu\mapsto\trunc{\bu}{j}$ maps $\bV(k)$ to $\bV(j)$, when
$j\le k$.  Beware of the index: $k$ is the \newterm{codimension} of
$\Omega(V,\bu)$ in $\ring{R}^3$, when $\bu\in \bV(k)$; it is not the
\newterm{length} of the list $\bu$ (which is $k+1$).\footnote{By
  convention aa $d$-simplex is presented as a $d+1$-tuple.  Because of
  this shift by one, the notation $\trunc{\bu}{j}$ also differs by the
  same shift.}


\begin{lemma}[Voronoi face]\guid{KHEJKCI}\rz{150}\label{lemma:omega-face}  
Let $V\subset\ring{R}^3$ be a saturated packing.
Let $\bu=[\u_0;\ldots;\u_{k}]\in \bV(k)$.  
Then $\Omega(V,\bu)$ is a face of $\Omega(V,\u_0)$.
\end{lemma}

\begin{proof} $\Omega(V,\bu)$ is the intersection of $\Omega(V,\u_0)$ with
the planes $A(\u_0,\u_i)$, where $i>0$.  
%Alternatively, it is the
%intersection of $\Omega(V,\u_0)$ with the planes $A(\u_i,\u_0)$, where $i>0$.
It follows directly from the definition of a face that each plane
$A(\u_0,\u_i)$ is a face of the polyhedron $A_+(\u_0,\u_i)$.  Thus, if
the ``open convex hull''
\begin{displaymath} 
\op{aff}^0_+(\emptyset,\{\p,\q\}) = \{ s \p + t \q \mid s>0,\quad t>0,~\quad s + t = 1\}
\end{displaymath} 
meets
$\Omega(V,\bu)$ for some $\p,\q\in \Omega(V,\u_0)\subset
A_+(\u_0,\u_i)$, then it also meets the face $A(\u_i,\u_0)$ and $\p,\q$
must also lie in the face $A(\u_0,\u_i)$ of the polyhedron
$A_+(\u_0,\u_i)$ containing $\p,\q$ (by
the definition of face).  Then $\p,\q$ also lie in
$\Omega(V,\bu)$.  By the definition of face, $\Omega(V,\bu)$ is a face
of $\Omega(V,\u_0)$.
\end{proof}

\begin{lemma}[facets]\guid{IDBEZAL}\rz{250}\label{lemma:omega-facet} 
  Let $V\subset\ring{R}^3$ be a saturated packing.  Let $\bu\in
  \bV(k)$ for some $k<3$.  Then $F$ is a facet of $\Omega(V,\bu)$ if
  and only if there exists $\bv\in \bV(k+1)$ such that $F =
  \Omega(V,\bv)$ and $\trunc{\bv}{k} = \bu$.
\end{lemma}

\begin{proof} 
  Use Lemma~\ref{lemma:V4} to write the polyhedron $\Omega(V,\bu)$ in
  the form of Equation~\ref{eqn:polyrep}:
\begin{displaymath} 
\Omega(V,\bu) = A \cap A_\pm(\v_1,\u_0) \cap \cdots\cap A_\pm(\v_r,\u_0),
\end{displaymath}
where $A$ is the affine hull of $\Omega(V,\bu)$, $\v_i\in V$, where
$A_-(\v,\w) = A_+(\w,\v)$ with the signs $\pm$ chosen as needed, and
$r$ is as small as possible.  By Lemma~\ref{lemma:webster}, if $F$ is
any facet of $\Omega(V,\bu)$, then there exists an $i\le r$ such that
\begin{displaymath} 
F = \Omega(V,\bu) \cap A(\v_i,\u_0) = \Omega(V,\bv),
\end{displaymath}
where $\bv = [\u_0;\cdots;\u_k;\v_i]$ is the list that appends $\v_i$ to $\bu$.
Also, since $F$ is a facet:
\begin{displaymath} 
\dimaff(\Omega(V,\bv)) = \dimaff(F) = \dimaff(\Omega(V,\bu)) - 1 = 3 - k - 1.
\end{displaymath}
So $\bv \in \bV(k+1)$.  This proves the implication in the forward direction.

To prove the converse, let $\bv\in \bV(k+1)$, where $\trunc{\bv}{k} =
\bu$.  Elementary verifications show that $\Omega(V,\bv)\subset
\Omega(V,\bu)$ and that this set is nonempty if $k<3$.  By
Lemma~\ref{lemma:omega-face} and Lemma~\ref{lemma:webster},
$\Omega(V,\bv)$ is a face of $\Omega(V,\bu)$.  By the definition of
$\bV(*)$,
\begin{displaymath} 
\dimaff(\Omega(V,\bv)) = 3 - (k+1) = \dimaff(\Omega(V,\bu)) -1,
\end{displaymath}
so $\Omega(V,\bv)$ is a facet of $\Omega(V,\bu))$.
\end{proof}


\subsection{partitioning space}

Each Rogers simplex is given as the convex hull of  its set of extreme points.  
The extreme points $\omega(d_i\bu)$
are defined by recursion.

\begin{definition}[$\omega$]\guid{JJGTQMN}
  \formaldef{$\omega_k$} {omega\_list\_n} \formaldef{$\omega$}
  {omega\_list} Let $V$ be a saturated packing and let
  $\bu=[\u_0;\ldots]\in \bV(k)$ for some $k$.  Define points
  $\omega_j=\omega_j(V,\bu)\in\ring{R}^3$ by recursion over $j$.
\begin{align*}
\omega_{0\phantom{+1}} &= \u_0,\\
\omega_{j+1} &=\text{the closest point to } \omega_j \text{ on }\Omega(V,\trunc{\bu}{j+1}).
\end{align*}
Set $\omega(V,\bu) = \omega_{k}(V,\bu)$, when $\bu\in \bV(k)$.  The set $V$ is generally fixed
and is dropped from the notation.
%By induction $\omega_j(\bu) = \
%Define $\omega$ \mid \coprod_{j=0}^3 \bV(j)\to \ring{R}^3$ by recursion over $j$ as follows.
%Let \begin{displaymath} 
%\omega([\u]) = \u,
%\end{displaymath} and
%let $\omega( \bu)$ be the closest point on $\Omega(V, \bu)$ to
%$\omega( \trunc{\bu}{j})$, when $\bu\in \bV(j+1)$.
\end{definition}

\claim{The point $\omega(\bu)$ exists when $\bu\in V(k)$.}  Indeed,
the set $\Omega(V, \bu)$ is nonempty, convex, and compact.  Thus, by
convex analysis, the closest point $\omega( \bu)$ exists uniquely.

The point $\omega_k$ depends on $\bu$ through its projection to
$\trunc{\bu}{k}$ so that
\[
\omega_k(\bu) = \omega_k(\trunc{\bu}{k})=\omega(\trunc{\bu}{k}).
\]
\indy{Notation}{zzomega@$\omega$ (extreme points of Rogers
  simplex)}%

\begin{definition}[R,~Rogers simplex]\guid{PHZVPFY} 
\formaldef{$R$}{rogers}
Let $V\subset\ring{R}^3$ be a saturated packing. For $\bu\in \bV(k)$, let 
 \begin{align*} 
 R(\bu) &= \op{conv}\{\omega( \trunc{\bu}{0}), \omega(
 \trunc{\bu}{1}),\ldots,\omega( \trunc{\bu}{k})\}.
% R^0(\bu) &= \op{aff}^0_+(\emptyset,\{\omega( \trunc{\bu}{0}), \omega(
% \trunc{\bu}{1}),\ldots,\omega( \trunc{\bu}{k})\}).
 \end{align*}  
%Set $R(\bu)=R(0,\bu)$.  
The set $R(\bu)$ is called the Rogers simplex of $\bu$.
\indy{Notation}{R@$R$ (Rogers simplex)}%
\end{definition}

Euclidean space can be partitioned into Rogers simplices.

\begin{lemma}[Rogers decomposition]\guid{GLTVHUM}\rz{700} 
For any saturated packing $V\subset\ring{R}^3$, and any $\u_0\in V$,
\begin{equation} 
\Omega(V,\u_0) = \bigcup \{ R(\bv) \mid \bv\in \bV(3),~\trunc{\bv}{0} =[\u_0]\}.
\end{equation}
Consequently,
\begin{displaymath}
\ring{R}^3 = \bigcup\, \{ R(\bv) \mid \bv\in\bV(3)\}.
\end{displaymath}
\end{lemma}

\begin{proof} 
The proof uses the following standard facts about convex sets and polyhedra from
Section~\ref{sec:poly}.


%
Now turn to the proof.  
By the covering of $\ring{R}^3$  by Voronoi cells by (\ref{eqn:vor-rn}),
%\begin{displaymath}\ring{R}^3 = \bigcup\, \{\Omega(V, \bu)\mid \bu\in
%\bV(0)\}.\end{displaymath}
it is enough to show that each Voronoi cell is covered by Rogers simplices.

Let $\bu\in \bV(j)$ for $j<3$.
Consider the following set:
\begin{displaymath} 
N = \left\{k\in\ring{N}\mid j\le k\le 3, ~~\Omega(V,\bu) 
= \bigcup_{\bv \in \bV(k),~\trunc{\bv}{j}=\bu}
\op{conv}(O_k \cup\Omega(V,\bv)) %\\&\qquad\qquad
%\mid
%\right\}
\right\},
\end{displaymath}
where $O_k = \{\omega(\trunc{\bv}{j}),\ldots,\omega(\trunc{\bv}{k-1})\}$.

\claim{We claim $N = \{j,\ldots,3\}$.}  Indeed, to see that $j\in N$, note that
\begin{displaymath} 
\Omega(V,\bu) = \op{conv}(\Omega(V,\bu)),
\end{displaymath}
which holds by the convexity of the polyhedron $\Omega(V,\bu)$.
Assume that $k\in N$, and consider the membership condition of $N$  for
$k+1$.  We may assume that $k+1\le 3$.
Then
\begin{align*} 
&\phantom{=}\bigcup _{\bv \in \bV(k+1),~\trunc{\bv}{j}=\bu}
%\left\{
\op{conv}(O_{k+1} \cup\Omega(V,\bv))
%\mid%
%
%\right\}
\vspace{6pt}\\
&=\bigcup _{\bv \in \bV(k+1),~\trunc{\bv}{j}=\bu}
%\left\{
\op{conv}(O_{k} \cup\op{conv}(\{\omega(\trunc{\bv}{k})\}\cup\Omega(V,\bv)))
%\mid
%\right\}
\vspace{6pt}\\
&=\bigcup _{\bw\in \bV(k),~\trunc{\bw}{j}=\bu}
%\left\{
\bigcup_{\bv \in \bV(k+1),~\trunc{\bv}{k}=\bw}
%\left\{
\op{conv}(O_{k} \cup\op{conv}(\{\omega(\trunc{\bv}{k})\}\cup\Omega(V,\bv)))
%\mid
%\right\}\mid 
%\right\}
\vspace{6pt}\\
&=\bigcup _{\bw\in \bV(k),~\trunc{\bw}{j}=\bu}
%\left\{
\op{conv}(O_{k} \cup \Omega(V,\bw)    )
%\mid 
%\right\}
\vspace{6pt}\\
&=\quad\Omega(V,\bu).
\end{align*}
The induction hypothesis is used in the last step.  
This proves $k+1\in N$, and induction gives $N=\{j,\ldots,3\}$.

Consider  the extreme case $j=0$ and $k=3$.  The set $\Omega(V,\bv)$
reduces to $\{\omega(\bv)\}$  and the convex hull becomes
\begin{displaymath} 
\op{conv}(O_{k}\cup \Omega(V,\bv)) = R(\bv),
\end{displaymath}
when $\bv\in \bV(3)$.
This gives
\begin{equation} 
\Omega(V,\u_0) = 
\bigcup \{ R(\bv) \mid \bv\in \bV(3),~\trunc{\bv}{0} =[\u_0]\}.
\end{equation}
This proves the lemma.
\end{proof}


%%%
%\begin{displaymath}\Omega(V,\trunc{\bu}{0}) = 
%\bigcup\, \{ R(\bv) \mid \trunc{\bu}{0}=\trunc{\bv}{0},~\bv\in  \bV(3)\}.
%\end{displaymath}
%By Lemma~\ref{lemma:webster}, the relative boundary of the bounded
%polyhedron $\Omega(V,\trunc{\bu}{0})$ is the union of its facets.  The
%polyhedron can be partition into the cones over these facets.
%\begin{displaymath}\Omega(V,\trunc{\bu}{0}) = \bigcup\, \left\{
%  ~\op{conv}(\{\omega(\trunc{\bu}{0})\}\cup \Omega(V,\trunc{\bu}{1}))   
%   \mid \trunc{\bu}{1}\in  \bV(1)
%   \right\}.
%\end{displaymath}
%It is enough to show that
%\begin{displaymath} 
%\Omega(V,\trunc{\bu}{1}) = \bigcup\, 
%\{ R(1,\bv) \mid \trunc{\bv}{1}=\trunc{\bu}{1},~\bv\in  \bV(3)\}.
%\end{displaymath}
%After successively partitioning each facet into cones over facets of
%facets, it is enough to show that
%\begin{displaymath}\Omega(V,\trunc{\bu}{3}) 
%= \bigcup\,\{R(3,\bv)\mid\trunc{\bv}{3}=\trunc{\bu}{3},~\bv\in \bV(3)\}.\end{displaymath}
%The right-hand side is the singleton $\{\omega(\trunc{\bu}{3})\}$.
% The left-hand side contains this point and is contained in a
% properly decreasing chain of affine sets: $\ring{R}^3$, the bisector
% of $ \u_0$ and $ \u_1$, and so forth.  This determines a unique
% point, so the two sides are equal.
%\end{proof}

The intersection of two different Rogers simplices is a null set.

\begin{lemma}[Rogers disjoint]\guid{DUUNHOR}\rz{300}  \label{lemma:R-inter} 
Let $V$ be a saturated packing, and let $\bu,\bv\in \bV(3)$ be lists such that 
$R(\bu)\ne R(\bv)$.  Then the intersection 
\begin{displaymath} 
R(\bu)\cap R(\bv)
\end{displaymath}
is contained in a plane (and hence has measure zero).
\end{lemma}

This result and the previous lemma show that the simplices $R(\bu)$
partition Euclidean three-space.

\begin{proof} Let $\bu = [\u_0;\ldots]$ and $\bv = [\v_0;\ldots]$.  
Let $k$ be the
first index such that
\begin{displaymath} 
\Omega(V,[\u_0;\ldots;\u_{k}]) \ne \Omega(V,[\v_0;\ldots;\v_{k}]).
%\op{conv}\{\omega( [\u_0]),\ldots,\omega( [\u_0;\ldots; \u_{k}])\}\ne
%\op{conv}\{\omega([\v_0]),\ldots,\omega([\v_0;\ldots;\v_{k}])\}.
\end{displaymath}

\claim{Such an index $k$ exists.}  Indeed, the definition of points
$\omega(\trunc{\bu}{i})$ depends on $\bu$ only through the sets
$\Omega(V,\trunc{\bu}{j})$.  Hence, $R(\u)\ne R(\v)$ implies that the
two sequences $\Omega(V,*)$ must differ at some index.

The intersection $R(\u)\cap R(\v)$ lies in the convex hull $C$ of
$\{\omega([\u_0]),\ldots,\omega( [\u_0;\ldots; \u_{k-1}])\}$ and
$\Omega'=\Omega(V,[\u_0;\ldots;\u_{k};\v_{k}])$.  The set $\Omega'$
lies in a facet of $\Omega(V,\trunc{\bu}{k})$.  Hence the affine
dimension of $\Omega'$ is at most $3-k-1=2-k$.  In general, if a set
$A$ has affine dimension $r$, then the affine dimension of
$\op{conv}(\{\p\}\cup A)$ is at most $r+1$.  It follows that the
affine dimension of $C$ is at most $(2-k)+ k = 2$.  The intersection
is thus contained in a plane.
\end{proof}


\subsection{circumcenter}

The extreme points of a Rogers simplex are closely related to the circumcenter
of various subsets of $V$.  This subsection develops the connection between
Rogers simplices and circumcenters.

\begin{definition}[circumcenter,~circumradius]\guid{IFLFHKT} 
\formaldef{circumcenter}{circumcenter}%
\formaldef{circumradius}{radV}%
Let $S\subset\ring{R}^N$.  
A point $\p$ is a \newterm{circumcenter} of $S$ if it is an element
in the affine hull of $S$ that is equidistant from every $\v\in S$.  If $S$ has
circumcenter $\p$, then the common distance $\norm{\p}{\v}$, for all $\v\in S$,
is the \newterm{circumradius} of  $S$.
\end{definition}

The circumcenter comes as a solution to a system of linear equations.
We pause to review a standard result from the theory of linear
algebra, asserting the existence of a solution to a system of
equations.  Recall that a finite set $S$ is \newterm{affinely
  independent} if $\dimaff(S) = \card(S) -1$.  
%
\formaldef{affinely  independent}{\textasciitilde affine\_dependent s}%

\begin{lemma}[linear systems]\guid{QXSKIIT}\rz{500}\label{lemma:affine-system} 
  Let $S=\{\v_0,\ldots,\v_n\}\subset \ring{R}^N$ be an affinely
  independent set of cardinality $n+1$.  Then every system of
  equations
\begin{displaymath} 
\p \cdot (\v_i - \v_0) = b_i-b_0,\qquad i=1,\ldots,n,
\end{displaymath}
has a unique solution in $\p$ that lies in the affine hull of $S$.
\end{lemma}

\begin{proof} This is a standard result from linear algebra.
We sketch a proof for the sake of completeness.  

Let $\w_i = \v_i-\v_0$ and replace $b_i-b_0$ with $b_i$.  
The lemma reduces to the following claim.
Let $S' = \{\w_1,\ldots,\w_n\}$ be a \newterm{linearly independent} set
of cardinality $n$.  Then every system of equations
\begin{displaymath} 
\p \cdot \w_i = b_i,\qquad i=1,\ldots,n,
\end{displaymath}
has a unique solution in $\p$ that lies in the linear span of $S'$.

\claim{A solution is unique}. Indeed, the difference 
$\p = \p'-\p'' = \sum s_i \w_i$ of two solutions
$\p',\p''$ satisfies
\begin{displaymath} 
\normo{\p}^2=\p\cdot\p = \sum s_i \w_i \cdot (\p' - \p'') =
\sum s_i (b_i-b_i)= 0.
\end{displaymath}
So that $\p=0$ and $\p'=\p''$.  This proves uniqueness.

Let $W$ be the linear span of $\w_1,\ldots,\w_n$.  The image of the
map $W\to\ring{R}^n$, $\p\mapsto (\p\cdot\w_1,\ldots,\p\cdot\w_n)$ is
a linear space, hence an affine set.

\claim{A solution exists; that is, the image is all of $\ring{R}^n$.}
Otherwise, by Lemma~\ref{lemma:aff-u} some equation must hold; that
is, there exists $\u\ne \orz$ such that $\u\cdot \q =b$ for every
point $\q$ in the image.  As $\orz$ lies in the image, $b=0$.  Write
$\p = \sum u_i \w_i\in W$.  Then
\begin{displaymath} 
\normo{\p}^2 = \p\cdot\p = \sum u_i (\p \cdot \w_i) = \u\cdot \q = 0,
\end{displaymath} 
where $\q\in\ring{R}^n$ is the image of $\p\in W$.
Thus $\p=\orz$ so that $\u=0$, and we have reached a contradiction.
\end{proof}

\begin{lemma}[circumcenter exists]\guid{OAPVION}\rz{100} 
  Let $S\subset \ring{R}^N$ be a non-empty affinely independent set.
  Then there exists a unique circumcenter of $S$.
\end{lemma}

\begin{proof} 
  A point $\p$ is a circumcenter if and only if it is a point in the
  affine hull of $S$ that satisfies the system of equations:
\begin{displaymath} 
\norm{\p}{\v_i}^2 = \norm{\p}{\v_0}^2,\qquad i = i,\ldots,n.
\end{displaymath}
Equivalently,
\begin{displaymath} 
\p\cdot (\v_i-\v_0) = b_i-b_0,\qquad i=1,\ldots,n,
\end{displaymath}
where $b_i = \normo{\v_i}^2/2$.  By
Lemma~\ref{lemma:affine-system}, this system of equations has a unique
solution.
\end{proof}

The following lemma describes the structure of the affine hull of a
face of a Voronoi cell.  It describes the affine hull as an intersection of half-spaces,
and shows that it meets $\op{aff}(S)$ orthogonally at the circumcenter of $S$.

\begin{lemma}[]\guid{MHFTTZN}\rz{800}\label{lemma:aff-center} 
Let $V$ be a saturated packing and let $k\le 3$.
Let $\bu=[\u_0,\ldots,\u_k]\in \bV(k)$, and set $S = \{\u_0,\ldots,\u_k\}$.
Then
\begin{itemize} 
\item $\dimaff (S)= k$.  (In particular, $\card\{\u_0,\ldots,\u_k\}=k+1$ and
$S$ is affinely independent.)
\item $\aff{\Omega(V,\bu)}= \cap_{i=1}^k A(\u_0,\u_i).$
\item $\aff{\Omega(V,\bu)} \cap \aff(S) = \{\q\}$, 
where $\q$ is the circumcenter of $S$.
\item $(\aff{\Omega(V,\bu)}-\q) \perp (\aff(S)-\q)$, where
  $X-\q$ denotes the translate of a set $X$ by $-\q$, and $(\perp)$ is
  the orthogonality relation.
\end{itemize}
\end{lemma}
\indy{Notation}{1@$\perp$}


\begin{proof}  The proof is by induction on $k$.  

  \claim{The lemma holds when $k=0$.}  Indeed, $\Omega(V,\u_0)$ is an
  open set, so that its affine hull is $\ring{R}^3$.  This is the
  first conclusion.  The other conclusions reduce to trivial facts:
  $\dimaff\ring{R}^3 = 3$; $\dimaff\{\u_0\}=0$; $\ring{R}^3\cap
  \{\u_0\} = \{\u_0\}$; and $\ring{R}^3\perp \{\orz\}$.

Assume the induction hypothesis for $k$.  We may assume that $k<3$,
because otherwise there is nothing further to prove.  Let $\bu\in
\bV(k+1)$.  Let $\bv = \trunc{\bu}{k}\in \bV(k)$.  Let $\q$ be the
circumcenter of (the point set of) $\bv$.  Write $A_j = \cap_{i\le j}
A(\u_0,\u_i)$; $B_j = \aff(\Omega(V,\trunc{\bu}{j}))$; $C_j =
\aff\{\u_0,\ldots,\u_j\}$.   
%; $S_j = \{\u_0,\ldots,\u_j\}$.
By the induction hypothesis $A_k = B_k$.

\claim{We claim $\dimaff\{\u_0,\ldots,\u_{k+1}\}=k+1$.}
Otherwise, by general background facts about affine sets, $\u_{k+1}\in C_k$.
Write $\u_{k+1}-\q=\sum_{i\le k} t_i (\u_i-\q)$.  If $\p\in A_k$, then
by the orthogonality induction hypothesis:
\begin{align*} 
(\u_{k+1}-\q)\cdot (\p-\q) &= \sum t_i (\u_i-\q)\cdot (\p-\q) = 0, \intertext{ and }
\norm{\u_{k+1}}{\p}^2 - \norm{\u_0}{\p}^2 &=
\norm{\u_{k+1}}{\q}^2 - \norm{\u_0}{\q}^2.
\end{align*}
Thus, if $A_k$ meets $A(\u_0,\u_{k+1})$, then $A_k\subset
A(\u_0,\u_{k+1})$.  This is contrary to $0\le \dimaff(A_{k+1}) =
\dimaff(A_k) - 1$, which holds because $\bu\in \bV(k+1)$ with $k<3$.

\claim{We claim that $B_{k+1} = A_{k+1}$.}  Indeed, by definition,
$B_{k+1}\subset A_{k+1}\subset A_k$.  Also,
\begin{displaymath} 
\dimaff B_{k+1} = 3 - (k+1) \le \dimaff{A_{k+1}} \le \dimaff A_k = 3 - k.
\end{displaymath}
Hence, by general background on affine sets, if $A_{k+1}\ne A_k$, then
$B_{k+1}=A_{k+1}$ follows.  Suppose for a contradiction that $A_k =
A_{k+1}$.  Then $\Omega(V,\bv) \subset \Omega(V,\bu) =
\Omega(V,\bv)\cap A(\u_0,\u_{k+1}) \subset \Omega(V,\bv)$, so that
$B_k = B_{k+1}$.  This contradicts the defining conditions of
$\bV(k+1)$.

\claim{We claim that $A_{k+1}\cap C_{k+1} = \{\q_{k+1}\}$, where
  $\q_{k+1}$ is the circumradius of $S_{k+1}$.}  Indeed, by the
definition of $A_{k+1}$, any point in this affine set is equidistant
from every point of $S_{k+1}$.  By the definition of $C_{k+1}$, the
point lies in the affine hull of $S_{k+1}$.  This uniquely
characterizes the circumcenter.

\claim{Finally, $(A_{k+1} -\q)\perp (C_{k+1}-\q)$, where $\q=\q_{k+1}$.}
Indeed, if $\p\in A_{k+1}$, then
\begin{align*} 
0 &=\norm{\p}{\u_i}^2 -\norm{\p}{\u_0}^2\\
&=\norm{(\p-\q)}{(\u_i-\q)}^2 -\norm{(\p-\q)}{(\u_0-\q)}^2\\
&=-2 (\p-\q)\cdot (\u_i-\u_0).
\end{align*}
Since the linear span of the points $\u_i-\u_0$ is all of
$C_{k+1}-\q$, the claim follows.

This completes the induction.
\end{proof}

\begin{definition}[h]\guid{CHNGQBD}
\formaldef{h}{hl}% 
%Let $h(\trunc{\bu}{i}) = \norm{\omega(\trunc{\bu}{i})}{ \u_0}$.  
If $\bu=[\u_0;\u_0;\ldots;\u_k]$ is a list of points in $\ring{R}^N$.
let $h(\bu)$ be the
circumradius of its point set $\{ \u_0,\ldots, \u_k\}$.
\end{definition}
\indy{Notation}{h@$h$ (circumradius)}%

The constant $r=\sqrt2$ is the smallest real number $r$ such that
there exist four cocircular points in the plane with circumradius $r$.
The four points are the vertices of a square of side length $2$.
Eight Rogers simplices meet at the circumcenter of the square, but
when $r<\sqrt2$ only six Rogers simplices meet at the circumcenter.
In general, at $r=\sqrt2$, certain degeneracies start to appear that
cannot occur for a smaller radius.  To avoid degeneracies, we often
assume in the following lemmas that the circumradius is less than
$\sqrt2$.

\begin{lemma}[nondegeneracy]\guid{XYOFCGX}\rz{500}\label{lemma:sqrt2-close} 
  Let $V\subset\ring{R}^3$ be a saturated packing.  Let $S\subset V$
  be an affinely independent set with circumcenter $\p$.  Assume that
  the circumradius of $S$ is less than $\sqrt2$.  Then
  $\norm{\v}{\p}>\norm{\u}{\p}$ for all $\u\in S$ and all $\v\in
  V\setminus S$.
\end{lemma}

\begin{proof} 
Otherwise
there is a point $\w\in V\setminus S$ satisfying
\begin{equation}\label{eqn:closest} 
\norm{\w}{\p}\le \norm{ \u}{\p}, \quad\text{for all }  \u\in S.
\end{equation}
The angles $\arc_V(\p,\{\v, \u\})$ are obtuse for distinct elements
$\v,\u$ of $ S$ because of the law of cosines and
\begin{displaymath} 
\norm{\p}{\u} < \sqrt2,\quad \norm{\p}{ \v} <\sqrt2,\quad \norm{\u}{ \v} \ge 2.
\end{displaymath} 
Let $S=\{\u_0,\ldots,\u_k\}$.
A case-by-case argument follows for each $k\in\{0,1,2,3\}$.

\claim{[$k=0$].}  This case is trivial.

\claim{[$k=1$].}  In this case, the points $\p, \u_0, \u_1$ are collinear and cannot give
two obtuse angles.

\claim{[$k=2$].} In this case, let $\w'$ be the projection of $\w$ to
the plane containing $\p, \u_0, \u_1, \u_2$.  Under orthogonal
projection, the angles remain obtuse:
\begin{displaymath} 
\arc_V(\p,\{\w,\u_i\}) = \arc_V(\p,\{\w',\u_i\}).
\end{displaymath}
The four points $\w', \u_0, \u_1, \u_2$ can
be arranged cyclically around $\p$, according to the polar cycle,
each forming an obtuse angle with
the next.  A circle around $\p$ cannot give four obtuse angles, because the sum is
$2\pi$.

\claim{[$k=3$].}
In this case, assume that $ \u_0,\ldots, \u_3$ are labeled according to the azimuth
cycle
around the line $\op{aff}\{\p,\w\}$.  Consider the dihedral angle
\begin{displaymath} 
\gamma=\gamma_i=\dih(\{\p,\w\},\{ \u_i, \u_{i+1}\})
\end{displaymath}
of the simplex $\{\p,\w,\u_i,\u_{i+1}\}$ along the edge $\{\p,\w\}$.
By the spherical law of cosines, the angle $\gamma$ of the
spherical triangle is given in terms of the edges as
\begin{displaymath} 
\cos c - \cos a \cos b = \sin a \sin b \cos \gamma.
\end{displaymath}
The angles $a,b,c$ are obtuse, so that both terms on the left-hand
side are negative. Thus, $\gamma>\pi/2$.  The azimuth angle
$\op{azim}(\p,\w,\u_i,\u_{i+1})$ is then also greater than $\pi/2$ by
Lemma~\ref{lemma:dih-azim}.  This is impossible, as the sum of the
four azimuth angles $\gamma$ is $2\pi$ by Lemma~\ref{lemma:2pi-sum}.
%This completes the proof that $\omega(\trunc{\bu}{j})$
%is the circumcenter.
\end{proof}

With nondegeneracy established, we can now give further details about
the extreme points of a Rogers simplex and their relationship to the
circumcenter of of a subset $S$ of the packing $V$.

\begin{lemma}[Rogers simplex and circumcenter]\guid{XNHPWAB}\rz{500}\label{lemma:v2} 
Let $V$ be a saturated packing.
Let $\bu=[\u_0;\ldots;\u_k]\in \bV(k)$ for some $k\le 3$,
and let $S=\{\u_0,\ldots,\u_k\}$ be the
point set of $\bu$.
Assume that $h(\bu)<\sqrt2$.
%\begin{displaymath} 
%\norm{\omega(\trunc{\bu}{j}) }{  \u_0} < \sqrt2.
%\end{displaymath}
Then 
%\begin{description} 
\begin{itemize} 
\item%[(circumcenter)]  
$\omega(\bu)$ is the circumcenter of $S$.
\item%[(convex hull)]  
$\omega(\bu)\in\op{conv}(S)$.
\item%[(distinctness)]  
The set $\{\omega(\trunc{\bu}{j})\mid j\le k\}$ has affine dimension $k$.
\item
The sequence $h(\trunc{\bu}{j})$ is
strictly increasing in $j$.
\end{itemize}
%\end{description}
\end{lemma}
\indy{Index}{convex hull}%

\begin{proof} The three conclusions of the lemma  are proved
separately.

\claim{$\omega(\bu)$ is the circumcenter of $S$.} Indeed, by
definition, if $\bu\in \bV(k)$, then
\begin{displaymath} 
\dimaff\Omega(V,[\u_0;\ldots;\u_k]) = 3-k.
\end{displaymath}  
The case $k=0$ of the lemma is trivially satisfied.  Assume by
induction the result holds for natural numbers up to $k$.

Now consider the case $k+1$.  Let $\bu\in \bV(k+1)$, and let $S$ be
the point set of $\bu$.  By the induction hypothesis
$\omega(\trunc{\bu}{k})$ is the circumcenter of the point set of
$\trunc{\bu}{k}$.  Let $\p$ be the point in
$A=\aff(\Omega(V,\bu))$ closest to $\omega(\trunc{\bu}{k})$.  By
Lemma~\ref{lemma:sqrt2-close}, the point $\p\in\Omega(V,S_{j})$.
Thus, $\p=\omega(\bu)$.  By Lemma~\ref{lemma:aff-center}, the
circumcenter of $S$ is the point of intersection of orthogonal affine
sets $\aff(S)$ and $A$.  Thus, the circumcenter equals the unique
point of $A$ closest to any point $\omega(\trunc{\bu}{k})$ in
$\aff(S)$.  The claim follows.

%The circumcenter $\p=\p_{j}$ of the set $S_{j}$ is
%the point on the plane $\op{aff}(\Omega(V,S_{j}))$ closest to
%$\p_{j-1}=\omega(\trunc{\bu}{j-1})$.

% If $j=0$, there is nothing to show.  If $j=1$, the point is the
% midpoint of the convex hull.  If $j=2$, the point is the
% circumcenter of an acute triangle.  If $j=3$, the point is the
% circumcenter of a simplex such that every face has positive
% orientation.  Thus, in every case the point lies in the convex hull.

\claim{We claim $\omega(\bu)\in\op{conv}(S)$.} 
Otherwise, there $\v\in S$ such that $\aff(S')$ separates $\omega(\bu)$ from
$ \v$, where $S'=S\setminus\{v\}$.  Let $\p'$ (resp. $\p=\omega(\bu)$)
be the circumcenter
of $S'$ (resp. $S$).  When $\u\in S'$, the law of cosines gives
\begin{align*} 
\norm{ \u}{\p}^2 &= \norm{\u}{\p'}^2 + \norm{\p'}{\p}^2\\ 
\norm{ \v}{\p}^2 &\ge \norm{\v}{\p'}^2 + \norm{\p'}{\p}^2.
\end{align*}
This gives $\norm{\v}{\p'}\le \norm{\u}{\p'}$.  This is contrary to
Lemma~\ref{lemma:sqrt2-close}.

\claim{The set $\{\omega(\trunc{\bu}{j})\mid j\le k\}$ has affine dimension $k$.}
%\claim{The points $\omega(\trunc{\bu}{j})$ for
%$j\le k$, are all distinct.}
It follows from  Lemma~\ref{lemma:aff-center} that the vectors 
${\omega(\trunc{\bu}{i+1})}-{\omega(\trunc{\bu}{i})}$ are mutually orthogonal.
Thus, the claim about affine dimension easily follows if we show that these vectors
are nonzero.
%Indeed, by the Pythagorean theorem,
%\begin{equation} 
%\norm{\omega(\trunc{\bu}{j})}{\omega(\trunc{\bu}{0})}^2 =
%\sum_{i=0}^{j-1} \norm{\omega(\trunc{\bu}{i+1})}{\omega(\trunc{\bu}{i})}^2.
%\end{equation}
%so it is enough to show that $\omega(\trunc{\bu}{i})\ne
%\omega(\trunc{\bu}{i+1})$.  
Otherwise, the
circumcenter $\omega(\trunc{\bu}{i})$ of $S_i=\{\u_0,\ldots,\u_i\}$
has an equally close point $ \u_{i+1}\in V\setminus S_i$, which is
impossible by Lemma~\ref{lemma:sqrt2-close}.

\claim{The sequence $h(\trunc{\bu}{j})$ is strictly increasing in
  $j$.}  
Indeed, by the Pythagorean theorem,
\begin{equation} 
\norm{\omega(\trunc{\bu}{j})}{\omega(\trunc{\bu}{0})}^2 =
\sum_{i=1}^{j} \norm{\omega(\trunc{\bu}{i})}{\omega(\trunc{\bu}{i-1})}^2.
\end{equation}
so the result follows from the
previous claim.
\end{proof}


% The concept of \newterm{positive orientation} is used in the proof.
% This is discussed in the 1998 proof and in \newterm{Lemmas in Geometry}.
% If a face has circumradius less than $\sqrt2$ it has positive
% orientation.  If every face has positive orientation, then the
% circumcenter of the simplex is contained in its convex hull.

\begin{lemma}\guid{WAUFCHE}\rz{100}
  Let $V$ be a saturated packing.  let $\bu =[\u_0;\ldots]\in
  \bV(k)$ for some $k$.  Then $h(\bu)\le
  \norm{\omega(\bu)}{\u_0}$.  Moreover, if $h(\bu)<\sqrt2$, then
  $h(\bu)=\norm{\omega(\bu)}{\u_0}$.
\end{lemma}

\begin{proof} By construction, the point $\omega(\bu)$ belongs to
  $\Omega(V,\bu)$, and is therefore equidistant to the points in
  $S=\{\u_0,\ldots,\u_k\}$.  The orthogonal projection of
  $\omega(\bu)$ to $\op{aff}(S)$ is the circumcenter of $S$.  The
  orthogonal projection cannot increase distances, so the inequality
  follows.  If $h(\bu)<\sqrt2$, then $\omega(\bu)$ is already the
  circumcenter by Lemma~\ref{lemma:v2}, so that equality holds.
\end{proof}

\subsection{Delaunay simplex}

The Delaunay decomposition of space into simplices is dual to the
Voronoi cell.  Two points in a packing $V$ are joined by an edge if
their Voronoi cells have a common facet.\footnote{The Delaunay
  decomposition may be degenerate if the points of $V$ are not in
  general position.  This book confines itself to the nondegenerate
  situation.}  A $2$-simplex is added with vertices at three points in
$V$ if their Voronoi cells have a common edge.  A $3$-simplex is added
for each common vertex.  A Delaunay $3$-simplex is the convex hull of
four points in the packing $V$.

Under a nondegeneracy condition (on the circumradius of the set of
points), we may construct a Delaunay simplex as a union of Rogers
simplices.  To this end, we examine the set of all Rogers simplices
around a common extreme point.  The convex hull of a nondegenerate set
$S\subset V$ of four points consists of $4!$ Rogers simplices, that
its facets consist of the facets of $3!$ Rogers simplices, and so
forth.  In brief, the Rogers simplices give every nondegenerate
Delaunay simplex an identical simplicial structure.


Recall that $\op{Sym}(k+1)$ be the \newterm{group} of all permutations on the
set $\{0,\ldots,k\}$.  Let $\bu = [\u_0,\ldots, \u_k]$ be a list of
length $k+1$.  For any \newterm{permutation} $\pi\in\op{Sym}(k+1)$,
let $\pi_*(\bu)$ be the \newterm{rearrangement} given by
\begin{displaymath} 
\pi_*(\bu)_i =  \u_{\pi^{-1} i}, % inverse added Feb 16, 2010. left regular action.
\end{displaymath}   
where $\u_i$ denotes the $i$th element of a list $\bu$.
\formaldef{permutation}{permutes}%
\formaldef{$\pi_*$}{left\_action\_list}%
\indy{Notation}{Xpi@$\pi$ (permutation)}
\indy{Notation}{Sym@$\op{Sym}$ (symmetric group)}

The following lemma shows that rearrangements have the same extreme
point of a Rogers simplex.

\begin{lemma}[extreme point rearrangement]\guid{YIFVQDV}\rz{500}   
\label{lemma:perm-Vk} 
  Let $V$ be a saturated packing.  Let $\bu\in \bV(k)$.  Assume that
  $h(\bu)<\sqrt2$. Let $\bv$ be any rearrangement of $\bu$ under a
  permutation.  Then $\bv\in \bV(k)$ and $\omega(\bu) = \omega(\bv)$.
\end{lemma}

\begin{proof} 
Let $\bv = [\v_0;\ldots;\v_k]$.  
Let $S_j = \{\v_0,\ldots,\v_j\}$,  
$\Omega_j = \Omega(V,\trunc{\bv}{j})$, 
$A_j=\cap_{i=1}^j A(\v_0,\v_i)$, $a_j = \dimaff(A_j)$.
By convention, set $A_0 = \ring{R}^3$ so that $a_0=\dimaff(A_0) = 3$.
Also, set $a_{-1} = 4$ by convention.

The set $S_k$ is the point set of $\bu$, which is affinely independent
by Lemma~\ref{lemma:aff-center}.  The set $S_j$ is then also affinely
independent.  Let $p_j$ be the circumcenter of $S_j$.  Then $p_k$ is
the circumcenter of $S_k$.  The circumradius of $S_j$ is at most the
circumradius of $S_k$, which by assumption is less than $\sqrt2$.

\claim{We claim that $\dimaff \Omega_j = a_j$, when $0\le j\le k$.}
By Lemma~\ref{lemma:sqrt2-close}, if $\p=\p_j$, then
\begin{equation}\label{eqn:sqrt2-close} 
\norm{\v}{\p} > \norm{\u}{\p}\text{ for all }\u\in S_j
\text{ and for all }\v\in V\setminus S_j.
\end{equation}   
Pick a small neighborhood $U$ of $\p_j$ such that (\ref{eqn:sqrt2-close}) holds
for all $\p\in U_j$.  By the definition of Voronoi cell, $\Omega_j \cap U=A_j\cap U$.
By background facts on affine sets $\dimaff\Omega_j = \dimaff A_j=a_j$.  This gives
the claim.

To prove the lemma, we prove the following claims by simultaneous induction on $j$.
For all $0\le j\le k$ we have
\begin{itemize} 
\item $a_j \ge a_{j-1} - 1\ge 3-j$.
\item $a_j = 3-j$ if and only if $a_i=3-i$ for all $0\le i\le j$.
\end{itemize}
The base case $j=0$ is trivial.  Assume the induction hypothesis for $j$.

$A_{j+1} = A_{j}\cap A(\v_0,\v_{j+1})$.  The intersection contains $\p_{j+1}$ and is therefore
nonempty.  By general background facts on the intersection of an
affine set with a hyperplane, $a_{j+1} \ge a_{j}-1$.  By the induction hypothesis,
$a_{j}-1\ge 3-(j+1)$.
If $a_{j+1}=3-(j+1)$, then $a_{j}=3-j$ and by the induction hypothesis
$a_{i}=3-i$ for all $0\le i\le j$. This completes the proofs of the claims by induction.

$a_k = \dimaff A_k=\dimaff \Omega_k$.  However, $\Omega_k=
\Omega(V,\bu)$, and since $\bu\in \bV(k)$, it follows that $0=\dimaff\Omega(V,\bu)=a_k$.
By the established claims, $a_i = 3-i$ for all $0\le i\le k$.  This proves $\bv\in\bV(k)$.

Finally $\omega(\bu) = \omega(\bv)$ because both equal the circumcenter of the
point set $S_k$.
%Since the sets $\Omega(V,\trunc{\bu}{j})$ satisfy \eqn{eqn:omega-dim}, it
%follows that $\Omega(V,\bu)\cap \op{conv}\{ \u_0,\ldots, \u_k\}$ is
%the singleton $\{\omega(\bu)\}$, which contains the circumcenter of
%the simplex with extreme points $\{ \u_0,\ldots, \u_k\}$.  This describes
%$\omega(\bu)$ in a way that does not depend on the ordering of $
%\u_0,\ldots, \u_k$.
%
%The condition \eqn{eqn:omega-dim} can be shown to hold for $\bv$.
%The proof of Lemma~\ref{lemma:v2} shows that the midpoint of $\v_0$
%and $\v_1\}$ lies in $\Omega(V,\trunc{\bv}{1})$.  By the distinctness
%conclusion of the same lemma, some neighborhood of this midpoint in
%the bisecting plane of $\{\v_0,\v_1\}$ lies in $\Omega(V,\trunc{\bv}{1})$.
%Thus, $\dimaff(\Omega(V,\trunc{\bv}{1}))=2$.  We continue in this fashion to show
%that
%\begin{displaymath} 
%\dimaff(\Omega(V,\trunc{\bv}{j}))=3-j.\end{displaymath}
\end{proof}

The map from permutations to Rogers simplices is one-to-one.

\begin{lemma}[permutations one-to-one]\guid{KSOQKWL}\rz{200} 
  Let $V$ be a saturated packing and let $\bu\in
  \bV(k)$.  Assume that $h(\bu)<\sqrt2$.  Let $\pi\in\op{Sym}(k+1)$ such that
 $R(\bu)= R(\pi_*\bu)$.  Then $\pi= I$.
\end{lemma}

\begin{proof} 
Write $\bv = \pi_*\bu$.  By Lemma~\ref{lemma:v2}, the sets
$\{\omega(\trunc{\bu}{j})\mid j\le k\}$ and 
$\{\omega(\trunc{\bv}{j})\mid j\le k\}$ are each affinely independent of
cardinality $k+1$.  By Lemma~\ref{lemma:simplex-poly}, these are these sets
of extreme points of $R(\bu)$ and $R(\bv)$, respectively.  Thus, it is enough
to show that the sets of extreme points are unequal.

Let $j$ be the largest index such that $\trunc{\bu}{j}=\trunc{\bv}{j}$.
The assumption $\pi\ne I$ implies that $j<k$.  Let $\p$ be the circumcenter
of $\{\u_0,\ldots,\u_{j+1}\}$.  By Lemma~\ref{lemma:sqrt2-close}, 
\begin{displaymath} 
\norm{\u_0}{\p} = \norm{\u_{j+1}}{\p} < \norm{\v_{j+1}}{\p}.
\end{displaymath}
Thus, $\omega(\trunc{\bu}{j+1}) \ne \omega(\trunc{\bv}{j+1})$.  The result follows.
\end{proof}

To prepare for Lemma~\ref{lemma:Rconv}, we need a preliminary lemma that
does some index shuffling for us.  It gives an  explicit representatives of the
cosets of $\op{Sym}(k+1)$ in $\op{Sym}(k+2)$.


\begin{definition}\guid{TSIVSKG}
\formaldef{$\bu^i$}{DROP}
Let $\bu$ be any list.
For each $i$,  let
$\bu^i = [\u_0;\ldots;\hat\u_i;\ldots]$, the list which drops the $i$th entry.
\end{definition}

\begin{lemma}[coset representatives]\guid{IVFICRK}\rz{300}\label{lemma:coset-bijection} 
There is a bijection between the set 
\begin{displaymath} 
\{(i,\sigma)\mid 0\le i\le k+1,\quad \sigma\in \op{Sym}(k+1)\}
\end{displaymath}
and $\op{Sym}(k+2)$ such that for any list $\bu$ of length $k+2$
\[
(\pi_*\bu)_j = \begin{cases} (\sigma_*(\bu^i))_j&0\le j \le k\\
  \bu_i & j=k+1.
\end{cases}
\]
\end{lemma}

\begin{proof} 
The bijection sends $(i,\sigma)$ to the permutation $\pi$, where
\begin{displaymath} 
\pi^{-1} j = \begin{cases} 
\sigma^{-1} j, & \sigma^{-1} j<i\\
(\sigma^{-1}j)+1 & \sigma^{-1} j \ge i\\
i& j=k+1.
\end{cases}
\end{displaymath}
This has the required properties.
\end{proof}

This lemma shows that each (nondegenerate) Delaunay simplex can be
partitioned as a union of Rogers simplices, indexed by the permutation
group.

\begin{lemma}[Delaunay simplex]\guid{WQPRRDY}\rz{450}\label{lemma:Rconv}  
  Let $V$ be a saturated packing and let $\bu = [\u_0;\ldots;\u_k]\in
  \bV(k)$.  Assume that $h(\bu)<\sqrt2$.  
  Then
\begin{displaymath} 
\op{conv}\{ \u_0,\ldots, \u_k\} = \bigcup \,\{ R(\pi_*\bu) \mid \pi\in \op{Sym}(k+1)\}.
\end{displaymath}
\end{lemma}
\indy{Notation}{Sym@$\op{Sym}$ (symmetric group)}%

\begin{proof} The proof is by induction on $k$.  The base case of the induction $k=0$
reduces to the trivial assertion: $\op{conv}\{\u_0\} = \op{conv}\{\u_0\}$.  
%Assume
%the result holds for $k$.


%For each $i$,  let
%$\bu^i = [\u_0;\ldots;\hat\u_i;\ldots;\u_{k+1}]$, which drops the $i$th entry.

\claim{We claim $\bu^i\in \bV(k)$, when $\bu=[\u_0;\ldots;\u_{k+1}]\in \bV(k+1)$.}  Indeed,  some permutation
$\pi\in \op{Sym}(k+2)$ carries $\bu$ to
$\bv=[\u_0;\ldots;\hat\u_i;\ldots;\u_{k+1};\u_i]$.  By
Lemma~\ref{lemma:perm-Vk}, $\bv\in \bV(k+1)$, so that $\bu^i =
\trunc{\bv}{k}\in \bV(k)$.

By the induction hypothesis 
\begin{equation}\label{eqn:sigma} 
\op{conv}(S\setminus\{\u_i\}) = \bigcup \,\{ R(\sigma_*\bu^i) \mid \sigma\in \op{Sym}(k+1)\},
\end{equation}
where $S = \{\u_0,\ldots,\u_{k+1}\}$.
By Lemma~\ref{lemma:simplex-poly}, the facets of the polyhedron
$\op{conv}(S)$ are the sets $\op{conv}(S\setminus\{\u_i\})$.
Lemma~\ref{lemma:facet-partition} gives the partition
\begin{equation} \label{eqn:convS}
\op{conv}(S) = \bigcup_{i=0}^{k+1} \op{conv}(\{\omega(\bu)\}\cup \op{conv}(S\setminus\{\u_i\})).
\end{equation}
Substitute the formula \eqn{eqn:sigma} into \eqn{eqn:convS}, then use the bijection of
Lemma~\ref{lemma:coset-bijection} to replace the double union by a single union
over $\pi\in \op{Sym}(k+2)$.  Background facts in affine
geometry then simplify the expression to the desired formula.  The
induction follows.
%
%
%Let $L = \{ \u_0,\ldots, \u_k\}$.  The proof is by
%induction on $k$.  When $k=0$, the result is trivial.  Now assume
%$k>0$.
%
%The circumcenter $\omega(\bu)$ of $L$ lies in the convex hull of
%these points.  (See the proof of Lemma~\ref{lemma:v2}.)  Thus, the
%left-hand side is the union of cones:
%\begin{displaymath} 
%\op{conv}(W) = \bigcup\,\{ \op{conv}(\omega(\bu),L\setminus \{ \u_i\}\mid i=0,\ldots,k) \}.
%\end{displaymath}
%The sets $L\setminus \{ \u_i\}$ can be identified with cosets of
%$\op{Sym}(k+1) /\op{Sym}(k)$.  By induction $L\setminus \{ \u_i\}$ is
%the union of $R(\bv)$ as $\bv$ runs over all permutations
%$\op{Sym}(k)$ of $L\setminus \{ \u_i\}$.  The result follows by
%induction.
\end{proof}



In summary of this section, by construction, the Rogers simplices
$R(\bu)$ are compatible with the Voronoi decomposition of space.
Under mild restrictions on the circumradius, they can also be
reassembled into simplices (the Delaunay simplices) with extreme
points at the centers of the packing, by Lemma~\ref{lemma:Rconv}.
\indy{Index}{simplex!Delaunay}%
\indy{Index}{partition!Rogers}%
\indy{Index}{decomposition!Voronoi} %




\section{Marchal Cells}



\cite{marchal:2009} has proposed an approach to sphere packings
 that gives significant improvements to the original proof
in~\cite{Hales:2006:DCG}.  The definition of $k$-cells,
conjecture~\ref{conj:m1}, Theorem~\ref{theorem:mk1}, and the method of
Lemma~\ref{lemma:13-14} are all due to him.  
\indy{Index}{Marchal, C.}%

% His articles claim to give a {\it demonstration} of the Kepler
% conjecture \cite{marchal:2009}.  However, the
% mathematically rigorous part of the article only gives a reduction
% of the problem to a difficult optimization problem in a finite
% number of variables.  The method of gradient descent is then used to
% explore the local minima of the optimization problem in finitely
% many variables.

Marchal's partition of space is a variant of Rogers's partition into
the simplices $R(\bu)$.  The main part of construction is the
decomposition obtained by truncating the Voronoi cells by a ball of
radius $\sqrt2$.  In a few carefully chosen situations, he assembles
the simplices $R(\bu)$ into larger convex cells (Delaunay cells), as
suggested by Lemma~\ref{lemma:Rconv}.




\begin{definition}[Marchal cells]\guid{QEEHXUB} 
  \formaldef{$i$-cell}{mcell} \formaldef{$\xi$}{mxi}%
  \indy{Index}{cell}%
  \indy{Index}{Marchal cell}%
  Let $V$ be a saturated packing.  Let $\bu=[\u_0;\ldots;\u_3]\in
  \bV(3)$.  Define $\xi(\bu)$ as follows.  If $\sqrt2\le
  h(\trunc{\bu}{2})$, then let $\xi(\bu)=\omega(\trunc{\bu}{2})$.  If
  $h(\trunc{\bu}{2})<\sqrt2\le h(\bu)$, define $\xi(\u)$
  to be the unique point in
\[
\op{conv}\{\omega(d_2\bu),\omega(\bu)\}
\]
at distance $\sqrt2$ from $\u_0$.  
%\begin{displaymath} 
%h(\trunc{\bu}{2}) <\sqrt2 \le h(\bu).
%\end{displaymath}
%When this inequality holds, there is a a unique point $\xi(\bu)$ in
%$\op{conv}\{\omega(\trunc{\bu}{2}),\omega(\bu)\}$ at distance exactly
%$\sqrt2$ from $ \u_0$.  
%\end{definition}
A set $\cell(\bu,i)\subset\ring{R}^3$ is associated with $\bu$ and $i=0,1,2,3,4$.  
\hfill\break\smallskip 
\case{The $0$-cell} of $\bu$
is
\begin{displaymath} 
\cell(\bu,0) = R(\bu)\setminus B(\u_0,\sqrt2).%
%\{\p\in R(\bu) \mid \norm{\p}{\u_0} \ge \sqrt2\}.
\end{displaymath}
\bigskip
\case{The $1$-cell} of $\bu$ is 
\begin{displaymath} 
\cell(\bu,1) = (R(\bu) \cap  \bar B(\u_0,\sqrt2))\setminus \op{rcone}^0(\u_0,\u_1,a),
\quad a={h(\trunc{\bu}{1})}/{\sqrt2}.
\end{displaymath}
%\op{conv}(\{\omega([ \u_0])\}\cup R_1),\hbox{ where } 
%R_1 = \{\p \in R(\bu) \mid \norm{\omega([ \u_0])}{\p}= \sqrt2\}.
%\end{displaymath}
%\indy{Notation}{R@$R_1$ (Rogers simplex)}%
\bigskip
\case{The $2$-cell} of $\bu$ is
empty unless $h(\trunc{\bu}{1})<\sqrt2\le h(\bu)$.  If nonempty, the $2$-cell is
(with $a$ as above)
\begin{align*} 
\cell(\bu,2) &= 
 \op{rcone}(\u_0,\u_1,a)\cap \op{rcone}(\u_1,\u_0,a)\cap 
\op{aff}_+(\{\u_0,\u_1\},\{\xi(\u),\omega(\u)\}).
%\op{conv}( \{\u_0, \u_1\}\cup R_2),\quad\text{where }  \vspace{6pt}\\
%R_2 &= \{\p \in R(\bu)\cap \Omega(V,\trunc{\bu}{1}) \mid 
%\norm{ \u_0}{\p}=\norm{ \u_1}{\p} =\sqrt2\}.
\end{align*}
\bigskip
\case{The $3$-cell} of $\bu$ is defined to be empty unless 
$h(\trunc{\bu}{2}) <\sqrt2 \le h(\bu)$.
If nonempty, $\xi(\bu)\in \op{conv}\{\omega(\trunc{\bu}{2}),\omega(\bu)\}$
%\formaldef{$\xi$}{mxi}%
and  the $3$ cell is
\begin{displaymath} 
\cell(\bu,3) = \op{conv}\{ \u_0, \u_1, \u_2,\xi(\bu)\}.
\end{displaymath}
\bigskip
\case{The $4$-cell} of $\bu$ is defined to be empty unless
$h(\bu) <\sqrt2$.
If nonempty, the $4$ cell is
\begin{displaymath} 
\cell(\bu,4) = \op{conv}\{ \u_0, \u_1, \u_2, \u_3\}.
\end{displaymath}
\end{definition}
\indy{Notation}{zzxi@$\xi(\cdot)$ (Marchal cell parameter)}%

The $0$ and $1$-cells are  subsets of a Rogers  simplex
$R$.  Yet, the $2$, $3$, and $4$-cells lie in a union of
simplices.  The index $i$ in  $\cell(\bu,i)$ indicates the number
of points of $V$ that are extreme points of the cell. 

This definition seems unmotivated at first.  Some history might help.
The 1998 proof of the Kepler conjecture partitioned space into a
hybrid of Voronoi cells and Delaunay simplices.  The Delaunay
simplices are fine tuned instruments for detail.  The Voronoi cells
are coarsely tuned, suitable for rough hewing.  Delaunay simplices
articulate the foreground; Voronoi cells recede into background.  The
solution to the problem lies in the right balance between foreground
and background.  Too many Delaunay simplices and the details
overwhelm.  Too many Voronoi cells and the estimates become too weak.
The central geometrical insights of the original proof are expessed as
rules that delineate foreground against background, Delaunay against
Voronoi.

Marchal takes the  hybrid decomposition and improves its
execution.  Marchal's $4$-cell is a Delaunay simplex.  His $0$- and
$1$-cells are parts of a Voronoi cell.  His $2$ and $3$-cells are
gradations between the two.

Examples show the shortcomings of a non-hybrid approach.
Recall that the density of the face centered cubic packing is
$\pi/\sqrt{18}=0.74048$\ldots.  Numerical evidence shows that an
approach based entirely on Delaunay simplices should give a bound of
about $0.740873$, a failure that comes tantalizingly
close~\cite{Hales:1992:JCAM}.  The dodecahedral theorem, asserting that the
Voronoi cell of smallest volume in the regular dodecahedron, gives the
bound of about $0.755$ ~\cite{Hales:2010:Dodec}.  Thus, the pure Voronoi cell
strategy fails as well.  The pure approaches can be modified in ways
that are conjectured to produce sharp bounds. The modified problems
are complex and daunting.
% constant 0.740873 checked 4/2010.
% constant 0.755 cross-checked with dodec.

Starting with Fejes T\'oth, it has become a common practice to
truncate Voronoi cells by intersecting them with a ball concentric
with the cell.  Different authors use different radii for the
truncating sphere: $7/\sqrt{27}=1.347\ldots$ (Fejes T\'oth), $\sqrt2$,
$1.385$ and $1.255$ (Ferguson and H.), $\sqrt2$ (Marchal),
$\sqrt{3}\tan{\pi/5}=1.258\ldots$ (McLaughlin).  A larger radius
retains more information (and complexity) than a smaller radius.
Marchal's $0$-cells are the refuse, lying outside the ball of radius
$\sqrt2$.  The $0$-cells are inconsequential to the proof.


\bigskip

\begin{definition}[$i$-rearrangement]\guid{BGXEVQU} 
\formaldef{$i$-rearrangement}{\_}
Let $\bu=[\u_0;\ldots;\u_k],\bv=[\v_0;\ldots;\v_k]$ be two lists of the same length.  
One is an
 $i$-\newterm{rearrangement} of the other if
$\pi_*\bu = \bv$ for some $\pi\in\op{Sym}(k+1)$ such
that 
$\pi(j) = j$  when $j \ge i$.
\end{definition}

In particular, if $\bu,\bv$ are $0$- or $1$-rearrangements of one another,
then $\bu = \bv$.


\begin{lemma}[]\guid{EMNWUUS}\rz{100}\label{lemma:M-complement4} 
Let $V$ be a saturated packing.  Let $\bu\in \bV(3)$.
The following are equivalent.
\begin{itemize} 
\item  $\cell(\bu,i)=\emptyset$ for $i=0,1,2,3$.
\item  $\cell(\bu,4)\ne\emptyset$.
\item  $h(\bu)<\sqrt2$.
\end{itemize}
\end{lemma}

\begin{proof} 
The diameter of $R(\bu)$ is easily seen to be $h(\bu)$.  Hence if $h(\bu)<\sqrt2$
all of the defining conditions are empty for $\cell(\bu,i)$ for $i<4$.  The result follows.
\end{proof}

\begin{lemma}[]\guid{SLTSTLO}\rz{200}\label{lemma:M-exhaust} 
Let $V$ be a saturated packing and let $\bu\in \bV(3)$. Then
%there exists a null set $Z$ such that 
every point in $R(\bu)$ 
belongs to $\cell(\bu,i)$  for some $0\le i\le 4$.  Furthermore, there is a null set  
$Z$ such that each point in $R(\bu)\setminus Z$ belongs to a unique
$\cell(\bu,i)$.
\end{lemma}

\begin{proof} 
Explicitly, the null set is the union of $R(\bu)\setminus R^0(\bu)$ (which lies in 
a finite union of planes), the sphere of radius $\sqrt2$ at $\u_0$, 
the difference $\op{rcone}(\u_0,\u_1,a)\setminus \op{rcone}^0(\u_0,\u_1,a)$,
and the
plane $\op{aff}\{\u_0,\u_1,\xi(\bu)\}$ when needed.  Let $p\in R(\bu)$.

\noindent
\claim{$[h(\bu)<\sqrt2]$.} In this case,  $\p\in\cell(\bu,4)$.

\noindent
\claim{$[\norm{\p}{\u_0} \ge\sqrt2]$.} In this case, $\p\in\cell(\bu,0)$.

\noindent
\claim{$[\p\not\in\op{rcone}^0(\u_0,\u_1,\omega(\trunc{\bu}{1})/\sqrt2)]$.} 
In this case, $\p\in \cell(\bu,1)$.

\noindent
\claim{$[\p\in \op{aff}_+(\{\u_0,\u_1\},\{\u_2,\xi(\bu)\})]$.} In this
case, $\p\in \cell(\bu,2)$.

\noindent
\claim{$[\p\in \op{aff}_+(\{\u_0,\u_1\},\{\xi(\bu),\u_3\})]$.} In this
case, $\p\in \cell(\bu,3)$.

When the corresponding strict inequalities are used, we obtain uniqueness for
$R(\bu)\setminus Z$.
\end{proof}




\begin{lemma}[]\guid{RVFXZBU}\rz{600}\label{lemma:marchal-equal} 
Let $V$ be a saturated packing, 
let $\bu,\bv\in \bV(3)$, and let $i,j\in \{0,1,2,3,4\}$.
%Assume that the affine dimension of $\cell(\bu,i)$ is three.
If the intersection, of
$\cell(\bu,i)$ with $\cell(\bv,j)$ has positive measure,
then $i=j$ and $\bu$ is an $i$-rearrangement of $\bv$.
Conversely, if $i=j$ and $\bu$ is an $i$-rearrangement of $\bv$, 
then $\cell(\bu,i)=\cell(\bv,j)$.
\end{lemma}

\begin{proof} 
The converse statement follows directly from the definition of Marchal cells.

Let $\cell(\bu,i)$ and $\cell(\bv,j)$ be two cells whose intersection
$X$ has positive measure.  There exists $\bw\in \bV(3)$ such that
$R(\bw)\cap X$ has positive measure.  In particular, $R(\bw)$ has
affine dimension three.  There are finitely may $R(\bw')$ for $\bw'\in
\bV(3)$ that meet $R(\bw)$.  By Lemma~\ref{lemma:R-inter}, by avoiding
finitely many planes (null sets), there exists subset $X'$ of positive measure
in
$X\cap R(\bw)$ such that $R(\bw)$ is the unique Rogers simplex that
meets $X'$.  Furthermore, by Lemma~\ref{lemma:M-exhaust}, there is a subset $X''$ of
positive measure in $X'$ and a natural number $k$ such that $\cell(\bw,k)$ is
the unique cell meeting $X''$.

Now $\cell(\bu,i)$, which contains $X''$, 
is contained in the union of the sets $R(\bu')$ as $\bu'$ runs over the
$i$-rearrangements of $\bu$.  Hence $\bw$ is an $i$-rearrangement of
$\bu$.  By the converse statement, replacing $\bu$ with an
$i$-rearrangement, we may assume that $\bu=\bw$.  Similarly, we may
assume that $\bv=\bw$.  By the exclusion of null sets as above, $i=j=k$.
\end{proof}



\begin{definition}\guid{LEPJBDJ}
\formaldef{$V(X)$}{VX}
  Let $V$ be a saturated packing and let $\bu=[\u_0;\ldots]\in
  \bV(3)$.  Let $X$ be a $k$-cell.  Define $V(X) =
  \{\u_0,\ldots,\u_{k-1}\}$.  In particular, if $X$ is a $0$-cell
  $V(X)=\emptyset$.
\end{definition}


\begin{lemma}\guid{URRPHBZ}\rz{220}\label{lemma:cell-radial}
  Let $V$ be a saturated packing and let $\bu\in \bV(3)$.  Then
  $X=\op{cell}(\bu,k)$ is measurable and eventually radial at each
  $\v\in V$.  Furthermore the cell $X$ is bounded away from every
  $v\in V\setminus V(X)$, so that the solid angle of $X$ is zero, except
  at $\v\in V(X)$.
\end{lemma}

\begin{proof} This follows from the fact that $R(\bu)$ is a simplex, and
$R(\bu)\cap V = \{\u_0\}$.
\end{proof}


%\begin{definition}\guid{ENBYLCBxo}
%Let $V$ be a saturated packing and let $X$ be a $k$-cell.
%Write $\op{solc}(\bu,k)$ for $\sol(\op{cell}(\bu,k),\u_0)$, where
%$\bu = [\u_0;\ldots]\in \bV(3)$.
%\indy{Notation}{solc@$\op{solc}$ (solid angle of a cell)}
%\end{definition}

\begin{lemma}\guid{QZYZMJC}\rz{200}
Let $V$ be a saturated packing.  For every $\v\in V$, 
\[
\sum_{X\mid \v\in V(X)}  \op{sol}(X,\v) = 4\pi.
\]
where the sum runs over all cells $X$ such that $\v\in V(X)$.
\end{lemma}

\begin{proof}  Indeed, the cells partition $\ring{R}^3$ and $\sol(B(\v,\epsilon)) = 4\pi$.
\end{proof}

\begin{definition}[$\op{tsol}$]\guid{LZYLTFY} 
\formaldef{$\op{tsol}$}{total\_solid}
  %Let $V\cap X$ be the intersection of $ V$ with the set of extreme
  %points of the $k$-cell $X$.  Explicitly, $V\cap X=\emptyset$ if $k=0$;
  %and $V\cap X = \{ \u_0,\ldots, \u_{k-1}\}$ in general.  Each $k$-cell
  %is measurable and eventually radial at each $\u\in V\cap X$.  
  Define
  the \newterm{total solid angle} of a cell $X$ to be
\begin{displaymath} 
\op{tsol}(X) = \sum_{\v \in V(X) } \op{sol}(X,\v).
\end{displaymath}
\end{definition}
\indy{Index}{angle!total solid}%
\indy{Index}{extreme point}%
%\indy{Notation}{VX@$V(X)$ (extreme points of a cell $X$)}
\indy{Notation}{tsol@$\op{tsol}$}%

%By Lemma~\ref{lemma:cell-radial}, the only terms that contribute to the sum are $\u\in V(X)$.

\begin{definition}[edge]\guid{WYORUNK}
\formaldef{$E(X)$}{edgeX}
  \indy{Notation}{EX@$E(X)$}%
  Let $E(X)$ be the set of \newterm{extremal edges} of the $k$-cell
  $X$ in a saturated packing $V$.  More precisely, let
\begin{displaymath}E(X)=\{\{ \u_i, \u_j\}\mid \u_i\ne \u_j\in
V( X)\}.\end{displaymath}
\indy{Notation}{1@$\tbinom{n}{k}$ (binomial coefficient)}%
\end{definition}

In particular, $E(X)$ is empty for $0$ and $1$-cells, and contains
$\tbinom{k}{2}$ pairs when $k\ge 2$.

\begin{definition}[$\op{dih}$]\guid{RSDYMHV} 
\formaldef{$\dih$}{dihX}
Let $V$ be a saturated packing, 
$\bu=[\u_0;\u_1;\u_2;\u_3]\in \bV(3)$, $X=\cell(\bu,k)$,
and $\ee=\{\u_0,\u_1\}\in E(X)$.
Let $\op{dih}(X,\ee)$ be the dihedral angle
along edge $\{\u_0,\u_1\}\in
E(X)$ in the $k$-cell $X = \op{cell}(\bu,k)$.
%be a $k$-cell.
%is defined to be the dihedral angle along that edge of the
%boundary.  
Explicitly, if $X$ is a null set, set $\op{dih}(X,\ee)=0$.  Otherwise,
set $\op{dih}(X,\ee)=\dih(\{\u_0,\u_1\},\{\v,\w\})$,
where 
\[
\{\v,\w\} = 
\begin{cases}
  \{\xi(\u),\omega(\bu) \} &  k=2\\
  \{\u_2,\xi(\u)\} & k=3\\
  \{\u_2,\u_3\} &k=4.
\end{cases}
\]
This is independent of the representation of the cell $X$ in the
form $\cell(\bu,k)$.
\index{Notation}{dih@$\dih$}%
\index{Index}{angle!dihedral}%
\index{Notation}{h@$h$ (half-edge length)} 
\end{definition}

Each
edge $\ee=\{ \u_i, \u_j\}\in E(X)$ determines half-length
$h(\ee) = \norm{ \u_i}{ \u_j}/2$.
This definition of $h$ is compatible with the previous definition for lists in the sense that
$h([\u_0;\u_1]) = h(\{\u_0,\u_1\})$.

\begin{lemma}\guid{GRUTOTI}\rz{500}
Let $V$ be a saturated packing.  Assume that $\v_0,\v_1\in V$ satisfy $\norm{\v_0}{\v_1}<\sqrt2$.  Set $\ee=\{\v_0,\v_1\}$.
Then
\[
\sum_{X\mid \ee\in E(X) } \op{dih}(X,\ee) = 2\pi.
\]
The sum runs over cells $X$ such that $\ee\in E(X)$.
\end{lemma}

\begin{proof} 
The sum of the azimuth angles of the cyclic set 
$S=\{\omega(\bu) \mid \bu =[\v_0;\v_1;\ldots]\in \bV(3)$ is $2\pi$
by Lemma~\ref{lemma:2pi-sum}.   Each azimuth angle is less than $\pi$ and is thus
equal to a dihedral angle.  The sum does not change, if we add additional points
$\xi(\bu)$ and $\omega(\trunc{\bu}{2})$ to $S$.  The result follows.
\end{proof}
%
%The dihedral angle of each $4$-cell $\op{cell}([\u_0;\u_1;\u_2;\u_3],4)$ is the sum of the two corresponding Rogers simplices
%$R([\u_0;\u_1;\u_3;\u_4])$ and $R([\u_0;\u_1;\u_4;\u_3])$.  The dihedral
%angles of $2$- and $3$-cells combine at the point $\xi(\bu)$ to give
%the dihedral angle of the Rogers simplex.  Thus, it is enough to
%show that 
%\[
%\sum_{\bv =[\v_0;\v_1;\ldots]\in \bV(3)} \op{dih}(\{\v_0,\v_1\},\{\omega(\trunc{\bv}{2}),\omega(\bv)\}) = 2\pi.
%\]
%The points $\omega(\trunc{\bv}{2})$ are by construction points on
%the edge of a Voronoi cell between two extreme points
%$\omega(\bv)$ and $\omega(\bv')$.  Thus, the two dihedral terms
%involving $\omega(\trunc{\bv}{2})$ can be combined into a single term
%$\dih(\{\v_0,\v_1\},\{\omega(\bv),\omega(\bv')\}$. The set of points
%\[
%\{\omega(\bv) \mid \bv\in \bV(3), \bv = [\v_0;\v_1;\ldots]\}
%\]
%is a cyclic set with respect to
%  $(\v_0,\v_1)$.   By Lemma~\ref{lemma:2pi-sum} the
%  sum of the azimuth (dihedral) angles is $2\pi$.



%
%
%
%
%First, the proof establishes that cells are either
%disjoint (up to a null set) or equal.  At the same time, it
%determines exactly when two $k$-cells are equal to one another.
%
%By Lemma~\ref{lemma:Rconv}, the $4$-cell $\cell(\bu,4)$ is a union
%of the simplices $R(\bv)$, as $\bv$ runs over rearrangements of $\bu$.
%Two $4$-cells that meet in a set of positive measure are equal.
%Every $4$-cell is a union of simplices $R(\bu)$ with
%$h(\bu)<\sqrt2$.  This condition gives $\cell(\bu,i)=\emptyset$ for
%$i=0,1,2,3$.
%
%Similarly, the $3$-cell is a union of the convex hulls
%$\op{conv}(\{\xi(\bv)\}\cup R(\trunc{\bv}{2}))$ as $\trunc{\bv}{2}$ runs over
%rearrangements of $\trunc{\bu}{2}$.  Note that the point
%$\xi(\bv)=\xi(\bu)$ is independent of the rearrangement, since
%it is determined as the point at distance $\sqrt2$ from $ \u_0$
%along the line through from $\omega(\trunc{\bv}{2})=\omega(\trunc{\bu}{2})$
%perpendicular to the plane $\op{aff}\{ \u_0, \u_1, \u_2\}$ (in the
%half-space of $\trunc{\bu}{3}$). Two $3$-cells that meet in a set of
%positive measure are equal.  Their parameters are equal:
%$\trunc{\bu}{2}=\trunc{\bv}{2}$.  The intersection $\cell(\bu,3)$ cannot meet
%$\cell(\bu,i)$ for $i<3$, in a set of affine dimension three, because the
%plane $\op{aff}\{ \u_0, \u_1,\p\}$ separates them.
%
%The $2$-cell with parameter $\bu$ is contained in the right-circular cone
%\begin{displaymath} 
%\op{rcone}( \u_0, \u_1,\cos(\op{arc}(\norm{ \u_0}{ \u_1},\sqrt2,\sqrt2)))
%\end{displaymath}
%This cone separates the $2$-cell from the $0$- and $1$-cells with the
%same parameter $\bu$.  The $1$-cell is separated from $0$-cells by the
%sphere of radius $\sqrt2$, centered at $ \u_0$.


%\begin{lemma}[]\guid{PDVQTUO}\rz{0} 
%Let $V$ be a saturated packing.  If two Marchal cells of $V$ are not equal, then
%their intersection lies in a null measure set.
%\end{lemma}






\subsection{cells revisited}

The Marchal cells can be described in an alternative intuitive way.
If $S\subset\ring{R}^3$, let
\[
\op{equi}(S,r) = \{ \p \mid \norm{\p}{\v} = r \text{ for all } \v \in S \}.
\]


Let $V$ be a saturated packing.
Define
\[
X(S) = \op{conv} (S\cup\op{equi}(S, \sqrt2) ),
\]
for $S\subset V$.  The set $X(S)$ is empty if the circumradius of $S$
is greater than $\sqrt2$.

The set $X(\emptyset)$ is $\ring{R}^3$.  The set $X(\{\w\})$ is a ball
of radius $\sqrt2$ with center $\w$.  The set $X(\{\v,\w\})$ is a
double cone, $X(\{\u,\v,\w\})$ a bipyramid, and $X(\{{\mathbf
  t},\u,\v,\w\})$ is a simplex.

\begin{lemma}
Let $V$ be a saturated packing.
If $S\subset V$ is not empty, then $X(S)$ is contained in the union of sets
\[
X(S\setminus \{\v\}),  \quad \v\in S.
\]
\end{lemma}

\begin{lemma}
Let $V$ be a saturated packing and let $S\subset V$. 
 If $\op{card}(S)=k$, then
\[Y(S) = X(S) \setminus \bigcup_{S\subset S'\subset V} X(S'),\]
where $S'\subset V$ runs over subsets of cardinality $k+1$ that contain $S$,
equals a finite union of Marchal $k$-cells, up to a null set.
\end{lemma}

It is possible to base a construction of Marchal cells on this lemma,
dispensing entirely with Voronoi cells and Rogers simplices.  This is
essentially what Marchal does.  It is quick and intuitive.  We have
followed a longer path that gives more detail about the structure of
cells.



\subsection{Marchal's conjecture}

This section shows how the existence of a FCC-compatible negligible
function is a consequence of an explicit
inequality related to the distances $h(\bu)$, where $\bu\in \bV(1)$.

%\begin{note}%
%Replace $\sol_0/\pi =\Delta_1$ to simplify formulas.
%\end{note}


\begin{definition}[$\sol_0$,~$\tau_0$,~$m_1$,~$m_2$,~$h_+$,~$M$]\guid{AOZUTMU} 
\formaldef{$\sol_0$}{sol0}%
\formaldef{$\tau_0$}{tau0}%
\formaldef{$m_1$}{mm1}%
\formaldef{$m_2$}{mm2}%
\formaldef{$h_+$}{hplus}%
\formaldef{$M$}{marchal}%
Define the following constants and functions: 
\begin{align}\label{eqn:m-def} 
\sol_0 &= 3\arccos(1/3)-\pi\\
%\Delta_1 &= (3\arccos(1/3)-\pi)/\pi\\
\tau_0 &= 4\pi  - 20\sol_0\\
m_1 &= \sol_0 2\sqrt2/\tau_0 = 1.012\ldots \\ %% K 
m_2  &=  (6\sol_0- \pi)\sqrt2/(6 \tau_0) = 0.0254\ldots\\ %% M 
h_+ &= 1.3254 \hbox{~(exact rational value)}
\end{align}
Let $M:\ring{R}\to\ring{R}$ 
be the following piecewise polynomial function (Figure~\ref{fig:M}):
\begin{equation}\label{eqn:M} 
M(h) =
\begin{cases} 
% (\sqrt2-h) (h-1.3254) (9h^2 - 17 h + 3)/(1.627 (\sqrt2-1))& h\le\sqrt2\\
\dfrac{\sqrt2-h}{\sqrt2-1}~ \dfrac{h_+-h}{h_+-1} ~\dfrac{17 h - 9 h^2 - 3}{5} 
& h \le \sqrt2.\vspace{3pt} \\
0 & h >\sqrt2.
\end{cases}
\\
\end{equation}
\end{definition}

\begin{figure}[htb]
\centering
\szincludegraphics[width=60mm]{\pdfp/Mfun.eps}
% Plot[Mfun[h], {h, 1, Sqrt[2]}]
% copied to Preview, then saved, then converted to eps via pdf2eps.
\caption{The quartic polynomial $M$.}
\label{fig:M}
\end{figure}

The constant $\sol_0$
is the area of a spherical triangle with sides $\pi/3$.
Simple calculations based on the definitions give
\begin{equation}\label{eqn:km}m_1 - 12m_2 = \sqrt{1/2}\end{equation} 
and
\begin{equation}M(1) = 1,\quad M(h_+)=0,\quad M(\sqrt2) =0.\end{equation} 

\begin{definition}[$\gamma$]\guid{KGFDCFM} 
\formaldef{$\gamma$}{gammaX}
For any cell $X$ of a saturated packing, 
define the functional $\gamma(X,\cdot)$ on  $\{f:\ring{R}\to\ring{R}\}$ by
\begin{equation}\label{eqn:gamma-def} 
\gamma(X,f) =  \op{vol}(X)
-\left(\frac{2m_1}{\pi}\right) \op{tsol}(X) + \left(\frac{8m_2}{\pi}\right)
\sum_{\ee\in E(X)} \dih(X,\ee)  f(h(\ee))
\indy{Notation}{ZZddgamma@$\gamma$ (fundamental estimate)}%
\end{equation}
\end{definition}


\begin{theorem}[Marchal's inequality]\guid{HJKDESR}\rz{0}\label{lemma:MI} 
Let $V$ be any saturated packing, and let $X$ be any Marchal cell of $V$.  Then
\begin{equation}\label{eqn:mfe} 
\gamma(X,M)\ge 0,
\end{equation}
where $M$ is the function defined in (\ref{eqn:M}).
\end{theorem}

\begin{proof}  See~\cite{marchal:2009}.  We return to this inequality in the appendix,
to provide our own (computer) proof.
\end{proof}
%% cc:mar are the k-cell estimates for non-cell clusters.
%By Calculation~\ref{calc:marchal}, Marchal's fundamental estimate
%holds for any cell


\begin{conjecture}[Marchal]\guid{PHNFUXP}\rz{0}\label{conj:m1} 
For any packing $ V$, and
any $ \u\in V$,
\begin{displaymath} 
\sum_{\v\in V\setminus\{\u\}} M(h([\u;\v])) \le 12.
\end{displaymath}
\end{conjecture}

Marchal's conjecture is still open.  This book proves a variant of
Marchal's conjecture.

\begin{theorem}\guid{KIZHLTL}\rz{750}\label{theorem:mk1} 
% formalization must follow the statements in pack_concl.hl, 250 each estimate.
Marchal's conjecture~\ref{conj:m1} implies
that for every saturated packing $V$, there exists a negligible FCC-compatible function
$G:V\to \ring{R}$.
\end{theorem}


\begin{proof} 
It is enough to show that $G( \u_0,f) = -\op{vol}(\Omega(V, \u_0)) + 8
m_1 - \sum 8 m_2 f(h([\u_0;\u]))$ is FCC-compatible and negligible, when $f=M$.
The sum runs over $\u\in V\setminus\{\u_0\}$.
%The result then follows from Lemma~\ref{lemma:deltabound}.  
The function $G(\cdot,M)$ is FCC-compatible directly
by equation~\eqn{eqn:km}
and conjecture~\ref{conj:m1}:
\indy{Index}{negligible}%
\indy{Index}{FCC-compatible}%
\begin{align*} 
\sqrt{32} &= 8 m_1 - 8\cdot 12 m_2\\
&\le 8 m_1 - 8 m_2 \sum M(h)\\
&= \op{vol}(\Omega(V, \u_0)) + G( \u_0,M).
\end{align*}
The issue is to prove that it is negligible.  In view of future applications, 
we work with a general bounded function $f$ instead
of the function $M$.   More explicitly, we show that there is a
constant $c$ such that for all $r\ge 1$:% and all $\p\in\ring{R}^3$:
\begin{equation}\label{eqn:neg} 
-\sum G( \u,f) = \sum \op{vol}(\Omega(V, \u)) 
-\sum 8m_1 + \sum \sum 8 m_2 f(h) \ge \sum_{X\subset B(\orz,r)} \gamma(X,f)  + c r^2,
\end{equation}
where the outer nested sum runs over $ \u\in  V(\orz,r)$.  

Lemmas~\ref{lemma:Zr2} and \ref{lemma:V-finite} show that the number
of points of $ V$ near the boundary of $B(\orz,r)$ is at most $c r^2$.


The sum of $\gamma(X,f)$ over all cells in a large ball
$B(\orz,r)$ has the form $T_1 + T_2 + T_3$ for
three terms $T_i = T_i(r)$ of \eqn{eqn:gamma-def}.  The desired
equation~\eqn{eqn:neg} consists of three corresponding terms
$T'_i(r)$.  It is enough to show that there exist constants $c_i$ such that
\begin{displaymath} 
T_i'(r) \ge T_i(r) + c_i r^2.
\end{displaymath}

The sum of the volumes of the Voronoi cells $ \u\in B(\orz,r)$ is not
exactly the volume of $B(\orz,r)$, because of the contribution at the
boundary of $B(\orz,r)$ of Voronoi cells that are only partly contained
in $B(\orz,r)$.  Similarly, the sum of the various $k$-cells for
$X\subset B(\orz,r)$ is not exactly the volume of $B(\orz,r)$, because of
contribution from the boundary. The boundary contributions have order
$r^2$. Thus,
\begin{displaymath} 
T_1'= \sum_{ \u\in  V(\orz,r)} \op{vol}(\Omega(V, \u)) 
\ge \sum_{X\subset B(\orz,r)} \op{vol}(X) + c_1 r^2 = T_1 + c_1 r^2.
\end{displaymath}


The estimates on the other terms are similar.  The solid angles
around each point sum to $4\pi$.
In Landau big O notation, this gives
\begin{align*} 
\sum_{X\subset B(\orz,r)} \op{tsol}(X) &= 
\sum_{X\subset B(\orz,r)} \sum_{ \u\in V( X)} \sol(X, \u)\\
&=\sum_{ \u\in  V(\orz,r)} \sum_{X\mid  \u\in V( X)} \sol(X, \u) + O(r^2)\\
&=\sum_{ \u\in  V(\orz,r)} 4\pi    + O(r^2).
\end{align*}
Hence
\begin{displaymath} 
T_2' = -\sum_{ V(\orz,r)} 8 m_1 = 
-\sum_{X\subset B(\orz,r)}\left(\frac{2m_1}\pi\right) \op{tsol(X)} + O(r^2) = T_2 + O(r^2).
\end{displaymath}
Similarly, the dihedral angles around each edge sum to $2\pi$.  A
factor of $2$ enters the following calculation, because there are two
ordered pairs for each unordered pair $\ee=\{ \u_0, \u_1\}$:
\begin{align*} 
&\phantom{=}\sum_{X\subset B(\orz,r)} \sum_{~~\ee\in E(X)} \dih(X,\ee)  f(h(\ee)) \vspace{3pt}\\
&=\sum_{\ee\subset B(\orz,r)} \sum_{~~X\mid \ee\in E(X)} \dih(X,\ee)  f(h(\ee)) +O(r^2)\vspace{3pt}\\
&=\sum_{\ee\subset B(\orz,r)} 2\pi f(h(\ee)) + O(r^2)\vspace{3pt} \\
&=\sum_{ \u_0\in  V(\orz,r)} \sum_{~~\u_1\in  V(\orz,r) } \pi f(h( \u_0, \u_1)) + O(r^2).\\
\end{align*}
Finally,
\begin{align*} 
T_3' &= \sum\sum 8 m_2 f(h(\bu)) \\
&\ge \left(\frac{8m_2}\pi\right)
\sum_{X\subset B(\orz,r)}\sum_{\ee\in E(X)}\dih(X,\ee) f(h(\ee)) + O(r^2) \\
&= T_3 + O(r^2).
\end{align*}
\end{proof}




\section{Clusters}

Marchal's conjecture~\ref{conj:m1} is still open.  This section
introduces a variant of Marchal's conjecture.  In this variant,
a piecewise linear function $L$ replaces the piecewise polynomial
function $M$.  More crucially, the support of the function $L$ is
contained in $\leftclosed2,2.52\rightclosed$.  By contrast, the
support of Marchal's function is much larger:
$\leftclosed2,2.6508\rightclosed$.  This small difference in the
support of the function creates an enormous difference in the
difficulty of the conjectures.

The conjecture formulated in this section also implies the existence
of FCC-compatible negligible functions.  To prove this existence result, it is
helpful to group Marchal cells together into new aggregates, called
\newterm{clusters}.  This section makes a detailed study of clusters
in order to produce a negligible function.

The sections and chapters that follow give a proof of the
conjecture in this section.  The proof of this conjecture is the
main intermediate result in the proof of the Kepler
conjecture.


% This section shows how to improve on the estimates of the previous
% section by combining various cells into \newterm{cell clusters}.
Recall that $M(h_+) = 0$, where   $h_+ = 1.3254$.
\indy{Index}{cell cluster}%

\begin{definition}[$L$,~$h_0$,~$h_-$]\guid{ULZRABY}\label{def:L} 
\formaldef{$h_0$}{h0}
\formaldef{$h_-$}{hminus}
\formaldef{$L$}{lmfun}
Set
\begin{displaymath} 
h_0 = 1.26\\  %%\hm
\end{displaymath}
Let $L:\ring{R}\to\ring{R}$ be the piecewise linear function 
\begin{displaymath} 
L(h) = \begin{cases} 
\dfrac{h_0-h}{h_0-1}, & h \le h_0 \\
0, & h\ge h_0. \\
\end{cases}
\end{displaymath}
It follows from the definition that
\begin{displaymath} 
L(1) = 1\quad L(\hm) = 0.
\end{displaymath}
Let $h_- = 1.23175\ldots$ be the unique root of the quartic polynomial
$M(h)-L(h)$ that lies in the interval $[1.231,1.232]$.
\indy{Notation}{L@$L$ (linear function)}%
\indy{Notation}{h@$h_- = 1.23175\ldots$}%
\indy{Notation}{h@$h_0 = 1.26$}%
\end{definition}

%%
\begin{figure}[htb]
\centering
\szincludegraphics[width=60mm]{\pdfp/Lfun.eps}
% Plot[{Mfun[h],Lfun[h]}, {h, 1.2, 1.35}]
% copied to Preview, then saved, then converted to eps via pdf2eps.
%% WW very big .eps file!
\caption{Detail of the quartic $M$ and linear function $L$.}
\label{fig:L}
\end{figure}

The inequality $L(h)\ge M(h)$ holds except when $h\in [h_-,h_+]$.  The
aim of this section is to prove a variant (Theorem~\ref{theorem:mk2})
of Theorem~\ref{theorem:mk1}
that uses the function $L$ rather than $M$.  For this, one needs to
combine cells into groups called cell clusters.

\begin{definition}[critical edge,~$\op{EC}$,~$\op{wt}$]\guid{MZSRVBC}\label{def:wt} 
A \newterm{critical edge} $\ee$ of a saturated packing $V$ is an unordered pair
that appears as an element of $E(X)$ for some 
$k$-cell $X$ of the packing $V$, and such that
$h(\ee)\in[h_-,h_+]$.  Let $\op{EC}(X)$ 
be the set of critical edges that belong to $E(X)$.  If $X$ is any cell such
that $\op{EC}(X)$ is nonempty, let the \newterm{weight} $\op{wt}(X)$ of $X$ be
$1/\card(\op{EC}(X))$.
\end{definition}
\indy{Index}{critical!edge}%
\indy{Notation}{EC@$\op{EC}$ (critical edges)}%
\indy{Notation}{wt@$\op{wt}$ (weight)}%

\begin{definition}[$\beta_0$,~$\beta$]\guid{PQFEXQN}\label{def:beta} 
\formaldef{$\beta_0$}{bump}
\formaldef{$\beta$}{beta\_bump}
Set 
\begin{displaymath} 
\beta_0(h) = 0.005 (1 - (h-h_0)^2/(h_+-h_0)^2).
\end{displaymath}
If $X$ is a $4$-cell with exactly two critical edges and if those edges
are opposite, then set
\begin{displaymath} 
\beta(\ee,X) = \beta_0(h(\ee)) - \beta_0(h(\ee')), 
\text{ where }\op{EC}(X) = \{\ee,\ee'\} .  
\end{displaymath}
Otherwise, for all other edges in all other cells, set $\beta(\ee,X) = 0$.
\end{definition}
\indy{Notation}{ZZbeta0@$\beta_0$}%
\indy{Notation}{ZZbeta@$\beta$ (bump)}%

\begin{definition}[cell cluster,~$\Gamma$]\guid{YSULGYR}\label{def:gammaL} 
\formaldef{cell cluster}{cell\_cluster}
\formaldef{$\Gamma$}{cluster\_gammaX}
  Let $V$ be a saturated packing.  Let $\ee\in \op{EC}(X)$ be a
  critical edge of a $k$-cell $X$ of $V$ for some $2\le k\le 4$.  A
  \newterm{cell cluster} is the set
\begin{displaymath} 
\op{CL}(\ee) = \{X\mid \ee\in \op{EC}(X)\} 
\end{displaymath}
\indy{Notation}{cluster}%
of all cells around $\ee$. 
If $Z$ is a subset of a cell cluster $\op{CL}(\ee)$, define
\indy{Notation}{CL@$\op{CL}$ (cell cluster)}%
\begin{displaymath} 
\Gamma(\ee,Z) = \sum_{X\in Z} \gamma(X,L) \op{wt}(X) +\beta(\ee,X).
\end{displaymath}
%and where $\op{wt}(X)$ is the weight of $X$.
\indy{Notation}{ZZddGamma@$\Gamma$}%


\end{definition}
\indy{Index}{cell cluster}%

\begin{theorem}[cell cluster estimate]\guid{OXLZLEZ}\rz{1500} 
\label{lemma:cluster}
Let $\op{CL}(\ee)$ be any cell cluster of a critical edge $\ee$ in a saturated packing $V$.  
Then $\Gamma(\ee,\op{CL}(\ee))\ge 0$.
\end{theorem}

The proof of this cell cluster estimate is a numerical calculation
that has been carried out by computer.  Further discussion about the
methods appears in the appendix. The proof of the following assertion
is extremely long and complex.  It relies on many computer
calculations.  The non-computer parts of the proof take up most of the
remainder of the book.%
%
%

\begin{assertion}\guid{BJERBNU}\rz{0}\label{conj:L12} 
  For any  saturated packing $ V$, and any $ \u_0\in V$,
\begin{equation}\label{eqn:L12} 
\sum_{ \u_1\in V\mid h( \u_0, \u_1)\le \hm} L(h\{\u_0, \u_1\}) \le 12.
\end{equation}
\end{assertion}

\begin{lemma}[]\guid{UPFZBZM}\rz{600}\label{theorem:mk2} 
Assertion~\ref{conj:L12} implies
that for every saturated packing $V$, there exists a negligible FCC-compatible function
$G:V\to \ring{R}$.
\end{lemma}

\begin{remark}\label{rem:L12KC}
In light of Lemma~\ref{lemma:deltabound}, assertion~\ref{conj:L12}
implies the Kepler conjecture. 
\end{remark}

\begin{proof} The proof imitates the proof of
Theorem~\ref{theorem:mk1}.  It is enough to show that $G_L( \u_0) =
-\op{vol}(\Omega(V, \u_0)) + 8 m_1 - \sum 8 m_2 L(h(\bu))$ is
FCC-compatible and negligible.  The function $G_L$ is FCC-compatible
directly from equation~\eqn{eqn:km} and Assertion~\ref{conj:L12}.

The theorem follows from a proof that $G_L$ is negligible.  More
precisely, one needs to show there exists a constant $c$ such that
for all $r\ge 1$:% and all $\orz\in\ring{R}^3$:
\begin{equation}\label{eqn:A2neg} 
\sum \op{vol}(\Omega(V, \u)) -\sum 8m_1 + \sum \sum 8 m_2 L(h) \ge c r^2,
\end{equation}

For cells $X$ that do not belong to any cell cluster,
the proof is just as in the proof of Theorem~\ref{theorem:mk1}.
If $\op{EC}(X)=\emptyset$, then 
$L(h(\ee))\ge M(h(\ee))$ for each edge $\ee\in E(X)$, and
\begin{displaymath}\gamma(X,L)\ge \gamma(X,M)\ge 0\end{displaymath} 
by inequality \eqn{eqn:mfe}.

Note that the function $\beta(\ee,X)$ averages to zero for any $4$-cell $X$:
\begin{displaymath} 
\sum_{\ee\in \op{EC}(X)} \beta(\ee,X) = 0.
\end{displaymath}
Hence the terms involving $\beta$ in sums may be disregarded in this proof.
(These terms may be disregarded here, but they are
needed for the proof of Lemma~\ref{lemma:cluster}.)

Theorem~\ref{lemma:cluster} gives the required inequality for cell
clusters.  Again, using big O notation,
\begin{align*} 
\sum_{X\subset B(\orz,r)} \gamma(X,L) &=
\sum_{X\subset B(\orz,r)\mid \op{EC}(X)\ne\emptyset} \gamma(X,L) +
\sum_{X\subset B(\orz,r)\mid \op{EC}(X)=\emptyset} \gamma(X,L) \vspace{6pt}\\
&\ge \sum_{X\subset B(\orz,r)\mid \op{EC}(X)\ne\emptyset} \gamma(X,L)\vspace{6pt} \\
&=\sum_{X\subset B(\orz,r)}\gamma(X,L)\sum_{\ee \in \op{EC}(X)}\op{wt}(X) + O(r^2)\vspace{6pt}\\
&=\sum_{\ee\subset B(\orz,r)}\sum_{X\mid \ee \in \op{EC}(X)}\gamma(X,L)\op{wt}(X) + O(r^2)\vspace{6pt}\\
&=\sum_{\ee\subset B(\orz,r)}\Gamma(\ee,\op{CL}(\ee)) + O(r^2)\vspace{6pt}\\
&\ge ~~O(r^2).
\end{align*}


From the definition of $\gamma$, the sum $\sum \gamma(X,L)$ may be
expanded as the sum of three terms, $T_1+T_2+T_3$, and compared term
by term with \eqn{eqn:A2neg}:
\begin{displaymath} 
T_i' \ge T_i + c_i r^2.
\end{displaymath}
This proceeds exactly as in the proof of Theorem~\ref{theorem:mk1}.
\end{proof}

\begin{definition}[$\BB$]\guid{WTKURHK} 
\formaldef{$\BB$}{ball\_annulus}
Let $\BB$ be the
\newterm{annulus} $\bar B(\orz,2h_0)\setminus B(\orz,2)$, where
$\bar B(\orz,r)$ is the closed ball of radius $r$.
\indy{Notation}{BB@$\BB$}
\end{definition}


\begin{corollary}\guid{RDWKARC}\rz{200}\label{cor:CE} 
  If the Kepler conjecture is false, there exists a finite packing
  $V\subset\BB$ such that
\begin{equation}\label{eqn:CE} 
\sum_{ \u\in V} L(h\{\orz, \u\}) > 12.
\end{equation}
\end{corollary}

The proof of the Kepler conjecture proceeds by assuming that there is
a countexample to Assertion~\ref{conj:L12} and then deriving a contradiction.
This corollary formulates the potential counterexample in slightly simpler terms.

\begin{proof} If the Kepler conjecture is false,
Assertion~\ref{conj:L12} is violated for some packing $ V$ and some
$ \u_0\in V$.  After the replacement of $ V$ with $ V - \u_0$ and $
\u_0$ with $\orz$, it follows without loss of generality that $
\u_0=\orz\in V$.  After the replacement of $ V$ with the finite
subset
$V\cap \BB$,
it follows without loss of generality that the packing is a finite subset of $\BB$.
\end{proof}



\section{Counting Spheres}

This section proves two estimates about a packing $V\subset \BB$ that
satisfies the inequality~\ref{eqn:CE}.  The first estimate
(Lemma~\ref{lemma:13-14}) shows that the cardinality of $V$ is thirteen,
fourteen, or fifteen.  The second estimate (See Lemma~\ref{lemma:D'}.)  shows
that no point $\v\in V$ can be strongly isolated from the other points
of $V$.    To prove these two estimates, we need a formula for the smallest
possible area of a spherical polygon that contains a disk.  This formula is
developed in the first subsection.

\subsection{solid angle}
\indy{Index}{polygon}%
\indy{Index}{polygon!regular}%


\begin{lemma}[]\guid{GOTCJAH}\rz{450}\label{lemma:ngon} 
Let $P$ be a bounded polyhedron in $\ring{R}^3$ that contains $\orz$
as an interior point.  Let $F$ be a facet of $P$, given by an
equation
\begin{displaymath} 
F = \{\p \mid \p \cdot \v = b\} \cap P.
\end{displaymath} 
Let $W_F$ be the corresponding topological component of $Y(V_P,E_P)$.  
Assume that $W_F$ contains the right-circular cone 
\begin{displaymath} 
\op{rcone}^0(\orz,\v,h)
\end{displaymath}
for some $h>0$.
Then 
\begin{displaymath} 
\sol(W_F) \ge 
2\pi - 2 k \,\arcsin\left(\,h\sin(\pi/k)\,\right),
\end{displaymath}
where $k$ is the number of edges of $F$.
\end{lemma}

%\begin{lemma}[]\guid{ZZ}\rz{0}\label{lemma:ngon:old} 
%  Let $C$ be a circle on the unit sphere with arcradius $a<\pi/2$.
%  Among all spherical $n$-gons that contain $C$ (that is, among all
%  $n$-fold intersections of hemispheres containing $C$), that of
%  minimal area is the regular $n$-gon.
%\end{lemma}
%
%In other words, the minimal configuration consists of the
% intersection of $n$-hemispheres whose bounding great circles are
% tangent to the circle $C$ at $n$-equally spaced bounds around $C$.

\begin{proof} 
\claim{Without loss of generality, we may assume that each edge of
$F$ meets $\op{rcone}(\orz,\v,h)$ in a unique point.}  Indeed,
each edge of $F$ is the intersection of $F$ with another facet
$F_i$.  Write
\begin{displaymath} 
F_i = \{\p \mid \p \cdot \v_i = b_i\} \cap P.
\end{displaymath}
The region $W_F$ consists of points $\p$ for which there exists a
$t>0$ such that $t\p \in\op{ri}(F)$ (by Lemma~\ref{lemma:WF}).  Hence
if we produce a second polyhedron $P'$ and facet $F'$ with
$\op{ri}(F')\subset \op{ri}(F)$, then
\begin{displaymath} 
\sol(W_F)\ge \sol(W_{F'}).
\end{displaymath}
Shift the facet $F_i$ to
\begin{displaymath} 
\op{aff}(F'_i) = \{\p \mid \p \cdot \v_i = b'_i\}.
\end{displaymath}
where $b'_i$ is chosen so that the point $\p_i\in \op{aff}(F'_i)\cap
\op{aff}(F)$ and lies on the boundary of $\op{rcone}(\orz,\v,h)$.
Define $P'$ by the intersection of $P$ with the half-spaces
\begin{displaymath} 
\{\p \mid \p \cdot \v_i \le b'_i\},
\end{displaymath}
where the signs are chosen so that $b'_i>0$.  Let $F' = P'\cap F$.
Then $F'$ is a facet of $P'$.  The polyhedron $P'$ and facet $F'$
satisfy the assumptions of the lemma with the same constant $k= k_F =
k_{F'}$.  This completes the proof that we may assume that each edge
of $F$ meets $\op{rcone}(\orz,\v,h)$ in a unique point.  That is, the
edge is tangent to the right-circular cone.

Drop the primes from the notation: $P=P'$, $F=F'$, and so forth.  The
Rogers partition gives a partition of the polyhedron $P$ into
simplices.  There are $2k$ simplices in (the closure of) $W_F$.  The
solid angle of each simplex is the area of a spherical triangle.

Consider a spherical triangle with sides $a,b,c$ and opposite angles
$\alpha,\beta,\gamma$.  If $\gamma=\pi/2$, then by Girard's formula,
the area of the triangle is
\begin{displaymath} 
\alpha+\beta-\pi/2,
\end{displaymath}
and by the law of cosines 
\begin{displaymath} 
\cos(\alpha) =\sin(\beta)\cos(a).
\end{displaymath}
This determines the area $g(a,\beta)$ of the triangle 
as a function of $a$ and $\beta$. 
\indy{Index}{Girard's formula}%
\indy{Notation}{g@$g$ (triangle area)}%
\indy{Notation}{ZZalpha@$\alpha$ (angle)}%
\indy{Notation}{ZZbeta@$\beta$ (angle)}%
\indy{Notation}{ZZddgamma@$\gamma$ (angle)}%
\indy{Index}{convex}%
\indy{Index}{Girard's formula}%
\indy{Index}{polygon}%

The solid angle of $W_F$ is the sum of the areas of the triangles:
\begin{displaymath} 
\sum_{i=1}^k g(a,\beta_i) 2,
\end{displaymath}
with angle sum
\begin{displaymath} 
\sum_{i=1}^k \beta_i = 2\pi.
\end{displaymath}
With  $a$ fixed, the second partial of $A$ with respect to $\beta$ is
\begin{displaymath} 
\frac{\partial^2 g(a,\beta)}{\partial \beta^2} = 
\frac{\cos(a)\sin^2(a)\sin(\beta)}{\sin^2(\alpha)} > 0.
\end{displaymath}
The function is convex.
By convexity, the minimum area occurs when all angles are equal
$\beta=\beta_i = \pi/k$.

The solid angle bound of the lemma is equal to 
\begin{displaymath} 
2 k g(a,\beta)
\end{displaymath}
where $\cos(a)=h$.  Alternatively, the polygon breaks into $2k$
triangles, each computed by Girard's formula to have area
\begin{displaymath} 
\beta - (\pi/2 - \alpha)  = \pi/k - \arcsin(\cos(\alpha)) = 
\pi/k - \arcsin(\cos(a)\sin(\beta)).
\end{displaymath}
\end{proof}

%\begin{lemma}[]\guid{BBEVFIC}\rz{0}\label{lemma:ngon-area} 
%  The minimum area of an intersection of $k$-hemispheres containing a
%  circle $C$ of arcradius $a<\pi/2$ is
%\begin{displaymath} 
%2\pi - 2 k \,\arcsin\left(\,\cos(a)\sin(\beta)\,\right),
%\end{displaymath}
%when $\beta = \pi/k$.
%\end{lemma}



\subsection{a polyhedral bound}

\begin{definition}[weakly saturated]\guid{HUCFLEB} 
\label{def:weakly-saturated}
Let $r$ and $r'$ be real numbers such that $2\le r\le r'$.  Define a
set $ V\subset\ring{R}^3\setminus B(\orz,2)$ to be \newterm{weakly saturated} with
parameters $(r,r')$ if for every $\p\in\ring{R}^3$
\begin{displaymath} 
2\le\normo{\p}\le r'~~~\implies~~~ \exists \u\in V.~\norm{ \u}{\p}< r.
\end{displaymath}
\end{definition}

\begin{lemma}[]\guid{TARJJUW}\rz{0}\label{lemma:poly-bounded} 
\oldrating{120}
\formalauthor{Dang Tat Dat}
Fix $r$ and $r'$ such that $2\le r\le r'$.
Let $ V$ be a weakly saturated finite packing with parameters $(r,r')$.
%such that $\orz\in  V$.
%where 
%   $\orz\in V$, and
%   $\normo{ \u}\le r'$ for all $ \u\in V$,
%Set $ V^*= V\setminus\{\orz\}$.
For any $g: V\to\ring{R}$, let $P( V,g)$ be the
polyhedron given by the intersection of half-spaces
\begin{displaymath} 
\{\p \mid  \u\cdot \p \le g( \u)\},\quad \u\in V.
\end{displaymath}
Then $P( V,g)$ is bounded.
\end{lemma}
\indy{Index}{polyhedron}%

\begin{proof} Assume for a contradiction that $P=P( V,g)$ is
  unbounded, and there exists $\p\in P$ such that $\normo{\p} > g( \u)
  r'/2$ for all $ \u\in V$.  Let $\v = r' \p/\normo{\p}$ so that
  $r'=\normo{\v}$.  By the weak saturation of $ V$, there exists $
  \u\in V$ such that $\norm{\v}{ \u}<r$.  Then,
\begin{align*} 
\normo{\p} &> g( \u) r'/2 \ge  \u\cdot (r' \p)/2 = \normo{\p}  \u\cdot \v /2\\
&= \normo{\p} (\normo{ \u}^2 + \normo{\v}^2 - \norm{ \u}{\v}^2)/4\\
&> \normo{\p}(4+r'^2-r^2)/4\\
&\ge \normo{\p}.
\end{align*}
This contradiction shows that $P$ is bounded.
\end{proof}




Since $L(h)\le 1$ when $h\ge1$, it is clear that a finite packing $V$
that satisfies Inequality~\ref{eqn:CE} has cardinality greater than twelve.
The following lemma also gives an upper bound on the cardinality of $V$.

\begin{lemma}[]\guid{DLWCHEM}\rz{500}\label{lemma:13-14}  %%
Let $V\subset \BB$ be a packing that satisfies
inequality~\ref{eqn:CE}.  Then the cardinality of $V$ is thirteen, fourteen, or fifteen.
\end{lemma}


\begin{proof} (Following Marchal)
Consider a finite packing $ V=\{\u_1,\ldots, \u_N\}\subset \BB$ satisfying
inequality~\ref{eqn:CE}.  The packing $V$ contains more than twelve points, because otherwise
the inequality~\ref{eqn:CE} cannot hold, as $L(h)\le 1$.

By adding points
as necessary, the packing becomes weakly saturated in the sense of
Definition~\ref{def:weakly-saturated}, with $r=2$ and $r'=2\hm$.  It is enough to show
that this enlarged set has cardinality less than sixteen.    Let
\begin{displaymath}%
g(h) = \arccos(h/2) - \pi/6,  %
\end{displaymath}%
and let $h_i =
\normo{ \u_i}/2$.  Then $h_i\le h_0=1.26$.
Consider the spherical disks $D_i$ of radii $g(h_i)$,
centered at $ \u_i/\normo{ \u_i}$ on the unit sphere.  These disks do not overlap; this
follows from the easy
Calculation~\ref{calc:cc:disks} %% Marchal disks are disjoint.  % cc:disks
\begin{equation}\label{eqn:disks} 
g(h_i) + g(h_j) \le \op{arc}(2h_i,2h_j,2).%
\end{equation}%
\indy{Notation}{D@$D$ (spherical disks)}%
For each $i$, the plane through the circular boundary of $D_i$ bounds
a half-space containing the origin.  The intersection of these
half-spaces is a polyhedron $P$, which is bounded by
Lemma~\ref{lemma:poly-bounded}.  Lemma~\ref{lemma:polyhedron}
associates a fan $(V_P,E_P)$ with $P$.  (The set $V_P$ is dual to $
V$; the set $V_P$ is in bijection with extreme points of $P$, whereas $
V$ is in bijection with the facets of $P$.)  There are natural
bijections between the following sets:
\begin{itemize} 
\item $ V = \{ \u_1,\ldots, \u_N\}$.
\item The  facets of $P$.
\item The set of  topological components of $Y(V_P,E_P)$.
\item The set of faces in the hypermaps $\op{hyp}(V_P,E_P)$.
\end{itemize}
The bijection of the first two sets follows from the first conclusion
of Lemma~\ref{lemma:webster}.  Lemmas~\ref{lemma:WF} and
~\ref{lemma:face} give the other two bijections.

Let $k_i$ be the cardinality of the face in $\op{hyp}(V_P,E_P)$
corresponding to the facet $i$.  By Lemma~\ref{lemma:ngon}, the solid
angle of the topological component $W_i$ of $Y(V_P,E_P)$ is at least
$\op{reg}(g(h_i),k_i)$, where \indy{Index}{half-plane}%
\indy{Index}{half-space}%
\indy{Notation}{reg (area of regular spherical polygon)}%
\begin{displaymath} 
\op{reg}(a,k) = 2\pi - 2 k (\arcsin(\cos(a)\sin(\pi/k))).
\end{displaymath}
By a computer calculation\footnote{\calc{1965189142}. 
The integer parameter $k$ can be replaced with a real variable.   If $k\ge34$, then
the right-hand-side is negative and the inequality is immediate.} 
%~\ref{calc:cc:alin}, %% Linear lower bound on regular
%% polygon. % cc:alin
\begin{equation}\label{eqn:alin} 
\op{reg}(g(h),k) \ge c_0 + c_1 k + c_2 L(h),\quad
k = 3,4,\ldots,\quad 1\le h\le \hm,
\end{equation}
where
\begin{displaymath}
%c_0 = 0.6327,\quad c_1 = -0.0333,\quad c_2 =0.4754.
c_0=0.591,\quad c_1=-0.0331,\quad c_2 = 0.506.
\end{displaymath} The sum $\sum_i k_i$ is the number of darts
in $\op{hyp}(V_P,E_P)$ by  Lemma~\ref{lemma:polyhedron}.  By
Lemma~\ref{lemma:dart-upper}, $\sum_i k_i \le (6N-12)$.  Summing over
$i$, an estimate on $N$ follows: 
\indy{Index}{polyhedron!convex}%
\indy{Index}{hypermap!planar}%
\begin{align*} 
4\pi &= \sum_i\op{sol}(W_i)\\
&\ge \sum_i \op{reg}(g(h_i),k_i) \\
&\ge c_0 N +c_1\sum_i k_i + c_2 \sum L(h_i)\\
&\ge c_0 N +c_1 (6N-12) + c_2 12\\
\end{align*}
This gives
$16 > N$.
\end{proof} 


\begin{lemma}[]\guid{XULJEPR}\rz{400}\label{lemma:D'}  
Let $ V\subset\BB$ be a finite packing.
Assume that there exists $ \v\in V$ such that $\normo{ \v}=2$ and
$\norm{ \v}{ \u}\ge 2\hm$ whenever $\v\ne\u\in V$.
Then  inequality~(\ref{eqn:CE}) does not hold on $ V$.
\end{lemma}

\begin{proof} Assume for a contradiction that a packing exists that
satisfies the assumptions and the inequality.  Without loss of
generality, assume that $N\ge 13$, since the inequality is known to
hold when $N\le 12$.  Create one large disk $D_1'$ centered at $
\v/2$ and repeat the proof of the previous lemma.  Extend the packing to a the weak
saturation with parameters $r=r'=2\hm$.  This can be done in a way that maintains
the assumptions on $\v$.  By
Lemma~\ref{lemma:poly-bounded}, the polyhedron is
bounded.  By the assumptions of the lemma, we may take
\begin{displaymath}a'=0.797 < \arc(2,2\hm,2\hm)-g(\hm)\end{displaymath} 
for the arcradius of the large disk $D_1'$.  
By a computer calculation\footnote{\calc{6096597438}.
The integer parameter $k$ can be replaced with a real variable.  If $k\ge64$, then
the right-hand-side is negative and the inequality is immediate.} %~\ref{calc:cc:alin2}, 
%% Linear lower bound on regular polygon (large disk) % cc:alin2
\begin{equation}\label{eqn:alin2} 
\op{reg}(a',k) \ge c_0 + c_1 k + c_2 L(1) +
c_3,\quad k=3,4,\ldots\end{equation}
where
$c_3 =  1$.  % was %0.85$.
Then 
\begin{align*} \label{eqn:D'}
4\pi &= \sum_i\op{sol}(W_i)\\
&\ge \op{reg}(a',k_1)+\sum_{i>1} \op{reg}(g(h_i),k_i) \\
&\ge  c_0 N +c_1\sum_i k + c_2 \sum L(h_i) + c_3\\
&\ge c_0 N +c_1 (6N-12) + c_2 12 + c_3\\
\end{align*}
This gives a contradiction
$13 > N \ge 13.$
\end{proof}
