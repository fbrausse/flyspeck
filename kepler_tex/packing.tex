% file started March 22, 2009

\chapter{Packing}

\begin{summary}
  This chapter comprises much of the core material of the book.  At
  last we take up the topic of dense sphere packings.  Associated with
  a sphere packing $V$ in $\ring{R}^3$ are various subsidiary
  decompositions of space.  This chapter focuses on three such
  decompositions: the Voronoi decomposition into polyhedra, the Rogers
  decomposition into simplices, and the Marchal decomposition into
  cells.  Each of these decompositions leads to a bound on the density
  of sphere packings.  The bounds in the first two cases are not
  sharp.  The third decomposition leads to a sharp bound
  $\pi/\nsqrt{18}$ on the density of sphere packing in three
  dimensions.  The final sections of this chapter undertake a detailed
  study of the properties of the Marchal cell decomposition.
\end{summary}

\section{The Primitive State of Our Subject Revealed}


\subsection{definition}



Informally, a \newterm{packing} is an arrangement of congruent
balls in Euclidean three space that are nonoverlapping in the sense
that the interiors of the balls are pairwise disjoint.  By convention,
we take the radius of the congruent balls to be $1$.
%the scale
%invariance of density, without loss of generality, units can be chosen
%so that each ball has radius $1$. 
Let $ V$ be the set of centers of the balls in a
packing. The choice of unit radius for the
balls implies that any two points in $ V$ have distance  at
least $2$ from each other. 
 Formally, the packing is identified
with the set of centers $V$.
\indy{Notation}{V@$V$ (packing)}%

%
%The density of a packing does not decrease when balls are added to the
%packing. Thus, to construct packings of maximal density, one may add
%nonoverlapping balls until there is no room to add further balls.  A
A packing in which no further balls can be added is said to be {\it
saturated} (Figure~\ref{fig:saturated}).

\figDEQCVQL % fig:saturated

\begin{definition}[saturated,~packing]\guid{XASMJUK} 
\formaldef{packing}{packing}
\formaldef{saturated}{saturated}
A \newterm{packing} $ V\subset \ring{R}^3$ is a set such that
\[ 
\forall  \u,~\v\in  V.~  \norm{ \u}{\v} < 2 \Rightarrow ( \u=\v).
\]  
A set $V$ is \newterm{saturated} if for every $\p\in\ring{R}^3$ there
exists some $ \u\in V$ such that $\norm{ \u}{\p}< 2$.
\end{definition}
\indy{Index}{saturated}%
\indy{Index}{packing}%



Let $B(\p,r)$ denote the open ball in
Euclidean three-space at center $\p$ and radius $r$.  The open ball
is measurable with measure $4\pi r^3/3$.
 Set $ V(\p,r) = V \cap
B(\p,r)$. %and $ V^*(\p,r) = V(\p,r)\setminus \{\p\}$.
\formaldef{ball}{ball}
\formaldef{$V(\p,r)$}{V INTER ball(p,r)}
\indy{Index}{measure}%
\indy{Notation}{B@$B(\v,r)$ (open ball)}%
\indy{Notation}{V1@$V(\p,r)=V\cap B(\p,r)$}%

\begin{lemma}[]\guid{KIUMVTC}
\formalauthor{Nguyen Tat Thang}
\label{lemma:V-finite}
Let $ V$ be a packing and let $\p\in\ring{R}^3$.
Then the set $ V(\p,r)$ is finite.
\end{lemma}

\begin{proof}  Let $\p = (p_1,p_2,p_3)$. The floor function gives the map
\[ (v_1,v_2,v_3)\mapsto (\lfloor 2(v_1-p_1)
  \rfloor, \lfloor 2(v_2-p_2) \rfloor, \lfloor 2(v_3-p_3) \rfloor).
\] 
It is a one-to-one map from $ V(\p,r)$ into the set $\ring{Z}^3\cap
B(\orz,2r + 1)$.  By Lemma~\ref{lemma:Zcount} the range of this
one-to-one map is finite.  Hence, the domain $ V(\p,r)$ of the map is
also finite.%
\footnote{An alternative proof uses the open cover of the compact ball
  $\bar B(\p,r)$ by the sets $\bar B(\p,r)\setminus V$ and $B(\v,1)$
  for $\v\in V$. By compactness, the cover is necessarily finite.}
\end{proof}
%\indy{Notation}{2@$\lfloor\wild\rfloor$ (floor)}% % doesn't parse






\subsection{Voronoi cell}

Geometric decompositions of space give a way to estimate the density
of sphere packings.  A popular decomposition of space
is the Voronoi cell decomposition (Figure~\ref{fig:voronoi}).

\begin{definition}[Voronoi~cell,~$\Omega$]\guid{YGFWXEH}\label{def:voronoi} 
\formaldef{$\Omega$}{voronoi\_closed}
%\indy{Index}{Voronoi cell}%
Let $V\subset\ring{R}^3$ and $\v\in V$.
The \fullterm{Voronoi cell}{decomposition!Voronoi}
$\Omega(V,\v)$
is the set of points at least as close to $\v$ as to
any other point in $V$. 
% Let $\Omega_t( V,\v) = \Omega( V,\v)
%\cap B(\v,t)$ be the truncated Voronoi cell at radius $t$.
\end{definition}
\indy{Notation}{zzZ@$\Omega$ (Voronoi cell)} %

\figXOHAZWO % fig:voronoi

\begin{lemma}[Voronoi partition]\guid{TIWWFYQ}
If $V$ is a saturated packing, then 
\begin{equation}\label{eqn:vor-rn} 
\ring{R}^3 = \bigcup \{\Omega(V,\v)\mid \v \in V\}.
\end{equation}
\end{lemma}

\begin{proof}
  If $V$ is a saturated packing, then every point $\p$ has distance 
  less than $2$ from some point of $V$.  The set $V(\p,2)$ is finite
  by Lemma~\ref{lemma:V-finite}.  Hence, $\p$ is at least as close to
  some $\v\in V$ as it is to any other $\w\in V$.  This means that
  $\p\in\Omega(V,\v)$.  
\end{proof}

We use half-spaces to separate one Voronoi cell from another.

\begin{definition}[half-space]\guid{BGXHPKY}
\formaldef{$A$}{bis}
\formaldef{$A_+$}{bis\_le}
\begin{align*} 
A(\u,\v) &= \{\p\in\ring{R}^3
\mid 2(\v-\u)\cdot \p = \normo{\v}^2 - \normo{\u}^2 \},\\
A_+(\u,\v) &= \{\p\in\ring{R}^3
\mid 2(\v-\u)\cdot \p \le \normo{\v}^2 - \normo{\u}^2 \},
\end{align*}
when $\u,\v\in\ring{R}^3$.  The plane $A(\u,\v)$ is the \newterm{bisector} of
$\{\u,\v\}$ and $A_+(\u,\v)$ is the \newterm{half-space} of points at least as
close to $\u$ as to $\v$.  
\end{definition}
\indy{Notation}{A1@$A(\u,\v)$ (bisector)}%
\indy{Notation}{A2@$A_+(\u,\v)$ (half-space)}%

Each Voronoi cell is a bounded polyhedron.

\begin{lemma}[Voronoi polyhedron]\guid{RHWVGNP}\label{lemma:V4} 
  Let $V\subset\ring{R}^3$ be a saturated packing.  Then
  $\Omega(V,\v)\subset B(\v,2)$.  Also, $\Omega(V,\v)$ is a polyhedron
  defined by the intersection of the finitely many half-spaces
  $A_+(\v,\u)$ for $\u\in V(\v,4)\setminus\{\v\}$.
\end{lemma}

\begin{proof} 
The Voronoi cell $\Omega(V,\v)$ is the
intersection of the half-spaces $A_+(\v,\u)$ as $\u$ runs over
$V\setminus \{\v\}$.

Let $\p\not\in B(\v,2)$.  
By saturation, there exists $\u\in V$ such that $\norm{\p}{\u}<2$.
Then 
\[  
\norm{\p}{\u} < 2 \le \norm{\p}{\v}.
\] 
Hence, $\p\not\in\Omega(V,\v)$.  This proves the first conclusion.


Let $\Omega'$ be the intersection of the half-spaces $A_+(\v,\u)$ as
$\u$ runs over $V(\v,4)$.  Clearly, $\Omega(V,\v)\subset \Omega'$.
Assume for a contradiction that $\p\in \Omega'\setminus\Omega(V,\v)$.
The intersection of the ray $\op{aff}_+\{\v,\{\p\}\}$ with
$\Omega(V,\v)$ is a closed and bounded convex subset of the line.  By
general principles of convex sets, this intersection is an interval
$\op{conv}\{\v,\p'\}$ for some $\p'\in\Omega(V,\v)\subset B(\v,2)$.
For some small $t>0$, the point lies beyond the interval but remains
within the ball:
\[  
\q = (1+t)\p' -t \v\in (B(\v,2)\cap \Omega')\setminus \Omega(V,\v).
\] 
Choose $\u\in V\setminus V(\v,4)$ such that $\q\in A_+(\u,\v)$.  By the
triangle inequality,
\[  
\norm{\u}{\v} \le \norm{\u}{\q} + \norm{\v}{\q} \le 2\norm{\v}{\q} < 4.
\] 
This contradicts the assumption $\u\not\in V(\v,4)$.

The number of half-spaces $A_+(\v,\u)$ for $\u\in V(\v,4)$ is finite by
Lemma~\ref{lemma:V-finite}.  A set defined by the intersection of a finite number
of closed half-spaces is a polyhedron.
\end{proof}

\begin{lemma}[Voronoi compact]\guid{DRUQUFE}
\formalauthor{Nguyen Tat Thang}
Let $ V$ be a saturated packing.  For every $\v\in  V$, 
the Voronoi cell $\Omega( V,\v)$  is
compact, convex, and measurable.
\end{lemma}

\begin{proof} By the previous lemma, it is a bounded polyhedron.
  Every bounded polyhedron is compact, convex, and measurable.
\end{proof}




\subsection{reduction to a finite packing}

We finally state the main result of this book, the Kepler conjecture.
The proof fills most of this book. This section describes the
outline of the proof and gives references to the sources of the
details of the proof.


\begin{theorem*}[Kepler's conjecture on dense packings]\guid{IJEKNGA} 
\label{theorem:kepler}   
\formaldef{Kepler conjecture}{kepler\_conjecture}%
No packing of congruent balls in Euclidean three space has density
greater than that of the face-centered cubic (FCC) packing.
\end{theorem*}
\indy{Index}{FCC}%
\indy{Index}{face-centered cubic|see{FCC}}%
\indy{Index}{HCP}%
\indy{Index}{hexagonal-close packing|see{HCP}}%
\indy{Index}{Kepler conjecture}%

\begin{remark}\guid{LLFORJR}
This density is $\pi/\nsqrt{18}\approx 0.74$.  There are other
packings, such as the HCP or the FCC
packing with finitely many balls removed, that attain this
same density.
\end{remark}

The Kepler conjecture is a statement about space-filling packings.  A
space-filling packing is specified by a countable number of real
coordinates -- three for the position of each of countably many balls.
The first task in resolving the conjecture is to reduce the problem to
one involving only a finite number of balls.  This is accomplished by
Lemma~\ref{lemma:deltabound}.

The relevant concepts are \fullterm{negligibility}{negligible} and {\it
  FCC-compatibility}, given as follows.  FCC-compatibility means that
the Voronoi cells on average have volume at least that of those in the
FCC-packing.  Negligibility means that the error term is insignificant.


\begin{definition}[negligible,~FCC-compatible]\guid{ZREKEVW}\label{def:negligible}
\formaldef{FCC compatible}{fcc\_compatible}
\formaldef{negligible}{negligible\_fun\_0}
A function $G:V\to \ring{R}$ on a set $V\subset\ring{R}^3$
is \fullterm{negligible\/}{negligible}
if there is a constant $c_1$ such that for all $r\ge1$,
% and all $\p\in\ring{R}^3$,
\[ \sum_{\v\in V(\orz,r)} G(\v) \le c_1
r^2.\] 
A function $G: V\to\ring{R}$ is
\fullterm{FCC-compatible\/}{FCC!compatible}
if for all $\v\in V$, 
\[ 4\nsqrt{2}\le \op{vol}(\Omega(V,\v)) +
G(\v).\] 
\indy{Notation}{G@$G$ (negligible function)}%
\end{definition}


\begin{remark}\guid{RTMZJVG}
  The value $\op{vol}(\Omega(V,\v)) + G(\v)$ may be interpreted as an
  \newterm{adjusted\/} volume of the Voronoi cell. The constant
  $4\nsqrt{2}$ that appears in the definition of FCC-compatibility is
  the volume of the Voronoi cell in the HCP and FCC packings.  (See
  Chapter~\ref{sec:close}.)  The corrected volume is at least the
  volume of these Voronoi cells when the correction term $G$ is
  FCC-compatible.
%
\indy{Index}{corrected    volume}%
\end{remark}

% \begin{remark}\guid{GRKKKGN} In \cite{Hales:2006:DCG}, the full Voronoi cell
%   $\Omega(V,\v)$ is used, rather than $\Omega(V,\v)$.  The truncation at
%   radius $2$ is just a matter of convenience to guarantee the
%   boundedness and hence the finite volume of the (truncated) Voronoi
%   cell.  In \cite{Hales:2006:DCG}, the same effect was achieved by
%   requiring all packing%s
%   to be saturated.  Drop the assumption of saturation on $ V$.
%\end{remark}



The density $\delta( V,\p,r)$ of a packing $ V$ within a bounded
region of space is defined as a ratio. The numerator is volume of
$B(V,\p,r)$, defined as the intersection with $B(\p,r)$ of the union
of all balls $B(\v,1)$ in the packing $V$.  The denominator is the volume of
$B(\p,r)$. 
%An ordered pair $( V,\v)$ with $\v\in V$ is called a \newterm{centered
%packing}.  \indy{Index}{packing!centered}%
\indy{Notation}{zzd@$\delta( V,\p,r)$}%
\indy{Notation}{V@$V$ (packing)}%



\begin{lemma}[reduction to finite dimensions]\guid{JGXZYGW} 
\formalauthor{Nguyen Tat Thang}
\label{lemma:deltabound} % 
If there exists a \newterm{negligible} \fullterm{FCC-compatible}{FCC!compatible}
function $G: V\to\ring{R}$ for a saturated packing $ V$, then there
exists a constant $c=c(V)$ such that for all $r\ge1$,
% and all $\p\in\ring{R}^3$, 
%%  removed on  Jan 16, 2009 after formalization.
\[  
\delta( V,\orz,r)
\le {\pi}\Big/{\nsqrt{18}} + c/r.
\] 
%The constant $c$ depends on $ V$ only through the constant
%$c_1$ of Definition~\ref{def:negligible}.
\end{lemma}

%\begin{remark}\label{conj:fcc-neg} 
%For every saturated packing $ V$, there exists a negligible
%FCC-compatible function $G: V\to R$.
%\end{remark}



\begin{proof} 
The volume of $B( V,\orz,r)$ is at most the product of the volume
$4\pi/3$ of each ball with the number of centers in
$B(\orz,r+1)$.  Hence,
\begin{equation} 
\op{vol}\, B( V,\orz,r) \le \card( V(\orz,r+1))\, 4\pi/3.
\label{eqn:Abound}
\end{equation}

%In a %saturated packing 
Each truncated Voronoi cell is contained in a ball of radius $2$ that
is concentric with the unit ball in that cell.  The volume of the
large ball $B(\orz,r+3)$ is at least the combined volume of all
truncated Voronoi cells centered in $B(\orz,r+1)$. This observation,
combined with FCC-compatibility and negligibility, gives
\begin{equation} 
\begin{split} 
4\nsqrt{2}\,\,\card( V(\orz,r+1))
&\le \sum_{\v\in V(\orz,r+1)} (G(\v) +
\op{vol}(\Omega(V,\v))) \\
&\le c_1 (r+1)^2 + \op{vol}\,B(\orz,r+3) \\
&\le c_1 (r+1)^2 + (1+3/r)^3 \op{vol}\,B(\orz,r).
\label{eqn:Bbound}
\end{split}
\end{equation}
\indy{Index}{FCC!compatible}%
Recall that $\delta( V,\orz,r)=
\op{vol}\,B( V,\orz,r)/\op{vol}\,B(\orz,r)$. Divide Inequality
\ref{eqn:Abound} through by $\op{vol}\,B(\orz,r)$.  Use
Inequality~\ref{eqn:Bbound} to eliminate $\card( V(\orz,r+1))$ from the
resulting inequality.  This gives
\[ \delta( V,\orz,r)
\le \frac{\pi}{\sqrt{18}} (1+3/r)^3 + c_1 \frac{(r+1)^2}{r^34\sqrt{2}}.
\] 
The result follows for an appropriately chosen constant $c$
(depending on $c_1$).
\end{proof}

\begin{remark}[Kepler conjecture in precise terms]\guid{ZHIQGGN}
\label{remark:precise} 
The precise meaning of the \newterm{sphere packing problem} or the
\newterm{Kepler conjecture} is to prove the bound bound $\delta(
V,\orz,r) \le \pi/\nsqrt{18} + c/r$ for every saturated packing $ V$.
The error term $c/r$ comes from the boundary effects of a bounded
container holding the balls.  The error tends to zero as the radius
$r$ of the container tends to infinity.  Thus, by the preceding lemma,
the existence of a negligible FCC-compatible function provides the
solution to the packing problem.  The strategy is to define a
negligible function and then to solve an optimization problem in
finitely many variables to establish that the function is also
FCC-compatible.
\end{remark}





\section{Rogers Simplex}\label{sec:rogers}

\indy{Index}{Rogers|see{decomposition}}%
\indy{Index}{decomposition!Rogers}%

% Think of $ V$ as the set of centers of a packing of congruent balls
% of unit radius. To be saturated means that there is no room for
% further balls to be added to the packing. There is no loss in
% generality in assuming that the packing is saturated, when searching
% for the greatest possible density of a packing.

Rogers gave a bound on the density of sphere packings in Euclidean
space of arbitrary dimension~\cite{Rogers:1958:Packing}.  His bound
states that the density of a packing in $n$-dimensions cannot exceed
the ratio of the volume of $A \cap T$ to the volume of $T$, where $T$
is a regular tetrahedron of side length $2$ and $A$ is the set of
$n+1$ balls of unit radius placed at the extreme points of $T$.  In two
dimensions, the Rogers's bound is sharp and gives a solution to the
sphere packing problem.  In three dimensions the bound is
approximately $0.7797$, which differs significantly from the optimal
value $\pi/\nsqrt{18}\approx 0.74$.  Rogers's bound is the unattainable
density that would result if regular tetrahedra could tile
space.\footnote{Aristotle erroneously believed that regular tetrahedra
  tile space: ``It is agreed that there are only three plane figures
  which can fill a space, the triangle, the square, and the hexagon,
  and only two solids, the pyramid and the cube.''~\cite{Aristotle}.}
\indy{Notation}{T@$T$ (regular tetrahedron)}%
\indy{Index}{decomposition!Rogers}%
\indy{Index}{Aristotle}%

To prove his bound, Rogers gives a partition of Euclidean space into
simplices with extreme points in a packing $V$.  This section develops
the basis properties of Rogers simplices.  The next section modifies
the simplices to obtain a sharp bound on the density of packings.




\subsection{faces}

The Rogers partition is a refinement of the Voronoi cell decomposition.
In preparation for this decomposition, this subsection goes into
greater detail about the structure of the faces of a Voronoi cell.  We
parameterize various faces of the Voronoi cell by lists of points in a
saturated packing $V$ (Figure~\ref{fig:vset}).

\figKFETCJS % fig:vset

\begin{definition}[$\Omega$ reprise]\guid{BBDTRGC} 
  \formaldef{$\Omega(V,W)$}{voronoi\_set}
  \formaldef{$\Omega(V,\bu)$}{voronoi\_list} Let $V$ be a saturated
  packing.  The notation $\Omega(V,\wild )$ can be
  \newterm{overloaded} to denote intersections of Voronoi cells, when
  the second argument is a set or list of points.  If $W\subset
  V$, 
  then the intersection of the family of Voronoi cells is
  $\Omega(V,W)$:
\[ \Omega(V,W) = \bigcap \{\Omega(V, \u)\mid \u\in W
\}.\] 
Define $\Omega$ on lists 
to be the same as its value on point sets: 
\begin{align*} 
\Omega(V,[\u_0;\ldots;\u_k]) = \Omega(V,\{\u_0;\ldots;\u_k\}).
\end{align*}
\end{definition}
\indy{Notation}{zzZ@$\Omega(V,W)$ (intersection of Voronoi cells)}%

An intersection of Voronoi cells can be written in many equivalent forms:
\[  
  \Omega(V,\v)\cap \Omega(V,\u) =\Omega(V,\{\u,\v\})
 = \Omega(V,\v)\cap A_+(\u,\v) 
  = \Omega(V,\v)\cap A(\u,\v) =  \cdots.
\] 





\begin{definition}[$\bV$]\guid{NOPZSEH} 
  \formaldef{$\bV(k)$}{barV} Let $V$ be a saturated packing.
  When $k=0,1,2,3$, let $ \bV(k)$ be the set of lists
  $\bu=[\u_0;\ldots;\u_k]$ of length $k+1$ with $ \u_i\in V$ such
  that
\begin{equation}\label{eqn:omega-dim} 
\dimaff(\Omega(V,[\u_0;\ldots;\u_j])) = 3-j
\end{equation}
for all $0<j\le k$.  (Recall that $\dimaff(X)$ is the affine dimension
of $X$ from Definition~\ref{def:affine}.)  Set $\bV(k)=\emptyset$ for
$k>3$.  
\end{definition}
%\indy{Notation}{V2@$\bV(k)$ (set of suitable lists of length $k+1$ in $V$)}% % doesn't parse
%\indy{Notation}{vz@$\bu$ (list of members of a packing)}% % doesn't parse.
\indy{Notation}{dimaff@$\dimaff$ (affine dimension)}%

In particular, $V$ can be identified with $\bV(0)$ under the natural
bijection $\v\mapsto[\v]$, and $\bV(1)$ is the set of lists $[\u;\v]$
of distinct elements such that the Voronoi cells at $ \u$ and $\v$
have a common facet (Lemma~\ref{lemma:omega-facet}).  

\begin{notation}[underscore]
  % Given a saturated packing $V$, and natural number $k$, the
  % definition gives $\bV(k)$ as a certain set of lists of points in
  % $V$.
  The underscores follow a special syntax.  In $\bV(k)$, the
  underscore is a function
\[  
\underline{\phantom V}:\{V \mid \text{$V$ saturated packing} \}
\times \ring{N} \to \wild.
\] 
 The syntax is somewhat different in $\bu$.  Here the
underscore is not a function, but part of its name, following a
general typographic convention to mark lists of points. The notations
are coherent because $\bu\in\bV(k)$.
\end{notation}

\begin{notation}[$\trunc{\bu}{j}$]\guid{JNRJQSM}
%\indy{Notation}{u@$\bu$ (list of points)}% index misformats
\indy{Notation}{d@$d_j$ (truncation of lists)}%
\formaldef{$\trunc{\bu}{j}$} {truncate\_simplex}
\hspace{-3pt}
When $\bu=[\u_0;\ldots;\u_k]$ and $j\le k$, write
%\footnote{The  notation follows the syntax of Python slices.} 
$\trunc{\bu}{j} = 
[\u_0;\ldots;\u_j]$ for the truncation of the list.  
%\indy{Notation}{1@$\trunc{\wild }{:\wild }$}%
\end{notation}

Truncation $\bu\mapsto\trunc{\bu}{j}$ maps $\bV(k)$ to $\bV(j)$ when
$j\le k$.  Beware of the index: $k$ is the \newterm{codimension} of
$\Omega(V,\bu)$ in $\ring{R}^3$, when $\bu\in \bV(k)$; it is not the
\newterm{length} of the list $\bu$ (which is $k+1$).\footnote{By
  convention aa $k$-simplex is presented as a $k+1$-tuple.  Because of
  this shift by one, the notation $\trunc{\bu}{j}$ also differs by the
  same shift.}


\begin{lemma}[Voronoi face]\guid{KHEJKCI}\label{lemma:omega-face}  
Let $V\subset\ring{R}^3$ be a saturated packing.
Let $\bu=[\u_0;\ldots;\u_{k}]\in \bV(k)$.  
Then $\Omega(V,\bu)$ is a face of $\Omega(V,\u_0)$.
\end{lemma}

\begin{proof} This follows directly from the definition of face on
  page~\pageref{def:face}.  The set $\Omega(V,\bu)$ is
an intersection of the convex sets $\Omega(V,\u_i)$ and
is therefore convex.  Also,  $\Omega(V,\bu)$ is the intersection of
  $\Omega(V,\u_0)$ with the planes $A(\u_0,\u_i)$, where $i>0$.  
 Let $\p,\q\in\Omega(V,\u_0)$ and
  assume
\[  
\p' = s \p + t \q\in\Omega(V,\bu),\quad 
\text{ for some } s>0,\quad t>0,~\quad s + t = 1.
\]  
Then $\p'\in A(\u_0,\u_i)$.  Each plane $A(\u_0,\u_i)$ is a face of
the corresponding half-space $A_+(\u_0,\u_i)$ containing $\p$ and
$\q$.  By the definition of face, $\p,\q$ must also lie in
$A(\u_0,\u_i)$.  It follows that $\p,\q$ also lie in $\Omega(V,\bu)$.
By the definition of face, $\Omega(V,\bu)$ is a face of
$\Omega(V,\u_0)$.
\end{proof}

\begin{lemma}[facets]\guid{IDBEZAL}\label{lemma:omega-facet} 
  Let $V\subset\ring{R}^3$ be a saturated packing.  Let $\bu\in
  \bV(k)$ for some $k<3$.  Then $F$ is a facet of $\Omega(V,\bu)$ if
  and only if there exists $\bv\in \bV(k+1)$ such that $F =
  \Omega(V,\bv)$ and $\trunc{\bv}{k} = \bu$.
\end{lemma}

\begin{proof} 
  Use Lemma~\ref{lemma:V4} to write the polyhedron $\Omega(V,\bu)$ in
  the form of Equation~\ref{eqn:polyrep}:
\[  
\Omega(V,\bu) = A \cap A_\pm(\v_1,\u_0) \cap \cdots\cap A_\pm(\v_r,\u_0),
\] 
where $A$ is the affine hull of $\Omega(V,\bu)$, $\v_i\in V$, where
$A_-(\v,\w) = A_+(\w,\v)$ with the signs $\pm$ chosen as needed, and
$r$ is as small as possible.  By Lemma~\ref{lemma:webster}, if $F$ is
any facet of $\Omega(V,\bu)$, then there exists an $i\le r$ such that
\[  
F = \Omega(V,\bu) \cap A(\v_i,\u_0) = \Omega(V,\bv),
\] 
where $\bv = [\u_0;\cdots;\u_k;\v_i]$ is the list that appends $\v_i$
to $\bu$.  Also,
\[  
\dimaff(\Omega(V,\bv)) = \dimaff(F) = \dimaff(\Omega(V,\bu)) - 1 = 3 - k - 1,
\] 
because $F$ is a facet.  It follows that $\bv \in \bV(k+1)$.  This
proves the implication in the forward direction.

To prove the converse, let $\bv\in \bV(k+1)$, where $\trunc{\bv}{k} =
\bu$.  Elementary verifications show that $\Omega(V,\bv)\subset
\Omega(V,\bu)$ and that this set is nonempty if $k<3$.  By
Lemma~\ref{lemma:omega-face} and Lemma~\ref{lemma:webster},
$\Omega(V,\bv)$ is a face of $\Omega(V,\bu)$.  By the definition of
$\bV(\wild )$,
\[  
\dimaff(\Omega(V,\bv)) = 3 - (k+1) = \dimaff(\Omega(V,\bu)) -1.
\] 
It follows that $\Omega(V,\bv)$ is a facet of $\Omega(V,\bu)$.
\end{proof}


\subsection{partitioning space}

Each Rogers simplex is given as the convex hull of  its set of extreme points.  
The extreme points $\omega(d_i\bu)$
are defined by recursion.

\begin{definition}[$\omega$]\guid{JJGTQMN}\label{def:omega}
  \formaldef{$\omega_k$} {omega\_list\_n} \formaldef{$\omega$}
  {omega\_list} Let $V$ be a saturated packing and let
  $\bu=[\u_0;\ldots]\in \bV(k)$ for some $k$.  Define points
  $\omega_j=\omega_j(V,\bu)\in\ring{R}^3$ by recursion over $j\le k$
  (Figure~\ref{fig:rogers-omega}).
\begin{align*}
\omega_{0\phantom{+1}} &= \u_0,\\
\omega_{j+1} &=\text{the closest point to } \omega_j 
\text{ on }\Omega(V,\trunc{\bu}{j+1}).
\end{align*}
Set $\omega(V,\bu) = \omega_{k}(V,\bu)$, when $\bu\in \bV(k)$.  The
set $V$ is generally fixed and is dropped from the notation.
%By induction $\omega_j(\bu) = \
%Define $\omega$ \mid \coprod_{j=0}^3 \bV(j)\to \ring{R}^3$ by recursion over $j$ as follows.
%Let \[  
%\omega([\u]) = \u,
%\]  and
%let $\omega( \bu)$ be the closest point on $\Omega(V, \bu)$ to
%$\omega( \trunc{\bu}{j})$, when $\bu\in \bV(j+1)$.
\end{definition}

\figHFFTUNW % fig:rogers-omega

\claim{The point $\omega(\bu)$ exists when $\bu\in V(k)$.}  Indeed,
the set $\Omega(V, \bu)$ is nonempty, convex, and compact.  Thus, by
convex analysis, the closest point $\omega( \bu)$ exists uniquely.

The point $\omega_k$ depends on $\bu$ through its projection to
$\trunc{\bu}{k}$ so that
\[
\omega_k(\bu) = \omega_k(\trunc{\bu}{k})=\omega(\trunc{\bu}{k}).
\]
\indy{Notation}{zzz@$\omega$ (extreme points of Rogers simplex)}%

\begin{definition}[R,~Rogers simplex]\guid{PHZVPFY} 
\formaldef{$R$}{rogers}
Let $V\subset\ring{R}^3$ be a saturated packing. For $\bu\in \bV(k)$, let 
\[
 R(\bu) = \op{conv}\{\omega( \trunc{\bu}{0}), \omega(
 \trunc{\bu}{1}),\ldots,\omega( \trunc{\bu}{k})\}.
% R^0(\bu) &= \op{aff}^0_+(\emptyset,\{\omega( \trunc{\bu}{0}), \omega(
% \trunc{\bu}{1}),\ldots,\omega( \trunc{\bu}{k})\}).
\]
%Set $R(\bu)=R(0,\bu)$.  
The set $R(\bu)$ is called the Rogers simplex of $\bu$.
\indy{Notation}{R@$R$ (Rogers simplex)}%
\end{definition}

Each Voronoi cell can be partitioned into Rogers simplices (Figure~\ref{fig:rogers-random}).

\figBUGZBTW % fig:rogers-random 

\begin{lemma}[Rogers decomposition]\guid{GLTVHUM} 
\label{lemma:Rogers-d}
For any saturated packing $V\subset\ring{R}^3$, and any $\u_0\in V$,
\begin{equation} 
\Omega(V,\u_0) = \bigcup \{ R(\bv) \mid \bv\in \bV(3),~\trunc{\bv}{0} =[\u_0]\}.
\end{equation}
Consequently,
\[ 
\ring{R}^3 = \bigcup\, \{ R(\bv) \mid \bv\in\bV(3)\}.
\] 
\end{lemma}

\begin{proof} 
The proof uses standard facts about convex sets and polyhedra from
Section~\ref{sec:poly}.


%
%Now we turn to the proof.  
By the covering of $\ring{R}^3$  by Voronoi cells by \eqref{eqn:vor-rn},
%\[ \ring{R}^3 = \bigcup\, \{\Omega(V, \bu)\mid \bu\in
%\bV(0)\}.\] 
it is enough to show that each Voronoi cell is covered by Rogers
simplices.

Let $\bu\in \bV(j)$ for $j<3$.
Consider the following set:
\[  
N = \left\{k\in\ring{N}\mid j\le k\le 3, ~~\Omega(V,\bu) 
= \bigcup_{\bv \in \bV(k),~\trunc{\bv}{j}=\bu}
\op{conv}(O_k \cup\Omega(V,\bv)) %\\&\qquad\qquad
%\mid
%\right\}
\right\},
\] 
where $O_k = \{\omega(\trunc{\bv}{j}),\ldots,\omega(\trunc{\bv}{k-1})\}$.

\claim{We claim $N = \{j,\ldots,3\}$.}  Indeed, to see that $j\in N$, we note that
\[  
\Omega(V,\bu) = \op{conv}(\Omega(V,\bu)),
\] 
which holds by the convexity of the polyhedron $\Omega(V,\bu)$.  We
assume that $k\in N$ and consider the membership condition of $N$ for
$k+1$.  We may assume that $k+1\le 3$.  Then
\begin{alignat*}{4}
&\phantom{=}&&\bigcup _{\bv \in \bV(k+1),~\trunc{\bv}{j}=\bu}
\op{conv}(O_{k+1} \cup\Omega(V,\bv))
\vspace{6pt}\\
&=&&\bigcup _{\bv \in \bV(k+1),~\trunc{\bv}{j}=\bu}
\op{conv}(O_{k} \cup\op{conv}(\{\omega(\trunc{\bv}{k})\}
\cup\Omega(V,\bv)))
\vspace{6pt}\\
&=&&\bigcup _{\bw\in \bV(k),~\trunc{\bw}{j}=\bu,~~}
\bigcup_{\bv \in \bV(k+1),~\trunc{\bv}{k}=\bw}
\op{conv}(O_{k} \cup\op{conv}(\{\omega(\trunc{\bv}{k})\}
\cup\Omega(V,\bv)))
\vspace{6pt}\\
&=&&\bigcup _{\bw\in \bV(k),~\trunc{\bw}{j}=\bu\phantom{+1}}
\op{conv}(O_{k} \cup \Omega(V,\bw)    )
\vspace{6pt}\\
&=&&\quad\Omega(V,\bu).
\end{alignat*}
The induction hypothesis is used in the last step.  
This proves $k+1\in N$, and induction gives $N=\{j,\ldots,3\}$.

Consider the extreme case $j=0$ and $k=3$.  The set $\Omega(V,\bv)$
reduces to $\{\omega(\bv)\}$ and the convex hull becomes
\[  
\op{conv}(O_{k}\cup \Omega(V,\bv)) = R(\bv)
\] 
when $\bv\in \bV(3)$.
This gives
\begin{equation} 
\Omega(V,\u_0) = 
\bigcup \{ R(\bv) \mid \bv\in \bV(3),~\trunc{\bv}{0} =[\u_0]\}.
\end{equation}
This proves the lemma.
\end{proof}


%%%
%\[ \Omega(V,\trunc{\bu}{0}) = 
%\bigcup\, \{ R(\bv) \mid \trunc{\bu}{0}=\trunc{\bv}{0},~\bv\in  \bV(3)\}.
%\] 
%By Lemma~\ref{lemma:webster}, the relative boundary of the bounded
%polyhedron $\Omega(V,\trunc{\bu}{0})$ is the union of its facets.  The
%polyhedron can be partition into the cones over these facets.
%\[ \Omega(V,\trunc{\bu}{0}) = \bigcup\, \left\{
%  ~\op{conv}(\{\omega(\trunc{\bu}{0})\}\cup \Omega(V,\trunc{\bu}{1}))   
%   \mid \trunc{\bu}{1}\in  \bV(1)
%   \right\}.
%\] 
%It is enough to show that
%\[  
%\Omega(V,\trunc{\bu}{1}) = \bigcup\, 
%\{ R(1,\bv) \mid \trunc{\bv}{1}=\trunc{\bu}{1},~\bv\in  \bV(3)\}.
%\] 
%After successively partitioning each facet into cones over facets of
%facets, it is enough to show that
%\[ \Omega(V,\trunc{\bu}{3}) 
%= \bigcup\,\{R(3,\bv)\mid\trunc{\bv}{3}=\trunc{\bu}{3},~\bv\in \bV(3)\}.\] 
%The right-hand side is the singleton $\{\omega(\trunc{\bu}{3})\}$.
% The left-hand side contains this point and is contained in a
% properly decreasing chain of affine sets: $\ring{R}^3$, the bisector
% of $ \u_0$ and $ \u_1$, and so forth.  This determines a unique
% point, so the two sides are equal.
%\end{proof}

\figELMXAFH % fig:rogers-ill-defined

Although the Rogers simplex $R(\bu)$ need not determine the parameter
$\bu$ (Figure~\ref{fig:rogers-ill-defined}), the intersection of two
different Rogers simplices is a null set.

\begin{lemma}[Rogers disjoint]\guid{DUUNHOR}  \label{lemma:R-inter} 
% Revised October 20, 2010.
  Let $V$ be a saturated packing and let $\bu,\bv\in \bV(3)$ be lists
  such that $R(\bu)\ne R(\bv)$.  Then the intersection
\[  
R(\bu)\cap R(\bv)
\] 
is contained in a plane (and hence has measure zero).
\end{lemma}

This result and the previous lemma show that the simplices $R(\bu)$
partition Euclidean three-space.

\begin{proof} 
We may assume that the affine dimension of $R(\bu)$ 
is three, for otherwise $R(\bu)$ is contained in a plane.  Similarly,
we may assume that the affine dimension of $R(\bv)$ is
three.

Let $\bu = [\u_0;\ldots]$ and $\bv = [\v_0;\ldots]$.  
Let $k$ be the
first index such that
\[  
\Omega(V,[\u_0;\ldots;\u_{k}]) \ne \Omega(V,[\v_0;\ldots;\v_{k}]).
%\op{conv}\{\omega( [\u_0]),\ldots,\omega( [\u_0;\ldots; \u_{k}])\}\ne
%\op{conv}\{\omega([\v_0]),\ldots,\omega([\v_0;\ldots;\v_{k}])\}.
\] 

\claim{Such an index $k$ exists.}  Indeed, the definition of points
$\omega(\trunc{\bu}{i})$ depends on $\bu$ only through the sets
$\Omega(V,\trunc{\bu}{j})$.  Hence, $R(\u)\ne R(\v)$ implies that the
two sequences $\Omega(V,\wild )$ must differ at some index.
We have $\omega_i(\bu) = \omega_i(\bv)$, for $i<k$.

Select $\w\in R(\bu)\cap R(\bv)$.  Write
\[
\w = \sum_{j=0}^3 s_j \omega_j(\bu)  = \sum_{j=0}^3 t_j \omega_j(\bv),
%\]
\text{ where } \sum_{j=0}^3 s_j = \sum_{j=0}^3 t_j = 1.
\]
Set $\sigma_i = \sum_{j=i}^3 s_j$.

\claim{We claim that $s_i = t_i$, and $\sigma_{i+1} = \sum_{j={i+1}}^3 t_j$, 
for $i=0,\ldots,k-1$.}  Indeed, the
proof is an induction on $i$.  Assume that the claim holds for all
indices less than $i$ so that
\[
\sum_{j=i}^3 s_j \omega_j(\bu)  = \sum_{j=i}^3 t_j \omega_j(\bv).
\]
%Set $\sigma_i = s_1+s_2+\cdots+s_i$.
We apply Lemma~\ref{lemma:scale2} to the points
\[
\p_0=\omega_i(\bu)=\omega_i(\bv),\quad
\p = \sum^3_{j={i+1}} \frac{s_j}{\sigma_i} \omega_j(\bu),\quad
\p' = \sum^3_{j={i+1}} \frac{t_j}{\sigma_i} \omega_j(\bv)
\]
in the polyhedron $\Omega(V,[\u_0;\ldots;\u_i])$
to obtain the induction step $s_i=t_i$.

Let
\[
\Omega'=\Omega(V,[\u_0;\ldots;\u_{k}]) 
\cap \Omega(V,[\v_0;\ldots;\v_{k}])=
\Omega(V,[\u_0;\ldots;\u_{k};\v_{k}]).
\]
The claim implies that 
\[
\frac{1}{\sigma_k}\sum_{j=k}^3 s_j \omega_j(\bu) = 
\frac{1}{\sigma_k}\sum_{j=k}^3 t_j \omega_j(\bv) \in\Omega'.
\]
It follows that the intersection $R(\u)\cap R(\v)$ lies in the convex
hull $C$ of
\[
\{\omega([\u_0]),\ldots,\omega( [\u_0;\ldots; \u_{k-1}])\}
\]
and $\Omega'$.  The set $\Omega'$ lies in a facet of
$\Omega(V,\trunc{\bu}{k})$.  Hence, the affine dimension of $\Omega'$
is at most $3-k-1=2-k$.  In general, if a set $A$ has affine dimension
$r$, then the affine dimension of $\op{conv}(\{\p\}\cup A)$ is at most
$r+1$.  It follows that the affine dimension of $C$ is at most 
$k + (2-k) = 2$.  The intersection is thus contained in a plane.
\end{proof}


\subsection{circumcenter}

The extreme points of a Rogers simplex are closely related to the
circumcenter of various subsets of $V$.  This subsection develops the
connection between Rogers simplices and circumcenters.

\begin{definition}[circumcenter,~circumradius]\guid{IFLFHKT} 
  \formaldef{circumcenter}{circumcenter}%
  \formaldef{circumradius}{radV}%
  Let $S\subset\ring{R}^n$.  A point $\p$ is a \newterm{circumcenter}
  of $S$ if it is an element in the affine hull of $S$ that is
  equidistant from every $\v\in S$.  If $S$ has circumcenter $\p$,
  then the common distance $\norm{\p}{\v}$ for all $\v\in S$ is the
  \newterm{circumradius} of $S$.
\end{definition}
\indy{Notation}{S@$S$ (finite subset of $\ring{R}^3$)}%

The circumcenter comes as a solution to a system of linear equations.
We pause to review a standard result from the theory of linear
algebra, asserting the existence of a solution to a system of
equations.  Recall that a finite set $S$ is 
\fullterm{affinely~independent}{affine!independence} 
if $\dimaff(S) = \card(S) -1$.  
%
\formaldef{affinely  independent}{\textasciitilde affine\_dependent s}%

\begin{lemma}[linear systems]\guid{QXSKIIT}\label{lemma:affine-system} 
  Let $S=\{\v_0,\ldots,\v_m\}\subset \ring{R}^n$ be an affinely
  independent set of cardinality $m+1$.  Then every system of
  equations
\[  
\p \cdot (\v_i - \v_0) = b_i-b_0,\quad \text{for } i=1,\ldots,m
\] 
has a unique solution  $\p$ that lies in the affine hull of $S$.
\end{lemma}

\begin{proof} This is a standard result from linear algebra.
We sketch a proof for the sake of completeness.  

Let $\w_i = \v_i-\v_0$ and replace $b_i-b_0$ with $b_i$.  
The lemma reduces to the following claim.
Let $S' = \{\w_1,\ldots,\w_m\}$ be a \newterm{linearly independent} set
of cardinality $m$.  Then every system of equations
\[  
\p \cdot \w_i = b_i,\quad \text{for }i=1,\ldots,m
\] 
has a unique solution in $\p$ that lies in the linear span of $S'$.

\claim{A solution is unique}. Indeed, the difference 
$\p = \p'-\p'' = \sum s_i \w_i$ of two solutions
$\p',\p''$ satisfies
\[  
\normo{\p}^2=\p\cdot\p = \sum s_i \w_i \cdot (\p' - \p'') =
\sum s_i (b_i-b_i)= 0.
\] 
It follows that $\p=\orz$ and $\p'=\p''$.  This proves uniqueness.

Let $W$ be the linear span of $\w_1,\ldots,\w_m$.  The image of the
map $W\to\ring{R}^m$, $\p\mapsto (\p\cdot\w_1,\ldots,\p\cdot\w_m)$ is
a linear space and is therefore an affine set.

\claim{A solution exists; that is, the image is all of $\,\ring{R}^m$.}
Otherwise, by Lemma~\ref{lemma:aff-u} some equation must hold; that
is, there exists $\u\ne \orz$ such that $\u\cdot \q =b$ for every
point $\q$ in the image.  As $\orz$ lies in the image, $b=0$.  Write
$\p = \sum u_i \w_i\in W$ and let $\q\in\ring{R}^m$ be the image of
$\p\in W$.  Then
\[  
\normo{\p}^2 = \p\cdot\p = \sum u_i (\p \cdot \w_i) = \u\cdot \q = 0.
\]  
Thus $\p=\orz$ so that $\u=\orz$.  We have reached a contradiction.
\end{proof}

\begin{lemma}[circumcenter exists]\guid{OAPVION} 
  Let $S\subset \ring{R}^n$ be a nonempty affinely independent set.
  Then there exists a unique circumcenter of $S$.
\end{lemma}

\begin{proof} 
  A point $\p$ is a circumcenter if and only if it is a point in the
  affine hull of $S$ that satisfies the system of equations:
\[  
\norm{\p}{\v_i}^2 = \norm{\p}{\v_0}^2,\qquad i = i,\ldots,m.
\] 
Equivalently,
\[  
\p\cdot (\v_i-\v_0) = b_i-b_0,\qquad i=1,\ldots,m,
\] 
where $b_i = \normo{\v_i}^2/2$.  By
Lemma~\ref{lemma:affine-system}, this system of equations has a unique
solution.
\end{proof}

The following lemma describes the structure of the affine hull of a
face of a Voronoi cell.  It describes the affine hull as an
intersection of half-spaces and shows that it meets $\op{aff}(S)$
orthogonally at the circumcenter of $S$.

\begin{lemma}[]\guid{MHFTTZN}\label{lemma:aff-center} 
Let $V$ be a saturated packing and let $k\le 3$.
Let $S = \{\u_0,\ldots,\u_k\}$, where $\bu=[\u_0,\ldots,\u_k]\in \bV(k)$.
Then
\begin{enumerate} 
\item $\dimaff (S)= k$.  
(In particular, $\card\{\u_0,\ldots,\u_k\}=k+1$, and
$S$ is affinely independent.)
\item $\aff{\Omega(V,\bu)}= \cap_{i=1}^k A(\u_0,\u_i).$
\item $\aff{\Omega(V,\bu)} \cap \aff(S) = \{\q\}$, 
where $\q$ is the circumcenter of $S$.
\item $(\aff{\Omega(V,\bu)}-\q) \perp (\aff(S)-\q)$, where
  $X-\q$ denotes the translate of a set $X$ by $-\q$, and $(\perp)$ is
  the orthogonality relation.
\end{enumerate}
\end{lemma}
\indy{Notation}{7@$\perp$}%


\begin{proof}  The proof is by induction on $k$.  

  \claim{The lemma holds when $k=0$.}  Indeed, $\Omega(V,\u_0)$ contains an open
 ball centered at $\u_0$, so its affine hull is $\ring{R}^3$.  This is the
  first conclusion.  The other conclusions reduce to trivial facts:
  $\dimaff\ring{R}^3 = 3$, $\dimaff\{\u_0\}=0$, $\ring{R}^3\cap
  \{\u_0\} = \{\u_0\}$, and $\ring{R}^3\perp \{\orz\}$.

  Assume the induction hypothesis for $k$.  We may assume that $k<3$
  because otherwise there is nothing further to prove.  Let $\bu\in
  \bV(k+1)$.  Let $\bv = \trunc{\bu}{k}\in \bV(k)$.  Let $\q_k$ be the
  circumcenter of (the point set of) $\bv$.  Write 
  $A_j = \cap_{i=1}^j A(\u_0,\u_i)$; $B_j = \aff(\Omega(V,\trunc{\bu}{j}))$; $C_j =
  \aff\{\u_0,\ldots,\u_j\}$; $S_j = \{\u_0,\ldots,\u_j\}$.
By the induction hypothesis $A_k = B_k$.

\claim{We claim $\dimaff S_{k+1} =k+1$.}
Otherwise, by general background facts about affine sets, $\u_{k+1}\in C_k$.
Write $\u_{k+1}-\q_k=\sum_{i\le k} t_i (\u_i-\q_k)$.  If $\p\in A_k$, then
by the orthogonality induction hypothesis:
\begin{align*} 
(\u_{k+1}-\q_k)\cdot (\p-\q_k) &= 
\sum t_i (\u_i-\q_k)\cdot (\p-\q_k) = 0, \intertext{ and consequently }
\norm{\u_{k+1}}{\p}^2 - \norm{\u_0}{\p}^2 &=
\norm{\u_{k+1}}{\q_k}^2 - \norm{\u_0}{\q_k}^2.
\end{align*}
Thus, if $A_k$ meets $A(\u_0,\u_{k+1})$ at some point $\p$, then 
both sides of this equation vanish and $A_k\subset
A(\u_0,\u_{k+1})$.  This is contrary to $0\le \dimaff(A_{k+1}) =
\dimaff(A_k) - 1$, which holds because $\bu\in \bV(k+1)$ with $k<3$.

\claim{We claim that $B_{k+1} = A_{k+1}$.}  Indeed, by definition,
$B_{k+1}\subset A_{k+1}\subset A_k$.  Also,
\[  
\dimaff B_{k+1} = 3 - (k+1) \le \dimaff{A_{k+1}} \le \dimaff A_k = 3 - k.
\] 
Hence, by general background on affine sets, if $A_{k+1}\ne A_k$, then
$B_{k+1}=A_{k+1}$.  Suppose for a contradiction that $A_k =
A_{k+1}$.  Then $\Omega(V,\bv) \subset \Omega(V,\bu) =
\Omega(V,\bv)\cap A(\u_0,\u_{k+1}) \subset \Omega(V,\bv)$, so that
$B_k = B_{k+1}$.  This contradicts the defining conditions of
$\bV(k+1)$.

\claim{We claim that $A_{k+1}\cap C_{k+1} = \{\q_{k+1}\}$.}  Indeed, by the
definition of $A_{k+1}$, any point in this affine set is equidistant
from every point of $S_{k+1}$.  By the definition of $C_{k+1}$, the
intersection lies in the affine hull of $S_{k+1}$.  This uniquely
characterizes the circumcenter.

\claim{Finally, $(A_{k+1} -\q_{k+1})\perp (C_{k+1}-\q_{k+1})$.}
Indeed, if $\p\in A_{k+1}$, then
\begin{align*} 
0 &=\norm{\p}{\u_i}^2 -\norm{\p}{\u_0}^2\\
&=\norm{(\p-\q_{k+1})}{(\u_i-\q_{k+1})}^2 -\norm{(\p-\q_{k+1})}{(\u_0-\q_{k+1})}^2\\
&=-2 (\p-\q_{k+1})\cdot (\u_i-\u_0).
\end{align*}
Since the linear span of the points $\u_i-\u_0$ is all of
$C_{k+1}-\q_{k+1}$, the final claim and the  proof by induction ensue.
\end{proof}

\begin{definition}[h]\guid{CHNGQBD}
\formaldef{h}{hl}% 
%Let $h(\trunc{\bu}{i}) = \norm{\omega(\trunc{\bu}{i})}{ \u_0}$.  
If $\bu=[\u_0;\u_0;\ldots;\u_k]$ is a list of points in $\ring{R}^n$, then
let $h(\bu)$ be the
circumradius of its point set $\{ \u_0,\ldots, \u_k\}$.
\end{definition}
\indy{Notation}{h@$h$ (circumradius)}%

\begin{remark}
The constant $r=\sqrt2$ is the smallest real number $r$ such that
there exist four cocircular points in the plane with pairwise
distances at least $2$ and with circumradius $r$ (Figure~\ref{fig:rogers-sqrt2}).  
The four points are
the vertices of a square of side length $2$.  Eight two-dimensional Rogers simplices
meet at the circumcenter of the square, but when $r<\sqrt2$, only six
Rogers simplices meet at the circumcenter.  In general, at $r=\sqrt2$,
certain degeneracies start to appear in $n$-dimensions that cannot occur for a smaller
radius.  To avoid degeneracies, many lemmas in this section
assume that the circumradius is less than $\sqrt2$.
\end{remark}

\figNOCHOTB % fig:rogers-sqrt2

\begin{lemma}[nondegeneracy]\guid{XYOFCGX}\label{lemma:sqrt2-close} 
  Let $V\subset\ring{R}^3$ be a saturated packing.  Let $S\subset V$
  be an affinely independent set with circumcenter $\p$.  Assume that
  the circumradius of $S$ is less than $\sqrt2$.  Then
  $\norm{\v}{\p}>\norm{\u}{\p}$ for all $\u\in S$ and all $\v\in
  V\setminus S$.
\end{lemma}

\begin{proof} 
Assume for a contradiction that 
there is a point $\w\in V\setminus S$ satisfying
\begin{equation}\label{eqn:closest} 
\norm{\w}{\p}\le \norm{ \u}{\p}, \quad\text{for all }  \u\in S.
\end{equation}
The angles $\arc_V(\p,\{\v, \u\})$ are obtuse for distinct elements
$\v,\u$ of $ S\cup\{\w\}$ because of the law of cosines and
\[  
\norm{\p}{\u} < \sqrt2,\quad \norm{\p}{ \v} <\sqrt2,\quad \norm{\u}{ \v} \ge 2.
\]  
Let $S=\{\u_0,\ldots,\u_k\}$.
A case-by-case argument follows for each $k\in\{0,1,2,3\}$.

\begin{enumerate}
\setcounter{enumi}{-1}
\item 
%\claim{[Case $k=0$].}  
The case $k=0$ is trivial.
\item
%\claim{[Case $k=1$].}  
  In the case $k=1$, the points $\p, \u_0, \u_1$ are collinear and
  cannot give two obtuse angles.
\item
%\claim{[$k=2$].} 
In this case, let $\w'$ be the projection of $\w$ to
the plane containing $\p, \u_0, \u_1, \u_2$.  Under orthogonal
projection, the angles remain obtuse:
\[
(\u_i-\p)\cdot (\w-\p) = (\u_i-\p)\cdot (\w'-\p) <0.
%\arc_V(\p,\{\w,\u_i\}) = \arc_V(\p,\{\w',\u_i\}).
\] 
The four points $\w', \u_0, \u_1$, and $\u_2$ can be arranged
cyclically around $\p$, according to the polar cycle, each forming an
obtuse angle with the next.  A circle around $\p$ cannot give four
obtuse angles because the sum is $2\pi$.
\item
%\claim{[$k=3$].}
In this case, assume that $ \u_0,\ldots, \u_3$ are labeled according to the azimuth
cycle
around the line $\op{aff}\{\p,\w\}$.  Consider the dihedral angle
\[  
\gamma=\gamma_i=\dih(\{\p,\w\},\{ \u_i, \u_{i+1}\})
\] 
of the simplex $\{\p,\w,\u_i,\u_{i+1}\}$ along the edge $\{\p,\w\}$.
By the spherical law of cosines, the angle $\gamma$ of the
spherical triangle with sphere center $\p$ is given in terms of the edges as
\[  
\cos c - \cos a \cos b = \sin a \sin b \cos \gamma.
\] 
The angles $a,b,c$ are obtuse, so that both terms on the left-hand
side are negative. Thus, $\gamma>\pi/2$.  The  angle
$\op{azim}(\p,\w,\u_i,\u_{i+1})$ is then also greater than $\pi/2$ by
Lemma~\ref{lemma:dih-azim}.  This is impossible, as the sum of the
four azimuth angles $\gamma$ is $2\pi$ by Lemma~\ref{lemma:2pi-sum}.
\end{enumerate}
%This completes the proof that $\omega(\trunc{\bu}{j})$
%is the circumcenter.
\end{proof}

With nondegeneracy established, we can now give further details about
the extreme points of a Rogers simplex and their relationship to the
circumcenter of a subset $S$ of the packing $V$.

\begin{lemma}[Rogers simplex and circumcenter]\guid{XNHPWAB}\label{lemma:v2} 
Let $V$ be a saturated packing.
Let $\bu=[\u_0;\ldots;\u_k]\in \bV(k)$ for some $k\le 3$,
and let $S=\{\u_0,\ldots,\u_k\}$ be the
point set of $\bu$.
Assume that $h(\bu)<\sqrt2$.
%\[  
%\norm{\omega(\trunc{\bu}{j}) }{  \u_0} < \sqrt2.
%\] 
Then 
\begin{enumerate} 
\item%[(circumcenter)]  
$\omega(\bu)$ is the circumcenter of $S$.
\item%[(convex hull)]  
$\omega(\bu)\in\op{conv}(S)$.
\item%[(distinctness)]  
The set $\{\omega(\trunc{\bu}{j})\mid j\le k\}$ has affine dimension $k$.
\item
The sequence $h(\trunc{\bu}{j})$ is
strictly increasing in $j$.
\end{enumerate}
\end{lemma}
\indy{Index}{convex hull}%

\begin{proof} The four conclusions of the lemma  are proved
separately.

\begin{enumerate}
\item \claim{We claim that $\omega(\bu)$ is the circumcenter of $S$.}
  Indeed, by definition, if $\bu\in \bV(k)$, then
\[  
\dimaff\Omega(V,[\u_0;\ldots;\u_k]) = 3-k.
\]   
The case $k=0$ of the claim is trivially satisfied.  Assume by
induction the result holds for natural numbers up to $k$.

Now consider the case $k+1$.  Let $\bu\in \bV(k+1)$ and let $S_{k+1}$ be the
point set of $\bu$.  By the induction hypothesis,
$\omega(\trunc{\bu}{k})$ is the circumcenter of the point set of
$\trunc{\bu}{k}$.  Let $\p$ be the point in $A=\aff(\Omega(V,\bu))$
closest to $\omega(\trunc{\bu}{k})$.  By Lemma~\ref{lemma:aff-center},
the circumcenter of $S_{k+1}$ is the point of intersection of orthogonal
affine sets $\aff(S_{k+1})$ and $A$.  Thus, the circumcenter equals the
unique point $\p$ of $A$ closest to the point $\omega(\trunc{\bu}{k})$ in
$\aff(S_{k+1})$.  By Lemma~\ref{lemma:sqrt2-close}, $\p\in\Omega(V,\bu)$.
Thus, $\p=\omega(\bu)$.  The claim ensues.

\item\claim{We claim $\omega(\bu)\in\op{conv}(S)$.}  Otherwise, select
  $\v\in S$ such that $\aff(S')$ separates $\omega(\bu)$ from $ \v$,
  where $S'=S\setminus\{\v\}$.  Let $\p'$ (resp. $\p=\omega(\bu)$) be
  the circumcenter of $S'$ (resp. $S$).  When $\u\in S'$, the law of
  cosines gives
\begin{align*} 
\norm{ \u}{\p}^2 &= \norm{\u}{\p'}^2 + \norm{\p'}{\p}^2\\ 
\norm{ \v}{\p}^2 &\ge \norm{\v}{\p'}^2 + \norm{\p'}{\p}^2.
\end{align*}
This gives $\norm{\v}{\p'}\le \norm{\u}{\p'}$, which is contrary to
Lemma~\ref{lemma:sqrt2-close}.

\item\claim{The set $\{\omega(\trunc{\bu}{j})\mid j\le k\}$ has affine dimension $k$.}
%\claim{The points $\omega(\trunc{\bu}{j})$ for
%$j\le k$, are all distinct.}
  Lemma~\ref{lemma:aff-center} implies that the vectors
  ${\omega(\trunc{\bu}{i+1})}-{\omega(\trunc{\bu}{i})}$ are mutually
  orthogonal.  Thus, the claim about affine dimension easily follows
  if we show that these vectors are nonzero.
Otherwise, the
circumcenter $\omega(\trunc{\bu}{i})$ of $S_i=\{\u_0,\ldots,\u_i\}$
has an equally close point $ \u_{i+1}\in V\setminus S_i$, which is
impossible by Lemma~\ref{lemma:sqrt2-close}.

\item\claim{The sequence $h(\trunc{\bu}{j})$ is strictly increasing in
  $j$.}  
Indeed, by the Pythagorean theorem,
\begin{equation} 
\norm{\omega(\trunc{\bu}{j})}{\omega(\trunc{\bu}{0})}^2 =
\sum_{i=1}^{j} \norm{\omega(\trunc{\bu}{i})}{\omega(\trunc{\bu}{i-1})}^2.
\end{equation}
So the result follows from the
previous claim.
\end{enumerate}
\end{proof}


\begin{lemma}\guid{WAUFCHE}\label{lemma:h-omega}
  Let $V$ be a saturated packing.  Let $\bu =[\u_0;\ldots]\in
  \bV(k)$ for some $k$.  Then $h(\bu)\le
  \norm{\omega(\bu)}{\u_0}$.  Moreover, if $h(\bu)<\sqrt2$, then
  $h(\bu)=\norm{\omega(\bu)}{\u_0}$.
\end{lemma}

\begin{proof} By construction, the point $\omega(\bu)$ belongs to
  $\Omega(V,\bu)$ and is therefore equidistant to the points in
  $S=\{\u_0,\ldots,\u_k\}$.  The orthogonal projection of
  $\omega(\bu)$ to $\op{aff}(S)$ is the circumcenter of $S$.  The
  orthogonal projection cannot increase distances, and the inequality
  ensues.  If $h(\bu)<\sqrt2$, then $\omega(\bu)$ is already the
  circumcenter by Lemma~\ref{lemma:v2}, so that equality holds.
\end{proof}

\begin{lemma}\guid{NJIUTIU}\label{lemma:omega-uv}  %% NEW
Let $V$ be a saturated packing.  Let $\bu,\bv\in \bV(3)$.
Suppose that $R(\bu)=R(\bv)$ and that $\dimaff R(\bu)=3$.
Then $\omega_i(\bu)=\omega_i(\bv)$, for $i=0,1,2,3$.
\end{lemma}

\begin{proof} Let $R=R(\bu)=R(\bv)$.
The set
\[
W =\{\omega_0(\bu),\cdots,\omega_3(\bu)\}
\]
is characterized as the set of extreme points of the simplex $R$.  It
is the same for both $\bu$ and $\bv$.  Since $R\subset
\Omega(V,\u_0)\cap \Omega(V,\v_0)$, and the Rogers simplex $R$ has
full dimension, we must have $\u_0=\v_0$.  Inductively, we may
determine $\omega_i = \omega_i(\bu)=\omega_i(\bv)$ as follows.  The
point $\omega_0$ is $\u_0=\v_0$.  The point $\omega_{i+1}$ is the
closest point of $\op{conv}(W\setminus\{\omega_0,\ldots,\omega_{i}\})$
to $\omega_{i}$.  Note that
$\op{conv}(W\setminus\{\omega_0,\ldots,\omega_{i}\})$ is a subset
containing $\omega_{i+1}$ of the set $\Omega(V,d_{i+1}\bu)$ that is
used to define $\omega_{i+1}(\bu)$
(Definition~\ref{def:omega}).  This description of the points
$\omega_i$ is independent of $\bu\in \bV(3)$ such that $R=R(\bu)$.
\end{proof}

\begin{lemma}\guid{TEZFFSK}\label{lemma:dk-uv}   %% NEW
Let $V$ be a saturated packing.
Let $\bu=[\u_0;\ldots]\in \bV(3)$ be such that $\dimaff R(\bu)=3$.  Select
$k\le 3$  such that $h(d_k\bu)  < \sqrt2$.
% for all $i\le k$.
% \norm{\omega_i(\bu)}{\u_0}
Suppose that $R(\bu)=R(\bv)$, for some $\bv\in \bV(3)$.  Then
\[
d_k\bu = d_k\bv.
\]
\end{lemma}

\begin{proof}
  Write $\omega_i$ for $\omega_i(\bu)$.  By
  Lemma~\ref{lemma:omega-uv}, these points are determined by $R(\bu)$.
  %By Lemma~\ref{lemma:h-omega}, $h(d_k\bu) <\sqrt2$.  
  By
  Lemma~\ref{lemma:v2}, $h(d_i\bu)<\sqrt2$, and $\omega_i$ is the
  circumcenter of $\{\u_0,\ldots,\u_i\}$, for all $i\le k$.

Since $R(\bu)=\op{conv}\{\omega_0,\ldots,\omega_3\}$ has affine
dimension $3$, the points $\omega_0,\ldots,\omega_k$ are affinely
independent.  These circumcenters are constructed as points in the
affine hull of $\{\u_0,\ldots,\u_k\}$.  Hence $\u_0,\ldots,\u_k$ are
also affinely independent.

By Lemma~\ref{lemma:sqrt2-close}, we have the following recursive
description of the points $\u_i$ in terms of $\omega_i$, for $i\le k$.
The point $\u_0$ is $\omega_0$.  The point $\u_{i+1}$ is the unique
$\v\in V\setminus \{\u_0,\ldots,\u_{i}\}$ such that
\[
\norm{\v}{\omega_{i+1}} = \norm{\u_0}{\omega_{i+1}}.
\]
This description of $[\u_0;\ldots;\u_k]=d_k\bu$ depends on $\u$ only
through $R(\bu)$.
\end{proof}

\subsection{Delaunay simplex}

The Delaunay decomposition of space into simplices is dual to the
Voronoi cell.  It is presented as a collection of $k$-simplices with
vertices in $V$, for $k=1,2,3$.  The Delaunay $1$-simplices are
defined as the edges between two points in a packing $V$ whose their
Voronoi cells have a common facet.\footnote{The Delaunay decomposition
  may be degenerate if the points of $V$ are not in general position.
  This book confines itself to the nondegenerate situation.}  A
$2$-simplex is given with vertices at three points in $V$ if their
Voronoi cells have a common edge.  A $3$-simplex is given for every
four points in $V$ whose Voronoi cells have a common extreme point.  A
Delaunay $3$-simplex is the convex hull of four points in the packing
$V$.

Under a nondegeneracy condition (on the circumradius of the set of
points), we may construct a Delaunay simplex as a union of Rogers
simplices.  To this end, we examine the set of all Rogers simplices
around a common extreme point.  The convex hull of a nondegenerate set
$S\subset V$ of four points consists of $4!$ Rogers simplices,  each
facet of the convex hull consists of $3!$ pieces,  and so
forth  (Lemma~\ref{lemma:Rconv}).  
In brief, the Rogers simplices give every nondegenerate
Delaunay simplex an identical simplicial structure.
%  

Recall that $\op{Sym}(k+1)$ is the \newterm{group} of all permutations
on the set $\{0,\ldots,k\}$.  Let $\bu = [\u_0,\ldots, \u_k]$ be a
list of length $k+1$.  For any \newterm{permutation}
$\rho\in\op{Sym}(k+1)$, let $\rho_*(\bu)$ be the \newterm{rearrangement}
given by
\[  
\rho_*(\bu)_i =  \u_{\rho^{-1} i}, % inverse added Feb 16, 2010. left regular action.
\]    
where $\u_i$ denotes the $i$th element of a list $\bu$.
\formaldef{permutation}{permutes}%
\formaldef{$\rho_*$}{left\_action\_list}%
\indy{Notation}{zzr@$\rho$ (permutation)}%
\indy{Notation}{Sym@$\op{Sym}$ (symmetric group)}%

The following lemma shows that rearrangements have the same extreme
point of a Rogers simplex.

\begin{lemma}[extreme point rearrangement]\guid{YIFVQDV}
\label{lemma:perm-Vk} 
  Let $V$ be a saturated packing.  Let $\bu\in \bV(k)$.  Assume that
  $h(\bu)<\sqrt2$. Let $\bv$ be any rearrangement of $\bu$ under a
  permutation.  Then $\bv\in \bV(k)$ and $\omega(\bu) = \omega(\bv)$.
\end{lemma}

\begin{proof} 
Let $\bv = [\v_0;\ldots;\v_k]$.  
Let $S_j = \{\v_0,\ldots,\v_j\}$,  
$\Omega_j = \Omega(V,\trunc{\bv}{j})$, 
$A_j=\cap_{i=1}^j A(\v_0,\v_i)$, and $a_j = \dimaff(A_j)$, for $0\le j\le k$.
By convention, set $A_0 = \ring{R}^3$, so that $a_0=\dimaff(A_0) = 3$.
Also, set $a_{-1} = 4$ by convention.

The set $S_k$ is the point set of $\bu$, which is affinely independent
by Lemma~\ref{lemma:aff-center}.  The set $S_j$ is also affinely
independent.  Let $\p_j$ be the circumcenter of $S_j$.  The
circumradius of $S_j$ is at most the circumradius of $S_k$, which by
assumption is less than $\sqrt2$.

\claim{We claim that $\dimaff \Omega_j = a_j$, when $0\le j\le k$.}
By Lemma~\ref{lemma:sqrt2-close}, if $\p=\p_j$, then
\begin{equation}\label{eqn:sqrt2-close} 
\norm{\v}{\p} > \norm{\u}{\p}\text{ for all }\u\in S_j
\text{ and for all }\v\in V\setminus S_j.
\end{equation}   
Select a small neighborhood $U$ of $\p_j$ such that \eqref{eqn:sqrt2-close} holds
for all $\p\in U_j$.  By the definition of Voronoi cell, $\Omega_j \cap U=A_j\cap U$.
By background facts on affine sets $\dimaff\Omega_j = \dimaff A_j=a_j$.  This gives
the claim.

To prove the lemma, we prove the following claim by simultaneous
induction on $j$.  For all $0\le j\le k$ we have
\begin{align*}
a_j &\ge a_{j-1} - 1\ge 3-j.\\
a_j &= 3-j \text{ if and only if } a_i=3-i \text{ for all } 0\le i\le j.
\end{align*}
The base case $j=0$ is trivial.  Assume the induction hypothesis for $j$.

We have $A_{j+1} = A_{j}\cap A(\v_0,\v_{j+1})$.  The intersection
contains $\p_{j+1}$ and is therefore nonempty.  By general background
facts on the intersection of an affine set with a hyperplane, $a_{j+1}
\ge a_{j}-1$.  By the induction hypothesis, $a_{j}-1\ge 3-(j+1)$.  If
$a_{j+1}=3-(j+1)$, then $a_{j}=3-j$ and by the induction hypothesis
$a_{i}=3-i$ for all $0\le i\le j$. This completes the proof of the
claim by induction.

We have $a_k = \dimaff A_k=\dimaff \Omega_k$.  However, $\Omega_k=
\Omega(V,\bu)$, and since $\bu\in \bV(k)$, it follows that
$3-k=\dimaff\Omega(V,\bu)=a_k$.  By the established claims, $a_i = 3-i$
for all $0\le i\le k$.  This proves $\bv\in\bV(k)$.

Finally, $\omega(\bu) = \omega(\bv)$ because both equal the
circumcenter of the point set $S_k$.
%Since the sets $\Omega(V,\trunc{\bu}{j})$ satisfy \eqref{eqn:omega-dim}, it
%follows that $\Omega(V,\bu)\cap \op{conv}\{ \u_0,\ldots, \u_k\}$ is
%the singleton $\{\omega(\bu)\}$, which contains the circumcenter of
%the simplex with extreme points $\{ \u_0,\ldots, \u_k\}$.  This describes
%$\omega(\bu)$ in a way that does not depend on the ordering of $
%\u_0,\ldots, \u_k$.
%
%The condition \eqref{eqn:omega-dim} can be shown to hold for $\bv$.
%The proof of Lemma~\ref{lemma:v2} shows that the midpoint of $\v_0$
%and $\v_1\}$ lies in $\Omega(V,\trunc{\bv}{1})$.  By the distinctness
%conclusion of the same lemma, some neighborhood of this midpoint in
%the bisecting plane of $\{\v_0,\v_1\}$ lies in $\Omega(V,\trunc{\bv}{1})$.
%Thus, $\dimaff(\Omega(V,\trunc{\bv}{1}))=2$.  We continue in this fashion to show
%that
%\[  
%\dimaff(\Omega(V,\trunc{\bv}{j}))=3-j.\] 
\end{proof}

The next lemma shows that the map from permutations to Rogers
simplices is one-to-one.

\begin{lemma}[permutations one-to-one]\guid{KSOQKWL} 
  Let $V$ be a saturated packing and let $\bu\in \bV(k)$.  Assume that
  $h(\bu)<\sqrt2$.  Let $\rho\in\op{Sym}(k+1)$ such that $R(\bu)=
  R(\rho_*\bu)$.  Then $\rho= I$.
\end{lemma}

\begin{proof} 
  We assume that $\rho\ne I$, write $\bv = \rho_*\bu$,  and prove that
  $R(\bu)\ne R(\bv)$.
  By Lemma~\ref{lemma:v2}, the sets
  $\{\omega(\trunc{\bu}{j})\mid j\le k\}$ and
  $\{\omega(\trunc{\bv}{j})\mid j\le k\}$ are each affinely
  independent of cardinality $k+1$.  By
  Lemma~\ref{lemma:simplex-poly}, these are  sets of extreme
  points of $R(\bu)$ and $R(\bv)$, respectively.  Thus, it is enough
  to show that the sets of extreme points are unequal.

  Let $j$ be the largest index such that
  $\trunc{\bu}{j}=\trunc{\bv}{j}$.  The assumption $\rho\ne I$ implies
  that $j<k$.  Let $\p$ be the circumcenter of
  $\{\u_0,\ldots,\u_{j+1}\}$.  By Lemma~\ref{lemma:sqrt2-close},
\[  
\norm{\u_0}{\p} = \norm{\u_{j+1}}{\p} < \norm{\v_{j+1}}{\p}.
\] 
Thus, $\omega(\trunc{\bu}{j+1}) \ne \omega(\trunc{\bv}{j+1})$.  The result ensues.
\end{proof}

To prepare for Lemma~\ref{lemma:Rconv}, we need a preliminary lemma
that does some index shuffling for us.  It gives an explicit
representatives of the cosets of $\op{Sym}(k+1)$ in $\op{Sym}(k+2)$.


\begin{definition}\guid{TSIVSKG}
\formaldef{$\bu^i$}{DROP}
Let $\bu$ be any list.
For each $i$,  let
$\bu^i = [\u_0;\ldots;\hat\u_i;\ldots]$ be the list that drops the $i$th entry.
\end{definition}

\begin{lemma}[coset representatives]\guid{IVFICRK}
\label{lemma:coset-bijection} 
There is a bijection between the set 
\[  
\{(i,\sigma)\mid 0\le i\le k+1,\quad \sigma\in \op{Sym}(k+1)\}
\] 
and $\op{Sym}(k+2)$ such that for any list $\bu$ of length $k+2$
\[
(\rho_*\bu)_j = \begin{cases} (\sigma_*(\bu^i))_j&0\le j \le k\\
  \bu_i & j=k+1.
\end{cases}
\]
\end{lemma}

\begin{proof} 
The bijection sends $(i,\sigma)$ to the permutation $\rho$, where
\[  
\rho^{-1} j = \begin{cases} 
\sigma^{-1} j, & \sigma^{-1} j<i\\
(\sigma^{-1}j)+1 & \sigma^{-1} j \ge i\\
i& j=k+1.
\end{cases}
\] 
This has the required properties.
\end{proof}

This lemma shows that each (nondegenerate) Delaunay simplex can be
partitioned as a union of Rogers simplices, indexed by the permutation
group (Figure~\ref{fig:rogers-fact}).

\figYAJOTSL % fig:rogers-fact

\begin{lemma}[Delaunay simplex]\guid{WQPRRDY}\label{lemma:Rconv}  
  Let $V$ be a saturated packing and let $\bu = [\u_0;\ldots;\u_k]\in
  \bV(k)$.  Assume that $h(\bu)<\sqrt2$.  
  Then
\[  
\op{conv}\{ \u_0,\ldots, \u_k\} = \bigcup \,\{ R(\rho_*\bu) \mid \rho\in \op{Sym}(k+1)\}.
\] 
\end{lemma}
\indy{Notation}{Sym@$\op{Sym}$ (symmetric group)}%

\begin{proof} The proof is by induction on $k$.  The base case of the
  induction $k=0$ reduces to the trivial assertion: $\op{conv}\{\u_0\}
  = \op{conv}\{\u_0\}$.
%Assume
%the result holds for $k$.


%For each $i$,  let
%$\bu^i = [\u_0;\ldots;\hat\u_i;\ldots;\u_{k+1}]$, which drops the $i$th entry.

  \claim{We claim $\bu^i\in \bV(k)$, when
    $\bu=[\u_0;\ldots;\u_{k+1}]\in \bV(k+1)$.}  Indeed, some
  permutation $\rho\in \op{Sym}(k+2)$ carries $\bu$ to
  $\bv=[\u_0;\ldots;\hat\u_i;\ldots;\u_{k+1};\u_i]$.  By
  Lemma~\ref{lemma:perm-Vk}, $\bv\in \bV(k+1)$, so that $\bu^i =
  \trunc{\bv}{k}\in \bV(k)$.

  By the induction hypothesis
\begin{equation}\label{eqn:sigma} 
\op{conv}(S\setminus\{\u_i\}) = \bigcup \,\{ R(\sigma_*\bu^i) \mid \sigma\in \op{Sym}(k+1)\},
\end{equation}
where $S = \{\u_0,\ldots,\u_{k+1}\}$.  By
Lemma~\ref{lemma:simplex-poly}, the facets of the polyhedron
$\op{conv}(S)$ are the sets $\op{conv}(S\setminus\{\u_i\})$.
Lemma~\ref{lemma:facet-partition} gives the partition
\begin{equation} \label{eqn:convS}
\op{conv}(S) = \bigcup_{i=0}^{k+1} \op{conv}(\{\omega(\bu)\}
\cup \op{conv}(S\setminus\{\u_i\})).
\end{equation}
Substitute the formula \eqref{eqn:sigma} into \eqref{eqn:convS} and
use the bijection of Lemma~\ref{lemma:coset-bijection} to replace the
double union by a single union over $\rho\in \op{Sym}(k+2)$.
Background facts in affine geometry then simplify the expression to
the desired formula.  The proof by induction ensues.
%
%
%Let $L = \{ \u_0,\ldots, \u_k\}$.  The proof is by
%induction on $k$.  When $k=0$, the result is trivial.  Now assume
%$k>0$.
%
%The circumcenter $\omega(\bu)$ of $L$ lies in the convex hull of
%these points.  (See the proof of Lemma~\ref{lemma:v2}.)  Thus, the
%left-hand side is the union of cones:
%\[  
%\op{conv}(W) = \bigcup\,\{ \op{conv}(\omega(\bu),L\setminus \{ \u_i\}\mid i=0,\ldots,k) \}.
%\] 
%The sets $L\setminus \{ \u_i\}$ can be identified with cosets of
%$\op{Sym}(k+1) /\op{Sym}(k)$.  By induction $L\setminus \{ \u_i\}$ is
%the union of $R(\bv)$ as $\bv$ runs over all permutations
%$\op{Sym}(k)$ of $L\setminus \{ \u_i\}$.  The result follows by
%induction.
\end{proof}



In summary of this section, by construction, the Rogers simplices
$R(\bu)$ are compatible with the Voronoi decomposition of space.
Under mild restrictions on the circumradius, they can also by
Lemma~\ref{lemma:Rconv} be reassembled into simplices (the Delaunay
simplices) with extreme points at the centers of the packing.
\indy{Index}{Delaunay|see{decomposition}}%
\indy{Index}{decomposition!Delaunay}%
\indy{Index}{decomposition!Voronoi} %




\section{Cells}

\subsection{definition}

%The definition of $k$-cells,
%conjecture~\ref{conj:m1}, Theorem~\ref{theorem:mk1}, and the method of
%Lemma~\ref{lemma:13-14} are all due to him.  

Marchal~\cite{marchal:2009} has proposed an approach to sphere
packings that gives some improvements to the original proof
in~\cite{Hales:2006:DCG}.  
He gives a partition of space into cells that is a variant of Rogers's partition into
the simplices $R(\bu)$.  The main part of the construction is the
decomposition obtained by truncating Voronoi cells with a ball of
radius $\sqrt2$.  In a few carefully chosen situations, the simplices
$R(\bu)$ are assembled into larger convex cells (Delaunay cells), as
suggested by Lemma~\ref{lemma:Rconv}.

\indy{Index}{Marchal,  C.}%
\indy{Index}{decomposition!Marchal}%


\begin{definition}[Marchal cells]\guid{QEEHXUB} 
  \formaldef{$i$-cell}{mcell} \formaldef{$\xi$}{mxi}%
  \indy{Index}{cell}%
  \indy{Index}{Marchal cell}%
% June 11, 2011. 
% Changed sqrt < h to sqrt2 <= h in 0-,1-,2-cells.  (This makes it more memorable.)
% It adds some null sets, but they don't matter anyway.
  Let $V$ be a saturated packing.  Let 
\[
\bu=[\u_0;\ldots;\u_3]\in
  \bV(3).
\]
  Define $\xi(\bu)$ as follows.  If $\sqrt2\le
  h(\trunc{\bu}{2})$, then let $\xi(\bu)=\omega(d_2\bu)$.  If
  $h(\trunc{\bu}{2})<\sqrt2\le h(\bu)$, define $\xi(\bu)$
  to be the unique point in
\[
\op{conv}\{\omega(d_2\bu),\omega(\bu)\}
\]
at distance $\sqrt2$ from $\u_0$.  
Set $a(\bu)={h(\trunc{\bu}{1})}/{\sqrt2}$.
%\[  
%h(\trunc{\bu}{2}) <\sqrt2 \le h(\bu).
%\] 
%When this inequality holds, there is a unique point $\xi(\bu)$ in
%$\op{conv}\{\omega(\trunc{\bu}{2}),\omega(\bu)\}$ at distance exactly
%$\sqrt2$ from $ \u_0$.  
%\end{definition}
A set $\cell(\bu,i)\subset\ring{R}^3$ is associated with $\bu$ and
$i=0,1,2,3,4$.  \hfill\break\smallskip
\begin{enumerate}
\setcounter{enumi}{-1}
\item %[~0]
The $0$-cell of $\bu$ is defined to be empty unless $\sqrt2 \le h(\bu)$.
If this inequality holds, then the $0$-cell is
\[  
\cell(\bu,0) = R(\bu)\setminus B(\u_0,\sqrt2).%
%\{\p\in R(\bu) \mid \norm{\p}{\u_0} \ge \sqrt2\}.
\] 
\bigskip
\item The $1$-cell of $\bu$ is defined to be empty unless $\sqrt2 \le
  h(\bu)$.  If this inequality holds, then the $1$-cell is
\[  
\cell(\bu,1) = (R(\bu) \cap  \bar B(\u_0,\sqrt2))
\setminus \op{rcone}^0(\u_0,\u_1,a(\bu)),
\] 
(The set $\op{rcone}^0(\u_0,\u_1,a)$ is empty when $a>1$.)
%\op{conv}(\{\omega([ \u_0])\}\cup R_1),\hbox{ where } 
%R_1 = \{\p \in R(\bu) \mid \norm{\omega([ \u_0])}{\p}= \sqrt2\}.
%\] 
%\indy{Notation}{R@$R_1$ (Rogers simplex)}%
\bigskip
\item The $2$-cell of $\bu$ is defined to be empty unless
  $h(\trunc{\bu}{1})<\sqrt2\le h(\bu)$.  If this inequality holds,
  then the $2$-cell is (with $a=a(\bu)$ as above)
\begin{align*} 
\cell(\bu,2) &= 
 \op{rcone}(\u_0,\u_1,a)\cap \op{rcone}(\u_1,\u_0,a)\cap 
\op{aff}_+(\{\u_0,\u_1\},\{\xi(\bu),\omega(\bu)\}).
%\op{conv}( \{\u_0, \u_1\}\cup R_2),\quad\text{where }  \vspace{6pt}\\
%R_2 &= \{\p \in R(\bu)\cap \Omega(V,\trunc{\bu}{1}) \mid 
%\norm{ \u_0}{\p}=\norm{ \u_1}{\p} =\sqrt2\}.
\end{align*}
\bigskip
\item
The $3$-cell of $\bu$ is defined to be empty unless 
$h(\trunc{\bu}{2}) <\sqrt2 \le h(\bu)$.
If this inequality holds, then $\xi(\bu)\in \op{conv}
\{\omega(\trunc{\bu}{2}),\omega(\bu)\}$
%\formaldef{$\xi$}{mxi}%
and  the $3$-cell is
\[  
\cell(\bu,3) = \op{conv}\{ \u_0, \u_1, \u_2,\xi(\bu)\}.
\] 
\bigskip
\item
The $4$-cell of $\bu$ is defined to be empty unless
$h(\bu) <\sqrt2$.
If this inequality holds, the $4$-cell is
\[  
\cell(\bu,4) = \op{conv}\{ \u_0, \u_1, \u_2, \u_3\}.
\] 
\end{enumerate}
\end{definition}
\indy{Notation}{zzo@$\xi$ (cell parameter)}%

\figKVIVUOT % fig:marchal-variety

The $0$- and $1$-cells are subsets of a Rogers simplex $R$
(Figure~\ref{fig:marchal-variety}).  Yet, the $2$-, $3$-, and
$4$-cells lie in a union of simplices.  The index $i$ in
$\cell(\bu,i)$ indicates the number of points of $V$ that are extreme
points of the cell (Figures~\ref{fig:marchal-variety} and
\ref{fig:marchal-2d}).

\figBWEYURN % fig:marchal-2d

\subsection{informal discussion}

A \newterm{cell}, short for Marchal cell,
can be described in an alternative intuitive way.
If $S\subset\ring{R}^3$, let
\[
\op{equi}(S,r) = \{ \p \mid \norm{\p}{\v} = r \text{ for all } \v \in S \}.
\]
If $S$ is a finite set of cardinality $k$, of affine dimension $k-1$, and
with circumradius less than $r$, then
$\op{equi}(S,r)$ is a sphere of dimension $3-k$.  In particular,
if $k=3$, then $\op{equi}(S,r)$ is a set of two points.
\indy{Notation}{equi@equi (intersection of spheres)}%
\indy{Notation}{C@$C(S)$ (cell-like subset of $\ring{R}^3$)}%

Let $V$ be a saturated packing.  Define
\[
C(S) = \op{conv} (S\cup\op{equi}(S, \sqrt2) )
\]
for $S\subset V$.  The set $C(S)$ is empty if the circumradius of $S$
is greater than $\sqrt2$.

The set $C(\emptyset)$ is $\ring{R}^3$.  The set $C(\{\w\})$ is a ball
of radius $\sqrt2$ with center $\w$.  The set $C(\{\v,\w\})$ is a
double cone, $C(\{\u,\v,\w\})$ a bipyramid, and $C(\{{\mathbf
  t},\u,\v,\w\})$ is a simplex.

\begin{lemma}\guid{VXIQEJC}
Let $V$ be a saturated packing.
If $S\subset V$ is not empty, then $C(S)$ is contained in the union of sets
\[
C(S\setminus \{\v\}),  \quad \v\in S.
\]
\end{lemma}

\begin{lemma}\guid{WHCFBMJ}
Let $V$ be a saturated packing and let $S\subset V$. 
Set
\[C'(S) = C(S) \setminus \bigcup_{S\subset S'\subset V} C(S'),\] where
$S'\subset V$ runs over subsets of cardinality $k=\op{card}(S)+1$ that
contain $S$.  Then $C'(S)$ equals a union of $k$-cells,
up to a null set.
\end{lemma}

In other words, up to a null set, the union of $0$-cells is the set of points
outside the balls $C(\{\v\})$, for $\v\in V$.  The union of the $1$-cells
is the set of points inside the balls $C(\{\v\})$ but outside the double cones
$C(\{\u,\w\})$, and so forth.

It is possible to base a construction of cells on this lemma,
dispensing entirely with Voronoi cells and Rogers simplices.  It is
quick and intuitive.  We have followed a longer path that gives more
detail about the structure of cells.

\bigskip

At first, the definition of  cells seems unmotivated.  Some
history might help.  The 1998 proof of the Kepler conjecture
partitioned space into a hybrid of truncated Voronoi cells and Delaunay-like
simplices (Figure~\ref{fig:ferguson-hales}).  
In vague terms, the Delaunay simplices are tuned for
detail.  The Voronoi cells are coarsely tuned, suitable for rough
hewing.  Delaunay simplices articulate the foreground, while Voronoi
cells fill the background.  The solution to the problem lies in the
right balance between foreground and background.  Too many Delaunay
simplices and the details overwhelm.  Too many Voronoi cells and the
estimates become too weak.  The central geometrical insights of the
original proof are expressed as rules that delineate foreground
against background, Delaunay against Voronoi.

\figFIFJALK % fig:ferguson-hales

Cells give a hybrid decomposition.  A $4$-cell is a Delaunay simplex.
The $0$- and $1$-cells are parts of a Voronoi cell.  The $2$- and
$3$-cells are gradations between the two.

Examples show the shortcomings of a nonhybrid approach.  Recall that
the density of the face centered cubic packing is
$\pi/\nsqrt{18}\approx 0.74048$.  Numerical evidence shows that an
approach based entirely on Delaunay simplices should give a bound of
about $0.740873$, a failure that comes tantalizingly
close~\cite{Hales:1992:JCAM}.  The dodecahedral theorem, which asserts
that the Voronoi cell of smallest volume in the regular dodecahedron,
gives the bound of about $0.755$ ~\cite{Hales:2010:Dodec}.  Thus, the
pure Voronoi cell strategy fails as well.  The pure approaches can be
modified in ways that are conjectured to produce sharp bounds. These
modifications are complex and daunting.
% constant 0.740873 checked 4/2010.
% constant 0.755 cross-checked with dodec.

A common practice that started with L. Fejes T\'oth is to truncate
 Voronoi cells by
intersecting them with a ball concentric with the cell.  
Different authors use different radii for the
truncating sphere: $7/\nsqrt{27}\approx 1.347$ \cite{Fej53},
$\sqrt2$, $1.385$, and $1.255$ \cite{Hales:2006:DCG},  $\sqrt2$ \cite{marchal:2009},
and $\sqrt{3}\tan{\pi/5}\approx 1.258$ \cite{Hales:2010:Dodec}.  A larger radius
retains more information and complexity than a smaller radius.
The $0$-cells are the refuse that lie outside the ball of
truncation and are inconsequential to the proof.


\bigskip
\subsection{cell partition}

\begin{lemma}[]\guid{EMNWUUS}\label{lemma:M-complement4} 
\formalauthor{Vu Khac Ky}\oldrating{100}
Let $V$ be a saturated packing.  Let $\bu\in \bV(3)$.
The following  are equivalent.
\begin{enumerate} 
\item  $\cell(\bu,i)=\emptyset$ for $i=0,1,2,3$.
\item  $\cell(\bu,4)\ne\emptyset$.
\item  $h(\bu)<\sqrt2$.
\end{enumerate}
%Moreover, $h(\bu)<\sqrt2$ implies these conditions, and these conditions
%imply $h(\bu)\le\sqrt2$.
\end{lemma}

\begin{proof} 
  The diameter of $R(\bu)$ is easily seen to be $h(\bu)$.  Hence, if
  $h(\bu)<\sqrt2$ all of the defining conditions are empty for
  $\cell(\bu,i)$ for $i<4$.  The result ensues.
\end{proof}

\begin{lemma}[]\guid{SLTSTLO}\label{lemma:M-exhaust} 
  Let $V$ be a saturated packing and let $\bu\in \bV(3)$. Then
%there exists a null set $Z$ such that 
  every point in $R(\bu)$ belongs to $\cell(\bu,i)$ for some $0\le
  i\le 4$.  Furthermore, there is a null set $Z$ such that each point
  in $R(\bu)\setminus Z$ belongs to a unique $\cell(\bu,i)$.
\end{lemma}

\begin{proof} 
  Explicitly, the null set is the union of $R(\bu)\setminus R^0(\bu)$
  (which lies in a finite union of planes), the sphere of radius
  $\sqrt2$ at $\u_0$, the difference $\op{rcone}(\u_0,\u_1,a)\setminus
  \op{rcone}^0(\u_0,\u_1,a)$, and the plane
  $\op{aff}\{\u_0,\u_1,\xi(\bu)\}$.  Let $\p\in R(\bu)$.  To make the
  cases disjoint, each of the following cases assumes that the
  conditions of preceding cases fail.  It is convenient to reorder the
  cases to make the $4$-cell appears first.
\begin{enumerate}
\setcounter{enumi}{3}
\item %[~4]
If $h(\bu)<\sqrt2$,  then $\p\in\cell(\bu,4)$.

\setcounter{enumi}{-1}
\item %[~0]\setcounter{enumi}{0}
If $\norm{\p}{\u_0} \ge\sqrt2$,  then $\p\in\cell(\bu,0)$.

\item
If $\p\not\in\op{rcone}^0(\u_0,\u_1,h(\trunc{\bu}{1})/\nsqrt2)$, then
 $\p\in \cell(\bu,1)$.

\item
If $\p\in \op{aff}_+(\{\u_0,\u_1\},\{\xi(\bu),\omega(\bu)\})$, then
 $\p\in \cell(\bu,2)$.

\item If $\p\in \op{aff}_+(\{\u_0,\u_1\},\{\u_2,\xi(\bu)\})$, then
  $\p\in \cell(\bu,3)$.
\end{enumerate}
When the corresponding strict inequalities are used, we obtain
uniqueness for $R(\bu)\setminus Z$.
\end{proof}



\begin{definition}[$i$-rearrangement]\guid{BGXEVQU} 
  \formaldef{$i$-rearrangement}{\_} Let
  $\bu=[\u_0;\ldots;\u_k],\bv=[\v_0;\ldots;\v_k]$ be two lists of the
  same length.  One is an $i$-\newterm{rearrangement} of the other if
  $\rho_*\bu = \bv$ for some $\rho\in\op{Sym}(k+1)$ such that $\rho(j) =
  j$ when $j \ge i$.
\end{definition}

In particular, if $\bu,\bv$ are $0$- or $1$-rearrangements of one
another, then $\bu = \bv$.  The constraint $\rho(j)=j$ is vacuous when
$j>k$.


% old version. Changed Oct 17, 2011.
%\begin{lemma}[]\guid{RVFXZBU}\label{lemma:marchal-equal} 
%Let $V$ be a saturated packing, 
%let $\bu,\bv\in \bV(3)$, and let $i,j\in \{0,1,2,3,4\}$.
%If the intersection of
%$\cell(\bu,i)$ with $\cell(\bv,j)$ has positive measure,
%then $i=j$ and $\bu$ is an $i$-rearrangement of $\bv$.
%Conversely, if $i=j$ and $\bu$ is an $i$-rearrangement of $\bv$, 
%then $\cell(\bu,i)=\cell(\bv,j)$.
%\end{lemma}

\begin{lemma}[]\guid{YNHYJIT}\label{lemma:i-omega} % NEW
  Let $V$ be a saturated packing, let $\bu\in \bV(3)$, and let
  $i\in\{2,3,4\}$.  Assume that $h(d_{i-1}\bu)<\sqrt2$.  Let $\bv$ be
  an $i$-rearrangement of $\bu$.  Then $\bv\in \bV(3)$ and
$\omega_j(\bu)=\omega_j(\bv)$,
  for $j=i-1,\ldots,3$.
\end{lemma}

\begin{proof}
Let $S_j =\{\u_0,\ldots,\u_{j}\}$, for $j\ge i-1$, where
  $\bu=[\u_0;\ldots]$.  Since $\bv=[\v_0;\ldots]$ is an
  $i$-rearrangement of $\bu$, we have $S_j =\{\v_0,\ldots,\v_{j}\}$ and
  $\Omega(V,d_j\bu) = \Omega(V,d_j\bv)$,
  %, and $h(d_{j}\bu) = h(d_{j}\bv)$, 
for all $j\ge i-1$. 
   By Lemma~\ref{lemma:perm-Vk},
  $\omega_{i-1}(\bu) = \omega_{i-1}(\bv)$.  By the recursive
  definition of the points $\omega_j$, we then have
  $\omega_j(\bu)=\omega_j(\bv)$, for $j= i-1,\ldots,3$.

  We show that $\bv\in \bV(3)$ by checking the defining condition
\[
\dimaff \Omega(V,d_j\bv) = 3 - j, \text{ for } 0 < j \le 3.
\]
When $j\le i-1$, the conclusion $d_{i-1}\bv\in \bV(i-1)$ of
Lemma~\ref{lemma:perm-Vk} implies the condition.  When $i-1<j$, the
identity $\Omega(V,d_J\bv)=\Omega(V,d_j\bu)$ and $\bu\in \bV(3)$ imply
the condition.  The result ensues.
\end{proof}

\begin{lemma}[]\guid{RVFXZBU}\label{lemma:marchal-equal} 
Let $V$ be a saturated packing, 
let $\bu,\bv\in \bV(3)$, and let $i\in \{0,1,2,3,4\}$.
If $\bu$ is an $i$-rearrangement of $\bv$, 
then $\cell(\bu,i)=\cell(\bv,i)$.
\end{lemma}

\begin{proof} 
The  statement follows from the definition of cells.
%Let $\cell(\bu,i)$ and $\cell(\bv,j)$ be two cells that have an
%intersection $C$ of positive measure.  There exists $\bw\in \bV(3)$
%such that $R(\bw)\cap C$ has positive measure.  In particular,
%$R(\bw)$ has affine dimension three.  There are finitely many
%$R(\bw')$ for $\bw'\in \bV(3)$ that meet $R(\bw)$.  By
%Lemma~\ref{lemma:R-inter}, by avoiding finitely many planes (null
%sets), there exists a subset $C'$ of positive measure in $C\cap
%R(\bw)$ such that $R(\bw)$ is the unique Rogers simplex that meets
%$C'$.  Furthermore, by Lemma~\ref{lemma:M-exhaust}, there is a subset
%$C''$ of positive measure in $C'$ and a natural number $k$ such that
%$\cell(\bw,k)$ is the unique cell meeting $C''$.
%
%Now $\cell(\bu,i)$, which contains $C''$, is contained in the union of
%the sets $R(\bu')$ as $\bu'$ runs over the $i$-rearrangements of
%$\bu$.  Hence, $\bw$ is an $i$-rearrangement of $\bu$.  By the
%converse statement, replacing $\bu$ with an $i$-rearrangement, we may
%assume that $\bu=\bw$.  Similarly, we may assume that $\bv=\bw$.  By
%the exclusion of null sets as above, $i=j=k$.
\end{proof}

\begin{lemma}[]\guid{QZKSYKG}\label{lemma:cell-in-rogers} %% NEW
  Let $V$ be a saturated packing, let $\bu\in \bV(3)$, and $k\in
  \{0,1,2,3,4\}$.  Assume that $\cell(\bu,k)$ is not empty.
  Then  each $k$-rearrangement $\bv$ of $\bu$ lies in $\bV(3)$.
  Moreover,
  $\cell(\bu,k)$ is contained in the union of
  $R(\bv)$, as $\bv$ runs over all $k$-rearrangements of $\bu$.
\end{lemma}

\begin{proof}  If $k=0$ or $k=1$, 
then by definition, $\cell(\bu,k)\subset R(\bu)$.

Assume that  $2\le k\le 4$.  The nonemptiness hypothesis
implies $h(d_{k-1}\bu)<\sqrt2$.  Lemma~\ref{lemma:i-omega} implies
that $\bv\in \bV(3)$.

The definition of the cells can be used to show directly that
\[
\cell(\bu,k)\subset 
\op{conv}\{\u_0,\u_1,\ldots,\u_{k-1},\omega_{k}(\bu),\ldots,
\omega_3(\bu)\}.
\]
The definition of $k$-rearrangement, Lemma~\ref{lemma:Rconv}, and
Lemma~\ref{lemma:i-omega} partition this convex hull according to
$k$-rearrangements of $\bu$.  The result ensues.
\end{proof}

\begin{lemma}[]\guid{DDZUPHJ}\label{lemma:cell-disjoint} %% NEW
Let $V$ be a saturated packing, let $\bu,\bv\in \bV(3)$, and let
$k\in \{0,1,2,3,4\}$.  Suppose that $R(\bu)=R(\bv)$, that $R(\bu)$
has affine dimension three,  and that
$\cell(\bu,k)$ is not empty.   Then
$\cell(\bu,k)=\cell(\bv,k)$.
\end{lemma}

\begin{proof} 
We break the proof into cases, according to $k$.
Assume $k=4$.  By the definition of the cell, the nonemptiness condition
implies that $h(\bu)<\sqrt2$.  By Lemma~\ref{lemma:h-omega} and Lemma~\ref{lemma:dk-uv},  $\bu=\bv$.

Assume that $k=3$.  The nonemptiness condition gives
$h(d_2\bu)<\sqrt2\le h(\bu)$.  By Lemma~\ref{lemma:h-omega} and
Lemma~\ref{lemma:dk-uv}, $d_2\bu=d_2\bv$.  The point $\omega(\bu)$ is
determined by $R(\u)$ by Lemma~\ref{lemma:omega-uv}.  The point
$\xi(\bu)$ is determined by $\omega(d_2\bu)$ and $\omega(\bu)$.
Finally, $\cell(\bu,3)$ is determined by $d_2\bu$ and $\xi(\bu)$.  The
conclusion $\cell(\bu,3)=\cell(\bv,3)$ ensues.

The other cases are similar.  Assume that $k=2$ and that
$\cell(\bu,2)$ is not empty.  If $h(d_2\bu)<\sqrt2$, then
$d_2\bu=d_2\bv$, and the cell is determined by $d_2\bu$ and
$\xi(\bu)$, as in the case $k=3$.  If $h(d_1\bu)<\sqrt2\le h(d_2\bu)$,
then $d_1\bu=d_1\bv$, and the cell is determined by $d_1\bu$ and the
points $\omega_j(\bu)$, which in turn depend only on $R(\bu)$, by
Lemma~\ref{lemma:omega-uv}.

The cases $k=0$ and $k=1$ are similar, but even more trivial, and are
left to the reader.
\end{proof}

The following lemma and Lemma~\ref{lemma:M-exhaust} 
show that the cells
partition $\ring{R}^3$.

\begin{lemma}[]\guid{AJRIPQN}\label{lemma:marchal-partition} %% NEW
Let $V$ be a saturated packing, let $\bu,\bv\in \bV(3)$, and let
$k,k'\in \{0,1,2,3,4\}$.  Suppose that $\cell(\bu,k)\cap\cell(\bv,k')$
has positive measure.  Then $k=k'$ and $\cell(\bu,k)=\cell(\bv,k)$.
\end{lemma}

\begin{proof}  Select $\bw\in \bV(3)$ such that 
\[
R(\bw)\cap \cell(\bu,k)\cap\cell(\bv,k')
\]
has positive measure.  In particular, $R(\bw)$ has affine dimension
three.  By Lemmas~\ref{lemma:cell-in-rogers}
and~\ref{lemma:marchal-equal} and \ref{lemma:R-inter}, we may replace $\bu$ with
a $k$-rearrangement, and $\bv$ with a $k'$-rearrangement to assume
without loss of generality that $R(\bw)=R(\bu)=R(\bv)$.

Lemma~\ref{lemma:cell-disjoint} implies that
$\cell(\bu,k)=\cell(\bw,k)$ and $\cell(\bv,k')=\cell(\bw,k')$.  Since
$R(\w)\cap\cell(\bw,k')\cap\cell(\bw,k)$ has positive measure, 
Lemma~\ref{lemma:M-exhaust} implies that $k=k'$.  The result ensues.
\end{proof}

\subsection{edges of cells}

\begin{definition}\guid{LEPJBDJ}
\formaldef{$V(X)$}{VX}
  Let $V$ be a saturated packing and let $\bu=[\u_0;\ldots]\in
  \bV(3)$.  Let $X=\cell(\bu,k)$.  When $X\ne\emptyset$, define $V(X) =
  \{\u_0,\ldots,\u_{k-1}\}$.  In particular, if $X$ is a $0$-cell
  $V(X)=\emptyset$.  
\end{definition}
\indy{Notation}{V@$V(X)=V\cap X$}%

\begin{lemma}[]\guid{HDTFNFZ}\label{lemma:VX}
Let $V$ be a saturated packing and let $\bu=[\u_0;\ldots]\in
  \bV(3)$.  Let $X=\cell(\bu,k)$.   If $X\ne\emptyset$, 
then $V(X) = V\cap X$.
In particular, the set $V(X)$ is well-defined.
\end{lemma}

\begin{proof}  
  If $i\le k-1$, then $\u_i\in V$ and $X=\cell(\bv,k)$, for some
  $k$-rearrangement $\bv=[\u_i;\ldots]$ of $\bu$.  By the definition
  of $\cell(\bv,k)$, we find that $\v_0=\u_i$ belongs to
  $\cell(\bv,k)$, when $k\ge 1$.  This implies that $V(X)\subset
  V\cap X$.

Conversely, let $\v\in V\cap \cell(\bu,k)$.  It can be checked from
definitions that $\cell(\bu,0)\subset \Omega(V,\u_0)$ and
\[
\cell(\bu,k)\subset \Omega(V,\u_0)\cup \cdots \cup\Omega(V,\u_{k-1}), \quad 
\text{when~} k\ge1.
\]
This implies $\v\in V\cap\Omega(V,\u_i)$ for some $i\le k$.  This
forces $\v=\u_i$.  Hence $V(X) = V\cap X$.
\end{proof}

\begin{lemma}\guid{URRPHBZ}\label{lemma:cell-radial}
  Let $V$ be a saturated packing and let $\bu\in \bV(3)$.  Then
  $X=\op{cell}(\bu,k)$ is measurable and eventually radial at each
  $\v\in V$.  Furthermore, the cell $X$ is bounded away from every
  $v\in V\setminus V(X)$, so that the solid angle of $X$ is zero, except
  at $\v\in V(X)$.
\end{lemma}

\begin{proof} The first claim of the
 lemma follows from the fact that $R(\bu)$ is a
  simplex, and $R(\bu)\cap V = \{\u_0\}$.
Each cell is compact, and is bounded away from every point not in
the cell. Lemma~\ref{lemma:VX} implies the second claim of the lemma.
\end{proof}


%\begin{definition}\guid{ENBYLCBxo}
%Let $V$ be a saturated packing and let $X$ be a $k$-cell.
%Write $\op{solc}(\bu,k)$ for $\sol(\op{cell}(\bu,k),\u_0)$, where
%$\bu = [\u_0;\ldots]\in \bV(3)$.
%\end{definition}

\begin{lemma}\guid{QZYZMJC}
Let $V$ be a saturated packing.  For every $\v\in V$, 
\[
\sum_{X\mid \v\in V(X)}  \op{sol}(X,\v) = 4\pi,
\]
where the sum runs over all cells $X$ such that $\v\in V(X)$.
\end{lemma}

\begin{proof} Indeed, the cells partition $\ring{R}^3$ and
  $\sol(B(\v,\epsilon)) = 4\pi$.
\end{proof}

\begin{definition}[$\op{tsol}$]\guid{LZYLTFY} 
\formaldef{$\op{tsol}$}{total\_solid}
  %Let $V\cap X$ be the intersection of $ V$ with the set of extreme
  %points of the $k$-cell $X$.  Explicitly, $V\cap X=\emptyset$ if $k=0$;
  %and $V\cap X = \{ \u_0,\ldots, \u_{k-1}\}$ in general.  Each $k$-cell
  %is measurable and eventually radial at each $\u\in V\cap X$.  
  Define
  the \newterm{total solid angle} of a cell $X$ to be
\[  
\op{tsol}(X) = \sum_{\v \in V(X) } \op{sol}(X,\v).
\] 
\end{definition}
\indy{Index}{angle!total solid}%
\indy{Index}{extreme point}%
\indy{Notation}{tsol@$\op{tsol}$}%

% By Lemma~\ref{lemma:cell-radial}, the only terms that contribute to
% the sum are $\u\in V(X)$.

\begin{definition}[edge]\guid{WYORUNK}
  \formaldef{$E(X)$}{edgeX} 
  Let $E(X)$ be the set of \newterm{extremal edges} of the $k$-cell
  $X$ in a saturated packing $V$.  More precisely, let
\[ E(X)=\{\{ \u_i, \u_j\}\mid \u_i\ne \u_j\in
V( X)\}.\] 
\end{definition}
\indy{Notation}{E3a@$E(X)$}%
\indy{Notation}{0@$\tbinom{n}{k}$ (binomial coefficient)}%

In particular, $E(X)$ is empty for $0$ and $1$-cells and contains
$\tbinom{k}{2}$ pairs when $2\le k\le 4$.

\begin{definition}[$\op{dih}$]\guid{RSDYMHV} \label{def:dihX}
  \formaldef{$\dih$}{dihX} 
Let $V$ be a saturated packing.  Let $X$ be
  a $k$-cell, where $2\le k\le 4$.  Let $\ee\in E(X)$.  We define the
  dihedral angle $\op{dih}(X,\ee)$ of $X$ along $\ee$ as follows.
  Explicitly, if $X$ is a null set, then set
  $\op{dih}(X,\ee)=0$. Otherwise, choose $\bu=[\u_0;\u_1;\u_2;\u_3]\in
  \bV(3)$ such that $X=\cell(\bu,k)$ and $\ee=\{\u_0,\u_1\}$.  Set
  $\op{dih}(X,\ee)=\dih_V(\{\u_0,\u_1\},\{\v,\w\})$, where
\[
\{\v,\w\} = 
\begin{cases}
  \{\xi(\bu),\omega(\bu) \} &  k=2\\
  \{\u_2,\xi(\bu)\} & k=3\\
  \{\u_2,\u_3\} &k=4.
\end{cases}
\]
This is independent of the choice $\bu$ defining $X$.
\indy{Notation}{dih}%
\indy{Index}{angle!dihedral}%
\indy{Notation}{h@$h$ (half-edge length)}%
\end{definition}

Each
edge $\ee=\{ \u, \v\}\in E(X)$ has a half-length
$h(\ee) = \norm{ \u}{ \v}/2$.
This definition of $h$ is compatible with the previous definition of the circumradius
of lists in the sense that
$h([\u;\v]) = h(\ee)$.

\begin{lemma}\guid{GRUTOTI}
  Let $V$ be a saturated packing.  Assume that $\u_0,\u_1\in V$
  satisfy $\norm{\u_0}{\u_1}<2\nsqrt2$.  Set $\ee=\{\u_0,\u_1\}$.  Then
\[
\sum_{X\mid \ee\in E(X) } \op{dih}(X,\ee) = 2\pi.
\]
The sum runs over cells $X$ such that $\ee\in E(X)$.
\end{lemma}

\begin{proof} 
  Consider the set $C=B(\u_0,r)\cap \op{rcone}^0(\u_0,\u_1,a)$, where
  $r$ and $a$ are small positive real numbers.  From the definition of
  $k$-cells, it follows that we can choose $r$ and $a$ sufficiently
  small so that if $X$ is a $k$-cell that meets $C$ in a set of
  positive measure, then $k\ge 2$ and there exists $\bu\in \bV(3)$
  such that $X=\cell(\bu,k)$ and $d_1\bu=[\u_0;\u_1]$.  Moreover,
\[
C\cap X = C\cap A, \quad A=\op{aff}_+(\{\u_0,\u_1\},\{\v,\w\}),
\]
where $A$ is the lune of Definition~\ref{def:lune} and $\v$, $\w$ are
chosen as in Definition~\ref{def:dihX}.  By
Lemma~\ref{lemma:wedge-sol} and Definition~\ref{def:dihX}, the volume
of this intersection is
\[
\op{vol}(C\cap A) = \op{vol}(C)\,
 {\op{dih}_V(\{\u_0,\u_1\},\{\v,\w\}) }/{(2\pi)} =
  \op{vol}(C)\, {\dih(X,\ee)}/{(2\pi)}.
\]
The set of cells meeting $C$ in a set of positive measure gives a 
partition of $C$ into finitely many measurable sets.
This gives
\[
\op{vol}(C) = \sum_{X\mid \ee\in E(X)} \op{vol}(C\cap X)  = 
\op{vol}(C)\sum_{X\mid \ee\in E(X)} \dih(X,\ee)/(2\pi).
\]
The calculation of volumes in Chapter~\ref{chapter:volume} gives
$\op{vol}(C)>0$.  The conclusion follows by canceling $\op{vol}(C)$
from both sides of the equation.
\end{proof}







\subsection{A conjecture}

This section shows how the existence of a FCC-compatible negligible
function is a consequence of an explicit
inequality related to the distances $h(\bu)$, where $\bu\in \bV(1)$.

\begin{definition}[$\sol_0$,~$\tau_0$,~$m_1$,~$m_2$,~$h_+$,~$M$]\guid{AOZUTMU} 
\formaldef{$\sol_0$}{sol0}%
\formaldef{$\tau_0$}{tau0}%
\formaldef{$m_1$}{mm1}%
\formaldef{$m_2$}{mm2}%
\formaldef{$h_+$}{hplus}%
\formaldef{$M$}{marchal}%
Define the following constants and functions: 
\begin{align}\label{eqn:m-def} 
\sol_0 &= 3\arccos(1/3)-\pi\\
%\Delta_1 &= (3\arccos(1/3)-\pi)/\pi\\
\tau_0 &= 4\pi  - 20\sol_0\\
m_1 &= \sol_0 2\nsqrt2/\tau_0 \approx 1.012 \\ %% K 
m_2  &=  (6\sol_0- \pi)\nsqrt2/(6 \tau_0) \approx 0.0254\\ %% M 
h_+ &= 1.3254 \hbox{~(exact rational value).}
\end{align}
Let $M:\ring{R}\to\ring{R}$ 
be the following piecewise polynomial function (Figure~\ref{fig:M}):
\begin{equation}\label{eqn:M} 
M(h) =
\begin{cases} 
% (\sqrt2-h) (h-1.3254) (9h^2 - 17 h + 3)/(1.627 (\sqrt2-1))& h\le\sqrt2\\
\dfrac{\sqrt2-h}{\sqrt2-1}~%
\dfrac{h_+-h}{h_+-1} ~\dfrac{17 h - 9 h^2 - 3}{5} 
& h \le \sqrt2\vspace{3pt} \\
0 & h >\sqrt2.
\end{cases}
\\
\end{equation}
\end{definition}
\indy{Notation}{h@$h_+=1.3254$}%
\indy{Notation}{sol0@$\sol_0 = 3\arccos(1/3)-\pi$}%
\indy{Notation}{m1@$m_1\approx 1.012$}%
\indy{Notation}{m2@$m_2\approx 0.0254$}%
\indy{Notation}{zzt1@$\tau_0\approx\op{tgt}$}%
\indy{Notation}{M@$M$ (Marchal's quartic)}%

\figTULIGLY % fig:M

%\guid{TULIGLY}
%\begin{figure}[htb]
%\centering
%\szincludegraphics[width=60mm]{\pdfp/Mfun.eps}
%% Plot[Mfun[h], {h, 1, Sqrt[2]}]
%% copied to Preview, then saved, then converted to eps via pdf2eps.
%\caption{The quartic polynomial $M$}
%\label{fig:M}
%\end{figure}

The constant $\sol_0$
is the area of a spherical triangle with sides $\pi/3$.
Simple calculations based on the definitions give
\begin{equation}\label{eqn:km}
m_1 - 12m_2 = \sqrt{1/2}
\end{equation} 
and
\begin{equation}
M(1) = 1,\quad M(h_+)=0,\quad M(\sqrt2) =0.
\end{equation} 

\begin{definition}[$\gamma$]\guid{KGFDCFM} 
  \formaldef{$\gamma$}{gammaX} For any cell $X$ of a saturated
  packing, define the functional $\gamma(X,\wild)$ on
  $\{f:\ring{R}\to\ring{R}\}$ by
\begin{equation}\label{eqn:gamma-def} 
\gamma(X,f) =  \op{vol}(X)
-\left(\frac{2m_1}{\pi}\right) \op{tsol}(X) + \left(\frac{8m_2}{\pi}\right)
\sum_{\ee\in E(X)} \dih(X,\ee)  f(h(\ee)).
\end{equation}
\indy{Notation}{zzc@$\gamma$ (packing inequality)}%
\end{definition}


\begin{theorem*}[]\guid{HJKDESR}\label{lemma:MI} 
  Let $V$ be any saturated packing and let $X$ be any cell of
  $V$.  Then
\begin{equation}\label{eqn:mfe} 
\gamma(X,M)\ge 0,
\end{equation}
where $M$ is the function defined in \eqref{eqn:M}.
\end{theorem*}

\begin{remark}  %This is a \cc{HJKDESR}{}. 
  We do not use this inequality, and its proof is omitted.  
  The only published proof~\cite{marchal:2009} is not satisfactory because
  it plots sample level curves of the function and reaches conclusions
  based on the visual appearance of these level curves.
\end{remark}
%% cc:mar are the k-cell estimates for noncell clusters.
%By Calculation~\ref{calc:marchal}, the fundamental estimate
%holds for any cell


\begin{conjecture}[Marchal]\guid{PHNFUXP}\label{conj:m1} 
For any packing $ V$ and
any $ \u\in V$,
\[  
\sum_{\v\in V\setminus\{\u\}} M(h([\u;\v])) \le 12.
\] 
\end{conjecture}

This book proves a variant of the 
conjecture.

\begin{theorem}\guid{QFUXEXR}\label{theorem:mk1} 
  % formalization must follow the statements in pack_concl.hl, 250
  % each estimate.
  The conjecture~\eqref{conj:m1} and 
  inequality~\eqref{eqn:mfe} imply that for every saturated packing
  $V$, there exists a negligible FCC-compatible function $G:V\to
  \ring{R}$.
\end{theorem}

The theorem follows from the following more general statement.

\begin{lemma}\guid{KIZHLTL}\label{lemma:mk1}
  Let $f$ be any bounded, compactly supported function.  Set
\[
G( \u_0,f) = -\op{vol}(\Omega(V, \u_0)) + 8
m_1 - \sum 8 m_2 f(h([\u_0;\u])),
\]
with sum running over $\u\in V\setminus\{\u_0\}$.
If
\[  
\sum_{\v\in V\setminus\{\u\}} f(h([\u;\v])) \le 12,
\] 
then $G(\wild,f)$ is FCC-compatible.
Moreover, if there exists a constant $c_0$ such
that for all  $r\ge1$
\[
\sum_{X\subset B(\orz,r)} \gamma(X,f) \ge c_0 r^2,
\]
then $G(\wild,f)$ is negligible.
\end{lemma}

Theorem~\ref{theorem:mk1} is the special case $f=M$.  
Inequality~\ref{eqn:mfe} implies that we can take $c_0=0$ in the
lemma.

\begin{proof} 
%It is enough to show that $G( \u_0,f) = -\op{vol}(\Omega(V, \u_0)) + 8
%m_1 - \sum 8 m_2 f(h([\u_0;\u]))$ is FCC-compatible and negligible, when $f=M$.
%
%The result then follows from Lemma~\ref{lemma:deltabound}.  
The function $G(\wild,f)$ is FCC-compatible (page \pageref{def:negligible}) directly
by equation~\eqref{eqn:km}
and the assumption of the lemma:
\indy{Index}{negligible}%
\indy{Index}{FCC!compatible}%
\begin{align*} 
4\nsqrt{2} &= 8 m_1 - 8 (12 m_2)\\
&\le 8 m_1 - 8 m_2 \sum_{\u\in V\setminus \{\u_0\} } f(h([\u_0;\u]))\\
&= \op{vol}(\Omega(V, \u_0)) + G( \u_0,f).
\end{align*}
The issue is to prove that it is negligible. 
More explicitly, we show that there is a
constant $c$ such that for all $r\ge 1$:% and all $\p\in\ring{R}^3$:
\begin{align}\label{eqn:neg} 
-\sum G( \u,f) &= 
\sum \op{vol}(\Omega(V, \u)) 
-\sum 8m_1 + 
\sum \sum_{\v\in V\setminus\{\u\} } 8 m_2 f(h([\u;\v]))\notag \\
&\ge \sum_{X\subset B(\orz,r)} \gamma(X,f)  + c r^2,
\end{align}
where all unmarked sums run over $ \u\in  V(\orz,r)$.  
The lemma follows from this inequality, the assumption of the lemma,
 and the definition of negligible (Definition~\ref{def:negligible}).

Lemmas~\ref{lemma:Zr2} and \ref{lemma:V-finite} show that the number
of points of $ V$ near the boundary of $B(\orz,r)$ is at most $c r^2$,
for some $c$.


The function $\gamma(X,f)$ is defined as a sum of three
terms~\eqref{eqn:gamma-def}.  The sum of $\gamma(X,f)$
over all cells in a large ball
$B(\orz,r)$ is a sum of the contributions $T_1(r) + T_2(r) + T_3(r)$ from
the three separate terms defining $\gamma$.  
The sum of $-G$ in  equation~\eqref{eqn:neg} is a sum of three
corresponding terms $T'_i(r)$.  It is enough to work term by term, producing
constants $c_i$ such that
\[  
T_i'(r) \ge T_i(r) + c_i r^2,\quad i=1,2,3.
\] 

The sum of the volumes of the Voronoi cells $ \u\in B(\orz,r)$ is not
exactly the volume of $B(\orz,r)$ because of the contribution at the
boundary of $B(\orz,r)$ of Voronoi cells that are only partly contained
in $B(\orz,r)$.  Similarly, the sum of the various $k$-cells for
$X\subset B(\orz,r)$ is not exactly the volume of $B(\orz,r)$ because of
contributions from the boundary. The boundary contributions have order
$r^2$.  See Section~\ref{sec:finiteness} for order $r^2$ calculations. 
Thus,
\[  
T_1'= \sum_{ \u\in  V(\orz,r)} \op{vol}(\Omega(V, \u)) 
\ge \sum_{X\subset B(\orz,r)} \op{vol}(X) + c_1 r^2 = T_1 + c_1 r^2.
\] 


The estimates on the other terms are similar.  The solid angles
around each point sum to $4\pi$.
In Landau big O notation, this gives
\begin{align*} 
\sum_{X\subset B(\orz,r)} \op{tsol}(X) &= 
\sum_{X\subset B(\orz,r)~~} \sum_{ \u\in V( X)} \sol(X, \u)\\
&=\sum_{ \u\in  V(\orz,r)~~} \sum_{X\mid  \u\in V( X)} \sol(X, \u) + O(r^2)\\
&=\sum_{ \u\in  V(\orz,r)} 4\pi    + O(r^2).
\end{align*}
\indy{Notation}{O@$O$ (Landau's big O)}%
Hence
\[  
T_2' = -\sum_{ V(\orz,r)} 8 m_1 = 
-\sum_{X\subset B(\orz,r)}\left(\frac{2m_1}\pi\right) 
\op{tsol(X)} + O(r^2) = T_2 + O(r^2).
\] 
Similarly, the dihedral angles around each edge sum to $2\pi$.  A
factor of two enters the following calculation because there are two
ordered pairs for each unordered pair $\ee=\{ \u_0, \u_1\}$:
\begin{alignat*}{6}
&\phantom{=}&&\sum_{X\subset B(\orz,r)} 
&&\sum_{~~\ee\in E(X)~~~~~} \dih(X,\ee)  f(h(\ee)) \vspace{3pt}\\
&=&&\sum_{\ee\subset B(\orz,r)} 
&&\sum_{~~X\mid \ee\in E(X)} \dih(X,\ee)  f(h(\ee)) +O(r^2)\vspace{3pt}\\
&=&&\sum_{\ee\subset B(\orz,r)} &&2\pi f(h(\ee)) + O(r^2)\vspace{3pt} \\
&=&&\sum_{ \u_0\in  V(\orz,r)} 
&&\sum_{~~\u_1\in  V(\orz,r)~~~} \pi f(h( \u_0, \u_1)) + O(r^2).\\
\end{alignat*}
Finally,
\arrayqed
\begin{align*} 
T_3' &= \sum\sum 8 m_2 f(h(\bu)) \\
&\ge \left(\frac{8m_2}\pi\right)
\sum_{X\subset B(\orz,r)~~}\sum_{\ee\in E(X)}\dih(X,\ee) f(h(\ee)) + O(r^2) \\
&= T_3 + O(r^2).\arrayqedhere
\end{align*}
\end{proof}




\section{Clusters}

This section introduces a variant of 
Conjecture~\ref{conj:m1}.  In this variant, a piecewise linear
function $L$ replaces the piecewise polynomial function $M$.  More
crucially, the support of the function $L$ is contained in
$\leftclosed2,2.52\rightclosed$.  By contrast, the function $M$ is
positive on a large interval: $\leftclosed2,2.6508\rightopen$.  This
difference in the support of the function creates a large difference
in the difficulty of the conjectures.

The conjecture formulated in this section also implies the existence
of FCC-compatible negligible functions.  To prove this existence
result, it is helpful to group cells together into new
aggregates, called \newterm{clusters}.  This section makes a detailed
study of clusters in order to produce a negligible function.
The aim of this section is to prove a variant
(Theorem~\ref{theorem:mk2}) of Theorem~\ref{theorem:mk1} that uses the
function $L$ rather than $M$.


% This section shows how to improve on the estimates of the previous
% section by combining various cells into \newterm{cell clusters}.
Recall that $M(h_+) = 0$, where   $h_+ = 1.3254$.
\indy{Index}{cell cluster}%

\begin{definition}[$L$,~$h_0$,~$h_-$]\guid{ULZRABY}\label{def:L} 
\formaldef{$h_0$}{h0}
\formaldef{$h_-$}{hminus}
\formaldef{$L$}{lmfun}
Set
\[  
h_0 = 1.26.%\\  %%\hm
\] 
Let $L:\ring{R}\to\ring{R}$ be the piecewise linear function 
\[  
L(h) = \begin{cases} 
\dfrac{h_0-h}{h_0-1}, & h \le h_0 \\
0, & h\ge h_0. \\
\end{cases}
\] 
It follows from the definition that
\[  
L(1) = 1\textand  L(\hm) = 0.
\] 
Let $h_- \approx 1.23175$ be the unique root of the quartic polynomial
$M(h)-L(h)$ that lies in the interval $[1.231,1.232]$.
\indy{Notation}{L@$L$ (linear function)}%
\indy{Notation}{h@$h_- \approx 1.23175$}%
\indy{Notation}{h@$h_0 = 1.26$}%
\end{definition}

The inequality $L(h)\ge M(h)$ holds except when $h\in [h_-,h_+]$.

\figBJLIEKB % fig:L

%%
%\begin{figure}[htb]
%\centering
%\szincludegraphics[width=60mm]{\pdfp/Lfun.eps}
% Plot[{Mfun[h],Lfun[h]}, {h, 1.2, 1.35}]
% copied to Preview, then saved, then converted to eps via pdf2eps.
%% WW very big .eps file!
%\caption{Detail of the quartic $M$ and linear function $L$}
%\label{fig:L}
%\end{figure}


\begin{definition}[critical~edge,~$\op{EC}$,~$\op{wt}$]\guid{MZSRVBC}\label{def:wt} 
  A \newterm{critical edge} $\ee$ of a saturated packing $V$ is an
  unordered pair that appears as an element of $E(X)$ for some
  $k$-cell $X$ of the packing $V$ such that $h(\ee)\in[h_-,h_+]$.  Let
  $\op{EC}(X)$ be the set of critical edges that belong to $E(X)$.  If
  $X$ is any cell such that $\op{EC}(X)$ is nonempty, let the
  \newterm{weight} $\op{wt}(X)$ of $X$ be $1/\card(\op{EC}(X))$.
\end{definition}
\indy{Notation}{E3b@$\op{EC}$ (critical edges)}%
\indy{Notation}{wt@$\op{wt}$ (weight)}%

\begin{definition}[$\beta_0$,~$\beta$]\guid{PQFEXQN}\label{def:beta} 
\formaldef{$\beta_0$}{bump}
\formaldef{$\beta$}{beta\_bump}
Set 
\[  
\beta_0(h) = 0.005 (1 - (h-h_0)^2/(h_+-h_0)^2).
\] 
 (See Figure~\ref{fig:fg1}.) 
If $X$ is a $4$-cell with exactly two critical edges and if those
edges are opposite, then set
\[  
\beta(\ee,X) = \beta_0(h(\ee)) - \beta_0(h(\ee')), 
\text{ where }\op{EC}(X) = \{\ee,\ee'\} .  
\] 
Otherwise, for all other edges in all other cells, set $\beta(\ee,X) = 0$.
\end{definition}
\indy{Notation}{zzb@$\beta,~\beta_0$ (bump)}%

\figPQFEXQN % fig:fg1

\begin{definition}[cell cluster,~$\Gamma$]\guid{YSULGYR}
\label{def:gammaL} 
\formaldef{cell cluster}{cell\_cluster}
\formaldef{$\Gamma$}{cluster\_gammaX}
  Let $V$ be a saturated packing.  Let $\ee\in \op{EC}(X)$ be a
  critical edge of a $k$-cell $X$ of $V$ for some $2\le k\le 4$.  A
  \newterm{cell cluster} is the set
\[  
\op{CL}(\ee) = \{X\mid \ee\in \op{EC}(X)\} 
\] 
\indy{Notation}{cluster}%
of all cells around $\ee$. 
%If $Z$ is a subset of a cell cluster $\op{CL}(\ee)$, 
Define
\[  
\Gamma(\ee) = \sum_{X\in \op{CL}(\ee)} \gamma(X,L) \op{wt}(X) +\beta(\ee,X).
\] 
%and where $\op{wt}(X)$ is the weight of $X$.
\end{definition}
\indy{Index}{cell cluster}%
\indy{Notation}{CL@$\op{CL}$ (cell cluster)}%
\indy{Notation}{zzC@$\Gamma$}%


The following weak form of Theorem~\ref{lemma:MI} is sufficient
for our needs.

\begin{lemma}\guid{TSKAJXY}\label{lemma:LI} 
Let $V$ be any saturated packing and let $X$ be any cell of $V$ such that $\op{EC}(X)$ is empty.   Then
\[
\gamma(X,L)\ge 0.
\]
\end{lemma}

\begin{proof} This is a \cc{TSKAJXY}{}.
\end{proof}


\begin{theorem}[cell cluster inequality]\guid{OXLZLEZ} 
\label{lemma:cluster}
Let $\op{CL}(\ee)$ be any cell cluster of a critical edge $\ee$ in a
saturated packing $V$.  Then $\Gamma(\ee)\ge 0$.
\end{theorem}

\begin{proof}[Proof sketch]
  The proof of this cell cluster inequality is a \cc{OXLZLEZ}{}, which
  is the most delicate computer estimate in the book.  It 
  %The
  %method of linear assembly (described in Section~\ref{sec:assembly})
  reduces the cell cluster inequality to hundreds of nonlinear
  inequalities in at most six variables.  In degenerate cells with a
  face of area zero, Euler's formula (Lemma~\ref{lemma:euler}) should
  be used to calculate solid angles, because the standard dihedral angle
  formula for solid angles can lead to the evaluation of $\atn$ at the
  branch point $(0,0)$, which is numerical unstable and is best
  avoided.
\end{proof}

\begin{example}\guid{JXEHXQY}
  We  construct an example of a cell cluster in the form of an
  octahedron, with four $4$-cells joined along a common critical edge
  $\ee$.  Assume that all of the edges of the octahedron have length
  $2$, except for one of length $y$ for some edge that does not meet
  $\ee$.  If $y\in \leftclosed 2 h_-,2 h_+\rightclosed$, then one of
  the four simplices has weight $1/2$ and the other three simplices
  have weight $1$.  The parameter $y$ determines the cell cluster up
  to isometry.  We plot the function $f(y)=\Gamma(\ee)$
  as a function of $y$.    We also plot the function
\[
g(y) = \sum_{X\in \op{CL}(\ee)} \gamma(X,L) \op{wt}(X).
\]
As the plot shows, the function $g$ is not positive.  This shows
that without the small correction term $\beta$, the cell cluster
inequality is false.  Numerical evidence suggests that the global
minimum of $\Gamma(\ee)$ occurs when the cell cluster has
the form of an octahedron with parameter $y=2h_+$ and value
$f(2h_+)\approx 0.0013$.
\end{example}
\indy{Notation}{g@$g$ (function name)}%

\figJXEHXQY % fig:fg

The proof of the following lemma is deferred, because it relies on
many computer calculations and is extremely long and complex.  The
non-computer parts of the proof take up most of the remainder of the
book.  In this chapter the lemma is treated as an unproved assertion.


\begin{lemma*}\guid{BJERBNU}\label{conj:L12} 
  For any  saturated packing $ V$ and any $ \u_0\in V$,
\begin{equation}\label{eqn:L12} 
\sum_{ \u_1\in V\mid h( \u_0, \u_1)\le \hm} L(h\{\u_0, \u_1\}) \le 12.
\end{equation}
\end{lemma*}

\begin{lemma}[]\guid{UPFZBZM}\label{theorem:mk2} 
Inequality~\eqref{eqn:L12} implies
that for every saturated packing $V$, there exists a negligible FCC-compatible function
$G:V\to \ring{R}$.
\end{lemma}

\begin{remark}\label{rem:L12KC}
In light of Lemma~\ref{lemma:deltabound}, inequality~\ref{eqn:L12}
implies the Kepler conjecture. 
\end{remark}

\begin{proof} 
  By Lemma~\ref{lemma:mk1}, the proof reduces to showing that there
  exists a constant $c_0$ such that for all $r\ge1$
\[
\sum_{X\subset B(\orz,r)} \gamma(X,L) \ge c_0 r^2.
\]

If a cell $X$ does not belong to any cell cluster, then
\[ 
\gamma(X,L)\ge 0
\] 
by Lemma~\ref{lemma:LI}.  Note that the function
$\beta(\ee,X)$ averages to zero for any $4$-cell $X$:
\[  
\sum_{\ee\in \op{EC}(X)} \beta(\ee,X) = 0.
\] 
Hence, the terms involving $\beta$ in sums may be disregarded in this
proof.  (These terms may be disregarded here, but they are needed in
 Lemma~\ref{lemma:cluster}.)

Theorem~\ref{lemma:cluster} gives the required inequality for cell
clusters.  Again, using big O notation,
\begin{alignat*}{8}
\sum_{X\subset B(\orz,r)} \gamma(X,L) 
&=&&\sum_{X\subset B(\orz,r)\mid \op{EC}(X)\ne\emptyset} \hspace{-2em}\gamma(X,L) \quad +
\sum_{X\subset B(\orz,r)\mid \op{EC}(X)=\emptyset} \hspace{-2em}\gamma(X,L) \vspace{6pt}\\
&\ge &&\sum_{X\subset B(\orz,r)\mid \op{EC}(X)\ne\emptyset} \hspace{-2em}\gamma(X,L)\vspace{6pt} \\
&= &&\sum_{\hspace{1.8em}X\subset B(\orz,r)\hspace{1.8em}}\hspace{-2em}\gamma(X,L)\sum_{\ee \in \op{EC}(X)}\op{wt}(X) \vspace{6pt}\\
&= &&\sum_{\hspace{1.85em}\ee\subset B(\orz,r)\hspace{1.85em}}\hspace{-2.5em}\sum_{~~~~X\!\mid\! \ee \in \op{EC}(X)}\hspace{-1.1em}\gamma(X,L)\op{wt}(X) &+ O(r^2)\vspace{6pt}\\
&= &&\sum_{\hspace{1.85em}\ee\subset B(\orz,r)\hspace{1.85em}}\hspace{-1em}\Gamma(\ee) &+ O(r^2)\vspace{6pt}\\
&\ge&& ~~O(r^2).
\end{alignat*}
\end{proof}

\begin{definition}[$\BB$]\guid{WTKURHK} 
  \formaldef{$\BB$}{ball\_annulus} Let $\BB$ be the \newterm{annulus}
  $\bar B(\orz,2h_0)\setminus B(\orz,2)$, where $\bar B(\orz,r)$ is
  the closed ball of radius $r$.  
\end{definition}
\indy{Notation}{BB@$\BB$}%


\begin{corollary}\guid{RDWKARC}\label{cor:CE} 
  If the Kepler conjecture is false, there exists a finite packing
  $V\subset\BB$ with the following properties.
\begin{equation}\label{eqn:CE} 
\sum_{ \u\in V} L(h\{\orz, \u\}) > 12.
\end{equation}
\end{corollary}

The proof of the Kepler conjecture proceeds by assuming that there is
a counterexample to Inequality~\ref{eqn:L12} and then deriving a
contradiction.  This corollary formulates the potential counterexample
in slightly simpler terms.

\begin{proof} If the Kepler conjecture is false,
  Inequality~\ref{eqn:L12} is violated for some packing $ V$ and some
  $ \u_0\in V$.  After translating  $ V$ to $ V - \u_0$ and $
  \u_0$ to $\orz$, it follows without loss of generality that $
  \u_0=\orz\in V$.  After the replacement of $ V$ with the finite
  subset $V\cap \BB$, it follows without loss of generality that the
  packing is a finite subset of $\BB$.
\end{proof}



\section{Counting Spheres}

This section proves two estimates about a packing $V\subset \BB$ that
satisfies Inequality~\ref{eqn:CE}.  The first estimate
(Lemma~\ref{lemma:13-14}) shows that the cardinality of $V$ is
thirteen, fourteen, or fifteen.  The second estimate (See
Lemma~\ref{lemma:D'}.)  shows that no point $\v\in V$ can be strongly
isolated from the other points of $V$.  To prove these two estimates,
we need a formula for the smallest possible area of a spherical
polygon that contains a disk.  This formula is developed in the first
subsection.

\subsection{solid angle}
\indy{Index}{polygon}%


The following lemma is analogous to the Rogers decomposition of a
polyhedron into simplices.  The lemma constructs $2k$ points to be
used to triangulate (a subset of) a polygon (Figure~\ref{fig:reference-w}).


\begin{lemma}[]\guid{EUSOTYP}\label{lemma:2D-poly}
Let $P$ be a two-dimensional bounded polyhedron in $\ring{R}^2$.
Let $k$ be the number of facets of $P$.  Let $r>0$.  Suppose that
\[
\{\v\in \ring{R}^2\mid  \normo{v}< r\} \subset P.
\]
Then there exist nonzero points $\w_j\in P$
for $j=0,\ldots,2 k-1$ such that
\begin{enumerate}
\item The polar cycle on  $\{\w_j\mid j\}$ is given by
$\sigma(\w_j) = \w_{j+1}$, with indexing mod $2k$.
\item $\theta(\w_{2i},\w_{2i+1}) = \theta(\w_{2i+1},\w_{2i+2}) < \pi/2$,  
where $\theta$ denotes the
relative polar coordinate of Lemma~\ref{lemma:polar-sum}.
\item $\normo{\w_{2i}}=r$ and 
$\normo{\w_{2i+1}} = r\sec \theta(\w_{2i},\w_{2i+1})$, 
for $i=0,\ldots,k-1$.
\item $\w_{2i}\cdot (\w_{2i\pm1}-\w_{2i}) = 0$.
\end{enumerate}
\end{lemma}

\figYAHDBVO % fig:reference-w

\begin{proof}
  Enumerate  the distinct facets $F_1,\ldots,F_k$ of $P$, and for each
  one select a defining equation
\[
F_i = P\cap \{\p \mid \u_i \cdot \p = b_i\},
\quad\text{where } \normo{\u_i}=1\text{ and } b_i\ge0.
\]
We may assume that ordering of facets by increasing subscripts is
the ordering by the polar cycle on $\u_i\in\ring{R}^2$.
The assumption that $P$ contains an open disk of radius $r$ gives
$r\le b_i$.

\claim{We claim that $r\u_i\in P$ and does not lie in any facet except
  possibly $F_i$.}  Otherwise, for some $j$,
\[
r\le b_j \le (r\u_i)\cdot \u_j\le r\normo{\u_i}\normo{\u_j}=r.
\]  
This is the case of equality in Cauchy--Schwarz, which implies that
$\u_i=\u_j$.  The definition of face implies that $b_i=b_j$, and $i=j$.   
The claim ensues.

\claim{We claim that $0<\theta(\u_i,\u_{i+1})<\pi$.}  Indeed, from the
previous claim it follows that $0<\theta(\u_i,\u_{i+1})$.  By the
boundedness of $P$, for any nonzero  $\v$ orthogonal to $\u_i$,
there exists $s>0$ such that $r\u_i + s \v$ lies in some facet
$F_j\ne F_i$.  This gives $r \u_j\cdot\u_i + s \u_j\cdot \v = b_j$.
The condition $r\u_i\in P\setminus F_j$ gives $r\u_j\cdot\u_i < b_j$.
Hence $\u_j\cdot\v > 0$.  For an appropriate choice of sign of $\v$, this
gives $\theta(\u_i,\u_{i+1})\le\theta(\u_i,\u_j)<\pi$.  

Suppressing the subscript $i$, we write $\psi = \theta(\u_i,\u_{i+1})/2$.
Let $\u_i'$ be the point in the plane given in polar coordinates by
\[
\normo{\u_i'} = r\sec\psi,\quad 
\theta(\u_i,\u'_{i}) = \theta(\u_i',\u_{i+1})=\psi.
\]

\claim{We claim that $\u'_i\in P$.}  Indeed, for every $j$, we have
\[
\u_j \cdot \u_i' = 
\normo{\u_j}\normo{\u_i'} \cos\varphi = 
r\sec\psi\cos\varphi,
\]
where $\varphi = \theta(\u_i',\u_j)$.  From the polar order, and the
construction of $\u_i'$ along the bisector of $\u_i,\u_{i+1}$, it follows
that
\[
\psi\le \varphi \le 2\pi - \psi.
\]
Hence, $\cos\varphi \le \cos\psi$.  This gives
\[
\u_j \cdot \u_i' \le r \le b_j.
\]
This shows that $\u'_i$ satisfies all the defining conditions of $P$.

Set $\w_{2i}=\u_i$ and $\w_{2i+1}=\u'_i$.  It is clear from
construction that the polar cycle on  $\{\w_j\mid j\}$ is
compatible with the indexing.  The enumerated properties of the lemma
now follow from Lemma~\ref{lemma:polar-sum}.
The lemma ensues.
\end{proof}

\begin{lemma}[]\guid{GOTCJAH}\label{lemma:ngon} 
  Let $P$ be a bounded polyhedron in $\ring{R}^3$ that contains $\orz$
  as an interior point.  Let $F$ be a facet of $P$, given by an
  equation
\[  
F = \{\p \mid \p \cdot \v = b_0\} \cap P.
\]  
Let $W_F$ be the corresponding topological component of $Y(V_P,E_P)$.
Assume that $W_F$ contains the right-circular cone
\begin{equation}\label{eqn:rW}
\op{rcone}^0(\orz,\v,t) \subset W_F
\end{equation}
for some $t$ such that $0<t<1$.
Then 
\[  
\sol(W_F) \ge 
2\pi - 2 k \,\arcsin\left(\,t\sin(\pi/k)\,\right),
\] 
where $k$ is the number of edges of $F$.
\end{lemma}

\begin{proof}
  Project the facet $F$ to $\ring{R}^2$ by  projecting onto
  the coordinates of $\e_2$ and $\e_3$ of  an orthonormal frame
  $(\e_1,\e_2,\e_3)$ adapted to $(\orz,\v,\ldots)$.  By the Pythagorean theorem,
  the hypothesis \eqref{eqn:rW} implies that a disk of radius
\[
b_1 \nsqrt{1-t^2}/t 
\]
is contained in the projected face, where $b_1= b_0/\normo{\v}$ is the
distance from $\op{aff}(F)$ to $\orz$.  Apply
Lemma~\ref{lemma:2D-poly} to the projected face, and pull the points
$\w_j$ back to points on $F$ with the same names.

By the additivity of measure over measurable sets that are disjoint up
to a null set, we may partition into wedges:
\begin{align*}
\sol(W_F) &= \sum_j \sol(W_F \cap W(\orz,\v,\w_j,\w_{j+1}))\\
  &\ge \sum_j \sol(\op{aff}_+^0(\orz,\{\v,\w_j,\w_{j+1}\})).
\end{align*}
The solid triangles that appear in the last sum are primitive volumes,
which are computed in terms of
dihedral angles  in Chapter~\ref{chapter:volume}.  Set
\[
\beta_{2j}=\beta_{2j+1}=\dih_V(\{\orz,\v\},\{\w_{2j},\w_{2j\pm1}\}) \text{ and }
a = \arc_V(\orz,\{\v,\w_{2j}\}) = \arccos t.
\]
The three vectors $\v$, $\w_{2j}-\v$, and $\w_{2j+1}-\w_{2j}$ are
mutually orthogonal by the final claim of Lemma~\ref{lemma:2D-poly}.
Lemma~\ref{lemma:dih-cross} gives
\[
\dih_V(\{\orz,\w_{2j}\},\{\v,\w_{2j+1}\}) =\pi/2,
\]
because
\begin{align*}
(\w_{2j} \times \v)\cdot (\w_{2j}\times \w_{2j+1}) &=
(\w_{2j} \times \v)\cdot (\w_{2j}\times (\w_{2j+1}-\w_{2j}))\\
 &=  (\w_{2j+1}-\w_{2j})\cdot ((\w_{2j} \times \v)\times \w_{2j})\\
&= 0.
\end{align*}

Consider a spherical triangle with sides $a,b,c$ and opposite angles
$\alpha,\beta,\gamma$.  If $\gamma=\pi/2$, then by Girard's formula,
the area of the triangle is
\[  
\alpha+\beta-\pi/2,
\] 
and by the spherical law of cosines (Lemma~\ref{lemma:sloc2})
\[  
\cos\alpha =\cos a\sin\beta.
\] 
This determines the area $g(a,\beta)$ of the triangle 
as a function of $a$ and $\beta$:
\[
g(a,\beta) = \beta - \arcsin(\cos a \sin \beta).
\]
\indy{Index}{Girard's formula}%
\indy{Notation}{g@$g$ (triangle area)}%
\indy{Notation}{zza@$\alpha$ (angle)}%
\indy{Notation}{zzb@$\beta$ (angle)}%
\indy{Notation}{zzc@$\gamma$ (angle)}%
\indy{Index}{convex}%
\indy{Index}{Girard's formula}%
\indy{Index}{polygon}%

The solid angle of $W_F$ is at least sum of the areas of the triangles:
\[  
\sum_{j=0}^{2k-1} g(a,\beta_j),
\] 
with angle sum
\[  
\sum_{j=0}^{2k-1} \beta_j = 2\pi.
\] 
The second partial of $g$ with respect to $\beta$ is
\[  
\frac{\partial^2 g(a,\beta)}{\partial \beta^2} = 
\frac{\cos a\sin^2 a\sin \beta}{\sin^2\alpha} \ge 0.
\] 
Thus, the function is convex in $\beta$.  By convexity, the minimum area
occurs when all angles are equal $\beta=\beta_j = \pi/k$.

The solid angle bound of the lemma is equal to 
\[  
2 k g(a,\pi/k)
\] 
where $\cos a=t$.  
%Alternatively, the polygon breaks into $2k$
%triangles, each computed by Girard's formula to have area
%\[  
%\beta - (\pi/2 - \alpha)  = \pi/k - \arcsin(\cos(\alpha)) = 
%\pi/k - \arcsin(\cos(a)\sin(\beta)).
%\] 
\end{proof}

%\begin{lemma}[]\guid{BBEVFIC}\label{lemma:ngon-area} 
%  The minimum area of an intersection of $k$-hemispheres containing a
%  circle $C$ of arcradius $a<\pi/2$ is
%\[  
%2\pi - 2 k \,\arcsin\left(\,\cos(a)\sin(\beta)\,\right),
%\] 
%when $\beta = \pi/k$.
%\end{lemma}



\subsection{a polyhedral bound}

\begin{definition}[weakly saturated]\guid{HUCFLEB} 
\label{def:weakly-saturated}
Let $r$ and $r'$ be real numbers such that $2\le r\le r'$.  Define a
set $ V\subset\ring{R}^3\setminus B(\orz,2)$ to be \newterm{weakly
  saturated} with parameters $(r,r')$ if for every $\p\in\ring{R}^3$
\[  
2\le\normo{\p}\le r'~~~\implies~~~ \exists \u\in V.~\norm{ \u}{\p}< r.
\] 
\end{definition}

\begin{lemma}[]\guid{TARJJUW}\label{lemma:poly-bounded} 
\formalauthor{Dang Tat Dat}
Fix $r$ and $r'$ such that $2\le r\le r'$.
Let $ V$ be a weakly saturated finite packing with parameters $(r,r')$.
%such that $\orz\in  V$.
%where 
%   $\orz\in V$, and
%   $\normo{ \u}\le r'$ for all $ \u\in V$,
%Set $ V^*= V\setminus\{\orz\}$.
For any $g: V\to\ring{R}$, let $P( V,g)$ be the
polyhedron given by the intersection of half-spaces
\[  
\{\p \mid  \u\cdot \p \le g( \u)\},\quad \u\in V.
\] 
Then $P( V,g)$ is bounded.
\end{lemma}
\indy{Index}{polyhedron}%

\begin{proof} Assume for a contradiction that $P=P( V,g)$ is unbounded
  and there exists $\p\in P$ such that $\normo{\p} > g( \u) r'/2$ for
  all $ \u\in V$.  Let $\v = r' \p/\normo{\p}$ so that
  $r'=\normo{\v}$.  By the weak saturation of $ V$, there exists $
  \u\in V$ such that $\norm{\v}{ \u}<r$.  Then,
\begin{align*} 
\normo{\p} &> g( \u) r'/2 \ge  
\u\cdot (r' \p)/2 = \normo{\p}  \u\cdot \v /2\\
&= \normo{\p} (\normo{ \u}^2 + \normo{\v}^2 - \norm{ \u}{\v}^2)/4\\
&> \normo{\p}(4+r'^2-r^2)/4\\
&\ge \normo{\p}.
\end{align*}
This contradiction shows that $P$ is bounded.
\end{proof}

\begin{lemma}\guid{YSSKQOY}\label{lemma:g-ineq}
  % \cc{4198769118}{This inequality gives the nonoverlap of
  %   disks. Deprecated.}
Let 
\[ 
g(h) = \arccos(h/2) - \pi/6.
\] 
Then
\begin{equation}\label{eqn:disks} 
\op{arc}(2h,2h',2)\ge g(h) + g(h') ,%
\end{equation}%
for all $h,h'\in\leftclosed 1,h_0\rightclosed$.
\end{lemma}

\begin{proof}
The function $g$ can be rewritten as
\[
g(h) = \op{arc}(2h,2,2) - \op{arc}(2,2,2)/2.
\]
It is enough to prove a  more general inequality in
  symmetrical form
\begin{equation}\label{eqn:disks-symmetrical}
f(a_2,b_2) - f(a_1,b_2) - f(a_2,b_1) + f(a_1,b_1)\ge0, 
\end{equation}
when
\[
2\le a_1 \le a_2 \le 2h_0,\text{ and } 2\le b_1\le b_2 \le 2h_0,
\]
where $f(a,b) = \op{arc}(a,b,2)$.  
%The left-hand side of
%\eqref{eqn:disks-symmetrical} is additive on rectangular meshes of
%$\leftclosed a_1,a_2\rightclosed \times \leftclosed
%b_1,b_2\rightclosed$.  Thus, it is enough to prove the inequality in
%the limit as $a_2\mapsto a_1$ and $b_2\mapsto b_1$.  In the limit, the
%general inequality follows from the leading term of the Taylor
%approximation
A calculation gives
\[
 \frac {\partial^2 f(a,b)}{\partial a\,\partial b} = \frac{32 a b}{\ups(a^2,b^2,4)^{3/2}} > 0.
\]
Thus, by holding $a$ fixed, $\partial f/\partial a$ is increasing in $b$:
\[
 \frac {\partial f(a,b_2) } {\partial a} -\frac{\partial f(a,b_1)}{\partial a} \ge 0.
\]
This shows that with $b_1$ and $b_2$ fixed, 
$f(a,b_2)-f(a,b_1)$ is increasing in $a$.
Equation~\ref{eqn:disks-symmetrical} ensues.
\end{proof}


Since $L(h)\le 1$ when $h\ge1$, it is clear that a finite packing $V$
that satisfies Inequality~\ref{eqn:CE} has cardinality greater than twelve.
The following lemma also gives an upper bound on the cardinality of $V$.

\begin{lemma}[]\guid{DLWCHEM}\label{lemma:13-14}  %%
  If $V\subset \BB$ is a packing that satisfies
  Inequality~\ref{eqn:CE}, then the cardinality of $V$ is thirteen,
  fourteen, or fifteen.
\end{lemma}


\begin{proof} (Following Marchal.) Consider a finite packing $
  V=\{\u_1,\ldots, \u_N\}\subset \BB$ satisfying
  Inequality~\ref{eqn:CE}.  The packing $V$ contains more than twelve
  points because otherwise Inequality~\ref{eqn:CE} cannot hold, as
  $L(h)\le 1$.

  By adding points as necessary, the packing becomes weakly saturated
  in the sense of Definition~\ref{def:weakly-saturated}, with $r=2$
  and $r'=2\hm$.  It is enough to show that this enlarged set has
  cardinality less than sixteen.  Let
\[ %
g(h) = \arccos(h/2) - \pi/6,  %
\] %
and let $h_i =
\normo{ \u_i}/2$.  Then $h_i\le h_0=1.26$.
Consider the spherical disks $D_i$ of radii $g(h_i)$,
centered at $ \u_i/\normo{ \u_i}$ on the unit sphere.  
These disks do not overlap by Lemma~\ref{lemma:g-ineq}.


\indy{Notation}{D@$D$ (spherical disk)}%
For each $i$, the plane through the circular boundary of $D_i$ bounds
a half-space containing the origin.  The intersection of these
half-spaces is a polyhedron $P$, which is bounded by
Lemma~\ref{lemma:poly-bounded}.  (See Figure~\ref{fig:marchal-polyhedron}.)
Lemma~\ref{lemma:polyhedron}
associates a fan $(V_P,E_P)$ with $P$.  (The set $V_P$ is dual to $
V$; the set $V_P$ is in bijection with extreme points of $P$, whereas
$ V$ is in bijection with the facets of $P$.)  There are natural
bijections between the following sets.
\begin{enumerate}\wasitemize  
\item $ V = \{ \u_1,\ldots, \u_N\}$.
\item The  facets of $P$.
\item The set of  topological components of $Y(V_P,E_P)$.
\item The set of faces in the hypermap $\op{hyp}(V_P,E_P)$.
\end{enumerate}\wasitemize 
The first conclusion
of Lemma~\ref{lemma:webster} gives 
the bijection of the first two sets.  Lemmas~\ref{lemma:WF} and
~\ref{lemma:face} give the other bijections.

\figZXEVDCA % fig:marchal-polyhedron

%Let $k_i$ be 
%associated with the facet $i$.  
By Lemma~\ref{lemma:edge-bi}, the number of edges of the facet $i$
is $k_i$, the cardinality of the corresponding face in $\op{hyp}(V_P,E_P)$.   
By Lemma~\ref{lemma:ngon}, the solid
angle of the topological component $W_i$ of $Y(V_P,E_P)$ is at least
$\op{reg}(g(h_i),k_i)$, where \indy{Index}{half-plane}%
\indy{Index}{half-space}%
\indy{Notation}{reg (area of regular spherical polygon)}%
\[  
\op{reg}(a,k) = 2\pi - 2 k (\arcsin(\cos(a)\sin(\pi/k))).
\] 
By a \cc{BIEFJHU}{%, cc:alin, old: 1965189142
This is a linear lower bound on the area of a regular polygon.} 
\begin{equation}\label{eqn:alin} 
\op{reg}(g(h),k) \ge c_0 + c_1 k + c_2 L(h),\quad \text{ for all }
k = 3,4,\ldots,\quad 1\le h\le \hm,
\end{equation}
where
\[ 
%c_0 = 0.6327,\quad c_1 = -0.0333,\quad c_2 =0.4754.
c_0=0.591,\quad c_1=-0.0331,\quad c_2 = 0.506.
\] The sum $\sum_i k_i$ is the number of darts in $\op{hyp}(V_P,E_P)$
by Lemma~\ref{lemma:polyhedron}.  By Lemma~\ref{lemma:dart-upper},
$\sum_i k_i \le (6N-12)$.  Summing over $i$, an estimate on $N$
follows: \indy{Index}{polyhedron}%
\indy{Index}{hypermap!planar}%
\begin{align*} 
4\pi &= \sum_i\op{sol}(W_i)\\
&\ge \sum_i \op{reg}(g(h_i),k_i) \\
&\ge c_0 N +c_1\sum_i k_i + c_2 \sum L(h_i)\\
&\ge c_0 N +c_1 (6N-12) + c_2 12.
\end{align*}
This gives
$16 > N$.
\end{proof} 


\begin{lemma}[]\guid{XULJEPR}\label{lemma:D'}  
  Assume that $V\subset \BB$ is a packing that satisfies
  Inequality~\ref{eqn:CE}.  Then
 for every $ \v\in V$ such that $\normo{ \v}=2$, there exists
  $\u\in V$ such that 
$0<\norm{ \v}{ \u}< 2\hm$. 
\end{lemma}

\begin{proof} Assume for a contradiction that a packing $V$ exists that
  satisfies the inequality for which there exists $\v\in V$ for which 
\begin{equation}\label{eqn:norm-hm}
2\hm\le\norm{\v}{\u},\quad \u\ne \v.
\end{equation}
The assumption that \eqref{eqn:CE} holds implies that $N\ge 13$.
Create one large disk $D_1'$ centered at $\v/2$ and repeat the proof
of the previous lemma.  Extend the packing to a  weak saturation
with parameters $r=r'=2\hm$.  This can be done in a way that maintains
the assumptions on $\v$.  By Lemma~\ref{lemma:poly-bounded}, the
polyhedron is bounded.  By a \cc{WAZLDCD}{} % old: 8055810915
% The nonoverlap of disks
\[ a'=0.797 < \arc(2,2h,2\hm)-g(h)\text{\ \ for } 1\le h \le \hm.\]  
By \eqref{eqn:norm-hm}, we may take $a'$
for the arcradius of the large disk $D_1'$.  
By a \cc{UKBRPFE}{}   %old:6096597438,  cc:alin2, 
%Linear lower bound on regular polygon (large disk)
\begin{equation}\label{eqn:alin2} 
\op{reg}(a',k) \ge c_0 + c_1 k + c_2 L(1) +
c_3,\quad k=3,4,\ldots\end{equation}
where
$c_3 =  1$.  % was %0.85$.
Then 
\begin{align*} \label{eqn:D'}
4\pi &= \sum_{i=1}^{N}\op{sol}(W_i)\\
&\ge \op{reg}(a',k_1)+\sum_{i=2}^{N} \op{reg}(g(h_i),k_i) \\
&\ge  c_0 N +c_1\sum_{i=1}^{N} k + c_2 \sum_{i=1}^{N} L(h_i) + c_3\\
&\ge c_0 N +c_1 (6N-12) + c_2 12 + c_3.
\end{align*}
This gives a contradiction
$13 > N \ge 13.$
\end{proof}
