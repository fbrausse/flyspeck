% file started March 22, 2009

\chapter{Packing}


\section{Definition}



Intuitively, a \newterm{packing} is an arrangement of congruent
balls in Euclidean three space that are nonoverlapping in the sense
that the interiors of the balls are pairwise disjoint.  By convention,
we take the radius of the congruent balls to be $1$.
%the scale
%invariance of density, without loss of generality, units can be chosen
%so that each ball has radius $1$. 
Let $ V$ be the set of centers of the balls in a
packing. The choice of unit radius for the
balls implies that any two points in $ V$ have distance at
least $2$ from each other. 
 Formally, the packing is identified
with the set of centers $V$.
\indy{Notation}{V@$ V$ (packing)}%

%
%The density of a packing does not decrease when balls are added to the
%packing. Thus, to construct packings of maximal density, one may add
%nonoverlapping balls until there is no room to add further balls.  A
A packing in which no further balls can be added is said to be {\it
saturated}.

\begin{definition}[saturated,~packing]\guid{XASMJUK}
A \newterm{packing} $ V\subset \ring{R}^3$ is a set such that
\begin{displaymath}\forall  \u,~\v\in  V.~  \norm{ \u}{\v} < 2 \Rightarrow ( \u=\v).\end{displaymath}
A set is \newterm{saturated} if for every $\p\in\ring{R}^3$, there
exists $ \u\in V$ such that $\norm{ \u}{\p}< 2$.
\end{definition}
\indy{Index}{saturated}%
\indy{Index}{packing}%



Let $B(\p,r)$ denote the open ball in
Euclidean three-space at center $\p$ and radius $r$.  The open ball
is measurable with measure $4\pi r^3/3$.
 Set $ V(\p,r) = V \cap
B(\p,r)$. %and $ V^*(\p,r) = V(\p,r)\setminus \{\p\}$.
\indy{Index}{measure}%
\indy{Notation}{B@$B(\p,r)$}%

\begin{lemma}[]\guid{KIUMVTC}
\oldrating{80}
\formalauthor{Nguyen Tat Thang}
\rating{0}
\label{lemma:V-finite}
Let $ V$ be a packing and let $\p\in\ring{R}^3$.
Then the set $ V(\p,r)$ is finite.
\end{lemma}

\begin{proof}  Let $\p = (p_1,p_2,p_3)$. The floor function gives the map
\begin{displaymath}\v=(v_1,v_2,v_3)\mapsto (\lfloor 2(v_1-p_1)
  \rfloor, \lfloor 2(v_2-p_2) \rfloor, \lfloor 2(v_3-p_3) \rfloor)\end{displaymath}
is a one-one map from $ V(\p,r)$ into the set $\ring{Z}^3\cap B(\orz,2r + 1)$.  
By Lemma~\ref{lemma:Zcount} the range of this one-to-one map is finite. 
Hence the domain $ V(\p,r)$ of the map is also finite.%
\footnote{An alternative proof uses the open cover of compact ball 
$\bar B(\p,r)$ by the sets $\bar B(\p,r)\setminus V$ and $B(\v,1)$, 
for $\v\in V$. By compactness, the cover is necessarily finite.}
\end{proof}
\indy{Notation}{1@$\lfloor\cdot\rfloor$ (floor)}%




\section{Rogers simplex}\label{sec:rogers}



% Think of $ V$ as the set of centers of a packing of congruent balls
% of radius $1$. To be saturated means that there is no room for
% further balls to be added to the packing. There is no loss in
% generality in assuming that the packing is saturated, when searching
% for the greatest possible density of a packing.


Given a packing $ V$, \cite{Rogers:1958:Packing} gives a partition of
Euclidean space into simplices with extreme points in $ V$.  He used
this partition to give a bound on the density of sphere packings in
Euclidean space of arbitrary dimension.  In two dimensions, the bound
is sharp and gives a solution to the sphere packing problem.


\subsection{Voronoi cell}

\begin{definition}[Voronoi~cell,~$\Omega$]\guid{YGFWXEH}\label{def:voronoi}
%\indy{Index}{Voronoi cell}% 
Let $V\subset\ring{R}^3$ and $\v\in V$.
The \newterm{Voronoi cell} 
$\Omega(V,\v)$
is the set of points at least as close to $\v$ as to
any other point in $V$. 
% Let $\Omega_t( V,\v) = \Omega( V,\v)
%\cap B(\v,t)$ be the truncated Voronoi cell at radius $t$.
\end{definition}
\indy{Notation}{ZZZomega@$\Omega$ (Voronoi cell)} %

\begin{remark}
  If $V$ is a saturated packing, then every point $\p$ has distance
  less than $2$ from some point of $V$.  The set $V(\p,2)$ is finite
  by Lemma~\ref{lemma:V-finite}.  Hence, $\p$ is at least as close to
  some $\v\in V$ as it is to any other $\w\in V$.  This means that
  $\p\in\Omega(V,\v)$.  It follows that
\begin{equation}\label{eqn:vor-rn}
\ring{R}^3 = \bigcup \{\Omega(V,\v)\mid \v \in V\}.
\end{equation}
\end{remark}

\begin{remark}
Set
\begin{eqnarray*}
A_+(\u,\v) &= \{\p\in\ring{R}^3
\mid 2(\v-\u)\cdot \p \le \normo{\v}^2 - \normo{\u}^2 \},\\
A(\u,\v) &= \{\p\in\ring{R}^3
\mid 2(\v-\u)\cdot \p = \normo{\v}^2 - \normo{\u}^2 \},\\
\end{eqnarray*}
when $\u,\v\in\ring{R}^3$.  The plane $A(\u,\v)$ is the bisector of
$\{\u,\v\}$ and $A_+(\u,\v)$ is the half-space of points at least as
close to $\u$ as to $\v$.  The Voronoi cell $\Omega(V,\v)$ is the
intersection of the half-spaces $A_+(\v,\u)$ as $\u$ runs over
$V\setminus \{\v\}$.
\end{remark}

\begin{lemma}[]\guid{RHWVGNP}\label{lemma:V4}
  Let $V\subset\ring{R}^3$ be a saturated packing.  Then
  $\Omega(V,\v)\subset B(\v,2)$.  Also, $\Omega(V,\v)$ is a polyhedron
  defined by the intersection of the finitely many half-spaces
  $A_+(\v,\u)$, for $\u\in V(\v,4)$.
\end{lemma}

\begin{proof} Let $\p\not\in B(\v,2)$.  
By saturation, there exists $\u\in V$ such that $\norm{\p}{\u}<2$.
Then 
\begin{displaymath}
\norm{\p}{\u} < 2 \le \norm{\p}{\v}.
\end{displaymath}
Hence, $\p\not\in\Omega(V,\v)$.  This proves the first conclusion.


Let $\Omega'$ be the intersection of the half-spaces $A_+(\v,\u)$ as
$\u$ runs over $V(\v,4)$.  Clearly, $\Omega(V,\v)\subset \Omega'$.
Assume for a contradiction that $\p\in \Omega'\setminus\Omega(V,\v)$.
The intersection of the ray $\op{aff}_+\{\v,\{\p\}\}$ with
$\Omega(V,\v)$ is a closed and bounded convex subset of the line.  By
general principles of convex sets, this intersection is an interval
$\op{conv}\{\v,\p'\}$, for some $\p'\in\Omega(V,\v)\subset B(\v,2)$.
For some small $t>0$, the point lies beyond the interval but remains
within the ball:
\begin{displaymath}
\q = (1+t)\p' -t \v\in (B(\v,2)\cap \Omega')\setminus \Omega(V,\v).
\end{displaymath}
Pick $\u\in V\setminus V(\v,4)$ such that $\q\in A_+(\u,\v)$.  By the
triangle inequality,
\begin{displaymath}
\norm{\u}{\v} \le \norm{\u}{\q} + \norm{\v}{\q} \le 2\norm{\v}{\q} < 4.
\end{displaymath}
This contradicts the assumption $\u\not\in V(\v,4)$.

The number of half-spaces $A_+(\v,\u)$, for $\u\in V(\v,4)$ is finite by
Lemma~\ref{lemma:V-finite}.  A set defined by the intersection of a finite number
of closed half-spaces is a polyhedron.
\end{proof}

\begin{lemma}[]\guid{DRUQUFE}
\formalauthor{Nguyen Tat Thang}
\oldrating{80} 
\rating{0}
Let $ V$ be a saturated packing.  For every $\v\in  V$, 
the Voronoi cell $\Omega( V,\v)$,  is
compact, convex, and measurable.
\end{lemma}

\begin{proof}  By the previous lemma, it is a bounded polyhedron.  Every bounded
polyhedron is compact, convex, and measurable.
\end{proof}




\begin{definition}[$\Omega$ reprise]\guid{BBDTRGC}
Let $V$ be a saturated packing.
The notation $\Omega(V,*)$ can be \newterm{overloaded} to denote intersections
of Voronoi cells, when the second argument is a set or list of points.
If $W\subset V$, %(or if $W$ is an ordered tuple of elements of $ V$), 
then the intersection of the family of Voronoi cells is  $\Omega(V,W)$:
\begin{displaymath}\Omega(V,W) = \bigcap \{\Omega(V, \u)\mid \u\in W
\}.\end{displaymath}
Define $\Omega$ on lists % by recursion:
%\begin{displaymath}
%\begin{array}{lll}
%\Omega(V,[]) &= \ring{R}^3,\\
%\Omega(V,\u_0\cooln[\u_1;\ldots;\u_k]) &= \Omega(V,\u_0)\cap \Omega(V,[\u_1;\ldots;\u_k]).\\
%\end{array}
%\end{displaymath}
to be the same as its value on point sets: 
\begin{eqnarray*}
%\Omega(V,[]) &= \ring{R}^3,\\
%\Omega(V,\u_0\cooln\bu) &= \Omega(V,\u_0)\cap \Omega(V,\bu).\\
\Omega(V,[\u_0;\ldots;\u_k]) = \Omega(V,\{\u_0;\ldots;\u_k\}).
\end{eqnarray*}
%Call $\Omega(V,W)$ the $W$-face of $\Omega(V, \u)$ when $ \u\in W$.
\end{definition}

An intersection of Voronoi cells can be written in many equivalent forms:
\begin{displaymath}
  \Omega(V,\v)\cap \Omega(V,\u) =\Omega(V,\{\u,\v\})= \Omega(V,\v)\cap A_+(\u,\v) 
  = \Omega(V,\v)\cap A(\u,\v) =  \cdots
\end{displaymath}





\begin{definition}[$\bV$]\guid{NOPZSEH}
Let $V$ be a saturated packing.
When $k=0,1,2,3$, let $ \bV(k)$ be the set of lists of length $k+1$:
$[\u_0;\ldots;\u_k]$, with $ \u_i\in V$ such that
\begin{equation}\label{eqn:omega-dim}
\dimaff(\Omega(V,[\u_0;\ldots;\u_j])) = 3-j,
\end{equation}
for all $0<j\le k$.  (Recall that $\dimaff(X)$ is the affine dimension
of $X$ from Definition~\ref{def:affine}.)  Set $\bV(k)=\emptyset$, for
$k>3$.  \indy{Notation}{dimaff@$\dimaff$}%
\end{definition}

In particular, $V$ can be identified with $\bV(0)$ under the natural
bijection $\v\mapsto[\v]$, and $\bV(1)$ is the set of lists $[\u;\v]$
of distinct elements such that the Voronoi cells at $ \u$ and $\v$
have a common facet, and so forth (Lemma~\ref{lemma:omega-facet}).  

\begin{notation}[underscore]
  % Given a saturated packing $V$, and natural number $k$, the
  % definition gives $\bV(k)$ as a certain set of lists of points in
  % $V$.
  We hope that the use of underscores does not lead to confusion.  The
  underscore in $\bV(k)$ is a function
\begin{displaymath}
\underline{\phantom V}:\{V \mid \text{$V$ saturated packing} \}
\times \ring{N} \to \ldots
\end{displaymath}
Contrast this with the use of the underscore in $\bu$.  Here the
underscore is not a function, but part of its name, following a
general typographic convention to mark lists of points. The two uses
are coherent in the sense that $\bu\in\bV(k)$.
\end{notation}

\begin{notation}[$\trunc{\bu}{j}$]
\indy{Notation}{u@$\bu$ (list of points)}%
When $\bu=[\u_0;\ldots;\u_k]$ and $j\le k$, write
%\footnote{The  notation follows the syntax of Python slices.} 
$\trunc{\bu}{j} = 
[\u_0;\ldots;\u_j]$ for the truncation of the list.  
\indy{Notation}{1@$\trunc{*}{:*}$}%
\end{notation}

Truncation $\bu\mapsto\trunc{\bu}{j}$ maps $\bV(k)$ to $\bV(j)$, when
$j\le k$.  Beware of the index: $k$ is the {\it codimension} of
$\Omega(V,\bu)$ in $\ring{R}^3$, when $\bu\in \bV(k)$; it is not the
{\it length} of the list $\bu$ (which is $k+1$).\footnote{Compare this
  to the convention that presents a $d$-simplex as a $d+1$-tuple.
  Because of this shift by $1$, the notation $\bu[:j]$ also differs by
  the same shift from Python slice notation.}


\begin{lemma}[]\guid{KHEJKCI}\label{lemma:omega-face}  
Let $V\subset\ring{R}^3$ be a saturated packing.
Let $\bu=[\u_0;\ldots;\u_{k}]\in \bV(k)$.  
Then $\Omega(V,\bu)$ is a face of $\Omega(V,\u_0)$.
\end{lemma}

\begin{proof} $\Omega(V,\bu)$ is the intersection of $\Omega(V,\u_0)$ with
the planes $A(\u_0,\u_i)$, where $i>0$.  
%Alternatively, it is the
%intersection of $\Omega(V,\u_0)$ with the planes $A(\u_i,\u_0)$, where $i>0$.
It follows directly from the definition of a face that each plane
$A(\u_0,\u_i)$ is a face of the polyhedron $A_+(\u_0,\u_i)$.  Thus, if
the ``open convex hull''
\begin{displaymath}
\op{aff}^0_+(\emptyset,\{\p,\q\}) = \{ s \p + t \q \mid s>0,\quad t>0,~\quad s + t = 1\}
\end{displaymath} 
meets
$\Omega(V,\bu)$ for some $\p,\q\in \Omega(V,\u_0)\subset
A_+(\u_0,\u_i)$, then it also meets the face $A(\u_i,\u_0)$ and $\p,\q$
must also lie in the face $A(\u_0,\u_i)$ of the polyhedron
$A_+(\u_0,\u_i)$ containing $\p,\q$ (by
the definition of face).  Then $\p,\q$ also lie in
$\Omega(V,\bu)$.  By the definition of face, $\Omega(V,\bu)$ is a face
of $\Omega(V,\u_0)$.
\end{proof}

\begin{lemma}[]\guid{IDBEZAL}\label{lemma:omega-facet} 
  Let $V\subset\ring{R}^3$ be a saturated packing.  Let $\bu\in
  \bV(k)$, for some $k<3$.  Then $F$ is a facet of $\Omega(V,\bu)$ if
  and only if there exists $\bv\in \bV(k+1)$ such that $F =
  \Omega(V,\bv)$ and $\trunc{\v}{k} = \bu$.
\end{lemma}

\begin{proof}
  Use Lemma~\ref{lemma:V4} to write the polyhedron $\Omega(V,\bu)$ in
  the form of Equation~\ref{eqn:polyrep}:
\begin{displaymath}
\Omega(V,\bu) = A \cap A_\pm(\v_1,\u_0) \cap \cdots\cap A_\pm(\v_r,\u_0),
\end{displaymath}
where $A$ is the affine hull of $\Omega(V,\bu)$, $\v_i\in V$, 
where $A_-(\v,\w) = A_+(\w,\v)$ with the signs $\pm$  chosen as
needed, 
and $r$ is as small as possible.
By Lemma~\ref{lemma:webster}, if $F$ is any facet of $\Omega(V,\bu)$, then there exists
an $i\le r$ such that 
\begin{displaymath}
F = \Omega(V,\bu) \cap A(\v_i,\u_0) = \Omega(V,\bv),
\end{displaymath}
where $\bv = [\u_0;\cdots;\u_k;\v_i]$ is the list that appends $\v_i$ to $\bu$.
Also, since $F$ is a facet:
\begin{displaymath}
\dimaff(\Omega(V,\bv)) = \dimaff(F) = \dimaff(\Omega(V,\bu)) - 1 = 3 - k - 1.
\end{displaymath}
So $\bv \in \bV(k+1)$.  This proves the implication in the forward direction.

To prove the converse, let $\bv\in \bV(k+1)$, where $\trunc{\bv}{k} =
\bu$.  Elementary verifications show that $\Omega(V,\bv)\subset
\Omega(V,\bu)$ and that this set is nonempty if $k<3$.  By
Lemma~\ref{lemma:omega-face} and Lemma~\ref{lemma:webster},
$\Omega(V,\bv)$ is a face of $\Omega(V,\bu)$.  By the definition of
$\bV(*)$,
\begin{displaymath}
\dimaff(\Omega(V,\bv)) = 3 - (k+1) = \dimaff(\Omega(V,\bu)) -1,
\end{displaymath}
so $\Omega(V,\bv)$ is a facet of $\Omega(V,\bu))$.
\end{proof}

\subsection{partition}

\begin{definition}[$\omega$]\guid{JJGTQMN}
Define $\omega: \coprod_{j=0}^3 \bV(j)\to \ring{R}^3$ by recursion over $j$ as follows.
Let \begin{displaymath}
\omega([\u]) = \u,
\end{displaymath} and
let $\omega( \bu)$ be the closest point on $\Omega(V, \bu)$ to
$\omega( \trunc{\bu}{j})$, when $\bu\in \bV(j+1)$.
\end{definition}

The point $\omega(\bu)$ exists when $\bu\in V(k)$, and $k\le 3$.
Indeed, the set $\Omega(V, \bu)$ is nonempty, convex, and compact.  Thus, by
convex analysis, the closest point $\omega( \bu)$ exists uniquely.
\indy{Notation}{zzomega@$\omega$ (extreme points of Rogers
  simplex)}%

\begin{definition}[R,~Rogers simplex]\guid{PHZVPFY}
Let $V\subset\ring{R}^3$ be a saturated packing. For $\bu\in \bV(k)$, let 
\begin{displaymath}
R(\bu) = \op{conv}\{\omega( \trunc{\bu}{0}), \omega(
\trunc{\bu}{j+1}),\ldots,\omega( \trunc{\bu}{k})\}.
\end{displaymath}
%Set $R(\bu)=R(0,\bu)$.  
The set $R(\u)$ is called the Rogers simplex of $\bu$.
\indy{Notation}{R@$R$ (Rogers simplex)}%
\end{definition}

\begin{lemma}[]\guid{GLTVHUM}\rating{200}
For any saturated packing $V\subset\ring{R}^3$, 
\begin{displaymath}\ring{R}^3 = \bigcup\, \{ R(\bu) \mid \bu\in
\bV(3)\}.\end{displaymath}
\end{lemma}

\begin{proof}
The proof uses the following standard facts about convex sets and polyhedra from
Section~\ref{sec:poly}.


%
Now turn to the proof.  
By the covering of $\ring{R}^3$  by Voronoi cells by (\ref{eqn:vor-rn}),
%\begin{displaymath}\ring{R}^3 = \bigcup\, \{\Omega(V, \bu)\mid \bu\in
%\bV(0)\}.\end{displaymath}
it is enough to show that each Voronoi cell is covered by Rogers simplices.

Let $\bu\in V(j)$, for $j<3$.
Consider the following set:
\begin{displaymath}
N = \left\{k\in\ring{N}\mid j\le k\le 3, ~~\Omega(V,\bu) 
= \bigcup_{\bv \in \bV(k),~\trunc{\bv}{j}=\bu}
\op{conv}(O_k \cup\Omega(V,\bv)) %\\&\qquad\qquad
%\mid
%\right\}
\right\},
\end{displaymath}
where $O_k = \{\omega(\trunc{\bv}{j}),\ldots,\omega(\trunc{\bv}{k-1})\}$.

\claim{We claim $N = \{j,\ldots,3\}$.}  Indeed, to see that $j\in N$, note that
\begin{displaymath}
\Omega(V,\bu) = \op{conv}(\Omega(V,\bu)),
\end{displaymath}
which holds by the convexity of the polyhedron $\Omega(V,\bu)$.
Assume that $k\in N$, and consider the membership condition of $N$  for
$k+1$.  We may assume that $k+1\le 3$.
Then
\begin{eqnarray*}
&\phantom{=}&\bigcup _{\bv \in \bV(k+1),~\trunc{\bv}{j}=\bu}
%\left\{
\op{conv}(O_{k+1} \cup\Omega(V,\bv))
%\mid%
%
%\right\}
\vspace{6pt}\\
&=&\bigcup _{\bv \in \bV(k+1),~\trunc{\bv}{j}=\bu}
%\left\{
\op{conv}(O_{k} \cup\op{conv}(\{\omega(\trunc{\bv}{k})\}\cup\Omega(V,\bv)))
%\mid
%\right\}
\vspace{6pt}\\
&=&\bigcup _{\bw\in V(k),~\trunc{\bw}{j}=\bu}
%\left\{
\bigcup_{\bv \in \bV(k+1),~\trunc{\bv}{k}=\bw}
%\left\{
\op{conv}(O_{k} \cup\op{conv}(\{\omega(\trunc{\bv}{k})\}\cup\Omega(V,\bv)))
%\mid
%\right\}\mid 
%\right\}
\vspace{6pt}\\
&=&\bigcup _{\bw\in V(k),~\trunc{\bw}{j}=\bu}
%\left\{
\op{conv}(O_{k} \cup \Omega(V,\bw)    
%\mid 
%\right\}
\vspace{6pt}\\
&=&\quad\Omega(V,\bu).
\end{eqnarray*}
The induction hypothesis is used in the last step.  
This proves $k+1\in N$, and induction gives $N=\{j,\ldots,3\}$.

Consider  the extreme case $j=0$ and $k=3$.  The set $\Omega(V,\bv)$
reduces to $\{\omega(\bv)\}$  and the convex hull becomes
\begin{displaymath}
\op{conv}(O_{k}\cup \Omega(V,\bv)) = R(\bv),
\end{displaymath}
when $\bv\in \bV(3)$.
This gives
\begin{equation}
\Omega(\u_0) = \bigcup \{ R(\bv) \mid \bv\in V(3),~\trunc{\bv}{0} =[\u_0]\}.
\end{equation}
This proves the lemma.
\end{proof}


%%%
%\begin{displaymath}\Omega(V,\trunc{\bu}{0}) = 
%\bigcup\, \{ R(\bv) \mid \trunc{\bu}{0}=\trunc{\bv}{0},~\bv\in  \bV(3)\}.
%\end{displaymath}
%By Lemma~\ref{lemma:webster}, the relative boundary of the bounded
%polyhedron $\Omega(V,\trunc{\bu}{0})$ is the union of its facets.  The
%polyhedron can be partition into the cones over these facets.
%\begin{displaymath}\Omega(V,\trunc{\bu}{0}) = \bigcup\, \left\{
%  ~\op{conv}(\{\omega(\trunc{\bu}{0})\}\cup \Omega(V,\trunc{\bu}{1}))   
%   \mid \trunc{\bu}{1}\in  \bV(1)
%   \right\}.
%\end{displaymath}
%It is enough to show that
%\begin{displaymath}
%\Omega(V,\trunc{\bu}{1}) = \bigcup\, \{ R(1,\bv) \mid \trunc{\bv}{1}=\trunc{\bu}{1},~\bv\in  \bV(3)\}.
%\end{displaymath}
%After successively partitioning each facet into cones over facets of
%facets, it is enough to show that
%\begin{displaymath}\Omega(V,\trunc{\bu}{3}) 
%= \bigcup\,\{R(3,\bv)\mid\trunc{\bv}{3}=\trunc{\bu}{3},~\bv\in \bV(3)\}.\end{displaymath}
%The right-hand side is the singleton $\{\omega(\trunc{\bu}{3})\}$.
% The left-hand side contains this point and is contained in a
% properly decreasing chain of affine sets: $\ring{R}^3$, the bisector
% of $ \u_0$ and $ \u_1$, and so forth.  This determines a unique
% point, so the two sides are equal.
%\end{proof}

\begin{lemma}[]\guid{DUUNHOR}\rating{140}  \label{lemma:R-inter}
Let $V$ be a saturated packing, and let $\bu,\bv\in \bV(3)$ be lists such that 
$R(\bu)\ne R(\bv)$.  Then the intersection 
\begin{displaymath}
R(\bu)\cap R(\bv)
\end{displaymath}
is contained in a plane (and hence has measure zero).
\end{lemma}

This result and the previous lemma show that the simplices $R(\bu)$
partition Euclidean three-space.

\begin{proof} Let $\bu = [\u_0;\ldots]$ and $\bv = [\v_0;\ldots]$.  
Let $k$ be the
first index such that
\begin{displaymath}
\Omega(V,[\u_0;\ldots;\u_{k}]) \ne \Omega(V,[\v_0;\ldots;\v_{k}]).
%\op{conv}\{\omega( [\u_0]),\ldots,\omega( [\u_0;\ldots; \u_{k}])\}\ne
%\op{conv}\{\omega([\v_0]),\ldots,\omega([\v_0;\ldots;\v_{k}])\}.
\end{displaymath}

\claim{Such an index $k$ exists.}  Indeed, the definition of points
$\omega(\trunc{\bu}{i})$ depends on $\bu$ only through the sets
$\Omega(V,\trunc{\bu}{j})$.  Hence, $R(\u)\ne R(\v)$ implies that the
two sequences $\Omega(V,*)$ must differ at some index.

The intersection $R(\u)\cap R(\v)$ lies in the convex hull $C$ of
$\{\omega([\u_0]),\ldots,\omega( [\u_0;\ldots; \u_{k-1}])\}$ and
$\Omega'=\Omega(V,[\u_0;\ldots;\u_{k};\v_{k}])$.  The set $\Omega'$
lies in a facet of $\Omega(V,\trunc{\bu}{k})$.  Hence the affine
dimension of $\Omega'$ is at most $3-k-1=2-k$.  In general, if a set
$A$ has affine dimension $r$, then the affine dimension of
$\op{conv}(\{\p\}\cup A)$ is at most $r+1$.  It follows that the
affine dimension of $C$ is at most $(2-k)+ k = 2$.  The intersection
is thus contained in a plane.
\end{proof}


\subsection{circumcenter}

\begin{definition}[circumcenter,~circumradius]\guid{IFLFHKT}
Let $S\subset\ring{R}^N$.  
A point $\p$ is a \newterm{circumcenter} of $S$ if it is an element
in the affine hull of $S$ that is equidistant from every $\v\in S$.  If $S$ has
circumcenter $\p$, then the common distance $\norm{\p}{\v}$, for all $\v\in S$,
is the \newterm{circumradius} of  $S$.
\end{definition}

Recall that a finite set $S$
is \newterm{affinely independent} if $\dimaff(S) = \card(S) -1$.

\begin{lemma}[]\guid{QXSKIIT}\label{lemma:affine-system}
Let $S=\{\v_0,\ldots,\v_n\}$ be an affinely independent set of cardinality $n+1$.
Then every system of equations
\begin{displaymath}
\p \cdot (\v_i - \v_0) = b_i,\qquad i=1,\ldots,n,
\end{displaymath}
has a unique solution in $\p$ that lies in the affine hull of $S$.
\end{lemma}

\begin{proof} This is a standard result from linear algebra~\cite{XX}.
We sketch a proof for the sake of completeness.  

Let $\w_i = \v_i-\v_0$.  The lemma reduces to the following claim.
Let $S' = \{\w_1,\ldots,\w_n\}$ be a \newterm{linearly independent} set
of cardinality $n$.  Then every system of equations
\begin{displaymath}
\p \cdot \w_i = b_i,\qquad i=1,\ldots,n,
\end{displaymath}
has a unique solution in $\p$ that lies in the linear span of $S'$.

\claim{A solution is unique}. Indeed, the difference 
$\p = \p'-\p'' = \sum s_i \w_i$ of two solutions
$\p',\p''$ satisfies
\begin{displaymath}
\normo{\p}^2=\p\cdot\p = \sum s_i \w_i \cdot (\p' - \p'') =
\sum s_i (b_i-b_i)= 0.
\end{displaymath}
So that $\p=0$ and $\p'=\p''$.  This proves uniqueness.

Let $W$ be the linear span of $\w_1,\ldots,\w_n$.  The image of the
map $W\to\ring{R}^n$, $\p\mapsto (\p\cdot\w_1,\ldots,\p\cdot\w_n)$ is
a linear space, hence an affine set.

\claim{A solution exists; that is, the image is all of $\ring{R}^n$.}
Otherwise, by Lemma~\ref{lemma:aff-u} some equation must hold;
that is, there exists $\u\ne \orz$  $\u\cdot \q =b$, for every point $\q$ in the image.  
As $\orz$ lies in the image, $b=0$.  Write $\p = \sum u_i \w_i\in W$.
Then
\begin{displaymath}
\normo{\p}^2 = \p\cdot\p = \sum u_i (\p \cdot \w_i) = \u\cdot \q = 0,
\end{displaymath} 
where $\q\in\ring{R}^n$ is the image of $\p\in W$.
Thus $\p=\orz$ so that $\u=0$, and we have reached a contradiction.
\end{proof}

\begin{lemma}[]\guid{OAPVION} 
Let $S$ be a non-empty affinely independent set.  Then there exists a unique
circumcenter of $S$.
\end{lemma}

\begin{proof}
A point $\p$ is a circumcenter if and only if it is a point in the affine hull of $S$
that satisfies the system of equations:
\begin{displaymath}
\norm{\p}{\v_i}^2 = \norm{\p}{\v_0}^2,\qquad i = i,\ldots,n.
\end{displaymath}
Equivalently,
\begin{displaymath}
\p\cdot (\v_i-\v_0) = b_i,\qquad i=1,\ldots,n,
\end{displaymath}
where $b_i = (\normo{\v_i}^2-\normo{\v_0}^2)/2$.  By
Lemma~\ref{lemma:affine-system}, this system of equations has a unique
solution.
\end{proof}

\begin{lemma}[]\guid{MHFTTZN}\label{lemma:affhull-center}
Let $V$ be a saturated packing and let $k\le 3$.
Let $\bu=[\u_0,\ldots,\u_k]\in \bV(k)$, and set $S = \{\u_0,\ldots,\u_k\}$.
Then
\begin{itemize}
\item $\dimaff (S)= k$.  (In particular, $\card\{\u_0,\ldots,\u_k\}=k+1$ and
$S$ is affinely independent.)
\item $\affhull{\Omega(V,\bu)}= \cap_{i=1}^k A(\u_0,\u_i).$
\item $\affhull{\Omega(V,\bu)} \cap \affhull(S) = \{\q\}$, 
where $\q$ is the circumcenter of $S$.
\item $(\affhull{\Omega(V,\bu)}-\q) \perp (\affhull(S)-\q)$, where
  $X-\q$ denotes the translate of a set $X$ by $-\q$, and $(\perp)$ is
  the orthogonality relation.
\end{itemize}
\end{lemma}
\indy{Notation}{1@$\perp$}


\begin{proof}  The proof is by induction on $k$.  

  \claim{The lemma holds when $k=0$.}  Indeed, $\Omega(V,\u_0)$ is an
  open set, so that its affine hull is $\ring{R}^3$.  This is the
  first conclusion.  The other conclusions reduce to trivial facts:
  $\dimaff\ring{R}^3 = 3$; $\dimaff\{\u_0\}=0$; $\ring{R}^3\cap
  \{\u_0\} = \{\u_0\}$; and $\ring{R}^3\perp \{\orz\}$.

Assume the induction hypothesis for $k$.  We may assume that $k<3$,
for otherwise there is nothing further to prove.  Let $\bu\in
\bV(k+1)$.  Let $\bv = \trunc{\bu}{k}\in \bV(k)$.  Let $\q$ be the
circumcenter of (the point set of) $\bv$.  Write $A_j = \cap_{i\le j}
A(\u_0,\u_i)$; $B_j = \affhull(\Omega(V,\trunc{\bu}{j}))$; $C_j =
\affhull\{\u_0,\ldots,\u_j\}$.%; $S_j = \{\u_0,\ldots,\u_j\}$.
By the induction hypothesis $A_k = B_k$.

\claim{We claim $\dimaff\{\u_0,\ldots,\u_{k+1}\}=k+1$.}
Otherwise, by general background facts about affine sets, $\u_{k+1}\in C_k$.
Write $\u_{k+1}-\q=\sum_{i\le k} t_i (\u_i-\q)$.  If $\p\in A_k$, then
by the orthogonality induction hypothesis:
\begin{eqnarray*}
(\u_{k+1}-\q)\cdot (\p-\q) &=& \sum t_i (\u_i-\q)\cdot (\p-\q) = 0, \text{ and }\\
\norm{\u_{k+1}}{\p}^2 - \norm{\u_0}{\p}^2 &=&  
\norm{\u_{k+1}}{\q}^2 - \norm{\u_0}{\q}^2.
\end{eqnarray*}
Thus, if $A_k$ meets $A(\u_0,\u_{k+1})$, then $A_k\subset A(\u_0,\u_{k+1})$.
This is contrary to 
$0\le \dimaff(A_{k+1}) = \dimaff(A_k) - 1$, which holds because $\bu\in \bV(k+1)$
with $k<3$.

\claim{We claim that $B_{k+1} = A_{k+1}$.}  Indeed, by definition,
$B_{k+1}\subset A_{k+1}\subset A_k$.  Also,
\begin{displaymath}
\dimaff B_{k+1} = 3 - (k+1) \le \dimaff{A_{k+1}} \le \dimaff A_k = 3 - k.
\end{displaymath}
Hence, by general background on affine sets, if $A_{k+1}\ne A_k$, then
$B_{k+1}=A_{k+1}$ follows.  Suppose for a contradiction that $A_k =
A_{k+1}$.  Then $\Omega(V,\bv) \subset \Omega(V,\bu) =
\Omega(V,\bv)\cap A(\u_0,\u_{k+1}) \subset \Omega(V,\bv)$, so that
$B_k = B_{k+1}$.  This contradicts the defining conditions of
$\bV(k+1)$.

\claim{We claim that $A_{k+1}\cap C_{k+1} = \{\q_{k+1}\}$, where
  $\q_{k+1}$ is the circumradius of $S_{k+1}$.}  Indeed, by the
definition of $A_{k+1}$, any point in this affine set is equidistant
from every point of $S_{k+1}$.  By the definition of $C_{k+1}$, the
point lies in the affine hull of $S_{k+1}$.  This uniquely
characterizes the circumcenter.

\claim{Finally, $(A_{k+1} -\q)\perp (C_{k+1}-\q)$, where $\q=\q_{k+1}$.}
Indeed, if $\p\in A_{k+1}$, then
\begin{eqnarray*}
0 &=&\norm{\p}{\u_i}^2 -\norm{\p}{\u_0}^2\\
&=&\norm{(\p-\q)}{(\u_i-\q)}^2 -\norm{(\p-\q)}{(\u_0-\q)}^2\\
&=&-2 (\p-\q)\cdot (\u_i-\u_0).
\end{eqnarray*}
Since the linear span of the points $\u_i-\u_0$ is all of $C_{k+1}-\q$, the claim follows.

This completes the induction.
\end{proof}

\begin{definition}[h]\guid{CHNGQBD}
%Let $h(\trunc{\bu}{i}) = \norm{\omega(\trunc{\bu}{i})}{ \u_0}$.  
If $\bu=[\u_0;\u_0;\ldots;\u_k]\in\bV(k)$,
let $h(\bu)$ be the
circumradius of its point set $\{ \u_0,\ldots, \u_k\}$.
\end{definition}
\indy{Notation}{h@$h$ (circumradius)}%

\begin{lemma}[]\guid{XXOFCGX}\rating{400}\label{lemma:sqrt2-close}
Let $V\subset\ring{R}^3$ be a saturated packing.  Let $S\subset V$ be an
affinely independent set with circumcenter $\p$.  Assume that the circumradius
of $S$ is less than $\sqrt2$.  Then $\norm{\v}{\p}>\norm{\u}{\p}$, for all $\u\in S$
and all $\v\in V\setminus S$.
\end{lemma}

\begin{proof}
Otherwise
there is a point $\w\in V\setminus S$ satisfying
\begin{equation}\label{eqn:closest}
\norm{\w}{\p}\le \norm{ \u}{\p}, \quad\text{for all }  \u\in S.
\end{equation}
The angles $\arc_V(\p,\{\v, \u\})$ are obtuse for distinct elements $\v,\u$ of $ S$ because
of the law of cosines and
\begin{displaymath}
\norm{\p}{\u} < \sqrt2,\quad \norm{\p}{ \v} <\sqrt2,\quad \norm{\u}{ \v} \ge 2.
\end{displaymath} 
Let $S=\{\u_0,\ldots,\u_k\}$.
A case-by-case argument follows, for each $k\in\{0,1,2,3\}$.

\claim{[$k=0$].}  This case is trivial.

\claim{[$k=1$].}  In this case, the points $\p, \u_0, \u_1$ are collinear and cannot give
two obtuse angles.

\claim{[$k=2$].} In this case, let $\w'$ be the projection of $\w$ to
the plane containing $\p, \u_0, \u_1, \u_2$.  Under orthogonal
projection, the angles remain obtuse:
\begin{displaymath}
\arc_V(\p,\{\w,\u_i\} = \arc_V(\p,\{\w',\u_i\}).
\end{displaymath}
The four points $\w', \u_0, \u_1, \u_2$ can
be arranged cyclically around $\p$, according to the polar cycle,
each forming an obtuse angle with
the next.  A circle around $\p$ cannot give four obtuse angles, because the sum is
$2\pi$.

\claim{[$k=3$].}
In this case, assume that $ \u_0,\ldots, \u_3$ are labeled according to the azimuth
cycle
around the line $\op{aff}\{\p,\w\}$.  Consider the dihedral angle
\begin{displaymath}
\gamma=\gamma_i=\dih(\{\p,\w\},\{ \u_i, \u_{i+1}\})
\end{displaymath}
of the simplex $\{\p,\w,\u_i,\u_{i+1}\}$ along the edge $\{\p,\w\}$.
By the spherical law of cosines, the angle $\gamma$ of the
spherical triangle is given in terms of the edges as
\begin{displaymath}
\cos c - \cos a \cos b = \sin a \sin b \cos \gamma.
\end{displaymath}
The angles $a,b,c$ are obtuse, so that
both terms on the left-hand side are negative. Thus, $\gamma>\pi/2$.
The azimuth angle $\op{azim}(\p,\w,\u_i,\u_{i+1})$ is then also greater than $\pi/2$
by Lemma~\ref{lemma:dih-azim}.
This is impossible, as the sum of the four azimuth angles $\gamma$
is $2\pi$ by Lemma~\ref{lemma:2pi-sum}.
%This completes the proof that $\omega(\trunc{\bu}{j})$
%is the circumcenter.
\end{proof}

\begin{lemma}[]\guid{XNHPWAB}\rating{200}\label{lemma:v2} 
Let $V$ be a saturated packing.
Let $\bu=[\u_0;\ldots;\u_k]\in \bV(k)$, for some $k\le 3$,
and let $S=\{\u_0,\ldots,\u_k\}$ be the
point set of $\bu$.
Assume that $h(\bu)<\sqrt2$.
%\begin{displaymath}
%\norm{\omega(\trunc{\bu}{j}) }{  \u_0} < \sqrt2.
%\end{displaymath}
Then 
%\begin{description}
\begin{itemize}
\item%[(circumcenter)]  
$\omega(\bu)$ is the circumcenter of $S$.
\item%[(convex hull)]  
$\omega(\bu)\in\op{conv}(S)$.
\item%[(distinctness)]  
The set $\{\omega(\trunc{\bu}{j})\mid j\le k\}$ has affine dimension $k$.
\item
The sequence $h(\trunc{\bu}{j})$ is
strictly increasing in $j$.
\end{itemize}
%\end{description}
\end{lemma}
\indy{Index}{convex hull}%

\begin{proof} The three conclusions of the lemma will be proved
separately.

\claim{$\omega(\bu)$ is the circumcenter of $S$.} Indeed, by
definition, if $\bu\in \bV(k)$, then
\begin{displaymath}
\dimaff\Omega(V,[\u_0;\ldots;\u_k]) = 3-k.
\end{displaymath}  
The case $k=0$ of the lemma is trivially
satisfied.  Assume by induction the result holds for natural numbers up to $k$.

Now consider the case $k+1$.  Let $\bu\in \bV(k+1)$, and let $S$ be
the point set of $\bu$.  By the induction hypothesis
$\omega(\trunc{\bu}{k})$ is the circumcenter of the point set of
$\trunc{\bu}{k}$.  Let $\p$ be the point in
$A=\affhull(\Omega(V,\bu))$ closest to $\omega(\trunc{\bu}{k})$.  By
Lemma~\ref{lemma:sqrt2-close}, the point $\p\in\Omega(V,S_{j})$.
Thus, $\p=\omega(\bu)$.  By Lemma~\ref{lemma:affhull-center}, the
circumcenter of $S$ is the point of intersection of orthogonal affine
sets $\affhull(S)$ and $A$.  Thus, the circumcenter equals the unique
point of $A$ closest to any point $\omega(\trunc{\bu}{k})$ in
$\affhull(S)$.  The claim follows.

%The circumcenter $\p=\p_{j}$ of the set $S_{j}$ is
%the point on the plane $\op{aff}(\Omega(V,S_{j}))$ closest to
%$\p_{j-1}=\omega(\trunc{\bu}{j-1})$.

% If $j=0$, there is nothing to show.  If $j=1$, the point is the
% midpoint of the convex hull.  If $j=2$, the point is the
% circumcenter of an acute triangle.  If $j=3$, the point is the
% circumcenter of a simplex such that every face has positive
% orientation.  Thus, in every case the point lies in the convex hull.

\claim{We claim $\omega(\bu)\in\op{conv}(S)$.} 
Otherwise, there $\v\in S$ such that $\affhull(S')$ separates $\omega(\bu)$ from
$ \v$, where $S'=S\setminus\{v\}$.  Let $\p'$ (resp. $\p=\omega(\bu)$)
be the circumcenter
of $S'$ (resp. $S$).  When $\u\in S'$, the law of cosines gives
\begin{eqnarray*}
\norm{ \u}{\p}^2 &=& \norm{\u}{\p'}^2 + \norm{\p'}{\p}^2\\ 
\norm{ \v}{\p}^2 &\ge& \norm{\v}{\p'}^2 + \norm{\p'}{\p}^2.
\end{eqnarray*}
This gives $\norm{\v}{\p'}\le \norm{\u}{\p'}$.  This is contrary to Lemma~\ref{sqrt2-close}.

\claim{The set $\{\omega(\trunc{\bu}{j})\mid j\le k\}$ has affine dimension $k$.}
%\claim{The points $\omega(\trunc{\bu}{j})$, for
%$j\le k$, are all distinct.}
It follows from  Lemma~\ref{lemma:affhull-center} that the vectors 
${\omega(\trunc{\bu}{i+1})}-{\omega(\trunc{\bu}{i})}$ are mutually orthogonal.
Thus, the claim about affine dimension easily follows if we show that these vectors
are nonzero.
%Indeed, by the Pythagorean theorem,
%\begin{equation}
%\norm{\omega(\trunc{\bu}{j})}{\omega(\trunc{\bu}{0})}^2 =
%\sum_{i=0}^{j-1} \norm{\omega(\trunc{\bu}{i+1})}{\omega(\trunc{\bu}{i})}^2.
%\end{equation}
%so it is enough to show that $\omega(\trunc{\bu}{i})\ne
%\omega(\trunc{\bu}{i+1})$.  
Otherwise, the
circumcenter $\omega(\trunc{\bu}{i})$ of $S_i=\{\u_0,\ldots,\u_i\}$
has an equally close point $ \u_{i+1}\in V\setminus S_i$, which is
impossible by Lemma~\ref{lemma:sqrt2-close}.

\claim{The sequence $h(\trunc{\bu}{j})$ is strictly increasing in
  $j$.}  
Indeed, by the Pythagorean theorem,
\begin{equation}
\norm{\omega(\trunc{\bu}{j})}{\omega(\trunc{\bu}{0})}^2 =
\sum_{i=0}^{j-1} \norm{\omega(\trunc{\bu}{i+1})}{\omega(\trunc{\bu}{i})}^2.
\end{equation}
so the result follows from the
previous claim.
\end{proof}


% The concept of {\it positive orientation} is used in the proof.
% This is discussed in the 1998 proof and in {\it Lemmas in Geometry}.
% If a face has circumradius less than $\sqrt2$ it has positive
% orientation.  If every face has positive orientation, then the
% circumcenter of the simplex is contained in its convex hull.


\subsection{rearrangement}


Let $\op{Sym}(k+1)$ be the \newterm{group} of all permutations on the
set $\{0,\ldots,k\}$.  Let $\bu = ( \u_0,\ldots, \u_k)\in \bV(k)$.  For any
\newterm{permutation} $\pi\in\op{Sym}(k+1)$, let $\pi_*(\bu)\in
\bV(k)$ be the \newterm{rearrangement} given by
\begin{displaymath}
\pi_*(\bu)_i =  \u_{\pi i},
\end{displaymath}   
where $\u_i$ denotes the $i$th element of a list $\bu$.
\indy{Notation}{Xpi@$\pi$ (permutation)}


\begin{lemma}[]\guid{YIFVQDV}\rating{140}   \label{lemma:perm-Vk}
  Let $V$ be a saturated packing.  Let $\bu\in \bV(k)$.  Assume that
  $h(\bu)<\sqrt2$. Let $\bv$ be any rearrangement of $\bu$ under a
  permutation.  Then $\bv\in \bV(k)$ and $\omega(\bu) = \omega(\bv)$.
\end{lemma}

\begin{proof} 
Let $\bv = [\v_0;\ldots;\v_k]$.  Let $S_j = \{\v_0,\ldots,\v_j\}$,  $\Omega_j = \Omega(V,\trunc{\bv}{j})$, $A_j=\cap_{i=1}^j A(\v_0,\v_i)$, $a_j = \dimaff(A_j)$.
By convention, set $A_0 = \ring{R}^3$ so that $a_0=\dimaff(A_0) = 3$.
Also, set $a_{-1} = 4$ by convention.

The set $S_k$ is the point set of $\bu$,
which is affinely independent by Lemma~\ref{lemma:affhull-center}.  The set $S_j$ is then also affinely independent.  Let $p_j$ be the circumcenter of $S_j$.  Then $p_k$ is the
circumcenter of $S_k$.  The circumradius of $S_j$ is at most the circumradius of $S_k$,
which by assumption is less than $\sqrt2$.

\claim{We claim that $\dimaff \Omega_j = a_j$, when $0\le j\le k$.}
By Lemma~\ref{lemma:sqrt2-close}, if $\p=\p_j$, then
\begin{equation}\label{eqn:sqrt2-close}
\norm{\v}{\p} > \norm{\u}{\p}\text{ for all }\u\in S_j
\text{ and for all }\v\in V\setminus S_j.
\end{equation}   
Pick a small neighborhood $U$ of $\p_j$ such that (\ref{eqn:sqrt2-close}) holds
for all $\p\in U_j$.  By the definition of Voronoi cell, $\Omega_j \cap U=A_j\cap U$.
By background facts on affine sets $\dimaff\Omega_j = \dimaff A_j=a_j$.  This gives
the claim.

To prove the lemma, we prove the following claims by simultaneous induction on $j$.
For all $0\le j\le k$ we have
\begin{itemize}
\item $a_j \ge a_{j-1} - 1\ge 3-j$.
\item $a_j = 3-j$ if and only if $a_i=3-i$ for all $0\le i\le j$.
\end{itemize}
The base case $j=0$ is trivial.  Assume the induction hypothesis for $j$.

$A_{j+1} = A_{j}\cap A(\v_0,\v_{j+1})$.  The intersection contains $\p_{j+1}$ and is therefore
nonempty.  By general background facts on the intersection of an
affine set with a hyperplane, $a_{j+1} \ge a_{j}-1$.  By the induction hypothesis,
$a_{j}-1\ge 3-(j+1)$.
If $a_{j+1}=3-(j+1)$, then $a_{j}=3-j$ and by the induction hypothesis
$a_{i}=3-i$, for all $0\le i\le j$. This completes the proofs of the claims by induction.

$a_k = \dimaff A_k=\dimaff \Omega_k$.  However, $\Omega_k=
\Omega(V,\bu)$, and since $\bu\in \bV(k)$, it follows that $0=\dimaff\Omega(V,\bu)=a_k$.
By the established claims, $a_i = 3-i$ for all $0\le i\le k$.  This proves $\bv\in\bV(k)$.

Finally $\omega(\bu) = \omega(\bv)$ because both equal the circumcenter of the
point set $S_k$.
%Since the sets $\Omega(V,\trunc{\bu}{j})$ satisfy \eqn{eqn:omega-dim}, it
%follows that $\Omega(V,\bu)\cap \op{conv}\{ \u_0,\ldots, \u_k\}$ is
%the singleton $\{\omega(\bu)\}$, which contains the circumcenter of
%the simplex with extreme points $\{ \u_0,\ldots, \u_k\}$.  This describes
%$\omega(\bu)$ in a way that does not depend on the ordering of $
%\u_0,\ldots, \u_k$.
%
%The condition \eqn{eqn:omega-dim} can be shown to hold for $\bv$.
%The proof of Lemma~\ref{lemma:v2} shows that the midpoint of $\v_0$
%and $\v_1\}$ lies in $\Omega(V,\trunc{\bv}{1})$.  By the distinctness
%conclusion of the same lemma, some neighborhood of this midpoint in
%the bisecting plane of $\{\v_0,\v_1\}$ lies in $\Omega(V,\trunc{\bv}{1})$.
%Thus, $\dimaff(\Omega(V,\trunc{\bv}{1}))=2$.  Continue in this fashion to show
%that
%\begin{displaymath}
%\dimaff(\Omega(V,\trunc{\bv}{j}))=3-j.\end{displaymath}
\end{proof}

\begin{lemma}[]\guid{KSOQKWL}
  Let $V$ be a saturated packing and let $\bu = [\u_0;\ldots;\u_k]\in
  \bV(k)$.  Assume that $h(\bu)<\sqrt2$.  Let $\pi\in\op{Sym}(k+1)$, with $\pi\ne I$.
  Then $R(\bu)\ne R(\pi_*\bu)$.
\end{lemma}

\begin{proof}
Write $\bv = \pi_*\bu$.  By Lemma~\ref{lemma:v2}, the sets
$\{\omega(\trunc{\bu}{j}\mid j\le k\}$ and 
$\{\omega(\trunc{\bv}{j}\mid j\le k\}$ are each affinely independent of
cardinality $k+1$.  By Lemma~\ref{lemma:simplex-poly}, these are these sets
of extreme points of $R(\bu)$ and $R(\bv)$, respectively.  Thus, it is enough
to show that the sets of extreme points are unequal.

Let $j$ be the largest index such that $\trunc{\bu}{j}=\trunc{\bv}{j}$.
The assumption $\pi\ne I$ implies that $j<k$.  Let $\p$ be the circumcenter
of $\{\u_0,\ldots,\u_{j+1}\}$.  By Lemma~\ref{lemma:sqrt2-close}, 
\begin{displaymath}
\norm{\u_0}{\p} = \norm{\u_{j+1}}{\p} < \norm{\v_{j+1}}{\p}.
\end{displaymath}
Thus, $\omega(\trunc{\bu}{j+1}) \ne \omega(\trunc{\bv}{j+1})$.  The result follows.
\end{proof}

To prepare for Lemma~\ref{lemma:Rconv}, we need a preliminary lemma that
does some index shuffling for us.

\begin{lemma}[]\guid{IVFICRK}\label{lemma:coset-bijection}
There is a bijection between the set 
\begin{displaymath}
\{(i,\sigma)\mid 0\le i\le k,\quad \sigma\in \op{Sym}(k)\},
\end{displaymath}
and $\op{Sym}(k+1)$ 
that sends $(i,\sigma)$ to the permutation $\pi$, where
\begin{displaymath}
\pi j = \begin{cases}
\sigma j, &j<i\\
k,&j=i\\
\sigma({j-1}),&j>i\\
\end{cases}
\end{displaymath}
Under this bijection $\pi_*[\u_0;\ldots;\u_{k+1}]=[\u_{\sigma 0};\ldots;\u_{\sigma (k-1)};\u_i]$.
\end{lemma}

\begin{proof}
This is left as an exercise to the reader.
\end{proof}

\begin{lemma}[]\guid{WQPRRDY}\rating{400}\label{lemma:Rconv}  
  Let $V$ be a saturated packing and let $\bu = [\u_0;\ldots;\u_k]\in
  \bV(k)$.  Assume that $h(\bu)<\sqrt2$.  Let $\pi\in\op{Sym}(k+1)$
  Then
\begin{displaymath}
\op{conv}\{ \u_0,\ldots, \u_k\} = \bigcup \,\{ R(\pi_*(\bu)) : \pi\in \op{Sym}(k+1)\}.
\end{displaymath}
\end{lemma}
\indy{Notation}{Sym@$\op{Sym}$ (symmetric group)}%

\begin{proof} The proof is by induction on $k$.  The base case of the induction $k=0$
reduces to the trivial assertion: $\op{conv}\{\u_0\} = \op{conv}\{\u_0\}$.  Assume
the result holds for $k$.

Let $\bu=[\u_0;\ldots;\u_{k+1}]\in \bV(k+1)$.  For each $i$,  let
$\bu^i = [\u_0;\ldots;\hat\u_i;\ldots;\u_{k+1}]$, which drops the entry with subscript $i$.

\claim{We claim $\bu^i\in \bV(k)$.}  Indeed, there is some permutation
$\pi\in \op{Sym}(k+2)$ that carries $\bu$ to
$\bv=[\u_0;\ldots;\hat\u_i;\ldots;\u_{k+1};\u_i]$.  By
Lemma~\ref{lemma:perm-Vk}, $\bv\in \bV(k+1)$, so that $\bu^i =
\trunc{\bv}{k}\in \bV(k)$.

By the induction hypothesis 
\begin{equation}\label{eqn:sigma}
\op{conv}(S\setminus\{\u_i\}) = \bigcup \,\{ R(\sigma_*(\bu^i)) : \sigma\in \op{Sym}(k+1)\}.
\end{equation}
By Lemma~\ref{lemma:simplex-poly}, the facets of the polyhedron
$\op{conv}(S)$ are the sets $\op{conv}(S\setminus\{\u_i\})$.
Lemma~\ref{lemma:facet-partition} gives the partition
\begin{displaymath}
\op{conv}(S) = \bigcup_{i=0}^{k+1} \op{conv}(\{\omega(\bu)\}\cup \op{conv}(S\setminus\{\u_i\})).
\end{displaymath}
Substitute the formula (\ref{eqn:sigma}) in, then use the bijection of Lemma~\ref{lemma:coset} to replace the double union by a single union over $\pi\in \op{Sym}(k+2)$.  Routine background
facts in affine geometry then simplify the expression to the desired formula.
The induction follows.
%
%
%Let $L = \{ \u_0,\ldots, \u_k\}$.  The proof is by
%induction on $k$.  When $k=0$, the result is trivial.  Now assume
%$k>0$.
%
%The circumcenter $\omega(\bu)$ of $L$ lies in the convex hull of
%these points.  (See the proof of Lemma~\ref{lemma:v2}.)  Thus, the
%left-hand side is the union of cones:
%\begin{displaymath}
%\op{conv}(W) = \bigcup\,\{ \op{conv}(\omega(\bu),L\setminus \{ \u_i\}\mid i=0,\ldots,k) \}.
%\end{displaymath}
%The sets $L\setminus \{ \u_i\}$ can be identified with cosets of
%$\op{Sym}(k+1) /\op{Sym}(k)$.  By induction $L\setminus \{ \u_i\}$ is
%the union of $R(\bv)$ as $\bv$ runs over all permutations
%$\op{Sym}(k)$ of $L\setminus \{ \u_i\}$.  The result follows by
%induction.
\end{proof}



In summary,
by construction, the Rogers simplices $R(\bu)$  are
compatible with the Voronoi decomposition of space.  Under mild
restrictions on the circumradius, they can also be reassembled into
simplices with extreme points at the centers of the packing (the Delaunay
simplices), by Lemma~\ref{lemma:Rconv}.
\indy{Index}{simplex!Delaunay}%
\indy{Index}{partition!Rogers}%
\indy{Index}{decomposition!Voronoi} %



\section{Marchal cells}

\cite{marchal:2008} has proposed an approach to sphere packings
 that gives significant improvements to the original proof
in~\cite{Hales:2006:DCG}.  The definition of $k$-cells,
Conjecture~\ref{conj:m1}, Theorem~\ref{theorem:mk1}, and the method of
Lemma~\ref{lemma:13-14} are all due to him.  
\indy{Index}{Marchal, C.}%

% His articles claim to give a {\it demonstration} of the Kepler
% conjecture \cite{marchal:2007}, \cite{marchal:2008}.  However, the
% mathematically rigorous part of the article only gives a reduction
% of the problem to a difficult optimization problem in a finite
% number of variables.  The method of gradient descent is then used to
% explore the local minima of the optimization problem in finitely
% many variables.

Marchal's partition of space is a variant of Rogers's partition into
the simplices $R(\bu)$.  The main part of construction is the
decomposition obtained by truncating the Voronoi cells by a ball of
radius $\sqrt2$.  In a few carefully chosen situations, he assembles
the simplices $R(\bu)$ into larger convex cells, as suggested by
Lemma~\ref{lemma:Rconv}.


\begin{definition}[Marchal cells]\guid{QEEHXUB} 
Let $V$ be a saturated packing.  
A set $\cell(\bu,i)\subset\ring{R}^3$ is associated with each $\bu=( \u_0,\ldots,
\u_3)\in \bV(3)$ and $i=0,1,2,3,4$.  
\hfill\break\smallskip \case{The $0$-cell} of $\bu$
is
\begin{displaymath}
\cell(\bu,0) = \{\p\in R(\bu) \mid \norm{\omega([ \u_0])}{\p} > \sqrt2\}.
\end{displaymath}
\bigskip
\case{The $1$-cell} of $\bu$ is 
\begin{displaymath}
\cell(\bu,1) = \op{conv}(\{\omega([ \u_0])\}\cup R_1),\hbox{ where } R_1 = \{\p \in R(\bu) \mid \norm{\omega([ \u_0])}{\p}= \sqrt2\}.
\end{displaymath}
\indy{Notation}{R@$R_1$ (Rogers simplex)}%
\indy{Index}{cell}%
\indy{Index}{Marchal cell}%
\bigskip
\case{The $2$-cell} of $\bu$ is
\begin{eqnarray*}
\cell(\bu,2) &=& \op{conv}( \{\u_0, \u_1\}\cup R_2),\quad\text{where }  \vspace{6pt}\\
R_2 &=& \{\p \in R(\bu)\cap \Omega(V,\trunc{\bu}{1}) \mid \norm{ \u_0}{\p}=\norm{ \u_1}{\p} =\sqrt2\}.
\end{eqnarray*}
\bigskip
\case{The $3$-cell} of $\bu$ is defined to be empty unless 
\begin{displaymath}
h(\trunc{\bu}{2}) <\sqrt2 \le h(\bu).
\end{displaymath}
When this inequality holds, there is a a unique point $\xi(\trunc{\bu}{2})$ in
$\op{conv}\{\omega(\trunc{\bu}{2}),\omega(\bu)\}$ at distance exactly
$\sqrt2$ from $ \u_0$.  Define the $3$ cell to be
\begin{displaymath}
\cell(\bu,3) = \op{conv}\{ \u_0, \u_1, \u_2,\xi(\trunc{\bu}{2})\}.
\end{displaymath}
\bigskip
\case{The $4$-cell} of $\bu$ is defined to be empty unless
\begin{displaymath}
h(\bu) <\sqrt2.
\end{displaymath}
When this inequality holds, define the $4$ cell to be
\begin{displaymath}
\cell(\bu,4) = \op{conv}\{ \u_0, \u_1, \u_2, \u_3\}.
\end{displaymath}
\end{definition}
\indy{Notation}{zzxi@$\xi(\cdot)$ (Marchal cell parameter)}%

Note the the $0$ and $1$-cells are always subsets of a single simplex
$R$.  However, the $2$, $3$, and $4$-cells lie in a union of
simplices.  The index $i$ in  $\cell(\bu,i)$ indicates the number
of points of $V$ that are extreme points of the cell. 

\begin{definition}[$i$-rearrangement]\guid{BGXEVQU}
Let $\bu=[\u_0;\ldots;\u_k],\bv=[\v_0;\ldots;\v_k]$ be two lists of the same length.  
One is an
 $i$-\newterm{rearrangement} of the other if
$\pi_*\bu = \bv$ for some $\pi\in\op{Sym}(k+1)$ such
that 
\begin{displaymath}
\pi(i) = \pi(i+1)=\cdots=\pi(k).
\end{displaymath}
\end{definition}

In particular, if $\bu,\bv$ are $0$- or $1$-rearrangements of one another,
then $\bu = \bv$.


\begin{lemma}[]\guid{EMNWUUS}\label{lemma:M-complement4}
Let $V$ be a saturated packing.  Let $\bu\in \bV(3)$.
The following are equivalent.
\begin{itemize}
\item  $\cell(\bu,i)=\emptyset$ for $i=0,1,2,3$.
\item  $\cell(\bu,4)\ne\emptyset$.
\item  $h(\bu)<\sqrt2$.
\end{itemize}
\end{lemma}

\begin{proof}
The diameter of $R(\bu)$ is easily seen to be $h(\bu)$.  Hence if $h(\bu)<\sqrt2$
all of the defining conditions are empty for $\cell(\bu,i)$, for $i<4$.  The result follows.
\end{proof}

\begin{lemma}[]\guid{SLTSTLO}\label{lemma:M-exhaust}
Let $V$ be a saturated packing.  Then every point in $\ring{R}^3$ belongs to
a Marchal cell of $V$.  In fact, if $\p\in R(\bu)$ for some $\bu\in \bV(k)$, 
then for some $0\le i\le 3$, $\p\in \cell(\bu,i)$.
\end{lemma}

\begin{proof}
By Rogers's partition, a point $\p$ belongs to a simplex
$R(\bu)$, for some $\bu\in\bV(3)$.  

\noindent
\claim{$[h(\bu)<\sqrt2]$.} In this case,  $\p\in\cell(\bu,4)$.

\noindent
\claim{$[\norm{\p}{\u_0} >\sqrt2]$.} In this case, $\p\in\cell(\bu,0)$.

\noindent
\claim{$[\p\not\in\op{rcone}(\u_0,\u_1,\omega(\trunc{\bu}{1})/\sqrt2)]$.} 
In this case, $\p\in \cell(\bu,1)$.

\noindent
\claim{$[\p\not\in W(\u_0,\u_1,\u_2,\xi(\trunc{\bu}{2}))]$.} In this
case, $\p\in \cell(\bu,2)$.

\noindent
\claim{$[\p\in W(\u_0,\u_1,\u_2,\xi(\trunc{\bu}{2}))]$.} In this
case, $\p\in \cell(\bu,3)$.
\end{proof}




\begin{lemma}[]\guid{RVFXZBU}\rating{400}\label{lemma:marchal-equal}
Let $V$ be a saturated packing, 
let $\bu,\bv\in \bV(3)$, and let $i,j\in \{0,1,2,3,4\}$.
%Assume that the affine dimension of $\cell(\bu,i)$ is $3$.
If the intersection, of
$\cell(\bu,i)$ with $\cell(\bv,j)$ is not a null measure set,
then $i=j$ and $\bu$ is an $i$-rearrangement of $\bv$.
Conversely, if $i=j$ and $\bu$ is an $i$-rearrangement of $\bv$, 
then $\cell(\bu,i)=\cell(\bv,j)$.
\end{lemma}

\begin{proof} 
The converse conclusion follows directly from the definition of Marchal cells.

Let $\cell(\bu,i)$ and $\cell(\bv,j)$ be two cells whose intersection $X$ is
not a null set.  There exists $\bw\in \bV(3)$ such that $R(\bw)\cap X$
is not a null set.  In particular, $R(\bw)$ has affine dimension $3$.  There are finitely
may $R(\bw')$, for $\bw'\in \bV(3)$ that meet $R(\bw)$.  
By Lemma~\ref{lemma:R-inter}, by avoiding finitely many planes (of null measure),
there exists a point $\p\in X\cap R(\bw)$ such that $R(\bw)$ is the unique
Rogers simplex that contains it.  Furthermore, by avoiding the null sets distinguishing
the cases in Lemma~\ref{lemma:M-exhaust} (that is, the sphere of radius $\sqrt2$,
the boundary of $\op{rcone}$, and the half-plane boundaries of wedges), we
may assume that $\p$ does not lie in any of the null sets
\begin{displaymath}
X\cap R(\bw)\cap\cell(\bw,i)\cap \cell(\bw,j),\quad i\ne j.
\end{displaymath} 


Now $\cell(\bu,i)$ is contained in the union of the sets $R(\bu')$ as $\bu'$ runs over the
$i$-rearrangements of $\bu$.  Hence $\bw$ is an $i$-rearrangement of
$\bu$.  By the converse statement, replacing $\bu$ with an
$i$-rearrangement, we may assume that $\bu=\bw$.  Similarly, we may
assume that $\bv=\bw$.  By the exclusion of null sets as above, $i=j$.
\end{proof}

%
%
%
%
%First, the proof establishes that cells are either
%disjoint (up to a null set) or equal.  At the same time, it
%determines exactly when two $k$-cells are equal to one another.
%
%By Lemma~\ref{lemma:Rconv}, the $4$-cell $\cell(\bu,4)$ is a union
%of the simplices $R(\bv)$, as $\bv$ runs over rearrangements of $\bu$.
%Two $4$-cells that meet in a set of positive measure are equal.
%Every $4$-cell is a union of simplices $R(\bu)$ with
%$h(\bu)<\sqrt2$.  This condition gives $\cell(\bu,i)=\emptyset$, for
%$i=0,1,2,3$.
%
%Similarly, the $3$-cell is a union of the convex hulls
%$\op{conv}(\{\xi(\trunc{\bv}{2})\}\cup R(\trunc{\bv}{2}))$ as $\trunc{\bv}{2}$ runs over
%rearrangements of $\trunc{\bu}{2}$.  Note that the point
%$\xi(\trunc{\bv}{2})=\xi(\trunc{\bu}{2})$ is independent of the rearrangement, since
%it is determined as the point at distance $\sqrt2$ from $ \u_0$
%along the line through from $\omega(\trunc{\bv}{2})=\omega(\trunc{\bu}{2})$
%perpendicular to the plane $\op{aff}\{ \u_0, \u_1, \u_2\}$ (in the
%half-space of $\trunc{\bu}{3}$). Two $3$-cells that meet in a set of
%positive measure are equal.  Their parameters are equal:
%$\trunc{\bu}{2}=\trunc{\bv}{2}$.  The intersection $\cell(\bu,3)$ cannot meet
%$\cell(\bu,i)$, for $i<3$, in a set of affine dimension three, because the
%plane $\op{aff}\{ \u_0, \u_1,\p\}$ separates them.
%
%The $2$-cell with parameter $\bu$ is contained in the right-circular cone
%\begin{displaymath}
%\op{rcone}( \u_0, \u_1,\cos(\op{arc}(\norm{ \u_0}{ \u_1},\sqrt2,\sqrt2)))
%\end{displaymath}
%This cone separates the $2$-cell from the $0$ and $1$-cells with the
%same parameter $\bu$.  The $1$-cell is separated from $0$-cells by the
%sphere of radius $\sqrt2$, centered at $ \u_0$.


%\begin{lemma}[]\guid{PDVQTUO}
%Let $V$ be a saturated packing.  If two Marchal cells of $V$ are not equal, then
%their intersection lies in a null measure set.
%\end{lemma}




\section{The Primitive State Of Our Subject Revealed}

\subsection{reduction to a finite packing}


\begin{theorem}[Kepler's Conjecture on Dense Packings]\guid{IJEKNGA}
\label{theorem:kepler}   No packing of congruent balls in
Euclidean three space has density greater than that of the
face-centered cubic packing.
\end{theorem}

\begin{remark}
This density is $\pi/\sqrt{18}\approx 0.74.$  There are other
packings, such as the hexagonal close packing, that attain this
same density.
\end{remark}

The proof of this result is presented in this book. This section
describes the outline of the proof and gives references to
the sources of the details of the proof.



\begin{definition}[negligible,~fcc-compatible]\guid{ZREKEVW}\label{def:negligible}
A function $G:V\to \ring{R}$ on a set $V\subset\ring{R}^3$
is \newterm{negligible\/}
if there is a constant $c_1$ such that for all $r\ge1$,
% and all $\p\in\ring{R}^3$,
\begin{displaymath}\sum_{\v\in V(\orz,r)} G(\v) \le c_1
r^2.\end{displaymath}
A function $G: V\to\ring{R}$ is
\newterm{fcc-compatible\/}
if for all $\v\in V$, 
\begin{displaymath}\sqrt{32}\le \op{vol}(\Omega(V,\v)) +
G(\v).\end{displaymath}
\indy{Index}{negligible}%
\indy{Index}{fcc-compatible}%
\indy{Notation}{G (negligible function)}%
\end{definition}


\begin{remark}
The value $\op{vol}(\Omega(V,\v)) + G(\v)$ may be interpreted as a
{\it corrected\/} volume of the Voronoi cell. The constant
$\sqrt{32}$ that appears in the definition of fcc-compatibility is
the volume of the Voronoi cell in the face-centered cubic and
hexagonal-close packings.  The corrected volume is at least the
volume of these Voronoi cells when the correction term $G$ is
fcc-compatible.  \indy{Index}{corrected volume}%
\end{remark}

% \begin{remark} In \cite{Hales:2006:DCG}, the full Voronoi cell
%   $\Omega(V,\v)$ is used, rather than $\Omega(V,\v)$.  The truncation at
%   radius $2$ is just a matter of convenience to guarantee the
%   boundedness and hence the finite volume of the (truncated) Voronoi
%   cell.  In \cite{Hales:2006:DCG}, the same effect was achieved by
%   requiring all packing%s
%   to be saturated.  Drop the assumption of saturation on $ V$.
%\end{remark}



The density $\delta( V,\p,r)$ of a packing $ V$ within a bounded
region of space is defined as a ratio. The numerator is volume of
$B(V,\p,r)$, defined as the intersection with $B(\p,r)$ of the union
of all balls in the packing.  The denominator is the volume of
$B(\p,r)$. 
%An ordered pair $( V,\v)$ with $\v\in V$ is called a \newterm{centered
%packing}.  \indy{Index}{packing!centered}%
\indy{Notation}{ZZdelta@$\delta( V,\p,r)$}%
\indy{Notation}{V@$ V$ (packing)}%
%\indy{Notation}{V@$ V^*$(packing)}%


\begin{lemma}[]\guid{JGXZYGW}
\oldrating{150}
\formalauthor{Nguyen Tat Thang}
\rating{0}
\label{lemma:deltabound} If there exists a 
\newterm{negligible} \newterm{fcc-compatible} function
$G: V\to\ring{R}$ for a 
%saturated 
packing $ V$, then there
exists a constant $c$ such that for all $r\ge1$,
% and all $\p\in\ring{R}^3$, %%  removed on  Jan 16, 2009 after formalization.
\begin{displaymath}
\delta( V,\orz,r)
\le \pi/\sqrt{18} + c/r.
\end{displaymath}
%The constant $c$ depends on $ V$ only through the constant
%$c_1$ of Definition~\ref{def:negligible}.
\end{lemma}

%\begin{remark}\label{conj:fcc-neg}
%For every saturated packing $ V$, there exists a negligible
%fcc-compatible function $G: V\to R$.
%\end{remark}



\begin{proof}
The volume of $B( V,\orz,r)$ is at most the product of the volume
$4\pi/3$ of a single ball $4\pi/3$ with the number of centers in
$B(\orz,r+1)$.  Hence
\begin{equation}
\op{vol}\, B( V,\orz,r) \le \card( V(\orz,r+1)) 4\pi/3.
\label{eqn:Abound}
\end{equation}

%In a %saturated packing 
Each truncated Voronoi cell is contained in a ball of
radius $2$ that is concentric with the unit ball in that cell.  The volume
of the large ball $B(\orz,r+3)$ is at least the combined volume of 
all truncated Voronoi
cells centered in $B(\orz,r+1)$. This observation,
combined with fcc-compatibility and negligibility, gives
\begin{equation}
\begin{split}
\sqrt{32}\,\,\card( V(\orz,r+1))
&\le \sum_{\v\in V(\orz,r+1)} (G(\v) +
\op{vol}(\Omega(V,\v))) \\
&\le c_1 (r+1)^2 + \op{vol}\,B(\orz,r+3) \\
&\le c_1 (r+1)^2 + (1+3/r)^3 \op{vol}\,B(\orz,r)
\label{eqn:Bbound}
\end{split}.
\end{equation}
\indy{Index}{fcc-compatible}%
Recall that $\delta( V,\orz,r)=
\op{vol}\,B( V,\orz,r)/\op{vol}\,B(\orz,r)$. Divide Inequality
\ref{eqn:Abound} through by $\op{vol}\,B(\orz,r)$.  Use
Inequality~\ref{eqn:Bbound} to eliminate $\card( V(\orz,r+1))$ from the
resulting inequality.  This gives
\begin{displaymath}\delta( V,\orz,r)
\le \frac{\pi}{\sqrt{18}} (1+3/r)^3 + c_1 \frac{(r+1)^2}{r^3\sqrt{32}}.
\end{displaymath}
The result follows for an appropriately chosen constant $c$
(depending on $c_1$).
\end{proof}

\begin{remark} \label{remark:precise} The precise meaning of the
sphere packing problem is to prove the bound bound $\delta( V,\orz,r)
\le \pi/\sqrt{18} + c/r$ for every packing $ V$.  Thus, by the
preceding lemma, the existence of a negligible fcc-compatible
function provides the solution to the packing problem.  The strategy
will be to define a negligible function and then to solve an
optimization problem in finitely many variables to establish that
the function is also fcc-compatible.
\end{remark}


\subsection{Marchal's conjecture}

This section shows how the existence of a fcc-compatible negligible
function (Remark~\ref{rem:precise})  would follow from an explicit
estimate related to the the distances $h(\bu)$, where $\bu\in \bV(1)$.

%\begin{note}%
%Replace $\sol_0/\pi =\Delta_1$ to simplify formulas.
%\end{note}


\begin{definition}[$\sol_0$,~$\tau_0$,~$m_1$,~$m_2$,~$h_+$,~$M$]\guid{AOZUTMU}
Define the following constants and functions: 
\begin{eqnarray}\label{eqn:m-def}
\sol_0 &=& 3\arccos(1/3)-\pi\\
%\Delta_1 &=& (3\arccos(1/3)-\pi)/\pi\\
\tau_0 &=& 4\pi  - 20\sol_0\\
m_1 &=& \sol_0 2\sqrt2/\tau_0 = 1.012\ldots \\ %% K 
m_2  &=&  (6\sol_0- \pi)\sqrt2/(6 \tau_0) = 0.0254\ldots\\ %% M 
h_+ &=& 1.3254 \hbox{~(exact rational value)}
\end{eqnarray}
Let $M:\ring{R}\to\ring{R}$ 
be the following piecewise polynomial function (Figure~\ref{fig:M}):
\begin{equation}\label{eqn:M}
M(h) =
\begin{cases}
% (\sqrt2-h) (h-1.3254) (9h^2 - 17 h + 3)/(1.627 (\sqrt2-1))& h\le\sqrt2\\
\dfrac{\sqrt2-h}{\sqrt2-2}~ \dfrac{h_+-h}{h_+-1} ~\dfrac{17 h - 9 h^2 - 3}{5} & h \le \sqrt2.\vspace{3pt} \\
0 & h >\sqrt2.
\end{cases}
\\
\end{equation}
\end{definition}

\begin{figure}[htb]
\centering
\szincludegraphics[width=60mm]{\pdfp/Mfun.eps}
% Plot[Mfun[h], {h, 1, Sqrt[2]}]
% copied to Preview, then saved, then converted to eps via pdf2eps.
\caption{The quartic polynomial $M$.}
\label{fig:M}
\end{figure}

The constant $\sol_0$
is the area of a spherical triangle with sides $\pi/3$.
Simple calculations based on the definitions give
\begin{equation}\label{eqn:km}m_1 - 12m_2 = \sqrt{1/2}\end{equation}
and
\begin{equation}M(1) = 1,\quad M(h_+)=0,\quad M(\sqrt2) =0.\end{equation}

\begin{definition}[$V(X)$,~$\op{tsol}$]\guid{LZYLTFY}
  Let $V(X)$ be the intersection of $ V$ with the set of extreme
  points of the $k$-cell $X$.  Explicitly, $V(X)=\emptyset$ if $k=0$;
  and $V(X) = \{ \u_0,\ldots, \u_{k-1}\}$ in general.  Each $k$-cell
  is measurable and eventually radial at each $\u\in V(X)$.  Define
  the \newterm{total solid angle} of $X$ to be
\begin{displaymath}
\op{tsol}(X) = \sum_{ \u\in V(X)} \sol(X, \u).
\end{displaymath}
\end{definition}
\indy{Index}{angle!total solid}%
\indy{Index}{extreme point}%
\indy{Notation}{VX@$V(X)$ (extreme points of a cell $X$)}
\indy{Notation}{tsol@$\op{tsol}$}%

\begin{definition}
Let $E(X)$ be the set of extremal edges of the $k$-cell $X$ in $ V$.
More precisely, let
\begin{displaymath}E(X)=\{\{ \u_i, \u_j\}\mid \u_i\ne \u_j\in
V(X)\}.\end{displaymath}
\indy{Notation}{1@$\tbinom{n}{k}$ (binomial coefficient)}%
\end{definition}

In particular, $E(X)$ is empty for $0$ and $1$-cells, and contains
$\tbinom{k}{2}$ pairs when $k\ge 2$.

\begin{definition}[$\dih(X,e)$,~$h$]\guid{RSDYMHV}
The value $\dih(X,e)$ of the dihedral angle indexed by an edge $e\in
E(X)$ is defined to be the dihedral angle along that edge of the
boundary.  Explicitly, the (single) dihedral angle of a $2$-cell is the same as
the radian angle subtended by the arc $R_2$, which was defined in the
construction of $2$-cells.  The dihedral angle of a $3$ or $4$-cell is
the dihedral angle along the given edge of the simplex $X$.
\index{Notation}{dih@$\dih$}%
\index{Index}{angle!dihedral}%
\index{Notation}{h@$h$ (half-edge length)} 
Each
element $e=\{ \u_i, \u_j\}\in E(X)$ also determines the real number
$h(e) = \norm{ \u_i}{ \u_j}/2$.
\end{definition}

This definition of $h$ is compatible with the previous definition in the sense that
$h([\u_0;\u_1]) = h(\{\u_0,\u_1\})$.

\begin{definition}[$\gamma$]\guid{KGFDCFM}
For any function $f:\ring{R}\to\ring{R}$ and any cell $X$ of a saturated packing, set
\begin{equation}\label{eqn:gamma-def}
\gamma(X,f) =  \op{vol}(X)
-\left(\frac{2m_1}{\pi}\right) \op{tsol}(X) + \left(\frac{8m_2}{\pi}\right)
\sum_{e\in E(X)} \dih(X,e)  f(h(e))
\indy{Notation}{ZZddgamma@$\gamma$ (fundamental estimate)}%
\end{equation}
\end{definition}


\begin{theorem}[Marchal's inequality]\guid{HJKDESR}\label{lemma:MI}
Let $V$ be any saturated packing, and let $X$ be any Marchal cell of $V$.  Then
\begin{equation}\label{eqn:mfe}
\gamma(X,M)\ge 0,
\end{equation}
where $M$ is the function defined in (\ref{eqn:M}).
\end{theorem}

\begin{proof}  See~\cite{marchal:2008}.  We return to this inequality in the appendix,
to provide our own (computer) proof.
\end{proof}
%% cc:mar are the k-cell estimates for non-cell clusters.
%By Calculation~\ref{calc:marchal}, Marchal's fundamental estimate
%holds for any cell


\begin{conjecture}[Marchal]\guid{PHNFUXP}\rating{ZZ}\label{conj:m1} 
For any packing $ V$, and
any $ \u_0\in V$,
\begin{displaymath}
\sum_{\bu\in \bV(1)} M(h(\bu)) \le 12.
\end{displaymath}
\end{conjecture}

Marchal's conjecture is still open.  This book proves a variant of
Marchal's conjecture.

\begin{theorem}\guid{KIZHLTL}\rating{300}\label{theorem:mk1}
Marchal's Conjecture~\ref{conj:m1} implies
that for every saturated packing $V$, there exists a negligible fcc-compatible function
$G:V\to \ring{R}$.
\end{theorem}


\begin{proof} 
It is enough to show that $G( \u_0) = -\op{vol}(\Omega(V, \u_0)) + 8
m_1 - \sum 8 m_2 M(h(\bu))$ is fcc-compatible and negligible.
%The result then follows from Lemma~\ref{lemma:deltabound}.  
The function $G$ is fcc-compatible directly
by equation~\eqn{eqn:km}
and Conjecture~\ref{conj:m1}:
\indy{Index}{negligible}%
\indy{Index}{fcc-compatible}%
\begin{eqnarray*}
\sqrt{32} &=& 8 m_1 - 8\cdot 12 m_2\\
&\le& 8 m_1 - 8 m_2 \sum M(h)\\
&=& \op{vol}(\Omega(V, \u_0)) + G( \u_0).
\end{eqnarray*}
The issue is to prove it negligible.  More explicitly, one must find a
constant $c$ such that for all $r\ge 1$:% and all $\p\in\ring{R}^3$:
\begin{equation}\label{eqn:neg}
-\sum G( \u) = \sum \op{vol}(\Omega(V, \u)) 
-\sum 8m_1 + \sum \sum 8 m_2 M(h) \ge c r^2,
\end{equation}
where the outer nested sum runs over $ \u\in  V(\orz,r)$.

Lemmas~\ref{lemma:Zr2} and \ref{lemma:V-finite} show that the number
of points of $ V$ near the boundary of $B(\orz,r)$ is bounded by $c
r^2$.


The sum of the inequality~\eqn{eqn:mfe} over all cells in a large ball
$B(\orz,r)$ gives an inequality of the form $T_1 + T_2 + T_3\ge 0$ for
three terms $T_i = T_i(r)$ of \eqn{eqn:gamma-def}.  The desired
equation~\eqn{eqn:neg} consists of three corresponding terms
$T'_i(r)$.  It is enough to show that
\begin{displaymath}
T_i'(r) \ge T_i(r) + c_i r^2,
\end{displaymath}
for some constants $c_i$.

The sum of the volumes of the Voronoi cells $ \u\in B(\orz,r)$ is not
exactly the volume of $B(\orz,r)$, because of the contribution at the
boundary of $B(\orz,r)$ of Voronoi cells that are only partly contained
in $B(\orz,r)$.  Similarly, the sum of the various $k$-cells, for
$X\subset B(\orz,r)$ is not exactly the volume of $B(\orz,r)$, because of
contribution from the boundary. The boundary contributions have order
$r^2$. Thus,
\begin{displaymath}
T_1'= \sum_{ \u\in  V(\orz,r)} \op{vol}(\Omega(V, \u)) 
\ge \sum_{X\subset B(\orz,r)} \op{vol}(X) + c_1 r^2 = T_1 + c_1 r^2.
\end{displaymath}


The estimates on the other terms are similar.  The solid angles
around each point sum to $4\pi$.
In Landau big O notation, this gives
\begin{eqnarray*}
\sum_{X\subset B(\orz,r)} \op{tsol}(X) &=& 
\sum_{X\subset B(\orz,r)} \sum_{ \u\in V(X)} \sol(X, \u)\\
&=&\sum_{ \u\in  V(\orz,r)} \sum_{X\mid  \u\in V(X)} \sol(X, \u) + O(r^2)\\
&=&\sum_{ \u\in  V(\orz,r)} 4\pi    + O(r^2).
\end{eqnarray*}
Hence
\begin{displaymath}
T_2' = -\sum_{ V(\orz,r)} 8 m_1 = 
-\sum_{X\subset B(\orz,r)}\left(\frac{2m_1}\pi\right) \op{tsol(X)} + O(r^2) = T_2 + O(r^2).
\end{displaymath}
Similarly, the dihedral angles around each edge sum to $2\pi$.  A
factor of $2$ enters the following calculation, because there are two
ordered pairs for each unordered pair $e=\{ \u_0, \u_1\}$:
\begin{eqnarray*}
&\phantom{=}&\sum_{X\subset B(\orz,r)} \sum_{~~e\in E(X)} \dih(X,e)  M(h(e)) \vspace{3pt}\\
&=&\sum_{e\subset B(\orz,r)} \sum_{~~X\mid e\in E(X)} \dih(X,e)  M(h(e)) +O(r^2)\vspace{3pt}\\
&=&\sum_{e\subset B(\orz,r)} 2\pi M(h(e)) + O(r^2)\vspace{3pt} \\
&=&\sum_{ \u_0\in  V(\orz,r)} \sum_{~~\u_1\in  V(\orz,r) } \pi M(h( \u_0, \u_1)) + O(r^2).\\
\end{eqnarray*}
Finally,
\begin{eqnarray*}
T_3' &=& \sum\sum 8 m_2 M(h(\bu)) \\
&\ge& \left(\frac{8m_2}\pi\right)
\sum_{X\subset B(\orz,r)}\sum_{e\in E(X)}\dih(X,e) M(h(e)) + O(r^2) \\
&=& T_3 + O(r^2).
\end{eqnarray*}
\end{proof}




\section{Clusters}

\begin{summary} Marchal's Conjecture is still open.  This section
gives another conjecture that is closely related to Marchal's
Conjecture.  In this new conjecture, a piecewise linear function $L$
replaces the piecewise polynomial function $M$.  More crucially, the
support of the function $L$ is contained in
$\leftclosed2,2.52\rightclosed$.  By contrast, the support of
Marchal's function is much larger:
$\leftclosed2,2.6508\rightclosed$.  This small difference in the
support of the function creates an enormous difference in the
difficulty of the conjectures.

The conjecture formulated in this section also implies the existence
of fcc-compatible negligible functions.  To prove this, it is
helpful to group Marchal cells together, into what are called
\newterm{clusters}.  This section makes a detailed study of clusters
in order to produce a negligible function.

The sections and chapters that follow give a proof of the
conjecture in this section.  The proof of this conjecture is the
main intermediate result in the journey to prove the Kepler
conjecture.
\end{summary}

% This section shows how to improve on the estimates of the previous
% section by combining various cells into {\it cell clusters}.
Recall that $M(h_+) = 0$, where   $h_+ = 1.3254$.
\indy{Index}{cell cluster}%

\begin{definition}[$L$,~$h_0$,~$h_-$]\guid{ULZRABY}\label{def:L}
Set
\begin{displaymath}
h_0 = 1.26\\  %%\hm
\end{displaymath}
Let $L:\ring{R}\to\ring{R}$ be the piecewise linear function 
\begin{displaymath}
L(h) = \begin{cases}
\dfrac{h_0-h}{h_0-1}, & h \le h_0 \\
0, & h\ge h_0. \\
\end{cases}
\end{displaymath}
It follows from the definition that
\begin{displaymath}
L(1) = 1\quad L(\hm) = 0.
\end{displaymath}
Let $h_- = 1.23175\ldots$ be the unique root of the quartic polynomial
$M(h)-L(h)$ lying in the interval $[1.2,1.3]$.
\indy{Notation}{L@$L$ (linear function)}%
\indy{Notation}{h@$h_- = 1.23175\ldots$}%
\indy{Notation}{h@$h_0 = 1.26$}%
\end{definition}

%%
\begin{figure}[htb]
\centering
\szincludegraphics[width=60mm]{\pdfp/Lfun.eps}
% Plot[{Mfun[h],Lfun[h]}, {h, 1.2, 1.35}]
% copied to Preview, then saved, then converted to eps via pdf2eps.
%% WW very big .eps file!
\caption{Detail of the quartic $M$ and linear function $L$.}
\label{fig:L}
\end{figure}

The inequality $L(h)\ge M(h)$ holds except when $h\in [h_-,h_+]$.  The
aim of this section is to prove a variant of Theorem~\ref{theorem:mk1}
that uses the function $L$ rather than $M$.  For this, one needs to
combine cells into groups called cell clusters.

\begin{definition}[critical edge,~$\op{EC}$,~$\op{wt}$]\guid{MZSRVBC}\label{def:wt}
A \newterm{critical edge} $e$ of a saturated packing $V$ is an ordered pair
that appears as an element of $E(X)$ for some 
$k$-cell $X$ of the packing $V$, and such that
$h(e)\in[h_-,h_+]$.  Let $\op{EC}(X)$ 
be the set of critical edges that belong to $E(X)$.  If $X$ is any cell such
that $\op{EC}(X)$ is nonempty, let the \newterm{weight} $\op{wt}(X)$ of $X$ be
$1/\card(\op{EC}(X))$.
\end{definition}
\indy{Index}{critical!edge}%
\indy{Notation}{EC@$\op{EC}$ (critical edges)}%
\indy{Notation}{wt@$\op{wt}$ (weight)}%

\begin{definition}[$\beta$,~$\op{bump}$]\guid{PQFEXQN}\label{def:beta}
Set 
\begin{displaymath}
\op{bump}(h) = 0.005 (1 - (h-h_0)^2/(h_+-h_0)^2).
\end{displaymath}
If $X$ is a $4$-cell with exactly two critical edges and if those edges
are opposite, then set
\begin{displaymath}
\beta(e,X) = \op{bump}(h(e)) - \op{bump}(h(e')), \text{ where } \{e,e'\} = \op{EC}(X).  
\end{displaymath}
Otherwise, for all other edges and all other cells, set $\beta(e,X) = 0$.
\end{definition}
\indy{Notation}{bump@$\op{bump}$}%
\indy{Notation}{ZZbeta@$\beta$ (bump)}%

\begin{definition}[cell cluster,~$\Gamma$,~$\gamma_L$]\guid{YSULGYR}\label{def:gammaL}
Let $e\in \op{EC}(X)$ be a critical edge of a $k$-cell for some $k\ge 1$.
A \newterm{cell cluster} is the set 
\begin{displaymath}
\op{CL}(e) = \{X\mid e\in E(X)\} 
\end{displaymath}
\indy{Notation}{cluster}%
of all cells around $e$. 
If $Z$ is a subset of a cell cluster $\op{CL}(e)$, define
\indy{Notation}{ZZddgamma@$\gamma_L$ (cell cluster function)}%
\indy{Notation}{CL@$\op{CL}$ (cell cluster)}%
\begin{displaymath}
\Gamma(e,Z) = \sum_{X\in Z} \gamma_L(X) \op{wt}(X) +\beta(e,X),\quad\hbox{ where }\quad
\gamma_L(X) = \gamma(X,L).
\end{displaymath}
%and where $\op{wt}(X)$ is the weight of $X$.
\indy{Notation}{ZZddGamma@$\Gamma$}%
\indy{Notation}{ZZcamma@$\gamma_L$}%

\end{definition}
\indy{Index}{cell cluster}%

\begin{theorem}[cell cluster estimate]\guid{OXLZLEZ}\rating{1500}
\label{lemma:cluster}
Let $\op{CL}(e)$ be any cell cluster of a critical edge $e$ in a saturated packing $V$.  
Then $\Gamma(e,\op{CL}(e))\ge 0$.
\end{theorem}

The proof of this cell cluster estimate is a numerical calculation
that has been carried out by computer.  Further discussion about the
methods appears in the appendix. The proof of the following assertion
is extremely long and complex.  It relies on many computer
calculations.  The non-computer parts of the proof take up most of the
remainder of the book.%
%
%
\textnote{ %% CLUSTER INEQUALITY
\subsubsection{cluster inequality}

Turn to the proof of Lemma~\ref{lemma:cluster}.  Let $Z$ be a cell
cluster and let $e$ be the critical edge shared by the cells in the
cluster.  Call this critical edge the \newterm{spine}.  Consider all
the faces along the spine $\{ \u_0, \u_1\}$, consisting of three
points $\{ \u_0, \u_1, \u_2\}$ in the packing with circumradius less
than $\sqrt2$.  Call such a face a \newterm{leaf}. 
\indy{Index}{edge!spine}%
\indy{Index}{edge!critical}%
\indy{Index}{leaf}%

Consecutive cells around the spine are \newterm{adjacent}.  A cell
adjacent to a $4$-cell is a $3$-cell or another $4$-cell.  A cell
adjacent to a $2$-cell is a $3$-cell.  There are no $0$ or $1$-cells
along a critical edge.  \indy{Index}{adjacent}%
\indy{Index}{quarter}%

Call a $4$-cell a \newterm{quarter}, when it has exactly one critical
edge and all other edges of the simplex have length at most $2 h_-$.
The weight of any quarter is $1$.


\subsubsection{two leaves}

The proof will be divided into cases, according to the number of
leaves along the critical edge.

\claim{It suffices to prove Lemma~\ref{lemma:cluster} when the cluster
contains at least one quarter and at least two leaves.} Indeed, by
Calculation~\ref{calc:cc:qtr}, if $X$ is any cell,
then % gammaL is nonneg on quarters. cc:qtr ~[GLFVCVK]
\begin{displaymath}
\gamma_L(X) \op{wt}(X) + \beta(e,X)\ge 0
\end{displaymath} 
when $X$ is not a quarter.  This is the desired inequality.  Also, a
quarter is flanked by two leaves.

Consider a cluster with exactly two leaves.
Without loss of generality, the two leaves flank a quarter
$X$. 
The azimuth angle of what remains outside the quarter
is greater than $\pi$.  Thus, there must be a $3$-cell
along each leaf.  Let $Y$ be one of these $3$-cells.
The cell $Y$ has weight $1$.
Then 
\begin{equation}\label{eqn:34}
\Gamma(Z)\ge \gamma_L(X)+\gamma_L(Y)\ge 0,
\end{equation}
by Calculation~\ref{calc:cc:2bl}.
%a calculation~\cite[FHBVYXZ]
% 2-leaf calculation, gammaL(fourcell)+gammaL(threecell) >=0. % cc:2bl:
\indy{Index}{azimuth}%
\indy{Index}{angle!azimuth}%

\subsubsection{five or more leaves}

Let $B$ be the set of cells in the cluster that lie between any two
consecutive leaves.  $B$ is either a singleton set containing a
$4$-cell, or a set of three cells: a $2$-cell and two adjacent
$3$-cells.  Write $\op{azim}(B)$ for the azimuth angle formed by the
two leaves.  By Calculation~\ref{calc:cc:5bl}, the cells between two
consecutive leaves satisfy an
inequality: % five or more leaves along a spine are found
% at~[ZTGIJCF].  % cc:5bl:

\begin{displaymath}
\sum_{X\in B} \gamma_L(X)\op{wt}(X) + \beta(e,X) \ge a + b\,\op{azim}(B),
\end{displaymath}
where
\begin{displaymath}
a= 0.0560305, \quad\text{and}\quad  b= -0.0445813.
\end{displaymath}
It follows that
\begin{displaymath}
\Gamma(Z) \ge 5 a + b\, (2\pi) > 0.
\end{displaymath}

\subsubsection{three or four leaves}\label{sec:3or4}

In the case of three and four leaves, the proof relies to an even
greater extent on computer calculations.  
Lemma~\ref{lemma:cluster} asserts a nonlinear inequality.  The inequality
asserts that some continuous function $f$ is positive on a compact
domain $D$ in Euclidean space.  Rigorous methods of global nonlinear
optimization can be used to show that the function $f$ is indeed
positive.

Details about global nonlinear optimization appear in the appendix.
The method is linear relaxation, which can be briefly described as
follows.\footnote{Linear relaxations appear again later in the
solution of significantly more difficult nonlinear optimzation
problems.}  The domain $D$ is partitioned into finitely many subsets
$D_1,\ldots, D_r$.  The positivity of $f$ is established on each
subset $D_i$.  The graph $\{(f(x), x)\mid x\in D_i\}$ of $f$ on $D_i$
is a subset of a polyhedron $P_i$.  The inequalities defining the
polyhedron can be determined explicitly.  A linear program computes
the minimum of the first coordinate over $P_i$.  This minimum is
positive.  Since the graph is contained in $P_i$, the values of $f$ on
$D_i$ must also be positive.

This strategy has been implemented and gives the desired lower bound.
The linear programming has been implemented as a {\tt MathProg} model
in Calculation~\ref{calc:shorts}.\footnote{{\tt MathProg} implements a
subset of the {\it AMPL} modeling language.  The computer code
giving the model appears in {\tt shorts.mod} and the computer code
controlling the branching appears in {\tt shorts.ml}.}  This model
contains the number of leaves, variables representing the edges and
the azimuth angles between consecutive leaves.

The proof that each graph is contained in a explicit polyhedron $P_i$
relies on nonlinear inequalities.  In fact, the proof requires over
one hundred nonlinear inequalities that have been established by
computer.

A function on the set $D$ with finite range partitions $D$ into
finitely many subsets, according to the image of a point.  Several
simple boolean functions and functions with finite range were used to
partition $D$ into subsets $D_i$: Are there three leaves or four?  Is
there a $2$-cell in the cluster?  Is the $4$-cell a quarter?  Is the
azimuth angle of a given cell greater than $2.3$?  Is the value of
$\gamma_L$ on a given quarter negative? Does a particular leaf have a
nonspline edge of length greater than $2h_-$?  What is the weight of a
given $4$-cell?  The inequalities that define the polyhedra $P_i$ have
been designed specifically for the subdomains $D_i$, according to the
answers to these questions.

These linear relaxations and are sufficient to prove the bound in all
but one case.  This is the case of four leaves, three quarters, and
one $4$-cell that has two critical edges: the spine and the edge
opposite the spine.  The other four edges of the $4$-cell have length
at most $2h_-$.  In this case as well, the inequality is established
by linear relaxation, but things are more involved.

The method to prove this case by computer is as follows.  Number the
four simplices $j=1,2,3,4$, with $j=1$ representing the $4$-cell of
weight $1/2$.  Write $\gamma^j$ in abbreviation of
$\gamma_L(X)\op{wt}(X) + \beta(e,X)$.  The desired inequality is
\begin{equation}\label{eqn:gpos}
\sum_{j=1}^4 \gamma^j > 0.
\end{equation}
The domain is a set of ordered four-tuples of simplices
$X_1,\ldots,X_4$ that fit together into an octahedron (the dihedral
angles along the spine sum to $2\pi$, the four simplices all have the
same spline length $y_1(X_j)$, and the shared edges are the same
length for the two simplices sharing the edge).  Instead of the single
inequality asserted by Lemma~\ref{lemma:cluster}, many inequalities of
the following form are established:
\begin{equation}\label{eqn:gpart}
\gamma^j + a_i \dih^j + b_i^j y_1 + c_i^j (y_2+y_3+y_5+y_6) + d_i^j > 0, 
\end{equation}
for all $X \in I_i^j$, \quad $i \in I$, and $j\in \{1,2,3,4\}$.  Here,
$\gamma^j$, $\dih^j$, and $y_i$ are all functions of $X$.  The simplex
$X$ has spine of length $y_1$ and other edges with lengths $y_k$.
Also, $\dih^j$ is the dihedral angle of $X$ along the spine.  $I$ is a
finite indexing set.  Each domain $I_i^j$ is a product of intervals in
$\ring{R}^6$, under the parameterization of a simplex by the lengths
of its six edges.  The union over $i$ of the sets
\begin{displaymath}
\{(X_1,\ldots,X_4)\mid~ X_j \in I_i^j,\text{ for } j=1,2,3,4.\}
\end{displaymath}
covers the entire domain of the desired inequality~\eqn{eqn:gpos}.   

\claim{A subset of this union already covers the domain.}  Indeed, the
dihedral angles of the four simplices along the spine sum to $2\pi$:
\begin{displaymath}
\sum_{j=1}^4 \gamma^j = 2\pi.
\end{displaymath}
Furthermore, the spine lengths of the simplices must agree: $y_1(X_j)
= y_1(X_k)$ for all $j,k\in\{1,2,3,4\}$.  Finally, each leaf flanks
two different simplices; the edge lengths of the leaf must agree.
This gives a collection of inequalities of the form $y_i(X_j) =
y_{i'}(X_{j'})$.  These are the subset relations.

The coefficients of the inequalities~\eqn{eqn:gpart}
are chosen so that for each $i$, the following inequality holds:
\begin{equation}\label{eqn:glin}
0 > a_i 2\pi + 
\sum_{j=1}^4 (b_i^j y_1 +  c_i^j (y_2(X_j)+y_3(X_j)+y_5(X_j)+y_6(X_j)) + d_i^j).
\end{equation}
(Notice that this inequality is linear in the variables $y_k(X_j)$ and
the domain is a product of intervals.  Hence this inequality is
particularly easy to check.)  The sum of the
inequality~\eqn{eqn:gpart} over $j$, the inequality~\ref{eqn:glin},
and the subset relation yield the desired inequality~\ref{eqn:gpos}.

A four-leafd cluster is an octahedron.  The set of octahedra is a
$13$-dimensional object, parameterized by $12$ external edges and one
diagonal (the spine).  By contrast, a simpliex in $\ring{R}^3$ is only
a $6$-dimensional object, parameterized by $6$ edges.  The four-leafed
cluster inequality is an inequality in $13$-dimensions.  The preceding
arguments reduce the $13$-dimensional inequality into a series of
$6$-dimensional inequalities.  The $6$-dimensional inequalities are
within within the reach of a computer.

The only question is how the magical coefficients
$a_i,b_i^j,c_i^j,d_i^j$ are obtained.  Clearly, the
inequalities~\eqn{eqn:glin} and \eqn{eqn:gpart} are linear in these
coefficients.  Thus, a they can be found by linear programming.  It is
somewhat troubling that there are infinitely many constraints, as each
point $X_j$ in the domain gives one constraint.  In practice, the
infinite number of constraints can be replaced by a finite collection.
Some information is lost in approximating the system with a finite
number of constraints.  However, the coefficients, once they are
guessed by approximations, can be checked independently by computer.
(An exact fit it not necessary; any coefficients that work will do.)
It was necessary to partition the domain into smaller pieces to
produce coefficients that work.  In other words, the indexing set $I$
contains more than one element.  After some experimentation, we found
a set $I$ of cardinality $23$ that works.  Through these methods,
coefficients were found.
} %CLUSTER INEQUALITY

\begin{assertion}\guid{BJERBNU}\rating{ZZ}\label{conj:L12} 
  For any  saturated packing $ V$, and any $ \u_0\in V$,
\begin{equation}\label{eqn:L12}
\sum_{ \u_1\in V\mid h( \u_0, \u_1)\le \hm} L(h\{\u_0, \u_1\}) \le 12.
\end{equation}
\end{assertion}

\begin{lemma}[]\guid{UPFZBZM}\rating{300}\label{theorem:mk2}
Assertion~\ref{conj:L12} implies the Kepler conjecture.
\end{lemma}

\begin{proof} The proof imitates the proof of
Theorem~\ref{theorem:mk1}.  It is enough to show that $G_L( \u_0) =
-\op{vol}(\Omega(V, \u_0)) + 8 m_1 - \sum 8 m_2 L(h(\bu))$ is
fcc-compatible and negligible.  The function $G_L$ is fcc-compatible
directly from equation~\eqn{eqn:km} and Assertion~\ref{conj:L12}.

The theorem follows from a proof that $G_L$ is negligible.  More
precisely, one needs to show there exists a constant $c$ such that
for all $r\ge 1$:% and all $\orz\in\ring{R}^3$:
\begin{equation}\label{eqn:A2neg}
\sum \op{vol}(\Omega(V, \u)) -\sum 8m_1 + \sum \sum 8 m_2 L(h) \ge c r^2,
\end{equation}

For cells $X$ that do not belong to a cell cluster,
the proof is just as in the proof of Theorem~\ref{theorem:mk1}.
If $\op{EC}(X)=\emptyset$, then 
$L(h(e))\ge M(h(e))$ for each edge $e\in E(X)$, and
\begin{displaymath}\gamma_L(X)\ge \gamma(X,M)\ge 0\end{displaymath} 
by inequality \eqn{eqn:mfe}.

Note that the function $\beta(e,X)$ averages to zero for any $4$-cell $X$:
\begin{displaymath}
\sum_{e\in \op{EC}(X)} \beta(e,X) = 0.
\end{displaymath}
Hence the terms involving $\beta$ in sums may be safely disregarded.
(The terms involving $\beta$ may be disregarded here, but they are
needed for the proof of Lemma~\ref{lemma:cluster} in
Section~\ref{sec:3or4}.)

Theorem~\ref{lemma:cluster} gives the required inequality for cell
clusters.  Again, using big O notation,
\begin{eqnarray*}
\sum_{X\subset B(\orz,r)} \gamma_L(X) &=&
\sum_{X\subset B(\orz,r)\mid \op{EC}(X)\ne\emptyset} \gamma_L(X) +
\sum_{X\subset B(\orz,r)\mid \op{EC}(X)=\emptyset} \gamma_L(X) \vspace{6pt}\\
&\ge& \sum_{X\subset B(\orz,r)\mid \op{EC}(X)\ne\emptyset} \gamma_L(X)\vspace{6pt} \\
&=&\sum_{X\subset B(\orz,r)}\gamma_L(X)\sum_{e \in \op{EC}(X)}\op{wt}(X) + O(r^2)\vspace{6pt}\\
&=&\sum_{e\subset B(\orz,r)}\sum_{X\mid e \in \op{EC}(X)}\gamma_L(X)\op{wt}(X) + O(r^2)\vspace{6pt}\\
&=&\sum_{e\subset B(\orz,r)}\Gamma(e,\op{CL}(e)) + O(r^2)\vspace{6pt}\\
&\ge& ~~O(r^2).
\end{eqnarray*}


From the definition of $\gamma_L$, the sum $\sum \gamma_L(X)$ may be
expanded as the sum of three terms, $T_1+T_2+T_3$, and compared term
by term with \eqn{eqn:A2neg}:
\begin{displaymath}
T_i' \ge T_i + c_i r^2.
\end{displaymath}
This proceeds exactly as in the proof of Theorem~\ref{theorem:mk1}.
\end{proof}

\begin{definition}[$\BB$]\guid{WTKURHK}
Let $\BB$ be the
\newterm{annulus} $\bar B(\orz,2h_0)\setminus B(\orz,2)$, where
$\bar B(\orz,r)$ is the closed ball of radius $r$.
\indy{Notation}{BB@$\BB$}
\end{definition}


\begin{corollary}\guid{RDWKARC}\rating{ZZ}\label{cor:CE} 
If the Kepler Conjecture is false,
there exists a finite packing $V\subset\BB$ such that
\begin{equation}\label{eqn:CE}
\sum_{ \u\in V} L(h\{\orz, \u\}) > 12.
\end{equation}
\end{corollary}

The proof of the Kepler conjecture proceeds by assuming that there is
a countexample to Assertion~\ref{conj:L12} and then deriving a contradiction.
This corollary formulates the potential counterexample in slightly simpler terms.

\begin{proof} If the Kepler conjecture is false,
Assertion~\ref{conj:L12} is violated for some packing $ V$ and some
$ \u_0\in V$.  After the replacement of $ V$ with $ V - \u_0$ and $
\u_0$ with $\orz$, it follows without loss of generality that $
\u_0=\orz\in V$.  After the replacement of $ V$ with the finite
subset
$V\cap \BB$,
it follows without loss of generality that the packing is a finite subset of $\BB$.
\end{proof}



\section{Counting spheres}


\subsection{solid angle}
\indy{Index}{polygon}%
\indy{Index}{polygon!regular}%


\begin{lemma}[]\guid{GOTCJAH}\rating{300}\label{lemma:ngon}
Let $P$ be a bounded polyhedron in $\ring{R}^3$ that contains $\orz$
as an interior point.  Let $F$ be a facet of $P$, given by an
equation
\begin{displaymath}
F = \{\p \mid \p \cdot \v = b\} \cap P.
\end{displaymath} 
Let $W_F$ be the corresponding topological component of $Y(V_P,E_P)$.  
Assume that $W_F$ contains the right-circular cone 
\begin{displaymath}
\op{rcone}^0(\orz,\v,h)
\end{displaymath}
for some $h>0$.
Then 
\begin{displaymath}
\sol(W_F) \ge 
2\pi - 2 k \,\arcsin\left(\,h\sin(\pi/k)\,\right),
\end{displaymath}
where $k$ is the number of edges of $F$.
\end{lemma}

%\begin{lemma}[]\guid{ZZ}\rating{0}\label{lemma:ngon:old}
%  Let $C$ be a circle on the unit sphere with arcradius $a<\pi/2$.
%  Among all spherical $n$-gons that contain $C$ (that is, among all
%  $n$-fold intersections of hemispheres containing $C$), that of
%  minimal area is the regular $n$-gon.
%\end{lemma}
%
%In other words, the minimal configuration consists of the
% intersection of $n$-hemispheres whose bounding great circles are
% tangent to the circle $C$ at $n$-equally spaced bounds around $C$.

\begin{proof} 
\claim{Without loss of generality, we may assume that each edge of
$F$ meets $\op{rcone}(\orz,\v,h)$ in a unique point.}  Indeed,
each edge of $F$ is the intersection of $F$ with another facet
$F_i$.  Write
\begin{displaymath}
F_i = \{\p \mid \p \cdot \v_i = b_i\} \cap P.
\end{displaymath}
The region $W_F$ consists of points $\p$ for which there exists a
$t>0$ such that $t\p \in\op{ri}(F)$ (by Lemma~\ref{lemma:WF}).  Hence
if we produce a second polyhedron $P'$ and facet $F'$ with
$\op{ri}(F')\subset \op{ri}(F)$, then
\begin{displaymath}
\sol(W_F)\ge \sol(W_{F'}).
\end{displaymath}
Shift the facet $F_i$ to
\begin{displaymath}
\op{aff}(F'_i) = \{\p \mid \p \cdot \v_i = b'_i\}.
\end{displaymath}
where $b'_i$ is chosen so that the point $\p_i\in \op{aff}(F'_i)\cap
\op{aff}(F)$ and lies on the boundary of $\op{rcone}(\orz,\v,h)$.
Define $P'$ by the intersection of $P$ with the half-spaces
\begin{displaymath}
\{\p \mid \p \cdot \v_i \le b'_i\},
\end{displaymath}
where the signs are chosen so that $b'_i>0$.  Let $F' = P'\cap F$.
Then $F'$ is a facet of $P'$.  The polyhedron $P'$ and facet $F'$
satisfy the assumptions of the lemma with the same constant $k= k_F =
k_{F'}$.  This completes the proof that we may assume that each edge
of $F$ meets $\op{rcone}(\orz,\v,h)$ in a unique point.  That is, the
edge is tangent to the right-circular cone.

Drop the primes from the notation: $P=P'$, $F=F'$, and so forth.  The
Rogers partition gives a partition of the polyhedron $P$ into
simplices.  There are $2k$ simplices in (the closure of) $W_F$.  The
solid angle of each simplex is the area of a spherical triangle.

Consider a spherical triangle with sides $a,b,c$ and opposite angles
$\alpha,\beta,\gamma$.  If $\gamma=\pi/2$, then by Girard's formula,
the area of the triangle is
\begin{displaymath}
\alpha+\beta-\pi/2,
\end{displaymath}
and by the law of cosines 
\begin{displaymath}
\cos(\alpha) =\sin(\beta)\cos(a).
\end{displaymath}
This determines the area $g(a,\beta)$ of the triangle 
as a function of $a$ and $\beta$. 
\indy{Index}{Girard's formula}%
\indy{Notation}{g@$g$ (triangle area)}%
\indy{Notation}{ZZalpha@$\alpha$ (angle)}%
\indy{Notation}{ZZbeta@$\beta$ (angle)}%
\indy{Notation}{ZZddgamma@$\gamma$ (angle)}%
\indy{Index}{convex}%
\indy{Index}{Girard's formula}%
\indy{Index}{polygon}%

The solid angle of $W_F$ is the sum of the areas of the triangles:
\begin{displaymath}
\sum_{i=1}^k g(a,\beta_i) 2,
\end{displaymath}
with angle sum
\begin{displaymath}
\sum_{i=1}^k \beta_i = 2\pi.
\end{displaymath}
With  $a$ fixed, the second partial of $A$ with respect to $\beta$ is
\begin{displaymath}
\frac{\partial^2 g(a,\beta)}{\partial \beta^2} = 
\frac{\cos(a)\sin^2(a)\sin(\beta)}{\sin^2(\alpha)} > 0.
\end{displaymath}
The function is convex.
By convexity, the minimum area occurs when all angles are equal
$\beta=\beta_i = \pi/k$.

The solid angle bound of the lemma is equal to 
\begin{displaymath}
2 k g(a,\beta)
\end{displaymath}
where $\cos(a)=h$.  Alternatively, the polygon breaks into $2k$
triangles, each computed by Girard's formula to have area
\begin{displaymath}
\beta - (\pi/2 - \alpha)  = \pi/k - \arcsin(\cos(\alpha)) = 
\pi/k - \arcsin(\cos(a)\sin(\beta)).
\end{displaymath}
\end{proof}

%\begin{lemma}[]\guid{BBEVFIC}\rating{0}\label{lemma:ngon-area}
%  The minimum area of an intersection of $k$-hemispheres containing a
%  circle $C$ of arcradius $a<\pi/2$ is
%\begin{displaymath}
%2\pi - 2 k \,\arcsin\left(\,\cos(a)\sin(\beta)\,\right),
%\end{displaymath}
%when $\beta = \pi/k$.
%\end{lemma}



\subsection{a polyhedral bound}

\begin{definition}[weakly saturated]\guid{HUCFLEB}
Let $r$ and $r'$ be real numbers such that $2\le r\le r'$.  Define a
set $ V\subset\ring{R}^3\setminus B(\orz,2)$ to be \newterm{weakly saturated} with
parameters $(r,r')$ if for every $\p\in\ring{R}^3$
\begin{displaymath}
2\le\normo{\p}\le r'~~~\implies~~~ \exists \u\in V.~\norm{ \u}{\p}< r.
\end{displaymath}
\end{definition}

\begin{lemma}[]\guid{TARJJUW}\rating{ZZ}\label{lemma:poly-bounded} 
Fix $r$ and $r'$ such that $2\le r\le r'$.
Let $ V$ be a weakly saturated finite packing with parameters $(r,r')$.
%such that $\orz\in  V$.
%where 
%   $\orz\in V$, and
%   $\normo{ \u}\le r'$ for all $ \u\in V$,
%Set $ V^*= V\setminus\{\orz\}$.
For any $g: V\to\ring{R}$, let $P( V,g)$ be the
polyhedron given by the intersection of half-spaces
\begin{displaymath}
\{\p \mid  \u\cdot \p \le g( \u)\},\quad \u\in V.
\end{displaymath}
Then $P( V,g)$ is bounded.
\end{lemma}
\indy{Index}{polyhedron}%

\begin{proof} Otherwise $P=P( V,g)$ is unbounded, and there exists
$\p\in P$ such that $\normo{\p} > g( \u) r'/2$ for all $ \u\in V$.
Let $\v = r' \p/\normo{\p}$ so that $r'=\normo{\v}$.  By the weak
saturation of $ V$, there exists $ \u\in V$ such that $\norm{\v}{
\u}<r$.  Then,
\begin{eqnarray*}
\normo{\p} &>& g( \u) r'/2 \ge  \u\cdot (r' \p)/2 = \normo{\p}  \u\cdot \v /2\\
&=& \normo{\p} (\normo{ \u}^2 + \normo{\v}^2 - \norm{ \u}{\v}^2)/4\\
&>& \normo{\p}(4+r'^2-r^2)/4\\
&\ge& \normo{\p}.
\end{eqnarray*}
This contradiction shows that $P$ is bounded.
\end{proof}




Since $L(h)\le 1$ when $h\ge1$, it is clear that a counerexample in
the sense of Corollary~\ref{cor:CE} has more than $12$  terms.
The following variant of a lemma of Marchal gives an upper bound on
the number of  terms.


\begin{lemma}[]\guid{DLWCHEM}\rating{300}\label{lemma:13-14}  %%
If the Kepler conjecture is false, there exists a packing $ V\subset\BB$ with 
containing either $13$ or $14$  points for which
inequality~\ref{eqn:CE} holds.
\end{lemma}


\begin{proof} If $ V$ contains at most $12$  points then
the inequality~\ref{eqn:CE} cannot hold, because $L\le 1$.

Consider a finite packing $ V=\{\u_1,\ldots, \u_N\}$ satisfying
inequality~\ref{eqn:CE}.  Without loss of generality, by adding points
as necessary, the packing becomes weakly saturated in the sense of
Lemma~\ref{lemma:poly-bounded}, with $r=2$ and $r'=2\hm$.  Set $h_i =
\normo{ \u_i}/2$.  Then $h_i\le h_0=1.26$.  Set
\begin{displaymath}%
g(h) = \arccos(h/2) - \pi/6.  %
\end{displaymath}%
On the unit sphere, consider the disks $D_i$ of radii $g(h_i)$,
centered at $ \u_i/\normo{ \u_i}$.  These disks do not overlap; this
follows from the easy
Calculation~\ref{calc:cc:disks} %% Marchal disks are disjoint.  % cc:disks
\begin{equation}\label{eqn:disks}
g(h_i) + g(h_j) \le \op{arc}(2h_i,2h_j,2).%
\end{equation}%
\indy{Notation}{D@$D$ (spherical disks)}%
For each $i$, the plane through the circular boundary of $D_i$ bounds
a half-space containing the origin.  The intersection of these
half-spaces is a polyhedron $P$.  The polyhedron is bounded by
Lemma~\ref{lemma:poly-bounded}.  Lemma~\ref{lemma:polyhedron}
associates a fan $(V_P,E_P)$ with $P$.  (The set $V_P$ is dual to $
V$; the set $V_P$ is in bijection with extreme points of $P$, whereas $
V$ is in bijection with the facets of $P$.)  There are natural
bijections between the following sets:
\begin{itemize}
\item $ V = \{ \u_1,\ldots, \u_N\}$.
\item The  facets of $P$.
\item The set of  topological components of $Y(V_P,E_P)$.
\item The set of faces in the hypermaps $\op{hyp}(V_P,E_P)$.
\end{itemize}
The bijection of the first two sets follows from the first conclusion
of Lemma~\ref{lemma:webster}.  Lemmas~\ref{lemma:WF} and
~\ref{lemma:face} give the other two bijections.

Let $k_i$ be the cadinality of the face in $\op{hyp}(V_P,E_P)$
corresponding to the facet $i$.  By Lemma~\ref{lemma:ngon}, the solid
angle of the topological component $W_i$ of $Y(V_P,E_P)$ is at least
$\op{reg}(g(h_i),k_i)$, where \indy{Index}{half-plane}%
\indy{Index}{half-space}%
\indy{Notation}{reg (area of regular spherical polygon)}%
\begin{displaymath}
\op{reg}(a,k) = 2\pi - 2 k (\arcsin(\cos(a)\sin(\pi/k))).
\end{displaymath}
By Calculation~\ref{calc:cc:alin}, %% Linear lower bound on regular
%% polygon. % cc:alin
\begin{equation}\label{eqn:alin}
\op{reg}(g(h),k) \ge c_0 + c_1 k + c_2 L(h),\quad
k = 3,4,\ldots,\quad 1\le h\le \hm,
\end{equation}
where
\begin{displaymath}c_0 = 0.6327,\quad c_1 = -0.0333,\quad c_2 =
0.4754.\end{displaymath} The sum $\sum_i k_i$ is the number of darts
in $\op{hyp}(V_P,E_P)$ by  Lemma~\ref{lemma:polyhedron}.  By
Lemma~\ref{lemma:dart-upper}, $\sum_i k_i \le (6N-12)$.  Summing over
$i$, an estimate on $N$ follows: 
\indy{Index}{polyhedron!convex}%
\indy{Index}{hypermap!planar}%
\begin{eqnarray*}
4\pi &=& \sum_i\op{sol}(W_i)\\
&\ge& \sum_i \op{reg}(g(h_i),k_i) \\
&\ge& c_0 N +c_1\sum_i k_i + c_2 \sum L(h_i)\\
&\ge& c_0 N +c_1 (6N-12) + c_2 12\\
\end{eqnarray*}
This gives
\begin{displaymath}
14.93 \ge N.
\end{displaymath}
\end{proof} 


\begin{lemma}[]\guid{XULJEPR}\label{300}\label{lemma:D'}  
Let $ V\subset\BB$ be a finite packing containing the origin.
Assume that there exists $ \u_1$ such that $\normo{ \u_1}=2$ and
$\norm{ \u_1}{ \u}\ge 2\hm$ for all $\u\in V$,
Then  inequality~(\ref{eqn:CE}) does not hold on $ V$.
\end{lemma}

\begin{proof} Assume for a contradiction that a packing exists that
satisfies the assumptions and the inequality.  Without loss of
generality, assume that $N\ge 13$, since the inequality is known to
hold when $N\le 12$.  Create one large disk $D_1'$ centered at $
\u_1/2$ and repeat the proof of the previous lemma.  Extend the packing to a the weak
saturation with parameters $r=r'=2\hm$.  This can be done in a way that maintains
the assumptions on $\u_1$.  By
Lemma~\ref{lemma:poly-bounded}, the polyhedron is
bounded.  By the assumptions of the lemma, take
\begin{displaymath}a'=\arc(2,2,2\hm)-g(\hm) \approx 0.797\end{displaymath}
for the arcradius of the large disk $D_1'$.  
By Calculation~\ref{calc:cc:alin2}, %% Linear lower bound on regular polygon (large disk) % cc:alin2
\begin{equation}\label{eqn:alin2}
\op{reg}(a',k) \ge c_0 + c_1 k + c_2 L(1) +
c_3\end{equation}
where $c_3 = 0.85$.
Then 
\begin{eqnarray*}
4\pi &=& \sum_i\op{sol}(W_i)\\
&\ge& \op{reg}(a',k_1)+\sum_{i>1} \op{reg}(g(h_i),k_i) \\
&\ge&  c_0 N +c_1\sum_i k + c_2 \sum L(h_i) + c_3\\
&\ge& c_0 N +c_1 (6N-12) + c_2 12 + c_3\\
\end{eqnarray*}
This gives a contradiction:
\begin{displaymath}
12.97 \ge N \ge 13.
\end{displaymath}
\end{proof}
