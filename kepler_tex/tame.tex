% Flyspeck
% Thomas C. Hales
% Starting with Chapter on Tame Hypermaps



\chapter{Tame Hypermap}

\label{sec:tame}


This chapter defines a class of hypermaps.  Hypermaps in this class
are said to be {\it tame}.  A complete
classification of all tame hypermaps has been carried out by computer.   This classification solves a
major step of the packing problem.\FIXX{Unresolved chapter issues: \begin{itemize}\item $A$ vs. $v$ for nodes
\end{itemize}}

\section{Definition}


\begin{definition}[triangle,~quadrilateral,~exceptional]
Let $H$ be a hypermap.
Faces of cardinality $3$ are called {\it triangles}, those of
cardinality $4$ are called {\it quadrilaterals}, and so forth. Let
$p_v$ be the number of triangles incident with a node $v$. A face of
cardinality at least $5$ is called an {\it exceptional\/} face.
 %
 \indy{Index}{triangle}
 \indy{Index}{exceptional}
 \indy{Index}{quadrilateral}
 \indy{Index}{exceptional!face}
 \indy{Index}{pZ@$p_v$}
\end{definition}

\begin{definition}[type,~$(p,q,r)$]\label{definition:type}
The {\it type\/} of a node is defined to be a triple of
non-negative integers $(p,q,r)$, where $p$ is the number of
triangles containing the node, $q$ is the number of quadrilaterals
containing it, and $r$ is the number of exceptional faces.
%
 \indy{Index}{type (of a node)}
\end{definition}


\subsection{weight assignment}\label{sec:wtassign}

We call the constant $\op{tgt}=1.541$, which arises repeatedly in
this chapter, the {\it target}. 
%
 \indy{Index}{target}\indy{Index}{tgt@$\op{tgt}=1.541$}
 \indy{Greek}{ZZdzeta@$\zeta= 1/(2\arctan(5,\sqrt{2}))$}

\begin{definition}[a]
  Define $a:\ring{N}\to \ring{R}$ by
  $$
  a(n) = \hbox{ if } (n=5) \hbox{ then } 0.63 \hbox{ else } \op{tgt}.
  $$
\end{definition}

\begin{definition}[b]
  Define $b:\ring{N}\times \ring{N}\to \ring{R}$ by $b\pqr{(p,q,0)}=1.541$,
  except for the values in the following table
  (with  $\op{tgt}=1.541$):
  {
  \def\tx{\op{tgt}}
  $$\begin{matrix}  &q=0&1&2&3&4\\
           p=0&\tx&\tx&\tx&0.618&1.0\\
           1&\tx&\tx&0.66&0.618&\tx\\
           2&\tx&0.8&0.412&1.2851&\tx\\
           3&\tx&0.315&0.83&\tx&\tx\\
           4&0.35&0.374&\tx&\tx&\tx\\
           5&0.04&1.144&\tx&\tx&\tx\\
           6&0.689&\tx&\tx&\tx&\tx\\
           7&1.450&\tx&\tx&\tx&\tx
   \end{matrix}
   $$
   }
\end{definition}


\begin{definition}[d]
    Define $d:\ring{N}\to \ring{R}$ by
  $$d(n) = \begin{cases}
    0 & n=3, \\
    0.206 & n=4, \\
    0.483 & n=5, \\
    0.760 & n=6, \\
    1.038 & n=7, \\
    1.315 & n=8,\\
    \op{tgt}=1.541 & \text{otherwise}.
  \end{cases}
  $$
\end{definition}


We write $\card(S)$ for the cardinality of a finite set $S$.\indy{Index}{card}
\indy{Index}{cardinality}


\begin{definition}[weight~assignment]
%
A {\it weight assignment\/} of a hypermap $H$ is a function $w$ on
the set of faces of $H$, taking values in the set of non-negative
real numbers. A weight assignment is {\it admissible} if the
following properties hold:
%
 \indy{Index}{weight assignment}
 \indy{Index}{admissible (weight assignment)}
\begin{enumerate}
  \item If the face $F$ has cardinality $n$, then
        $w(F) \ge d(n)$
  \item If a node $v$ has type $(p,q,0)$, then
        $$\sum_{F:\,v\cap F\ne\emptyset} w(F) \ge b{\pqr{(p,q,0)}}.$$
        \label{admissible:b}
  \item Let $v$ be any node of type $(5,0,1)$, and let $A$ be the set of
triangles meeting that node.
        Then
        $$\sum_{F\in A} w(F)
            \ge  a(5).$$
        \label{definition:admissible:excess}
\end{enumerate}
The sum $\sum_F w(F)$ is called the {\it total weight} of $w$.
\indy{Index}{total weight}
\end{definition}





\subsection{hypermap property}
\label{sec:graphproperty}

\begin{definition}\footnote{Move to the chapter
on hypermap or fan}
An articulation node of a hypermap is a node for which
there exists two darts $x,y$, 
neither belonging to the node,
such that every contour path from $x$ to $y$ passes
through the given node.
A connected hypermap is biconnected
if it contains no articulation node.  
\end{definition}


We say that a hypermap is {\it tame\/} if it satisfies the following
conditions.
%
 \indy{Index}{tame}

\begin{enumerate}
    \label{definition:tame}
    %1
    \item {\bf (planar)} The hypermap is plain, planar.
    \item {\bf (biconnected)} The hypermap is connected and biconnected.  In particular, every face meets every node in at most one dart.
    \item {\bf (nondegenerate)} The edge map $e$ has no fixed points.
    \item {\bf (no loops)} The two darts of each edge lie in different nodes.
    \item {\bf (no double joins)} The set of edges meeting any two given nodes has cardinality at most $1$.
    \label{definition:tame:40}

    \item {\bf (triangles)} If $L$ is a contour loop with $3$ face steps, and if there exists a node in
    the exterior of $L$, then $L$ is a face of the hypermap.
    \label{definition:tame:3-circuit}

    \item {\bf (quadrilaterals)} If $L$ is a contour loop  $4$ face steps, and there are at least two nodes
    in the exterior of $L$, then the interior of $L$ takes one of the forms
    illustrated in Figure
    \ref{fig:fourcircuit}.\FIXX{Make this more precise.}
    \label{definition:tame:4-circuit}
    \begin{figure}[htb]
        \centering
        \myincludegraphics{\ps/tame4circuit.eps}
        \caption{Tame $4$-circuits}
        \label{fig:fourcircuit}
    \end{figure}
  \item {\bf (face)} There are at least three faces.
    \item {\bf (face)} The cardinality of each face is at least $3$ and at most $6$.
    \label{definition:tame:length}
    \item {\bf (node)} There are either $13$ or $14$ nodes.
    \item {\bf (node)} The cardinality of every node is at least $3$ and at most
    $7$.
    \label{definition:tame:degree}
    \item {\bf (node)} If a node is incident with an exceptional face,
        then the cardinality of the node is at most $6$.
    \label{definition:tame:degreeE}
    \item {\bf (weights)} There exists an admissible weight assignment
        of total weight less than the target, $\op{tgt}=1.541$.
    \label{definition:tame:squander}
\end{enumerate}
%

\section{Classification}
    \label{sec:proof-classification}

%\section{Statement of the Theorem}
\label{sec:classification}

A list of several thousand hypermaps appears at \cite{website:Hales:1998:Code}. The
following theorem is listed as one of the central claims in the
proof in Section~\ref{sec:logic}.

\begin{definition}[opposite] The opposite of a hypermap $(D,e,n,f)$ is the
hypermap $(D,f n,n^{-1},f^{-1})$.
\end{definition}

\begin{lemma}\guid{PPHEUFG} If a hypermap has properties\FIXX{Give reference}, 
then so does its opposite.
\end{lemma}

\begin{theorem}
\label{theorem:classification} Every tame hypermap is isomorphic to
a hypermap in this list, or is isomorphic to the opposite of a
hypermap in this list.
\end{theorem}

The results of this section are not needed except in the proof of
Theorem \ref{theorem:classification}.

\smallskip

Computers are used to generate a list of all hypermaps and to check
them against the archive of tame hypermaps.  The computer program is
based on the face-insertion construction of Lemma~\ref{XX}.  There it is
proved that all sufficiently nice hypermaps can be generated by an
elementary face-insertion process.  Tame hypermaps satisfy all the
hypotheses of that lemma.

\section{Contravening hypermap}

Our aim is to prove conjecture~(XX).  For that purpose, we assume that
we have a counterexample to inequality~(XX).  The purpose of this section is to select a special class of counterexamples.

Associated with a centered packing $(\Lambda,\vartheta)$ is a contact fan and a standard fan.  For both fans, set
$$
V = \{v\in \Lambda\mid \|v-\vartheta\|\le 2\hm\}.
$$
Let
\begin{equation}\label{eqn:fan-edge}
\begin{array}{lll}
 E_{std} &= \{\{v,w\}\in V^2\mid |v-w|\le 2\hm \}\\
 E_{cnt} &= \{\{v,w\}\in V^2\mid |v-w|= 2 \}\subset E_{std}.
\end{array}
\end{equation}

\begin{lemma}
$(\vartheta,V,E)$ is a fan, if $E=E_{std}$ or $E_{cnt}$.
\end{lemma}
They are called the standard fan and contact fan.
Let $\op{hyp}(\vartheta,V,E)$ be the associated hypermap.

\begin{definition}
Let $(\vartheta,V,E)$ be a fan.
We say that $v\in V$ is isolated in the fan if $E(v)$ is empty.
(That is, the degree of the corresponding node in the hypermap is $0$.) We say that $v\in V$ is rigid in the fan if the azimuth angles of the darts in the fan at $v$ are all $<\pi$.  (In particular, the cardinality of $E(v)$ is at least three.)
\end{definition}

The following lemma appear in Sch\"utte and van der Waerden~\cite{vanderWaerden:1951}.

\begin{lemma}
Given any $V\subset \Lambda(\vartheta,2\hm)$,
there exists a finite packing $V'$ 
in bijective correspondence with $V$:
$$
f:V\to V'
$$
such that $\|v-\vartheta\| = \|f(v)-\vartheta\|$ and
such that every $v'\in V'$
is isolated or rigid in the contact fan of $V'$.
\end{lemma}

\begin{proof} Consider all finite packings in 
bijective correspondence with $V$, such that the
bijection preserves distances to $\vartheta$.
Among these packings pick one $V'$ with the largest number
of isolated points in the contact graph.  If there is a point $v\in V'$ that
is not isolated and not rigid, then it has a contact
dart with angle at least $\pi$.  We can then perturb $v$ away from the contacts, making it isolated, while preserving its distance to $\vartheta$.  The perturbed packing $V''$ has one more isolated point than $V'$, contrary to the supposed maximality of $V'$.  Hence, the conclusion of the 
lemma holds for $V'$.
\end{proof}

The following lemma shows that the corresponding result
holds for the standard fan.

\begin{lemma}\label{lemma:rigid}  
Given any $V\subset \Lambda(\vartheta,2\hm)$,
there exists a finite packing $V'$ 
in bijective correspondence with $V$:
$$
f:V\to V'
$$
such that $\|v-\vartheta\| = \|f(v)-\vartheta\|$ and
such that every $v'\in V'$
is isolated or rigid in the standard fan of $V'$.
\end{lemma}

\begin{proof}  
Let $V'_{iso},V'_{rid}$ be the
sets of isolated and rigid vertices of $V'$ in the contact
fan.
By the previous lemma, we may assume
without generality that every $v\in V$ is isolated
or rigid in the contact fan of $V$.    
We consider the set $\CalV$ finite
packings $V'$ in bijective correspondence with $V$:
$$
f:V\to V'
$$
preserving distances to $\vartheta$, with $V'_{rid}=V_{rig}$.
For any such packing, define constants $d_i=d_i(V')$,
for $i=1,\ldots,n=\op{card}(V_{iso})$ by letting 
$d(f(v))$, for $v\in V_{iso}$, 
be the minimum distance of $f(v)$ to a point
in $V'\setminus \{f(v)\}$, and then ordering the real numbers $d(f(v))$ in increasing order:
$$
d_0 \le d_1 \le d_2 \le \cdots \le d_n.
$$
The constants $d_i:\CalV\to\ring{R}$ are continuous functions on the compact configuration space $\CalV$.
We have $V\in \CalV$, so $\CalV\ne\emptyset$.
There is a nonempty 
compact subset $\CalV_0$ of $\CalV$ on which
$d_0$ attains its maximum value. Continuing recursively,
there is a nonempty compact subset $\CalV_{i+1}$ of
$\CalV_{i}$ on which $d_i$ attains its maximum value.

For the configuration $V\in \CalV$, we have $d_0 >2$.
Hence, $d_0(V')>2$ for $V'\in\CalV_0$.  It follows
that $V'_{iso}$ is in bijective correspondence with
$V_{iso}$.

We claim that any $V'\in \CalV_n$ has the desired property.
Suppose on the contrary, that some vertex $v'\in V'$
is neither isolated nor rigid in the standard fan.  
Any such vertex satisfies $d(v')=d_i(V')$, for some $i$.
Among such vertices, pick the vertex with smallest index
$i$.  Write 
$$
d(v') = d_i(V') = d_{i+1}(V') =\cdots= d_j(V') < d_{j+1}(V'),
$$
for some $j\ge i$.  As $v'$ is not isolated in the
standard fan, we have $d_j(v') < 2.52$.  It is not rigid,
so the cyclic order on
$$
\{(v',w')\in V'^2 \mid \|v'-w'\| = d_j(v'),
$$
has an azimuth angle that is at least $\pi$.
Thus, there is a direction in which $v'$ can be perturbed
that increases $d_j$, while keeping $d_0,\ldots,d_{j-1}$
fixed.  This is contrary to the defining property of
$\CalV_n\subset\CalV_j$.
\end{proof}


We may assume that we have an extremal counterexample
in the sense that it maximizes the left-hand side of inequality~(\ref{XX}).  We may assume that the set of vertices $V$ has cardinality $13$ or $14$ (Lemma~{XX}).
We may assume that every vertex is rigid or isolated in
the standard fan (Lemma~\ref{lemma:rigid}).  By Lemma~\ref{XX}, if there are any isolated verices in the standard fan, then the inequality~(\ref{XX}) holds.  Hence, we may in fact assume that every vertex is rigid in the standard fan.
By extremality and construction, we may assume that every vertex $v$ that is not rigid in the contact fan satisfies $\|v-\vartheta\|=2$.

Among all extremal counterexamples with these properties,
we may chose those that minimize the number of connected components
of the standard hypermap.
If the standard hypermap contains at least two components, let $W\subset V$ be the set of vertices of one of the components.  There is an axis $\ell$ such that the orthogonal transformations $T(\theta)$ around axis $\ell$ and small angle $\theta$  have the effect of rotating $W$ in the direction of the vertices of another component, until for some angle $\theta$, the condition~(\ref{eqn:fan-edge}) for
a new edge joining the components is met.  Thus, we may assume the standard hypermap has a single connected component.

In summary, we have constructed a counterexample with the following properties.

\begin{lemma} Assume that a counterexample exists to inequality~(XX).  Then there also exists a counterexample with the following properties.
\begin{itemize}
\item It maximizes the sum XX.
\item There are $13$ or $14$ vertices.
\item The standard hypermap is connected.
\item Every vertex is rigid in the standard fan.
\item Every vertex $v$ that is not rigid in the contact
fan satisfies $\|v-\vartheta\|=2$.
\end{itemize}
\end{lemma}







In the next section, we study further properties of contravening hypermaps $H$.





\section{Contravention is Tame}
    \label{sec:contraproof}

Let $(\Lambda,v_0)$ be a centered packing with
fan $(v_0,V,E)$.  Let  hypermap $H=(D,e,n,f)$
be the planar hypermap attached to $(v_0,V,E)$.
The hypermap $H$ is connected and biconnected.

The connected components of $Y(v_0,V,E)$ are in bijection with
faces of $H$.  
The fan gives a azimuth angle function
$$
\op{azim} : D \to (0,2\pi).
$$
For each face of $H$, the corresponding component $R$
is eventually radial with solid
angle
  $$
  2\pi + \sum_{x\in F} (\op{azim}(x) -\pi).
  $$
We write $\sol(F)$ for the solid angle of the connected component
of $Y(v_0,V,E)$ associated with a face $F$ of the hypermap.
We have
    $$\sum_{F} \sol(F) = 4\pi.$$


For each face, there is a
real number $\tau(F)=\tau(\Lambda,F)$ such that
$$
  \sum_{F : \text{face}}\tau(\Lambda,F)
$$
We define a weight function $w(F)$ on the faces of the hypermap
by $w(F) = \tau(F)$.  In this way, we attach
a pair $(H,w)$ to each contravening centered packing $(\Lambda,v_0)$.


\begin{theorem} \label{theorem:contravene}
Let $(\Lambda,v_0)$ be a contravening centered packing.  Let $(H,w)$ be
the hypermap and function on its faces attached to $(\Lambda,v_0)$ as above.
Then $H$ is a tame hypermap with admissible weight function $w$.
\end{theorem}




\subsection{general properties}
    \label{sec:startame}


Recall that we say that a node $v$ has {\it type\/} $(p,q,r)$ if
there are exactly $p+q+r$ faces that meet the node, of which exactly
$p$ triangles and $q$ quadrilateral faces (see
Definition~\ref{definition:type}).  We write $(p_v,q_v,r_v)$ for the
type of a node $v$.

\begin{lemma}\guid{JGTDEBU} The contravening hypermap $H$ satisfies Conditions~\ref{XX}-\ref{XX} 
of tameness.
Explicitly, it is a plain, planar, connected, and biconnected. 
The edge map $e$
has no fixed points. There are at least three faces. Every face meets
every node in at most one dart.  There are never two nodes of type
$(4,0,0)$ that are adjacent to each other.  Every face has
cardinality at least $3$ and at most $8$.  If $L$ is a contour loop
with $3$ face steps, and if there exists a node in the exterior of
$L$, then $L$ is a face of the hypermap.
\end{lemma}

\begin{proof}
It is connected and biconnected by properties of contravening hypermaps.
It is plain and planar by general properties that we have established
about the hypermaps associated with fans.
A fixed point for the edge map is a degenerate dart, which does not occur in a connected hypermap.
\end{proof}





\subsection{properties of nodes}


\begin{lemma}\guid{SZIPOAS}\dcg{Lemma~21.4}{223} 
Formally contravening hypermaps satisfy Property
\ref{definition:tame:degree} of tameness: The cardinality of every
node is at least $2$ and at most $7$.
\end{lemma}

\begin{proof}  There is no node of cardinality one by
Lemma~\ref{lemma:nondegen}.  There is no node of degree
greater than $7$ by Lemma~\ref{a:6}.
\end{proof}


\begin{lemma}\guid{GNCEGFN}\label{lemma:no-2}
Let $(H,\azim,\flat,\sigma)$ be a formally contravening hypermap.
Suppose that $L$ is a contour loop with at most four face steps.
Suppose that there are at least two nodes in the exterior of $L$.
Then there at most one node interior to $L$.
\end{lemma}


\begin{lemma}\guid{CDTETAT} \label{lemma:0.852}
Let $(H,\azim,\flat,\sigma)$ be a formally contravening
hypermap. For every dart $x$,
    $$0.852\le \azim(x)\le 1.9.$$
For every dart $x$ whose face is not a triangle, we have
    $$1.15\le\azim(x)\le 3.27.$$
Consequently, if a vertex $v$ has type $(p,q,0)$, then $(p,q)$
must be one of the following pairs:
$$
\begin{array}{lll}
&(0,2),~(0,3),~(0,4),~(0,5),~(1,2),~(1,3),~(1,4),~(2,1),~(2,2),~(2,3),\\
&(3,1),~(3,2),~(3,3),~(4,0),~(4,1),~(4,2),~(5,0),~(5,1),~(6,0),~(6,1),~(7,0)
\end{array}
$$
\end{lemma}
 %
 \indy{Index}{ZZZZ1.15@$1.15$}
 \indy{Index}{ZZZZ0.852@$0.852$}
\begin{proof}
The angle bounds are a calculation.  The sum of the azimuth angles
around a vertex satisfies:
$$
  p (0.852) + q (1.15) \le 2\pi \le p (1.9) + q (3.27),
$$
and the pairs satisfying these constraints are listed.
\end{proof}



\begin{lemma}\guid{RTOIQJC}\label{lemma:nobad4}
Let $H$ be a contravening hypermap.
There does not exist a node of 
type $(1,0,1)$ with precisely one triangle and
one pentagon, as show in Figure~\ref{fig:no4circuit}. 
\end{lemma}

\begin{proof}  Every node in
the hypermap has degree at least three, but such a node
has degree two.
\end{proof}

\begin{lemma}\guid{BDJYFFB} \label{lemma:deg5}
Every formally contravening hypermap satisfies Property
\ref{definition:tame:degreeE} of tameness: If a node meets an
exceptional face, then the cardinality of the node is at most $6$.
\end{lemma}

\subsection{faces}

\begin{lemma}  Every face has has size at least $3$ and at most $6$.
\end{lemma}

\begin{proof} It is enough to show that there does not exist a face (of the hypermap associated with the standard fan)
with the following two properties:
\begin{enumerate}
\item Every dart has angle less than $\pi$.
\item The sum of the heights $(|v|-2)$, as $v$ runs
over the vertices of the face, is at most $2(0.52)$.
\end{enumerate}
Indeed, we have shown that the first property holds
of a contravening hypermap.  If the second property fails,
then since $L_2$ is the linear interpolation between
the points $(1,1)$ and $(1.26,0)$, and there are at most
$14$ points in $V$, we have
$$\sum_{v\in V}L_2(|v|/2) \le 12 L_2(1) + 2 L_2(1.26) =12$$
and the main inequality holds.

By \cite{vanderWaerden:1951}, we have that the sum of
the arclengths of the edges of the cycle is less than
$2\pi$.  Their argument briefly goes as follows. Relaxing the constraint so that the angles are $\le \pi$, we deform the polygon so that it has the maximum number of angles flattened to $\pi$.  As long as there are more than three angles $<\i$, the polygon can be further flattened, so it is not maximal.  There results a triangle.  The three arclengths of a spherical triangle sum to less than $2\pi$.  (The extreme case occurs when all three points lie on a great circle and all angles equal $\pi$.)

The edge $\{v,w\}$ has arclength at least
$$
\arc(|v|,|w|,|v-w|) \ge \arc(|v|,|w|,2). 
$$

A calculation\footnote{XX} gives
$$
\arc(|v|,|w|,2)\ge 1 - 0.6076 (|w|/2 - 1) - 0.6076 (|v|/2 - 1).
$$
Summing over a face of size at least $7$, we obtain:
$$
\sum \arc(|v_i|,|v_{i+1}|,|v_i-v_{i+1}|)\ge
7 - 0.6076 \sum (|v_i|-2) \ge 7 - 0.6076 (2) (0.52) > 2\pi,
$$
which is contrary to the upper bound on the edge length
sum.
\end{proof}

\begin{lemma}  The function $\tau(F)$ on a face of
size $n$ satisfies 
$$
\tau(F) \ge d(n),
$$
where $d(n)$ is the constant in XX.
\end{lemma}

\begin{proof}  We consider the cycle in the fan
giving the face $F$.  For the purposes of this
estimate, we may discard all the vertices of $V$
that do not belong to the cycle as well as all edges
of $E_{std}$ that lie outside the cycle.
Write $\tau(V)$ for the value of $\tau(F)$ as a
function of the positions of the vertices in $V$,
keeping the set of edges of the cycle fixed.  The
azimuth angles associated with the face $F$ will be
call the interior angles.  

We will deform the cycle by moving the points of $V$ in such a way that is nonincreasing in $\tau(V)$.
The deformation is required to maintain the following
constraints:
\begin{itemize}
\item $V\cup\{\theta\}$ is a packing.
\item The interior angles are at most $\pi$.
\item The distances satisfy $\|v-w\|\ge 2.52$, if $\{v,w\}$ is
not an edge.  (Call these the diagonals.)
\end{itemize}
\end{proof}

Initially, the distances $\|v-w\|>2.52$ for diagonals.
If after deformation, equality holds; we stop the deformation, and cut the cycle into two smaller along the diagonal
and continue with deformations for each part separately.
Any edge along XXD


\subsection{triangles and quadrilaterals}


\label{sec:2.2}  To continue with the proof that formally
contravening hypermaps are tame, we need to introduce some more
notation and methods.



Every contour loop partitions the faces into the interior and
exterior.  Every contour loop partitions the nodes that do not meet
the loop into exterior and interior nodes.
%
 \indy{Index}{interior node}

Lemma~\ref{lemma:no-2} asserts that either the interior or the
exterior has at most $1$ enclosed vertex.   When choosing which
aggregate is to be called the interior, we may make our choice so
that the interior has area at most $2\pi$, and hence contains at
most $1$ node. With this choice, we have the following lemma.

\begin{lemma}\guid{CHOMXMX}
Let $(H,\azim,\flat,\sigma)$ be a formally contravening hypermap. If
$L$ is a contour loop with $4$ face steps, and there are at least
two nodes in the exterior of $L$, then the interior of $L$ takes one
of the forms illustrated in Figure~\ref{XX} in Property
    \ref{definition:tame:4-circuit} of tameness.
\end{lemma}

\begin{proof}
By Lemma~\ref{XX}, the interior of $L$ contains at most one node.

$H$ is a connected plane planar map.  We form a normal family of
contour loops ${\cal L}$ by taking the contour loop $L^{-1}$
reversing $L$\FIXX{Explain} and all the faces in the interior of $L$.
(Check this is a normal family.)  The quotient $H' = H/{\cal L}$ is
a plane planar map.  There is a further quotient of $H'$ with normal
family $\{L,L^{-1}\}$, which is isomorphic to $P_4$ with the natural
flag coming from $H'$.  The niceness conditions of Lemma~\ref{XX} are
satisfied, so we can recover $H'$ from $P_4$ by a sequence of
face-insertions.  Since the interior of $L$ contains at most one
node, this gives restrictions on the partitions that can be used in
face-insertion.

If there are no enclosed vertices, then the only possibilities are
for it to be a single quadrilateral face or a pair of adjacent
triangles.

Assume there is one enclosed vertex $v$.  If $v$ is connected to $3$
or $4$ nodes of the quadrilateral, then that possibility is listed
as part of the conclusion.

If $v$ is connected to $2$ opposite nodes in the $4$-cycle, then the
node $v$ has type $(0,2,0)$ and the bounds of
Lemma~\ref{lemma:pq} show that the hypermap cannot be formally
contravening.

If $v$ is connected to $2$ adjacent nodes in the $4$-cycle, then we
appeal to Lemma~\ref{lemma:nobad4} to conclude that the hypermap
does not contravene.

If $v$ is connected to $0$ or $1$ nodes, then we appeal to
Lemma~\ref{lemma:enclosed:bis}.  This completes the proof.
\end{proof}


\subsection{weight assignment}
    \label{sec:weight}

The purpose of this section is to prove the existence of a good
admissible weight assignment for formally contravening hypermaps.
This will complete the proof that all formally contravening
hypermaps are tame.

\begin{theorem}  Every formally contravening hypermap has an admissible
weight assignment of total weight less than $\op{tgt}=1.541$.
\end{theorem}

Given a formally contravening hypermap $(H,\azim,\flat,\sigma)$, we
define a weight assignment $w$ by
    $$F \mapsto w(F) = \tau(F).$$
The challenge of the theorem will be to prove that $w$, when
defined by this formula, is admissible.

\subsection{admissibility}
\label{sec:admissibility}

The next three lemmas establish that this definition of $w(F)$ for
formally contravening hypermaps satisfies the first three defining
properties of an admissible weight assignment.

\begin{lemma}\guid{OLNSWLK}  Let $F$ be a face of cardinality $n$ in a formally contravening hypermap.
Define $w(F)$ as above. Then
        $w(F) \ge d(n)$.
\end{lemma}

\begin{proof} This is Lemma~\ref{XX}. %\ref{proposition:wttau}.
\end{proof}

\begin{lemma}\guid{RVFTYCI} Let $v$ be a node of type $(p,q,0)$ in a
formally contravening hypermap.  Define $w(F)$ as above. Then
        $$\sum_{v\in F} w(F) \ge b{\pqr{(p,q,0)}}.$$
\end{lemma}


\begin{proof} This is Lemma~\ref{lemma:pq}.
\end{proof}



The proof that formally contravening hypermaps are tame is complete.




\begin{lemma}\guid{FUJCTEI}\label{lemma:enclosed:bis} % {Lemma 2.2}
A quadrilateral component does not enclose any vertices of height at
most $2t_0$.
\end{lemma}