% Flyspeck
% Thomas C. Hales
% Starting with Chapter on Tame Hypermaps



\chapter{Tame Hypermap}

\label{sec:tame}


This chapter defines a class of hypermaps.  Hypermaps in this class
are said to be {\it tame}.  A complete
classification of all tame hypermaps has been carried out by computer.   This classification solves a
major step of the packing problem.\FIXX{Unresolved chapter issues: \begin{itemize}\item $A$ vs. $v$ for nodes
\end{itemize}}

\section{Definition}


\begin{definition}[triangle,~quadrilateral,~exceptional]
Let $H$ be a hypermap.
Faces of cardinality $3$ are called {\it triangles}, those of
cardinality $4$ are called {\it quadrilaterals}, and so forth. Let
$p_v$ be the number of triangles incident with a node $v$. A face of
cardinality at least $5$ is called an {\it exceptional\/} face.
 %
 \indy{Index}{triangle}
 \indy{Index}{exceptional}
 \indy{Index}{quadrilateral}
 \indy{Index}{exceptional!face}
 \indy{Index}{pZ@$p_v$}
\end{definition}

\begin{definition}[type,~$(p,q,r)$]\label{definition:type}
The {\it type\/} of a node is defined to be a triple of
non-negative integers $(p,q,r)$, where $p$ is the number of
triangles containing the node, $q$ is the number of quadrilaterals
containing it, and $r$ is the number of exceptional faces.
%
 \indy{Index}{type (of a node)}
\end{definition}


\subsection{weight assignment}\label{sec:wtassign}

We call the constant $\op{tgt}=1.541$, which arises repeatedly in
this chapter, the {\it target}. 
%
 \indy{Index}{target}\indy{Index}{tgt@$\op{tgt}=1.541$}
 \indy{Greek}{ZZdzeta@$\zeta= 1/(2\arctan(5,\sqrt{2}))$}

\begin{definition}[a]
  Define $a:\ring{N}\to \ring{R}$ by
  $$
  a(n) = \hbox{ if } (n=5) \hbox{ then } 0.63 \hbox{ else } \op{tgt}.
  $$
\end{definition}

\begin{definition}[b]
  Define $b:\ring{N}\times \ring{N}\to \ring{R}$ by $b\pqr{(p,q,0)}=1.541$,
  except for the values in the following table
  (with  $\op{tgt}=1.541$):
  {
  \def\tx{\op{tgt}}
  $$\begin{matrix}  &q=0&1&2&3&4\\
           p=0&\tx&\tx&\tx&0.618&1.0\\
           1&\tx&\tx&0.66&0.618&\tx\\
           2&\tx&0.8&0.412&1.2851&\tx\\
           3&\tx&0.315&0.83&\tx&\tx\\
           4&0.35&0.374&\tx&\tx&\tx\\
           5&0.04&1.144&\tx&\tx&\tx\\
           6&0.689&\tx&\tx&\tx&\tx\\
           7&1.450&\tx&\tx&\tx&\tx
   \end{matrix}
   $$
   }
\end{definition}


\begin{definition}[d]
    Define $d:\ring{N}\to \ring{R}$ by
  $$d(n) = \begin{cases}
    0 & n=3, \\
    0.206 & n=4, \\
    0.483 & n=5, \\
    0.760 & n=6, \\
    %1.038 & n=7, \\
    %1.315 & n=8,\\
    \op{tgt}=1.541 & \text{otherwise}.
  \end{cases}
  $$
\end{definition}


We write $\card(S)$ for the cardinality of a finite set $S$.\indy{Index}{card}
\indy{Index}{cardinality}


\begin{definition}[weight~assignment]
%
A {\it weight assignment\/} of a hypermap $H$ is a function $w$ on
the set of faces of $H$, taking values in the set of non-negative
real numbers. A weight assignment is {\it admissible} if the
following properties hold:
%
 \indy{Index}{weight assignment}
 \indy{Index}{admissible (weight assignment)}
\begin{enumerate}
  \item If the face $F$ has cardinality $n$, then
        $w(F) \ge d(n)$
  \item If a node $v$ has type $(p,q,0)$, then
        $$\sum_{F:\,v\cap F\ne\emptyset} w(F) \ge b{\pqr{(p,q,0)}}.$$
        \label{admissible:b}
  \item Let $v$ be any node of type $(5,0,1)$, and let $A$ be the set of
triangles meeting that node.
        Then
        $$\sum_{F\in A} w(F)
            \ge  a(5).$$
        \label{definition:admissible:excess}
\end{enumerate}
The sum $\sum_F w(F)$ is called the {\it total weight} of $w$.
\indy{Index}{total weight}
\end{definition}





\subsection{hypermap property}
\label{sec:graphproperty}

\begin{definition}\footnote{Move to the chapter
on hypermap or fan}
An articulation node of a hypermap is a node for which
there exists two darts $x,y$, 
neither belonging to the node,
such that every contour path from $x$ to $y$ passes
through the given node.
A connected hypermap is biconnected
if it contains no articulation node.  
\end{definition}


We say that a hypermap is {\it tame\/} if it satisfies the following
conditions.
%
 \indy{Index}{tame}

\begin{enumerate}
    \label{definition:tame}
    %1
    \item {\bf (planar)} The hypermap is plain, planar.
    \item {\bf (biconnected)} The hypermap is connected and biconnected.  In particular, every face meets every node in at most one dart.
    \item {\bf (nondegenerate)} The edge map $e$ has no fixed points.
    \item {\bf (no loops)} The two darts of each edge lie in different nodes.
    \item {\bf (no double joins)} The set of edges meeting any two given nodes has cardinality at most $1$.
    \label{definition:tame:40}

    \item {\bf (triangles)} If $L$ is a contour loop with $3$ face steps, and if there exists a node in
    the exterior of $L$, then $L$ is a face of the hypermap.
    \label{definition:tame:3-circuit}

    \item {\bf (quadrilaterals)} If $L$ is a contour loop  $4$ face steps, and there are at least two nodes
    in the exterior of $L$, then the interior of $L$ takes one of the forms
    illustrated in Figure
    \ref{fig:fourcircuit}.\FIXX{Make this more precise.}
    \label{definition:tame:4-circuit}
    \begin{figure}[htb]
        \centering
        \myincludegraphics{\ps/tame4circuit.eps}
        \caption{Tame $4$-circuits}
        \label{fig:fourcircuit}
    \end{figure}
  \item {\bf (face)} There are at least three faces.
    \item {\bf (face)} The cardinality of each face is at least $3$ and at most $6$.
    \label{definition:tame:length}
    \item {\bf (node)} There are either $13$ or $14$ nodes.
    \item {\bf (node)} The cardinality of every node is at least $3$ and at most
    $7$.
    \label{definition:tame:degree}
    \item {\bf (node)} If a node is incident with an exceptional face,
        then the cardinality of the node is at most $6$.
    \label{definition:tame:degreeE}
    \item {\bf (weights)} There exists an admissible weight assignment
        of total weight less than the target, $\op{tgt}=1.541$.
    \label{definition:tame:squander}
\end{enumerate}
%

\section{Classification}
    \label{sec:proof-classification}

%\section{Statement of the Theorem}
\label{sec:classification}

A list of several thousand hypermaps appears at \cite{website:Hales:1998:Code}. The
following theorem is listed as one of the central claims in the
proof in Section~\ref{sec:logic}.

\begin{definition}[opposite] The opposite of a hypermap $(D,e,n,f)$ is the
hypermap $(D,f n,n^{-1},f^{-1})$.
\end{definition}

\begin{lemma}\guid{PPHEUFG} If a hypermap has properties\FIXX{Give reference}, 
then so does its opposite.
\end{lemma}

\begin{theorem}
\label{theorem:classification} Every tame hypermap is isomorphic to
a hypermap in this list, or is isomorphic to the opposite of a
hypermap in this list.
\end{theorem}

The results of this section are not needed except in the proof of
Theorem \ref{theorem:classification}.

\smallskip

Computers are used to generate a list of all hypermaps and to check
them against the archive of tame hypermaps.  The computer program is
based on the face-insertion construction of Lemma~\ref{XX}.  There it is
proved that all sufficiently nice hypermaps can be generated by an
elementary face-insertion process.  Tame hypermaps satisfy all the
hypotheses of that lemma.

\section{Contravening hypermap}

Our aim is to prove conjecture~(XX).  For that purpose, we assume that
we have a counterexample to inequality~(XX).  The purpose of this section is to select a special class of counterexamples.

Associated with a centered packing $(\Lambda,\vartheta)$ is a contact fan and a standard fan.  For both fans, set
$$
V = \{v\in \Lambda\mid \|v-\vartheta\|\le 2\hm\}.
$$
Let
\begin{equation}\label{eqn:fan-edge}
\begin{array}{lll}
 E_{std} &= \{\{v,w\}\in V^2\mid |v-w|\le 2\hm \}\\
 E_{cnt} &= \{\{v,w\}\in V^2\mid |v-w|= 2 \}\subset E_{std}.
\end{array}
\end{equation}

\begin{lemma}
$(\vartheta,V,E)$ is a fan, if $E=E_{std}$ or $E_{cnt}$.
\end{lemma}
They are called the standard fan and contact fan.
Let $\op{hyp}(\vartheta,V,E)$ be the associated hypermap.

\begin{definition}
Let $(\vartheta,V,E)$ be a fan.
We say that $v\in V$ is isolated in the fan if $E(v)$ is empty.
(That is, the degree of the corresponding node in the hypermap is $0$.) We say that $v\in V$ is surrounded in the fan if the azimuth angles of the darts in the fan at $v$ are all $<\pi$.  (In particular, the cardinality of $E(v)$ is at least three.)
\end{definition}

The following lemma appear in Sch\"utte and van der Waerden~\cite{vanderWaerden:1951}.

\begin{lemma}
Given any $V\subset \Lambda(\vartheta,2\hm)$,
there exists a finite packing $V'$ 
in bijective correspondence with $V$:
$$
f:V\to V'
$$
such that $\|v-\vartheta\| = \|f(v)-\vartheta\|$ and
such that every $v'\in V'$
is isolated or surrounded in the contact fan of $V'$.
\end{lemma}

\begin{proof} Consider all finite packings in 
bijective correspondence with $V$, such that the
bijection preserves distances to $\vartheta$.
Among these packings pick one $V'$ with the largest number
of isolated points in the contact graph.  If there is a point $v\in V'$ that
is not isolated and not surrounded, then it has a contact
dart with angle at least $\pi$.  We can then perturb $v$ away from the contacts, making it isolated, while preserving its distance to $\vartheta$.  The perturbed packing $V''$ has one more isolated point than $V'$, contrary to the supposed maximality of $V'$.  Hence, the conclusion of the 
lemma holds for $V'$.
\end{proof}

The following lemma shows that the corresponding result
holds for the standard fan.

\begin{lemma}\label{lemma:surrounded}  
Given any $V\subset \Lambda(\vartheta,2\hm)$,
there exists a finite packing $V'$ 
in bijective correspondence with $V$:
$$
f:V\to V'
$$
such that $\|v-\vartheta\| = \|f(v)-\vartheta\|$ and
such that every $v'\in V'$
is isolated or surrounded in the standard fan of $V'$.
\end{lemma}

\begin{proof}  
Let $V'_{iso},V'_{rid}$ be the
sets of isolated and surrounded vertices of $V'$ in the contact
fan.
By the previous lemma, we may assume
without generality that every $v\in V$ is isolated
or surrounded in the contact fan of $V$.    
We consider the set $\CalV$ finite
packings $V'$ in bijective correspondence with $V$:
$$
f:V\to V'
$$
preserving distances to $\vartheta$, with $V'_{rid}=V_{rig}$.
For any such packing, define constants $d_i=d_i(V')$,
for $i=1,\ldots,n=\op{card}(V_{iso})$ by letting 
$d(f(v))$, for $v\in V_{iso}$, 
be the minimum distance of $f(v)$ to a point
in $V'\setminus \{f(v)\}$, and then ordering the real numbers $d(f(v))$ in increasing order:
$$
d_0 \le d_1 \le d_2 \le \cdots \le d_n.
$$
The constants $d_i:\CalV\to\ring{R}$ are continuous functions on the compact configuration space $\CalV$.
We have $V\in \CalV$, so $\CalV\ne\emptyset$.
There is a nonempty 
compact subset $\CalV_0$ of $\CalV$ on which
$d_0$ attains its maximum value. Continuing recursively,
there is a nonempty compact subset $\CalV_{i+1}$ of
$\CalV_{i}$ on which $d_i$ attains its maximum value.

For the configuration $V\in \CalV$, we have $d_0 >2$.
Hence, $d_0(V')>2$ for $V'\in\CalV_0$.  It follows
that $V'_{iso}$ is in bijective correspondence with
$V_{iso}$.

We claim that any $V'\in \CalV_n$ has the desired property.
Suppose on the contrary, that some vertex $v'\in V'$
is neither isolated nor surrounded in the standard fan.  
Any such vertex satisfies $d(v')=d_i(V')$, for some $i$.
Among such vertices, pick the vertex with smallest index
$i$.  Write 
$$
d(v') = d_i(V') = d_{i+1}(V') =\cdots= d_j(V') < d_{j+1}(V'),
$$
for some $j\ge i$.  As $v'$ is not isolated in the
standard fan, we have $d_j(v') < 2.52$.  It is not surrounded,
so the cyclic order on
$$
\{(v',w')\in V'^2 \mid \|v'-w'\| = d_j(v'),
$$
has an azimuth angle that is at least $\pi$.
Thus, there is a direction in which $v'$ can be perturbed
that increases $d_j$, while keeping $d_0,\ldots,d_{j-1}$
fixed.  This is contrary to the defining property of
$\CalV_n\subset\CalV_j$.
\end{proof}


We may assume that we have an extremal counterexample
in the sense that it maximizes the left-hand side of inequality~(\ref{XX}).  We may assume that the set of vertices $V$ has cardinality $13$ or $14$ (Lemma~{XX}).
We may assume that every vertex is surrounded or isolated in
the standard fan (Lemma~\ref{lemma:surrounded}).  By Lemma~\ref{XX}, if there are any isolated verices in the standard fan, then the inequality~(\ref{XX}) holds.  Hence, we may in fact assume that every vertex is surrounded in the standard fan.
By extremality and construction, we may assume that every vertex $v$ that is not surrounded in the contact fan satisfies $\|v-\vartheta\|=2$.

Among all extremal counterexamples with these properties,
we may chose those that minimize the number of connected components
of the standard hypermap.
If the standard hypermap contains at least two components, let $W\subset V$ be the set of vertices of one of the components.  There is an axis $\ell$ such that the orthogonal transformations $T(\theta)$ around axis $\ell$ and small angle $\theta$  have the effect of rotating $W$ in the direction of the vertices of another component, until for some angle $\theta$, the condition~(\ref{eqn:fan-edge}) for
a new edge joining the components is met.  Thus, we may assume the standard hypermap has a single connected component.

In summary, we have constructed a counterexample with the following properties.

\begin{lemma} Assume that a counterexample exists to inequality~(XX).  Then there also exists a counterexample with the following properties.
\begin{itemize}
\item It maximizes the sum XX.
\item There are $13$ or $14$ vertices.
\item The standard hypermap is connected.
\item Every vertex is surrounded in the standard fan.
\item Every vertex $v$ that is not surrounded in the contact
fan satisfies $\|v-\vartheta\|=2$.
\end{itemize}
\end{lemma}







In the next section, we study further properties of contravening hypermaps $H$.





\section{Contravention is Tame}
    \label{sec:contraproof}

Let $(\Lambda,v_0)$ be a centered packing with
fan $(v_0,V,E)$.  Let  hypermap $H=(D,e,n,f)$
be the planar hypermap attached to $(v_0,V,E)$.
The hypermap $H$ is connected and biconnected.

The connected components of $Y(v_0,V,E)$ are in bijection with
faces of $H$.  
The fan gives a azimuth angle function
$$
\op{azim} : D \to (0,2\pi).
$$
For each face of $H$, the corresponding component $R$
is eventually radial with solid
angle
  $$
  2\pi + \sum_{x\in F} (\op{azim}(x) -\pi).
  $$
We write $\sol(F)$ for the solid angle of the connected component
of $Y(v_0,V,E)$ associated with a face $F$ of the hypermap.
We have
    $$\sum_{F} \sol(F) = 4\pi.$$


For each face, there is a
real number $\tau(F)=\tau(\Lambda,F)$ such that
$$
  \sum_{F : \text{face}}\tau(\Lambda,F)
$$
We define a weight function $w(F)$ on the faces of the hypermap
by $w(F) = \tau(F)$.  In this way, we attach
a pair $(H,w)$ to each contravening centered packing $(\Lambda,v_0)$.


\begin{theorem} \label{theorem:contravene}
Let $(\Lambda,v_0)$ be a contravening centered packing.  Let $(H,w)$ be
the hypermap and function on its faces attached to $(\Lambda,v_0)$ as above.
Then $H$ is a tame hypermap with admissible weight function $w$.
\end{theorem}




\subsection{general properties}
    \label{sec:startame}


Recall that we say that a node $v$ has {\it type\/} $(p,q,r)$ if
there are exactly $p+q+r$ faces that meet the node, of which exactly
$p$ triangles and $q$ quadrilateral faces (see
Definition~\ref{definition:type}).  We write $(p_v,q_v,r_v)$ for the
type of a node $v$.

\begin{lemma}\guid{JGTDEBU} The contravening hypermap $H$ satisfies Conditions~\ref{XX}-\ref{XX} 
of tameness.
Explicitly, it is a plain, planar, connected, and biconnected. 
The edge map $e$
has no fixed points. There are at least three faces. Every face meets
every node in at most one dart.  There are never two nodes of type
$(4,0,0)$ that are adjacent to each other.  Every face has
cardinality at least $3$ and at most $8$.  If $L$ is a contour loop
with $3$ face steps, and if there exists a node in the exterior of
$L$, then $L$ is a face of the hypermap.
\end{lemma}

\begin{proof}
It is connected and biconnected by properties of contravening hypermaps.
It is plain and planar by general properties that we have established
about the hypermaps associated with fans.
A fixed point for the edge map is a degenerate dart, which does not occur in a connected hypermap.
\end{proof}





\subsection{properties of nodes}


\begin{lemma}\guid{SZIPOAS}\dcg{Lemma~21.4}{223} 
Formally contravening hypermaps satisfy Property
\ref{definition:tame:degree} of tameness: The cardinality of every
node is at least $2$ and at most $7$.
\end{lemma}

\begin{proof}  There is no node of cardinality one by
Lemma~\ref{lemma:nondegen}.  There is no node of degree
greater than $7$ by Lemma~\ref{a:6}.
\end{proof}


\begin{lemma}\guid{GNCEGFN}\label{lemma:no-2}
Let $H$ be a contravening hypermap.
Suppose that $L$ is a contour loop with at most four face steps.
Suppose that there are at least two nodes in the exterior of $L$.
Then there at most one node interior to $L$.
\end{lemma}


\begin{lemma}\guid{CDTETAT} \label{lemma:0.852}
Let $H$ be a contravening
hypermap. For every dart $x$,
    $$0.852\le \azim(x)\le 1.9.$$
For every dart $x$ whose face is not a triangle, we have
    $$1.15\le\azim(x)\le 3.27.$$
Consequently, if a vertex $v$ has type $(p,q,0)$, then $(p,q)$
must be one of the following pairs:
$$
\begin{array}{lll}
&(0,2),~(0,3),~(0,4),~(0,5),~(1,2),~(1,3),~(1,4),~(2,1),~(2,2),~(2,3),\\
&(3,1),~(3,2),~(3,3),~(4,0),~(4,1),~(4,2),~(5,0),~(5,1),~(6,0),~(6,1),~(7,0)
\end{array}
$$
\end{lemma}
 %
 \indy{Index}{ZZZZ1.15@$1.15$}
 \indy{Index}{ZZZZ0.852@$0.852$}
\begin{proof}
The angle bounds are a calculation.  The sum of the azimuth angles
around a vertex satisfies:
$$
  p (0.852) + q (1.15) \le 2\pi \le p (1.9) + q (3.27),
$$
and the pairs satisfying these constraints are listed.
\end{proof}



\begin{lemma}\guid{RTOIQJC}\label{lemma:nobad4}
Let $H$ be a contravening hypermap.
There does not exist a node of 
type $(1,0,1)$ with precisely one triangle and
one pentagon, as show in Figure~\ref{fig:no4circuit}. 
\end{lemma}

\begin{proof}  Every node in
the hypermap has degree at least three, but such a node
has degree two.
\end{proof}

\begin{lemma}\guid{BDJYFFB} \label{lemma:deg5}
Every contravening hypermap satisfies Property
\ref{definition:tame:degreeE} of tameness: If a node meets an
exceptional face, then the cardinality of the node is at most $6$.
\end{lemma}

\subsection{triangles and quadrilaterals}


\label{sec:2.2}  To continue with the proof that
contravening hypermaps are tame, we need to introduce some more
notation and methods.



Every contour loop partitions the faces into the interior and
exterior.  Every contour loop partitions the nodes that do not meet
the loop into exterior and interior nodes.
%
 \indy{Index}{interior node}

Lemma~\ref{lemma:no-2} asserts that either the interior or the
exterior has at most $1$ enclosed vertex.   When choosing which
aggregate is to be called the interior, we may make our choice so
that the interior has area at most $2\pi$, and hence contains at
most $1$ node. With this choice, we have the following lemma.

\begin{lemma}\guid{CHOMXMX}
Let $H$ be a contravening hypermap. If
$L$ is a contour loop with $4$ face steps, and there are at least
two nodes in the exterior of $L$, then the interior of $L$ takes one
of the forms illustrated in Figure~\ref{XX} in Property
    \ref{definition:tame:4-circuit} of tameness.
\end{lemma}

\begin{proof}
By Lemma~\ref{XX}, the interior of $L$ contains at most one node.

$H$ is a connected plane planar map.  We form a normal family of
contour loops ${\cal L}$ by taking the contour loop $L^{-1}$
reversing $L$\FIXX{Explain} and all the faces in the interior of $L$.
(Check this is a normal family.)  The quotient $H' = H/{\cal L}$ is
a plane planar map.  There is a further quotient of $H'$ with normal
family $\{L,L^{-1}\}$, which is isomorphic to $P_4$ with the natural
flag coming from $H'$.  The niceness conditions of Lemma~\ref{XX} are
satisfied, so we can recover $H'$ from $P_4$ by a sequence of
face-insertions.  Since the interior of $L$ contains at most one
node, this gives restrictions on the partitions that can be used in
face-insertion.

If there are no enclosed vertices, then the only possibilities are
for it to be a single quadrilateral face or a pair of adjacent
triangles.

Assume there is one enclosed vertex $v$.  If $v$ is connected to $3$
or $4$ nodes of the quadrilateral, then that possibility is listed
as part of the conclusion.

If $v$ is connected to $2$ nodes in the $4$-cycle, then the
node $v$ has type $(0,2,0)$, which has degree $2$, contrary to the lower bound of $3$ on degrees.

If $v$ is connected to $0$ or $1$ nodes, this is contrary to the proposition that the hypermap is connected and biconnected.  This completes the proof.
\end{proof}


\subsection{faces}

\begin{lemma}  Every face has has size at least $3$ and at most $6$.
\end{lemma}

\begin{proof} It is enough to show that there does not exist a face (of the hypermap associated with the standard fan)
with the following two properties:
\begin{enumerate}
\item Every dart has angle less than $\pi$.
\item The sum of the heights $(|v|-2)$, as $v$ runs
over the vertices of the face, is at most $2(0.52)$.
\end{enumerate}
Indeed, we have shown that the first property holds
of a contravening hypermap.  If the second property fails,
then since $L_2$ is the linear interpolation between
the points $(1,1)$ and $(1.26,0)$, and there are at most
$14$ points in $V$, we have
$$\sum_{v\in V}L_2(|v|/2) \le 12 L_2(1) + 2 L_2(1.26) =12$$
and the main inequality holds.

By \cite{vanderWaerden:1951}, we have that the sum of
the arclengths of the edges of the cycle is less than
$2\pi$.  Their argument briefly goes as follows. Relaxing the constraint so that the angles are $\le \pi$, we deform the polygon so that it has the maximum number of angles flattened to $\pi$.  As long as there are more than three angles $<\i$, the polygon can be further flattened, so it is not maximal.  There results a triangle.  The three arclengths of a spherical triangle sum to less than $2\pi$.  (The extreme case occurs when all three points lie on a great circle and all angles equal $\pi$.)

The edge $\{v,w\}$ has arclength at least
$$
\arc(|v|,|w|,|v-w|) \ge \arc(|v|,|w|,2). 
$$

A calculation\footnote{XX} gives
$$
\arc(|v|,|w|,2)\ge 1 - 0.6076 (|w|/2 - 1) - 0.6076 (|v|/2 - 1).
$$
Summing over a face of size at least $7$, we obtain:
$$
\sum \arc(|v_i|,|v_{i+1}|,|v_i-v_{i+1}|)\ge
7 - 0.6076 \sum (|v_i|-2) \ge 7 - 0.6076 (2) (0.52) > 2\pi,
$$
which is contrary to the upper bound on the edge length
sum.
\end{proof}







\subsection{weight assignment}
    \label{sec:weight}

The purpose of this section is to prove the existence of a good
admissible weight assignment for contravening hypermaps.
This will complete the proof that all contravening
hypermaps are tame.

\begin{theorem}  Every contravening hypermap has an admissible
weight assignment of total weight less than $\op{tgt}=1.541$.
\end{theorem}

Given a contravening hypermap $H$, we
define a weight assignment $w$ by
    $$F \mapsto w(F) = \tau(F).$$
The challenge of the theorem will be to prove that $w$, when
defined by this formula, is admissible.

\subsection{admissibility}
\label{sec:admissibility}

The next three lemmas establish that this definition of $w(F)$ for contravening hypermaps satisfies the first three defining
properties of an admissible weight assignment.

\begin{lemma}\guid{OLNSWLK}  Let $F$ be a face of cardinality $n$ in a contravening hypermap.
Define $w(F)$ as above. Then
        $w(F) \ge d(n)$.
\end{lemma}

\begin{proof} This is Lemma~\ref{XX}. %\ref{proposition:wttau}.
\end{proof}

\begin{lemma}\guid{RVFTYCI} Let $v$ be a node of type $(p,q,0)$ in a contravening hypermap.  Define $w(F)$ as above. Then
        $$\sum_{v\in F} w(F) \ge b{\pqr{(p,q,0)}}.$$
\end{lemma}


\begin{proof} This is Lemma~\ref{lemma:pq}.
\end{proof}

\begin{lemma}\guid{FUJCTEI}\label{lemma:enclosed:bis} % {Lemma 2.2}
A quadrilateral component does not enclose any vertices of height at
most $2t_0$.
\end{lemma}

The proof that contravening hypermaps are tame is complete.


\section{Main Estimate}


\begin{lemma}  The function $\tau(F)$ on a face of
size $n$ satisfies 
$$
\tau(F) \ge d(n),
$$
where $d(n)$ is the constant in XX.
\end{lemma}

In the original 1998 proof, the corresponding result
is called the ``Main Theorem.''  The proof of that 
theorem takes about 30 pages and relies on many
long computer calculations.  The proof given here
is substantially simpler than the proof of the
Main Theorem, but
it is still nontrivial.  Here, we have the advantages
of knowing that the polygons are convex, the hypermap
is biconnected, and each face has at most six sides.

In the proof, we will adopt the following convention for
configurations.  We let $v_0,\ldots,v_{n-1}$ be the vertices
of the polygon, and write
$$
y_i = \|v_i-\vartheta\|,\quad y_{ij} = \|v_i- v_j\|.
$$

It is helpful to keep in mind the origin of the constants $d(n)$.
Although, we do not produce sharp lower bounds on $\tau(F)$, the
statement of the lemma is motivated by the following configuration.
Consider a polygon with $n$ sides all of length $y_{i,i+1}=2$, heights
$y_i=2$, and $n-3$ diagonals of length $2.52$: $y_{0,j}=2.52$, for
$j=2,\ldots,n-2$.  Evaluating $\tau$ on these configurations gives
$$
\tau(F) = \begin{cases}
0.20612\ldots & n=4\\
0.48356\ldots & n=5\\
0.760993\ldots &n=6
\end{cases}
$$
These constants $d(n)$ were chosen to be slightly smaller than these values.


\begin{proof}  We consider the cycle in the fan
giving the face $F$.  For the purposes of this
estimate, we may discard all the vertices of $V$
that do not belong to the cycle as well as all edges
of $E_{std}$ that lie outside the cycle.
Write $\tau(V)$ for the value of $\tau(F)$ as a
function of the positions of the vertices in $V$,
keeping the set of edges of the cycle fixed.  The
azimuth angles associated with the face $F$ will be
call the interior angles.  

The idea of the proof is simple.  We may deform $V$ in a way to decrease $\tau(V)$.  We pick the deformations to lie in subspaces of small dimension to allow us to verify directly that the deformations are nonincreasing in $\tau(V)$.  As the deformations continue, the configuration $V$ moves into a subspace of smaller and smaller dimension.  Eventually, the dimension becomes sufficiently small that we can make a direct interval arithmetic calculation to see that $\tau(V)$ satisfies the desired bounds.

We break the proof into several parts in the following
subsections
\end{proof}


\subsection{halting conditions}

The deformation is required to maintain the following
constraints:
\begin{itemize}
\item $V\cup\{\theta\}$ is a packing.
\item The interior angles are at most $\pi$.
\item The distances satisfy $\|v-w\|\ge 2.52$, if $\{v,w\}$ is
not an edge.  (Call these the diagonals.)
\end{itemize}
If any of the constraints become binding, we freeze that
constraint, and continue the deformation along the remaining degrees of freedom of the configuration.  The deformations are described in detail below.

\subsection{recursion}

Initially, the distances $\|v-w\|>2.52$ for diagonals.
If after deformation, equality holds; we stop the deformation, and cut the cycle into two smaller along the diagonal
and continue recursively with deformations for each smaller cycle.  A cut edge has length $\|v-w\|=2.52$, and
we require that all further deformations must keep this distance fixed.  

On a smaller cycle, let $r$ be the number of original edges and $s$ be the number of edges produces by cuts along a diagonal fixed at $2.52$.  (Call this second kind of edge a {\it cut} edge.  Let $\tau(V(r,s))$ be the functon $\tau$ on a configuration $V$ with parameters $r$ and $s$.  Starting from polygon with at most six sides, the values $(r,s)$ that might be obtained are
$$
(3,0),~(2,1),~(1,2),~(0,3),~
(4,0),~(3,1),~(2,2),~
(5,0),~(4,1),~
(6,0)
$$
That is, $0\le s\le 3$ and $3-s\le r\le 6-2s$.
The recursive bound we prove is
\begin{equation}\label{eqn:drs}
\tau(V(r,s)) \ge d(r,s) = 0.103 (2-s) + 0.277 (r+2s-4) 
\end{equation}
Note that $d(n,0) = d(n)$. Also, note that if we cut
$V(r,s)$ along a new diagonal to produce $V(r_1,s_1)$
and $V(r_2,s_2)$, we have $r_1+r_2=r$ and $s_1+s_2 = 2+s$.
Also,
$$
\begin{array}{lll}
d(r,s) &= d(r_1+r_2,s_1+s_2-2) \\
  &=0.103 (4-s_1-s_2) + 0.277 (r_1+r_2+2s_1+2s_2-8) \\
  &=d(r_1,s_1) + d(r_2,s_2).
\end{array}
$$
So we see that the definition of $d(r,s)$ has been
chosen so that the recursive 
bound~(\ref{eqn:drs}) implies the
lemma.

In the proof that follows, we may assume for a contradiction that the
inequality~(\ref{eqn:drs}) fails, and that we have chosen a minimal counterexample in the sense that $r+s$ as small as possible.   Minimality allows us to assume that no diagonals develop as we deform, because cutting along the diagonal produces pieces on which the inequality holds.

\subsection{deformations}

We have the following $\tau$-nonincreasing deformations:
\begin{itemize}
\item {\bf (Vertex push)} Push one vertex radially toward $\theta$.  By the formula for $\tau$, this deformation decreases $\tau$.
\item {\bf (Lexell)} Fix all the heights $\|v-\vartheta\|$. Then $\tau$ depends only on the area of the convex spherical polygon.  Consider an {\it ear} of the polygon (the triangle formed by two adjacent edges and a diagonal).  By Lexell's theorem, as we increase the length of one of the edges of the polygon, the area of the ear has a unique local maximum and no local minimum.  Thus, we may always deform until each edge is as long or as short as possible.
\end{itemize}

\subsection{flat vertices}

If the interior angle has increased to $\pi$, call the vertex {\it flat}. 
If there are $k$ consecutive cut edges, there are  $k+1$ corresponding edges that form a linear series, all lying in a common plane through the origin $\vartheta$.  When the vertex is flat, the Lexell triangle deformation must preserve the flat vertex.  Thus, we deform along the linear series as a whole, until each is as long or as short as possible.  

The next few lemmas use the triangle inequality to constrain configurations with flat vertices.

\begin{lemma}
There cannot be three consecutive flat vertices.
\end{lemma}

\begin{proof} This is because of the triangle inequality.  Three flat vertices produces a linear series of length at least
$$
4\arc(2.52,2.52,2) > 3,
$$
but the remaining two edges (on a hexagon) have combined length at most
$$
2\arc(2,2,2.52) < 3.
$$
\end{proof}

\begin{lemma}
If there are two consecutive flat vertices, there is no cut edge among the corresponding set of three edges.
\end{lemma}

\begin{proof}  From the constraints on $r$ and $s$ given above, if $s>0$, then $r+s\le 5$.  Thus, $r+s=5$, forming a triangle with a linear series of three, and then the two other edges.  If $y=\|v-\vartheta\|$ where $v$ is a vertex of the triangle formed by an end of the linear series, the linear series has length at least
$$
\arc(y,2.52,2)+\arc(2.52,2.52,2) +\arc(2.52,2.52,2.52)
$$
which is greater than the maximum sum of the other two lengths:
$$
\arc(y,2,2.52)+\arc(2,2,2.52).
$$
This violates the triangle inequality.
\end{proof}

\begin{lemma}  If there are two consecutive flat vertices, then (following the Lexell triangle deformations) the edges are as short as possible
\end{lemma}

\begin{proof} The Lexell argument either increases edges as much as possible or compresses them as much as possible.  If the vertices $v_1,v_2,v_3,v_4$ of the linear series are stretched, it has length at least
\begin{equation}\label{eqn:3side}
\arc(y_1,2,2.52)+\arc(2,2,2.52)+\arc(2,y_4,2.52),
\end{equation}
where $\|v_i-\vartheta\|=y_i$.
However, there are at most two other vertices and three other edges.  By the triangle inequality, the sume of these three lengths is at most (\ref{eqn:3side}).
Thus, equality is obtained in the triangle inequality, and the polygon reduces to a linear segment.  This forces the two other vertices to be precisely equal to $v_2$ and $v_3$, which is contrary to the assumption that we have a packing with distance separations at least $2$.
\end{proof}

\begin{lemma} If there are two consecutive flat vertices, then without loss of generality, we may assume that one of them has minimal or maximal length:
$$\|v-\vartheta\|\in \{2,2.52\}.$$
\end{lemma}

\begin{proof}
Assume there are two adjacent flat angles $v_2,v_3$, forming a linear series $v_1,v_2,v_3,v_4$.
Assume without loss of generality (by previous reductions) that
$|v_i-v_{i+1}|=2$, for $i=1,2,3$.
Let $y_i = \|v_i-\vartheta\|$.
We may pull one vertex $v_2$ away from $\vartheta$ and push the other $v_3$ in such a way that fixes $y_2+y_3$, while decreasing the arclength of the linear series.  This follows by concavity.
In fact, the arclength is given by a sum of three terms:
  $$
  \sum_{i=1}^3\arc(y_i,y_{i+1},2).
  $$
This follows from a calculation showing that the second derivative of $\arc(t,s,2)$ with respect to $t$ is negative, for $t,s\in[2,2.52]$.  Thus, the
sum is also concave, so that we shorten arclength as much as possible when the heights are extremal.
\end{proof}

In summary, we have the follow configurations of flat angles:
\begin{itemize}
\item There are two consecutive flat angles contracted as much as possible.  They form a linear series $v_1,v_2,v_3,v_4$ where
$$
\|v_2-\vartheta\|\in\{2,2.52\},\quad
\|v_i-v_{i+1}\|=2,\quad i=1,2,3.
$$
\item There is a single flat angle contracted as much as possible.  There
is a linear series $v_1,v_2,v_3$ where
$$
\|v_i-v_{i+1}\|=2,\quad i=1,2.
$$
\item There is a single flat angle stretched as much as possible.  There
is a linear series $v_1,v_2,v_3$ where
$$
\|v_2-\vartheta\|=2,\quad
\|v_i-v_{i+1}\|=2.52,\quad i=1,2.
$$
\end{itemize}

More than one of these can be combined in a single polygon.  We represent the sizes of the linear series in a polygon with $n$ sides as a partition of $n$.
The number of parts of the partition gives the number of sides of the polygon when each linear series is considered a single side.  The partition must contain at least three parts.
Possibilities containing at least one flat vertex are
$$
(2,1,1),~(3,1,1),~(2,2,1),~(2,1,1,1),~(3,2,1),~(3,1,1,1),~(2,2,2),~(2,2,1,1),~(2,1,1,1,1).
$$
There are two distinct configurations with partition $(2,2,1,1)$: either
alternating $(2,1,2,1)$ or adjacent $(2,2,1,1)$.

\subsection{Adjusting heights}

Fix three consecutive vertices $u,v,w$ of the cycle.
Assume $\|v-u\|$ and $\|v-w\|$ are already at their extremal value (either $2$ for  $2.52$).  Then the function $\tau(V)$ may be considered as a function
of the edges of the simplex $\vartheta,u,v,w$ as $v$ moves with the other points of $V$ fixed.  Fix $5$ edges of the simplex as parameters and vary $\|\theta-v\|$, so that $\tau$ becomes a function of a single variable.

\begin{lemma} As a function of $\|v-\vartheta\|$
 $\tau$ has negative second derivative whenever the derivative is zero.  Thus, $\tau$ has no local minimum.
\end{lemma}

Consequently, we may deform by either increasing or decreasing $\|v-\theta\|$ as much as possible.  The proof is an interval arithmetic calculation over a four-dimensional space.\FIXX{do}


We can do the same when there is one flat vertex forming a linear series of length $2$, with compressed edges.

\begin{lemma}
As a function of $\|v-\vartheta\|$
 $\tau$ has negative second derivative whenever the derivative is zero.  Thus, $\tau$ has no local minimum.
\end{lemma}

Again, this is an interval calculation.\FIXX{do}

Thus, if there a sequence of a flat vertex $u$, an acute vertex $v$, and another acute vertex, then we may assume that $u$ or $v$ has extremal height $\in\{2,2.52\}$.



\subsection{Adjusting quadrilaterals}

Suppose that there are two consecutive vertices $v_2,v_3$ that are not flat,
and consider the chain of four consecutive vertices $v_1,v_2,v_3,v_4$.
By Lexell arguments, each edge $\|v_i-v_{i+1}\|$ is $2$ or $2.52$.
Set $y_i = \|v_i-\vartheta\|$ and $y_{ij} = \|v_i-v_j\|$.
After adjusting heights $y_i$ is $2$ or $2.52$, for $i=2,3$.
This leaves four degrees of freedom:
$$
y_1,y_4,y_{14},
$$
and a diagonal to the quadrilateral, say $y_{13}$.
We assume that the $y_{14}$ is at least $2.52$.  (If is is smaller, it is an uncut edge of the polygonal, of length $2$, so that the polygon is actually a convex quadrilateral.  This special case is most easily dealt with separately.)

\begin{lemma}
In this context, the function $\tau$ as a function of $y_1,y_4,y_{14},y_{13}$ does not have an interior point local minimum.
\end{lemma}

\begin{proof} This is an interval arithmetic calculation in four variables.\FIXX{do}
\end{proof}

As a consequence, the minimum occurs when a new flat vertex forms.  We may then eliminate any case with two consecutive vertices that are not flat (after treating the quadrilateral with four edges $y_{ij}=2$ as a special case).

\subsection{Cases}

We review the combinatorial possibilities.  In each case, we may check that the inequality holds, to complete the proof of the main estimate.  We label each 
case by the partition.  If the partition is $(\mu_1,\mu_2,\ldots)$, we
index the vertices in the same order as the partition, with $v_0$
the vertex occurring just before the first flat vertex.  For example,
if the partition is $(3,1,1)$, the polygon is a pentagon, flattened into
an effective triangle, with vertices $v_0,\ldots,v_4$, where $v_1$ and $v_2$
are the flat vertices.

\begin{itemize}
\item {\bf (1,1,1,1)}  This is a convex quadrilateral with four sides $y_{ij}=2$ and four extremal heights $y_i$.  There is one degree of freedom given by the diagonal.
\item {\bf (3,1,1)}  This is a pentagon, flattened into an effective triangle.  There are three degrees of freedom $y_0,y_{03},y_3$ along the flattened side, and no freedom in the rest of the figure.  We have $y_{01}=y_{12}=y_{23}=2$.
\item {\bf (2,2,1)} This is a pentagon, flattened into an effective triangle.  The edge lengths $y_{i,i+1}$ are all extremal.  Because of height adjustments, there are three degrees of freedom in the heights $y_i$.
\item {\bf (3,2,1)} This is a hexagon, flattened into an effective triangle.  There are four degrees of freedom.
\item {\bf (2,2,2)}  This is a hexagon, flattened into an effective triangle.  There are six degrees of freedom, given by all heights $y_i$.
\item {\bf (2,2,1,1)} This is a hexagon, flattened into an effective quadrilateral.  There are five degrees of freedom: a diagonal $y_{03}$ to the quadrilateral and four independent heights.
\item {\bf (2,2,1,1)}  This is a hexagon, flattened into an effective quadrilateral.  There are four degrees of freedom: a diagonal $y_{03}$ to the quadrilateral and three independent heights.
\end{itemize}


Interval arithmetic calculations for each of these cases completes the proof.


