% Flyspeck
% Thomas C. Hales
% Sep 3, Removed 2 enclosed in quad properties.


\chapter{Tame Hypermap}
\indy{Index}{hypermap!tame}%

\label{sec:tame}
\indy{Index}{tame}%

\begin{summary}
  This chapter is the second of the two core chapters that are devoted
  to the proof of the Kepler Conjecture.  If $V$ is a finite set of
  vectors in $\ring{R}^3$, let
  \begin{displaymath}\CalL(V) = \sum_{\v\in V}
    L(\normo{\v}/2).\end{displaymath}
Let $\BB$ be the
annulus $\bar B(\orz,2h_0)\setminus B(\orz,2)$, where
$\bar B(\orz,r)$ is the closed ball of radius $r$.
By Corollary~\ref{cor:CE}, if every packing $V$
contained in $\BB$
satisfies
\begin{equation}\label{eqn:CCE}
\CalL(V) \le 12,
\end{equation}
then the Kepler conjecture follows.

This chapter assumes that there exists a counterexample $V$ to this
inequality and reaches a contradiction.  A subset of extremal
counterexamples will be selected that are particularly well-suited for
further analysis.  Every extremal counterexample gives rise to a fan
and a corresponding hypermap.  A detailed study of these hypermaps
leads to a long list of properties that all such hypermaps must
possess.  A \newterm{tame} hypermap is defined to be precisely the set
of hypermaps that have all of these properties.  Tameness is thus an
umbrella term that covers a long list of loosely related properties.

An earlier chapter on hypermaps gives an algorithm that generates all
restricted hypermaps (with a given bound on the number of nodes).
Every tame hypermap is restricted and has at most $14$ nodes.  Hence a
list of all tame hypermaps can be obtained by generating all
restricted hypermaps and filtering out those that are not tame.  This
algorithm has been implemented in computer code and run.  The result
is that there is an explicit complete list of all tame hypermaps (up
to isomorphism).  This list gives the classification of tame hypermaps
up to isomorphism.  This classification solves a major step of the
packing problem.

Each tame hypermap $H$ gives rise to a nonlinear optimization problem.
Maximize $\CalL(V)$ subject to the constraint that the hypermap
associated with $V$ is isomorphic to $H$.  This nonlinear optimization
problem has a linear relaxation.  This linear program has a maximum
that is at least as large as the maximum of the nonlinear program.
The linear programs have been solved by computer.  In every case, the
maximum is less than $12$.  This means that inequality~\ref{eqn:CCE}
always holds, so that the Kepler Conjecture is confirmed.
\end{summary}
\indy{Index}{hypermap!tame}%
\indy{Index}{tame!hypermap}%



\section{Definition}


\begin{definition}[triangle,~quadrilateral]
  Faces of cardinality $3$ in a hypermap are called
  \newterm{triangles}, those of cardinality $4$ are called
  \newterm{quadrilaterals}, and so forth.
%
\indy{Index}{triangle}%
\indy{Index}{quadrilateral}%
\indy{Notation}{pZ@$p_\v$}%
\end{definition}

\begin{definition}[type,~$(p,q,r)$]\label{definition:type}
The \newterm{type\/} of a node is defined to be a triple of
non-negative integers $(p,q,r)$, where $p$ is the number of
triangles meeting the node, $q$ is the number of quadrilaterals
meeting it, and $r$ is the number of other faces meeting it, so that
$p+q+r$ is the total number of faces meeting the node.
%
\indy{Index}{node!type}%
\indy{Notation}{pqr@$(p,q,r)$}%
\end{definition}


\subsection{weight assignment}\label{sec:wtassign}
\indy{Index}{weight assignment}%

Call the constant $\op{tgt}=1.541$, which arises repeatedly in
this chapter, the \newterm{target}. 
%
\indy{Index}{target}%
\indy{Notation}{tgt@$\op{tgt}=1.541$}%


\begin{definition}[b]
Define $b:\ring{N}\times \ring{N}\to \ring{R}$ by $b(p,q)=\op{tgt}$,
except for the values in the following table:
{
\def\tx{\op{tgt}}
\begin{displaymath}\begin{matrix}  &q=0&1&2&3&4\\
p=0&\tx&\tx&\tx&0.618&1.0\\
1&\tx&\tx&0.66&0.618&\tx\\
2&\tx&0.8&0.412&1.2851&\tx\\
3&\tx&0.315&0.83&\tx&\tx\\
4&0.35&0.374&\tx&\tx&\tx\\
5&0.04&1.144&\tx&\tx&\tx\\
6&0.689&\tx&\tx&\tx&\tx\\
7&1.450&\tx&\tx&\tx&\tx
\end{matrix}
\end{displaymath}
}
\indy{Notation}{b@$b$ (tame)}%
\end{definition}


\begin{definition}[d]
Define $d:\ring{N}\to \ring{R}$ by
\begin{displaymath}d(k) = \begin{cases}
0 & k\le3, \\
0.206 & k=4, \\
0.4819 & k=5, \\
0.7578 & k=6, \\
%1.038 & k=7, \\
%1.315 & k=8,\\
\op{tgt}=1.541 & \text{otherwise}.
\end{cases}
\end{displaymath}
\indy{Notation}{d@$d$ (tame)}%
\end{definition}




\begin{definition}[weight~assignment]
%
  A \newterm{weight assignment\/} of a hypermap $H$ is a real-valued
  function $\tau$ on the set of faces of $H$. A weight assignment
  $\tau$ is \newterm{admissible} if the following properties hold:
%
\indy{Index}{weight assignment}%
\indy{Index}{weight assignment!admissible}%
\indy{Index}{cardinality}%
\indy{Index}{node}%
\indy{Notation}{ZZtau@$\tau$ (weight assignment)}%
\begin{nomerate}
\item \case{bound~a} Let $v$ be any node of type $(5,0,1)$, and let
  $A$ be the set of triangles meeting that node.  Then
\begin{displaymath}\sum_{F\in A} \tau(F)
\ge  0.63.\end{displaymath}
%   where $a=0.63$.
%        \label{definition:admissible:excess}
\item \case{bound~b} If a node $v$ has type $(p,q,0)$, then
  \begin{displaymath}\sum_{F:\,v\cap F\ne\emptyset} \tau(F) \ge
    b(p,q).\end{displaymath}
%        \label{admissible:b}
\item \case{bound~d} If the face $F$ has cardinality $k$, then
$\tau(F) \ge d(k)$.
\end{nomerate}
\indy{Notation}{v@$v$ (node)}%
\indy{Notation}{A@$A$ (set of triangles)}%

The sum $\sum_F \tau(F)$ (over all faces) is called the \newterm{total
  weight}. % of $\tau$.
\indy{Index}{weight!total}%
%\indy{Notation}{total@$\sum\tau$}%
\end{definition}





\subsection{hypermap property}
\label{sec:graphproperty}

A hypermap is \newterm{tame\/} if it satisfies the following conditions.
%
\indy{Index}{tame}%

\begin{itemize}
\label{definition:tame}
%1
\item \case{planar}  The hypermap is plain and planar.
\item \case{simple} The hypermap is connected and simple.  In
  particular, every intersection of a face with a node contains at
  most one dart.
\item \case{nondegenerate} The edge map $e$ has no fixed points.
\item \case{no loops} The two darts of each edge lie in different nodes.
\item \case{no double joins} At most one edge meets any two given nodes.
%    \label{definition:tame:40}
%    \item \case{blank}
\indy{Notation}{edgemapz@$e$ (edge map)}%
% \item \case{triangles} If $L$ is a contour loop with three face
%   steps, and if there exists a node in the exterior of $L$, then $L$
%   is a face of the hypermap.
%    \label{definition:tame:3-circuit}
%\item \case{blank}
% \item \case{quadrilaterals} If $L$ is a contour loop with four face
%   steps, and there are at least two nodes in the exterior of $L$,
%   then the interior of $L$ takes one of the forms illustrated in
%   Figure \ref{fig:fourcircuit}.
%   \label{definition:tame:4-circuit}
%  \begin{figure}[htb]
%      \centering
%      \myincludegraphics{\pdfp/fourcircuit.eps}
%      \caption{Tame $4$-circuits}
%      \label{fig:fourcircuit}
%  \end{figure}
\item \case{face count} The hypermap has at least three faces.
\item \case{face size} The cardinality of each face is at least $3$
  and at most $6$.
\item \case{node count} There are either $13$ or $14$ nodes.
\item \case{node size} The cardinality of every node is at least $3$
  and at most $7$.
%    \label{definition:tame:degree}
\item \case{node types} If a node has cardinality $7$, then the type
  of the node is $(p,q,0)$ for some $p,q\ge0$.  If the cardinality of
  the node is exactly $6$, then the type of the node is $(6,0,0)$,
  $(5,1,0)$, or $(5,0,1)$.
%    \label{definition:tame:degreeE}
\item \case{weights} There exists an admissible weight assignment
of total weight less than the target, $\op{tgt}=1.541$.
\end{itemize}
%
\indy{Index}{planar}%
\indy{Index}{simple}%
\indy{Index}{nondegenerate}%
\indy{Index}{no loops}%
\indy{Index}{no double joins}%
\indy{Index}{face}%
\indy{Index}{weight}%

\section{Classification}
\label{sec:proof-classification}

%\section{Statement of the Theorem}
\label{sec:classification}




\begin{definition}[opposite] The opposite of a hypermap $(D,e,n,f)$ is the
hypermap $(D,f n,n^{-1},f^{-1})$.
\indy{Index}{hypermap!opposite}%
\end{definition}

\begin{lemma}\guid{PPHEUFG}
\oldrating{300}
\rating{0}
\formalauthor{Diep Trieu Thi} 
A hypermap is tame if and only if its opposite hypermap is tame as well.
\end{lemma}

\begin{lemma}\guid{RUNOQPQ}\rating{ZZ}
Every tame hypermap is a restricted hypermap.
\end{lemma}

\begin{proof}
  By definition, a tame hypermap is nonempty, connected, plain,
  planar, and simple.  The edge map has no fixed points.  The node map
  has no fixed points.  The cardinality of every face is at least
  three.  These are precisely the defining properties of a restricted
  hypermap.
\end{proof}


A list of hypermaps appears at \cite{website:Hales:1998:Code}.

\begin{theorem}\guid{WTEMDTA}
  \label{theorem:classification} Every tame hypermap is isomorphic to
  a hypermap in the list~\cite{website:Hales:1998:Code}, or is
  isomorphic to the opposite of a hypermap in the list.
  \indy{Index}{isomorphic}%
\end{theorem}

\begin{note}%XX
  The (java version of the) graph generator has been run with the new
  parameters described in this version of the book.  It produces
  finitely many graphs (about 25,000).  This new list will eventually
  replace the archive from 1998.  The Bauer-Nipkow formalization of
  the classification has not yet been adapted to this new definition
  of tameness.  \indy{Index}{Java}%
\end{note}


Computers are used to generate a list of all hypermaps and to check
them against the archive of tame hypermaps.  The computer program is
based on the hypermap generation construction of
Section~\ref{sec:face-insert}.  According to the results of that
section, every restricted hypermap can be generated by an elementary
edge-insertion algorithm.  Every tame hypermap is a restricted
hypermap.  The list of restricted hypermaps can be filtered to obtain
a complete list of tame hypermaps.  \indy{Index}{computer
  calculation}%

\section{Contravening Hypermap}

\indy{Index}{hypermap!contravening}%
% The aim is to prove Conjecture~\ref{conj:L12}.  For a contradiction,
% assume a counterexample.
A counterexample to the Kepler conjecture leads to a finite packing
$V\subset \BB$ that violates the inequality~(\ref{eqn:CCE}).  The
purpose of this section is to select a special class of
counterexamples to this inequality.  Let
\begin{equation}\label{eqn:fan-edge}
\begin{array}{lll}
E_{std} &= \{\{\v,\w\}\in V^2\mid 0 < \norm{\v}{\w}\le 2\hm \}\\
E_{ctc} &= \{\{\v,\w\}\in V^2\mid \norm{\v}{\w}= 2 \}\subset E_{std}.
\end{array}
\end{equation}

\begin{lemma}\guid{UBHDEUU}\rating{140}
$(V,E)$ is a fan, if $E=E_{std}$ or $E=E_{ctc}$.
\end{lemma}
$(V,E_{std})$ and $(V,E_{ctc})$ are called the standard fan and
contact fan, respectively.  Let $\op{hyp}(V,E)$ be the associated
hypermap.  \indy{Index}{fan!standard}%
\indy{Index}{fan!contact}%
\indy{Notation}{V@$V$ (packing)}%
\indy{Notation}{V@$( V, E ) $ (fan)}%
\indy{Notation}{hyp@$\op{hyp}$ (hypermap)}%
\indy{Notation}{E@$E_{std}$}%
\indy{Notation}{E@$E_{ctc}$}%

\begin{proof} 
$(V,E_{ctc})$ is a fan by Lemma~\ref{lemma:ctc-fan}.

Since $E_{ctc}\subset E_{std}$, it follows from
Lemma~\ref{lemma:subset-fan} that $(V,E_{ctc})$ is a fan.
\end{proof}

\begin{definition}[isolated,~surrounded]
  Let $(V,E)$ be a fan.  Say that $\v\in V$ is \newterm{isolated} in
  the fan if $E(\v)$ is empty.  (That is, the degree of the
  corresponding node in the hypermap is $0$.) Say that $\v\in V$ is
  \newterm{surrounded} in the fan if the azimuth angles of all darts
  at the node $\v$ are less than $\pi$.  (In particular, the
  cardinality of $E(\v)$ is at least three.)
\end{definition}
\indy{Index}{isolated}%
\indy{Index}{surrounded}%
\indy{Index}{angle!azimuth}%
\indy{Index}{azimuth}%
\indy{Index}{cardinality}%
\indy{Notation}{E@$E(\v)$ (fan)}%

The following lemma appears in Sch\"utte and van der
Waerden~\cite{vanderWaerden:1951}.

\begin{lemma}\guid{FATUGPD}\rating{200}
Given any packing $V\subset \BB$,
there exists a  packing $V'$ 
in bijective correspondence with $V$:
\begin{displaymath}
\phi:V\to V'
\end{displaymath}
such that $\normo{\v} = \normo{\phi(\v)}$ and
such that every vertex $\v'$ in the contact fan of $V'$
is either isolated or surrounded.
\end{lemma}
\indy{Index}{packing!finite}%
\indy{Index}{fan!contact}%
\indy{Notation}{V@$V'$ (packing)}%

\begin{proof} Consider all finite packings in bijective correspondence
  with $V$, such that the bijection preserves distances of points from
  the origin.  Among these packings, pick one $V'$ with the largest
  number of isolated points in the contact graph.  If there is a point
  $\v\in V'$ that is not isolated and not surrounded, then it has a
  contact dart with angle at least $\pi$.  Perturb $\v$ away from the
  contacts, making it isolated, while preserving its distance to
  $\orz$.  The perturbed packing $V''$ has one more isolated point
  than $V'$, which is contrary to the supposed maximality of $V'$.
  Hence, the conclusion of the lemma holds for $V'$.
\end{proof}

The following lemma shows that the corresponding result holds for the
standard fan.  Reader beware!  The set of isolated and surrounded
vertices depends on a choice of fan.  The next proof makes heavy use
of two different fans $(V,E_{std})$ and $(V,E_{ctc})$, which have
different sets of isolated and surrounded vertices, even though the
set $V$ is the same in both cases.



\begin{lemma}\guid{FJLBXS}\rating{1000}
\label{lemma:surrounded}  % was rating{500}.
Given any packing $V\subset \BB$,
there exists a  packing $V'$ 
in bijective correspondence with $V$:
\begin{displaymath}
\phi:V\to V'
\end{displaymath}
such that $\normo{\v} = \normo{\phi(\v)}$ and
such that every vertex $\v'$ in the standard fan of $V'$
is isolated or surrounded.
\indy{Index}{packing!finite}%
\end{lemma}

\begin{proof} For any packing $V'\subset \BB$, let $V'_{iso},V'_{sur}$
  be the sets of isolated and surrounded vertices of $V'$
  respectively, in the {\bf contact} fan.  By the previous lemma,
  assume without generality that every vertex $\v$ in the contact fan
  of $V$ is either isolated or surrounded:
\begin{displaymath}
V = V_{iso} \cup V_{sur}.
\end{displaymath}
Consider the set $\CalV$ of pairs $(V',\phi)$ consisting of a packing $V'$
and bijection $\phi:V\to V'$ such that
\begin{itemize}
\item $\phi$ preserves distances from the origin, 
\item $V_{sur}=V'_{sur}$, and 
\item the restriction of $\phi$ to $V_{sur}$ is the identity map $I$.
\end{itemize}
\indy{Notation}{V@$\CalV$ (family of packings)}%%
\indy{Notation}{V@$V_{iso}$ (contact isolated vertices)}%%
\indy{Notation}{V@$V_{sur}$ (contact surrounded vertices)}%%
\indy{Notation}{I@$I$ (identity map)}%

\claim{The set $\CalV$ is a nonempty compact topological space.}
Indeed, by fixing an enumeration $V = \{\v_0,\ldots,\v_r\}$, the set
$\CalV$ embeds into Euclidean space of dimension $3r$ under the map
that sends the pair $(V',\phi)$ to the point
$(\phi^{-1}\v_0,\ldots,\phi^{-1}\v_r)$.  Thus, $\CalV$ carries a
metric space topology as a subset of Euclidean space.  With a little
argument, it follows from the boundedness of $V\subset B(0,2\hm)$ that
$\CalV$ is compact.  Also, $(V,I)\in \CalV$, so $\CalV\ne\emptyset$.


For any pair $(V',\phi)$ in $\CalV$ and $\w\in \phi(V_{iso})$, let
$c(V',\phi,\w)$, be the minimum distance of $\w$ to a point in
$V'\setminus \{\w\}$.  For $i=1,\ldots,n=\card(V_{iso})$ define
constants $c_i=c_i(V',\phi)$, by ordering the real numbers
$c(V',\phi,\w)$, $\w\in\phi(V_{iso})$, in increasing order:
\begin{displaymath}
c_0 \le c_1 \le c_2 \le \cdots \le c_n.
\end{displaymath}
The constants $c_i:\CalV\to\ring{R}$ are continuous functions on the
compact configuration space $\CalV$.  There is a nonempty compact
subset $\CalV_0$ of $\CalV$ on which $c_0$ attains its maximum
value. Continuing recursively, there is a nonempty compact subset
$\CalV_{i+1}$ of $\CalV_{i}$ on which $c_i$ attains its maximum value.
\indy{Notation}{c i@$c_i$ (constant)}%
\indy{Notation}{V@$\CalV_0$}%

For the configuration $(V,I)\in \CalV$, it follows that $c_0(V,I) >2$.
Hence, $c_0(V',\phi)>2$ for $(V',\phi)\in\CalV_0$.  It follows
that $V'_{iso}$ is in bijective correspondence with
$V_{iso}$, for all $(V',\phi)\in \CalV$.

\claim{Any $(V',\phi)\in \CalV_n$ has the desired property.}
Otherwise, there exists some vertex $\v'\in V'_{iso}$ (in contact
isolation) is neither isolated nor surrounded in the standard fan.
Any such vertex satisfies $c(V',\phi,\phi(\v'))=c_i(V',\phi)$ for some
$i$.  Among such vertices, pick the vertex with smallest index $i$.
Write
\begin{displaymath}
  c(V',\phi,\phi(\v')) = c_i(V',\phi) = c_{i+1}(V',\phi) 
=\cdots= c_j(V',\phi) < c_{j+1}(V',\phi),
\end{displaymath}
for some $j\ge i$.  As $\v'$ is not isolated in the standard fan, it
follows that $c_j(\v') < 2\hm$.  As $\v'$ is not surrounded in the
standard fan, in the cyclic order on
\begin{displaymath}
\{(\v',\w')\in V'^2 \mid \norm{\v'}{\w'} = c_j(\v')\},
\end{displaymath}
some azimuth angle is at least $\pi$.
Thus, there is a direction in which $\v'$ can be perturbed
that increases $c_j$, while keeping $c_0,\ldots,c_{j-1}$
fixed.  This is contrary to the defining property of
$\CalV_n\subset\CalV_j$.  This establishes the claim.
\end{proof}



\begin{lemma}\guid{FCDJDOT}\rating{100}\label{lemma:CE} 
  Assume that there exists a counterexample to
  inequality~\ref{eqn:CCE}.  Then there also exists a counterexample
  $V$ with the following properties.
\begin{itemize}
\item $V\subset \BB$
\item $\CalL(V) > 12$, and no finite packing in $\BB$ attains a value
  larger than $\CalL(V)$.
\item The cardinality of $V$ is $13$ or $14$.
\item Every vertex $\v$ is surrounded in the standard fan $(V,E_{std})$.
\item Every vertex $\v$ that is not surrounded in the contact
fan $(V,E_{ctc})$ satisfies $\normo{\v}=2$.
\end{itemize}
\end{lemma}

\begin{proof} Assume that a counterexample exists.  The set of
  counterexamples $V\subset\BB$ is a compact set.  The function
  $\CalL$ is a continuous function on this compact set.  Hence, there
  exists $V$ that maximizes $\CalL(V)$.

  The set $V$ has cardinality $13$ or $14$
  (Lemma~\ref{lemma:13-14}). Lemma~\ref{lemma:surrounded} gives the
  existence of a counterexample $V$ in which every vertex is
  surrounded or isolated in the standard fan.  By
  Lemma~\ref{lemma:D'}, if there are any isolated verices in the
  standard fan, then it is not a counterexample.  Hence, in fact every
  vertex is surrounded in the standard fan.

  \claim{A vertex $\v$ that is not surrounded in the contact fan
    satisfies $\normo{\v}=2$.}  Otherwise, the counterexample does not
  maximize $\CalL$.  \indy{Notation}{V@$V$ (set of vertices)}%
\end{proof}


\begin{definition}[contravening]
  A finite packing is $V$ a \newterm{contravening} packing if it
  satisfies the properties of Lemma~\ref{lemma:CE}.  The fan
  $(V,E_{std})$ and the hypermap $\op{hyp}(V,E_{std})$ are said to be
  \newterm{contravening} when $V$ is contravening.
  \indy{Index}{packing!centered contravening}%
  \indy{Index}{hypermap!contravening}%
\end{definition}



The next section studies further properties of contravening hypermaps $H$.



\section{Contravention is Tame}
\indy{Index}{tame!contravention}%
\label{sec:contraproof}

Let $V$ be a centered packing with
standard fan $(V,E)$ and hypermap $H=\op{hyp}(V,E)=(D,e,n,f)$
be the hypermap attached to $(V,E)$.
The hypermap $H$ is plain, planar, connected, and simple.
The set of topological components of $Y(V,E)$ is in bijection with
the set of faces of $H$.  
\indy{Notation}{H@$H$ (hypermap)}%
For each face of $H$, the corresponding component $U_F$
is eventually radial with solid
angle
\indy{Notation}{U@$U_F$ (component)}%
\begin{displaymath}
\sol(U_F) = 2\pi + \sum_{x\in F} (\op{azim}(x) -\pi).
\end{displaymath}
Recall that
\begin{displaymath}\sum_{F} \sol(U_F) = 4\pi.\end{displaymath}
Recall the a map $\v:D\to V$ that maps each dart to its vertex:
\begin{displaymath}
\v \mapsto \v(x); \quad   x = (\v(x),\ldots).
\end{displaymath}
Set 
\begin{displaymath}h(x) = \normo{\v(x)}/2.\end{displaymath}
Define the weight function
\begin{equation}
\begin{array}{lll}
  \tau(V,E,F) &=\sum_{x\in F} \op{azim}(x)
  \left(1 + \dfrac{\sol_0}{\pi}(1- L(h(x)))\right) 
  + \left(\pi+{\sol_0}\right) (2- k(F))\vspace{6pt}\\
  &= \sol(U_F) + (2- k(F))\sol_0 - \dfrac{\sol_0}{\pi}
\sum_{x\in F}\op{azim}(x) (L(h(x)) - 1)\vspace{6pt}\\
  &= \sol(U_F) \left( 1 + \dfrac{\sol_0}{\pi}\right) 
- \dfrac{\sol_0}{\pi} \sum_{x\in F} \op{azim}(x)(L(h(x))),\\
\end{array}
\end{equation}
where $\sol_0$ is the solid angle of a spherical equilateral triangle
of side $\pi/3$, and $k(F)$ is the cardinality of $F$.
% 
These formulas are equivalent.  The proof of equivalence rests on the
Euler formula for planar hypermaps and the solid angle formula for
topological components $U_F$.  The first expression for $\tau(V,E,F)$
is particularly convenient, because it expresses $\tau$ as a sum of
local contributions from each dart.  \indy{Notation}{L@$L$}%
\indy{Notation}{ZZtau@$\tau$}%
\indy{Notation}{ZZDeltanaught@$\sol_0$}%
The main conjecture may be expressed in the following alternative
form:

\begin{lemma}\guid{HRXEFDM}\rating{80}
\begin{displaymath}
\sum \tau (V,E,F) \ge 4\pi - 20\sol_0
\end{displaymath}
if and only if
\begin{displaymath}
\sum L(h(x)) \le 12.
\end{displaymath}
\end{lemma}

\begin{proof}
  The solid angles over the sphere sum to $4\pi$ and the azimuth
  angles at each vertex sum to $2\pi$.
% The Euler relation for connected plain planar hypermaps gives
%\begin{displaymath}
%\sum_F (2- k(F)) = 2\#f - 2\#e = 4 - 2\#n.
%\end{displaymath}
Thus,
\begin{equation}\label{eqn:delta0}
\begin{array}{lll}
  \sum \tau (V,E,F) 
  &= 4\pi (1 + \dfrac{\sol_0}{\pi}) 
- (\dfrac{\sol_0}{\pi}) 2\pi \sum_{V} L(h)\vspace{6pt}\\
&= (4\pi - 20\sol_0) + 2\sol_0 (12 - \sum_V L(h(\v,\orz))).
\end{array}
\end{equation}
The result follows.
\end{proof}

The significance of the constant $\op{tgt}$ is that it is
approximately equal to $4\pi - 20\sol_0$.
\indy{Notation}{tgt@$\op{tgt}=1.541$}%
The following theorem is one of the main results of this chapter.  The
proof has been broken into a series of steps in the sections that
follow.

\begin{theorem}\guid{MQMSMAB} \label{theorem:contravene}
  Let $V$ be a contravening packing.  Then the weight assignment
  $\tau=\tau(V,E_{std},\cdot)$ on $H=\op{hyp}(V,E_{std})$ is
  admissible.  Moreover, the hypermap $H$ is tame with weight
  assignment $\tau$.
\end{theorem}
\indy{Notation}{H@$(H, \tau)$}%
\indy{Notation}{H@$H$ (hypermap)}%
\indy{Notation}{ZZtau@$\tau$ (weight)}%



\subsection{general properties}
\label{sec:startame}


Many of the properties of tameness are trivial or have been
established in earlier sections.  The following lemma quickly disposes
of many of the properties of tameness.

\begin{lemma}\guid{JGTDEBU}\rating{100}\label{lemma:multi}
                                                           %%100=without
                                                           %(vi),
                                                           %%(deprecated:
                                                           %300=with
                                                           %(vi))
  A contravening hypermap $H$ satisfies Properties \case{planar},
  \case{simple}, \case{nondegenerate}, \case{no loops}, \case{no
    double joins}, \case{face count}, \case{node count}, and the first
  part of \case{node size}
%~(i)--(\v),  % was (i)-(vi).
%(viii),  (x), and the first part of (xi).
of tameness.
\end{lemma}

\begin{proof}
  The hypermap is plain, planar, connected, and simple by the general
  results established in the chapter on fan.  That chapter also shows
  that the hypermap attached to a fan satisfies properties
  \case{nondegenerate}, \case{no loops}, and \case{no double joins}.

%The property~(vi) is established in \cite[Lemma~3.7]{sp1}.
  \claim{Properties~\case{face count} and the first half of
    property~\case{node size} hold}.  Indeed, every vertex is
  surrounded, meaning that the azimuth angles of the darts at the
  vertex are less than $\pi$.  As the angles around the vertex sum to
  $2\pi$, there are at least three darts in the node. Each of the
  darts in the node leads into a different face by
  property~\case{simple}.

%% 13 or 14 nodes.
  Finally, property~\case{node count} has already been established in
  Lemma~\ref{lemma:CE}.
\end{proof}

There remain properties \case{face size}
(Lemma~\ref{lemma:face-size}), %(vii), (ix), the second part of (xi),
                               %(xii), and (xiii).
\case{node types} (Lemma~\ref{lemma:degE}), \case{weights bound a}
(Lemma~\ref{lemma:degE}), \case{weights bound b}
(Lemma~\ref{lemma:weightB}), \case{weights bound d}
(Lemma~\ref{lemma:main}), and the second part of \case{node size}
(Lemma~\ref{lemma:node-upper}).


\subsection{properties of nodes}
\indy{Index}{node!properties}%



\begin{lemma}\guid{CDTETAT}\rating{140} \label{lemma:0.852}
Let $H$ be a contravening
hypermap. For every dart $x$ in a triangular face of $H$,
\begin{displaymath}0.852\le \azim(x)\le 1.9.\end{displaymath}
For every dart $x$ in a nontriangular face of $H$, 
\begin{displaymath}1.15\le\azim(x)\le 3.27.\end{displaymath}
\indy{Notation}{H@$H$ (hypermap)}%
\indy{Notation}{x@$x$ (dart)}%
Consequently, if a vertex $\v$ has type $(p,q,0)$, then $(p,q)$
must be one of the following pairs:
\begin{displaymath}
\begin{array}{lll}
  &(0,2),~(0,3),~(0,4),~(0,5),~(1,2),~(1,3),~(1,4),~(2,1),~(2,2),~(2,3),\\
  &(3,1),~(3,2),~(3,3),~(4,0),~(4,1),~(4,2),
  ~(5,0),~(5,1),~(6,0),~(6,1),~(7,0)
\end{array}
\end{displaymath}
\end{lemma}
%

\begin{proof}
The angle bounds are a calculation.  The sum of the azimuth angles
around a vertex satisfies:
\begin{displaymath}
p (0.852) + q (1.15) \le 2\pi \le p (1.9) + q (3.27),
\end{displaymath}
and the pairs satisfying these constraints are listed.
\end{proof}

\begin{lemma}\guid{SZIPOAS}\oldrating{80}\label{lemma:node-upper}
\rating{0}
\formalauthor{Vu Thanh}
%dcg{Lemma~21.4}{223} 
Contravening hypermaps satisfy the second part of property \case{node
  size}
%\ref{definition:tame:degree} 
of tameness.  That is, the cardinality of every
node is at most $7$.
\end{lemma}

\begin{proof}  The azimuth angle bound
\begin{displaymath}
(p+q+r) 0.852 \le 2\pi
\end{displaymath}
implies $p+q+r < 8$.
\end{proof}




\begin{lemma}\guid{KCBLRQC}\rating{300} \label{lemma:weightB}
  Let $\v$ be a node of type $(p,q,0)$ in a contravening hypermap.
  Then the property~\case{bound b} of a admissible weight assignment
  holds:
\begin{displaymath}
\sum_{ F\cap \v\ne \emptyset} \tau(V,E,F) \ge  b(p,q).
\end{displaymath}
\end{lemma}
\indy{Notation}{A@$A$ (faces)}%
\indy{Notation}{pqr@$(p,q,r)$}%

\begin{proof} Let $A =\{F\mid F\cap \v\ne\emptyset\}$.  The archive
  \cite[FUSDSPJ]{hales:2009:nonlinear} contains a list of nonlinear
  inequalities for $\tau(V,E,F)$ when $F$ is a triangle or
  quadrilateral. Each nonlinear inequality has the form
  \indy{Notation}{F@$F$ (polygon)}%
\begin{displaymath}\tau(V,E,F) \ge a~\op{azim}(x) + b,\end{displaymath}
for some $a,b\in\ring{R}$, where $x$ is the uniquely determined dart
at the node $\v$ in the face $F$.  These nonlinear inequalities admit
a linear relaxation as follows.  For each $a,b$, write a corresponding
linear inequality \indy{Notation}{ZZtau@$\tau$}%
\indy{Notation}{x@$x$ (dart)}%
\begin{displaymath}
t(F) \ge a~z(F) + b,
\end{displaymath}
where $t(F)$ and $z(F)$ are variables indexed by $F\in A$.
\indy{Notation}{A@$A$ (index set)}%
\indy{Notation}{t@$t$ (variable)}%
\indy{Notation}{z@$z$ (variable)}%
Then  minimize 
\begin{displaymath}\sum_{F\in A} t(F)\end{displaymath}
subject to these linear inequalities and the constraint
\begin{displaymath}
2\pi = \sum_{F\in A} z(F).
\end{displaymath}
This linear program has been executed on a computer for each of the
types $(p,q,0)$ of Lemma~\ref{lemma:0.852}. The given constants are
obtained from the (downward rounded) solutions to these linear
programs.
\end{proof}

\begin{lemma}\guid{BDJYFFB}\rating{200}\label{lemma:degE}
  Every contravening hypermap satisfies Properties \case{node types}
  and \case{weight bound A}
%\ref{definition:tame:degreeE} 
of tameness: 
If a node has cardinality $7$, then the type of the
node is $(p,q,0)$ for some $p,q\ge0$.   If the
cardinality of the node is exactly $6$, then the type of the node
is $(6,0,0)$, $(5,0,1)$, or $(5,1,0)$.
If the type is $(5,0,1)$, let $A$ be the set of five triangles at the
node $\v$.  Then
\indy{Notation}{A@$A$ (set of triangles)}%
\begin{displaymath}
\sum_{F\in A} \tau(V,E,F) > a,
\end{displaymath}
where $a=0.63$.
\end{lemma}



\begin{proof} These conclusions also come from linear programming.
  The same set of nonlinear inequalities is used, and the linear
  relaxation is constructed in the same way.  The linear programming
  bounds exceed the constant $\op{tgt}$ in the cases excluded in the
  conclusion of the lemma.  The constant $a$ is the downward rounding
  of the solution to the linear program for $(5,0,1)$.
\end{proof}
\indy{Notation}{tgt@$\op{tgt}=1.541$}%

\indy{Notation}{a@$a$ (constant)}%

\subsection{faces}



\begin{lemma}\guid{CRTTXAT}\rating{140}  \label{lemma:face-size}
  Property~\case{face size} holds.  That is, Every face of a
  contravening hypermap $\op{hyp}(V,E_{std})$ has cardinality at least
  $3$ and at most $6$.
\end{lemma}

\begin{proof} The lower bound holds because the hypermap has no loops
  or double joins.  For a contradiction, let $F$ be a face of the
  hypermap of cardinality at least $7$.  Divide the proof into cases
  depending on whether the following inequality holds:
\begin{displaymath}
\sum _{x\in F} (\normo{\v(x)}-2) \ge 4(\hm-1).
\end{displaymath}
If the inequality holds, then since $L$ is the linear interpolation
between the points $(1,1)$ and $(\hm,0)$, and there are at most $14$
points in $V$, it follows that
\begin{displaymath}
\CalL(V) = \sum_{\v\in V}L(\normo{\v}/2) \le 12 L(1) + 2 L(\hm) =12,
\end{displaymath}
and the main inequality holds.

Now assume that the inequality is false.
The edge $\{\v,\w\}$ has arclength at least
\begin{displaymath}
  \arc(\normo{\v},\normo{\w},\norm{\v}{\w}) 
\ge \arc(\normo{\v},\normo{\w},2). 
\end{displaymath}

A calculation~\cite[cc:arc]{hales:2009:nonlinear} gives
\begin{displaymath}
\arc(\normo{\v},\normo{\w},2)
\ge 1 - 0.6076 (\normo{\w}/2 - 1) - 0.6076 (\normo{\v}/2 - 1).
\end{displaymath} %%CC:arc
The sum over a face of size at least $7$ gives
\begin{displaymath}
\begin{array}{lll}
\sum \arc(\normo{\v_i},\normo{\v_{i+1}},\norm{\v_i}{\v_{i+1}})&\ge
7 - 0.6076 \sum (\normo{\v_i}-2) \\
&\ge 7 - 0.6076 (2) (0.52) \\
&> 2\pi.
\end{array}
\end{displaymath}
The left-hand side is the perimeter of the localization of $(V,E)$
along $F$.  This perimeter estimate is contrary to the upper bound on
perimeter in Lemma~\ref{lemma:convex-hyp}.
\end{proof}




\section{Admissibility}


%\begin{theorem}\guid{THPJDQA}\rating{0}  %points for OLNSWLK below.
%  The weight assignment $\tau$ on a contravening hypermap is
%  admissible, and the total weight of $\tau$ is less than
%  $\op{tgt}=1.541$.  \indy{Index}{weight assignment}%
%\end{theorem}
%\indy{Notation}{tgt@$\op{tgt}=1.541$}%
%\subsection{admissibility}
%\label{sec:admissibility}


The main result (Lemma~\ref{lemma:main}) of this section is a proof
that for every contravening hypermap $H=\op{hyp}(V,E)$, the function
$\tau(V,E,\cdot)$ is an admissible weight assignment on $H$.



\begin{lemma}\guid{GBMLQWW}\rating{ZZ}  \label{lemma:tau-local}
  Let $V$ be a contravening packing with standard fan $(V,E_{std})$.
  Let $F$ be any face of $\op{hyp}(V,E_{std})$.  Then
\begin{displaymath}
\tau(V,E_{std},F) = \tau(V',E',F),
\end{displaymath}
where $(V',E',F)$ is the localization of $(V,E_{std})$ along the face $F$.
\end{lemma}

\begin{proof} The value $\tau(V,E,F)$ is expressed entirely in terms
  of $\normo{\v}$ for $\v\in V'\subset V$ and in terms of
  $\op{azim}(x)$, for $x\in F$.  By Lemma~\ref{lemma:localization},
  the terms $\op{azim}(x)$ are the same, whether calculated in terms
  of the hypermap of $(V,E_{std})$ or in terms of that of $(V',E')$.
\end{proof}


\begin{lemma} \guid{RNSYJXM}\rating{ZZ} Let $(V,E)$ be any fan, and
  let $F$ be a face of its hypermap.  Let $(V',E')$ be the
  localization of $(V,E)$ along $F$.  Assume that $\op{azim}(x)<\pi$,
  for every dart $x$ in $\op{hyp}(V,E)$.  Then $(V',E',F)$ is a cyclic
  fan.
\end{lemma}

\begin{proof}
  Lemma~\ref{lemma:localization} gives all the properties of a cyclic
  fan except for property \case{wedge}: $V'\subset \bWdart(x)$ for
  every dart $x\in F$.  By Lemma~\ref{lemma:face}, $U_F \subset
  \Wdart(x)$.  The wedge $\bWdart(x)$ is closed and contains
  $\Wdart(x)$. Hence, the closure $\bar U_F$ is contained in
  $\bWdart(x)$.  Let $\v\in V'$ and choose $\w\in V$ such that
  $y=(\v,\w)\in F$.
  % There is a dart of $F$ of the form $y=(\v,\rho \v)$.  Since $y$
  % leads into
  Since the dart $y$ leads into $U_F$, every neighborhood of $\v$
  meets $U_F$.  Thus, $\v\in \bar U_F\subset \bWdart(x)$.  This
  completes the proof.
\end{proof}








%
%\begin{lemma}\guid{BGDPIZY}\rating{ZZ}
%Suppose that
%\begin{displaymath}
%\tau(V,E,F) \ge d (r,s)
%\end{displaymath}
%for every special fan $(V,E,F,G)$,
%where $(r,s)$ are the parameters of the special fan.
%Then Lemma~\ref{lemma:main} holds.
%\end{lemma}



%The proof adopts the following convention for
%configurations.  Let $v_0,\ldots,v_{n-1}$ be the vertices
%of the polygon, and write
%\begin{displaymath}
%y_i = \normo{v_i},\quad y_{ij} = \norm{v_i}{ v_j}.
%\end{displaymath}
%\indy{Notation}{v@$v$ (vertex)}%



\begin{lemma}\guid{OLNSWLK}\rating{3000}\label{lemma:main} %including
                                                           %lemmas
                                                           %that lead
                                                           %up to it.
                                                           %Interval
                                                           %ineq may
                                                           %be
                                                           %assumed.
Property~\case{weight bound d} holds.  That is,
let $V_0$ be a contravening packing, and  let $F$ be a
face of $\op{hyp}(V_0,E_{std})$ of cardinality $k$.  Then
$\tau(V_0,E_{std},F) \ge d(k)$.
\end{lemma}
\indy{Notation}{F@$F$ (face)}%
\indy{Notation}{k@$k$ (cardinality of a face)}%


\begin{remark}
  In the original 1998 proof, the corresponding result is called the
  ``Main Estimate.''  The proof of that theorem takes about 30 pages
  and relies on many long computer calculations.  The proof given here
  is substantially simpler than the proof of the original main
  estimate, but it is still nontrivial. In fact, nearly the entire
  chapter on cyclic fans consists of a proof of the main estimate.
  The proof presented in this book has the advantage of various
  special features that were not present in the original proof:
  polygons are convex, the hypermap is simple, and that each face has
  at most six sides.
\end{remark}




\begin{proof}
  Let $V_0$ be a contravening packing.  By
  Lemma~\ref{lemma:tau-local}, $\tau(V_0,E_{std},F)=\tau(V,E,F)$ where
  $(V,E,F)$ is the localization of $(V_0,E_{std})$ along $F$.

  \claim{The tuple $(V,E,F,\emptyset)$ is a special fan.}  Indeed,
  every property can be verified in turn.  The properties
  \case{packing} and \case{annulus} result from the assumption that
  $V$ is contravening packing.

\case{cyclic fan}  The localization $(V,E,F)$ is indeed a cyclic fan.

\case{diagonal} If $\norm{v}{w}<2\hm$ and $v\ne w$, then $\{v,w\}\in
E_{std}$.
% Assume that $v,w\in V$ and $C^0\{v,w\}\subset \Wdart(x)$ for all
% $x\in F$.  The intersection of the sets $\Wdart(x)$ is the
% topological component $U_F$ of $Y(V_0,E_{std})$.  Thus,
% $C^0\{v,w\}\subset Y(V_0,E_{std})$.  Assume for a contradiction that
% $\norm{v}{w}< 2\hm$. Then by the definition of $E_{std}$,
% $\{v,w\}\in E_{std}$.  Hence $C^0\{v,w\}\subset
% X(V_0,E_{std})$. This leads to a contradiction, because $C^0\{v,w\}$
% is nonempty, and $X(V_0,E_{std})\cap Y(V_0,E_{std})=\emptyset$.

\case{subset} \case{g norm} These properties are trivial, because
$G=\emptyset$.

\case{e norm} If $\{v,w\}\in E$, then $\{v,w\}\in E$.  It follows from
the definition of standard fan that $2\le \norm{v}{w}\le 2\hm$.

\case{card} Let $s=\card(\emptyset) = 0$, and $r=\card(E)$.  Then
$0\le s\le 3$ is trivial. Also,
\begin{displaymath}3-s \le 3\le r \le 6=6 - 2s\end{displaymath}
follows from the defining property \case{face size} of a tame
hypermap.  This proves the claim that $(V,E,F,\emptyset)$ is a special
fan with parameters $r=k$ and $s=0$.

%By Lemma~\ref{lemma:min-empty}, the set of minimal fans is empty.
By Lemma~\ref{lemma:empty-d}, 
\begin{displaymath}
\tau(V,E,F) \ge d (r,s)
\end{displaymath}
for every special fan $(V,E,F,G)$. 
Finally, 
\begin{displaymath}
\tau(V_0,E_{std},F)=\tau(V,E,F) \ge d(r,s) = d(k,0) = d(k).
\end{displaymath}  
This completes the proof.
\end{proof}


\begin{remark}
  It is helpful to keep in mind the origin of the constants $d(k)$.
  Although the proof of Lemma~\ref{lemma:main} does not produce sharp
  lower bounds on $\tau(V,E,F)$, the statement of the lemma is
  motivated by the following configurations.  Consider a nonplanar
  polygon contained in a sphere of radius $2$, with $k$ sides all of
  length $y_{i,i+1}=2$, heights $y_i=2$, and $k-3$ diagonals of length
  $2\hm$: $y_{0,j}=2\hm$, for $j=2,\ldots,k-2$.  Let $V$ be the set of
  vertices of the polygon, let $(V,E_{ctc})$ be its contact fan, and
  let $F$ be the face of $\op{hyp}(V,E_{ctc})$ representing the
  ``interior'' of the polygon.  Evaluating $\tau$ on these rigid
  configurations gives
\begin{displaymath}
\tau(V,E_{ctc},F) = 
\begin{cases}
0.20612\ldots & k=4\\
0.48356\ldots & k=5\\
0.760993\ldots &k=6
\end{cases}
\end{displaymath}
These calculations suggested the values of constants $d(k)$.  The
constants $d(k)$ are slightly smaller than these calculated values.%
\footnote{Note $\tau(2.1028,2,2,2,2.52,2.52) = 0.275951\ldots <
  0.277433\ldots = \tau(2,2,2,2,2.52,2.52)$.}  \indy{Notation}{d@$d$
  ($\tau$ bound)}%
\indy{Notation}{ZZtau@$\tau$}%
\indy{Notation}{k@$k$ (face size)}%
\end{remark}


\section{Linear Programs}

One can attach a linear program to each tame hypermap.
For each tame hypermap $H$ there is a configuration space $D(H)$ of all
finite packings $V\subset \BB$ whose standard fan is
isomorphic to $H$.
\indy{Notation}{H@$H$ (hypermap)}%
\indy{Notation}{D@$D(H)$ (configuration space)}%

A nonlinear optimization problem asks for the maximum of
\begin{equation}\label{eqn:L2}
\sum_{\v\in V} L(\normo{\v}/2)
\end{equation}
over all $V\in D(H)$.

The linear program comes as a linear relaxation of this nonlinear
optimization problem on $D(H)$. That is, the optimal solution of the
the linear program has value at least as great as the corresponding
nonlinear problem.  By showing that the value of each linear program
is at most $12$, one may conclude that the maximum of \eqn{eqn:L2}
is at most $12$.


\begin{note}%XX
  The linear programming part of the proof has not been
  completed. There are two cases that remain.
\end{note}

