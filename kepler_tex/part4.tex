

\chapter{Local Bounds in Exceptional Regions}% DCG Sec.12. p. 125
    \oldlabel{4}
    \label{sec:BER}

%\section{Positivity}
%    \oldlabel{4.1}





\section{The Main Theorem} %DCG 12.1, p125
    \label{sec:the-main-theorem}
    \oldlabel{4.4}

Let $(R,D)$ be a standard cluster. Let $U$ be the set of corners,
that is, the set of vertices in the cone over $R$ that have height
at most $2t_0$.  Consider the set $E$ of edges of length at most
$2t_0$ between vertices of $U$. We attach a multiplicity to each
edge. We let the multiplicity be $2$ when the edge projects
radially to the interior of the standard region, and $0$ when the
edge projects radially to the complement of the standard region.
The other edges, those bounding the standard region, are counted
with multiplicity $1$.

Let $n_1$ be the number of edges in $E$, counted with multiplicities.
Let $c$ be the number of classes of vertices under the equivalence
relation $v\sim v'$ if there is a sequence of edges in $E$ from $v$ to
$v'$. Let $n(R)=n_1+2(c-1)$. If the standard region under $R$ is a
polygon, then $n(R)$ is the number of sides.

\begin{theorem}
    \label{thm:the-main-theorem}
    Let $(\Lambda,v)$ be a contravening centered packing.
$\tau_R(\Lambda,v) > t_n$, where $n=n(R)$ and
    $$
    \begin{array}{lll}
    t_4&=0.1317,\quad t_5=0.27113,\quad
    t_6=0.41056,\\
    t_7&=0.54999,\quad t_8=0.6045.
    \end{array}
    $$
The centered packing scores less than $8\,\pt$, if $n(R)\ge 9$,
for some standard cluster $R$. The scores satisfy
$\sigma_R(\Lambda,v)<s_n$, for $5\le n\le 8$, where
    $$
    s_5=-0.05704,\quad s_6=-0.11408,\quad
    s_7=-0.17112,\quad s_8=-0.22816.
    $$
\end{theorem}

Sometimes, it is convenient to calculate these bounds as a multiple
of $\pt$.  We have
    $$
    \begin{array}{lll}
    t_4&>2.378\,\pt,\quad t_5>4.896\,\pt,\quad
    t_6>7.414\,\pt,\\
    t_7&>9.932\,\pt,\quad
    t_8>10.916\,\pt.
    \end{array}
    $$
    %
    $$
    s_5 < -1.03\,\pt,\quad s_6<-2.06\,\pt,\quad
    s_7<-3.09\,\pt,\quad s_8<-4.12\,\pt.
    $$




\begin{corollary}
    \label{cor:std-aggregate-list}
Every standard region is a either a polygon or one shown in
Figure~\ref{fig:std-aggregates}
\end{corollary}


%\gram|2.2||diag7.1.ps|
%\gram|0.5||SAMF/samfigp20.eps|

\begin{figure}[htb]
  \centering
  \myincludegraphics{\ps/diag7.1.ps}
  \caption{}
  \label{fig:std-aggregates}
\end{figure}


In the cases that are not (simple) polygons, we call the {\it polygonal
hull\/} the polygon obtained by removing the internal edges and
vertices. We have $m(R)\le n(R)$, where the constant $m(R)$ is the
number of sides of the polygonal hull.

\begin{proof}
By the theorem, if the standard region is not a polygon, then $8\ge
n_1\ge m\ge 5$. (Quad clusters and quasi-regular tetrahedra have no
enclosed vertices. See Lemma~\ref{lemma:no-enclosed-tri} and
Lemma~\ref{tarski:at-most-one-negative}.) If $c>1$, then $8\ge
n=n_1+2(c-1)\ge 5+2(c-1)$, so $c=2$, and $n_1=5,6$ (frames $2$ and $5$
of the figure).

Now take $c=1$.    Then $8\ge n\ge 5+(n-m)$, so $n-m\le 3$.  If $n-m=3$,
we get frame $3$. If $n-m=2$, we have $8\ge m+2\ge 5+2$, so $m=5,6$
(frames $1$, $4$).

But $n-m=1$ cannot occur, because a single edge that does not bound the
polygonal hull has even multiplicity.  Finally, if $n-m=0$, we have a
polygon.
\end{proof}

\begin{corollary} \label{lemma:70}
If the type of a vertex of a centered packing is $(7,0)$, then it
does not contravene.
\end{corollary}

\begin{proof} By Theorem~\ref{thm:the-main-theorem},
if there is a non-triangular region, we have
    $$\tau(\Lambda,v)\ge\tlp(7,0)+t_4>\squander.$$
Assume that all standard regions are triangular.  If there is a
vertex that does not lie on one of seven triangles, we have by
Lemma~\ref{lemma:pq}:
    $$\tau(\Lambda,v)\ge\tlp(7,0)+0.55\,\pt>\squander.$$
Thus, all vertices lie on one of the seven triangles.  The
complement of these seven triangles is a region triangulation by
five standard regions.  There is some vertex of these five that does
not lie on any of the other four standard regions in the complement.
That vertex has type $(3,0)$, which is contrary to
Lemma~\ref{lemma:pq-impossible}.
%By the results of Part I, which
%treats the case in which all standard regions are triangles, we
%may assume that the centered packing has at least one quadrilateral.  We then
%have $\tau(\Lambda,v)\ge\tlp(7,0) + t_4
%>\squander$.  The result follows.
\end{proof}

\subsection{Nonagons} %DCG 12.2 p127
    \label{sec:nonagon}
    \oldlabel{4.6}

A few additional comments are needed to eliminate $n=9$ and $10$,
even after the bounds $t_9$, $t_{10}$ are established.

\begin{lemma} \label{lemma:s9-t9}
Let $F$ be a set of one or more standard regions bounded by a simple
polygon with at most nine edges.  Assume  that
    $$\sigma_F(\Lambda,v) \le s_9\quad\text{and }\tau_F(\Lambda,v)\ge t_9,$$
where $s_9=-0.1972$ and $t_9=0.6978$.  Then $(\Lambda,v)$ does not
contravene.
\end{lemma}

\begin{proof}
Suppose that $n=9$, and that $R$ squanders at least $t_9$ and
scores less than $s_9$.  This bound is already sufficient to
conclude that there are no other standard clusters except
quasi-regular tetrahedra ($t_9+t_4>\squander$). There are no
vertices of type $(4,0)$ or $(6,0)$: $t_9+4.14\,\pt>\squander$ by
Lemma~\ref{lemma:pq}.   So all vertices not over the exceptional
cluster are of type $(5,0)$. Suppose that there are $\ell$
vertices of type $(5,0)$. The polygonal hull of $R$ has $m\le 9$
edges. There are $m-2+2\ell$ quasi-regular tetrahedra. If $\ell\le
3$, then by Lemma~\ref{lemma:0.55}, the score is less than
    $$s_9+ (m-2+2\ell)\,\pt -0.48 \ell\,\pt < \scoregoal.$$
If on the other hand, $\ell\ge 4$, the centered packing squanders
more than
    $$t_9+ 4(0.55)\,\pt > \squander.$$
\end{proof}


The bound $s_9$ will be established as part of the proof of
Theorem~\ref{thm:the-main-theorem}.

The case $n=10$ is similar.  If $\ell=0$, the score is less than
    $(m-2)\,\pt\le \scoregoal$,
because the score of an exceptional cluster is strictly negative,
Theorem~\ref{lemma:quad0}.  If $\ell>0$, we squander at least
    $t_{10}+ 0.55\,\pt > \squander$ (Lemma~\ref{lemma:0.55}).


\subsection{Distinguished edge conditions} %DCG 12.3, p128
    \oldlabel{4.2}

Take an exceptional cluster.  We prepare the cluster by erasing
upright diagonals, including those that are $3$-unconfined,
$3$-crowded, or $4$-crowded.  The only upright diagonals that we
leave unerased are loops.  When the upright diagonal is erased, we
score with the truncated function $\op{sovo}(\cdot,\lambda_{oct})$ 
away from flat
quarters.  Flat quarters are scored with the function
$\hat\sigma$. The exceptional clusters in \Chaps~\ref{x-4} and
\ref{x-5} are assumed to be prepared in this way.


A simplex $S$ is {\it special\/} if the fourth edge has length at
least $2\sqrt{2}$ and at most $3.2$, and the others have length at
most $2t_0$. The fourth edge will be called its diagonal.


We draw a system of edges between vertices.  Each vertex will have
height at most $2t_0$.  The radial projections of the edges to the
unit sphere will divide the standard region into subregions. We
call an edge {\it nonexternal\/} if the radial projection of the
edge lies entirely in the (closed) exceptional region.

\begin{enumerate}
\item Draw all nonexternal edges of length at most $2\sqrt{2}$
except those between nonconsecutive anchors of a remaining upright
diagonal. These edges do not cross (Lemma~\ref{tarski:skew-quad}).
These edges do not cross the edges of slices
(Lemma~\ref{tarski:qrtet-pair-pass} and
Lemma~\ref{tarski:pass-anchor}).

\item Draw all edges of (remaining) slices
that are opposite the upright diagonal, except when the edge gives
a special simplex. The interiors of distinct slices do
not meet (Lemma~\ref{lemma:anchor-no-overlap}), so these edges do
not cross. These edges are nonexternal (Lemma~\ref{x-3.6} and
Lemma~\ref{tarski:2t0-doesnt-pass-through}).

\item Draw as many additional nonexternal edges as possible of
length at most $3.2$ subject to not crossing another edge, not
crossing any edge of a slice, and not being the
diagonal of a special simplex.
\end{enumerate}

We fix once and for all a maximal collection of edges subject to
these constraints. Edges in this collection are called {\it
distinguished\/} edges. The radial projection of the distinguished
edges to the unit sphere gives the bounding edges of regions
called the {\it subregions}.  Each standard region is a union of
subregions. The vertices of height at most $2t_0$ and the vertices
of the remaining upright diagonals are said to form a {\it
subcluster}.


By construction, the special simplices and slices
around an upright quarter form a subcluster.  Flat quarters in the
$Q$-system, flat quarters of an isolated pair, and simplices of
type $\SA$ and $\SB$ are subclusters.  Other subclusters are
scored by the function $\op{sovo}(\cdot,\lambda_{oct})$. 
For these subclusters,
Formula~\ref{eqn:3.7} extends without modification.

\subsection{Scoring subclusters} %DCG 12.4, p128
    \oldlabel{4.3}

XX move this entire section much earlier to a general discussion
of regions (standard or otherwise).

The terms of Formula~\ref{eqn:3.7} defining
$\op{svR}(v,P,\Lambda,t_0)$ have a clear geometric
interpretation as quoins, wedges of $t_0$-cones, and solid angles
(see Section~\ref{sec:scoring}). There is a quoin for each Rogers
simplex. There is a somewhat delicate point that arises in
connection with the geometry of subclusters.  It is not true in
general that the Rogers simplices entering into the truncation
$(v,P,\Lambda,t_0)$ of $(P,D)$ lie in the cone over $P$.
Formula~\ref{eqn:3.7} should be viewed as an analytic continuation
that has a nice geometric interpretation when things are nice, and
which always gives the right answer when summed over all the
subclusters in the cluster, but which may exhibit unusual behavior
in general. The following lemma shows that the simple geometric
interpretation of Formula~\ref{eqn:3.7} is valid when the
subregion is not triangular.

XX at various places, we use a version of the following 
for sqrt2-truncation
and also for t0-truncation with barriers other than qrtriangles.


\begin{lemma}
    \label{lemma:no-cross}
If a subregion is not a triangle and is not  the subregion
containing the slices around an upright diagonal, then
   $$\op{cone}^0(0,v,|v-v_0|/(2t_0)),$$
where $v$ is a corner of the subcluster,
does not cross out of the subregion.
\end{lemma}

\begin{proof}
For a contradiction, let $\{v_1,v_2\}$ be a distinguished edge that
the cone crosses. If both edges $\{v,v_1\}$ and $\{v,v_2\}$ have
length less than $2t_0$, by Lemma~\ref{tarski:E:part4:4},
there can be no enclosed vertex $w$ of
height at most $2t_0$, unless its distance from $v_1$ and $v_2$ is
less than $2t_0$.
In this case, we can replace $\{v_1,v_2\}$ by an edge of the
subregion closer to $v$, so without loss of generality we may
assume that there are no enclosed vertices when both edges
$\{v,v_1\}$ and $\{v,v_2\}$ have length less than $2t_0$.

The subregion is not a triangle, so $|v-v_1|\ge 2t_0$, or
$|v-v_2|\ge 2t_0$, say $|v-v_1|\ge 2t_0$.  The result now follows
from Lemma~\ref{tarski:beta:dcg-p129}.
\end{proof}

As a consequence, in nonspecial standard regions, the terms in the
Formula~\ref{eqn:3.7} for $\op{svR}_0$ retain their interpretations as
quoins, Rogers simplices, $t_0$-cones, and solid angles, all lying
in the cone over the standard region.


\subsection{Proof} %DCG 12.5, p 129
    \oldlabel{4.5}

The proof of the theorem occupies the rest of the \chap. The
inequalities for triangular and quadrilateral regions have already
been proved. The bounds on $t_3$, $t_4$, $s_3$, and $s_4$ are
found in Lemma~\ref{lemma:roger0}, Section~\ref{x-3.2},
Lemma~\ref{lemma:1pt}, and Theorem~\ref{lemma:quad0},
respectively. Thus, we may assume throughout the proof that the
standard region is exceptional

We begin with a slightly simplified account of the method of
proof. Set $t_9=0.6978$, $t_{10}= 0.7891$, $t_n=\squander$, for
$n\ge 11$. Set $D(n,k) = t_{n+k} - 0.06585\,k$, for $0\le k\le n$,
and $n+k\ge 4$. This function satisfies
    \begin{equation}
    D(n_1,k_1)+D(n_2,k_2)\ge D(n_1+n_2-2,k_1+k_2-2).
    \label{eqn:D-superadd}
    %\oldlabel{eqn:4.5.1}
    \end{equation}
In fact, this inequality unwinds to $t_r+0.13943\ge t_{r+1}$,
$D(3,2)=0.13943$, and $t_n =(0.06585)2+(n-4)D(3,2)$, for $n=4,5,6,7$.
These hold  by inspection.

Call an edge between two vertices of height at most $2t_0$ {\it long\/}
if it has length greater than $2t_0$. Add the distinguished edges to
break the standard regions into subregions. We say that a subregion has
{\it edge parameters} $(n,k)$ if there are $n$ bounding edges, where $k$
of them are long. (We count edges with multiplicities as in
Section~\ref{sec:the-main-theorem}, if the subregion is not a polygon.)
Combining two subregions of edge parameters $(n_1,k_1)$ and $(n_2,k_2)$
along a long edge $e$ gives a union with edge parameters
$(n_1+n_2-2,k_1+k_2-2)$, where we agree not to count the internal edge
$e$ that no longer bounds. Inequality~\ref{eqn:D-superadd} localizes the
main theorem to what is squandered by subclusters. Suppose we break the
standard cluster into groups of subregions such that if the group has
edge parameters $(n,k)$, it squanders at least $D(n,k)$. Then by
superadditivity (Sec.~\ref{x-4.5}, Formula~\ref{eqn:D-superadd}), the
full standard cluster $R$ must squander $D(n,0) = t_n$, $n=n(R)$, giving
the result.

Similarly, define constants $s_4=0$, $s_9 = -0.1972$, $s_{n}=0$, for
$n\ge10$.  Set $Z(n,k) = s_{n+k}-k\epsilon$, for $(n,k)\ne (3,1)$, and
$Z(3,1)=\epsilon$, where\footnote{Compare \calc{193836552}.} %A1
 $\epsilon=0.00005$. The function
$Z(n,k)$ is subadditive:
    $$Z(n_1,k_1)+Z(n_2,k_2) \le Z(n_1+n_2-2,k_1+k_2-2).$$
In fact, this easily follows from $s_a+s_b\le s_{a+b-4}$, for $a,b\ge
4$, and $\epsilon>0$. It will be enough in the proof of
Theorem~\ref{thm:the-main-theorem} to show that the score of a union of
subregions with edge parameters $(n,k)$ is at most $Z(n,k)$.


\subsection{Preparation of the standard cluster} %DCG 12.6, p 130
   \label{sec:prep-cluster}
    \oldlabel{4.7}

Fix a standard cluster.  We return to the construction of
subregions and distinguished edges, to describe the penalties.
Take the penalty of $0.008$ for each $3$-unconfined upright
diagonal. Take the penalty $0.03344 = 3\xiG+\xikG$ for $4$-crowded
upright diagonals. Take the penalty $0.04683=3\xiG$ for
$3$-crowded upright diagonals. Set $\maxpi=0.06688$. The penalty
in the next lemma refers to the combined penalty from erasing all
$3$-unconfined, $3$-crowded, and $4$-crowded upright diagonals in
the centered packing. The upright quarters that completely
surround an upright diagonal (loops) are not erased.

\begin{lemma}
The total penalty from a contravening centered packing is at most
$\maxpi$.
\end{lemma}

\begin{proof}
Before any upright quarters are erased, each quarter
squanders\footnote{\calc{148776243}} %A13
$>0.033$, so the centered packing squanders $>\squander$ if there
are $\ge25$ quarters.  Assume there are at most $24$ quarters. If
the only penalties are $0.008$, we have $8(0.008)<\maxpi$. If we
have the penalty $0.04683$, there are at most seven other quarters
($0.5606+8(0.033)>\squander$) (Lemma~\ref{x-3.7}), and no other
penalties from this type or from $4$-crowded upright diagonals, so
the total penalty is at most $2(0.008)+ 0.04683 < \maxpi$.
Finally, if there is one $4$-crowded upright diagonal, there are
at most twelve other quarters (Section~\ref{x-3.8}), and erasing
gives the penalty $0.03344+4 (0.008)<\maxpi$.
\end{proof}

The remaining upright diagonals are surrounded by slices. If
the edge opposite the diagonal in a slice has length
$\ge2\sqrt2$, then there may be an adjacent special simplex whose
diagonal is that edge.  Section~\ref{x-5.11} will give bounds on the
aggregate of these slices and special simplices.  In all
other contexts, the upright quarters have been erased with penalties.

Break the standard cluster into subclusters as in Section~\ref{x-4.2}.
If the subregion is a triangle, we refer to the bounds of \ref{x-5.7}.
Sections~\ref{x-4.8}--\ref{x-5.10} give bounds for subregions that are
not triangles in which all the upright quarters have been erased. We
follow the strategy outlined in Section~\ref{x-4.5}, although the
penalties will add certain complications.

We now assume that we have a subcluster without quarters and whose
region is not triangular.  The truncated function $\op{svR}_0$ is an
upper bound on the score.  Penalties are largely disregarded until
Section~\ref{x-5.4}.

We describe a series of deformations of the subcluster that
increase $\op{svR}(v,P,\Lambda,t_0)$ and decrease $\tau_{0,P}(\Lambda,v)$.  These
deformations disregard the broader geometric context of the
subcluster. Consequently, we cannot claim that the deformed
subcluster exists in any centered packing $(\Lambda,v)$.  As the deformation
progresses, an edge $\{v_1,v_2\}$, not previously distinguished,
can emerge with the properties of a distinguished edge. If so, we
add it to the collection of distinguished edges, use it if
possible to divide the subcluster into smaller subclusters, and
continue to deform the smaller pieces.  When triangular regions
are obtained, they are set aside until Section~\ref{x-5.7}.

\subsection{Reduction to polygons} %DCG 12.7, p131
    \oldlabel{4.8}
    % text moved to hypermaap and geometry chapter.


%% MOVED DCG 12.7 (visibility) to hypermap.

%% MOVED DCG 12.8 TO reduction.tex

%% MOVED DCG 12.9, DCG 12.10. pp.134--136.

\begin{definition}\index{penalty-free}\index{penalty-inclusive}
(By {\it penalty-free\/} score, we mean the part of the scoring
bound that does not include any of the penalty terms.  We will
sometimes call the full score, including the penalty terms, the
{\it penalty-inclusive\/} score.)
\end{definition}

Lemma~\ref{x-4.3} was stated in the context of a subregion before
deformation, but a cursory inspection of the proof shows that the
geometric conditions required for the proof remain valid by our
deformations. (This assumes that the subregion is not a triangle, which
we assumed at the beginning of Section~\ref{x-4.10}.) In more detail,
there is a solid $CP_0$ contained in the ball of radius of $t_0$ at the
origin, and lying over the cone of the subregion $P$ such that a bound
on the penalty-free subcluster score is 
$\op{svR}^g_0(CP_0)$ and squander $\tau^g_0(CP_0)$. 
XX WHAT ARE THESE?


Let $\{y_1,\ldots,y_r\}$ be a decomposition of the subregion into
disjoint regions whose union is $X$. Then if we let $CP_0(y_i)$ denote
the intersection of $CP_0(y_i)$ with the cone over $y_i$, we can write
    $$\tau^g_0(CP_0) =\sum_i \tau^g_0(CP_0(y_i)).$$

These lemmas allow us to express bounds on the score (and
squander) of a subcluster as a sum of terms associated with
individual (truncated) corner cells. By Lemmas~\ref{XX} %% \ref{x-4.12.1}
through \ref{lemma:4.12.5}, these objects do not meet at interior
points under suitable conditions. Moreover, by the interpretation
of terms provided by Section~\ref{x-4.3}, the cones over these
objects do not meet at interior points, when the objects
themselves do not. In other words, under the various conditions,
we can take the (truncated) corner cells to be among the sets
$CP_0(y_i)$.

To work a typical example, let us place a truncated corner cell with
parameter $\lambda=1.6$ at each concave corner.  Place a $t_0$-cone
wedge $X_0$ at each convex corner. The cone over each object lies in the
cone over the subregion. By Lemma~\ref{lemma:no-cross} and
Lemma~\ref{lemma:tau-positive} (see the proof), the $t_0$-cone wedge
$X_0$ squanders a positive amount.  The part $P'$ of the subregion
outside all truncated corner cells and outside the $t_0$-cone wedges
squanders
    $$\sol(P')(\zeta\pt-\phi_0) > 0.$$
where $\sol(P')$ is the part of the solid angle of the subregion
lying outside the tccs. Dropping these positive terms, we obtain a
lower bound on the penalty-free squander:
    $$\tau^g_0(CP_0) \ge \sum_{C_0} \tau^g_0(C_0).$$
There is one summand for each concave corner of the subregion.
Other cases proceed similarly.


\subsection{Convexity} %DCG 12.12, p139
    \oldlabel{4.13}

\begin{lemma}
    \oldlabel{4.13.1}
There are at most two concave corners.
\end{lemma}

\begin{proof}
Use the parameter $\lambda=1.6$ and place a truncated corner cell $C_0$
at each concave corner $v$. Let $C_0^u(|v-v_0|,\dih)$ denote the
corresponding untruncated cell.  By Lemma~\ref{lemma:tcc-est},
$\tau_0(C_0) > 0.297$ for each corner cell at a concave corner.
If there are three or more concave corners, then the penalty-free corner
cells squander at least $3(0.297)$. The penalty is at most $\maxpi$
(Section~\ref{x-4.7}). So the penalty-inclusive squander is more than
    $3(0.297) - \maxpi >\squander$.
\end{proof}

\begin{lemma}
    \oldlabel{4.13.2}
There are no concave corners of height at most $2.2$.
\end{lemma}

\begin{proof}
Suppose there is a corner of height at most $2.2$. Place an
untruncated corner cell $C^u_0(|v-v_0|,\dih)$ with parameter $\lambda
=1.815$ at that corner and a $t_0$-cone wedge at every other corner. 
The
subcluster squanders at least
    $\tau_0(C^u_0(|v-v_0|,\pi))-\maxpi$.
By Lemma~\ref{lemma:tcc-1815}, this is at least $\squander$.
\end{proof}

By the assumptions at the beginning of Section~\ref{x-4.10}, the lemma
implies that each concave corner has distance at least $3.2$ from every
other visible corner.

If there are two concave corners, we may put a corner cell at
each corner, with parameter $\lambda=1.945$.  By Lemma~\ref{lemma:2tcc},
they give $\tau > \squander + \maxpi$.
We conclude that there is at most one concave corner. 

Let $v$ be such a
corner.   
If we push $v$ toward the origin, the solid angle is unchanged
and $\op{svR}_0$ is increased.  Following this by the deformation of
Section~\ref{x-4.9}, we maintain the constraints $|v-w|=3.2$, for
adjacent corners $w$, while moving $v$ toward the origin. Eventually
$|v-v_0|=2.2$. This is impossible by Lemma~\ref{x-4.13.2}.

We verify that this deformation preserves the constraint
$|v-w|\ge2$, for all corners $w$ such that $\{v,w\}$ lies entirely
outside the subregion.  
By Lemma~\ref{lemma:details}, every other vertex $w$
is visible to $v$.  Thus, if a length drops below $2\sqrt2$ we may
add a distinguished edge and continue recursively.

We conclude that all subregions can be deformed into convex polygons.





\section{Proof of main theorem for Convex Polygons}%DCG Sec.13,p.143.
    \oldlabel{5}

\subsection{Deformations} %DCG 13.1, p143
    \oldlabel{5.1}
We divide the bounding edges over the polygon according to length
$[2,2t_0]$, $[2t_0,2\sqrt{2}]$, $[2\sqrt{2},3.2]$. The deformations of
Section~\ref{x-4.9} contract edges to the lower bound of the intervals
($2,2t_0$, or $2\sqrt{2}$) unless a new distinguished edge is formed. By
deforming the polygon, we assume that the bounding edges have length
$2,2t_0$, or $2\sqrt{2}$. (There are a few instances of triangles or
quadrilaterals that do not satisfy the hypotheses needed for the
deformations. These instances will be treated in Sections~\ref{x-5.7}
and \ref{x-5.8}.)

\begin{lemma}
    \oldlabel{5.1.1}
Let $S=S(y_1,\ldots,y_6)$ be a simplex, with $x_i=y_i^2$,
as usual.  Let $y_4\ge 2$,
    $y_5,y_6\in\{2,2t_0,2\sqrt{2}\}$.
Assume that the coordinates $y_1,\ldots,y_6$ are realized by
some simplex $\{v_1,\ldots,v_4\}$.  % Avoid mention of Deelta. 
Fixing all the variables but $x_1$, let $f(x_1)$ be one of the
functions $\op{sovo}_0(v_0,S,\lambda_{oct})$ or 
$-\op{sovo}_0(v_0,S,\lambda_{sq})$. We have $f''(x_1)>0$
whenever $f'(x_1)=0$.
\end{lemma}

\begin{proof} This is an interval calculation.%
\footnote{\calc{311189443}} %A15
% We put the condition about realization to avoid mention of Deelta.
\end{proof}

XX Notational problem in following paragraph. $v_0$ as origin,
% v_0 as non origin, 0 as origin.

The lemma implies that $f$ does not have an interior point local maximum
for $x_1\in[2^2,2t_0^2]$.  Fix three consecutive corners, $v_0,v_1,v_2$
of the convex polygon, and apply the lemma to the variable $x_1 =
|v_1-v_0|^2$ of the simplex $S=\{0,v_0,v_1,v_2\}$. We deform the simplex,
increasing $f$.  If the deformation produces flattens out the simplex, 
so that some
dihedral angle is $\pi$, then the arguments for nonconvex regions bring
us eventually back to the convex situation. Eventually $y_1$ is $2$ or
$2t_0$.  Applying the lemma at each corner, we may assume that the
height of every corner is $2$ or $2t_0$.   (There are a few cases where
the hypotheses of the lemma are not met, and these are discussed in
Sections~\ref{x-5.7} and \ref{x-5.8}.)


\subsection{Truncated corner cells} %DCG 13.2, p143
    \oldlabel{5.2}

The following lemma justifies using tccs at the corners as an upper
bound on the score (and lower bound on what is squandered). We fix the
truncation parameter at $\lambda=1.6$.

\begin{lemma}
Take a convex subregion that is not a triangle.  Assume edges between
adjacent corners have lengths $\in\{2,2t_0,2\sqrt{2},3.2\}$. Assume
nonadjacent corners are separated by distances $\ge3.2$.  Then the
truncated corner cell at each vertex lies in the cone over the
subregion.
\end{lemma}

\begin{proof}
Place a tcc at $v_1$. For a contradiction, let $\{v_2,v_3\}$ be an
edge that the tcc meets at an interior point.  
If $|v_1-v_i|\ge 2t_0$, then the result follows from
Lemma~\ref{tarski:beta:dcg-144a}.  

Assume that
 $|v_1-v_2|<2t_0$.  By the hypotheses of the lemma,
$|v_1-v_2|=2$.  If $|v_1-v_3|<3.2$, then $\{0,v_1,v_2,v_3\}$ is
triangular, contrary to hypothesis.  So $|v_1-v_3|\ge3.2$.
The result now follows from Lemma~\ref{tarski:beta:dcg-144b} when
$y_1\ge 2.2$ and from Lemma~\ref{tarski:beta:dcg-144c} when
$y_1\le 2.2$.
\end{proof}


\subsection{Penalties} %DCG 13.4, p145
    \oldlabel{5.4}
    \label{sec:penalty1}

In Section~\ref{x-4.7}, we determined the bound of
$\maxpi=0.06688$ on penalties. In this section, we give a more
thorough treatment of penalties. Until now a penalty has been
associated with a given standard region, but by taking the worst
case on each subregion, we can move the penalties to the level of
subregions.   Roughly, each subregion should incur the penalties
from the upright quarters that were erased along edges of that
subregion.  Each upright quarter of the original standard region
is attached at an edge between adjacent corners of the standard
cluster. The edges have lengths between $2$ and $2t_0$.  The
deformations shrink the edges to length $2$.  We attach the
penalty from the upright quarter to this edge of this subregion.
In general, we divide the penalty evenly among the upright
quarters along a common diagonal, without trying to determine a
more detailed accounting. For example, the penalty $0.008$ in
Lemma~\ref{lemma:0.008} comes from three upright quarters.  Thus,
we give each of three edges a penalty of $0.008/3$. Or, if there
are only two upright quarters around the $3$-unconfined upright
diagonal, then each of the two upright quarters is assigned the
penalty $0.00222/2$ (see Lemma~\ref{x-3.9.2}).

The penalty $0.04683 = 3\xiG$ in Section~\ref{x-4.7} comes from
three upright quarters around a $3$-crowded upright diagonal. Each
of three edges is assigned a penalty of $\xiG$.  The penalty
$0.03344=3\xiG+\xikG$ comes from a $4$-crowded upright diagonal of
Section~\ref{x-3.8}. It is divided among $4$ edges. These are the
only upright quarters that take a penalty when erased. (The case
of two upright quarters over a flat quarter as in
Lemma~\ref{lemma:unerased}, are treated by a separate argument in
Section~\ref{x-5.7}. Loops will be discussed in
Section~\ref{x-5.11}.)

The penalty can be reduced in various situations involving a
masked flat quarter.  For example, around a $3$-crowded upright
diagonal, if there is a masked flat quarter, two of the upright
quarters are scored by the analytic  function $\op{svan}$, so that the
penalty plus adjustment is only%
\footnote{\calc{73974037}} %A10
\footnote{\calc{764978100}} %A11
 $0.034052=2\xiV+\xiG+0.0114$.
The adjustment $0.0114$ reflects the scoring
rules for masked flat quarters (Lemma~\ref{lemma:0.008}).  This we
divide evenly among the three edges that carried the upright
quarters. If $e$ is an edge of the subregion $R$, let $\pi_0(R,e)$
denote the penalty and score adjustment along edge $e$ of $R$.

In summary, we have the penalties,
    $$\xik,\xiV,\xiG,\ 0.008,$$
combined in various ways in the upright diagonals that are
$3$-unconfined, $3$-crowded, or $4$-crowded.  There are score
adjustments
    $$0.0114\quad \text{ and }\quad 0.0063$$
from Section~\ref{x-3.10} for masked flat quarters.  If the sum of these
contributions is $s$, we set $\pi_0(R,e)=s/n$, for each edge $e$ of $R$
originating from an erased upright quarter of
    $\mathcal{\mathbf S}_n^\pm$.

\subsection{Penalties and Bounds} %DCG 13.5, p146
    \oldlabel{5.5}

Recall that the bounds for flat quarters we wish to establish from
Section~\ref{x-4.5} are $Z(3,1)=0.00005$ and $D(3,1)=0.06585$. Flat
quarters arise in two different ways.  Some flat quarters are present
before the deformations begin.  They are scored by the rules of
Section~\ref{x-3.10}. Others are formed by the deformations.  In this
case, they are scored by $\op{svR}_0$. Since the flat quarter is broken away
from the subregion as soon as the diagonal reaches $2\sqrt{2}$, and then
is not deformed further, the diagonal is fixed at $2\sqrt{2}$.  Such
flat quarters can violate our desired inequalities. For example,
    $$
    Z(3,1)<\op{sovo}_0(S(2,2,2,2\sqrt{2},2,2),\lambda_{oct}) 
      \approx 0.00898,\quad
        \op{sovo}_0(S(2,2,2,2\sqrt{2},2,2),\lambda_{sq})\approx 0.0593.
    $$
On the other hand, as we will see, the adjacent subregion satisfies the
inequality by a comfortable margin.  Therefore, we define a transfer
$\epsilon$ from flat quarters to the adjacent subregion. (In an
exceptional region, the subregion next to a flat quarter along the
diagonal is not a flat quarter.)

For a flat quarter $Q$, set
    $$
    \epsilon_\tau(Q) =
        \begin{cases} 0.0066,&\text{(deformation),}\\
            0,&\text{(otherwise)}.
        \end{cases}
    $$
    %
    $$
    \epsilon_\sigma(Q) =
        \begin{cases}
         0.009,&\text{(deformation),}\\
            0,&\text{(otherwise)}.
        \end{cases}
    $$
The nonzero value occurs when the flat quarter $Q$ is obtained by
deformation from an initial configuration in which $Q$ is not a quarter.
The value is zero when the flat quarter $Q$ appears already in the
undeformed standard cluster. Set
    $$
    \begin{array}{lll}
    \pi_\tau(R) &= \sum_e \pi_0(R,e) +
    \sum_e\pi_0(Q,e)+\sum_Q \epsilon_\tau(Q),\\
    \pi_\sigma(R)&=\sum_e \pi_0(R,e) +
    \sum_e\pi_0(Q,e)+\sum_Q \epsilon_\sigma(Q).
    \end{array}
    $$
The first sum runs over the edges of a subregion $R$.  The second sum
runs over the edges of the flat quarters $Q$ that lie adjacent to $R$
along the diagonal of $Q$.

The edges between corners of the polygon have lengths $2$, $2t_0$, or
$2\sqrt{2}$.  Let $k_0$, $k_1$, and $k_2$ be the number of edges of
these three lengths respectively.  By Lemma~\ref{lemma:7-sides}, we have
$k_0+k_1+k_2\le7$. Let $\tilde\sigma$ denote any of the functions of
Section~\ref{x-3.10}.2(a)--(f). Let $\tilde\tau = \sol\zeta\pt -
\tilde\sigma$.

To prove Theorem~\ref{thm:the-main-theorem}, refining the strategy
proposed in Section~\ref{x-4.5}, we must show that for each flat quarter
$Q$ and each subregion $R$ that is not a flat quarter, we have
    \begin{equation}
    \begin{array}{lll}
    \tilde\tau(Q) &> D(3,1) - \epsilon_\tau(Q),\\
    \tau_0(Q) &> D(3,1)-\epsilon_\tau(Q),\quad\text{if }y_4(Q)=2\sqrt2,\\
    \tau_V(R) &> D(3,2),\quad\text{(type $\SA$)},\\
    \tau_0(R) &> D(k_0+k_1+k_2,k_1+k_2)+\pi_\tau(R),
    %oldtag 5.5.1
    \label{eqn:tau>D(n,k)}
    \end{array}
    \end{equation}
where $D(n,k)$ is the function defined in Section~\ref{x-4.5}. The first
of these inequalities follows.%
% from $\A_{1},\A_{13},\A_{16}$.
\footnote{\calc{193836552}} %A1
\footnote{\calc{148776243}} %A13
\footnote{\calc{163548682}} %A16
In general,
we are given a subregion without explicit information about what the
adjacent subregions are.  Similarly, we have discarded all information
about what upright quarters have been erased.  Because of this, we
assume the worst, and use the largest feasible values of $\pi_\tau$.

\begin{lemma}
We have
    $\pi_\tau(R)\le 0.04683 + (k_0+2k_2-3)0.008/3 +0.0066k_2$.
\end{lemma}

\begin{proof}
The worst penalty $0.04683=3\xiG$ per edge comes from a
$3$-crowded upright diagonal. The number of penalized edges not on
a simplex around a $3$-crowded upright diagonal is at most
$k_0+2k_2-3$. For every three edges, we might have one
$3$-unconfined upright diagonal. The other cases such as
$4$-crowded upright diagonals or situations with a masked flat
quarter are readily seen to give smaller penalties.
\end{proof}

For bounds on the score, the situation is similar.  The only
penalties we need to consider are $0.008$ from
Lemma~\ref{lemma:0.008}. If either of the other configurations of
$3$-crowded or $4$-crowded upright diagonals occur, then the score
of the standard cluster is less than $s_8=-0.228$, by
Sections~\ref{x-3.7} and \ref{x-3.8}. This is the desired bound.
So it is enough to consider subregions that do not have these
upright configurations. Moreover, the penalty $0.008$ does not
occur in connection with masked flats. So we can take
$\pi_\sigma(R)$ to be
    $$(k_0+2k_2)0.008/3 + 0.009 k_2.$$
If $k_0+2k_2<3$, we can strengthen this to
    $\pi_\sigma(R)=0.009 k_2$.
Let $\tilde\sigma$ be any of the functions of Section~\ref{x-3.10}.2
parts (a)--(f). To prove Theorem~\ref{thm:the-main-theorem}, we will
show
    \begin{equation}
    \begin{array}{lll}
    \tilde\sigma(Q)&< Z(3,1)+\epsilon_\sigma(Q),\\
    \op{sovo}_0(Q,\lambda_{oct})
    &< Z(3,1)+\epsilon_\sigma(Q),\quad\text{if }y_4(Q)=2\sqrt2,\\
    \op{svR}_0(R)&< Z(3,2),\quad\text{(type $\SA$)},\\
    \op{svR}_0(R) &< Z(k_0+k_1+k_2,k_1+k_2) - \pi_\sigma(R).
    %oldtag 5.5.2
    \label{eqn:sigma<Z(n,k)}
    \end{array}
    \end{equation}
The first of these inequalities follows.%
\footnote{\calc{193836552}} %A1
\footnote{\calc{148776243}} %A13
\footnote{\calc{163548682}} %A16
% form $\A_1,\A_{13},\A_{16}$.


\subsection{Penalties} %DCG 13.6, p148
\label{sec:4.2} \label{sec:penalty}

Erasing a compressed upright quarter gives a penalty of
at most $\xiG$ and a decompressed one gives at most $\xiV$. We
take the worst possible penalty.  It is at most $n\xiG$ in an
$n$-gon. If there is a masked flat quarter, the penalty is at most
$2\xi_V$ from the two upright quarters along the flat quarter.  We
note in this connection that both edges of a polygon along a flat
quarter lie on upright quarters, or neither does.

If an upright diagonal appears enclosed over a flat quarter, the
flat quarter is part of a loop with context $(n,k)=(4,1)$, for a
penalty at most $2\xi'_\Gamma+\xi_V$.  This is smaller than the
bound on the penalty obtained from a loop with context
$(n,k)=(4,1)$, when the upright diagonal is not enclosed over the
flat quarter:
    $$\xi_\Gamma + 2\xi_V.$$
So we calculate the worst-case penalties under the assumption that
the upright diagonals are not enclosed over flat quarters.

A loop of context $(n,k)=(4,1)$ gives $\xi_\Gamma+2\xi_V$ or
$3\xi_\Gamma$.  A loop of context $(n,k)=(4,2)$ gives
$2\xi_\Gamma$ or $2\xi_V$.

If we erase a $3$-unconfined upright diagonal, there is a penalty
of $0.008$ (or 0 if it masks a flat quarter.) This is dominated by
the penalty $3\xi_\Gamma$ of context $(n,k)=(4,1)$.

Suppose we have an octagonal standard region.  We claim that a loop
does not occur in context $(n,k)=(4,2)$. If there are at most three
vertices that are not corners of the octagon, then there are at most
twelve quasi-regular tetrahedra, and the score is at most
$$s_8 + 12\,\pt<8\,\pt.$$
Assume there are more than three vertices that are not corners
over the octagon. We squander
$$t_8+ \dloop(4,2)+4\tlp(5,0) > \squander.$$
As a consequence, context $(n,k)=(4,2)$ does not occur.

So there are at most two upright diagonals and at most six quarters,
and the penalty is at most $6\xi_\Gamma$. Let $f$ be the number of
flat quarters This leads to
    $$
    \piF = \begin{cases} 6\xiG, & f=0,1,\\
                   4\xiG+2\xiV, & f=2,\\
                    2\xiG+4\xiV, & f=3,\\
                    0, & f=4.
            \end{cases}
    $$
The 0 is justified by a parity argument.  Each upright quarter
occurs in a pair at each masked flat quarter.  But there is an odd
number of quarters along the upright diagonal, so no penalty at
all can occur.

Suppose we have a heptagonal standard region.  Three loops are a
geometric impossibility. Assume there are at most two upright
diagonals.
 If there is no context $(n,k)=(4,2)$,
 then we have the following bounds on the penalty
    $$
    \piF = \begin{cases} 6\xiG, & f=0,\\
                 4\xiG+2\xiV, & f=1,\\
                3\xiG, & f=2,\\
                \xiG+2\xiV, & f=3.
            \end{cases}
    $$
If an upright diagonal has context $(n,k)=(4,2)$, then
    $$
    \piF = \begin{cases} 5\xiG, & f=0,1,\\
                3\xiG + 2\xiV, & f=2,\\
                \xiG + 4 \xiV, &f = 3.\\
            \end{cases}
    $$
This gives the bounds used in the diagrams of cases.



\subsection{Constants} %DCG 13.7, p149
    \oldlabel{5.6}

Theorem~\ref{thm:the-main-theorem} now results from the calculation of a
host of constants. Perhaps there are simpler ways to do it, but it was a
routine matter to run through the long list of constants by computer.
What must be checked is that the Inequalities~\ref{eqn:tau>D(n,k)}
and~\ref{eqn:sigma<Z(n,k)} of Section~\ref{x-5.5} hold for all possible
convex subregions. Call these inequalities the {\it D} and {\it Z}
inequalities.  This section describes in detail the constants to check.

We begin with a subregion given as a convex $n$-gon, with at least
four sides.   The heights of the corners and the lengths of edges
between adjacent edges have been reduced by deformation to a finite
number of possibilities (lengths $2$, $2t_0$, or lengths $2$,
$2t_0$, $2\sqrt{2}$, respectively). By Lemma~\ref{lemma:7-sides}, we
may take $n=4,5,6,7$. Not all possible assignments of lengths
correspond to a geometrically viable configuration. One constraint
that eliminates many possibilities, especially heptagons, is that of
Section~\ref{x-5.1}: the perimeter of the convex polygon is at most
a great circle.  Eliminate all length-combinations that do not
satisfy this condition.  When there is a special simplex it can be
broken
from the subregion and scored\footnote{\calc{148776243}} %A13
separately unless the two heights along the diagonal are $2$.
We assume in all that follows that all specials that can be
broken off have been. There is a second condition related to special
simplices.  By Lemma~\ref{tarski:311}, if
$(y_1,y_2,y_3,y_5,y_6)= (2t_0,2,2,2,2)$, then
the simplex must be special
($y_4\in[2\sqrt{2},3.2]$).


The easiest cases to check are those with no special simplices over the
polygon.  In other words, these are subregions for which the distances
between nonadjacent corners are at least $3.2$.  In this case we
approximate the score (and what is squandered) by tccs at the corners.
We use monotonicity to bring the fourth edge to length $3.2$. We
calculate the tcc constant bounding the score, checking that it is less
than the constant
    $ Z(k_0+k_1+k_2,k_1+k_2) - \pi_\sigma$,
from the Z inequalities. The D inequalities  are verified in the same
way.

When $n=5,6,7,$ and there is one special simplex, the situation is not
much more difficult.  By our deformations,  we decrease the lengths of
edges $2,3,5,6$ of the special to 2. We remove the special by cutting
along its fourth edge $e$ (the diagonal).  We score the special with
weak bounds.\footnote{\calc{148776243}} %A13  found in $\A_{13}$.
Along the edge $e$, we then apply deformations to the $(n-1)$-gon
that remains. If this deformation brings $e$ to length
$2\sqrt{2}$, then the $(n-1)$-gon may be scored with tccs as in
the previous paragraph.  But there are other possibilities. Before
$e$ drops to $2\sqrt{2}$, a new distinguished edge of length $3.2$
may form between two corners (one of the corners will be a chosen
endpoint of $e$).  The subregion breaks in two. By deformations,
we eventually arrive at $e=2\sqrt2$ and a subregion with diagonals
of length at least $3.2$.  (There is one case that may fail to be
deformable to $e=2\sqrt2$, a pentagonal cases discussed further in
Section~\ref{x-5.9}.) The process terminates because the number of
sides to the polygon drops at every step. A simple recursive
computer procedure runs through all possible ways the subregion
might break into pieces and checks that the tcc-bound gives the
$D$ and $Z$ inequalities. The same argument works if there is a
special simplex that meets at an interior point with each of the
other special simplices in the subcluster.

When $n=6,7$ and there are two nonoverlapping special simplices, a
similar argument can be applied. Remove both specials by cutting along
the diagonals. Then deform both diagonals to length $2\sqrt{2}$, taking
into account the possible ways that the subregion can break into pieces
in the process.  In every case the $D$ and $Z$ inequalities are
satisfied.

There are a number of situations that arise that escape this generic
argument and were analyzed individually. These include the cases
involving more than two special simplices over a given subregion, two
special simplices over a pentagon, or a special simplex over a
quadrilateral.  Also, the deformation lemmas are insufficient to bring
all of the edges between adjacent corners to one of the three standard
lengths $2,2t_0,2\sqrt{2}$ for certain triangular and quadrilateral
regions.  These are treated individually.

The next few sections describe the cases treated individually. The cases
not mentioned in the sections that follow fall within the generic
procedure just described.

\subsection{Triangles} %DCG 13.8, p151
    \oldlabel{5.7}

With triangular subregions, there is no need to use any of the
deformation arguments because the dimension is already sufficiently
small to apply interval arithmetic directly to obtain our bounds.
There is no need for the tcc-bound approximations.

Flat quarters and simplices of type $\SA$ are treated by a computer
calculation.\footnote{\calc{163548682}} %A16
Other simplices are scored by the truncated function
$\op{sovo}_0(\cdot,\lambda_{oct})$. We break the edges between corners into the cases
    %((
    $[2,2t_0)$, $[2t_0,2\sqrt{2})$, $[2\sqrt{2},3.2]$.
    %]]
Let $k_0$, $k_1$, and $k_2$, with $k_0+k_1+k_2=3$, be the number
of edges  in the respective intervals.

If $k_2=0$, we can improve the penalties,
    $$\pi_\tau = \pi_\sigma=0.$$
To see this, first we observe that there can be no $3$-crowded or
$4$-crowded upright diagonals. By placing $\ge3$ quarters around
an upright diagonal, if the subregion is triangular, the upright
diagonal becomes surrounded by slices, a case deferred
until Section~\ref{x-5.11}.

If $k_0=k_1=k_2=1$, we can take $\pi'_\tau=
\xiG+2\xiV+0.0114=0.034052$. A few cases are needed to justify
this constant. If there are no $3$-crowded upright diagonals,
$\pi'_\tau$ is at most
    $$
    \begin{array}{lll}
    &[\xiG + 2 \xiV +\xikG ]3/4 < 0.0254,\\
    \hbox{or\quad }&[\xiG+2\xiV+\xikG]2/4 + 0.008/3 < 0.0254
    \end{array}
    $$
If there are at most two edges in the subregion coming from an
$3$-crowded upright diagonal,
    $$(\xiG+2\xiV+0.0114)2/3 + 0.008/3 < 0.0254.$$
If three edges come from the simplices of a $3$-crowded upright
diagonal, we get $0.034052$. To get somewhat sharper bounds, we
consider how the edge $k_2$ was formed.  If it is obtained by
deformation from an edge in the standard region of length
$\ge3.2$, then it becomes a distinguished edge when the length
drops to $3.2$.  If the edge in the standard region already has
length $\le3.2$, then it is distinguished before the deformation
process begins, so that the subregion can be treated in isolation
from the other subregions. We conclude that when
$\pi'_\tau=0.034052$ we can take $y_4\ge2.6$ or $y_5=3.2$
(Lemma~\ref{tarski:last:E}).

The $D$ and $Z$ inequalities now follow.%
\footnote{\calc{852270725}} %A17
\footnote{\calc{819209129}} %A18
% from $\A_{17}$ and $\A_{18}$.

\subsection{Quadrilaterals} %DCG 13.9, p152
    \oldlabel{5.8}

We introduce some notation for the heights and edge lengths of a convex
polygon.  The heights will generally be $2$ or $2t_0$, the edge lengths
between consecutive corners will generally be $2$, $2t_0$, or
$2\sqrt{2}$.  We represent the edge lengths by a vector
    $$(a_1,b_1,a_2,b_2,\ldots,a_n,b_n),$$
if the corners of an $n$-gon, ordered cyclically have heights $a_i$ and
if the edge length between corner $i$ and $i+1$ is $b_i$.  We say two
vectors are equivalent if they are related by a different cyclic
ordering on the corners of the polygon, that is, by the action of the
dihedral group.

The vector of a polygon with a special simplex is equivalent to one of
the form
    $$(2,2,a_2,2,2,\ldots).$$  If $a_2=2t_0$, then what we have is
necessarily special (Section~\ref{x-5.6}). However, if $a_2=2$, it is
possible for the edge opposite $a_2$ to have length greater than $3.2$.


Turning to quadrilateral regions, we use tcc scoring if both diagonals
are greater than $3.2$.   Suppose that both diagonals are between
$[2\sqrt{2},3.2]$, creating a pair of overlapping special simplices. The
deformation lemma requires a diagonal longer than $3.2$, so although we
can bring the quadrilateral to the form
    $$(a_1,2,2,2,2,2,a_4,b_4),$$
the edges $a_1,a_4,b_4$ and the diagonal vary%
\footnote{\calc{148776243}} %A13  (see $\A_{13}$).
continuously.
We have bounds\footnote{\calc{128523606}} %A19
 on the score
    $$
    \begin{array}{lll}
    \tau_0 &> 0.235, \quad \op{svR}_0 < -0.075,
                \hbox{ if } b_4\in[2t_0,2\sqrt{2}],\\
    \tau_0 &> 0.3109, \quad \op{svR}_0 < -0.137,
                \hbox{ if } b_4\in[2\sqrt{2},3.2],\\
    \end{array}
    $$
We have $D(4,1)=0.2052$, $Z(4,1)=-0.05705$. When
$b_4\in[2t_0,2\sqrt{2}]$, we can take $\pi_\tau=\pi_\sigma=0$. (We are
excluding loops here.) When $b_4\in[2\sqrt{2},3.2]$, we can take
    $$
    \begin{array}{lll}
    \pi_\tau &= \maxpi+ 0.0066, \\
    \pi_\sigma &= 0.008 (5/3)+ 0.009. \\
    \end{array}
    $$
It follows that the $D$ and $Z$ Inequalities are satisfied.

Suppose that one diagonal has length $[2\sqrt{2},3.2]$ and the other has
length at least $3.2$.  The quadrilateral is represented by the vector
    $$(2,2,a_2,2,2,b_3,a_4,b_4).$$
The hypotheses of the deformation lemma hold, so that $a_i\in\{2,2t_0\}$
and $b_j\in\{2,2t_0,2\sqrt2\}$. To avoid quad clusters, we assume
$b_4\ge\max(b_3,2t_0)$. These are one-dimensional with a diagonal of
length $[2\sqrt{2},3.2]$ as parameter.
 The required verifications\footnote{\calc{874876755}} %A20
have been made by interval arithmetic.
%appear in $\A_{20}$.


\subsection{Pentagons} %DCG 13.10, p153
    \oldlabel{5.9}

Some extra comments are needed when there is a special simplex. The
general argument outlined above removes the special, leaving a
quadrilateral.  The quadrilateral is deformed, bringing the edge that
was the diagonal of the special to $2\sqrt{2}$. This section discusses
how this argument might break down.

Suppose first that there is a special and that both diagonals on the
resulting quadrilateral are at least $3.2$.  We can deform using either
diagonal, keeping both diagonals at least $3.2$. The argument breaks
down if both diagonals drop to $3.2$ before the edge of the special
reaches $2\sqrt{2}$ and both diagonals of the quadrilateral lie on
specials. When this happens, the quadrilateral has the form
    $$(2,2,2,2,2,2,2,b_4),$$
where $b_4$ is the edge originally on the special simplex.  If both
diagonals are $3.2$, this is rigid, with $b_4= 3.12$. We find its score
to be
    $$
    \begin{array}{lll}
    &\op{sovo}_0(S(2,2,2,b_4,3.2,2),\lambda_{oct})+\op{sovo}_0(S(2,2,2,3.2,2,2),\lambda_{oct})+0.0461<-0.205,\\
    &\op{sovo}_0(S(2,2,2,b_4,3.2,2),\lambda_{sq})+\op{sovo}_0(S(2,2,2,3.2,2,2),\lambda_{sq})2> 0.4645.\\
    \end{array}
    $$
So the $D$ and $Z$ Inequalities hold easily.

If there is a special and there is a diagonal on resulting quadrilateral
$\le3.2$, we have two nonoverlapping specials.  It has the form
    $$(2,2,a_2,2,2,2,a_4,2,2,b_5).$$
The edges $a_2$ and $a_4$ lie on the special.  If $b_5>2$, cut
away one of the special simplices.  What is left can be reduced to
a triangle, or a quadrilateral case and then treated%
\footnote{\calc{874876755}} %A20
by computer. % in $\A_{20}$.
Assume
$b_5=2$.  We have a pentagonal standard region. We may assume that
there is no $3$-crowded or $4$-crowded upright diagonal, for
otherwise Theorem~\ref{thm:the-main-theorem} follows trivially
from the bounds in Section~\ref{x-2}. A pentagon can then have at
most a $3$-unconfined upright diagonal for a penalty of $0.008$.

If $a_2=2t_0$ or $a_4=2t_0$, we again remove a special simplex and
produce triangles, quadrilaterals, or the special cases treated
by computer.%
\footnote{\calc{874876755}} %A20
We may impose the condition $a_2=a_4=b_5=2$. We score this full pentagonal
arrangement by computer,\footnote{\calc{692155251}} %A21 $\A_{21}$,
using the edge lengths of the two diagonals of
the specials as variables. The inequalities follow.

\subsection{Hexagons and heptagons} %DCG 13.11, p154
    \oldlabel{5.10}

We turn to hexagons. There may be three specials whose diagonals do not
cross.  Such a subcluster is represented by the vector
    $$(2,2,a_2,2,2,2,a_4,2,2,2,a_6,2).$$
The heights $a_{2i}$ are $2$ or $2t_0$.  Draw the diagonals between
corners $1$, $3$, and $5$.  This is a three-dimensional configuration,
determined by the lengths of the three diagonals, which is treated
by computer.\footnote{\calc{692155251}} %A21
%The required bound follows from $\A_{21}$.

There is one case with a special simplex that
did not satisfy the generic computer-checked inequalities for
what is to be squandered.  Its vector is
    $$(a_1,2,2,2,2,2,2,b_4,2,2,2,2),$$
with $a_1=b_4=2t_0$. A vertex of the special simplex has height
$a_1=2t_0$ and all other corners have height $2$.  The subregion
is a hexagon with one edge longer than $2$.  We have $D(6,1)=
0.48414$. This is certainly obtained if the subregion contains a
$3$-crowded upright diagonal, squandering $0.5606$. But if this
configuration does not appear, we can decrease $\pi_\tau$ to
    $0.03344 + (2/3) 0.008$,
a constant coming from $4$-crowded upright diagonals in
Section~\ref{x-4.7}. With this smaller penalty the inequality is
satisfied.

Now turn to heptagons.The bound $2\pi$  on the perimeter of the polygon,
eliminates all but one equivalence class of vectors associated with a
polygon that has two or more potentially specials simplices. The vector
is
    $$(2,2,a_2,2,2,2,a_4,2,2,2,a_6,2,a_7,2),$$
$a_2=a_4=a_6=a_7=2t_0$. In other words, the edges between adjacent
corners are $2$ and four heights are $2t_0$. There are two specials.
This case is treated by the procedure outlined for subregions with two
specials whose diagonals do not cross.

\subsection{Loops} %DCG 13.12, p154
    \oldlabel{5.11}
    \label{sec:loops}

We now return to a collection of slices that surround
the upright diagonal.  This is the last case needed to complete
the proof of Theorem~\ref{x-4.3}. There are four or five 
slices around the upright diagonal.
There are linear inequalities%
\footnote{\calc{815492935}} %A2
\footnote{\calc{729988292}} %A3
\footnote{\calc{531888597}} %A4
\footnote{\calc{628964355}} %A5
\footnote{\calc{934150983}} %A6
\footnote{\calc{187932932}} %A7
%$\A_2$--$\A_7$ give a list
%of linear inequalities
satisfied by the slices, broken
up according to type: upright, type $\SC$, opposite edge $>3.2$,
etc. The slices are related by the constraint that the
sum of the dihedral angles around the upright diagonal is $2\pi$.
We run a linear program in each case based on these linear
inequalities, subject to this constraint to obtain bounds on the
score and what is squandered by the slices.

When the edge opposite the diagonal of a slice has length
$\in[2\sqrt{2},3.2]$ and the simplex adjacent to the slice
across that edge is a special simplex, we use inequalities%
\footnote{\calc{485049042}} %A22
\footnote{\calc{209361863}} %A23
% $\A_{22}$ and $\A_{23}$
that run parallel to the similar system\footnote{\calc{531888597}} %A4
\footnote{\calc{628964355}} %A5
%$\A_4$ and $\A_5$.
It is not
necessary to run separate linear programs for these.  It is enough to
observe that the constants for what is squandered improve on those from
the similar system\footnote{\calc{531888597}} %A4
%$\A_4$ by at least $0.06445$
and that the constants for the score in one system%
\footnote{\calc{485049042}} %A22
differ with those of the other%
\footnote{\calc{531888597}} %A4
by no more than $0.009$.

When the dihedral angle of a slice is greater than $2.46$,
the simplex is dropped, and the remaining slices are subject
to the constraint that their dihedral angles sum to at most $2\pi-2.46$.
There can not be a slice with dihedral angle greater than
$2.46$ when there are five darts: $2.46+4 (0.956)>2\pi$. There cannot%
\footnote{\calc{83777706}} %A8
be two slices with dihedral angle greater than $2.46$:
$2(2.46+0.956)>2\pi$.

The following table summarizes the linear programming results.

$$
\begin{matrix}
(n,k)   &   \DLP(n,k) & D(n,k)      &\ZLP(n,k)  &Z(n,k)\\
(4,0)   &   0.1362  &   0.1317  &   0   &   0\\
(4,1)   &   0.208   &   0.20528 &-0.0536&   -0.05709\\
(4,2)   &   0.3992  &   0.27886 &-0.2   &   -0.11418\\
(4,3)   &  0.6467   &   0.35244 &-0.424 &   -0.17127\\
(5,0)   &   0.3665  &   0.27113 &-0.157 &   -0.05704\\
(5,1)   &  0.5941   &   0.34471 &-0.376 &   -0.11413\\
(5,\ge2)&  0.9706   &  \squander    &*          &   *
\end{matrix}
$$

The bound for $D(4,0)$ comes from Lemma~\ref{lemma:0.1317}. A few
more comments are needed for $Z(4,1)$.  Let $S=S(y_1,\ldots,y_6)$
be the slice that is not a quarter.  If $y_4\ge2\sqrt2$
or $\dih(S)\ge 2.2$, the linear programming bound is $<Z(4,1)$.
With this, if $y_1\le 2.75$, we have\footnote{\calc{855294746}} %A12
    $\sigma(S) < Z(4,1)$.
But if $y_1\ge2.75$, the three upright quarters along the upright
diagonal satisfy
    $$\nu< -0.3429+0.24573\dih.$$
With this stronger inequality, the linear programming bound becomes
$<Z(4,1)$. This completes the proof of
Theorem~\ref{thm:the-main-theorem}.

\begin{lemma}\label{lemma:loop}
Consider an upright diagonal that is a loop.  Let $R$ be the
standard region that contains the upright diagonal and its
surrounding simplices.   Then the following contexts $(m,k)$ are the
only ones possible.  Moreover, the constants that appear in the
columns marked $\sigma$ and $\tau$ are upper and lower bounds
respectively for $\tau_R(\Lambda,v)$ when $R$ contains one loop of that
context.
    $$
    \begin{array}{llll}
        n=n(R)&(m,k) &\sigma &\tau \\
        &&&\\
        4& & &\\
        &(4,0) &-0.0536 & 0.1362 \\
        5 & & &\\
        &(4,1) &s_5 &0.27385\\
        &(5,0) &-0.157   &0.3665\\
        6 & & &\\
        &(4,1) &s_6 &0.41328\\
        &(4,2) &-0.1999  &0.5309\\
        &(5,1) &-0.37595 &0.65995\\
        7 & & &\\
        &(4,1) &s_7 &0.55271\\
        &(4,2) &-0.25694 &0.67033\\
        8 & & &\\
        &(4,1) &s_8 &0.60722\\
        &(4,2) &-0.31398 &0.72484
    \end{array}
    $$
\end{lemma}

\begin{proof} In context $(m,k)$, and if $n=n(R)$, we have
    $$
    \sigma_R(\Lambda,v) < s_{n} + \ZLP(m,k)-Z(m,k)\quad
    \tau_R(\Lambda,v)> t_{n}+\DLP(m,k)-D(m,k).
    $$
The result follows.
\end{proof}

In the context $(n,k)=(4,3)$, the standard region $R$ must have at
least seven sides $n(R)\ge7$.   Then
    $$
    \begin{array}{lll}
    \tau(\Lambda,v)&\ge t_7+\dloop(4,3)\\
            &>\squander.
    \end{array}
    $$
Thus, we may assume that this context does not occur.

If the context $(5,1)$ appears in an octagon, we have
    $$\tau(\Lambda,v) >\dloop(5,1)+t_8>\squander.$$
If this appears in a heptagon, we have
$$\tau(\Lambda,v) >\dloop(5,1) + t_7+ 0.55\,\pt > \squander,$$
because there must be a vertex that is not a corner of the
heptagon. It cannot appear on a pentagon.



