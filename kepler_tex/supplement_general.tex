% May 15, 2013.  Unsorted Appendices.




\newpage
\section{Appendix on deformations of local fans}\label{sec:sup-deformation}

This is an appendix to Section 7.2 of Dense Sphere Packings.
This appendix gives further details about the proof of the wedge property in  Lemma 7.25 (ZLZTHIC).

\begin{lemma} \guid{FSQKWKK}\label{lemma:rot}
If $\op{azim}(\orz,\v_1,\v_2,\v_3)\le \pi$ then $\op{azim}(\orz,\v_2,\v_3,\v_1)\le\pi$.
\end{lemma}

\begin{proof}  
$\azim(\orz,\v_1,\v_2,\v_3)\le\pi$  iff $0\le\sin(\azim(\orz,\v_1,\v_2,\v_3))$ iff
$0\le (\v_1\times \v_2)\cdot \v_3$.  The final condition is invariant under even permutations of
the subscripts.
\end{proof}

Recall that if we have $\op{azim}(\orz,\v_1,\v_2,\v_3)\le \pi$ and appropriate noncollinearity constraints, then
$\op{azim}(\orz,\v_1,\v_2,\v_3) = \op{dih}_V (\orz,\v_1,\v_2,\v_3)$.

\begin{lemma} \guid{MKIFWJT}\label{lemma:sum3-azim-fan}
Let $\v,\w_1,\w_2,\w_3\in\ring{R}^3$. Assume that none of the sets $\{\orz,\v,\w_i\}$ are collinear
and that $\op{azim}(\orz,\v,\w_1,\w_2)+\op{azim}(\orz,\v,\w_2,\w_3)< 2\pi$.  Then
\[
\op{azim}(\orz,\v,\w_1,\w_2)+\op{azim}(\orz,\v,\w_2,\w_3) = \azim(\orz,\v,\w_1,\w_3).
\]
\end{lemma}

\begin{proof} This is (Fan.sum3\_azim\_fan).
\end{proof}



\begin{lemma}[deform-wedge] \label{lemma:deform-wedge} \guid{XIVPHKS}
Let $\w_0,\ldots,\w_{n}\in \ring{R}^3$, for some $n\ge 1$.  Assume that $i\ne j$ implies
$\{\orz,\w_i,\w_j\}$ is not collinear.   Let $r=n-1$.
Set
\[
d(i,j,k)=\op{dih}_V(\orz,\w_i,\w_j,\w_k) \quad\text{and}\quad a(i,j,k) = \azim(\orz,\w_i,\w_j,\w_k).
\]
Introduce abbreviations for  wedges $W(i) = W(\orz,\w_i,\w_{i+1},\w_{i-1})$.
Let $\epsilon\in\ring{R}$ satisfy $0 < 2 \epsilon < a(i,i+1,i-1)$, for $i=1,\ldots,r-1$.
Assume $a(i,i+1,i-1) \le \pi$ for $i=1,\ldots,r$.
Assume that $d(p,q,q+1)<\epsilon$ for all sets of three distinct elements $\{p,q,q+1\}\subset \{0,\ldots,r\}$.
Assume that $d(p,p+1,q)<\epsilon$ for all sets of three distinct elements $\{p,p+1,q\}\subset \{0,\ldots,r\}$ with
$q > p+1$.
Assume that $d(p+1,p,q)<\epsilon$ for all sets of three distinct elements $\{p+1,p,q\}\subset \{0,\ldots,r\}$ with
$q < p$. 

Then for all $k$ we have the statement $S_k$: 
for all $j\le r - k$ we have $\w_j\in W({j+k})$ and $\w_{j+k}\in W(j)$.
\end{lemma}

\begin{proof} We prove $S_k$ by induction on $k$.  The cases $k=0,1$ are trivially satisfied.
Assume that $S_k$, for some $k\ge1$.  We show $S_{k+1}$.  Fix $j\le r-(k+1)$.

By induction, $\w_j\in W({j+k})$, giving
\[
a(j+k,j+k+1,j)\le a(j+k,j+k+1,j+k-1)\le\pi.
\]
By Lemma~\ref{lemma:rot}, $a(j,j+k,j+k+1)\le\pi$, and converting to dihedral, $a(j,j+k,j+k+1)< \epsilon$.

By induction, $\w_{j+k}\in W(j)$, giving
\[
a(j,j+1,j+k)\le a(j,j+1,j-1)\le \pi,
\]
and converting to dihedral, $a(j,j+1,j+k)<\epsilon$.

Since $2\epsilon < 2\pi$, we can add the angles (Lemma~\ref{lemma:sum3-azim-fan}),
\[
a(j,j+1,j+k) + a(j,j+k,j+k+1) = a(j,j+1,j+k+1) < 2\epsilon < a(j,j+1,j-1).
\]
This says that $\w_{j+k+1}\in W(j)$.

The proof that $\w_j\in W({j+k+1})$ is similar.

By induction, $\w_{j+k+1}\in W({j+1})$, giving
\[
a(j+1,j+k+1,j)\le a(j+1,j+2,j)\le \pi.
\]
By Lemma~\ref{lemma:rot}, $a(j+k+1,j,j+1)\le \pi$, and converting to dihedral $a(j+k+1,j,j+1)<\epsilon$.

By induction, $\w_{j+1}\in W({j+k+1})$, giving
\[
a(j+k+1,j+1,j+k)\le a (j+k+1,j+k+2,j+k)\le \pi,
\]
and converting to dihedral $a(j+k+1,j+1,j+k) < \epsilon$.

Since $2\epsilon < 2\pi$, we can add angles,
\[
a(j+k+1,j,j+1) + a(j+k+1,j+1,j+k) = a(j+k+1,j,j+k) < 2 \epsilon < a(j+k+1,j+k+2,j+k).
\]
This says that $\w_{j}\in W({j+k+1})$.
\end{proof}

We apply this lemma to prove the wedge property in ZLZTHIC (Dense Sphere Packings Lemma 7.25).
We return to the context and notation of that lemma.

\begin{corollary}  Let $(\varphi,V,I)$ be a deformation of a generic local fan $(V,E,F)$ over an
interval $I$.  Assume that the azimuth angle of $\v_i(t)$ is at most $\pi$ for all  $t\in I$,
whenever $\v_i$ is straight.  Assume that $(V(t),E(t),F(t))$ is a generic fan for all $t\in I$.
Then for all sufficiently small $t\in I$,  we have that $(V(t),E(t),F(t))$ satisfies the wedge property
of local fans.  That is, $V(t)\subset \Wdart(x(t))$ for all $x(t)\in F(t)$.
\end{corollary}

\begin{proof}
As in the proof in Dense Sphere Packings,  by continuity,
the proof of the wedge property reduces to the case $\v\in \Wdart(\u)\setminus \Wdarto(\u)$.
We have a sequence $\v_0,\ldots,\v_r$, with $\v_i =\rho^i\v$,
where all of the terms $\v_i$ ($i=1,\ldots,r-1$) are straight.
Pick $\epsilon$ such that  $0 < 2\epsilon < \azim(\v_i,\v_{i+1},\v_{i-1})$ for $i=1,\ldots,r-1$.
By continuity, there is some $\epsilon > 0$ such that
$2\epsilon < \azim(\v_i(t),\v_{i+1}(t),\v_{i-1}(t))$ for $i=1,\ldots,r-1$ when $|t|<\epsilon$.

When $t=0$, the straight conditions give
\[
\dih(\orz,\v_p,\v_q,\v_{q+1}) = 0,
\]
for distinct triples $\{p,q,q+1\} \subset \{0,\ldots,r\}$,
and
\[
\dih(\orz,\v_p,\v_{p+1},\v_q)=0,
\]
for distinct triples $\{p,p+1,q\}\subset \{0,\ldots,r\}$ with $p+1 < q$.
By continuity, shrinking $\epsilon$ as needed, we may assume the corresponding
dihedral angles are less than $\epsilon$, when evaluated at $\v_i(t)$, for $|t|<\epsilon$.

All of the conditions of Lemma~\ref{lemma:deform-wedge} are satisfied for
$\w_i = \v_i(t)$ for any $|t|<\epsilon$.  Hence for all $i$ and all $j$ such that $i,j\le r$, we have
\[
\v_i(t) \in \Wdart(\v_{j}(t),\v_{j+1}(t)).
\]
This is the wedge condition.
\end{proof}

\newpage
\section{Appendix on explicit deformations}\label{sec:sup-deformation}

This is an appendix to Section 7.2 of Dense Sphere Packings.
This appendix gives the explicit construction of particular deformations.

The first lemma constructs a simplex $\{\v_0,\v_1,\v_2,\v_3\}\subset\ring{R}^3$ on
a given base triangle $\{\v_0,\v_1,\v_2\}$.  Our intention is to define a new constant
to equal the right-hand side of Equation~\ref{eqn:v3}.  The variable $x_5$ will run over
an interval to define a continuous deformation of a local fan $(V,E,F)$.


\begin{lemma}\guid{PQCSXWG}  Let $\{\v_0,\v_1,\v_2\}\subset \ring{R}^3$.  Assume that $\{\v_0,\v_1,\v_2\}$ is
not collinear.  Let $x_1,\ldots,x_6\in\ring{R}^3$ be given with $x_i> 0$ and
\[
x_1 = \norm{\v_1}{\v_0}^2, \quad x_2 = \norm{\v_2}{\v_0}^2,\quad x_6 = \norm{\v_1}{\v_2}^2.
\]
Assume that $\Delta(x_1,\ldots,x_6) > 0$  (that is, {\tt delta\_x x1 x2 ...}).
Then there exists  $\v_3$ such that
\begin{itemize}
\item $x_3 = \norm{\v_3}{\v_0}^2,\quad x_5 = \norm{\v_3}{\v_1}^2,\quad x_4 = \norm{\v_3}{\v_2}^2$, and
\item $(\v_1 - \v_0) \cdot (\v_2 - \v_0) \times (\v_3 - \v_0) > 0$.
\end{itemize}
Explicitly, the following vector works:
\begin{equation}\label{eqn:v3}
\v_3 = \v_0 +  \dfrac{2\sqrt{\Delta} (\v_1-\v_0) \times (\v_2 - \v_0) + 
\Delta_5  (\v_1-\v_0) + \Delta_6 (\v_2-\v_0)} {\ups(x_1,x_2,x_6)},
\end{equation}
with subscripts on $\Delta$ indicating partial derivatives and $\ups = \text{\tt ups\_x}$, the upsilon function.
Moreover, fixing $\v_0,\v_1,\v_2$ and fixing all the variables $x_i$ except $x_5$, the vector $\v_3\in\ring{R}^3$
in Equation~\ref{eqn:v3}
depends continuously on $x_5$ on the domain
\[
\{x_5 \mid x_5 > 0,\quad \Delta(x_1,\ldots,x_6)>0\}.
\]
\end{lemma}

\begin{remark} There is a symmetry to the data fixing $\v_0$, $x_2$, $x_5$, $\ups(x_1,x_2,x_6)$  and swapping
$\v_1\leftrightarrow \v_2$, $x_3\leftrightarrow x_4$, $x_1\leftrightarrow x_6$, $\Delta_5 \leftrightarrow \Delta_6$.
Under this symmetry, the vector $\v_3$ is given by the same formula, except that $\sqrt{\Delta}$ is
replaced with $-\sqrt{\Delta}$, and the sign of the triple product is reversed.
\end{remark}

\begin{proof}
It can be shown by direct computation that the vector $\v_3$ works.  This proof gives details about how
$\v_3$ is found.

Without loss of generality, we may move $\v_0$ to the origin.  Explicitly,
we let $\w_i = \v_i - \v_0$.  We will construct a unique $\w_3$ (for $\w_0=0$) and then set $\v_3 = \v_0 + \w_3$.

Note that the non-collinearity condition on $\{\v_0,\v_1,\v_2\}$ gives $\w_1\times\w_2\ne 0$.
We may write $\w_3$ in terms of the basis $\w_1\times\w_2$, $\w_1$, and $\w_2$:
\[
\w_3 = \alpha (\w_1\times \w_2) + \beta \w_1 + \gamma \w_2.
\]
We compute the norms of $\w_3$, $\w_3 - \w_1$ and $\w_3-\w_2$ in this basis:
\begin{align*}
\normo{\w_3}^2 = x_3 &= \alpha^2 \normo{\w_1\times\w_2}^2 + \normo{\beta\w_1+\gamma\w_2}^2 \\
\norm{\w_3}{\w_1}^2 = x_5 &= \alpha^2 \normo{\w_1\times\w_2}^2 + \normo{(\beta-1)\w_1 + \gamma\w_2}^2\\
\norm{\w_3}{\w_2}^2 = x_4 &= \alpha^2\normo{\w_1\times\w_2}^2 +\normo{\beta\w_1 + (\gamma-1)\w_2}^2.
\end{align*}
We eliminate $\alpha^2$ from the equations and write the resulting equations for $\beta$ and $\gamma$ as
a linear system:
\[
\left(\begin{matrix} \normo{\w_1}^2 & \w_1\cdot \w_2 \\ \w_1\cdot \w_2 & \normo{\w_2} \end{matrix}\right)
\left(\begin{matrix} \beta \\ \gamma \end{matrix}\right) = 
\left(\begin{matrix} (x_1 + x_3 - x_5)/2 \\ (x_2 + x_3 - x4)/2 \end{matrix}\right).
\]
The determinant of the system is 
\[
\normo{\w_1}^2\normo{\w_2}^2 - (\w_1\cdot \w_2)^2 = \normo{\w_1\times \w_2}^2 >  0.
\]
By the law of sines or cosines (Lemma~2.59), this determinant can also be written in terms of $x_i$:
\[
x_1 x_2 - ((x_6 - x_1 - x_2)/2)^2 = \ups(x_1,x_2,x_6)/4 > 0.
\]
Thus, there are unique solutions $\beta$ and $\gamma$ as functions of the variables $x_i$. Explicitly,
\[
b = \Delta_5/\ups(x_1,x_2,x_6),\quad c = \Delta_4/\ups(x_1,x_2,x_6),
\]
where $\Delta_i=\Delta_i(x_1,\ldots,x_6)$ is the $i$th partial derivative of $\Delta$.

With $\beta$ and $\gamma$ in hand, we solve for $\alpha$ using the equation:
\[
x_3 - \normo{\beta\w_1 + \gamma\w_2}^2 = \alpha^2 \normo{\w_1\times \w_2}^2.
\]
The left-hand side of this equation, expressed in terms of the variables $x_i$ is precisely $\Delta/\ups(x_1,x_2,x_6) >0$.
Hence 
\[
\alpha = \pm 2\frac{\sqrt{\Delta(x_1,\ldots,x_6)}} {\ups(x_1,x_2,x_6)}.
\]
The triple product is
\[
((\v_1 - \v_0) \times (\v_2 - \v_0)) \cdot (\v_3 - \v_0) = (\w_1 \times \w_2) \cdot \w_3 = \alpha |\w_1\times \w_2|^2.
\]
This is positive exactly when $\alpha$ is chosen to be positive.
This shows the existence of a unique vector $\w_3$ subject to the given conditions.

Continuity follows from the continuous dependence of $\alpha$, $\beta$, and $\gamma$ on $x_5$.
In fact, $\beta$ and $\gamma$ are polynomials in $x_5$ and $\alpha$ requires the extraction of a square
root of a positive polynomial in $x_5$.
\end{proof}


We will need a second lemma for deformations that occur within a plane.
Our intention is to define a new constant
to equal the right-hand side of Equation~\ref{eqn:v3-planar}.  It will be used in two
contexts.  Sometimes, the variable $x_3$ will run over
an interval to define a continuous deformation of a local fan $(V,E,F)$.
At other times, the vector $\v_2$ will vary in a continuous deformation constructed in the previous lemma,
and $\v_3$ will be carried continuously along in the plane of $\{\v_0,\v_1,\v_2(t)\}$ by the construction of this lemma.



\begin{lemma}\guid{EYYPQDW}
Let $\{\v_0,\v_1,\v_2\}\subset \ring{R}^3$.  Assume that $\{\v_0,\v_1,\v_2\}$ is
not collinear.  Let $x_1, x_2,x_3,x_5,x_6 \in\ring{R}^3$ be given with $x_i> 0$ and
\[
x_1 = \norm{\v_1}{\v_0}^2, \quad x_2 = \norm{\v_2}{\v_0}^2,\quad x_6 = \norm{\v_1}{\v_2}^2.
\]
Assume that $\ups(x_1,x_3,x_5)>0$.  Let $\sigma\in\{\pm 1\}$ be a choice of sign.
Then there exists  $\v_3$ such that
\begin{itemize}
\item $\{\v_0,\v_1,\v_2,\v_3\}$ is coplanar.
\item $x_3 = \norm{\v_3}{\v_0}^2,\quad x_5 = \norm{\v_3}{\v_1}^2$, and 
\item $(\v_3 - \v_0) \times (\v_1-\v_0)$ is a positive scalar times $\sigma (\v_1-\v_0)\times (\v_2 - \v_0)$.
\end{itemize}
Explicitly, the following vector works:
\begin{equation}\label{eqn:v3-planar}
\v_3 = \v_0 +  \frac{x_1 + x_3 - x_5}{2x_1}  (\v_1 - \v_0) + \frac{\sigma}{x_1} \sqrt{\frac {\ups(x_1,x_3,x_5)}{\ups(x_1,x_2,x_6)}}
   (\v_1 - \v_0) \times ((\v_1-\v_0) \times (\v_2-\v_0)).
\end{equation}
Moreover, fixing $\v_0,\v_1,\v_2$ and fixing all the variables $x_i$ except $x_3$, the vector $\v_3\in\ring{R}^3$
depends continuously on $x_3$ on the domain
\[
\{x_3 \mid x_3 > 0,\quad \ups(x_1,x_3,x_5) > 0\}.
\]
Moreover, fixing $\v_0,\v_1$ and fixing all the variables $x_i$, the vector $\v_3\in\ring{R}^3$ depends continuously
on $\v_2$ on the domain
\[
\{ \v_2\in\ring{R}^3\mid \v_2 \text{~~ is not on the line through~~ } \v_0 \text{~ and ~} \v_1 \}.
\]
\end{lemma}

\begin{remark} There is a symmetry in the data $\v_0\leftrightarrow \v_1$, $x_3\leftrightarrow x_5$,
$x_2\leftrightarrow x_6$ fixing $\v_2$ and $x_1$.  The symmetry preserves the solution $\v_3$.
\end{remark}

\begin{proof}
Without loss of generality, we may move $\v_0$ to the origin.  Explicitly,
we let $\w_i = \v_i - \v_0$.  We will construct a unique $\w_3$ (for $\w_0=0$) and then set $\v_3 = \v_0 + \w_3$.

Let $\n = \w_1 \times (\w_1\times \w_2)$.  By the non-collinearity assumption and the positivity of $x_i$,
we have $\n\ne 0$.  In fact, $\w_1$ are  $\n$ are orthogonal and span $\op{aff}\{\orz,\w_1,\w_2\}$.
The norm of $\n$ is computed as in the previous lemma by the law of cosiines:
\[
\normo{\n}^2 = \normo{\w_1}^2 \normo{\w_1\times \w_2}^2  = x_1 \ups(x_1,x_2,x_6)/4 > 0.
\]
We solve for $\w_3$ as a combination of $\w_1$ and $\n$:
\[
\w_3 = \alpha \w_1 + \beta \n.
\]
The norms of $\w_3$ and $\w_3-\w_1$ are computed as 
\begin{align*}
\normo{\w_3}^2 &= x_3 = \alpha^2 \normo {\w_1}^2 + \beta^2 \normo{\n}\\
\normo{\w_3-\w_1}^2 &= x_5 = \alpha^2 \normo {\w_1}^2 + \beta^2 \normo{\n}\\
\end{align*}
Eliminate $\beta$ and solve the resulting linear equation uniquely for $\alpha$:
\[
\alpha = (x_1 + x_3 - x_5)/(2 x_1).
\]
The right-hand side of $\beta^2 \normo{\n}^2 = x_3 - \alpha^2 x_1$, expressed in terms of the variables $x_i$
is $\ups(x_1,x_3,x_5)/(4 x_1)$.  Hence,
\[
\beta = \pm\frac{1}{x_1} \sqrt{\frac {\ups(x_1,x_3,x_5)}{\ups(x_1,x_2,x_6)}}.
\]

To compute the sign of $\beta$, we examine the cross-product condition.
\begin{align*}
\w_3\times \w_1 &= (\alpha \w_1 + \beta \n)\times \w_1 \\
 &= \beta  (\w_1\times (\w_1\times \w_2)) \times \w_1\\
&= \beta (\w_1 \times \w_2) (\w_1\cdot \w_1)\\
&= \beta x_1 (\w_1\times \w_2).
\end{align*}
This is a positive multiple of $\sigma \w_1\times \w_2$ when $\beta$ has sign $\sigma$.
This establishes the unique existence of $\w_3$.  

Continuity follows from the explicit formulas for $\alpha$ and $\beta$.
\end{proof}





\newpage
\section{Appendix on the proof of BGMIFTE}

This is an appendix to Section 7.3 of Dense Sphere Packings.

This page contains some notes on the verification that the polar $(V',E',F')$ is
a local fan.   This is an expanded version of  the last paragraph
of the proof of Lemma 7.34 (BGMIFTE).
We check the intersection property of fans, the dihedral property
of local fans, and the face property of local fans.

Since we are treating the last paragraph of the
proof, we may assume that all the other parts of that lemma dealing
with arcs and azimuth angles have been established.

We place ourselves in the context of Lemma BGMIFTE, adopting the
notation from that lemma.  
In particular, $\v_i = \rho^i\v$ and
$\w_i = \v_i \times \v_{i+1}$.  
We take the indices modulo $k=\card(V)$, so that $\v_{i+k}=\v_i$.
By earlier parts of the proof, we have
\begin{equation}\label{eqn:wj'}
\w_j\in W^0(\orz,\w_{i},\w_{i+1},\w_{i-1}),\quad j\ne i-1, i, i+1
\end{equation}
and
\begin{equation}\label{eqn:colw'}
\{\orz,\w_i,\w_j\} \text{ is not collinear, when } i\ne j
\end{equation}  
and
\begin{equation}\label{eqn:arcV'}
\arc_V(\orz,\{\w_i,\w_{i+1}\}) 
=\pi - \angle(\v_{i+1}) >0
\end{equation}
and
\begin{equation}\label{eqn:angle'}
\angle'(\w_{i+1}) 
=\pi -\arc_V(\orz,\{\v_{i+1},\v_{i+2}\})  \in\leftopen 0,\pi\rightopen.
\end{equation}

We verify the fan intersection property of $(V',E',F')$.  Remark GMLWKPK gives
some hints about verifying the intersection property, and notes that it comes
down to two cases:
\begin{enumerate}
\item $\ee\cap \ee' = \emptyset$ implies $C(\ee)\cap C(\ee') = \emptyset$.
\item $\ee\cap\ee' = \{\v\}$ implies $C(\ee)\cap C(\ee') = C\{v\}$.
\end{enumerate}
If $\ee$ and $\ee'$ are both singletons then the intersection property
follows from \eqref{eqn:colw'}.  Without loss of generality assume that $\ee$ is
not a singleton.  By the definition of $E'$, we have 
 $\ee=\{\w_i,\w_{i+1} \}$, for some $i$.  We may partition $C(\ee)$ as
\[
C(\ee) = C^0(\ee) \cup C^0(\w_i) \cup C^0(\w_{i+1})\cup \{\orz\}.
\]
Since we know the intersection property for singletons, we are reduced to showing
the following.
\begin{enumerate}
\item if $\ee'=\{\w_j,\w_{j+1}\}\ne\ee$ is also an edge, then  $C^0(\ee)\cap C^0(\ee')=\emptyset$.
\item for every $\w_j\in V'$, we have $C^0(\ee) \cap C^0(\w_j)=\emptyset$.
\end{enumerate}

Consider the first of these two enumerated 
cases.  Exhanging $i$ with $j$ if necessary,
we may assume that $j\ne i,i+1$.  
Set $\alpha(\p)=\op{azim}(\orz,\w_i,\w_{i+1},\p)$.  For every point
$\p$ in $C^0(\ee)$ we have
$\alpha(\p)=0$ and
$C^0(\ee)\subset A:= \op{aff}\{\orz,\w_i,\w_{i+1}\}$.
We separate this from $C^0(\ee')$ by showing that
$C^0(\ee')\subset A_+^0:= \op{aff}_+^0(\{\orz,\w_i,\w_{i+1}\},\{\w_{i-1}\})$,
and using the disjointness of this open half-space $A_+^0$ from its bounding
plane $A$.

 If $j=i-1$, then by \eqref{eqn:angle'}, for every $\q\in C^0(\ee')$, 
we have 
\[
\alpha(\q)=\alpha(\w_j) = \alpha(\w_{i-1}) \in\leftopen 0,\pi\rightopen.
\]
The values of $\alpha$ separate $C^0(\ee)$ from $C^0(\ee')$.  Note also
that this gives 
\begin{equation}\label{eqn:a+}
\w_{i-1}\in A_+^0
\end{equation}

If $\ell\ne i,i+1$, then by \eqref{eqn:wj'} and \eqref{eqn:a+}, we have
$\w_\ell\in A_+^0$.  From this, we obtain $\ee'\subset A_+^0$ and from
the conic structure of the halfspace $A_+^0$, it follows that $C^0(\ee')\subset A_+^0$.


The second enumerated case is similar,  if $j\in\{i,i+1\}$, then the empty
intersection property $C^0\{\w_i,\w_{i+1}\}\cap C^0(\w_j)=\emptyset$
follows from the strict inequality in the definition of $C^0$ and the
linear independence of $\w_i$ and $\w_{i+1}$.  For example,
\[
t_0 \w_i + t_1 \w_{i+1} = s \w_i,
\]
has no solution in positive real numbers $t_0,t_1,s$.
Otherwise, if $j\not\in\{i,i+1\}$,
we separate $C^0(\ee)$ from $C^0(\w_j)$ by the disjointness of $A_+^0$ and $A$, 
as in the first case.

This completes the proof of the intersection property of fans for the polar.
Next we verify the dihedral property of local fans. 


For this, we review how
a hypermap is attached to the fan $(V',E',F')$.
We have sets $(V',E',F')$ defined to be
\[
V' = \{ \w_i = \v_i \times \v_{i+1}\mid i\},
\]
\[
E' = \{\{\w_i,\w_{i+1}\}\mid i\},
\]
\[
F' = \{(\w_i,\w_{i+1})\mid i\}.
\]
The set of darts of the hypermap consists of all orderings of edges:
\[
D  =\{(\w_i,\w_{i+1}) \mid i\} \cup \{(\w_{i+1},\w_i)\mid i\} = F' \cup F'' \text{ say }.
\]
The set $E'(\w_i)$, with overloaded notation, is defined as the set
\[
\{\w\in V' \mid \{\w,\w_i\}\in E'\},
\]
which in this situation reduces to
\[
E'(\w_i) = \{\w_{i-1},\w_{i+1}\}.
\]
The permutation $\sigma(\w_i)$ of $E'(\w_i)$ is defined as the azimuth
cycle on this set.  Since the set has only two elements, the permutation is
forced to swap $\w_{i-1}$ and $\w_{i+1}$.
The hypermap is a tuple $(D,e,n,f)$, where the permutations of $D$ are
generally defined as follows.
\begin{align*}n(\v,\w) &= (\v,\sigma(\v,\w)),\\
f (\v,\w) &= (\w,\sigma(\w)^{-1} \v),\\
e (\v,\w) &= (\w,\v).
\end{align*}
In the present context, this reduces to
\begin{align*}n(\w_i,\w_{i\pm 1}) &= (\w_i,\w_{i\mp 1}),\\
f (\w_i,\w_{i\pm1}) &= (\w_{i\pm1},\w_{i\pm2}),\\
e (\w_i,\w_{i\pm 1}) &= (\w_{i\pm 1},\w_i).
\end{align*}

To prove the dihedral property, by Lemma QQYVCFM, it is enough
to prove the following properties
\begin{enumerate}
\item the hypermap is connected.
\item the number of darts is $2k$.
\item the orders of $f,n,e$ are $k$, $2$, $2$, respectively.
\end{enumerate}

We have 
\[
f^j(\w_i,\w_{i\pm1}) = (\w_{i\pm j},\w_{i\pm (j+1)}).
\]
Note that the orbit of $f$ on $(\w_i,\w_{i\pm1})$ is $F'$ or $F''\subset D$.
(This proves in particular the face property of local fans: $F'$ is a face of the
hypermap.)
and $n$ exchanges darts in $F'$ and $F''$.  Hence the hypermap is connected.

The sets $F'$ and $F''$ are disjoint and each contain $k$ darts.  Hence
the number of darts is $2k$.

The smallest positive $j$ such that $f^j(\w_i,\w_{i\pm1}) = (\w_i,\w_{i\pm1})$
is $k$.  Hence $f$ has order $k$.  The orders of $e$ and $n$ are $2$ by
inspection.  This completes the verification of the dihedral property of
local fans.

While proving the dihedral property, the face property fell out as well.

This completes the verification of properties intersection, fan, and dihedral.

