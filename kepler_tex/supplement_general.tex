% May 15, 2013.  Unsorted Appendices.




\newpage
\asection{Appendix on deformations of local fans}\label{sec:sup-deformation}

This is an appendix to Section 7.2 of Dense Sphere Packings.
This appendix gives further details about the proof of the wedge property in  Lemma 7.25 (ZLZTHIC).

\begin{lemma}\guid{WEECNNS}
\formalauthor{J. Harrison}
The function $\dih_V$ is continuous on the set
\[
\{(\v_1,\v_2,\v_3,\v_4) \mid \text{not collinear } \{\v_1,\v_2,\v_3\} \text{ and } \text{not collinear } \{\v_1,\v_2,\v_4\} \}.
\]
\end{lemma}

\begin{lemma}\guid{XBJRPHC}
The function $\op{azim}$ is continuous on the set
\[
\{(\v1,\v_2,\v_3,\v_4) \mid \text{not collinear } \{\v_1,\v_2,\v_3\} \text{ and } \text{not collinear } \{\v_1,\v_2,\v_4\} 
\v_4\not\in \op{aff}_+^0(\{\v_1,\v_2\},\v_3)\}.
\]
\end{lemma}

The following lemma rewrites the generic blade condition as a union of two sets that are
manifestly open.

\begin{lemma}[generic\_alt]\guid{IHWVUIZ}
\formalauthor{T. Hales}
Assume that $\{0,\v,\w\}$ is not collinear and that $\u\ne\orz$.
Then the generic blade condition holds
\[
\op{aff}_+ (\orz,\{\v,\w\}) \cap \op{aff}_-^0(\{\orz\},\{\u\}) = \emptyset
\]
iff $\{\orz,\u,\v,\w\}$ is not coplanar or $-\u\in W^0 (\orz,\v\times\w,\w,\v)$.
\end{lemma}

\begin{lemma}\guid{NHCXLRV} 
Let $(V,E,F)$ be a generic convex local fan.
Let $(\varphi,I,V)$ be a deformation of the fan.   Let $\v,\w \in V$.  Assume that
$\v\in \Wdarto(\w,\rho\w)$.  Then for sufficiently small $t$, we have
\[
\v(t)\in \Wdarto(\w(t),\rho\w(t)).
\]
\end{lemma}

\begin{proof} 
\claim{We claim that $(\v,\w)$ is not a pole of the fan.}  In fact, by {\tt WEDGE\_ALT}, if a pole, then 
$\v\not\in \Wdarto(\w,\rho\w)$.

The condition $\v\in \Wdarto(\w,\rho\w)$ gives
\[
0 < \op{azim}(\orz,\w,\rho\w,\v) < \op{azim}(\orz,\w,\rho\w,\rho^{-1}\w).
\]
Pick $c$ such that
\begin{equation}\label{eqn:azim-c}
0 < \op{azim}(\orz,\w,\rho\w,\v) < c < \op{azim}(\orz,\w,\rho\w,\rho^{-1}\w).
\end{equation}
The pole claim gives that the domain conditions for the continuity of $\op{azim}$ are met.
Therefore Equation~\eqref{eqn:azim-c} holds under a small deformation.
This gives the conclusion.
\end{proof}

\begin{lemma}[]\guid{WNWSHJT}
  Let $(V,E,F)$ be generic fan.
% with pole $(\v,\w)$.  Assume that the interior angle at the pole
%is less than $\pi$.  Let $\u\ne \v,\w$.  
Let $(\varphi,I,V)$ be a deformation of the fan.
Let $\u\in V$ have interior angle $\angle(\u)<\pi$.  Then for sufficiently small $t$, we have
$\angle(\u(t))<\pi$.
\end{lemma}

\begin{proof}
By {\tt INTERIOR\_ANGLE1\_POS}, we have $0 < \angle(\u) <\pi$.  The continuity domain conditions for $\op{azim}$
are met at $(\orz,\u,\rho\u,\rho^{-1}\u$.   Hence
\[
0 < \azim(\u(t)) < \pi
\]
for $t$ sufficiently small.
\end{proof}

The following case is not needed, because it is a special case of {\tt MHAEYJN}.

\begin{lemma}[]\guid{CREUXCQ}
  Let $(V,E,F)$ be a generic fan.
 with pole $(\v,\w)$.  Assume that the interior angle at the pole
is less than $\pi$.  Let $\u\ne \v,\w$.  
Let $(\varphi,I,V)$ be a deformation of the fan that moves a single $\u\in V\setminus\{\v,\w\}$ and
 such that  $\u\in V$, we have $\u(t)\in \op{aff}\{\orz,\v,\w,\u\}$.
Fix $\u'\in V$ such that $\angle(\u')=\pi$.  Then for sufficiently small $t$, we have
$\angle(\u'(t))\le\pi$.
\end{lemma}

\begin{proof}  Let $\u_1=\u'$, $\u_2=\rho\u'$, and $\u_0 = \rho^{-1}\u'$.  
Let $S=\{\orz,\u_0,\u_1,\u_2,\v,\w\}$.

We consider the case $\u\not\in S$.  
Then the set $S$ is fixed under the deformation, so
that 
\[
\angle(\u_1(t)) = \angle(\u'(t)) =\angle(\u') =\pi.
\]

In the remaining case $\u\in S$.  
By the lunar geometry
lemma, $S \subset A$, where $A = \op{aff}\{\orz,\v,\w,\u\}$, a plane.
The deformation moves $\u$ within this plane, so $S(t)\subset A$.  This implies that $S(t)$ is coplanar,
which gives 
\[
\angle(\u'(t))=\azim(\orz,\u_0(t),\u_1(t),\u_2(t))\in \{\orz,\pi\}.
\]

We claim the the domain conditions for the continuity of $\op{azim}$ are met.  Hence
by continuity, $\angle(\u'(t))=\pi$ for small $\pi$.
\end{proof}

In the next series of lemmas we introduce an assumption {\tt ECAU} given as follows.

\begin{remark}[ECAU]
Let $\u_0,\ldots,\u_r\in\ring{R}^3$, and define $\e = \u_0\times\u_1$.  We make
the following assumptions.
\begin{enumerate}
\item (C) The set $\{\orz,\u_i,\u_j\}$ is not collinear for $i\ne j$.
\item (A) For all $i\ge 1$, we have $\u_i\in \op{aff}_+^0(\{\orz,\u_0\},\u_1)$.
\item (U) The set $U = \{\u_0,\ldots,\u_r\}$ is cyclic with respect to $(\orz,\e)$, with cycle
\[
\sigma \u_i = \u_{i+1},\text { if } i< r.
\]
We let $A$ be the plane $\op{aff}\{\orz,\u_0,\u_1\}$.
\end{enumerate}
\end{remark}

\begin{lemma}[]\guid{VUYCADE}  
\formalauthor{Hales}
Let $\u_1,\ldots,\u_r$ with cross product $\e$  satisfy the conditions {\tt ECAU}.  
Then for all $i\le r$, we have constants $t_0$ and $t_1$, with $t_1>0$ such that 
\[
\u_i = t_0 \u_0 + t_1 \u_1,\quad \e\cdot\u_i = 0,\quad \op{azim}(\orz,\e,\u_0,\u_i)<\pi.
\]
Also, $\{\orz,\e,\u_i\}$ is not collinear.
\end{lemma}

\begin{proof} Condition A gives $\u_i = t_0 \u_0 + t_1\u_1$, with $t_1\ge0$.  We get $t_1>0$ from C.
Taking the dot product of both sides of this identity gives $\e\cdot\u_i=0$.
We have that $\{\orz,\e,\u_i\}$ is not collinear because $\e$ and $\u_i$ are orthogonal and nonzero.
The condition $\op{azim}(\orz,\e,\u_0,\u_i)<\pi$ is equivalent to $\e\cdot (\u_0\times\u_i)\ne0$.
The left-hand-side of this equation evaluates to $\normo{\e}t_1 >0$.
\end{proof}

\begin{lemma}[]\guid{YBTASCZ}
\formalauthor{Hales}
Let $\u_1,\ldots,\u_r$ with cross product $\e$  satisfy the conditions {\tt ECAU}.  
For all $i<j\le r$, we have
\[
\op{azim}(\orz,\e,\u_0,\u_i)<\azim(\orz,\e,\u_0,\u_j).
\]
Furthermore, $\op{azim}(\orz,\e,\u_i,\u_j)<\pi$.
\end{lemma}

\begin{proof}
Let $j\le r$.  We prove the result by induction on $i$.

When $i=0$, we have
\[
0 = \azim(\orz,\e,\u_0,\u_0)\le \azim(\orz,\e,\u_0,\u_j).
\]
We do not have equality because of A and C.

Assume for $i$ and assume for a contradiction $i+1$ fails. 
That is,
\[
\azim(\orz,\e,\u_0,\u_i) < \azim(\orz,\e,\u_0,\u_j)\le \azim(\orz,\e,\u_0,\u_{i+1}).
\]
By AC, the inequality is strict.  By U, this contradicts $\sigma\u_i = \u_{i+1}$.
This proves the main statement.

The further claim follows from the addition law
\[
\azim(\orz,\e,\u_0,\u_i)+\azim(\orz,\e,\u_i,\u_j) = \azim(\orz,\e,\u_0,\u_j).
\]
\end{proof}

\begin{lemma}[]\guid{KCZXLLE}
\formalauthor{Hales}
Let $\u_1,\ldots,\u_r$ with cross product $\e$  satisfy the conditions {\tt ECAU}.  
Assume $j<k\le r$.  If $i<j$ or $k<i$ then $\azim(\orz,\u_i,\u_j,\u_k)=\orz$.
\end{lemma}

\begin{proof}  By $A$, the set $\{\orz,\u_i,\u_j,\u_k\}$ is coplanar.  This implies that
the azimuth angle is $0$ or $\pi$.  By {\tt LDURDPN}, using that $\{\orz,\u_i,\u_j,\u_k\}$ is coplanar,
we get that $\azim(\orz,\u_i,\u_j,\u_k)\ne \pi$ if and only iff
\begin{equation}\label{eqn:aff00}
\op{aff}\{\orz,\u_i\} \cap \op{conv}^0\{\u_j,\u_k\}=\emptyset.
\end{equation}
Since $\op{azim}(\orz,\e,\u_j,\u_k)<\pi$, we can relate the wedge to a lune:
\[
\op{conv}^0\{\u_j,\u_k\}\subset \op{aff}_+^0\{\orz,\e\}\{\u_j,\u_k\} = W^0(\orz,\e,\u_j,\u_k)\subset
\op{aff}_+^0(\{\orz,\e,\u_0\},\u_1)
\]
and by the $t_1>0$ calculation above, we also have
\[
\op{aff}\{\orz,\u_i\}\cap \op{aff}_+^0(\{\orz,\e,\u_0\},\u_1)\subset \op{aff}_+^0\{\orz\}\{\u_i\}.
\]
Hence the intersection \eqref{eqn:aff00} is contained in $W^0(\orz,\e,\u_j,\u_k)\cap \op{aff}_+^0\{\orz\}\{\u_i\}$.
A point $\p$ in the intersection on the right gives
\[
0 < \azim (\orz,\e,\u_j,\p) < \azim(\orz,\e,\u_j,\u_k) \text{ and } \azim(\orz,\u_j,\p) = \azim(\orz,\u_j,\u_i).
\]
However, when $i<j$ or $k<i$, these inequalities fail.
\end{proof}

\begin{lemma} \guid{FSQKWKK}\label{lemma:rot}
\formalauthor{JH}
If $\op{azim}(\orz,\v_1,\v_2,\v_3)\le \pi$ then 
\[
\op{azim}(\orz,\v_2,\v_3,\v_1)\le\pi.
\]
\end{lemma}

\begin{proof}  
$\azim(\orz,\v_1,\v_2,\v_3)\le\pi$  iff $0\le\sin(\azim(\orz,\v_1,\v_2,\v_3))$ iff
$0\le (\v_1\times \v_2)\cdot \v_3$.  The final condition is invariant under even permutations of
the subscripts.
\end{proof}

Recall that if we have $\op{azim}(\orz,\v_1,\v_2,\v_3)\le \pi$ and appropriate noncollinearity constraints, then
$\op{azim}(\orz,\v_1,\v_2,\v_3) = \op{dih}_V (\orz,\v_1,\v_2,\v_3)$.

\begin{lemma} \guid{MKIFWJT}\label{lemma:sum3-azim-fan}
Let $\v,\w_1,\w_2,\w_3\in\ring{R}^3$. Assume that none of the sets $\{\orz,\v,\w_i\}$ are collinear
and that $\op{azim}(\orz,\v,\w_1,\w_2)+\op{azim}(\orz,\v,\w_2,\w_3)< 2\pi$.  Then
\[
\op{azim}(\orz,\v,\w_1,\w_2)+\op{azim}(\orz,\v,\w_2,\w_3) = \azim(\orz,\v,\w_1,\w_3).
\]
\end{lemma}

\begin{proof} This is (Fan.sum3\_azim\_fan).
\end{proof}



\begin{lemma}[deform-wedge] \label{lemma:deform-wedge} \guid{XIVPHKS}
\formalauthor{John Harrison}
Let $\w_0,\ldots,\w_{n}\in \ring{R}^3$, for some $n\ge 1$.  Assume that $i\ne j$ implies
$\{\orz,\w_i,\w_j\}$ is not collinear.   Let $r=n-1$.
Set
\[
d(i,j,k)=\op{dih}_V(\orz,\w_i,\w_j,\w_k) \quad\text{and}\quad a(i,j,k) = \azim(\orz,\w_i,\w_j,\w_k).
\]
Introduce abbreviations for  wedges $W(i) = W(\orz,\w_i,\w_{i+1},\w_{i-1})$.
Let $\epsilon\in\ring{R}$ satisfy $0 < 2 \epsilon < a(i,i+1,i-1)$, for $i=1,\ldots,r$.
Assume $a(i,i+1,i-1) \le \pi$ for $i=1,\ldots,r$.
Assume that $d(p,q,q+1)<\epsilon$ for all sets of three distinct elements $\{p,q,q+1\}\subset \{0,\ldots,r\}$.
Assume that $d(p,p+1,q)<\epsilon$ for all sets of three distinct elements $\{p,p+1,q\}\subset \{0,\ldots,r\}$ with
$q > p+1$.
Assume that $d(p+1,p,q)<\epsilon$ for all sets of three distinct elements $\{p+1,p,q\}\subset \{0,\ldots,r\}$ with
$q < p$. 

Then for all $k$ we have the statement $S_k$: 
for all $j\le r - k$ we have $\w_j\in W({j+k})$ and $\w_{j+k}\in W(j)$.
\end{lemma}

\begin{proof} We prove $S_k$ by induction on $k$.  The cases $k=0,1$ are trivially satisfied.
Assume that $S_k$, for some $k\ge1$.  We show $S_{k+1}$.  Fix $j\le r-(k+1)$.

By induction, $\w_j\in W({j+k})$, giving
\[
a(j+k,j+k+1,j)\le a(j+k,j+k+1,j+k-1)\le\pi.
\]
By Lemma~\ref{lemma:rot}, $a(j,j+k,j+k+1)\le\pi$, and converting to dihedral, $a(j,j+k,j+k+1)< \epsilon$.

By induction, $\w_{j+k}\in W(j)$, giving
\[
a(j,j+1,j+k)\le a(j,j+1,j-1)\le \pi,
\]
and converting to dihedral, $a(j,j+1,j+k)<\epsilon$.

Since $2\epsilon < 2\pi$, we can add the angles (Lemma~\ref{lemma:sum3-azim-fan}),
\[
a(j,j+1,j+k) + a(j,j+k,j+k+1) = a(j,j+1,j+k+1) < 2\epsilon < a(j,j+1,j-1).
\]
This says that $\w_{j+k+1}\in W(j)$.

The proof that $\w_j\in W({j+k+1})$ is similar.

By induction, $\w_{j+k+1}\in W({j+1})$, giving
\[
a(j+1,j+k+1,j)\le a(j+1,j+2,j)\le \pi.
\]
By Lemma~\ref{lemma:rot}, $a(j+k+1,j,j+1)\le \pi$, and converting to dihedral $a(j+k+1,j,j+1)<\epsilon$.

By induction, $\w_{j+1}\in W({j+k+1})$, giving
\[
a(j+k+1,j+1,j+k)\le a (j+k+1,j+k+2,j+k)\le \pi,
\]
and converting to dihedral $a(j+k+1,j+1,j+k) < \epsilon$.

Since $2\epsilon < 2\pi$, we can add angles,
\[
a(j+k+1,j,j+1) + a(j+k+1,j+1,j+k) = a(j+k+1,j,j+k) < 2 \epsilon < a(j+k+1,j+k+2,j+k).
\]
This says that $\w_{j}\in W({j+k+1})$.
\end{proof}

We apply this lemma to prove the wedge property in ZLZTHIC (Dense Sphere Packings Lemma 7.25).
We return to the context and notation of that lemma.

\begin{lemma}\guid{ITNZZRD}  
Let $(\varphi,V,I)$ be a deformation of a generic local fan $(V,E,F)$ over an
interval $I$.  Assume that the azimuth angle of $\v_i(t)$ is at most $\pi$ for all  $t\in I$,
whenever $\v_i$ is straight.  Assume that $(V(t),E(t),F(t))$ is a generic for all $t\in I$.  (Or assume
that $(V,E,F)$ is a local fan and a deformation that moves a single nonpolar element of $V$.)
Then for all sufficiently small $t\in I$,  we have that $(V(t),E(t),F(t))$ satisfies the wedge property
of local fans.  That is, $V(t)\subset \Wdart(x(t))$ for all $x(t)\in F(t)$.
\end{lemma}

\begin{proof}
As in the proof in Dense Sphere Packings,  by continuity,
the proof of the wedge property reduces to the case $\w\in \Wdart(\u,\rho\u)\setminus \Wdarto(\u,\rho\u)$.
We may prove the more symmetrical statement
\[
\u(t)\in \Wdart(\w(t),\rho\w(t)) \text{ and } \w(t) \in \Wdart(\u(t),\rho\u(t)).
\]
Exchanging $\u$ and $\w$ as needed, 
we have a sequence $\u=\v_0,\ldots,\v_r =\w$, with $\v_i =\rho^i\v$,
where all of the terms $\v_i$ ($i=1,\ldots,r-1$) are straight.  We use Lemmas 7.15 and 7.19
to check the properties {\tt ECAU}.

The interior angles of the fan are positive.
Pick $\epsilon$ such that  $0 < 2\epsilon < \azim(\orz,\v_i,\v_{i+1},\v_{i-1})$ for $i=0,\ldots,r$.
By the continuity of $\azim$, there is some $\epsilon > 0$ such that
$2\epsilon < \azim(\orz,\v_i(t),\v_{i+1}(t),\v_{i-1}(t))$ for $i=0,\ldots,r$ when $|t|<\epsilon$.

When $t=0$, the straight conditions give
\[
\dih(\orz,\v_p,\v_q,\v_{q+1}) = 0,
\]
for distinct triples $\{p,q,q+1\} \subset \{0,\ldots,r\}$,
and
\[
\dih(\orz,\v_p,\v_{p+1},\v_q)=0,
\]
for distinct triples $\{p,p+1,q\}\subset \{0,\ldots,r\}$ with $p+1 < q$.  (This follows from
the lemma above, which shows that $\azim  = 0$.)
By the continuity of $\op{dih}_V$, 
shrinking $\epsilon$ as needed, we may assume the corresponding
dihedral angles are less than $\epsilon$, when evaluated at $\v_i(t)$, for $|t|<\epsilon$.

All of the conditions of Lemma~\ref{lemma:deform-wedge} are satisfied for
vectors $\v_i(t)$ for any $|t|<\epsilon$.  Hence for all $i$ and all $j$ such that $i,j\le r$, we have
\[
\v_i(t) \in \Wdart(\v_{j}(t),\v_{j+1}(t)) \text{ and } \v_j(t) \in \Wdart(\v_{i}(t),\v_{i+1}(t)).
\]
If we take $i=0$ and $j=r$, this is the desired wedge condition.
\end{proof}





\newpage
\asection{Appendix on the proof of BGMIFTE}

This is an appendix to Section 7.3 of Dense Sphere Packings.

This page contains some notes on the verification that the polar $(V',E',F')$ is
a local fan.   This is an expanded version of  the last paragraph
of the proof of Lemma 7.34 (BGMIFTE).
We check the intersection property of fans, the dihedral property
of local fans, and the face property of local fans.

Since we are treating the last paragraph of the
proof, we may assume that all the other parts of that lemma dealing
with arcs and azimuth angles have been established.

We place ourselves in the context of Lemma BGMIFTE, adopting the
notation from that lemma.  
In particular, $\v_i = \rho^i\v$ and
$\w_i = \v_i \times \v_{i+1}$.  
We take the indices modulo $k=\card(V)$, so that $\v_{i+k}=\v_i$.
By earlier parts of the proof, we have
\begin{equation}\label{eqn:wj'}
\w_j\in W^0(\orz,\w_{i},\w_{i+1},\w_{i-1}),\quad j\ne i-1, i, i+1
\end{equation}
and
\begin{equation}\label{eqn:colw'}
\{\orz,\w_i,\w_j\} \text{ is not collinear, when } i\ne j
\end{equation}  
and
\begin{equation}\label{eqn:arcV'}
\arc_V(\orz,\{\w_i,\w_{i+1}\}) 
=\pi - \angle(\v_{i+1}) >0
\end{equation}
and
\begin{equation}\label{eqn:angle'}
\angle'(\w_{i+1}) 
=\pi -\arc_V(\orz,\{\v_{i+1},\v_{i+2}\})  \in\leftopen 0,\pi\rightopen.
\end{equation}

We verify the fan intersection property of $(V',E',F')$.  Remark GMLWKPK gives
some hints about verifying the intersection property, and notes that it comes
down to two cases:
\begin{enumerate}
\item $\ee\cap \ee' = \emptyset$ implies $C(\ee)\cap C(\ee') = \emptyset$.
\item $\ee\cap\ee' = \{\v\}$ implies $C(\ee)\cap C(\ee') = C\{v\}$.
\end{enumerate}
If $\ee$ and $\ee'$ are both singletons then the intersection property
follows from \eqref{eqn:colw'}.  Without loss of generality assume that $\ee$ is
not a singleton.  By the definition of $E'$, we have 
 $\ee=\{\w_i,\w_{i+1} \}$, for some $i$.  We may partition $C(\ee)$ as
\[
C(\ee) = C^0(\ee) \cup C^0(\w_i) \cup C^0(\w_{i+1})\cup \{\orz\}.
\]
Since we know the intersection property for singletons, we are reduced to showing
the following.
\begin{enumerate}
\item if $\ee'=\{\w_j,\w_{j+1}\}\ne\ee$ is also an edge, then  $C^0(\ee)\cap C^0(\ee')=\emptyset$.
\item for every $\w_j\in V'$, we have $C^0(\ee) \cap C^0(\w_j)=\emptyset$.
\end{enumerate}

Consider the first of these two enumerated 
cases.  Exhanging $i$ with $j$ if necessary,
we may assume that $j\ne i,i+1$.  
Set $\alpha(\p)=\op{azim}(\orz,\w_i,\w_{i+1},\p)$.  For every point
$\p$ in $C^0(\ee)$ we have
$\alpha(\p)=0$ and
$C^0(\ee)\subset A:= \op{aff}\{\orz,\w_i,\w_{i+1}\}$.
We separate this from $C^0(\ee')$ by showing that
$C^0(\ee')\subset A_+^0:= \op{aff}_+^0(\{\orz,\w_i,\w_{i+1}\},\{\w_{i-1}\})$,
and using the disjointness of this open half-space $A_+^0$ from its bounding
plane $A$.

 If $j=i-1$, then by \eqref{eqn:angle'}, for every $\q\in C^0(\ee')$, 
we have 
\[
\alpha(\q)=\alpha(\w_j) = \alpha(\w_{i-1}) \in\leftopen 0,\pi\rightopen.
\]
The values of $\alpha$ separate $C^0(\ee)$ from $C^0(\ee')$.  Note also
that this gives 
\begin{equation}\label{eqn:a+}
\w_{i-1}\in A_+^0
\end{equation}

If $\ell\ne i,i+1$, then by \eqref{eqn:wj'} and \eqref{eqn:a+}, we have
$\w_\ell\in A_+^0$.  From this, we obtain $\ee'\subset A_+^0$ and from
the conic structure of the halfspace $A_+^0$, it follows that $C^0(\ee')\subset A_+^0$.


The second enumerated case is similar,  if $j\in\{i,i+1\}$, then the empty
intersection property $C^0\{\w_i,\w_{i+1}\}\cap C^0(\w_j)=\emptyset$
follows from the strict inequality in the definition of $C^0$ and the
linear independence of $\w_i$ and $\w_{i+1}$.  For example,
\[
t_0 \w_i + t_1 \w_{i+1} = s \w_i,
\]
has no solution in positive real numbers $t_0,t_1,s$.
Otherwise, if $j\not\in\{i,i+1\}$,
we separate $C^0(\ee)$ from $C^0(\w_j)$ by the disjointness of $A_+^0$ and $A$, 
as in the first case.

This completes the proof of the intersection property of fans for the polar.
Next we verify the dihedral property of local fans. 


For this, we review how
a hypermap is attached to the fan $(V',E',F')$.
We have sets $(V',E',F')$ defined to be
\[
V' = \{ \w_i = \v_i \times \v_{i+1}\mid i\},
\]
\[
E' = \{\{\w_i,\w_{i+1}\}\mid i\},
\]
\[
F' = \{(\w_i,\w_{i+1})\mid i\}.
\]
The set of darts of the hypermap consists of all orderings of edges:
\[
D  =\{(\w_i,\w_{i+1}) \mid i\} \cup \{(\w_{i+1},\w_i)\mid i\} = F' \cup F'' \text{ say }.
\]
The set $E'(\w_i)$, with overloaded notation, is defined as the set
\[
\{\w\in V' \mid \{\w,\w_i\}\in E'\},
\]
which in this situation reduces to
\[
E'(\w_i) = \{\w_{i-1},\w_{i+1}\}.
\]
The permutation $\sigma(\w_i)$ of $E'(\w_i)$ is defined as the azimuth
cycle on this set.  Since the set has only two elements, the permutation is
forced to swap $\w_{i-1}$ and $\w_{i+1}$.
The hypermap is a tuple $(D,e,n,f)$, where the permutations of $D$ are
generally defined as follows.
\begin{align*}n(\v,\w) &= (\v,\sigma(\v,\w)),\\
f (\v,\w) &= (\w,\sigma(\w)^{-1} \v),\\
e (\v,\w) &= (\w,\v).
\end{align*}
In the present context, this reduces to
\begin{align*}n(\w_i,\w_{i\pm 1}) &= (\w_i,\w_{i\mp 1}),\\
f (\w_i,\w_{i\pm1}) &= (\w_{i\pm1},\w_{i\pm2}),\\
e (\w_i,\w_{i\pm 1}) &= (\w_{i\pm 1},\w_i).
\end{align*}

To prove the dihedral property, by Lemma QQYVCFM, it is enough
to prove the following properties
\begin{enumerate}
\item the hypermap is connected.
\item the number of darts is $2k$.
\item the orders of $f,n,e$ are $k$, $2$, $2$, respectively.
\end{enumerate}

We have 
\[
f^j(\w_i,\w_{i\pm1}) = (\w_{i\pm j},\w_{i\pm (j+1)}).
\]
Note that the orbit of $f$ on $(\w_i,\w_{i\pm1})$ is $F'$ or $F''\subset D$.
(This proves in particular the face property of local fans: $F'$ is a face of the
hypermap.)
and $n$ exchanges darts in $F'$ and $F''$.  Hence the hypermap is connected.

The sets $F'$ and $F''$ are disjoint and each contain $k$ darts.  Hence
the number of darts is $2k$.

The smallest positive $j$ such that $f^j(\w_i,\w_{i\pm1}) = (\w_i,\w_{i\pm1})$
is $k$.  Hence $f$ has order $k$.  The orders of $e$ and $n$ are $2$ by
inspection.  This completes the verification of the dihedral property of
local fans.

While proving the dihedral property, the face property fell out as well.

This completes the verification of properties intersection, fan, and dihedral.

\newpage

\asection{Appendix on saturation}

\begin{lemma}\guid{CPNKNXN}  Let $V\subset\ring{R}^3$ be any packing.  
Then there exists a saturated packing $V_{sat}$ that contains $V$.
\end{lemma}

\begin{proof}
If $V$ is any packing and $r$ is a real number, let $m(V,r)$ be the maximum over the following
set of natural numbers:
\[
\{ \card(V'\cap B(\orz,r)) \mid V \subset V' \text{ and } V' \text{ is a packing } \}.
\]
By (the proof of) Lemma 6.2 of DSP  (KIUMVTC) and Lemma (WQZISRI), there is a one-to-one map from
$V'\cap B(\orz,r)$ to $\ring{Z}^3\cap B(\orz,2r+1)$, which gives an upper bound on the cardinality
of $V'\cap B(\orz,r)$ depending only on $r$.  Thus the maximum exists.

Skolemizing variables, we have a function $W$ of $V$ and $r$ with the property that for any
packing $V$ and and any real number $r$, we have that $W(V,r)$ is a packing with $V \subset W(V,r)$ that
realizes the maximum cardinality $m(V,r)$.

We define a packing by recursion.
Let $V_0 = V$ and let $V_{n+1} = W(V_n,n)$.  Set $V_{sat}= \bigcup_n V_n$.  We will show that $V_{sat}$ is
a saturated packing that contains $V$.   It will take a few steps to reach the conclusion.

We show that $V\subset V_{sat}$.  In fact, $V = V_0 \subset \bigcup_n V_n = V_{sat}$.

It is easily checked that if $V' \subset V''$ is an inclusion of packings, then for all $r$, we have $V' \subset W(V'',r)$.
From this, it follows that if $V' \subset V_n$ then $V' \subset V_{n+1}$.
By induction, it now follows that if $n \le m$, then $V_n \subset V_m$.

We show that $V_{sat}$ is a packing.
If $\v,\w\in V_{sat}$, then $\v$, $\w\in V_m$ for some $m$.  The set $V_m$ is a packing, so that
if $\v\ne\w$, we have $\norm{\v}{\w}\ge 2$.  Hence $V_{sat}$ is a packing.

%If $n\le m$, we have $B(\orz,n) \cap V_n \subset B(\orz,n) \cap V_m$.  By the cardinality maximizing
%property of the function $W$, we have $B(\orz,n)\cap V_n = B(\orz,n)\cap V_m$.  Hence also,
%$B(\orz,n)\cap V_{sat} = B(\orz,n)\cap V_n$.

We claim that $V_{sat}$ is saturated.  Otherwise, there exists $\w\in\ring{R}^3$, with $\w\not\in V_{sat}$ such that
$\norm{\v}{\w}\ge 2$ for all $\v\in V_{sat}$.   Choose $n$ such that $\w\in B(\orz,n)$.
Let $V' = V_{n+1} \cup \{\w\}$.  Then $V'$ is a packing containing $V_n$
that achieves a larger cardinality of intersection
with $B(\orz,n)$ than does $V_{n+1}$.  This contradicts the maximality of $V_{n+1}$.
\end{proof}

