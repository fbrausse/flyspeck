%% Elementary Theory of the Reals
%% Total file rating 15455 as of 2/21/2008.

%% Next actions:
% Extract any remaining blueprint statements and add to this collection.
% Proofread.
% Insert graphics. Stick figures, and 3D.
% Fix XX.
% Fix messy ones at the end that haven't been converted to tarski language.
% Introduce some higher level abbreviations such as anchor, diagonal. Rewrite accordingly.
% Type out formal statements.
% Write javascript to do tag searches.
% Post to archive.
% Submit for publication.






% summary file:
\newwrite\mywrite
\immediate\openout\mywrite=summary_tarski.txt
\newwrite\myhtml
\immediate\openout\myhtml=array_tarski.js

% end the file with this line:
\def\filetarskiaway{
\immediate\write\mywrite{}\closeout\mywrite 
\immediate\write\myhtml{//fin}\closeout\myhtml
}


\section{Introduction}
\label{tarski:XX}


This is a collection of problems
in geometry.  Each problem can in fact be expressed in 
Tarksi's language of the real numbers.  There are decision
procedures to decide 
the truth of
any statement expressed in Tarski's language.  In this
sense, every problem in this collection is elementary.



\subsection{Flyspeck}

The sphere packing problem asserts that no packing of congruent
balls in three dimensions has density greater than $\pi/\sqrt{18}$.
This density is achieved by various packings, including the
face-centered cubic packing.

A long computer-assisted proof of this theorem appears
in \cite{DCG}.  The aim of an project called {\it Flyspeck} is
to give a complete formalization of this proof of the theorem.
  (The name {\it Flyspeck} comes from the acronym {\it FPK},
for the {\it Formal Proof of the Kepler} conjecture.)

This collection is a contribution to the Flyspeck project.  The problems
in this collection were extracted from the proof of the theorem.   
All of the extracted problems share certain 
features.  They can all be expressed as statements in Tarski's
language of the real numbers.  They are all statements
about the geometry of configurations in three dimensional Euclidean
space.  They can all be expressed with a small number of quantifiers.

\subsection{complexity}

What is meant by a {\it small} number of quantifiers?  Most of
the statements are about configurations of points.  Each point
is described by three coordinates, and hence three quantifiers.
In addition to the quantifiers that specify the coordinates there
may be a few additional quantifiers that specify the internal
relations among points.  For example, many statements assert that
two affine spaces intersect, and this is expressed through existential
quantifiers.
  
The collection is generally
arranged in order of increasing number of points.  This ordering
is not strict, because we have also tried to order the results
according to logical dependence, and to group together results 
that bear an affinity to one another.  The longest section of
the collection deals with five-point configurations.  
There is a smaller group of results about six-point configurations,
and then just a few results that involve seven or eight points.

From the point of view of quantifier elimination over the real
numbers, a configuration
of eight points is a daunting problem.  It involves 24 quantifiers
for the coordinates,  together with the additional
quantifiers tied to the internal relations among the points.



However, these problems are all rather simple from a geometric
point of view.  Eight points, for example, specify the
vertices of two tetrahedra in Euclidean space.    There is only
one result in this collection that concerns eight points, and
it is the assertion that two tetrahedra, subject to various
simple constraints, have disjoint interiors.  It is a completely
trivial
matter to give a direct geometric proof of this result.


\subsection{methods of proof}

Each statement in this collection is accompoanied by a proof.
In keeping with the elementary nature of the statements, 
we have generally tried to restrict our methods of proof to
elementary arguments.  We avoid quoting results from external
sources to make this collection self-contained.

As a rule, we avoid limits, calculus, analysis, and measure theory.
There are rare exceptions, where we look at the sign of derivative in a proof
to determine whether a function is increasing. 
There are a few compactness arguments implicit in some
of the proofs, where we move some of the points in a configuration
until a constraint is satisfied.

Non-elementary functions have been avoided as a rule.  A few
arguments are based on consideration of angle, which are defined
in terms of point configurations by means of an inverse trigonometric
function.  (See Section~\ref{sec:trig}.)  However, as we point
out in that section, the inverse trigonometric functions can be
eliminated from equations that are linear in the angle, by making
suitable use of trigonometric addition formulas.  Thus, the only purpose of
the inverse trigonometric functions is to make the
proofs more legible.

A few results elementary results in \cite{DCG} were established
by computer.  In a few cases, we have retained a computer-assisted
proof.  Each computer-verified inequality carries a nine-digit 
identification
number \calc{123456789} that helps to locate the computer code
that specifies the inequality and that performs the verification.
The identification number is used in the formal statement of the
inequalities.\footnote{\tt
http://flyspeck.googlecode.com/svn/trunk/inequalities/kep\_inequalities.ml.}


\subsection{formalization}

Although proofs have been provided for all the results, it is
my hope that other proofs might be found that are better adapted
to formalization.  In the best of all possible
worlds, we might hope for
algorithms that entirely automate the proofs in this collection.
This collection is not altogether different from the large collection
of problems in geometry that have been successfully automated
by Wu's method. 

After all, a quick glance at the problems shows that these are
mostly simple results about incidence in affine geometry, convexity, cones,
and the circumradius.  If such a collection lies beyond
the limits of automation, then this does not bode well for 
the general discipline of automated theorem proving.  That then
is the hope and challenge of this piece of the flyspeck project:
to find algorithms to automate the proofs in this collection.


Some general proof methods have been suggested
in a companion article \cite{method}.  These proposed methods include
quantifier elimination, energy minimization of tensegrities,
sums-of-squares decompositions, and global optimization.  


\subsection{figures}

Some of the results are accompanied by figures.\footnote{The number $2.51$ appears
throughout this collection.  It appears here, because it appers throughout
the proof of the packing problem.  It would be more natural to
rewrite all of the results so that the constants that appear are
optimal for the given configurations.  This would give
the results in this collection a more natural and definitive
form.  However, this would complicate some of
the proofs.  For that reason the results are left in their
current form.}
 
% Some of these
%figures are called ``stick figures.'' These are highly schematicized
%two dimensional representations of the configurations.
%
%In a stick figure, there is generally one point, called the
%base point.  (As a general rule, it is the point called $v_0$ in
%the statement of the result.) 
%
%The base point does not appear in the stick figure.  Rather, it
%is the point of perspective, from which every other point is viewed.
%More precisely, let $S=\{v_0,v_1,\ldots,v_k\}$ be a set of points
%in $\ring{R}^3$.  Assume that there is a plane $P$ that separates
%$v_0$ from all the points $\{v_1,\ldots,v_k\}$.  Then each point
%$v_j$, for $j\ne 0$, gives a segment from $v_j$ to $v_0$ that intersects
%$P$ in a point $\bar v_j$.  The stick figure draws the points
%$\{\bar v_1,\ldots ,\bar v_k\}\subset P$ in the plane $P$.
%A line segment from $v_i$ to $v_j$, with $i,j>0$, is represented by a
%line segment from $\bar v_i$ to $\bar v_j$ in the stick figure.
%A line segment from $v_i$ to $v_0$ degenerates to a single point $\bar v_i$
%in the stick figure.
%
%The stick figures are not drawn to scale.  Rather they give
%a quick and dirty schematic representation of the points of a configuration
%in a way that is represents the primary incidence relations.
%In fact, in general it would be impossible to draw the stick figures
%to scale, because many of the results assert that the given
%configurations do not exist!
%
%A stick figure should be imagined with the viewer located to one
%side of $P$ and the base point $v_0$ to the other side of $P$.  Thus,
%the viewer sits in front of the printed page, and the
%the base point is located {\it behind} it.  With
%this perspective in mind, 
%when a segment $\bar v_i$ to $\bar v_j$ is shown as
%passing over a segment $\bar v_k$ to $\bar v_\ell$, this is intended
%to convey the geometric fact that the segment $v_k$ to $v_\ell$ meets
%the triangular convex hull of the three points $v_i,v_j,v_0$.  That
%is, the viewer perceives the segment $v_i$ to $v_j$ as lying in front
%of the segment $v_k$ to $v_\ell$.
%
%There are certain conventions for representing various length constraints.
%Various segments are constrained to have lengths between $2$ and $2.51$.\footnote{The number $2.51$ appears
%throughout this collection.  It appears here, because it appers throughout
%the proof of the packing problem.  It would be more natural to
%rewrite all of the results so that the constants that appear are
%optimal for the given configurations.  This would give
%the results in this collection a more natural and definitive
%form.  However, I fear that this would complicate some of
%the proofs.  For that reason the results are left in their
%current form.}  When a segment $v_j$ to $v_0$
%has unspecified length the corresponding vertex $\bar v_j$ is represented
%as a black dot.  If $v_j$ to $v_0$ has length between $2$ and $2.51$
%(or thereabouts), then $\bar v_j$ is represented as a unfilled dot.
%If $v_j$ to $v_0$ has length between $2.51$ and $\sqrt8$ (or
%thereabouts), then $\bar v_j$ is represented as a crossed dot.
%
%A segment $v_i$ to $v_j$ in the approximate range $[2,2.51]$ is marked with
%an unfilled segment $\bar v_i$ to $\bar v_j$.  In the approximate
%range $[2.51,\sqrt8]$, the segment $\bar v_i$ to $\bar v_j$ is marked
%with an $X$.  There are no markings on segments of unknown length
%constraints.
%
%
%\pdffig{stick_template}{stick}{some features of stick figures}
%
%
%

\subsection{database}

\subsubsection{tags}

Each problem carries a list of tags to make it easier to search the database for
problems with various properties.

Here is a list of tags and their intended meaning. Many of the intended meanings are a matter
of definitions that appear elsewhere in this collection.

\newenvironment{taglist}{}{}

\begin{taglist}

\marker{def}{The statement is related to a definition, rather than a proposition.}

\marker{conv\,$n$}{The problem involves a convex hull of $n$ points, $n=2$ a segment, $n=3$ a triangle, $n=4$ a simplex, etc.}

\marker{aff\,$n$}{The problem involves the affine hull of $n$ points, $n=2$ a line, $n=3$, a plane, etc.}

\marker{pt\,$n$}{A set of $n$ points occurs.}
\marker{bary\,$n$}{Barycentric coordinates relative to $n$ points in an affice space of dimension $n-1$.}
\marker{circum\,$n$}{The circumcenter of $n$ points appears.}

\marker{2.696}{Various constants that appear in the statements of lemmas appear as tags.}
\marker{t0}{The constants $1.255$ and $2.51$.}
\marker{sqrt2}{The constants $\sqrt2$ and its integer multiples.}
\marker{dodec}{An inequality that may be of use in the proof of the Dodecahedral Conjecture.}


\marker{cm3-ups}{$\ups$}

\marker{cm4-delta}{$\Delta$}

\marker{cm4-E}{either $\mathcal E$ or $R$.}

\marker{packing}{a packing of points in $\ring{R}^3$.}
\marker{voronoi}{a Voronoi cell appears.}
\marker{plane}{Involves a planarity condition.}

\marker{poly-id}{a polynomial identity.}

\marker{poly-ineq}{a polynomial inequality.}

\marker{condA, condC, condF, condS, ball, rcone, collinear, bis, rad, eta, beta, epsilon, arc, anchor, oct, rogers}{A term with the same name appears.}

\marker{ray, halfplane, blade, halfspace, lune, cone}{Involves $\op{aff}_+(S,S')$, with $(\#S,\#S')=(1,1),(2,1),(1,2),(3,1),(2,2),(1,3)$, respectively.}


\marker{segs-meet}{the intersection of two segments.}
\marker{triseg-meet}{the intersection of a segment with the convex hull of three points.}
\marker{aff-meet}{the intersection of affine sets.}
\marker{inside}{a point insides a region formed by a convex hull, a lune, or a cone.}
\marker{deprecated}{The statement has been deprecated and should not be used in the Flyspeck project.}

\end{taglist}

\subsubsection{summary}

Each problem carries a summary statement. This is not meant to be a faithful statement
of all the hypotheses and conclusion.  It is intended as a label to help identify which
problems might be relevant in a given context.

\subsubsection{rating}

The rating is a rough guess about how difficult the result will be to formalize.   These numbers
are of heuristic value only.  Wiedijk's heuristic
is a week of labor to formalize one-page of textbook mathematics.  A rating of $100$ roughly corresponds
to one page of textbook mathematics.   The total rating of the entire collection is about $14,500$.


\subsubsection{usage}

By extracting various results from the main body of the proof,
there is danger that their original context might be forgotten.
For that reason, many of the results are accompanied by usage
notes that indicate the purpose of the result in relation to \cite{DCG}.

The collection of problems and their proofs are completely independent
of these usage notes.   


%>>>>>
\begin{tarskidata}
\begin{tarski}
\name{XX}
\summary{This is a sample of how each page is formatted.}
\tag{bis}
\rating{0}
\guid{SPPYNGU}

\begin{lemma}[generic lemma]\tlabel{tarski:XX}
This lemma is nonsense.
\end{lemma}
\end{tarski}
%<<<<<

%>>>>>
\begin{tarski}
%\section{Function}
\section{Function definitions}\tlabel{def:poly:tarski}
\name{def:tar:delta}
\summary{The polynomials $\ups,\rho,\chi,\Delta$ are defined.}
\tag{def, cm3-ups, cm4-delta, chi, rho}
\rating{0}
\guid{EDSFZOT}

\begin{definition}
We define the following polynomials.
$$
\begin{array}{lll}
\ups(x,y,z) &= -x^2 - y^2 - z^2 + 2 x y + 2 y z + 2 z x.\\
\\
\rho(x_{ij}) &=
   -x_{14}^2 x_{23}^2 - x_{13}^2x_{24}^2 - x_{12}^2 x_{34}^2   \\
 &\quad +2x_{12}x_{14}x_{23}x_{34} + 2x_{12}x_{13}x_{24}x_{34} 
     + 2 x_{13} x_{14}x_{23}x_{24}\\
\\
 \chi(x_{ij}) &= \chi(x_{12},x_{13},x_{14},x_{34},x_{24},x_{23})\\
     &=
      x_{13} x_{23} x_{24} + x_{14} x_{23} x_{24}  + 
      x_{12} x_{23} x_{34} + x_{14} x_{23} x_{34} + x_{12} x_{24} x_{34}\\ 
      &\quad + x_{13} x_{24} x_{34} - 
      2 x_{23} x_{24} x_{34} - x_{12} x_{34}^2 
      - x_{14} x_{23}^2 - x_{13} x_{24}^2\\
   % &= x_1 x_4 x_5 + x_1 x_6 x_4 + x_2 x_6 x_5 + x_2 x_4 x_5 + x_5 x_3 x_6 \\
   %&\quad+ x_3 x_4 x_6 - 2 x_5 x_6 x_4 - x_1 x_4^2 - x_2 x_5^2 - x_3 x_6^2.\\
\\
\Delta(x_{ij}) &= 
   -x_{12} x_{13} x_{23} - x_{12} x_{14} x_{24} - x_{13} x_{14} x_{34} 
    - x_{23} x_{24} x_{34}\\
    &\quad + x_{12} x_{34} (-x_{12} + x_{13} + x_{14} + x_{23} + x_{24} - x_{34}) \\
  &\quad + x_{13} x_{24} (x_{12} - x_{13} + x_{14} + x_{23} - x_{24} + x_{34})\\
    &\quad + x_{14} x_{23} (x_{12} + x_{13} - x_{14} - x_{23} + x_{24} + x_{34})\\
   %x_1 x_4 (- x_1+x_2+x_3- x_4+x_5+x_6)+\\&
   %         x_2 x_5 (x_1- x_2+x_3+x_4- x_5+x_6)
   %         +x_3 x_6 (x_1+x_2- x_3+x_4+x_5- x_6)
   %         - \\&x_2 x_3 x_4- x_1 x_3 x_5- x_1 x_2 x_6- x_4 x_5 x_6\\
\end{array}
    \tlabel{def:chi}\usage{DCG-[5.14]{ def:chi}}
\tlabel{def:tar:delta}
\tlabel{def:ups}
   \indy{Greek}{ZZwchi@$\chi$}
$$
Also, define the following polynomial.
$$
R(x_{ij}) = \det\,\begin{pmatrix}
 0 & x_{12} & x_{13} & x_{14} & x_{15} & 1 \\
 x_{12} & 0 & x_{23} & x_{24} & x_{25} & 1 \\
 x_{13} & x_{23} & 0 & x_{34} & x_{35} & 1 \\
 x_{14} & x_{24} & x_{34} & 0 & x_{45} & 1 \\
 x_{15} & x_{25} & x_{35} & x_{45} & 0 & 1 \\
 1 & 1 & 1 & 1 & 1 & 0
\end{pmatrix}
$$
which we often treat as a quadratic polynomial in $x_{45}$:
$$
R(x_{ij}) = a x_{45}^2  +  b x_{45} + c
$$
where
$$
\begin{array}{lll}
  a &= \ups(x_{12},x_{13},x_{23}) \\
  b &= \ldots\\
  c &= \ldots\\
\end{array}
$$
\end{definition}
Similarly, fix the values of all the variables $x_{ij}\ge 0$ except
$x_{34}$ to view $\mu_\Delta(x_{34})=\Delta(x_{ij})$ as a function of $x_{34}$.  % 4th argument
The polynomial $\chi$ has less symmetry that $\rho$ and $\Delta$.  For that
reason, we need to take greater care with $\chi$ about the order in which the variables are
listed as arguments.  Note that the last three arguments of
$\chi$ contain the three variables $x_{ij}$ whose indices $ij$ satisfy $i\ne 1\ne j$.
Also, the $i$th and $(i+3)$rd argument of $\chi$ contain distinct indices.

We write $\Delta_{(ij)}(x_{ij})$ for the partial derivative of $\Delta$
with respect to $x_{ij}$.
\end{tarski}
%<<<<<





%>>>>>
\begin{tarski}
\name{def:eta}
\summary{The function $\eta$ is defined.}
\tag{def, eta, pt3}
\rating{0}
\guid{MVHAIMQ}

\begin{definition}[circumradius,~$\eta$]\tlabel{def:eta}\usage{DCG-[4.20]{ def:eta}}
\indy{Index}{circumradius}
%
 \indy{Greek}{Zeta@$\eta$}
Let $\eta$ be defined as
	$$\eta(x,y,z) = x y z/\sqrt{\ups(x^2,y^2,z^2)}.$$
Also, if $v_1,v_2,v_3\in\ring{R}^3$, set
   $\eta_V(v_1,v_2,v_3) =\eta(|v_1-v_2|,|v_2-v_3|,|v_1-v_3|)$.
\end{definition}
\end{tarski}
%<<<<<

%>>>>>
\begin{tarski}
\name{def:CM}
\summary{The Cayley-Menger square is defined.}
\tag{def, cm3-ups, cm4-delta, cm5-E}
\rating{0}
\guid{GHCICOE}

%\begin{definition}[Cayley-Menger~square,~CM]
\tlabel{def:CM}
\indy{Index}{Cayley-Menger}
\indy{Index}{CM}
If $v_1,\ldots,v_n\in\ring{R}^3$, define
the Cayley-Menger square $\op{CM}_n$
to be the square
of the $n-1$-dimensional volume of the parallelopiped spanned by the
vectors
	$$v_2-v_1,\ v_3-v_1,\ \ldots\ v_n-v_1.$$
Defined as a square, its value is always non-negative.
It is known for general $n$ that $\op{CM}_n$ is a polynomial in
$x_{ij} = |v_i-v_j|^2$.  (See for example,
\cite{EZ}.)
We will only need the cases, $n=3,4,5$.
In fact, we do not make direct use of the Cayley-Menger determinant at all.  However,
it is implicit in many of the facts about the polynomials $\chi,\Delta,R$ (which
are equal to the Cayley-Menger squares up to scalar factors, for $n=3,4,5$). 
%\end{definition}
\end{tarski}
%<<<<<



%>>>>>
\begin{tarski}
\section{Polynomial identity}

\name{rho-ups}
\summary{This is a polynomial identity in $\rho$, $\ups$, $\chi$, and $\Delta$.}
\tag{chi, ups, rho, delta}
\rating{40}
\guid{RHUFIIB}

\begin{lemma}\tlabel{tarski:rho-ups}
  $$\rho(x_{ij}) \ups(x_{34},x_{24},x_{23}) = \chi(x_{ij})^2 +
  4 \Delta(x_{ij}) x_{34} x_{24} x_{23}$$
\end{lemma}

\begin{proved} Expand both sides.
%It comes from page 61 of Kep2004.
\swallowed\end{proved}
\end{tarski}
%<<<<<


%>>>>>
\begin{tarski}
\name{cayley-menger-pos}
\summary{The positivity of Cayley-Menger.}
\tag{cm3-ups, cm4-delta, cm4-E}
\rating{120}
\guid{NUHSVLM}

\begin{lemma}\tlabel{tarski:cayley-menger-pos}
If $x_{ij} = |v_i-v_j|$ for some points $\{v_1,v_2,v_3\}\subset\ring{R}^3$,  then
$\ups(x_{ij})\ge 0$.
If $x_{ij} = |v_i-v_j|$ for some points $\{v_1,v_2,v_3,v_4\}\subset\ring{R}^3$,  then
$\Delta(x_{ij})\ge 0$.
If $x_{ij} = |v_i-v_j|$ for some points $\{v_1,v_2,v_3,v_4,v_5\}\subset\ring{R}^3$,  then
$R(x_{ij})= 0$.
\end{lemma}

\begin{proved}
The statement $R=0$ can be interpreted as saying the four-dimensional volume of a set in $\ring{R}^3$
is zero.
To see that the polynomials $\ups$ and $\Delta$ are non-negative and that $R=0$, 
volumes are not required.
Write $w_i=v_i-v_1$, and write out components for $w_2,w_3,w_4,w_5$:
$$
w_2 =(r_1,r_2,r_3),\quad w_3 = (s_1,s_2,s_3),\quad w_4= (t_1,t_2,t_3),
\quad w_5 = (u_1,u_2,u_3).
$$
Set $x_{ij} = w_j\cdot w_j$ if $i=1$ and $x_{ij} = 
w_i\cdot w_i + w_j\cdot w_j - 2 w_i\cdot w_j$, to write $x_{ij}$ as a 
polynomial in the components $r_i,s_i,t_i,u_i$.  We obtain the following
positivity results.
   $$
   0 \le 4((r_2s_3-s_2r_3)^2 + (r_1s_3-s_1r_3)^2 + (r_1s_2-s_1r_2)^2) = 
   \ups(x_{12},x_{13},x_{23}),
   $$
    $$
    0\le 4\det(w_2,w_3,w_4)^2 =\Delta(x_{ij}).
    $$
We also, get the following vanishing result when $x_{ij}$ is expressed
as a polynomials in the components.
$$
    0 = R(x_{ij}),\quad x_{ij} = x_{ij}(r_*,s_*,t_*,u_*).
    $$
\swallowed\end{proved}
\end{tarski}
%<<<<<



%>>>>>
\begin{tarski}
\name{heron-ups}
\summary{This is a polynomial identity that relates $\ups$ to Heron's formula.}
\tag{ups}
\rating{40}
\guid{JVUNDLC}

\begin{lemma}\tlabel{tarski:heron-ups}
Let $s = (a+b+c)/2$.  Then $16 s(s-a)(s-b)(s-c) = \ups(a^2,b^2,c^2)$.
\end{lemma}

\begin{proved} Expand both sides. 
\swallowed\end{proved}
\end{tarski}
%<<<<<









%>>>>>
\begin{tarski}
\section{Geometry definition}  
\name{def:packing}
\summary{A packing is defined.}
\tag{def, packing}
\rating{0}
\guid{PUSACOU}

\begin{definition}[packing]\tlabel{def:tar:packing}
If $S$ is a set of point in $\ring{R}^3$, then
$S$ is a packing if $|v-w| < 2$, for $v,w\in S$ implies that $v=w$.
\end{definition}
In many of the definitions and lemmas we give a set $\{v_1,\ldots,v_k\}$
of size $k$.  It is to be understood from this  
that the enumerated elements are distinct: $v_i\ne v_j$, if $i\ne j$.
On the other hand, if we enumerate a set $\{v_1,\ldots,v_k\}$, without
specifying that the size is $k$, then repetitions are allowed.
A packing $S$ of $n$ points means that $S$ is a packing and that $S$
is a finite set of size $n$.
\end{tarski}
%<<<<<



%>>>>>
\begin{tarski}
\name{def:tar:aff}
\summary{The affine hull of a set is defined.}
\tag{def, pt2, pt3, pt4, aff2, aff3, aff4, ray, blade, cone, conv2, conv3, conv4, halfplane, lune, halfspace }
\rating{0}
\guid{QWRUQSK}

\begin{definition}[affine,~aff]\tlabel{def:tar:aff}
\indy{Index}{aff}\indy{Index}{affine}
If $S = \{v_1,v_2,\ldots,v_k\}$ 
and $S'=\{v_{k+1},\ldots,v_n\}$ are  finite sets, then
set
	$$\begin{array}{lll}
      \op{aff}\, S &= \{t_1 v_1 +\cdots t_k v_k \mid
	t_1 +\cdots+t_k = 1\}.\\
        \op{aff}_{\pm} (S,S') &= \{t_1 v_1 +\cdots t_n v_n \mid
	t_1 +\cdots+t_n = 1, \pm t_j \ge 0, \text{ for } j>k.\}.\\
        \op{aff}^0_{\pm} (S,S') &= \{t_1 v_1 +\cdots t_n v_n \mid
	t_1 +\cdots+t_n = 1, \pm t_j > 0, \text{ for } j>k.\}.\\
		\end{array}
        $$
\end{definition}
\end{tarski}
%<<<<<

%>>>>>
\begin{tarski}
\name{def:conv}
\summary{The convex hull of a set is defined.}
\tag{def, conv2, conv3, conv4, pt2, pt3, pt4}
\rating{0}
\guid{BINWIYA}

\begin{definition}[convex hull,~conv]\tlabel{def:conv}
\indy{Index}{conv}\indy{Index}{convex hull}  
If $S = \{v_1,v_2,\ldots,v_n\}$ is a finite set
of points in $\ring{R}^3$, then
set
	$$
        \begin{array}{lll}
          \op{conv}\, S &= \op{aff}_+\, (\emptyset,S)\\
	   \op{conv}^0 S &= \op{aff}^0_+\, (\emptyset,S).\\
           \end{array}
        $$
\end{definition}
\end{tarski}
%<<<<<


%>>>>>
\begin{tarski}
\name{def:cone}
\summary{A ray, blade, and cone are defined.}
\tag{def, ray, blade, cone, pt2, pt3, pt4}
\rating{0}
\guid{PLXSHHI}

\begin{definition}[cone]\tlabel{def:cone}
\indy{Index}{cone}
Let $S=\{v_1,\ldots,v_n\}$ be a finite set of points in 
$\ring{R}^3$.  Let $v\in\ring{R}^3$. Set
  $$\begin{array}{lll}
  \op{cone}(v,S) &= \op{aff}_+(\{v\},S)\\
  \op{cone}^0(v,S) &= \op{aff}^0_+(\{v\},S)\\
  \end{array}
  $$
When $n=1$, we also call the set a ray; and when $n=2$ a blade.
\end{definition}
\end{tarski}
%<<<<<

%>>>>>
\begin{tarski}
\name{def:tar:ball}
\summary{A ball is defined.}
\tag{def}
\rating{0}
\guid{HZMKSUU}
\begin{definition}[ball,~B]\indy{Index}{ball}\tlabel{def:ball}
Set 
$$
B(v_0,r) = \{ x\mid |x - v| < r \},
$$
the open ball of radius $r$ and center $v_0$.
\end{definition}
\end{tarski}
%<<<<<

%>>>>>
\begin{tarski}
\name{def:tar:wedge}
\summary{A wedge is defined.}
\tag{def, lune, pt4}
\rating{0}
\guid{SIDEXYO}

\begin{definition}[wedge,~W]\indy{Index}{wedge}\tlabel{def:tar:wedge}
Let $S=\{v_1,v_2,w_1,w_2\}$ be a set of points in
$\ring{R}^3$.  Assume that $\{v_1,v_2,w_i\}$ is not collinear
for $i=1,2$. Let $D = \det(w_1-v_1,w_2-v_1,v_2-v_1)$.  
Let $W = W(v_1,v_2,w_1,w_2)$ (the wedge) be defined
as follows.  
$$
W=\begin{cases}
\emptyset,& w_2\in \op{aff}_+(\{v_1,v_2\},w_1)\\
\op{aff}_+^0(\{v_1,v_2\},\{w_1,w_2\}),& D > 0\\
\op{aff}_+(\{v_1,v_2,w_1\},(w_1-v_1)\times (v_2-v_1)),&
    w_2\in\op{aff}_-^0(\{v_1,v_2\},w_1)\\
\ring{R}^3\setminus\op{aff}_+(\{v_1,v_2\},\{w_1,w_2\}),& D < 0\\
\end{cases}
$$
where 
$$
(x_1,y_1,z_1)\times (x_2,y_2,z_2) = 
(y_1 z_2 - y_2 z_1, z_1 x_2 - z_2 x_1, x_1 y_2 - x_2 y_1).
$$
\end{definition}
\end{tarski}
%<<<<<

%>>>>>
\begin{tarski}
\name{def:omega}
\summary{The Voronoi cell is defined.}
\tag{def, voronoi}
\rating{0}
\guid{SWSAMQE}

\begin{definition}[Voronoi~cell,~$\Omega$] \tlabel{def:omega}
\indy{Index}{Voronoi}\indy{Greek}{ZZzomega@$\Omega$}
Let $S$ be a finite set of points in 
$\ring{R}^3$.  Let $v\in\ring{R}^3$. Set
  $$
  \Omega(v,S) = \{x \mid |v-x| < |w-x|, \forall w\in S\setminus\{v\}\}
  $$
and
  $$
  \bar\Omega(v,S) = \{x \mid |v-x| \le |w-x|, \forall w\in S\setminus\{v\}\}
  $$
\end{definition}
\end{tarski}
%<<<<<
	
%>>>>>
\begin{tarski}
\name{def:line}
\summary{A line is defined.}
\tag{def, aff2, pt2, collinear}
\rating{0}
\guid{LFQMLPU}

\begin{definition}[line]\tlabel{def:line}
\indy{Index}{line}
Any set of the form $\op{aff}\{v,w\}$ for some $v\ne w$ is called a 
 {\it line.}
\end{definition}
\end{tarski}
%<<<<<

%>>>>>
\begin{tarski}
\name{def:collinear}
\summary{A collinear set is defined.}
\tag{def, collinear}
\rating{0}
\guid{PPZSAYG}

\begin{definition}[collinear]\tlabel{def:collinear}  
\indy{Index}{collinear}
A set $S$ is collinear if there exists
a line that contains every point of $S$.
\end{definition}
\end{tarski}
%<<<<<

%>>>>>
\begin{tarski}
\name{def:plane}
\summary{A plane and half-plane are defined.}
\tag{def, pt3, plane, collinear, halfplane}
\rating{0}
\guid{BUGLQNN}

\begin{definition}[plane,~half~plane]\tlabel{def:plane}
\indy{Index}{plane}\indy{Index}{half-plane}
If $A=\op{aff}\{u,v,w\}$ for some set $\{u,v,w\}$ that is not collinear,
then $A$ is a {\it plane.}  Sets $\op{aff}_+(\{u,v\},\{w\})$ (and $\op{aff}_-^0(\{u,v\},\{w\})$) with
$\{u,v,w\}$ not collinear, are called half-planes (and open-half planes respectively).  
\end{definition}
\end{tarski}
%<<<<<

%>>>>>
\begin{tarski}
\name{def:coplanar}
\summary{A coplanar set is defined.}
\tag{def, plane}
\rating{0}
\guid{MHHXNTW}

\begin{definition}[coplanar]\tlabel{def:coplanar}
\indy{Index}{coplanar}
A set $S$ is  coplanar if there exists
a plane that contains every point of $S$.
\end{definition}
\end{tarski}
%<<<<<

%>>>>>
\begin{tarski}
\name{def:bisector}
\summary{The bisecting plane of two points is defined.}
\tag{def, bis, pt2}
\rating{0}
\guid{WMJHKBL}

\begin{definition}[bisector]\tlabel{def:bisector}
\indy{Index}{bisector}
The set of points 
   $$
   \{ x \mid |x - u | = |x-v|\}
   $$
is called the bisector of $\{u,v\}$ and is denoted
$\op{bis}\{u,v\}$.
\end{definition}
\end{tarski}
%<<<<<

%>>>>>
\begin{tarski}
\name{def:bispace}
\summary{The halfspace bounded by the bisecting plane is  defined.}
\tag{def, bis, voronoi, pt2}
\rating{0}
\guid{TIWZVEW}

\begin{definition}[$\op{bis}_+$]\tlabel{def:bispace}
\indy{Index}{bisector}
The set of points
   $$
   \{ x \mid |x- u | \le |x-v|\}
   $$
of points at least as close to $u$ as to $v$ is denoted
$\op{bis}_+(u,v)$.  If the inequality is strict, it is denoted
$\op{bis}_+^0(u,v)$ or $\Omega(u,\{u,v\})$.
\end{definition}
\end{tarski}
%<<<<<

%>>>>>
\begin{tarski}
\name{def:half-space}
\summary{A half-space is defined.}
\tag{def, halfspace}
\rating{0}
\guid{QTQNLKK}

\begin{definition}[half~space]\tlabel{def:half-space}
\indy{Index}{half-space}
A set $\op{aff}_{\pm}(\{u,v,w\},\{v'\})$,
with $\{u,v,w,v'\}$ not coplanar, is called a half-space.  If
we replace $\op{aff}_{\pm}$ with $\op{aff}_{\pm}^0$ we have an
{\it open half-space}.
\end{definition}
\end{tarski}
%<<<<<






%>>>>>
\begin{tarski}
\name{def:circum2}
\summary{The circumcenter of a set of three points is defined.}
\tag{def, pt3, circum3}
\rating{0}
\guid{XCJABYH}

\begin{definition}[circumcenter]\tlabel{def:circum2}
\indy{Index}{circumcenter}
Let $S=\{x,v_1,v_2,v_3\}$ be a coplanar set in $\ring{R}^3$.  If $x$ 
is equidistant from all three points $v_i$, then $x$ is called a
circumcenter of $\{v_1,v_2,v_3\}$.
\end{definition}
\end{tarski}
%<<<<<

%>>>>>
\begin{tarski}
\name{def:circumrad2}
\summary{The circumradius of a set of three points is defined.}
\tag{def, pt3, circum3, eta}
\rating{0}
\guid{XPLPHNG}

\begin{definition}[circumradius]\tlabel{def:circumrad2}  
\indy{Index}{circumradius}
Let $S=\{v_1,v_2,v_3\}$ be 
a set of three points in $\ring{R}^3$.
The common distance from a circumcenter of $S$ to $v_i$ is called a
circumradius of $S$.  
\end{definition}
\end{tarski}
%<<<<<

%>>>>>
\begin{tarski}
\name{def:circum3}
\summary{The circumcenter of a set of four points is defined.}
\tag{def, pt4, circum4}
\rating{0}
\guid{ZSCVHXC}

\begin{definition}[circumcenter]\tlabel{def:circum3}
\indy{Index}{circumcenter}
Let $S=\{v_1,v_2,v_3,v_4\}\subset \ring{R}^3$.
A point $x\in\ring{R}^3$ 
that is equidistant from every $v_i\in S$ is called a
circumcenter of $S$.  
\end{definition}
\end{tarski}
%<<<<<

%>>>>>
\begin{tarski}
\name{def:rad}
\summary{This gives notation for the circumradius.}
\tag{def, circum4, pt4, rad}
\rating{0}
\guid{PPNZQUS}

\begin{definition}[circumradius,~$\rad$]\tlabel{def:rad}  
\indy{Index}{circumradius}\indy{Index}{rad}
Let $S=\{v_1,v_2,v_3,v_4\}\subset\ring{R}^3$.
The common distance from a circumcenter of $S$ to $v_i$ is called a
circumradius of $S$.  We write
$\rad_V(S)$ or $\rad_V(v_1,v_2,v_3,v_4)$ for a circumradius of $S$.
\end{definition}
\end{tarski}
%<<<<<


%>>>>>
\begin{tarski}
\name{def:orientation}
\summary{The orientation of vertex of a simplex is given by the relation between the vertex, the circumcenter, and the opposite face.}
\tag{def, chi, pt4, circum4}
\rating{0}
\guid{EOBLRCS}

\begin{definition}[orientation] \tlabel{def:orientation}\usage{DCG-[5.12]{ def:orientation}}
\indy{Index}{orientation}
Let $S\subset\ring{R}^3$ be a set of four points.
We say that the {\it orientation\/} of $v\in S$ is
{\it negative} if the plane separates the
circumcenter of the simplex from the vertex of the simplex that
does not lie on the face.  The orientation is positive if the
circumcenter and the vertex lie on the same side of the plane. The
orientation is zero if the circumcenter lies in the plane.
More precisely, let $p$ be a circumcenter of $S$.  Let $S'=S\setminus\{v\}$.  
Then
   $$
   \begin{array}{lll}
     p\in \op{aff}_-^0(S',\{v\}) &\Leftrightarrow &\text{negative}\\
     p\in \op{aff}_+^0(S',\{v\}) &\Leftrightarrow &\text{positive}\\
     p\in \op{aff} (S') &\Leftrightarrow &\text{zero}\\
     \end{array}
   $$
\indy{Index}{orientation}
\end{definition}
\end{tarski}
%<<<<<



%>>>>>
\begin{tarski}
\name{def:rcone}
\summary{A right circular cone is defined.}
\tag{def, rcone, pt2}
\rating{0}
\guid{NKOATPV}

\begin{definition}[rcone]\tlabel{def:rcone} 
\indy{Index}{rcone}\indy{Index}{right-circular cone}
If $v$ and $w$ are points in $\ring{R}^3$, and
  $h\in\ring{R}$, then set
  $$\begin{array}{lll}
    \op{rcone}(v,w,h) &= \{x\mid (x-v)\cdot (w-v) \ge |x-v|\,|w-v| h\},\\
    \op{rcone}^0(v,w,h) &= \{x\mid (x-v)\cdot (w-v) > |x-v|\,|w-v| h\}.\\
    \op{rcone}_-^0(v,w,h) &= \{x\mid (x-v)\cdot (w-v) < |x-v|\,|w-v| h\}.\\
    \partial\op{rcone}(v,w,h) &= \{x\mid (x-v)\cdot (w-v) = |x-v|\,|w-v| h\}.\\
    \end{array}
    $$
\end{definition}
\end{tarski}
%<<<<<


%>>>>>
\begin{tarski}
\name{def:rog-repeat}
\summary{A Rogers simplex can be defined as a special type of orthosimplex.}
\tag{def, rogers, pt4, eta, circum3}
\rating{0}
\guid{SJHNHRR}

\begin{definition}[Rogers~simplex,~rog] \tlabel{def:rog-repeat}
\indy{Index}{rog}
\indy{Index}{rogers simplex}
\indy{Index}{ortho}
% REPEATED DEFINITION from primitive volume
% edited to eliminate ortho.
Let $\{v_0,v_1,v_2,v_3\}$ be a set of four points in $\ring{R}^3$.
Assume that they are not coplanar.  Let $p$ be the circumcenter
of $\{v_0,v_1,v_2\}$ and $\eta$ its circumradius.  Let $c\ge r$.
By Lemma~\tref{tarski:rog-exist}, there exists a unique
point $p'$ in $A=\op{aff}_+(\{v_0,v_1,v_2\},v_3)$ at equal distance $c$
from $v_0,v_1,v_2$.
Let $$
    \begin{array}{rll}
    \op{rog}^0(v_0,v_1,v_2,v_3,c) &= 
    &= \op{conv}^0(v_0,(v_0+v_1)/2,p,p'),\\
    \end{array}
    $$
(We also define $\op{rog}(v_0,\ldots,v_3,c)$, where we use
$\op{conv}$ instead of $\op{conv}^0$.)
We take $\op{rog}^0$ to be the empty set, if $c< r$.
 \indy{Index}{rogers simplex}
\end{definition}
This completes the listing of the main definitions that
are used in this collection.  There are a number of other
definitions of a technical nature of limited use that
are interspersed with the collection of problems.
This includes the definition of the barycentric coordintes,
a trigonometric function $\beta_\psi$ (Definition~\ref{def:beta}),
two functions $\epsilon$ and $\epsilon'$ related to the
facets of Voronoi cells $\Omega$ (Definitions~\ref{def:epsilon}
and \ref{def:epsilonp}), and the notion of an octahedron
(Definition~\ref{def:tarski:oct}).  A detailed index lists
all defined terms.
\end{tarski}
%<<<<<










%>>>>>
\begin{tarski}
%\section{Two Points}
\section{Two points: line}
\name{two-line}
\summary{Two distinct points determine a unique line.}
\tag{aff2, pt2}
\rating{40}
\guid{RCEABUJ}

\begin{lemma}\tlabel{tarski:two-line}
	If $v_1\ne v_2$ are points in $\ring{R}^3$, then $\op{aff}\{v_1,v_2\}$ is the unique
line containing $v_1$ and $v_2$.
\end{lemma}

\begin{proved}
\swallowed\end{proved}
\end{tarski}
%<<<<<





%>>>>>
\begin{tarski}
\name{bis-plane}
\summary{The bisector is a plane.}
\tag{bis, pt2}
\rating{60}
\guid{BXVMKNF}

\begin{lemma}
Let $\{u,v\}$ be a set of two distinct points in 
$\ring{R}^3$.  Then $\op{bis}\{u,v\}$ is a plane.
\end{lemma}

\begin{proved}
\swallowed\end{proved}
\end{tarski}
%<<<<<






%>>>>>
\begin{tarski}
%\section{Three Points}
\section{Three points: Cayley-Menger}
\name{three-plane}
\summary{Three non-collinear points determine a unique plane.}
\tag{aff3, pt3}
\rating{80}
\guid{SMWTDMU}

\begin{lemma}
	Let $v_1,v_2,v_3$ be points in $\ring{R}^3$.  Assume that they are not collinear.
Then $\op{aff}\{v_1,v_2,v_3\}$ is the unique plane containing all three points.
\end{lemma}

\begin{proved}
\swallowed\end{proved}
\end{tarski}
%<<<<<




%\name{cm3-ups}
%\summary{The polynomial $\ups$ can be expressed as a sum of squares.}
%\tag{cm3-ups, pt3, deprecated}
%\rating{0} % was 100
%\guid{JITVRKC}
%
%\begin{lemma}
%	Let $\op{CM}_3(x_{ij})$ be the Cayley-Menger square for
%three points $v_1,\ldots,v_3$ in $\ring{R}^3$, with $|v_i-v_j|^2 = x_{ij}$,
%for $1\le i < j \le 3$.  Then
%	$$\op{CM}_3(x_{ij}) = \ups(x_{12},x_{23},x_{13})/4.$$
%\end{lemma}
%
%\begin{proved}
%\swallowed\end{proved}



%\name{ups-nonneg}
%\summary{The polynomial $\ups$ is non-negative. Deprecated because it duplicates an earlier result.}
%\tag{ups, pt3, deprecated}
%\rating{40}
%\guid{KXTAOXF}
%
%\begin{lemma}
%Let $v_1,\ldots,v_3$ be three points
%in $\ring{R}^3$.  Let $x_{ij} = |v_i-v_j|^2$.  Then
%	$$\ups(x_{12},x_{23},x_{x13})\ge0.$$
%\end{lemma}
%
%\begin{proved}
%\swallowed\end{proved}
%\end{tarski}






%>>>>>
\begin{tarski}
\name{dependent-linear}
\summary{A linear relation among three points implies collinearity.}
\tag{collinear, pt3}
\rating{80}
\guid{SGFCDZO}

\begin{lemma}
Let $S = \{v_1,v_2,v_3\}$ be a set of three
points in $\ring{R}^3$.  If there exist $t_1,t_2,t_3$, not all zero,
such that  $$
           t_1 v_1 + t_2 v_2 + t_3 v_3 = 0,\quad t_1+t_2+t_3=0,
           $$
then $\{v_1,v_2,v_3\}$ is a collinear set.
\end{lemma}

\begin{proved}
If say $t_1\ne0$, then $v_1\in \op{aff}\{v_2,v_3\}$ and all three points
lie in the line $\op{aff}\{v_2,v_3\}$.
\swallowed\end{proved}
\end{tarski}
%<<<<<


%>>>>>
\begin{tarski}
\name{ups0}
\summary{Three points are collinear exactly when $\ups$ is zero.}
\tag{ups, pt3, collinear.}
\rating{60}
\guid{FHFMKIY}

\begin{lemma}\tlabel{tarski:ups0}
Let $v_1,v_2,v_3$ be three points
in $\ring{R}^3$.  Let $x_{ij} = |v_i-v_j|^2$.
The points $v_1,v_2,v_3$ are collinear if and only if
	$$\ups(x_{12},x_{23},x_{13}) = 0.$$
\end{lemma}

\begin{proved}
By Lemma~\ref{tarski:cayley-menger-pos},
$\ups=0$ iff $r\times s = 0$ iff $\alpha r = \beta s$, where $\alpha$ and $\beta$ are
not both zero.
\swallowed\end{proved}
\end{tarski}
%<<<<<






%>>>>>
\begin{tarski}
\section{Triangle inequality}
\name{triangle-eq}
\summary{This is the case of equality in the triangle inequality.}
\tag{conv2, pt3}
\rating{40}
\guid{ZPGPXNN}

\begin{lemma} \tlabel{tarski:triangle-eq}
Let $v_1,v_2,v\in\ring{R}^3$.  
Assume that
	$$|v_1-v_2| < |v_1-v| + |v - v_2|.$$
Then $v\not\in\op{conv}\{v_1,v_2\}$.
\end{lemma}

\begin{proved}
For a contradiction, assume $v= t_1 v_1 + t_2 v_2$ with $t_1+t_2=1$, $t_i\ge0$.
Then 
$$
|v_1-v| + |v-v_2| = t_2 |v_1-v_2| + t_1 |v_1-v_2| = |v_1-v_2|.
$$
\swallowed\end{proved}
\end{tarski}
%<<<<<







%>>>>>
\begin{tarski}
%\section{Four Points}
\section{Four points: barycentric coordinate}
\tlabel{tarski:sec:pc2}
\indy{Index}{barycentric coordinates}
This section is nearly identical to Section~\tref{tarski:sec:pc3}, which adds one more point
to the mix.

\name{bary3}
\summary{In a plane, a fourth point can be expressed in the barycentric coordinates of three non-collinear points.}
\tag{bary3, pt4, plane}
\rating{80}
\guid{FAFKVLR}

\begin{lemma}\tlabel{tarski:bary3}
Let $S=\{v_1,\ldots,v_4\}\subset\ring{R}^3$.  
Suppose
that $\{v_1,\ldots,v_3\}$ is not a collinear
set.   Suppose that $v_4$ lies in the plane of
$\{v_1,\ldots,v_3\}$.  
Then there exist unique real numbers
$t_1,\ldots,t_3$ such that $\sum_i t_i = 1$ and
	$$v_4 = \sum_i t_i v_i.$$
\end{lemma}

\begin{proved}  By the definition of plane,
there 
exists a solution to this system of equations.
This is a linear system with
two unknowns:
	$$(v_4- v_3) = t_1 (v_1-v_3) +
		t_2 (v_2-v_3).
	$$
If this has a different expression as
   $$
   (v_4- v_3) = t'_1 (v_1-v_3) +
		t'_2 (v_2-v_3),
   $$
then 
  $$
  (t_1-t_1') v_1 + (t_2-t_2') v_2 + (t'_1-t_1+t_2'-t_2) v_3 = 0,
  $$
and $\{v_1,v_2,v_3\}$ is collinear.
\swallowed\end{proved}
\end{tarski}
%<<<<<



%>>>>>
\begin{tarski}
\name{def:coef}
\summary{Some notation can be introduced for barycentric coordinates.}
\tag{def, bary3, pt4}
\rating{0}
\guid{ZAMOYCJ}

\begin{definition}[coef]\tlabel{def:coef} 
\indy{Index}{coef}
Let $\op{coef}_i(v_1,\ldots,v_4)$
be the constant $t_i$ from Lemma~\ref{tarski:bary3}.
\end{definition}
\end{tarski}
%<<<<<




%>>>>>
\begin{tarski}
\name{coef-plane-sign}
\summary{The signs of barycentric coordinates determine whether a point lies in a halfspace or line.}
\tag{bary3, pt4, halfplane, aff2}
\rating{80}
\guid{CNXIFFC}

\begin{lemma}\tlabel{tarski:coef-plane-sign}
Suppose
that $\{v_1,\ldots,v_3\}\subset\ring{R}^3$ is not a collinear
set. 
Suppose that $v_4$ lies in the plane of $S$.
Let $S_i = S\setminus \{v_i\}$ and $c_i = \op{coef}_i(v_1,v_2,v_3,v_4)$. 
Then for $i=1,2,3$, 
   $$
   \begin{array}{lll}
     c_i > 0  & \Leftrightarrow & v_i \in \op{aff}_+^0(S_i,\{v_4\})\\
     c_i = 0 & \Leftrightarrow & v_i \in \op{aff}(S_i)\\
     c_i < 0 & \Leftrightarrow & v_i \in \op{aff}_-^0(S_i,\{v_4\})\\
     \end{array}
   $$
\end{lemma}

\begin{proved}
  Let $L_i =\op{aff}(S_i)$.
If $\op{coef}_i(v_1,\ldots,v_4)=0$, then $v_4$
has the form
	$$
	v_4 = t_1 v_1 + \cdots + t_3 v_3,
	$$
with $t_i=0$.  It is clear that these are
precisely the points in the given line $L_i$.
Now if $t_i>0$,
elements of
$\op{conv}\{v_4,v_i\}$ have the form
	$$s v_4 + (1-s) v_i,\quad 0\le s \le 1.$$
The coefficient of $v_i$ in this expression
is $1-s(1-t_i)>0$.  Thus, $\op{conv}\{v_4,v_i\}$
does not meet the line $L_i$, and $v_i$ and $v_4$
lie in the same half-plane.  Similarly,
if $t_i<0$, then $1-s(1-t_i)=0$ has a solution
in $s\in(0,1)$, so that the line $L_i$ separates the
points $v_i$ and $v_4$.
\swallowed\end{proved}
\end{tarski}
%<<<<<




%>>>>>
\begin{tarski}
\name{conv-bary0}
\summary{A point lies in the convex hull of three other points when its barycentric coordinates are non-negative. A similar
  statement holds when the barycentric coordinates are positive.}
\tag{collinear, conv3, bary3, pt4}
\rating{30}
\guid{MYOQCBS}

\begin{lemma}
Suppose
that $T=\{v_1,\ldots,v_3\}$ is not a collinear
set. 
Suppose that $v_4$ lies in the plane of $T$.  
 Then $v_4\in\op{conv}(T)$ if and only
if  
$\op{coef}_i(v_1,\ldots,v_4)\ge0$ 
for $i=1,2,3$.
Similarly, $v_4\in\op{conv}^0(T)$ if and only
if  
$\op{coef}_i(v_1,\ldots,v_4)>0$ 
for $i=1,2,3$.
\end{lemma}

\begin{proved}  This is trivial by the definitions
of $\op{conv}$, $\op{conv}^0$ and $\op{coef}$.
\swallowed\end{proved}
\end{tarski}
%<<<<<





%>>>>>
\begin{tarski}
\name{convex-ball}
\summary{A simplex lies in a ball if its vertices do.}
\tag{pt4, conv4, ball}
\rating{40}
\guid{TXDIACY}

\begin{lemma}\tlabel{tarski:convex-ball}
Let $S$ be a set of four points in $\ring{R}^3$.
Let $r>0$ and $v_0\in\ring{R}^3$.  If $S\subset B(v_0,r)$, then
$\op{conv}(S)\subset B(v_0,r)$.
\end{lemma}

\begin{proved}  Let $x\in \op{conv}(S)$.
Since $B=B(v_0,r)$ is convex,  a convex combination of points in $B$
again lies in $B$.  Hence $x\in B$.
\swallowed\end{proved}
\end{tarski}
%<<<<<


%>>>>>
\begin{tarski}
\name{delta0}
\summary{Four points are coplanar exactly when $\Delta$ is zero.}
\tag{pt4, plane, cm4-delta}
\rating{80}
\guid{POLFLZY}

\begin{lemma}\tlabel{tarski:delta0}
Let $v_1,\ldots,v_4$ be four points
in $\ring{R}^3$.  Let $x_{ij} = |v_i-v_j|^2$.
The points $v_1,\ldots,v_4$ are coplanar if and only if
	$$\Delta(x_{12},\ldots,x_{23}) = 0.$$
\end{lemma}

\begin{proved}  From Lemma~\ref{tarski:cayley-menger-pos},
  $$
  \Delta = 4\det(v_2-v_1,v_3-v_1,v_4-v_1)^2.
  $$
The determinant is zero if and only if the points are coplanar.
\swallowed\end{proved}
\end{tarski}
%<<<<<


%>>>>>
\section{Circumradius}
\begin{tarski}
\name{construct-pt-plane}
\summary{Construct a point in the plane.}
\tag{pt4, delta, ups}
\rating{100}
\guid{SDIHJZK}

\begin{lemma}\tlabel{tarski:construct-pt-plane}  
Assume that $S=\{v_1,v_2,v_3\}\subset\ring{R}^3$ is not collinear.
Let $a_{01},a_{02},a_{03}\in\ring{R}$ be such that
$$
   \Delta(a_{01},a_{02},a_{03},x_{23},x_{13}x_{12})=0,
$$
where $x_{ij}=|v_i-v_j|^2$.
Then there exists a unique point $v_0$ such that $|v_0-v_i|^2 = a_{0i}$, for $i=1,2,3$.
In fact, 
  $$
  v_0 = t_1 v_1 + t_2 v_2 + t_3 v_3, \hbox{ where } t_i = \frac{\Delta_{(01)}}{\ups},
  $$
$\ups=\ups(x_{23},x_{13},x_{12})$ and $\Delta_{(ij)} = \partial\Delta/\partial x_{ij}(a_{01},\ldots,x_{12})$.
Moreover, $v_0$ lies in the plane $\op{aff}\{v_1,v_2,v_3\}$.
\end{lemma}

\begin{proved}  
%Let $\ups=\ups(x_{23},x_{13},x_{12})$.  Write $\Delta_{(ij)} = \partial\Delta/\partial x_{ij}(a_{01},\ldots,x_{12})$.
%Set 
%$$
%  v_0 = t_1 v_1 + t_2 v_2 + t_3 v_3,\hbox{ where } t_i = \frac{\Delta_{(01)}}{\ups}.
%$$
Note that $t_1+t_2+t_3=1$, so that $v_0\in\op{aff}\{v_1,v_2,v_3\}$.
A direct calculation shows that $v_0$ has the required properties.
For example,
$$
 |v_0-v_1|^2 = |t_2 (v_2-v_1) + t_3 (v_3-v_1)|^2 = t_2^2 x_{12} + t_2 t_3 (x_{13}+x_{12}-x_{23})  + t_3^2 x_{13} =
 a_{01} - \frac{\Delta}u = a_{01}.
$$
If $v_0'$ is a second solution, then $v_0'\in\op{aff}\{v_1,v_2,v_3\}$ by Lemma~\ref{tarski:delta0}.
Also, if $w=v_0-v_0'$ then, we may write
   $$w = t_1 (v_1-v_3) + t_2 (v_2-v_3)$$
From the equations 
    $$
    |v_0-v_i|^2 = |v_0'-v_i|^2,
    $$
it follows that $w\cdot (v_i-v_3) = 0$, so also that
    $$w\cdot w = \sum t_i w\cdot (v_i-v_3) = 0.$$
Hence $w=0$ and $v_0'=v_0$.  (Compare the uniqueness proof in Lemma~\ref{tarski:circumcenter}.)
\swallowed\end{proved}
\end{tarski}
%<<<<<








%>>>>>
\begin{tarski}
\name{circum2}
\summary{This  calculates the barycentric coordinates of the circumcenter of a triangle.  It also asserts the
unique existence of the circumcenter, and identifies the circumradius with the function $\eta$.}
\tag{pt3, circum3, ups, eta}
\rating{80}
\guid{CDEUSDF}

\begin{lemma}\tlabel{tarski:circum2}
Let $S= \{v_a,v_b,v_c\}$ be a set of three points in $\ring{R}^3$.
Set $a = |v_b
- v_c|$, $b = |v_a - v_c|$, $c = |v_a  - v_b |$.  If $v_a,v_b,v_c$
are not collinear then a unique circumcenter exists.  The circumcenter is
    $$p = \alpha_a v_a + \alpha_b v_b + \alpha_c v_c$$
where
    $$%\begin{array}{lll}
    \alpha_a = \frac{a^2(- a^2 + b^2 + c^2)}{2\ups(a^2,b^2,c^2)},\quad
    \alpha_b = \frac{b^2(a^2 - b^2 + c^2)}{2\ups(a^2,b^2,c^2)},\quad
    \alpha_c = \frac{c^2(a^2 + b^2 - c^2)}{2\ups(a^2,b^2,c^2)}.
    %\end{array}
    $$
The circumradius equals $\eta(a,b,c)$.
\end{lemma}


\begin{proved} Note that for $r\ge 0$,
$$\Delta(r,r,r,a^2,b^2,c^2) = -a^2 b^2 c^2 + r^2\ups(a^2,b^2,c^2),$$
which vanishes exactly when $r = \eta(a^2,b^2,c^2)$.  (If $r$ is not this value, 
then $\Delta\ne0$ and by Lemma~\ref{tarski:delta0} there is no point in the plane at distance
$r$ from $v_a,v_b,v_c$.)  By Lemma~\ref{tarski:construct-pt-plane}, the circumcenter exists and
we find that $\eta(a^2,b^2,c^2)$ is the circumradius.  That lemma also gives a formula
for the circumcenter, which specializes to the given formula in the lemma.
Compare
Lemma~\tref{tarski:circumcenter}.
\swallowed\end{proved}
\end{tarski}
%<<<<<




%>>>>>
\begin{tarski}
\name{circum-acute}
\summary{In an acute triangle, the circumcenter lies in the interior of the triangle.}
\tag{pt3, circum3, halfplane, conv2}
\rating{60}
\guid{WSMRDKN}

\begin{lemma}\tlabel{tarski:circum-acute}
Let  $S=\{v_1,v_2,v_3\}$ be a set of three points
in $\ring{R}^3$. Assume 
   $$
   |v_1-v_2|^2 + |v_2-v_3|^2 > |v_1-v_3|^2.
   $$
Let $p$ be the circumcenter of $S$.
Then $p$ and $v_2$ lie on the same side of the line $\op{aff}\{v_1,v_3\}$
in the plane $\op{aff}\{v_1,v_2,v_3\}$.
\end{lemma}

\pdffig{WSMRDKN}{circum-acute}{$p$ and $v_2$ lie in the same shaded region.}

\begin{proved} It is enough to look at the signs of the terms in
Lemma~\tref{tarski:circum2} and use Lemma~\tref{tarski:coef-plane-sign}.
\swallowed\end{proved}
\end{tarski}
%<<<<<





%>>>>>
\begin{tarski}
\name{eta:mono}
\summary{The circumradius is monotonic on acute triangles.}
\tag{eta, pt3, circum3}
\rating{60}
\guid{BYOWBDF}

\begin{lemma}\tlabel{tarski:eta:mono}
Let $(a,b,c)$ and $(a',b',c')$ be two triples of real numbers.
Suppose that $0 < a\le a'$, $0 < b\le b'$, $0< c\le c'$.  Suppose that
   $$
   a'^2 \le b^2 + c^2,\quad b'^2 \le a^2 + c^2,\quad c'^2\le a^2 + b^2.
   $$
Then
   $$
   \eta(a,b,c) \le \eta(a',b',c').
   $$
\end{lemma}

\pdffig{BYOWBDF}{eta:mono}{Monotonicity of the circumradius.}

\begin{proved} We compute
	$$\partial\eta(a,b,c)/\partial a = 
        \frac{b c (a^2 - b^2 + c^2)(a^2 + b^2 - c^2)}{\ups(a^2,b^2,c^2)^{3/2}}.
	$$
This is nonnegative.  
\swallowed\end{proved}
\end{tarski}
%<<<<<





%>>>>>
\begin{tarski}
\name{circum-sqrt2}
\summary{With tight bounds on the lengths of the edges of a triangle, the circumradius is less than $\sqrt2$.}
\tag{pt3, sqrt2, eta}
\rating{60}
\guid{BFYVLKP}

\begin{lemma}
\tlabel{tarski:eta-2.2}
Let $S=\{v_1,v_2,v_3\}\subset\ring{R}^3$.
Assume that
	$$
	\begin{array}{rlrlrl}
		2\le|v_1-v_2|&\le 2.52, &2\le|v_1-v_3|&\le 2.2, &2\le|v_2-v_3|&\le 2.2\\
	\end{array}
	$$
Then the $S$ is not a collinear set and the circumradius of 
$S$ is less than $\sqrt2$.
\end{lemma}

\begin{proved}
By the monotonicity of the circumradius (Lemma~\ref{tarski:eta:mono}), we have
	$$\eta(x,y,z)\le \eta(2.2,2.2,2.52) < \sqrt2.$$
\swallowed\end{proved}
\end{tarski}
%<<<<<



%>>>>>
\begin{tarski}
\name{1453}
\summary{With bounds on the edges of a triangle, the circumradius is at most $1.453$.}
\tag{pt3, eta, sqrt2, 1.453}
\rating{40}
\guid{WDOMZXH}

\begin{lemma}\tlabel{tarski:1453}
If $2\le y\le\sqrt8$, then $\eta(y,2,2.51) < 1.453$.
\end{lemma}

\begin{proved}
\swallowed\end{proved}
\end{tarski}
%<<<<<



%>>>>>
\begin{tarski}
\name{eta245}
\summary{With bounds on the edges of a triangle, the circumradius is at least $\sqrt2$.}
\tag{pt3, eta, t0, 2.77, sqrt2}
\rating{40}
\guid{ZEDIDCF}

\begin{lemma}\tlabel{tarski:eta245}
Let $S=\{v_0,v_1,v_2\}$ be a set of three points in $\ring{R}^3$.
Assume that 
  $$
  2\le|v_0-v_1|\le 2.51,\quad 2.45\le|v_1-v_2|\le 2.51,
  \quad 2.77\le |v_0-v_2|\le \sqrt8.
  $$
Then the circumradius of $S$ is greater than $\sqrt2$.
\end{lemma}


\begin{proved}
By the monotonicity of the circumradius (Lemma~\tref{tarski:eta:mono}), 
we have
that it is at least $\eta(2,2.45,2.77) > \sqrt2$.
\swallowed\end{proved}
\end{tarski}
%<<<<<





%>>>>>
\begin{tarski}
\name{eta696}
\summary{This lemma compares the circumradius of two constrained triangles.}
\tag{eta, 2.45, t0, 2.696, sqrt2}
\rating{80}
\guid{NHSJMDH}

\begin{lemma}\tlabel{tarski:eta696}
If $y\in[2.696,\sqrt8]$, then
$\eta(y,2.45,2.45)\ge\eta(y,2,2.51)$.
\end{lemma}

\begin{proved}
We argue as follows. The function
  $$f(x) = x(\eta(\sqrt{x},2.45,2.45)^{-2}-\eta(\sqrt{x},2,2.51)^{-2})
  $$
is a quadratic
polynomial in $x$ with negative values for
$\sqrt{x}=y\in[2.696,\sqrt{8}]$. From $f(x)\le 0$, the result follows.
\swallowed\end{proved}
\end{tarski}
%<<<<<





%>>>>>
\begin{tarski}
\name{eta-min}
\summary{The circumradius is at least the length of a longest side of a triangle.}
\tag{eta, pt3, ups}
\rating{40}
\guid{HVXIKHW}

\begin{lemma}\tlabel{tarski:eta-min}
Assume $\ups(x^2,y^2,z^2)>0$ and that $x,y,z\ge0$.  Then
$\eta(x,y,z)\ge \max(x,y,z)/2$.
\end{lemma}

\begin{proved} 
  $$\eta(x,y,z)^2 - (x/2)^2 =
  \frac{x^2 (x^2-y^2-z^2)^2}{\ups(x^2,y^2,z^2)} \ge0.
  $$
\swallowed\end{proved}
\end{tarski}
%<<<<<



\pdffig{HVXIKHW}{tarski:eta-min}{The circumradius is at least half 
an edge length.}


%>>>>>
\begin{tarski}
\name{eta-root3}
\summary{The minimum circumradius of a triangle in a packing is calculated.}
\tag{packing, pt3, eta}
\rating{40}
\guid{HMWTCNS}

\begin{lemma}\tlabel{tarski:eta-root3}
Let $S$ be a packing of three points.
 Then its circumradius is at least $2/\sqrt3$.
\end{lemma}

\begin{proved} If any side has length at least $4/\sqrt3$, then
it follows from Lemma~\tref{tarski:eta-min}.  Otherwise, it
follows from the monotonicity of $\eta$ that
$\eta(x,y,z)\ge \eta(2,2,2) = 2/\sqrt3$ (Lemma~\tref{tarski:eta:mono}).
\swallowed\end{proved}
\end{tarski}
%<<<<<




%>>>>>
\begin{tarski}
\name{eta-ortho}
\summary{The segment from the circumcenter to the midpoint of a triangle's side is perpendicular to the side.}
\tag{pt3, circum3, collinear, bis}
\rating{40}
\guid{POXDVXO}

\begin{lemma}\tlabel{tarski:eta-ortho}
Let $S=\{v_1,v_2,v_3\}$ be a set of three points
in $\ring{R}^3$.  Assume that the points are not collinear.
Let $p$ be the circumcenter of $\{v_1,v_2,v_3\}$.   Then
  $(p-(v_i+v_j)/2)\cdot (v_i-v_j)=0$, for $1\le i<j\le 3$.
\end{lemma}

\begin{proved}  The circumcenter is equidistant from $v_i$
and $v_j$, so it lies in $\op{bis}\{v_i,v_j\}$.
This equation expresses that fact.
\swallowed\end{proved}
\end{tarski}
%<<<<<














%%>>>>>
%\begin{tarski}
%\name{cm4}
%\summary{Cayley and Menger express $\Delta$ as the square of a determinant.}
%\tag{cm4-delta, pt4, deprecated}
%\rating{0} % was 80
%\guid{EZHSEQJ}
%
%\begin{lemma}\tlabel{tarski:cm4}
%	Let $\op{CM}_4(x_{ij})$ be the Cayley-Menger square for
%four points $v_1,\ldots,v_4$ in $\ring{R}^3$, with $|v_i-v_j|^2 = x_{ij}$,
%for $1\le i < j \le 4$.  Then
%	$$\op{CM}_4(x_{ij}) = \Delta(x_{12},x_{13},x_{14},x_{34},x_{24},x_{23})/4.$$
%\end{lemma}
%
%\begin{proved}
%This polynomial is a determinant  $$\Delta(x_{12},\ldots,x_{23}) = \frac{d}{2},$$
%where $$
%     d=\left|\begin{matrix}
%     0 & 1 & 1 & 1 & 1\\
%     1 & 0 & x_{12} & x_{13} & x_{14} \\
%     1 & x_{12} & 0 & x_{23} & x_{24} \\
%     1 & x_{13} & x_{23} & 0 & x_{34} \\
%     1 & x_{14} & x_{24} & x_{34} & 0
%  \end{matrix}\right|.
%  $$
%The explicit formula for the Cayley-Menger
%square is
%	$$
%	\op{CM}_4(x_{ij}) = \frac{d}{2^3}.
%	$$
%\swallowed\end{proved}
%\end{tarski}
%%<<<<<




%%>>>>>
%\begin{tarski}
%\name{delta-ge0}
%\summary{$\Delta$, evaluated on a set of four points, is non-negative.}
%\tag{pt4, cm4-delta, deprecated}
%\rating{40}
%\guid{KGFFYWQ}
%
%\begin{lemma}
%Let $v_1,\ldots,v_4$ be four points
%in $\ring{R}^3$.  Let $x_{ij} = |v_i-v_j|^2$.  Then
%	$$\Delta(x_{12},x_{13},x_{14},x_{34},x_{24},x_{23})\ge0.$$
%\end{lemma}
%
%\begin{proved}
%\swallowed\end{proved}
%\end{tarski}
%%<<<<<










%>>>>>
\begin{tarski}
\section{Cayley-Menger}
\name{x12}
\summary{$\Delta$ is quadratic in each variable.  The leading coefficient is calculated.}
\tag{cm4-delta}
\rating{40}
\guid{MAEWNPU}

View $\Delta(x_{ij})$ as a polynomial $\mu_\Delta(x_{34})$ of $x_{34}$.

\begin{lemma} \tlabel{tarski:x12}
$\mu_\Delta$  is a quadratic
polynomial in $x_{34}$ with leading coefficient $-x_{12}$. 
\end{lemma}

\begin{proved}
\swallowed\end{proved}
\end{tarski}
%<<<<<



%>>>>>
\begin{tarski}
\name{delta-discrim}
\summary{The discriminant of $\Delta$, viewed as a quadratic polynomial, is calculated in terms of $\ups$.}
\tag{cm4-delta, ups}
\rating{40}
\guid{AGBWHRD}

\begin{lemma}\tlabel{tarski:delta-discrim}
The discriminant of the quadratic polynomial $\mu_\Delta$ is
	$$
	\ups(x_{12},x_{23},x_{13}) \ups(x_{12},x_{24},x_{14}).
	$$
\end{lemma}

\begin{proved}
\swallowed\end{proved}
\end{tarski}
%<<<<<




%\def\condA{\op{condA}}
%>>>>>
\begin{tarski}
\name{def:asm-cm4}
\summary{Several lemmas use a technical hypothesis, called $\op{condA}$.}
\tag{def, condA}
\rating{0}
\guid{HOXTPPH}

\begin{definition}\tlabel{def:asm-cm4}
Define $\op{condA}(v_1,\ldots,v_4,x_{ij})$ to be the following condition:
There exist
coplanar points $v_1,\ldots,v_4\in\ring{R}^3$ where 
\begin{itemize}
\item  $v_1\ne v_2$, and
\item $|v_i-v_j|^2 = x_{ij}$ for $1\le i < j \le 4$  $(i,j)\ne (34)$.
\end{itemize} 
\end{definition}
\end{tarski}
%<<<<<


%>>>>>
\begin{tarski}
\name{plane-delta}
\summary{The vanishing of $\Delta$ determines the sixth edge length of four points in terms of the other five.}
\tag{cm4-delta, pt4, plane}
\rating{40}
\guid{VCRJIHC}

\begin{lemma}\tlabel{tarski:plane-delta}
Assume $\op{condA}(v_1,\ldots,v_4,x_{ij})$.
Then $|v_3-v_4|^2$ is a root of the polynomial $\mu_\Delta(x)$. 
\end{lemma}

\begin{proved}  See Lemma~\ref{tarski:delta0}.
\swallowed\end{proved}
\end{tarski}
%<<<<<



%>>>>>
\begin{tarski}
\name{line-delta}
\summary{The discriminant of $\Delta$ is zero when three points are collinear.}
\tag{pt4, cm4-delta, plane, collinear}
\rating{40}
\guid{EWVIFXW}

\begin{lemma}
Assume $\op{condA}(v_1,\ldots,v_4,x_{ij})$.
At least one of $v_3$ or $v_4$ lies on the line through $v_1,v_2$ if and only if
the discriminant of the polynomial $\mu_\Delta$ is $0$.
\end{lemma}

\begin{proved} Lemma~\ref{tarski:ups0} and Lemma~\ref{tarski:delta-discrim}. 
\swallowed\end{proved}
\end{tarski}
%<<<<<



%>>>>>
\begin{tarski}
\name{cm4-large}
\summary{Given four coplanar points, the larger root of $\Delta$ is the correct root to use, when a line separation property holds.}
\tag{plane, pt4. cm4-delta, aff2}
\rating{40}
\guid{FFBNQOB}

\begin{lemma} \tlabel{tarski:cm4-large}
Assume $\op{condA}(v_1,\ldots,v_4,x_{ij})$.
If the line through $v_1,v_2$ separates
$v_3$ from $v_4$, then $|v_3-v_4|^2$ is the larger root of $\mu_\Delta$.
\end{lemma}

\begin{proved}
Let $v_4'$ be the reflection of $v_r$ through $\op{aff}(v_1,v_2)$.  Then by
Lemma~\ref{tarski:plane-delta}, 
$|v_3-v_4'|^2$ and $|v_3-v_4|^2$ are both  roots and $|v_3-v_4'|^2 < |v_3-v_4|^2$.
\swallowed\end{proved}
\end{tarski}
%<<<<<



%>>>>>
\begin{tarski}
\name{cm4-small}
\summary{Given four coplanar points, the smaller root of $\Delta$ is the correct one to use, when a line separation property fails.}
\tag{plane, pt4, cm4-delta, aff2}
\rating{40}
\guid{CHHSZEO}

\begin{lemma}\tlabel{tarski:cm4-small}
Assume $\op{condA}(v_1,\ldots,v_4,x_{ij})$.
If $v_3$ and $v_4$ lie in the same half-plane of the line through $v_1,v_2$, then 
 $|v_3-v_4|^2$ is the smaller root of $\mu_\Delta$.
\end{lemma}

\begin{proved}  The proof is similar to that of Lemma~\ref{tarski:cm4-large}.
\swallowed\end{proved}
\end{tarski}
%<<<<<




%>>>>>
\begin{tarski}
\name{partial-delta-ups}
\summary{This gives a formula for the partial derivative of $\Delta$ in terms of $\ups$ on a set of four coplanar points.}
\tag{plane, pt4, cm4-delta, ups}
\rating{60}
\guid{CMUDPKT}

\begin{lemma}
Assume $\op{condA}(v_1,\ldots,v_4,x_{ij})$.
Let $x'_{34}$ be the smaller root
of $\mu_\Delta$.  Then
  $$\frac{\partial\Delta}{\partial x_{34}} (x_{12},x_{13},x_{14},
   x'_{34},x_{24},x_{23}) = 
    \sqrt{\ups(x_{12},x_{23},x_{13}) \ups(x_{12},x_{24},x_{14})}.
  $$
Let $x''_{34}$ be the larger root
of $\mu_\Delta$.  Then
  $$\frac{\partial\Delta}{\partial x_{34}} (x_{12},x_{13},x_{14},
   x''_{34},x_{24},x_{23}) = 
    -\sqrt{\ups(x_{12},x_{23},x_{13}) \ups(x_{12},x_{24},x_{14})}.
  $$
\end{lemma}

\begin{proved} Solve the quadratic equation  $\mu_\Delta=0$ and substitute the root
into the formula for the partial derivative.
\swallowed\end{proved}
\end{tarski}
%<<<<<








%\def\condface{\op{condF}}
%\def\condcross{\op{condC}}
%\def\condskinny{\op{condS}}

%>>>>>
\begin{tarski}
\section{Crossing segments}
\name{def:condC}
\summary{Several lemmas rely on a technical hypothesis $\op{condC}$, which has been formlated as a definition.}
\tag{def, condC, cm4-delta}
\rating{0}
\guid{KICBNLF}

\begin{definition}[$\op{condC}$]\tlabel{def:condC}
\indy{Index}{condC}
We write
	$$
	\op{condC}(M_{13},m_{12},m_{14},M_{24},m_{34},m_{23})
	$$
for the following condition.
The constants  $m_{12},m_{23},m_{34}$,  $m_{14}$, 
$M_{13}$, and $M_{24}$ are positive and satisfy
	$$
	\begin{array}{rll}
	m_{12} + m_{23} &\ge M_{13}\\
	m_{14} + m_{34} &\ge M_{13}\\
	m_{12} + m_{14} &> M_{24}\\
	m_{23} + m_{34} &> M_{24}\\
	\Delta(M^2_{13},m^2_{12},m^2_{14},M^2_{24},
		m^2_{34},m^2_{23}) &\ge 0.
	\end{array}
	$$
\end{definition}
\end{tarski}
%<<<<<



%>>>>>
\begin{tarski}
\name{cross}
\summary{This give edge length conditions, with general parameters, that prevent two segments from meeting.}
\tag{segs-meet, conv2, condC}
\rating{250}
\guid{CXWOCGN}

\begin{lemma} \tlabel{tarski:cross}
Assume that  $m_{12},m_{23},m_{34}$,  $m_{14}$, 
$M_{13}$, and $M_{24}$ satisfy 
  $\op{condC}(M_{13},m_{12},m_{14},M_{24},m_{34},m_{23})$.
Let $S=\{v_1,v_2,v_3,v_4\}$ be a set of four points in $\ring{R}^3$.
Assume that
	$$
	\begin{array}{lll}
	|v_1-v_2|&\ge m_{12},\\
	|v_2-v_3|&\ge m_{23},\\
	|v_3-v_4|&\ge m_{34},\\
	|v_4-v_1|&\ge m_{14},\\
	|v_1-v_3|&< M_{13},\\
	|v_2-v_4|&\le M_{24},\\
	\end{array}
	$$  
Let $e = \{v_1,v_3\}$ and $e'=\{v_2,v_4\}$.
Then $\op{conv} (e)$ does not meet $\op{conv} (e')$.    	
\end{lemma}

\pdffig{CXWOCGN}{cross}{conditions that prevent two line segments
from meeting.}

\begin{proved} 
We assume for a contradiction that there exists a set $S$ for
which $\op{conv}(e)$ meets $\op{conv}(e')$.  Set $|v_i-v_j|=x_{ij}$. Contract $|v_1-v_3|$
preserving constraints and intersection until we
reduce to configurations in which equality
holds in the lower bounds:
	$$|v_1-v_2|=m_{12},\ |v_2-v_3|=m_{23},\ |v_3-v_4|=m_{34},\ 
      |v_4-v_1|=m_{14},$$   
By Lemma~\tref{tarski:cm4-large} and Lemma~\ref{tarski:partial-delta-ups}, we find that
$\partial\Delta/\partial x_{13}<0$.   Similarly for the $x_{24}$ partial.
Along the curve $\Delta=0$, the implicit derivative $d x_{13}/d x_{24} <0$.  
We may therefore contract $|v_1-v_3|$ until $|v_2-v_4|=M_{24}$.  Then we use
the sign of $\partial\Delta/\partial x_{13}$ again to see that
   $$0 = \Delta(x_{13},\ldots) > \Delta(M^2_{13},\ldots) \ge 0.$$
This is a contradiction.
\swallowed\end{proved}
\end{tarski}
%<<<<<





%>>>>>
\begin{tarski}
\name{cross-edge-generic}
\summary{This give edge length conditions, with general parameters, that prevent two segments from meeting.}
\tag{segs-meet, conv2, condC}
\rating{100}
\guid{THADGSB}

\begin{lemma}  \tlabel{tarski:cross-edge-generic}
In Lemma~\ref{tarski:cross}, modify the conditions to make
    $$
    \begin{array}{lll}
      	m_{12} + m_{23} &> M_{13}\\
	m_{14} + m_{34} &> M_{13}\\
  \Delta(M^2_{13},m^2_{12},m^2_{14},M^2_{24},
		m^2_{34},m^2_{23}) &> 0,\\
                |v_1-v_3|\le M_{13},
      \end{array}
    $$
while keeping the other conditions unchanged (including $\op{CondC}$).  Then the same conclusion
holds.
\end{lemma}

\begin{proved}  We arrive at a nearly identical contradiction:
  $$ 0 = \Delta(x_{13},\ldots) \ge \Delta(M^2_{13},\ldots) > 0.$$
\swallowed\end{proved}
\end{tarski}
%<<<<<







%>>>>>
\begin{tarski}
\name{mid-Voronoi}
% From DCG Sec 9, to justify the Def of epsilon.
\summary{In a packing, the midpoint of a segment of length less than $\sqrt8$ cannot be closer to a third
point than to an endpoint.}
\tag{voronoi,packing,pt3,sqrt2}
\rating{40}
\guid{ZZSBSIO}

\begin{lemma}\tlabel{tarski:mid-Voronoi}
  Let $\{u,v,w\}$ be a packing of three points.  Suppose that 
$|u-v| <\sqrt8$.  Then $|w- (u+v)/2| > |u-v|/2$.
\end{lemma}

\begin{proved} Suppose that $|u-v|/2\le |w-(u+v)/2|$.
Let $w'$ be the reflection of $w$ through the point
$(u+v)/2$.  Then $|w-w'| = 2|w-(u+v)/2|\le |u-v| <\sqrt8$.
This is contrary to Lemma~\tref{tarski:cross} applied to
$e=\{w,w'\}$, $e'=\{u,v\}$, with $m_{ij}=2$.  
\swallowed\end{proved}
\pdffig{ZZSBSIO}{mid-Voronoi}{A point far from the endpoints 
of a segment cannot
be close to the midpoint.}
\end{tarski}
%<<<<<







%% From the proof of DCG, Lemma 11.9.
%>>>>>
\begin{tarski}
\name{336}
\summary{If two segments meet and one has length at most $\sqrt8$, then the other has length at least $3.2$.}
\tag{segs-meet, pt4, conv2}
\rating{40}
\guid{JGYWWBX}

\begin{lemma} \tlabel{tarski:336}
There does not exist a set
$S=\{v_1,v_2,w_1,w_2\}$ of four distinct points
in $\ring{R}^3$ with the following properties.
\begin{itemize}
	\item $2\le |u-v|$ for every distinct $u,v\in S$.
          \item $|v_1-w_2|\ge \sqrt{8}$.
		\item $ |v_1-v_2| \le 3.2$.
	\item $|w_1-w_2|\le \sqrt{8}$.
	\item $\op{conv}\{v_1,v_2\}$ meets
		$\op{conv}\{w_1,w_2\}$.
\end{itemize}
\end{lemma}

\pdffig{JGYWWBX}{336}{The segments cannot meet.}

\begin{proved} The result follows from Lemma~\tref{tarski:cross-edge-generic},
because
  $$
  \Delta(3.2^2,8,4,8,4,4) > 0.
  $$
\swallowed\end{proved}
\end{tarski}
%<<<<<





%>>>>>
\begin{tarski}
\name{delta-2}
\summary{Two segments of a packing cannot cross, under particular constraints on the distances.}
\tag{cm4-delta,conv2,segs-meet,pt4, t0, 3.2}
\rating{40}
\guid{PAHFWSI}

\begin{lemma}\tlabel{tarski:delta-2}
Let $S=\{v_1,v_2,v_3,v_4\}$ be a packing of four points.
  Assume that
	$|v_1-v_3|\le 3.2$.  Assume $|v_1-v_2|\ge 2.51$.
Assume $|v_2-v_4|\le 2.51$.
Then $\op{conv}\{v_1,v_3\}$ does not meet
$\op{conv}\{v_2,v_4\}$.
\end{lemma}

\pdffig{JGYWWBX}{delta-2}{The segments cannot meet.}

\begin{proved}  This follows from Lemma~\tref{tarski:cross-edge-generic},
because    
   $$
    \Delta(x_{ij})\ge \Delta(2.51^2,4,4,3.2^2,4,2.51^2)>0.
    %\tlabel{eqn:D>0}
    $$
\swallowed\end{proved}
\end{tarski}
%<<<<<


    


%% From DCG 12.8, page 134.
%>>>>>
\begin{tarski}
\name{307}
\summary{Two segments of a packing cannot cross, under particular constraints on the distances.}
\tag{pt4, segs-meet, conv2, cm4-delta, t0, 3.07}
\rating{40}
\guid{UVGVIXB}

\begin{lemma}\tlabel{tarski:307}
Let $S=\{v_1,v_2,w_1,w_2\}$ be a packing of four points.
Assume that
	$$|w_1-w_2|\le 2.51.$$
Assume that $|v_1-v_2|\le 3.07$.  
Then
$\op{conv}\{w_1,w_2\}$ does not meet $\op{conv}\{v_1,v_2\}$.
\end{lemma}

\pdffig{JGYWWBX}{307}{The segments cannot meet.}

\begin{proved}
Assume for a contradiction that $|w_1-w_2|\le2.51$.
The result follows from  Lemma~\tref{tarski:cross-edge-generic},
because 
    $$
    \Delta(4,4,2.51^2,4,4,3.07^2) > 0.
    $$
\swallowed\end{proved}
\end{tarski}
%<<<<<





%>>>>>
\begin{tarski}
\name{22}
\summary{Two segments of a packing cannot cross, under particular constraints on the distances.}
\tag{pt4, cm4-delta, conv2, segs-meet, t0, 2.2, 3.2}
\rating{40}
\guid{PJFAYXI}

\begin{lemma}\tlabel{tarski:22}
Let $S=\{v_1,v_2,w_1,w_2\}$ be a packing of four points. 
Assume that $|v_1-v_2|\le 3.2$.  Assume $|w_1-w_2|\le 2.51$. 
Assume that
	$|w_1-v_1|\ge 2.2$.
Then
$\op{conv}\{w_1,w_2\}$ does not meet $\op{conv}\{v_1,v_2\}$.
\end{lemma}

\pdffig{JGYWWBX}{22}{The segments cannot meet.}

\begin{proved}
The result follows from Lemma~\tref{tarski:cross-edge-generic},
because
    $$
    \Delta(2.2^2,4,2.51^2,4,4,3.2^2) > 0.
    $$
\swallowed\end{proved}
\end{tarski}
%<<<<<





%>>>>>
\begin{tarski}
\name{311}
\summary{An upper bound on the length of the sixth edge of a simplex is computed, given the other five.}
\tag{cm4-delta, 3.114467, pt4, t0}
\rating{40}
\guid{YXWIPMH}

\begin{lemma}\tlabel{tarski:311}
Let $S=\{v_1,v_2,v_3,v_4\}$ be a set of four distinct points in $\ring{R}^3$.  Assume that
the distances $y_{ij}=|v_i-v_j|$ are given by
	$$
	y_{12}=2.51, \quad y_{13} = y_{14}=
	y_{23}=y_{24}=2
	$$
Then $y_{34}\le 3.114467$.
\end{lemma}

\pdffig{YXWIPMH}{311}{An upper bound on the length of an edge of a
tetrahedron comes by stretching it, until the configuration becomes
planar, and then measuring its length.}

\begin{proved}
We have
    $\Delta(2.51^2,2^2,2^2,x^2,2^2,2^2)<0$, if $x> 3.114467$.
\swallowed\end{proved}
\end{tarski}
%<<<<<





% FROM DCG p.162 (Lemma 14.5)
%>>>>>
\begin{tarski}
\name{3488}
\summary{An upper bound on the length of the sixth edge of a simplex is computed, given bounds on the other five.}
\tag{cm4-delta, 3.488, pt4, t0, sqrt2}
\rating{40}
\guid{PAATDXJ}

\begin{lemma}\tlabel{tarski:3488}
Let $S=\{v_1,v_2,v_3,v_4\}$ be a set of four distinct points in $\ring{R}^3$.  Assume that
the distances $y_{ij}=|v_i-v_j|$ are given by
	$$
        y_{12}=y_{14}=2.51,\quad y_{23}=y_{34}=2,\quad y_{24}\ge\sqrt8.
	$$
Then $y_{13}< 3.488$.
\end{lemma}

\pdffig{YXWIPMH}{3488}{An upper bound the length of an edge of a
tetrahedron comes by stretching it, until the configuration becomes
planar, and then measuring its length.}


\begin{proved}
For a contradiction, if $y_{13}\ge 3.488$, then
we have
    $$0\le \Delta(x_{ij}) < \Delta(3.488^2,4,4,8,2.51^2,2.51^2)<0.$$
\swallowed\end{proved}
\end{tarski}
%<<<<<






%\begin{lem}\guid{LDGXZGT}\tlabel{tarski:flat-Q}\usage{ lemma:voronoi-truncation-over-Q  }
%Let $S=\{v_1,v_2,v_3,v_4\}$ be a set of four points in $\ring{R}^3$.
%Assume that the distanes $y_{ij}=|v_i-v_j|$ are given by
% $$
% 2 \le y_{ij} \le 2.51, \text{ for } \{i,j\}\ne \{1,2\},\qquad
% 2 \le y_{12} \le \sqrt8.
% $$
%Then $S$ is not coplanar.
%\end{lemma}
%
%\begin{proved}  The minimum of $\Delta$ subject to these
%constraints is positive.
%\swallowed\end{proved}




%>>>>>
\begin{tarski}
\section{Point in a triangle}
\name{condF}
\summary{Several lemmas rely on a technical hypothesis $\op{condF}$.}
\tag{def, condF, cm4-delta}
\rating{0}
\guid{VQKSFLW}

\begin{definition}[$\op{condF}$]\tlabel{tarski:condF}
\indy{Index}{condF}
Let
	$$
	\op{condF}(m_{14},m_{24},m_{34},M_{23},M_{13},M_{12}).
	$$
be the following condition:
The constants $M_{12},M_{13},M_{23},m_{14},m_{24},m_{34}$ are positive and
satisfy 
	$$
	\begin{array}{rll}
		M_{12} &< m_{14} + m_{24}\\
		M_{13} &< m_{14} + m_{34}\\
		M_{23} &< m_{24} + m_{34}\\
	\end{array}
	$$
$$
		\Delta(m_{14}^2,m_{24}^2,m_{34}^2,M_{23}^2,M_{13}^2,M_{12}^2) > 0.
$$
\end{definition}
\end{tarski}
%<<<<<



%>>>>>
\begin{tarski}
\name{face}
\summary{Under general conditions on distances, a point cannot fit inside the convex hull of three points.}
\tag{pt4, condF, conv3, inside, plane, cm4-delta}
\rating{200}
\guid{YJHQPAL}

\begin{lemma} \tlabel{tarski:face}
Assume that $M_{12},M_{13},M_{23},m_{14},m_{24},m_{34}$ satisfy
condition 
$$\op{condF}(m_{14},m_{24},m_{34},M_{23},M_{13},M_{12}).$$
Then there do not exist four vectors
$v_1,\ldots,v_4\in\ring{R}^3$ satisfying the conditions
	\begin{itemize}
		\item $v_1,\ldots,v_4$ are coplanar.
		\item $v_4\in\op{conv}\{v_1,v_2,v_3\}$.
		\item $$
	\begin{array}{rlrlrl}
		y_{14} &\ge m_{14}, &y_{24} &\ge m_{24}, &y_{34}&\ge m_{34},\\
		y_{23} &\le  M_{23}, &y_{13} & \le M_{13}, &y_{12}&\le M_{12},\\
	\end{array}
	$$
	\end{itemize}
where
$x_{ij}=y_{ij}^2$ and $y_{ij}=|v_i-v_j|$,
for $1\le i < j \le 4$.  
\end{lemma}


\pdffig{YJHQPAL}{face}{A point that is far away from all three
vertices of a triangle cannot lie in its interior.}

\begin{proved}
	Let $v_1,\ldots,v_4$ be such a configuration, assuming it exists.  
We will deform the points $v_1,\ldots,v_4$ in such a way that $\op{condF}$ and
the constraints continue
to be satisfied.  By Lemma~\tref{tarski:triangle-eq},
the original configuration and those produced by deformation do not have
	$v_4\in\op{conv}\{v_i,v_j\}$, for $i,j\in\{1,2,3\}$.
We may deform $v_1$ away from $v_3$ until $y_{13}=M_{13}$.  Similarly,
deform until $y_{12}=M_{12}$ and $y_{23}=M_{23}$.    Deform $v_4$ until two of the
lower bound constraints $y_{i4}\ge m_{i4}$ reach their minimum. 
Say $y_{14}=m_{14}$ and $y_{24}=m_{24}$.
The configuration is planar, so
	$$
	0=\Delta(x_{ij}) = \Delta(m_{14}^2,m_{24}^2,x_{34},M_{23}^2,M_{13}^2,M_{12}^2).
	$$
By Lemma~\tref{tarski:cm4-small}, $x_{34}$ is the smaller root of this quadratic equation.
The quadratic polynomial has negative leading coefficient by Lemma~\tref{tarski:x12}.
Any quadratic polynomial with negative leading coefficent is an increasing function on
	$$\{x \mid x \le r\},$$ where $r$ is the smaller root.
We have $m_{34}^2\le x_{34}$. Thus, we reach the contradiction
	$$0<\Delta(m_{14}^2,m_{24}^2,m_{34}^2,M_{23}^2,M_{13}^2,M_{12}^2)
        \le \Delta(\ldots,x_{34},\ldots)= 0.$$
\swallowed\end{proved}
\end{tarski}
%<<<<<







%>>>>>
\begin{tarski}
\section{Point in a skinny triangle}
\name{def:condS}
\summary{Some lemmas rely on a technical hypothesis $\op{condS}$, which has been separated out as a definition.}
\tag{def, condS}
\rating{0}
\guid{GRVDLVL}

\begin{definition}[condS]\tlabel{tarski:condS}
\indy{Index}{condS}  
Let 
	$$
	\op{condS}(m,m_{34},M_{13},M_{23}). 
	$$
be the following condition.
 $M_{13},M_{23},m,m_{34}$ are positive real constants
that satisfy 
	$$
	\begin{array}{rll}
		M_{13} &< m + m_{34}\\
		M_{23} &< m + m_{34}\\
		M_{13}^2 &< M_{23}^2 + 4m^2\\
		M_{23}^2 &< M_{13}^2 + 4m^2\\
	\end{array}
	$$
Furthermore, let $r$ be the largest root of the quadratic polynomial in $x$:
	$$\Delta(4m^2,M_{13}^2,M_{23}^2,x,M_{13}^2,M_{23}^2),$$
then $r < (2 m_{34})^2$.
\end{definition}
\end{tarski}
%<<<<<




%>>>>>
\begin{tarski}
\name{skinny-tri}
\summary{Under general conditions, a point cannot fit inside the convex hull of three other points.}
\tag{pt4, conv3, plane, inside, condS, cm4-delta}
\rating{200}
\guid{MPXSJDI}

\begin{lemma}\tlabel{tarski:skinny-tri}
Assume that $M_{13},M_{23},m,m_{34}$ satisfy condition
   $$\op{condS}(m,m_{34},M_{13},M_{23}).$$
Then there do not exist four vectors
$v_1,\ldots,v_4\in\ring{R}^3$ satisfying the conditions
	\begin{itemize}
		\item $v_1,\ldots,v_4$ are coplanar.
		\item $v_4\in\op{conv}\{v_1,v_2,v_3\}$.
	\end{itemize}
	$$
	\begin{array}{rlrlrl}
		y_{14} &\ge m, &y_{24} &\ge m, &y_{34}&\ge m_{34}\\
		y_{23} &\le  M_{23}, &y_{13} & \le M_{13}, &\\
	\end{array}
	$$
where
$y_{ij}=|v_i-v_j|$,
for $1\le i < j \le 4$.  
\end{lemma}

\pdffig{YJHQPAL}{skinny-tri}{A point that is far away from all three
vertices of a triangle cannot lie in its interior, even if no
upper bound is given for one of the edge lengths of the triangle.}

\begin{proved}  Suppose for a contradiction that a figure exists.
We deform the figure by pivoting $v_3$ 
(as in the proof of Lemma~\tref{tarski:face})
until $|v_3-v_1| = M_{13}$ and $|v_3-v_2|=M_{23}$.   Next pivot $v_1$, increasing $|v_1-v_2|$
until $v_1,v_2,$ and $v_4$ are collinear.  Slide $v_4$ along $\op{conv}\{v_1,v_2\}$ until
it has distance $m$ from $v_1$ or $v_2$ (say $v_1$).   Constrain $v_4$ so that further
deformations keep it along  $\op{conv}\{v_1,v_2\}$ at distance $m$ from $v_1$.  
Fix $v_1,v_3$ and pivot $v_2$ around $v_3$ toward $v_1$.
%Deform the figure, by moving $v_2$ directly toward $v_1$ and $v_4$, while moving
%$v_3$ at the same time to preserve the constraints on $|v_3-v_1|$ and $|v_3-v_2|$. 
By the assumptions on the constants $M_*$ and $m$, this is increasing in $|v_3-v_4|$.  The conditions
on the constants also insure that the deformation cannot lead to a situation in which
$v_1,v_2,v_3$ are collinear.  Continue
until $|v_2-v_4|=m$.

Let $v_3' = v_4 - (v_3-v_4)$ be the reflection of $v_3$ about $v_4$.  The four vectors
$v_1,v_2,v_3,v_3'$ form a planar quadrilateral with diagonals $|v_1-v_2|=2m$ and $|v_3-v_3'| \ge 2m_{34}$.
The line through $v_1,v_2$ separates $v_3$ from $v_3'$.  By Lemma~\tref{tarski:cm4-large},  
$r = |v_3-v_3'|^2$ is
the largest root of the given quadratic polynomial.  So $r \ge (2 m_{34})^2$.  This is contrary
to hypothesis.
\swallowed\end{proved}
\end{tarski}
%<<<<<






\section{Trigonometry}
\tlabel{sec:trig}


It is often useful to make arguments based on angle in the proofs
of elementary problems in geometry. The angle $\gamma$
of a triangle with sides
$a,b,c$ on the side opposite $c$ is
    $$
    \gamma = \arc(a,b,c)=\arccos(\frac{a^2 + b^2 - c^2}{2 a b}).
    $$
This cannot be used in the Tarski language because of the trigonometric
function $\arccos$.  We will avoid mentioning the function
$\arc(a,b,c)$ in the statements of the lemmas.
In fact, we will often work with $\cosarc(a,b,c) = \cos(\arc(a,b,c))$.
\indy{Index}{cosarc} 
However, whenever, we have a comparison of
two angles
    $$
    \arc(a,b,c) \le \arc(a',b',c'),
    $$
we may take the cosine of both sides to express the inequality in
elementary terms.  In this way, some non-elementary arguments
can be reduced to inequalities of rational functions.  When
$S=\{v_1,v_2,v_3\}$ is a set of three points in $\ring{R}^3$, we write
   $$
   \arc_V(v_1,v_2,v_3) = \arc(|v_1-v_2|,|v_1-v_3|,|v_2-v_3|).
   $$

Another useful definition is that of dihedral angle.  Let 
$S=\{v_1,v_2,v_3,v_4\}$ be a set of four points in $\ring{R}^3$.
The lune $\op{aff}_+(\{v_1,v_2\},\{v_3,v_4\})$ is the intersection
of two half-spaces $\op{aff}_+(\{v_1,v_2,v_i\},\{v_j\})$ with
$\{i,j\}=\{3,4\}$.  The angle formed by these two half-spaces along
the line $\op{aff}\{v_1,v_2\}$ is the dihedral angle of the lune,
and is denoted by $\dih_V(\{v_1,v_2\},\{v_3,v_4\})$.  It is a function
of the distances $y_{ij} = |v_i-v_j|$.  It is known that
 $$
 \dih_V(\{v_1,v_2\},\{v_3,v_4\}) = 
  \dih(y_{12},y_{13},y_{14},y_{34},y_{24},y_{12}),
 $$
where setting $x_{ij}=y_{ij}^2$, we have
 $$\begin{array}{lll}
  \dih(y_{12},y_{13},y_{14},y_{34},y_{24},y_{12}) &=
  \arccos\left(\frac{\Delta_{(34)}(x_{ij})}{\sqrt{
    \ups(x_{12},x_{13},x_{23})\ups(x_{12},x_{14},x_{24})}}\right)\\
    &=\frac{\pi}{2} - \atn
     \left({\sqrt{4 x_{12} \Delta(x_{ij})}},{\Delta_{(34)}(x_{ij})}\right).
  \end{array} 
 $$
Recall that $\Delta_{(ij)}$ is an abbreviation for 
$\partial\Delta/\partial x_{ij}$.  We use $\dih_V$ for the
vector version of the function and $\dih$ for the function of
edge lengths.
It is fortunate that angles are expressed in terms of the 
core functions ($\ups$, $\Delta$) of this collection.  Of course,
this is no coincidence.


A comparison of two dihedral angles can be reduced to an inequality
of rational functions. We also see that statements such as
  $$
  \dih_V(\{v_1,v_2\},\{v_3,v_4\}) < \pi/2
  $$
can be rewritten in elementary terms.  From the explicit formula
for dihedral angle, this particular inequality
is equivalent to $\Delta_{(34)}(x_{ij}) > 0$.

The partial derivatives of $\dih$ are again readily expressed in
terms of the core functions.
% - x12 D[del, x24]/(Ups[x12, x13, x23] Sqrt[x12 del]);
To give one example, we have
  $$
  \frac{\partial\dih(y_{12},y_{13},y_{14},y_{34},y_{24},\sqrt{x_{23}})}
  {\partial x_{23}} = 
  \frac{ -x_{12} \Delta_{(24)}(x_{ij})} {\ups(x_{12},x_{13},x_{23})
   \sqrt{x_{12}\Delta(x_{ij})}}
  $$
From this, we see that the monotonicity of $\dih$
in the $x_{23}$ direction is controlled by the sign of
$-\Delta_{(24)}$.




In summary, we will refrain from mentioning functions such as
$\arc(a,b,c)$ and $\dih_V$ in the statements of lemmas.  However,
we will to use them in proofs.  Most of the time, we use
these functions in proofs in such a way that the inverse
trigonometric functions
can be easily eliminated.



\section{Pivot}

A number of arguments are based on a consraint preserving deformation
that is called a {\it pivot}.   This has already come up in a two-dimensional
context in Lemma~\ref{tarski:skinny-tri}.  Here we give a detailed description. 
Let $\{v_1,v_2,v_3\}$ be a set
of three points.  Let $L=\op{aff}\{v_1,v_2\}$ and let $A$ be
the plane orthogonal to $L$ that passes through the point $v_3$.
Let $r$ be the distance from $v_3$ to $L$.  Let $A\cap L = \{p\}$.

The point $v_3$ lies on the circle
   $$
   \{u \in A \mid r = |u - p|\}.
   $$
The motion of $v_3$ along this circle is called pivoting $v_3$
around the axis $\{v_1,v_2\}$.   Pivoting preserves the distance from
$v_3$ to $v_1$ and $v_2$.   In general, the pivot can move
either clockwise or counterclockwise around the circle.  To fix
the direction of rotation, we will often specify that the pivot
should move towards (or away from) a given point $v_4\in\ring{R}^3$.

\pdffig{pivot}{pivot}{A pivot is the motion of a point
around a circle with given axis.}


Usually, it is understood that the pivoting motion should continue
until one of the constraints of the lemma prevents further motion.
This often takes the form of an upper or lower bound distance
constraint involving $v_3$.  For example, we might have a constraint
$|v_3-v_4|\le M$, which would force the pivot to stop when $|v_3-v_4|=M$.
Often, there are several possible constraints that may become active
and halt the pivot.  When this happens, the proof may branch into several
cases, depending on which constraint becomes active.

It is useful to be able to detect by inspection whether a given
distance $|v_3-v_4|$ is increasing or decreasing under a pivot $v_3$
around an axis $\{v_1,v_2\}$.   The infinitesimal pivot of $v_3$ 
around the axis $\{v_1,v_2\}$ is in the direction $n$ for some normal
to the plane
$\op{aff}\{v_1,v_2,v_3\}$. 
To first order $v_3$ at time $t$, $v_3(t)$ can be given as $v_3+t n$.
We have the approximation
  $$
  |(v_3(t)-v_4|^2 \approx |(v_3+t n)-v_4|^2 
  \approx |v_3-v_4|^2 + 2 t n \cdot (v_3-v_4).
  $$
Thus, the sign of the derivative of $|v_3(t)-v_4|^2$ is determined by the
sign of $n\cdot (v_3-v_4)$.


Pivots do not appear in any of the statements of the lemmas.  They
are way of deforming the figure presented in a lemma in a way to make
the lemma easier to prove.







%>>>>>
\section{Beta cone}
The following function is used in some of the proofs in this
section.  It is not an elementary function, so we avoid mentioning
it in any of the statements of the results.  
\begin{tarski}
\name{def:beta}
\summary{A function $\beta$ measures the angle formed by two planes, one tangent to a right-circular cone and the other
passing through its axis.}
\tag{def, beta, rcone, pt4}
\rating{0}
\guid{YKIYGNA}

\begin{definition}[$\beta$]\tlabel{def:beta}
Consider the cone $R=\op{rcone}(v_0,v_2,\cos\psi)$.  
%Pick
%$v_3\ne v_0$ on the boundary of this cone: $v_3\in R\setminus R^0$.
Let $v_1\not\in R$.  
Let $A$ be the tangent plane to $\partial R$ through $v_1$.
Define $\beta$ to be the angle formed by the lune
   $\op{aff}_+(\{v_0,v_1\},\{v_2,v_3\})$. It is a function
$\beta_\psi(\theta)\in[0,\pi/2]$ depending only on $\psi$ and
$\theta = \arc_V(v_0,v_1,v_2)$.
\end{definition}\indy{Greek}{ZZbeta@$\beta_\psi$}
\pdffig{beta}{beta}{$\beta_\psi$ is the dihedral angle formed by
two planes, one that is tangent to the 
right-circular cone, and the other 
passing through its axis.}
\end{tarski}
%<<<<<





%>>>>>
\begin{tarski}
\name{beta:dcg-p129}
\summary{Under suitable conditions, a blade and a right-circular cone meet only at the common apex.}
\tag{pt4, packing, t0, 3.2, rcone, blade, beta}
\rating{120}
\guid{VMZBXSQ}

\begin{lemma}\tlabel{tarski:beta:dcg-p129}
 Let $S=\{v_0,v_1,v_2,v_3\}$ be a packing of four points.
Assume
that 
  $$
    |v_1-v_0|,\ |v_2-v_0|, |v_3-v_0|\le 2.51.\quad 
    |v_1-v_2|\le 3.2,\quad |v_1-v_3|\ge 2.51.
   $$
Then $R^0=\op{rcone}^0(v_0,v_3,|v_3-v_0|/2.51)$ does not meet
$C=\op{cone}(v_0,\{v_1,v_2\})$.
\end{lemma}

\pdffig{VMZBXSQ}{beta:dcg-p129}{The separation of a right circular
cone from an affine cone (in blue).}

\begin{proved} (We give a non-elementary proof.)  Let $u\in\ R^0\cap C$.
Pivot so that $|v_1-v_2|=3.2$, $|v_3-v_1|=2.51$, $|v_3-v_2|=2$.
Let $\cos\psi  =|v_3-v_0|/2.51$ and $\theta=\arc_V(v_0,v_2,v_3)$. 
By Lemma~\tref{tarski:delta-2}, the
set $\{v_0,v_1,v_2,v_3\}$ is not coplanar. 
Simple estimates show that $v_2\not\in R^0$.
Then use\footnote{\calc{193836552} or more precisely, \calc{735258244} } %A1
% This is a composite calc.  I think a more precise ref is the second.
 $\beta_\psi(\theta) < \dih_V(\{v_0,v_2\},\{v_1,v_3\})$.
However, the point $u\in R^0\cap C$ gives the contradiction
$$
  \dih_V(\{v_0,v_2\},\{v_1,v_3\}) = dih_V(\{v_0,v_2\},\{v_1,u\}) < \beta_\psi(\theta).
$$
\swallowed\end{proved}
\end{tarski}
%<<<<<





%% Expanded the definition of $\SB$.

%>>>>>
\begin{tarski}
\name{beta:B}
\summary{Under given conditions, a blade and right-circular cone are disjoint.}
%\endnote{Beta lemma, alphabet simplex of type B, tarski:beta:B}
\tag{packing, pt4, 2.23, 2.77, rcone, blade, beta, lune, dih}
\rating{80}
\guid{QNXDWNU}

\begin{lemma}\tlabel{tarski:beta:B}
Let $S=\{v_0,v_1,v_2,v_3\}$ be a packing of four points.
Assume that
   $$
   |v_2-v_0|\le 2.23,\quad |v_3-v_0|\le 2.23,\quad 2.77\le |v_2-v_3|\le\sqrt8.
   $$
Assume $|v_1-v_i|\le 2.51$, for $i=0,2,3$.
Then $R^0=\op{rcone}^0(v_0,v_1,|v_1-v_0|/2.77)$ does not meet
$C=\op{cone}(v_0,\{v_2,v_3\})$.
\end{lemma}

\pdffig{VMZBXSQ}{beta:B}{The separation of a right circular
cone from a blade.}

\begin{proved} (We give a non-elementary proof.)  Assume $u\in R^0\cap C$.
Pivot $v_1$ toward $v_2$
along the axis $\{v_0,v_3\}$, until $|v_1-v_2|=2$.
Set $\theta = \arc_V(v_0,v_1,v_3)$.
Define $\psi$ by $\cos\psi = |v_1-v_0|/2.77$.
Simple estimates show that $v_3\not\in R^0$.
We have that\footnote{\calc{757995764}}
%\calc{757995764} is part of the \calc{193836552}  composite.
    $\beta_\psi(\theta) < \dih_V(\{v_0,v_3\},\{v_1,v_2\})$,
where $\dih_V(\{v_0,v_3\},\{v_1,v_2\})$ is the dihedral angle 
of the lune $\op{aff}_+(\{v_0,v_3\},\{v_1,v_2\})$. 
As in the proof of Lemma~\ref{tarski:beta:dcg-p129}, this leads to a contradiction.
\swallowed\end{proved}
\end{tarski}
%<<<<<




%>>>>>
\begin{tarski}
\name{beta:dcg-144a}
\summary{Under suitable conditions, a blade meets a right-circular cone only at their common apex.}
\tag{pt4, packing, 3.2, t0, blade, rcone, beta}
\rating{80}
\guid{PNUMEHN}

\begin{lemma}\tlabel{tarski:beta:dcg-144a}
Let $S=\{v_0,v_1,v_2,v_3\}$ be a packing of four points.
Assume that
$|v_1-v_i|\ge 2.51$, $i=2,3$.
Assume that $|v_2-v_3|\le 3.2$.
Let $y = |v_1-v_0|$ and 
$$
  h=\cos\arc(y,1.255,1.6).
$$
Then $C=\op{cone}(v_0,\{v_2,v_3\})$ does not meet
$R^0=\op{rcone}^0(v_0,v_1,h)$.
\end{lemma}

\pdffig{VMZBXSQ}{beta:dcg-144a}{The separation of a right circular
cone from an affine cone (in blue).}

\begin{proved}  Pivot $v_1$ until
$|v_1-v_2|=|v_1-v_3|=2.51$.   
Set
$h=\cos\psi$. 
Simple estimates show that $v_2\not\in R^0$.
A calculation\footnote{\calc{193836552}} %A1
gives $\beta_\psi(\arc_V(v_0,v_1,v_2))<\dih_v(\{v_0,v_2\},\{v_1,v_3\})$.
As in the proof of Lemma~\ref{tarski:beta:dcg-p129}, this leads to a contradiction.
\swallowed\end{proved}
\end{tarski}
%<<<<<




%>>>>>
\begin{tarski}
\name{beta:dcg-144b}
\summary{Under suitable conditions, a right-circular cone and a blade meet only at their common apex.}
\tag{pt4, packing, 3.2, t0, rcone, blade, beta}
\rating{80}
\guid{HLAHAUS}

\begin{lemma} \tlabel{tarski:beta:dcg-144b}
Let $S=\{v_0,v_1,v_2,v_3\}$ be a packing of four points.
Assume that
$|v_1-v_2|=2$ and $|v_1-v_3|\ge3.2$.
Assume that $|v_2-v_3|\le 3.2$. Let $y = |v_1-v_0|$
Assume that $2.2\le y\le 2.51$.
Let
$$
  h= \cos\arc(y,1.255,1.6).
$$
Then $C=\op{cone}(v_0,\{v_2,v_3\})$ does not meet
$R^0=\op{rcone}^0(v_0,v_1,h)$.
\end{lemma}

\pdffig{VMZBXSQ}{beta:dcg-144b}{The separation of a right circular
cone from a blade.}

\begin{proved}
Pivot until
$|v_1-v_3|=3.2$. Then use the calculation\footnote{\calc{193836552}} %A1
    $$\beta_\psi(\arc_V(v_0,v_1,v_2))< \dih_V(\{v_0,v_2\},\{v_1,v_3\}),$$
where $\cos\psi=h$,
provided $y\in[2.2,2.51]$. Note further, that
    $$\arc(y,1.255,1.6)<\arc_V(v_0,v_1,v_2),$$
for $|v_1-v_0|=y\in[2.2,2.51]$, 
so $v_2\not\in\op{cone}(v_0,v_1,h)$.
As in the proof of Lemma~\ref{tarski:beta:dcg-p129}, this leads to a contradiction.
\swallowed\end{proved}
\end{tarski}
%<<<<<






%>>>>>
\begin{tarski}
\name{beta:dcg-144c}
\summary{Under suitable conditions, a blade meets the intersection of a right-circular cone with a half-space only at the common apex.}
\tag{pt4, packing, aff3, halfspace, rcone, 3.2, 2.2, t0, blade, dih, circum3, circum4}
\rating{140}
\guid{RISDLIH}

\begin{lemma} \tlabel{tarski:beta:dcg-144c}
Let $S=\{v_0,v_1,v_2,v_3\}$ be a packing of four points.
Assume $|v_0-v_2|,|v_0-v_3|\le 2.51$.
Assume that
$|v_1-v_2|=2$ and $|v_1-v_3|\ge3.2$.
Let $y = |v_1-v_0|$.
Assume that $2\le y\le 2.2$.
Assume that $|v_2-v_3|\le 3.2$.
Let 
$$
  h=\cos\arc(y,1.255,1.6).
$$
Let  $p$ be the circumcenter
of $\{v_0,v_1,v_2\}$.
Let $H = \op{aff}_+(\{v_0,p,p'\},v_1)$ where $p'\not\in\op{aff}\{v_0,p\}$ lies in
plane $P\supset\op{aff}\{v_0,p\} $ orthogonal to $\op{aff}\{v_0,v_1,v_2\}$.
Then $\op{cone}(v_0,\{v_2,v_3\})$ does not meet
$$
  T=\op{rcone}(v_0,v_1,h) \cap H.
$$
\end{lemma}

\begin{proved} Assume for a contradiction that the two sets meet.  
We start with some notation.
Set $y_{ij}=|v_i-v_j|$ and $x_{ij} =y_{ij}^2$.  In particular, $y_{01}=y$.
Write $d$ for $\dih_V(\{v_0,v_2\},\{v_1,v_3\})$, viewed as a function
of $x_{ij}$.
Similarly, write $d'$ for $\dih_V(\{v_0,v_3\},\{v_1,v_2\})$, 
viewed as a function
of $x_{ij}$.
Write $f_{(ij)}$ for the
partial derivative of a function $f$ with respect to $x_{ij}$. 
%Let 
%  $$\psi=\arccos(h)= \arc(y,1.255,1.6).$$

Start out as in Lemma~\tref{tarski:beta:dcg-144b}. 
If $y\le 2.2$, then $\Delta_{(01)}\ge0$, so
    $d_{(03)}\le 0$.
Without loss of generality, $x_{03}=2.51^2$. Also, $\Delta_{(12)}\ge0$, so
    $d_{(23)}\le0$.
Without loss of generality, $x_{23}=3.2^2$.

Let $c$ be a point of intersection of the plane $P=\op{aff}(v_0,p,p')$ with
the circle at distance $\lambda=1.6$ from $v_1$ on the sphere centered
at the origin of radius $1.255$.  That is, let $c$ be a point on the boundary
of $T$.  The angle of the lune $\op{aff}_+(\{v_0,v_2\},\{v_1,c\})$
is 
    $$\dih(\op{rog}(v_0,v_2,v_1,y/(2 h),v_3)).$$
This angle is less\footnote{\calc{193836552}} %A1
than $d$. 

Also, $\Delta_{(01)}\ge0$, so $d'_{02}\le 0$.  Thus, without loss of generality, 
$x_{02}=2.51^2$. Then
$\Delta_{(13)}<0$, so $d >\pi/2$.  This means that $P$
separates $T$ from $\op{cone}(v_0,\{v_2,v_3\})$. 
\swallowed\end{proved}
\end{tarski}
%<<<<<





%>>>>>
\begin{tarski}
%\section{Five Points}
\section{Five points: Cayley-Menger}
Let $R(x_{ij})$ be  Cayley-Menger polynomial given above
 ($1\le i < j\le 5$).  Fix the values of all the variables $x_{ij}\ge 0$ except
$x_{45}$ to view $\mu_R(x_{45})=R(x_{ij})$ as a function of $x_{45}$.
\name{cm5-quadratic}
\summary{The Cayley-Menger polynomial $R$ is a quadratic polynomial in each variable.  This lemma calculates the leading coefficient.}
\tag{cm5-E, pt5, cm3-ups}
\rating{80}
\guid{LTCTBAN}

\begin{lemma}
$\mu_R$  is a quadratic
polynomial in $x_{45}$ with leading coefficient 
$\ups(x_{12},x_{23},x_{13})$. 
\end{lemma}

\begin{proved}
\swallowed\end{proved}
\end{tarski}
%<<<<<





%>>>>>
\begin{tarski}
\name{cm5-discrim-delta}
\summary{This is a calculation of the discriminant of $R$, viewed as a quadratic polynomial.}
\tag{cm5-E, pt5, cm4-delta}
\rating{80}
\guid{GJWYYPS}

\begin{lemma}\tlabel{tarski:cm5-discrim-delta}
The discriminant of the quadratic polynomial $\mu_R$ is
	$$
	16\Delta(x_{12}, x_{13}, x_{14}, x_{34}, x_{24}, x_{23}) 
	\Delta(x_{15}, x_{25}, x_{35}, x_{23}, 
          x_{13}, x_{12}).
	$$
\end{lemma}

\begin{proved}
\swallowed\end{proved}
\end{tarski}
%<<<<<




%>>>>>
\begin{tarski}
\name{def:calE}
\summary{When five points are given, A function $\CalE$ measures one distance in terms of the other nine.}
\tag{def, cm5-E}
\rating{0}
\guid{RPFVZDI}

\begin{definition}[$\CalE$] \tlabel{def:calE}
\indy{Index}{E}
Let 
$$\CalE(y_{14},y_{24},y_{34},y_{23},y_{13},y_{12},y_{15},y_{25},y_{35})=\sqrt{r}$$
where $r$ is the largest root of the quadratic function in $x=x_{45}$:
	$$
	\mu_R(x;\{x_{ij\ne45}\} )=R(x_{ij}),\quad x_{ij} = y_{ij}^2.
	$$
This is well-defined on the domain for which the leading coefficient of $\mu_R$ is nonzero, and its
discriminant is non-negative.  
\end{definition}
We note that we have not specified the order of the arguments in $R$.   This is in
contrast with the function $\CalE$.  The order of the arguments is entirely significant.
There are some symmetries in the
arguments of the function.  
If the arguments are grouped in three consecutive groups of three, a permutation
can be applied to each group of three, and the value of the function is unchanged.  Here
are some symmetries of the arguments:
	$$
	\begin{array}{lll}
	\CalE(a,a',a'',b,b',b'',c,c',c'') &= \CalE(a',a'',a,b',b'',b,c',c'',c)\\
			&=\CalE(a,a'',a',b,b'',b',c,c'',c')\\
			&= \CalE(c,c',c'',b,b',b'',a,a',a'')\\
	\end{array}
	$$
\end{tarski}
%<<<<<


%>>>>>
\begin{tarski}
\name{def:fSS}
\summary{Five points in space, determine a quadratic polynomial, derived from the Cayley-Menger polynomial $R$.}
\tag{def, pt5, cm5-E}
\rating{0}
\guid{EQXUUDS}

\begin{definition}[$\mu_V$]\tlabel{def:fSS}
\indy{Index}{mu@$\mu_V[S,S']$}
Let $S=\{v_1,v_2,v_3\}\subset\ring{R}^3$ and $S'=\{v_4,v_5\}\subset\ring{R}^3$.
Let $\mu_V[S,S'](x_{45})$ be the quadratic polynomial 
   $$x_{45}\mapsto R(x_{ij}),$$
where $x_{ij} = |v_i-v_j|^2$, for ${i,j}\ne \{4,5\}$.  
\end{definition}
\end{tarski}
%<<<<<



%>>>>>
\begin{tarski}
\name{cm5-root}
\summary{For any set of five points, one length can be calculated from the others, 
using the polynomial $R$.}
\tag{CM5-E, pt5}
\rating{40}
\guid{PFDFWWV}

\begin{lemma}\tlabel{tarski:cm5-root}
Let $\{v_1,v_2,v_3,v_4,v_5\}\subset\ring{R}^3$.
Then $|v_4-v_5|^2$ is a root of the polynomial 
$\mu_V[\{v_1,v_2,v_3\},\{v_4,v_5\}]$. 
\end{lemma}

\begin{proved}
Let $x_{ij}=|v_i-v_j|^2$.  By Lemma~\ref{tarski:cayley-menger-pos},
$$
\mu_V[\{v_1,v_2,v_3\},\{v_4,v_5\}](x_{45}) =R(x_{ij})=0.
$$
\swallowed\end{proved}
\end{tarski}
%<<<<<




%>>>>>
\begin{tarski}
\name{cm5-discrim}
\summary{This gives a criterion for four of five points to lie in a common plane, in terms
of the vanishing of the discriminant of $R$.}
\tag{pt5, cm5-E, plane}
\rating{40}
\guid{GDLRUZB}

\begin{lemma}
Let $\{v_1,v_2,v_3,v_4,v_5\}\subset\ring{R}^5$.
Then $\{v_1,v_2,v_3,v_4\}$ or $\{v_1,v_2,v_3,v_5\}$ is a coplanar set
if and only if
the discriminant of the polynomial $\mu_V[\{v_1,v_2,v_3\},\{v_4,v_5\}]$ is $0$.
\end{lemma}

\begin{proved} This follows by Lemmas~\ref{tarski:cm5-discrim-delta} and \ref{tarski:delta0}.
\swallowed\end{proved}
\end{tarski}
%<<<<<



%>>>>>
\begin{tarski}
\name{cm5-separate}
\summary{Given five points in the plane, if the plane through three separates the other two points, then
the distance between those two points is the larger root of $R$.}
\tag{pt5, cm5-E, plane, halfspace}
\rating{40}
\guid{KGSNYDS}

\begin{lemma}\tlabel{tarski:cm5-separate}
Let $\{v_1,v_2,v_3,v_4,v_5\}\subset\ring{R}^3$.  Assume that $\{v_1,v_2,v_3\}$ is not coplanar.
If the plane through $v_1,v_2,v_3$ separates
$v_4$ from $v_5$, then $|v_4-v_5|^2$ is the larger root of 
$\mu_V[\{v_1,v_2,v_3\},\{v_4,v_5\}]$.
\end{lemma}

\begin{proved}  Let $v_4'$ be the reflection of $v_4$ through $\op{aff}\{v_1,v_2,v_3\}$.
Set $x_{45}=|v_4-v_5|$ and $x'_{45}=|v_4-v_5|$.
Then $x'_{45} < x_{45}$ and both are roots by Lemma~\ref{tarski:cm5-root}.
\swallowed\end{proved}
\end{tarski}
%<<<<<



%>>>>>
\begin{tarski}
\name{cm5-small}
\summary{Given five points in the plane, if the plane through three doesn't separate the other two points, then
the distance between those two points is the smaller root of $R$.}
\tag{pt5, plane, halfspace, cm5-E}
\rating{40}
\guid{YOEHSIY}

\begin{lemma} \tlabel{tarski:cm5-small}
Let $\{v_1,v_2,v_3,v_5\}$ be a set of five
points in $\ring{R}^3$.  Assume that $\{v_1,v_2,v_3\}$ are not
coplanar.
If $v_4$ and $v_5$ lie in the same half-space of the plane through $v_1,v_2,v_3$, then 
 $|v_4-v_5|^2$ is the smaller root of $\mu_V[\{v_1,v_2,v_3\},\{v_4,v_5\}]$.
\end{lemma}

\begin{proved} This is proved as in Lemma~\ref{tarski:cm5-separate}.
\swallowed\end{proved}
\end{tarski}
%<<<<<










%>>>>>
\begin{tarski}
\section{Pass through}
\name{pass-cone}
\summary{If a segment passes through the convex hull of three points, then the endpoint of the segment lies in the cone
formed at the other endpoint and generated by the three points.}
%\endnote{Five points, An edge passes through a triangle.  Then the endpoint of the edge lies in $\op{aff}_+(w_1,\{v_1,v_2,v_3\})$. tarski:pass-cone}
\tag{pt5, conv2, conv3, cone}
\rating{60}
\guid{QHSEWMI}

\begin{lemma}\tlabel{tarski:pass-cone}
Suppose that $S=\{v_1,v_2,v_3,w_1,w_2\}$
is a set of five points in  $\ring{R}^3$.  Suppose
that  $\op{conv}\{w_1,w_2\}$ meets $\op{conv}\{v_1,v_2,v_3\}$,
but $w_1\not\in\op{conv}\{v_1,v_2,v_3\}$.
Then $w_2\in\op{cone}(w_1,\{v_1,v_2,v_3\})$.
\end{lemma}

\begin{proved}
We have 
  $$
  s_1 w_1 + s_2 w_2 = t_1 v_1 + t_2 v_2 + t_3 v_3
  $$
with $s_2>0$,
and hence also
  $$
  w_2 = w_1 + (t_1/s_2) (v_1-w_1) + 
       (t_2/s_2) (v_2 - w_1) + (t_3/s_2) (v_3-w_1)\in 
   \op{cone}(w_1,\{v_1,v_2,v_3\}).
  $$
\swallowed\end{proved}
\end{tarski}
%<<<<<





%>>>>>
\begin{tarski}
\name{tri-eta-small}
\summary{In a packing, a sufficiently short segment cannot pierce a triangle of small circumradius.}
\tag{pt5, eta, sqrt2, tri-seg-meet}
\rating{80}
\guid{DOUFCOI}

\begin{lemma}\tlabel{no-pass-sqrt2}\usage{DCG-[4.21]{ lemma:no-pass-sqrt2}}
%{\bf Lemma I 3.2.}
Suppose that  $S=\{v_1,v_2,v_3,w_1,w_2\}$
is a packing of five points.
Suppose that $S'=\{v_1,v_2,v_3\}$ is not collinear and that 
the circumradius of $S'$ is less than
$\sqrt2$.  Suppose that $|w_1-w_2|\le \sqrt8$.  Then
$\op{conv}(S')$ does not meet $\op{conv}\{w_1,w_2\}$.
\end{lemma}

\begin{proved}  Assume for a contradiction that they meet. 
By Lemma~\tref{tarski:eta-min}, the condition on the circumradius implies that
	$$|v_i-v_j| < \sqrt8, \text{ for } i\ne j.$$
By Lemma~\tref{tarski:cross}, the constraints prevent 	$\op{conv}\{w_1,w_2\}$
from meeting $\op{conv}\{v_i,v_j\}$ for $i\ne j$.  By Lemma~\tref{tarski:face},
the points $w_i$ do not meet $\op{conv}(S')$.

Pivot in a constraint satisfying way until $|w_i-v_j|=2$, for
$i=1,2$, $j=1,2,3$.  Let $p$ be the circumcenter of $S'$ with circumradius
% Let $v'_1$ be the
%point on the circumscribing circle of $\{v_1,v_2,v_3\}$ diametrically
%opposite $v_1$.  Then $w_1,w_2,v_1,v_1'$ give the vertices of a planar
%rhombus with side $2$.
$r=\eta_V(v_1,v_2,v_3)$. The Pythagorean theorem gives
   $$
   4 = |v_1-w_2|^2 = |v_1-p|^2 + |p-w_2|^2 = r^2 + \frac{|w_1-w_2|^2}4 \le r^2 + 2.
   $$
So $r\ge\sqrt2$, which is contrary to assumption.
\swallowed\end{proved}
\end{tarski}
%<<<<<








%>>>>>
\begin{tarski}
\section{Line crossing a triangle}
\name{line-tri}
\summary{Under general parametric conditions, a segment does not meet the convex hull of three points.}
\tag{condC, condF, pt5, collinear, conv2, conv3, cal5-E, triseg-meet}
\rating{200}
\guid{UQQVJON}

\begin{lemma}\tlabel{tarski:line-tri}
Let $$m_{12},m_{13},m_{14},m_{25},m_{35},m_{45},M_{23},M_{34},M_{24},M_{15}$$ 
be positive constants that satisfy the following constraints:
	$$
	\begin{array}{clll}
	\op{condC}(M_{15},m_{12},m_{13},M_{23},m_{35},m_{25})\\
	\op{condC}(M_{15},m_{13},m_{14},M_{34},m_{45},m_{35})\\
	\op{condC}(M_{15},m_{12},m_{14},M_{24},m_{45},m_{25})\\
	\op{condF}(m_{25},m_{35},m_{45},M_{34},M_{24},M_{23})\\
        \op{condF}(m_{12},m_{13},m_{14},M_{34},M_{24},M_{23})\\
	M_{15} < \CalE(m_{12},m_{13},m_{14},M_{34},M_{24},M_{23},m_{25},m_{35},m_{45}).
	\end{array}
	$$
Then there does not exist a set $S=\{v_1,v_2,\ldots,v_5\}$ of five
points  in $\ring{R}^3$ that satisfies the following conditions.  
Set $y_{ij}^2 = x_{ij} = |v_i-v_j|^2$.
	\begin{itemize}
	\item $\{v_2,v_3,v_4\}$ is not collinear.
	\item $\op{conv}\{v_1,v_5\}$ meets $\op{conv}\{v_2,v_3,v_4\}$.
	\item 
		$$
		\begin{array}{rlrlrl}
		|v_1-v_5| &\le M_{15},\\
		|v_2-v_3| &\le M_{23}, &|v_2-v_4| &\le M_{24}, & |v_3-v_4|&\le M_{34},\\
		|v_1-v_2| &\ge m_{12},&|v_1-v_3| &\ge m_{13},& |v_1-v_4|&\ge m_{14},\\
		|v_5-v_2| &\ge m_{25},&|v_5-v_3| &\ge m_{35},& |v_5-v_4|&\ge m_{45}.\\
		\end{array}
		$$
	\end{itemize}
\end{lemma}

% compare \cite[Sec. 4.2]{DCG}.


\begin{proved}
We assume such a figure exists.  We deform in such a way that the constraints continue
to hold and that is non-increasing in the length $|v_1-v_5|$.  By the conditions of the
lemma and Lemmas~\tref{tarski:cross} and \tref{tarski:face}, 
the edge $\op{conv}\{v_1,v_5\}$
does not meet the
 edges $\op{conv}\{v_i,v_j\}$, $(i,j)=(2,3),(2,4),(3,4)$ 
and the points $v_1,v_5$
do not meet $\op{conv}\{v_2,v_3,v_4\}$.  As we deform in a way that preserves constraints,
these conditions continue to hold.  By these constraints we see that the condition
  $\op{conv}\{v_1,v_5\}$ meets $\op{conv}\{v_2,v_3,v_4\}$
continues to hold under any deformation that preserves the other constraints.

By a series of constraint preserving pivots, we bring the figure to one that satisfies
	$$
	\begin{array}{rlrlrl}
	|v_2-v_3|&=M_{23}, &|v_2-v_4| &= M_{24}, &|v_3-v_4| &= M_{34}\\
	|v_1-v_2|&=m_{12}, &|v_1-v_3| &= m_{13}, &|v_1-v_4| &= m_{14}\\
	|v_5-v_2|&=m_{25}, &|v_5-v_3| &= m_{35}, &|v_5-v_4| &= m_{45}.
	\end{array}
	$$
Under these conditions, we find by the definition of $\CalE$ 
that $\CalE = |v_1-v_5|$.  Thus, we reach the contradiction
	$$
	\CalE = |v_1-v_5| \le M_{15} < \CalE.
	$$
\swallowed\end{proved}
\end{tarski}
%<<<<<





%>>>>>
\begin{tarski}
\name{qrtet-pair-pass}
\summary{If a segment meets a quasi-regular triangle, then the endpoints of the segment have distance at most $2.2$
from the vertices of the triangle.}
\tag{pt5, packing, t0, 2.52, sqrt2, conv2, conv3, 2.2}
\rating{40}
\guid{FRCXQKB}

\begin{lemma} \tlabel{tarski:qrtet-pair-pass}\usage{DCG-[4.22]{ lemma:qrtet-pair-pass}}
%{Lemma 1.4}
Let $S=\{v_1,v_2,v_3,v_4,v_5\}$ be a packing of five points.
Assume that
	$$
	\begin{array}{lll}
	|v_2-v_3|, |v_3-v_4|, |v_2-v_4|\le 2.52.\\
	|v_1-v_5|\le \sqrt8.
	\end{array}
	$$
Assume that $\op{conv}\{v_1,v_5\}$ meets $\op{conv}\{v_2,v_3,v_4\}$.
Then $|w-v|\le 2.2$ for $w\in\{v_1,v_5\}$ and $v\in\{v_2,v_3,v_4\}$.
\end{lemma}

\begin{proved}
This is a corollary of Lemma~\tref{tarski:line-tri} (if say $|v_1-v_2|\ge2.2$).
%Let the diagonal edge be $\{v_0,v_0'\}$ and the vertices of the
%face be $\{v_1,v_2,v_3\}$.  If $|v_i-v_0|>2.2$ or $|v_i-v_0'|>2.2$
%for some $i>0$, then constraint preserving deformations give the
%contradiction
%$$|v_0-v_0'|\ge{\CalE}(2,2,2,2.51,2.51,2.51,2,2,2.2) > \sqr8.$$
\swallowed\end{proved}
\end{tarski}
%<<<<<





%>>>>>
\begin{tarski}
\name{blades-disjoint}
\summary{In a packing, two blades at a common vertex meet only at that vertex, under suitable restrictions
on distances.}
\tag{pt5, t0, blade, aff-meet, cm5-E}
\rating{80}
\guid{ZHBBLXP}

\begin{lemma}
\tlabel{tarski:nocross}\usage{DCG-[5.25]{ lemma:nocross}}
% DCG: edges of a graph do not cross.
Let $S=\{v_0,v_1,v_2,v_3,v_4\}$ be a packing of five points.
Assume  that $|v_0-v_i|\le 2.51$ for $i=1,2,3,4$ and
$|v_1-v_3|,|v_2-v_4|\le 2.51$. Then the intersection of
$\op{cone}(v_0,\{v_1,v_3\})$ with
$\op{cone}(v_0,\{v_2,v_4\})$ is $\{v_0\}$.
\end{lemma}

\begin{proved} Assume for a contradiction that they meet at $x$.
We have
  $$x = t_0 v_0 + t_1 v_1 + t_3 v_3 = s_0 v_0 + s_2 v_2 + s_4 v_4,
  \quad t_1,t_3,s_2,s_4\ge0\quad t_0+t_1+t_3=s_0+s_2+s_4=1.
  $$
If $s_0\ge t_0$, then $\op{conv}\{v_1,v_3\}$ meets 
$\op{conv}\{v_0,v_2,v_4\}$.  
Lemma~\ref{tarski:line-tri} gives a 
contradiction
    $$2.51 < \CalE(2,2,2,2.51,2.51,2.51,2,2,2) \le |v_1-v_3|\le 2.51.$$
Similarly, if $t_0\ge s_0$, we get the contradiction
    $$2.51 < |v_2-v_4|\le 2.51.$$
\swallowed\end{proved}
\end{tarski}
%<<<<<








%>>>>>
\begin{tarski}
\section{Line passing through a triangle (tables)}
\name{def:E}
\summary{A predicate $E$ asserts that a segment does not meet a triangular region.}
\tag{def, pt5, cm5-E, conv2, conv3, triseg-meet}
\rating{0}
\guid{CUNATVU}

\begin{definition}
In the next few lemmas, we let
$E(m_{12},m_{13},m_{14},M_{34},M_{24},M_{23},m_{25},m_{45},m_{35};M_{15})$ 
be the following assertion:
There does not exist a set of five points $S=\{v_1,\ldots,v_5\}$ in
$\ring{R}^3$ with the following properties.
\begin{itemize}
  \item For every distinct $u,v\in S$, we have $|u-v|\ge 2$.
    \item $\op{conv}\{v_1,v_5\}$ meets $\op{conv}\{v_2,v_3,v_4\}$.
  \item $$
    \begin{array}{lll}
      |v_1-v_2|\ge m_{12}, &|v_1-v_3|\ge m_{13}, &|v_1-v_4|\ge m_{14}\\
      |v_3-v_4|\le M_{34}, &|v_2-v_4|\le M_{24}, &|v_2-v_3|\le M_{23}\\
      |v_2-v_5|\ge m_{25}, &|v_3-v_5|\ge m_{35}, &|v_4-v_5|\ge m_{45}\\
      |v_1-v_5|\le M_{15},\\
      \end{array}
    $$
\end{itemize}
\end{definition}
Statements of this form will be immediate corollaries of Lemma~\tref{tarski:line-tri}.
\end{tarski}
%<<<<<



%>>>>>
\begin{tarski}
\name{EE}
\summary{A large table of values of $\CalE$ is computed, which gives the non-existence of many different arrangments of five points, in which
a segment crosses the convex hull of three points.}
%\endnote{{\tt Enclosed Calculations:} tarski:EE (generic name), tarski:245, tarski:245bis, tarski:277, tarski:2t0-doesnt-pass-through, tarski:E:part4:1, tarski:enclosed-v, tarski:last:E, tarski:pass-anchor }
\tag{cm5-E, dodec, triseg-meet, conv2, conv3, pt5, sqrt2, 2.55, t0, 2.906, 2.77, 2.45, 3.2, 2.6, 2.72, 2.91, 2.517 }
\rating{230}
\guid{RLYYNDE}

\begin{lemma}\tlabel{tarski:EE}
We have 
   $$
   E(y_1,y_2,y_3,y_4,y_5,y_6,y_6,y_8,y_9;y_{10}),
   $$
for the following $10$-tuples:
$$
\def\tuid#1{\relax} %{\text{\guid{#1}}}
\def\rr#1{\text{\tt \# #1:}}
\begin{array}{llllllllllll}
%
 \rr{12}&\tuid{EKTYKKO}\tlabel{tarski:E:part4:4}
   &    2 &2 &2 &  2.51& 2.51 &3.2&   2.51 & 2 & 2 & 2.51\\
%
 \rr{13}&\tuid{TZKIBQD}\tlabel{tarski:E:part4:5}
   &2 &2 &2 &2.51& \sqrt{8} &\sqrt{8}& 2.51 & 2.51 & 2.51 & 2.51\\
%
 \rr{14}&\tuid{NYHXWOU}\tlabel{tarski:E:part4:6}
   &2 &2 &2 &2.51& 2.51 &3.2& 2.51 & 2.51 & 2 & 2.51\\
%
 \rr{15}&\tuid{HCEDPSK}\tlabel{tarski:E:part4:7}
   &2 &2 &2 &2.51& 2.51 &3.2& 2 & 2.51 & 2 & 2.51\\
%
 \rr{16}&\tuid{ZDKFOUK}\tlabel{tarski:E:part4:8}
   &2 &2 &2 &\sqrt{8}& \sqrt{8} &2.51& 2.51 & 2.51 & 2 &  2.51\\
%
 \rr{17}&\tuid{OQXGZCQ}\tlabel{tarski:E:part4:9}
   &2 &2 &2 &\sqrt{8}& \sqrt{8} &2.51& 2 & 2.51 & 2.51 &  2.51\\
%
 \rr{02}&\tuid{SDUXZXL}\tlabel{tarski:E1453}
   &2 &2 &2 &2.51& 2.51 &2.906& 2 & 2 &2.51 & 2.51\\
%
 \rr{19}&\tuid{QFJLWUC}\tlabel{tarski:dcg-p142}
   &2 &2 &2 &   3.2& 3.2 &3.2&   2 & \sqrt8 & 3.2 &  2.51\\
%
 \rr{08}&\tuid{TVBTKEK}\tlabel{tarski:node}
   &2 &2 &2 &\sqrt8& \sqrt8 &\sqrt8& 2.51 & 2.51 & 2 & 2.51\\ 
%
 \rr{01}&\tuid{GVKETKP} \tlabel{tarski:2t0-doesnt-pass-through}%\usage{DCG-[4.19]{ lemma:2t0-doesnt-pass-through}}
   &2 &2 &2 & 2.55&  2.55 & \sqrt{8}&  2 & 2 & 2 & 2.55\\
 %
 \rr{21}&\tuid{RPIOZZC}\tlabel{tarski:last:E}%\usage{DCG-[11.22]{ remark:2.6}}
   &2 &2 &2 &  2.51& 2.51 &\sqrt8&  2 & 2 & 2 & 2.6\\
%
 \rr{10}&\tuid{MYWEFEM}\tlabel{tarski:E:part4:2}
   &2 &2 &2 &2.51 & 2.51  &\sqrt{8}&   2 & 2 & 2 &  2.6\\
%
 \rr{22}&\tuid{JBUAAWJ} \tlabel{tarski:rem2.7}
     &2 &2 &2 &2.2& 2.51 &\sqrt{8}& 2 & 2 & 2 & 2.7\\
%
 \rr{11}&\tuid{EVWERXQ}\tlabel{tarski:E:part4:3}
   &2 &2 &2 &2.1& 2.51  &\sqrt{8}&   2 & 2 & 2 &  2.72\\
%
 \rr{04}&\tuid{HYEOHXS}\tlabel{tarski:enclosed-v}
   &2 &2 &2 &\sqrt{8}& 2.51 &2.51& 2.51 & 2 & 2 &  2.77\\
%
 \rr{05}&\tuid{MCWJKBZ}\tlabel{tarski:277}
   &2 &2 &2 &\sqrt{8}& 2.51 &2.51& 2.51 & 2 & 2 & 2.77\\
%
 \rr{06}&\tuid{TVORNRR}\tlabel{tarski:245}
   &2 &2 &2 &\sqrt{8}& 2.51 &2.51& 2 & 2 & 2.45 & \sqrt{8}\\
%
 \rr{07}&\tuid{AMVZLGB}\tlabel{tarski:245bis}
   &2 &2 &2 &\sqrt{8}& 2.51 &2.51& 2.51 & 2.45 & 2 & \sqrt{8}\\
%
 \rr{09}&\tuid{UREAOFU}\tlabel{tarski:E:part4:1}
   &2 &2 &2 &  2& 2 &3.2&   \sqrt{8} & 2 & 2 &  \sqrt{8}\\
%
 \rr{20}&\tuid{TADGJXO}\tlabel{tarski:convex-quad}
   &2 &2 &2 &   2.517& 2.517 &2.517&  2.3 & 2 & 2 &  \sqrt8\\
%
 \rr{23}&\tuid{NNPXALF} \tlabel{tarski:pass-anchor}%\usage{DCG-[4.24]{ lemma:pass-anchor}}
     &2&2&2&\sqrt8 &2.51 &2.51 &2 &2 &2.51&\sqrt8\\
%
 \rr{18}&\tuid{GCRJNIH}\tlabel{tarski:E:part4:10}
   &2 &2 &2 &  2& 2.51 &2.51&   2 & 2 & 2 &  2.91\\
%
 \rr{03}&\tuid{DIKGYMO}\tlabel{tarski:dcg-p89}
   &2 &2 &2 &\sqrt8& 2 &2& 2 & 2 & 2.77 &2.906\\
\end{array}$$
\end{lemma}

\begin{proved}
\swallowed\end{proved}
\end{tarski}
%<<<<<


 






%>>>>>
\begin{tarski}
\name{dcg-1220}
\summary{Under suitable conditions, a segment does not meet a blade.}
\tag{pt5, conv2, blade, aff-meet, 3.2, cal5-E}
\rating{300}
\guid{VNZLYWT}

\begin{lemma}\tlabel{tarski:dcg-1220}
  There does not exist a set $\{v_0,v_1,v_2,v,w\}$ of five points
in $\ring{R}^3$ with the following properties:
\begin{itemize}
  \item $\op{conv}\{v,w\}$ meets $\op{cone}(v_0,\{v_1,v_2\})$.
  \item $2\le |u-v_0| \le 2.51$, for $u=v_1,w,v,v_2$.
    \item $|v-v_1|\ge 3.2$,  $|v-v_2|\ge 2$,
    \item $|w-v_1|\ge2$, $|w-v_2|\ge2$,
    \item   $|v-w|\le2$, $2\le |v_1-v_2|\le 3.2$.
\end{itemize}
\end{lemma}

\begin{proved} Assume the set of solutions is nonempty.  The constraints are
translation invariant.  Thus, without loss of generality $v_0=0$.  The space $X$
of solutions in $(v_1,v_2,v,w)\in\ring{R}^{12}$ 
is closed and bounded hence compact.  The function
$$
   g(v_0,v,w) = |v-w| + |w-v_0| + |v_0-v|
$$
is continuous on $X$.  Let $X_1\subset X$ be the nonempty compact set on which
the function $g$ attains its maximum.
The function
$$
   g_1(v_0,v_1,v_2,v,w) = \sum_{u = v_0,v,w}\sum_{i=1}^2 |v_i-u|
$$
is continuous on $X_1$.  Let $(v_0,v_1,v_2,v,w)$ be a point that
minimizes $g_1$ over $X_1$.  We study the properties of this particular solution.

If the segment $\op{conv}^0\{v,w\}$ meets $\op{conv}\{v_0,v_1,v_2\}$,
then the desired impossibility proof follows by
   $$
   E(2,2,2,  3.2,2.51,2.51,  2,2, 3.2; 2).
   $$
Hence, the segment $\op{conv}^0\{v_1,v_2\}$ 
meets $\op{conv}\{v_0,v,w\}$.

We distinguish two cases:
\begin{itemize}
\item $\op{conv}\{v_1,v_2\}$ meets $\op{conv}\{v_0,v,w\}\setminus \op{conv}^0\{v_0,v,w\}$.
\item $\op{conv}\{v_1,v_2\}$ meets $\op{conv}^0\{v_0,v,w\}$.
\end{itemize}
In the second case, the figure is rigid, with $|v-w|$, $|w-v_0|$ and $|v_0-v|$ attaining
their upper bounds and $|v_i-u|$ attaining their lower bounds.  By direct calculation of
dihedral angles, we have the contradiction
$$
\pi\ge \dih_V(\{v_0,w\},\{v_1,v_2\})= \dih_V(\{v_0,w\},\{v_1,v\})+ \dih_V(\{v_0,w\},\{v,v_2\})
> 2.4+1.2 >\pi.
$$

We now turn to the first case.  The edges $\op{conv}\{v_1,v_2\}$ and $\op{conv}\{v,w\}$
cannot meet by Lemma~\ref{tarski:cross-edge-generic}.  Swapping $v$ and $w$ if necessary,
we may assume that $\op{conv}\{v_1,v_2\}$ meets $\op{conv}\{v_0,w\}$.  We have
$|w-v|=2$, for otherwise $g$ is not maximal (pivot $v_1$).  Similarly, $|v-v_0|=2.51$,
$|v_2-v|=2$, and $|v_1-v|=3.2$.

%The proof
%follows by a constraint preserving deformation, 
%provided that $\{v_0,v_1,v_2,w\}$
%are not coplanar. Assume for a contradiction that 
%   $$
%   w\in P=\op{aff}\{v_0,v_1,v_2\}.
%   $$ 
%We move back to the nonplanar case if
%$|v_2-v|$ is not $2$ (pivot $v_2$ around $\{v_0,w\}$ toward $v$), if
%$|v_1-v|$ is not $3.2$ (pivot $v_1$ around $\{v_0,w\}$ toward $v$), if
%$|w-v|$ is not $2$ (pivot $w$ around $\{v_1,v_2\}$ away from $v$),
%or $v$ is not $2.51$ (pivot $v$ and $w$ simultaneously preserving
%$|w-v|$ around $\{v_1,v_2\}$).  Therefore, we may assume without
%loss of generality that $|v_2-v|=2$, $|v_1-v|=3.2$, $|w-v|=2$, and
%$|v-v_0|=2.51$.


Let $p$ be the orthogonal projection of $v$ to the plane $P$.  Let
$h=|v-p|$. 
The distances from $p$ to $u\in P$ is
$f_h(|v-u|)$, where $f_h(x)=\sqrt{x^2-h^2}$.
Note that
$$
2\le |v_2-w|\le |v_2-p|+|p-w|=f_h(|v_2-v|)+f_h(|v-w|) = 2 f_h(2) = 2\sqrt{4-h^2}.
$$
Hence, $0\le h \le\sqrt3$.
%(The upper bound
%$\sqrt3$ is determined by the condition that the triangle
%$\{w,v_1,v\}$, which is equilateral in the extreme case, must exist under
%the given edge constraints.)

%We consider two cases depending
%on whether we can find a line in $P$ through $p$ dividing the
%plane into a half-plane containing $v_1$, $v_0$, and $v_2$, or into
%a half-plane containing $v_1$, $w$, and $v_2$.  In the first case
%we have

Note that $\op{aff}\{v_1,v_2\}$ separates $v_0$ from $w$.  If
$p\in \op{aff}_+^0(\{v_1,v_2\},v_0)$, then 
 $$
\begin{array}{lll}
    0&=\arc_V(p,v_0,v_1)+
    \arc_V(p,v_0,v_2)- \arc_V(p,v_1,v_2)\\
    &\ge \arc(f_h(3.2),f_h(2.51),2)+
    \arc(f_h(2),f_h(2.51),2)-\arc(f_h(3.2),f_h(2),3.2)
\end{array}
 $$
The function $\arc$ is monotonic in the arguments and from this it
follows easily that this function of $h$ is positive on its domain
$0\le h\le \sqrt3$. This is a contradiction.  

Otherwise, we obtain the
related contradiction
  $$
\begin{array}{lll}
    0&=\arc_V(p,v_1,w)+
    \arc_V(p,w,v_2)-
    \arc_v(p,v_1,v_2)\\
    &\ge \arc(f_h(3.2),f_h(2),2)+
    \arc(f_h(2),f_h(2),2)-\arc(f_h(3.2),f_h(2),3.2)\\
    &>0
\end{array}
  $$
\swallowed\end{proved}
\end{tarski}
%<<<<<





%>>>>>
\begin{tarski}
\name{pass-makes-quarter}
\summary{If a segment of length at most $\sqrt8$ meets the convex hull formed by an anchor triangle, then a quarter is formed.}
%
\tag{pt5, packing, conv2, conv3, t0, sqrt2, anchor, cm5-E}
\rating{100}
\guid{VAXNRNE}

\begin{lemma} \tlabel{tarski:pass-makes-quarter}\usage{DCG-[4.34]{ tarski:pass-makes-quarter}}
%{Lemma 1.9}
%\FIWW{I'm not sure the second part gets used.  Deleting it would simplify the proof.}
Let $\{v_0,v_1,v_2,v_3,w\}$  be a packing of five points.
Assume $2.51\le|w-v_0|\le\sqr8$, $|v_1-v_3|\le\sqr8$.
Assume that $|u-v|\le 2.51$, with $u=v_1,v_2,v_3$,
$v=v_0,w$.
Assume that $\op{conv}\{v_1,v_3\}$ meets $\op{conv}\{v_0,w,v_2\}$.
Then
$\min(|v_1-v_2|,|v_2-v_3|)\le2.51$. Furthermore, if the minimum is
$2.51$, then $|v_1-v_2|=|v_2-v_3|=2.51$.
\end{lemma}


\begin{proved}
Assume that $|v_1-v_2|\ge 2.51$ and $|v_2-v_3|\ge2.51$.

We have $|v_1-v_3|=\sqrt8$.  Otherwise,
$|v_1-v_3|=r<\sqrt8$, and the configuration does not exist by
exist by the calculation
  $$
  E(2.51,2,2,\sqrt8,2.51,2.51,2.51,2,2; r).
  $$
We have $|v_0-w|=\sqrt8$, for otherwise $|v_0-w|=r<\sqrt8$ and
$$
   E(2.51,2,2,r,2.51,2.51,2.51,2,2,\sqrt8).
$$
We have $|w-v_3|=2$, for otherwise,
$|w-v_3|=2+s$, with $s>0$, and  
the configuration does not exist:
  $$
  E(2.51,2,2+s,\sqrt8,2.51,2.51,2.51,2,2;\sqrt8). 
  $$
Similarly, $|w-v_i|=|v_0-v_i|=2$, for $i=1,3$.

Thus, $(v_0,v_1,w,v_3)$ is a square.
 We may also assume, without loss of generality, that
$|w-v_2|=|v_2-v_0|=2.51$. This forces $|v_2-v_i|=2.51$, for $i=1,3$.
This is rigid,  and is the unique figure that satisfies the
constraints. The lemma follows.
\swallowed\end{proved}
\end{tarski}
%<<<<<









%>>>>>
\begin{tarski}
\name{dcg-p122}
\summary{Under suitable conditions, a segment does not meet the convex hull of three points.}
%\endnote{Five points, part of the erasing with penalty 0.008 proof, tarski:dcg-p122}
\tag{pt5, packing, t0, sqrt2, conv2, conv3, triseg-meet, 3.2}
\rating{120}
\guid{GMOKOTN}

\begin{lemma}\tlabel{tarski:dcg-p122}
% DCG part of proof of Lemma 11.23.
% DCG big gap on anchors.
Let $S=\{v_0,v_1,v_2,v_3,v_4\}$ be a packing of five points.
Assume that $|u-w|\le 2.51$, for
$u\in\{v_0,v_2\}$ and $w\in\{v_1,v_3,v_4\}$.
Assume that $|v_1-v_3|\le \sqrt8$ and $|v_0-v_2|\le\sqrt8$.
Assume that $\op{conv}\{v_1,v_3\}$ meets $\op{conv}\{v_0,v_2,v_4\}$.
Then $|v_1-v_4|< 3.2$.
\end{lemma}

\begin{proved} Assume for a contradiction that $|v_1-v_4|\ge 3.2$.
By
constraint preserving deformations, we arrive at the rigid figure
    $$
    \begin{array}{lll}
    |v_2-v_0|&=\sqrt{8},\ |v_1-v_0|=|v_1-v_2|=|v_2-v_3|=|v_3-v_0|=|v_3-v_4|=2
    \\ |v_2-v_4|&=|v_4-v_0|=2.51,\ |v_1-v_4|=3.2.
    \end{array}
    $$
The dihedral angles give
    $$
    \begin{array}{rll}
    &\phantom{=}\dih_V(\{v_0,v_2\},\{v_1,v_4\}) &+ \dih_V(\{v_0,v_2\},\{v_3,v_4\})\\
    &= \dih(\sqrt{8},2,2.51,3.2,2.51,2) &+
    \dih(\sqrt{8},2,2.51,2,2.51,2)\\
    &> 2.3 &+ 1.16 \\
    &>\pi .
    \end{array}
    $$
This is contrary to the claim that the edge
$\op{conv}^0\{v_1,v_3\}$ meets $\op{conv}\{v_0,v_2,v_4\}$.
\swallowed\end{proved}
\end{tarski}
%<<<<<










% I think that XILK is no longer needed, now that sigma<=0 argument is gone.



%\begn{lemma}\guid{XILKJJV}
%[First separation lemma]
%\tlabel{tarski:sqrt2-cone-avoidance}\usage{DCG-[8.17]{ lemma:sqrt2-cone-avoidance}}\dcg{Lemma~8.17}{78}
%Let $S=\{v_0,v,v_2,v_3\}$ be a set of four points in $\ring{R}^3$.
%Assume that the separation of points in $S$ is at least $2$.
%    Suppose $|v-v_0|\le\sqrt8$.  Suppose $\eta(v_0,v_2,v_3)<\sqrt2$.
%    Then 
%    $\op{rcone}^0(v_0,v,|v-v_0|/\sqrt8)$ does not meet 
%    $C=\op{cone}(v_0,\{v_2,v_3\})$.
%\end{lemma}
%
%\begin{proved}
%Let $D$ be the open disk spanning the circle of intersection of
%$B(v_0,\sqrt2)$ and $B(v,\sqrt2)$.  The union of the rays from $v_0$
%through $D$ is all of $\op{rcone}^0$.  It is enough to show that this
%disk does not meet $C$.  This disk is contained in $B(v,\sqrt2)$,
%and so we bound this ball away from the given cone.
%
%Assume for a contradiction that these two sets meet.  Let $v'$ be
%the reflection of $v$ through the plane $P = \op{aff}\{v_0,v_2,v_3\}$.
%Let $p=(v'+v)/2$ be the closed point of $P$ to $v$.
%
%If $p$  lies outside $C$, then the
%edge constraint $|v-v_0|\le\sqrt8$ forces the closest point in $C$ to
%lie in $\op{conv}\{v_0,v_2\}\cup\op{conv}\{v_0,v_3\}$.  Since
%$|v_2-v_0|,|v_3-v_0|\le\sqrt8$, this closest point has distance at least
%$\sqrt2$ from $v$. Thus, we may assume that the closest point in
%$P$ to $v$ lies in $C$.
%
%Assume next that $p$ lies in 
%$\op{conv}\{v_0,v_2,v_3\}$.  We obtain an edge
%$\op{conv}^0\{v,v'\}$ of length at most $\sqrt8$ that meets a
%triangle $\op{conv}\{v_0,v_2,v_3\}$ 
%of circumradius less than $\sqrt2$. This contradicts
%Lemma \tref{no-pass-sqrt2}.
%
%Assume finally that $p\in\op{cone}(v_0,\{v_2,v_3\})$,
%but $p\not\in\op{conv}\{v_0,v_2,v_3\}$. By
%moving $v$ toward $C$ (preserving $|v-v_0|$), we may assume that
%$|v-v_2|=|v-v_3|=2$.  Stretching the edge $\{v_2,v_3\}$, we may
%assume that the circumradius of $\{v_0,v_2,v_3\}$ is precisely
%$\sqrt2$.  Since the closest point in $P$ is not in 
% $\op{conv}\{v_0,v_2,v_3\}$, we may move $v_2$ and $v_3$ away from $v$ while
%preserving the circumradius and increasing the lengths $|v-v_2|$ and
%$|v-v_3|$.  By moving $v$ again toward $C$, we may assume without
%loss of generality that $|v_2-v_0|=|v_3-v_0|=2$ and $|v_2-v_3|=\sqrt8$. We
%have reduced to a one-parameter family of arrangements, parametrized
%by $|v-v_0|$. We observe that the disk $D$ is
%tangent to the segment $\op{conv}\{v_2,v_3\}$ at its midpoint, no matter what
%the value of $|v-v_0|$ is.  Thus, in the extremal case, the open disk
%does not intersect the segment $\{v_2,v_3\}$ or the cone $C$ that it
%generates.  This completes the proof.
%\swallowed\end{proved}




%>>>>>
\begin{tarski}
\section{Cone separation}
\name{cone-avoidance}
\summary{Under suitable conditions, a blade meets a right-circular cone only at their common apex.}
\tag{packing, pt4, t0, sqrt2, rcone, blade}
\rating{120}
\guid{EFXWFNQ}

\begin{lemma}\tlabel{tarski:cone-avoidance}\dcg{Lemma~8.18}{79}
Let $S=\{v_0,v_1,v_2,v_3\}$ be a packing of four points.
Assume $|v_1-v_0|\le 2.51$. Assume $|v_1-v_2|>2.51$.
Assume that two of the separations in $\{v_0,v_2,v_3\}$ are at
          most $2.51$ and the third is at most $\sqrt8$.
     Then $\op{rcone}^0(v_0,v_1,|v_1-v_0|/\sqrt8)$ does not meet
$C = \op{cone}(v_0,\{v_2,v_3\})$.
\end{lemma}

\begin{proved}
Let $D$ be the open disk spanning the circle of intersection of
$B(v_0,\sqrt2)$ and $B(v_1,\sqrt2)$.  The union of the rays from $v_0$
through $D$ is all of $\op{rcone}^0$.  It is enough to show that this
disk does not meet $C$.  This disk is contained in $B(v,\sqrt2)$,
and  we bound this ball away from the given cone.

Let $p$ be the orthogonal 
projection of $v_1$ to the plane $P=\op{aff}\{v_0,v_2,v_3\}$.
If $p$  lies outside $C$, then the
edge constraint $|v_1-v_0|\le 2.51$ forces the closest point in $C$ to
lie in $\op{conv}\{v_0,v_2\}\cup\op{conv}\{v_0,v_3\}$.  Since
$|v_2-v_0|,|v_3-v_0|\le\sqrt8$, this closest point has distance at least
$\sqrt2$ from $v_1$. Thus, we may assume that the closest point in
$P$ to $v_1$ lies in $C$.
We may now assume that $p\in C$.

Let $v_1'$ be the reflection of $v_1$ through $P$.  We
have that either $\op{conv}^0\{v_2,v_3\}$ meets $\op{conv}\{v_0,v_1,v'_1\}$ or
$\op{conv}^0\{v_1,v'_1\}$ meets $\op{conv}\{v_0,v_2,v_3\}$.  We may assume for
a contradiction that $|v_1-v'_1|<\sqrt8$.

If $\op{conv}^0\{v_2,v_3\}$ meets $\op{conv}\{v_0,v_1,v'_1\}$, then
Lemma~\tref{tarski:pass-anchor} gives the contradiction
$|v_1-v_2|\le2.51$.

If $\op{conv}^0\{v_1,v'_1\}$ meets $\op{conv}\{v_0,v_2,v_3\}$, then by
Lemma~\tref{tarski:qrtet-pair-pass} the longest edge of $\{v_0,v_2,v_3\}$
has length greater than $2.51$. 
Moreover, $v_1$ and $v'_1$ both have distances at most $2.51$
from both endpoints of the longest edge of $\{v_0,v_2,v_3\}$, 
by Lemma~\tref{tarski:pass-anchor}.  Since
$|v_1-v_2|>2.51$, the longest edge must not have $v_2$ as an endpoint,
so the longest edge is $\{v_0,v_3\}$.
Lemma~\tref{tarski:pass-makes-quarter} forces one of $|v_1-v_2|$ or
$|v_1'-v_2|$ to be at most $2.51$. But these are both equal to
$|v_1-v_2|>2.51$, a contradiction.
\swallowed\end{proved}
\end{tarski}
%<<<<<












%\move{lemma}{tarski:at-most-one-negative}\usage{DCG-[5.13]{ lemma:at-most-one-negative}}

%>>>>>
\begin{tarski}
\section{Barrier}
\name{sqrt2-unobstructed}
\summary{In a packing, if a ball meets the convex hull of three points, then the ball must also be close to all three points.}
\tag{pt4, packing, t0, sqrt2, ball, conv3}
\rating{80}
\guid{LOKDUSU}

\begin{lemma}\tlabel{tarski:sqrt2-unobstructed}\usage{DCG-[5.20]{ lemma:sqrt2-unobstructed}}
%DCG p48, L.5.20.
Let $\{v_0,v_1,v_2,v_3\}$ be a packing of four points.
Assume that 
$$
   |v_1-v_2|\le 2.51,\quad |v_2-v_3|\le 2.51,\quad |v_1-v_3|<\sqrt8.
$$
Assume that $B(v_0,\sqrt2)$ meets $T=\op{conv}\{v_1,v_2,v_3\}$ at $x$.
Then $|v_0-v_i|\le 2.51$, for $i=1,2,3$.
\end{lemma}

% OLD VERSION:
%\begin{lem}\guid{TLKHAUY} \tlabel{tarski:sqrt2-unobstructed}
%DCG p48, L.5.20.
%If $x$ lies in the open ball of radius $\sqrt2$ at the origin, and
%if $x$ is not in the closed cone over any simplex in $\CalQ_0$,
%then the origin is unobstructed at $x$.
%\end{lemma}

\begin{proved}
No point of $\op{conv}\{v_i,v_j\}$ meets $B(v_0,\sqrt2)$.  (The
extreme case is a triangle with sides $2,2,\sqrt8$: the hypotenuse
has distance exactly $\sqrt2$ from the opposite vertex.)
Suppose that the closest point $v_0$ in
$T$ is an interior point $p$. Reflect $v_0$
through the plane of $T$ to get $v_0'$. The assumptions imply
that $\op{conv}\{v_0,v_0'\}$ meets $T$ and has
length less than $\sqrt8$. If $|v_1-v_3|\le2.51$, then
then Lemma~\tref{tarski:qrtet-pair-pass} implies the result.
Assume that $|v_1-v_3|>2.51$.
By Lemma~\tref{tarski:pass-makes-quarter},  
either $|v_0-v_2|\le2.51$ or $|v_0'-v_2|\le 2.51$.  Given that
$v_0'$ is the mirror image of $v_0$, these distances are equal;
hence $|v_0-v_2|\le 2.51$.  By Lemma~\tref{tarski:pass-anchor},
$|v_0-v_1|,|v_0-v_3|\le 2.51$.
\swallowed\end{proved}
\end{tarski}
%<<<<<


%>>>>>
\begin{tarski}
\name{unobstr-t0}
\summary{pt4, packing, t0, sqrt2, ball, conv3}
%\endnote{Four points, Ball $B(v_0,1.255)$ does not meet a barrier. tarski:unobstr-t0}
\tag{A ball does not meet the convex hull of three points, under suitable distance constraints.}
\rating{80}
\guid{FFSXOWD}

\begin{lemma}\tlabel{tarski:unobstr-t0}\usage{DCG-[5.22]{ lemma:unobstr-t0}}
Let $\{v_0,v_1,v_2,v_3\}$ be a packing of four points. Assume that
$$
   |v_1-v_2|\le 2.51,\quad |v_2-v_3|\le 2.51,\quad |v_1-v_3|<\sqrt8.
$$
Then $B(v_0,1.255)$ does not meet $T=\op{conv}\{v_1,v_2,v_3\}$.
\end{lemma}

%\begin{lem}\guid{OXYYCVX}
%If $x\in B(v,1.255)$, then $x$ is unobstructed at $v$.
%\end{lemma}

\begin{proved}   For a contradiction, assume they meet. 
We imitate the proof of Lemma~\tref{tarski:sqrt2-unobstructed}.
Let $v_0'$ be the reflection of $v_0$ through the plane of $T$.
By
Lemma~\tref{tarski:2t0-doesnt-pass-through}, we have $|v_1-v_3|>2.51$.
This implies the result.
\swallowed\end{proved}
\end{tarski}
%<<<<<













%>>>>>
\begin{tarski}
\section{Point in a tetrahedron}
\name{v-interior}
\summary{In a packing, a point does not fit inside a simplex whose edge lengths are at most $\sqrt8$.}
\tag{pt5, packing, sqrt2, inside, conv4, cm5-E}
\rating{80}
\guid{ZODWCKJ}

\begin{lemma} \tlabel{tarski:v-interior}\usage{DCG-[4.16]{ lemma:v-interior}}
Let $\{v,v_1,\ldots,v_4\}$ be a packing of five points.  Assume
	$$ |v_i-v_j| \le \sqrt8, \text{ for } i\ne j.$$
Then $v\not\in \op{conv}\{v_1,v_2,v_3,v_4\}$.
\end{lemma}

%An alternate proof based on spherical geometry is provided
%in \cite[Lemma 4.16]{DCG}.

\begin{proved}  Assume for a contradiction that $\{v,v_1,v_2,v_3,v_4\}$ is a set
of five points satisfying the hypotheses with $v\in\op{conv}\{v_1,v_2,v_3,v_4\}$.
By Lemma~\tref{tarski:face}, the point $v$ cannot lie on any face
	$$\op{conv}\{v_i,v_j,v_k\}.$$
We pivot points in the figure in a constraint preserving way until 
	$$
	|v_i-v_j = \sqrt8,\text{ for } 1 \le i < j \le 4.
	$$
Re-indexing the vectors $v_i$ if necessary, we then pivot $v$ until
	$$
	|v-v_i| = 2, \text{ for } i=1,2,3.
	$$
By Lemma~\tref{tarski:cm5-small}, $x_{45}=|v-v_4|^2$ is the smaller root $r$ of the quadratic
polynomial
	$\mu_V[\{v_1,v_2,v_3\},\{v_4,v\}]$.
However, we compute this root and find it satisfies $r < 4$, contrary to the
constraint $|v-v_4|^2 \ge 4$.
\swallowed\end{proved}
\end{tarski}
%<<<<<

%>>>>>
\begin{tarski}
\name{point-in-simplex}
\summary{In a packing, a point does not fit inside a simplex all of whose edge lengths but one are at most $\sqrt8$.}
\tag{pt5, packing, sqrt2, inside, conv4, cm5-E}
\rating{80}
\guid{KCGHSFF}

\begin{lemma}\usage{DCG-[4.15]{ lemma:v-interior-alt}}\tlabel{tarski:point-in-simplex} % BN
Let $\{v_1,v_2,v_3,v_4,w\}$ be a packing of five points.  If
$|v_i-v_j|<\sqrt8$ for all $\{i,j\}\ne\{1,2\}$, then $w\not\in\op{conv}\{v_1,v_2,v_3,v_4\}$.
\end{lemma}

\begin{proved}
Assume for a contradiction that $w\in\op{conv}\{v_1,v_2,v_3,v_4\}$.
Pivot $v_1$ around the axis $\{v_3,v_4\}$ away from $v_2$ until $w\not\in\op{conv}\{v_1,v_2,v_3,v_4\}$.
Reindexing if needed, we have $w\in\op{conv}\{v_1,v_2,v_3\}$.
By Lemmas~\ref{tarski:face} and~\ref{tarski:skinny-tri}, we have $w\not\in\op{conv}\{v_1,v_2,v_3\}$.
%
%We have $|w-v_i|\ge 2$ and $|v_1-v_2|<\sqrt8$.  Hence $\arc_V(w,v_1,v_3)<\pi/2$,
%and similarly, $\arc_V(w,v_2,v_3)<\pi/2$.
%Then 
%$$
%2\pi = \sum_{i\ne j} \arc_V(w,v_i,v_j) < \pi + \pi/2 + \pi = 2\pi.
%$$
\swallowed\end{proved}
\end{tarski}
%<<<<<

% constant was 2.6 in DCG.
%>>>>>
\begin{tarski}
%\section{Point in a skinny tetrahedron}
\name{v-interior-alt}
\summary{In a packing, a point cannot lie inside the convex hull of
  four points, if distances are suitably constrained.
This is essentially the same as Lemma~\ref{tarski:point-in-simplex},
so it will be deleted.}
\tag{pt5, packing, conv4, inside, deprecated}
\rating{0}
\guid{TEAULMK}

\begin{lemma}
$\{v_1,v_2,v_3,v_4,v_5\}$ be a packing of
five distinct points.  Suppose that $|v_i-v_j|\le
\sqrt8$ for $i\ne j$ and $\{i,j\}\ne\{1,4\}$. Then $v_5$ does not lie
in $\op{conv}\{v_1,v_2,v_3,v_4\}$.
\end{lemma}

%The proof in \cite[Lemma 3.5]{part1} is based on spherical trigonometry.
%Here, we give an alternate proof.

\begin{proved}
The conditions of Lemma~\tref{tarski:skinny-tri} are met. This implies
that $v_5$ does not lie in $\op{conv}\{v_1,v_4,v_2\}$ or
	$\op{conv}\{v_1,v_4,v_3\}$.
The conditions of Lemma~\tref{tarski:face} are met.  This implies that
$v_5$ does not lie in $\op{conv}\{v_2,v_3,v_4\}$ or
	$\op{conv}\{v_2,v_3,v_1\}$.
By a series of pivots, we may assume that
	$$|v_1-v_2|=|v_1-v_3|=|v_2-v_3|=|v_2-v_4|=|v_3-v_4|=\sqrt8.$$
By a series of pivots of $v_5$, we may assume that its distance is exactly
$2$ from three of $S=\{v_1,v_2,v_3,v_4\}$.  
Pivot in a constraint satisfying way to decrease $|v_5-v_4|$.
If $|v_1-v_4|<\sqrt8$, then Lemma~\tref{XX} % not skinny
gives the
result.
Eventually, $v_5$ has distance exactly $2$ from all four other points.
The set $S$ has circumradius $2$ and circumcenter $v_5\in\op{conv}(S)$.  
	We have with $M=\sqrt8$:
	$$0 < \chi(M^2,M^2,M^2,y^2,M^2,M^2)= M^2 y^2 (2M^2 - y^2).$$
Hence $|v_1-v_4|=y < 4$.
Let $r(y)$ be the circumradius of a simplex with edges $\{M,M,M,y,M,M)$.
Then we have the contradiction
	$$2 = r(|v_1-v_4|) < r(\sqrt2) = 2.$$
\swallowed\end{proved}
\end{tarski}
%<<<<<





%>>>>>
\begin{tarski}
\name{anc-simplex-not-enc}
\summary{In a packing, if a point of height less than $\sqrt8$
is enclosed over a suitably constrained cone (slice), then it forms an anchor of one of the generators of the cone.}
%\endnote{Five points, nothing is enclosed over a slice. tarski:anc-simplex-not-enc}
\tag{pt5, cone, packing, inside, anchor, cm4-delta, cm5-E}
\rating{120}
\guid{EIYPZVL}

\begin{lemma}\tlabel{tarski:anc-simplex-not-enc}\usage{DCG-[11.5] page 115}
% This states that there is nothing enclosed over a slice.
Let $S=\{w,v_0,v,v_1,v_2\}$ be a packing of five points.
Suppose that
  $$\begin{array}{rlrlrlrl}
    |v-v_1|&\le 2.51, &
    |v-v_2|&\le 2.51, &
    |v_0-v_1|&\le 2.51,&
    |v_0-v_2|&\le 2.51,\\
    && |v_0-v|&\le \sqrt8,&
    &&2.51\le|v_0-w|&< \sqrt8,\\
  \end{array}
  $$
Assume that $w\in\op{cone}(v_0,\{v_1,v_2,v\})$. 
Then $|w-v|\le 2.51$.
\end{lemma}

\begin{proved}  By Lemma~\tref{tarski:point-in-simplex}, 
$w\not\in\op{conv}\{v_0,v_1,v_2,v\}$.  
If $|v_1-v_2|\le \sqrt{8}$, the result follows from
Lemma~\tref{tarski:pass-makes-quarter}. 

Assume that a figure exists with $|v_1-v_2|>\sqrt{8}$. Suppose for a
contradiction that $|v-w|>2.51$.    Pivot $v_1$ around $\{v_0,v_2\}$ until
$|v-v_1|=2.51$ and $v_2$ around $\{v_0,v_1\}$ until $|v-v_2|=2.51$.  Rescale
$w$ so that $|w-v_0|=\sqrt{8}$. Set $x = |v_1-v_2|$. If, through constraint
preserving deformations, $w$ is not deformed into the plane of 
$\{v_0,v_2,v_1\}$,
then we are left with the one-dimensional family $|w'-v_0|=|w'-w|=2$, for
$w'=v_2,v_1$, $|v-w|=|v-v_0|=|v_1-v|=|v_2-v|=2.51$, depending on  $x$. This
gives a contradiction
    $$
    \begin{array}{lll}
        \pi &\ge \dih_V(\{v_2,v_1\},\{v_0,v\}) + \dih_V(\{v_2,v_1\},\{v,w\})\\
        &= 2\dih(x,2,2.51,2.51,2.51,2)
         > \pi,
    \end{array}
    $$
for $x>\sqrt{8}$.
(Equality is attained if $x=\sqrt{8}$.)

Thus, we may assume that $w$ lies in the plane $P=\op{aff}\{v_0,v_1,v_2\}$. Take the
circle in $P$ at distance $2.51$ from $v$. The vertices $v_0$ and $w$ lie
on or outside the circle. The vertices $v_1$ and $v_2$ lie on the
circle, so the diameter is at least $x>\sqrt{8}$.  The distance from
$v$ to $P$ is less than $x_0= \sqrt{2.51^2-2}$.  The edge 
$\op{conv}\{v_0,w\}$ cannot
contain the center of the circle, because $|w-v_0|$ is less than the
diameter.
%
Reflect $v$ through $P$ to get $v'$.  Then $|v-v'|< 2x_0$. Swapping
$v_1$ and $v_2$ as necessary, we may assume that 
  $$w\in\op{cone}(v_0,\{v,v',v_2\}).$$  
The desired bound $|v-w|\le 2.51$ now follows from
  $$
  \Delta(2x,2.51,2.51,\sqrt8,2.51,2.51)>0,
  $$
for $0\le x< x_0$ and
  $$
  E(2,2.51,2.51,2x_0,2.51,2.51,2,2.51,2.51; \sqrt8).
  $$
%%Lemma~\tref{tarski:x0}.
\swallowed\end{proved}
\end{tarski}
%<<<<<







%>>>>>
\begin{tarski}
\name{old372}
\summary{This is a calculation to justify truncation at $\sqrt2$.  This is deprecated.}
\tag{pt5, packing, cone, inside, t0, sqrt2, cm5-E, deprecated}
\rating{0}
\guid{VZETXZC}

\begin{lemma}\tlabel{tarski:old372}
Let $S=\{w,v_1,v_2,v_3,v_4\}$ be a packing of five points.
Assume that
  $$
  \begin{array}{rlrlrlrlrllll}
  |v_2-v_3|&\le 2.51,& |v_3-v_4|&\le 2.51,  &|v_2-v_4|&\le 2.51,\\
  |v_1-v_2|\ge 2.51,\quad &|v_1-v_3|\ge 2.51,&|v_1-v_4|&\le 2.51\\
  |v_1-w|&\le \sqrt8,&|v_1-v_2|&\le \sqrt8,&|v_1-v_3|&\le\sqrt8.\\
  \end{array}
$$
Then $w\not\in \op{cone}(v_1,\{v_2,v_3,v_4\})$.
\end{lemma}

\begin{proved}
By Lemma~\tref{tarski:v-interior}, we have $w\not\in\op{conv}\{v_1,v_2,v_3,v_4\}$.
We may assume without loss of generality that
$\op{conv}\{v_1,w\}$ meets $\op{conv}\{v_2,v_3,v_4\}$.  The result
now follows from
    $$
    E(2,2.51,2.51,2.51,2.51,2.51,2,2,2;  \sqrt{8}).
    $$
\swallowed\end{proved}
\end{tarski}
%<<<<<







%>>>>>
\begin{tarski}
\section{Constructed point}
\name{mk-point}
\summary{This constructs the fourth vertex of a simplex, given specifications of its distances from the three vertices of a given triangle.  It is determined when that simplex degenerates to a planar arrangement.}
\tag{pt4, cm4-delta, cm3-ups}
\rating{100}
\guid{OFGJQUS}

\begin{lemma}\tlabel{tarski:mk-point}
Let $S=\{v_1,v_2,v_3,v_4\}$ be a set of four points
in $\ring{R}^3$.  Assume that the points are not coplanar.
Let $x_{ij} = |v_i-v_j|^2$, for $1\le i < j\le 3$.
Let $a_{01},a_{02},a_{03}$ be three positive real numbers such that
$$
  \Delta(a_{01},a_{02},a_{03},x_{23},x_{13},x_{12})\ge0.
$$
Then there exists a unique point $v_0\in\op{aff}^+(\{v_1,v_2,v_3\},v_4)$
such that
   $|v_0-v_i|^2 = a_{0i}$, for $i=1,2,3$.
Furthermore, $v_0\in\op{aff}\{v_1,v_2,v_3\}$ if and only if 
 $$
  \Delta(a_{01},a_{02},a_{03},x_{23},x_{13},x_{12})=0.
$$
\end{lemma}

\begin{proved} 
%If we pick coordinates of the form $v_1=(0,0,0)$,
%$v_2=(*,0,0)$, $v_3=(*,*,0)$, then we have a quadratic equation to
%solve for $v_0$.  It gives solutions of the form
%   $$
%   v_0 = (*,*,\pm\sqrt{\Delta/\ups}),
%   $$
%where $\ups = \ups(x_{23},x_{13},x_{12})$.  The assumption that the points
%are not coplanar, implies that $v_2,v_3,v_4$ are not collinear, and that
%$\ups>0$.  We pick the sign of the square root, so that it lands in the
%same half-space as $v_4$. 
$$
\Delta(a_{01}-h,a_{02}-h,a_{03}-h,x_{23},x_{13},x_{12})=
\Delta(a_{01},a_{02},a_{03},x_{23},x_{13},x_{12})-h\ups(x_{23},x_{13},x_{12}).
$$
The coefficent of $h$ is positive by Lemma~\ref{tarski:cayley-menger-pos} and \ref{tarski:ups0}.
So there exists a unique $h\ge 0$ that makes the left-hand side zero.
By Lemma~\ref{tarski:construct-pt-plane} there exists a unique $w\in\op{aff}\{v_1,v_2,v_3\}$ such that
 $$
 |w-v_i|^2 = a_{0i} - h.
 $$
Then a solution $v_0$ is given by
 $$
 v_0 = w + \sqrt{h} n,
 $$
where $n$ is a unit normal to the plane $\op{aff}\{v_1,v_2,v_3\}$ whose sign is chosen
so that $v_0\in\op{aff}_+(\{v_1,v_2,v_3\},v_4)$.  Any other solution can be written in the
form 
  $$
  v_0' = w' + \sqrt{h'} n,
  $$
where $w'\in\op{aff}\{v_1,v_2,v_3\}$.  This forces $h'=h$ and the uniqueness of $w$ gives
$w'=w$, so that $v_0=v_0'$.  Finally, $v_0\in\op{aff}\{v_1,v_2,v_3\}$ exactly when $h=0$,
or equivalently, $\Delta(a_{01},\ldots)=0$.
\swallowed\end{proved}
\end{tarski}
%<<<<<





%>>>>>
\begin{tarski}
\name{rog-exist}
\summary{The vertices of a Rogers simplex exist under general conditions.}
\tag{rogers, pt4, cm4-delta, ups}
\rating{40}
\guid{LFYTDXC}

\begin{lemma}\tlabel{tarski:rog-exist}
Let $S=\{v_1,v_2,v_3,v_4\}$ be a set of four points
in $\ring{R}^3$.  Assume that the points are not coplanar.
Let $x_{ij} = |v_i-v_j|^2$, for $1\le i < j\le 3$.
Let $r\ge \eta(x_{12},x_{23},x_{13})$ 
Then there exists a unique point $v_0\in\op{aff}_+(\{v_1,v_2,v_3\},v_4)$
such that
   $|v_0-v_i| = r$, for $i=1,2,3$.
\end{lemma}

\begin{proved}
$$\Delta(r^2,r^2,r^2,x_{23},x_{13},x_{12})=
   \ups(x_{12},x_{23},x_{13}) (r^2-\eta(x_{12},x_{23},x_{13})^2).$$
The condition on non coplanarity gives $\ups>0$, so
$\Delta\ge0$.  The result now follows from Lemma~\tref{tarski:mk-point}. 
\swallowed\end{proved}

%\begin{remark}\tlabel{tarski:delr3} 
%The relation 
%$\Delta(r^2,r^2,r^2,x_{23},x_{13},x_{12})=\ups (r^2-\eta^2)$
%can be used to derive the formula for the circumradius. In fact,
%by Lemma~\tref{tarski:mk-point}, there exists a unique point
%in the plane with distances $\eta$ from all three vertices.
%\end{remark}
%
\end{tarski}
%<<<<<







%>>>>>
\begin{tarski}
\name{rog-ortho}
\summary{The segment between two points, each equidistant from the three vertices of
a triangle, is orthogonal to the plane of the triangle.}
\tag{pt4, eta, plane, halfspace, circum3, aff3}
\rating{60}
\guid{TIEEBHT}

\begin{lemma}\tlabel{tarski:rog-ortho}
Let $S=\{v_1,v_2,v_3,v_4\}$ be a set of four points
in $\ring{R}^3$.  Assume that $S$ is not not coplanar.
Let $x_{ij} = |v_i-v_j|^2$, for $1\le i < j\le 3$.
Let $r > \eta(x_{12},x_{23},x_{13})$.
Let $p'$ be the unique point in
$\op{aff}^+(\{v_1,v_2,v_3\},v_4)$
such that
   $|p'-v_i| = r$, for $i=1,2,3$, provided by Lemma~\ref{tarski:rog-exist}.
Let $p$ be the circumcenter of $\{v_1,v_2,v_3\}$.  Let 
$u\in \op{aff}\{v_1,v_2,v_3\}$.  Then
  $(p'-p)\cdot (u-p)=0$.
\end{lemma}

\begin{proved}  Let $P_i=\op{bis}\{v_1,v_i\}$ 
be the bisector
of $\{v_1,v_i\}$.  Then $p,p'$ lie on these planes and
  $$(p'-p)\cdot (v_1-v_i)=0,\quad i=2,3.$$
Every point $u$ in $\op{aff}\{v_1,v_2,v_3\}$
has the form $u=p + t_2(v_1-v_2) + t_3(v_1-v_3)$, so the result
follows.
\swallowed\end{proved}
\end{tarski}
%<<<<<








%>>>>>
\begin{tarski}
\section{Barycentric coordinate}
\tlabel{tarski:sec:pc3}
\indy{Index}{barycentric coordinates}
This section is nearly identical to Section~\tref{tarski:sec:pc2}, which treats one fewer point.

\name{bary4-exists}
\summary{This lemma gives the unique existence of barycentric coordinates.}
\tag{pt5, bary4, plane}
\rating{60}
\guid{ECSEVNC}

\begin{lemma}\tlabel{bary4-exists}
Let $S=\{v_1,\ldots,v_5\}$ be
a set of five points in $\ring{R}^3$.  Suppose
that $\{v_1,\ldots,v_4\}$ is not a coplanar
set.    Then there exist unique real numbers
$t_1,\ldots,t_4$ such that $\sum_i t_i = 1$ and
	$$v_5 = \sum_i t_i v_i.$$
\end{lemma}

\begin{proved}  This is a linear system with
three unknowns:
	$$(v_5- v_4) = t_1 (v_1-v_4) +
		t_2 (v_2-v_4) + t_3 (v_3-v_4).
	$$
The
square of the
determinant of the system is
$\Delta(x_{ij}) >0$, where $x_{ij}=|v_i-v_j|^2$.
Hence it has a unique solution.
\swallowed\end{proved}
\end{tarski}
%<<<<<



%>>>>>
\begin{tarski}
\name{def:coef3}
\summary{A function can be defined giving the barycentric coordinates of a point in space.}
\tag{def, pt5, bary4, plane }
\rating{0}
\guid{DHIMZUM}

\begin{definition}[coef]\tlabel{def:coef3}
\indy{Index}{coef}
Let $\op{coef}_i(v_1,\ldots,v_5)$
be the constant $t_i$ from Lemma~\ref{bary4-exists}.
\end{definition}
\end{tarski}
%<<<<<




%>>>>>
\begin{tarski}
\name{coeff-sign}
\summary{The sign of a barycentric coordinate determines which half-space it lies in.}
\tag{pt5, plane, bary4, halfspace}
\rating{80}
\guid{SRGTIHY}

\begin{lemma}\tlabel{tarski:coeff-sign}
Let $S=\{v_1,\ldots,v_5\}$ be
a set of five points in $\ring{R}^3$.  Suppose
that $\{v_1,\ldots,v_4\}$ is not a coplanar
set.  Set $S_i=S\setminus \{v_i,v_5\}$. Then the sign of 
$c_i=\op{coef}_i(v_1,\ldots,v_5)$ is positive,
zero, or negative according to whether $v_5$
and $v_i$ are in the same half-space, $v_5$
is in the separating plane, or $v_5$ and $v_i$
are in opposite half-spaces of the plane
through the three points $S_i=S\setminus\{v_i,v_5\}$.
That is,
  $$
  \begin{array}{lll}
   c_i > 0 &\Leftrightarrow   &v_5\in\op{aff}_+^0(S_i,\{v_i\})\\
   c_i = 0 &\Leftrightarrow   &v_5\in\op{aff}(S_i)\\
   c_i < 0 &\Leftrightarrow   &v_5\in\op{aff}_-^0(S_i,\{v_i\})\\
    \end{array}
  $$
\end{lemma}

\begin{proved}
If $\op{coef}_i(v_1,\ldots,v_5)=0$, then $v_5$
has the form
	$$
	v_5 = t_1 v_1 + \cdots + t_4 v_4,
	$$
with $t_i=0$.  These are
precisely the points in the given plane.
Now if $t_i$ is not assumed to be positive, 
elements of
$\op{conv}\{v_5,v_i\}$ have the form
	$$s v_5 + (1-s) v_i,\quad 0\le s \le 1.$$
The coefficient of $v_i$ in this expression
is $1-s(1-t_i)>0$.  Thus, $\op{conv}\{v_5,v_i\}$
does not meet the plane, and $v_i$ and $v_5$
lie in the same half-space.  Similarly,
if $t_i<0$, then $1-s(1-t_i)=0$ has a solution
in $s\in[0,1]$, so that the plane separates the
points $v_i$ and $v_5$.
\swallowed\end{proved}
\end{tarski}
%<<<<<




%>>>>>
\begin{tarski}
\name{bary4-convex}
\summary{The defining conditions of a convex hull can be expressed in barycentric coordinates.}
\tag{bary4, conv4, plane}
\rating{40}
\guid{ZRFMKPY}

\begin{lemma}
Let $S=\{v_1,\ldots,v_5\}$ be
a set of five points in $\ring{R}^3$.  Let $S'=S\setminus\{v_5\}$.
Suppose
that $\{v_1,\ldots,v_5\}$ is not a coplanar
set.  Then $v_5\in\op{conv}(S')$ if and only
if  
$\op{coef}_i(v_1,\ldots,v_5)\ge0$ 
for $i=1,2,3,4$.
Similarly, $v_5\in\op{conv}^0(S')$ if and only
if  
$\op{coef}_i(v_1,\ldots,v_5)>0$ 
for $i=1,2,3,4$.
\end{lemma}

\begin{proved}  This is trivial by the definitions
of $\op{conv}$, $\op{conv}^0$ and $\op{coef}$.
\swallowed\end{proved}
\end{tarski}
%<<<<<














%>>>>>
\begin{tarski}
% Used in Fine Decomposition BigD analysis.
\section{Barycentric application}
\name{consec-anchors}
\summary{In a packing, under suitable distance constraints, a point lies in a lune.  This can force two anchors to be consecutive.}
%\endnote{Five points, conditions on when two points $v_2$ $v_4$ are consecutive in the cyclic order around an upright diagonal $v_0,v_1$. tarski:consec-anchors}
\tag{pt5, packing, t0, sqrt2, lune, inside}
\rating{100}
\guid{COEBRMF}

\begin{lemma}\tlabel{tarski:consec-anchors}
Let $S=\{v_0,v_1,v_2,v_3,v_4\}$ be a packing of five points.
Assume that
  $$2\le |v_1-v|\le 2.51,\quad \text{ for } v = v_2,v_3,v_4.$$
Assume that
  $$2.51 < |v_1-v_0| < \sqrt8,\quad |v_3-v_0|\le 2.51,
   \quad |v_2-v_4|\le2.51.$$
Assume that $|v_0-v_2|,|v_0-v_4|<\sqrt8$ and that
at most one of these two lengths is greater than $2.51$.
Then $v_3\not\in\op{aff}_+(\{v_0,v_1\},\{v_2,v_4\})$.
\end{lemma}



\begin{proved}
\FIXX{The references to Lemma~\ref{tarski:convex-quad}, are to the
right lemma but wrong line of the lemma.  The lines of this lemma should
be displayed in the proof (globally).}
The constraints on lengths imply that $\{v_0,v_1,v_2,v_4\}$
is not planar.  We use barycentric 
coordinates to write
  $$
  v_3 = t_0 v_0 + t_1 v_1 + t_2 v_2 + t_4 v_4, \text{ with } t_0+t_1+t_2+t_3=1.
  $$
Assume for a contradiction, that $t_2\ge0$, $t_4\ge0$.
We consider four cases, depending on the signs of $t_0,t_1$.

If $t_0,t_1\ge0$, then $v_3\in\op{conv}\{v_0,v_1,v_2,v_4\}$.
This is contrary to Lemma~\tref{tarski:v-interior}.
If $t_1\ge0,t_0\le0$, then $\op{conv}\{v_0,v_3\}$ meets
$\op{conv}\{v_1,v_2,v_4\}$.  
This is contrary to Lemma~\tref{tarski:convex-quad}.
If $t_0\ge0,t_1\le0$, then $\op{conv}\{v_1,v_3\}$ meets
$\op{conv}\{v_0,v_2,v_4\}$.  
This is again contrary to Lemma~\tref{tarski:convex-quad}.  Finally, if
$t_0\le0,t_1\le0$, then $\op{conv}\{v_0,v_1,v_3\}$ meets 
$\op{conv}\{v_2,v_4\}$.  
Use Lemma~\tref{tarski:convex-quad} again.
\swallowed\end{proved}
\end{tarski}
%<<<<<





%>>>>>
\begin{tarski}
\name{over-under}
\summary{If two blades meet, then the generating segment of one meets the other blade.}
\tag{pt5, blade, aff-meet, conv2}
\rating{40}
\guid{JVDAFRS}

\begin{lemma}\tlabel{tarski:over-under}
Let $\{v_0,v_1,v_2,v_3,v_4\}$ be a set of five points in $\ring{R}^3$.
Assume that $\op{cone}(v_0,\{v_1,v_3\})$ meets
  $\op{cone}(v_0,\{v_2,v_4\})$.  Then either
$\op{conv}\{v_1,v_3\}$ meets $\op{cone}(v_0,\{v_2,v_4\})$ or
$\op{conv}\{v_2,v_4\}$ meets $\op{cone}(v_0,\{v_1,v_3\})$.
\end{lemma}

\begin{proved} 
If $x$ is a point of intersection,
  $$
  x = s_0 v_0 + s_1 v_1 + s_3 v_ 3 = t_0 v_0 + t_2 v_2 + t_4 v_4.
  $$
The two cases of the conclusion are precisely the two cases
  $t_0\ge s_0$ and $s_0\ge t_0$.
\swallowed\end{proved}
\end{tarski}
%<<<<<





%>>>>>
\begin{tarski}
\section{Polyhedron vs. polytope}
%% Needed for volume calcs that present simplex by constraints.
The following lemma reconciles the polyhedral view of a simplex
(as an intersection of half-planes), with the polytope view (as
the convex hull of extreme points.

\name{hedra-tope}
\summary{In general, the convex hull of four points is the intersection of the four half-spaces determined by the four faces (closed case).}
\tag{pt4, plane, halfspace, conv4}
\rating{40}
\guid{ARIKWRQ}

\begin{lemma} \tlabel{tarski:hedra-tope}
Let $S=\{v_1,\ldots, v_4\}$ be a set of four points
in $\ring{R}^3$.  Suppose that $S$ is not coplanar.
Let $A_i = \op{aff}_+(S\setminus\{v_i\},v_i)$.
Then $$\op{conv}(S)  = A_1\cap A_2\cap A_3\cap A_4.$$
%
\end{lemma}

\begin{proved}
Both consist of points $x\in\ring{R}^3$ such that
  $$x = t_1 v_1 +\cdots+ t_4 v_4,\quad t_i\ge 0,\quad t_1+\cdots +t_4 =1.
  $$
\swallowed\end{proved}
\end{tarski}
%<<<<<



%>>>>>
\begin{tarski}
\name{conv-halfspace}
\summary{In general, the convex hull of four point is the intersection of the four halfspaces determined by the four faces (open case).}
\tag{pt4, plane, conv4, halfspace}
\rating{40}
\guid{MXHKOXR}

\begin{lemma}
Let $S=\{v_1,\ldots, v_4\}$ be a set of four points
in $\ring{R}^3$.  Suppose that $S$ is not coplanar.
Let $A^0_i = \op{aff}^0_+(S\setminus\{v_i\},v_i)$.
Then $$\op{conv}^0(S)  = A^0_1 \cap A^0_2 \cap A^0_3 \cap A^0_4.$$
\end{lemma}

\begin{proved}  Both consist of points $x\in\ring{R}^3$ such that
  $$x = t_1 v_1 +\cdots+ t_4 v_4,\quad t_i> 0,\quad t_1+\cdots +t_4 =1.
  $$
\swallowed\end{proved}
\end{tarski}
%<<<<<






%>>>>>
\begin{tarski}
\section{Rogers's lemma}
\name{rog-lemma}
\summary{This is the polynomial inequality in Rogers's celebrated lemma of 1958, giving the monotonicity of the density on a Rogers simplex.}
\tag{rogers, pt4, poly-ineq}
\rating{40}
\guid{TYUNJLA}

\begin{lemma}\tlabel{tarski:rog-lemma}
Let $e_1$, $e_2$, and $e_3$ be the standard basis of
$\ring{R}^3$.  Let  $a,b,c$ and $a',b',c'$
are real numbers that satisfy $0 <a \le b \le c$, $0 \le a'\le b'\le c'$,
$a \le a'$, $b \le b'$, $c \le c'$. 
Let $t_1,t_2,t_3>0$ and $t_1+t_2+t_3< 1$.  Let
   $$
   \begin{array}{lll}
   v &= (t_1+t_2+t_3) a e_1 + (t_2+t_3) \sqrt{b^2-a^2} e_2 + t_3
   \sqrt{c^2-b^2} e_3\\
   v' &= (t_1+t_2+t_3) a' e_1 + (t_2+t_3) \sqrt{b'^2-a'^2} e_2 + t_3
   \sqrt{c'^2-b'^2} e_3\\
    \end{array}
    $$
    Then $|v| \le |v'|$.
\end{lemma}

\begin{proved}
  We calculate
  $$
  |v'|^2-|v|^2 = (t_1+2t_2+2t_3)(a'^2-a^2) + t_2 (t_2+2t_3)(b'^2-b^2)
    +t_3^2 (c'^2-c^2)\ge0.
  $$
\swallowed\end{proved}
\end{tarski}
%<<<<<






%>>>>>
\begin{tarski}
\section{Circumradius}
\name{rho-sign}
\summary{The polynomial rho is non-negative when evaluated on the distance set of four points.}
\tag{rho, pt4}
\rating{20}
\guid{SHOGYBS}

\begin{lemma}\tlabel{tarski:rho-sign}
Let $S=\{v_1,v_2,v_3,v_4\}$ be
a set of four points in $\ring{R}^3$.  Let
$x_{ij}=|v_i-v_j|^2$.  Then
	$$\rho(x_{ij})\ge 0.$$
\end{lemma}

\begin{proved} By Lemma~\tref{tarski:rho-ups},
$\rho$ is the sum of two non-negative terms, hence
non-negative.
\swallowed\end{proved}
\end{tarski}
%<<<<<





%>>>>>
\begin{tarski}
\name{circumcenter}
\summary{The barycentric coordinates of the circumcenter of four points is computed in terms of $\chi$ and $\Delta$.}
\tag{pt4, cm4-delta, plane, circum4, chi, bary4}
\rating{80}
\guid{VBVYGGT}

\begin{lemma}\tlabel{tarski:circumcenter}
Let $S=\{v_1,v_2,v_3,v_4\}$ be a set of four points in $\ring{R}^3$.
Assume that $S$ is not a coplanar set.  
Then there exists a unique circumcenter $p$
of $S$.  Set $x_{ij} = |v_i-v_j|^2$.
Then 
    $$
    p = (\chi_1 v_1 + \chi_2 v_2 + \chi_3 v_3 + \chi_4
    v_4)/(2\Delta(x_{ij})),
    $$
where
    $$
    \begin{array}{lll}
    \chi_1 &= \chi(x_{12},x_{13},x_{14},x_{34},x_{24},x_{23})\\
    \chi_2 &= \chi(x_{12},x_{24},x_{23},x_{34},x_{13},x_{14})\\
    \chi_3 &= \chi(x_{34},x_{13},x_{23},x_{12},x_{24},x_{14})\\
    \chi_4 &= \chi(x_{34},x_{24},x_{14},x_{12},x_{13},x_{23}).
    \end{array}
    $$
\end{lemma}


\begin{proved}
Note that the last three arguments of $\chi_k$ 
contain the variables $x_{ij}$ whose
indices $ij$ satisfy $i\ne k\ne j$.
Under the assumption that $S$ does not lie
in a plane, we have $\Delta(x_{ij})>0$ (by Lemmas~\ref{tarski:cayley-menger-pos} and \ref{tarski:delta0}). 
We compute directly that $p$ given by this
formula has the same distance from each point
$v_i$.  If $p'$ is a second circumradius, then in 
barycentric coordinates we have that
  $$p'-p = \sum_i t_i v_i.$$
The condition that $p$ and $p'$ have the same distance to each $v_i$
gives
  $$(p'-p)\cdot v_i=0.$$
Then $$|p'-p|^2 = \sum_it_i (p'-p)\cdot v_i = 0.$$  
Thus, $p'=p$.
\swallowed\end{proved}
\end{tarski}
%<<<<<





%>>>>>
\begin{tarski}
\name{rad-rho}
\summary{There is a formula for the circumradius of four points in terms of the polynomials $\rho$ and $\Delta$.}
\tag{pt4, rho, cm4-delta, plane}
\rating{60}
\guid{GDRQXLG}

\begin{lemma}
Let $S=\{v_1,v_2,v_3,v_4\}$ be a set of four points in $\ring{R}^3$.
Assume that $S$ is not a coplanar set.  
Let $x_{ij}=|v_i-v_j|^2$.
Then the circumradius of
	$S$ is equal to 
		$$
		\sqrt{\rho(x_{ij})}/
		(2\sqrt{\Delta(x_{ij})})
		$$
\end{lemma}

\begin{proved} It is enough to compare the
squares of both sides.  (We know the sign of
$\rho$ by Lemma~\tref{tarski:rho-sign}.)  We
compute $|p-v_1|^2$ from the formula
in Lemma~\tref{tarski:circumcenter}.
\swallowed\end{proved}
\end{tarski}
%<<<<<




%>>>>>
\begin{tarski}
\name{eta-rad}
\summary{The circumradius of a simplex is at least the circumradius of any of its faces.}
\tag{pt4, rad, eta, ups, cm4-delta}
\rating{50}
\guid{ZJEWPAP}

\begin{lemma}\tlabel{tarski:eta-rad}
Let $S=\{v_0,v_1,v_2,v_3\}$ be a set of four points in $\ring{R}^3$.
Assume that $S$ is not coplanar.
Then $\rad_V(S) \ge \eta_V(v_0,v_1,v_2)$.
\end{lemma}

\begin{proved}  Let $r=\rad_V(S)$.  Let $p$ be the circumcenter of $S$.
Lemma~\ref{tarski:cayley-menger-pos}, applied to the set $\{p,v_1,v_2,v_3\}$, gives
  $$
  0\le \Delta(r^2,r^2,r^2,x_{12},x_{23},x_{13}).
  $$
Computing the right-hand-side, we find that
$$
\Delta = \ups(x_{12},x_{23},x_{13}) (r^2 - \eta(x_{12},x_{23},x_{13})^2).
$$
The set $\{v_1,v_2,v_3\}$ is not coplanar, so that $\ups(x_{12},x_{23},x_{13})>0$
(by Lemmas~\ref{tarski:cayley-menger-pos} and \ref{tarski:ups0}).
The result follows.
\swallowed\end{proved}
\end{tarski}
%<<<<<








%>>>>>
\begin{tarski}
\section{Orientation}
\usage{DCG-[5.2]{ sec:orientation}}

\name{chi}
\summary{The negative orientation of a vertex in a simplex is equivalent to a negative value of $\chi$.}
\tag{pt4, chi, circum4}
\rating{40}
\guid{VSMPQYO}

\begin{lemma} \tlabel{tarski:chi}\usage{DCG-[5.15]{ lemma:chi}}
Let $S=\{v_1,\ldots,v_4\}$ be a set of four points
in $\ring{R}^3$.  
Let $x_{ij}=|v_i-v_j|^2$, for $1\le i< j\le 4$.
The point $v_1$ has negative
orientation with respect to $S$ if and only if
    $\chi(x_{12},x_{13},x_{14},x_{34},x_{24},
    x_{23})<0$.
\end{lemma}

\begin{proved} 
This follows immediately from  Lemma~\tref{tarski:circumcenter} and 
Lemma~\tref{tarski:coeff-sign}.
\swallowed\end{proved}
\end{tarski}
%<<<<<



%>>>>>
\begin{tarski}
\name{orientation-gt0}
\summary{The circumcenter lies in the convex hull of four points, when all orientations are positive.}
\tag{pt4, circum4, conv4, chi}
\rating{40}
\guid{PVLJZLA}

\begin{lemma}
The circumenter of $\{v_1,\ldots,v_4\}$ lies in
$\op{conv}^0\{v_1,\ldots,v_4\}$ if all four
orientations are positive.
\end{lemma}

\begin{proved}  This again follows from
Lemma~\tref{tarski:circumcenter} and 
Lemma~\tref{tarski:coeff-sign}.
\swallowed\end{proved}
\end{tarski}
%<<<<<






%>>>>>
\begin{tarski}
\name{at-most-one-negative}
\summary{In a suitably constrained simplex of a packing, at most one vertex has negative orientation.}
\tag{pt4, sqrt2, chi}
\rating{80}
\guid{IAALCFJ}

\begin{lemma} \tlabel{tarski:at-most-one-negative}
%proclaim{Lemma 2.1}
Let $S=\{v_1,v_2,v_3,v_4\}$ be a packing of four points.
   $$
   |v_i-v_j| < \sqrt8,
   $$
for every $i\ne j$.
Assume that $v_1$ has negative orientation in $S$.  Then
$v_2,v_3,v_4$ have non-negative orientation in $S$.
%DCG: At most one face of a quarter $Q$ has negative orientation.
\end{lemma}


\begin{proved}
The proof applies to any simplex with nonobtuse faces. 
Fix an edge and project $S$ orthogonally
to a triangle in a plane perpendicular to that edge. The faces
$F_1$ and $F_2$ of $S$ along the edge project to edges $e_1$ and
$e_2$ of the triangular projection of $S$. The line equidistant
from the three vertices of $F_i$ projects to a line perpendicular
to $e_i$, for $i=1,2$. These two perpendiculars intersect at the
projection of the circumcenter of $S$.  If the faces of $S$ are
nonobtuse, the perpendiculars pass through the segments $e_1$ and
$e_2$ respectively; and the two vertices opposite $F_1$ and $F_2$ cannot both
be negatively oriented.
\swallowed\end{proved}
\end{tarski}
%<<<<<





%>>>>>
\begin{tarski}
\name{neg-orient-quad}
\summary{If a point has negative orientation with respect to an anchor, then a quarter is formed.}
%\endnote{Four points, A negative orientation on a simplex, forces edges $\le 2.51$.  Quarter case.  tarski:neg-orient-quad}
\tag{pt4, packing, t0, sqrt2, chi, anchor}
\rating{80}
\guid{YOEEQPC}

\begin{lemma}\tlabel{tarski:neg-orient-quad}\usage{DCG-[5.16]{ lemma:neg-orient-quad}}
%\proclaim{Lemma 2.2}
Let $S=\{v_1,v_2,v_3,v_4\}$ be a packing of four points.
%Let $F=\{v_2,v_3,v_4\}$.
%Assume that one edge between
%pairs of vertices in $F$
%has length between $2.51$ and $\sqrt8$ and that
%the other two edges have length at most $2.51$.  
Assume
 $$
 2.51\le |v_2-v_4|\le \sqrt8,\quad |v_2-v_3|\le 2.51,\quad |v_3-v_4|\le 2.51.
 $$
Assume that the $v_1$ has negative orientation
along $S$. Then
   $$
   |v_1-v_i|\le 2.51, \quad i=2,3,4.
   $$
\end{lemma}

\begin{proved}  
Assume for a contradiciton that $|v_1-v_i|> 2.51$ for some $i$.
The orientation of $v_1$ is determined by the sign of the function
$\chi$ (see Lemma~\tref{tarski:chi}). 
Let $x_{ij}=|v_i-v_j|^2$.  
Note that $\partial\chi/\partial x_{12} = x_{34}
(-x_{34}+x_{24}+x_{23})$.  By our
constraints on the lengths of edges, 
$-x_{34}+x_{24}+x_{23}\ge0$. Thus, we have monotonicity in the variable $x_{12}$,
and the same is true of $x_{13}$, and $x_{14}$. Also, $\chi$ is
quadratic with negative leading coefficient in each of the
variables $x_{34}$, $x_{24}$, $x_{23}$. Thus, to check positivity, when any
of the lengths is greater than $2.51$, it is enough to evaluate
$$\chi(2^2,2^2,2.51^2,x^2,y^2,z^2), \quad
\chi(2^2,2.51^2,2^2,x^2,y^2,z^2),  \quad
\chi(2.51^2,2^2,2^2,x^2,y^2,z^2),$$ for $x\in\{2,2.51\}$,
$y\in\{2,2.51\}$, and $z\in\{2.51,\sqr8\}$, and verify that these
values are nonnegative. (The minimum is  $0$.)
\swallowed\end{proved}
\end{tarski}
%<<<<<




%>>>>>
\begin{tarski}
\name{neg-orient-tet}
\summary{In a packing, if a point has negative orientation with respect to a quasi-regular face, then a quasi-regular tetrahedron is formed.}
\tag{pt4, chi, packing, t0}
\rating{60}
\guid{YJTLEGD}

\begin{lemma}\tlabel{tarski:neg-orient-tet}\usage{DCG-[5.17]{ lemma:neg-orient-tet}} Let $S=\{v_0,v_1,v_2,v_3\}$ be a
packing of four points.  Assume that 
  $$
  |v_i-v_j|\le 2.51,\quad 1\le i < j \le 3,
  %\quad |v_0-v_k|,\quad 1\le k\le 3.
  $$
 If $v_0$
has negative orientation along
$S$, then
  $$
  2 \le |v_0-v_i | < 2.51, \text{ for } i=1,2,3.
  $$
\end{lemma}

\begin{proved} The proof is similar to the proof of
Lemma~\tref{tarski:neg-orient-quad}. It comes down to checking that
    $$
    \chi(2^2,2^2,2.51^2,x^2,y^2,z^2)>0,$$
    for $x,y,z\in[2,2.51]$.
\swallowed\end{proved}
\end{tarski}
%<<<<<




%>>>>>
\begin{tarski}
\name{sqrt2-chi-plus}
\summary{If a face of a simplex has circumradius less than $\sqrt2$, then the orientation of the opposite vertex is positive.}
\tag{pt4, sqrt2, packing, circum3, chi, eta}
\rating{50}
\guid{NJBVVWG}

\begin{lemma}\tlabel{tarski:sqrt2-chi-plus}\usage{DCG-[5.18]{lemma:sqrt2-chi+}}
%
Let $S=\{v_1,v_2,v_3,v_4\}$ be a packing of four
points.
If $S\setminus\{v_i\}$
has circumradius less than $\sqrt2$, then
the orientation of $v_i$ is positive in $S$.
\end{lemma}

\begin{proved}
Let $y_{ij}=|v_i-v_j|$ and $x_{ij} = y_{ij}^2$.
If the face has circumradius less than $\sqrt2$, by monotonicity (as in Lemma~\ref{tarski:neg-orient-quad}),
    $$\chi(x_{12},\ldots,x_{23}) \ge \chi(4,4,4,x_{34},x_{24},x_{23})
    = 2 x_{34}x_{24}x_{23}
    (2/\eta(y_{34},y_{24},y_{23})^2 - 1) >0.$$
\swallowed\end{proved}
\end{tarski}
%<<<<<





% Lemma for GLQHFGH.
%>>>>>
\begin{tarski}
\name{bounded-tri}
\summary{If the intersection of three half-planes is nonempty and bounded, then that intersection is the convex hull of three points.}
\tag{pt3, halfplane, aff3, conv3}
\rating{110}
\guid{VUYPDCF}

\begin{lemma}\tlabel{tarski:bounded-tri}
Let $\{v_1,v_2,v_3\}$ be a set of three points
in $\ring{R}^3$ that is not collinear.  Let $H_1,H_2,H_3$ be three open half-planes in
$\op{aff}\{v_1,v_2,v_3\}$, with bounding line $L_i$ of $H_i$.  
Suppose that the intersection
   $$H_1\cap H_2\cap H_3$$
is nonempty and bounded.  Let $w_i\in L_j\cap L_k$.  Then
$w_i\in H_i$ and
   $$H_1\cap H_2\cap H_3 = \op{conv}^0\{w_1,w_2,w_3\}.$$
\end{lemma}

\begin{proved}  If some $w_i=w_j$, then all three lines meet
In this case, since the half-planes
are open, $H_1\cap H_2\cap H_3$ is empty or unbounded, which is contrary to 
hypothesis.  Now assume that $w_i$ are all distinct.

If $\{w_1,w_2,w_3\}$ is collinear, then we have $L_1=L_2=L_3$.
In this case, $H_1\cap H_2\cap H_3$ is empty or the entire
unbounded set $H_1$.  This is contrary to hypothesis.

Thus, we have $L_i = \op{aff}\{w_j,w_k\}$ and
$H_i = \op{aff}^0_{\sigma(i)}(\{w_j,w_k\},w_i)$, for some choice of
signs $\sigma(i)\in\{\pm\}$.  
Depending on the signs $\sigma(i)$, we find that in barycentric
coordinates with respect to $\{w_1,w_2,w_3\}$, we have
  $$H_1\cap H_2\cap H_3 =
    \{ t_1 w_1 + t_2 w_2 +t_3 w_3 \mid
         t_1 + t_2 + t_3 = 1,\quad  \sigma(i)t_i > 0\}.$$
If two signs differ, then the set is unbounded.  If all signs
are negative, then the set is empty.  So all signs are positive.
\swallowed\end{proved}
\end{tarski}
%<<<<<






% This is used in lemma:voronoi-truncation-over-Q.
%>>>>>
\begin{tarski}
\name{vor-bar}
\summary{Under suitable constraints, if the Voronoi cell at the vertex of a simplex meets the opposite face, then that vertex has negative orientation.}
\tag{pt4, packing, plane, sqrt2, conv3, voronoi, chi}
\rating{110}
\guid{GLQHFGH}

\begin{lemma}\tlabel{tarski:vor-bar}
Let $S=\{v_0,v_1,v_2,v_3\}$ be a packing of four points.
Assume that they are not coplanar.
Assume that 
$$
|v_i-v_j|<\sqrt8,\quad 1\le i < j \le 3.
$$
  Suppose that
$\op{conv}\{v_1,v_2,v_3\}$ meets $\Omega(v_0,S)$.  Then $v_0$ has negative
orientation with respect to $S$.  Similarly, if 
$\op{conv}\{v_1,v_2,v_3\}$ meets $\bar\Omega(v_0,S)$.  Then $v_0$ has nonpositive
orientation with respect to $S$.
\end{lemma}

\begin{proved} Let $R = \op{aff}\{v_1,v_2,v_3\}\cap\Omega(v_0,S)$.
It is not empty.  It is convex, but does not meet the edges
$\op{conv}\{v_i,v_j\}$ of $\op{conv}\{v_1,v_2,v_3\}$.  Therefore it
is bounded.  By Lemma~\tref{tarski:bounded-tri}, we have that
$R = \op{conv}^0\{w_{12},w_{23},w_{13}\}$, where $w_{ij}$ is the point
on $\op{aff}\{v_1,v_2,v_3\}$ that is equidistant from $v_0,v_j,v_k$,
with $j,k$ distinct subscripts.  By an explicit calculation,
the barycentric coordinates of $w_{23}$, we find that
   $$w_{23} = t_1 v_1 + t_2 v_2 + t_3 v_3,$$
%% Done in Mathematica, grep for GLQHFGH.
where $$t_1 = \frac{x_{23} (x_{02} + x_{03} - x_{23})}{\Delta_{(01)}(x_{ij})}.$$
Since $w_{23}\in\op{conv}\{v_1,v_2,v_3\}$, the sign is non-negative.
Also, the numerator is positive, by the edge-length constraints
$4\le x_{ij} < 8$.  Thus, $\Delta_{(01)}(x_{ij}) > 0$.
Again, by Lemma~\tref{tarski:bounded-tri}, $w_{23}$ is on the `positive'
side of the line in $\op{aff}\{v_1,v_2,v_3\}$ formed by the plane
equidistant from $v_0$ and $v_1$.  That is,
   $$|w_{23}-v_0|^2 - |w_{23}-v_1|^2 > 0.$$
Computing this quantity using barycentric coordinates for $w_{23}$
with respect
to $\{v_1,v_2,v_3\}$, we obtain that
   $$|w_{23}-v_1|^2 - |w_{23}-v_0|^2 = 
   -\chi(x_{01},x_{02},x_{03},x_{23},x_{13},x_{12})/{\Delta_{(01)}(x_{ij})} >0.$$
Thus, $\chi < 0$, and the orientation is negative.

The final statement about $\bar\Omega$ is proved similarly.
\swallowed\end{proved}
\end{tarski}
%<<<<<







%>>>>>
\begin{tarski}
\name{vor-bar-sqrt2}
\summary{In a packing, if a Voronoi cell meets the convex hull of three points, then the circumradius at least $\sqrt2$.}
%\endnote{  tarski:vor-bar-sqrt2}
\tag{pt4, packing, voronoi, conv3, eta, sqrt2}
\rating{40}
\guid{SGUCIKJ}

\begin{lemma}\tlabel{tarski:vor-bar-sqrt2}
Let $\{v_1,v_2,v_3,v\}$ be a packing of four points in $\ring{R}^3$.
Assume that they are not collinear.
Let $S=\{v_1,v_2,v_3\}$.
Assume $v$ does not lie in $\op{aff}(S)$.  Suppose that
$\op{conv}(S)$ meets $\Omega(v,S)$.  Then the circumradius
of $S$ is at least $\sqrt2$.
\end{lemma}

\begin{proved} Combine Lemma~\tref{tarski:vor-bar} with
Lemma~\tref{tarski:sqrt2-chi-plus}.
\swallowed\end{proved}
\end{tarski}
%<<<<<



%>>>>>
\begin{tarski}
\name{vor-bar-tet}
\summary{If the Voronoi cell at a vertex meets the opposite face (a qrtet), then the simplex is quasi-regular.}
\tag{pt4, packing, collinear, voronoi, conv3, t0}
\rating{40}
\guid{TCQPONA}

\begin{lemma}\tlabel{tarski:vor-bar-tet}
Let $\{v,v_1,v_2,v_3\}$ be a
packing of four points.  Assume that
  $$
   |v_i-v_j|\le 2.51, \quad\text{ for } i\ne j.
  $$
Assume that $S=\{v_1,v_2,v_3\}$ not collinear. Assume $v$ 
does not lie in $\op{aff}(S)$.  Suppose that
$\op{conv}(S)$ meets $\Omega(v,S)$.
Then
  $$
  2 \le |v-v_i | < 2.51, \text{ for } i=1,2,3.
  $$
\end{lemma}

\begin{proved}
Combine Lemma~\tref{tarski:vor-bar} with Lemma~\tref{tarski:neg-orient-tet}.
\swallowed\end{proved}
\end{tarski}
%<<<<<






%>>>>>
\begin{tarski}
\name{vor-bar-quad}
\summary{If the Voronoi cell at a vertex meets the opposite face (an anchor triangle), then the simplex is a quarter.}
\tag{pt4, packing, t0, sqrt2, voronoi, conv3}
\rating{40}
\guid{CEWWWDQ}

\begin{lemma}\tlabel{tarski:vor-bar-quad}
Let $S=\{v,v_1,v_2,v_3\}$ be a
packing of four points. Suppose
  $$
  |v_1-v_2|\le 2.51,\quad 
  |v_2-v_3|\le 2.51,\quad
  |v_1-v_3|\le \sqrt{8}.
  $$
Suppose that
$\op{conv}(S\setminus \{v\})$ meets $\Omega(v,S)$.
Then
  $$
  2 \le |v-v_i | \le 2.51, \text{ for } i=1,2,3.
  $$
\end{lemma}

\begin{proved}
Combine Lemma~\tref{tarski:vor-bar} with Lemma~\tref{tarski:neg-orient-quad}.
\swallowed\end{proved}
\end{tarski}
%<<<<<






% Extracted from the Separation Lemma of DCG.
%>>>>>
\begin{tarski}
\name{decouple-orient}
\summary{If a suitably constrained segment meets a blade, then it meets the corresponding convex hull of three points.}
\tag{pt4, pt5, blade, conv2, voronoi, conv3}
\rating{160}
\guid{CKLAKTB}

\begin{lemma}\tlabel{tarski:decouple-orient}
Let $S=\{w,v_0,v_1,v_2\}$ be a packing of four points.
Assume that
   $$
   |v_0-v_1| < \sqrt8,\quad |v_0-v_2| < \sqrt8,\quad |v_1-v_2| < \sqrt8.
   $$
Suppose that there exists a point $x\in\ring{R}^3$
such that 
  $\op{conv}\{x,w\}$ meets $\op{cone}(v_0,\{v_1,v_2\})$.  
Assume furthermore that
  $$
  |x-v_0| < |x-v_1|,\quad 
  |x-v_0| < |x-v_2|,\quad
  |x-w| < |x-v_0|;
  $$
that is, $x\in\Omega(v_0,\{v_1,v_2\})\cap \Omega(w,\{v_0\})$.
%Then $w$ has nonpositive orientation with respect to $\{w,v_0,v_1,v_2\}$.
Then $\op{conv}(x,w)$ meets $F=\op{conv}\{v_0,v_1,v_2\}$.
\end{lemma}


\begin{proved} 
Assume for a contradiction that $\op{conv}(x,w)$ meets $\op{cone}$,
but not $F$.  Let $A = \op{aff}\{v_0,v_1,v_2\}$ and $A^-=\op{aff}+-(\{v_0,v_1,v_2\},w)$
be the half-space bounded by $A$ not containing $w$. 
Let $P$
be the plane orthogonal to $A$ containing the line $\op{aff}\{v_1,v_2\}$.
Let $P^+$ be the half-space bounded by $P$ containing $v_0$.  Let
$P^-$ be the complementary half-space.

We note that $\Omega(v_0,\{v_1,v_2\})$ is a subset of $P^+$.  Since
$v_0,v_1,v_2\in A$ and $P$ is orthogonal to $A$, this reduces to
a calculation in in plane $A$.  In the plane $A$ it follows from
the fact that the triangle $\{v_0,v_1,v_2\}$ is not obtuse.  So $x\in P^+$.

If $w\in P^+$, then $\op{conv}\{x,w\}\subset P^+$, and
$P^+\cap \op{cone} \subset F$.  The result follows in this case.
We take $w\in P^-$.  Since $|v_1-v_2|<\sqrt8$, 
this implies that $P^+\cap A^-\cap \Omega(w,\{v_1,v_2\})$
is empty.  This contradicts the assumption that $x$ belongs to this
intersection.
%
%% (This proof is a minor adaptation of \cite[Lemma~2.2]{part2}.)
%Let $F=\{v_0,v_1,v_2\}$ and $C(F) = \op{cone}(v_0,\{v_1,v_2\})$.
%
%Consider the set $X$ containing $x$ and bounded by the planes
%$H_1$ through $\{v_0,v_1,w\}$, $H_2$ through $\{v_0,v_2,w\}$,  $H_3$
%through $\{v_0,v_1,v_2\}$, $H_4 = \{x: x\cdot v_1 = v_1\cdot
%v_1/2\}$, and $H_5 = \{x: x\cdot v_2 = v_2\cdot v_2/2\}$.  The
%planes $H_4$ and $H_5$ contain the 
%bounding faces of $\Omega(v_0,\{v_1,v_2\})$. The plane $H_3$
%contains the triangle $F$. The planes $H_1$ and $H_2$ bound the
%set containing points, such as $x$, that can be connected to $w$
%by a segment that passes through $C(F)$.
%
%We define a half-space $P$ by
%  $$
%  P = \{x \mid |x-w| < |x-v_0|\}.
%  $$ 
%The choice of $w$ implies
%that $X\cap P$ is nonempty. We leave it as an exercise to check
%that $X\cap P$ is bounded.  If the intersection of a bounded
%polyhedron with a half-space is nonempty, then some vertex of the
%polyhedron lies in the half-space.  Thus, some vertex of $X$ lies
%in $P$.
%
%We claim that the vertex of $X$ lying in $P$ cannot lie on $H_1$.
%To see this, pick coordinates $(x_1,x_2)$ on the plane $H_1$ with
%origin $v_0=0$ so that $v_1 = (0,z)$ (with $z>0$) and
% $X\cap H_1\subset X':=\{(x_1,x_2) : x_1\ge 0, \ x_2\le z/2\}$.
%If the quadrant $X'$ meets $P$, then the point $v_1/2$ lies in
%$P$. This is impossible, because every point between $0$ and $v_1$
%lies in the Voronoi cell at $0$ or $v_1$, and not in the Voronoi
%cell of $w$. (Recall that for every point $v_1$ on a barrier at
%the origin, $|v_1|<\sqrt8$.)
%
%Similarly, the vertex of $X$ in $P$ cannot lie on $H_2$.  Thus,
%the vertex must be the unique vertex of $X$ that is not on $H_1$
%or $H_2$, namely, the point of intersection of $H_3$, $H_4$, and
%$H_5$. This point is the circumcenter $c$ of the face $F$.  We
%conclude that the polyhedron $X_0:= X\cap P$ contains $c$. Since
%$c\in X_0$, the point $w$ has nonpositive
%orientation.
%
% 
\swallowed\end{proved}
\end{tarski}
%<<<<<

 



%>>>>>
\begin{tarski}
\name{dcg-page82}
\summary{This result is deprecated.  It was used in DCG (page 82) 
for the $\sqrt2$-based score0 bounds.  It gives conditions for a segment not to meet a triangle.}
\tag{pt4, circum4, sqrt2, plane, eta, halfspace, deprecated}
\rating{100}
\guid{XBNRPGQ}

\begin{lemma}\tlabel{tarski:dcg-page82}
Let $S=\{v_0,v_1,w_0,w_1\}$ be a packing of four points.
Assume that the circumradius of $S$ is $r\ge \sqrt2$.  Assume
that $S$ is not coplanar.   Assume
that the circumradius of $F=\{v_0,v_1,w_0\}$ is
  $\eta < \sqrt2$.   Let $c_0$ be the circumcenter of $F$.
Let $c_1$ be the point whose distances from $v_0,v_1$, and $w_0$ are
all exactly $\sqrt2$ and which lies in
$\op{aff}^0_+(F,w_2)$.
Let $F'=\{v_0,v_1,w_1\}$.  Assume that at least one of
the following
properties holds.
\begin{itemize}
  \item The circumradius of $F'$ is less than $\sqrt2$.
  \item $F'=\{u,v,w\}$ with $|u-v| \le 2.51$, $|v-w|\le 2.51$,
        $|u-w|\le\sqrt8$. Also, $|w_0-u|>2.51$ for some $u\in F'$.
\end{itemize}

Then $e=\op{conv}\{c_0,c_1\}$ does not
meet $\op{aff}(F)$.
\end{lemma}

\begin{proved}   We note that the point $c_1$ exists 
by Lemma~\tref{tarski:rog-exist}.

Assume for a contradiction that they meet.
The circumcenter $c_S$ of $S$ lies on
the line $\op{aff}\{c_0,c_1\}$ (through $e$) at a
distance $r$ from each point in $F$.  
Since $\eta < \sqrt2$, the orientation of $w_1$ is positive
with respect to $S$ (by Lemma~\tref{tarski:sqrt2-chi-plus}). 
Thus, $c_S$ lies along
the ray emanating from $c_0$ through $c_1$.  

This implies that $\op{conv}\{c_0,c_S\}$ meets $\op{aff}(F)$.
This in turn implies that $w_0$ has nonpositive orientation with
respect to $S$.  Again, by Lemma~\tref{tarski:sqrt2-chi-plus}, the
circumradius of $F'=\{v_0,v_1,w_1\}$ is at least $\sqrt2$.
If the orientation is
nonpositive, then 
   $$|w_0-u|\le 2.51,\quad\text{ for all } u\in F'$$
by Lemmas~\tref{tarski:neg-orient-quad} and
\tref{tarski:neg-orient-tet}. This is contrary to our assumption.
\swallowed\end{proved}
\end{tarski}
%<<<<<






%>>>>>
\begin{tarski}
\section{Rogers in a simplex}
\name{rog-in-S}
\summary{In a packing, under various constraints on distances, a Rogers's simplex is contained in the convex hull of four points.}
\tag{t0, sqrt2, pt4, packing, rad, conv4, deprecated}
\rating{0} % was 80 before deprecated.
\guid{MITDERY}

\begin{lemma}\tlabel{tarski:rog-in-S}
Let $S=\{v_0,v,w,w'\}$ be a packing of four points.
Suppose that
   $$
   2.51\le |v-v_0|\le\sqrt8,\quad 2.51< |w-w'|\le\sqrt8,
   %% second 2.51 was 2.77.
   $$
and the other four pairs in $S$ have separation at most $2.51$.
Let $c\le \rad_V(S)$.
Then $\op{rog}^0(v_0,w,v,w',c)$ is contained in
$\op{conv}^0(v_0,w,v,w')$.
\end{lemma}

\begin{proved}  Since $c\le \rad_V(S)$, we have
  $$
  \op{rog}^0(v_0,w,v,w',c)\subset \op{rog}^0(v_0,w,v,w',\rad_V(S)).
  $$
So it is enough to prove the result when $c=\rad_V(S)$.
The vertices of $\op{rog}^0$ are the circumcenter of $S$ and points
on $\op{conv}\{v_0,w,v\}$.
Since $|w-w'|\ge 2.51$, we see that the orientation of each face
of $\{v_0,v,w,w'\}$ is positive (Lemma~\ref{tarski:neg-orient-quad}).   It follows that the circumcenter
and hence all
the vertices of $\op{rog}^0$ lie in $\op{conv}^0(v_0,w,v,w')$.
The result follows.
\swallowed\end{proved}
\end{tarski}
%<<<<<





%% Lemma for use in UREVUCX.
%>>>>>
\begin{tarski}
\name{split-lune}
\summary{A point in a lune can be used to split it into two disjoint lunes.}
\tag{pt5, lune, plane, bary4}
\rating{80}
\guid{BAJSVHC}

\begin{lemma}\tlabel{tarski:split-lune}
Let $\{v_1,v_2,v_3,v_4,v_5\}$ be a set of five
points in $\ring{R}^3$.  Suppose that $\{v_1,v_2,v_3,v_4\}$ is
not coplanar.  Suppose that 
    $$v_5\in \op{aff}_+(\{v_1,v_3\},\{v_2,v_4\})$$
and $v_5\not\in\op{aff}\{v_1,v_3\}$.
Then 
    $$
    \op{aff}_+(\{v_1,v_3\},\{v_2,v_4\}) = 
\op{aff}_+(\{v_1,v_3\},\{v_2,v_5\}) \cup
\op{aff}_+(\{v_1,v_3\},\{v_5,v_4\}).
    $$
Moreover,
   $$
\op{aff}^0_+(\{v_1,v_3\},\{v_2,v_5\}) \cap
\op{aff}^0_+(\{v_1,v_3\},\{v_5,v_4\}) =\emptyset .
   $$
\end{lemma}

\begin{proved}
Use barycentric coordinates to write
   $$
   v_5 = t_1 v_1 + t_2 v_2 + t_3 v_3 + t_4 v_4,
   $$
and 
    $$
    \begin{array}{lll}
    x &= s_1 v_1 + s_2 v_2 + s_3 v_3 + s_4 v_4 \\
      &= (s_1 - s_2 t_1/t_2) v_1 + (s_3 - s_2 t_3/t_2) v_3 +
        (s_4 - s_2 t_4/t_2) v_4 + s_2/t_2 v_5\\
    \end{array}
    $$
with $t_2,t_4,s_2,s_4\ge 0$,
for any $x\in\op{aff}_+(\{v_1,v_3\},\{v_2,v_4\})$. 
This final expression for $x$ holds only if $t_2>0$. 
Let $r=t_2s_4-s_2t_4$.  
Note $t_2>0$ and $r<0$ imply $t_4\ge0$. In fact, we find that the following
three cases exhaust the possibilities:
\begin{itemize}
\item $t_2>0$ and $r\ge 0$,
\item $t_4>0$ and $r\le 0$,
\item $t_2=t_4=0$.
\end{itemize}
In the first case, this shows that 
$x\in\op{aff}_+(\{v_1,v_3\},\{v_4,v_5\})$.
Similarly, in the second case,
$x\in\op{aff}_+(\{v_1,v_3\},\{v_2,v_5\}$.
Finally if $t_2=t_4=0$, then $v_5\in\op{aff}\{v_1,v_3\}$, which is
contrary to hypothesis.  It follows that one of these two
conditions hold.  If both conditions hold, then $t_2,t_4>0$
and $t_2 s_4 - s_2 t_4 =0$, so that $s_2=s_4 = 0$.  This means
that $x\not\in\op{aff}^0$.
\swallowed\end{proved}
\end{tarski}
%<<<<<





%% Lemma for UREVUCX.
%>>>>>
\begin{tarski}
\name{three-lune}
\summary{Space can be partitioned into three lunes along a given line and determined by three points not on the line.}
\tag{pt5, plane, cone, lune}
\rating{100}
\guid{TBMYVLM}

\begin{lemma}\tlabel{tarski:three-lune}
Suppose that $\{v_0,v_1,v_2,v_3,v_4\}$ is a set
of five points in $\ring{R}^3$.  Assume that $\{v_0,v_4,v_i,v_j\}$
is not coplanar for any $1\le i < j \le 3$.
Assume that $v_4\in\op{cone}^0(v_0,\{v_1,v_2,v_3\})$.
Then 
   $$
   \ring{R}^3 = \bigcup_{1\le i<j\le 3}
    \op{aff}_+(\{v_0,v_4\},\{v_i,v_j\}).
   $$
Moreover,
   $$
   \op{aff}_+^0(\{v_0,v_4\},\{v_i,v_j\}) \cap
   \op{aff}_+^0(\{v_0,v_4\},\{v_\ell,v_m\}) =\emptyset,
   $$
for $\{i,j\}\ne \{\ell,m\}$, $1\le i<j\le 3$, $1\le \ell<m\le 3$.
\end{lemma}

\begin{proved} Write $A_{ij} = \op{aff}_+(\{v_0,v_4\},\{v_i,v_j\})$.
  Let $x\in\ring{R}^3$.  In barycentric coordinates, we have
  $$
  v_3 = t_0 v_0 + t_1 v_1 + t_2 v_2 + t_4 v_4
  $$
and
$$
  x = s_0 v_0 + s_1 v_1 + s_2 v_2 + s_4 v_4.
$$
Since $v_4\in\op{cone}^0(v_0,\{v_1,v_2,v_3\})$, we have
$t_1<0$, $t_2<0$, $t_4>0$.  
We have $s_1\ge0,s_2\ge0$ if and only if $x\in A_{12}$.
We have $s_2\le0,s_1t_2\le t_1s_2$ if and only if $x\in A_{13}$.
We have $s_1\le0,s_1t_2\ge t_1s_2$ if and only if $x\in A_{23}$.
It is easily checked that these three conditions cannot simultaneously
fail.  So $\ring{R}^3 = A_{12}\cup A_{13}\cup A_{23}$.

If $x\in A_{12}\cap A_{13}$, then $s_2=0$, so $x\not\in\op{aff}_+^0(\{v_0,v_4\},\{v_1,v_2\}$.  The other cases are similar.
\swallowed\end{proved}
\end{tarski}
%<<<<<





%% Helper for UREVUCX.
%>>>>>
\begin{tarski}
\name{omega-cone-rog}
\summary{A Rogers simplex is the intersection of a cone, a lune, and Voronoi cell.}
\tag{pt4, packing, chi, sqrt2, circum4, circum3, rogers, voronoi, cone, lune }
\rating{120}
\guid{TQLZOUG}

\begin{lemma}\tlabel{tarski:omega-cone-rog}
Suppose that $S=\{v_0,v_1,v_2,v_3\}$ is a packing
of four points.  Suppose that 
the orientations of $v_1$, $v_2$, and $v_3$ are positive with respect
to $S$.  Suppose that 
  $$
  2 \le |v_i - v_j | < \sqrt8, \quad 0\le i < j \le 3.
  $$
Let $p$ be the circumcenter of $S$.  Let $c_3$
be the circumcenter of $\{v_0,v_1,v_2\}$.
Let 
   $$
   R = \op{rog}(v_0,v_1,v_2,v_3,\rad_V(S)).
   $$
and $R^0$ the corresponding region with $\op{rog}$ replaced
with $\op{rog}^0$.
Then 
$$
 R \supset \op{cone}(v_0,\{v_1,v_2,v_3\})\cap
  \op{aff}_+(\{v_0,p\},\{v_1,c_3\}) \cap
  \Omega(S,v_0)
$$ 
and 
$$
 R^0 = \op{cone}^0(v_0,\{v_1,v_2,v_3\})\cap
  \op{aff}^0_+(\{v_0,p\},\{v_1,c_3\}) \cap
  \Omega(v_0,S)
$$ 
\end{lemma}

\begin{proved}
Let $w=(v_0+v_1)/2$.  We have $R=\op{conv}\{v_0,w,c_3,p\}$.
Write $C = \op{cone}(v_0,\{v_1,v_2,v_3\})$ and $C^0 = \op{cone}^0(\cdots)$.
By Lemma~\ref{tarski:hedra-tope}, 
   $$
   R=\op{aff}_+(\{v_0,p,c_3\},w)\cap \op{aff}_+(\{v_0,p,w\},c_3)\cap
   \op{aff}_+(\{p,w,c_3\},v_0)\cap \op{aff}_+(\{v_0,w,c_3\},p).
   $$
The intersection of the first two sets on the right is
$$
   \op{aff}_+(\{v_0,p\},\{c_3,w\}) = \op{aff}_+(\{v_0,p\},\{v_1,c_3\}).
$$
We have
$$
C = \cap_i \op{aff}_+(\{v_0,v_i,v_j\},v_k)
  \subset \op{aff}_+(\{v_0,v_1,v_2\},v_3) = \op{aff}_+(\{v_0,v_1,v_2\},p).
$$
Also,
$$
\Omega(S,v_0)\subset \Omega(\{v_1\},v_0) = \op{aff}_+^0(\{p,w,c_3\},v_0).
$$
The first claim follows.
A similar argument shows that
$$
 R^0 \supset C^0\cap
  \op{aff}_+^0(\{v_0,p\},\{v_1,c_3\}) \cap
  \Omega(S,v_0).
$$ 

In the other direction, assume that $x\in R^0$.
By the orientation constraints
$$
p\in \op{conv}\{v_0,v_1,v_2,v_3\}\subset C.
$$
By the edge constraints,
$$
v_0,w,c_3\in \op{cone}(v_0,\{v_1,v_2\}) \subset C.
$$
So $R=\op{conv}(p,v_0,w,c_3) \subset C$ and $R^0\subset C^0$.
As above,
$$
R^0 = \op{aff}^0_+(\{v_0,p\},\{c_3,w\}) \cap \cdots.
$$

We have 
$$R^0\subset \Omega(\{v_1\},v_0) = \op{aff}_+^+(\{p,w,c_3\},v_0)$$
and $$v_0,w,c_3,p\in \bar\Omega(\{v_2\},v_0)$$ so that $R^0\subset \Omega(\{v_2\},v_0)$.
Similarly, $$v_0,w,c_3,p\in \bar\Omega(\{v_3\},v_0)$$ so that $R^0\subset \Omega(\{v_3\},v_0)$.
Combining these inclusions gives the lemma.
\swallowed\end{proved}
\end{tarski}
%<<<<<




%>>>>>
\begin{tarski}
\name{omega-rog}
\summary{If the three generators have positive orientation, then the intersection of a Voronoi cell with a cone is a union of six Rogers simplices.}
\tag{pt4, packing, chi, sqrt2, rad, rogers, voronoi, cone}
\rating{80}
\guid{UREVUCX}

\begin{lemma}\tlabel{tarski:omega-rog}
Suppose that $S=\{v_0,v_1,v_2,v_3\}$ is a packing
of four points.  Suppose that 
the orientations of $v_1$, $v_2$, and $v_3$ are positive with respect
to $S$.  Suppose that 
  $$
  |v_i - v_j | < \sqrt8, \quad 0\le i < j \le 3.
  $$
Let 
   $$
   \begin{array}{lll}
   R_{ijk} &= \op{rog}(v_0,v_i,v_j,v_k,\rad_V(S)),\\
   R^0_{ij} &= \op{rog}(v_0,v_i,v_j,v_k,\rad_V(S)),
   \end{array}
   $$
where $(i,j,k)\in P$,  the set of six permutations $P$ of $(1,2,3)$.
Then
   $$
   \bigcup_{(i,j,k)\in P} R_{ijk}^0\subset
   \op{cone}(v_0,\{v_1,v_2,v_3\})\cap \Omega(v_0,S) \subset
   \bigcup_{(i,j,k)\in P} R_{ijk}.
   $$
Furthermore,
   $$
   R_{ijk}^0 \cap R_{i'j'k'} =\emptyset,\quad
   \{i,j,k\}\ne\{i',j',k'\}\in P.
   $$
\end{lemma}

\begin{proved} Let $p$ be the circumcenter of $S$.  Let $c_k$
be the circumcenter of $\{v_0,v_i,v_j\}$, for $(i,j,k)\in P$.
By Lemma~\tref{tarski:three-lune}, we have that
   $$
   \ring{R}^3 = \op{aff}_+(\{v_0,p\},\{v_1,v_2\}) \cup
     \op{aff}_+(\{v_0,p\},\{v_2,v_3\}) \cup
     \op{aff}_+(\{v_0,p\},\{v_3,v_1\}),
   $$
and that the pairwise intersections are empty when we replace
$\op{aff}_+$ with $\op{aff}_+^0$.  By Lemma~\tref{tarski:split-lune},
we have that 
  $$
  \op{aff}_+(\{v_0,p\},\{v_i,v_j\}) =
\op{aff}_+(\{v_0,p\},\{v_i,c_k\}) \cup
\op{aff}_+(\{v_0,p\},\{c_k,v_j\})
  $$
and the intersections of the corresponding terms $\op{aff}_+^0$ 
are empty.
The result now follows from Lemma~\tref{tarski:omega-cone-rog}.
\swallowed\end{proved}
\end{tarski}
%<<<<<





%>>>>>
\begin{tarski}
\name{tip-cone}
\summary{The tip of a Voronoi cell lies in the suitably constrained cone.}
\tag{pt4, chi, voronoi, halfspace, cone}
\rating{80}
\guid{RKWVONN}

\begin{lemma}\tlabel{tarski:tip-cone}
Suppose that $S=\{v_0,v_1,v_2,v_3\}$ is a packing
of four points.  Suppose that
the orientations of $v_1$, $v_2$, and $v_3$ are positive with respect
to $S$.  Suppose that
  $$
  |v_i - v_j | < \sqrt8, \quad 0\le i < j \le 3.
  $$
Then
  $$
  \bar\Omega(v_0,S) \cap \op{aff}_-(\{v_1,v_2,v_3\},v_0) \subset
  \op{cone}(v_0,\{v_1,v_2,v_3\}).
  $$
\end{lemma}




\begin{proved}  If the orientation of $v_0$ is positive,
then the left-hand side is empty, and there is nothing to prove.
Supose that 
 $$x\in
   \Omega(v_0,S) \cap \op{aff}_-(\{v_1,v_2,v_3\},v_0)
   \setminus
  \op{cone}(v_0,\{v_1,v_2,v_3\}).
 $$
Let $c$ be the circumcenter of $S$, and let $r$ be the
circumradius.  The segment $\op{conv}(x,c)$
lies in $\op{aff}_-(\{v_1,v_2,v_3\},v_0)$ and crosses into
$\op{cone}(v_0,\{v_1,v_2,v_3\})$ at some point $x'$.  
By Lemma~\tref{tarski:omega-rog}, up to indexing,
we have 
  $$x'\in\op{cone}(v_0,\{v_1,v_2\})\cap\op{rog}(v_0,v_1,v_2,v_3,r).$$
By the definition of the Rogers simplex, this intersection is
  $$\op{conv}\{v_0,(v_0+v_1)/2,c'\}$$
where $c'$ is the circumcenter of $\{v_0,v_1,v_2\}$.
By Lemma~\tref{tarski:circum-acute}, the points $v_0,v_1/2,c'$ all lie
on the same side of $\op{aff}\{v_1,v_2\}$ in the plane $\op{aff}\{v_0,v_1,v_2\}$.  This is contrary to assumption.
\swallowed\end{proved}
\end{tarski}
%<<<<<



%>>>>>
\begin{tarski}
\name{def:epsilon}
\summary{A relation $\epsilon$ relates a point in a Voronoi cell to the faces of the cell.}
\tag{def, epsilon, voronoi}
\rating{0}
\guid{IZKASBY}

\section{Epsilon}
In the next few lemmas we make use of a relation
$\epsilon$, which is defined as follows.
\begin{definition}[$\epsilon$]\tlabel{def:epsilon}
\indy{Greek}{ZZeps@$\epsilon$}
We define a relation $\epsilon(x,S,v,u)$, where $S$ is
a finite set of points in $\ring{R}^3$ containing $v$.  It is true if
the following conditions hold, and false otherwise.
\begin{itemize}
 \item $x\in \Omega(v,S)$.
 \item $v + t(x-v)\not\in\Omega(v,S)$ for
$t>0$ sufficiently large.
  \item $u,v\in S$, $u\ne v$.
 \item \begin{equation}\tlabel{fine:eps-test}
     \begin{array}{lll}
   |x'-u| &= |x'-v|\\
   |x'-u| &\le |x'-w|,\text{ for } w\in S\setminus\{u,v\},
   \end{array}
   \end{equation}
  where $x' = v+t_m (x-v)$ and $t_m$ is the supremum of $t$
  such that $v+t(x-v)\in\Omega(v,S)$.
\end{itemize}
\end{definition}
\end{tarski}
%<<<<<

%>>>>>
\begin{tarski}
\name{line-lem}
\summary{The locus equidistant from three points is a line.}
\tag{epsilon}
\rating{40}
\guid{EMLLARA}

\begin{lemma}\tlabel{tarski:line-lem}
Suppose that $S=\{u,v,w\}\subset \ring{R}^3$ is a set of three points.
Let 
$$
L = \{x \mid |x-u| = |x-v| = |x-w|\}.
$$
If the points are collinear then $L$ is empty.  Otherwise, it is a line.
\end{lemma}

\begin{proved} 
If the points are collinear, then
$$
L\subset \op{bis}\{u,v\} \cap \op{bis}\{u,w\}=\emptyset.
$$
Assume the points are not collinear.  
Lemma~\ref{tarski:mk-point} constructs two distinct points $p,p'$ on $L$. 
A simple calculation shows
that $\op{aff}\{p,p'\}=L$. 
\swallowed\end{proved}
\end{tarski}
%<<<<<

%>>>>>
\begin{tarski}
\name{epsilon-tie}
\summary{The relation $\epsilon$ holds for two different faces along a degenerate plane.}
\tag{epsilon}
\rating{80}
\guid{DKCSJPZ}

\begin{lemma}
If $\epsilon(x,S,v,u)$ and $\epsilon(x,S,v,w)$,
with $u\ne w$, then
   $x$ lies on a plane containing $v$ and the line equidistant
from $u,v,w$.  
\end{lemma}

\begin{proved}  The point $x'$ lies on the line (Lemma~\ref{tarski:line-lem})
   $$\{x' \mid \ |x'-u|=|x'-u'|=|x'-v|\}.$$
So $x$ lies in the given plane.
\swallowed\end{proved}
\end{tarski}
%<<<<<






%>>>>>
\begin{tarski}
\name{def:epsilonp}
\summary{A relation $\epsilon'$ relates a point in a Voronoi cell to a nearby edge of the cell.}
\tag{def, epsilon, voronoi, packing, sqrt8}
\rating{0}
\guid{RCTDSUH}
\begin{definition}[$\epsilon'$]\tlabel{def:epsilonp}
\indy{Greek}{ZZeps'@$\epsilon'$}
We define a relation $\epsilon'(x,S,v,u,w)$, where $S$ is
a finite set of points  $\ring{R}^3$
containing $v$.  It is true if
the following conditions hold, and false otherwise.
\begin{itemize}
  \item $\epsilon(x,S,v,u)$.
  \item Let $x' = v + t_m(x-v)$ be the point given in the
   definition of $\epsilon$.
  \item The points in $S$ have distance at least $2$ from
one another.
  \item  $|u-v|<\sqrt8$.
  \item  $(v+u)/2 + s (x'- (u+v)/2)$ 
  does not satisfy Condition~\tref{fine:eps-test}
of Definition~\tref{def:epsilon}
for $s>0$ sufficiently large.   (The condition is satisfied for
$s=0$ by Lemma~\tref{tarski:mid-Voronoi}.)  
  \item Let $s_m$ be the supremum of $s$ such that these
conditions are met.  Set $x''=(u+v)/2+s_m(x'-(u+v_/2)$.
  \item $$
  \begin{array}{lll}
   |x''-u| &= |x''-v| = |x''-w|\\
   |x''-w| &\le |x''-w'|,\text{ for } w'\in S\setminus\{u,v,w\}.
   \end{array}
  $$
\end{itemize}
\end{definition}\indy{Greek}{zzepsilon@$\epsilon(\Lambda,x)$}
 %  Assume futher
%that $x$ does not lie on a plane through $v$, $u$ and the circumcenter
%of $\{v,u,w,w'\}$ for any $\{w,w'\}\subset S\setminus\{u,v\}$.  
The conditions $\epsilon$ and $\epsilon'$ on $x$ 
correspond to the intuitive notion of $x$ lying
in the cone over the face of $\Omega(v,S)$ determined by the point $u$
with  
$\epsilon(x,S,v,u)$ and $x'$ lying in
the face and in the cone over the edge determined by the point $w$ such
that 
$\epsilon'(x,S,v,u,w)$.
\end{tarski}
%<<<<<



%>>>>>
\begin{tarski}
\name{vor-bar-23}
\summary{If a point of a blade is related by $\epsilon$ to anoter vertex, then that vertex does not have positive orientation.}
\tag{pt4, packing, epsilon, chi, sqrt2, plane, blade}
\rating{160}
\guid{KMTAMFH}

\begin{lemma}\tlabel{tarski:vor-bar-23}
\usage{DCG-[9.8]} % Fine decomposition page 89.
\usage{local -- tarski:fine:Rw:5}
Let $S=\{v_0,v_1,v_2,v_3\}$ be a packing of four points.
Assume that they are not coplanar.
Assume that 
$$
|v_1-v_2|<\sqrt8,\quad  |v_2-v_3|<\sqrt8.
$$
Suppose that
$C=\op{cone}(v_2,\{v_1,v_3\})$ has a point $x$ with 
  $\epsilon(x,\{v_0,v_1,v_3\},v_2,v_0)$.    
Then $v_0$ has non-positive orientation with respect to $S$.
\end{lemma}



\begin{proved} \FIXX{Recheck this proof}
Let Let $P_{ij}$ be the plane equidistant from $v_i$ and
$v_j$ and let $L_{ij} = \op{aff}\{v_1,v_2,v_3\}\cap P_{ij}$.
Let $L=L_{02}$.
The hypothesis implies that there exists a point 
$x'\in I \cap C$, where
   $$I= L\cap \{x'\mid |x'-v_2|\le|x'-v_1|,\ |x'-v_2|\le |x'-v_3|\}.$$  
The line $L$ meets either $\op{aff}\{v_1,v_2\}$ or $\op{aff}\{v_2,v_3\}$,
because it cannot be parallel to both.  Say $\op{aff}\{v_2,v_3\}$.
Since $|v_2-v_3|<\sqrt8$, the midpoint of $\{v_2,v_3\}$ is closer
to $v_2$ and $v_3$ than to $v_0$ by Lemma~\ref{tarski:mid-Voronoi}.
This implies that $I$ does not meet $\op{aff}\{v_2,v_3\}$.
We find that the intersection $w_{23}$ of the line $L$ with
the line $L_{23}$ lies in $\op{aff}_+^0(\{v_2,v_3\},v_2)$.  Similarly,
if $L$ meets $\op{aff}\{v_1,v_2\}$, then $w_{12}$ lies in
$\op{aff}_+^0(\{v_1,v_2\},v_3)$.  These points are the endpoints of 
$I$ (or one of them is, if $I$ is a ray).  It follows that $w_{23}\in C$.

We compute the barycentric coordinates of $w_{23}$ as in the proof
of Lemma~\ref{tarski:vor-bar}, to conclude that the orientation
of $v_0$ is non-positive.
\swallowed\end{proved}
\end{tarski}
%<<<<<





%>>>>>
\begin{tarski}
\name{eps-bigd-}
\summary{If a blade meets a right-circular cone (at more than their common apex), then various distance and circumradius constraints hold.}
%\endnote{Four points, if a blade meets a cone, then distances are small.  (For epsilon stuff).  tarski:eps-bigd-}
\tag{pt4, packing, t0, sqrt2, eta, blade, rcone, rad}
\rating{120}
\guid{JHOQMMR}

\begin{lemma}\tlabel{tarski:eps-bigd-}
 Let $S=\{v_0,v_1,u_1,u_2\}$ be a packing of four points.
Suppose all distances between pairs in $S$, except possibly $\{v_1,u_1\}$
$\{v_1,u_2\}$, are
less than $\sqrt8$. Suppose that $2.51 < |v_1-v_0|$.
Suppose that at most one of lengths $|v_0-u_1|$,
$|v_0-u_2|$, $|u_1-u_2|$ is greater than $2.51$.
%
Set $y=|v_1-v_0|$ and $b=\eta(2,2.51,y)$.
Suppose that $F=\op{cone}(v_0,\{u_1,u_2\})$ meets
$C=\op{rcone}^0(v_0,v_1,y/(2b))$ at $x$.
Then 
  $$
  |v_1-u_1|,|v_1-u_2|,|v_0-u_1|,|v_0-u_2|\le 2.51.
  $$
Furthermore, $\rad_V(S) < b$. 
\end{lemma}

\begin{proved}  
The orientation
of $v_1$ in $S$ is positive by Lemma~\tref{XX}.
 If $\rad_V(S)\ge b$, then for all sufficientlly small $r>0$, we have
$\epsilon(x,S,v_0,v_1)$ for $x\in C\cap B(v_0,r)$, 
but $\epsilon(x,S,v_0,u)$ implies $u\in\{u_1,u_2\}$,
for $x\in F\cap B(v_0,r)$, giving inconsistent values for $C\cap F$.  So
the sets are disjoint in this case.  

We conclude that $\rad_V(S) < b$.
If $|v_1-u_1| > 2.51$, then we get the contradiction
$$b = \eta(2,2.51,y) < \eta_V(v_0,v_1,u_1)\le \rad_V(S) < b.$$
Similarly, we find that
   $$
   |v_1-u_1|,|v_1-u_2|,|v_0-u_1|,|v_0-u_2| \le 2.51.
   $$
\swallowed\end{proved}
\end{tarski}
%<<<<<





%>>>>>
\begin{tarski}
\name{eps:fine:Rw}
\summary{If a blade meets a Rogers simplex at more than their common apex, then various distance and circumradius constraints hold.}
%\endnote{Four points, If a blade meets a Rogers, then distances are small.  For fitted crowns. tarski:eps:fine:Rw}
\tag{pt4, eta, packing, blade, rogers, rad}
\rating{120}
\guid{YFTQMLF}

\begin{lemma}\tlabel{tarski:eps:fine:Rw}
\usage{DCG-[9.8]} % fine decomposition page 89.
 Let $S=\{v_0,v_1,u_1,u_2\}$ be a packing of four points.
Suppose all distances between pairs in $S$, except possibly $\{v_1,u_1\}$
$\{v_1,u_2\}$, are
less than $\sqrt8$.
%Suppose that all distances between pairs in $S$ are at least $2$
%and at most $\sqrt8$. 
Suppose that $2.51 < |v_1-v_0|$.
Suppose that at most one of lengths $|v_0-u_1|$,
$|v_0-u_2|$, $|u_1-u_2|$ is greater than $2.51$.
%
Set $y=|v_1-v_0|$ and $b=\eta(2,2.51,y)$.
Suppose that $F=\op{cone}(v_0,\{u_1,u_2\})$ meets
$R=\op{rog}^0(v_0,u_1,v_1,w',b)$ at $x$, (choosing any $w'$ that
does not lie in the plane of $\{v_0,u_1,v_1\}$).
Then 
  $$
  |v_1-u_1|,|v_1-u_2|,|v_0-u_1|,|v_0-u_2|\le 2.51.
  $$
Furthermore, $\rad_V(S) < b$. 
\end{lemma}

\begin{proved}  
The orientation
at $v_1$ in $S$ is positive by Lemmas~\tref{XX}.
If $\rad_V(S)\ge b$, then for $r>0$ sufficiently small, we have
$\epsilon'(x,S,v_0,u_1,v_1)$ for $x\in R\cap B(v_0,r)$, 
but $\epsilon'(x,S,v_0,u_1,u)$ implies $u\in\{u_1,u_2\}$,
for $x\in F\cap B(v_0,r)$, giving inconsistent values for $R\cap F$.  So
the sets are disjoint in this case.  

We conclude that $\rad_V(S) < b$.
As in the proof of Lemma~\tref{tarski:eps-bigd-},
this implies that
   $$
   |v_1-u_1|,|v_1-u_2|,|v_0-u_1|,|v_0-u_2| \le 2.51.
   $$
\swallowed\end{proved}
\end{tarski}
%<<<<<





%% Use in fine decomposition. Extracted
% Fine decomposition page 89.
%>>>>>
\begin{tarski}
\name{fine:Rw:5}
\summary{This is one of the most complex statements in the collection.  Under a large list of assumptions, if
a blade meets a Rogers simplex at more than their common apex, then the configuration can take three possible forms.}
\tag{pt5, packing, eta, epsilon, blade, rogers, cone, lune}
\rating{220}
\guid{GQMZTHN}

\begin{lemma}\tlabel{tarski:fine:Rw:5}\usage{DCG-[9.8]}
 Let $S=\{v_0,v_1,w,u_1,u_2\}$ be a packing of five points.
Suppose that 
   $$
   \begin{array}{rlrlrll}
   |v_0-u_1|&<\sqrt8,& |v_0-u_2|&<\sqrt8,& |u_1-u_2|&<\sqrt8,\\
   |w-v_0|&\le 2.51, &2.51&<|v_0-v_1|<\sqrt8.\\
   |v_1-w|&\le 2.51.
   \end{array}
   $$
Suppose that at most one of $|v_0-u_1|,|v_0-u_2|,|u_1-u_2|$ is
greater than $2.51$.
%
Set $y=|v_1-v_0|$ and $b=\eta(2,2.51,y)$.
Suppose that $\epsilon(x,\{v_1,w,u_1,u_2\},v_0,w)$ for some point
%%
$x\in F\cap R$, where $F=\op{cone}(v_0,\{u_1,u_2\})$ and
$R=\op{rog}^0(v_0,w,v_1,p,b)$ (for some $p$ that
does not lie in the plane of $\{v_0,w,v_1\}$).
Then $|w-u_1|\le 2.51$ and  $|w-u_2|\le 2.51$.
Moreover, one of the following holds:
  \begin{itemize}
  \item
  For $u=u_1$ or $u=u_2$, we have that
  $R$ meets $\op{cone}(v_0,\{w,u\})$.
  %
  \item  We have $2.51<|u_1-u_2|<\sqrt8$, 
  $v_1\in\op{cone}^0(v_0,\{w,u_1,u_2\})$, 
  $|v_1-u_1|\le 2.51$, and $|v_1-u_2|\le 2.51$.
  \item  We have $2.51<|u_1-u_2|<\sqrt8$, and 
  $w\in\op{aff}_+(\{v_0,v_1\},\{u_1,u_2\})$.
  \end{itemize}
\end{lemma}

\begin{proved}  
Note that $\epsilon(x,\{v_1,w,u_1,u_2\},v_0,v_1)$ never holds
on points of $R$, so we can use the relation
$\epsilon(x,\{w,u_1,u_2\},v_0,-)$ instead.

By Lemma~\ref{tarski:vor-bar-23}, the orientation
of $w$ in $S=\{v_0,w,u_1,u_2\}$ is negative.  This implies that
$|w-u_1|\le 2.51$ and  $|w-u_2|\le 2.51$, by
 Lemmas~\ref{tarski:neg-orient-tet} and \ref{tarski:neg-orient-quad}. 


Assume first that 
$v_1\not\in \op{aff}_+(\{v_0,w\},\{u_1,u_2\})$.  
Thus, $v_1\in \op{aff}_-^0(\{v_0,w\},u)$, for $u=u_1$ or $u=u_2$.
If $R$ does not meet $P=\op{aff}\{v_0,w,u\}$, then this plane separates
$R$ from $F\subset \op{aff}_+(\{v_0,w\},\{u_1,u_2\})$, which contradicts
the hypotheses.  So $R$ meets $P$ (say at $y$).  
If $y'$ is the midpoint $(v_0+w)/2$, then
%projection of $y$ to the line $\op{aff}\{v_0,w\}$, then
$y' + t (y - y')\in R\cap P$, for $0<t<1$.  Choosing $t$ sufficiently
small, we get $y'\in R\cap \op{cone}(v_0,\{w,u\})$.
This is the first of the possibilities enumerated in the statement
of the lemma.

Now assume that
$v_1\in \op{aff}_+(\{v_0,w\},\{u_1,u_2\})$.  
By Lemma~\ref{tarski:consec-anchors}, we have $2.51<|u_1-u_2|$.
The hypotheses of the lemma give $|v_0-u_1|,|v_0-u_2|\le 2.51$.
Either $\op{cone}^0(v_0,\{w,v_1\})$ meets $\op{cone}^0(v_0,\{u_1,u_2\})$
or it does not.  If they do not meet, then
$v\in\op{cone}^0(v_0,\{w,u_1,u_2\})$.  Moreover, the
estimates $|v_1-u_1|\le 2.51$ and $|v_1-u_2|\le 2.51$ hold
by Lemma~\ref{tarski:pass-anchor}.  % E().

If they meet, by Lemma~\ref{tarski:quad-types}, we have that
the signs of the barycentric coordinates are such that
$w\in\op{aff}_+(\{v_0,v_1\},\{u_1,u_2\})$.  This completes the proof.
\swallowed\end{proved}
\end{tarski}
%<<<<<






%% Use in fine decomposition. Extracted
% Fine decomposition page 89.
%>>>>>
\begin{tarski}
\name{fine:Rw:u}
\summary{This is one of the most complex statements in the collection.  Under a long list of assumptions, if a blade meets a Rogers simplex, then the configuration can take three possible forms.}
\tag{pt5, pt6, packing, t0, sqrt2, 2.77, epsilon, blade, rogers, lune, rad, eta}
\rating{320}
\guid{KWOHVUP}

\begin{lemma}\tlabel{tarski:fine:Rw:u}\usage{DCG-[9.8]}
 Let $S=\{v_0,v_1,w,u_1,u_2\}$ be a packing of five points.
Suppose that 
   $$
   \begin{array}{rlrlrll}
   |v_0-u_1|&<\sqrt8,& |v_0-u_2|&<\sqrt8,& |u_1-u_2|&<\sqrt8,\\
   |w-v_0|&\le 2.51, &2.51&<|v_0-v_1|<\sqrt8.\\
   |v_1-w|&\le 2.51.
   \end{array}
   $$
Suppose that at most one of $|v_0-u_1|,|v_0-u_2|,|u_1-u_2|$ is
greater than $2.51$.
%
Set $y=|v_1-v_0|$ and $b=\eta(2,2.51,y)$.
Suppose that $\epsilon(x,\{v_1,w,u_1,u_2\},v_0,u_1)$ for some point
%%
$x\in F\cap R$, where $F=\op{cone}(v_0,\{u_1,u_2\})$ and
$R=\op{rog}^0(v_0,w,v_1,p,b)$ (for some $p$ that
does not lie in the plane of $\{v_0,w,v_1\}$).
Suppose further 
that if $\rad_V\{v_0,v_1,w,u_1\} < \eta(|v-v_0|,2,2.51)$, 
and if $R\subset \op{aff}_+(\{v_0,v_1\},\{w,u_1\})$,
then there exists
$w'\in\op{aff}_+^0(\{v_0,v\},\{w,u_1\})$ such that 
  \begin{enumerate}
  \item $\rad_V\{v_0,v_1,u_1,w'\} \ge \eta(|v_0-v_1|,2,2.51)$, 
  \item $2\le |u'-w'|$, for $u'=v_0,v_1,w,u_1,u_2$, and
  \item $|w'-v_1|\le 2.51$, $|w'-v_0|\le 2.51$, $|w'-w|\ge 2.77$.
  \end{enumerate}
Then one of the following holds:
  \begin{itemize}
  \item $|v_0-u_1|\le 2.51$, $|v_1-u_1|\le 2.51$, $|u_1-w|\le 2.51$.
   Also, $u_2\in \op{aff}_+^0(\{v_0,u_1\},\{v_1,w\})$.
   %Also, $R\subset \op{cone}(u,\{v_0,v_1,w\})$,
   %and $R\subset \op{aff}_-(\{v_0,v_1,w\},u)$.
  %
  \item $|v_0-u_1|\le 2.51$, $|v_1-u_1|\le 2.51$, $|u_1-w|\le 2.51$.
  %% quarter
  Also, $R\subset \op{aff}_+(\{v_0,v_1,w\},u)$.
  \item $|v-u|\le 2.51$, for $v=v_0,v_1$, and $u=u_1,u_2$.  Also,
   there exists a point $w''\in\op{aff}_+^0(\{v_0,v_1\},\{u_1,u_2\})$ 
   that has distance at least $2$ from
   $v_0,v_1,u_1,u_2$ and such that
   $|w''-v|\le 2.51$, for $v=v_0,v_1$.
  \end{itemize}
\end{lemma}

\begin{proved} $R$ lies entirely in
$A_\pm=\op{aff}_\pm(\{v_0,v_1,w\},u)$ for appropriate sign.
We separate the proof into two cases, according to sign.
Let $S'=\{v_0,v_1,w,u_1\}$.  Abbreviate $v=v_1$, $u=u_1$.

Assume that $R\subset A_-$.
For any nonzero vertices $v',v''$,
Let $L(v',v'')$ be the line of points equidistant from $\{v_0,v',v''\}$.
The three lines $L(u,v)$, $L(v,w)$, $L(u,w)$ meet at the circumcenter
$c$ of $S'$.  The rays $L^+(u,v)$, $L^+(v,w)$, $L^+(u,w)$ demarcate
the regions between $\epsilon(x,\{w,u,v\},v_0,p)$, for $p=w,u,v$.  
(Pick the direction
of ray so that it runs through the circumcenter of the face $\{v_0,v',v''\}$
if the remaining vertex has positive orientation, and so that it
runs in the opposite direction otherwise.)

If $u$ has positive orientation in $S'$, then $L^+(v,w)$ runs through
the circumcenter of $\{v_0,v,w\}$ and along an edge of $R_w$.
The point $w/2$ in the closure of  $R_w$ also has 
$\epsilon(w/2,\ldots,w)$.
It follows that $\epsilon(x,\ldots,w)$ and not $\epsilon(x,\ldots,u)$
on $R_w$, which is contrary
to our assumption.

Thus, $u$ has negative orientation in $S$.  
This implies that $|u-v|,|u-w|,|u-v_0|\le 2.51$.  In particular,
$S'$ is a quarter.  By Lemma~\ref{tarski:tip-cone}, we have
that $x\in \op{cone}(u,\{v,w,v_0\})$.  As $x\in F = \op{cone}(v_0,\{u_1,u_2\})$, it follows that $u_2\in \op{aff}_+^0(\{v_0,u_1\},\{w,v_1\})$.\FIXX{Add detail}
This falls within the first case of the lemma.

%%WW MOVE THIS BACK TO FINE.TEX
%Since it has the same diagonal as a quarter in
%the $Q$-system, we have that $S$ is in the $Q$-system, so that
%$\{v_0,v,w\}$ is a barrier.  By Lemma~{tarski:tip-cone}, we have
%that $x\in \op{cone}(u,\{v,w,v_0\})$.  In particular, $x$ is obstructed
%from $u$ by the barrier $\{v,w,v_0\}$.  Take $w'=w$ in this case.

Now assume that $R\subset A_+$.
First assume additionally that
$\rad_V(S')\ge \eta_{V0}(v,v_0)$ and that
the orientation of $u$ is non-positive in $S'$.  It
follows that we are in the second case of the lemma.  
%By the rule
%for constructing crown tuples, there is no $R_w$ along $\{v_0,v,w\}$
%in this case.  

Next assume additionally that
$\rad_V(S')\ge \eta_{V0}(v,v_0)$ and the orientation of $u$ is positive
in $S'$.  The ray $L^+(v,w)$ runs along the edge of 
$R_w$ as before, and we cannot have $\epsilon(x,\ldots,u)$.
%see that $\epsilon_0=w$ on $R_w$.

Finally assume additionally that $\rad_V(S')<\eta_{V0}(v,v_0)$.
It follows that $u$ is an anchor of $\{v_0,v\}$.
There exists $w'$ between $u$ and $w$
satisfying the the hypotheses of the lemma.
From $|w-w|\ge 2.77$ and $\rad_V\{v_0,v_1,w,w'\} \ge \eta_{V0}(v,v_0)$,
it follows that $x\in \op{aff}_-(\{v_0,v_1,w'\},u)$.  By Lemma~{tarski:tip-cone}, we have $x\in\op{cone}(u_1,\{v_0,v_1,w'\})$.
As above, $u_2\in\op{aff}_+^0(\{v_0,u_1\},\{v_1,w'\})$. 
It follows that we are in the third case of the lemma.\FIXX{Add Details.}
%By Lemma~\ref{tarski:prev},
%$\op{conv}\{x,u\}$ meets $\op{conv}\{v_0,v,w'\}$, and
% $|u-w'|\le 2.51$.  In particular $S'=\{v_0,v,w',u\}$ is
%an upright quarter and $\{v_0,v,w'\}$ is a barrier.
\swallowed\end{proved}
\end{tarski}
%<<<<<





%>>>>>
\begin{tarski}
\name{fine-Rw-split}
\summary{This is the most complex statement in the collection.  If a blade meets a lune at more than their common apex, then the configuration can take six different forms.}
%\endnote{Five Points, If a Rogers meet a blade then strange things happen. tarski:fine-Rw-split}
\tag{pt5, packing, t0, sqrt2, blade, rogers, eta, lune, rad}
\rating{80}
\guid{UJCUNAS}

\begin{lemma}\tlabel{tarski:fine-Rw-split}
\usage{DCG-[9.8]} % Fine decomposition page 89.
 Let $S=\{v_0,v_1,w,u_1,u_2\}$ be a packing of five points.
Suppose that 
   $$
   \begin{array}{rlrlrll}
   |v_0-u_1|&<\sqrt8,& |v_0-u_2|&<\sqrt8,& |u_1-u_2|&<\sqrt8,\\
   |w-v_0|&\le 2.51, &2.51&<|v_0-v_1|<\sqrt8.\\
   |v_1-w|&\le 2.51.
   \end{array}
   $$
Suppose that at most one of $|v_0-u_1|,|v_0-u_2|,|u_1-u_2|$ is
greater than $2.51$.
%
Set $y=|v_1-v_0|$ and $b=\eta(2,2.51,y)$.
%Suppose that $\epsilon(x,\{v_1,w,u_1,u_2\},v_0,u_1)$ for some point
%%
Suppose that $F\cap R\ne\emptyset$, 
where $F=\op{cone}(v_0,\{u_1,u_2\})$ and
$R=\op{rog}^0(v_0,w,v_1,p,b)$ (for some $p$ that
does not lie in the plane of $\{v_0,w,v_1\}$).
Suppose further 
that if $\rad_V\{v_0,v_1,w,u_1\} < \eta(|v-v_0|,2,2.51)$, 
and if $R\subset \op{aff}_+(\{v_0,v_1\},\{w,u_1\})$,
then there exists
$w'\in\op{aff}_+^0(\{v_0,v\},\{w,u_1\})$ such that 
  \begin{enumerate}
  \item $\rad_V\{v_0,v_1,u_1,w'\} \ge \eta(|v_0-v_1|,2,2.51)$, 
  \item $2\le |u'-w'|$, for $u'=v_0,v_1,w,u_1,u_2$, and
  \item $|w'-v_1|\le 2.51$, $|w'-v_0|\le 2.51$, $|w'-w|\ge 2.77$.
  \end{enumerate}
Then one of the following holds:
  \begin{itemize}
  \item
  $|w-u_1|\le 2.51$ and  $|w-u_2|\le 2.51$.
  Moreover, for $u=u_1$ or $u=u_2$, we have that
  $R$ meets $\op{cone}(v_0,\{w,u\})$.
  %
  \item  We have $|w-u_1|\le 2.51$ and  $|w-u_2|\le 2.51$.
  We have $2.51<|u_1-u_2|<\sqrt8$, 
  $v_1\in\op{cone}^0(v_0,\{w,u_1,u_2\})$, 
  $|v_1-u_1|\le 2.51$, and $|v_1-u_2|\le 2.51$.
  \item  We have $|w-u_1|\le 2.51$ and  $|w-u_2|\le 2.51$.
  We have $2.51<|u_1-u_2|<\sqrt8$, and 
  $w\in\op{aff}_+(\{v_0,v_1\},\{u_1,u_2\})$.
  \item For some matching $\{u,u'\}=\{u_1,u_2\}$, we have
  $|v_0-u|\le 2.51$, $|v_1-u|\le 2.51$, $|u-w|\le 2.51$.
   Also, $u'\in \op{aff}_+^0(\{v_0,u\},\{v_1,w\})$.
   %Also, $R\subset \op{cone}(u,\{v_0,v_1,w\})$,
   %and $R\subset \op{aff}_-(\{v_0,v_1,w\},u)$.
  %
  \item $|v_0-u|\le 2.51$, $|v_1-u|\le 2.51$, $|u-w|\le 2.51$.
  %% quarter
  Also, $R\subset \op{aff}_+(\{v_0,v_1,w\},u)$.
  \item $|v-u''|\le 2.51$, for $v=v_0,v_1$, and $u''=u_1,u_2$.  Also,
   there exists a point $w''\in\op{aff}_+^0(\{v_0,v_1\},\{u_1,u_2\})$ 
   that has distance at least $2$ from
   $v_0,v_1,u_1,u_2$ and such that
   $|w''-v|\le 2.51$, for $v=v_0,v_1$.
  \end{itemize}
\end{lemma}


\begin{proved}  This follows directly from Lemma~\ref{tarski:fine:Rw:5}
and Lemma~\ref{tarski:fine:Rw:u}, where the first three cases
come when some $x\in F\cap R$ satsifies
  $\epsilon(x,\{v_1,w,u_1,u_2\},w)$ and the final three cases
come when $\epsilon(x,\{v_1,u_1,u_2\},u)$ for some $u\in \{u_1,u_2\}$.
\swallowed\end{proved}
\end{tarski}
%<<<<<






%>>>>>
\begin{tarski}
\name{prev}
\summary{If a point in Roger's simplex is related by $\epsilon$ to another point, then under special assumptions, the segment between these two points must pass through a triangle.}
%\endnote{Five points, for fitted crowns, if a Rogers has a bad epsilon, then a segment crosses through a triangle, tarski:prev}
\tag{pt5, packing, t0, eta, rad, conv2, halfplane, aff-meet, rogers, voronoi, conv3, epsilon}
\rating{160}
\guid{DVLHHMF}

\begin{lemma}\tlabel{tarski:prev}\usage{DCG-[9.7]{lemma:prev}}
Let $S=\{v_0,v,w,w',u\}$ be a packing of five points.
Assume that $2.51<|v_0-v|<\sqrt8$, $2.51<|w-w'|$, and
  $$
  |v'-w''| \le 2.51, \text{ for } v'=v_0,v \text{ and } w''=w,w',u.
  $$
Assume $\rad_V\{v_0,v,w,w'\}\ge\eta(2,2.51,|v-v_0|)$
Assume that $\op{conv}\{w,u\}$ meets the
half plane $\op{aff}_+(\{v_0,v\},w')$.
Set
  $$R_w=\op{rog}^0(v_0,w,v,u,\eta(2,2.51,|v-v_0|)).$$
Let 
   $$R'_w = \{x\in R_w \cap\Omega(v_0,\{w,u\}) \mid
     \epsilon(x,\{v_0,w,u\},v_0,u)\}.$$
Assume $R_w'\ne \emptyset$.
Then
\begin{enumerate}
 \item  For every point $x\in R'_w$, the segment $\op{conv}\{x,u\}$ meets
  $\op{conv}\{v_0,v,w'\}$.
\item $|u-w'|\le 2.51$.
\end{enumerate}
\end{lemma}

\begin{proved} 
Let $x\in R'_w$. Set $S'=\{v_0,v,w,w'\}$ and $Q=\{v_0,v,w',u\}$.
Let $x' = t x$ be the point as in the definition of $\epsilon$ such
that $|x'-u|=|x'-v_0|$.  By the definition of $\epsilon$,
$|x'-u| \le |x'-w|$.  Since $\rad_V(S')\ge\eta(2,2.51,|v-v_0|)$, we have
$|x'-v_0| \le |x-v|,|x-w'|$.  Thus, $x'$ is at least as close to $u$
as any of the other vertices of $Q$.  Also, $x'$
and $u$ lie on opposite sides of the plane $\op{aff}\{v_0,v,w'\}$
(again because of the condition on $\rad_V(S')$).  Thus, $u$ has
negative orientation in $Q$.  This forces 
$|u-w'|\le 2.51$ by Lemma~\tref{XX}.

We have that $x'\in\Omega(u,Q)$.  By Lemma~\tref{tarski:tip-cone},
we have that $x'\in\op{cone}(u,\{v_0,v,w'\})$.  This completes
the proof.
\swallowed\end{proved}
\end{tarski}
%<<<<<






%>>>>>
\begin{tarski}
\name{eps-inner}
\summary{Under special conditions, 
a Rogers's simplex is disjoint from a right-circular cone with the same apex.}
%\endnote{Five points, disjointness of a Rogers simplex with an rcone.  For fitted crown  stuff.  tarski:eps-inner}
\tag{packing, pt5, eta, rogers, rcone, t0, sqrt2, plane}
\rating{80}
\guid{UIXOFDB}

\begin{lemma}\tlabel{tarski:eps-inner}\usage{lemma:FC-no-over}
Let $S=\{v_0,v_1,v_1',w_1,w_2\}$ be a packing of five points.
  Assume that
  $$\begin{array}{lll}
  2.51 \le |v_0-v_1| < \sqrt8,\\
  2.51\le |v_0-v_1'| < \sqrt8,\\
  |v-w|\le 2.51, \text{ for } v=v_0,v_1, w=w_1,w_2.\\
  \end{array}
  $$
Assume $\{v_0,v_1,w_1,w_2\}$ is not coplanar.  
%Assume that
%$$\op{rad}_V\{v_0,v_1,w_1,w_2\} > \eta(|v_0-v_1|,2,2.51).$$
Let $b' = |v_0-v'_1|/(2\eta(|v_0-v'_1|,2,2.51))$.  
Let $b== |v_0-v_1|/(2\eta(|v_0-v_1|,2,2.51))$.  
Then
$\op{rog}^0(v_0,w_1,v_1,w_2,b)$ is disjoint from $\op{rcone}(v_0,v_1',b')$.
\end{lemma}


\begin{proved} The relation $\epsilon$ takes incompatable on
$\op{rog}^0$ and $\op{rcone}$, so they are disjoint.
\swallowed\end{proved}
\end{tarski}
%<<<<<






%>>>>>
\begin{tarski}
\name{eps-outer}
\summary{This is one of the most complex statements of the entire collection.  The main
assumption is that the interior two Rogers simplices meet.  The conclusion asserts that
points must be configured in one of three ways.}
%\endnote{Five points: If two Rogers meet, then strange things happen.  For fitted crowns. tarski:eps-outer}
\tag{packing, pt5, t0, sqrt2, eta, plane, blade, rogers, halfspace, epsilon}
\rating{140}
\guid{MJNUTQH}

\begin{lemma}\tlabel{tarski:eps-outer}\usage{lemma:FC-no-over}
Let $S=\{v_0,v_1,v_2,w_1,w_2\}$ be a packing of five points.
  Assume that
  $$\begin{array}{lll}
  2.51 \le |v_0-v_i| < \sqrt8,\quad i=1,2\\
  |v_i-w_i|\le 2.51, \quad i=1,2.\\
  |v_0-w_i|\le 2.51,\quad i=1,2.
  \end{array}
  $$
pick $p_i$ so that $\{v_0,v_i,w_i,p_i\}$ is not coplanar.
Let $b_i=\eta(|v_0-v_i|,2,2.51)$.  
%
%Assume that
%$$\op{rad}_V\{v_0,v_1,w_1,w_2\} > \eta(|v_0-v_1|,2,2.51).$$
%For $i=1,2$, let $u_i\in \ring{R}^3$ satisfy
%the following condition:
%If 
%  \begin{itemize}
%  \item $|w_1-w_2|>2.51$,
%  \item $\rad_V\{v_0,v_i,w_1,w_2\} < \eta(2,2.51,|v_0-v_i|)$, and
%  \item $\op{aff}_+(\{v_0,v_i,w_j\},p_j) = \op{aff}_+(\{v_0,v_i,w_j\},w_k)$,
%     for $\{j,k\}=\{1,2\}$,
%  \end{itemize}
%then
%  \begin{itemize}
%   \item $u_i \in\op{aff}_+^0(\{v_0,v_i\},\{w_1,w_2\}$,
%   \item the distance from $u_i$ to each point of $\{v_0,v_i,w_1,w_2\}$ is at least $2$,
%   \item $|u_i - v_0| \le 2.51$, 
%   \item $|u_i - v_i| \le 2.51$, and
%   \item $|u_i - w| \le 2.51$, for some $w\in\{w_1,w_2\}$.
%  \end{itemize}
Assume that $\op{rog}^0(v_0,w_1,v_1,p_1,b_1)$ meets
  $\op{rog}^0(v_0,w_2,v_2,p_2,b_2)$.
Then there exists a matching $\{i,j\}=\{1,2\}$ 
such that $|v_i-w_j|\le 2.51$ and one of the following holds:
\begin{itemize}
     \item $|w_1-w_2|\le 2.51$ and $\op{rog}^0(v_0,w_2,v_2,p_2,b_2)$ meets
           $\op{aff}_+^0(v_0,\{v_1,w_1\})$.
       \item $|w_1-w_2|\le 2.51$ and 
     $$
     \op{aff}_+(\{v_0,v_i,w_i\},w_j) = \op{aff}_+(\{v_0,v_i,w_i\},p_i);
     $$
              \item $|w_1-w_2|> 2.51$, $\epsilon(x,\{v_0,v_1,w_1,w_2\},v_0,w_2)$,
                for some $x\in \op{rog}^0(v_0,w_1,\ldots)$, 
             $\rad_V\{v_0,v_i,w_1,w_2\}<\eta(2,2.51,|v_0-v_i|)$, and
    $$
     \op{aff}_+(\{v_0,v_i,w_i\},w_j) = \op{aff}_+(\{v_0,v_i,w_i\},p_i).
     $$
\end{itemize}
\end{lemma}

\begin{proved}
Let $x$ be a point in the intersection of $\op{rog}^0(v_0,w_1,v_1,p_1,b_1)$ 
with $\op{rog}^0(v_0,w_2,v_2,p_2,b_2)$.  Up to renaming indices and perturbing $x$,
we may assume that
$\epsilon(x,\{v_0,v_1,w_1,w_2\},v_0,u)$ when $u=w_2$ but not when $u=w_1$.  
Let $S=\{v_0,v_1,w_1,w_2\}$.
We choose the index $i=1$.

We have $|v_1-w_2|\le 2.51$. 
Otherwise $\rad_V(S) \ge \eta(2,2.51,|v_1-v_0|)$ and
the orientation of $S$ at $w_2$ is positive.  In this case, we have
$\epsilon(x,\{v_0,v_1,w_1,w_2\},v_0,w_1)$ contrary to hypothesis.  

We will say that $w_1$ points {\it inward} if
     $$
     \op{aff}_+(\{v_0,v_1,w_1\},w_2) = \op{aff}_+(\{v_0,v_1,w_1\},p_1),
     $$
and outward otherwise.

If $|w_1-w_2|> 2.51$, then the orientation of $w_2$ in $S$ is positive.
We have $\rad_V(S) < \eta(2,2.51,|v_1-v_0|)$, for otherwise
$\epsilon(x,\ldots,w_1)$.   Also, $w_1$ points inward, for otherwise
$\epsilon(x,\ldots,w_1)$.

Now take $|w_1-w_2|\le 2.51$. 
We assume that the second case of the conclusion does not hold and prove that
the first case holds.  Thus, we assume that $w_1$ points outward.
By the definition of $\epsilon$, there exists $t\ge 1$ such
that $$u = v_0 + t (x- v_0)$$ satisfies
$$|u - w_2|=|u-v_0| < |u-w_1|,\ |u-v_1|.$$
Generally, the inequality is not strict here, but by perturbing the initial
choice of $x$ as we have, we may assume it is strict.
For $t'>t$ sufficiently small, and setting $u' = v_0 + t'(x-v_0)$, we
have
$$
  |u - w_2| < |u-v|, \text{ for } v = v_0,w_1,v_1.
$$
By Lemma~{tarski:tip-cone}, we have
$$
  u' \in \op{aff}_+^0(w_2,\{v_0,w_1,v_1\}).
$$
Then also $x \in\op{aff}_+^0(w_2,\{v_0,w_1,v_1\})$.
%$$
%     H=\op{aff}^0_-(\{v_0,v_1,w_1\},w_2) = \op{aff}^0_+(\{v_0,v_1,w_1\},p_1);
%%     $$
%The dihedral angle $\dih_V(\{v_0,w_1\},\{v_1,w_2\}) < \pi/2$ (justified by
%the arguments of Section~\ref{sec:trig}),\FIWW{Counterexample:
%$\dih(2.51, 2.51, 2, 2.51, 2, 2) > \pi/2$.  Need to fix this, perhaps with interval calculations.}
%so its complementary
%angle is greater than $\pi/2$.  The dihedral angle of $R_1=\op{rog}^0(v_0,w_1\ldots)$
%along $\{v_0,w_1\}$ is less than $\pi/2$.  Thus, $R_1$ is contained in
%$A=\op{aff}_+^0(\{v_0,w_2\},\{v_1,w_1\})$.  In particular, $x\in A$.
%Our assumption gives  $x \in R_1 \subset H$.  
Also, $R_2 = \op{rog}^0(v_0,w_2,\ldots)$
is convex, contains $x$, and extends to $y=(v_0+w_2)/2$.  
The plane
$$P  =\op{aff}\{v_0,v_1,w_1\}$$ separates $x$ from $y$.  The point $z$
of intersection of $\op{conv}\{x,y\}$ with $P$ lies in
   $$R_2 \cap \op{aff}_+^0(v_0,\{v_1,w_1\}),$$
as claimed by the lemma.
\swallowed\end{proved}
\end{tarski}
%<<<<<










%>>>>>
\begin{tarski}
\section{Decoupling}
\name{decouple}
\summary{If a blade meets a segment, then under suitable conditions, the corresponding triangle also meets the segment.}
%\endnote{Five points, a segment meets a blade, then a segment meets the convex hull of the three points giving blade, tarski:decouple}
\tag{pt5, aff-meet, conv2, blade, t0}
\rating{80}
\guid{UMMNOJN}

\begin{lemma}\tlabel{tarski:decouple}
\FIXX{Revisit this lemma.  Move it right after 
 [CKLAKTB], Lemma~\tref{tarski:decouple-orient}.  Keep it together with tarski:back that follows.}
Let $S=\{x,w,v_0,v_1,v_2\}$ be a set of five points
in $\ring{R}^3$.  Assume that
  $\op{conv}\{x,w\}$ meets $\op{cone}(v_0,\{v_1,v_2\})$.  
Assume that
  $$
  |x-v_0| < |x-v_1|,\quad 
  |x-v_0| < |x-v_2|,\quad
  |x-w| < |x-v_0|
  $$
Assume that 
  $$
  2\le |v_i-v_j|\le \sqrt{8}, \text{ for } 0\le i < j \le 2.
  $$
Assume that at most one of $|v_i-v_j|$ is greater than $2.51$.
Then
  $$|w-v_i|\le 2.51,\text{ for } i=1,2,3$$
and $\op{conv}\{v_0,v_1,v_2\}$ meets $\op{conv}(x,w)$.
\end{lemma}


\begin{proved}
By Lemma~\tref{tarski:decouple-orient}, $w$ has negative orientation
in $Q=\{w,v_0,v_1,v_2\}$.
By
Lemmas~\tref{tarski:neg-orient-quad} and \tref{tarski:neg-orient-tet},
we have
  $$|w-v_i|\le 2.51,\text{ for } i=1,2,3.$$

We adopt the notation $H_i$ from the proof of
Lemma~\tref{tarski:decouple-orient}.
Let $c$ be the circumcenter of the triangle $F=\{v_0,v_1,v_2\}$ and
let $c_2$ be the circumcenter of the simplex $\{v_0,v_1,v_2,w\}$.
Let $C=\op{conv}\{v_0,v_1/2,v_2/2,c,c_2\}$.  The set
$C$ contains the set of points separated from $w$ by the
half-plane $H_3$, closer to $w$ than to $v_0$, and closer to $v_0$
than both $v_1$ and $v_2$. The point $x$ lies in
$C$. Since $C$ is nonempty, the simplex $S$ has
negative orientation along the face $\{v_0,v_1,v_2\}$.

Assume for a contradiction that $\op{conv}\{v_0,v_1,v_2\}$
does not meet $\op{conv}(x,w)$.
The set $C'$ of points $y\in C$ such that 
  $\op{conv}(F)$ does not meet $\op{conv}\{y,w\}$ is thus
nonempty. The set $C'$ must include the extreme point $c_2$ of
$C$. This means that the plane $\{w,v_1,v_2\}$ separates $c_2$
from $v_0$, so that the simplex $Q$ has negative orientation
also along the face $\{w,v_1,v_2\}$.  This contradicts
Lemma~\tref{tarski:at-most-one-negative}.
\swallowed\end{proved}
\end{tarski}
%<<<<<






% DCG Lemma 5.31, page52.
%>>>>>
\begin{tarski}
\name{back}
\summary{This is not yet expressed in Tarski's language.  It asserts the existence of an obstructed vertex.}
%\endnote{Corollary to Decoupling Lemma. tarski:back}
\tag{pt4, circum3, circum4, conv5, halfspace, trisegmeet}
\rating{80}
\guid{BIQFYVG}

\begin{lemma}\tlabel{tarski:back}\usage{DCG-[5.31]{ lemma:back}}
\FIXX{Revisit}
We draw out a simple consequence of the proof.
\FIXX{Write this as a separate lemma.} 
Let
$F=\{v_0,v_1,v_2\}$ with edges of length between $2$ and $\sqrt8$.
Let $S=\{v_0,w,v_1,v_2\}$, and assume that $S$ has negative
orientation along $F$. Let $c$ be the circumcenter of the triangle
$F=\{v_0,v_1,v_2\}$ and let $c_2$ be the circumcenter of the simplex
$\{v_0,v_1,v_2,w\}$. Let
$C=\op{conv}\{v_0,v_1/2,v_2/2,c,c_2\}$.  The set $C$ contains the set of points
separated from $w$ by the half-plane $H_3$, closer to $w$ than to
$0$, and closer to $v_0$ than both $v_1$ and $v_2$. Let $x\in C$.
In this context, $w$ is obstructed at $x$.
\end{lemma}

\begin{proved} This is what the final paragraph of the  proof of Lemma~\ref{tarski:decouple} proves by contradiction.
\swallowed\end{proved}
\end{tarski}
%<<<<<



















%>>>>>
\begin{tarski}
\section{Nonconvex quadrilateral}
\name{quad-types}
\summary{All possible sign combinations of barycentric coordinates can be interpreted geometrically in terms
of incidence relations of blades, cones, and convex hulls.}
\tag{pt5, cone, blade, aff-meet, bary4, inside, conv4}
\rating{80}
\guid{MPJEZGP}

\begin{lemma}\tlabel{tarski:quad-types}
Let $S=\{v_0,v_1,v_2,v_3,v_4\}$ be a set of five points in
$\ring{R}^3$.
Assume that
   $$
   t_0 v_0 + t_1 v_1 + t_2 v_2 + t_3 v_3 + t_4 v_4 = 0,
   $$
where $t_4 < 0$ and $t_0+t_1+t_2+t_3+t_4=0$.
Then
   $$
   \def\lr{\Leftrightarrow}
   \def\m{ \text{ meets } }
   \def\cc{\op{cone}}
   \begin{array}{llcl}
  t_1,t_2,t_3\ge 0  &      &\lr& v_4\in\cc(v_0,\{v_1,v_2,v_3\})\\
  t_1,t_3\ge 0 & t_2\le 0  &\lr& \cc(v_0,\{v_2,v_4\})\m\cc(v_0,\{v_1,v_3\})\\
  t_1,t_2\ge 0 & t_3\le 0  &\lr& \cc(v_0,\{v_3,v_4\})\m\cc(v_0,\{v_1,v_2\})\\
  t_2,t_3\ge 0 & t_1\le 0  &\lr& \cc(v_0,\{v_1,v_4\})\m\cc(v_0,\{v_2,v_3\})\\
  t_1\ge0     & t_2,t_3\le0&\lr& v_1\in\cc(v_0,\{v_2,v_3,v_4\})\\
  t_2\ge0     & t_1,t_3\le0&\lr& v_2\in\cc(v_0,\{v_1,v_3,v_4\})\\
  t_3\ge0     & t_1,t_2\le0&\lr& v_3\in\cc(v_0,\{v_1,v_2,v_3\})\\
  t_0\ge0  &t_1,t_2,t_3\le0&\lr& v_0\in\op{conv}\{v_1,v_2,v_3,v_4\}\\
  %&t_0,t_1,t_2,t_3\le0&\lr& \text{ convex position }
   \end{array}
   $$
\end{lemma}

\begin{proved}
The follow immediately from the definitions.
Note that the cases exhaust the complement of $t_1,t_2,t_3,t_4\le 0$.
\swallowed\end{proved}
\end{tarski}
%<<<<<







%% These are not in DCG.  They clarify what happens for nonconvex quads.

%>>>>>
\begin{tarski}
\name{d4}
\summary{If the edges of a simplex are suitably constrained, a partial derivative of $\Delta$ must be positive.}
\tag{pt4, packing, cm4-delta, 2.65, 2.517, poly-ineq}
\rating{80}
\guid{GFVQUPP}

\begin{lemma}\tlabel{tarski:d4}
Let $S=\{v_0,v_1,v_2,v_3\}$ be a packing of four
points.
Let $x_{ij}=|v_i-v_j|^2$.
 Suppose that
$$
x_{01},x_{02},x_{03},x_{23},x_{13}\le 2.517^2,\quad x_{12}\ge 2.65^2.
$$
Then $\Delta_{(23)}(x_{01},x_{02},x_{03},x_{23},x_{13},x_{12})>0$.
\end{lemma}

\begin{proved} We minimize over the given domain.
The partial derivative with respect to $x_{23}$ is negative.
Set $x_{23}=2.157^2$.  Continue with partial derivatives with
respect to $x_{02}$, $x_{03}$, $x_{12}$, $x_{13}$, $x_{01}$, in that order.
The minimum is
$$\Delta_{(23)}(2.517^2,4,4,2.517^2,4,2.65^2) >0.$$ 
\swallowed\end{proved}
\end{tarski}
%<<<<<





%>>>>>
\begin{tarski}
\name{d4bis}
\summary{If the edges of a simplex are suitably constrained, the partial derivative of $\Delta$ on that simplex is positive.}
\tag{pt4, t0, 2.3, 2.517, packing, cm4-delta, poly-ineq}
\rating{80}
\guid{TFKALQL}

\begin{lemma}\tlabel{tarski:d4bis}
Let $S=\{v_0,v_1,v_2,v_3\}$ be a packing of four
points. 
Let $x_{ij}=|v_i-v_j|^2$.
 Suppose that
$$
x_{01}\le 2.3^2,\
\quad x_{02},x_{03},x_{23},x_{13}\le 2.517^2,\quad x_{12}\ge 2.51^2.
$$
Then $\Delta_{(23)}(x_{01},x_{02},x_{03},x_{23},x_{13},x_{12})>0$.
\end{lemma}

\begin{proved}
We minimize over the given domain.
The partial derivative with respect to $x_{23}$ is negative.
Set $x_{23}=2.517^2$.  Continue with partial derivatives with
respect to $x_{02}$, $x_{03}$, $x_{12}$, $x_{13}$, $x_{01}$, in that order.
The minimum is
$$\Delta_{(23)}(2.3^2,4,4,2.517^2,4,2.51^2) >0.$$ 
\swallowed\end{proved}
\end{tarski}
%<<<<<





%>>>>>
\begin{tarski}
\name{flat-then-convexq}
\summary{If a point of a packing lies in a cone, then under suitable constraints, it is near a generator of the cone.}
\tag{pt5, packing, t0, cone, 2.65, cm4-delta, dih}
\rating{60}
\guid{AAGNQFL}

\begin{lemma}\tlabel{tarski:flat-then-convexq}
Let $S=\{v_0,v_1,v_2,v_3,v_4\}$ be a packing of
five points.  Suppose that the pairwise
distances are at most $2.51$ for distinct $u,v\in S$, with
$\{u,v\}\ne \{v_1,v_3\}, \{v_2,v_4\}$.
Suppose that $v_1\in\op{cone}(v_0,\{v_2,v_3,v_4\})$.
Then $|v_1-v_3|< 2.65$.  
\end{lemma}

\begin{proved}
We have $\dih_V(\{v_0,v_1\},\{v_j,v_3\})>\pi/2$ for $j=2,4$.
  By the formula for the dihedral angle, this
implies that $\Delta_{(23)}<0$.  This is contrary to Lemma~\tref{tarski:d4}.
\swallowed\end{proved}
\end{tarski}
%<<<<<




%The Lemma~\tref{tarski:convex-quad} is also relevant. Combined
%with these results it gives that a point of ht <= sqrt8 can't
%be enclosed over a non-convex quad.



%>>>>>
\begin{tarski}
\name{inside-cone}
\summary{If a point lies in a cone, then under certain length constraints, it is close to a generator of the cone.}
\tag{pt5, 2.3, t0, cone, inside }
\rating{60}
\guid{QOKQFRE}

\begin{lemma}
Let $S=\{v_0,v_1,v_2,v_3,v_4\}$ be a packing of
five points.  Suppose that the pairwise
distances are at most $2.51$ for distinct $u,v\in S$, with
$\{u,v\}\ne \{v_1,v_3\}, \{v_2,v_4\}$.  Suppose that $|v_0-v_1|\le 2.3$.
Suppose that $v_1\in\op{cone}(v_0,\{v_2,v_3,v_4\})$.
Then $|v_1-v_3|< 2.51$.  
\end{lemma}

\begin{proved}
We have $\dih_V(\{v_0,v_1\},\{v_j,v_3\})>\pi/2$ for $j=2,4$.
  By the formula for the dihedral angle, this
implies that $\Delta_{(23)}<0$.  This is contrary to Lemma~\tref{tarski:d4bis}.
\swallowed\end{proved}
\end{tarski}
%<<<<<









%>>>>>
\begin{tarski}
\section{Convex quadrilateral}
\name{def:convex-quad}
\summary{A convex quad cluster can be expressed in Tarski arithmetic.}
\tag{def, pt5, quadp, quadc, cone, blade, aff-meet}
\rating{0}
\guid{SVMHNRP}

\begin{definition}[convex~quad,~quadp]\tlabel{def:convex-quad}
\indy{Index}{convex quad}
\indy{Index}{quadp}
\indy{Index}{quadc}
We say that property $\op{quadp}(v_0,(v_1,v_2,v_3,v_4))$  holds if
$\op{cone}^0(v_0,\{v_1,v_3\})$ meets $\op{cone}^0(v_0,\{v_2,v_4\})$.
In this case, we let 
 $$
  \begin{array}{lll}
   \op{quadc}^0(v_0,(v_1,v_2,v_3,v_4))&= 
    \op{cone}^0(v_0,\{v_1,v_2,v_3\})  \\
    &\cup \ \op{cone}^0(v_0,\{v_1,v_3\})\ \cup \ 
   \op{cone}^0(v_0,\{v_1,v_4,v_3\}).
  \end{array}
  $$
\end{definition}
It follows immediately from definitions that
  $$\op{quadp}(v_0,(v_1,v_2,v_3,v_4))=\op{quadp}(v_0,(v_2,v_3,v_4,v_1)).$$
\end{tarski}
%<<<<<





%>>>>>
\begin{tarski}
\name{quad-cluster-triangulate}
\summary{The cone
over a quad cluster is independent of the choice of triangulation.}
\tag{pt5, quadp, quadc, cone}
\rating{80}
\guid{DFLUMBW}

\begin{lemma}
% qcone well-defined.
Let $\{v_0,v_1,v_2,v_3,v_4\}$ be a set of four points in $\ring{R}^3$.
Assume that $\op{quadp}(v_0,(v_1,v_2,v_3,v_4))$ holds.
Then 
 $$\op{quadc}^0(v_0,(v_1,v_2,v_3,v_4))  
  = \op{quadc}^0(v_0,(v_2,v_3,v_4,v_1)).
 $$
\end{lemma}

\begin{proved}
By the symmetry in the statement, it is enough to prove 
  $$\op{cone}^0(v_0,\{v_1,v_2,v_3\})\cup\op{cone}^0(v_0,\{v_1,v_3\})
  \subset \op{quadc}^0(v_0,(v_2,v_3,v_4,v_1)).$$
Let $x$ belong to the left-hand side:
  $$
  x = a_0 v_0 + a_1 v_1 + a_2 v_2 + a_3 v_3,
  \quad a_1,a_3 > 0,\quad a_2\ge 0,\quad a_0+a_1+a_2+a_3=1.
  $$
By the definition of $\op{quadp}$ we have
$$
  s_0 v_0 + s_1 v_1 + s_3 v_3 = t_0 v_0 + t_2 v_2 + t_4 v_4,
$$
with $s_0 + s_1 + s_3 = t_0 + t_2 + t_4 = 1$, $s_1,s_3,t_2,t_4>0$.
If $a_1 s_3 - a_3 s_1 \ge 0$, then
  $$s_3 x = (s_3 a_0 + a_3 t_0 - a_3 s_0) v_0 + 
          (s_3 a_1 - a_3 s_1) v_1 + (s_2 a_2 + a_3 t_2) v_2 +
          (s_3 a_3 + a_3 t_4) v_4,$$
which presents $x$ as an element of 
 $$
 \op{cone}^0(v_0,\{v_2,v_4,v_1\}) \cup\op{cone}^0(v_0,\{v_2,v_4\}).
 $$
Similarly, if $a_1 s_3 - a_3 s_1 < 0$, then
  $$
  x\in \op{cone}^0(v_0,\{v_2,v_4,v_3\}).
  $$
The result follows.
\swallowed\end{proved}
\end{tarski}
%<<<<<






%>>>>>
\begin{tarski}
\name{skew-quad}
\summary{In an arrangement of five ponts, if the two diagonals have length less than $\sqrt8$, then a quad cluster forms.}
%\endnote{Five points, Quad, two diagonals of length at most $\sqrt8$ cross, then you have a quad cluster. tarski:skew-quad}
\tag{pt5, packing, t0, quadp, sqrt2}
\rating{80}
\guid{HTYDGWI}

\begin{lemma}  \tlabel{tarski:skew-quad}\usage{DCG-[4.30]{ lemma:skew-quad}}
%\proclaim{Lemma 1.6}
Let $\{v_0,v_1,v_2,v_3,v_4\}$ be a packing of five points.
Suppose that $|v_0-v_i|\le 2.51$, for $i=1,2,3,4$.
Suppose $\op{quadp}(v_0,(v_1,v_2,v_3,v_4))$.
Suppose that
  $$|v_1-v_3|<\sqrt8,\quad |v_2-v_4| <\sqrt8.$$
Then $|v_i-v_j|\le 2.51$, for $(i,j)=(1,2),(2,3),(3,4),(4,1)$.
\end{lemma}

%\begin{lem}\guid{XMSECYQ}
%Suppose that there exist four nonzero vertices $v_1,\ldots,v_4$ of
%height at most $2.51$ (that is, $|v_i-v_0|\le 2.51$) forming a skew
%quadrilateral. Suppose that the diagonals $\{v_1,v_3\}$ and
%$\{v_2,v_4\}$ have lengths between $2.51$ and $\sqr8$. Suppose the
%diagonals $\{v_1,v_3\}$ and $\{v_2,v_4\}$ cross. Then the four
%vertices are the corners of an adjacent pair of quarters with base
%point at the origin.
%\end{lemma}

\begin{proved}
Up to symmetry (Lemma~\tref{tarski:over-under}), we may assume
that $\op{conv}\{v_1,v_3\}$ meets $\op{conv}\{v_0,v_2,v_4\}$.
The calculation
   $$
   E(2,2,2,2.51,\sqrt8,2.51,2.51,2,2; \sqrt8)
   $$
gives $|v_1-v_2|\le 2.51$. The other inequalities are similar.
\swallowed\end{proved}
\end{tarski}
%<<<<<





%>>>>>
\begin{tarski}
\name{oct-t0}
\summary{If a vertex of height less than $\sqrt8$ is enclosed over a quad cluster having both diagonals of length at most $\sqrt8$, then an octahedron is formed.}
\tag{pt6, packing, quadp, quadc, t0, sqrt2}
\rating{65}
\guid{XLHACRX}

\begin{lemma} \tlabel{tarski:oct-t0}\usage{DCG-[4.31]{ lemma:oct}}
%\proclaim{Lemma 1.7}
Let $\{v_0,v_1,v_2,v_3,v_4,w\}$ be a packing of six points.
Suppose that $|v_0-v_i|\le 2.51$, for $i=1,2,3,4$.
Suppose $\op{quadp}(v_0,(v_1,v_2,v_3,v_4))$.
Suppose that
  $$|v_1-v_3|<\sqrt8,\quad |v_2-v_4| <\sqrt8.$$
Assume $w\in\op{quadc}(v_0,(v_1,v_2,v_3,v_4))$ and that $|w-v_0|<\sqrt8$.
Then $|w-v_i|\le 2.51$, for $i=1,2,3,4$.
\end{lemma}


\begin{proved}
If, say, $w\in \op{cone}(v_0,\{v_1,v_2,v_3\})$, then
$|w-v_1|$, $|w-v_3|\le 2.51$ (Lemma~\tref{tarski:pass-anchor}).
 Similarly, the distance from
$w$ to the other two corners is at most $2.51$.
\swallowed\end{proved}
\end{tarski}
%<<<<<







%>>>>>
\begin{tarski}
\section{Quad enclosed}
\name{gc}
\summary{If a point  enclosed over a quad cluster satisfies various distance constraints, then a more rigid configuration
can be produced.}
\tag{pt6, quadp, quadc, t0, sqrt2}
\rating{400}
\guid{ZTFABJM}

\begin{lemma}\tlabel{tarski:gc}\usage{DCG-[10.12]{ lemma:gc}}
\FIXX{State the general case, if it is needed.}
Let $\{v_0,v_1,v_2,v_3,v_4,v\}$ be a set of six points in
$\ring{R}^3$.  Assume $\op{quadp}(v_0,(v_1,v_2,v_3,v_4))$.
Assume $v\in \op{quadc}(v_0,(v_1,v_2,v_3,v_4))$.
Assume 
    $$\begin{array}{lll}
    2\le &|v_i-v_0|\le 2.51,\\
    2\le&|v_i-v_{i+1}|\le 2.51, \\
    2\le&|v_i-v_{i+2}|,\\
    h_i\le &|v-v_i|, \\
    2\le &|v-v_0|\le 2.51, \hbox{ for }
        i=1,\ldots,4 \ (\hbox{mod } 4)\\
    \end{array}
    $$
where $h_i$ are fixed constants that satisfy
$h_i\in[2,\sqrt{8})$.  
Then another figure exists made of
a collection of vectors $w_0,w_1,\ldots,w_4$ and $w$ subject to
the constraints as given above together with the additional constraints
    $$\begin{array}{lll}
    &|w_i-w_{i+1}|=2.51\\
    &|w_0-w_i|=2, \hbox { for } i=1,\ldots,4,\\
    &|w_0-w|=2.51.
    \end{array}
    $$
\end{lemma}

\begin{proved} This  lemma is a special case of
\cite[Lemma~4.3]{part1}.
\swallowed\end{proved}
\end{tarski}
%<<<<<





% Lemma 2.2.
%>>>>>
\begin{tarski}
\name{enclosed}
\summary{A point enclosed over a quad cluster, with at most one corner as anchor, has height greater than $2.51$.}
\tag{pt6, quadp, quadc, t0}
\rating{180}
\guid{VTIVSIF}

\begin{lemma}\tlabel{tarski:enclosed}\usage{DCG-[10.13]{ lemma:enclosed}}
Let $\{v_0,v_1,v_2,v_3,v_4,v_5\}$ be a packing of six points.
Suppose that 
  $$\op{quadp}(v_0,(v_1,v_2,v_3,v_4)) \text{ and }
   v_5\in \op{quadc}(v_0,(v_1,v_2,v_3,v_4)).$$
Suppose that $|u-v|\le 2.51$ for
  $$(u,v)=(v_0,v_1), (v_0,v_2), (v_0,v_3), (v_0,v_4), 
  (v_1,v_2), (v_2,v_3), (v_3,v_4), (v_4,v_1).$$
Suppose that
$|v_5-v_i|\le 2.51$ for at most one $i\in\{1,2,3,4\}$.  Then
$|v_0-v_5| > 2.51$.
\end{lemma}

%A quadrilateral component does not enclose any vertices of height at
%most $2.51$.


\begin{proved} Assume for a contradiction that $|v_0-v_5|\le2.51$>
We apply Lemma~\tref{tarski:gc} to assume
    $$|v_i-v_{i+1}|=2.51,\quad |v_i-v_0|=2, \quad |v-v_0|=2.51,$$
for $i=1,\ldots,4$. Reindexing and perturbing $v_5$ as necessary, we
may assume that $2\le |v_1-v_5|\le2.51$ and $|v_i-v_5|\ge2.51$, for
$i=2,3,4$.  Pivoting $v_5$ around appropriate axes, 
we may assume it reaches the minimal
distance to two adjacent corners ($2$ for $v_1$ or $2.51$ for
$v_i$, $i>1$).  Keeping $v_5$ fixed at this minimal distance,
perturb $\{v_1,v_2,v_3,v_4\}$ along its remaining degree of freedom
until $v_5$ attains its minimal distance to three of the corners.
This is a rigid figure.  There are four possibilities depending on
which three corners are chosen. Pick coordinates to show that the
distance from $v_5$ to the remaining point violates its inequality.
\swallowed\end{proved}
\end{tarski}
%<<<<<





%% DCG PENTAGON-TRIANGLE STUFF.
%>>>>>
\begin{tarski}
\name{4circuit}
\summary{A point enclosed over a quad cluster has height at least $2.3$ and cannot
be too far from the corners.}
\tag{pt6, t0, packing, quadp, quadc, sqrt2, 3.02, 2.3}
\rating{160}
\guid{TPXUMUZ}

\begin{lemma}\tlabel{tarski:4circuit}
Let $S=\{v_0,v_1,v_2,v_3,v_4,v_5\}$ be a packing of six points.  
Suppose that $|u-v|\le 2.51$, for all $u,v\in S$, 
except
when $\{u,v\}=\{v_1,v_3\}$, $\{v_2,v_4\}$, $\{v_5,v_4\}$,
$\{v_5,v_3\}$.  
Assume that $|v_3-v_5|,|v_4-v_5|>2.51$.
Suppose that $\op{quadp}(v_0,(v_1,v_2,v_3,v_4))$
and $v_5\in\op{quadc}^0(v_0,(v_1,v_2,v_3,v_4))$.
Then
\begin{itemize}
  \item At least one of $|v_5-v_4|$ $|v_5-v_3|$ is at most $\sqrt8$.
    \item Both are at most $3.02$.
  \item $|v_0-v_5|\ge 2.3$.
\end{itemize}
\end{lemma}

\begin{proved}
This is a standard exercise in constraint preserving deformations.
We deform the figure using pivots to a configuration
$v_1,\ldots,v_4$ at height $2$, and $|v_i-v_j|=2.51$,
$(i,j)=(1,2),(2,3),(3,4),(4,1)$. (See Lemma~\tref{tarski:gc}.)
We scale $v_5$ until $|v_5-v_0|=2.51
$. We can also take the distance from $v_5$ to $v_2$ and to $v_1$
to be $2$. If we have $|v_5-v_3|\ge \sqrt{8}$, then we stretch
the edge $|v_5-v_4|$ until $|v_5-v_3|=\sqrt{8}$. The resulting
configuration is rigid.  Pick coordinates to find that
$|v_5-v_4|<\sqrt{8}$. If we have $|v_5-v_3|\ge 2.51 $, follow a
similar procedure to reduce to the rigid configuration
$|v_5-v_3|=2.51$, to find that $|v_5-v_4|<3.02$. The estimate
$|v_5-v_0|\ge2.3$ is similar.
\swallowed\end{proved}
\end{tarski}
%<<<<<












%>>>>>
\begin{tarski}
\section{Miscellaneous}

\name{double-face}
\summary{If a segment passes through the convex hull of two anchor triangles along the same diagonal, then the segment has length greater than $\sqrt8$.}
%\endnote{Six points, if a segment pases through two anchors, then it is longer than $\sqrt8$. tarski:double-face}
\tag{pt6, packing, sqrt2, t0, anchor, conv2, conv3, conv3}
\rating{160}
\guid{YWPHYZU}

\begin{lemma} \tlabel{tarski:double-face}\usage{DCG-[4.32]{ lemma:double-face}}
%proclaim{Lemma 1.8}
Let $\{v_0,v_1,v_2,v_3,v_4,v_5\}$ be a packing of six points.
Assume that $|v_0-v_5|<\sqrt8$ and that
  $$|u-v| \le 2.51,\quad u=v_0,v_5,\quad v = v_1,v_2$$
If $\op{conv}\{v_3,v_4\}$ meets 
$\op{conv}\{v_0,v_5,v_1\}$ and $\op{conv}\{v_0,v_5,v_2\}$,
then $|v_3-v_4|>\sqrt8$.
\end{lemma}

\begin{proved}
Suppose the figure exists with $|v_3-v_4|\le\sqr8$. Label vertices
so $v_3$ lies on the same side of the figure as $v_1$. Contract
$\{v_3,v_4\}$ by moving $v_3$ and $v_4$ until
    $\{v_i,u\}$ has length $2$,
for $u=v_0,v_5,v_{i-2}$, and $i=3,4$. Pivot $v_5$ away from $v_3$ and
$v_4$ around the axis $\{v_1,v_2\}$ until
    $|v_5-v_0|=\sqr8$.
Contract $\{v_3,v_4\}$ again. By stretching $\{v_1,v_2\}$, we
obtain a square of edge two and vertices $\{v_0,v_3,v_5,v_4\}$. Short
calculations based on explicit formulas for the dihedral angle and
its partial derivatives give
    $$
        \dih(\sqr8,2,y_3,2,y_5,2) > 1.075,\quad
        y_3,y_5\in[2,2.51],
    $$
    %
    $$
    \dih(\sqr8,y_2,y_3,2,y_5,y_6) >1,\quad
        y_2,y_3,y_5,y_6\in[2,2.51].
   $$
Then
$$\pi\ge \dih_V(\{v_0,v_5\},\{v_3,v_1\}) + \dih_V(\{v_0,v_5\},\{v_1,v_2\}) + 
\dih_V(\{v_0,v_5\},\{v_2,v_4\})
    > 1.075 + 1 + 1.075 > \pi .$$
Therefore, the figure does not exist.
\swallowed\end{proved}
\end{tarski}
%<<<<<










%>>>>>
\begin{tarski}
\section{Miscellaneous (continued)}
\name{single-enclosed}
\summary{In a suitably constrained packing, two segments with a common endpoint cannot both meet the convex hull of an anchor triangle.}
\tag{pt6, t0, packing, sqrt2, cone, anchor}
\rating{80}
\guid{PYURAKS}

\begin{lemma} \tlabel{tarski:single-enclosed}\usage{DCG-[4.33]{ lemma:single-enclosed}}
%proclaim{Lemma 1.11}
Let $\{v_0,w,w',v_1,v_2,v_3\}$ be a packing of six points.
Assume that  $|v_1-v_2|\le\sqrt8$,
$|v_1-v_3|\le 2.51$, and $|v_2-v_3|\le 2.51$.
Assume that $w,w'\in\op{cone}(v_0,\{v_1,v_2,v_3\})$.
Assume that $|v_0-w|\le\sqrt8$.  Then $|v_0-w'| > \sqrt8$.
\end{lemma}

\begin{proved}
For a contradiction, assume $|v_0-w'|\le\sqrt8$.  By
Lemma~\tref{tarski:qrtet-pair-pass}, we have $|v_1-v_2|\ge 2.51$.
By Lemma~\tref{XX},\FIXX{$\CalE$}
we have $|v_i-u|\le 2.51$, for
$i=1,2$ and $u=v_0,v_3,w,w'$.  
We do not have $w,w'\in\op{conv}\{v_0,v_1,v_2,v_3\}$ by Lemma~\tref{XX}.  Either 
$\op{conv}(w,v_0)$ meets $\op{aff}_+(\{v_1,v_2\},w')$ or
$\op{conv}(w',v_0)$ meets $\op{aff}_+(\{v_1,v_2\},w)$.
Without loss of generality, we assume the latter case.
We have that $\op{conv}\{w',v_0\}$ does not meet $\op{conv}\{w,v_1,v_2\}$
by Lemma~\tref{XX},%%note:136 
and it already meets $\op{conv}\{v_1,v_2,v_3\}$.
We have that $w\not\in\op{cone}(v_0,\{w',v,v_2\})$, because this
gives $w\in \op{conv}(v_0,v_1,v_2,w')$.  This implies that
$\op{cone}(v_0,\{w',v_i\})$ meets $\op{cone}^0(v_0,\{w,v_i,v_j\})$
for $(i,j)=(1,2)$ or $(i,j)=(2,1)$.  Without loss of generality,
we may assume that $i=1$ and $j=2$.   This implies that
$\op{conv}(w,v_2)$ meets $\op{conv}\{v_0,v_1,w'\}$ or that
$\op{conv}(w',v_1)$ meets $\op{conv}\{v_0,v_2,w\}$.  Both
are impossible by Lemma~\tref{XX}.\FIXX{$\CalE$}
%
% OLD PROOF:
%The diagonal $(v_1,v_2)$ has anchors
%$\{v_0,v_3,w,w'\}$. Assume that the cyclic order of vertices around
%the line $\{v_1,v_2\}$ is $v_0,v_3,w,w'$. We see that $\op{conv}^0\{v_1,w\}$ is
%too short to meet $\op{conv}\{v_0,v_2,w'\}$, and $w$ is not in
%$\op{conv}\{v_0,v_1,v_2,w'\}$.  Thus, 
%$\op{conv}(v_1,\{v_2,w\})$ meets $\op{conv}(v_1,\{v_0,w'\})$.
%Then $\op{conv}\{v_0,w'\}$ meets
%$\op{conv}\{v_1,v_2,w\}$, 
%or $\op{conv}\{v_2,w\}$ meets $\op{conv}\{v_1,v_0,w'\}$.
%But $\op{conv}\{v_2,w\}$ is too short to meet $\op{conv}\{v_1,v_0,w'\}$.
%Thus, $\op{conv}\{v_0,w'\}$ meets booth $\op{conv}\{v_1,v_2,w\}$ and
%$\op{conv}\{v_1,v_2,v_3\}$. Lemma~\tref{tarski:double-face} gives the
%contradiction $|v_0-w'|>\sqrt8$.
\swallowed\end{proved}
\end{tarski}
%<<<<<















%>>>>>
\begin{tarski}
\section{Octahedra}
The following definition will be used for the next few lemmas.
\name{def:tarski:special}
\summary{A set of three points is said to be special under certain restrictive conditions on edge lengths and circumradius.}
\tag{def,special,pt3,t0,sqrt2,eta}
\rating{0}
\guid{ESRFPRV}

\begin{definition}[special]\tlabel{def:tarski:special} 
\indy{Index}{special}
Let $S=\{v_0,v_1,v_2\}$ be a packing of three
points.  We say that $S$ is special, if
the following conditions hold.  
\begin{itemize}
  \item $|v_i-v_j|<\sqrt8$, for
$0\le i< j \le 2$.  
   \item If $2.51 < |v_i-v_j|$ and $2.51 < |v_j-v_k|$ ,
   where $\{i,j,k\} = \{0,1,2\}$, then
  the circumradius of $\{v_0,v_1,v_2\}$ is less than $\sqrt2$.
  \end{itemize}
\end{definition}
\end{tarski}
%<<<<<



%>>>>>
\begin{tarski}
\name{lin-sim-6-not}
\summary{In a suitably constrained packing, a segment does not meet the interior of a simplex.}
\tag{pt6, packing, sqrt2, t0, conv2, conv4}
\rating{200}
\guid{MLTDJJV}

\begin{lemma}\tlabel{tarski:lin-sim-6-not}
% Line and simplex.  % NEW LEMMA NOT IN DCG.
% Case where there are not opposite long edges in simplex.
Let $\{v_0,v_1,v_2,v_3,w,w'\}$ be a packing of six points. Suppose that
$|w-w'| < \sqrt8$.  Suppose that each triple $\{v_i,v_j,v_k\}$
in $\{v_0,v_1,v_2,v_3\}$ is special. 
Suppose   %% rule out octahedron.
   $$
   |v_1-v_2|,|v_2-v_3|,|v_1-v_3|\le 2.51.
   $$
Then $I=\op{conv}\{w,w'\}$ does not meet $C=\op{conv}^0\{v_0,v_1,v_2,v_3\}$.
%Then $2.51 \le |w-w'|$ and
%  $$
%  |u-v_i|\le 2.51,\quad u=w,w',\quad i=0,1,2,3.
%  $$
%Furthermore, there is a reindexing $\{w_0,w_1,w_2,w_3\} =\{v_0,v_1,v_2,v_3\}$
%such that
%  $$
%  |w_i-w_j|\le 2.51,\quad (i,j)=(1,2),(2,3),(3,4),(4,1),
%  $$
%and $2.51 < |w_1-w_3|, |w_2-w_4|$.
\end{lemma}

%(That, is the figure is an octahedron.)

\begin{proved}  Assume for a contradiction that the sets meet at $x$.
We cannot have $w,w'\in C$ by
Lemma~\tref{tarski:v-interior}.  So the endpoints $w,w'$ of $I$
lie outside $C$, and  $x\in I\cap C$.
The segment $I$ does not meet an edge $\op{conv}\{v_i,v_j\}$
by Lemma~\tref{tarski:cross}.  So the segment $I$ meets
two different faces of $C$, one on entry and another on exit.
By the definition of special face, each of these two faces has at most
one edge of length greater than $2.51$.  

Suppose that the edge $\{w_1,w_2\}$
shared by these two faces has length greater
than $2.51$.  By Lemma~\tref{tarski:double-face}, 
we reach a contradiction $|w-w'|>\sqrt8$.
So the edge $\{w_1,w_2\}$ has length at most $2.51$.

Let $\{w_3,w_4\} = \{v_0,v_1,v_2,v_3\}\setminus \{w_1,w_2\}$ be the
edge opposite the shared edge.  We can decrease $|w-w'|$, contracting
along $\{w_3,w_4\}$ until $|w_3-w_4|=2$.  The assumptions now allow
us to label the vertices so that $\{w_3,w_4\}=\{v_1,v_2\}$, and so
that
$\{v_0,v_3\}$ is the shared edge (as we see by running through
all possibilities).  Increase the dihedral angle
of the simplex along $\{v_1,v_2\}$ until $|v_0-v_3|=2.51$.

Fix the simplex $\{v_0,v_1,v_2,v_3\}$ and contract along $\{w,w'\}$
until (switching the prime with the unprimed if necessary so that) we have
  $$
  |u-v_i| = 2,\quad (u,i)=(w,0),(w,3),(w,1), (w',0), (w',3),(w',2).
  $$
The projections of $w$ and $v_2$ to the plane $A=\op{aff}(v_0,v_3,v_1)$
lie in $\op{conv}^0\{v_0,v_3,v_1\}$ (because the appropriate dihedral
angles are less than $\pi/2$).\FIXX{Check this.}  Hence, we may move
$v_1$ in the plane $A$, preserving constraints,
until $|v_0-v_1|=\sqrt8$ and $|v_1-v_3|=2.51$.
Similarly, we deform until $|v_0-v_2|=\sqrt8$ and $|v_3-v_2|=2.51$.

We arrive at a rigid figure that satisfies all the constraints.
Pick coordinates and compute $|w-w'| > \sqrt8$, a contradiction.
\swallowed\end{proved}
\end{tarski}
%<<<<<







%>>>>>
\begin{tarski}
\name{def:tarski:oct}
\summary{A quasi-regular octahedron is defined.}
\tag{def, t0, quadp, quadc, pt6, conv4, oct}
\rating{0}
\guid{EOAJRDV}

\begin{definition}[octahedron]\tlabel{def:tarski:oct}
\indy{Index}{octahedron}
\indy{Index}{oct}
Let $S=\{v_0,v_1,v_2,v_3,w,w'\}$ be a set of six points in $\ring{R}^3$.
We say that $(v_0,v_1,v_2,v_3)$ forms an octahedron along $\{w,w'\}$ and write
$\op{oct}(\{w,w'\},(v_0,v_1,v_2,v_3))$ if
$$
  \op{quadp}(w,(v_0,v_1,v_2,v_3)),\quad
   w'\in\op{quadc}^0(w,(v_0,v_1,v_2,v_3)),
$$
and $w'\not\in\op{conv}\{w,v_i,v_j,v_k\}$ for any triple
$\{v_i,v_j,v_k\}\subset \{v_0,v_1,v_2,v_3\}$.
If, furthermore, $2\le |u-v_i|\le 2.51$, for $u=w,w'$ and $i=0,1,2,3$,
and $2.51<|w-w'|<\sqrt8$, then we say that 
$(v_0,v_1,v_2,v_3)$ forms a quasi-regular octahedron along
 $\{w,w'\}$ and write
$\op{qroct}(\{w,w'\},(v_0,v_1,v_2,v_3))$.
\indy{Index}{octahedron}\indy{Index}{quasi-regular octahedron}\indy{Index}{oct}\indy{Index}{qroct}
\end{definition}
\end{tarski}
%<<<<<



%>>>>>
\begin{tarski}
\name{oct-quad-cluster}
\summary{Five of the points of a quasi-regular octahedron form a quad cluster.  A vertex of the qr-octahedron does
not lie in the convex hull formed by four other vertices.}
\tag{pt6, quadp, quadc, oct, conv4}
\rating{80}
\guid{WKMPKQS}

\begin{lemma}
Let $S=\{v_0,v_1,v_2,v_3,w,w'\}$ be a set of six points
in $\ring{R}^3$.  Assume that $op{oct}(\{w,w'\},(v_0,v_1,v_2,v_3)$.
Then 
$$
  \op{quadp}(w',(v_0,v_1,v_2,v_3)),\quad
   w'\in\op{quadc}^0(w,(v_0,v_1,v_2,v_3)),
$$
and $w\not\in\op{conv}\{w',v_i,v_j,v_k\}$ for any triple
$\{v_i,v_j,v_k\}\subset \{v_0,v_1,v_2,v_3\}$.
\end{lemma}

\begin{proved}
\FIXX{Insert proof of octahedron symmetry.}
\swallowed\end{proved}
(Thus, the definition of octahedron is symmetric in $w$ and $w'$.)%
\FIXX{Add lemma: 
A quasi-regular octahedron is a union of four disjoint quarters. Continue
with more}
\end{tarski}
%<<<<<




%>>>>>
\begin{tarski}
\section{Line and simplex}
% point and line,
% point and triangle,
% point and simplex,
% line and line, 
% line and triangle, done in CalE.
\name{lin-sim-6}
\summary{In a packing, if segment of length less than $\sqrt8$ meets the interior of a simplex,
then special distance constraints hold.}
\tag{pt6, packing, sqrt2, t0, conv2, conv4}
\rating{80}
\guid{ZDKDXFM}

\begin{lemma}\tlabel{tarski:lin-sim-6}
% Line and simplex, oct case.  % NEW LEMMA NOT IN DCG.
Let $\{v_0,v_1,v_2,v_3,w,w'\}$ be a packing of six points. Suppose that
$|w-w'| < \sqrt8$.  Suppose that each triple $\{v_i,v_j,v_k\}$
in $\{v_0,v_1,v_2,v_3\}$ is special. 
Suppose   
   $$
   2.51 < |v_1-v_3|,|v_2-v_0|.
   $$
Suppose that $I=\op{conv}\{w,w'\}$  meets $C=\op{conv}^0\{v_0,v_1,v_2,v_3\}$.
Then $2.51 < |w-w'|$ and
  $$
  |u-v_i|\le 2.51,\quad u=w,w',\quad i=0,1,2,3.
  $$
\end{lemma}
\FIXX{Eventually, 
we might want to add a furthermore clause: One of the anchors $w,w'$ lies in
the lune $L(v_0,v_2,v_1,v_3)$.  That is, the anchors $v_1$ and $v_3$
are not consecutive.  This is for applications to overlap of type C simplices.
If you first locate the other anchor $w'$, forming two quarters then
this is readily proved.}%
\FIXX{A stronger result actually, holds; the six points are vertices of an
octahedron with three diagaonals of length less than $\sqrt8$ and
edges of length at most $2.51$.}


\begin{proved} As in the proof of Lemma~\tref{tarski:lin-sim-6-not},
the segment $I$ meets
two different faces of $C$, one on entry and another on exit.
By the definition of special, each of these two faces has at most
one edge of length greater than $2.51$.  Furthermore, the edge
shared by these two faces has length less than $2.51$.  We may
label it $\{v_0,v_1\}$, so that $I$ meets $\op{conv}\{v_0,v_1,v_2\}$
and $\op{conv}\{v_0,v_1,v_3\}$.   The circumradius
of each face crossed by $I$ is at least $\sqrt2$.
Since the faces are special,
we have $|v_0-v_1|,|v_1-v_2|,|v_0-v_3|\le 2.51$.  

Applying Lemma~\tref{tarski:pass-anchor}
to both of these faces, we get that 
   $$
   |u-v_i|\le 2.51,\quad u=w,w',\quad i=0,1,2,3.
   $$
We have $|w-w'|\ge 2.51$, by Lemma~\tref{XX}.\FIXX{$\CalE$}
%
%% WW CUT BECAUSE IT RELATES TO LONGER THM STATEMENT.
%Suppose that the edge $\{w_1,w_2\}$
%shared by these two faces has length greater
%than $2.51$.  By Lemma~\tref{tarski:double-face}, 
%we reach a contradiction $|w-w'|>\sqrt8$.
%So the edge $\{w_1,w_2\}$ has length at most $2.51$.
%It is readily checked using the classification of Lemma~\tref{tarski:quad-types},
%that $\op{cone}(v_0,\{v_1,v_3\})$ meets $\op{cone}(v_0,\{v_2.v_4\})$
%(by ruling out the other possibilities).  So
%$\op{quadp}(w,(v_0,v_1,v_2,v_3))$ and similarly,
%$\op{quadp}(w',(v_0,v_1,v_2,v_3))$.
%
%By Lemma~\tref{tarski:skew-quad}, we have that $|v_2-v_3|\le 2.51$.
%This completes the proof.
\swallowed\end{proved}
\end{tarski}
%<<<<<





%>>>>>
\begin{tarski}
\name{lin-sim-5}
\summary{In a suitably constrained packing, if a segment starting from a vertex of a simplex passes into the interior of the simplex, then it also meets the opposite face.}
\tag{pt5, packing, sqrt2, conv4, conv2, conv3}
\rating{60}
\guid{XHMHKIZ}

\begin{lemma}\tlabel{tarski:lin-sim-5}
% New Lemma, not in DCG.
Let $S=\{v_0,v_1,v_2,v_3,w\}$ be a packing of five points.
Assume that 
  $$
  |v_1-v_2|,|v_1-v_3|,|v_2-v_3|\le\sqrt8.
  $$
Assume that $\op{conv}\{v_0,w\}$ meets $\op{conv}^0\{v_0,v_1,v_2,v_3\}$ at $x$.
Then $\op{conv}\{v_0,w\}$ meets $\op{conv}^0\{v_1,v_2,v_3\}$.
\end{lemma}

\begin{proved}  Extend $\op{conv}\{v_0,x\}$ until it meets
$\op{conv}$ of the opposite face at $y$.  The point $y$ must lie between
$v_0$ and $w$, for otherwise $w\in\op{conv}\{v_0,v_1,v_2,v_3\}$,
which is impossible, by Lemma~\tref{XX}.
\swallowed\end{proved}
\end{tarski}
%<<<<<







%>>>>>
\begin{tarski}
\section{Separation triangle triangle}
\name{tri-tri-6}
\summary{In a suitably constrained packing, 
if two convex hulls of three points meet, then a side of each meets the other hull.}
\tag{pt6, packing, t0, conv3, sqrt2, conv2}
\rating{120}
\guid{TIMIDQM}

\begin{lemma}\tlabel{tarski:tri-tri-6}
% New lemma, not in DCG. \tlabel{tarski:two-face}
Let $S=\{v_0,v_1,v_2,w_0,w_1,w_2\}$ be a packing of six points. 
Suppose that the triangles $\{v_0,v_1,v_2\}$ and $\{w_0,w_1,w_2\}$
are special.  Suppose that $\op{conv}\{v_0,v_1,v_2\}$ meets
$\op{conv}\{w_0,w_1,w_2\}$ at $x$.  Then after reindexing
($\{v'_0,v'_1,v'_2\}=\{v_0,v_1,v_2\}$, $\{w'_0,w'_1,w'_2\} = \{w_0,w_1,w_2\}$),
we have 
   $$
   \begin{array}{lll}
   |v'_0-v'_1|\le 2.51,\quad |v'_1-v'_2|\le 2.51,\quad 2.51 < |v'_0-v'_2|<\sqrt8,\\
   |w'_0-w'_1|\le 2.51,\quad |w'_1-w'_2|\le 2.51,\quad 2.51 < |w'_0-w'_2|<\sqrt8.
   \end{array}
   $$
Furthermore, $\op{conv}\{v'_0-v'_2\}$ meets $\op{conv}\{w_0,w_1,w_2\}$
and  $\op{conv}\{w'_0-w'_2\}$ meets $\op{conv}\{v_0,v_1,v_2\}$.
Furthermore, these are the only incidences of edge with face.
\end{lemma}

\begin{proved}
It can be readily checked that $\op{aff}\{v_0,v_1,v_2\}\ne
\op{aff}\{w_0,w_1,w-2\}$.  Also $x\not\in S$.
Let $L$ be the line of intersection.
The line $L$ meets $\op{conv}\{v_0,v_1,v_2\}$ in an interval $I_v$
and $\op{conv}\{w_0,w_1,w_2\}$ in an interval $I_w$.  We have
$x\in I_v\cap I_w$.  Some endpoint $y$ of one of the intervals lies
in $I_v\cap I_w$.  This implies that an edge, say $\op{conv}\{v_0,v_1\}$,
meets the other face $\op{conv}\{w_0,w_1,w_2\}$.  
By Lemma~\tref{no-pass-sqrt2},
the circumradius of $\{w_0,w_1,w_2\}$ is at least $\sqrt2$.  By
the definition of special face, the triangle $\{w_0,w_1,w_2\}$
has at most one edge longer than $2.51$.

By Lemma~\tref{XX}, we have $|v_0-v_1|>2.51$.  
$v_2$ lies in one of the three lunes
$$L(v_0,v_1,w_1,w_2),\quad L(v_0,v_1,w_2,w_3),\quad L(v_0,v_1,w_1,w_3)$$
Say $L(v_0,v_1,w_1,w_2)$.   We have $v_2\not\in\op{cone}(v_0,\{v_1,w_1,w_2\})$
because the edges $\op{conv}^0\{v_0,v_2\}$ 
and $\op{conv}^0\{v_1,v_2\}$ are too short to meet
the face $\op{conv}\{w_0,w_1,w_2\}$.  Thus, $\op{cone}\{v_0,v_1,v_2\}$ meets
$\op{cone}\{v_0,w_1,w_2\}$.

By Lemma~\tref{XX},\FIXX{proj-coord.}
either $\op{conv}\{w_1,w_2\}$ meets $\op{conv}\{v_1,v_2,v_3\}$ or
$\op{conv}\{v_1,v_2\}$ meets $\op{conv}\{v_0,w_1,w_2\}$.
In the first case, we get that $|w_1-w_2|>2.51$, $|w_0-w_1|,|w_0-w_2|\le 2.51$.
This is the situation described by the conclusion of the Lemma.

We rule out the second case.  If 
$\op{conv}\{v_1,v_2\}$ meets $\op{conv}\{v_0,w_1,w_2\}$,
then $\op{conv}\{v_1,v_2\}$ also meets $\op{conv}\{w_0,w_1,w_2\}$.
However, $\op{conv}\{v_1,v_2\}$ is too short for that.  Thus,
this case does not occur.
\swallowed\end{proved}
\end{tarski}
%<<<<<





 % NEW, NOT IN DCG. 
%>>>>>
\begin{tarski}
\name{tri-tri-5}
\summary{In a suitably constrained packing, if two convex hulls of three points with a shared vertex meet, then the unshared
side of one meets the other hull.}
\tag{pt5, packing, t0, conv3, aff-meet}
\rating{80}
\guid{KGTJGLX}

\begin{lemma}\tlabel{tarski:tri-tri-5}
%\tlabel{tarski:two-face-shared}\
% FACE FACE WITH A SHARED VERTEX
Let $S=\{u_0,v_1,v_2,w_1,w_2\}$ be a packing of five points.
Assume that the triangles $\{u_0,v_1,v_2\}$ and $\{u_0,w_1,w_2\}$ are
special.
Assume that $\op{conv}^0\{u_0,v_1,v_2\}$ meets
$\op{conv}^0\{u_0,w_1,w_2\}$.  Then
$\op{conv}\{v_1,v_2\}$ meets $\op{conv}\{u_0,w_1,w_2\}$, or
$\op{conv}\{w_1,w_2\}$ meets $\op{conv}\{u_0,v_1,v_2\}$.
\end{lemma}

\begin{proved}
Let $x$ be a point of intersection.  As in the proof of 
Lemma~\tref{tarski:tri-tri-6}, the planes of the two faces
meet in a line $L$.  The open interval $I\subset L$ of intersection of
$\op{conv}^0\{u_0,v_1,v_2\}$ with $\op{conv}^0\{u_0,w_1,w_2\}$ has
$u_0$ as one endpoint.  Let $y$ be the other.  This point lies
in one of the two given intersections.
\swallowed\end{proved}
\end{tarski}
%<<<<<






%>>>>>
\begin{tarski}
\section{Separation simplex and triangle}
\name{tri-sim-7}
\summary{In a suitably constrained packing, the convex hull of a simplex is disjoint from the convex hull of three points.}
\tag{pt7, t0, packing, conv4, conv3, aff-meet}
\rating{120}
\guid{RTBONNT}

\begin{lemma}\tlabel{tarski:tri-sim-7}
%% Not in DCG. Used to prove simplices in Q-sys disjoint.
Let $S=\{v_1,v_2,v_3,v_4,w_1,w_2,w_3\}$ be a set of seven points
in $\ring{R}^3$.  Assume that the distainces are at least $2$.
Let $S_V = \{v_1,v_2,v_3,v_4\}$ and $S_W=\{w_1,w_2,w_3\}$.  Assume
that the sets $S_W$ and $S_V\setminus\{v_i\}$ are special for
$i=1,2,3,4$.
Then $C_V=\op{conv}(S_V)$ is disjoint from $C_W=\op{conv}(S_W)$.
\end{lemma}

\begin{proved}  Let $x$ be in the intersection.
Let $L$ be any line through $x$.  $L$ meets $C_V$ in an interval
$I_V$ and $C_W$ in an interval $I_W$.  Pick an endpoint $z$
of $I_V$ or $I_W$ that lies in $I_V\cap I_W$.  

After reindexing, if $z$ is an endpoint of $I_V$, then
$z\in\op{conv}\{v_1,v_2,v_3\}\cap C_W$.  After reindexing, if $z$
is an endpiont of $I_W$, then $C_V$ meets $\op{conv}\{w_1,w_2\}$.

In the first case, $\op{conv}\{v_1,v_2,v_3\}$ meets $C_V$.
This is the intersection of two special faces, which is described
by Lemma~\tref{tarski:tri-tri-6}.  With appropriate indexing, we have
$$
  |v_1-v_3|>2.51,\quad |w_1-w_3|>2.51.
$$
Also, $\op{conv}\{v_1,v_3,v_4\}$ meets $C_W$, since $\op{conv}\{v_1,v_3\}$
does.  This is again, the situation of Lemma~\tref{tarski:tri-tri-6}.
This gives that $\op{conv}\{w_1,w_3\}$ meets $\op{conv}\{v_1,v_3,v_4\}$
(and $\op{conv}|[v_1,v_2,v_3\}$).  This is impossible by 
Lemma~\tref{tarski:double-face}.
\swallowed\end{proved}
\end{tarski}
%<<<<<





%>>>>>
\begin{tarski}
\name{tri-sim-6}
\summary{When, in a suitable packing, a simplex and a triangle sharing a vertex meet in more than 
the shared vertex, the configuration must take one of two forms.}
\tag{pt6, packing, t0, aff-meet, conv4, conv3}
\rating{200}
\guid{JMHCAKG}

\begin{lemma}\tlabel{tarski:tri-sim-6}
% NOT IN DCG, used to separate simplices, etc.
Let $S=\{v_1,v_2,v_3,v_4,w_2,w_3\}$ be a packing of six points.
Let $S_V = \{v_1,v_2,v_3,v_4\}$ and $S_W=\{w_1,w_2,w_3\}$, where $w_1=v_1$.  
Assume
that the sets $S_W$ and $S_V\setminus\{v_i\}$ are special for
$i=1,2,3,4$.
Let $C_V=\op{conv}(S_V)$ and $C_W=\op{conv}(S_W)$.
Assume that $C_V\cap C_W\ne \{v_1\}$.
Then, up to reindexing points, the incidence structure 
is described by either A or B.
\begin{itemize}
\item{A} We have that $\op{conv}\{w_1,w_3\}$ meets $\op{conv}\{v_2,v_3,v_4\}$
and $\op{conv}\{v_2,v_4\}$ meets $\op{conv}\{w_1,w_2,w_3\}$.
Also, $|w_1-w_3|,|v_2-v_4|>2.51$ and 
  $$
  |w_1-v_4|,|v_4-w_3|,|w_3-v_2|,|v_2-w_1|\le 2.51.
  $$
\item{B} We have that $\op{conv}\{w_2,w_3\}$ meets $\op{conv}(S_V\setminus\{v_i\})$ for two different indices $i=i_1,i_2$.  Furthermore,
$|v_2-v_4|,|v_1-v_3|>2.51$ and 
$|w_i-v_j|\le 2.51$, for $i=2,3$ and $j=1,2,3,4$.
\end{itemize}
\end{lemma}

\begin{proved}  Note that Case B states that we are in the context
of Lemma~\tref{tarski:lin-sim-6}.

Let $x\ne v_1$ be an element of $C_V\cap C_W$.  Let $L=\op{aff}\{x,v_1\}$.
Let $L\cap C_V = I_V$, $L\cap C_W = I_W$.  Pick an endpoint $z$ of $I_W$
or $I_V$ that lies in $I_W\cap I_V$.  
Up to reindexing, we have four possibilities, depending on whether
$z$ is an endpoint of $I_W$ or $I_V$ and on whether the facet containing
$z$ has $w_1=v_1$ as an extreme point or not.
\begin{itemize}
  \item Case 1: $z\in \op{conv}\{v_2,v_3,v_4\}\cap C_W$.
  \item Case 2: $z\in C_V \cap \op{conv}\{w_2,w_3\}$.
  \item Case 3: $z\in C_V \cap \op{conv}\{w_1,w_2\}$.
  \item Case 4: $z\in \op{conv}\{v_1,v_2,v_3\}\cap C_W$.
\end{itemize}
Start with Case~1.  This situation is described by 
Lemma~\tref{tarski:tri-tri-6}.  It states that there is a uniquely
determined pair $\{w_i,w_j\}$ such that $|w_i-w_j|>2.51$.
It follows by Lemma~\tref{tarski:tri-tri-6}
that we are in situation A, provided $\{i,j\}\ne\{2,3\}$.
If $\{i,j\}=\{2,3\}$,  then by Lemma~\tref{tarski:lin-sim-6-not}, the
set $\{v_1,v_2,v_3,v_4\}$ can be written $\{v_a,v_b\}\cup \{v_c,v_d\}$
with $|v_a-v_b|>2.51$ and $|v_c-v_d|>2.51$.  The conditions of
Lemma~\tref{tarski:lin-sim-6} are now satisfied, placing us in
situations B.

In Case~2, it follows from Lemma~\tref{tarski:lin-sim-6-not} that
the hypotheses of Lemma~\tref{tarski:lin-sim-6} hold.  It follows
from Lemma~\tref{tarski:lin-sim-6} that we are in situation B.

In Case~3, we have by Lemma~\tref{XX} that $\op{conv}\{w_1,w_2\}$
meets $\op{conv}\{v_2,v_3,v_4\}$. In particular $C_W$ meets
$\op{conv}\{v_2,v_3,v_4\}$.  This situation was treated in Case~1.

In Case~4, by Lemma~\tref{XX}, we either have that $\op{conv}\{w_2,w_3\}$
meets $\op{conv}\{v_1,v_2,v_3\}$ (which was treated in Case~2),
or that $\op{conv}\{v_2,v_3\}$ meets $\op{conv}\{w_1,w_2,w_3\}$
(which was treated in Case~1).  This completes the proof.
\swallowed\end{proved}
\end{tarski}
%<<<<<





%>>>>>
\begin{tarski}
\name{tri-sim-5}
\summary{When, in a suitable packing, a simplex and a triangle sharing an meet in more than 
that edge, the configuration must take one of two forms.}
\tag{pt5, packing, aff-meet, conv2, conv4, conv3, t0}
\rating{160}
\guid{UGQMJJA}

\begin{lemma}\tlabel{tarski:tri-sim-5}
Let $S=\{v_1,v_2,v_3,v_4,w_3\}$ be a packing of five points.
Let $S_V =\{v_1,v_2,v_3,v_4\}$ and $S_W=\{w_1,w_2,w_3\}$, where
$w_1=v_1$ and $w_2=v_2$.  
Assume that $S_W$ and $S_V\setminus\{v_i\}$ is special for $i=1,\ldots,4$.
Let $C_V=\op{conv}(S_V)$ and $C_W=\op{conv}(S_W)$.
Assume that $C_V\cap C_W\ne \op{conv}\{v_1,v_2\}$.
Then, up to reindexing, we are in one of the two following situations.
\begin{itemize}
\item A: $\op{conv}\{w_1,w_3\}$ meets $\op{conv}\{v_2,v_3,v_4\}$
and $|w_1-w_3|>2.51$.
\item B:  $\op{conv}\{v_3,v_4\}$ meets $C_W$ and $|v_3-v_4|>2.51$.
\end{itemize}
\end{lemma}

\begin{proved}
Let $x\in C_V\cap C_W$.  For a contradiction, assume that
$x\not\in\op{conv}\{v_1,v_2\}$.  Let $L=\op{aff}\{x,v_1\}$.
Let $L\cap C_W = I_W$ and $L\cap C_V = I_V$.  Let $z$ be an endpoint
other than $v_1$
of $I_V$ or $I_W$ in $I_V\cap I_W$.  Then $z\not\in\op{conv}\{v_1,v_2\}$. 
Moreover, $x,z\not\in\op{aff}\{v_1,v_3,v_4\},\op{aff}\{v_1,v_2,v_4\}$. 
There are two possibilities, up to reindexing.
Either $z\in \op{conv}\{w_1,w_3\}\cap C_V$ or 
$z\in\op{conv}\{v_1,v_3,v_4\}\cap C_W$.

In the first case, Lemma~\tref{XX} gives that $\op{conv}\{w_1,w_3\}$
meets $\op{conv}\{v_2,v_3,v_4\}$.  This is situation~A.  The estimate
$|w_1-w_3|>2.51$ follows from Lemma~\tref{XX}.\FIXX{$\CalE$}

In the second case, there is a common vertex $v_1=w_1$.  We have
by Lemma~\tref{XX}, that either $\op{conv}\{w_2,w_3\}$ meets
$\op{conv}\{v_1,v_3,v_4\}$ or $\op{conv}\{v_3,v_4\}$ meets
$C_W$.  If the former holds, then swapping indices $1$ and $2$,
this is a case treated in Case~1.  If the latter holds, then
we are in situation B.  The estimate $|v_3-v_4|>2.51$ follows
from Lemma~\tref{XX}.\FIXX{$\CalE$}
\swallowed\end{proved}
\end{tarski}
%<<<<<







%>>>>>
\begin{tarski}
\section{Simplex simplex}
\name{sim-sim-8}
\summary{Under suitable conditions, the two convex hulls of four points cannot meet.}
\tag{pt8, packing, t0, conv4, aff-meet}
\rating{80}
\guid{ZILQMDQ}

\begin{lemma}\tlabel{tarski:sim-sim-8}
Let $S=\{v_1,v_2,v_3,v_4,w_1,w_2,w_3,w_4\}$ be a packing of eight
points.
Assume that $\{v_1,v_2,v_3,v_4\}\setminus\{v_i\}$ and
$\{w_1,w_2,w_3,w_4\}\setminus\{w_i\}$ are special, for $i=1,2,3,4$.
Then $C_W =\op{conv}\{w_1,w_2,w_3,w_4\}$ does not meet
$C_V = \op{conv}\{v_1,v_2,v_3,v_4\}$.
\end{lemma}

\begin{proved}
Let $x$ belong to the intersection.  Let $L$ be a line through
$x$.  let $z\in L\cap C_W\cap C_V$ be an endpoint of one of the
intervals $L\cap C_W$ or $L\cap C_V$.  Up to evident symmetries,
$z\in \op{conv}\{w_1,w_2,w_3\}\cap C_V$.  This does not exist,
by Lemma~\tref{tarski:tri-sim-7}.
\swallowed\end{proved}
\end{tarski}
%<<<<<





%>>>>>
\begin{tarski}
\name{sim-sim-7}
\summary{Under suitable conditions, two convex hulls of four points meet only at their shared vertex.}
\tag{pt7, packing, t0, conv4, aff-meet}
\rating{160}
\guid{AQKANYN}

\begin{lemma}\tlabel{tarski:sim-sim-7}
Let $S=\{v_1,v_2,v_3,v_4,w_2,w_3,w_4\}$ be a packing of seven
points.
Let $w_1=v_1$.
Assume that $\{v_1,v_2,v_3,v_4\}\setminus\{v_i\}$ and
$\{w_1,w_2,w_3,w_4\}\setminus\{w_i\}$ are special, for $i=1,2,3,4$.
Let 
   $C_W =\op{conv}\{w_1,w_2,w_3,w_4\}$ and
$C_V = \op{conv}\{v_1,v_2,v_3,v_4\}$.  
Then
  $$
  C_W \cap C_V = \{v_1\}.
  $$
\end{lemma}

\begin{proved}  Assume that $v_1\ne x\in C_W\cap C_V$.
Let $L=\op{aff}\{x,v_1\}$.  Let $z\in L\cap C_W\cap C_V$ be an endpoint
of one of the intervals $L\cap C_W$ or $L\cap C_V$.  Then, up
to reindexing, and the symmetry $v_i\leftrightarrow w_i$,  we have
that $z$
belongs to $\op{conv}\{w_1,w_2,w_3\}\cap C_V$ or to
$\op{conv}\{w_2,w_3,w_4\}\cap C_V$.  The latter case is impossible
by Lemma~\tref{tarski:tri-sim-7}.  Hence,  
   $$z\in\op{conv}\{w_1,w_2,w_3\}\cap C_V.$$  This situation
is described by Lemma~\tref{tarski:tri-sim-6}, which gives two cases (A)
and (B). 

In case~A of Lemma~\tref{tarski:tri-sim-6}, we have that
$\op{conv}\{w_1,w_3\}$ meets $C_V$, and also (reindexing if necessary)
$\op{conv}\{v_2,v_3\}$ meets $\{w_1,w_2,w_3\}$.
Moreover, $|v_2-v_3|>2.51$ and $|v_4-v_2|,|v_4-v_3|\le 2.51$.
In particular, $\op{conv}\{w_1,w_3,w_4\}$
meets $C_V$.  We apply Lemma~\tref{tarski:tri-sim-6} again, this
time to $\{w_1,w_3,w_4\}$ to get that $\op{conv}\{v_2,v_3\}$ meets
$\{w_1,w_3,w_4\}$.  However, this contradicts Lemma~\tref{tarski:double-face}.

In case~B of Lemma~\tref{tarski:tri-sim-6}, 
we have that $\op{conv}\{w_2,w_3\}$
meets a face $\op{conv}\{v_i,v_j,v_k\}$ of $S_V$.  In particular,
$\op{conv}\{w_2,w_3,w_4\}$ meets $C_V$.  This is impossible
by Lemma~\tref{tarski:tri-sim-7}.
\swallowed\end{proved}
\end{tarski}
%<<<<<





%>>>>>
\begin{tarski}
\name{sim-sim-6}
\summary{Under suitable conditions, two convex hulls of four points meeting at more than their shared edge must take one of two special forms.}
\tag{pt6, packing, conv4, conv2, aff-meet}
\rating{160}
\guid{IHENCNY}

\begin{lemma}\tlabel{tarski:sim-sim-6}
%% DCG isolated pair.
Let $S=\{v_1,v_2,v_3,v_4,w_3,w_4\}$ be a packing of six points.
Let $w_i=v_i$, for $i=1,2$.  Let $S_V=\{v_1,v_2,v_3,v_4\}$
and $S_W=\{w_1,w_2,w_3,w_4\}$. 
Assume that $S_W\setminus \{w_i\}$ and $S_V\setminus\{v_i\}$
are special for $i=1,2,3,4$.
 Let $C_V = \op{conv}(S_V)$ and
$C_W=\op{conv}(S_W)$.  Assume that 
   $$
   C_W \cap C_V \ne \op{conv}\{v_1,v_2\}.
   $$
Then reindexing,
we have that $\op{conv}\{w_1,w_3\}$ meets $\op{conv}\{v_2,v_3,v_4\}$
and $\op{conv}\{v_2,v_4\}$ meets $\op{conv}\{w_1,w_3,w_4\}$.  Moroever,
$|w_1-w_3|>2.51$, $|v_2-v_4|>2.51$,
and
  $$
  \begin{array}{llll}
  |w_1-v_2|\le 2.51,&|v_2-w_3|\le 2.51,& |w_3-v_4|\le 2.51,&
  |v_4-w_1|\le 2.51.\\
  |v_2-v_3|\le 2.51,&|v_4-v_3|\le 2.51,& |w_1-w_4|\le 2.51,&
  |w_4-w_3|\le 2.51.
  \end{array}
  $$
\end{lemma}

\begin{proved}
Let $x\in (C_W\cap C_V)\setminus \op{conv}\{v_1,v_2\}$.
Let $L=\op{aff}\{v_1,x\}$.  Let $z$ be a point in 
$L\cap (C_W\cap C_V)\setminus\op{conv}\{v_1,v_2\}$ that is
an endpoint of an interval $L\cap C_W$ or $L\cap C_V$.   Exchanging
$S_V$ and $S_W$ under the symmetry $v_i\leftrightarrow w_i$,
we may assume that $z$ is an endpoint of $L\cap C_W$.  
Then $\op{conv}\{w_i,w_j,w_k\}$ meets $C_V$, for some $i,j,k$.
Up to reindexing, 
whave two possibilities, according to the cardinality of
$\{i,j,k\}\cap\{1,2\}$.
\begin{enumerate}
  \item $\op{conv}\{w_1,w_2,w_4\}$ meets $C_V$.
  \item $\op{conv}\{w_1,w_3,w_4\}$ meets $C_V$.
\end{enumerate}

The first case is treated in Lemma~\tref{tarski:tri-sim-5}.
\FIXX{Finish this argument.}

The second case is treated in Lemma~\tref{tarski:tri-sim-6}.
\FIXX{Finish this argument.}
\swallowed\end{proved}
\end{tarski}
%<<<<<








%>>>>>
\begin{tarski}
\name{sim-sim-5}
\summary{Under suitable conditions, if two convex hulls of four points meet at more than their shared face, then some other edge of one simplex meets another face of the other simplex.}
\tag{pt5, packing, conv4, conv3}
\rating{80}
\guid{DWGDFMQ}

\begin{lemma}\tlabel{tarski:sim-sim-5}
Let $S=\{v_1,v_2,v_3,v_4,w_4\}$ be a packing of five points.
Let $w_i=v_i$, for $i=1,2,3$.  Let $S_V=\{v_1,v_2,v_3,v_4\}$
and $S_W=\{w_1,w_2,w_3,w_4\}$. 
Assume that $S_W\setminus \{w_i\}$ and $S_V\setminus\{v_i\}$
are special for $i=1,2,3,4$.
 Let $C_V = \op{conv}(S_V)$ and
$C_W=\op{conv}(S_W)$.  Assume that 
   $$
   C_W \cap C_V \ne \op{conv}\{v_1,v_2,v_3\}.
   $$
Then reindexing and up to symmetry under $v_i\leftrightarrow w_i$,
we have that $\op{conv}\{w_1,w_4\}$ meets $\op{conv}\{v_2,v_3,v_4\}$.
Morevover, $|w_1-w_4|>2.51$.
\end{lemma}

\begin{proved}
\FIXX{Add proof.}
\swallowed\end{proved}
\end{tarski}
%<<<<<






%>>>>>
\begin{tarski}
\name{sim-sim-qrtet}
\summary{Under suitable conditions, two convex hulls of four points meet at the convex hull of the set of shared extreme points.}
\tag{pt4, pt5, pt6, conv2, conv3, conv4, aff-meet}
\rating{80}
\guid{CZXNFAQ}

\begin{lemma}\tlabel{tarski:sim-sim-qrtet}
Let $S_V=\{v_1,v_2,v_3,v_4\}$ be a set of four
points in $\ring{R}^3$.  Let $S_W=\{w_1,w_2,w_3,w_4\}$ be a set
of four points in $\ring{R}^3$.  Assume that that distances
between points in $S_V\cup S_W$ are at least $2$.  Assume that
  $$
  |v_i-v_j|\le 2.51,\quad |w_i -w_j|\le 2.51,\quad 1\le i<j\le 4.
  $$
Then $$\op{conv}(S_V)\cap \op{conv}(S_W) = \op{conv}(S_V\cap S_W).$$
\end{lemma}

\begin{proved}
We have that $S_W\setminus\{w_i\}$ and $S_V\setminus\{v_i\}$ are
special for $i=1,2,3,4$.

Let $n$ be the cardinality of $S_W\cup S_V$.  By Lemmas~\tref{tarski:sim-sim-8}
and \tref{tarski:sim-sim-7}, we have $4\le n\le 6$.  

If $n=6$,
then Lemma~\tref{tarski:sim-sim-6} treats the situation.  
This
lemma gives the desired conclusion.
If $n=5$, then Lemma~\tref{tarski:sim-sim-5} gives the desired conclusion.
Finally, if $n=4$, then the result is trivial.
\swallowed\end{proved}
\end{tarski}
%<<<<<





%>>>>>
\begin{tarski}
\name{sim-sim-qrtet-special}
\summary{Under suitable conditions, if two convex hulls of four points meet in more than the convex hull of their common extreme points, and if one is a quasi-regular tetrahedron, then they share a face.  Moreover, various edge, circumradius, and incidence relations hold.}
\tag{pt5, pt6, pt7, conv2, conv3, eta, conv4, aff-meet}
\rating{140}
\guid{BYUOUXO}

\begin{lemma}\tlabel{tarski:qrtet-over}\usage{DCG-[4.35]{ lemma:qrtet-over}}\tlabel{tarski:sim-sim-qrtet-special}
Let $S_V=\{v_1,v_2,v_3,v_4\}$ be a set of four
points in $\ring{R}^3$.  Let $S_W=\{w_1,w_2,w_3,w_4\}$ be a set
of four points in $\ring{R}^3$.  Assume that that distances
between points in $S_V\cup S_W$ are at least $2$.  Assume that
  $$
  |v_i-v_j|\le 2.51\quad 1\le i<j\le 4.
  $$
Assume that $S_W\setminus\{w_i\}$ is special for $i=1,2,3,4$.
Assume that 
  \begin{equation}\tlabel{tarski:eqn:cap}
  \op{conv}(S_W)\cap \op{conv}(S_V)\ne \op{conv}(S_W\cap S_V).
  \end{equation}
Then the cardinality of
$S_W\cap S_V$ is five.  Moreover, up to reindexing, we have
that $w_i=v_i$ for $i=1,2,3$ and that
  $$
  \op{conv}\{w_1,w_4\} \text{ meets } \op{conv}\{v_2,v_3,v_4\}.
  $$
Moreover, we have
  $$
  \begin{array}{lll}
  &|w_1-w_4|>2.51,\quad \\
  &|w_j-v_i|\le 2.2,\quad i=2,3,4;\ j=1,4\\
  &\eta_V(v_1,v_i,v_j) < \sqrt2,\quad 2\le i < j \le 4\\
  &\eta_V(v_2,v_3,v_4) \ge \sqrt2\\
  \end{array}
  $$
\end{lemma}

\begin{remark}
Because of the conclusion,
we see that $S_V'=\{w_4,v_2,v_3,v_4\}$ also satisfies the
conditions of the lemma (if $S_V$ does). 
Also, because of the conclusion,
$S_W' = \{w_1,w_4,v_2,v_4\}$ and $S_W''=\{w_1,w_4,v_3,v_4\}$
satisfy the conditions of the lemma (on $S_W$); that is,
their faces are special, and so forth.
Again, by the conclusion, $S_W$, $S_W'$, and $S_W''$ are the
only three sets of four points in $\ring{R}^3$ that satisfy
the conditions of the lemma (with respect to 
$S_W$ for some fixed $S_V$).
\end{remark}

\begin{proved}
We have that $S_V\setminus\{v_i\}$ are
special for $i=1,2,3,4$.

Let $n$ be the cardinality of $S_W\cup S_V$.  By Lemmas~\tref{tarski:sim-sim-8}
and \tref{tarski:sim-sim-7}, we have $4\le n\le 6$.  When
$n=6$, Lemma~\tref{tarski:sim-sim-6} implies that $S_V$ has an edge
$|v_i-v_j|>2.51$, which is contrary to hypothesis.  So $n\le 5$.

Also, $n=4$ is contrary to the assumption~\tref{tarski:eqn:cap}. 
Hence $n=5$.  
Choose indices so that $w_i=v_i$, for $i=1,2,3$. 
This situation is treated by Lemma~\tref{tarski:sim-sim-5}.
That lemma gives that $|w_1-w_4|>2.51$ and that
  $\op{conv}\{w_1,w_4\}$ meets $\op{conv}\{v_2,v_3,v_4\}$.
This implies by Lemma~\tref{no-pass-sqrt2} that 
$\eta_V(v_2,v_3,v_4)\ge\sqrt2$.

By Lemma~\tref{tarski:qrtet-pair-pass}, we have that 
$$|w_j-v_i|\le 2.2,\quad i=2,3,4;\ j=1,4.$$
By Lemma~\tref{tarski:eta-2.2}, we have that 
 $$\eta_V(v_1,v_i,v_j) < \sqrt2,\quad 2\le i < j \le 4.$$
The conclusion follows.
\swallowed\end{proved}
\end{tarski}
%<<<<<



%>>>>>
\begin{tarski}
\section{Dumped without context}
\FIXX{Many of the Lemmas in this section need to be rewritten in the
Tarski's language.}
\name{oct-over}
\summary{This is not yet in Tarski's language.  It asserts that a quarter in an octahedron is disjoint from the rest of the Q-system.}
\tag{pt6, oct, aff-meet}
\rating{80}
\guid{ZYMHOWN}

\begin{lemma} \tlabel{tarski:oct-over}\usage{DCG-[4.36]{ lemma:oct-over}}
The interior of a quarter in the $Q$-system that is part of a
quartered octahedron is disjoint from the interior of every other
simplex in the $Q$-system.
\end{lemma}

\begin{proved} By construction, the quarters that lie along a
different diagonal of the octahedron do not belong to the
$Q$-system.  Edges $\op{conv}^0\{v_1,v_2\}$  
of length at most $2.51$ are too short to meet
an external face $\op{conv}\{w_1,w_2,w_3\}$ of the octahedron
(Lemma~\tref{tarski:2t0-doesnt-pass-through}). Similarly, a diagonal of a
strict quarter meet an external face,
because of Lemma~\tref{tarski:qrtet-pair-pass}.
\swallowed\end{proved}
\end{tarski}
%<<<<<



%>>>>>
\begin{tarski}
\name{adj-over}
\summary{This is not yet in Tarski's language.  It asserts that a quarter in an adjacent pair is disjoint from the Q-system.}
\tag{pt5, pt6, pt7, pt8, packing, aff-meet, conv4}
\rating{80}
\guid{OOHHKOI}

\begin{lemma}\tlabel{tarski:adj-over}\usage{DCG-[4.37]{ lemma:adj-over}}
Let $Q$ be a strict quarter that is part of an adjacent pair.
Assume that $Q$ is not part of a quartered octahedron.  If $Q$
belongs to the $Q$-system, then its interior is disjoint from the
interior of every other simplex in the $Q$-system.
\end{lemma}

The proof of this lemma will give valuable details about how the
interior of one strict quarter meets another.

\begin{proved}
Fix the origin at the base point of an adjacent pair of quarters.
We investigate the local geometry when the interior of another
quarter overlaps meets one of their interiors.  (This happens, for
example, when there is a conflicting diagonal in the sense of
Definition~\tref{def:corner}.)

Label the base point of the pair of quarters $v_0$, and the four
corners  $v_1$, $v_2$, $v_3$, $v_4$, with $\{v_1,v_3\}$ the common
diagonal. Assume that $|v_1-v_3|<\sqrt8$.\indy{Index}{conflicting
diagonals}

If two quarters share an interior point then a face on one of them
``overlaps'' a face on the other.  By
Lemmas~\tref{tarski:single-enclosed} and \tref{tarski:double-face}, we
actually have that some edge (in fact the diagonal) of each passes
through a face of the other.  This edge cannot exit through
another face by Lemma~\tref{tarski:double-face} and it cannot end
inside the simplex by Lemma~\tref{tarski:v-interior}. Thus, it must
end at a vertex of the other simplex.  We break the proof into
cases according to which vertex of the simplex it terminates at.
In Case 1, the edge has the base point as an endpoint.  In Case 2,
the edge has a corner as an endpoint.

\noindent{\bf Case 1.} {\it The edge $\op{conv}^0\{v_0,w\}$ meets 
the
triangle $\op{conv}\{v_1,v_2,v_3\}$, where $\{v_0,w\}$ is a diagonal of a
strict quarter.}

Lemma~\tref{tarski:pass-anchor} implies that $v_1$ and $v_3$ are
anchors of $\{v_0,w\}$. The only other possible anchors of $\{v_0,w\}$
are $v_2$ or $v_4$, for otherwise an edge of length at most $2.51$
passes through a face formed by $\{v_0,w\}$ and one of its anchors.
If both $v_2$ and $v_4$ are anchors, then we have a quartered
octahedron, which has been excluded by the hypotheses of the
lemma. Otherwise, $\{v_0,w\}$ has at most three anchors: $v_1$,
$v_3$, and either $v_2$ or $v_4$. In fact, it must have exactly
three anchors, for otherwise there is no quarter along the edge
$\{v_0,w\}$. So there are exactly two quarters along the edge
$\{v_0,w\}$. There are at least four anchors along $\{v_1,v_3\}$:
$0$, $w$, $v_2$, and $v_4$. The quarters along the diagonal
$\{v_1,v_3\}$ lie in the $Q$-system. (None of these quarters is
isolated.)  The other two quarters, along the diagonal $\{v_0,w\}$,
are not in the $Q$-system. They form an adjacent pair of quarters
(with base point $v_4$ or $v_2$) that has conflicting diagonals,
$\{v_0,w\}$ and $\{v_1,v_3\}$, of length at most $\sqr8$.

\noindent {\bf Case 2.}  {\it $\{v_2,v_4\}$ is a diagonal of
length less than $\sqr8$ (conflicting with $\{v_1,v_3\}$).}

(Note that if an edge of a quarter meets the shared face
of an adjacent pair of quarters, then that edge must be
$\{v_2,v_4\}$, so that Case 1 and Case 2 are exhaustive.) The two
diagonals $\op{conv}^0\{v_1,v_3\}$ and $\op{conv}^0\{v_2,v_4\}$ 
do not meet. By
symmetry, we may assume that $\op{conv}^0\{v_2,v_4\}$ meets the face
$\op{conv}\{v_0,v_1,v_3\}$. Assume (for a contradiction) that both diagonals
have an anchor other than $0$ and the corners $v_i$. Let the
anchor of $\{v_2,v_4\}$ be denoted $v_{24}$ and that of
$\{v_1,v_3\}$ be $v_{13}$. Assume the figure is not a quartered
octahedron, so that $v_{13}\ne v_{24}$. By
Lemma~\tref{tarski:2t0-doesnt-pass-through}, it is impossible to
draw the edges $\{v_1,v_{13}\}$ and $\{v_{13},v_3\}$ between $v_1$
and $v_3$.  In fact, if the edges pass outside the quadrilateral
$\{v_0,v_2,v_{24},v_4\}$, one of the edges of length at most $2.51$
(that is,
    $\{v_0,v_2\}$, $\{v_2,v_{24}\}$, $\{v_{24},v_4\}$,
or $\{v_4,0\}$) violates the lemma applied to the face
$\{v_1,v_3,v_{13}\}$. If they pass inside the quadrilateral, one
of the edges $\{v_1,v_{13}\}$, $\{v_{13},v_3\}$ violates the lemma
applied to the face
    $\{v_0,v_{2},v_4\}$ or $\{v_{24},v_2,v_4\}$.
We conclude that at most one of the two diagonals has additional
anchors.

If neither of the two diagonals has more than three anchors, we
have nothing more than two overlapping adjacent pairs of quarters
along conflicting diagonals.  The two quarters along the lower
edge $\{v_2,v_4\}$ lie in the $Q$-system.  Another way of
expressing this ``lower-edge'' condition is to require that the
two adjacent quarters $Q_1$ and $Q_2$ satisfy
$\dih_V(Q_1)+\dih_V(Q_2)>\pi$, when the dihedral angles are measured
along the diagonal. The pair $(Q_1',Q_2')$ along the upper edge
will have $\dih_V(Q_1')+\dih_V(Q_2')<\pi$.

If there is a diagonal with more than three anchors,  the quarters
along the diagonal with more than three anchors lie in the
$Q$-system.  Any additional quarters along the diagonal
$\{v_2,v_4\}$ belong to an adjacent pair. Any additional quarters
along the diagonal $\{v_1,v_3\}$ cannot intersect the adjacent
pair along $\{v_2,v_4\}$.  Thus, every quarter intersecting an
adjacent pair also belongs to an adjacent pair.

In both possibilities of case 2, the two quarters left out of the
$Q$-system correspond to a conflicting diagonal.
\swallowed\end{proved}
\end{tarski}
%<<<<<



\begin{remark}\tlabel{tarski:iso}\usage{DCG-[4.38]{ remark:iso}}
We have seen in the proof of Lemma~\tref{tarski:adj-over} that if
the interior of a strict quarter $Q$ meets the interior of a
strict quarter that is part of an adjacent pair, then $Q$ is also
part of an adjacent pair. Thus, if the interior of an isolated
strict quarter meets the interior of another strict quarter, then
both strict quarters are necessarily isolated.
\end{remark}

%>>>>>
\begin{tarski}
\name{iso-over}
\summary{This is not yet in Tarski's language.  The geometry of an isolated quarter is explored.}
%\endnote{If an isolated strict quarter meets the interior of another strict quarter, then three anchors.  Used for geometry of isolated pairs.  tarski:iso-over}
\tag{aff-meet, conv4, anchor}
\rating{80}
\guid{VXMZWGM}

\begin{lemma}\tlabel{tarski:iso-over}\usage{DCG-[4.39]{ lemma:iso-over}}
If the interior of an isolated strict quarter $Q$ meets the
interior of another strict quarter, then the diagonal of $Q$ has
exactly three anchors.
\end{lemma}

The proof of the lemma will give detailed information about the
geometric configuration that is obtained when the interior of an
isolated quarter meets the interior of another strict quarter.

\begin{proved}  Assume that there are two strict quarters $Q_1$ and $Q_2$
that meet in their interiors.  Following Remark~\tref{tarski:iso},
assume that neither is adjacent to another quarter. Let $\{v_0,u\}$
and $\{v_1,v_2\}$ be the diagonals of $Q_1$ and $Q_2$. Suppose the
diagonal $\op{conv}^0\{v_1,v_2\}$ meets a face $\op{conv}\{v_0,u,w\}$ of $Q_1$.
By Lemma~\tref{tarski:pass-anchor}, $v_1$ and $v_2$ are anchors of
$\{v_0,u\}$. Again, either the length of $\{v_1,w\}$ is at most
$2.51$ or the length of $\{v_2,w\}$ is at most $2.51$, say
$\{w,v_2\}$ (by Lemma~\tref{tarski:pass-makes-quarter}). It follows
that
    $Q_1=\{v_0,u,w,v_2\}$ and $|v_1-w|\ge2.51$.
($Q_1$ is not adjacent to another quarter.)  So $w$ is not an
anchor of $\{v_1,v_2\}$.

Let $\{v_1,v_2,w'\}$ be a face of $Q_2$ with $w'\ne v_0,u$. If
$\{v_1,w',v_2\}$ does not link $\{v_0,u,w\}$, then $\op{conv}^0\{v_1,w'\}$ or
$\op{conv}^0\{v_2,w'\}$ meets the face $\op{conv}\{v_0,u,w\}$, which is
impossible by Lemma~\tref{tarski:2t0-doesnt-pass-through}. So
$\op{conv}\{v_1,v_2,w'\}$  meets $\op{conv}\{v_0,u,w\}$
and an edge of $\op{conv}\{v_0,u,w\}$
meets the face $\op{conv}\{v_1,v_2,w'\}$. It is not the edge
$\{u,w\}$ or $\{v_0,w\}$, for they are too short by
Lemma~\tref{tarski:2t0-doesnt-pass-through}.  So $\op{conv}^0\{v_0,u\}$ meets 
$\op{conv}\{w',v_1,v_2\}$. The only anchors of $\{v_1,v_2\}$ (other
than $w'$) are $u$ and $0$ (by Lemma~\tref{tarski:double-face}).
Either $\{u,w'\}$ or $\{w',0\}$ has length at most $2.51$ by
Lemma~\tref{tarski:pass-makes-quarter}, but not both, because this
would create a quarter adjacent to $Q_2$. By symmetry,
$Q_2=\{v_1,v_2,w',0\}$ and the length of $\{u,w'\}$ is greater
than $2.51$. By symmetry, $\{v_0,u\}$ has no other anchors either.
This determines the local geometry when there are two quarters
that intersect without belonging to an adjacent pair of quarters.
It follows that the two quarters form an isolated pair.
\swallowed\end{proved}

%WW Include later.
%\begin{figure}[htb]
%  \centering
%  %\myincludegraphics{\ps/isolatedpair.eps}
%  \caption{An isolated pair.  The isolated pair consists of two simplices
%   $Q_1=\{v_0,u,w,v_2\}$ and $Q_2=\{v_0,w',v_1,v_2\}$.  The six extremal vertices
%   form an octahedron. This is not a quartered octahedron because the edges
%   $\{u,w'\}$ and $\{w,v_1\}$ have length greater than $2.51$.}
%  \tlabel{fig:diag19}\usage{DCG-[4.3]{ fig:diag19}}
%\end{figure}

Isolated quarters that meet in the interior with another strict
quarter do not belong to the $Q$-system.
\end{tarski}
%<<<<<








%>>>>>
\begin{tarski}
\section{Dumped from DCG page 119, 11.5, 11.6, 11.7}
\FIXX{The following bit of geometry should be moved to Tarski
incidence tables.}
\name{def:mask}
\summary{This is a misplaced definition (of masked quarters).}
\tag{def}
\rating{0}
\guid{KCLLHRU}
\begin{definition}[mask]\indy{Index}{masked}
An upright quarter $Q_1$ in the $Q$-system
{\it masks} a flat quarter $Q_2$, if
$\op{conv}^0(Q_1)$ meets $\op{conv}^0(Q_2)$.   A set of upright
quarters in the $Q$-system masks a flat quarter if at least 
one of them does.
\end{definition}

By the basic properties of the $Q$-system, a masked flat quarter is
not in the $Q$-system.
\end{tarski}
%<<<<<


%>>>>>
\begin{tarski}
\name{quarter-slice}
\summary{This is not yet expressed in the Tarski language.  It needs to be cleaned up.}
\tag{t0, anchor}
\rating{80}
\guid{FAGLBKX}

\begin{lemma}
%    \oldlabel{3.8.2}
Let $\{v_0,v\}$ be an upright diagonal with at least four anchors.
If $Q$ is a flat quarter such that $\op{conv}^0(Q)$  meets  
an interior point of 
a slice along $\{v_0,v\}$, then the vertices of $Q$ are the
origin and three consecutive anchors of $\{v_0,v\}$.
\end{lemma}

\begin{proved}
For there to be overlap, the diagonal $\op{conv}^0\{w_1,w_2\}$ of $Q$ 
must meet
 $\op{conv}\{v_0,v,v_1\}$ formed by some anchor $v_1$  (see
Lemma~\ref{tarski:2t0-doesnt-pass-through}).  By
Lemma~\ref{tarski:pass-anchor}, $w_1$ and $w_2$ are anchors of
$\{v_0,v\}$. By Lemma~\ref{tarski:double-face}, $w_2,v_1$, and $w_1$
are consecutive anchors. If $v_1$ is a vertex of $Q$ we are done.
Otherwise, let $w_3\ne v_0,w_1,w_2$ be the remaining vertex of $Q$.
The edges $\op{conv}^0\{v,v_1\}$ and $\op{conv}^0\{v_1,0\}$ do not
meet
$\op{conv}\{w_1,w_2,w_3\}$ by Lemma~\ref{tarski:2t0-doesnt-pass-through}.
Likewise, the edges $\op{conv}\{w_2,w_3\}$ and $\op{conv}\{w_3,w_1\}$ 
do not meet
 $\op{conv}\{v_0,v,v_1\}$. Thus, $v$ is enclosed over the
quarter $Q$.

Let $w_3'\ne w_1,v_1,w_2$ be a fourth anchor of $\{v_0,v\}$. By
Lemma~\ref{tarski:2t0-doesnt-pass-through}, we have $w_3'=w_3$.
\swallowed\end{proved}
\end{tarski}
%<<<<<



%>>>>>
\begin{tarski}
\name{quarter-anchor}
\summary{This needs to be cleaned up.  It is not expressed in the Tarski language.}
\tag{t0, anchor}
\rating{80}
\guid{EANBDBV}

\begin{lemma}
If $v$ is enclosed over a flat quarter, then $\{v_0,v\}$ has at most four
anchors.
\end{lemma}

\begin{proved}
\swallowed\end{proved}
\end{tarski}
%<<<<<









%>>>>>
\begin{tarski}
\section{Extracted from volume}
\name{BCDE}
\summary{A particular intersection of four half-spaces lies in a given ball.}
\tag{pt4, plane, circum3, eta, halfspace, ball}
\rating{}
\guid{QMRAQFC}

\begin{lemma}\tlabel{tarski:BCDE}\usage{Extracted from Volumes chapter, quoin
volume}
Let $\{v_0,v_1,v_2,v_3\}$ be a set of four points in $\ring{R}^3$.
Assume that $\{v_0,v_1,v_2,v_3\}$ is not coplanar.  Let $p$
be the circumcenter of $\{v_0,v_1,v_2\}$, and $r$ its circumradius.  
Let $p'$ be the
point in $\op{aff}_+^0(\{v_0,v_1,v_2\},v_3\}$ at distance
$c > r$ from $v_0$, $v_1$, and $v_2$ (provided by Lemma~\ref{XX}). %% TARSKI
Then the intersection
  $$
  \op{aff}^0_+(\{v_0,v_1,v_2\},v_3) \cap
  \op{aff}^0_+(\{(v_0+v_1)/2,p,p' \},v_0) \cap
  \op{aff}^0_+(\{v_0,p,p'\},v_1\}\cap
  \op{aff}^0_0(\{v_0,v_1,p'\},p)
  $$
is a subset of $B(v_0,c)$.
\end{lemma}

\begin{proved}
\swallowed\end{proved}
\end{tarski}
%<<<<<





%>>>>>
\begin{tarski}
\name{BCEF}
\summary{A particular intersection of three half-spaces with a right-circular cone lies in a given ball.}
\tag{pt4, plane, circum3, eta, halfspace, rcone, ball}
\rating{80}
\guid{RAJHSDQ}

\begin{lemma}\tlabel{tarski:BCEF}\usage{Extracted from Volumes chapter, quoin volume}
Let $\{v_0,v_1,v_2,v_3\}$ be a set of four points in $\ring{R}^3$.
Assume that $\{v_0,v_1,v_2,v_3\}$ is not coplanar.  Let $p$
be the circumcenter of $\{v_0,v_1,v_2\}$, and $r$ its circumradius.  
Let $p'$ be the
point in $\op{aff}_+^0(\{v_0,v_1,v_2\},v_3\}$ at distance
$c > r$ from $v_0$, $v_1$, and $v_2$ (provided by Lemma~\ref{tarski:XX}).
Let $a = |v_0-v_1|/2$.
Then the intersection
  $$
  \op{aff}^0_+(\{v_0,v_1,v_2\},v_3) \cap
  \op{aff}^0_+(\{(v_0+v_1)/2,p,p' \},v_0) \cap
  \op{aff}^0_0(\{v_0,v_1,p'\},p)\cap
  \op{rcone}^0(v_0,v_1,a/c)
  $$
is a subset of $B(v_0,c)$.
\end{lemma}

\begin{proved}
\swallowed\end{proved}
\end{tarski}
%<<<<<






%>>>>>
\begin{tarski}
\name{ABCD}
\summary{The intersection of particular halfspaces with a ball lies in a given halfspace.}
\tag{pt4, ball, halfspace, circum3, eta}
\rating{80}
\guid{XQXFRZY}

\begin{lemma}\tlabel{tarski:ABCD}\usage{Extracted from Volumes chapter, quoin volume}
Let $\{v_0,v_1,v_2,v_3\}$ be a set of four points in $\ring{R}^3$.
Assume that $\{v_0,v_1,v_2,v_3\}$ is not coplanar.  Let $p$
be the circumcenter of $\{v_0,v_1,v_2\}$, and $r$ its circumradius.  
Let $p'$ be the
point in $\op{aff}_+^0(\{v_0,v_1,v_2\},v_3\}$ at distance
$c > r$ from $v_0$, $v_1$, and $v_2$ (provided by Lemma~\ref{tarski:XX}).
Let $a = |v_0-v_1|/2$.
Then the intersection
  $$
  B(v_0,c)\cap
  \op{aff}^0_+(\{v_0,v_1,v_2\},v_3) \cap
  \op{aff}^0_-(\{(v_0+v_1)/2,p,p' \},v_0) \cap
  \op{aff}^0_-(\{v_0,p,p'\},v_1)
  $$
is a subset of $\op{aff}^0_+(\{v_0,v_1,p'\},p)$.
\end{lemma}

\begin{proved}
\swallowed\end{proved}
\end{tarski}
%<<<<<






%>>>>>
\begin{tarski}
\name{ABCE}
\summary{The intersection of particular halfspaces with a ball lies in a given right-circular cone.}
\tag{pt4, ball, halfspace, rcone, circum3, eta}
\rating{80}
\guid{ZULMEGV}

\begin{lemma}\tlabel{tarski:ABCE}\usage{Extracted from Volumes chapter, quoin volume}
Let $\{v_0,v_1,v_2,v_3\}$ be a set of four points in $\ring{R}^3$.
Assume that $\{v_0,v_1,v_2,v_3\}$ is not coplanar.  Let $p$
be the circumcenter of $\{v_0,v_1,v_2\}$, and $r$ its circumradius.  
Let $p'$ be the
point in $\op{aff}_+^0(\{v_0,v_1,v_2\},v_3\}$ at distance
$c > r$ from $v_0$, $v_1$, and $v_2$ (provided by Lemma~\ref{tarski:XX}).
Let $a = |v_0-v_1|/2$.
Then the intersection
  $$
  B(v_0,c)\cap
  \op{aff}^0_+(\{v_0,v_1,v_2\},v_3) \cap
  \op{aff}^0_-(\{(v_0+v_1)/2,p,p' \},v_0) \cap
  \op{aff}^0_0(\{v_0,v_1,p'\},p)
  $$
is a subset of $\op{rcone}^0(v_0,v_1,a/c)$.
\end{lemma}

\begin{proved}
\swallowed\end{proved}
\end{tarski}
%<<<<<







%>>>>>
\begin{tarski}
\section{Extracted from truncating rogers}
\name{wedge-union}
\summary{A wedge breaks into two smaller wedges and a separating half-plane. These pieces are disjoint.}
\tag{lune, aff-meet, pt5, inside}
\rating{80}
\guid{WJFHKSY}

\begin{lemma}\tlabel{tarski:wedge-union}\usage{lemma:sovo:truncRog}
Let $\{v_0,v_1,v_2,v_3,w\}$ be a set of five points in $\ring{R}^3$.
Assume that $w\in W(v_0,v_1,v_2,v_3)$. Then
$$
W(v_0,v_1,v_2,v_3) = W(v_0,v_1,v_2,w) \cup \op{aff}_+^0(\{v_0,v_1\},w)
\cup W(v_0,v_1,w,v_3).
$$
Furthermore, the sets
$$
W(v_0,v_1,v_2,w),\quad \op{aff}_+^0(\{v_0,v_1\},w),\quad
W(v_0,v_1,w,v_3)
$$
are disjoint from one another.
\end{lemma}

\begin{proved}
\FIXX{A similar result for splitting $\op{aff}_+^0$ appears earlier.}
\swallowed\end{proved}
\end{tarski}
%<<<<<





%>>>>>
\begin{tarski}
\name{rogers-ball}
\summary{A Rogers simplex lies in a ball defined by the same center and radius parameter.}
\tag{pt4, rogers, ball, plane}
\rating{60}
\guid{FKFJBPG}

\begin{lemma}\tlabel{tarski:rogers-ball}\usage{lemma:sovo:truncRog}
Let $\{v_0,v_1,v_2,v_3\}$ be a set of four points in $\ring{R}^3$.
Assume that the set is not coplanar.
Let $t>0$.  We have
$$
   \op{rog}^0(v_0,v_1,v_2,v_3,t) \subset B(v_0,t).
$$
\end{lemma}

\begin{proved}
\swallowed\end{proved}
\end{tarski}
%<<<<<





%>>>>>
\begin{tarski}
\name{rogers2}
\summary{A Rogers simplex intersected with a suitable wedge is again a Rogers simplex.}
\tag{pt4, rogers, plane, eta, halfspace, lune}
\rating{80}
\guid{KRETJIK}

\begin{lemma}\tlabel{tarski:rogers2}\usage{lemma:sovo:truncRog}
Let $\{v_0,v_1,v_2,v_3\}$ be a set of four points in $\ring{R}^3$.
Assume that the set is not coplanar.
Let $\eta_V(v_0,v_1,v_2) \le b \le c$.  Let $q$ be the point
in $\op{aff}_+(\{v_0,v_1,v_2\},v_3)$ at equidistance $b$
from $v_0,v_1,v_2$.  (The unique existence of this point is given
by Lemma~\ref{tarski:rog-exist}.)  Then
$$
W(v_0,v_1,v_2,q) \cap \op{rog}^0(v_0,v_1,v_2,v_3,c) = 
  \op{rog}^0(v_0,v_1,v_2,v_3,b).
$$
\end{lemma}

\begin{proved}
\swallowed\end{proved}
\end{tarski}
%<<<<<





%>>>>>
\begin{tarski}
\name{rcone-ball}
\summary{The intersection of a Rogers simplex with a right-circular cone lies in a suitable ball.}
\tag{pt4, plane, eta, rogers, rcone, ball}
\rating{60}
\guid{KMQHURS}

\begin{lemma}\tlabel{tarski:rcone-ball}\usage{lemma:sovo:truncRog}
Let $\{v_0,v_1,v_2,v_3\}$ be a set of four points in $\ring{R}^3$.
Assume that the set is not coplanar.
Let $\eta_V(v_0,v_1,v_2)\le b\le c$.  Let $y=|v_1-v_0|$.
Then
$$
\op{rog}^0(v_0,v_1,v_2,v_3,c) \cap \op{rcone}^0(v_0,v_1,y/2,y/(2b))
\subset B(v_0,b).
$$
\end{lemma}

\begin{proved}
\swallowed\end{proved}
\end{tarski}
%<<<<<





%>>>>>
\begin{tarski}
\name{rogers-FR}
\summary{Under suitable conditions, the intersection of a wedge, right-circular cone, and rogers simplex equals the intersection of a wedge and a frustum.}
\tag{pt4, plane, eta, halfspace, lune, rcone}
\rating{80}
\guid{GBEJKWT}

\begin{lemma}\tlabel{tarski:rogers-FR}\usage{lemma:sovo:truncRog}
Let $\{v_0,v_1,v_2,v_3\}$ be a set of four points in $\ring{R}^3$.
Assume that the set is not coplanar.
Let $\eta_V(v_0,v_1,v_2)\le b \le c$.  
Let $q_b$ (resp. $q_c$) be the point
in $\op{aff}_+(\{v_0,v_1,v_2\},v_3)$ at equidistance $b$ (resp. $c$)
from $v_0,v_1,v_2$.  (The unique existence of this point is given
by Lemma~\ref{tarski:rog-exist}.)   Let $y = |v_1-v_0|$.
$$
\begin{array}{lll}
W(v_0,v_1,q_b,q_c) \cap \op{rcone}^0(v_0,v_1,y/(2b))
\cap \op{rog}^0(v_0,v_1,v_2,v_3,c) &=\\
\quad FR(v_0,v_1,y/2,y/(2b))\cap W(v_0,v_1,q_b,q_c).
\end{array}
$$
\end{lemma}

\begin{proved}
\swallowed\end{proved}
\end{tarski}
%<<<<<





%>>>>>
\begin{tarski}
\name{rogers-rad}
\summary{Under suitable conditions, the intersection of a wedge, rogers simplex, ball, and complement of a right-circular cone equals the intersection of a wedge, half-space, ball, and complement of a right-circular cone.}
\tag{pt4, plane, eta, halfspace, lune, rogers, ball, rcone}
\rating{80}
\guid{KCNGLUP}

\begin{lemma}\tlabel{tarski:rogers-rad}\usage{lemma:sovo:truncRog}
Let $\{v_0,v_1,v_2,v_3\}$ be a set of four points in $\ring{R}^3$.
Assume that the set is not coplanar.
Let $\eta_V(v_0,v_1,v_2)\le b \le c$.  
Let $q_b$ (resp. $q_c$) be the point
in $\op{aff}_+(\{v_0,v_1,v_2\},v_3)$ at equidistance $b$ (resp. $c$)
from $v_0,v_1,v_2$.  (The unique existence of this point is given
by Lemma~\ref{tarski:rog-exist}.)   Let $y = |v_1-v_0|$.
Then
$$
\begin{array}{lll}
W(v_0,v_1,q_b,q_c)  \cap \op{rog}^0(v_0,v_1,v_2,v_3,c)\cap
B(v_0,b) \cap \op{rcone}_-^0(v_0,v_1,y/(2b)) =\\
\quad
W(v_0,v_1,q_b,q_c)  \cap \op{aff}_+^0(\{v_0,q_b,q_c\},v_1)\cap
B(v_0,b) \cap \op{rcone}_-^0(v_0,v_1,y/(2b)).
\end{array}
$$
\end{lemma}

\begin{proved}
\swallowed\end{proved}
\end{tarski}
%<<<<<






%>>>>>
\begin{tarski}
\name{CC}
\summary{The intersection of a ball, right-circular cone and half-space can give a frustum.}
\tag{pt2, rcone, halfspace, bis, ball}
\rating{80}
\guid{WBHWSGD}

\begin{lemma}\tlabel{tarski:CC}\usage{lemma:sovo:CC}
Let $\{v_0,v_1\}$ be a set of two points in $\ring{R}^3$>
let $t,\mu > 0$.  Set $y=|v_0-v_1|$
and $r=\op{cosarc}(y,t,\mu)$.
Let $C=\op{rcone}^0(v_0,v_1,r)$.
Let $P$ be the half-space containing $v_0$ bounded by
the perpendicular bisector of  $\{v_0,v_1\}$.
Let $CC_1 = C\cap P \cap B(v_0,t)$.
Suppose that $y/(2t) \ge r$.
Then
  $$
  \op{rcone}^0(v_0,v_1,y/(2t))\cap CC_1 = FR(v_0,v_1,h,h/t)
  $$
\end{lemma}

\begin{proved}
\swallowed\end{proved}
\end{tarski}
%<<<<<





%>>>>>
\begin{tarski}
\name{CCbar}
\summary{The part inside one right-circular cone and outside another intersected with a ball
and half-space equals the same region without the half-space constraint.}
\tag{pt2, rcone, ball, halfspace}
\rating{80}
\guid{DJPAABI}

\begin{lemma}\tlabel{tarski:CCbar}\usage{lemma:sovo:CC}
Let $\{v_0,v_1\}$ be a set of two points in $\ring{R}^3$>
let $t,\mu > 0$.  Set $y=|v_0-v_1|$
and $r=\op{cosarc}(y,t,\mu)$.
Let $C=\op{rcone}^0(v_0,v_1,r)$.
Let $P$ be the half-space containing $v_0$ bounded by
the perpendicular bisector of  $\{v_0,v_1\}$.
Let $CC_1 = C\cap P \cap B(v_0,t)$.
Let $\bar A = \op{rcone}^0_-(v_0,v_1,y/(2t))$
Suppose that $y/(2t) \ge r$.
Then
  $$
  \bar A \cap CC_1 = \bar A \cap C \cap B(v_0,t).
  $$
\end{lemma}

\begin{proved}
\swallowed\end{proved}
\end{tarski}
%<<<<<





%>>>>>
\begin{tarski}
\name{rCCinvert-rad}
\summary{A particular set defined by a right-circular cone and half-spaces is radial.}
\tag{pt4, rcone, halfspace, bis, ball, circum3, collinear, conv2, halfspace}
\rating{80}
\guid{STPHWLD}

\begin{lemma}\tlabel{tarski:rCCinvert-rad}\usage{sec:inverted}
Let $\{v_0,v_1,w_1,w_2\}$ be a set of four points in $\ring{R}^3$>
let $t,\mu > 0$.  Set $y=|v_0-v_1|$
and $r=\op{cosarc}(y,t,\mu)$.
Let $C=\op{rcone}^0(v_0,v_1,r)$.
Let $P$ be the half-space containing $v_0$ bounded by
the perpendicular bisector of  $\{v_0,v_1\}$.
Let $CC = C\cap P \cap B(v_0,t)$. % subscript 1 deleted on CC, 4/7/2008.
Let $p_i$ be the circumcenter of $\{v_0,v_1,w_i\}$.  Let
$q_i$ be any point not collinear with $\{v_0,v_1\}$ that
lies on the plane perpendicular to $\op{aff}(\{v_0,v_1,w_i\})$
through the line $\op{aff}\{v_0,v_1\}$.
Let $A_i = \op{aff}_-^0(\{v_0,p_i,q_i\},v_1)$.
Suppose that $y/(2t) \ge r$ and $|w_1-w_2| > 2t$.
Then
  $$
  (CC \cap A_1\cap A_2) =   (CC \cap A_1\cap A_2) \cap B(v_0,t).
  $$
\end{lemma}

%% WWInclude later
%\begin{figure}[htb]
% \centering
%  \myincludegraphics{\ps/samfigA54.eps}
%  \caption{Different forms of given region.   The
%  structure shown in the middle frame cannot occur.}
%  \tlabel{fig:chi-anal-vs-geom}
%\end{figure}


\begin{proved}
We let $c_0$ be the point at distance $t$ from $v_0$ 
on the intersection of the
planes $\{v_0,v_1,w_1\}^\perp$ and $\{v_0,v_1,w_2\}^\perp$. 
consider the figure as a function of $y_4=|w_1-w_2|$. When $y_4$ is
sufficiently large the claim is certainly true.  Contract $y_4$ until
$c_0=c_0(y_4)$ meets the perpendicular bisector of $\{v_0,v_1\}$. Then $c_0$
is equidistant from $v_0,v_1,w_1$ and $w_2$ so it is the circumcenter of
$\{v_0,v_1,w_1,w_2\}$. It has distance $t$ from the origin, so the
circumradius is $t$. This implies that $y_4\le 2t$.
\swallowed\end{proved}
\end{tarski}
%<<<<<






%>>>>>
\begin{tarski}
\name{2CCrad}
\summary{Two sufficiently separated right-circular cones meet only at their common apex.}
\tag{t0, 3.2, 1.945, rcone}
\rating{80}
\guid{JNNCRMJ}

\begin{lemma}\tlabel{tarski:2CCrad}
Let $\{v_0,v_1,v_1'\}$ be a set of three points in $\ring{R}^3$.
Let  
$\mu = 3.2 - 1.255 = 1.945$.  Let $y = |v_1-v_0|$ and $y'=|v_1'-v_0|$.
Set $\psi = \arc(y,1.255,\mu)$.
Assume that $|v_1-v'_1|\ge 3.2$ and that $y,y'\le 2.51$.
Then 
   $$
   \op{rcone}(v_0,v_1,y/(2.51))\cap \op{rcone}(v_0,v_1',\cos\psi)
   =\emptyset .
   $$
\end{lemma}


\begin{proved}
Let $p$ be a point in the intersection.  Both sets are cones
that are symmetrical through the plane $A=\op{aff}(v_0,v_1,v_1')$.
Thus, the reflection $p'$ of $p$ through $A$ is also
in the intersection.  By the convexity of the two sets so is
the midpoint $(p+p')/2$.  Hence, we may assume without loss of
generality that $p\in A$.

Since both sets are cones,
we can scale $p$ to get another point in the intersection  
that belongs to the bisector of $\{v_0,v_1\}$.  We may
move along the bisector towards $\op{aff}(v_0,v_1')$ until
the boundary of $\op{rcone}(v_0,v_1,y/(2t))$ is reached.
Changing notation,
we assume that $p$ is this point on the bisector.
We then have $|p-v_1|=|p-v_0|=t$, $|p-v_0|< \mu$.
This violates the triangle inequality:
  $$
  3.2\le |v_1-v_1'| \le |v_1-p| + |p-v_1'| < t + \mu = 3.2. 
  $$
\swallowed\end{proved}
\end{tarski}
%<<<<<





%>>>>>
\begin{tarski}
\name{AA-prime}
\summary{The intersection of two suitably restrained right-circular cones is confined to the union of two cones.}
\tag{pt3, rcone, halfspace, cone, aff3}
\rating{80}
\guid{LIBCELL}

\begin{lemma}\tlabel{tarski:AA-prime}\usage{lemma:2tcc}
Let $\{v_0,w_1,w_2\}$ be a set of three points in $\ring{R}^3$.
let $t,\mu > 0$.  Set $y_i=|v_0-w_i|$
and $r_i=\op{cosarc}(y_i,t,\mu)$.
Let $C_i=\op{rcone}^0(v_0,w_i,r_i)$.
%Let $P_i$ be the half-space containing $v_0$ bounded by
%the perpendicular bisector of  $\{v_0,w_i\}$.
% Next two lines commented out 4/7/2008.
%Let $CC_i = C_i\cap %P_i \cap 
% B(v_0,t)$.
Let $q$ and $q'$ be the two points defined by distances
$t$ from $v_0$, $\mu$ from $w_1$, and $\mu$ from $w_2$.
(The existence of two such points is given by Lemma~\ref{tarski:mk-point}.)
Let $A=\op{aff}_+(v_0,\{q,w_1,w_2\})$ and
$A'=\op{aff}_+(v_0,\{q',w_1,w_2\})$.
Suppose that $y_i/(2t) \ge r_i$.
% 4/7/2008 CC_ changed to C_, If the right-hand side is conic, then why should we intersect $C_i$ with a ball?
Then $$C_1\cap C_2 \subset A\cup A'\cup
      \op{aff}\{v_0,w_1,w_2\}.$$
\end{lemma}

\begin{proved}
\swallowed\end{proved}
\end{tarski}
%<<<<<





%>>>>>
\begin{tarski}
\name{rcone2}
\summary{The intersection of two right-circular cones increases as their directions move closer together.}
\tag{pt3, pt4, rcone}
\rating{80}
\guid{RJUGGRQ}

\begin{lemma}\tlabel{tarski:rcone2}\usage{lemma:2tcc}
Let $\{v_0,w_1,w_2\}$ be a set of three points in $\ring{R}^3$.
Let $w_3 \in \op{aff}_+^0(v_0,\{w_1,w_2\})$.
let $t,\mu > 0$.  Set $y_i=|v_0-w_i|$
and $r_i=\op{cosarc}(y_i,t,\mu)$.
Let $C_i=\op{rcone}^0(v_0,w_i,r_i)$.
Then
  $$
  C_1 \cap C_2 \subset C_1 \cap C_3.
  $$
\end{lemma}

\begin{proved}
\swallowed\end{proved}
\end{tarski}
%<<<<<






%>>>>>
\begin{tarski}
\name{RCFR}
\summary{The intesection of a ball with a right-circular cone is contained in a truncated right-circular cone.}
\tag{pt2, ball, rcone}
\rating{80}
\guid{EFDFOTF}

\begin{lemma}\tlabel{tarski:RCFR}\usage{lemma:sovo:CR}
Let $\{v_0,v_1\}$ be a set of two points in $\ring{R}^3$.
Assume that $0 < t \le h \le c$.  Then
$$
B(v_0,t) \cap \op{rcone}^0(v_0,v_1,h/c) \subset
FR(v_0,v_1,h,h/c).
$$
\end{lemma}

\begin{proved}
\swallowed\end{proved}
\end{tarski}
%<<<<<








%>>>>>
\begin{tarski}
\section{Extracted from hypermap}
\name{dot-pos}
\summary{If two points lie in the same blade, their dot product is positive.}
\tag{pt2, collinear,blade}
\rating{40}
\guid{TMDMILH}

\begin{lemma}\tlabel{tarski:dot-pos}
Let $\{u,v\}$ be a set of two points in $\ring{R}^3$.
Assume that $\{0,u,v\}$ are not collinear.
If $w,w'\in\op{aff}_+(0,\{u,v\})$, then
$w\cdot w' > 0$.
\end{lemma}

\begin{proved}
\swallowed\end{proved}
\end{tarski}
%<<<<<



%>>>>>
\begin{tarski}
\name{miss-plane}
\summary{Given a blade, a plane can be found that meets the blade only at its apex.}
\tag{pt3, pt5, collinear}
\rating{80}
\guid{TDVIYHO}

\begin{lemma}\tlabel{tarski:miss-plane}\usage{lemma:dart-curve}
Let $\{v_0,v_1,v_2\}$ be a set of three points in $\ring{R}^3$.
Assume that the set is not collinear.  Then there exists points
$w_1,w_2$ such that $\{v_0,w_1,w_2\}$ are not collinear and such
that 
   $$
   \op{aff}(v_0,w_1,w_2)\cap \op{aff}_+^0(v_0,\{v_1,v_2\})=\emptyset .
   $$
\end{lemma}

\begin{proved}
Let $e_i = (v_i-v_0)/|v_i-v_0|$, for $i=1,2$. Let 
$e = e_1+e_2$.  By the first collinearity assumption, we have
$e\ne 0$.  Let $w_1,w_2$ be any two points such that
$(w_i-v_0)\cdot e = 0$ and $\{v_0,w_1,w_2\}$ are not collinear.

If 
 $$
 v_0 + t_1 (w_1-v_0) + t_2 (w_2-v_0)=v_0 + s_1 (v_1-v_0) + s_2 (v_2-v_0),
 $$
then taking the dot product of both sides with $e$, we get
$$
  0 = (s_1 (w_1-v_0) + s_2 (w_2-v_0))\cdot e.
$$
However, $(s_1 (w_1-v_0) + s_2 (w_2-v_0)$ and $e$ are both
in $\op{aff}_+^0(0,\{w_1-v_0,w_2-v_0\})$, so the dot product
must be positive (by Lemma~\ref{tarski:dot-pos}).
\swallowed\end{proved}
\end{tarski}
%<<<<<






%>>>>>
\begin{tarski}
\section{Extracted from fan}
\name{subvertex}
\summary{Two points can be shown distinct, by separating them into different regions of a lune.}
\tag{pt5, pt7, aff-meet}
\rating{40}
\guid{EZLYYDN}

\begin{lemma}\tlabel{tarski:subvertex}\usage{dist2,hypermap.tex}
Let $\{v_0,v_1,v_2,v,w\}$ be a set of five points in $\ring{R}^3$.
Assume that $\op{aff}_+^0(v_0,\{v,w\})$ meets $\op{aff}_+(v_0,\{v_1,v_2\})$.
Let $w_1\in \op{aff}_+(v_0,\{v,v_1,v_2\})$ and $u$ be points such
that $\op{aff}_+^0(v_0,\{w_1,u\})\cap \op{aff}_+(v_0,\{v_1,v_2\})=\emptyset$.
Then $u\ne w$.
\end{lemma}

\begin{proved} 
We have that $w$ is contained in the lune $\op{aff}_+(\{v_0,v\},\{v_1,v_2\})$.
It is not contained in $\op{aff}_+(v_0,\{v,v_1,v_2\})$.  Hence,
$\op{aff}_+(v_0,\{w_1,w\})$ meets $\op{aff}_+(v_0,\{v_1,v_2\})$.  However,
if $w=u$, these two sets are disjoint.  Hence $u\ne w$.
\swallowed\end{proved}
\end{tarski}
%<<<<<



%>>>>>
\begin{tarski}
\name{cone4.12.1}
\summary{Under suitable constraints, the intersection of a right-circular cone, blade, and lune is empty.}
\tag{pt4, pt6, packing, t0, 3.2, rcone, lune, blade, 1.945}
\rating{80}
\guid{NLZKMES}

\begin{lemma}\tlabel{tarski:cone4.12.1}
%    \oldlabel{4.12.1}
Let $\{v_0,v_1,v_2,v_3\}$ be a packing of four points.
Assume that $|v_1-v_2|\le 3.2$.  Let $w_1,w_2$ be points (not necessarily
distinct from $v_1,v_2$)
in $\op{aff}_+(v_0,\{v_1,v_2,v_3\})$ such that
  $$|w_1-w_2|\ge 2,\ |v_3-w_1|\ge 3.2,\ |v_3-w_2|\ge 3.2.
  $$
Then 
  $$\op{rcone}^0(v_0,v_3,\cosarc(2,1.255,1.945)) \cap \op{aff}_+^0(v_0,\{v_1,v_2\})
  \cap \op{aff}_+(\{v_0,v_3\},\{w_1,w_2\})
=\emptyset .
  $$
\end{lemma}

%\begin{lem}\guid{WIRABFB}\tlabel{tarski:4.12.1}
%%    \oldlabel{4.12.1}
%Let $v$ be a concave vertex with $|v-\orgn|\ge2.2$. The truncated
%corner cell at $v$ with parameter $\lambda=1.945$ lies in the truncated
%$V$-cell over $R$.
%\end{lemma}

\begin{proved}
Set $\lambda=1.945$, $T=2.51$, and
    $\theta = \arc(2,1.255,\lambda)< 1.21 <\pi/2$.  
They angle between $v_3,w_i$ at $v_0$ is at least
$\arc(T,T,3.2)>1.38$. The angle between
$v_1,v_2$ at $v_0$ is at most
$\arc(2,2,3.2)<1.86$.

Consider the spherical triangle on the unit sphere centered at $v_0$
formed by
 $\op{aff}_+(v_0,\{v_3,w_1\})$, $\op{aff}_+(v_0,\{v_3,w_2\})$ 
(extended as needed) and $\op{aff}_+(v_0,\{v_1,v_2\})$. 
Let $C$ be the radial projection of
$v_3$ to the unit sphere, 
and let $AB$ be the edge of the spherical triangle corresponding to
$\{v_1,v_2\}$. Pivot $A$ and $B$ toward $C$ until the edges $AC$ and
$BC$ have arclength $1.38$.  The perpendicular from $C$ to $AB$
has length at least
    $$\arccos(\cos(1.38)/\cos(1.86/2))>1.21>\theta .$$
This proves that the intersection is empty.
\swallowed\end{proved}
\end{tarski}
%<<<<<




%>>>>>
\begin{tarski}
\name{old4.12.2}
\summary{Under suitable constraints, the intersection of a right-circular cone, blade, and lune is empty.}
\tag{t0, pt4, pt6, packing, 3.2, 3.07, 1.815, rcone, blade, lune}
\rating{80}
\guid{CNISQSN}

\begin{lemma}\tlabel{tarski:old4.12.2}
%%    \oldlabel{4.12.2}
%Let $v$ be a concave vertex. The truncated corner cell at $v$ with
%parameter $\lambda=1.815$ lies in the truncated $V$-cell over $R$.
Let $\{v_0,v_1,v_2,v_3\}$ be a packing of four points.
Assume that $|v_1-v_2|\le 3.2$.  Let $w_1,w_2$ be points (not necessarily
distinct from $v_1,v_2$)
in $\op{aff}_+(v_0,\{v_1,v_2,v_3\})$ such that
  $$|w_1-w_2|\ge 2,\ |v_3-w_1|\ge 3.07,\ |v_3-w_2|\ge 3.07.
  $$
Then 
  $$\op{rcone}^0(v_0,v_3,\cosarc(2,1.255,1.815)) \cap \op{aff}_+^0(v_0,\{v_1,v_2\})
  \cap \op{aff}_+(\{v_0,v_3\},\{w_1,w_2\})
=\emptyset .
  $$
\end{lemma}

\begin{proved}
The proof proceeds along the same lines as the previous lemma, with
slightly different constants. Replace $1.945$ with $1.815$, $1.38$ with
$1.316$, $1.21$ with $1.1$. Replace $3.2$ with $3.07$ in contexts related
to the lower bound on $|v_3-w_i|$ and keep it at $3.2$ in
remaining contexts. The constant $1.86$ remains unchanged.
\swallowed\end{proved}
\end{tarski}
%<<<<<



%>>>>>
\begin{tarski}
\name{cone-meet}
\summary{If two right-circular cones meet at more than a common apex, then their generators must be suitably close to each other.}
%\endnote{Three points, corner cells do not meet, tarski:cone-meet}
\tag{pt3, rcone}
\rating{80}
\guid{OXAYQXU}

\begin{lemma}\tlabel{tarski:cone-meet}
Let $\{v_0,v,w\}$ be a set of three points in $\ring{R}^3$.  Let $t,\lambda,\lambda'>0$.  If 
$$\op{rcone}(v_0,v,\cosarc(|v-v_0|,t,\lambda)) \text{ meets }
\op{rcone}^0(v_0,w,\cosarc(|v-v_0|,t,\lambda')),$$ then
$|v-w| < \lambda+\lambda'$.
\end{lemma}

\begin{proved} Pull $v$ away from $w$ keeping $|v|$ fixed
until the intersection of the two
cones is empty, and the intersection of the corresponding closed cones
is a ray tangent to both cones.  Let $v'$ be the point into which
$v$ is transformed.  The point $p$ at distance $t$ from $v_0$ along
the ray has distance $\lambda$ from $v'$ and $\lambda'$ from $w$.
Then
  $$
  |v-w| < |v'-w| \le |v'-p| + |p-w| \le \lambda + \lambda'.
  $$
\swallowed\end{proved}
\end{tarski}
%<<<<<
\end{tarskidata}



\filetarskiaway