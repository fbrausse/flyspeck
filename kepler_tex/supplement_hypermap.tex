% Created Dec 13, 2013.
% Supplementary notes on the hypermap chapter.

Revision 3/16/2014

\asection{Appendix on connecting with Bauer-Nipkow 
work in Isabelle}

This section builds on the HOL-Light files hypermap.hl
and import-tame-classificaton.hl.

For simplicity of typesetting, we use a dash in this text to
represent an undescore in HOL Light.

\asubsection{basic definitions}

(This section has been formalized 2/2014.)

In the Bauer-Nipkow work, there is a type for Isabelle graphs,
which we abbreviate to lgraph.   There is a predicate
planegraph that expresses lgraph planarity.  There is also a slightly
broader class planegraph-relaxed.

\begin{lemma}\guid{DPZGBYF} 
\formalauthor{tchales}
If $g$ is planegraph, then it is also planegraph-relaxed.
\end{lemma}

There is a function, fgraph, that maps an lgraph to
the list of list representation of the lgraph.  For example,
the list of list representation of a planar graph consisting
of a square triangulated into four squares with common
vertex $0$ is
\[
[[0;1;2];[0;2;3];[0;3;4];[0;4;1];[4;3;2;1]].
\]
We refer to these entries $[0;1;2]$, $[0;2;3]$, etc. as the
faces of the lgraph.

There is a function, hypermap-of-list, that maps the
list of list representation of the planar graph into the
corresponding hypermap.  The darts of the hypermap
are consecutive pairs of elements in the list of list
representation.  For example, the example above gives
the dart set
\[
\{(0,1),(1,2),(2,0),~(0,2),(2,3),(3,0),~(0,3),(3,4),(4,0),~
(0,4),(4,1),(1,0),~(4,3),(3,2),(2,1),(1,4)\}.
\]
The face map follows each pair around the face in which it occurs
$f(0,1) = (1,2)$, etc.  The edge map reverses an ordered pair $e(0,1)
= (1,0)$, etc.  The set of integers that appear in this list of list
is in bijection with the set of vertices of the lgraph.  In fact, we
can take this is as the definition of the set of vertices of the
lgraph.  When we refer to the hypermap of $g$, we refer to this
construction.

There is a function, finals, that returns a sublist of the
list of list representaton of the lgraph.

In the Isabelle development, there is a function facesAt.



Since our aim is relate hypermaps to the constructions in Isabelle, which
are based on lists, we give a series of list-based definitions that run
in parallel with hypermap-based definitions.

We make a table of notions about hypermaps and their corresponding list-based notion.
This table serves as a dictionary translating between the different data structures.

\bigskip
\begin{tabular}{l l l}
\hline
{\bf hypermap} & {\bf lists} & {\bf theorem}\\ \hline
loop & loop-list \\
is-contour & contour-list & contour-list-is-contour\\
is-normal & normal-list & normal-list-normal\\
hyp'm, hyp'p, hyp'q & l'm, l'p, l'q \\
hyp'y, hyp'z & l'y, l'z \\
final-quotient-face   & final-list \\
quotient & quotient-list \\
- & final-dart-list \\
- & split-normal-list \\
transform & transform-list \\
hyp'S & s-list \\
s-flagged & s-flag-list \\
empty-flagged & flag-list \\
is-marked & marked-list \\
iso & iso-list \\
hyp-iso & isop-list \\
f-map, n-map, e-map & f-list, n-list, e-list\\
dih2k & dih2k-list\\
atom-choice & find-atom \\
atom & atom-list \\
head-of-atom & TAIL \\
tail-of-atom & HD \\
is-split-condition & split-condition-list \\
index & indexf & index-indexf \\
loop-of-face & loop-of-list & loop-of-face-list \\
\hline
\end{tabular}

\bigskip
We also describe the general correspondence of variable names.

\begin{tabular}{l l l}
\hline
{\bf hypermap} & {\bf lists} & {\bf function}\\ \hline
H (hypermap) & L (list of lists) & hypermap-of-list\\
NF (normal family) & N & loop-family-of-list \\
L (loop) & r & loop-of-list \\
x (dart) & x \\
\hline
\end{tabular}

\bigskip
We also have conversions from lists to hypermaps.  The constant loop-of-list converts a list to a loop,
and loop-family-of-list converts a list of lists to a family of loops.

There is a function match-core-list that is true when there is a correspondence between
faces of an lgraph g and those of the quotient of L by N, such that the final faces of g are included in those of $N$.
A function match-list also measures the corespondence.  The function loop-choice generates a pair (r,x)
to be used as part of a marked-list, from a lgraph g.





\asubsection{properties of planegraph}

\begin{definition}[good-list,~good-list-nodes,~good-graph]
Recall that a list of lists $L$ is a good-list if it has three properties:
\begin{enumerate}
\item The list of darts of $L$ has no duplicates.
\item Every member of $L$ is non-nil.
\item If $(x,y)$ is a dart of $L$, then so is $(y,x)$.
\end{enumerate}
Recall that a good list $L$ is a good-list-nodes if 
the number of nodes in its hypermap is equal to the number
of vertices of $L$.
We say that an lgraph is is good-graph if it has the properties:
\begin{enumerate}
\item Its fgraph $L$ is a good-list and a good-list-nodes.
\item all faces of the fgraph are final.
\item In each face, each vertex occur at most once: (all uniq L).
\item The vertex set coincides with the set of elements of the fgraph.
\item The facesAt a vertex $v$ is the same as the list of faces that contain $v$.
\end{enumerate}
\end{definition}

(We don't need to formalize JUJUWAT, CGGZYRC, EAHHATZ, ETDLJXT, and HWDMZDM
 because we will get them as a corollary
of the correspondence between restricted hypermaps and plane-graphs.)

\begin{lemma}\guid{JUJUWAT} 
If $g$ is planegraph-relaxed, then its fgraph is a good-list.
\end{lemma}  

\begin{lemma}\guid{CGGZYRC} 
Let $g$ be a planegraph.  Let $F$ be a face of $g$
that is not final.  Let $(x,y)$ be a dart of (the hypermap of) $g$ on face
$F$.
Then $(y,x)$ is a dart on a face that is final.
\end{lemma} 
 
\begin{lemma}\guid{EAHHATZ}
If $g$ is a Plane-graph, then it is a good-graph.
\end{lemma}

\begin{lemma}\guid{ETDLJXT}
If $g$ is planegraph-relaxed, then  the vertices of $g$ are $0,\ldots,n-1$, where
$n$ is the number of nodes of $g$. 
\end{lemma} 

\begin{lemma}\guid{HWDMZDM}
Let $g$ be planegraph-relaxed, and $v$ in the vertex setof $g$.
Then $facesAt~g~v$ is the set of faces of $g$ that $v$ is a member of.
\end{lemma}


In general we want to have the properties of good-list, good-list-nodes, and all-uniq.
The following lemmas show how these properties progagate.
The following properties can be proved by structural induction
for planegraph-relaxed.  (We no longer need to use planegraph-relaxed; planegraph will
be sufficient for our purposes.)



\begin{lemma}\guid{PMBRINH}  If $L$ is good-list-nodes and normal-list-L-N, then good-list-nodes
 holds of quotient-list-L-N.
\end{lemma}

\begin{proof}  It is enough to show that all of the darts with a given first coordinate form a single node of the quotient.
This is clear, because by assumption all the darts of $L$ with a given first coordinate for a single node for $L$.
\end{proof}

\begin{lemma}\guid{SNVACWG}
\formalauthor{tch 3/2014}  
If $L$ is a good-list and good-list-nodes, its hypermap is restricted,
and normal-list-L-N, then good-list holds of quotient-list-L-N.
\end{lemma}

\begin{proof} We examine each of the defining properties of a good-list.  

\case{uniq}  
If a dart $(a,b)$ occurs more than once  then there are two darts $(x_1,y_1)$ and $(x_2,y_2)$ in $N$, that run from
nodes $a$ to $b$.  This is impossible by the no-double-joins property of restricted hypermaps.

\case{all-non-nil}  It is a property of normal that its members are not nil.  The same is then true of the quotient.

\case{sym}.  Let $(a,b)$ be a dart in the quotient.  Then there is a dart $(a,b)$ in $N$.  In fact, we see that $(a,b)$ is the last member of
its atom in $N$.  The family $N$ meets the node $b$,
and $(b,a)$ is a dart at the node $b$.  By properties of normal, all darts at the node, including $(b,a)$, are on $N$.
We have $n^{-1}(b,a) = f (a,b)$.  By unicity, $(b,a)$ is followed by an $f$-step in $N$.  This implies that $(b,a)$ is a dart in the
quotient. 
\end{proof}

\begin{lemma}\guid{LYNVPSU}
\formalauthor{tch 3/3/2014}  
$(L,N,r,s)$ be a marked list.  Then the quotient of $L$ by $N$ is all-uniq.
\end{lemma}

\begin{proof}  This is a defining property of a marked hypermap.
\end{proof}

\begin{lemma}\guid{LEBHIJR}
Let $L$ be a good-list and good-list-nodes.
Let $(L,N,r,x)$ be a marked list.  Let $(N',r')$ be the transform.  Then the set of elements of $L//N'$
is the union of the set of elements of $L//N$, together with the set of nodes that visit $N'$ but not $N$.
\end{lemma}

\begin{lemma}\guid{HOJODCM}
Let $L$ be a good-list and good-list-nodes.
Let $(L,N,r,x)$ be a marked list with transform $(N',r')$.  Let $r''$ be the offshoot of the transform.
Let $V$ be the set of nodes that visit $N'$ but not $N$.
Let $s$ be a loop in $N'$, and $v$ a node of $N'$.  Then we have the following
rules describing when $v$ is visited by $s$.
\begin{enumerate}
\item If $s\ne r',r''$, then $v$ is visited by $s$ iff $v$ is visited by $s$ in $N$.  
\item If $v\in V$, then $v$ is visited by $s$ iff $s = r'$ or $r''$.  Now assume $v\not\in V$.
\item $v$ is visited by $r'$ iff $v$ is at the node of $l'z$, $l'y$ or between them.
\item $v$ is visited by $r''$ iff $v$ is at the node of $l'y$, $l'z$ or between them.
\end{enumerate}
\end{lemma}


\asubsection{main result}

This result is imported from Isabelle.

\begin{theorem}[Import~Tame~Classification]  Let $g$ be a final
planegraph that is tame.  Then there exists a $y$ in the archive such
that the fgraph of g is fgraph congruent to $y$.
\end{theorem}

The main result to be proved is the following lemma.

\begin{lemma}[LSKOKJE]  Let $H$ be a restricted hypermap.  Then there exists a good-graph g that is a PlaneGraphs,
such that $H$ is isomorphic to the hypermap of $g$.
\end{lemma}

\asubsection{isomorphism}


We develop the properties of iso-list, which is the list-based version of isomorphism.
It is a particularly rigid notion, because it does not allow the rearrangement of the elements of a list.
However, it is sufficient for our purposes.

We often need to consider $L$ together with a normal-list $N$.  The definition of iso-list carries with it
a second argument that checks that $N$ is sent to $N'$ under the isomorphism.

\begin{lemma}\guid{GNBEVVU}
\formalauthor{tchales} 
If $L$ is a good-list and $L'$ is an iso-list, then the hypermaps of $L$ and $L'$ are isomorphic.
\end{lemma}

\begin{proof} This follows directly from the theorem hypermap-of-list-map.
\end{proof}

\begin{lemma} [iso-list-refl,~iso-list-sym,~iso-list-trans] 
\formalauthor{tchales}
The relation iso-list is reflexive, symmetric, and transitive.
\end{lemma}

The properties of a list $L$ that are invariant under an injective map remain true when $L$ is replaced
with an iso-list.  This gives the following results.

\begin{lemma}\guid{PEUTLZH}
\formalauthor{tchales, 2/2014} 
Let $L$ be a good-list, and $L'$ an iso-list.  Then $L'$ is a good-list.
\end{lemma}

\begin{lemma}\guid{OISRWOF}
\formalauthor{tchales, 2/2014} 
Let $L$ be good-list and good-list-nodes, and $L'$ an iso-list.  Then $L'$ is a good-list-nodes.
\end{lemma}

\begin{lemma}\guid{UEYETNI} 
\formalauthor{tchales, 2/2014}
Let $L$ be all-uniq, and $L'$ an iso-list.  Then $L'$ is a all-uniq.
\end{lemma}

\begin{lemma}\guid{XKDZKWV}
\formalauthor{tch, 2/25/2014}
 Let $L$ be a good-list with normal-list $N$.  If $(L',N')$ is iso-list, then $N'$ is a normal-list of $L'$.
\end{lemma}

\begin{lemma}\guid{DAKEFCC}\formalauthor{tch 3/10/2014}
If $H$ and $K$ are isomorphic hypermaps, and if $H$ is restricted, then $K$ is restricted.
\end{lemma}

\begin{lemma}\guid{MEEIXJO}
Let $L$ be a good-list with marked-list $(L,N,r,x)$.  If $(L',N',r',x')$ is iso-list, then $(L',N',r',x')$ is a marked-list.
\end{lemma}



\asubsection{representing hypermap as lists}

It is convenient to work with lists of list $L$ rather than hypermaps $H$.  This means
we should give a list version of notions such as loop and normal family.
We have loop-list as the list version of a loop and norm-list as the list
version of  a normal family of loops.  




\begin{lemma}\guid{RXOKSKC}
\formalauthor{HLT, 2/26/2014}
Let $H$ be a restricted hypermap.  Then there exists a list $L$ such that $H$ is its hypermap.
In fact, there exists such an $L$ that is good-list, good-list-nodes, all-uniq.
\end{lemma}

\begin{proof} We may convert a hypermap to list $L$ as follows.  From each face $F$ of $H$
pick a dart $x$ and form the list $[\op{node}(x),\op{node}(f x),\op{node}(f^2 x),\ldots]$.  
The list of all these lists for each face is defined to be $L$.

A restricted hypermap is simple, hence the elements of the list are uniq.  This gives the all-uniq property.

By construction the elements of these lists are the nodes.  This gives the good-list-nodes property.

We check that it is a good list.  
For the uniqueness property, note that we show below that the map between darts sets is one-to-one.
This means that each dart appears in a face once.

For the non-nil property of good-lists: each list contains $\op{node}(x)$ as the first element. Hence it is
not the nil-list.

For the symmetry property of good-lists: if $(\op{node}(x),\op{node}(f x))$ is a dart, then
\[
(\op{node}(n f x),\op{node}(f n f x)) = (\op{node}(f x),\op{node}(x))
\]
is also a dart.

Finally, we check that $H$ is isomorphic to the hypermap of $L$.  The darts of the hypermap of $L$ are formed by
consecutive pairs $(\op{node}(f^i x),\op{node}(f^{i+1} x))$.  

We claim the map between dart sets, $y\mapsto (\op{node}(y),\op{node}(f y))$, is one-to-one.
This follows from the no-double-joins property of restricted hypermaps.  Specifically,
if $(\op{node}(y),\op{node}(f y)) = (\op{node}(z),\op{node}(f z))$, then the edges $\{y,e y\}$ and $\{z,ez\}$ run
between the same nodes, forcing $y=z$.

The face maps are compatible on the two hypermaps by construction. Finally, the edge map is compatible because
\[
(\op{node}(e y),\op{node}(f (e y))) = (\op{node}(f y),\op{node}(y)) = e-list (\op{node}(y),\op{node}(f y)).
\]
\end{proof}

\begin{lemma}\guid{JXBJOAB}
\formalauthor{tchales, 2/2014}  
If $L$ is any list of lists, then there is a iso-list $L'$ whose elements are natural numbers.
\end{lemma}

\begin{proof} Pick any injective map from the elements of $L$ to $\ring{N}$ and take the resulting iso-list.
\end{proof}

For the correspondence with Isabelle, we will need to use lists $L$ whose elements are natural numbers and
whose elements are ordered in a particular way.  By taking a permutation on the natural numbers we can obtain
any convenient ordering.  A particular ordering is given by lemma SHXWKXQ.  We refer to the formal statement
in the HOL Light files.  What follows is an informal summary.

\begin{lemma}\guid{SHXWKXQ}.  For any $L$ (good-list, good-list-nodes, all-uniq)
and normal-list $N$.  By passing to an iso-list, we can arrange that the elements appearing
in $N$ are the natural numbers $\{0,1,\ldots,n-1\}$, where $n$ is the number of vertices visited by $N$.
Furthermore, if $x$ is any dart, and $F$ is the face in $L$ containing $x$ listed so that $x$ appears at the head of the
list.  Let $F'$ be the filtered list of $F$ containing all the darts of $F$ that are not in $N$.  Then we may also assume that
$F' = [n+1;n+2;n+k]$, where $k$ is the length of $F'$.
\end{lemma}

The following lemma gives a reduction of the main result LSKOKJE.

\begin{lemma}[JCAJYDU]\formalauthor{HLT, 2/28/2014}
 It is enough to prove LSKOKJE in the case when $H$ is the hypermap of $L$,
where $L$ is a good-list, good-list-nodes, and having elements in the natural numbers.
\end{lemma}


\asubsection{translating notions between hypermaps and lists}


We skip the following proofs.  In each case the definitions involved in the correspondence
are direct translations of one another.  Thus, the proofs are a matter of comparing defintions.

\begin{lemma}\guid{GZLJIGN}\formalauthor{tch 3/9/2014} 
Under the correspondence between $L$ and its hypermap, the constants
$l'm$, $l'p$, $l'q$, $l'y$, $l'z$ correspond respectively to hyp'm, hyp'p, hyp'q, hyp'y, hyp'z.
\end{lemma}

\begin{lemma}\guid{EVNAPDQ}\formalauthor{tch 2/28/2014} 
Let $L$ be a restricted, good-list and good-list-nodes.  Let $N$ be a normal list of $L$.
Then the loop family is a normal family of the hypermap of $L$.  
\end{lemma}

\begin{lemma}\guid{ABKCJWD}\formalauthor{tch 3/8/2014} 
Let $L$ be a good-list.  Let $N$ be a normal list of $L$.
Then the hypermap of the quotient-list is isomorphic to the quotient of the data
on the hypermap side.
\end{lemma}

\begin{proof} Their respective definitions are direct translations of one another.
\end{proof}

\begin{lemma}\guid{ODWAFRG}\formalauthor{tch 3/8/2014} 
 Let $L$ be a good-list and $(L,N,r,x)$ a marked-list.
Then the corresponding data in the hypermap is is-marked.
\end{lemma}

There are also specific lemmas about hypermaps that we wish to translate back into theorems
about lists.  Specifically, we have lemmas HQYMRTX1-list, HQYMRTX2-list, HQYMRTX3-list,
QRDYXYJ-list, AQIUNPP1-list, and AQIUNPP2-list that are list versions of lemmas given in {\it Dense Sphere Packings}.
We omit the proofs here, because the proofs are direct translations from hypermap language to list language.
They have been deduced form the hypermap versions of the lemams using transform-assumption-v2.

\begin{lemma}[transform-assumption-v2]\guid{AQ} 
Let $L$ be good-list, good-list-nodes, with marked list $(L,N,r,x)$ and transform $(N',r')$.
Then marked-list $(L,N',r',x)$.
\end{lemma}


\asubsection{dihedral  initialization}

The fgraph of Seed~p is the list of lists $[[0;1;...;n+2];[n+2;n+1;...;0]]$.

\begin{lemma}[good-list-seed,~all-uniq-seed] \guid{FOEGZEQ}
\formalauthor{tch}
 Seed~p is a good-list and all-uniq.
\end{lemma}



\begin{lemma}\guid{ENWCUED}
\formalauthor{tch}  
Seed~p is dih2k-list.
\end{lemma}

\begin{lemma}\guid{TAGYMWJ}
\formalauthor{tch} 
Seed~p is a good-list-nodes.  More generally, for any $L$ that is dih2k-list,
it is good-list-nodes.
\end{lemma}


\begin{lemma}
\formalauthor{tch, 3/14/2014} 
If $L$ is dih2k-list, then it is all uniq.
\end{lemma}

\begin{lemma}
\formalauthor{tch, 3/14/2014} 
If $L$ is dih2k-list, then it is good-list.
\end{lemma}

The following is not needed:

\begin{lemma}\guid{DLWOJBB} If $L$ is dih2k-list, then its hypermap is dih2k.
\end{lemma}

\begin{lemma}\guid{KUKASGD}
\formalauthor{tch 3/14/2014} 
Any two dih2k-list's $L$ and $L'$ of the same face size are iso-lists.
\end{lemma}

\begin{lemma}\guid{UNHQYQM}
\formalauthor{tch, 3/14/2014} 
If $L$ and $L'$ are iso-lists and $L$ is dih2k-list, then so is $L'$.
\end{lemma}

This is a version of AUQTZYZ for lists.  It can be deduced from the hypermap version.

\begin{lemma}[AUQTZYZ-list]  Let $L$ be good-list with restricted hypermap.
Let $f$ be a face of $L$.  Then there exists $f'$ such that $[f;f']$ is normal and the quotient
is $[f,rev~f]$ (which is dih2k-list).
\end{lemma}

\begin{proof} Pick $f'$ to be the complementary path (a concatenation of complementary nodes) of $f$.
\end{proof}

The following is a more precise version of the correspondence of a restricted hypermap with
the seed.  We will apply this lemma to a face with the largest size of $L$.

\begin{lemma}\guid{UYOUIXG}  Let $L$ be a good-list with restricted hypermap.   Let $f$ be any face of $L$.  Then there exists data $(L',N')$ and $f'$
such that $(L,[f;f'])$ is an iso-list of $(L',N')$ and such that $N'$ is a normal-list of $L'$ and
match-quotient-list between the seed and $(L',N')$.
\end{lemma}

\begin{proof} This is obtained by renaming the elements of $L$ to agree exactly with those of the seed.
\end{proof}

\asubsection{termination}

We will obtain termination after a finite sequence of iterations of subdivFace.  Each subdivFace
will increase the number of final faces by one.  The number of final faces
 is no more than the number of faces in the original.  Finally,
if every face  is final, then every loop in $N$ is also final, and the quotient list of $L$ by $N$ is equal to $L$.

This is not needed. We will count final faces rather than darts.

\begin{lemma}\guid{ADACDYF}  Let $(L,N,r,x)$ be a marked-list in which $r$ is not final.
Let $(N',r')$ be the transform of the marked-list.  Then the number of darts in the
quotient of $L$ by $N$ is less than the number of darts in the quotient of $L$ by $N'$.
\end{lemma}

\begin{proof}  The number of darts in a quotient hypermap is equal
to the number of quotient darts (atoms) in the normal family $N$.
The normal family $N'$ replaces a loop $r$ with $r_1$ and $r_2$
and keeps the other loops the same.  The atoms of $r$ at
$y$ and $z$ are split into two atoms to create $r_1$ and $r_2$.
Hence the number increases.
\end{proof}


\begin{lemma}\guid{ZBHENEI}  The number of final faces in a quotient is no more than
the number of faces in the original $L$.
\end{lemma}

\begin{proof} 
Each final face in the quotient is a face in the original $L$, up to rotation.
\end{proof}

The following upper bound on the number of elements that are final in 
an lgraph, combined with a monotonicity result for $g$, will give termination
of the the procedure.

\begin{lemma}\guid{XZAJELF} If we have a match-quotient-list $(g,L,N)$,
then the number of final faces  in $g$ is at most the number of faces in $L$.
\end{lemma}

\begin{proof}  The set of elements that are final in $g$ injects into the
set of darts in the quotient.  Then apply ZBHENEI.
\end{proof}

We can continue to subdivFace until every face is final.  Note that match-quotient-list
allows for there to be loops that are final in N, whose corresponding face in $g$ is not final.
In this case, the iteration of subdivFace has no effect other than to make the face final.
In this case,  no transform of $N$ occurs.

The following lemma shows that when we terminate, the fgraph of $g$
will equal $L$.

\begin{lemma}\guid{XWCNBMA} If $(L,N,r,x)$ is a marked-list, 
and if every path of $N$
is a final-list, then $L$ is equal to its quotient-list by $N$.
\end{lemma}

This has been proved at the level of hypermaps in STKBEPH.  We can deduce
the lemma from that.

\begin{proof} This is Example 4.51 (maximal normal family), translated into lists.
If $f$ is a loop in the family, it is canonically true.  Hence its
darts are singletons and the darts in the loop form a face.  Thus,
the set of darts visited by the family $N$ is a union of faces.

By the third, property of normal family, the set of darts visited by
the family $N$ is a union of nodes.  Hence the set of such darts is
a connected component. 

$L$ is restricted, hence connected.  
Our conditions force $N$ to be a connected set of faces, hence all of $L$
\end{proof}

%(Alternatively, we get the same conclusion if we assume that $N$ is obtained as a transform.  In this case,
%the flag-list condition, forces $N$ to be a connected set of faces, hence all of $L$.)

Thus, a series of transforms leads in a finite number of steps to a quotient-list that equals $L$.

\asubsection{finals and nonFinals}

This section shows that showDups and hideDups are inverse operations, enum-of-VertexList and indexToVertexList are inverse
operations, mk-triple and dest-triple are inverse operations.  It analyzes containsDuplicateEdge and gives
an alternative to subdivFace.

We also analyze enumerator and simplify formulas for splitFace.

\asubsection{enumeration lists}



\asubsection{index calculus of higher transforms}

We establish a common notation for the following lemmas.

Let $\iota(y,z,s)$ be the index of the dart $z$ in the list $s$ relative
to the base point $y$ (where both $y$ and $z$ are members of $s$).

\begin{lemma}\guid{QYHXIVZ}
\formalauthor{tchales, 2/2014} 
\formalnote{indexf-refl, indexf1, next-el-indexf, indexf-rotn, indexf-rotn, indexf-add-right, indexf-add-left, indexf-add-sum,
indexf-antisym, size-betwn, indexl-betwn, indexf-betwn-eq, next-eln-indexf, indexf-n, betwn-cases, indexf-add-betwn}
The index satisfies the following properties.  Assume
that $a,b,c$ are members of $s$, and that $s$ is uniq.  Let $n$ be the length of $s$.
\begin{itemize}
\item $\iota(a,a,s)=0$.
\item $\iota$ is invariant under rotations of $s$.
\item $\iota(a,b,s) + \iota(b,c,s) = \iota(a,c,s)$ if $\iota(b,c,s)\le \iota(a,c,s)$.
\item $\iota(a,b,s) + \iota(b,c,s) = \iota(a,c,s)$ if $\iota(a,b,s)\le \iota(a,c,s)$.
\item $\iota(a,b,s) + \iota(b,c,s) = \iota(a,c,s)$ if $\iota(a,b,s)+\iota(b,c,s) < n$.
\item $\iota(a,b,s) + \iota(b,a,s) = n$, if $~(a=b)$.
\item $n^i a = b$, where $i=\iota(a,b,s)$ and $n$ is the next element map.
\item $\iota(a,b,s)=1$ iff $b$ is the next element after $a$ in $s$.
\item $b$ is between $a$ and $c$ on $s$ iff $a\ne c$ and $0 < \iota(a,b,s) < \iota(a,c,s)$.
\item We have one of the following: $b=a$, $b=c$, $b$ is between $a$ and $c$, or $b$ is between $c$ and $a$.
\item if $b$ is not between $c$ and $a$ on $s$, then $\iota(a,b,s) + \iota(b,c,s) = \iota(a,c,s)$.
\end{itemize}
\end{lemma}

\begin{lemma}\guid{FWDDPHY}
\formalauthor{tchales, 2/2014}
\formalnote{indexf-prev}   Assume that $b$ follows $a$ in $s$ and that $a\ne b$. Then
$\iota(b,a,s) + 1$ is the length of $s$.
\end{lemma}

\begin{proof} This is a special case of the preceeding lemma, since $\iota(a,b,s)=1$.
\end{proof}


\begin{remark}[F,~T,~w,~$N_i$,~$r_i$,~$y_i$,~$z_i$]
Let $(L,N,r,x)$ be a marked list.  Let $F$ be the face of $x$ in $L$.
 We let $w$ = $f x$ be the dart following $x$ in $F$.  It will serve
as a base point.

Let $T = T_{L,x}$ be the transform.  Set $(N_i,r_i) = T^i (N,r)$.  In particular,
$(N_0,r_0) = (N,r)$.   Let $y_i = l'y(L,r_i,x)$ and $z_i = l'z (L,N_i,r_i,x)$.
By construction $y_i$ and $z_i$ are members of $r_i$.
Also, let $z_i^-$ be the predecessor of $z_i$ in $r_i$ and $y_i^+$ the successor of
$y_i$ in $r_i$.
\end{remark}

\begin{lemma}\guid{DANGEYJ}
\formalauthor{HLT 3/2/2014}
  Let $(L,N,r,x)$ be a marked list with notation as established above.
Then every dart of $N_i$ is a dart of $N_{i+1}$.
\end{lemma}

\begin{proof} The darts of $N_i$ in each face are the same in each face, except the face which is split.
Its darts are divided between the two new faces.
\end{proof}

\begin{lemma}\guid{PWSSRAT} 
\formalauthor{HLT 3/5/2014}
Let $(L,N,r,x)$ be a marked list with notation as established above.
Then a dart $d$ of $r_i$ appears in $r_{i+1}$ if and only if $\iota(z_i,d,r_i)\le \iota(z_i,y_i,r_i)$.
\end{lemma}

(In particular, $z_i$ appears in $r_{i+1}$.)

\begin{proof} This follows from the explicit ordering of darts on faces and the transform.
\end{proof}


\begin{lemma}\guid{OHCGKFU}
\formalauthor{HLT 3/2014}
Let $(L,N,r,x)$ be a marked list with notation as established above.
Assume that $d$ is a dart in both $r_i$ and $r_{i+1}$.  Then 
\[
\iota(z_i,d,r_i) = \iota(z_i,d,r_{i+1}).
\]
\end{lemma}

\begin{proof} This follows from the explicit ordering of darts on faces and the transform.
\end{proof}

\begin{lemma}\guid{PPLHULJ}
\formalauthor{HLT 3/2014}
$(L,N,r,x)$ be a marked list with notation as established above.
Then $y_i$ and $z_i$ belong to $F$, and
 $\iota(w,y_i,F) < \iota(w,z_i,F)$.
\end{lemma}

\begin{proof} This follows from the face map power applied to $y_i$ to obtain $z_i$,
an the result that $z_i$ is not in the s-list (HQYMRTX1).
\end{proof}

\begin{lemma}\guid{NCVIBWU}
\formalauthor{HLT 3/2014}
$(L,N,r,x)$ be a marked list with notation as established above.
If $j\le \iota(w,z_i,F)$, then $f^jw$ belongs to $r_{i+1}$ and
$j = \iota(w,f^j w,r_{i+1})$.
\end{lemma}

\begin{proof} This follows from the explicit ordering of darts on faces and the transform.
\end{proof}

\begin{lemma}\guid{QCDVKEA}
$(L,N,r,x)$ be a marked list with notation as established above.
Then $\iota(w,z_i,F) \le \iota(w,y_{i+1},F)$.
\end{lemma}

\begin{proof} By construction,  $\iota(w,y_{i+1},F)$ is the 
largest index $j$ such that the indexing on $f$ and $r_{i+1}$ agree up to $j$.
By the previous lemma, the indexing agrees at least up to $\iota(w,z_i,F)$.
\end{proof}

By PWSSRAT, $y_0$ belongs to $r_i$ for all $i$.

\begin{lemma}\guid{PBFLHET}
$(L,N,r,x)$ be a marked list with notation as established above.
Let $d$ be a dart in $r_i$ such that
$\iota(z_i,d,r_i) \le \iota(z_i,y_0,r_i)$. Then $d$ belongs to $r$.
Moreover, $\iota(z_i,d,r_i) = \iota(z_i,d,r)$.
\end{lemma}

\begin{proof}
By assumption $d$ belongs to $r_i$.
Fix such a $d$.  By PWSSRAT, and the monotonicity of the parameters PPLHULJ and QCDVKEA,
by induction, $d$ belongs to $r_{i-j}$, for $j=0,\ldots,i$.  In particular, $d$ belongs to $r_0 = r$.

Taking $j=0$, we get $d=z_i$ belongs to $r$.

The index calculation is also backwards induction on indices $r_{i-j}$, using 
OHCGKFU.
\end{proof}

\begin{lemma}\guid{PNXVWFS}
$(L,N,r,x)$ be a marked list with notation as established above.
For all $i$, we have $y_i$ and $z_i$ belong to $r$.
\end{lemma}

\begin{proof}
Apply the previous lemma to $d = y_{i+1}$ and $d=z_i$, which belong to $r_i$ by construction.
\end{proof}

\begin{lemma}\guid{DIOWAAS}
$(L,N,r,x)$ be a marked list with notation as established above.
Then $\iota(w,y_i,r) < \iota(w,z_i,r)$.
\end{lemma}

\begin{proof}
This is equivalent to $\iota(z_i,w,r) \le \iota(z_i,y_i,r)$,  or to $\iota(z_i,w,r_i)\le \iota(z_i,y_i,r_i)$. 
For this it is enough to show that $w$ is not between $y_i$ and $z_i$ on $r_i$.  This is clear,
because the segment between $y_i$ and $z_i$ does not meet $r_{i+1}$, but $w$ belongs to $r_{i+1}$.
\end{proof}

\begin{lemma}\guid{RYIUUVK}
$(L,N,r,x)$ be a marked list with notation as established above.
Then $\iota(w,z_i,r) \le \iota(w,y_{i+1},r)$.
\end{lemma}

\begin{proof}  
The dart $y_{i+1}$ belongs to $r_{i+1}$.  We claim that it lies on the segment of $r_{i+1}$ between $z_i$ and $w$.
Otherwise, it is between $w$ and $z_i$.  But from $w$ to $z_i$, the list $r_{i+1}$ coincides with $f$, by NCVIBWU.  This is impossible,
since $y_{i+1}$ comes after $z_i$ on $f$.

On the segment of $r_{i+1}$ between $z_i$ and $w$, the ordering is the same as on $r$.
So $\iota(z_i,y_{i+1},r_{i+1}) < \iota(z_i,w,r_{i+1})$ gives $\iota(z_i,y_{i+1},r) < \iota(z_i,w,r)$ and hence
$\iota(w,z_i,r) \le \iota(w,y_{i+1},r)$.
\end{proof}

\begin{lemma}\guid{KBWPBHQ} 
$(L,N,r,x)$ be a marked list with notation as established above.
The set $\{y_i\}$ consists of darts $y$ such that
$y$ on $F$ and $N$ such that $y$ is followed by an 
$n^{-1}$-step on $r$.
The set $\{z_i\}$ consists of darts $z$ on $F$ and $N$ such that
$z$ is preceded by an $n^{-1}$-step on $r$.
\end{lemma} 

\begin{proof}  We have seen that the darts $y_i$ and $z_i$ lie on $F$ and $r$.
By construction the darts $y_i$ are followed by an $n^{-1}$-step, and the darts
$z_i$ are preceded by an $n^{-1}$ step.

Conversely, let $y$ be any dart on $F$ and $N$ that is followed by an $n^{-1}$ step on $N$.
If $y$ is between $y_i$ and $z_i$, then these darts are not in $N_i$, hence not in $N$.  
This is contrary
to the assumption.

If $y$ is between $z_i$ and $y_{i+1}$, then $y$ is reached from $z_i$ by $f$-steps on $r_i$.
So they are also $f$-steps on $r$.
\end{proof}

%\begin{lemma}\guid{XBXFJPH}
%$(L,N,r,x)$ be a marked list with notation as established above.
%The darts on both  $F$ and $N$ all lie on the loop $L$.
%\end{lemma}


\asubsection{match $g$ $L$ and $N$}

\asubsection{tame hypermaps}

The formalization of this section is complete.

\begin{lemma}\guid{OXAXUCS} 
\formalauthor{tch}
If a hypermap $H$ is isomorphic to one that has property
tame $k$, for $k\in \{9a, 10, 11a, 11b, 12o, 13a\}$, then
$H$ has that property as well.
\end{lemma} 

Each of the tame properties $I=\{9a,10,11a,11b,12o,13a\}$ has
a corresponding definition for lgraphs, say $I'$.  This correspondence is
defined in such a way that the following holds.

\begin{lemma}\guid{WMLNYMD}\formalauthor{tch}  Let $g$ be a good graph.  
Let $H$ be a tame hypermap that is isomorphic to the hypermap of $g$.
Then $g$ is tame.
\end{lemma} 

Let $H$ be any tame hypermap.  It is restricted, so there exists
a final planegraph $g$ and an isomorphism between $H$ and the
hypermap of $g$.  It follows that the hypermap of $g$ has all of
the tameness properties $I$.  Hence $g$ itself has all of the tameness
properties $I'$.  

By the Bauer-Nipkow formalization on tame graphs, $g$ is 
fgraph congruent
to an fgraph $y$ in the archive.

\begin{lemma}\guid{XRFJNDO}
\formalauthor{Solovyev}
\formalnote{See tame/good\_list\_archive.hl}
 Every member of the archive 
is a good-list.
\end{lemma} 

\begin{proof} This is by direct enumeration of the archive.
\end{proof}

\begin{lemma}\guid{ELLLNYZ}
\formalauthor{tch} 
Let $x$ and $y$ be two good-lists that are fgraph congruent.
Then their hypermaps are isomorphic, or the opposite hypermap of $x$
is isomorphic to the hypermap of $y$.
\end{lemma} 

Putting these results together we have that $H$ isomorphic to the
hypermap of $y$ or the opposite of $H$ is isomorphic to the hypermap
of $y$.

The main linear programming result, formalized by Solovyev, shows
that if $H$ is isomorphic to the hypermap of $y$, then it is not
contravening.  We need the opposite as well.

\begin{lemma}\guid{ASFUTBF}\formalauthor{tch}
 If the opposite of $H$ is contravening, then  $H$
is also contravening.  
\end{lemma} 

\begin{proof}
  If $H$ is contravening, this means there is a finite packing $V$
  which is contravening and whose associated hypermap
  $\op{hyp}(V,E_{std}(V))$ is $H$.  The finite packing $-V$ obtained
  by negating all the coordinates is also contravening.  Its hypermap
  is isomorphic to the opposite of $H$.
\end{proof}

\newpage
\asubsection{notes on AQ}

Here we discuss the proof of AQ12 and AQ13.
Specifically, we discuss the proof of the second part of those statements (not
involving the contour list).

Let $N' = \op{ntrans}~ L~N~r~x~1$, and $r' = \op{rtrans}~ L ~N~r~x~1$.
Write AQ12' for the goal $AQ(L,N',r',x)$ and AQ12 for the assumption $AQ(L,N,r,x)$.
Similarly, for AQ13' and AQ13.
We write $S = $s-list L r x, and $S' = $s-list L r' x. Note that every member of $S$
is a member of $S'$.

\begin{proof}
We start with the proof of AQ12'.  We may assume (final-list L r') and not(final-list L r).
Otherwise, there is nothing to prove.

We let u = LAST p, where p is a part of s in N'.  We assume that not(final-list L s),
for otherwise there is nothing to prove.  Our goal is to prove that e-list u is a member
of final-dart-list L N'.

(A) We may assume that e-list u is not a member of r', because otherwise e-list u is in the
a final face r', and we are done.  Note that every member of $S$ is in r'.  Thus,
we may also assume that e-list u is not a member of $S'$.  Since not(final-list L s)
and final-list L r', we have $s\ne r'$; so that u is not a member of r' and that u is not a member of 
$S'$.

We have a slightly weaker statement:
(A') e-list u is not a member $S'$, and u is not a member of $S'$.

We will see that if we give a proof of AQ12' based on (A'), then exactly the same proof
can be used for AQ13'.  The following condition also holds:

(B)  Every member of final-dart-list L N is a member of final-dart-list L N'.

\claim{Case 1.  u is a member of  N' but not N.}  Comparing N' with N, we see
that u is a new dart that is added at new nodes of degree 2 on the quotient
 hypermap.  (These are the darts at the nodes between l'y and l'z along the face of $x$.)
By (A'), u is not a member of $S'$.  This implies that u is not in r'.
It follows that u lies at  a node of degree 2 on the
 quotient hypermap.
 Hence e-list u is in $S'$, which is contrary to (A').

\claim{Case 2. u is a member of r.}
Recall that u has the form LAST p, for p a part of s.
If p is also a part of
 r, then using that
  r is not final, and that e-list u does not depend on N, N', we see
that by AQ13, e-list u is a member of $S$ or a final-dart-list L N.
By (A'), we know that e-list u is not a member of $S'$.  So it is a member of final-dart-list N.
The result follows by (B).

If  u has the form LAST p, where p is a part of s, but not a part of r, then there are only
two possbilities for u.  Either it is the dart l'y, or the dart node-list (l'z).  The first case u = l'y is
incompatible with (A'), since l'y is a member of $S'$.  The second case u =node-list (l'z), gives
e-list u = $f^{-1} l'z$, which is also a member of $S'$.  This is also incompatible with (A').

\claim{Case 3. u is a member of N but not r.}  For such u, we have that u is
 final-dart-list for N iff it is
 in the final-dart-list for N'.  
Recall that s  is not-final. 

In view of (A'), the assumption AQ13 tells us that since s is a face of both N and N',
and since not(final-list s), we have that e-list u is a member of the final-dart-list L N.
By (B), e-list u is a member of final-dart-list L N' as well. This completes the proof.
\end{proof}

\bigskip

\begin{proof} Now we turn to the proof of AQ13'.
We may assume that not(final-list L r') and not(final-list L r).  Otherwise,
there is nothing to prove.

We let u = LAST p, where p is a part of s in N'.  We assume that not(final-list L s),
for otherwise there is nothing to prove.  To avoid a trivial case,
we may also assume that u is not a member of $S'$.

Our goal is to prove that e-list u is a member
of $S'$  or final-dart-list L N'.  In other words, assuming that e-list u is not a member of $S'$,
we wish to prove it is a member of final-dart-list L N'.  In this form, we have assumption (A'), as
stated in the previous proof.  Assumption (B) also holds.

We have written the proof of AQ12' in such a way that the three claims from the previous proof
work word-for-word to give the proof of AQ13' as well.
\end{proof}
