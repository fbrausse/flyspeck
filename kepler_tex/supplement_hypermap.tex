% Created Dec 13, 2013.
% Supplementary notes on the hypermap chapter.

\section{Connecting with Bauer-Nipkow 
work in Isabelle}

This section builds on the HOL-Light files hypermap.hl
and import-tame-classificaton.hl.

For simplicity of typesetting, we use a dash in this text to
represent an undescore in HOL Light.

In the Bauer-Nipkow work, there is a type for Isabelle graphs,
which we abbreviate to lgraph.   There is a predicate
planegraph that expresses lgraph planarity.  There is also a slightly
broader class planegraph-relaxed.

\begin{lemma}\guid{DPZGBYF} 
If $g$ is planegraph, then it is also planegraph-relaxed.
\end{lemma}

There is a function, fgraph, that maps an lgraph to
the list of list representation of the lgraph.  For example,
the list of list representation of a planar graph consisting
of a square triangulated into four squares with common
vertex $0$ is
\[
[[0;1;2];[0;2;3];[0;3;4];[0;4;1];[4;3;2;1]].
\]
We refer to these entries $[0;1;2]$, $[0;2;3]$, etc. as the
faces of the lgraph.

There is a function, hypermap-of-list, that maps the
list of list representation of the planar graph into the
corresponding hypermap.  The darts of the hypermap
are consecutive pairs of elements in the list of list
representation.  For example, the example above gives
the dart set
\[
\{(0,1),(1,2),(2,0),~(0,2),(2,3),(3,0),~(0,3),(3,4),(4,0),~
(0,4),(4,1),(1,0),~(4,3),(3,2),(2,1),(1,4)\}.
\]
The face map follows each pair around the face in which it occurs
$f(0,1) = (1,2)$, etc.  The edge map reverses an ordered pair $e(0,1)
= (1,0)$, etc.  The set of integers that appear in this list of list
is in bijection with the set of vertices of the lgraph.  In fact, we
can take this is as the definition of the set of vertices of the
lgraph.  When we refer to the hypermap of $g$, we refer to this
construction.

There is a function, finals, that returns a sublist of the
list of list representaton of the lgraph.

In the Isabelle development, there is a function facesAt.

\subsection{basic definitions}

Since our aim is relate hypermaps to the constructions in Isabelle, which
are based on lists, we give a series of list-based definitions that run
in parallel with hypermap-based definitions.

We make a table of notions about hypermaps and their corresponding list-based notion.
This table serves as a dictionary translating between the different data structures.

\bigskip
\begin{tabular}{l l}
\hline
{\bf hypermap} & {\bf lists} \\ \hline
loop & loop-list \\
is-contour & contour-list \\
is-normal & normal-list \\
hyp'm, hyp'p, hyp'q & l'm, l'p, l'q \\
hyp'y, hyp'z & l'y, l'z \\
final-quotient-face   & final-list \\
quotient & quotient-list \\
- & final-dart-list \\
- & split-normal-list \\
transform & transform-list \\
hyp'S & s-list \\
s-flagged & s-flag-list \\
empty-flagged & flag-list \\
is-marked & marked-list \\
iso & iso-list \\
hyp-iso & isop-list \\
f-map, n-map, e-map & f-list, n-list, e-list\\
dih2k & dih2k-list\\
\hline
\end{tabular}

\bigskip
We also describe the general correspondence of variable names.

\begin{tabular}{l l}
\hline
{\bf hypermap} & {\bf lists} \\ \hline
H (hypermap) & L (list of lists)\\
NF (normal family) & N \\
L (loop) & r \\
x (dart) & x \\
\hline
\end{tabular}

\bigskip
We also have conversions from lists to hypermaps.  The constant loop-of-list converts a list to a loop,
and loop-family-of-list converts a list of lists to a family of loops.

There is a function match-core-list that is true when there is a correspondence between
faces of an lgraph g and those of the quotient of L by N, such that the final faces of g are included in those of $N$.
A function match-list also measures the corespondence.  The function loop-choice generates a pair (r,x)
to be used as part of a marked-list, from a lgraph g.





\subsection{properties of planegraph}

\begin{definition}[good-list,~good-list-nodes,~good-graph]
Recall that a list of lists $L$ is a good-list if it has three properties:
\begin{enumerate}
\item The list of darts of $L$ has no duplicates.
\item Every member of $L$ is non-nil.
\item If $(x,y)$ is a dart of $L$, then so is $(y,x)$.
\end{enumerate}
Recall that a good list $L$ is a good-list-nodes if 
the number of nodes in its hypermap is equal to the number
of vertices of $L$.
We say that an lgraph is is good-graph if it has the properties:
\begin{enumerate}
\item Its fgraph $L$ is a good-list and a good-list-nodes.
\item all faces of the fgraph are final.
\item In each face, each vertex occur at most once.
\item The vertex set coincides with the set of elements of the fgraph.
\item The facesAt a vertex $v$ is the same as the list of faces that contain $v$.
\end{enumerate}
\end{definition}

The following properties can be proved by structural induction
for planegraph-relaxed.

\begin{lemma}\guid{JUJUWAT} 
If $g$ is planegraph-relaxed, then its fgraph is a good-list.
\end{lemma}  
 


\begin{lemma}\guid{ETDLJXT}
If $g$ is planegraph-relaxed, then  the vertices of $g$ are $0,\ldots,n-1$, where
$n$ is the number of nodes of $g$. 
\end{lemma} 

\begin{lemma}\guid{CGGZYRC} 
Let $g$ be a planegraph.  Let $F$ be a face of $g$
that is not final.  Let $(x,y)$ be a dart of (the hypermap of) $g$ on face
$F$.
Then $(y,x)$ is a dart on a face that is final.
\end{lemma} 

(We don't need to formalize CGGZYRC, because we will get it as a corollary
of the correspondence between restricted hypermaps and plane-graphs.)

\begin{lemma}\guid{HWDMZDM}
Let $g$ be planegraph-relaxed, and $v$ in the vertex setof $g$.
Then $facesAt~g~v$ is the set of faces of $g$ that $v$ is a member of.
\end{lemma}

\begin{lemma}\guid{EAHHATZ}
If $g$ is a Plane-graph, then it is a good-graph.
\end{lemma}

(We don't need to formalize EAHHATZ, because we will get it as a corollary
of the correspondence between restricted hypermaps and plane-graphs.)

In general we want to have the properties of good-list, good-list-nodes, and all-uniq.
The following lemmas show how these properties progagate.

\begin{lemma}\guid{PMBRINH}  If $L$ is good-list-nodes and normal-list-L-N, then good-list-nodes
all holds of quotient-list-L-N.
\end{lemma}

\begin{proof}  It is enough to show that all of the darts with a given first coordinate form a single node of the quotient.
This is clear, because by assumption all the darts of $L$ with a given first coordinate for a single node for $L$.
\end{proof}

\begin{lemma}\guid{SNVACWG}  If $L$ is a good-list and good-list-nodes, its hypermap is restricted,
and normal-list-L-N, then good-list holds of quotient-list-L-N.
\end{lemma}

\begin{proof} We examine each of the defining properties of a good-list.  

\case{uniq}  
If a dart $(a,b)$ occurs more than once  then there are two darts $(x_1,y_1)$ and $(x_2,y_2)$ in $N$, that run from
nodes $a$ to $b$.  This is impossible by the no-double-joins property of restricted hypermaps.

\case{all-non-nil}  It is a property of normal that its members are not nil.  The same is then true of the quotient.

\case{sym}.  Let $(a,b)$ be a dart in the quotient.  Then there is a dart $(a,b)$ in $N$.  In fact, we see that $(a,b)$ is the last member of
its atom in $N$.  The family $N$ meets the node $b$,
and $(b,a)$ is a dart at the node $b$.  By properties of normal, all darts at the node, including $(b,a)$, are on $N$.
We have $n^{-1}(b,a) = f (a,b)$.  By unicity, $(b,a)$ is followed by an $f$-step in $N$.  This implies that $(b,a)$ is a dart in the
quotient. 
\end{proof}

\begin{lemma}\guid{LYNVPSU}  $(L,N,r,s)$ be a marked hypermap.  Then the quotient of $L$ by $N$ is all-uniq.
\end{lemma}

\begin{proof}  This is a defining property of a marked hypermap.
\end{proof}

\subsection{main result}

This result is imported from Isabelle.

\begin{theorem}[Import~Tame~Classification]  Let $g$ be a final
planegraph that is tame.  Then there exists a $y$ in the archive such
that the fgraph of g is fgraph congruent to $y$.
\end{theorem}

The main result to be proved is the following lemma.

\begin{lemma}[LSKOKJE]  Let $H$ be a restricted hypermap.  Then there exists a good-graph g that is a PlaneGraphs,
such that $H$ is isomorphic to the hypermap of $g$.
\end{lemma}

\subsection{isomorphism}


We develop the properties of iso-list, which is the list-based version of isomorphism.
It is a particularly rigid notion, because it does not allow the rearrangement of the elements of a list.
However, it is sufficient for our purposes.

We often need to consider $L$ together with a normal-list $N$.  The definition of iso-list carries with it
a second argument that checks that $N$ is sent to $N'$ under the isomorphism.

\begin{lemma}\guid{GNBEVVU} If $L$ is a good-list and $L'$ is an iso-list, then the hypermaps of $L$ and $L'$ are isomorphic.
\end{lemma}

\begin{proof} This follows directly from the theorem hypermap-of-list-map.
\end{proof}

\begin{lemma} [iso-list-refl,~iso-list-sym,~iso-list-trans] The relation iso-list is reflexive, symmetric, and transitive.
\end{lemma}

The properties of a list $L$ that are invariant under an injective map remain true when $L$ is replaced
with an iso-list.  This gives the following results.

\begin{lemma}\guid{PEUTLZH} Let $L$ be a good-list, and $L'$ an iso-list.  Then $L'$ is a good-list.
\end{lemma}

\begin{lemma}\guid{OISRWOF} Let $L$ be a good-list-nodes, and $L'$ an iso-list.  Then $L'$ is a good-list-nodes.
\end{lemma}

\begin{lemma}\guid{UEYETNI} Let $L$ be all-uniq, and $L'$ an iso-list.  Then $L'$ is a all-uniq.
\end{lemma}

\begin{lemma}\guid{XKDZKWV}
 Let $L$ be a good-list with normal-list $N$.  If $(L',N')$ is iso-list, then $N'$ is a normal-list of $L'$.
\end{lemma}

\begin{lemma}\guid{MEEIXJO}
Let $L$ be a good-list with marked-list $(L,N,r,x)$.  If $(L',N',r',x')$ is iso-list, then $(L',N',r',x')$ is a marked-list.
\end{lemma}



\subsection{representing hypermap as lists}

It is convenient to work with lists of list $L$ rather than hypermaps $H$.  This means
we should give a list version of notions such as loop and normal family.
We have loop-list as the list version of a loop and norm-list as the list
version of  a normal family of loops.  

We often need to pass to isomorphic data.  The following lemma is the starting point.

\begin{lemma}\guid{DAKEFCC}
If $H$ and $K$ are isomorphic hypermaps, and if $H$ is restricted, then $K$ is restricted.
\end{lemma}

\begin{lemma}\guid{RXOKSKC}
Let $H$ be a restricted hypermap.  Then there exists a list $L$ such that $H$ is its hypermap.
In fact, there exists such an $L$ that is good-list, good-list-nodes, all-uniq.
\end{lemma}

\begin{proof} We may convert a hypermap to list $L$ as follows.  From each face $F$ of $H$
pick a dart $x$ and form the list $[\op{node}(x),\op{node}(f x),\op{node}(f^2 x),\ldots]$.  
The list of all these lists for each face is defined to be $L$.

A restricted hypermap is simple, hence the elements of the list are uniq.  This gives the all-uniq property.

By construction the elements of these lists are the nodes.  This gives the good-list-nodes property.

We check that it is a good list.  
For the uniqueness property, note that we show below that the map between darts sets is one-to-one.
This means that each dart appears in a face once.

For the non-nil property of good-lists: each list contains $\op{node}(x)$ as the first element. Hence it is
not the nil-list.

For the symmetry property of good-lists: if $(\op{node}(x),\op{node}(f x))$ is a dart, then
\[
(\op{node}(n f x),\op{node}(f n f x)) = (\op{node}(f x),\op{node}(x))
\]
is also a dart.

Finally, we check that $H$ is isomorphic to the hypermap of $L$.  The darts of the hypermap of $L$ are formed by
consecutive pairs $(\op{node}(f^i x),\op{node}(f^{i+1} x))$.  

We claim the map between dart sets, $y\mapsto (\op{node}(y),\op{node}(f y))$, is one-to-one.
This follows from the no-double-joins property of restricted hypermaps.  Specifically,
if $(\op{node}(y),\op{node}(f y)) = (\op{node}(z),\op{node}(f z))$, then the edges $\{y,e y\}$ and $\{z,ez\}$ run
between the same nodes, forcing $y=z$.

The face maps are compatible on the two hypermaps by construction. Finally, the edge map is compatible because
\[
(\op{node}(e y),\op{node}(f (e y))) = (\op{node}(f y),\op{node}(y)) = e-list (\op{node}(y),\op{node}(f y)).
\]
\end{proof}

\begin{lemma}\guid{JXBJOAB}  If $L$ is any list of lists, then there is a iso-list $L'$ whose elements are natural numbers.
\end{lemma}

\begin{proof} Pick any injective map from the elements of $L$ to $\ring{N}$ and take the resulting iso-list.
\end{proof}

For the correspondence with Isabelle, we will need to use lists $L$ whose elements are natural numbers and
whose elements are ordered in a particular way.  By taking a permutation on the natural numbers we can obtain
any convenient ordering.  A particular ordering is given by lemma SHXWKXQ.  We refer to the formal statement
in the HOL Light files.  What follows is an informal summary.

\begin{lemma}\guid{SHXWKXQ}.  For any $L$ (good-list, good-list-nodes, all-uniq)
and normal-list $N$.  By passing to an iso-list, we can arrange that the elements appearing
in $N$ are the natural numbers $\{0,1,\ldots,n-1\}$, where $n$ is the number of vertices visited by $N$.
Furthermore, if $x$ is any dart, and $F$ is the face in $L$ containing $x$ listed so that $x$ appears at the head of the
list.  Let $F'$ be the filtered list of $F$ containing all the darts of $F$ that are not in $N$.  Then we may also assume that
$F' = [n+1;n+2;n+k]$, where $k$ is the length of $F'$.
\end{lemma}

The following lemma gives a reducion of the main result LSKOKJE.

\begin{lemma}[JCAJYDU]  It is enough to prove LSKOKJE in the case when $H$ is the hypermap of $L$,
where $L$ is a good-list, good-list-nodes, all-uniq, and having elements in the natural numbers.
\end{lemma}


\subsection{translating notions between hypermaps and lists}


We skip the following proofs.  In each case the definitions involved in the correspondence
are direct translations of one another.  Thus, the proofs are a matter of comparing defintions.

\begin{lemma}\guid{GZLJIGN} Under the correspondence between $L$ and its hypermap, the constants
$l'm$, $l'p$, $l'q$, $l'y$, $l'z$ correspond respectively to hyp'm, hyp'p, hyp'q, hyp'y, hyp'z.
\end{lemma}

\begin{lemma}\guid{EVNAPDQ} Let $L$ be a good-list.  Let $N$ be a normal list of $L$.
Then the loop family is a normal family of the hypermap of $L$.  
\end{lemma}

\begin{lemma}\guid{ABKCJWD} Let $L$ be a good-list.  Let $N$ be a normal list of $L$.
Then the hypermap of the quotient-list is isomorphic to the quotient of the data
on the hypermap side.
\end{lemma}

\begin{proof} Their respective definitions are direct translations of one another.
\end{proof}

\begin{lemma}\guid{ODWAFRG}  Let $L$ be a good-list and $(L,N,r,x)$ a marked-list.
Then the corresponding data in the hypermap is is-marked.
\end{lemma}

There are also specific lemmas about hypermaps that we wish to translate back into theorems
about lists.  Specifically, we have lemmas HQYMRTX1-list, HQYMRTX2-list, HQYMRTX3-list,
QRDYXYJ-list, AQIUNPP1-list, and AQIUNPP2-list that are list versions of lemmas given in {\it Dense Sphere Packings}.
We omit the proofs here, because the proofs are direct translations from hypermap language to list language.
In fact, it should be possible to deduce them directly from the hypermap version of the lemma.


\subsubsection{dihedral  initialization}

The fgraph of Seed~p is the list of lists $[[0;1;...;n+2];[n+2;n+1;...;0]]$.

\begin{lemma}[good-list-seed,~all-uniq-seed] \guid{FOEGZEQ}
\formalauthor{tch}
 Seed~p is a good-list and all-uniq.
\end{lemma}

\begin{lemma}\guid{TAGYMWJ} Seed~p is a good-list-nodes.
\end{lemma}

\begin{lemma}\guid{ENWCUED}  Seed~p is dih2k-list.
\end{lemma}

\begin{lemma}\guid{DLWOJBB} If $L$ is dih2k-list, then its hypermap is dih2k.
\end{lemma}

\begin{lemma}\guid{KUKASGD} Any two dih2k-list's $L$ and $L'$ of the same face size are iso-lists.
\end{lemma}

\begin{lemma}\guid{UNHQYQM} If $L$ and $L'$ are iso-lists and $L$ is dih2k-list, then so is $L'$.
\end{lemma}

This is a version of AUQTZYZ for lists.

\begin{lemma}[AUQTZYZ-list]  Let $L$ be good-list with restricted hypermap.
Let $f$ be a face of $L$.  Then there exists a loop-list $f'$ such that $[f;f']$ is normal and the quotient
is $[f,rev~f]$ (which is dih2k-list).
\end{lemma}

\begin{proof} Pick $f'$ to be the complementary path (a concatenation of complementary nodes) of $f$.
\end{proof}

The following is a more precise version of the correspondence of a restricted hypermap with
the seed.  We will apply this lemma to a face with the largest size of $L$.

\begin{lemma}\guid{UYOUIXG}  Let $L$ be a good-list with restricted hypermap.  Assume
that $L$ itself it not dih2k-list.  Let $f$ be any face of $L$.  Then there exists data $(L',N')$ and $f'$
such that $(L,[f;f2])$ is an iso-list of $(L',N')$ and such that $N'$ is a normal-list of $L'$ and
match-core-list between the seed and $(L',N')$.
\end{lemma}

\begin{proof} This is obtained by renaming the elements of $L$ to agree exactly with those of the seed.
Note that the heads of both lists $N'$ and the fgraph of the seed are the faces that are final.
\end{proof}





\subsection{termination}

We will obtain termination after a finite sequence of transforms.  Each transform
will increase the number of darts in the the quotient hypermap.  The number of
darts in the quotient is no more than the number of darts in the original.  Finally,
if every path in a normal family is final, then the quotient list of $L$ by $N$ is equal to $L$.

\begin{lemma}\guid{ADACDYF}  Let $(L,N,r,x)$ be a marked-list in which $r$ is not final.
Let $(N',r')$ be the transform of the marked-list.  Then the number of darts in the
quotient of $L$ by $N$ is less than the number of darts in the quotient of $L$ by $N'$.
\end{lemma}

\begin{proof}  The number of darts in a quotient hypermap is equal
to the number of quotient darts (atoms) in the normal family $N$.
The normal family $N'$ replaces a loop $r$ with $r_1$ and $r_2$
and keeps the other loops the same.  The atoms of $r$ at
$y$ and $z$ are split into two atoms to create $r_1$ and $r_2$.
Hence the number increases.
\end{proof}


\begin{lemma}\guid{ZBHENEI}  The number of darts in a quotient is no more than
the number of darts in the original $L$.
\end{lemma}

\begin{proof} The darts of the quotient  are obtained by taking a subset
of the darts of $L$ visitited by $N$ and then combining them into
atoms.  Taking subsets and combining both are non-increasing
in the number of darts.  
\end{proof}

The following upper bound on the number of elements that are final in 
an lgraph, combined with a monotonicity result for $g$, will give termination
of the the procedure.

\begin{lemma}\guid{XZAJELF} If we have a match-core-list $(g,L,N)$,
then the number of elements that are final in $g$ is at most the number
of darts in $L$.
\end{lemma}

\begin{proof}  The set of elements that are final in $g$ injects into the
set of darts in the quotient.  Then apply ZBHENEI.
\end{proof}

The following lemma shows that when we terminate, the fgraph of $g$
will equal $L$.

\begin{lemma}\guid{XWCNBMA} If $(L,N,r,x)$ is a marked-list, 
and if every path of $N$
is a final-list, then $L$ is equal to its quotient-list by $N$.
\end{lemma}

\begin{proof} This is Example 4.51 (maximal normal family), translated into lists.
If $f$ is a loop in the family, it is canonically true.  Hence its
darts are singletons and the darts in the loop form a face.  Thus,
the set of darts visited by the family $N$ is a union of faces.

By the third, property of normal family, the set of darts visited by
the family $N$ is a union of nodes.  Hence the set of such darts is
a connected component. 

Part of the definition of marked-list gives that $(L,N)$ is a flag-list.  This implies
that the set of darts is a connected set,
so that all darts are visited by $N$.  
\end{proof}

%(Alternatively, we get the same conclusion if we assume that $N$ is obtained as a transform.  In this case,
%the flag-list condition, forces $N$ to be a connected set of faces, hence all of $L$.)

Thus, a series of transforms leads in a finite number of steps to a quotient-list that equals $L$.

\subsection{finals and nonFinals}


\subsection{The index calculus of higher transforms}


Let $(H,\LL,L,x)$ be a marked hypermap.  
Let $T$ be the transform operator on marked hypermaps
(Definition 4.69 YQANQNF).  
Let $T^i(H,\LL,L,x) = (H,\MM_i,M_i,x)$ be the $i$th transform of
$(H,\LL,L,x)$.  We inductively describe the structure of $(H,\MM_i,M_i,x)$.

The natural number $i$ is bounded by the condition that $\MM_{i+1},M_{i+1}$ is defined only for those $i$ such that $M_i$ is canonically false.
We assume this condition without explicit mention.

Definition BVUFRRE associates constants $m_i,p_i,q_i$ and darts $y_i,z_i$
to the marked hypermap $(H,\MM_i,M_i,x)$, where we have added
subscripts to indicate dependence on $i$.


Recall from Section~4.7.3 the construction of $(\MM_{i+1},M_{i+1})$
from $(\MM_i,M_i)$. 
In general, we write modify-loop $(L,x,y,p)$ for the loop obtained
by replacing the segment of $L$ from $x$ to $y$ with the segment
of the path $p$ running from $x$ to $y$.
Let $F_x$ denote the face of $H$ containing the dart $x$. 

\begin{lemma}\guid{LPWFYMU} 
We have 
\[
\MM_{i+1} = (\MM_i \setminus \{M_i\}) \cup \{M_i^-,M_i^+\}
\]
where 
\[
M_i^- = \text{modify-loop} (M_i,x,z_i,F_x): 
  \text{follow } M_i \text{ from } z_i \text{ to } x
  \text{ and follow } F_x \text{ from } x \text{ to } z_i,
\]
and
\[
M_i^+ = \text{modify-loop} (M_i,nz_i,n^{-1}y_i,F^c_x): 
\text{follow } M_i \text{ from } n^{-1} y_i \text{ to } n z_i
  \text{ and follow } F^c_x \text{ from } nz_i \text{ to } n^{-1} y_i,
\]
and 
\[
M_{i+1} = M_i^-,
\]
where $F^c_x$ represents the complementary path of the path on $F_x$.
\end{lemma} 

We order darts on $F_x$ according to the $f$-ordering starting with $x$.
We order darts on $M_i^-$ 
according to the loop ordering starting with $x$.
Write $z <_{L,x} z'$ if a dart $z$ appears before dart $z'$ on the loop $L$
(starting from $x$).  We drop $x$ from notation when it is fixed.

\begin{lemma}\guid{RYIUUVK}  $z_i \le _F y_{i+1}$.
\end{lemma} 

\begin{proof} $y_{i+1}$ is defined as the first dart on $M_{i+1}$
that is followed by a $n^{-1}$-step on $M_{i+1} = M_i^-$.  By the
construction of $M_i^-$,
the steps from $x$ to $z_i$ are all $f$-steps.
\end{proof}

By construction $z_i$ appears at or after $y_i$ on $F_x$.  Thus,
$y_0 < z_0 \le y_1 < z_1\cdots$ is in order on $F_x$ (with gaps).

By loop confinement (Lemma~4.67), the dart $z_{i+1}$ appears on
$M_i$ in the segment strictly from $z_i$ back to $x$:
$z_i <_{M_i} z_{i+1}$.

\begin{lemma}\guid{CESHTIN} 
The darts $y_i$ and $z_i$ are visited by $L$.  Their order with
respect to $L$ is the same as their order with respect to $F_x$:
\[
y_0 <_L z_0 \le_L y_1 <_L z_1\cdots \le y_i <_L z_i.
\]
The segment of $L$ from $n^{-1} y_i$ to $n z_i$ is the same
as the segment of $M_i$ from $n^{-1}y_i$ to $n z_i$.
The segment of $L$ from $z_i$ to $x$ is the same as the
segment of $M_{i+1}$ from $z_i$ to $x$.
\end{lemma} 

\begin{proof} By complete induction on $i$, proving all statements
at once.  
Assume that all of these statements hold for $j<i$. We show they
hold at $i$.

%When $i=0$, the fact that $z_0$ is visited by $L$ and that $y_0 <_L z_0$
%in the ordering on $L$ follows from loop confinement. The statements
%about $M_0$ are trivial since $L=M_0$.

By the loop confinement lemma, $y_i$ and $z_i$ are visited by $M_i$.
$y_i$ is obtained from $f$-steps along $M_i$ from $x$, hence
$z_{i-1} <_{F_x} y_i$ implies $z_{i-1} <_{M_i} y_i$.  Also, $y_i <_{M_i} z_i$
by loop confinement.  By induction, from $z_{i-1}$ to $x$, we have that
$L$ and $M_i$ are the same, hence $y_i$ and $z_i$ are visited by $L$, and $z_{i-1} <_L < y_i <_L < z_i$.

The segment of $L$ from $z_i$ to $x$ is a subset of the segment
from $z_{i-1}$ to $x$, hence agrees with $M_i$ on that segment.
By construction, from $z_{i}$ to $x$, we have agreement of $M_i$ with
$M_{i+1}$.  Use transitivity.

Finally, from $y_i$ to $z_i$, we have agreement of $M_i$ with $L$.
By construction $M_i^+$ follows $M_i$ from $n^{-1}y_i$ to $n z_i$.
The conclusion follows.
\end{proof}

\begin{lemma}\guid{LFWKMQW} 
We have
\[
M_i^+: \text{follow } L \text{ from } n^{-1} y_i \text{ to } n z_i
  \text{ and follow } F^c_x \text{ from } nz_i \text{ to } n^{-1} y_i,
\]
\end{lemma} 

\begin{proof} The previous lemma allows us to replace $M_i$
with $L$ in the given  segment.
\end{proof}

Our description of the points $y_i$ and $z_i$ gives the following
conclusion.

\begin{lemma}\guid{KBWPBHQ} 
The set $\{y_i\}$ consists of darts $y$ such that
$y$ on $F$ and $L$ such that $y$ is followed by an 
$n^{-1}$-step on $L$.
The set $\{z_i\}$ consists of darts $z$ on $F$ and $L$ such that
$z$ is preceded by an $n^{-1}$-step on $L$.
\end{lemma} 

\begin{lemma}\guid{XBXFJPH}
The darts of $F_x$ visited by $\LL$ all lie on the loop $L$.
\end{lemma}

\begin{proof}  Suppose to the contrary that some dart $u$ of $F_x$ is
visited by $\LL$ but not $L$.  For some power $i$, the transform
$(H,\MM,M,x) = T^i(H,\LL,L,x)$. Has the property that $u$ is
the first dart encountered on $M$, starting after $y_i$ that lies on $\LL$.
By definition, this is $z_i$, which as we have seen lies on a segment of
$L$.  This is a contradiction.
\end{proof}

%  Let $k$ be
%the number of darts $y$ in the loop $L$ such that $y$ is on $F_x$, but
%the dart following $y$ on the contour loop $L$ is $n^{-1}y$ rather
%than $f y$.  Ordering according to the loop order starting at $x$,
%let $y_i$, for $i=1,\ldots,k$, be the darts in the loop $L$ such
%that $y_i$ is on $F_x$, but followed on $L$ by $n^{-1}y_i$.
%Following $f$-steps from $y_i$, let $z_i$ be the first dart of $H$
%after $y_i$
%that is again visited by $\LL$.


%Here is the main structural theorem.

%\begin{lemma} The darts $z_i$ of $H$ are visited by $L$.  
%The atoms of $\MM_i$ (in order) are the singleton darts $x$,
%$f x$, taking $f$-steps until the atom of $\MM_i$ containing $z_i$
%is reached, then from there following the atoms of the loop $L$
%back to $x$.  The darts $y_j$, for $j\le i$,
%occur in increasing order along the segment before $z_i$.
%The dart $y_{i+1}$ is the first dart encountered on $\MM_i$
%that is followed by a $n^{-1}$-step.
%The dart $z_{i+1}$ occurs on $\MM_i$ (or $L$) in the segment after
%$z_i$.  
%\end{lemma} 

%\begin{proof} This is an induction, using the transform both for
%the base case and induction step.  
%It uses Lemma 4.67 (loop confinement, HQYMRTX).
%\end{proof}



\subsection{The induction step}

The induction step is quite involved.  It goes from
$(\LL_i,g_i,\phi_i)$ to $(\LL_{i+1},g_{i+1},\phi_{i+1})$.
We describe all of these data.

We will give a function $\Phi$ that transforms $(\LL_i,g_i,\phi_i)$ to
$(\LL_{i+1},g_{i+1},\phi_{i+1})$.  For now, we drop the subscripts and
describe the function $\Phi$ on an arbitrary $(\LL,g,\phi)$ such that
$g$ is planegraph, and $\phi$ is an isomorphism between $H/\LL$ and
the hypermap coming from $g$.  Note that $H$ is fixed throughout the
construction.

\subsubsection{construction of $\MM$}

By construction, the loop $L$ is canonically false.
We write $T^k(H,\LL,L,x) = (H,\MM,M,x)$, where $k$ is
the number of iterates required to produce a canonically
true face $M$, (which equals the number of darts $y$ 
on the loop $L$
that are preceded by an $f$-step and followed by an $n^{-1}$-step).
We define the first coordinate of $\Phi(\LL,g,\phi)$ to be $\MM$,
obtained from the $k$th iterate of the transform.

\subsubsection{construction of $g'$}

We next describe the lgraph $g'$ obtained as the second
coordinate of $\Phi(\LL,g,\phi) = (\MM,g',\phi')$.  
In the Isabelle development there are functions, generatePolygon
and nextPlane, that generate a new $g'$ from what is called
an enumeration.  An enumeraton is an increasing list of integers:
\[
a_1\le a_2\le\cdots \le a_r
\]
satisfying a few simple inequalities.   
Thus, to specify $g'$, it is enough to give the list $a_i$.

We say that $(\phi,g,H,NF)$ are corresponding hyperdata if
the following properties hold.
\begin{enumerate}
\item $\LL$ is a normal family in $H$.
\item $g$ is a planegraph.
\item $\phi$ is an isomorphism between the hypermap of $g$  and
$H/\LL$.  
\item  The set of canonically true faces of $H/\LL$
corresponds under $\phi$ with the set of faces obtained from 
finals~$g$.  
\item  $H/\LL$ is a simple hypermap.
\item The node map has no fixed points on the subquotient $H/\LL$.
\item The canonical function is a $\emptyset$-flag on $H/\LL$.
\end{enumerate}

We will need to show that if the input is corresponding hyperdata, then
the output is as well.

There is a function, finalGraph, which is true exactly when
the set of faces of $g$ that are not final is empty.  By our
termination condition, we will stop modifying $(\LL,g,\phi)$
when we have finalGraph~g.  Hence, if finalGraph~g holds, we
set
\[
\Phi(\LL,g,\phi) = (\LL,g,\phi).
\]
Now we assume that finalGraph~g does not hold.

There is a function, minimalFace, that picks out a face $F$ from 
$g$ that is not final.  Under the isomorphism $\phi$ this
corresponds with a loop $L$ of $\LL$ that is canonically false.

There is a function, minimalVertex, that picks out a vertex $v$
of $g$ on the face $F$.  We describe how a face $F$ and a vertex $v$
on the face give a dart $d(F,v)$ of the hypermap of of $g$.
The vertex $v$ is an element of $F$.  The nextVertex $w$ on $F$
gives a dart $(v,w)$ of the hypermap.

Under the isomorphism $\phi$, the dart $(v,w)$ maps to an
atom in $H/\LL$.
Write this atom as $\bar x = [\ldots;x]$, ending in the dart $x$ of $H$.
Then $x$ is a dart visited by $L$, such that $x$ is
followed by an $f$-step in the loop $L$.

\begin{lemma}\guid{HKBGWJI}
In this construction, $(H,\LL,L,x)$ is a marked hypermap.
\end{lemma} 

\begin{proof}  
We verify each of the properties of a marked hypermap.

$H$ is restricted, so it has no M\"obius contours and $e$ acts
without fixed points.

By construction $\LL$ is a normal family, $L$ is a loop of $\LL$
and $x$ is a dart visited by $L$.

1. By assumption, $H/\LL$ is simple. 2. Also, by assumption, the
node map has no fixed points on $H/\LL$.  3.  The face $F$ of
the planegraph $g$ is not final, so the loop $L$ is not canonically
true.  The dart $x = \phi(a,b)$ is visited by $F$.  By an earlier lemma,
the dart $(b,a)$ lies on a face that is canonically true.  Then $\phi(b,a)
= e x$ is visited by a contour loop $L'$ that is canonically true.
5.  It is enough to show that the canonical function is a $\empty$-flag
on the subquotient $H/\LL$.  This is also true by assumption.
\end{proof}



We use $\LL,F,x$ to define the list $a_i$.  The length $r$ will be the
cardinality of $F_x$.  Label the darts in the face $F_x$ as $x_i =
f^{i+1} x$, for $i=0,\ldots,r-1$.  Note the shift in indexing: $x_0 =
f x$.

We use the SOME, NONE type for the sequence $c_i$.
Set $c_i = \text{NONE}$
if $x_i$ is not visited by  $\LL$, and otherwise set 
$c_i = \text{SOME}~Q_i$,
where $Q_i$ is the atom of $\LL$ containing $x_i$.  

Let $t$ be the number of atoms in the loop $L$.  Number the quotient
darts of $H/\LL$ on the face corresponding to the loop $L$ in
consecutive order:
\[
q_0,q_1,\ldots,q_{t-1},
\]
where $q_0$ is the atom of $x_0$ in $L$.

By the previous lemma, each $Q_i = q_j$ for some $j$.
Set $b_i = \text{NONE}$
if $x_i$ is not visited by the loop $L$, and otherwise set 
$b_i = \text{SOME}~j$,
where $Q_i = q_j$.
By the  lemma in the previous subsection, 
the integers $j$ are strictly increasing.

Set $a_i' = j$, where $b_{i'} = \text{SOME}~j$, and
where $i'\le i$ is the largest index less than or equal to $i$ such
that $L$ visits $x_{i'}$.  Then $a_i$ is weakly increasing and
generatePolygon constructs $g'$ from this sequence.

We recall exactly how generatePolygon uses $a_i$ to construct
$g'$.  The function hideDups expands $a_i$ back into $b_i$.
The function indexToVertexList generates the
vertices $v_i$ of $g$ corresponding to the integers $b_i$.
The function subdivFace then creates $g'$ from the vertex list $v_i$.

The function subdivFace is recursive, stepping through the vertex list
$v=[v_0;\ldots]$ one by one.  Only $k$ of the steps modify the lgraph.
This is done by splitting the face in a way that corresponds precisely
to the transform described above.

If the vertex $v_i$ is NONE, then a counter is incremented (starting
from 0).  If the vertex $v_i$ is followed in the list $v$ by the next
vertex on the face $F$ of $g$ and the counter is $0$, then nothing
happens and we move on to $v_{i+1}$.  In the remaining case, the face
$F$ is split into two by running new edges from $v_i$ to $v_{i+1}$,
adding new vertices according to the size of the counter.  The counter
is then reset.  Let $g=h_0,h_1,\ldots,h_k$ be the graphs constructed
by this process, with $g' = h_k$.  When the face $F$ is split into two
$F'$ and $F''$, one of the two faces $F''$ replaces $F$ in the next
step of the iteration.  As we iterate through the list $v_i$, we get a
sequence $F_0=F$, $F_1 =F''$, $F_2 = (F'')''$, etc.  of faces.  Note
that the intermediate lgraphs $h_i$ are not planegraphs, only the
initial $g=h_0$ and $g'=h_k$.

The Isabelle function, subdivFace0, marks one new face final $F_k$.
However, on the hypermap side, the canonical function is updated at
every iteration $\MM_i$.  This means that the list of final faces can
fall out of sync with (lag behind) the canonical function.  At stage
$k$, we can resync using the following lemma.

\begin{lemma}\guid{UWAHKWU}
  Let $g$ be a planegraph with parameter $p$.  If $F$ is any face of
  $g$ with at most $p$ vertices, there is a planegraph $g'$, which is
  identical to $g$ except that it makes the face $F$ final.
\end{lemma} 

\begin{proof}
  Applying the function generatePolygon (with the full enumeration
  $a_i=i$) to the face $F$.  No splits are made and the only effect is
  to make the face final.
\end{proof}

\begin{lemma}\guid{RKXPIXF} Let $H$ be restricted and $\LL$ a normal family.
Then $H/\LL$ has no double joins.
\end{lemma} 

\begin{proof}
By definition, the restricted hypermap $H$  has no double joins.
The set of nodes of $H/\LL$ is a subset of the set of nodes of $H$.
Thus a double join in $H/\LL$ would create a double join in $H$.
\end{proof}


\begin{lemma}\guid{XIZEQEV}  
Let $a_i$, for $i=1,\ldots,r$, be an enumeration.  
Then the lgraph $g_j$ obtained by a partial application of
the enumeration $[a_0;\ldots;a_j]$, for $0<j\le r$, is 
planegraph-relaxed provided $a_j \ne a_{j-1}$.
When $j=r$, the  full application is a planegraph.
\end{lemma} 

\begin{proof} A partial application is the same as a full application
of a modified enumeration
\[
a_1,a_2,\ldots,a_j,(a_j+1),(a_j+2),\ldots,t.
\]
\end{proof}

\begin{remark}
A function, containsDuplicateEdge, is used to eliminate certain
enumerations.  We must check that our enumeration $a_i$ is not
eliminated.  Those eliminated are those that create a double edge
between two vertices.  Thus, the enumeration $a_i$ will pass through
the containsDuplicateEdge filter.
\end{remark}

As a consequence the fgraph of $g_j$ is a good-list and a
good-list-node.

\subsubsection{construction of $\phi'$}

Let $\Phi(\LL,g,\phi) = (\MM,g',\phi')$, where $\phi'$ still
needs to be defined in a way so that $H/\MM$ is isomorphic
under $\phi'$ to the hypermap constructed from $g'$.

We can refine the assertion of an isomorphism so that there is an
isomorphism at every step from the hypermap constructed from $h_i$ and $H/\MM_i$, where $T^i(H,\LL,L,x) = (H,\MM_i,M_i,x)$.  Also, $M_i$ will
correspond with $F_i$ under the isomorphism.  The isomorphism $\phi'$
is constructed from $\phi$ in $k$ steps.  $\psi_0 = \phi,\ldots,\psi_k
= \phi'$.

%(Warning: we have shifted notation from earlier.  The
%subscript $i$ on $h_i$, $\psi_i$ is running from $0$ to $k$
%for the iterates of the transform.  This is all part of one step
%of $\Phi$.  Earlier the subscripts were indexing the iterates of
%$\Phi$ up to $N$.)

Inductively, assume that we have defined the isomorphism $\psi_i$
starting with the base case $\psi_0=\phi$.  We construct $\psi_{i+1}$
from $\psi_i$.  
%Since $H/\LL$ and the higher transforms $H/\MM_i$
%are simple hypermaps, we can specify a dart of $H/\MM_i$ by
%the face and node that it lies at.  That is, we can specify $\psi_{i+1}$
%by saying where it maps each face and vertex of $g$.

Inductively, we know $\psi_i$.  In going from $h_i$ to $h_{i+1}$, the
face $F_i$ splits into two $F'$ and $F_{i+1}=F''$ and the other faces
are the same in the two lgraphs.  The darts of $F'$ and $F''$ are the
same as those of $F_i$ except along a new sequence of edges running
between vertices $v$ and $w$ of $F_i$.  The counter specifies the
number of inserted vertices.  Each of the inserted vertices have
degree two, hence two darts at the level of hypermaps.

Recall that  $T(H,\MM_i,M_i,x) = (H,\MM_{i+1},M_{i+1},x)$.  The
faces of $H/\MM_i$ are in bijection with the loops in $\MM_i$. 
The transform splits $M_i$ into $M_{i+1}$ and $M_i^+$.  At the level
of faces, we have $\psi_{i+1}$ map $F_{i+1}$ to $M_{i+1}$ and $F'$
to $M_i^+$ and leaving other faces unchanged from $\psi_i$.

In going from $h_i$ to $h_{i+1}$, all of the vertices are the same
except for the new vertices $u$ added from the counter (which counts
the number of NONE entries, which equals a suite of consecutive
indices $j$ such that $b_j = \text{NONE}$).  Each entry NONE was
defined as an index $j$ such that $x_j$ does not lie at a node visited
by $M_i$.  By construction, $x_j$ does not
lie at a node visited by $\MM_i$.  Define $\phi_{i+1}$ to map the dart
$d(F'',u)$ to $x_j$, and $d(F',u)$ to $n x_j$.

At the joining vertices $v$ and $w$, we have atoms $\phi_i(d(F_i,v))$
and $\phi_i(d(F_i,w))$ which are each split into two atoms in
$H/\MM_{i+1}$.  by the transform map of marked hypermaps.  We define
$\phi_{i+1}$ on the darts
\[
d(F',v), d(F'',v), d(F',w), d(F'',w)
\]
to map to these four atoms, in the unique way that preserves nodes and
faces.  We leave it as an exercise, that $\phi_{i+1}$ defined on darts
this way is an isomorphism of hypermaps.

\begin{lemma}\guid{EPWRLGS} For all $i$, we have that 
$\psi_{i}$ is an isomorphism of the hypermap
of $h_{i}$  with $H/\MM_i$.
\end{lemma} 

Taking $i=k$, we obtain an isomorphism $\phi'=\psi_k$ of
the hypermap of $g'=h_k$ with $H/\MM$, where $\MM=\MM_k$.

This completes the description of $\Phi(\LL,g,\phi) = (\MM,g',\phi')$.
By construction, if $g$ is a planegraph, then $h_i$ is a planegraph,
and $g' = h_k$ is a planegraph.

As said above, we iterate $\Phi$ for sufficiently many times, to obtain a
hypermap quotient that is isomorphic to $H$ itself.
This completes the description of the isomorphism between
$H$ and the hypermap of a  lgraph $g$ that is a final planegraph.

\subsection{Tame hypermaps}

The formalization of this section is complete.

\begin{lemma}\guid{OXAXUCS} 
\formalauthor{tch}
If a hypermap $H$ is isomorphic to one that has property
tame $k$, for $k\in \{9a, 10, 11a, 11b, 12o, 13a\}$, then
$H$ has that property as well.
\end{lemma} 

Each of the tame properties $I=\{9a,10,11a,11b,12o,13a\}$ has
a corresponding definition for lgraphs, say $I'$.  This correspondence is
defined in such as way that the following holds.

\begin{lemma}\guid{WMLNYMD}\formalauthor{tch}  Let $g$ be a good graph.  
Let $H$ be a tame hypermap that is isomorphic to the hypermap of $g$.
Then $g$ is tame.
\end{lemma} 

Let $H$ be any tame hypermap.  It is restricted, so there exists
a final planegraph $g$ and an isomorphism between $H$ and the
hypermap of $g$.  It follows that the hypermap of $g$ has all of
the tameness properties $I$.  Hence $g$ itself has all of the tameness
properties $I'$.  

By the Bauer-Nipkow formalization on tame graphs, $g$ is 
fgraph congruent
to an fgraph $y$ in the archive.

\begin{lemma}\guid{XRFJNDO}
\formalauthor{Solovyev}
\formalnote{See tame/good\_list\_archive.hl}
 Every member of the archive 
is a good-list.
\end{lemma} 

\begin{proof} This is by direct enumeration of the archive.
\end{proof}

\begin{lemma}\guid{ELLLNYZ}
\formalauthor{tch} 
Let $x$ and $y$ be two good-lists that are fgraph congruent.
Then their hypermaps are isomorphic, or the opposite hypermap of $x$
is isomorphic to the hypermap of $y$.
\end{lemma} 

Putting these results together we have that $H$ isomorphic to the
hypermap of $y$ or the opposite of $H$ is isomorphic to the hypermap
of $y$.

The main linear programming result, formalized by Solovyev, shows
that if $H$ is isomorphic to the hypermap of $y$, then it is not
contravening.  We need the opposite as well.

\begin{lemma}\guid{ASFUTBF}\formalauthor{tch}
 If the opposite of $H$ is contravening, then  $H$
is also contravening.  
\end{lemma} 

\begin{proof}
  If $H$ is contravening, this means there is a finite packing $V$
  which is contravening and whose associated hypermap
  $\op{hyp}(V,E_{std}(V))$ is $H$.  The finite packing $-V$ obtained
  by negating all the coordinates is also contravening.  Its hypermap
  is isomorphic to the opposite of $H$.
\end{proof}















