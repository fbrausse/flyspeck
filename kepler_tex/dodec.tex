%% XX unstable, instability??
%% XX Define mu_S (single simplex version)

\section{Introduction}

% XX http://www.sojamo.de/blog/2007/04/  PUT IMAGE OF 3D VORONOI.

\label{sec:dodec}

A packing of congruent unit radius balls 
in Euclidean space determines a region called the Voronoi cell 
around each ball.  
A packing is determined by and is identified  with the set $\Lambda$ of centers of the balls.  The Voronoi
cell $\Omega(\Lambda,v)$ around a ball at $v\in \Lambda$ 
consists of points of space that are closer to $v$ than
to any other $w\in\Lambda$.  The Voronoi cell is a
convex polyhedron containing $v$.

The Dodecahedral conjecture asserts that in any packing of congruent balls of Euclidean
space every Voronoi cell has volume at least that of a regular dodecahedron
circumscribing a unit ball.    This bound is realized by a finite 
packing $\Lambda_{dod}$
(of twelve balls and a thirteenth  at the origin) obtained
by placing a ball at the center of each face of a regular dodecahedron.  The
theorem can then be stated as the inequality
  $$
  \op{vol}(\Omega(\Lambda,v)) \ge \op{vol}(\Omega(\Lambda_{dod},0))
  $$
for every $v\in\Lambda$, and for every set of points $\Lambda\subset \ring{R}^3$
whose pairwise distances are at least the diameter $2$.
The case of equality occurs exactly when $\Omega(\Lambda,v)$ is
congruent to a regular dodecahedron of inradius $1$.



\subsection{history}


L. Fejes T\'oth made the conjecture in 1943 \cite{Toth1}.  
In that article, L. Fejes
T'oth sketches a proof based on an unproved hypothesis.  This
hypothesis is an explicit version of the kissing number problem
in three dimensions.   This unproved hypothesis
is now generally regarded as being 
nearly as difficult as the  Dodecahedral conjecture itself.  

L. Fejes T\'oth returned to the Dodecahedral conjecture in a number
of publications.  It is a prominent part of his two books \cite{Fej72},
\cite{Toth2}.  According to the strategy of \cite{Fej72}, 
the Dodecaehdral conjecture forms a
step towards the solution of the sphere packing problem
(discussed below).   In \cite{Toth2}, he proved that the Dodecahedral conjecture
holds for every Voronoi cell with at most twelve faces.  
This result is reviewed in Section~\ref{sec:12sphere}.  It
is an ingredient in our proof. 

Lower bounds on the volume $M$ of a Voronoi cell can be expressed
as upper bounds on the density $4\pi M/3$ of packings of congruent balls
in three dimensions. The Dodecahedral conjecture gives an upper
bound on density of $0.755$.  
Upper bounds on the density based on lower bounds on the volume of a Voronoi cell
literature include Rogers's upper bound of 0.7797 \cite{Rog}, and Muder's upper 
bounds of 0.77836 \cite{Muder1} and 0.7731 \cite{Muder2}. 

In 1993, Hsiang published what seemed to be a proof of the Kepler
conjecture which would prove the Dodecahedral conjecture 
as well \cite{Hsiang}.
However, the proof did not hold up to careful analysis.  ``As of this
writing, Kepler's conjecture as well as the dodecahedral conjecture
are still unproven'' \cite[p761]{Bezdek}.  See also, \cite{Hal94}.

An alternative approach to the Dodecahedral conjecture is described
in \cite{Bezdek}.  Unfortunately, a counterexample has been found to
both parts of the third conjecture of that paper.  The counterexample
is described in \cite{arx}, the first version of this paper.  We
do not repeat the counterexample here.


K. Bezdek conjectures that the surface area of any Voronoi cell in a packing
of unit balls is at least that of a regular dodecahedron of inradius $1$.
This strengthened version of the Dodecahedral problem is still
open \cite{Bez04}.   



\subsection{sphere packing problem}

The Kepler conjecture, also known as the sphere packing problem, 
asserts that no packing of congruent balls
in three dimensions has density greater than the density of the
face-centered cubic packing.  
The research for the proof of the Dodecahedral Conjecture 
was carried out at the University
of Michigan at the same time that S. Ferguson and T. Hales were
carrying out their work on the sphere packing problem.  Both
problems were solved in 1998.  

There is no strict logical connection between the two problems.
The Dodecahedral conjecture does not follow from the Kepler conjecture
and is not an intermediate step in the solution to the Kepler conjecture.
(In Fejes T\'oth strategy, it was an intermediate step; however, that
strategy was not followed in the solution of the sphere packing problem.)
Nevertheless,
the two solutions follow a similar outline and share a significant number of 
methods.
Both are based 
on long computer calculations.  Computer
code was freely exchanged between the two projects.

This article is written in a way that it is not necessary to
read or understand the solution of the sphere packing problem before
reading this article.  However, for the benefit of the reader,
at various points in the proof
of the Dodecahedral conjecture, we point out 
parallels with the sphere packing problem.  We also cite various results
from that proof.



\subsection{differences}

Although the proof of the Dodecahedral conjecture runs parallel
to the solution to the sphere packing problem, there are special
difficulties that arise in the proof of the Dodecahedral conjecture.
In no sense is it a corollary of the sphere packing problem.
In the packing problem, there turn out to be many ways to
reduce the infinite ball problem to a problem about finite clusters
of balls.  This multiplicity of choices makes it
possible to design many difficulties away.  If one reduction is
not satisfactory one can work with
another.
With the Dodecahedral problem, there is no such flexibility.
The problem about finite clusters of balls is fixed from the outset.
This gives the problem a degree of rigidity that is not present
in the sphere packing problem.


\subsection{ten years later}

Over ten years have elapsed from the completion of the research until 
publication.  A few words of explanation are in order.
The review and 
publication process for the Kepler conjecture extended from 1998
until 2006.  Because of significant sharing between the Kepler conjecture
and the Dodecahedral conjecture, the editors adopted a ``wait and see''
attitude toward the Dodecahedral conjecture.  Once the Kepler conjecture
was published, the path became clear for the publication of Dodecahedral
conjecture.  

The details of the proof in the current publication are essentially the
same as those in the preprint posted on the arXiv in 1998.  No
significant errors have emerged in the original 1998 preprint.   However,
the paper has undergone three rewrites since then, an expanded version in 2002, followed by an abridged version in 2003, then a newly expanded version in 2008, all coming at the request by the editors.  It was as if a complex computer proof caused a discomfort that might be relived by a long series of
rewrites. The computer code has
also been entirely rewritten.

As an abridgement of a longer version, this article replaces 
some proofs with summaries and references.  The most complete version.
is \cite{arx}.  That version is written in such a way that it is
entirely independent of the proof of the Kepler conjecture. 
We refer the reader when necessary to relevant
passages in that article.

A formalization project, called Flyspeck, aims to
provide a complete formalization of the proof of the Kepler conjecture \cite{fly},\cite{Fl}.  (A formalized proof is one in which every logical inference of the proof has been independently checked by computer, all the way to the primitive axioms at the foundations of mathematics.)
A parallel project, called Flyspeck Light, aims to do the same
for the proof of the Dodecahedral conjecture.  These long-term projects
will take many years to complete.  Nevertheless, significant progress
has already been made toward the formal verification of the computer
code \cite{BN}, \cite{Ob}.   The revisions in this paper 
incorporate the parts of Flyspeck Light that have
already been completed.

\subsection{truncation}

The distance from the center of the regular dodecahedron to a vertex is
$t_{dod}=\sqrt{3}\tan(\pi/5)\approx 1.258$.   This parameter is used to truncate
Voronoi cells; it makes  volumes easier to estimate.  A similar truncation
takes place in the solution to the packing problem with truncation parameter
$t_0 = 1.255$.  It is a happy coincidence that these two truncation parameters 
are so close to one another.  (Section~\ref{sec:experiment} 
gives the reasons behind the parameter $t_0$.)  A great deal of duplicated effort
might have been avoided if these two parameters were equal.  However, the parameter
$t_{dod}$ cannot be replaced with anything smaller, 
and although the parameter $t_0$ could 
easily have been made larger, 
its value was already too deeply entrenched in published papers 
by the time  work started
on the Dodecahedral conjecture.  

As a result, there are several definitions in the proof of the Dodecahedral conjecture
and in the solution to the packing problem, identical in every respect except
for the choice of parameter: $t_0,t_{dod}$.  This is indeed unfortunate.  However,
there is no easy remedy. 



 As a first step towards unifying the proofs,
 many results can be stated in a form that holds for all $t\in[t_0,t_{dod}]$.
To transfer a lemma from \cite{DCG} to this article, a simple process is
involved.  The first step is generalization, replacing the constant $t_0$
with a free parameter $t\in[t_0,t_{dod}]$.  The second is specialization,
$t\mapsto t_{dod}$.  

The number $t_0$, although rational, 
can be consistently treated as an independent real transcendental 
in the solution to the sphere packing problem; that is,
none of the proofs involving $t_0$ rely on its exact numerical value.
The constant $t_0$ can always be replaced by a constant $t$
in a suitably small interval about $t_0$.  However, the only way to
know that this small interval is wide enough to contain $t_{dod}$ is to study the
details of the proof.  For this article, 
we have made a detailed study of each relevant proof in \cite{DCG}.

As a matter of terminology, we say a proposition for the Dodecahedral conjecture
is a $[t_0,t_{dod}]$-translation of a proposition of \cite{DCG},
if it is obtained by mindlessly replacing $t_0$ with $t_{dod}$, wherever that constant appears, and if the proof goes through verbatim with this minor change.
When this occurs, 
there is nothing to further to be learned by repeating the proof,
and we refer the reader  for the proof to \cite{DCG}.

\subsection{terminology}

Various notation and terminology is shared between the solution
of the sphere packing problem and this article.  

A large technical vocabulary has been developed 
to describe the solution to the sphere packing problem.  
Vocabulary can be imported from the sphere packing problem three
different ways.  The simplest way to import a term is for the
term to have precisely the same meaning in both places.  For
example, the terms ``orientation,'' ``packing,'' have the
same meaning in both places. 

The second way to import a term is by making a $[t_0,t_{dod}]$-translation of a term that depends on the parameter $t_0$.  For example,
in the solution of the sphere packing problem, a {\it quasi-regular
tetrahedron} is defined to be a set $S\subset\ring{R}^3$ of four
points forming a packing, such that
   $$
   \forall\ v,\ w \in S.\ |v - w| \le 2t_0.
   $$
That is, the edge lengths vary between $2$ and $2t_0$.
The $[t_0,t_{dod}]$-translation of this definition is a set $S\subset\ring{R}^3$
of four points forming a packing, such that
   $$
   \forall\ v,\ w \in S.\ |v - w| \le 2t_{dod}.
   $$

What to call such a set?  In the original version of this article,
this perturbed definition is again called a quasi-regular tetrahedron. 
A referee writes, ``Invent a new name
for `quasi-regular tetrahedron'; its definition is set in stone
in the Kepler conjecture papers.  This is a new definition\ldots'' 
This observation is entirely correct.  The referee's comment applies not
just to a solitary term `quasi-regular tetrahedron,' but to
a large technical vocabulary that has been developed to describe
the geometry of sphere packings.  The mastery of this specialized
vocabulary already places a mild burden on the reader.  The mastery
of two specialized vocabularies, one a $[t_0,t_{dod}]$-translation
of the other, would double that burden.  A new term,
such as pseudo-quasi-regular, would lead to a further burden
as the reader ascertains the exact relationship between
pseudo-quasi and quasi.  All this we wish to avoid.

Since this vocabulary is highly specialized for the proof of these
two problems in sphere packings, it is highly unlikely for these terms to find applications outside this problem domain.  They are not 
general fundamental notions such as set, manifold, or group, where
a large mathematical literature depends on a uniform definition.
The main danger is that a reader might improperly mix results from the
two articles.


The compromise that we have settled upon in this paper is to
retain the same technical vocabulary that is used in the solution
to the sphere packing problem, but to remind the reader regularly
of any variations in meaning that occur.  We ask the reader to
imagine a distinguishing subscript for each occurrence of {\it quasi-regular tetrahedron$_{dod}$}
and 
{\it quasi-regular tetrahedron$_0$} in the two articles.  
Subscript $dod$ in this article, 
and subscript $0$
there, never mixing the two.
For the convenience
of the reader, we give the following complete 
list of definitions and constructions in this
paper that are $[t_0,t_{dod}]$-translations of those in the sphere
packing problem.

\begin{itemize}
\item quasi-regular tetrahedron, standard region, standard component, special.
\end{itemize}

The third way in which terms have been imported into this proof
from the proof of the Kepler conjecture has been by structural
analogy.  For example, an important step in the solution to the
sphere packing problem is a classification of all tame planar graphs.
The term {\it tame graph} is a technical notion that arises in
the solution of the sphere packing problems.  In this article,
there is an analogous classification of planar graphs, this time
related to potential counterexamples to the Dodecahedral conjecture.
The planar graphs that arise here are not tame and they are
not even a $[t_0,t_{dod}]$-translation of the definition of tame.  
Nevertheless, a strong parallel exists between tame planar graphs 
and the graphs that are studied in this article.  
In fact, the same piece of computer
code was used to carry out both classifications. 

What to call them?  In this case, it would be too much of a stretch
to keep an identical name for this new type of planar graph.
In these cases, when we import the term, we modify it with
the adjective {\it structural}, to remind us that the way it functions 
bears a structural
analogy with the solution to the sphere
packing problem.  Thus, in this paper, we classify all
{\it structurally tame planar graphs}.

For the convenience of the reader, here is a list of terms that
have been imported by analogy, and qualified with the adjective
{\it structural} or adverb {\it structurally}.

\begin{itemize}
\item tame, squander, squander target, contravening, basic
\end{itemize}




\section{Outline}

In this section, we give a precise statement of the main theorem
and describe the broad outline of the proof.


\subsection{formulation}\label{sec:form}

This article proves the Dodecahedral conjecture in a slightly
stronger version than that stated in the abstract.  It proves
that a certain truncation of the Voronoi cell already has volume
at least as great as that of the regular dodecahedron.  This
section describes the truncation and states the stronger version
of the main theorem in a precise form.

Let $\Lambda$ be a packing and let $v_0\in\Lambda$.
Let $\Lambda(v_0,r)$ be the set of points of $\Lambda$ at distance at most $r$ from $v_0$.
Let $B(v,r)$ be an open ball
of radius $r$ centered at $v$.  

Let $\CalQ_{dod}(\Lambda,v_0)$ 
be the set of all $S=\{v_0,v_1,v_2,v_3\}\subset\Lambda(v_0,2t_{dod})$
consisting of four distinct points such that $|v_i-v_j|\le 2t_{dod}$ for all $i,j$
and such that the circumradius of each triangle $\{v_i,v_j,v_k\}\subset S$ is at most
$\sqrt2$.   Write $\op{conv}(S)$ for the convex hull of $S\in\CalQ_{dod}(\Lambda,v_0)$. 
% and $\op{conv}^0(S)$ for its interior.
%A simple lemma (Lemma~\ref{})
%shows that distinct $S,S'\in\CalQ_{dod}(\Lambda,v_0)$ have disjoint
%interiors:
%   $$\op{conv}^0(S) \cap \op{conv}^0(S') \ne \emptyset 
%   \quad\Rightarrow (S = S').
%   $$


Define the following truncation $\Omega_{trunc}(\Lambda,v_0)$ 
of the Voronoi cell $\Omega(\Lambda,v_0)$:
   $$
   \{x \in \Omega(\Lambda,v_0) \mid   x\in B(v_0,t_{dod}) \quad\text{\bf  or }\quad x \in \op{conv}(S)
     \text{ for some } S\in \CalQ_{dod}(\Lambda,v_0) \}. 
   $$
That is, we truncate the Voronoi cell by intersecting it with a ball of radius $t_{dod}$,
except inside regions protected by the sets $S\in\CalQ_{dod}(\Lambda,v_0)$.
Note that for the special packing $\Lambda_{dod}$, we have 
$\Omega_{trunc}(\Lambda_{dod},0) = \Omega(\Lambda_{dod},0)$.
We prove the  Dodecahedral conjecture in the following form.

\begin{theorem}\label{thm:main}
For every packing $\Lambda$ and every $v_0\in\Lambda$,
   $$
   \op{vol}(\Omega_{trunc}(\Lambda,v_0))\ge \op{vol}(\Omega(\Lambda_{dod},0)).
   $$
Equality holds exactly when $\Omega(\Lambda,v_0)$ is congruent to
$\Omega(\Lambda_{dod},0)$.
\end{theorem}





\subsection{proof outline}

The Lebesgue measure is translation invariant.  Thus it does no
harm to assume that the center point $v_0$ of the Voronoi cell
lies at the origin: $v_0 = 0 \in \Lambda$.  The assumption that
$v_0=0 \in\Lambda$ remains in force for the rest of this paper.


The next reduction is to replace the set $\Lambda$ with $\Lambda(0,2t_{dod})$.
This is accomplished by the following lemma, which shows that the
volumes in Theorem~\ref{thm:main} are insensitive to points of $\Lambda$
outside $\Lambda(0,2t_{dod})$.  The proof appears in Lemma~\ref{lemma:trunc}.


\begin{lemma} Let $\Lambda$ be any packing (with $0\in \Lambda$).
Then
$$\Omega_{trunc}(\Lambda,0) = \Omega_{trunc}(\Lambda(0,2t_{dod}),0).$$
\end{lemma}

The standing assumption that $\Lambda=\Lambda(0,2t_{dod})$ remains in force
for the rest of the paper.

The proof of Theorem~\ref{thm:main}
splits into two main cases, according to whether
$\Lambda\setminus\{0\}$ has at most $12$ points, or at least $13$.
Let $n=|\Lambda\setminus\{0\}|$.  In fact, the case $n\le 12$ was
settled by L. Fejes T\'oth in his book \cite{Toth2}.  Fejes T\'oth's proof
is sketched in Section~\ref{sec:12sphere}.

Completely different methods treat the case when $n\ge 13$.  
This part of the proof is
much more difficult than the case treated by Fejes T\'oth.

We give a sketch of the proof of the case $n\ge 13$.   This
rough sketch will be expanded in considerably greater detail later
in the article.
Let $\Lambda$ be a packing satisfying our standing assumptions
that $0\in\Lambda$ and $\Lambda = \Lambda(0,2t_{dod})$.
We seek to minimize the objective function 
$$
\op{vol}(\Omega_{trunc}(\Lambda,0)).
$$
This is a nonlinear optimization problem in finitely many variables.
The target value for the minimization is $b=\op{vol}(\Omega(\Lambda_{dod},0))$.  When $n\ge 13$, we prove\footnote{An examination of the proof shows the right-hand side can be improved to $b+10^{-10}$.  In fact, in terms of notation to be described later in the article, the proof only
relies on the bound $\mu(\Lambda,U_F) >0$ for triangles $F$, but in fact every configuration has at least one triangle $F$ with $\mu(\Lambda,U_F) > 10^{-10}$},
   $$
   \op{vol}(\Omega_{trunc}(\Lambda,0))  > b.
   $$



Some combinatorial information about each packing $\Lambda$ is encoded
as a  graph.  The vertex set of the graph is $\Lambda\setminus\{0\}$.
The edge set is 
  $$
  E = \{(v,w) \mid v,w\in\Lambda\setminus\{0\},\quad   |v-w| \le 2t_{dod}\}.
  $$
This is a planar graph.  
We may reduce to the case where this graph is connected.  In fact,
if the graph is not connected, we construct another packing
$\Lambda'$ whose graph is connected, $|\Lambda|=|\Lambda'|$, 
and such that
$$
   \op{vol}(\Omega_{trunc}(\Lambda,0)) = \op{vol}(\Omega_{trunc}(\Lambda',0)). 
$$
Similarly, we may reduce to the case where every face of the graph
is a simple polygon.  This again involves constructing an auxilliary 
packing $\Lambda''$ of the same cardinality and whose  truncated Voronoi cell has the same volume.  We now assume that the graph of $\Lambda$ is 
connected with simple polygonal faces.

We assume the existence of a counterexample $\Lambda$ to the Dodechedral
conjecture, and make a detailed study of the properties of its
graph.  We define a class of graphs (called structurally tame) and
prove that the graph of every counterexample is structurally tame.

Structurally tame graphs can be described in purely combinatorial
terms, without reference to packings, Voronoi cells, and volumes.
All structurally tame graphs can be classified up to isomorphism.
This classification is one of the main steps of the proof.
There are only finitely many possibilities.  Thus, the graph
of any counterexample
to the Dodecahedral conjecture must be one of these finitely
many cases.

Each structurally tame graph can be encoded as a hypermap $H$.  (A hypermap can be defined
as a finite set, together with two permutations on that set.  The elements
of the given finite set are called darts.)  If $H$ is a hypermap,
let $V$ be the finite dimensional
vector space of real-valued functions on its set of darts.
We call $(H,\Phi)$ a hypermap system, if $\Phi$ is a set of boolean
valued functions $\phi:V^\ell \to \{\op{true},\op{false}\}$ for
some $\ell$ that is independent of $\phi\in\Phi$.  We say that
a hypermap system $(H,\Phi)$ 
is feasible if there is some $x=(x_1,\ldots,x_\ell)\in V^\ell$
such that $\phi(x)$ holds for all $\phi\in\Phi$.  Otherwise,
it is infeasible.   Appendix~\ref{XX} defines
a hypermap system, called the dodecahedral hypermap system, 
for each case $H$ arising in the classification of
structurally tame graphs.

Another major step of the proof is the proof that every dodecahedral hypermap
system is infeasible (Theorem~\ref{XX}).   The proof of this theorem
is a case-by-case analysis based on the explicit enumeration of
structurally tame graphs, up to isomorphism.   The feasibility problem for 
each dodecahedral hypermap system is converted to a system of linear programs.
The infeasibility of the dodecahedral hypermap system follows from the
infeasibility of the corresponding linear program.

If there exists a counterexample to the Dodecahedral conjecture, then
we can consider its dodecahedral hypermap system $(H,\Phi)$.  By the preceding
result, this hypermap system is infeasible.  On the other hand, we can
use the counterexample to construct a feasible solution to the system 
(Theorem~\ref{XX}).
This contradiction shows that a counterexample cannot exist.
In this way, the Dodecahedral conjecture is proved.







\section{Computation}

The proof of the Dodecahedral conjecture is based on a series
of computer calculations.  This section briefly describes the
computer algorithms, the code implementing those algorithms,
and issues of the reliability of the computer code.

There are three main computer programs that are used in the proof.
The first is a graph generator that generates, up to isomorphism,
 all planar graphs satisfying a list of properties.  The second is
a linear programming package.  The third is a piece of code based
on interval arithmetic that
automatically proves nonlinear inequalities over the real numbers.
We discuss each in turn.  

This section discusses  also some
additional computer programs.  Although these computer programs,
strictly speaking, are not part of the proof, they are relevant
to understanding the structure of the proof and the reliability of
the computer implementation.   We include a brief discussion of
nonlinear optimization software, Tarski's decision procedure for
real-closed fields, and formal theorem proving packages.

\subsection{Tarski arithmetic}


The Dodecahedral conjecture, after a few preliminary reductions, 
can be expressed as a statement
in the elementary theory of the real numbers.  By a fundamental
result of Tarski, the elementary theory of the real numbers
is decidable.  Thus, the
truth of the Dodecahedral 
conjecture can be decided in theory by standard
algorithms such as Collin's cylindrical algebraic decomposition \cite{ColXX}, or the Cohen-H\"ormander algorithm \cite{CHX}.  However,
in practice, these decision procedures are exponential time in the
number of quantifiers, and thus are far too slow to be of practical
value for this conjecture.

To formulate the Dodecahedral conjecture as a statement in the
elementary theory of the reals, we use Voronoi cells without
truncation, centered at the origin.  We remark first of all that the
volume of the regular dodecahedron is an algebraic number $V_{dod}$,
hence definable in Tarski arithmetic.  Also, we can give an
a priori bound $n$ on the number of faces of the Voronoi cell, and thus also
on the size of the clusters of balls that give candidates for counterexamples. (For example, by adding balls to the packing to decrease the volume, we may assume the packing is saturated.  Assuming saturation, E. Harshbarger gives a quick calculation of $n\le58$ faces \cite{Har}. The exact value of this constant is not important as long as it is explicitly given.)
The assertion of the dodecahedral conjecture is then expressed as an enormous conjunction
of cases, with conjuncts indexed by an explicit enumeration of all possible combinatorial structures of a Voronoi cell, including a fixed triangulation of each face of the cell.  The fixed triangulation determines
a partition of the Voronoi cell into tetrahedra.  The volume of each tetrahedron is expressed by means of a Cayley-Menger determinant as a definable function of the the edge lengths of the tetrahedron.  Thus, for each combinatorial structure $X$ with $m\le 58$ faces, we can express the statement that
for all vectors $v_1,\ldots,v_m\in\ring{R}^3$, if the Voronoi cell defined by the vectors $v_i$ has combinatorial structure $X$, then the volume of the cell is at least $V_{dod}$.  The outer block of $3\cdot 58$ universal quantifiers -- not to mention the nested existential quantifiers -- is hopelessly beyond the practical reach of current algorithms.  The statement that the regular dodecahedron is the unique minimizer can be similarly expressed.

If the entire dodecahedral conjecture can be expressed
in Tarski arithmetic, then perhaps it is not so surprising that
many of the intermediate steps in the proof can also be so expressed.
These intermediate steps also tend to be beyond the reach of
current decision procedures.  
But here the situation is not so hopeless.  Many of these
intermediate problems involve no more than a dozen quantifiers.
One can imagine the day that these problems might fall within the
reach of decision procedures for Tarski arithmetic.

We find that describing various intermediate steps 
of the proof as exercises
in Tarski arithmetic is a useful point of view.  Doing so, identifies
a family of subproblems that can be expressed in a common language,
and that can often be solved by similar techniques.  
The complexity of the
problems can be measured objectively by counting the number
quantifiers.

Here are some geometrical 
objects that are definable within the Tarski arithmetic
that appear in the proof of the Dodecahedral conjecture.
The following three polynomials appear repeatedly.
\begin{definition}
$$
\begin{array}{lll}
\ups(x,y,z) &= -x^2 - y^2 - z^2 + 2 x y + 2 y z + 2 z x.\\
\\
 \chi(x_{ij}) &= \chi(x_{12},x_{13},x_{14},x_{34},x_{24},x_{23})\\
     &=
      x_{13} x_{23} x_{24} + x_{14} x_{23} x_{24}  + 
      x_{12} x_{23} x_{34} + x_{14} x_{23} x_{34} + x_{12} x_{24} x_{34}\\ 
      &\quad + x_{13} x_{24} x_{34} - 
      2 x_{23} x_{24} x_{34} - x_{12} x_{34}^2 
      - x_{14} x_{23}^2 - x_{13} x_{24}^2\\
   % &= x_1 x_4 x_5 + x_1 x_6 x_4 + x_2 x_6 x_5 + x_2 x_4 x_5 + x_5 x_3 x_6 \\
   %&\quad+ x_3 x_4 x_6 - 2 x_5 x_6 x_4 - x_1 x_4^2 - x_2 x_5^2 - x_3 x_6^2.\\
\\
\Delta(x_{ij}) &= 
   -x_{12} x_{13} x_{23} - x_{12} x_{14} x_{24} - x_{13} x_{14} x_{34} 
    - x_{23} x_{24} x_{34}\\
    &\quad + x_{12} x_{34} (-x_{12} + x_{13} + x_{14} + x_{23} + x_{24} - x_{34}) \\
  &\quad + x_{13} x_{24} (x_{12} - x_{13} + x_{14} + x_{23} - x_{24} + x_{34})\\
    &\quad + x_{14} x_{23} (x_{12} + x_{13} - x_{14} - x_{23} + x_{24} + x_{34})\\
   %x_1 x_4 (- x_1+x_2+x_3- x_4+x_5+x_6)+\\&
   %         x_2 x_5 (x_1- x_2+x_3+x_4- x_5+x_6)
   %         +x_3 x_6 (x_1+x_2- x_3+x_4+x_5- x_6)
   %         - \\&x_2 x_3 x_4- x_1 x_3 x_5- x_1 x_2 x_6- x_4 x_5 x_6\\
\end{array}
    \label{def:chi}
\label{def:tar:delta}
\label{def:ups}
$$
\end{definition}

By Heron's formula, the circumradius $\eta(x,y,z)$ of a triangle with sides $x,y,z$ is elementary definable:
$$\eta(x,y,z) = x y z/\sqrt{\ups(x^2,y^2,z^2)}.$$
Set $\eta_V(u,v,w) = \eta(|u-v|,|v-w|,|u-w|)$, for three points
$u,v,w\in\ring{R}^3$.

By Cayley-Menger determinants, the volume  of a tetrahedron with vertices $v_1,\ldots,v_4$ is elementary definable:
   $$
   \sqrt{\Delta(x_{ij})}/12,\quad x_{ij}=|v_i-v_j|^2.
   $$
In particular, $\Delta(x_{ij})\ge0$, whenever $x_{ij}$ have
the form $|v_i-v_j|^2$ for some points $v_i\in\ring{R}^3$
\cite[Lemma~8.1.4]{SPI}.

Again, if $x_{ij}=|v_i-v_j|^2$, then we say that the orientation
of $v_1$ is positive (zero, negative) in $S=\{v_1,\ldots,v_4\}$,
according to the sign of $\chi(x_{12},x_{13},x_{14},x_{34},x_{24},x_{23})$.
It is know that the sign of the orientation is positive (negative) exactly
when $v_1$ and the circumcenter of $S$ lie on the same side (opposite sides) of the
plane passing through $\{v_2,v_3,v_4\}$ \cite[Lemma~5.15]{DCG}.  This is an elementary condition. 

For $S=\{v_1,\ldots,v_r\}\subset\ring{R}^3$, let $\op{aff}_+(0,S)$
be the cone with apex $0$ generated by $S$:
  $$
  \{ t_1 v_1 + \cdots t_r v_r \mid  t_i \ge 0\}.
  $$
If we replace the inequalities with strict inequalities, we
write $\op{aff}_+^0(0,S)$.
Let $\op{conv}(S)$ be the convex hull
  $$
  \{ t_1 v_1 + \cdots + t_r v_r \mid t_i \ge 0,\ \sum_{i=1}^r t_i=1\}.
  $$
If we replace the inequalities with strict inequalities, we write
$\op{conv}^0(S)$.





\subsection{formal proof}

A formal proof is a proof in which every logical inference has
been checked, back to the foundational axioms of mathematics. 
Except in trivial cases, a computer is used to generate a formal
proof, because of the large number of inferences involved.
Both conventional proofs and computer assisted proofs can be
formalized.  In a computer assisted proof, this amounts to a
formal verification of the correctness of the computer code.
Formal verification of computer code is a difficult task.
For that reason,  formal verification tends to be reserved
for situations where correct performance 
is critically important, such as the verification of aircraft
control systems, cryptography algorithms, 
and security protocols for the internet.
There is no other means of checking 
computer software that can assure reliability at levels that
remotely compare with the
assurance afforded by formal verification. 



We are involved in a long term project, 
called Flyspeck Light, to give a formal verification of the
Dodecahedral conjecture.  Although this project is far from
complete, parts of this project have already
been carried out.  This means that some of the computer code
for this project now carries a proof of correctness,
according to formal mathematical standards.  One such
program is discussed in the next subsection.




%what it is, graph generator BN, Obua, McLaughlin, Zumkeller.
%Where Trust is important, aircraft control, 
%security protocals for the internet.  Difficult time-consuming
%to do this kind of analysis of software.
%There is no higher standard of software reliability in existence.

%Ongoing project (Flyspeck Light).

\subsection{graph generation}

The proof of the Dodecahedral conjecture is based on three separate
computer programs.  The first of these is a planar graph generator.
It generates all planar graphs, up to isomorphism, that satisfy
a given list of restrictions.  The restrictions include a bound on
the number of vertices in the graph, so that it is obvious that there are only a finite number of possible graphs.

The correctness of the graph generator program has been the 
subject of extensive
mathematical investigation.  Early versions of the program were
written by Hales in 1994 (in Mathematica), 1997 (in Java), and 2000 (in Java).  The same computer program is used in the Kepler conjecture
and Dodecahedral conjecture.  They differ only in their input parameters.
This computer program became the subject of  G. Bauer's
dissertation in computer science at the Technical University of Munich
\cite{Bauer}.
This 172-page dissertation translates the Java code into the
formal theorem proving system Isabelle/HOL and gives a detailed 
mathematical treatment of the graph theory underpinning the
computer code.  The dissertation analyzes every line of code.  
Building on the work of this thesis, B. Bauer
and T. Nipkow have completed the formal correctness proof of the HOL implementation
of the graph generator \cite{BN06}.  (Their published paper mentions
only the Kepler conjecture, but the formal verification 
has been extended to
apply to the input parameters of the Dodecahedral conjecture as well.)
As a result of this work, the graph generation is currently the
most scrupulously checked part of the proof of the Dodecahedral
conjecture.

There are several published sources that provide details of this
algorithm, and there is no need to repeat details here.   The basic idea is to start
with a small set of planar graphs  (called `seed' graphs) with
the property that every planar graph to be classified is known to have
one of the seed graphs as a subgraph.  The seed graphs are then
extended by adding one face at a time.  Faces are added in all
possible ways so that it is clear at every step of the algorithm
that every connected planar graph will be generated.  At the same time,
pruning operations discard partially completed graphs when it can
be shown that the partial graph is not a subgraph of any of the
graphs to be classified.  The pruning operations prevent a
combinatorial explosion of cases.
See \cite[sec.5]{alg}, \cite[sec.19]{DCG}, \cite{Bauer}, \cite{BN06}.


%What it is, several implementations, bug 2000,  Mathematica, java,
%java, ML (Isabelle), formal proof of correctness.

\subsection{linear programs}

The second computer 
program that is used in the proof of the Dodecahedral
conjecture is linear programming.  There are several hundred
linear programs that appear in the proof.

Formal verification has not yet been extended to this portion
of the computer code.  However, the recent dissertation of
S. Obua takes the first steps in this direction \cite{Ob}.
That work gives a formal correctness proof of the {\it basic} linear
programs that appear in the proof of the Kepler conjecture.
In particular, he has developed all the infrastructure needed to
carry out formal correctness proofs of linear programming problems.
The formalization  completed by Obua is a larger project than the
formalization of the linear programming segment of the Dodecahedral
conjecture.
Thus, we can expect that the formalization of this piece of
computer code  will soon follow suit.

In the 1998 proof, computer code written in C was used to generate
the linear programs, which were then fed to the 
commercial linear programming
package CPLEX.  In preparation for a formal proof, the computer
code has been rewritten in the programming language ML, with
an external interface to the lp solver GLPK.

We stress that it is not necessary to trust the algorithms of the
linear programming packages (such as CPLEX and GLPK) that solve
the linear programs.  These packages produce dual certificates that
can be used to give independent verification of the solutions
of the linear programming problems.

In the proof of the Dodecahedral conjecture, we encounter the
following situation.  We wish to prove that the maximum of a
linear function 
$x\mapsto c x$ is less than given constant $M$, when subject
to a system of linear constraints $A x \le b$.  (Here $x$, $c$, $b$
are vectors with real entries and $A$ is a matrix with real
entries.  The products are given by matrix multiplication of compatibly sized matrices and vectors. 
We write
a vector inequality $u\le v$
to mean that $u_i \le v_i$ for each coordinate $i$.) We are given
explicit lower and upper bounds on the variables: $\ell \le x \le u$.
Expressed equivalently, we wish to show that the linear 
system of inequalities
  $$
  A x \le b,\quad \ell \le x\le u,\quad c x \ge M
  $$
has no solutions in $x$.
The external linear programming package produces a dual certificate
in the form of a vector $y$, which that package claims to have
the properties 
  \begin{equation}
  y A = c,\quad y\ge 0,\quad y b < M.
  \end{equation}
If $y$ indeed has these properties, then for any $x$ satisfying
$A x \le b$, we have
   \begin{equation}\label{eqn:cxM}
   c x = y A x \le y b < M
   \end{equation}
as desired.

Because of inexact arithmetic used by the external packages, 
these identities will only be
approximately correct.   The imprecision in the dual
certificate can be readily eliminated as follows. If $u$ is
any vector, let $u^+$ be the vector obtained by replacing the
negative entries of $u$ with $0$, and let $u^-$ be the vector
obtained by replacing the positive entries of $u$ with $0$.
By replacing the vector $y$ with $y^+$, we may always assume
that $0\le y$.  In the following lemma, $s_1$ and $s_2$ are
small error terms that result from machine approximation.
By including them in the bounds on $c x$, a rigorous bound
can be recovered.


\begin{lemma}  Suppose that the real-valued vectors and matrices
$A,A_1,A_2$, $c,c_1,c_2$, $x,b,\ell,u$, $y$ satisfy the following
relations
  $$
  A x\le b, \quad A_1 \le A \le A_2,
  \quad c_1 \le c \le c_2,\quad \ell\le x\le u,\quad
  0\le y.
  $$
Define residuals
  $$
   s_1 = c_1 - y A_2,\quad s_2 = c_2  - y A_1.
  $$
If
$$
y b + s_2^+ u^+ + s_1^+ u^- + s_2^- l^+ + s_1^- l^- < M,
$$
then $c x < M$.
\end{lemma}

\begin{proof}  S. Obua has given a formal proof of this lemma in
the Isabelle/HOL system \cite[3.7.2]{Ob}.  In fact, the proof is
nothing but a slight
embellishment of Inequality~\ref{eqn:cxM}.
\end{proof}

We note that $A_1,A_2,c_1,c_2,\ell,u,y,b$ are all explicitly given
numerical data, so that explicit bounds are obtained by this
method.  It is not necessary to trust the package
that produces the certificate $y$, because
all of the assumptions of the lemma can be checked directly with
simple matrix multiplications.


XX note about checking certificates with interval arithmetic.

\subsection{interval arithmetic}

The third major computer program that is used in the proof of
the Dodecahedral conjecture is an nonlinear-inequality prover
over the real numbers based on interval arithmetic.  
This subsection describes the
methods involved and the computer implementation.

A finite number of nonlinear functions $f_1,\ldots,f_k$ are given.
It is assumed that all functions have the same domain
   $$R= [a_1,b_1] \times [a_2,b_2] \times [a_m,b_m],$$
given as a product of intervals in $\ring{R}^m$, for some $m$.
The computer program verifies that
  \begin{equation}\label{eqn:fpos}
  (f_1(x) >0) \lor (f_2(x) >0) \lor \cdots \lor (f_r(x) >0),
  \end{equation}
for every point $x\in R$.
The approach we take is similar to the approach described in
R. B. Kearfott \cite{Kea96}, based on interval arithmetic.
Our methods are similar to algorithms in widespread use for
rigorous global optimization.  Closely related algorithms
are also described in \cite{Zu}.

The method is based on an iteratively refined 
partition of the domain into $R$
into a finite number of smaller and smaller rectangles that cover $R$.
% Grothendieck topologies again!
We give a somewhat simplified description of the algorithm.
At iteration $p$, we have a cover of $R$ by smaller rectangles
$R^p_1,\ldots,R^p_{k_p}$ and determine lower bounds
   $$
   f_j(x) > a_{p,i,j},\quad \text{ for } j=1,\ldots,r;\quad i=1,\ldots,k_p.
   $$
%If $a_{p,i,j}\ge 0$ for some $j\le r$, then the condition~\ref{eqn:fpos}
%holds for $x\in R^p_i$.  
If for every $i$, there exists a $j$ such that $a_{p,i,j}\ge 0$,
then condition~\ref{eqn:fpos} holds for $x\in R$, and the algorithm
terminates.

If not, pick $i' \le k_p$ for which no such $j$ exists.  We
pass to the next iterate $p+1$ of the algorithm.  Cover $R^p_{i'}$
with finitely many rectangles $S^{p+1}_1,\ldots$, and let
  $$
  R^{p+1}_1,\ldots,R^{p+1}_{k_{p+1}}
  $$
be the collection consisting of $S^{p+1}_1,\ldots$ and $R^p_1,\ldots$
(omitting $R^p_{i'}$).  We compute new lower bounds $a_{p+1,i,j}$ 
for $f_j$ on each new rectangle, and continue as before.


When we subdivide rectangles to obtain smaller covers, we do
it in such a way the the width of the rectangle tends to zero
with increasing $p$.
In this way, we are able to arrange for the approximations
$a_{p,i,j}$ to $f$ to converge to the true minimum of $f$ on the
rectangle.

The lower bounds $a_{p,i,j}$ to a function $f_j$ on a rectangle
are obtained by methods of interval arithmetic.  The function
$f=f_j$ is generally $C^2$ on its domain.\footnote{The functions
we encounter in practice are usually $C^2$, but not always so.  When functions are not
$C^2$, we avoid the use of Taylor approximations.}  We expand
the function $f$ in a Taylor polynomial approximation with
explicit error bounds.  Derivatives are calculated by automatic
differentiation.  The error bounds are based on the Lagrange form of
the error term in the Taylor approximation.  Interval arithmetic is used to produce rigorous bounds on the error terms.

We take comfort in the fact that we have
been able to give rigorous proofs of difficult inequalities,
especially when confronted with the fact that 
nonlinear optimization is NP hard.  Considerable work has gone
into the implementation of these algorithms.

There have been several separate implementations of the interval
arithmetic package.  
The source code for all of these packages
is publicly available. (The code that is used for the proof of 
the Kepler conjecture is the same as the code that is used for
the proof of the Dodecahedral conjecture.  Only the statements
of the inequalities to be proved differ.) 
The first version, written in C++, 
was developed by T. Hales
over the period 1994-1998.  
A second version, written in C, was developed by S. Ferguson 
1995-1997.  A third version, written in ML, was developed by S. McLaughlin 2006-2008.  A interval arithmetic package has also been developed
by R. Zumkeller for the theorem proving system COQ, although it has not  been used to give a formal verification of any of the inequalities that arise in the proof of the Dodecahedral conjecture \cite{Zu}.
These implementations are all independent of one another. (Algorithms
were shared among us, but the code was independently implemented.) 
By comparing the proofs of different inequalities in different 
systems, we have developed 
a high degree of confidence that the implementations
of the algorithms are essentially correct.  Of course, it would
be desirable to have a formal correctness proof, but this part
of the Flyspeck Light project has not been completed.  (The ML
implementation and Zumkeller's research are partial steps
in this direction.)

The list of nonlinear inequalities that are used in the
proof of the Dodecahedral conjecture appears at \cite{XX}.
The domains of the functions are subsets of $\ring{R}^m$, for
$m\le 7$.  The complexity of the verification increases
rapidly with $m$.  We have used some of the tricks introduced
in \cite{DCG} to reduce the dimension of the domain wherever
possible.  Dimension reduction is based on established
monotonicity properties of the functions.  (For example, the
volume of a Voronoi cell does not increase when it is intersected
with a half-space.)  Whenever the functions $f_j$ are twice continuously
differentiable, we use a first order Taylor polynomial with
explicit error bounds on the second derivatives. 
The functions $f_j$ represent elementary geometric quantities such
as linear combinations of angles, dihedral angles, solid angles,
and volumes.  Explicit formulas for these functions are known
involving rational functions, the square root, and $\arctan$ functions.
The typical form of a function $f_j$ is a linear combination of terms
of the form
    $$
    \arctan(a/\sqrt{b}),
    $$
where $a,b$ are explicit polynomials on $\ring{R}^m$.
In the 
formula (\ref{eqn:fpos}), the number of disjuncts is usually one
$r=1$, but in some cases there are two disjuncts.

The computer calculations use interval arithmetic to
control for floating-point round-off errors.  Every real number $x$
is represented on the computer as an interval $[a,b]$ containing $x$, where $a$ and $b$ are
exactly representable floating point numbers \cite{Interval}, \cite{Numerics}.
The calculations conform to IEEE-754 standards \cite{Float}.
Approximations to inverse trigonometric functions are based on 
published approximations \cite{Approx}.

XX give references to package locations.




\subsection{nonlinear optimization}

Previous subsections describe the three main pieces of computer
code used in the proof of the Dodecahedral conjecture: graph 
generation, linear programming, and interval-arithmetic inequality
proving.  This subsection describes one additional software
package indirectly involved in the proof: nonlinear-optimization.  
We wish to emphasize that this package was used only in a heuristic way
and does form part of the proof tree.

The disjunction of inequalities in formula (\ref{eqn:fpos}) can
be represented as a constrained minimization problem: show that
the global minimum of $f_1$ on the domain
$$
\{x\in[a_1,b_1]\times\cdots\times[a_m,b_m] \mid  f_2(x)\le 0,\ldots,
  f_r(x)\le 0 \}
$$
is positive. Nonlinear optimization packages have been
used to test the validity of the constrained minimization
problems in our collection.

The nonlinear optimization library {\it cfsqp} has been used to
test all the inequalities in our collection \cite{cfsqp}.  
The algorithm
is based on sequential quadratic programming, which
searches for a local minimum of an objective function
according to a quadratic model of the objective
function.  We generate a large random set $X$ of points in the
domain, run the algorithm for each initial point $x\in X$ to
find a local minimum to the objective function $f_1$.  If $X$
is sufficiently large and sufficiently random, we expect one
of the local minima produced to be the  global minimum.

In practice, this approach works remarkably well on our
collection of problems, largely because the functions $f_j$
tend to be rather bland from the point of view of nonlinear
optimization.  (Typically, the second derivatives of $f_j$
are small; the level surfaces of $f_j$ are approximately planar;
there are no local minima in the interior of the domain;
the global minimum occurs at a corner of the domain; and
every run of the algorithm produces the same local minimum.)
Thus, we can usually determine the true global minimum with
high probability.


If this nonlinear optimization is not part of the proof tree,
what purpose does it serve? First of all, although we have
tried to be careful to avoid any errors in the computer code,
an independent check of the results is certainly welcome.
It makes the proof more robust against possible errors.  
In fact,
this independent check has helped us to spot and correct 
 data entry errors.  Secondly, the package was used to discover
inequalities that were likely to be true, and to discard quickly
inequalities that were false.  The plausibly true inequalities
then became candidates for  rigorous nonlinear optimization
with interval arithmetic.

More recently, 
S. McLaughlin has produced Objective CAML bindings for the
cfsqp library \cite{McLp}.  The nonlinear inequalities in our
collection have also been retested with a second package
Knitro.  This package implements a variety of algorithms
(barrier techniques, conjugate gradient mthods, an active set algorithm).


\subsection{summary}

This section has described various computer programs 
and algorithms that
have been used in the proof of the Dodecahedral conjecture.
For the rest of the paper, we assume that the types of
computations described in this paper can be reliably performed
by computer.

We briefly remind the reader how the three main computer programs
enter into the proof.  A planar graph is associated with
each potential counterexample of the Dodecahedral conjecture.
The properties of this graph is studied, and it is shown to
be a structurally tame graph.  Using the graph generator program,
all such graphs are classified up to isomorphism.  This
reduces the proof to a finite enumeration of cases. Linear programs
are then used to show that each case in this enumeration is 
infeasible.  The nonlinear inequalities appear throughout the proof.
They are used, for example to establish that the graph associated
with a counterexample is a structurally tame graph.  Nonlinear
inequalities are also used to justify the list of inequalities
used in the linear programs. (The linear programming inequalities
come as linear relaxations of nonlinear inequalities.)



\section{Fejes T\'oth's Reduction}\label{sec:12sphere}

L. Fejes T\'oth proved  the Dodecahedral conjecture 
under the extra hypothesis that $\Lambda(v_0,2t_{dod})\setminus\{v_0\}$
has at most $12$  elements.  The proof occupies about
eight pages of the book {\it Regular Figures}.  See the main
theorem of
Section~33 and the main theorem of Section~41 (including
the note on page 265) in \cite{Toth2}.  Fejes T\'oth's
bound is a general bound about the volume of truncated polyhedra.
The polyhedra do not have to be Voronoi cells in a sphere packing.
We sketch a proof of his theorem.

\begin{theorem}  Let $P$ be any polyhedron with at most $12$ sides
that contains a unit sphere $S^2$.  Let $B$ be the ball of
radius $t_0$ concentric with $S^2$.  The volume of the intersection
of $P$ with $B$ is
at least that of a regular
dodecahedron $D$ with inradius $1$.
Equality holds exactly $P$ is congruent to $D$.
\end{theorem}

\begin{proof} (Sketch)  By translation, we may assume that
the origin is the center of $S^2$.
We may consider the case of $k<12$ faces as a degenerate
case of a polyhedron with $12$ faces, where some of the faces
have degenerated to a single point at the vertex.  Similarly,
we may consider the case in which some vertices of the polyhedron
have degree greater than $3$ as degenerate cases of polyhedra 
where all degrees are three, where some of the vertices have
coalesced.   With these conventions, 
we may assume there are $12$ faces, $30$ edges,
and $20$ vertices.

Let 
  $$g(x) = \begin{cases}
          \frac13 \sec^2 (x), & x \le \theta_0\\
          \frac13 \sec^2 (\theta_0), & x \ge \theta_0,
          \end{cases}
   $$
where $\theta_0$ is defined by $\sec(\theta_0) = t_0$.
For each face $F_i$ of the polyhedron $i=1,\ldots, 12$, 
let $w_i$ be the point 
on the unit sphere $S^2$ at the origin closest to the plane through $F_i$.
Let $S_i$ be the radial projection of $F_i$ to a spherical polygon
on $S^2$.   The volume $V$ of the $t_0$-truncated polyhedron satisfies
    $$
    V \ge \sum_{i=1}^{12} \int_{S_i} g(\theta(w_i,x))\,dx
    $$
where $dx$ is the usual measure on $S^2$,
and $\theta(w_i,x)$ is the geodesic length of the arc on $S^2$ joining
$w_i$ to $x$. The integral on the
right is exactly the volume of the truncated polyhedron obtained
by projecting each polygon $S_j$ back out to a plane parallel to $F_i$
through $w_i$.

By the estimate of Sec.~33 of the book, 
the integral on the right is at least
   $$
   120 \int_{T} g(\theta(w,x))\, dx,
   $$
where $T$ is a spherical triangle with angles $\pi/2$,
$\pi/5$, $\pi/3$.  Here $w$ is the vertex of the triangle $T$
that has angle $\pi/5$.  (This estimate 
holds in fact for any non-decreasing function $g$.)

We claim that this integral is precisely the volume of a regular
dodecahedron of inradius $1$.  Since the angle at $w$ is $\pi/5$,
$10$ triangles $T$ form a regular pentagonal face.  By Girard's
formula for the area of a spherical triangle, the area of $T$ is
   $$
   \pi/2 + \pi/5 + \pi/3 - \pi = (4\pi)/(120).
   $$
Thus, $120$ triangles $T$ tile the sphere as $12$ congruent
regular pentagons.
Also, the maximum of $\theta(w,x)$ over the triangle is exactly $\theta_0$.  (This fact is equivalent to the definition of $t_0$ as the circumradius of the regular
dodecahedron.)
So $g(|w-x|) = (\sec^2\theta(w,x))/3$. The interpretation of the integral
as the volume of a regular dodecahedron now follows easily.

L. Fejes T\'oth also considers the case of equality and finds that
the only minimizing polyhedron is the regular dodecahedron.
\end{proof}






\section{Geometry of Voronoi Cells}

In this section, we describe the basic geometry of the
 Voronoi cell and
its truncation.  

\subsection{basic truncation}

Let $\Lambda = \Lambda(0,2t_{dod})$ be a finite
packing containing $0$, and let $\Omega(\Lambda,0)$ be the Voronoi cell.
Let $B(x,r)$ be the open ball of
radius $r$ centered at $x\in\ring{R}^3$.
Let $\Omega_0(\Lambda,0) = \Omega(\Lambda,0)\cap B(0,t_0)$.
Also, $\Omega_{trunc}(\Lambda,0)$ 
has been defined in Section~\ref{sec:form}.


Set $\Lambda^* = \Lambda\setminus\{0\}$.  There is a graph $G(\Lambda)$
with vertex set $\Lambda^*$, whose edges are formed by pairs
$\{v,w\}$ such that $0<|v-w|\le 2t_0$.  

We begin with a discussion of the geometry of $\Omega_0(\Lambda,0)$,
then adapt it to $\Omega_{trunc}(\Lambda,0)$.  The truncated cell
$\Omega_0(\Lambda,0)$ is obtained from the ball $B(0,t_0)$ by
removing a spherical cap
   $$
   C(v) = \{x \in B(0,t_0) \mid  |x - v| \le |x| \}
   $$
for each $v\in\Lambda^*$.
Each spherical cap is bounded by a sphere of radius $t_0$ and
a planar disk formed by the intersection of the bisector of 
$\{0,v\}$ with the ball $B(0,t_0)$.

When $\Lambda^* = \{v\}$ has a single point, the truncated Voronoi
cell is a ball with a single cap removed.  Its volume depends
only on $|v|$.   

XX give the volume formula.

When $\Lambda^*=\{v,w\}$ contains two points, the two spherical
caps meet if and only if $\eta_V(0,v,w) < t_0$.  (Recall that
$\eta_V$ is the circumradius.)   The volume formula for
$\Omega_0(\Lambda,0)$ as a function of $v,w$ is continuous
across the level curve $\eta_v(0,v,w)=t_0$, but not analytic.
When $\eta_v(0,v,w) > t_0$ the caps are disjoint and the
volume is independent of $|v-w|$, depending only on $|v|$ and $|w|$.
That is, the volume does not depend on the location of the caps,
provided they are disjoint.

When $\eta_V(0,v,w) < t_0$, the volume depends on
$|v|,|w|,|v-w|$.  Note that $\eta_V(0,v,w) < t_0$ implies
$|v-w|\le 2t_0$, so that the graph $G(\Lambda^*)$ contains
the edge $\{v,w\}$.
By inclusion-exclusion, the volume of $\Omega_0(\Lambda,0)$ is
   \begin{equation}\label{eqn:in-ex}
   \op{vol}(B(0,t_0)) - \op{vol}(C(v)) - \op{vol}(C(w)) +
 \op{vol}(C(v) \cap C(w)).
   \end{equation}
We partition $C(v)\cap C(w)$ into four regions (called quoins)
Let $P_1$ be the plane through $0,v,w$.  Let $P_2$ be the
plane orthogonal to $P_1$ that passes through $0$ and the circumcenter
of the triangle $\{0,v,w\}$.  Then $\ring{R}^3\setminus (P_1\cup P_2)$ 
contains four connected components, partitioning $C(v)\cap C(w)$
into four quoins.  The volume of a quoin is computed in \cite[7.3]{DCG}.
This gives the following volume formula.

XX give the volume formula.

When $\Lambda^*=\{v_1,v_2,v_3\}$ contains three points, the three spherical
caps meet if and only if the circumradius of the simplex
$\{v_1,v_2,v_3\}$ is less than $t_0$.  When this happens, each
edge $|v_i-v_j|\le 2t_0$, so that the graph $G(\Lambda)$ is a
triangle.  Also, the circumradius of each face is at most
$t_0 < \sqrt2$.  Thus, 
  $$\{0,v_1,v_2,v_3\}\in\CalQ_{dod}(\Lambda,0).$$
These are the sets singled out for special truncation in $\Omega_{trunc}$.
We return to this in a moment.

If the three spherical caps do not meet or meet in at most pairs,
then the volume, by inclusion-exclusion is given by a formula
similar to Formula~\ref{eqn:in-ex}.  In particular, the volume
is given in terms of quoins, and so forth.

The case when the graph $G(\Lambda)$ is connected, but not a triangle
has particular interest.  Suppose that there is no edge between
$v_1$ and $v_3$. In this case, the volume of $\Omega_0$ depends on
$|v_i|$, $|v_1-v_2|$ and $|v_2-v_3|$, but not on $|v_1-v_3|$.
Thus, the point $v_3$ can be moved, subject to the constraints
fixing $|v_3|$ and $|v_2-v_3|$ without changing the volume.
In particular, $v_3$ can be moved until $|v_3-v_1|=2t_0$.  Thus,
we obtain the following simple lemma.  

\begin{lemma}\label{lemma:3tri}  
Suppose $G(\Lambda)$ contains three vertices,
is connected, but not a triangle.  Then there is another packing
$\Lambda_1$ whose graph $G(\Lambda_1)$ is a triangle and such
that
  $$
  \op{vol}(\Omega_0(\Lambda,0)) = \op{vol}(\Omega_0(\Lambda_1,0)).
  $$
\end{lemma}

The same lemma holds
for $\Omega_{trunc}$ because in the situation at hand $\Omega_0=\Omega_{trunc}$.

We check in a moment that it is not possible for four or more
spherical caps to meet.

\subsection{planarity}


\begin{lemma}  For each edge $e=\{v,w\}$ of a graph $G(\Lambda)$,
let $A_e$ be the arc on the unit sphere at the origin formed
by the intersection of the sphere with $\op{aff}^0_+(0,\{v,w\})$
Then the sets $A_e$ do not meet one another.  In particular,
$G(\Lambda)$ is a planar graph.
\end{lemma}

\begin{proof}  Equivalently, we may show that the sets
$\op{aff}^0_+(0,\{v,w\})$ do not meet one another, as $e=\{v,w\}$
runs over edges.  This is a statement in Tarski arithmetic.
A detailed proof appears in \cite[Lemma~3.2]{arx}.  We do not
repeat the proof, because it is the $[t_0,t_{dod}]$-translation
of the published theorem \cite[Lemma~3.10]{Part1}, hence
repetitive.
\end{proof}

If $G'$ is any graph on the edge set $\Lambda^*$, we will call
it geometrically planar if the sets $A_e$ do not meet one
another, as $e$ runs over the edges of $G'$.  A geometrically
planar graph is clearly a planar graph.

\subsection{triangles}

This subsection describes the geometry associated with
triangles in the graph $G(\Lambda)$.  For each triangle,
there is a set $\{0,v_1,v_2,v_3\}$ such that
\begin{equation}\label{eqn:qrtet}
  |v_i| \le 2t_0,\quad |v_i-v_j | \le 2 t_0,\quad i,j\le 3.
\end{equation}

\begin{lemma} Let $S=\{0,v_1,v_2,v_3\}$ be a set of four points
that satisfies (\ref{eqn:qrtet}).  Then
there does not exist $w\in\op{conv}(S)$
(the convex hull of $S$) that satisfies $|w-u|\ge 2$ for 
$u\in S$.
\end{lemma}

\begin{proof}  This is a statement in Tarski arithmetic 
that can be expressed with $12$ quantifiers ($3$ coordinates
for each of $v_1,v_2,v_3,w$).  We do not repeat the proof
because it is the $[t_0,t_{dod}]$-translation of \cite[Lemma~4.15]{DCG}.
It is written out in full in \cite[Lemma~3.3]{arx}.
\end{proof}

\begin{lemma}\label{lemma:enclosed} 
Let $S=\{0,v_1,v_2,v_3\}$ be a set of four points
that satisfies (\ref{eqn:qrtet}).  Then
there does not exist $w\in\op{aff}_+(0,\{v_1,v_2,v_3\})$
(the cone with apex $0$ 
generated by positive linear combinations of $v_i$) 
that satisfies $|w-u|\ge 2$ for 
$u\in S$ and $|w|\le 2t_0$.
\end{lemma}

\begin{proof} This is a statement in Tarski arithmetic.
The case when $w\in\op{conv}(S)$ is covered by the previous lemma.
The remaining case is a $[t_0,t_{dod}]$-translation of
\cite[Lemma~4.19]{DCG}.  See also,  \cite[Cor~3.7]{arx}.
\end{proof}

\begin{lemma}  Let $0,v_1,v_2,v_3,v_4\in\Lambda$, with
$v_i$ distinct.  The intersection $\cap_{i=1}^4 C(v_i)$
of spherical caps is empty.
\end{lemma}

\begin{proof}  A nonempty intersection implies that each
edge $\{v_i,v_j\}$ in  $G(\Lambda)$, forming a complete graph
on four vertices.  The graph is planar by Lemma~\ref{XX}.
Its planar representation is a triangle with one vertex inside,
connected to all three vertices of the triangle.
Geometrically, this corresponds to a point $v_4\in\op{aff}_+(0,\{v_1,v_2,v_3\})$, which is impossible by the previous lemma.
\end{proof}

\begin{lemma}  Let $\{v_1,v_2,v_3\}$ and $\{v_1',v_2',v_3'\}$
be two distinct triangles in $G(\Lambda)$.  Then
$\op{aff}_+^0(0,\{v_1,v_2,v_3\})$ is disjoint from
$\op{aff}_+^0(0,\{v_1',v_2',v_3'\})$.  In particular,
$\op{cone}(\{0,v_1,v_2,v_3\})$ is disjoint from
$\op{cone}(\{0,v_1',v_2',v_3'\})$.
\end{lemma}

\begin{proof} Since the graph is planar, if the two cones
intersect, then one triangle must be contained in the other
triangle.  That is, one cone is contained in the other.  This
leads to a vertex $w$ of one triangle in the other
cone, $\op{aff}_+(0,\{v_1,v_2,v_3\})$.  This is prohibited
by Lemma~\ref{lemma:enclosed}.
\end{proof}

As mentioned earlier, a three-fold intersection of spherical
caps produces a triangle in the graph $G(\Lambda)$ and a 
set 
$$\{0,v_1,v_2,v_3\}\in \CalQ_{dod}(\Lambda,0).$$
We investigate the geometry of such $\{0,v_1,v_2,v_3\}$.

\begin{lemma}\label{lemma:Q}
If $S=\{0,v_1,v_2,v_3\}$ is a set of four points such that each
face has circumradius at most $\sqrt2$ and such that
$|u-v|\le 2t_0$ for $u,v\in S$, then
the circumcenter of $S$ lies in $\op{conv}(S)$.  Also, if $x
\in \op{conv}(S)$ and $w$ has distance at least $2$ from each
point of $S$, then $x$ is at least as close to some point of $S$ than to
$w$.
\end{lemma}

\begin{proof} This is a statement in Tarski arithmetic.
The statement about the circumcenter is \cite[Lemma~5.18]{DCG}.
(No translation is needed.)
If $x$  comes closer to  to $w$  than to each point of $S$, then
the Voronoi cell $\Omega(S\cup\{w\},w)$ meets $\op{conv}(S)$.
Again by \cite[Lemma~5.18]{DCG}, this implies that the circumradius
of some face of $S$ is greater than $\sqrt2$, which is contrary
to hypothesis. An alternative proof 
of both parts of the lemma
is contained in \cite[Lemma~3.5,3.6]{arx}.
\end{proof}

The following lemma justifies limiting packings to those
satisfying $\Lambda=\Lambda(0,2t_0)$.  Extending the packing
$\Lambda$ beyond radius $2t_0$ cannot decrease the volume of the
truncated Voronoi cell.

\begin{lemma}\label{lemma:trunc}  
Let $\Lambda=\Lambda(0,2t_0)$.  Let $\Lambda\subset\Lambda'$,
where $\Lambda'(0,2t_0) = \Lambda(0,2t_0)$.  Then
$$\Omega_{trunc}(\Lambda',0) = \Omega_{trunc}(\Lambda,0).
$$
That is, the truncated Voronoi cell cannot be decreased in volume by
adding additional points to the packing outside the ball $B(0,2t_0)$.
\end{lemma}

\begin{proof} Let $w\in \Lambda'\setminus\Lambda$, with $|w|>2t_0$.
Every point in $B(0,t_0)\cap \Omega_{trunc}(\Lambda,0)$ 
is clearly closer to $0$ than to $w$ so
belongs also to $\Omega_{trunc}(\Lambda',0)$.

Assume that $x\not\in B(0,t_0)$ and $x\in \Omega_{trunc}(\Lambda,0)$.
Then by the definition of $\Omega_{trunc}$,
there is some $S=\{0,v_1,v_2,v_3\}\in \CalQ_{dod}(\Lambda,0)$ such that
$x\in \Omega(\Lambda,0)\cap \op{conv}(S)$.  By Lemma~\ref{lemma:Q},
$x$ is closer to $0$ than to $w$.  Thus, $x\in\Omega(\Lambda',0)$.
The result follows.
\end{proof}

From Lemma~\ref{lemma:Q}, it follows that
\begin{equation}\label{eqn:omega-3}
\op{aff}^0_+(0,\{v_1,v_2,v_3\}) \cap \Omega(\Lambda,0) = 
\op{aff}^0_+(0,\{v_1,v_2,v_3\}) \cap \Omega_{trunc}(\Lambda,0) = 
\op{conv}(S)\cap\Omega(S,0).
\end{equation}
That is, the calculation of volume can be made locally in $S$ without
reference to the position of the packing $\Lambda$.
A formula for this volume calculation appears in 
\cite[sec.~8.6.3]{Part1}.  (This gives the formula in the case
when the circumradius is at most $t_{dod}$.  When the circumradius
is at least $t_{dod}$, the spherical caps intersect in pairs,
and the inclusion-exclusion formula (using quoins)
can be used.

In summary, we can use the inclusion-exclusion formula for all
calculations of truncated Voronoi cells $\Omega_{trunc}(\Lambda,0)$,
except for $\{0,v_1,v_2,v_3\}\in \CalQ_{dod}(\Lambda,0)$ with
a circumradius less than $t_0$.  In this case, we use
the explicit formula just mentioned.

\subsection{connecting the graph}

We have seen that the volume formula for $\Omega_{trunc}(\Lambda,0)$
depends only on $|v|$, for $v\in\Lambda^*$ and on
$|v-w|$ for $\{v,w\}$ and edge of $G(\Lambda)$.
In particular, if the graph $G(\Lambda)$ is not connected, the
collections spherical caps for two different connected components
of the graph do not intersect one another.  Thus they form
two or more non-interacting ``islands'' of spherical caps that
can be moved independently around the globe $B(0,t_0)$
without changing the volume of
the truncated Voronoi cell.  In particular, one island of spherical
caps can be moved rigidly until some vertex $v$ of on connected
component of the graph has distance exactly $2t_0$ from some
vertex in another component.  This connects two components of the
graph without changing volume.  Thus, every truncated Voronoi cell
has the same volume of another with a connected graph.  We
assume without loss of generality that the graph is connected.

Now consider a graph $G(\Lambda)$ that is not two-connected.
Then there exists some $w\in \Lambda^*$ that disconnects
the graph.  Write the vertex set $\Lambda^*$
as a disjoint union of three sets
$\{w\}$, $\Lambda_1$, $\Lambda_2$ such that there is no edge between
$\Lambda_1$ and $\Lambda_2$.  The volume of $\Omega_{trunc}(\Lambda,0)$
is independent of the distances between points of $\Lambda_1$ and
$\Lambda_2$.  This means that we can move $\Lambda_2$ rigidly,
while constrained to preserve $|v|$ for $v\in\Lambda_2$, $|v-u|$
for $u,v\in \Lambda_2\cup\{w\}$. 
We may continue the rigid motion of $\Lambda_2$ until some distance
$|u-v|$ decreases to $2t_0$, for
$u\in\Lambda_1$ and $v\in\Lambda_2$. (See Lemma~\ref{lemma:3tri}.) 
We may repeat this construction until $G(\Lambda)$ becomes two-connected.  The process does not alter the volume of $\Omega_{trunc}(\Lambda,0)$.
We now assume without loss of generality that $G(\Lambda)$ is
two-connected.



\subsection{standard components}

In this subsection, we continue to assume 
the running list of assumptions on $\Lambda$.  We have
$0\in\Lambda = \Lambda(0,t_0)$, $\Lambda^*$ has cardinality
at least $13$, and $G(\Lambda)$ is a connected,
two-connected planar graph.

Under these assumptions, each face of $G(\Lambda)$ is a simple
polygon.  In particular, there are no vertices of degree one 
in the graph.

For the moment, we generalize the situation somewhat and
allow $G'$ to be any graph on the vertex set $\Lambda$ that
is planar, connected,  two-connected, and contains $G(\Lambda)$
as a subgraph.

For each edge $\{u,v\}$ of $G'$ form the cone
$\op{aff}^0(0,\{u,v\})$.  Let $X=X(G')$ be the union of these
cones and let $Y(G')$ be the complement of $X$ in $\ring{R}^3$.
The open set $Y(G')$ breaks into a finite set of connected
components. 
We write $[Y(G')]$ for this set of standard
components.  The set $[Y(G')]$ is in natural bijection with
the set of faces of the graph $G'$.  If $F$ is a face
of the graph $G'$, we write $U_F$ for the corresponding
standard component, and say that $U$ is indexed by $F$.

We can identify each face $F$ with a sequence of vertices
$(v_1,\ldots,v_r)$ with $v_i\in G'$, giving the cyclic
order of the vertices around the face.  The sequence is well-
defined up to cyclic permutation, so that $(v_1,\ldots,v_r)$
defines the same face as $(v_r,v_1,\ldots)$.  We pick the
order of the cycle counterclockwise around each face.  
%In other words, if $w\in U_F\cap S^2$, the winding number
%of the path $(v_1/|v_1|,\ldots,v_r/|v_r|)$ in $S^2$ 
%(oriented with outward
%facing normal) around
%$w$ should be one. 
 (Figure XX.)

Each connected component $U$ has a solid angle $\sol(U)$, which
is defined to be the area of $U\cap S^2$.  The sum of the
solid angles is the area of $S^2$:
 $$
 \sum_{U\in[Y(G')]} \sol(U) = 4\pi.
 $$

If $U=U_F$ is a connected component and $v$ a vertex of $F$, then there
is an {\it azimuth angle} assigned to $(U,v)$ with the property
that if $F_1,\ldots,F_k$ are all the faces of $G(\Lambda)$
that contain $v$, the sum of the azimuth angles around
$v$ is $2\pi$:
  $$
  \sum_{i=1}^k \op{azim}(U_{F_i},v) = 2\pi.
  $$
Equivalently, 
the azimuth angle equals the interior angle of the spherical
polygon $U\cap S^2$ at $v/|v|$.  By Girard's formula for
the area of a triangle or polygon, we have for
$F = (v_1,\ldots,v_r)$:
  $$
  \sol(U_F) + (r-2)\pi = \sum_{i=1}^k \op{azim}(U_F,v_i)
  $$
When $\op{azim}(U_F,v) \ge \pi$, we say that $v$ is a concave
vertex of $F$.  Otherwise, we say it is convex.

Now specialize again to the situation where $G'=G(\Lambda)$.
In this case we write $X(\Lambda)=X(G(\Lambda))$, $Y(\Lambda)=Y(G(\Lambda))$,
and so forth.
 We call an connected component of $Y(\Lambda)$
a {\it standard component.}  (This term is a
$[t_0,t_{dod}]$-translation of a term by the same name
in the proof of the Kepler conjecture.)

If $U$ is a standard component, set
  $$\omega(\Lambda,U) = \op{vol}(U\cap \Omega_{trunc}(\Lambda,0)).$$
We then have
  $$\op{vol}(\Omega_{trunc}(\Lambda,0)) = 
  \sum_{U\in[Y(\Lambda)]} \omega(\Lambda,U).$$
Write $\omega(\Lambda)$ for the left-hand side of this equation.

If $U$ is indexed by a triangle $F=\{v_1,v_2,v_3\}$ in the graph,
then $U = \op{aff}_+^0(0,\{v_1,v_2,v_3\})$.  In this case,
$\omega(\Lambda,U)$ is precisely the volume of the region
already considered in Equation~\ref{eqn:omega-3}.

If $U$ is not indexed by a triangle, then
   $$\omega(\Lambda,U) = \op{vol}(U\cap\Omega_0(\Lambda,0)).$$
The formula for $\omega(\Lambda,U)$ in this case follows
by inclusion-exclusion as in Section~\ref{XX}.  Suppose
that $F$ is a face of $G(\Lambda)$ whose vertices are given
by $(v_1,\ldots,v_r)$ (listed consecutively around the face).
We have (XX give formula and give citation for justification)
$$
\omega(\Lambda,U_F) 
$$

Set $M=0.42755$ and let
 $$\mu(\Lambda,U)= \omega(\Lambda,U) - M \sol(U).$$
We have
$$
\sum_{U\in[Y(\Lambda)]} \mu(\Lambda,U) = \omega(\Lambda) -  4\pi M.
$$
Write $\mu(\Lambda)= \omega(\Lambda)- 4\pi M$ for the right-hand side of this equation.  The inequality that we are trying to prove is then
that $\mu(\Lambda) > \mu(\Lambda_{dod})$.  
We have 
$$
  \mu(\Lambda_{dod}) \approx 0.177540.
$$
We call the number on the left the {\it squander target}.  (This term
is also used in the proof of the Kepler conjecture with
analogous but different meaning.  See Remark~\ref{rem:sq}.)  We show
later that $\mu(\Lambda,U)$ is always positive.  The constant
$M$ is chosen so that the minimum of $\mu(\Lambda,U)$ -- as
both $\Lambda$ and $U$  vary -- is very close to zero (about $10^{-7}$).
The function $\mu$ tends to have better numerical behavior
than $\omega$.  For that reason, even though the two
functions carry essentially the same information,
we express estimates in terms
of $\mu$ rather than $\omega$, whenever possible.

\begin{remark}\label{rem:sq} The function $\mu$ is closely related to a function
$\tau(\cdot,t)$ that is used in the proof of the Kepler conjecture.  The
function $\mu$ is, up to a small error term, a positive multiple
of $\tau(\cdot,t_{dod})$.   The small
error term comes from the fact that in this article, the constant
$M$ is used, and in \cite{DCG} the constant
$M_0=1/(3 \delta_{tet})$ is used, where $\delta_{tet} = \sqrt8 \arctan(\sqrt2/5)$.  (The constant $\delta_{tet}\approx 0.7796$ is Rogers's famous bound on the density of sphere packings.)  The difference is small:
   $$M_0 - M \approx 1.86 \times 10^{-7}.$$
Because of the close similarity between $\mu$ and $\tau(\cdot,t_0)$,
for every estimate involving $\tau(\cdot,t_0)$ there is apt to
be an analogous estimate involving $\mu$.  The translation involves
replacing $M_0$ with $M$, $t_0$ with $t_{dod}$ and rescaling the
resulting function by an explicit positive scalar to get $\mu$.
One other difference is that the squander targets are entirely different
in the two problems (because the Dodecahedral conjecture and the
Kepler conjecture lead to entirely different bounds on the density
of packings).  In general, weaker estimates are sufficient in the
case of the Dodecahedral conjecture.
To draw out the parallels between the two articles, we will call
an estimate in this article a $\mu$-$\tau$ analogy of an estimate
in \cite{DCG} if this translation between functions applies.
\end{remark}



\section{The Main Estimate}

In this section, we prove the main estimate, which gives a lower
bound on the function $\mu(\Lambda,U)$ for any standard component $U$.
The running
assumptions on $\Lambda$ remain in effect: $0\in\Lambda= \Lambda(0,2t_0)$, $G(\Lambda)$ is connected and two-connected, and the cardinality
of $\Lambda^*$ is at least $13$.

\begin{theorem}\label{thm:main}  
Let $\Lambda$ be a finite packing satisfying the
running assumptions.  Let $U_F$ be a standard component indexed by
a face $F$ of $G(\Lambda)$.  Suppose that the polygon 
$F$ has $n$ vertices.  Then
   $\mu(\Lambda,U_F) > t_n$, where 
$$
\begin{array}{lll}
 t_3 &= 0\\
 t_4 &= 0.031\\
 t_5 &= 0.076\\
 t_6 &= 0.121\\
 t_7 &= 0.166\\
 t_n &= \mu(\Lambda_{dod}),\quad n\ge 8.
\end{array}
$$
\end{theorem}

The proof of this theorem is long and technical.  The proof extends for 
twenty pages in  \cite[pp.19-38]{arx}.  The analogous estimate in
the proof of the Kepler conjecture takes a full thirty pages 
\cite[pp.126-156]{DCG}.
We cannot pretend to give justice to the proof under the page constraints
imposed on this article by the editors.  We refer the reader to the
two articles just cited for full details of the proof.  In this article,
we give a general summary of the ideas of the proof, with 
references for the reader who wishes to pursue the proof in greater detail.

\subsection{verifications in low dimension}

The first two cases $n=3,4$ of the theorem can be handled directly with
interval arithmetic, because they are explicit nonlinear 
inequalities involving a small number of variables.  The case $n=3$ can
be expressed as a nonlinear optimization problem over a tetrahedron
whose edge lengths vary in length between $2$ and $2t_0$.  In other
words, it is a minimization problem on the six-dimensional 
domain $[2,2t_0]^6$. This is readily treated by interval arithmetic (XX reference).  

The case $n=4$ can also be directly proved with interval arithmetic.  
Here the optimization runs over a nine-dimensional domain.  The
quadrilateral face $F=(v_1,v_2,v_3,v_4)$ is parameterized up to rigid
motion by the nine variables (3 coordinates for each of four points
minus the 3 dimensional group of rotations).  Monotonicity arguments
reduce the configuration to a seven-dimensional domain.  (Two of the
points $v_i$ can be rescaled $v_i \mapsto \lambda v_i$ with $0 < \lambda \le 1$ until a constraint is met, 
because parallel shifts in faces of a Voronoi cell towards the origin are decreasing in volume.)  The inequality $\mu(\Lambda,U)> t_4$ on
a seven-dimensional domain can be proved directly by interval arithmetic.

\subsection{strategy: superadditivity}

% XX There are problems with the constants in version2.
% For instance, p22 says D(3,2)=0.181, but page 20 gives D(3,2)=0.0496.
% page22, (n,k)=(3,2) is incompatible with page 20. I am reverting to
% the treatment in version 1.
Define constants $D_{dod}(3,1) = 0.0161$ and $D_{dod}(n,k) = t_{n+k} - D_{dod}(3,1)k$,
for $n\ge 3$, $0\le k\le n$ and $n+k\ge 5$.  With these definitions,
when $n_1,n_2\ge 3$, $0\le k_1\le n_1$, $0\le k_2\le n_2$ and $n_1+k_1,n_2+k_2\ge 5$, we have the following supperadditivity:
\begin{equation}\label{eqn:super}
  D_{dod}(n_1,k_1) + D_{dod}(n_2,k_2) \ge D_{dod}(n_1+n_2-2,k_1+k_2-2).
\end{equation}
In fact, this follows immediately from the definitions and the
easily verified inequality,
for $m,n\ge 5$,
$$
t_m + t_n \ge t_{m+n-4} + 2 D_{dod}(3,1).
$$
(Note that there are only finitely many cases involved in the
verification of this identity,
because $t_n$ is constant for $n\ge 8$.)
One of the basic strategies of the proof is to give a partial triangulation
of the face $F$ (with $n$ sides)
into smaller polygons $F_1,\ldots,F_r$.  The polygon
$F_i$ will have $n_i$ sides.  We let $k_i$ be the number of edges
of $F_i$ that are not one of the original edges of $F$.  We define
a decomposition of $U_F$ into smaller regions $U_i$ corresponding
to each $F_i$, give a bound $\mu(\Lambda,U_i) > D_{dod}(n_i,k_i)$
and use superadditivity (\ref{eqn:super}) to prove the identities:
\begin{equation}\label{eqn:mu}
  \mu(\Lambda,U_F) = \sum_{i=1}^r \mu(\Lambda,U_i).
\end{equation}
\begin{equation}\label{eqn:super-mu}
\sum_{i=1}^r \mu(\Lambda,U_i) > \sum_{i=1}^4 D_{dod}(n_i,k_i)
 > D_{dod}(n,0) = t_n.
\end{equation}

The idea is that the objects $U_i$ are lower-dimensional objects
than $U_F$ (that is, the polygons  have fewer edges).
The dimension of $U_i$ controls the complexity
of the estimates.  Thus a series of inequalities $\mu(\Lambda,U_i) > D_{dod}(n_i,k_i)$
is expected to be easier to prove than a single inequality
$\mu(\Lambda,U_F) > t_n$ in higher dimension.  

On the other hand, if the edges of the polygons $F_i$ are allowed
to get too long, numerical experiments show that function $\mu(\Lambda,U_i)$
tends to become numerically unstable.  This prevents us from
becoming overly aggresive in triangulating $F$.
These experiments have led
us to restrict the edge lengths to be at most $3.2$.

We say that a triple $(u,v,w)$ is {\it unstable}
if $u,v,w$ are distinct vertices of $\Lambda^*$ such
that $|u-v|\le 2t_0$, $|v-w|\le 2t_0$, $|u-w|>\sqrt8$.
We say $\{u,v\}$ is unstable if there exists $w$ such that $(u,w,v)$
is unstable.  Otherwise it is said to be stable.
We also avoid unstable edges $\{u,w\}$.
because of the numerical instabilities they create.


\subsection{construction of subcomponents}

Let $F$ be a face of the graph $G(\Lambda)$ and let $U_F$ be the
corresponding standard component.  We represent $F$ as a cycle
$(v_1,\ldots,v_n)$ with $v_i\in\Lambda^*$.  The function
$\mu(\Lambda,U_F)$ depends only on $\Lambda^*$ through
$v_1,\ldots,v_n$.  Thus, for the purpose of the proof of
Theorem~\ref{thm:main}, we may assume without loss of generality
that $\Lambda^* = \{v_1,\ldots,v_n\}$.

We say that $u\in\Lambda^*$ is visible from $v\in\Lambda^*$ if
$\{0,u,v\}$ is not a collinear set and if
$\op{aff}_+^0(0,\{u,v\})\subset U_F$.  When this occurs, we
also call $\{u,v\}$ {\it internal}.  When $\{u,v\}$
is  internal, if the edge $\{u,v\}$ is added to the
graph $G(\Lambda)$, the graph continues to be (geometrically) planar.



%% XX Two stages are not needed in Dodec. Right?
We add edges to the graph $G(\Lambda)$ 
%in two stages 
as follows.  
%In
%the first stage,
%we add as many internal  $\{u,v\}$ as possible, subject
%to the constraint that $|u-v|\le\sqrt8$ and subject
%to the non-crossing condition:
%  \begin{equation}\label{eqn:nonc}
%  \op{aff}_+^0(0,\{u,v\}) \cap \op{aff}_+^0(0,\{u',v'\}) = \emptyset,
%  \end{equation}
%for any two edges $\{u,v\}$ and $\{u',v'\}$ that are added to the
%graph.  To describe the second stage, we need a definition.
%We say that two distinct vertices $u,v$ of $F$ are
%{\it subadjacent} if they are both adjacent to the same vertex $w$.
%In other words two vertices are subadjacent if adding the edge
%$\{u,v\}$ to the graph $G(\Lambda)$ forms a triangle.
%For example, $u=v_i,v=v_{i+2}$ are subadjacent for $i+2\le n$.
%In the second stage, 
%we 
Add as many internal edges $\{u,v\}$ as possible to $G(\Lambda)$, subject
to the following conditions:
\begin{enumerate}
\item  $|u-v|\le3.2$, 
\item The edges do no cross:
  \begin{equation}\label{eqn:nonc}
  \op{aff}_+^0(0,\{u,v\}) \cap \op{aff}_+^0(0,\{u',v'\}) = \emptyset,
  \end{equation}
for any two edges $\{u,v\}$ $\{u',v'\}$ added to the graph.
% including non-crossing
%with the edges added in the first stage,
\item There $\{u,v\}$ is stable.
\end{enumerate}  

We let $G'(\Lambda)$ be the graph on vertex set $\Lambda^*=\{v_1,\ldots,v_n\}$ obtained by adding these additional edges.  By the non-crossing
conditions, $G'(\Lambda)$ is a geometrically planar graph.  
%By \cite[Lemma~4.15]{arx}
%(which is a $[t_0,t_{dod}]$-translation of Lemma~\ref{}), every
%internal edge $\{u,v\}$ of $U_F$ such that $|u-v|\le$
By the Jordan curve theorem for polygons, $Y(\Lambda)$ has two connected 
components,
$U_F$ and the complementary region $U_{F'}$.
We consider the set of connected components of $Y(G')$.  Each
component $U'\subset Y(G')$ is either a subset of $U_F$ or $U_{F'}$.
Since all the added edges are internal to $U_F$, there is exactly
one component of $Y(G')$ that lies in $U_{F'}$, and that component
is equal to $U_{F'}$.  Write $[Y(G')]^* = [Y(G')]\setminus\{U_{F'}\}$
for the set of components of $Y(G')$ internal to $U_F$.

We may enumerate them $U_1,\ldots,U_r$.  We define
$\mu(\Lambda,U_i)$ to be given by Formula~XX.  Then
Equation~\ref{eqn:mu} holds.  Let $n_i$ be the number of edges of the face
$F_i$ of $G'$ corresponding to $U_i$.  Let $k_i$ be the number of edges
$\{u,v\}$
of $F_i$ such that $|u-v|> 2t_0$. (These are edges of $G'$ that do not
belong to $G(\Lambda)$.)  

XX describe geometry. Show decomposition is geometric.

The function $\mu(\Lambda,U_i)$ depends on $\Lambda$ only through
the vertices of $\Lambda$ on $F_i$.  Thus, for purposes of estimating
$\mu(\Lambda,U_i)$ for fixed $i$, we may assume that $\Lambda$
is equal to the set of vertices of $F_i$.  This allows us to express
estimates locally.  We formulate this as a definition.

\begin{definition}
Let $\Lambda$ be a packing such that $0\in\Lambda=\Lambda(0,2t_0)$.  
Let $G$ is a graph on vertex set $\Lambda^*$ 
consisting
of a single cycle containing $n\ge 3$ vertices.  
Suppose that $G$ is geometrically planar. 
Let $U\in [Y(G)]$ be a connected component of $Y(G)$.  
We call $(\Lambda,G,U)$ a {\it local configuration} if the
following conditions hold:
\begin{itemize}
\item Every edge $\{u,v\}$ of $G'$ satisfies $|u-v|\le 3.2$.
\item If
$\{u,v\}$ is internal in  $U$, then $|u-v|\ge \sqrt2$.
\item If $\{u,v\}$ is internal in $U$ and stable,
then $|u-v|\ge 3.2$.


\end{itemize}
\end{definition}

%XX Need a hypothesis that keeps the region from being exchanged with
%its complement on triangles, quadrilaterals, etc. ?? I don't think it is needed.

To each local configuration, we can attach two numbers $(n,k)$.
Let $n=n(\Lambda,G,U)$ be the cardinality of $\Lambda^*$.  Equivalently,
$n$ is the number of edges in the graph $G$.  Let
$k=k(\Lambda,G,U)$ be the number of edges $\{u,v\}$ of $G$ such
that $|u-v|> 2t_0$. We have $k\le n$.

The main estimate (Theorem~\ref{thm:main}) now follows from the following
refined version of the estimate and superadditivity.

\begin{theorem}\label{thm:main'}  
Let $(\Lambda,G,U)$ be any local configuration.
Let $n=n(\Lambda,G,U)$ and $k=k(\Lambda,G,U)$ be the corresponding constants.
Assume that $n+k\ge 4$.  Then
   $$
   \mu(\Lambda,U) > D_{dod}(n,k).
   $$
\end{theorem}

\subsection{deformations}

Let $(\Lambda,G,U)$ be a local configuration.  We say that $v\in\Lambda^*$
is concave in $U$, if the azimuth angle satisfies
$\op{azim}(U,v)\ge\pi$.  Otherwise,
$v$ is convex in $U$.  
We say that $U$ is convex if every $v\in\Lambda^*$ is convex in
$U$.

The proof of Theorem~\ref{thm:main'} is a total induction argument
on  the cardinality $n$ of $\Lambda^*$.  We can arrange for
the base case of the induction to be vacuous by making the
base case $n=2$. By definition, if $(\Lambda,G,U)$
is a local configuration, then $n\ge 3$, so the theorem is
vacuously true in this case.  Now take $n\ge 3$. We may now assume that
Theorem~\ref{thm:main'} holds for any local configuration with
cardinality less than $n$.  

The strategy of the proof is to deform the local configuration $(\Lambda,G,U)$ by moving a single vertex $v\in\Lambda^*$ at a time in a way
that preserves the constraint of being a local configuration,
preserves $n$, 
is non-increasing in $\mu(\Lambda,U)$, 
and is non-decreasing in $D_{dod}(n,k)$.
Under these conditions, any counterexample to the theorem propagates 
to a new counterexample under the deformation.
Note that it is easily checked that $D(n,k) \le D(n,k+1)$, for all
$n\ge 3$ and all $n\ge k\ge0$.  Thus, the condition that $D_{dod}(n,k)$
is non-decreasing can be replaced with the constraint that $k$
is non-decreasing under deformation.

Recall for $v\in\Lambda^*$, we have defined the azimuth angle
$\op{azim}(U,v)$ of $v$ in $U$.  We say that $v$ is concave in $U$
if $\op{azim}(U,v)\ge\pi$.  Otherwise, we say that $v$ is convex
in $U$.  If every vertex $v$ is convex in $U$, then we say that $U$
is convex.  When $U$ is convex in this sense, it is known that $U$
is contained in some open half-space whose bounding plane
contains the origin.  Moreover,
if $U$ is convex,
it is known that $u$ is visible from $v$ in $U$
for any two vertices $u,v\in\Lambda^*$.  Note that when $U$ is 
convex, the conditions on local configurations require that
$|u-v|\ge\sqrt8$, for any two non-adjacent vertices $u,v\in\Lambda^*$,
because $\{u,v\}$ is automatically internal.

\subsection{deformation types}

We consider several types of continuous 
deformations of a local configuration
$(\Lambda,G,U)$.  In each case a single vertex $v\in\Lambda^*$ is
moved at a time.

\subsubsection{type 1}  
Let $v$ be concave in $U$.  Let $u,w$ be the two vertices of
$\Lambda^*$ adjacent to $v$ in $G$.  Suppose that  
Suppose that 
and 

We require the deformation to stop if any of the following conditions
are met.
\begin{enumerate}
\item $|u-v|=3.2$.  
\item For some $v'\in\Lambda^*\setminus\{u,v,w\}$, 
$\{v,v'\}$ is internal and $|v-v'|\le \sqrt2$.
\item For some $v'\in\Lambda^*\setminus\{u,v,w\}$,
$\{v,v'\}$ is internal, stable, and $|v-v'|\le 3.2$.
\end{enumerate}


\subsubsection{type 2}

\subsubsection{type 3}

\subsubsection{type 4}






For each of the different deformation types, we consider 
how a continuous deformation of a single vertex $v\in\Lambda^*$
affects each of the conditions
defining a local configuration $(\Lambda,G,U)$  (Definition~\ref{XX}). 

We only consider deformations that maintain the constraint $|v|\le 2t_0$,
so that the constraint $0\in\Lambda=\Lambda(0,2t_0)$ is preserved.
For $\Lambda$ to remain a packing, we require the condition
 $|u-v|\ge 2$, for $u,v\in \Lambda$.
We only consider deformations that are non-decreasing in $|u|$ so that
the packing constraint $|u|\ge 2$ is preserved.  The preservation of the
condition $|u-v|\ge 2$, for $u,v\in\Lambda^*$ is more subtle. 
The proof that this constraint is maintained for the deformations
considered  takes  a couple of pages.  It appears as \cite[Lemma~7.6]{arx}.
It is the $[t_0,t_{dod}]$-translation of \cite[Lemma~12.20]{DCG}.

The next constraint is that the deformation should preserve the condition
that $G$ is geometrically planar.

The cardinality $n$ of the vertex set $\Lambda^*$ is preserved.
The set of edges of the graph of $G$ is combinatorially determined and
remains fixed under deformation.  In particular, $G$ remains a single
cycle.  
%The proof of Theorem~\ref{thm:main'} breaks into two separate cases
%depending on whether $U$ is convex.  We treat the convex case first.
%The proof in the nonconvex case is an inductive argument that
%assumes the validity of the theorem
%in the convex case.

%The convex case is further divided into subcases according to
%the cardinality $n$ of $\Lambda^*$.  








\section{Classification of Tame Hypermaps}

\subsection{hypermap}

A hypermap is a tuple $(D,n,e,f)$, where $D$ is a finite
set, and $n,e,f$ are three permutations on that set that
compose to the identity:
$e\circ n\circ f = I$.  The elements of $D$ are called darts.
The permutations $n,e,f$ are called the node permutation,
edge permutation, and face permuation, respectively.
(We previously defined a hypermap as a finite set $D$ with
two permutations $f,n$, which amounts to the same thing,
since $e$ is uniquely determined by $f,n$.)

If $m$ is any permutation on $D$, we write $D/m$ for the
set of orbits in $D$ under $m$.  Similarly, if $G$ is any
group of permutations on $D$, we write $D/G$ for the set
of orbits of $D$ under $G$.  In particular, $D/\tangle{n,e,f}$
is the set of orbits under the group generated by $n,e,f$.

A planar graph gives  a hypermap
by the following procedure.  Starting with a planar graph,
place a dart at each angle.  That is, at a vertex of degree $k$,
place $k$ darts, one between each consecutive pair of edges.
The face permutation has a cycle for
each face of the planar graph and  traverses the
darts in a counterclockwise direction around each face.
The node permutation has a cycle for each vertex and traverses
the darts in a counterclockwise direction around each vertex.
The edge permutation is defined by the relation $e\circ n\circ f=I$.
It can be interpreted as an involution that pairs a dart next
to one endpoint of an edge with a dart at the other endpoint.
See Figure~\ref{XX}.

Hypermaps are the primary combinatorial object used by Gonthier
in the formalization of the four-color theorem in COQ~\ref{Gon}.
Hypermaps, by being purely combinatorial, are more convenient
to represent on a computer than planar graphs.  (In the 1998
preprint, we encoded the combinatorics as planar maps rather
than hypermaps.  It is trivial to translating between the two notions.
Recent work on the formalization of the Kepler conjecture works
consistently with hypermaps; thus, we deprecate the use of
planar maps.  See~\cite{XXObua},\cite{XXBlue}.)

Not all hypermaps arise from a planar graph in this way.
Those that do have two special properties.  They are plain
and planar in the following sense.  (Note the surprising spelling
of `plain', particularly in the context of planar graphs!  
To avoid misunderstandings, 
we avoid the homophone `plane' in this article.)  The definition
of planar hypermap is the standard condition on the Euler
characteristic, translated into the language of hypermpas.

\begin{definition}\label{def:plain}

\begin{itemize}
\item The hypermap $(D,e,n,f)$ is plain, if $e$ is an involution:
$$
 e^2 = I.
$$.
\item The hypermap $(D,e,n,f)$ is planar, if
   $$
   \#(D/e) + \#(D/n) + \#(D/f) = \# D + 2\#(D/\tangle{e,n,f}).
   $$
\end{itemize}
\end{definition}

\subsection{tameness}

\subsection{classification}


\section{Counterexample implies Tameness}

\section{Linear Programs}

This section discusses Theorem~\ref{thm:graph-system}.  It is one of
the main steps in the proof of the Dodecahedral conjecture.
We begin with a discussion of the terminology used in the
statement of the theorem.
The following definition is influenced by S. Obua's thesis~\cite{Ob}.

\begin{definition} A hypermap system is a pair $(H,\Phi)$,
where $H=(D,e,n,f)$ is a hypermap, and $\Phi$ is a finite set constraints on $H$.  More precisely, let $V$ be the vector space of
real-valued functions on $D$.  A finite set of constraints $\Phi$ is a finite
set of boolean valued functions $\phi:V^\ell\to \{\op{true},\op{false}\}$
for some $\ell$. (We assume $\ell$ is independent of $\phi\in \Phi$).
\end{definition}

We say that the hypermap system $(H,\Phi)$ is feasible, if
there is some $x=(x_1,\ldots,x_\ell)\in V^\ell$ such that
$\phi(x)$ holds for all $\phi\in\Phi$. Otherwise, we say that
the system is infeasible.

Appendix~\ref{XX} gives a finite set of constraints $\Phi$, 
specified
in a uniform way for every structurally tame hypermap $H$.  This determines,
for every structurally tame hypermap $H$, we then have a well-defined extension
to a hypermap system $(H,\Phi)$.  We call this particular hypermap
system, by structural analogy with the Kepler conjecture,
the {\it dodecahedral hypermap system}.  


\begin{theorem}\label{thm:graph-system}  Let 
$H$ be any structurally tame hypermap with corresponding
dodecahedral hypermap system $(H,\Phi)$.  Then $(H,\Phi)$ is infeasible.
\end{theorem}

This proof is carried out by computer as a collection of linear
programs.  This is one of the three major parts of the proof
of the Dodecahedral conjecture that has been carried out by computer.
In this article, we describe the relationship between the
feasibility of $(H,\Phi)$ and a linear programming feasibility
problem.  We also describe some details of the implementation of 
the code.

We have obtained an enumeration of all structurally tame
hypermap systems in Section~\ref{XX}.  Thus,  we may prove
Theorem~\ref{thm:graph-system} through a case-by-case treatment
of cases.  This is how we proceed.

Let's turn our attention for a moment, to one structurally tame hypermap $H=(D,e,n,f)$ and the corresponding dodecahedral hypermap system $(H,\Phi)$.
We describe the strategy that we use to show that it is infeasible.
For some $\ell\in\ring{N}$,
each constraint $\phi$ is a function on $V^\ell$, where $V$ is
the vector space of real-valued functions on $D$.  Thus,
$V^\ell$ can be identified with $\ring{R}^m$, where $m= \ell \#(D)$.
If we examine the form of the constraints $\phi\in \Phi$, as listed
in Appendix~\ref{XX}, we notice that the constraints all have a 
very special form.  With a minor qualification, they are all linear constraints
on $\ring{R}^m$.  

We allow a minor qualification to allow some of the constraints
to carry guard conditions.  That is, some constraints have the form
  \begin{equation}\label{eqn:guard}
  (A x < b)  \Rightarrow (A' x \le b'),
  \end{equation}
for $x\in\ring{R}^m$, and various matrices $A,A'$ and vectors
$b,b'$.  (We write $a \le b$ to mean
that $a_i\le b_i$ for every component of the vectors $a,b$.)
We call the constraint $(A x < b)$ a guard condition.
We allow variations in which some of the inequalities in the
guard condition are weak and some of the inequalities in the
consequent are strict.

The collection of all inequalities that do not have a guard 
condition is a system of linear inequalities.  Standard linear
programming packages can be used to determine whether this
system of linear inequalities has a feasible solution.  If this
linear program is infeasible, then the hypermap system $(H,\Phi)$
is clearly also infeasible.  When this happens, we have a
proof of infeasibility for $(H,\Phi)$.

When this fails, we turn to the constraints that carry guard
conditions.
The introduction of a constraint that has a nontrivial guard condition
involves multiple steps.  
The constraint (\ref{eqn:guard}) can be rewritten in logically
equivalent form as
  $$
   (A_{1} x \ge b_{1}) \lor \cdots \lor
   (A_{r} x \ge b_{r}) \lor (A_2 x \le b_2),
  $$
where $A_{i}$ and $b_{i}$ are the rows of $A$ and $b$.
Taking each disjunct in turn, one linear at a time
is added to the system
of linear inequalities, and the resulting system is shown to be
infeasible.  When each
systems are infeasible, then $(H,\Phi)$ itself is infeasible.

In general, more than one guarded constraint may be added.  If
$k$ constraints are added with $r_1,\ldots,r_k$ guards
respectively, then as many as $(r_1+1)\cdots (r_k+1)$ systems
of linear equalities are created.  If they are all infeasible,
then $(H,\Phi)$ itself is infeasible.

This discussion may give the impression 
that a great many linear programming feasibility
problems are created in this manner.  In practice, nearly all
of the hypermap systems are eliminated in the first pass, without
requiring recourse to the guarded conditions.  Only
$14$ cases (XX?) use a guard condition.



\section{A Counterexample implies Feasibility}




\section{Appendix: Dodecahedral Hypermap System}

In this appendix, we list the system of constraints determining
the dodecahedral hypermap system.  As the main body of the text explains
a hypermap system is a pair $(H,\Phi)$, where $H$ is a hypermap
and $\Phi$ is a finite set of constraints on $H$.  To describe
the constraints $\Phi$ in precise terms requires some notation.

Let $H=(D,e,n,f)$ be a hypermap.  We use Greek letters
$\alpha,\beta \in D$ for darts.  Let $V$ be the vector space 
of real-valued  functions on $D$.  A constraint $\phi$ is a boolean-valued function on $V^\ell$ for some $\ell$.  We use a suggestive
notation $(\optt{sol},\optt{mu},\optt{tau},\optt{y},\ldots)\in V^\ell$ for $\ell$-tuples
of elements of $V$.  The main text relates each coordinate back
with its namesake.  For example, the linear constraints on
$\optt{sol}\in V$, mirror  nonlinear relations satisfied by the
nonlinear function $\sol$ in the main text.  This correspondence
between functions in $V$ and functions does not enter into the
definition of the dodecahedral hypermap system.  For the purposes of this
appendix, this correspondence is simply an aid to the
intuition.

If $m$ is  a permutation on $D$, let $\op{ord}(m,\alpha,i)$
be the predicate that asserts that the cardinality of the $m$-orbit
of $\alpha$ is $i\in\ring{N}$.
\bigskip

%XX Move to definitions.
\def\sland{\ \land\ }

\subsection{general bounds and relations}

$$
\begin{array}{lll}
\leftalign\\
   \forall \alpha\in D.\\
    2 \le \optt{yn}(\alpha) \sland
    \optt{yn}(\alpha) \le 2t_{dod} \\
    2 \le \optt{ye}(\alpha) \sland
    \optt{ye}(\alpha) \le 2t_{dod}\\
   0 \le \optt{sol}(\alpha) \sland
     \optt{sol}(\alpha) \le 4 \pi \\
   0 \le \optt{dih}(\alpha) \sland
     \optt{dih}(\alpha) \le 2\pi \\
   0 \le \optt{mu}(\alpha) \sland
      \optt{mu}(\alpha) \le 0.2 \\
   0 \le \optt{omega}(\alpha) \sland
      \optt{omega}(\alpha) \le XX \\
  \optt{yn}(\alpha) = \optt{yn}(n\alpha) \\
  \optt{ye}(\alpha) = \optt{ye}(e\alpha) \\
  \optt{omega}(\alpha) = \optt{omega}(f\alpha) \\
  \optt{mu}(\alpha) = \optt{mu}(f\alpha) \\
  \optt{sol}(\alpha)= \optt{sol}(f\alpha) \\
  \optt{mu}(\alpha) = \optt{omega}(\alpha) - 0.42755\ \optt{sol}(\alpha) \\

\end{array}
$$



$$
\begin{array}{lll}
\leftalign\\
\forall \alpha\in D.\\ 
   \optt{ord}(n,\alpha,k) \Rightarrow
   \sum_{i=1}^k \optt{dih}(n^i\alpha )  = 2\pi.\\
   \optt{ord}(f,\alpha,k) \Rightarrow
   \sum_{i=1}^k \optt{dih}(f^i\alpha ) = \optt{sol} + (k -2)\pi.\\
   \end{array}
$$
\noindent
Let $D'\subset D$ be a set of representatives of the faces in $D$.
$$
\begin{array}{lll}
\leftalign\\
\sum_{\alpha\in D'} \optt{omega}(\alpha) \le 5.5503 \\
\end{array}
$$
%\FIXX{Check the constant 5.5503 against what the LPs use.}
$$
\begin{array}{lll}
\leftalign\\
\forall \alpha\in D.\\
   \optt{ord}(f,\alpha,4) \Rightarrow
   \optt{mu}(\alpha) > 0.031 \\
   \optt{ord}(f,\alpha,5) \Rightarrow
   \optt{mu}(\alpha) > 0.076 \\
   \optt{ord}(f,\alpha,6) \Rightarrow
   \optt{mu}(\alpha) > 0.121 \\
   \optt{ord}(f,\alpha,7) \Rightarrow
   \optt{mu}(\alpha) > 0.166 \\
\end{array}
$$

\FIXX{Add the inequalities corresponding to Obua page 71,
the multiface inequalities, see page 14 McL 2003, Lemma 5.2,
Theorem 8.1 vertex adjustments.  Inequalities for Guards page 65.
}



\subsection{tetrahedral constraints}

\noindent
Function bounds:
$$
\begin{array}{lll}
\leftalign\\
  \forall \alpha\in D. \ \op{ord}(f,\alpha,3) 
   \Rightarrow \\ 
      \optt{omega}(\alpha) > 0.202804  \sland \\
   \optt{sol}(\alpha) > 0.315696 \sland \\
   \sol(S) < 1.051232\\
   \optt{dih}(\alpha) > 0.856147\\
   \optt{dih}(\alpha) < 1.886730\\
   \optt{mu}(\alpha) > 0
\end{array}
$$

$$
\begin{array}{lll}
\leftalign\\
\forall\alpha\in D. \ \op{ord}(f,\alpha,3) \Rightarrow \\
   \optt{omega}(\alpha) - 0.68\ \optt{sol}(\alpha) + 1.88718\ \optt{dih}(\alpha) > 1.54551 \\
   \optt{omega}(\alpha) - 0.68\ \optt{sol}(\alpha) + 0.90746\ \optt{dih}(\alpha) > 0.706725\\
   \optt{omega}(\alpha) - 0.68\ \optt{sol}(\alpha) + 0.46654\ \optt{dih}(\alpha) > 0.329233\\
   \optt{omega}(\alpha) - 0.55889\ \optt{sol}(\alpha) - 0\ \optt{dih}(\alpha) > -0.0736486\\
   \optt{omega}(\alpha) - 0.63214\ \optt{sol}(\alpha) - 0\ \optt{dih}(\alpha) > -0.13034\\
   \optt{omega}(\alpha) - 0.73256\ \optt{sol}(\alpha) - 0\ \optt{dih}(\alpha) > -0.23591\\
   \optt{omega}(\alpha) - 0.89346\ \optt{sol}(\alpha) - 0\ \optt{dih}(\alpha) > -0.40505\\
   \optt{omega}(\alpha) - 0.3\ \optt{sol}(\alpha) - 0.5734\ \optt{dih}(\alpha) > -0.978221\\
   \optt{omega}(\alpha) - 0.3\ \optt{sol}(\alpha) - 0.03668\ \optt{dih}(\alpha) > 0.024767\\
   \optt{omega}(\alpha) - 0.3\ \optt{sol}(\alpha) + 0.04165\ \optt{dih}(\alpha) > 0.121199\\
   \optt{omega}(\alpha) - 0.3\ \optt{sol}(\alpha) + 0.1234\ \optt{dih}(\alpha) > 0.209279\\
   \optt{omega}(\alpha) - 0.42755\ \optt{sol}(\alpha) - 0.11509\ \optt{dih}(\alpha) > -0.171859\\
   \optt{omega}(\alpha) - 0.42755\ \optt{sol}(\alpha) - 0.04078\ \optt{dih}(\alpha) > -0.050713\\
   \optt{omega}(\alpha) - 0.42755\ \optt{sol}(\alpha) + 0.11031\ \optt{dih}(\alpha) > 0.135633\\
   \optt{omega}(\alpha) - 0.42755\ \optt{sol}(\alpha) + 0.13091\ \optt{dih}(\alpha) > 0.157363\\
   \optt{omega}(\alpha) - 0.55792\ \optt{sol}(\alpha) - 0.21394\ \optt{dih}(\alpha) > -0.417998\\
   \optt{omega}(\alpha) - 0.55792\ \optt{sol}(\alpha) - 0.0068\ \optt{dih}(\alpha) > -0.081902\\
   \optt{omega}(\alpha) - 0.55792\ \optt{sol}(\alpha) + 0.0184\ \optt{dih}(\alpha) > -0.051224\\
   \optt{omega}(\alpha) - 0.55792\ \optt{sol}(\alpha) + 0.24335\ \optt{dih}(\alpha) > 0.193993\\
   \optt{omega}(\alpha) - 0.68\ \optt{sol}(\alpha) - 0.30651\ \optt{dih}(\alpha) > -0.648496\\
   \optt{omega}(\alpha) - 0.68\ \optt{sol}(\alpha) - 0.06965\ \optt{dih}(\alpha) > -0.278\\
   \optt{omega}(\alpha) - 0.68\ \optt{sol}(\alpha) + 0.0172\ \optt{dih}(\alpha) > -0.15662\\
   \optt{omega}(\alpha) - 0.68\ \optt{sol}(\alpha) + 0.41812\ \optt{dih}(\alpha) > 0.287778\\
   \optt{omega}(\alpha) - 0.64934\ \optt{sol}(\alpha) - 0\ \optt{dih}(\alpha) > -0.14843\\
   \optt{omega}(\alpha) - 0.6196\ \optt{sol}(\alpha) - 0\ \optt{dih}(\alpha) > -0.118\\
   \optt{omega}(\alpha) - 0.58402\ \optt{sol}(\alpha) - 0\ \optt{dih}(\alpha) > -0.090290\\
   \optt{omega}(\alpha) - 0.25181\ \optt{sol}(\alpha) - 0\ \optt{dih}(\alpha) > 0.096509\\
   \optt{omega}(\alpha) - 0.00909\ \optt{sol}(\alpha) - 0\ \optt{dih}(\alpha) > 0.199559\\
   \optt{omega}(\alpha) + 0.93877\ \optt{sol}(\alpha) - 0\ \optt{dih}(\alpha) > 0.537892\\
   \optt{omega}(\alpha) + 0.93877\ \optt{sol}(\alpha) - 0.20211\ \optt{dih}(\alpha) > 0.27313\\
   \optt{omega}(\alpha) + 0.93877\ \optt{sol}(\alpha) + 0.63517\ \optt{dih}(\alpha) > 1.20578\\
   \optt{omega}(\alpha) + 1.93877\ \optt{sol}(\alpha) - 0\ \optt{dih}(\alpha) > 0.854804\\
   \optt{omega}(\alpha) + 1.93877\ \optt{sol}(\alpha) - 0.20211\ \optt{dih}(\alpha) > 0.621886\\
   \optt{omega}(\alpha) + 1.93877\ \optt{sol}(\alpha) + 0.63517\ \optt{dih}(\alpha) > 1.57648\\
   \optt{omega}(\alpha) - 0.42775\ \optt{sol}(\alpha) - 0\ \optt{dih}(\alpha) > -0.000111\\
   \optt{omega}(\alpha) - 0.55792\ \optt{sol}(\alpha) - 0\ \optt{dih}(\alpha) > -0.073037\\
   \optt{omega}(\alpha) - 0\ \optt{sol}(\alpha) - 0.07853\ \optt{dih}(\alpha) > 0.08865\\
   \optt{omega}(\alpha) - 0\ \optt{sol}(\alpha) - 0.00339\ \optt{dih}(\alpha) > 0.198693\\
   \optt{omega}(\alpha) - 0\ \optt{sol}(\alpha) + 0.18199\ \optt{dih}(\alpha) > 0.396670\\
   \optt{omega}(\alpha) - 0.42755\ \optt{sol}(\alpha) - 0.2\ \optt{dih}(\alpha) > -0.332061\\
   \optt{omega}(\alpha) - 0.3\ \optt{sol}(\alpha) - 0.36373\ \optt{dih}(\alpha) > -0.58263\\
   \optt{omega}(\alpha) - 0.3\ \optt{sol}(\alpha) + 0.20583\ \optt{dih}(\alpha) > 0.279851\\
   \optt{omega}(\alpha) - 0.3\ \optt{sol}(\alpha) + 0.40035\ \optt{dih}(\alpha) > 0.446389\\
   \optt{omega}(\alpha) - 0.3\ \optt{sol}(\alpha) + 0.83259\ \optt{dih}(\alpha) > 0.816450\\
   \optt{omega}(\alpha) - 0.42755\ \optt{sol}(\alpha) - 0.51838\ \optt{dih}(\alpha) > -0.932759\\
   \optt{omega}(\alpha) - 0.42755\ \optt{sol}(\alpha) + 0.29344\ \optt{dih}(\alpha) > 0.296513\\
   \optt{omega}(\alpha) - 0.42755\ \optt{sol}(\alpha) + 0.57056\ \optt{dih}(\alpha) > 0.533768\\
   \optt{omega}(\alpha) - 0.42755\ \optt{sol}(\alpha) + 1.18656\ \optt{dih}(\alpha) > 1.06115\\
   \optt{omega}(\alpha) - 0.55792\ \optt{sol}(\alpha) - 0.67644\ \optt{dih}(\alpha) > -1.29062\\
   \optt{omega}(\alpha) - 0.55792\ \optt{sol}(\alpha) + 0.38278\ \optt{dih}(\alpha) > 0.313365\\
   \optt{omega}(\alpha) - 0.55792\ \optt{sol}(\alpha) + 0.74454\ \optt{dih}(\alpha) > 0.623085\\
   \optt{omega}(\alpha) - 0.55792\ \optt{sol}(\alpha) + 1.54837\ \optt{dih}(\alpha) > 1.31128\\
   \optt{omega}(\alpha) - 0.68\ \optt{sol}(\alpha) - 0.82445\ \optt{dih}(\alpha) > -1.62571\\
\end{array}
$$

$$
\begin{array}{lll}
\leftalign\\
\forall\alpha\in D. \ \op{ord}(f,\alpha,3) \Rightarrow \\
  \optt{sol}(\alpha) > 0.551285 - 0.245 (y_1+y_2+y_3-6) + 0.063 (y_4+y_5+y_6-6)\\
  \optt{sol}(\alpha) > 0.551285 - 0.3798 (y_1+y_2+y_3-6) + 0.198 (y_4+y_5+y_6-6)\\
  \optt{sol}(\alpha) < 0.551286 - 0.151 (y_1+y_2+y_3-6) + 0.323 (y_4+y_5+y_6-6)\\

  \optt{mu}(\alpha) > 0.0392 (y_1+y_2+y_3-6) + 0.0101 (y_4+y_5+y_6-6) \\
  \optt{omega}(\alpha) > 0.235702 -0.107 (y_1+y_2+y_3-6) + 0.116 (y_4+y_5+y_6-6)\\
  \optt{omega}(\alpha) > 0.235702 -0.0623 (y_1+y_2+y_3-6) + 0.0722 (y_4+y_5+y_6-6)\\
  \optt{dih}(\alpha) > 1.23095 + 0.237 (y_1-2) - 0.372 (y_2+y_3+y_5+y_6-8) + 0.708 (y_4-2) \\
  \optt{dih}(\alpha) > 1.23095 + 0.237 (y_1-2) - 0.363 (y_2+y_3+y_5+y_6-8) + 0.688 (y_4-2)\\
  \optt{dih}(\alpha) < 1.23096 + 0.505 (y_1-2) - 0.152(y_2+y_3+y_5+y_6-8) + 0.766 (y_4-2)\\

\end{array}
$$


$$
\begin{array}{lll}
\leftalign\\
\forall\alpha\in D. \ \op{ord}(f,\alpha,4) \Rightarrow \\
   \optt{omega}(\alpha) > 0.455149\\
   \optt{sol}(\alpha) > 0.731937\\
   \optt{sol}(\alpha) < 2.85860\\
   \optt{dih}(\alpha) > 1.15242\\
   \optt{dih}(\alpha) < 3.25887\\
   \optt{mu}(\alpha) > 0.031350\\
\end{array}
$$


$$
\begin{array}{lll}
\leftalign\\
\forall\alpha\in D. \ \op{ord}(f,\alpha,4) \Rightarrow \\
   \optt{omega}(\alpha) - 0.42775\ \optt{sol}(\alpha) - 0.15098\ \optt{dih}(\alpha) > -0.3670\\
   \optt{omega}(\alpha) - 0.42775\ \optt{sol}(\alpha) - 0.09098\ \optt{dih}(\alpha) > -0.1737\\
   \optt{omega}(\alpha) - 0.42775\ \optt{sol}(\alpha) - 0.00000\ \optt{dih}(\alpha) > 0.0310\\
   \optt{omega}(\alpha) - 0.42775\ \optt{sol}(\alpha) + 0.18519\ \optt{dih}(\alpha) > 0.3183\\
   \optt{omega}(\alpha) - 0.42775\ \optt{sol}(\alpha) + 0.20622\ \optt{dih}(\alpha) > 0.3438\\
   \optt{omega}(\alpha) - 0.55792\ \optt{sol}(\alpha) - 0.30124\ \optt{dih}(\alpha) > -1.0173\\
   \optt{omega}(\alpha) - 0.55792\ \optt{sol}(\alpha) - 0.02921\ \optt{dih}(\alpha) > -0.2101\\
   \optt{omega}(\alpha) - 0.55792\ \optt{sol}(\alpha) - 0.00000\ \optt{dih}(\alpha) > -0.1393\\
   \optt{omega}(\alpha) - 0.55792\ \optt{sol}(\alpha) + 0.05947\ \optt{dih}(\alpha) > -0.0470\\
   \optt{omega}(\alpha) - 0.55792\ \optt{sol}(\alpha) + 0.39938\ \optt{dih}(\alpha) > 0.4305\\
   \optt{omega}(\alpha) - 0.55792\ \optt{sol}(\alpha) + 2.50210\ \optt{dih}(\alpha) > 2.8976\\
   \optt{omega}(\alpha) - 0.68000\ \optt{sol}(\alpha) - 0.44194\ \optt{dih}(\alpha) > -1.6264\\
   \optt{omega}(\alpha) - 0.68000\ \optt{sol}(\alpha) - 0.10957\ \optt{dih}(\alpha) > -0.6753\\
   \optt{omega}(\alpha) - 0.68000\ \optt{sol}(\alpha) - 0.00000\ \optt{dih}(\alpha) > -0.4029\\
   \optt{omega}(\alpha) - 0.68000\ \optt{sol}(\alpha) + 0.86096\ \optt{dih}(\alpha) > 0.8262\\
   \optt{omega}(\alpha) - 0.68000\ \optt{sol}(\alpha) + 2.44439\ \optt{dih}(\alpha) > 2.7002\\
   \optt{omega}(\alpha) - 0.30000\ \optt{sol}(\alpha) - 0.12596\ \optt{dih}(\alpha) > -0.1279\\
   \optt{omega}(\alpha) - 0.30000\ \optt{sol}(\alpha) - 0.02576\ \optt{dih}(\alpha) > 0.1320\\
   \optt{omega}(\alpha) - 0.30000\ \optt{sol}(\alpha) + 0.00000\ \optt{dih}(\alpha) > 0.1945\\
   \optt{omega}(\alpha) - 0.30000\ \optt{sol}(\alpha) + 0.03700\ \optt{dih}(\alpha) > 0.2480\\
   \optt{omega}(\alpha) - 0.30000\ \optt{sol}(\alpha) + 0.22476\ \optt{dih}(\alpha) > 0.5111\\
   \optt{omega}(\alpha) - 0.30000\ \optt{sol}(\alpha) + 2.31852\ \optt{dih}(\alpha) > 2.9625\\
   \optt{omega}(\alpha) - 0.23227\ \optt{dih}(\alpha) > -0.1042\\
   \optt{omega}(\alpha) + 0.07448\ \optt{dih}(\alpha) > 0.5591\\
   \optt{omega}(\alpha) + 0.22019\ \optt{dih}(\alpha) > 0.7627\\
   \optt{omega}(\alpha) + 0.80927\ \optt{dih}(\alpha) > 1.5048\\
   \optt{omega}(\alpha) + 5.84380\ \optt{dih}(\alpha) > 7.3468\\
\end{array}
$$




\subsection{guard conditions}

\noindent
Condition H.14.1:  % Use alpha= y1.
$$
\begin{array}{lll}
\leftalign\\
\forall \alpha\in D.\\
  \op{ord}(f,\alpha,5) \sland 
  \optt{dih}(\alpha) < 1.342 \sland
  \optt{dih}(f^3\alpha) < 1.684 \sland
  \optt{yn}(\alpha) < 2.153 \sland \\
   \optt{y}(f\alpha) < 2.174 \sland
  \optt{yn}(f^2\alpha) < 2.26 \sland
  \optt{yn}(f^3 \alpha) < 2.194 \sland
  \optt{yn}(f^4 \alpha) < 2.314  \\
  \Rightarrow
  \optt{omega}(\alpha) > 0.950
\end{array}
$$

\noindent
Condition H.14.2:  (XX Potential problem, in H.14.1, the
condition is $y(10)< 2.26$, the index changes in H.14.2
to $y(5) > 2.26$.)
$$
\begin{array}{lll}
\leftalign\\
\forall \alpha\in D.\\
  \op{ord}(f,\alpha,5) \sland 
  \optt{dih}(\alpha) < 1.342 \sland
  \optt{dih}(f^3\alpha) < 1.684 \sland
  \optt{y}(\alpha) < 2.153 \sland \\
   \optt{y}(f\alpha) < 2.174 \sland
  \optt{y}(f^2\alpha) \ge 2.26 \sland
  \optt{y}(f^3 \alpha) < 2.194 \sland
  \optt{y}(f^4 \alpha) < 2.314  \\
   \Rightarrow
  \optt{mu}(\alpha) > 0.1234
\end{array}
$$
