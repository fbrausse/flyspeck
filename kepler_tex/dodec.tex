
\section{Introduction}

% XX http://www.sojamo.de/blog/2007/04/  PUT IMAGE OF 3D VORONOI.

\label{sec:dodec}

A packing of congruent unit radius balls 
in Euclidean space determines a region called the Voronoi cell 
around each ball.  
A packing is determined by and is identified  with the set $\Lambda$ of centers of the balls.  The Voronoi
cell $\Omega(\Lambda,v)$ around a ball at $v\in \Lambda$ 
consists of points of space that are closer to $v$ than
to any other $w\in\Lambda$.  The Voronoi cell is a
convex polyhedron containing $v$.

The Dodecahedral conjecture asserts that in any packing of congruent balls of Euclidean
space every Voronoi cell has volume at least that of a regular dodecahedron
circumscribing a unit ball.    This bound is realized by a finite 
packing $\Lambda_{dod}$
(of twelve balls and a thirteenth  at the origin) obtained
by placing a ball at the center of each face of a regular dodecahedron.  The
theorem can then be stated as the inequality
  $$
  \op{vol}(\Omega(\Lambda,v)) \ge \op{vol}(\Omega(\Lambda_{dod},0))
  $$
for every $v\in\Lambda$, and for every set of points $\Lambda\subset \ring{R}^3$
whose pairwise distances are at least the diameter $2$.
The case of equality occurs exactly when $\Omega(\Lambda,v)$ is
congruent to a regular dodecahedron of inradius $1$.



\subsection{history}


L. Fejes T\'oth made the conjecture in 1943 \cite{Toth1}.  
In that article, L. Fejes
T'oth sketches a proof based on an unproved hypothesis.  This
hypothesis is an explicit version of the kissing number problem
in three dimensions.   This unproved hypothesis
is now generally regarded as being 
nearly as difficult as the  Dodecahedral conjecture itself.  

L. Fejes T\'oth returned to the Dodecahedral conjecture in a number
of publications.  It is a prominent part of his two books \cite{Fej72},
\cite{Toth2}.  According to the strategy of \cite{Fej72}, 
the Dodecaehdral conjecture forms a
step towards the solution of the sphere packing problem
(discussed below).   In \cite{Toth2}, he proved that the Dodecahedral conjecture
holds for every Voronoi cell with at most twelve facets.  
This result is reviewed in Section~\ref{XX}.  It
is an ingredient in our proof. 

In 1993, Hsiang published what seemed to be a proof of the Kepler
conjecture which would prove the Dodecahedral conjecture 
as well \cite{Hsiang}.
However, the proof did not hold up to careful analysis.  ``As of this
writing, Kepler's conjecture as well as the dodecahedral conjecture
are still unproven'' \cite[p761]{Bezdek}.  See also, \cite{Hal94}.

An alternative approach to the Dodecahedral conjecture is described
in \cite{Bezdek}.  Unfortunately, a counterexample has been found to
both parts of the third conjecture of that paper.  The counterexample
is described in \cite{arx}, the first version of this paper.  We
do not repeat the counterexample here.


K. Bezdek conjectures that the surface area of any Voronoi cell in a packing
of unit balls is at least that of a regular dodecahedron of inradius $1$.
This strengthened version of the Dodecahedral problem is still
open \cite{Bez04}.   



\subsection{sphere packing problem}

The Kepler conjecture, also known as the sphere packing problem, 
asserts that no packing of congruent balls
in three dimensions has density greater than the density of the
face-centered cubic packing.  
The research for the proof of the Dodecahedral Conjecture 
was carried out at the University
of Michigan at the same time that S. Ferguson and T. Hales were
carrying out their work on the sphere packing problem.  Both
problems were solved in 1998.  

There is no strict logical connection between the two problems.
The Dodecahedral conjecture does not follow from the Kepler conjecture
and is not an intermediate step in the solution to the Kepler conjecture.
(In Fejes T\'oth strategy, it was an intermediate step; however, that
strategy was not followed in the solution of the sphere packing problem.)
Nevertheless,
the two solutions follow a similar outline and share a significant number of 
methods.
Both are based 
on long computer calculations.  Computer
code was freely exchanged between the two projects.

This article is written in a way that it is not necessary to
read or understand the solution of the sphere packing problem before
reading this article.  However, for the benefit of the reader,
at various points in the proof
of the Dodecahedral conjecture, we point out 
parallels with the sphere packing problem.  We also cite various results
from that proof.



\subsection{differences}

Although the proof of the Dodecahedral conjecture runs parallel
to the solution to the sphere packing problem, there are special
difficulties that arise in the proof of the Dodecahedral conjecture.
In no sense is it a corollary of the sphere packing problem.
In the packing problem, there turn out to be many ways to
reduce the infinite ball problem to a problem about finite clusters
of balls.  This multiplicity of choices makes it
possible to design many difficulties away.  If one reduction is
not satisfactory one can work with
another.
With the Dodecahedral problem, there is no such flexibility.
The problem about finite clusters of balls is fixed from the outset.
This gives the problem a degree of rigidity that is not present
in the sphere packing problem.


\subsection{ten years later}

Over ten years have elapsed from the completion of the research until 
publication.  A few words of explanation are in order.
The review and 
publication process for the Kepler conjecture extended from 1998
until 2006.  Because of significant sharing between the Kepler conjecture
and the Dodecahedral conjecture, the editors adopted a ``wait and see''
attitude toward the Dodecahedral conjecture.  Once the Kepler conjecture
was published, the path became clear for the publication of Dodecahedral
conjecture.  

The details of the proof in the current publication are essentially the
same as those in the preprint posted on the arXiv in 1998 \cite{arx}.  No
significant errors have emerged in the original 1998 preprint.   However,
the paper has undergone three rewrites since then, an expanded version in 2002, followed by an abridged version in 2003, then a newly expanded version in 2008, all coming at the request by the editors.  It was as if a complex computer proof caused a discomfort that might be relived by a long series of
rewrites. The computer code has
also been entirely rewritten.

A formalization project, called Flyspeck, aims to
provide a complete formalization of the proof of the Kepler conjecture \cite{fly},\cite{Fl}.  (A formalized proof is one in which every logical inference of the proof has been independently checked by computer, all the way to the primitive axioms at the foundations of mathematics.)
A parallel project, called Flyspeck Light, aims to do the same
for the proof of the Dodecahedral conjecture.  These long-term projects
will take many years to complete.  Nevertheless, significant progress
has already been made toward the formal verification of the computer
code \cite{BN}, \cite{Ob}.   The revisions in this paper 
incorporate the parts of Flyspeck Light that have
already been completed.

\subsection{truncation}

The distance from the center of the regular dodecahedron to a vertex is
$t_{dod}=\sqrt{3}\tan(\pi/5)\approx 1.258$.   This parameter is used to truncate
Voronoi cells; it makes  volumes easier to estimate.  A similar truncation
takes place in the solution to the packing problem with truncation parameter
$t_0 = 1.255$.  It is a happy coincidence that these two truncation parameters 
are so close to one another.  (Section~\ref{sec:experiment} 
gives the reasons behind the parameter $t_0$.)  A great deal of duplicated effort
might have been avoided if these two parameters were equal.  However, the parameter
$t_{dod}$ cannot be replaced with anything smaller, 
and although the parameter $t_0$ could 
easily have been made larger, 
its value was already too deeply entrenched in published papers 
by the time  work started
on the Dodecahedral conjecture.  

As a result, there are several definitions in the proof of the Dodecahedral conjecture
and in the solution to the packing problem, identical in every respect except
for the choice of parameter: $t_0,t_{dod}$.  This is indeed unfortunate.  However,
there is no easy remedy. 



 As a first step towards unifying the proofs,
 many results can be stated in a form that holds for all $t\in[t_0,t_{dod}]$.
To transfer a lemma from \cite{DCG} to this article, a simple process is
involved.  The first step is generalization, replacing the constant $t_0$
with a free parameter $t\in[t_0,t_{dod}]$.  The second is specialization,
$t\mapsto t_{dod}$.  

The number $t_0$, although rational, 
can be consistently treated as an independent real transcendental 
in the solution to the sphere packing problem; that is,
none of the proofs involving $t_0$ rely on its exact numerical value.
The constant $t_0$ can always be replaced by a constant $t$
in a suitably small interval about $t_0$.  However, the only way to
know that this small interval is wide enough to contain $t_{dod}$ is to study the
details of the proof.  For this article, 
we have made a detailed study of each relevant proof in \cite{DCG}.

As a matter of terminology, we say a proposition for the Dodecahedral conjecture
is a $[t_0,t_{dod}]$-perturbation of a proposition of \cite{DCG},
if it is obtained by mindlessly replacing $t_0$ with $t_{dod}$, wherever that constant appears, and if the proof goes through verbatim with this minor change.
When this occurs, 
there is nothing to further to be learned by repeating the proof,
and we refer the reader  for the proof to \cite{DCG}.

\subsection{terminology}

Various notation and terminology is shared between the solution
of the sphere packing problem and this article.  

A large technical vocabulary has been developed 
to describe the solution to the sphere packing problem.  
Vocabulary can be imported from the sphere packing problem three
different ways.  The simplest way to import a term is for the
term to have precisely the same meaning in both places.  For
example, the terms ``orientation,'' ``packing,'' have the
same meaning in both places. 

The second way to import a term is by making a $[t_0,t_{dod}]$-perturbation of a term that depends on the parameter $t_0$.  For example,
in the solution of the sphere packing problem, a {\it quasi-regular
tetrahedron} is defined to be a set $S\subset\ring{R}^3$ of four
points forming a packing, such that
   $$
   \forall\ v,\ w \in S.\ |v - w| \le 2t_0.
   $$
That is, the edge lengths vary between $2$ and $2t_0$.
The $[t_0,t_{dod}]$-perturbation of this definition is a set $S\subset\ring{R}^3$
of four points forming a packing, such that
   $$
   \forall\ v,\ w \in S.\ |v - w| \le 2t_{dod}.
   $$

What to call such a set?  In the original version of this article,
this perturbed definition is again called a quasi-regular tetrahedron. 
A referee writes, ``Invent a new name
for `quasi-regular tetrahedron'; its definition is set in stone
in the Kepler conjecture papers.  This is a new definition\ldots'' 
This observation is entirely correct.  The referee's comment applies not
just to a solitary term `quasi-regular tetrahedron,' but to
a large technical vocabulary that has been developed to describe
the geometry of sphere packings.  The mastery of this specialized
vocabulary already places a mild burden on the reader.  The mastery
of two specialized vocabularies, one a $[t_0,t_{dod}]$-perturbation
of the other, would double that burden.  A new term,
such as pseudo-quasi-regular, would lead to a further burden
as the reader ascertains the exact relationship between
pseudo-quasi and quasi.  All this we wish to avoid.

Since this vocabulary is highly specialized for the proof of these
two problems in sphere packings, it is highly unlikely for these terms to find applications outside this problem domain.  They are not 
general fundamental notions such as set, manifold, or group, where
a large mathematical literature depends on a uniform definition.
The main danger is that a reader might improperly mix results from the
two articles.


The compromise that we have settled upon in this paper is to
retain the same technical vocabulary that is used in the solution
to the sphere packing problem, but to remind the reader regularly
of any variations in meaning that occur.  We ask the reader to
imagine a distinguishing subscript for each occurrence of {\it quasi-regular tetrahedron$_{dod}$}
and 
{\it quasi-regular tetrahedron$_0$} in the two articles.  
Subscript $dod$ in this article, 
and subscript $0$
there, never mixing the two.
For the convenience
of the reader, we give the following complete 
list of definitions and constructions in this
paper that are $[t_0,t_{dod}]$-perturbations of those in the sphere
packing problem.

\begin{itemize}
\item quasi-regular tetrahedron, standard region, 
\end{itemize}

The third way in which terms have been imported into this proof
from the proof of the Kepler conjecture has been by structural
analogy.  For example, an important step in the solution to the
sphere packing problem is a classification of all tame planar graphs.
The term {\it tame graph} is a technical notion that arises in
the solution of the sphere packing problems.  In this article,
there is an analogous classification of planar graphs, this time
related to potential counterexamples to the Dodecahedral conjecture.
The planar graphs that arise here are not tame and they are
not even a $[t_0,t_{dod}]$-perturbation of the definition of tame.  
Nevertheless, a strong parallel exists between tame planar graphs 
and the graphs that are studied in this article.  
In fact, the same piece of computer
code was used to carry out both classifications. 

What to call them?  In this case, it would be too much of a stretch
to keep an identical name for this new type of planar graph.
In these cases, when we import the term, we modify it with
the adjective {\it structural}, to remind us that the way it functions 
bears a structural
analogy with the solution to the sphere
packing problem.  Thus, in this paper, we classify all
{\it structurally tame planar graphs}.

For the convenience of the reader, here is a list of terms that
have been imported by analogy, and qualified with the adjective
{\it structural} or adverb {\it structurally}.

\begin{itemize}
\item tame, squander, contravening, basic
\end{itemize}

\section{Outline}

In this section, we give a precise statement of the main theorem
and describe the broad outline of the proof.


\subsection{formulation}

This article proves the Dodecahedral conjecture in a slightly
stronger version than that stated in the abstract.  It proves
that a certain truncation of the Voronoi cell already has volume
at least as great as that of the regular dodecahedron.  This
section describes the truncation and states the stronger version
of the main theorem in a precise form.

Let $\Lambda$ be a packing and let $v_0\in\Lambda$.
Let $\Lambda(v_0,r)$ be the set of points of $\Lambda$ at distance at most $r$ from $v_0$.
Let $B(v,r)$ be an open ball
of radius $r$ centered at $v$.  

Let $\CalQ_{dod}(\Lambda,v_0)$ 
be the set of all $S=\{v_0,v_1,v_2,v_3\}\subset\Lambda(v_0,2t_{dod})$
consisting of four distinct points such that $|v_i-v_j|\le 2t_{dod}$ for all $i,j$
and such that the circumradius of each triangle $\{v_i,v_j,v_k\}\subset S$ is at most
$\sqrt2$.   Write $\op{conv}(S)$ for the convex hull of $S\in\CalQ_{dod}(\Lambda,v_0)$ and $\op{conv}^0(S)$ for its interior.
A simple lemma (Lemma~\ref{XX})
shows that distinct $S,S'\in\CalQ_{dod}(\Lambda,v_0)$ have disjoint
interiors:
   $$\op{conv}^0(S) \cap \op{conv}^0(S') \ne \emptyset 
   \quad\Rightarrow (S = S').
   $$


Define the following truncation $\Omega_{trunc}(\Lambda,v_0)$ 
of the Voronoi cell $\Omega(\Lambda,v_0)$:
   $$
   \{x \in \Omega(\Lambda,v_0) \mid   x\in B(v_0,t_{dod}) \quad\text{\bf  or }\quad x \in \op{conv}(S)
     \text{ for some } S\in \CalQ_{dod}(\Lambda,v_0) \}. 
   $$
That is, we truncate the Voronoi cell by intersecting it with a ball of radius $t_{dod}$,
except inside regions protected by the sets $S\in\CalQ_{dod}(\Lambda,v_0)$.
Note that for the special packing $\Lambda_{dod}$, we have 
$\Omega_{trunc}(\Lambda_{dod},0) = \Omega(\Lambda_{dod},0)$.
We prove the  Dodecahedral conjecture in the following form.

\begin{theorem}\label{thm:main}
For every packing $\Lambda$ and every $v_0\in\Lambda$,
   $$
   \op{vol}(\Omega_{trunc}(\Lambda,v_0))\ge \op{vol}(\Omega(\Lambda_{dod},0)).
   $$
Equality holds exactly when $\Omega(\Lambda,v_0)$ is congruent to
$\Omega(\Lambda_{dod},0)$.
\end{theorem}





\subsection{proof outline}

The Lebesgue measure is translation invariant.  Thus it does no
harm to assume that the center point $v_0$ of the Voronoi cell
lies at the origin: $v_0 = 0 \in \Lambda$.  The assumption that
$v_0=0 \in\Lambda$ remains in force for the rest of this paper.


The next reduction is to replace the set $\Lambda$ with $\Lambda(0,2t_{dod})$.
This is accomplished by the following lemma, which shows that the
volumes in Theorem~\ref{thm:main} are insensitive to points of $\Lambda$
outside $\Lambda(0,2t_{dod})$.  The proof appears in Section~\ref{XX}.


\begin{lemma} Let $\Lambda$ be any packing (with $0\in \Lambda$).
Then
$$\Omega_{trunc}(\Lambda,0) = \Omega_{trunc}(\Lambda(0,2t_{dod}),0).$$
\end{lemma}

The standing assumption that $\Lambda=\Lambda(0,2t_{dod})$ remains in force
for the rest of the paper.

The proof of Theorem~\ref{thm:main}
splits into two main cases, according to whether
$\Lambda\setminus\{0\}$ has at most $12$ points, or at least $13$.
Let $n=|\Lambda\setminus\{0\}|$.  In fact, the case $n\le 12$ was
settled by L. Fejes T\'oth in his book \cite{Toth2}.  Fejes T\'oth's proof
is sketched in Section~\ref{XX}.

Completely different methods treat the case when $n\ge 13$.  
This part of the proof is
much more difficult than the case treated by Fejes T\'oth.

We give a sketch of the proof of the case $n\ge 13$.   This
rough sketch will be expanded in considerably greater detail later
in the article.
Let $\Lambda$ be a packing satisfying our standing assumptions
that $0\in\Lambda$ and $\Lambda = \Lambda(0,2t_{dod})$.
We seek to minimize the objective function 
$$
\op{vol}(\Omega_{trunc}(\Lambda,0)).
$$
This is a nonlinear optimization problem in finitely many variables.
The target value for the minimization is $b=\op{vol}(\Omega(\Lambda_{dod},0))$.  When $n\ge 13$, we prove\footnote{An examination of the proof shows that when $n\ge13$, the left-hand side of this
equation is at least $b+10^{-16}$.},
   $$
   \op{vol}(\Omega_{trunc}(\Lambda,0))  > b.
   $$



Some combinatorial information about each packing $\Lambda$ is encoded
as a  graph.  The vertex set of the graph is $\Lambda\setminus\{0\}$.
The edge set is 
  $$
  E = \{(v,w) \mid v,w\in\Lambda\setminus\{0\},\quad   |v-w| \le 2t_{dod}\}.
  $$
This is a planar graph.  
We may reduce to the case where this graph is connected.  In fact,
if the graph is not connected, we construct another packing
$\Lambda'$ whose graph is connected, $|\Lambda|=|\Lambda'|$, 
and such that
$$
   \op{vol}(\Omega_{trunc}(\Lambda,0)) = \op{vol}(\Omega_{trunc}(\Lambda',0)). 
$$
Similarly, we may reduce to the case where every face of the graph
is a simple polygon.  This again involves constructing an auxilliary 
packing $\Lambda''$ of the same cardinality and whose  truncated Voronoi cell has the same volume.  We now assume that the graph of $\Lambda$ is 
connected with simple polygonal faces.

We assume the existence of a counterexample $\Lambda$ to the Dodechedral
conjecture, and make a detailed study of the properties of its
graph.  We define a class of graphs (called structurally tame) and
prove that the graph of every counterexample is structurally tame.

Structurally tame graphs can be described in purely combinatorial
terms, without reference to packings, Voronoi cells, and volumes.
All structurally tame graphs can be classified up to isomorphism.
This classification is one of the main steps of the proof.
There are only finitely many possibilities.  Thus, the graph
of any counterexample
to the Dodecahedral conjecture must be one of these finitely
many cases.

Each structurally tame graph can be encoded as a hypermap $H$.  (A hypermap can be defined
as a finite set, together with two permutations on that set.  The elements
of the given finite set are called darts.)  If $H$ is a hypermap,
let $V$ be the finite dimensional
vector space of real-valued functions on its set of darts.
We call $(H,\Phi)$ a hypermap system, if $\Phi$ is a set of boolean
valued functions $\phi:V^\ell \to \{\op{true},\op{false}\}$ for
some $\ell$ that is independent of $\phi\in\Phi$.  We say that
a hypermap system $(H,\Phi)$ 
is feasible if there is some $x=(x_1,\ldots,x_\ell)\in V^\ell$
such that $\phi(x)$ holds for all $\phi\in\Phi$.  Otherwise,
it is infeasible.   Appendix~\ref{XX} defines
a hypermap system, called the basic hypermap system, 
for each case $H$ arising in the classification of
structurally tame graphs.

Another major step of the proof is the proof that every basic hypermap
system is infeasible (Theorem~\ref{XX}).   The proof of this theorem
is a case-by-case analysis based on the explicit enumeration of
structurally tame graphs, up to isomorphism.   The feasibility problem for 
each basic hypermap system is converted to a system of linear programs.
The infeasibility of the basic hypermap system follows from the
infeasibility of the corresponding linear program.

If there exists a counterexample to the Dodecahedral conjecture, then
we can consider its basic hypermap system $(H,\Phi)$.  By the preceding
result, this hypermap system is infeasible.  On the other hand, we can
use the counterexample to construct a feasible solution to the system 
(Theorem~\ref{XX}).
This contradiction shows that a counterexample cannot exist.
In this way, the Dodecahedral conjecture is proved.






\section{Fejes T\'oth's Reduction}

L. Fejes T\'oth proved  the Dodecahedral conjecture 
under the extra hypothesis that $\Lambda(v_0,2t_{dod})\setminus\{v_0\}$
has at most $12$  elements.  He never published the details when
there is truncation. Here
we sketch a proof of Fejes T\'oth's theorem.

\begin{lemma}  If $\Lambda(v_0,2t_{dod})\setminus\{v_0\}$ has at most
$12$ elements, then
     $$
   \op{vol}(\Omega_{trunc}(\Lambda,v_0))\ge \op{vol}(\Omega(\Lambda_{dod},0)).
   $$
Equality holds exactly when $\Omega(\Lambda,v_0)$ is congruent to
$\Omega(\Lambda_{dod},0)$.
\end{lemma}

\begin{proof} (Sketch) By translation, we may assume that $v_0=0$.
For each nonzero $v$ in $\Lambda(v_0,2t_{dod})$, 
let $v'=2v/|v|$,We have $|v'|\le|v|$.
The Voronoi cell $\Omega'$ at the origin for $X = \{v'\mid v\in\Lambda(0,2t_{dod})\}$ is a subset of $\Omega(\Lambda,v_0)$ because face planes move to parallel planes closer to the origin.  
Set $\Omega'' = \Omega'\cap B(0,t_{dod})$.
Thus, it is enough to show
that 
   $$\op{vol}(\Omega'')\ge \op{vol}(\Omega(\Lambda_{dod},0))$$
and to analyze cases of equality.

As in Leech's solution to the kissing number problem, 
take the Delaunay triangulation of $X$ (now on the sphere of radius $2$).
For each triangle, there is a triple of points $\{v_1,v_2,v_3\}\subset X$ (dropping primes from the notation).
In Leech's proof, the area of the sphere
is a sum
of contributions from each face of the planar graph.
In this proof, the volume of $\Omega''$ is a sum of contributions
from each face of the planar graph.  Explicit formulas for the
contributions are developed later in this book.  They are as follows.
Let $t$ be the circumradius of $S=\{0,v_1,v_2,v_3\}$.  Set
   $$\op{vol}_{dod}(S) =
     \begin{cases}
       \op{volan}(0,S) & t < t_{dod}\\
       \op{vol}_{dod,trunc}(S) & t\ge t_{dod}\\
     \end{cases}
     $$
where 
  $$\op{vol}_{dod,trunc}(S) = \op{sv}_0(0,S,t_{dod},\lambda),\quad
     \lambda = (\lambda_v,\lambda_s) = (1,0).
  $$
The functions $\op{volan}$ and $\op{sv}_0$ 
are explicitly defined in Definitions~\ref{def:volan}
and~\ref{def:svor}. 
%% vol_dod is the basic quoin-Adih formula with t_dod truncation,
%% without regard for whether it is "geometric".
The sum of $\op{vol}_{dod}$ over all Delaunay triangles is 
the volume of $\Omega''$.

The function $\op{vol}_{dod}(\{0,v_1,v_2,v_3\})$ can be viewed as a function of three
variables $y_i = |v_j-v_k|$, for $(i,j,k)=(1,2,3),(2,3,1),(3,1,2)$.
The area $A$ of the Delaunay triangle can also be viewed as a function of
$y_1,y_2,y_3$.
We hold $y_1$ constant, vary $y_2$, and define $y_3$ as an implicit
function of $y_2$ on a level curve of $A(y_1,y_2,y_3)$.

By computing the derivative of $\op{vol}_{dod}(y_1,y_2,y_3)$ along
the level curve, we see that if $y_2 < y_3 \le y_1$, 
then $\op{vol}_{dod}$
is decreasing in $y_2$.  (This involves two calculations, one
when $t<t_{dod}$ and another when $t\ge t_{dod}$.)  Thus, we 
may decrease volume for constant Delaunay triangle areas by making
$y_1=y_2=y_3$.\endnote{The code for the Dodecahedral problem for $12$ balls
appears in a Mathematica Notebook at 
http://flyspeck.googlecode.com/svn/trunk/book\_code/sphereBook.nb.  This notebook supplements the
arguments in the text 
with calculations that show that the circumradius of the simplex
is properly behaved under these deformations.}
(The triangles no longer fit together, but this
does not matter.  Their areas still sum to $4\pi$.)

By computing the derivative of $\op{vol}_{dod}(y,y,y)+\op{vol}_{dod}(z,z,z)$
along a level curve of $A(y,y,y)+A(z,z,z)$, we see that if $y<z$,
then the volume may be decreased by increasing $y$ (and decreasing $z$).
(This involves three calculations, depending how the two circumradii
compare with $t_{dod}$.)  Thus, we may decrease volume by
making all triangles congruent and equilateral.  The areas of the
triangles still sum to $4\pi$.  

If $V$ is the cardinality
of $X$, then by the Euler formula the number $F$ 
of triangles is $(2V-4)$, which is at most $20$,
and each triangle has area $4\pi/F$.  Our bound on volume is
   $$
   F\, \op{vol}_{dod}(y,y,y), \text { where } F\, A(y,y,y)=4\pi.
   $$
Extend $F$ as a real-valued variable, and
let the area formula define $y$ as an implicit function of $F$.
The function $F\op{vol}_{dod}(y(F),y(F),y(F))$ 
is decreasing in $F$, so a lower bound is obtained
when $F$ is as large as possible.  The lower bound 
$20\,\op{vol}_{dod}(y(20),y(20),y(20))$ equals the volume of
$\Omega(\Lambda_{dod},0)$ (as the Delaunay triangulation
for $\Lambda_{dod}$ also consists of $20$
congruent triangles with total area $4\pi$).  
This proves the bound. 
All the derivatives involved in these
calculations are explicit.  Hence the case of equality is easily
traced.
\end{proof}

Fejes Toth's proof uses simple convexity arguments to prove an
apparently difficult result.  This method of proof has become
 is an essential part of the solution to the packing
problem.   Ferguson's thesis calculates the volume of a truncated
Voronoi cell, subject to a fixed area constraint 
\cite[sec.~16.9.5]{Fer97}.  In fact, his thesis provides details
to the sketch provided above, by giving the derivatives explicitly,
for truncated Voronoi cells.  Many of his calculations have been
reproduced in this book.  See, in particular, 
Lemma~\ref{lemma:quoin-equilize}.




\section{Classification of Tame Hypermaps}

\subsection{hypermap}

A hypermap is a tuple $(D,n,e,f)$, where $D$ is a finite
set, and $n,e,f$ are three permutations on that set that
compose to the identity:
$e\circ n\circ f = I$.  The elements of $D$ are called darts.
The permutations $n,e,f$ are called the node permutation,
edge permutation, and face permuation, respectively.

If $m$ is any permutation on $D$, we write $D/m$ for the
set of orbits in $D$ under $m$.  Similarly, if $G$ is any
group of permutations on $D$, we write $D/G$ for the set
of orbits of $D$ under $G$.  In particular, $D/\tangle{n,e,f}$
is the set of orbits under the group generated by $n,e,f$.

A planar graph gives  a hypermap
by the following procedure.  Starting with a planar graph,
place a dart at each angle.  That is, at a vertex of degree $k$,
place $k$ darts, one between each consecutive pair of edges.
The face permutation has a cycle for
each face of the planar graph and  traverses the
darts in a counterclockwise direction around each face.
The node permutation has a cycle for each vertex and traverses
the darts in a counterclockwise direction around each vertex.
The edge permutation is defined by the relation $e\circ n\circ f=I$.
It can be interpreted as an involution that pairs a dart next
to one endpoint of an edge with a dart at the other endpoint.
See Figure~\ref{XX}.

Hypermaps are the primary combinatorial object used by Gonthier
in the formalization of the four-color theorem in COQ~\ref{XX}.
Hypermaps, by being purely combinatorial, are more convenient
to represent on a computer than planar graphs.  (In the 1998
preprint, we encoded the combinatorics as planar maps rather
than hypermaps.  It is trivial to translating between the two notions.
Recent work on the formalization of the Kepler conjecture works
consistently with hypermaps; thus, we deprecate the use of
planar maps.  See~\cite{XXObua},\cite{XXBlue}.)

Not all hypermaps arise from a planar graph in this way.
Those that do have two special properties.  They are plain
and planar in the following sense.  (Note the surprising spelling
of `plain', particularly in the context of planar graphs!  
To avoid misunderstandings, 
we avoid the homophone `plane' in this article.)  The definition
of planar hypermap is the standard condition on the Euler
characteristic, translated into the language of hypermpas.

\begin{definition}\label{def:plain}

\begin{itemize}
\item The hypermap $(D,e,n,f)$ is plain, if $e$ is an involution:
$$
 e^2 = I.
$$.
\item The hypermap $(D,e,n,f)$ is planar, if
   $$
   \#(D/e) + \#(D/n) + \#(D/f) = \# D + 2\#(D/\tangle{e,n,f}).
   $$
\end{itemize}
\end{definition}



\section{Linear Programs}

This section discusses Theorem~\ref{thm:graph-system}.  It is one of
the main steps in the proof of the Dodecahedral conjecture.
We begin with a discussion of the terminology used in the
statement of the theorem.

The following definition is influenced by Obua's thesis~\cite{Ob}.

\begin{definition} A hypermap system is a pair $(H,\Phi)$,
where $H=(D,e,n,f)$ is a hypermap, and $\Phi$ is a finite set constraints on $H$.  More precisely, let $V$ be the vector space of
real-valued functions on $D$.  A finite set of constraints $\Phi$ is a finite
set of boolean valued functions $\phi:V^\ell\to \{\op{true},\op{false}\}$
for some $\ell$. (We assume $\ell$ is independent of $\phi\in \Phi$).
\end{definition}

We say that the hypermap system $(H,\Phi)$ is feasible, if
there is some $x=(x_1,\ldots,x_\ell)\in V^\ell$ such that
$\phi(x)$ holds for all $\phi\in\Phi$. Otherwise, we say that
the system is infeasible.

Appendix~\ref{XX} gives a finite set of constraints $\Phi$, 
specified
in a uniform way for every structurally tame hypermap $H$.  This determines,
for every structurally tame hypermap $H$, we then have a well-defined extension
to a hypermap system $(H,\Phi)$.  We call this particular hypermap
system, by structural analogy with the Kepler conjecture,
the {\it basic hypermap system}.  


\begin{theorem}\label{thm:graph-system}  Let 
$H$ be any structurally tame hypermap with corresponding
basic hypermap system $(H,\Phi)$.  Then $(H,\Phi)$ is infeasible.
\end{theorem}

This proof is carried out by computer as a collection of linear
programs.  This is one of the three major parts of the proof
of the Dodecahedral conjecture that has been carried out by computer.
In this article, we describe the relationship between the
feasibility of $(H,\Phi)$ and a linear programming feasibility
problem.  We also describe some details of the implementation of 
the code.

We have obtained an enumeration of all structurally tame
hypermap systems in Section~\ref{XX}.  Thus,  we may prove
Theorem~\ref{thm:graph-system} through a case-by-case treatment
of cases.  This is how we proceed.

Let's turn our attention for a moment, to one structurally tame hypermap $H=(D,e,n,f)$ and the corresponding basic hypermap system $(H,\Phi)$.
We describe the strategy that we use to show that it is infeasible.
For some $\ell\in\ring{N}$,
each constraint $\phi$ is a function on $V^\ell$, where $V$ is
the vector space of real-valued functions on $D$.  Thus,
$V^\ell$ can be identified with $\ring{R}^m$, where $m= \ell \#(D)$.
If we examine the form of the constraints $\phi\in \Phi$, as listed
in Appendix~\ref{XX}, we notice that the constraints all have a 
very special form.  With a minor qualification, they are all linear constraints
on $\ring{R}^m$.  

We allow a minor qualification to allow some of the constraints
to carry guard conditions.  That is, some constraints have the form
  \begin{equation}\label{eqn:guard}
  (A x < b)  \Rightarrow (A' x \le b'),
  \end{equation}
for $x\in\ring{R}^m$, and various matrices $A,A'$ and vectors
$b,b'$.  (We write $a \le b$ to mean
that $a_i\le b_i$ for every component of the vectors $a,b$.)
We call the constraint $(A x < b)$ a guard condition.
We allow variations in which some of the inequalities in the
guard condition are weak and some of the inequalities in the
consequent are strict.

The collection of all inequalities that do not have a guard 
condition is a system of linear inequalities.  Standard linear
programming packages can be used to determine whether this
system of linear inequalities has a feasible solution.  If this
linear program is infeasible, then the hypermap system $(H,\Phi)$
is clearly also infeasible.  When this happens, we have a
proof of infeasibility for $(H,\Phi)$.

When this fails, we turn to the constraints that carry guard
conditions.
The introduction of a constraint that has a nontrivial guard condition
involves multiple steps.  
The constraint (\ref{eqn:guard}) can be rewritten in logically
equivalent form as
  $$
   (A_{1} x \ge b_{1}) \lor \cdots \lor
   (A_{r} x \ge b_{r}) \lor (A_2 x \le b_2),
  $$
where $A_{i}$ and $b_{i}$ are the rows of $A$ and $b$.
Taking each disjunct in turn, one linear at a time
is added to the system
of linear inequalities, and the resulting system is shown to be
infeasible.  When each
systems are infeasible, then $(H,\Phi)$ itself is infeasible.

In general, more than one guarded constraint may be added.  If
$k$ constraints are added with $r_1,\ldots,r_k$ guards
respectively, then as many as $(r_1+1)\cdots (r_k+1)$ systems
of linear equalities are created.  If they are all infeasible,
then $(H,\Phi)$ itself is infeasible.

This discussion may give the impression 
that a great many linear programming feasibility
problems are created in this manner.  In practice, nearly all
of the hypermap systems are eliminated in the first pass, without
requiring recourse to the guarded conditions.  Only
$14$ cases (XX?) use a guard condition.




\section{Interval Arithmetic}

\section{Tarski Arithmetic}


\section{Appendix: Basic Hypermap System}

In this appendix, we list the system of constraints determining
the basic hypermap system.  As the main body of the text explains
a hypermap system is a pair $(H,\Phi)$, where $H$ is a hypermap
and $\Phi$ is a finite set of constraints on $H$.  To describe
the constraints $\Phi$ in precise terms requires some notation.

Let $H=(D,e,n,f)$ be a hypermap.  We use Greek letters
$\alpha,\beta \in D$ for darts.  Let $V$ be the vector space 
of real-valued  functions on $D$.  A constraint $\phi$ is a boolean-valued function on $V^\ell$ for some $\ell$.  We use a suggestive
notation $(\optt{sol},\optt{mu},\optt{tau},\optt{y},\ldots)\in V^\ell$ for $\ell$-tuples
of elements of $V$.  The main text relates each coordinate back
with its namesake.  For example, the linear constraints on
$\optt{sol}\in V$, mirror  nonlinear relations satisfied by the
nonlinear function $\sol$ in the main text.  This correspondence
between functions in $V$ and functions does not enter into the
definition of the basic hypermap system.  For the purposes of this
appendix, this correspondence is simply an aid to the
intuition.

If $m$ is  a permutation on $D$, let $\op{ord}(m,\alpha,i)$
be the predicate that asserts that the cardinality of the $m$-orbit
of $\alpha$ is $i\in\ring{N}$.
\bigskip

%XX Move to definitions.
\def\sland{\ \land\ }

\subsection{general bounds and relations}

$$
\begin{array}{lll}
\leftalign\\
   \forall \alpha\in D.\\
    2 \le \optt{yn}(\alpha) \sland
    \optt{yn}(\alpha) \le 2t_{dod} \\
    2 \le \optt{ye}(\alpha) \sland
    \optt{ye}(\alpha) \le 2t_{dod}\\
   0 \le \optt{sol}(\alpha) \sland
     \optt{sol}(\alpha) \le 4 \pi \\
   0 \le \optt{dih}(\alpha) \sland
     \optt{dih}(\alpha) \le 2\pi \\
   0 \le \optt{mu}(\alpha) \sland
      \optt{mu}(\alpha) \le 0.2 \\
   0 \le \optt{omega}(\alpha) \sland
      \optt{omega}(\alpha) \le XX \\
  \optt{yn}(\alpha) = \optt{yn}(n\alpha) \\
  \optt{ye}(\alpha) = \optt{ye}(e\alpha) \\
  \optt{omega}(\alpha) = \optt{omega}(f\alpha) \\
  \optt{mu}(\alpha) = \optt{mu}(f\alpha) \\
  \optt{sol}(\alpha)= \optt{sol}(f\alpha) \\
  \optt{mu}(\alpha) = \optt{omega}(\alpha) - 0.42755\ \optt{sol}(\alpha) \\

\end{array}
$$



$$
\begin{array}{lll}
\leftalign\\
\forall \alpha\in D.\\ 
   \optt{ord}(n,\alpha,k) \Rightarrow
   \sum_{i=1}^k \optt{dih}(n^i\alpha )  = 2\pi.\\
   \optt{ord}(f,\alpha,k) \Rightarrow
   \sum_{i=1}^k \optt{dih}(f^i\alpha ) = \optt{sol} + (k -2)\pi.\\
   \end{array}
$$
\noindent
Let $D'\subset D$ be a set of representatives of the faces in $D$.
$$
\begin{array}{lll}
\leftalign\\
\sum_{\alpha\in D'} \optt{omega}(\alpha) \le 5.5503 \\
\end{array}
$$
%\FIXX{Check the constant 5.5503 against what the LPs use.}
$$
\begin{array}{lll}
\leftalign\\
\forall \alpha\in D.\\
   \optt{ord}(f,\alpha,4) \Rightarrow
   \optt{mu}(\alpha) > 0.031 \\
   \optt{ord}(f,\alpha,5) \Rightarrow
   \optt{mu}(\alpha) > 0.076 \\
   \optt{ord}(f,\alpha,6) \Rightarrow
   \optt{mu}(\alpha) > 0.121 \\
   \optt{ord}(f,\alpha,7) \Rightarrow
   \optt{mu}(\alpha) > 0.166 \\
\end{array}
$$

\FIXX{Add the inequalities corresponding to Obua page 71,
the multiface inequalities, see page 14 McL 2003, Lemma 5.2,
Theorem 8.1 vertex adjustments.  Inequalities for Guards page 65.
}



\subsection{tetrahedral constraints}

\noindent
Function bounds:
$$
\begin{array}{lll}
\leftalign\\
  \forall \alpha\in D. \ \op{ord}(f,\alpha,3) 
   \Rightarrow \\ 
      \optt{omega}(\alpha) > 0.202804  \sland \\
   \optt{sol}(\alpha) > 0.315696 \sland \\
   \sol(S) < 1.051232\\
   \optt{dih}(\alpha) > 0.856147\\
   \optt{dih}(\alpha) < 1.886730\\
   \optt{mu}(\alpha) > 0
\end{array}
$$

$$
\begin{array}{lll}
\leftalign\\
\forall\alpha\in D. \ \op{ord}(f,\alpha,3) \Rightarrow \\
   \optt{omega}(\alpha) - 0.68\ \optt{sol}(\alpha) + 1.88718\ \optt{dih}(\alpha) > 1.54551 \\
   \optt{omega}(\alpha) - 0.68\ \optt{sol}(\alpha) + 0.90746\ \optt{dih}(\alpha) > 0.706725\\
   \optt{omega}(\alpha) - 0.68\ \optt{sol}(\alpha) + 0.46654\ \optt{dih}(\alpha) > 0.329233\\
   \optt{omega}(\alpha) - 0.55889\ \optt{sol}(\alpha) - 0\ \optt{dih}(\alpha) > -0.0736486\\
   \optt{omega}(\alpha) - 0.63214\ \optt{sol}(\alpha) - 0\ \optt{dih}(\alpha) > -0.13034\\
   \optt{omega}(\alpha) - 0.73256\ \optt{sol}(\alpha) - 0\ \optt{dih}(\alpha) > -0.23591\\
   \optt{omega}(\alpha) - 0.89346\ \optt{sol}(\alpha) - 0\ \optt{dih}(\alpha) > -0.40505\\
   \optt{omega}(\alpha) - 0.3\ \optt{sol}(\alpha) - 0.5734\ \optt{dih}(\alpha) > -0.978221\\
   \optt{omega}(\alpha) - 0.3\ \optt{sol}(\alpha) - 0.03668\ \optt{dih}(\alpha) > 0.024767\\
   \optt{omega}(\alpha) - 0.3\ \optt{sol}(\alpha) + 0.04165\ \optt{dih}(\alpha) > 0.121199\\
   \optt{omega}(\alpha) - 0.3\ \optt{sol}(\alpha) + 0.1234\ \optt{dih}(\alpha) > 0.209279\\
   \optt{omega}(\alpha) - 0.42755\ \optt{sol}(\alpha) - 0.11509\ \optt{dih}(\alpha) > -0.171859\\
   \optt{omega}(\alpha) - 0.42755\ \optt{sol}(\alpha) - 0.04078\ \optt{dih}(\alpha) > -0.050713\\
   \optt{omega}(\alpha) - 0.42755\ \optt{sol}(\alpha) + 0.11031\ \optt{dih}(\alpha) > 0.135633\\
   \optt{omega}(\alpha) - 0.42755\ \optt{sol}(\alpha) + 0.13091\ \optt{dih}(\alpha) > 0.157363\\
   \optt{omega}(\alpha) - 0.55792\ \optt{sol}(\alpha) - 0.21394\ \optt{dih}(\alpha) > -0.417998\\
   \optt{omega}(\alpha) - 0.55792\ \optt{sol}(\alpha) - 0.0068\ \optt{dih}(\alpha) > -0.081902\\
   \optt{omega}(\alpha) - 0.55792\ \optt{sol}(\alpha) + 0.0184\ \optt{dih}(\alpha) > -0.051224\\
   \optt{omega}(\alpha) - 0.55792\ \optt{sol}(\alpha) + 0.24335\ \optt{dih}(\alpha) > 0.193993\\
   \optt{omega}(\alpha) - 0.68\ \optt{sol}(\alpha) - 0.30651\ \optt{dih}(\alpha) > -0.648496\\
   \optt{omega}(\alpha) - 0.68\ \optt{sol}(\alpha) - 0.06965\ \optt{dih}(\alpha) > -0.278\\
   \optt{omega}(\alpha) - 0.68\ \optt{sol}(\alpha) + 0.0172\ \optt{dih}(\alpha) > -0.15662\\
   \optt{omega}(\alpha) - 0.68\ \optt{sol}(\alpha) + 0.41812\ \optt{dih}(\alpha) > 0.287778\\
   \optt{omega}(\alpha) - 0.64934\ \optt{sol}(\alpha) - 0\ \optt{dih}(\alpha) > -0.14843\\
   \optt{omega}(\alpha) - 0.6196\ \optt{sol}(\alpha) - 0\ \optt{dih}(\alpha) > -0.118\\
   \optt{omega}(\alpha) - 0.58402\ \optt{sol}(\alpha) - 0\ \optt{dih}(\alpha) > -0.090290\\
   \optt{omega}(\alpha) - 0.25181\ \optt{sol}(\alpha) - 0\ \optt{dih}(\alpha) > 0.096509\\
   \optt{omega}(\alpha) - 0.00909\ \optt{sol}(\alpha) - 0\ \optt{dih}(\alpha) > 0.199559\\
   \optt{omega}(\alpha) + 0.93877\ \optt{sol}(\alpha) - 0\ \optt{dih}(\alpha) > 0.537892\\
   \optt{omega}(\alpha) + 0.93877\ \optt{sol}(\alpha) - 0.20211\ \optt{dih}(\alpha) > 0.27313\\
   \optt{omega}(\alpha) + 0.93877\ \optt{sol}(\alpha) + 0.63517\ \optt{dih}(\alpha) > 1.20578\\
   \optt{omega}(\alpha) + 1.93877\ \optt{sol}(\alpha) - 0\ \optt{dih}(\alpha) > 0.854804\\
   \optt{omega}(\alpha) + 1.93877\ \optt{sol}(\alpha) - 0.20211\ \optt{dih}(\alpha) > 0.621886\\
   \optt{omega}(\alpha) + 1.93877\ \optt{sol}(\alpha) + 0.63517\ \optt{dih}(\alpha) > 1.57648\\
   \optt{omega}(\alpha) - 0.42775\ \optt{sol}(\alpha) - 0\ \optt{dih}(\alpha) > -0.000111\\
   \optt{omega}(\alpha) - 0.55792\ \optt{sol}(\alpha) - 0\ \optt{dih}(\alpha) > -0.073037\\
   \optt{omega}(\alpha) - 0\ \optt{sol}(\alpha) - 0.07853\ \optt{dih}(\alpha) > 0.08865\\
   \optt{omega}(\alpha) - 0\ \optt{sol}(\alpha) - 0.00339\ \optt{dih}(\alpha) > 0.198693\\
   \optt{omega}(\alpha) - 0\ \optt{sol}(\alpha) + 0.18199\ \optt{dih}(\alpha) > 0.396670\\
   \optt{omega}(\alpha) - 0.42755\ \optt{sol}(\alpha) - 0.2\ \optt{dih}(\alpha) > -0.332061\\
   \optt{omega}(\alpha) - 0.3\ \optt{sol}(\alpha) - 0.36373\ \optt{dih}(\alpha) > -0.58263\\
   \optt{omega}(\alpha) - 0.3\ \optt{sol}(\alpha) + 0.20583\ \optt{dih}(\alpha) > 0.279851\\
   \optt{omega}(\alpha) - 0.3\ \optt{sol}(\alpha) + 0.40035\ \optt{dih}(\alpha) > 0.446389\\
   \optt{omega}(\alpha) - 0.3\ \optt{sol}(\alpha) + 0.83259\ \optt{dih}(\alpha) > 0.816450\\
   \optt{omega}(\alpha) - 0.42755\ \optt{sol}(\alpha) - 0.51838\ \optt{dih}(\alpha) > -0.932759\\
   \optt{omega}(\alpha) - 0.42755\ \optt{sol}(\alpha) + 0.29344\ \optt{dih}(\alpha) > 0.296513\\
   \optt{omega}(\alpha) - 0.42755\ \optt{sol}(\alpha) + 0.57056\ \optt{dih}(\alpha) > 0.533768\\
   \optt{omega}(\alpha) - 0.42755\ \optt{sol}(\alpha) + 1.18656\ \optt{dih}(\alpha) > 1.06115\\
   \optt{omega}(\alpha) - 0.55792\ \optt{sol}(\alpha) - 0.67644\ \optt{dih}(\alpha) > -1.29062\\
   \optt{omega}(\alpha) - 0.55792\ \optt{sol}(\alpha) + 0.38278\ \optt{dih}(\alpha) > 0.313365\\
   \optt{omega}(\alpha) - 0.55792\ \optt{sol}(\alpha) + 0.74454\ \optt{dih}(\alpha) > 0.623085\\
   \optt{omega}(\alpha) - 0.55792\ \optt{sol}(\alpha) + 1.54837\ \optt{dih}(\alpha) > 1.31128\\
   \optt{omega}(\alpha) - 0.68\ \optt{sol}(\alpha) - 0.82445\ \optt{dih}(\alpha) > -1.62571\\
\end{array}
$$

$$
\begin{array}{lll}
\leftalign\\
\forall\alpha\in D. \ \op{ord}(f,\alpha,3) \Rightarrow \\
  \optt{sol}(\alpha) > 0.551285 - 0.245 (y_1+y_2+y_3-6) + 0.063 (y_4+y_5+y_6-6)\\
  \optt{sol}(\alpha) > 0.551285 - 0.3798 (y_1+y_2+y_3-6) + 0.198 (y_4+y_5+y_6-6)\\
  \optt{sol}(\alpha) < 0.551286 - 0.151 (y_1+y_2+y_3-6) + 0.323 (y_4+y_5+y_6-6)\\

  \optt{mu}(\alpha) > 0.0392 (y_1+y_2+y_3-6) + 0.0101 (y_4+y_5+y_6-6) \\
  \optt{omega}(\alpha) > 0.235702 -0.107 (y_1+y_2+y_3-6) + 0.116 (y_4+y_5+y_6-6)\\
  \optt{omega}(\alpha) > 0.235702 -0.0623 (y_1+y_2+y_3-6) + 0.0722 (y_4+y_5+y_6-6)\\
  \optt{dih}(\alpha) > 1.23095 + 0.237 (y_1-2) - 0.372 (y_2+y_3+y_5+y_6-8) + 0.708 (y_4-2) \\
  \optt{dih}(\alpha) > 1.23095 + 0.237 (y_1-2) - 0.363 (y_2+y_3+y_5+y_6-8) + 0.688 (y_4-2)\\
  \optt{dih}(\alpha) < 1.23096 + 0.505 (y_1-2) - 0.152(y_2+y_3+y_5+y_6-8) + 0.766 (y_4-2)\\

\end{array}
$$


$$
\begin{array}{lll}
\leftalign\\
\forall\alpha\in D. \ \op{ord}(f,\alpha,4) \Rightarrow \\
   \optt{omega}(\alpha) > 0.455149\\
   \optt{sol}(\alpha) > 0.731937\\
   \optt{sol}(\alpha) < 2.85860\\
   \optt{dih}(\alpha) > 1.15242\\
   \optt{dih}(\alpha) < 3.25887\\
   \optt{mu}(\alpha) > 0.031350\\
\end{array}
$$


$$
\begin{array}{lll}
\leftalign\\
\forall\alpha\in D. \ \op{ord}(f,\alpha,4) \Rightarrow \\
   \optt{omega}(\alpha) - 0.42775\ \optt{sol}(\alpha) - 0.15098\ \optt{dih}(\alpha) > -0.3670\\
   \optt{omega}(\alpha) - 0.42775\ \optt{sol}(\alpha) - 0.09098\ \optt{dih}(\alpha) > -0.1737\\
   \optt{omega}(\alpha) - 0.42775\ \optt{sol}(\alpha) - 0.00000\ \optt{dih}(\alpha) > 0.0310\\
   \optt{omega}(\alpha) - 0.42775\ \optt{sol}(\alpha) + 0.18519\ \optt{dih}(\alpha) > 0.3183\\
   \optt{omega}(\alpha) - 0.42775\ \optt{sol}(\alpha) + 0.20622\ \optt{dih}(\alpha) > 0.3438\\
   \optt{omega}(\alpha) - 0.55792\ \optt{sol}(\alpha) - 0.30124\ \optt{dih}(\alpha) > -1.0173\\
   \optt{omega}(\alpha) - 0.55792\ \optt{sol}(\alpha) - 0.02921\ \optt{dih}(\alpha) > -0.2101\\
   \optt{omega}(\alpha) - 0.55792\ \optt{sol}(\alpha) - 0.00000\ \optt{dih}(\alpha) > -0.1393\\
   \optt{omega}(\alpha) - 0.55792\ \optt{sol}(\alpha) + 0.05947\ \optt{dih}(\alpha) > -0.0470\\
   \optt{omega}(\alpha) - 0.55792\ \optt{sol}(\alpha) + 0.39938\ \optt{dih}(\alpha) > 0.4305\\
   \optt{omega}(\alpha) - 0.55792\ \optt{sol}(\alpha) + 2.50210\ \optt{dih}(\alpha) > 2.8976\\
   \optt{omega}(\alpha) - 0.68000\ \optt{sol}(\alpha) - 0.44194\ \optt{dih}(\alpha) > -1.6264\\
   \optt{omega}(\alpha) - 0.68000\ \optt{sol}(\alpha) - 0.10957\ \optt{dih}(\alpha) > -0.6753\\
   \optt{omega}(\alpha) - 0.68000\ \optt{sol}(\alpha) - 0.00000\ \optt{dih}(\alpha) > -0.4029\\
   \optt{omega}(\alpha) - 0.68000\ \optt{sol}(\alpha) + 0.86096\ \optt{dih}(\alpha) > 0.8262\\
   \optt{omega}(\alpha) - 0.68000\ \optt{sol}(\alpha) + 2.44439\ \optt{dih}(\alpha) > 2.7002\\
   \optt{omega}(\alpha) - 0.30000\ \optt{sol}(\alpha) - 0.12596\ \optt{dih}(\alpha) > -0.1279\\
   \optt{omega}(\alpha) - 0.30000\ \optt{sol}(\alpha) - 0.02576\ \optt{dih}(\alpha) > 0.1320\\
   \optt{omega}(\alpha) - 0.30000\ \optt{sol}(\alpha) + 0.00000\ \optt{dih}(\alpha) > 0.1945\\
   \optt{omega}(\alpha) - 0.30000\ \optt{sol}(\alpha) + 0.03700\ \optt{dih}(\alpha) > 0.2480\\
   \optt{omega}(\alpha) - 0.30000\ \optt{sol}(\alpha) + 0.22476\ \optt{dih}(\alpha) > 0.5111\\
   \optt{omega}(\alpha) - 0.30000\ \optt{sol}(\alpha) + 2.31852\ \optt{dih}(\alpha) > 2.9625\\
   \optt{omega}(\alpha) - 0.23227\ \optt{dih}(\alpha) > -0.1042\\
   \optt{omega}(\alpha) + 0.07448\ \optt{dih}(\alpha) > 0.5591\\
   \optt{omega}(\alpha) + 0.22019\ \optt{dih}(\alpha) > 0.7627\\
   \optt{omega}(\alpha) + 0.80927\ \optt{dih}(\alpha) > 1.5048\\
   \optt{omega}(\alpha) + 5.84380\ \optt{dih}(\alpha) > 7.3468\\
\end{array}
$$




\subsection{guard conditions}

\noindent
Condition H.14.1:  % Use alpha= y1.
$$
\begin{array}{lll}
\leftalign\\
\forall \alpha\in D.\\
  \op{ord}(f,\alpha,5) \sland 
  \optt{dih}(\alpha) < 1.342 \sland
  \optt{dih}(f^3\alpha) < 1.684 \sland
  \optt{yn}(\alpha) < 2.153 \sland \\
   \optt{y}(f\alpha) < 2.174 \sland
  \optt{yn}(f^2\alpha) < 2.26 \sland
  \optt{yn}(f^3 \alpha) < 2.194 \sland
  \optt{yn}(f^4 \alpha) < 2.314  \\
  \Rightarrow
  \optt{omega}(\alpha) > 0.950
\end{array}
$$

\noindent
Condition H.14.2:  (XX Potential problem, in H.14.1, the
condition is $y(10)< 2.26$, the index changes in H.14.2
to $y(5) > 2.26$.)
$$
\begin{array}{lll}
\leftalign\\
\forall \alpha\in D.\\
  \op{ord}(f,\alpha,5) \sland 
  \optt{dih}(\alpha) < 1.342 \sland
  \optt{dih}(f^3\alpha) < 1.684 \sland
  \optt{y}(\alpha) < 2.153 \sland \\
   \optt{y}(f\alpha) < 2.174 \sland
  \optt{y}(f^2\alpha) \ge 2.26 \sland
  \optt{y}(f^3 \alpha) < 2.194 \sland
  \optt{y}(f^4 \alpha) < 2.314  \\
   \Rightarrow
  \optt{mu}(\alpha) > 0.1234
\end{array}
$$
