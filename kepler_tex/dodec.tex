
\section{Introduction}

\label{sec:dodec}

A packing of congruent unit radius balls 
in Euclidean space determines a region called the Voronoi cell 
around each ball.  
The packing is determined by and is identified  with the set $\Lambda$ of centers of the balls.  The Voronoi
cell $\Omega(\Lambda,v)$ around a ball at $v\in \Lambda$ 
consists of points of space that are closer to its center than
to any other ball center in the packing.  The Voronoi cell is a
convex polyhedron containing $v$.

The dodecahedral theorem asserts that in any packing of congruent balls of Euclidean
space every Voronoi cell has volume at least that of a regular dodecahedron
circumscribing a unit ball.    This bound is realized by a finite 
packing $\Lambda_{dod}$
(of twelve balls and a thirteenth  at the origin) obtained
by placing a ball at the center of each face of a regular dodecahedron.  The
theorem can then be stated as the inequality
  $$
  \op{vol}(\Omega(\Lambda,v)) \ge \op{vol}(\Omega(\Lambda_{dod},0))
  $$
for every $v\in\Lambda$, and for every set of points $\Lambda\subset \ring{R}^3$
whose pairwise distances are at least the diameter $2$.
The case of equality occurs exactly when $\Omega(\Lambda,v)$ is
congruent to a regular dodecahedron of inradius $1$.



\subsection{history}


L. Fejes T\'oth made the conjecture in 1943.  In that article, L. Fejes
T'oth sketches a proof based on an unproved hypothesis.  This
hypothesis is an explicit version of the kissing number problem
in three dimensions.   Many decades later, this unproved hypothesis
is now generally regarded as being 
nearly as difficult as the full dodecahedral conjecture.  

L. Fejes T\'oth returned to the dodecahedral conjecture in a number
of publications.  It is a prominent part of his two books \cite{FTXX},
\cite{FTXX}.  According to the strategy of \cite{FTXX}, 
the Dodecaehdral conjecture forms a
step towards the solution of the sphere packing problem
(discussed below).   In \cite{FTXX}, he proved that the Dodecahedral conjecture
holds for every Voronoi cell with at most twelve facets.  
This result is reviewed in Section~\ref{XX}.  It
is an ingredient in our proof. 


In early 1990s, 
W.-Y. Hsiang claimed a proof of the dodecahedral conjecture (and various
other conjectures).  These claims did not hold up to careful examination
\cite{XX Status}.  
The first version of this paper \cite{arx} gives a counterexample to the approach to the Dodecahedral conjecture described in \cite{KBezdek}.
[XX COPY FROM 2003 VERSION.]


S. McLaughlin gave the first proof in 1998, 
by S. McLaughlin in 1998 \cite{McL98} while still an undergraduate, 
under the supervision of T. Hales.

K. Bezdek conjectures that the surface area of any Voronoi cell in a packing
of unit balls is at least that of a regular dodecahedron of inradius $1$.
This strengthened version of the dodecahedral problem is still
open \cite{Bez04}.   



\subsection{sphere packing problem}

The Kepler conjecture, also known as the sphere packing problem, 
asserts that no packing of congruent balls
in three dimensions has density greater than the density of the
face-centered cubic packing.  
The research for the proof of the Dodecahedral Conjecture 
was carried out at the University
of Michigan at the same time that S. Ferguson and T. Hales were
carrying out their work on the sphere packing problem.  Both
problems were solved in 1998.  

There is no strict logical connection between the two problems.
The Dodecahedral conjecture does not follow from the Kepler conjecture
and is not an intermediate step in the solution to the Kepler conjecture.
(In Fejes T\'oth strategy, it was an intermediate step; however, that
strategy was not followed in the solution of the sphere packing problem.)
Nevertheless,
the two solutions follow a similar outline and share a significant number of 
methods.
Both are based 
on long computer calculations.  Computer
code was freely exchanged between the two projects.

This article is written in a way thta it is not necessary to
read or understand the solution of the sphere packing problem before
reading this article.  However, at various points in the proof
of the Dodecahedral conjecture, we point out 
parallels with the sphere packing problem.  We also cite various results
from that proof.



\subsection{differences}

Although the proof of the dodecahedral theorem runs parallel
to the solution to the packing problem, there are special
difficulties that arise in the proof of the dodecahedral theorem.
In no sense is it a corollary of the packing problem.
In the packing problem, there turn out to be many ways to
reduce the infinite ball problem to a problem about finite clusters
of balls.  Through this multiplicity of reductions, it is
possible to design many difficulties away.  If one reduction is
not satisfactory, one can walk away from it and work with
another.
With the dodecahedral problem, there is no such flexibility.
The problem about finite clusters of balls is fixed from the outset.


\subsection{ten years later}

Over ten years have elapsed from the completion of the research until 
publication.  A few words of explanation are in order.
The review and 
publication process for the Kepler conjecture extended from 1998
until 2006.  Because of significant sharing between the Kepler conjecture
and the Dodecahedral conjecture, the editors adopted a ``wait and see''
attitude toward the Dodecahedral conjecture.  Once the Kepler conjecture
was published, the path became clear for the publication of Dodecahedral
conjecture.  

The details of the proof in the current publication are essentially the
same as those in the preprint posted on the arXiv in 1998.  No
significant errors have emerged in the original 1998 preprint.   However,
the paper has undergone three rewrites since then, an expanded version in 2002, followed by an abridged version in 2003, then a newly expanded version in 2008, all coming at the request by the editors.  It was as if a complex computer proof caused a discomfort that might be relived by a long series of
rewrites. The computer code has
also been entirely rewritten.

A formalization project, called Flyspeck, aims to
provide a complete formalization of the proof of the Kepler conjecture.
A parallel project, called Flyspeck Light, aims to do the same
for the proof of the Dodecahedral conjecture.  These long-term projects
will take many years to complete.  Nevertheless, significant progress
has already been made toward the formal verification of the computer
code \cite{BN}, \cite{O}.   This revisions in this paper 
incorporate the parts of Flyspeck Light that have
already been completed.

\subsection{truncation}

The distance from the center of the regular dodecahedron to a vertex is
$t_{dod}=\sqrt{3}\tan(\pi/5)\approx 1.258$.   This parameter is used to truncate
Voronoi cells; it makes  volumes easier to estimate.  A similar truncation
takes place in the solution to the packing problem with truncation parameter
$t_0 = 1.255$.  It is a happy coincidence that these two truncation parameters 
are so close to one another.  (Section~\ref{sec:experiment} 
gives the reasons behind the parameter $t_0$.)  A great deal of duplicated effort
might have been avoided if these two parameters were equal.  However, the parameter
$t_{dod}$ cannot be replaced with anything smaller, 
and although the parameter $t_0$ could 
easily have been made larger, 
its value was already too deeply entrenched in published papers 
by the time McLaughlin started work
on the dodecahedral conjecture.  

As a result, there are several definitions in the proof of the dodecahedral theorem
and in the solution to the packing problem, identical in every respect except
for the choice of parameter: $t_0,t_{dod}$.  This is indeed unfortunate.  However,
there is no easy remedy. 



 As a first step towards unifying the proofs,
 many results can be stated in a form that holds for all $t\in[t_0,t_{dod}]$.
To transfer a lemma from \cite{XX} to this article, a simple process is
involved.  The first step is generalization, replacing the constant $t_0$
with a free parameter $t\in[t_0,t_{dod}]$.  The second is specialization,
$t\mapsto t_{dod}$.  

The number $t_0$, although rational, 
can be consistently treated as an independent real transcendental 
in the solution to the packing problem; that is,
none of the proofs involving $t_0$ rely on its exact numerical value.
The constant $t_0$ can always be replaced by a constant $t$
in a suitably small interval about $t_0$.  However, the only way to
know that this small interval is wide enough to contain $t_{dod}$ is to study the
details of the proof.  For this article, 
we have made a detailed study of each relevant proof.

As a matter of terminology, we will say a proposition for the dodecahedral theorem
is a $[t_0,t_{dod}]$-perturbation of a proposition
in the solution of the packing problem,
if it is obtained by mindlessly replacing $t_0$ with $t_{dod}$, wherever that constant appears, and if the proof goes through verbatim with this minor change.
When this occurs, 
there is nothing to further to be learned by repeating the proof,
and we refer the reader to \cite{XX} for the proof.

\subsection{terminology}

Various notation and terminology is shared between the solution
of the sphere packing problem and this article.  

A large technical vocabulary has been developed 
to describe the solution to the sphere packing problem.  
Vocabulary can be imported frm the sphere packing problem three
different ways.  The simplest way to import a term is for the
term to have precisely the same meaning in both places.  For
example, the terms ``orientation,'' ``packing,'' have the
same meaning in both places. 

The second way to import a term is by making a $[t_0,t_{dod}]$-perturbation of a term that depends on the parameter $t_0$.  For example,
a in the solution of the sphere packing problem, a {\it quasi-regular
tetrahedron} is defined to be a set $S\subset\ring{R}^3$ of four
points forming a packing, such that
   $$
   \forall\ v,\ w \in S.\ |v - w| \le 2t_0.
   $$
That is, the edge lengths vary between $2$ and $2t_0$.
The $[t_0,t_{dod}]$-perturbation of this definition is a set $S\subset\ring{R}^3$
of four points forming a packing, such that
   $$
   \forall\ v,\ w \in S.\ |v - w| \le 2t_{dod}.
   $$

What to call such a set?  In the original version of this article,
this perturbed definition is again called a quasi-regular tetrahedron. 
A referee writes, ``Invent a new name
for `quasi-regular tetrahedron'; its definition is set in stone
in the Kepler conjecture papers.  This is a new definition\ldots'' 
This observation is entirely correct.  The referee's comment applies not
just to a solitary term `quasi-regular tetrahedron,' but to
a large technical vocabulary that has been developed to describe
the geometry of sphere packings.  The mastery of this specialized
vocabulary already places a mild burden on the reader.  The mastery
of two specialized vocabularies, one a $[t_0,t_{dod}]$-perturbation
of the other, would double that burden.  A new term,
such as pseudo-quasi-regular, would lead to a further burden
as the reader ascertains the exact relationship between
pseudo-quasi and quasi.  All this we wish to avoid.

Since this vocabulary is highly specialized for the proof of these
two problems in sphere packings, it is highly unlikely for these terms to find applications outside this problem domain.  They are not 
general fundamental notions such as set, manifold, or group, where
a large mathematical literature depends on a uniform definition.
The main danger of using the same term with a minor variation
in meaning is that a reader might improperly mix results from the
two articles.


The compromise that we have settled upon in this paper is to
retain the same technical vocabulary that is used in the solution
to the sphere packing problem, but to remind the reader regularly
of any variations in meaning that occur.  We ask the reader to
imagine a distinguishing subscript for each occurrence of {\it quasi-regular tetrahedron$_{dod}$}
and 
{\it quasi-regular tetrahedron$_0$} in the two articles.  
Every occurrence within the article carries the same subscript;
the two variants never mix; Subscript $dod$ in this article, 
and subscript $0$
there.
For the convenience
of the reader, we give the following complete 
list of definitions and constructions in this
paper that are $[t_0,t_{dod}]$-perturbations of those in the sphere
packing problem.

\begin{itemize}
\item quasi-regular tetrahedron, standard region, 
\end{itemize}

The third way in which terms have been imported into this proof
from the proof of the Kepler conjecture has been by structural
analogy.  For example, an important step in the solution to the
sphere packing problem is a classification of all tame planar graphs.
The term {\it tame graph} is a technical notion that arises in
the solution of the sphere packing problems related to the structure
of potential counterexamples to the conjecture. In this article,
there is an analogous classification of planar graphs, this time
related to potential counterexamples to the Dodecahedral conjecture.
The planar graphs that arise here are not tame and they are
not even a $[t_0,t_{dod}]$-perturbation of the definition of tame.  
Nevertheless, a strong parallel exists between tame planar graphs 
and this collection of graphs.  In fact, the same piece of computer
code was used to carry out both classifications. 

What to call them?  In this case, it would be too much of a stretch
to keep an identical name for this new type of planar graph.
In these cases, when we import the term, we modify it with
the adjective {\it structural}, to remind us that the way it functions 
bears a structural
analogy with the solution to the sphere
packing problem.  Thus, in this paper, we classify all
{\it structurally tame planar graphs}.

For the convenience of the reader, here is a list of terms that
have been imported by analogy, and qualified with the adjective
{\it structural} or adverb {\it structurally}.

\begin{itemize}
\item tame, squander, contravening, basic
\end{itemize}


\subsection{formulation}

This article proves the Dodecahedral conjecture in a slightly
stronger version than that stated in the abstract.  It proves
that a certain truncation of the Voronoi cell already has volume
at least as great as that of the regular dodecahedron.  This
section describes the truncation and states the stronger version
of the main theorem in a precise form.

Let $\Lambda$ be a packing and let $v_0\in\Lambda$.
Let $\Lambda(v_0,r)$ be the set of points of $\Lambda$ at distance at most $r$ from $v_0$.
Let $B(v,r)$ be an open ball
of radius $r$ centered at $v$.  

Let $\CalQ_{dod}(\Lambda,v_0)$ 
be the set of all $S=\{v_0,v_1,v_2,v_3\}\subset\Lambda(v_0,2t_{dod})$
consisting of four distinct points such that $|v_i-v_j|\le 2t_{dod}$ for all $i,j$
and such that the circumradius of each triangle $\{v_i,v_j,v_k\}\subset S$ is at most
$\sqrt2$.  A simple lemma (the dodecahedral analogue of Claim~\ref{thm:nonoverlap})
shows that the interiors of the convex hulls of distinct $S,S'\in\CalQ_{dod}(\Lambda,v_0)$
are disjoint. Write $\op{conv}(S)$ for the convex hull of $S\in\CalQ_{dod}(\Lambda,v_0)$.

Define a truncation $\Omega_{trunc}(\Lambda,v_0)$ 
of the Voronoi cell $\Omega(\Lambda,v_0)$:
   $$
   \{x \in \Omega(\Lambda,v_0) \mid   x\in B(v_0,t_{dod}) \quad\text{\bf  or }\quad x \in \op{conv}(S)
     \text{ for some } S\in \CalQ_{dod}(\Lambda,v_0) \}. 
   $$
That is, we truncate the Voronoi cell by intersecting it with a ball of radius $t_{dod}$,
except inside regions protected by the sets $S\in\CalQ_{dod}(\Lambda,v_0)$.
Note that for the special packing $\Lambda_{dod}$, we have 
$\Omega_{trunc}(\Lambda_{dod},0) = \Omega(\Lambda_{dod},0)$.

The strong form of the dodecahedral theorem is the following.

\begin{theorem}
For every packing $\Lambda$ and every $v_0\in\Lambda$,
   $$
   \op{vol}(\Omega_{trunc}(\Lambda,v_0))\ge \op{vol}(\Omega(\Lambda_{dod},0)).
   $$
Equality holds exactly when $\Omega(\Lambda,v_0)$ is congruent to
$\Omega(\Lambda_{dod},0)$.
\end{theorem}

Let us compare what we have done so far with what we will do in
 the solution of the packing problem. In Definition~\ref{def:q-system}
 a collection of ``protected simplices''
$\CalQ(\Lambda,v_0)$ is defined.  A modification of the Voronoi cell (called the
$V$-cell) is defined.  A truncation of the $V$-cell is defined.
The main step in the solution of the packing problem
is to show that the volume of a truncated $V$-cell, with modifications coming from
$\CalQ(\Lambda,v_0)$, is always at least as great as the volume of a rhombic dodecahedron.
So far the parallel is strong.






\section{Proof Outline}

Completely different methods treat the case when $\Lambda(v_0,2t_{dod})\setminus\{v_0\}$ contains at least $13$ points.  This part of the proof is
much more difficult than the case treated by Fejes T\'oth.
Although the details are
entirely different, the top-level description of the proof of the
dodecahedral conjecure sounds strickingly similar to the top-level
description of the solution to the packing problem.  Here
we describe the common top-level structure.  We use substantially
the same language that was used to give a summary of Leech's proof
of the kissing number problem in three dimensions in Section~\ref{sec:summary}.

Let $\Lambda$ be a packing, fix $v_0\in\Lambda$.
Let $t$ be the truncation
parameter, which $t_{dod}=1.258\ldots$ here, and  $t_0=1.255$ for the
packing problem.  We seek to minimize the objective function 
$$
\op{vol}(\Omega_{trunc}(\Lambda,v_0)).
$$
The target value for the minimization is $b=\op{vol}(\Omega(\Lambda_{dod},0))$.

As in Leech's proof, a planar graph is used to partition the
unit sphere.  It is formed as follows.  Consider all sets
with three elements $\{v_0,v_1,v_2\}\subset \Lambda(v_0,2t)$
such that $|v_1-v_2|\le 2t$.  Form a {\it blade}:
   $$\{v_0 + s_1 (v_1-v_0) + s_2 (v_2-v_0) \mid s_1,s_2\ge 0\}$$
from each triple $\{v_0,v_1,v_2\}$.  The complement of the union of
blades breaks into finitely many connected componenents  $U_1,\ldots,U_r$
called {\it standard components}.    The combinatorial structure of
the blades determines a planar graph (or more precisely a hypermap; see Definition~\ref{def:hypermap}).  When the graph is connected, the
set of standard components is in bijection with the set of faces of the
planar graph (Lemma~\ref{lemma:hypermap-planar}).  For simplicity assume that the graph
is connected.


An objective function is defined.  The objective function is a sum
of contributions from each face of the planar graph.  For the dodecahedral
problem, these contributions
are defined as $$\op{vol}(\Omega_{trunc}(\Lambda,v_0)\cap U_i).$$

A computer calculates bounds
on the contribution from each face.  In an assembly phase of the proof, 
the target total contribution $b$
is compared
with to the sum of bounds assembled from different faces. This comparison places
strong combinatorial restrictions on the planar graph.  A graph that satisfies these restrictions is called a tame planar graph.  (There is one class of tame planar graphs for the dodecahedral problem and a second class of graphs for the packing problem. Calling them by the same name is an abuse of language.)  A computer classification of all 
tame planar graphs shows what possibilities remain.   

Case-by-case analysis excludes the final possibilities.  A linear program is
attached to each tame planar graph.  This linear program is a linear relaxation
of a  nonlinear optimization problem: minimize the objective function
 constrained to having the combinatorics of the given planar graph.
The linear programming bound gives a bound on the nonlinear optimization problem. In most cases, this linear programming bound is good enough to exclude
the tame planar graph as potential counterexample.  When this linear program does not  bound the objective function away from $b$, 
it is replaced with a finite sequence of linear
programs, obtained by branch and bound methods.  
Every tame graph is excluded in this way.  
No packing can have a Voronoi cell with volume smaller than that
of the regular dodecahedron.  The Voronoi cells of minimal volume are all
congruent to a regular dodecahedron.






\section{Fejes T\'oth's Reduction}

L. Fejes T\'oth proved  the dodecahedral theorem 
under the extra hypothesis that $\Lambda(v_0,2t_{dod})\setminus\{v_0\}$
has at most $12$  elements.  He never published the details when
there is truncation. Here
we sketch a proof of Fejes T\'oth's theorem.

\begin{lemma}  If $\Lambda(v_0,2t_{dod})\setminus\{v_0\}$ has at most
$12$ elements, then
     $$
   \op{vol}(\Omega_{trunc}(\Lambda,v_0))\ge \op{vol}(\Omega(\Lambda_{dod},0)).
   $$
Equality holds exactly when $\Omega(\Lambda,v_0)$ is congruent to
$\Omega(\Lambda_{dod},0)$.
\end{lemma}

\begin{proof} (Sketch) By translation, we may assume that $v_0=0$.
For each nonzero $v$ in $\Lambda(v_0,2t_{dod})$, 
let $v'=2v/|v|$,We have $|v'|\le|v|$.
The Voronoi cell $\Omega'$ at the origin for $X = \{v'\mid v\in\Lambda(0,2t_{dod})\}$ is a subset of $\Omega(\Lambda,v_0)$ because face planes move to parallel planes closer to the origin.  
Set $\Omega'' = \Omega'\cap B(0,t_{dod})$.
Thus, it is enough to show
that 
   $$\op{vol}(\Omega'')\ge \op{vol}(\Omega(\Lambda_{dod},0))$$
and to analyze cases of equality.

As in Leech's solution to the kissing number problem, 
take the Delaunay triangulation of $X$ (now on the sphere of radius $2$).
For each triangle, there is a triple of points $\{v_1,v_2,v_3\}\subset X$ (dropping primes from the notation).
In Leech's proof, the area of the sphere
is a sum
of contributions from each face of the planar graph.
In this proof, the volume of $\Omega''$ is a sum of contributions
from each face of the planar graph.  Explicit formulas for the
contributions are developed later in this book.  They are as follows.
Let $t$ be the circumradius of $S=\{0,v_1,v_2,v_3\}$.  Set
   $$\op{vol}_{dod}(S) =
     \begin{cases}
       \op{volan}(0,S) & t < t_{dod}\\
       \op{vol}_{dod,trunc}(S) & t\ge t_{dod}\\
     \end{cases}
     $$
where 
  $$\op{vol}_{dod,trunc}(S) = \op{sv}_0(0,S,t_{dod},\lambda),\quad
     \lambda = (\lambda_v,\lambda_s) = (1,0).
  $$
The functions $\op{volan}$ and $\op{sv}_0$ 
are explicitly defined in Definitions~\ref{def:volan}
and~\ref{def:svor}. 
%% vol_dod is the basic quoin-Adih formula with t_dod truncation,
%% without regard for whether it is "geometric".
The sum of $\op{vol}_{dod}$ over all Delaunay triangles is 
the volume of $\Omega''$.

The function $\op{vol}_{dod}(\{0,v_1,v_2,v_3\})$ can be viewed as a function of three
variables $y_i = |v_j-v_k|$, for $(i,j,k)=(1,2,3),(2,3,1),(3,1,2)$.
The area $A$ of the Delaunay triangle can also be viewed as a function of
$y_1,y_2,y_3$.
We hold $y_1$ constant, vary $y_2$, and define $y_3$ as an implicit
function of $y_2$ on a level curve of $A(y_1,y_2,y_3)$.

By computing the derivative of $\op{vol}_{dod}(y_1,y_2,y_3)$ along
the level curve, we see that if $y_2 < y_3 \le y_1$, 
then $\op{vol}_{dod}$
is decreasing in $y_2$.  (This involves two calculations, one
when $t<t_{dod}$ and another when $t\ge t_{dod}$.)  Thus, we 
may decrease volume for constant Delaunay triangle areas by making
$y_1=y_2=y_3$.\endnote{The code for the dodecahedral problem for $12$ balls
appears in a Mathematica Notebook at 
http://flyspeck.googlecode.com/svn/trunk/book\_code/sphereBook.nb.  This notebook supplements the
arguments in the text 
with calculations that show that the circumradius of the simplex
is properly behaved under these deformations.}
(The triangles no longer fit together, but this
does not matter.  Their areas still sum to $4\pi$.)

By computing the derivative of $\op{vol}_{dod}(y,y,y)+\op{vol}_{dod}(z,z,z)$
along a level curve of $A(y,y,y)+A(z,z,z)$, we see that if $y<z$,
then the volume may be decreased by increasing $y$ (and decreasing $z$).
(This involves three calculations, depending how the two circumradii
compare with $t_{dod}$.)  Thus, we may decrease volume by
making all triangles congruent and equilateral.  The areas of the
triangles still sum to $4\pi$.  

If $V$ is the cardinality
of $X$, then by the Euler formula the number $F$ 
of triangles is $(2V-4)$, which is at most $20$,
and each triangle has area $4\pi/F$.  Our bound on volume is
   $$
   F\, \op{vol}_{dod}(y,y,y), \text { where } F\, A(y,y,y)=4\pi.
   $$
Extend $F$ as a real-valued variable, and
let the area formula define $y$ as an implicit function of $F$.
The function $F\op{vol}_{dod}(y(F),y(F),y(F))$ 
is decreasing in $F$, so a lower bound is obtained
when $F$ is as large as possible.  The lower bound 
$20\,\op{vol}_{dod}(y(20),y(20),y(20))$ equals the volume of
$\Omega(\Lambda_{dod},0)$ (as the Delaunay triangulation
for $\Lambda_{dod}$ also consists of $20$
congruent triangles with total area $4\pi$).  
This proves the bound. 
All the derivatives involved in these
calculations are explicit.  Hence the case of equality is easily
traced.
\end{proof}

Fejes Toth's proof uses simple convexity arguments to prove an
apparently difficult result.  This method of proof has become
 is an essential part of the solution to the packing
problem.   Ferguson's thesis calculates the volume of a truncated
Voronoi cell, subject to a fixed area constraint 
\cite[sec.~16.9.5]{Fer97}.  In fact, his thesis provides details
to the sketch provided above, by giving the derivatives explicitly,
for truncated Voronoi cells.  Many of his calculations have been
reproduced in this book.  See, in particular, 
Lemma~\ref{lemma:quoin-equilize}.




\section{Classification of Tame Hypermaps}

\subsection{hypermap}

A hypermap is a tuple $(D,n,e,f)$, where $D$ is a finite
set, and $n,e,f$ are three permutations on that set that
compose to the identity:
$e\circ n\circ f = I$.  The elements of $D$ are called darts.
The permutations $n,e,f$ are called the node permutation,
edge permutation, and face permuation, respectively.

If $m$ is any permutation on $D$, we write $D/m$ for the
set of orbits in $D$ under $m$.  Similarly, if $G$ is any
group of permutations on $D$, we write $D/G$ for the set
of orbits of $D$ under $G$.  In particular, $D/\tangle{n,e,f}$
is the set of orbits under the group generated by $n,e,f$.

A planar graph gives  a hypermap
by the following procedure.  Starting with a planar graph,
place a dart at each angle.  That is, at a vertex of degree $k$,
place $k$ darts, one between each consecutive pair of edges.
The face permutation has a cycle for
each face of the planar graph and  traverses the
darts in a counterclockwise direction around each face.
The node permutation has a cycle for each vertex and traverses
the darts in a counterclockwise direction around each vertex.
The edge permutation is defined by the relation $e\circ n\circ f=I$.
It can be interpreted as an involution that pairs a dart next
to one endpoint of an edge with a dart at the other endpoint.
See Figure~\ref{XX}.

Hypermaps are the primary combinatorial object used by Gonthier
in the formalization of the four-color theorem in COQ~\ref{XX}.
Hypermaps, by being purely combinatorial, are more convenient
to represent on a computer than planar graphs.  (In the 1998
preprint, we encoded the combinatorics as planar maps rather
than hypermaps.  It is trivial to translating between the two notions.
Recent work on the formalization of the Kepler conjecture works
consistently with hypermaps; thus, we deprecate the use of
planar maps.  See~\cite{XXObua},\cite{XXBlue}.)

Not all hypermaps arise from a planar graph in this way.
Those that do have two special properties.  They are plain
and planar in the following sense.  (Note the surprising spelling
of `plain', particularly in the context of planar graphs!  
To avoid misunderstandings, 
we avoid the homophone `plane' in this article.)  The definition
of planar hypermap is the standard condition on the Euler
characteristic, translated into the language of hypermpas.

\begin{definition}\label{def:plain}

\begin{itemize}
\item The hypermap $(D,e,n,f)$ is plain, if $e$ is an involution:
$$
 e^2 = I.
$$.
\item The hypermap $(D,e,n,f)$ is planar, if
   $$
   \#(D/e) + \#(D/n) + \#(D/f) = \# D + 2\#(D/\tangle{e,n,f}).
   $$
\end{itemize}
\end{definition}



\section{Linear Programs}

This section discusses Theorem~\ref{thm:graph-system}.  It is one of
the main steps in the proof of the dodecahedral conjecture.
We begin with a discussion of the terminology used in the
statement of the theorem.

The following definition is influenced by Obua's thesis~\cite{Ob}.

\begin{definition} A hypermap system is a pair $(H,\Phi)$,
where $H=(D,e,n,f)$ is a hypermap, and $\Phi$ is a finite set constraints on $H$.  More precisely, let $F$ be the vector space of
real-valued functions on $D$.  A finite set of constraints $\Phi$ is a finite
set of boolean valued functions $\phi:F^\ell\to \{\op{true},\op{false}\}$
for some $\ell$. (We assume $\ell$ is independent of $\phi\in \Phi$).
\end{definition}

We say that the hypermap system $(H,\Phi)$ is feasible, if
there is some $x=(x_1,\ldots,x_\ell)\in F^\ell$ such that
$\phi(x)$ holds for all $\phi\in\Phi$. Otherwise, we say that
the system is infeasible.

Appendix~\ref{XX} gives a finite set of constraints $\Phi$, 
specified
in a uniform way for every structurally tame hypermap $H$.  This determines,
for every tame hypermap $H$, we then have a well-defined extension
to a hypermap system $(H,\Phi)$.  We call this particular hypermap
system, by structural analogy with the Kepler conjecture,
the {\it basic hypermap system}.  


\begin{theorem}\label{thm:graph-system}  Let 
$H$ be any structurally tame hypermap with corresponding
basic hypermap system $(H,\Phi)$.  Then $(H,\Phi)$ is infeasible.
\end{theorem}

This proof is carried out by computer as a collection of linear
programs.  This is one of the three major parts of the proof
of the Dodecahedral conjecture that has been carried out by computer.
In this article, we describe the relationship between the
feasibility of $(H,\Phi)$ and a linear programming feasibility
problem.  We also describe some details of the implementation of 
the code.

We have obtained an enumeration of all structurally tame
hypermap systems in Section~\ref{XX}.  Thus,  we may prove
Theorem~\ref{thm:graph-system} through a case-by-case treatment
of cases.  This is how we proceed.

Let's turn our attention for a moment, to one structurally tame hypermap $H=(D,e,n,f)$ and the corresponding basic hypermap system $(H,\Phi)$.
We describe the strategy that we use to show that it is infeasible.
For some $\ell\in\ring{N}$,
each constraint $\phi$ is a function on $F^\ell$, where $F$ is
the vector space of real-valued functions on $D$.  Thus,
$F^\ell$ can be identified with $\ring{R}^m$, where $m= \ell \#(D)$.
If we examine the form of the constraints $\phi\in \Phi$, as listed
in Appendix~\ref{XX}, we notice that the constraints all have a 
very special form.  With a minor qualification, they are all linear constraints
on $\ring{R}^m$.  

We allow a minor qualification to allow some of the constraints
to carry guard conditions.  That is, some constraints have the form
  \begin{equation}\label{eqn:guard}
  (A x < b)  \Rightarrow (A' x \le b'),
  \end{equation}
for $x\in\ring{R}^m$, and various matrices $A,A'$ and vectors
$b,b'$.  (We write $a \le b$ to mean
that $a_i\le b_i$ for every component of the vectors $a,b$.)
We call the constraint $(A x < b)$ a guard condition.
We allow variations in which some of the inequalities in the
guard condition are weak and some of the inequalities in the
consequent are strict.

The collection of all inequalities that do not have a guard 
condition is a system of linear inequalities.  Standard linear
programming packages can be used to determine whether this
system of linear inequalities has a feasible solution.  If this
linear program is infeasible, then the hypermap system $(H,\Phi)$
is clearly also infeasible.  When this happens, we have a
proof of infeasibility for $(H,\Phi)$.

When this fails, we turn to the constraints that carry guard
conditions.
The introduction of a constraint that has a nontrivial guard condition
involves multiple steps.  
The constraint (\ref{eqn:guard}) can be rewritten in logically
equivalent form as
  $$
   (A_{1} x \ge b_{1}) \lor \cdots \lor
   (A_{r} x \ge b_{r}) \lor (A_2 x \le b_2),
  $$
where $A_{i}$ and $b_{i}$ are the rows of $A$ and $b$.
Taking each disjunct in turn, one linear at a time
is added to the system
of linear inequalities, and the resulting system is shown to be
infeasible.  When each
systems are infeasible, then $(H,\Phi)$ itself is infeasible.

In general, more than one guarded constraint may be added.  If
$k$ constraints are added with $r_1,\ldots,r_k$ guards
respectively, then as many as $(r_1+1)\cdots (r_k+1)$ systems
of linear equalities are created.  If they are all infeasible,
then $(H,\Phi)$ itself is infeasible.

This discussion may give the impression 
that a great many linear programming feasibility
problems are created in this manner.  In practice, nearly all
of the hypermap systems are eliminated in the first pass, without
requiring recourse to the guarded conditions.  Only
$14$ cases (XX?) use a guard condition.




\section{Interval Arithmetic}

\section{Tarski Arithmetic}


\section{Appendix: Basic Hypermap System}

In this appendix, we list the system of constraints determining
the basic hypermap system.  As the main body of the text explains
a hypermap system is a pair $(H,\Phi)$, where $H$ is a hypermap
and $\Phi$ is a finite set of constraints on $H$.  To describe
the constraints $\Phi$ in precise terms requires some notation.

Let $H=(D,e,n,f)$ be a hypermap.  We use Greek letters
$\alpha,\beta \in D$ for darts.  Let $F$ be the vector space 
of real-valued  functions on $D$.  A constraint $\phi$ is a boolean-valued function on $F^\ell$ for some $\ell$.  We use a suggestive
notation $(\optt{sol},\optt{mu},\optt{tau},\optt{y},\ldots)\in F^\ell$ for $\ell$-tuples
of elements of $F$.  The main text relates each coordinate back
with its namesake.  For example, the linear constraints on
$\optt{sol}\in F$, mirror  nonlinear relations satisfied by the
nonlinear function $\sol$ in the main text.  This correspondence
between functions in $F$ and functions does not enter into the
definition of the basic hypermap system.  For the purposes of this
appendix, this correspondence is simply an aid to the
intuition.

If $m$ is  a permutation on $D$, let $\op{ord}(m,\alpha,i)$
be the predicate that asserts that the cardinality of the $m$-orbit
of $\alpha$ is $i\in\ring{N}$.
\bigskip

%XX Move to definitions.
\def\sland{\ \land\ }

\subsection{general bounds}

$$
\begin{array}{lll}
   \forall \alpha\in D.\ &
    2 \le \optt{yn}(\alpha) \sland
    \optt{yn}(\alpha) \le 2t_{dod} \sland
    2 \le \optt{ye}(\alpha) \sland
    \optt{ye}(\alpha) \le 2t_{dod}
\end{array}
$$

\subsection{tetrahedral constraints}

\noindent
Function bounds:
$$
\begin{array}{lll}
  \forall \alpha\in D. \ \op{ord}(f,\alpha,3) &
   \Rightarrow (
   (\optt{omega}(\alpha) = \optt{omega}(f\alpha) = \optt{omega}(f^2\alpha)) \sland
   \\
   &\optt{omega}(\alpha) > 0.202804  \sland
   \optt{sol}(\alpha) > 0.315696 \sland
   )
\end{array}
$$

\subsection{guard conditions}

\noindent
Condition H.14.1:  % Use alpha= y1.
$$
\begin{array}{lll}
\forall \alpha\in D.\ &
  \op{ord}(f,\alpha,5) \sland 
  \optt{dih}(\alpha) < 1.342 \sland
  \optt{dih}(f^3\alpha) < 1.684 \sland
  \optt{yn}(\alpha) < 2.153 \sland \\
  &\quad \optt{y}(f\alpha) < 2.174 \sland
  \optt{yn}(f^2\alpha) < 2.26 \sland
  \optt{yn}(f^3 \alpha) < 2.194 \sland
  \optt{yn}(f^4 \alpha) < 2.314  \\
  &\quad \Rightarrow
  \optt{omega}(\alpha) > 0.950
\end{array}
$$

\noindent
Condition H.14.2:  (XX Potential problem, in H.14.1, the
condition is $y(10)< 2.26$, the index changes in H.14.2
to $y(5) > 2.26$.)
$$
\begin{array}{lll}
\forall \alpha\in D.\ &
  \op{ord}(f,\alpha,5) \sland 
  \optt{dih}(\alpha) < 1.342 \sland
  \optt{dih}(f^3\alpha) < 1.684 \sland
  \optt{y}(\alpha) < 2.153 \sland \\
  &\quad \optt{y}(f\alpha) < 2.174 \sland
  \optt{y}(f^2\alpha) \ge 2.26 \sland
  \optt{y}(f^3 \alpha) < 2.194 \sland
  \optt{y}(f^4 \alpha) < 2.314  \\
  &\quad \Rightarrow
  \optt{mu}(\alpha) > 0.1234
\end{array}
$$
