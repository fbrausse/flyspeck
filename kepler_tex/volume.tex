%------------------------------------------------------------
% File : volume.tex
% Author: Thomas C. Hales
% Book Chapter: Dense Sphere Packings
% File  created: 3/22/07
%------------------------------------------------------------



\chapter{Volume}\label{chapter:volume}

\section{Measure}
\indy{Index}{measure}%

Nowhere does this book need a notion of integration.  Measure alone
suffices.  Most of the volumes to be considered (such as that of a
ball, a tetrahedron, and a frustum) were known in antiquity.

J. Harrison has already implemented gauge integration, which is much
more than what this book requires, inside the proof assistant {\it HOL
Light}.\footnote{J. Harrison has contributed large bodies of formal
mathematics to the libraries of the HOL Light theorem prover in
order to make the Flyspeck project possible.}  As it will become
clear in a moment, one needs considerably less than the gauge integral
(and even less than the Riemann integral). However, since gauge
integration available, a theory of measure will be assumed.

This book uses notions of null set, measurable set, and volume in
three dimensions; the existence of these mathematical objects with
stated properties will be assumed without proof.  The existence
follows from the known properties of gauge integrals: a null sets can
be defined to be a set of zero gauge measure, a measurable set to be a
bounded gauge measurable set, and the volume of a measurable set to be
its gauge measure.  \indy{Index}{measure!Lebesgue}%


\subsection{null set}\label{sec:null}
\indy{Index}{null set}%

Assume a notion of null set with the following
properties.

\begin{enumerate}%[Null Set]
\item A finite union of null sets is a null set.
\item A plane is a null set.
\item A sphere is a null set.
\item A circular cone is a null set; that is, a union of all
lines through a fixed point $p$ and at
a fixed angle to a given line through $p$.
\tlabel{enum:null}
\indy{Index}{null set}%
\indy{Index}{plane}%
\indy{Index}{sphere}%
\indy{Index}{circular cone}%
\end{enumerate}
\indy{Notation}{p@$p$ (point)}%


Write $X\equiv Y$ when sets $X$ and $Y$ are equal up to a null set;
that is, when there exists a null set $Z$ such that
$(X\setminus Y) \cup (Y\setminus X) \subset Z$.
\indy{Index}{null set}%
\indy{Notation}{1@$\equiv$}%

\subsection{measurability}\label{sec:measure}
\indy{Index}{measure}%

Assume a notion of measurability that has the following properties.

\begin{enumerate}%[Measurable set]
\item Null sets are measurable.
\item The union of two measurable sets is measurable.
\item The intersection of two measurable sets is measurable.
\item The difference of two measurable sets is measurable.
\item Primitive regions are measurable (Definition~\ref{def:primitive}).
\tlabel{enum:measure}
\indy{Index}{null set}%
\indy{Index}{primitive regions}%
\end{enumerate}

\subsection{volume}\label{sec:volume}

Certain primitive regions are given in Definition~\ref{def:primitive}.
Assume a notion of volume that has the following properties.
\indy{Index}{primitive}%

\begin{enumerate}%[Volume]
\item The volume is defined for every measurable set.  It is
a non-negative real number.
\item The volume of a null set is $0$.
\item If $X$ and $Y$ are  measurable, and if
the symmetric difference of
$X$ and $Y$ is contained in a null set, then 
$X$ and $Y$ have the same volume.
\item If $X$ and $Y$ are measurable sets, and if $X\cap
Y$ is contained in a null set, then
\begin{displaymath}
\op{vol}(X\cup Y) = \op{vol}(X) + \op{vol}(Y).
\end{displaymath}
\indy{Notation}{X@$X$ (set)}%
\indy{Notation}{Y@$Y$ (set)}%
\item (linear stretch) For $X\subset \ring{R}^3$, and
$\v\in\ring{R}^3$, set
\begin{displaymath}
T_\v(X) = \{ (v_1u_1,v_2u_2,v_3u_3) \mid \u\in X\}.
\end{displaymath}
\indy{Notation}{T@$T_\v$ (linear stretch)}%
If $X$ and $T_\v(X)$ are measurable, then
$\op{vol}(T_\v(X)) = |v_1v_2v_3|\op{vol}(X)$.
\item (translation) If $X\subset \ring{R}^3$ and $\v\in\ring{R}^3$,
then let $X+\v = \{p + \v\mid p\in X\}$.  If $X$ is measurable,
then $X+\v$ is as well, and $\op{vol}(X) = \op{vol}(X+\v)$.
\item (primitives) If $X$ is a primitive, then $X$ is measurable
and $\op{vol}(X)$ is given by the formulas of
Lemma~\ref{lemma:prim-volume}.  \tlabel{enum:volume}
\end{enumerate}
\indy{Index}{volume}%
\indy{Index}{linear!stretch}%
\indy{Index}{translation}%
Volume calculations of primitive regions are given in 
Lemmas~\ref{lemma:prim-volume} and~\ref{lemma:wedge-vol}.
%% NB Don't need lemma:wedge-sol because solid of a \op{FR} is a \op{SC}.
%% Don't need lemma:prim-sol, because solds are \op{SC} or \op{ST}.

\begin{lemma}\guid{ATOAPUN}\oldrating{300}
\formalauthor{John Harrison}  
Let $\op{vol}$ be the Lebesgue measure
on $\ring{R}^3$. Let $B$ be the collection of all bounded Lebesgue
measurable subsets of $\ring{R}^3$.  Let $N$ be the collection of all
Lebesgue measurable subsets of $\ring{R}^3$ of measure zero.
Then
\begin{itemize}
\item The collection $N$ satisfies the null set properties of
Section~\ref{sec:null}
\item The collection $B$ satisfies the measurability properties of
Section~\ref{sec:measure}
\item $\op{vol}$ on $B$ satisfies the volume properties of
Section~\ref{sec:volume}.  \indy{Index}{measure!Lebesgue}%
\end{itemize}
\end{lemma}
\indy{Notation}{B@$B$ (Lebesgue measurable subsets)}%
\indy{Notation}{vol@$\op{vol}$}%

\subsection{radial set and solid angle}\label{sec:solid}

Surface integrals are not required in this book.  Although
the `solid angle' is traditionally defined as a surface integral,
it is simpler to give an alternative definition based on volume.


\begin{definition}[open ball] The open ball $B(\p,r)$ with center $\p$
and radius $r$ is the set
\begin{displaymath}
\{ \q\in\ring{R}^3 \mid \norm{\p}{\q} < r.\}
\end{displaymath}
\indy{Index}{open ball}%
\indy{Notation}{B@$B(\p,r)$}%
\end{definition}



\begin{definition}[radial]
A set $C$ is $r$-radial at center $\p$ if  $C\subset B(\p,r)$
and if
$\p + \u \in C$ implies
$\p + t \u \in C$ for all $t$ satisfying $0 < \normo{\u} t < r$.
A set $C$ is eventually radial at center $\p$ if $C\cap B(\p,r)$ is
$r$-radial at center $\p$, for some $r>0$.
\indy{Index}{radial}%
\end{definition}
\indy{Notation}{C@$C$ (set)}%

\begin{lemma}\guid{PUACSHX}\tlabel{lemma:r-r'}\oldrating{140}
\formalauthor{Nguyen Tat Thang, Nov 2008}
Assume that $C$ is measurable and $r$-radial at $\p$.  Let $0\le
r'<r$, Then $C\cap B(\p,r')$ is measurable and $\op{vol}(C\cap
B(\p,r')) = \op{vol}(C) (r'/r)^3$.
\end{lemma}

\begin{proof}  The radial set $C$ transforms into $C\cap B(\p,r')$ by
a series of translations and linear stretches.
\end{proof}


\begin{definition}[solid angle]\tlabel{def:sol}
When $C$ is measurable and eventually radial at center $\p$, 
define the \newterm{solid angle} of $C$ at $\p$ to be
\begin{displaymath}
\op{sol}(\p,C) = 3 \op{vol}(C\cap B(\p,r))/r^3,
\end{displaymath}
where $r$ is as in the definition of eventually radial. 
By Lemma~\ref{lemma:r-r'}, this
definition is independent of any such $r$.  When the center $\p$ is
clear from the context, write $\op{sol}(C)$ for
$\op{sol}(\p,C)$.
\indy{Index}{solid angle}%
\indy{Notation}{solid angle@$\op{sol}$ (solid angle)}%
\end{definition}



The following properties follow immediately from the definitions.
If $C$ is $r$-radial for some $r>0$ then it is eventually radial.
If $C$ is measurable and $r$-radial, then the volume of $C$
satisfies
\begin{displaymath}
\op{vol}(C) = \op{sol}(C) r^3/3.
\end{displaymath}
If $C$ is bounded away from $\p$, then $C$ is eventually radial at
$\p$, and $\op{sol}(C) = 0$.
\indy{Index}{radial}%

\begin{lemma}\guid{KODOBRF}\rating{60}  If $C$ and $C'$ are  $r$-radial
at $\p$, then $C\cap C'$ is also $r$-radial at
$\p$.
\end{lemma}






\section{Primitive Volume}
\indy{Index}{volume!primitive}%

This book accepts
certain elementary volume calculations as axioms.  
These regions will be called primitive volumes.  There are only
a few primitive volumes.
All further
volumes calculations will be obtained from these through the basic
properties of measure.   
% This treatment of volume is hardly about measure at all.  The focus
% is rather on the geometry of the various regions and how to
% decompose them into primitives.

To fix a convention, this book prefers to take the volume of open sets
whenever that can be arranged.  It begins with a description of some
of the primitive regions.

\subsection{wedge}
\indy{Index}{wedge}%


\begin{definition}[wedge]
Assume that $\v_0\ne \v_1$ and that
$\w_1$ and $\w_2$ do not lie on
the line $\op{aff}\{\v_0,\v_1\}$.  Set
\begin{displaymath}
\begin{array}{rll}
W^0(\v_0,\v_1,\w_1,\w_2) &= 
\{\p %\not\in\op{aff}\{\v_0,\v_1\} 
\mid 
0< \op{azim}(\v_0,\v_1,\w_1,\p) < \op{azim}(\v_0,\v_1,\w_1,\w_2)\}.\\
W(\v_0,\v_1,\w_1,\w_2) &= 
\{\p \mid 
0\le \op{azim}(\v_0,\v_1,\w_1,\p) \le \op{azim}(\v_0,\v_1,\w_1,\w_2)\}.\\
\end{array}
\end{displaymath}
It is a \newterm{wedge}.
\end{definition}

The set $\op{aff}^0_+(\{\v_0,\v_1\},\{\v_2,\v_3\})$ was defined
in Definition~\ref{def:aff}.  Call it a lune.  It is the intersection
of two open half-spaces
\begin{displaymath}
\op{aff}^0_+(\{\v_0,\v_1\},\{\v_2,\v_3\})
=\op{aff}^0_+(\{\v_0,\v_1,\v_2\},\v_3)\cap
\op{aff}^0_+(\{\v_0,\v_1,\v_3\},\v_2).
\end{displaymath}
\indy{Notation}{aff0@$\op{aff}^0$ (lune)}%%
A lune has a dihedral angle $\dih(\{\v_0,\v_1\},\{\v_2,\v_3\})$ between
$0$ and $\pi$.   For angles that are larger than $\pi$,   a wedge
$W^0(\v_0,\v_1,\w_1,\w_2)$ is used instead of the lune.  
When the angle is less than $\pi$, there is no difference between
a wedge and a lune.
\indy{Index}{lune}%
\indy{Index}{half-space}%
\indy{Index}{angle!dihedral}%
\indy{Index}{wedge}%
\indy{Notation}{dih}%
\indy{Notation}{W@$W^0$ (wedge)}%

\begin{lemma}\guid{VICUATE} Let $\{\v_0,\v_1,\v_2,\v_3\}$ be a set of four points
in $\ring{R}^3$.  Assume that the set is not coplanar.  Assume that
$\op{azim}(\v_0,\v_1,\v_2,\v_3)<\pi$.  Then,
\begin{displaymath}W^0(\v_0,\v_1,\v_2,\v_3) =
\op{aff}^0_+(\{\v_0,\v_1\},\{\v_2,\v_3\}).\end{displaymath}
\indy{Index}{coplanar}%
\end{lemma}


\subsection{solid triangle}
\indy{Index}{solid triangle}%

\begin{definition}[solid triangle] The \newterm{solid triangle}
$\op{ST}(\v_0,V,r)$ is determined by $\v_0\in\ring{R}^3$, a finite
subset $V\subset\ring{R}^3$, and a radius $r\ge0$.
\begin{displaymath}
\op{ST}(\v_0,V,r) = 
B(\v_0,r)\cap \op{cone}(\v_0,V).
\end{displaymath}
(Usually, $V=\{\v_1,\v_2,\v_3\}$, and this accounts for
the term {\it triangle}.)
\indy{Index}{solid triangle}%
\indy{Notation}{ST@$\op{ST}$ (solid triangle)}%
\indy{Notation}{r@$r$ (radius)}%
\end{definition}



\subsection{conic cap}
\indy{Index}{conic cap}%

% renamed from spherical cap.

\begin{definition}[conic cap]
The conic cap $\op{SC}(\v_0,\v_1,r,a)$ is specified by an apex
$\v_0\in\ring{R}^3$,  radius $r\ge0$,  non-zero vector $\v_1-\v_0$ giving
direction, and constant $a$.  The conic cap is the intersection of
the ball $B(\v_0,r)$ with a solid right-circular cone:
\begin{displaymath}
\op{SC}(\v_0,\v_1,r,a)=\{\p \in B(\v_0,r) 
\mid (\p-\v_0)\cdot (\v_1-\v_0) > \norm{\p}{\v_0}\, \norm{\v_1}{\v_0}\, a\}.
\end{displaymath}
\indy{Index}{conic cap}%
\indy{Index}{apex}%
\indy{Index}{cone}%
\indy{Notation}{SC@$\op{SC}$}%
\end{definition}

\subsection{frustum}
\indy{Index}{frustum}%

\begin{definition}[rcone]\tlabel{def:p:rcone}
%\tlabel{def:rcone} % from Tarski collection..
\indy{Index}{cone!right-circular}%
Define the following collection of right-circular cones.
If $\v$ and $\w$ are points in $\ring{R}^3$, and
$h\in\ring{R}$, then set
\begin{displaymath}\begin{array}{lll}
\op{rcone}(\v,\w,h) 
&= \{\p\mid (\p-\v)\cdot (\w-\v) \ge \norm{\p}{\v}\,\norm{\w}{\v} h\},\\
\op{rcone}^0(\v,\w,h) 
&= \{\p\mid (\p-\v)\cdot (\w-\v) > \norm{\p}{\v}\,\norm{\w}{\v} h\}.\\
% \op{rcone}_-^0(\v,\w,h) &= \{\p\mid (\p-\v)\cdot (\w-\v) < \norm{\p}{\v}\,\norm{\w}{\v} h\}.\\
% \partial\op{rcone}(\v,\w,h) &= \{\p\mid (\p-\v)\cdot (\w-\v) = \norm{\p}{\v}\,\norm{\w}{\v} h\}.\\
\end{array}
\end{displaymath}
\end{definition}
\indy{Notation}{rcone@$\op{rcone}$}%
\indy{Notation}{rcone2@$\op{rcone}^0$}%
%\indy{Notation}{rcone3@$\op{rcone}_-^0$}%
%\indy{Notation}{rcone4@$\partial\op{rcone}$}%
\indy{Notation}{v@$\v$ (point)}%
\indy{Notation}{wz@$\w$ (point)}%


\begin{definition}[frustum, FR] The frustum
$\op{FR}(\v_0,\v_1,h',h,a)$ is specified by an apex
$\v_0\in\ring{R}^3$, heights $0\le h'\le h$, a vector $\v_1-\v_0$
giving its direction, and $a\in[0,1]$. The set $\op{FR}$ is given as
\begin{displaymath}
\{ \p \in\op{rcone}^0(\v_0,\v_1,a) \mid \ 
h'\norm{\v_1}{\v_0} < (\p-\v_0) \cdot (\v_1-\v_0) < h\norm{\v_1}{\v_0} \}.
\end{displaymath}
\indy{Index}{frustum}%
\indy{Notation}{frustum@$\op{FR}$ (frustum)}%
\indy{Notation}{vapex@${\mathbf v}_0$ (apex)}%
%\indy{Notation}{v1@${\mathbf v}_1$ (vector)}%
%\indy{Notation}{v0@${\mathbf v}_0$ (vector)}%
\end{definition}

That is, the frustum is the part of a right-circular cone between two
parallel planes that cut the axis of the cone at a right angle.  When
$h'=0$, the frustum extends to the apex of the cone.  When $h'=0$, it
is dropped from the notation:
$\op{FR}(\v_0,\v_1,h,a)=\op{FR}(\v_0,\v_1,0,h,a)$.

\subsection{tetrahedron}
\indy{Index}{tetrahedron}%

\begin{definition}[tetrahedron] A tetrahedron is a set of the form
\begin{displaymath}\op{conv}^0\{\v_1,\v_2,\v_3,\v_4\}.\end{displaymath}
\indy{Index}{tetrahedron}%
\end{definition}

By Tarski arithmetic, %tarski: {hedra-tope}, 
this set can also be described
as the intersection of four open half-spaces, with each bounding
plane defined by three of the four points.
By taking this into account, it follows that
the sets in this section have all been defined by linear and quadratic
constraints.
\indy{Index}{Tarski arithmetic}%

\subsection{primitive}
\indy{Index}{primitive}%

\begin{definition}[primitive]\label{def:primitive} 
A primitive region is any of the following.

\begin{enumerate}%%[Primitive Volumes]
\item A solid triangle $\op{ST}$.
\indy{Notation}{ST@$\op{ST}$ (solid triangle)}%
\item A tetrahedron $conv^0\{\v_0,\v_1,\v_2,\v_3\}$.
\item A wedge of a frustum;
that is, the intersection of a frustum with
a wedge:
\begin{displaymath}
\op{FR}(\v_0,\v_1,h_1,h_2,a) \cap W^0(\v_0,\v_1,\v_2,\v_3).
\end{displaymath}
\item A wedge of a conic cap; that is, the intersection of a conic cap
with
a wedge:
\begin{displaymath}
\op{SC}(\v_0,\v_1,r,c) \cap W^0(\v_0,\v_1,\v_2,\v_3).
\end{displaymath}
\item a rectangle
\begin{displaymath}
\{\u \mid a_i < u_i < b_i,\text { for } i=1,2,3.\}.
\end{displaymath}
\item A wedge of a ball
\begin{displaymath}
B(\v_0,r) \cap W^0(\v_0,\v_1,\v_2,\v_3).
\end{displaymath}
%\item An ellipsoid
%  \begin{displaymath}
% \{\v \mid t_1 x_1^2 + t_2 x_2^2 + t_3 x_3^2 < r\},
% \end{displaymath}
%where $t_1,t_2,t_3,r>0$.
\indy{Index}{solid triangle}%
\indy{Index}{tetrahedron}%
\indy{Index}{wedge}%
\indy{Index}{frustum}%
\indy{Index}{conic cap}%
\indy{Index}{rectangle}%
\indy{Index}{ball}%
%\indy{Index}{ellipsoid}%
\tlabel{enum:volume-prim}
\end{enumerate}
One could find a smaller set of primitives.  In particular, the
rectangle is a union of tetrahedra, and wedge of a ball is a union of
solid triangles (up to a null set).
%But this would only make the definition more difficult to apply.

\end{definition}

\subsection{primitive volume calculation}\label{sec:primitive}

\begin{lemma}\guid{PAZNHPZ}\tlabel{lemma:prim-volume} 
\begin{enumerate} 
\item \rating{250} Let $\v_1,\v_2,\v_3$ be unit vectors.  A solid
triangle $\op{ST}(\v_0,\{\v_1,\v_2,\v_3\},r)$ is measurable and has
volume
\begin{displaymath}
(\alpha_1+\alpha_2+\alpha_3-\pi)r^3/3,
\end{displaymath}
where $\alpha_i = \dih_V(\{\v_0,\v_i\},\{\v_j,\v_k\})$.
\item \rating{250} The conic cap $\op{SC}(\v_0,\v_1,r,a)$ is
measurable and has volume
\begin{displaymath}
2\pi(1-a) r^3/3.
\end{displaymath}
\item \rating{250} A frustum $\op{FR}(\v_0,\v_1,h,a)$ (with $h'=0$) 
is measurable and has volume
\begin{displaymath}
\pi (t^2-h^2) h/3,\quad h = t a.
\end{displaymath}
\item\rating{250} A tetrahedron $\op{conv}^0\{\v_1,\v_2,\v_3,\v_4\}$
is measurable and has volume
\begin{displaymath}
\sqrt{\Delta(x_{12},x_{13},x_{14},x_{34},x_{24},x_{23})}/12,
\end{displaymath}
where $x_{ij} = \norm{\v_i}{\v_j}^2$.
\item\rating{100} A rectangle $\op{rect}(a,b)$ is measurable and
has volume $0$ unless $a_i<b_i$ for all $i$.  In this case, the
volume is
\begin{displaymath}(b_3-a_3)(b_2-a_2)(b_1-a_1).\end{displaymath}
\item\rating{250} The intersection of a ball $B(\v_0,r)$ with a wedge
$W^0(\v_0,\v_1,\v_2,\v_3)$ is measurable.  The volume of the intersection
is 
\begin{displaymath}
2 \theta r^3/3,\text{ for } r \ge 0,
\end{displaymath}
where $\theta = \op{azim}(\v_0,\v_1,\v_2,\v_3)$.

\end{enumerate}
\end{lemma}

The formula for the volume of the solid triangle is known as Girard's
formula.  Euler's formula (Lemma~\ref{lemma:euler}) gives an
equivalent expression for $(\alpha_1+\alpha_2+\alpha_3-\pi)$, which is
sometimes more convenient.  \indy{Index}{Euler's formula for solid
angle}%
\indy{Index}{Girard's formula}%

\begin{proof}
The formula for the volume of a solid triangle is $r^3/3$ times
its solid angle.  The formula 
\begin{displaymath}\alpha_1+\alpha_2+\alpha_3-\pi\end{displaymath}
for the area of a spherical triangle is classical.    
The conic cap volume is
$r^3/3$ times its solid angle.  
The volume of a right-circular cone is $1/3$ its base times height.
The volume of a tetrahedron is
\begin{displaymath}
|\det(\v_2-\v_1,\v_3-\v_1,\v_4-\v_1)|/6.
\end{displaymath}
Cayley and Menger evaluated the square of this determinant (See
Section~\ref{sec:piped}).  The square is $\Delta/4$, with
$\Delta\ge0$.  The result follows.
\end{proof}
\indy{Index}{Cayley-Menter determinant}%



\subsection{wedge}\label{sec:wedge}
\indy{Index}{wedge}%

When the region is realized by revolution along an axis
$\op{aff}\{\v_0,\v_1\}$, one can also give the volume of the
intersection of the region with a wedge
$W^0=W^0(\v_0,\v_1,\v_2,\v_3)$.  These intersections are measurable by
the last statement of Lemma~\ref{lemma:prim-volume} (with $r$
sufficiently large).  In the following let $\theta =
\azim(\v_0,\v_1,\v_2,\v_3)$.  \indy{Notation}{aff@$\op{aff}$}%
\indy{Notation}{ZZtheta@$\theta$ (azimuth)}%

\begin{lemma}\guid{DFNVMFM}\tlabel{lemma:wedge-vol}  
Let $C$ be $\op{SC}(\v_0,\v_1,r,a)$, $B(\v_0,r)$, or
$\op{FR}(\v_0,\v_1,h,a)$.  Let $m$ be the volume of $C$.  Then
$C\cap W^0$ has volume $m\,\theta/(2\pi)$.
\end{lemma}
\indy{Notation}{C@$C$ (wedge)}%

\begin{lemma}\guid{FMSWMVO}\tlabel{lemma:wedge-sol}  
Let $C$ be either $\op{SC}(\v_0,\v_1,r,a)$, $B(\v_0,r)$, or
$\op{FR}(\v_0,\v_1,h,a)$.  Then $C$ and $C\cap W^0$ are eventually
radial at $\v_0$. Furthermore, $C\cap W^0$ has solid angle
$s\,\theta/(2\pi)$, where $s$ is the solid angle of $C$.
\end{lemma}


\begin{proof}
These are elementary integrals.
\end{proof}


\subsection{primitive solid angle}

All of the primitive sets are eventually radial at the natural
base point $\v_0$, and have a
solid angle.  By Lemma~\ref{lemma:wedge-sol}, it is enough to compute
the solid angle before intersecting with a wedge.
\indy{Notation}{v0@$\v_0$ (base point)}%

\begin{lemma}\guid{FUPXNLC} \tlabel{lemma:prim-sol}
\begin{enumerate}
\item  $\op{ST}(\v_0,\{\v_1,\v_2,\v_3\})$ is eventually radial at $\v_0$
with solid angle 
\begin{displaymath}
(\alpha_1+\alpha_2+\alpha_3-\pi),\quad
\alpha_i=\dih_V(\{\v_0,\v_i\},\{\v_j,\v_k\}).
\end{displaymath}
\item $\op{conv}^0\{\v_0,\v_1,\v_2,\v_3\}$ is eventually radial at $\v_0$
with solid angle
\begin{displaymath}
(\alpha_1+\alpha_2+\alpha_3-\pi),\quad
\alpha_i=\dih_V(\{\v_0,\v_i\},\{\v_j,\v_k\}).
\end{displaymath}
\item $\op{SC}(\v_0,\v_1,r,a)$ is eventually radial at $\v_0$ with solid
angle 
$2\pi(1-a)$.
\item $\op{FR}(\v_0,\v_1,h,a)$ is eventually radial at $\v_0$ with solid
angle        $2\pi (1-a)$.
\indy{Index}{eventually radial}%
\end{enumerate}
\end{lemma}
\indy{Notation}{conv0@$\op{conv^0}$}%

\indy{Notation}{SC@$\op{SC}$}%
\indy{Notation}{frustum@$\op{FR}$ (frustum)}%
\indy{Notation}{ZZalpha@$\alpha$ (angle)}%
\indy{Notation}{dih}%

\begin{proof} In every case, the intersection of 
the region with $B(\v_0,r')$, for $r'>0$ sufficiently small, is
a conic cap or a solid triangle.  These two volumes have
already been calculated.  This gives the results as stated.
\end{proof}


\section{Scissor and Volume}
\tlabel{sec:measure-second}

There are many other volumes that can be computed from the
primitive ones enumerated in Definition~\ref{enum:volume-prim}.

\subsection{lune}  
\indy{Index}{lune}%

Consider the simple example of a derived volume given by a
lune $A=\op{aff}_+^0(\{\v_0,\v_1\},\{\v_2,\v_3\})$.  It is eventually
radial at $\v_0$ and has a solid angle.
\indy{Notation}{A@$A$ (lune)}%

\begin{lemma}\guid{WFFVGVF} $\sol(\v_0,A) =
2\dih_V(\{\v_0,\v_1\},\{\v_2,\v_3\})$.
\end{lemma}

\begin{proof}
Let $B_{\pm} = \op{aff}_\pm(\{\v_0,\v_2,\v_3\},\v_1)$.  $B_- \cap B_+$
is a null set.  The intersections $A\cap B_{\pm}\cap B(\v_0,r)$ 
are solid triangles.  This gives the solid angle of $A$ as
follows:
\begin{displaymath}\begin{array}{lll}
\sol(\v_0,A) &= \sol(\v_0,A\cap B_+)+\sol(\v_0,A\cap B_-) \\
&= 
\sol(\op{ST}(\v_0,\{\v_1,\v_2,\v_3\})) + \sol(\op{ST}(\v_0,\{\v_1',\v_2,\v_3\})) \\
&=
2\dih_V(\{\v_0,\v_1\},\{\v_2,\v_3\}).
\end{array}
\end{displaymath}
Here,  $\v_1'= 2 \v_0 - \v_1$, the reflection of $\v_1$
through $\v_0$.  The usual calculation of the volume of a solid triangle
inverts this proof, 
and derives the volume from the solid angle of a lune.
\end{proof}



\begin{lemma}\guid{CZOQFNL}\tlabel{lemma:wedge:sol} 
Assume that the sets $\{\v_0,\v_1,\w_1\}$ and
$\{\v_0,\v_1,\w_2\}$ are not collinear. 
\begin{displaymath}
\sol(\v_0,W^0(\v_0,\v_1,\w_1,\w_2)) = 2\,\op{azim}(\v_0,\v_1,\w_1,\w_2).
\end{displaymath}
\indy{Index}{collinear}%
\end{lemma}    

\begin{proof} Every wedge is a union of two lunes, up to a null set.
\end{proof}

\begin{lemma}\guid{OXQZBJG}  
\begin{displaymath}
\begin{array}{lll}
\op{sol}(\v_0,B(\v_0,t)) &= 4\pi\\
\op{vol}(\v_0,B(\v_0,t)) &= 4\pi t^3/3\\
\op{sovo}(\v_0,B(\v_0,t),\lambda) &= 4 \pi \phi(t,t,\lambda).
\end{array}
\end{displaymath}
\end{lemma}

\begin{proof}
The ball is $t$-radial at $\v_0$, so the volume is given by
Definition~\ref{def:sol} in terms of solid angle.  It is enough
to check that a hemisphere has solid angle $2\pi$.  This follows
from Lemma~\ref{lemma:wedge:sol}.
\end{proof}  




\subsection{orthosimplex}
\indy{Index}{orthosimplex}%



\begin{definition}[orth] \tlabel{def:orth} Let
$\{\v_0,\v_1,\v_2,\v_3\}$ be a set of four points in $\ring{R}^3$.
Assume that they are not coplanar.  Let $\p$ be the circumcenter of
$\{\v_0,\v_1,\v_2\}$ and $r$ its circumradius.  Let $c\ge r$.  By
Tarski arithmetic, %tarski: {rog-exist},
there exists a unique point $\p'$ in
$A=\op{aff}_+(\{\v_0,\v_1,\v_2\},\v_3)$ at equal distance $c$ from
$\v_0,\v_1,\v_2$.  Define the \newterm{orthosimplex}
\indy{Notation}{p@$\p$ (circumcenter)}%
\indy{Index}{Tarski arithmetic}%
\begin{displaymath}
\op{orth}^0(\v_0,\v_1,\v_2,\v_3,c) = 
\op{conv}^0(\v_0,\v_0+\w_1,\v_0+\w_1+\w_2,\v_0+\w_1+\w_2+\w_3),
\end{displaymath}
\begin{equation}\label{eqn:ortho}
\w_1=(\v_0+\v_1)/2,\quad \w_1+\w_2=\p,\quad \w_1+\w_2+\w_3=\p'.
\end{equation}
%(Define $\op{orth}(\v_0,\ldots,\v_3,c)$ similarly, with
%$\op{conv}$ instead of $\op{conv}^0$.)
Take $\op{orth}^0$ to be the empty set, if $c< r$.
\indy{Index}{orthosimplex}%
\indy{Notation}{orthosimplex@$\op{orth}^0$}%
%\indy{Notation}{rogers simplex@$\op{orth}^0$}%
%\indy{Notation}{rogers simplex1@$\op{orth}$}%
\end{definition}

%\begin{definition}[orthosimplex, orth] \tlabel{def:ortho}
%An \newterm{orthosimplex} is a tetrahedron
%    \begin{displaymath}
%\op{conv}^0(\p,\p+\v_1,\p+\v_1+\v_2,\p+\v_1+\v_2+\v_3),
%\end{displaymath}
%    when $\v_i\cdot \v_j=0$, for $1\le i<j\le 3$.
%%Write $\op{orth-old}^0(\p,\v_1,\v_2,\v_3)$ for this orthosimplex.
% \indy{Index}{orthosimplex}%
%\end{definition}

\begin{figure}[htb]
\centering
%\myincludegraphics{\ps/rogers.eps}
\caption{An orthosimplex}
\tlabel{fig:rogers}
\end{figure}


\begin{lemma}\guid{XQBOVQF} 
The vectors $\w_1,\w_2,\w_3$ of Equation~\ref{eqn:ortho}
are mutually orthogonal.  \indy{Index}{orthogonality!mutual}%
\end{lemma}

\begin{proof} The orthogonality of $\w_1$ and $\w_2$ is 
Tarski arithmetic. %tarski: {eta-ortho}.  
The orthogonality of $\w_3$ with the others is also Tarski
arithmetic. %tarski: {rog-ortho}.
\end{proof}
\indy{Index}{Tarski arithmetic}%

\begin{definition}[circumradius, $\eta$]\label{def:etaV}
Let $\eta(x,y,z)$ be the circumradius of a triangle with sides
$x,y,z$, and let $\eta_V(\v_0,\v_1,\v_2) =
\eta(\norm{\v_0}{\v_1},\norm{\v_0}{\v_2},\norm{\v_1}{\v_2})$.
\indy{Index}{circumradius}%
\indy{Notation}{ZZeta@$\eta$(circumradius)}%
\indy{Notation}{ZZetav@$\eta_\v$}%
\end{definition}

\begin{definition}[$abc$ parameter]\tlabel{def:abc}
Associate with $\op{orth}^0(\v_0,\v_1,\v_2,\v_3,c)$ the constants
$a=\norm{\v_1}{\v_0}/2$, $b=\eta_V(\v_0,\v_1,\v_2)$, and $c$.
Call these the $abc$ parameters of $\op{orth}^0$.
\indy{Index}{abc parameter} %
\indy{Notation}{a@$a$ (constant)}%
\indy{Notation}{b@$b$ (constant)}%
\indy{Notation}{c@$c$ (constant)}%
\end{definition}

The simplex $\op{orth}^0$ is a tetrahedron.  Hence it is one of our
primitive regions.  It is eventually radial at $\v_0$, hence
it has a solid angle at $\v_0$.  The dihedral
angle, mentioned without further context, is understood to refer to 
\begin{displaymath}
\dih_V(\{\v_0,\v_1\},\{\v_2,\p'\})=\dih_V(\{\v_0,(\v_0+\v_1)/2\},\{\p,\p'\}),
\end{displaymath}
\indy{Notation}{dihv@$\dih_V$}%
where $\p$ and $\p'$ are the points 
constructed in Definition~\ref{def:orth}.

The squares of the edge lengths of the tetrahedron are
\begin{displaymath}
(a^2,b^2,c^2,c^2-b^2,c^2-a^2,b^2-a^2).
\end{displaymath}
Define the functions
\begin{displaymath}
\begin{array}{lll}
\op{volR}(a,b,c) &= \begin{cases}
a\sqrt{(b^2-a^2)(c^2-b^2)}/6& 0 < a < b < c,\\
0,&\text{otherwise}
\end{cases}\vspace{6pt}\\
\op{solR}(a,b,c) &= \begin{cases}
2 \atn\left(\sqrt{{(a+b)(b+c)}},
\sqrt{{(b-a) (c-b)}}\right).& 0 < a < b < c,\\
0,&\text{otherwise}
\end{cases} \vspace{6pt}\\
\op{dihR}(a,b,c) &= \begin{cases}
\atn\left(\sqrt{b^2 - a^2},\sqrt{c^2 - b^2}\right)
& 0 < a < b < c,\\
0,&\text{otherwise}
\end{cases}
\end{array}
\end{displaymath}
\indy{Notation}{volR@$\op{volR}$ (function)}%
\indy{Notation}{solR@$\op{solR}$ (function)}%
\indy{Notation}{dihR@$\op{dihR}$ (function)}%
Specialization of the formulas for dihedral angle, volume, and solid
angle to this setting gives following expressions for volume and solid
angle.  (The calculation of the solid angle formula is based on
Euler's formula in Lemma~\ref{lemma:euler}.)

\begin{lemma}\guid{JJFHOIW}\label{lemma:orth:abc} 
Let $R=\op{orth}^0(\v_0,\v_1,\v_2,\v_3,c)$ and let $a$, $b$,
$c$ be the $abc$-parameters of $R$.  Let $\op{dih}(R)$ be the dihedral
angle of $R$ along the edge extending along $\op{aff}\{\v_0,\v_1\}$.  Then
\begin{displaymath}
\begin{array}{lll}
\op{vol}(R) &= \op{volR}(a,b,c)\\
\op{sol}(\v_0,R) &= \op{solR}(a,b,c)\\
\op{dih}(R) &= \op{dihR}(a,b,c)\\
\end{array}
\end{displaymath}
\end{lemma}
\indy{Notation}{a@$a$ (constant)}%
\indy{Notation}{b@$b$ (constant)}%
\indy{Notation}{c@$c$ (constant)}%
\indy{Notation}{R@$R$}%

\section{Finiteness and Volume}

Previous sections have developed all of the volume calculations that will
be needed in this book.  This chapter concludes with some 
elementary estimates based on the volumes of  cubes and balls.

\begin{lemma}\guid{WQZISRI}
\oldrating{100}
\formalauthor{Nguyen Tat Thang}
\rating{0}
\tlabel{lemma:Zcount}
For all $\p\in\ring{R}^3$ and all $r\ge 0$, the set
$\ring{Z}^3\cap B(\p,r)$ is finite of cardinality at most
$4\pi (r+\sqrt3)^3/3$.
\indy{Index}{cardinality!finite}%
\end{lemma}

\begin{proof}  If $\v\in\ring{Z}^3\cap B(\p,r)$, then the ith
coordinate $v_i$ of $\v$ must lie in the finite range
\begin{displaymath}
p_i - r \le v_i \le p_i + r.
\end{displaymath}
Hence there are only finitely many possibilities for $\v$.


Place an open unit cube at each point of $\ring{Z}^3\cap B(\p,r)$.
The cubes are measurable, disjoint, and contained in
$B(\p,r+\sqrt3)$.  Thus, the combined volume of the cubes, which is
$\card(\ring{Z}^3\cap B(\p,r))$,  is no greater than the volume of the
containing ball.  The result follows.
\end{proof}

\begin{lemma}\guid{PWVIIOL}
\oldrating{100}
\formalauthor{Nguyen Tat Thang}
\rating{0}
\tlabel{lemma:Zlow-count}
For all $\p\in\ring{R}^3$ and all $r\ge\sqrt3$, the set
$\ring{Z}^3\cap B(\p,r)$ is finite of cardinality at least
$4\pi (r-\sqrt3)^3/3$.
\end{lemma}

\begin{proof} Lemma~\ref{lemma:Zcount} establishes finiteness.  Place
a closed unit cube at each point of $\ring{Z}^3\cap B(\p,r)$.  The
cubes are measurable and cover $B(\p,r-\sqrt3)$.  Thus, the combined
volume of the cubes is at least the volume of the covered ball.  The
result follows.
\end{proof}

\begin{lemma}\guid{TXIWYHI}
\oldrating{50}
\rating{0}
\formalauthor{Nguyen Tat Thang}
\tlabel{lemma:Zr2}
For all $\p\in\ring{R}^3$, and $r_0,r_1>0$, there exists a $C$ such
that for all $r\ge r_1$, 
\begin{displaymath}
\ring{Z}^3 \cap (B(\p,r+r_0) \setminus B(\p,r-r_1)) \le C r^2.
\end{displaymath}
\end{lemma}

\begin{proof}  When $r \ge r_1+\sqrt3$, the previous two lemmas show
that the cardinality is at most $4\pi/3$ times
\begin{displaymath}
(r +r_0 + \sqrt3)^3 - (r - r_1 - \sqrt3)^3 \le C' r^2
\end{displaymath}
for some $C'$.  Similarly, if $r_1\le r\le r_1+\sqrt3$, the
cardinality is at most some fixed constant $C''$.  The result
easily follows.
\end{proof}

