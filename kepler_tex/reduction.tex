%% Dimension Reduction and Interval Arithmetic

\chapter{Interval Arithmetic}%DCG 8.3, p75
\label{sec:bounds-simplex}


There is an archive of hundreds of
inequalities that have been proved by computer.  This full archive
appears in \cite{web}.  The justification of these inequalities
appears in the same archive.  (The proofs of these inequalities were
executed by computer.)  An explanation of how computers are able to
prove inequalities can be found in \cite{algorithm} and
\cite{part1}. Each inequality carries a nine-digit identifying
number. To invoke an inequality, we state it precisely, and give its
identifying number, e.g. \calc{123456789}. The first of these
appears in Lemma \ref{lemma:1pt}.  Some results rely on a simple
combination of inequalities, rather than a single inequality.  To
make it easier to reference a group of inequalities, the archive at
\cite{web} gives a separate nine-digit identifier to certain groups
of inequalities.  This permits us to reference such a group by a
single number.
%
 \index{calc@\calc{123456789}}


\chapter{Dimension Reduction}\label{chap:reduction}

There are a number of inequalities in this \paper that have been
established by interval arithmetic.  These interval arithmetic
calculations tend to require significant computer resources.  To
make these calculations run more quickly, we introduce various
simplifications that we call {\it dimension reduction}.

The basic idea is quite simple.  Suppose that we are trying to
bound the maximum of a continuous function on a compact space.  If
we know that the function is differentiable on a dense open subset
$U$ of the domain, and if a partial derivative has fixed sign on
that dense open subset, then we can conclude that the maximum is
achieved on the complement of $U$.   This complement generally has
lower dimension that the original compact space.  In this way, we
reduce the maximization problem to a set of smaller dimension.
This is what we call dimension reduction.

There are two parts to a dimension reduction argument.  The first
part is an analysis of the partial derivatives to prove
monotonicity.  The second part is a study of the complement of
$U$.  This complement often consists of several distinct
``irreducible'' components.  Each of these components becomes a
separate case of an optimization problem in lower dimension. Thus,
a dimension reduction arguments generally trades a single
optimization problem in higher dimension for a list of
optimization problems in lower dimension.

Dimension reduction arguments would never be necessary if
computers were sufficiently powerful to establish the desired
bounds directly on the full domain.  In practice, we run up
against the limitations of computers quite rapidly.  This \chap
provides the dimension reduction arguments that compensate for
these limitations.  It can be hoped with the continued improvement
of computers that this \chap will become an increasingly less
important component of the proof of the Kepler conjecture.

\section{The vanishing gradient}\label{sec:gradient}

This is the basic result in multivariable calculus that we will
use repeatedly.

\begin{lemma}  Let $X\subset\ring{R}^n$ be a compact domain
defined by a finite set $S$ of continuous constraints:
    $$X =\{x\in\ring{R}^n\mid f(x)\ge 0, \forall f\in S\}.$$
Let $f$ be a continuous function on $X$.  Suppose that $f$ is
continuously differentiable in the direction $x_i$ on
    $$X^0= \{x\in\ring{R}^n\mid f(x) > 0, \forall f\in S\}.$$
Then every point in $X$ that achieves the maximum of $f$ lies in
    $$X \setminus X^0.$$
\end{lemma}

%% XX There are variants. What if f goes to infinity at boundary.
%% XX What if the direction is not a coordinate direction, etc.
\begin{proof} This is basic calculus.
\end{proof}

\section{Vertex Push}\label{sec:push}

In this section, we analyze of the reduced domain in a simple
case.

\begin{lemma}  Let $X$ be a set of simplices with non-obtuse faces constrained by edge
lengths $y_i\in[a_i,b_i]$.  Let $b_i$ be the predicate $y_i =
a_i$.  Vertex pushes along the vertices $v_1$ $v_2$, and $v_3$
encounter the following seven boundary cases:
    $$
    \begin{array}{rlll}
    b_1 &\land b_4 \\
    b_2 &\land b_5 \\
    b_3 &\land b_6 \\
    b_4 &\land b_5 \\
    b_5 &\land b_6 \\
    b_6 &\land b_4 \\
    b_1 &\land b_2 \land b_3
    \end{array}
    $$
\end{lemma}

\begin{proof} Since the faces are non-obtuse, pushing along the
vertex $v_1$ decreases the edge lengths of $y_1$, $y_5$, and
$y_6$.  The boundary is reached with $b_1\lor b_5\lor b_6$.
Similarly, pushing the vertex $v_2$ (resp. $v_3$) gives the
boundary conditions
    $$b_2\lor b_4 \lor b_6 \quad \text{resp. } b_3\lor b_4\lor
    b_5.$$
Let $b$ be the disjunction of the seven conditions of the lemma.
The result follows from the tautology
    $$
    (b_1\lor b_5\lor b_6)\land (b_2\lor b_4 \lor b_6)\land (b_3\lor b_4\lor
    b_5) \Rightarrow b.
    $$
\end{proof}

\section{Dihedral Angle}\label{sec:reduction:dih}

The dihedral angle of a simplex along an edge $e$ is monotonic in
the edge length of the edge opposite $e$.  Thus, when computing
lower bounds on the dihedral angle, it is enough to consider the
case in which the edge opposite $e$ is as small as possible.

\section{Compression}\label{sec:reduction:grad}

Let $S$ be a quarter or quasi-regular tetrahedron. Suppose that
the lengths of the edges $y_1$, $y_5$, and $y_6$ are greater than
$2$. Let $S'$ be a simplex formed by contracting the vertex
joining edges $1$, $5$, and $6$ along the first edge by a small
amount. We assume that the new lengths  $y'_1$, $y'_5$, and $y'_6$
$S'$ again form a quarter or quasi-regular tetrahedron.

\begin{remark}  This deformation of simplices will reoccur several
times.  We call it {\it pushing a vertex}.
\end{remark}

\begin{lemma}  $\Gamma(S') > \Gamma(S)$.
\end{lemma}

We write $\sol_i$, for $i=1,2,3$, for the solid angles at the
three vertices $p_1$, $p_2$, and $p_3$ of $S$ terminating the
edges $1$, $2$, and $3$. Let $p_1'$ be the vertex terminating edge
$1$ of $S'$. Similarly, we write $\sol'_i$, for $i=1,2,3$, for the
solid angles at the corresponding vertices of $S'$.  We set
$\op{vol}(V) = \op{vol}(S)-\op{vol}(S')$ and $w_i =
\sol_i-\sol'_i$. It follows directly from the construction of $S'$
that $w_2$ and $w_3$ are positive. The dihedral angle $\alpha$
along the first edge is the same for $S$ and $S'$. The angle
$\beta_i$ of the triangle $(0,p_1,p_i)$ at $p_1$ is less than the
angle $\beta'_i$ of the triangle $(0,p'_1,p_i)$ at $p'_1$, for
$i=2,3$.  It follows that $w_1$ is negative, since $-w_1$ is the
area of the quadrilateral region of Figure~\ref{XX} on the unit
sphere.

%\gram|2|8.7.2|dia872.ps|

\begin{lemma}
$\delta_{oct} \op{vol}(V) > w_2/3 + w_3/3 $.
\end{lemma}

The lemma immediately implies the proposition because $w_1<0$ and
$$\Gamma(S') -\Gamma(S) = -w_1/3 -w_2/3 -w_3/3 + \delta_{oct}\op{vol}(V).$$


\begin{proof} Let $T'$ be the face of $S'$ with vertices $p'_1$,
$p_2$, and $p_3$. We consider $S'$ as a function of $t$, where $t$
is the distance from $p_1$ to the plane containing $T'$. (See
Figure~\ref{XX}.) It is enough to establish the lemma for $t$
infinitesimal.

As shown in the figure,  let $V_1$ be the pyramid formed by
intersecting $V$ with the plane through $p_1$ that meets the fifth
edge at distance $t_0=1.15$ from $p_3$ and the sixth edge at
distance $t_0$ from $p_2$. Also, let the intersection of $V$ with
a ball of radius $t_0$ centered at $p_i$ be denoted $V_i$, for
$i=2,3$.  For $t$ sufficiently small, the region $V_1$ is
(essentially) disjoint from $V_2$ and $V_3$.

We claim that $\op{vol}(V_1) > \op{vol}(V_2\cap V_3)$.
 Let $\theta$ be the angle of $T'$
subtended by the fifth and sixth edges of $S'$.  Then
$\op{vol}(V_1) = B t/3$, where $B$ is the area of the intersection
of $V_1$ and $T'$:
    \begin{equation}
    \label{eqn:874}
    B = \sin\theta (y_5'-t_0)(y_6'-t_0)/2.  %%\tag 8.7.4
    \end{equation}

Since $S'$ is a Delaunay simplex, the estimates
$\pi/6\le\theta\le2\pi/3$ from \cite{H1,2.3} hold.  In particular,
$\sin\theta\ge 0.5$, so $B\ge 0.25(2-1.15)^2$, and $\op{vol}(V_1)>
0.06\,t$.

If $\op{vol}(V_2\cap V_3)$ is nonempty, the fourth edge of $S$
must have length less than $2t_0$.  The dihedral angle $\alpha'$
of $V$ along the fourth edge is then less than the tangent of the
angle, which is at most $t+ {O}(t^2)$, in Landau's notation. As in
\cite{H1,5}, we obtain the estimate
$$\op{vol}(V_2\cap V_3) \le {\alpha'} \int_1^{t_0}
        t_0^2-t^2\,dt = \frac{ (t_0-1)^2(2t_0+1)\alpha'}{3}
        < 0.025\, t + {O}(t^2).$$
This establishes the claim.

Thus, for $t$ sufficiently small
   $$
   \begin{array}{lll}
   \delta_{oct}\op{vol}(V) &\ge \delta_{oct}(\op{vol}(V_1) +\op{vol}(V_2)
        +\op{vol}(V_3) - \op{vol}(V_2\cap V_3)) \\
        &>
        (\delta_{oct} t_0^3) \left( \frac{w_2}{3} + \frac{w_3}{3}\right)
        > 1.09 \left( \frac{w_2}{3} + \frac{w_3}{3}\right).
    \end{array}
    $$
\end{proof}


\section{Voronoi functions}

\begin{lemma}  Let $S= (v_1,v_2,v_3,v_4)$ be a quarter.  For each $i$,
pushing the vertex $v_i$ increases the function $\op{volan}(S)$.
\end{lemma}

\begin{proof}  An explicit formula for $\op{volan}(S)$ appears in
\ref{XX}.  From this explicit expressions for the derivatives can
be derived.
\end{proof}


The Voronoi function is also monotonic in a ``vertex push''


The truncated Voronoi function is also monotonic in a ``vertex
push''

The dihedral and solid angle functions are fixed under an ``vertex
push''

%% XX FILL IN

\section{Voronoi and Solid}

This section studies a deformation that preserves the area of a
quadrilateral, but changes its shape.  We will find that we can
deform in a direction that increases the truncated Voronoi
function.

The formula for the truncated Voronoi function on a quadrilateral
region is given by Equation~\ref{XX}.  We assume in this section
that truncation is given by $t=\sqrt2$.  Recall that the area of a
quadrilateral is given by $\sum_{i=1}^4 \dih_i - 2\pi$, where
$\dih_i$ are the dihedral angles.  Assume the corners
$v_1,\ldots,v_4$ of the quadrilateral all have the same height:
    \begin{equation}\label{eqn:ht2}
    |v_i-v_0| = 2,\text{ for } i=1,2,3,4
    \end{equation}
then the coefficients of the dihedral angles $\op{dih}_i$ are all
equal (to $c=(1-1/t)(\phi(1,t)-\phi(t,t))$), and the terms may be
combined into a term depending only on the solid angle:
    $$c (\sol + 2\pi).$$
Under these assumptions, the truncated Voronoi function can be
written as a the sum of function depending only on the solid angle
with
    $$
    -4 \doct \sum_R \quo(R).
    $$
Thus, if we deform the quadrilateral in a solid-angle preserving
way, the change in the truncated Voronoi function is the same as
the change in the quoin terms.  This sum runs over two quoins
along each of four faces.  Under the hypothesis of
Equation~\ref{eqn:ht2}, the two quoins on each face are equal. The
dependence then takes the form $-8\doct$ times the sum of four
terms $\quo(R)$.

We fix a diagonal of the quadrilateral cluster to break it into
two simplices.

\begin{lemma}  Let $S=S(2,2,2,y_4,y_5,y_6)$ be a simplex whose
edges satisfy the constraints $\sqrt2\le y_4\le 2.51\sqrt2$,
$y_5,y_6\in [4/2.51,2.51]$.  Assume that $y_5 < y_6$.  The
truncated Voronoi function (with truncation $t=\sqrt2$) increases
with increasing $y_5$ and decreasing $y_6$ along a curve in
$(y_5,y_6)$ that preserves the solid angle of $S$.
\end{lemma}



\begin{proof}
We study the deformation of a single simplex under the constraint
of fixed solid angle
    $$\sol(S(2,2,2,y_4,y_5,y_6)) = c.$$
Set $x_i = y_i^2$.  If we fix $y_4$ in this constraint, we can
take the implicit derivative of $y_6$ with respect to $y_5$ to
obtain (via the explicit formula for $\sol$ in Section~\ref{XX}):
    $$
    \frac{y_6}{y_5} = -\frac{y_5 (16- x_6)(x_6 + x_4 - x_5 )}
    {y_6 (16-x_5)(x_5+x_4 - x_6)}
    $$

As we mentioned the dependence on quoins takes the form $-8\doct$
times the sum of four terms $\quo(R)$.  Each of the two simplices
carries two of the four quoins.  The change along a
solid-angle-preserving deformation  is
    $$
    \frac{d \op{quo}(R(1,\eta(2,2,y_5),\sqrt2))}{d y_5} + \frac{d
    \op{quo}(R(1,\eta(2,2,y_6),\sqrt2))} {d y_6} \frac{d y_6}{d y_5} =
    $$
This may be calculated explicitly from Equation~\ref{XX}.
Simplifying and clearing a positive denominator, we find that the
truncated Voronoi function of a single simplex is increasing along
a solid angle-preserving deformation exactly when for $x_5 < x_6$
we have
    $$
    (16-x_5)^2 x_5 (8-x_6)^3 (x_4 - x_5 + x_6)^2 <
    (16-x_6)^2 x_6 (8-x_5)^3 (x_4 - x_6 + x_5)^2.
    $$

This follows from the fact that the polynomials
    $$
    \frac{h(x_5,x_6) - h(x_6,x_5)}{x_5 - x_6}
    $$
are positive on the indicated range for
    $$
    h(x_5,x_6) = (16-x_5) x_5 (8-x_6) (x_4 - x_5 + x_6)
    $$
and
    $$
    h(x_5,x_6) = (16-x_5) (8-x_6)^2 (x_4 - x_5 + x_6)
    $$
This is easily checked.
\end{proof}



\section{Some deformations} %DCG 12.8, p132
    \oldlabel{4.9}

\begin{definition} %\label{def:concave} % REPEATED IN hypermap.tex
Consider three consecutive corners $v_3,v_1,v_2$ of a subcluster
$R$ such that the dihedral angle of $R$ at $v_1$ is at least
$\pi$.  We call such an corner {\it concave}.  (If the angle is
less than $\pi$, we call it {\it convex}.)  Similarly, the angle
of a subregion is said to be convex or concave depending on
whether it is less than or greater than $\pi$.
\index{concave}\index{convex}
\end{definition}

Let
    $S=S(y_1,\ldots,y_6)=\{0,v_1,v_2,v_3\}$, $y_i=|v_i-v_0|$.
Suppose that $y_6>y_5$.  Let $x_i=y_i^2$.

\begin{lemma}
    \oldlabel{4.9.1}
At a concave vertex, $\partial \op{sv}_0/\partial x_5 >0$ and
    $\partial \op{sovo}(v,\cdot,\lambda_{sq})_0/\partial x_5<0$.
%% XX Check notation on the function
\end{lemma}

\begin{proof}
As $x_5$ varies, $\dih_i(S)+\dih_i(R)$ is constant for $i=1,2,3$. The
part of Formula~\ref{eqn:3.7} for $\op{sv}_0$ that depends on $x_5$ can be
written
    $$-B(y_1)\dih(S)-B(y_2)\dih_2(S)-B(y_3)\dih_3(S)-4\doct
        (\quo(R_{135})+\quo(R_{315})),
    $$
where $B(y_i)=A(y_i/2)+\phi_0$, $R_{135}=R(y_1/2,b,t_0)$,
$R_{315}=R(y_3/2,b,t_0)$, $b=\eta(y_1,y_3,y_5)$, and $A(h) =
(1-h/t_0)(\phi(h,t_0)-\phi_0)$. Set $\ups_{135}=\ups(x_1,x_3,x_5)$, and
$\Delta_i = \partial \Delta/\partial x_i$. (The notation comes from
\cite[Sec.~8]{part1} and Section~\ref{sec:scoring}.) We have
    $$\frac{\partial \quo(R(a,b,c))}{\partial b} =
        \frac{-a (c^2-b^2)^{3/2}}{3 b (b^2-a^2)^{1/2}}\le 0
    $$
and $\partial b/\partial x_5\ge0$.  Also, $u\ge0$, $\Delta\ge0$ (see
\cite[Sec.~8]{part1}).  So it is enough to show
    $$V_0(S) = \ups_{135}\Delta^{1/2}
        \frac{\partial}{\partial x_5} (B(y_1)\dih(S)+B(y_2)\dih_2(S)
        + B(y_3)\dih_3(S))< 0.
    $$
By the explicit formulas of \cite[Sec.~8]{part1}, we have
    $$
    V_0(S) = -B(y_1)y_1\Delta_6 + B(y_2)y_2 \ups_{135} - B(y_3)y_3 \Delta_4.
    $$
For $\tau_0$, we replace $B$ with $B-\zeta\pt$. It is enough to
show that
    $$
    V_1(S) = -(B(y_1)-\zeta\pt)y_1\Delta_6 + (B(y_2)-\zeta\pt)y_2 \ups_{135} -
        (B(y_3)-\zeta\pt)y_3 \Delta_4<0.
    $$
The lemma now follows from an interval calculation.
%\footnote{\calc{984628285}} %A14 $\A_{14}$.
We note that the polynomials $V_i$
are linear in $x_4$, and $x_6$, and this may be used to reduce the
dimension of the calculation.
\end{proof}

We give a second form of the lemma when the dihedral angle of $R$ is
less than $\pi$, that is, at a convex corner.


\begin{lemma}
    \oldlabel{4.9.2}
At a convex corner, $\partial \op{sv}_0/\partial x_5 <0$ and
    $\partial \tau_0/\partial x_5>0$, if $y_1,y_2,y_3\in[2,2t_0]$,
$\Delta\ge0$, and (i) $y_4\in[2\sqrt{2},3.2]$, $y_5,y_6\in[2,2t_0]$, or
(ii) $y_4\ge 3.2, y_5,y_6\in[2,3.2]$.
\end{lemma}

\begin{proof}
We adapt the proof of the previous lemma.  Now
$\dih_i(S)-\dih_i(R)$ is constant, for $i=1,2,3$, so the signs change.
$\op{sv}_0$ depends on $x_5$ through
$$\sum B(y_i)\dih_i(S) - 4\doct (\quo(R_{135})+\quo(R_{315})).$$
So it is enough to show that
    $$
    V_0 - 4\doct\Delta^{1/2}\ups_{135}\frac{\partial}{\partial x_5}
    (\quo(R_{135})+\quo(R_{315}))<0.
    $$
Similarly, for $\tau_0$, it is enough to show that
    $$
        V_1 - 4\doct\Delta^{1/2}\ups_{135}\frac{\partial}{\partial x_5}
    (\quo(R_{135})+\quo(R_{315}))<0.
    $$
By an interval calculation\footnote{\calc{984628285}} %A14
    $$
    \begin{array}{lll}
    -4\doct  \ups_{135}\frac{\partial\phantom{x}}{\partial x_5}
    (\quo(R_{135})+\quo(R_{315}))&< 0.82,\quad\hbox{on } [2,2t_0]^3,\\
                            &<0.5,\quad\hbox{on }[2,2t_0]^3, y_5\ge2.189.
    \end{array}
    $$
The result now follows from
the inequalities.\footnote{\calc{984628285}} %A14
\end{proof}

Return to the situation of concave corner $v_1$. Let $v_2$, $v_3$ be the
adjacent corners. By increasing $x_5$, the vertex $v_1$ moves away from
every corner $w$ for which $\{v_1,w\}$ lies outside the region.  This
deformation then satisfies the constraint of
Remark~\ref{remark:proof-2}. Stretch the shorter of $\{v_1,v_2\}$,
$\{v_1,v_3\}$ until $|v_1-v_2|=|v_1-v_3|=3.07$ (or until a new
distinguished edge forms, etc.).  Do this at all concave corners.

By stopping at $3.07$, Lemma~\ref{tarski:307} implies that we prevent a corner crossing an edge from
outside-in. 

If $|v_1-v_0|\ge2.2$, we can continue the deformations even further. We
stretch the shorter of $\{v_1,v_2\}$ and $\{v_1,v_3\}$ until
$|v_1-v_2|=|v_1-v_3|=3.2$ (or until a new distinguished edge forms,
etc.).  Do this at all concave corners $v_1$ for which $|v_1-v_0|\ge2.2$.  In this case, Lemma~\ref{tarski:22} implies that that corners cannot cross an edge from the outside-in.






