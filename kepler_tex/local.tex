%
\chapter{Local Fan}\label{sec:local}


\begin{summary}
  The difficult technical estimates that we need for the proof of the
  Kepler conjecture are found in this chapter.  The main estimate
  (Lemma~\ref{lemma:empty-d}) takes the form
\begin{equation}\label{eqn:main}
\tau(V,E,F) \ge d(r,s),
\end{equation}
for every special fan $(V,E,F,S)$ with parameters $(r,s)$.  Here
$(V,E)$ is a fan, $F$ is a face of the hypermap of $(V,E)$, $S$ is a
subset of $E$, and $(r,s)$ is an ordered pair of natural numbers.  The
function $d$ is defined by a table of non-negative values for various
$(r,s)$.  Heuristically, the real-valued function $\tau$ measures the
looseness of a packing.  Large values of $\tau$ indicate that the
points of $V$ are loosely arranged around the face and large values of
$\tau$ indicate a tight packing.  The main estimate~\ref{eqn:main}
gives limits to the tightness of a packing, with the eventual aim of
showing that no packing can have density greater than the
face-centered cubic packing.  The next chapter will show that all
packings of concern to us are special, so that the main estimate is in
fact a general estimate.

This chapter also proves the well-know result that the perimeter of a
geodesically convex spherical polygon is never greater than $2\pi$,
the length of a great circle.

Because of the technical nature of this chapter, the final section gives a summary
of the results.
\end{summary}


\section{Localization}



In various proofs, it is useful to focus attention on a single face
in a fan.  This leads to the notion of a localization of a fan along
a face, which is again a fan.  Useful properties of the
localized fan are captured in the notion of a convex local fan.  


\subsection{basics}

A local fan is to a fan what a polygon is to a plane graph.  


\begin{definition}[local fan]\guid{FTNGOGF} \label{def:convex-local}
\formaldef{local fan}{local\_fan}
\formaldef{convex local fan}{convex\_local\_fan}
A triple $(V,E,F)$ is a \newterm{local fan} if the following conditions hold.
\begin{description} 
\item \case{fan} $(V,E)$ is a fan;
\item \case{face} $F$ is a face of $H = \op{hyp}(V,E)$;
\item \case{dihedral} $H$ is isomorphic to $\op{Dih}_{2k}$, where $k =
\card(F)$;
\end{description}
A local fan $(V,E,F)$ is said to be \newterm{convex} if the following
additional conditions hold.
\begin{description}
\item \case{angle} $\op{azim}(x)\le \pi$ for all darts $x\in F$;
\item \case{wedge} $V\subset \bWdart(x)$ for all $x\in F$.
% \item If $\{\v,\w\}\in E$, then $\{\orz,\v,\w\}$ is not
%   collinear. %% part of def of fan.
\end{description}
\end{definition}
\indy{Index}{fan!local}%

In the proof of the Kepler conjecture in this chapter and the next,
all local fans are convex.  Local fans (that are not convex) appear in
applications to other packing problems in Chapter~\ref{sec:further}.

\begin{remark}[visualization]\guid{PNCVUMY}
  The intersection of $X(V,E)$ with the unit sphere is a spherical
  polygon, when $(V,E,F)$ is a convex local fan.  The spherical
  polygon gives a visual representation of the convex local fan. The
  choice of $F$ distinguishes the interior of the polygon from its
  exterior.
\end{remark}

%% XX background cyclic permutation.

\begin{lemma}[]\guid{WRGCVDR}\rating{ZZ}  
For any convex local fan $(V,E,F)$, there is a bijection from $F$ onto $V$
given by
\begin{displaymath}
(\v,\w) \mapsto \v
\end{displaymath}
Moreover, write $\v\mapsto(\v,\rho\v)$ for the inverse map. 
Then $\rho:V\to V$ is a cyclic permutation.
\end{lemma}
\indy{Index}{cyclic permutation}%

\begin{proof} The map from a face to the set of nodes is a bijection
  for the dihedral hypermap $\op{Dih}_{2k}$. It is a also bijection
  for a fan isomorphic to $\op{Dih}_{2k}$.

For all $(\v,\rho \v)\in F$,
\begin{displaymath}
f(\v,\rho \v) = (\rho\v,\rho^2\v),
\end{displaymath}
so that the order of $\rho$ on $V$ is the order of $f$ on $F$, which
is $k=\card(F)$.  Thus, $\rho$ is a cyclic permutation of $V$ of order
$k=\card(V)$.
\end{proof}

\begin{definition}[$\rho$,~$\nd$]\guid{MFMPCVM} 
\formaldef{$\rho$}{rho\_node}
\formaldef{$\nd$}{FST}
For any convex local fan $(V,E,F)$, write
$\nd:F\to V$ and $\rho:V\to V$ for the bijections of the preceding
lemma.
\end{definition}
\indy{Notation}{ZZrho@$\rho$}%
\indy{Notation}{node@$\nd$}%

\begin{definition}[interior angle,~$\angle$,~$\bWdart$]\guid{PJRIMCV}
\formaldef{interior angle}{interior\_angle}
\formaldef{$\bWdart$}{cw\_node\_fan}
\formaldef{$\Wdart$}{w\_node\_fan}
For any convex local fan $(V,E,F)$,
write
\begin{displaymath}
\angle(\v) = \op{azim}((\v,\rho\v)),
\end{displaymath}
for all $\v\in V$.  This is the \newterm{interior} angle of the convex
local fan at $\v$.  Also, write
\begin{displaymath}
  \Wdart(F,\v) = \Wdart((\v,\rho \v)),\quad 
\bWdart(F,\v) =\bWdart((\v,\rho \v)).
\end{displaymath}
\indy{Notation}{1@$\angle$}%
\indy{Notation}{Wdart@$\Wdart(F,\v)$}%
\indy{Notation}{Wdart@$\bWdart(F,\v)$}%
%\indy{Notation}{node@$\nd(x)$}%
\end{definition}


\begin{definition}[localization]\guid{BIFQATK}
\formaldef{localization}{localization}
 Let $(V,E)$ be a fan, and let $F$ be
a face of $\op{hyp}(V,E)$.  Let
\begin{displaymath}
\begin{array}{rll}
V' &= \{\v\in V \mid \exists~\w\in V.~(\v,\w)\in F\}.\\
E' &= \{\{\v,\w\} \in E\mid (\v,\w)\in F\}.
\end{array}
\end{displaymath}
The triple $(V',E',F)$ is called the \newterm{localization} of $(V,E)$ along $F$.
\end{definition}
\indy{Index}{localization}%


\begin{lemma}[localization]\guid{LVDUCXU}\rating{ZZ}
\label{lemma:localization}
Let $(V,E)$ be any fan, and let $F$ be a face of its hypermap that is
simple and has cardinality at least $3$.  Then the localization
$(V',E',F)$ is a local fan.  Moreover, the angle $\op{azim}(x)$ and
the wedges $\bWdart(x)$ and $\Wdart(x)$ do not depend whether they
computed relative to $\op{hyp}(V,E)$ or to $\op{hyp}(V',E')$, for all
$x\in F$.
%That is, 
%\begin{displaymath}
%\begin{array}{lll}
%\op{azim}(V,E,x) &=\op{azim}(V',E',x)\\
%\Wdart(V,E,x) &=\Wdart(V',E',x)\\
%\end{array}
%\end{displaymath}
\end{lemma}



\begin{proof}
The proof that $(V',E')$ is a fan consists of various simple
verifications based on the techniques of
Remark~\ref{remark:fan-verify}.  The details are left to the reader.

The dart set $D'$ of $\op{hyp}(V',E')$ is naturally identified with
the disjoint union $F\coprod F'$, where $F = \{(\v,\rho\v) \mid \v\in
V\}$ and $F'=\{(\v,\rho^{-1}\v) \mid \v\in V\}$.  Under this
identification, $F$ is a face of $\op{hyp}(V',E')$.  The face, node,
and edge permutations have orders $k$, $2$, and $2$, respectively.  By
Lemma~\ref{lemma:dih-iso}, this bijection extends to an isomorphism of
hypermaps $\op{Dih}_{2k}$ onto $\op{hyp}(V',E')$.

The proof that the $\op{azim}(x)$ and $\Wdart(x)$ do not depend on the
choice of fan is a consequence of their definitions:
\begin{displaymath}
\begin{array}{lll}
\op{azim}(x) &= \op{azim}(\orz,\v,\w,\sigma(\v,\w)),\\
\Wdart(x) &= \Wdart(\orz,\v,\w,\sigma(\v,\w)).\\
\end{array}
\end{displaymath}
where $x = (\v,\w)$.  It is enough to check that $\sigma(\v,\w)\in
E'(\v)$.  But $\{\sigma(\v,\w),\v\}\in F$, so this is indeed the case.
% In particular $\op{azim}(x)\le\pi$ follow by the assumptions of the
% lemma.
\end{proof}



\subsection{geometric type}\label{sec:types}

\begin{definition}[generic,~lunar,~circular]\guid{RTPRRJS}
\formaldef{generic}{is\_generic\_clf}
\formaldef{lunar}{is\_lunar\_clf}
\formaldef{circular}{is\_circular\_clf}
A convex local fan $(V,E,F)$ is \newterm{generic}  if for every $\{\v,\w\}\in E$
and every $\u\in V$, 
\begin{displaymath}
C\{\v,\w\}\cap C^0_-\{\u\} = \emptyset.
\end{displaymath}
A convex local fan is \newterm{circular}  if there exists $\u\in V$ and
$\{\v,\w\}\in E$ such that
\begin{displaymath}
C^0\{\v,\w\}\cap C^0_-\{\u\}\ne \emptyset.
\end{displaymath}
A convex local fan is \newterm{lunar} with pole $\{\v,\w\} \subset V$ if it is not
circular, if $\v\ne\w$, and if $\{\v,\w\}$ is a parallel set.
\end{definition}
\indy{Index}{generic}%
\indy{Index}{lunar}%
\indy{Index}{circular}%


\begin{lemma}[trichotomy]\guid{CIZMRRH}\rating{ZZ} Every convex local fan is either
generic, lunar, or circular.  Moreover, these three properties are
mutually exclusive.
\end{lemma}
\indy{Index}{fan!local}%
\indy{Index}{convex local fan}%
\indy{Index}{generic}%
\indy{Index}{lunar}%
\indy{Index}{circular}%

\begin{proof} If $(V,E,F)$ is not generic,  select some $\{\v,\w\}\in E$
and some $\u\in V$ such that
\begin{equation}\label{eqn:non-generic}
C\{\v,\w\}\cap C^0_-\{\u\} \ne \emptyset.
\end{equation}
Now $C\{\v,\w\} = C^0\{\v,\w\} \cup C\{\v\}\cup C\{\w\}$.  If, for
some such triple $(\u,\v,\w)$, the
intersection~(\ref{eqn:non-generic}) meets $C^0\{\v,\w\}$, then the
convex local fan is circular.  Otherwise, the convex local fan is lunar.
\end{proof}

\begin{definition}[flat]\guid{YPSTLXA}
\formaldef{flat}{is\_flat}
 Let $(V,E,F)$ be a convex local fan.
If $\angle(\v)=\pi$, then $\v$ is \newterm{flat}.
\end{definition}
\indy{Index}{flat (node of a fan)}


\begin{lemma}[]\guid{LDURDPN}\rating{ZZ}  \label{lemma:coplanar}
Assume that $\{\orz,\u,\w\}$ and $\{\orz,\u,\v\}$ are not collinear sets.
Then $\op{azim}(\orz,\u,\v,\w)=\pi$ if and only if
there exists a plane $A$ such that $\{\orz,\u,\v,\w\}\subset A$
and such that the line $\op{aff}\{\orz,\u\}$ separates $\v$ from
$\w$ in $A$.
\end{lemma}

\begin{proof} The given azimuth angle is $\pi$ if and only if
$\dih(\{\orz,\u\},\{\v,\w\})=\pi$.  This holds exactly when $\{\orz,\u,\v,\w\}$ is
coplanar and the line $\op{aff}\{\orz,\u\}$ separates $\v$ from $\w$
in $A$.
\end{proof}

\begin{lemma}[]\guid{KOMWBWC}\rating{ZZ}\label{lemma:kom}
Let $(V,E,F)$ be a convex local fan.  Let $k=\card(F)$.  Assume that for
some $0<r\le k-1$ and some $\v\in V$, the set $U=\{\v,\rho
\v,\ldots,\rho^r \v\}$ is contained in a plane $A$ passing through
$\orz$.  Let $\e$ be the unit normal to $A$ in the direction
$\v\times \rho \v$.  Then the set $U$ is cyclic with respect to
$(\orz,\e)$, and the azimuth cycle $\sigma$ on $U$ is
\begin{displaymath}
  \sigma \u = 
\begin{cases} 
\rho \u, & \u\ne \rho^r\v,\\ \v, & \u = \rho^r\v.
\end{cases}
\end{displaymath}
Furthermore, for all $0\le i\le r-1$,
\begin{displaymath}
(\rho^i \v\times\rho^{i+1}\v)\cdot \e > 0.
\end{displaymath}
\end{lemma}

\begin{proof} 
Write $\v_i = \rho^i \v$, for $i=0,\ldots,r$.

\claim{ $(\v_i \times \v_{i+1})\cdot \e > 0$, for all $i\le r-1$. }
Indeed, the base case $(\v_0\times \v_1)\cdot \e > 0$ of an induction
argument holds by assumption.  Assume for a contradiction that the
inequality holds for $i$, but not for $i+1$.  Then
\begin{displaymath}
  \op{aff}^0_+(\{\orz,\v_{i+1}\},\v_i) 
= \op{aff}^0_+(\{\orz,\v_{i+1}\},\v_{i+2}).
\end{displaymath} 
This forces $C^0\{\v_i,\v_{i+1}\}$ to meet $C^0\{\v_{i+1},\v_{i+2}\}$,
which is contrary to the definition of a fan.  Thus, the claim holds.

The fact that $U$ is cyclic follows trivially from the fact that $U$
is contained in a plane $A$ through $\orz$ and that $\e$ is orthogonal
to $A$.

\claim{For all $0\le i \le r-1$,  $\sigma \v_i = \v_{i+1}$.}
Otherwise, there is some 
\begin{displaymath}
  \u \in (U\setminus \{\v_i,\v_{i+1}\}) 
\cap W^0(\orz,\e,\v_i,\v_{i+1}) \cap A.
%  ~~\subset~~ 
\end{displaymath}
However, by the claim, this intersection is a subset of $C^0\{\v_i,\v_{i+1}\}$, and
$\u\in C^0\{\v_i,\v_{i+1}\}$ is contrary to the property
\case{intersection} of fans.  The result follows.
% The membership of $\u$ in the rightmost term is contrary to the
% definition of fan.  Let $\e_3$ be the unit vector in the direction
% $\v\times \u$.  Let $\e_1$ be the unit vector in the direction $\u$.
% Let $\e_2 = \e_3 \times \e_1$.  The coordinates of $\u,\v,\w$ with
% respect to the frame $(\e_1,\e_2,\e_3)$ take the form
%\begin{displaymath}
%\begin{array}{rllrl}
%\v &= * \e_1 + a \e_2,  &\quad & a &< 0\\
%\u &= a' \e_1, &\quad & a' &>0\\
%\w&= * \e_1 + a'' \e_2, &\quad & a'' &>0\\
%\end{array}
%\end{displaymath}
%From this representation it is clear that $\v\times \u$ points in the
% same direction as $\u\times \w$.  The set $\{\v,\u,\w\}$ is clearly
% cyclic and the counterclockwise cycle $\sigma$ in the
% $\{\e_1,\e_2\}$ plane takes $\v$ to $\u$ and $\u$ to $\w$.
\end{proof}

\begin{lemma}[]\guid{OZQVSFF}\rating{ZZ} \label{lemma:A}  
Let $(V,E,F)$ be a convex local fan and let
  $\u,\v,\w\in V$ satisfy
\begin{itemize}
\item $\{\orz,\u,\v,\w\}$ is contained in a plane $A$; \vspace{3pt}
\item $\u,\w\not\in\op{aff}\{\orz,\v\}$; and \vspace{3pt}
\item $\op{aff}^0_+(\{\orz,\v\},\u) \ne \op{aff}^0_+(\{\orz,\v\},\w)$.
\end{itemize}
Then $\v$ is flat.  Moreover, $\rho \v,\rho^{-1} \v\in A$.
\end{lemma}

\begin{proof} Let $x = (\v,\rho\v)\in F$.  
Order $\u$ and $\w$ so that
\begin{displaymath}
\op{azim}(\orz,\v,\rho\v,\u) \le \op{azim}(\orz,\v,\rho\v,\w).
\end{displaymath}
By the definition of convex local fan, by the conditions $\u,\w\in \bWdart(x)$, 
and by  Lemma~\ref{lemma:coplanar},
%By the assumptions, $\dih(\{\orz,\v\},\{\u,\w\})=\pi$.  Since
%$\u,\w\in \bWdart(x)$, it follows that
%\begin{displaymath}\pi = \dih(\{\orz,\v\},\{\u,\w\}) \le 
%\op{azim}(x) = \angle(\v) \le \pi.\end{displaymath}
%The first conclusion follows.
\begin{eqnarray*}
%\begin{array}{rll}
0 &\le& \op{azim}(\orz,\v,\rho\v,\u) \\
&=& \op{azim}(\orz,\v,\rho\v,\w) - \op{azim}(\orz,\v,\u,\w)\\
&=& \op{azim}(\orz,\v,\rho\v,\w)-\pi \\
&\le& \op{azim}(\orz,\v,\rho\v,\rho^{-1}\v) - \pi \\
&=&\op{azim}(x) - \pi \\
&=&\angle(\v)-\pi\\
&\le& 0. 
%\end{array}
\end{eqnarray*}
Hence each inequality is equality.  In particular, $\v$ is flat.
In particular, $0 =\op{azim}(\orz,\v,\rho\v,\u)$, so that 
\begin{displaymath}
\rho\v\in \op{aff}_+(\{\orz,\v\},\u) \subset A.
\end{displaymath}
Similarly,
\begin{displaymath}
\rho^{-1}\v\in \op{aff}_+(\{\orz,\v\},\w) \subset A.
\end{displaymath}
\end{proof}

If Lemma~\ref{lemma:A} can be applied once to a set of vectors, then
it can often be applied repeatedly along a chain of vectors.  For
example, the conclusion of the lemma implies that $\rho^{-1} \v \in
A$.  In fact, by the definition of fan,
\begin{displaymath}
  \rho^{-1}\v \in A \setminus \op{aff}\{\orz,\v\} 
= \op{aff}^0_+(\{\orz,\v\},\u) \cup \op{aff}^0_+(\{\orz,\v\},\w).
\end{displaymath}
Suppose that $\rho^{-1} \v$ lies in the second term of the union.  If
$\w\ne \rho^{-1}\v$, then the assumptions of the lemma are met for
$\{\w,\rho^{-1} \v,\v\}$, giving the conclusions that $\rho^{-1} \v$
is flat, and $\rho^{-2}\v\in A$.  Repeating the argument
on a new set of vectors, we obtain a chain
\begin{displaymath}
\pi=\angle(\v) = \angle(\rho^{-1} \v) = \cdots,
\end{displaymath}
with $\v,\rho^{-1}\v,\ldots\in A$.  Another chain $\v,\rho \v,\ldots$
of vectors can be constructed in the other direction.  This process of
chaining gives the following lemma.

\begin{lemma}[circular geometry]\guid{KCHMAMG}\rating{ZZ}
\label{lemma:circular}
Let $(V,E,F)$ be a circular fan. Then
\begin{itemize}
\item $\v$ is flat for all $\v\in V$.
\item The set $V$ lies in a plane $A$ through $\orz$.
\item For some choice of unit vector $\e$ orthogonal to $A$, the set
$V$ is cyclic with respect to $(\orz,\e)$, and the azimuth cycle on
$V$ coincides with $\rho:V\to V$.  
\item 
$
\op{azim}(\orz,\e,\v,\rho\v) = \dih(\{\orz,\e\},\{\v,\rho\v\})
=\op{arc}_V(\orz,\{\v,\rho \v\}) <\pi
$.
\end{itemize}
\end{lemma}

\begin{proof} Let $\v, \u\in V$ be such that $C^0\{\u,\rho \u\}$ meets
$C^0_-\{\v\}$.  Apply Lemma~\ref{lemma:A} to $\{\u,\v,\rho \u\}$ to
conclude that $\v$ is flat, and some plane $A$ contains
$\{\orz,\u,\rho\u,\v,\rho \v,\rho^{-1} \v\}$.  If
%\begin{displaymath}
%\{\orz,\u,\rho \u,\v,\rho \v,\rho^{-1} \v\} \subset V\cap A,
%\end{displaymath}
%and 
$\w\in V\cap A$, then there exists $\w_1,\w_2\in (V\cap
A)\setminus\{\w\}$ for which the assumptions of Lemma~\ref{lemma:A}
hold for $\w_1,\w,\w_2$.  
Then $\w$ is flat, and $\rho \w \in V\cap A$.  The set $V\cap
A$ is therefore preserved by $\rho$.  By observing that $V$ is the
only nonempty subset of $V$ that is preserved by $\rho$, it follows
that $V\subset A$ and that $\w$ is flat for all $\w\in V$.


By Lemma~\ref{lemma:coplanar}, $V$ is cyclic with respect to a unit
vector $\e$ orthogonal to $A$.  The azimuth cycle on $V$ is $\v
\mapsto \rho \v$.

We turn to the final conclusion.  By the final conclusion of
Lemma~\ref{lemma:kom}, and Lemma~\ref{lemma:sim}, the azimuth angle is
less than $\pi$.  Under this constraint, the azimuth angle equals the
dihedral angle by Lemma~\ref{lemma:dih-azim}.  By definition, the
dihedral angle is the angle $\arc_V(\orz,*)$ of an orthogonal
projection of $\{\v,\rho\v\}$ to a plane with normal $\e$.  But
$\{\v,\rho\v\}$ is already a subset of the plane $A$, so that the
projection is the identity map, and the dihedral angle is
$\arc_V(\orz,\{\v,\rho\v\})$.
%
%\begin{displaymath}
%%  \op{azim}(\orz,\e,\v,\rho\v) = \dih(\{\orz,\e\},\{\v,\rho\v\}) =
%%  \op{arc}_V(\orz,\{\v,\rho\v\}),
%\end{displaymath}
%because the azimuth angle is less than $\pi$ and
%the dihedral angle is defined as the angle obtained by orthogonal projection
%to the plane $A$.  Since the points $\v,\rho\in A$, orthogonal projection has no effect.
\end{proof}

\begin{lemma}[lunar geometry]\guid{HKIRPEP}\rating{ZZ}\label{lemma:lunar}
Let $(V,E,F)$ be a lunar fan with pole $\{\v,\w\}\subset V$.  
%Assume that $\rho^r \v = \w$, for some $0< r < k$.
Then
\begin{itemize}
\item $\u$ is flat, for all $\u\in V\setminus \{\v,\w\}$; \vspace{3pt}
\item $0< \angle(\v) = \angle(\w)\le \pi$; \vspace{3pt}
\item $V\cap \op{aff}_+(\{\orz,\v\},\rho \v) = \{\v,\rho \v,\ldots,
\w\}$; \vspace{3pt}
\item $V \cap \op{aff}_+(\{\orz,\v\},\rho^{-1} \v) = \{\w,\rho
\w,\ldots \v\}$; \vspace{3pt}
\end{itemize}
\end{lemma}

\begin{proof} Set $V_1 = \{\v,\rho \v,\ldots,\w\}$ and $V_2 =
\{\w,\rho \w,\ldots,\v\}$.  Let $\u\in V\setminus\{\v,\w\}$ be
arbitrary.  Apply Lemma~\ref{lemma:A} to the set $\{\v,\u,\w\}$ to
find that $\u$ is flat, and that $\{\orz,\u,\rho \u,\rho^{-1} \u\}$
belongs to a plane $A(\u)$.  Now $A(\u)$ and $A(\rho \u)$ are both
the unique plane containing $\{\orz,\u,\rho \u\}$, hence $A(\u) =
A(\rho \u)$ when $\rho \u\not\in \{\v,\w\}$.  By induction, there
are planes $A_1, A_2$ such that $V_i\subset A_i$.  There is an
azimuth cycle $\sigma_i$ on $V_i$ such that $\sigma_i \u = \rho \u$,
when $\u \in A_i\setminus \{\v,\w\}$.

The angles $\angle(\v)$ and $\angle(\w)$ are both equal to the
dihedral angle between the half-planes
$\op{aff}_+(\{\orz,\v\},\rho^{\pm}\v)$.  In particular,
$0<\angle(\v)=\angle(\w)\le\pi$.
\end{proof}




\begin{lemma}[monotonicity]\guid{EGHNAVX}\rating{ZZ} 
\label{lemma:monotone}
Let $(V,E,F)$ be a convex local fan. Fix $\v_0\in V$.  Assume that
$\{\orz,\v_0,\u\}$ is not collinear for any $\u\in
V\setminus\{\v_0\}$.  For all $i$, set $\v_i = \rho^i \v_0$ and
$\beta(i) = \op{azim}(\orz,\v_0,\v_1,\v_i)$.  Then
\begin{displaymath}0=\beta(1)\le \beta(2)\le \cdots\le
\beta(k-1)\le\pi.\end{displaymath}
Moreover, if $\beta(i)=0$ for some $1<i \le k-1$, then
\begin{displaymath}
\angle(\v_1) = \cdots = \angle(\v_{i-1}) = \pi,
\end{displaymath}
and $\{\orz,\v_0,\ldots,\v_i\} \subset \op{aff}^0_+(\{\orz,\v_0\},\v_1)$.
Finally, if $\beta(i)=\beta(k-1)$ for some $1\le i<k-1$ then 
\begin{displaymath}
\angle(\v_{i+1}) = \cdots = \angle(\v_{k-1}) = \pi,
\end{displaymath}
and $\{\orz,\v_i,\ldots,\v_{k-1}\} \subset
\op{aff}^0_+(\{\orz,\v_0\},\v_{k-1})$.
\end{lemma}

\begin{proof}  
  With respect to a frame, the points $\v_j$ can be represented in
  spherical coordinates $(r_j,\theta_j,\phi_j)$.  In an appropriate
  frame, $\phi_0=0$, and $\theta_j=\beta(j)$, for all $j$.  From
  $\v_j\in \bWdart(F,\v_0)$ and $\angle(\v_0)\le\pi$, it follows that
  $0\le\theta_j\le\theta_{k-1}\le\pi$ when $0\le j\le k-1$.

One may assume the induction hypothesis that $0\le \beta(1)\le\cdots\le
\beta(i)$.  The condition
\begin{displaymath}
\v_0\in \bWdart(F,\v_i)
\end{displaymath}
implies that
\begin{displaymath}
  0 \le \op{azim}(\orz,\v_i,\v_{i+1},\v_0)
\le \op{azim}(\orz,\v_i,\v_{i+1},\v_{i-1})\le\pi.
\end{displaymath}
By Lemma~\ref{lemma:sim}, the resulting inequality
\begin{displaymath}
\sin(\op{azim}(\orz,\v_i,\v_{i+1},\v_0))\ge 0
\end{displaymath}
reduces to a triple-product:
\begin{displaymath}
(\v_0 \times \v_i)\cdot \v_{i+1}\ge 0.
\end{displaymath}
In spherical coordinates, this inequality becomes
\begin{displaymath}
r_0r_ir_{i+1}\sin\phi_i\sin\phi_{i+1}\sin(\theta_{i+1}-\theta_i)\ge0.
\end{displaymath}
Under the non-collinearity assumption, $\sin\phi_i\sin\phi_{i+1}\ne0$
(when $0< i < k-1$).  These inequalities give
$\theta_i\le\theta_{i+1}$ (with a small extra argument to exclude the
degenerate case $\theta_{i+1}=0,\theta_i=\pi$).  The conclusion
follows by induction.

Assume that $\beta(i)=\theta_i=0$, for some $i>1$.  Then by the first
conclusion, $\theta(j)=0$, for $0\le j\le i$.  That is, 
$\v_1,\ldots,\v_i$ all lie in the half-plane
$\op{aff}^0_+(\{\orz,\v_0\},\v_1)$.  In particular, they are coplanar.
A chaining argument based on Lemma~\ref{lemma:A} gives the result.

The final conclusion is similar.
\end{proof}



\subsection{deformation}\label{sec:deformation}


This subsection develop a theory of deformations of a convex local fan
$(V,E,F)$, including sufficient conditions for the deformation of a
convex local fan to remain a convex local fan.


\begin{definition}[deformation]\guid{YWNHMBP}
\formaldef{deformation}{is\_deformation\_clf}
A \newterm{deformation} 
of a convex local fan $(V,E,F)$ over an interval
$I\subset\ring{R}$ is a function $\varphi:V\times I
\to\ring{R}^3$ that is continuous $\varphi(v,\cdot):I\to\ring{R}^3$
for each $v\in V$.
\end{definition}
\indy{Index}{deformation}%
\indy{Index}{fan!local}%

\begin{notation}
  Beware of the notational distinction between the zenith angle $\phi$
  and the deformation $\varphi$.  When a deformation $\varphi$ is
  given, write $\v(t)$ as an abbreviation of $\varphi(\v,t)$, for
  $t\in I$.  Also, set
\begin{displaymath}
\begin{array}{lll}
V(t)&=\{\v(t) \mid \v\in V\},\\
E(t)&=\{\{\v(t),\w(t)\}\mid \{\v,\w\}\in E\},\\
F(t)&= \{(\v(t),\w(t)) \mid  (\v,\w)\in F\}.
\end{array}
\end{displaymath}
\indy{Notation}{vt@$\v(t)$ (deformation of $\v$)}
\end{notation}

A deformation does not require $(V(t),E(t),F(t))$ to be a convex local fan
for all $t\in I$, although this will often be the case. The
permutation $\rho:V\to V$ gives $\varphi(\rho \v,t)\in V(t)$, for
every $\v\in V$.


\begin{example}[lunar deformation]\label{example:lunar}
Consider a lunar fan $(V,E,F)$ with pole $\{\v,\w\}\subset V$.
Pick a frame and spherical coordinates $(r,\theta,\phi)$ such that $\phi(\v)=0$,
$\phi(\w)=\pi$,  $\theta(\rho \v)=0$, and $\theta(\rho^{-1}
\v)=\theta_{k-1}\le\pi$.  Consider the deformation $\varphi$ over $I
= \{t \mid 0 \le t < 1\}$ such that radial $r$ and zenith $\phi$
coordinates of $\varphi(\u,t)$ are constant as functions of $t$, and
the azimuth angle $\theta$ of $\varphi(\u,t)$ equals

\begin{displaymath}
\begin{cases} 
  (1-t) \theta_{k-1} & \text{if } \u\in 
\{\rho \w,\rho^2 \w,\ldots, \rho^{-1} \v\};\\
  0 & \text{if } \u\in \{\rho \v,\rho^2 \v,\ldots,\rho^{-1} \w\}.\\
\end{cases}
\end{displaymath}
%This deformation $(V(t),E(t),F(t))$ is a convex local fan for all $t\in I$.
%The cardinality of $V$ is independent of $t\in I$.
Note that $(V(0),E(0),F(0)) = (V,E,F)$.
\end{example}
\indy{Index}{lunar}%
\indy{Index}{spherical coordinates}%

\begin{lemma}[]\guid{HZIYFIZ}\rating{ZZ}\label{lemma:lunar-deform} 
Let $(V,E,F)$ be a lunar fan with pole $\{\v,\w\}\subset V$.  In
the deformation described above, the triple $(V(t),E(t),F(t))$ is a
lunar fan for all $t\in I$.  The cardinality of $V$ is
independent of $t\in I$.
\end{lemma}

\begin{proof} The proof consists of checking the defining properties
  of a lunar fan, one by one.  The verification of the property
  \case{cardinality} follows from the methods of
  Remark~\ref{remark:fan-verify}.

The map $\varphi(\cdot,t)$ is invertible, so that $V(t)$ is in
bijection with $V$.  In particular, the cardinality of $V(t)$ does not depend
on $t$.

\case{origin} The radial spherical coordinate is nonzero for every
element of $V(t)$.  Hence $V(t)$ does not contain $\orz$.

\case{non-parallel} For $\{\v,\rho \v\}\in E$, the angle
$\alpha=\op{arc}_V(\orz,\{\varphi(\v,t),\varphi(\rho \v,t)\})$ is
independent of $t$.  The non-parallel property is equivalent to
$\alpha\ne0,\pi$.  Thus, the  property for $E$ implies
the property for $E(t)$.

\case{intersection} The points $V(t)$ are contained in the union of
two half-planes $A_1,A_t$.  The deformation is the identity on $A_1$
and an isometry $A_0\to A_t$ on the second half-plane.  These
bijections preserve the incidence relations of blades of the cones
$C(\ee)$.

\case{dihedral}~\case{face} The combinatorial structure of
the hypermap does not depend on $t$.  In particular, the hypermap has
face $F(t)$, and the hypermap is isomorphic to $\op{Dih}_{2k}$.

\case{angle} The azimuth angle $\angle(\varphi(\u,t))$ is fixed when
$\u\ne \v,\w$ and is decreasing in $t$ when $\u\in \{\v,\w\}$.
Hence, the upper bound on the angle is preserved.

\case{wedge} $V(t)\subset A_0\cup A_t$, where $A_0$ and $A_t$ are
the half-planes described above.  Now $\bWdart(F,\varphi(\u,t))$ is
a half-space containing $A_0$ and $A_t$ when $\u\ne \v,\w$.  Also,
$\bWdart(F,\varphi(\u,t))$ is a wedge with bounding half-planes
$A_0$ and $A_t$ when $\u\in \{\v,\w\}$.  Hence $V(t)\subset A_0\cup
A_t\subset \bWdart(F,\varphi(\u,t))$ for all $\u\in V$.

\case{lunar} The points $\v,\w\in V(t)$ remain fixed and hence
remain parallel.  Thus, the convex local fan remains lunar.
\end{proof}
\indy{Index}{angle!azimuth}%

Next consider a deformation of a convex local fan.  The following lemma
gives a list of conditions that ensure that the deformed fan remains
convex local throughout the deformation.


\begin{lemma}[]\guid{XRECQNS}\rating{ZZ}\label{lemma:fan-open}
Let $\varphi$ be a deformation of a convex local fan $(V,E,F)$ over an
interval $I$.  Assume that $0\in I$ and that $\varphi(\v,0)=\v$ for
all $\v\in V$.  For all $\v\in V$ assume that if
$\v$ is flat, then $\v(t)$ is flat for all sufficiently small $t$.
%$\{\orz,\rho^{-1}\v,\v,\rho\v\}$ is coplanar, then
%\begin{displaymath}
%\{\orz,\varphi(\rho^{-1}\v,t),\varphi(\v,t),\varphi(\rho\v,t)\}
%\end{displaymath} is coplanar for all sufficiently small $t$.
%If the property \case{wedge} is maintained for sufficiently small $t$,
Then exists $\epsilon>0$ such that $(V(t),E(t),F(t))$ is a convex local fan
for all $t\in I\cap \leftopen-\epsilon,\epsilon\rightopen$.  
%Moreover,
%if $(V,E,F)$ is generic, then the property \case{wedge} is in fact
%maintained for sufficiently small $t$.  
Moreover,  if $(V,E,F)$ is
generic, then the deformed fan is also generic for sufficiently small
$t$.
\end{lemma}

\begin{proof} Each of the defining properties of a generic fan will be
examined in turn.

\case{cardinality}: The set $V(t)$ is the image of $V$ and is
therefore finite and nonempty.

\case{origin} Since $\varphi$ is continuous and
$\orz\not\in V$, it follows that $\orz\not\in V(t)$ for sufficiently
small $t$.

\case{non-parallel}: If $\v,\w$ are non-parallel, then $\v(t)$ and
$\w(t)$ are non-parallel for sufficiently small $t$.

\case{intersection}: If $\ee \cap \ee'=\emptyset$, then $C(\ee)\cap
S^2$ has a positive distance from $C(\ee')\cap S^2$.  Hence for
sufficiently small times, the deformation of these sets remain
disjoint.  If $\ee=\{\u,\v\}$ and $\ee'=\{\v,\w\}$ where $\u\ne\w$,
then again the deformations of $C(\ee)\cap C(\ee')$ is
$C(\{\v(t)\})$ for sufficiently small $t$.  The other cases are
similar.

\case{face},~\case{dihedral}: The azimuth cycle on $E(\v(t))$
is preserved; hence the combinatorial properties of the hypermap do
not change when $t$ is sufficiently small.

\case{angle}: If $\op{azim}(x)<\pi$, then the inequality remains
strict for sufficiently small $t$.  If $\op{azim}(x)=\pi$, then the
flatness assumption of the lemma forces the equality to be
preserved for all sufficiently small $t$.

\case{wedge}: The property $\u\in \Wdart(x)$ is an open condition.
It holds for sufficiently small $t$. Consider the case $\u\in
\bWdart(x)\setminus \Wdart(x)$, where $x= (\v_0,\rho\v_0)$.  Write
$\v_i = \rho^i\v_0$ and pick $r\le k-1$ such that $\u = \v_r$.  The
wedge property holds trivially when $r\in\{0,1,k-1\}$. Assume that
$r\not\in\{0,1,k-1\}$.  We finish the argument in cases, according to 
the type of the convex local fan  $(V,E,F)$.

If the fan $(V,E,F)$ is circular, then there is a plane $A$ through
the origin that contains $V$.  By the flatness condition, the
deformation maintains a coplanarity condition $V(t)\subset A(t)$.  The
sets $\bWdart(x(t))$ are half-spaces bounded by $A(t)$.  Thus
$V(t)\subset A(t)\subset \bWdart(x(t))$, as desired.

If the fan $(V,E,F)$ is lunar with pole $\{v,w\}$, the argument is
similar, but there are two half-planes $A_1(t)$ and $A_2(t)$ such that
$V(t)\subset A_1(t)\cup A_2(t)$.  Arguing as in the previous case,
each wedge $\bWdart(x(t))$ contains $A_1(t)\cup A_2(t))$.  The result
follows.

If the fan $(V,E,F)$ is generic, then
by the genericity assumption $\v_r$ and
$\v_0$ are non-parallel.  In the notation of
Lemma~\ref{lemma:monotone}, $\beta(r) = 0$ or $\beta(r) =
\beta(k-1)$.  This proof treats the case $\beta(r)=0$. (The case
$\beta(r)=\beta(k-1)$ is similar.)  By Lemma~\ref{lemma:monotone},
\begin{displaymath}0=\beta(1)=\beta(2)=\cdots=\beta(r),\end{displaymath}
and 
\begin{displaymath}
\{\orz,\v_0,\v_1,\ldots,\v_r\} \subset \op{aff}^0_+(\{\orz,\v_0\},\v_1).
\end{displaymath}
In particular, the set is coplanar.  By the flatness assumption of
the lemma, $\{\orz,\v_0(t),\v_1(t),\ldots,\v_r(t)\}$ is coplanar for
sufficiently small $t$.  In fact, the condition $\v_r(t)\not\in
\op{aff}_+(\{\orz,\v_0(t)\})$ is an open condition, so that
$\v_r(t)\in \op{aff}^0_+(\{\orz,\v_0(t)\},\v_1(t))$ for sufficiently
small $t$.  This half-plane is the bounding half-plane of
$\bWdart(F,\v_0(t))$.  Hence $\u = \v_r\in \bWdart(F,\v_0(t))$ for
sufficiently small $t$.

\case{generic}: Genericity is stated as open conditions $\v\not\in
C\{\u,\w\}$.  These conditions continue to hold for sufficiently small
$t$.
\end{proof}


%%%%%%%%%%%




\section{Perimeter}

A convex local fan has a perimeter.  One of the results of this
chapter is a proof of the upper bound $2\pi$ on the perimeter of a
convex local fan (Lemma~\ref{lemma:convex-hyp}).  A great circle has
perimeter $2\pi$.  The strategy of the proof is deform the convex
local fan in a way that flattens out the interior angles and increases
the perimeter.  Eventually a circular convex local fan or lunar convex
local fan is reached.  Both of these have perimeter $2\pi$.

This perimeter majorization result, stated for convex spherical
polygons, is classical.  For example, a proof appears
in~\cite{unknown}.


%\subsection{perimeter}

\begin{definition}[perimeter]\guid{IQCPCGW}\label{lemma:perim}
Let $(V,E,F)$ be a convex local fan.    Set
\begin{displaymath}
  \op{per}(V,E,F) 
= \sum_{i=0}^{k-1} \arc_V(\orz,\{\rho^i \v,\rho^{i+1} \v\}), 
\end{displaymath}
where $k=\card(F)$.  The right-hand side of this formula is easily
seen to be independent of the choice of $\v\in V$.  Call $\op{per}$
the \newterm{perimeter} of the convex local fan.  If $\v,\w\in V$ are
distinct nodes, define the \newterm{partial perimeter}
\begin{displaymath}
  \op{per}(V,E,F,\v,\w) 
= \sum_{i=0}^{r-1} \arc_V(\orz,\{\rho^i \v,\rho^{i+1} \v\}), 
\end{displaymath}
where $r$ is chosen so that $\w=\rho^r \v$ and $0<r\le k-1$.
\end{definition}
\indy{Index}{perimeter!convex local fan}%
\indy{Index}{fan!local}%
\indy{Notation}{per@$\op{per}$ (perimeter)}%



\begin{lemma}[perimeter majorization]\guid{WSEWPCH}\rating{400}
\label{lemma:convex-hyp}
The perimeter of every convex local fan is at most $2\pi$.  
\end{lemma}
\indy{Index}{fan!local}%
\indy{Index}{perimeter}%

\begin{proof} 
\claim{If the convex local fan is circular, then its perimeter is
$\op{per}(V,E,F) =2\pi$.}  Indeed, by Lemma~\ref{lemma:circular},
the arcs making up the perimeter all lie in a common plane.  The
azimuth cycle on $V$ coincides with $\rho:V\to V$.  The sum of the
terms in the formula defining the perimeter is the sum of the
azimuth angles in the azimuth cycle.  The sum is $2\pi$ by
Lemma~\ref{lemma:2pi-sum}.


\claim{if the convex local fan is lunar, then its perimeter is
$\op{per}(V,E,F) =2\pi$.}  Indeed, by Lemma~\ref{lemma:lunar}, the
set $V$ is contained in the union of two half-planes.  The perimeter
is the sum of arcs in a half-circle in the first half-plane plus the
sum of arcs in a half-circle in the second half-plane. This sum is
$2\pi$.

Finally, assume that the convex local fan is generic.  Suppose for a
contradiction that the lemma is false.  Consider all counterexamples
that minimize the cardinality of $V$.  
%Among all such
%counterexamples, pick a counterexample with the smallest number of
%darts $x\in F$ such that $\op{azim}(x) = \pi$.

A convex local fan $(V,E,F)$ is determined by $V$ and the cyclic
permutation $\rho:V\to V$: $E=\{\{\v,\rho \v\}\mid \v\in V\}$ and $F
= \{(\v,\rho \v)\mid \v\in V\}$.

In such a counterexample, if there is any flat dart $x=(\v,\w)\in F$,
then there is a new convex local fan $(V',E',F')$ with $V' =
V\setminus\{\v\}$ and $\rho':V'\to V'$ given by
\begin{displaymath}
\rho'(\u) = \begin{cases}
\rho(\u), & \text{if } \rho(\u)\ne \v;\\
\rho(\v), & \text{if }\rho(\u) = \v.\\
\end{cases}
\end{displaymath}
This is a convex local fan with the same perimeter, contrary to the presumed
minimality of the counterexample.  Thus, in the minimal counterexample
$\op{azim}(x) <\pi$, for all $x\in F$.

If $\card(V) <3$, then the convex local fan is circular or lunar.  The
circular and lunar cases have 
already been treated.  If $\card(V)=3$, then $V=\{\v_1,\v_2,\v_3\}$.
By the triangle inequality, $\arc_V(\orz,\{\v_2,\v_3\}) \le
\arc_V(\orz,\{\v_2,-\v_1\})+\arc_V(\orz,\{-\v_1,\v_3\})$.  Thus,
\begin{displaymath}
\begin{array}{rll}
  \op{per} &=\arc_V(\orz,\{\v_1,\v_2\}) 
  + \arc_V(\orz,\{\v_2,\v_3\}) 
  + \arc_V(\orz,\{\v_1,\v_3\})\\
  &\le(\arc_V(\orz,\{\v_1,\v_2\})+\arc(\orz,\{\v_2,-\v_1\}))
  \\&\qquad\qquad+(\arc_V(\orz,\{\v_1,\v_3\})
+\arc_V(\orz,\{-\v_1,\v_3\})) \\
  &= \pi+\pi.
\end{array}
\end{displaymath}

Now assume that $\card(V)\ge 4$.  Select $\v\in V$.  Consider a
deformation of the convex local fan $\varphi:V\times I \to \ring{R}^3$ that
fixes $V\setminus\{\v\}$ and moves $\v$:
\begin{displaymath}
\varphi(\v,t) = \cos(t) \v - \sin(t) \rho \v,\quad
 t \in I=\leftclosed0,\frac{\pi}{2}\rightclosed.
\end{displaymath}
By the spherical triangle inequality, this is increasing in the
perimeter.  For sufficiently small $t$, it remains a generic convex local fan
(Lemma~\ref{lemma:fan-open}).  For sufficiently small $t$, the
minimality gives $\angle(\varphi(\u,t))<\pi$.  Eventually, for some
smallest $t=t_0$ the deformed value is no longer a generic convex local fan.
% or for some $\u\in V$, $\angle(\varphi(\u,t_0))=\pi$.  In the latter
% case, the minimality condition fails, and the result follows.

Note that at $t=\pi/2$, $\rho\v,-\rho\v\in V(t)$, which violates the
genericity condition.  This gives an upper bound on the time $t_0$ of
``first failure.''  Let $L$ be the line of intersection of two planes
throug the origin:
\begin{displaymath}
L= \op{aff}\{\orz,\v,\rho\v\}
\cap\op{aff}\{\orz,\rho^{-1}\v,\rho^{-2}\v\}.
\end{displaymath}
(These two planes are not equal by the non-flatness at $\v$ and
Lemma~\ref{lemma:A}.)

\claim{The line $L$ meets the segment
  $\op{aff}^0_+(\emptyset,\{\v,-\rho\v\})$.}  Otherwise, it meets
$\op{conv}\{\v,\rho\v\}$.  Then $\v$ and $\rho\v$ lie in distinct
half-spaces bounded by $\op{aff}\{\orz,\rho^{-1}\v,\rho^{-2}\v\}$.  By
the non-flatness conditions, they even lie in different open
half-spaces.  Since $\bWdart(F,\rho^{-1}\v)$ is contained in one of
the two half-spaces, the condition $\v,\rho\v\in
\bWdart(F,\rho^{-1}\v)$ fails, and $(V,E,F)$ is not a convex local
fan.  This contradiction establishes the claim.

\claim{The time of $t_0$ of first failure satisfies $t_0 \le t_1
  <\pi/2$, where $t_1$ is the first time at which $\varphi(\v,t)$
  meets $L$.}  Indeed, by the previous claim, $t_1 <\pi/2$.  Also, at
time $t_1$, the set $\{\orz,\varphi(\v,t_1),\rho^{-1}\v,\rho^{-2}\v\}$
is coplanar, so that the dart of $F$ at $\rho^{-1}$ is flat, which is
contrary to the established flatness property.  This gives the claim.

\claim{The \case{non-parallel} property holds at $t=t_0$}.  Indeed, it
is enough to check non-parallelism when one of the points is
$\varphi(\v,t_0)$ and the other is $\rho\v$ or $\rho^{-1}\v$.  Begin
with $\rho^{-1}\v$.  If the set $\{\orz,\rho^{-1}\v,\varphi(\v,t)\}$
is collinear, then that line is contained in $L$, which is impossible
since $\rho^{-1}\v\not\in \op{aff}\{\orz,\v,\rho\v\}$.  Thus,
$\rho^{-1}\v$ and $\varphi(\v,t)$ are not parallel.  Also $\pi/2 >
t_0$ implies that $\varphi(\v,t_0)$ and $\rho\v$ are not
parallel. This gives non-parallelism.

\claim{The \case{intersection} property holds.}  If $\ee\subset V$,
write $\ee(t) = \{\varphi(\u,t) \mid \u\in \ee\}$.  Let $W
=W(\orz,\v,\rho\v,\rho^{-1}\v)$. The verification of the intersection
property is based on the following facts (when $t>0$):
\begin{itemize} 
\item If $\v\not\in \ee$, then $C(\ee(t))=C(\ee)\subset W$.
\item $C^0\{\v(t)\} \cap W = \emptyset$.
\item $C^0\{\v(t),\rho^{-1}\v\}\cap W = \emptyset$.
\item $C\{\v(t),\rho\v\}\cap W = C\{\v,\rho\v\}.$
\item $C\{\v(t),\rho\v\}\cap C\{\v(t),\rho^{-1}\v\} = C\{\v(t)\}$.
\end{itemize}

\claim{At time $t=t_0$, the deformed value is a convex local fan.}
Otherwise, if the object fails to be a convex local fan at time
$t=t_0$, then the \case{non-parallel} property, \case{intersection}
property, or the \case{dihedral} property fails.  The first two
properties have already been checked, so that the deformed value must
be a fan.  Furthermore, the conditions for a fan to be convex local
are closed conditions, so they must also hold.

In summary, at $t=t_0$, the deformed fan is convex local but not
generic.  The perimeter bound follows from circular and lunar cases.
\end{proof}

Here is a second proof of the same lemma.  It is conceptually much
simpler, but possibly more difficult to formalize.  The proof is based
on polar polygons (a generalization of polar triangles to spherical
polygons).

\begin{proof} A fan does not have any faces of cardinality less than
three.  Every blade of the fan has radian measure less than $\pi$.
\indy{Index}{polygon!polar}%

Consider the case of a spherical triangle with edges $a_i$
and polar triangle with angles $\beta_i$. Then $\beta_i=\pi-a_i$.
The perimeter is 
\begin{displaymath}a_1+a_2+a_3 = 2\pi - (\beta_1 -\beta_2 -
\beta_3-\pi)= 2\pi-\op{sol} < 2\pi,\end{displaymath} because the
solid angle $\op{sol}$ of the polar triangle is always strictly
positive.  \indy{Index}{triangle!spherical}%

Similarly, if the edges of the spherical polygon are
$a_i$, then the angles of the polar polygon are $\beta_i = \pi-a_i$.
The perimeter is
\begin{displaymath}
a_1+\cdots+a_n  = 2\pi- \op{sol}< 2\pi,
\end{displaymath}
where $\op{sol} = 2\pi-\sum a_i$ is the solid angle of the polar polygon.
%~\cite[\p.261]{williamson:2008}.
\indy{Notation}{sol $\op{sol}$ (solid angle)}%
\end{proof}


\section{Special Fan}\label{sec:weight}  

Our aim become single-minded throughout the rest of the chapter; we
wish to give a proof of the main estimate (Lemma~\ref{lemma:empty-d}).
This is the longest proof in the book, it requires substantial
preparation.  We will assume the existence of a counterexample to the
main estimate.  Assuming the existence of some counterexample, a
compactness argument gives the existence of a minimal counterexample.
The properties of minimal counterexamples are developed in a long
series of lemmas.  Eventually, enough properties of a minimal
counterexample are established to conclude that it cannot exist.




\subsection{definition}

This subsection gives the definitions that are needed to state the
main estimate.

\begin{definition}[special~fan,~$\hm$]\guid{FIJJSLP}
Let $\hm = 1.26$.
A \newterm{special fan} is a tuple $(V,E,F,S)$, where
\begin{description}
\item \case{packing} $V$ is a packing.  That is, for every $\v,\w\in
V$, if $\norm{\v}{\w}<2$, then $\v=\w$.
\item \case{annulus} $V\subset \BB$.
\item \case{local~fan} $(V,E,F)$ is a convex local fan (Definition~\ref{def:convex-local}).
\item \case{subset} $S\subset E$.
\item \case{s~norm} If $\{\v,\w\}\in S$, then $\norm{\v}{\w}=2\hm$.
\item \case{e~norm} If $\{\v,\w\}\in E$, then $\norm{\v}{\w}\le 2\hm$.
\item \case{diagonal} For all distinct elements $\v,\w\in V$, if
$\{\v,\w\}\not\in E$, then \begin{displaymath}\norm{\v}{\w}\ge
2\hm.\end{displaymath}
\item \case{card} %$k=\card(F)$,
Let      $s=\card(S)$ and $r=\card(E) - s = \card(F)-s$.  Then
\begin{displaymath}0\le s \le 3,\quad\text{and}\quad3-s \le r \le 6 -
2s.\end{displaymath}
\end{description}
The constants $r$ and $s$ are called the \newterm{parameters} of the special fan.
\end{definition}
\indy{Index}{special fan}
\indy{Index}{parameters (of a special fan)}
\indy{Notation}{r (special fan parameter)}
\indy{Notation}{s (special fan parameter)}


\begin{definition}[d]\guid{EFWASJQ}
\begin{displaymath}d(r,s) = \begin{cases}
0.103 (2-s) + 0.2759 (r+2s-4), & r + 2s > 3\\
0, & r + 2s \le 3.\\
\end{cases}\end{displaymath}
\end{definition}
%d(3,0)=0, d(4,0)= 0.206; d(5,0)= 0.4819,   .7578
\indy{Notation}{d (lower bound for $\tau$)}

\begin{definition}[$\hm$,~$\tau$,~$\dih_i$]\guid{CUFCNHB}\label{def:tau}
Let $(V,E,F)$ be a convex local fan.  Set $\hm = 1.26$.  Set
\begin{displaymath}
  \tau(V,E,F) =\sum_{x\in F} \op{azim}(x)\left(1 + \dfrac{\sol_0}{\pi}  
    \dfrac{\normo{\nd(x)}-2}{2\hm-2}\right) 
+ \left(\pi+{\sol_0}\right) (2- k(F)),
\end{displaymath}
where $\sol_0=3\arccos(1/3)-\pi\approx0.551$ is the solid angle of a
spherical equilateral triangle of side $\pi/3$, and $k(F)$ is the
cardinality of $F$.  The function $\tau$ is extended to special fans by
disregarding $S$:
\begin{displaymath}
\tau(V,E,F,S) = \tau(V,E,F).
\end{displaymath}
Let 
\begin{displaymath}
  \tau_{tri}(y_1,y_2,y_3,y_4,y_5,y_6) =
  \sum_{i=1}^3 \dih_i(y_1,\ldots,y_6)
\left(1 + \dfrac{\sol_0}{\pi}  \dfrac{y_i -2}{2\hm-2}\right) 
- \left(\pi+{\sol_0}\right),
\end{displaymath}
where
\begin{displaymath}
\begin{array}{lll}
\dih_1(y_1,y_2,y_3,y_4,y_5,y_6) &= \dih(y_1,y_2,y_3,y_4,y_5,y_6),\\
\dih_2(y_1,y_2,y_3,y_4,y_5,y_6) &= \dih(y_2,y_3,y_1,y_5,y_6,y_4),\\
\dih_3(y_1,y_2,y_3,y_4,y_5,y_6) &= \dih(y_3,y_1,y_2,y_6,y_4,y_5).\\
\end{array}
\end{displaymath}
\indy{Notation}{h0@$\hm$}
\indy{Notation}{zzt@$\tau$}
\indy{Notation}{sol@$\sol_0$}
\indy{Notation}{dih@$\dih_i$}
\end{definition}



\subsection{compactness}

This subsection continues with preparations for the main estimate.
The proof of the main estimate is a proof by contradiction. Assuming a
counterexample exists, compactness results are used to prove the
existence of a minimal counterexample.  This subsection prepares for
the proof of the main estimate by proving a compactness result.

Let 
\begin{displaymath}
\begin{array}{lll}
V(\p) &= \{\p_{i}\mid i=0,\ldots,k-1\},\\
E(\p) &= \{\{\p_{i},\p_{i+1}\}\mid i=0,\ldots,k-1\},\\
F(\p) &= \{(\p_{i},\p_{i+1}) \mid i=0,\ldots,k-1\},\\
\end{array}
\end{displaymath}
where $\p:\{0,\ldots,k-1\}\to \ring{R}^3$ is any function, and $\p_k =
\p_0$.  (That is, for purposes of indexing, identify
$\{0,\ldots,k-1\}$ with the cyclic group $Z_k$.)  If
$I\subset\{0,\ldots,k-1\}$ then set
\begin{displaymath}S(\p,I) = \{(\p_i ,\p_{i+1}) \mid i\in
I\}\end{displaymath}

\begin{definition}[fan~datum]\guid{PJWMYDB}
 Let $k\ge3$ and $I\subset
\{0,\ldots,k-1\}$.  A fan datum of shape $(k,I)$ is a function
$\p:\{0,\ldots,k-1\}\to \BB$ such that
\begin{itemize}
\item \case{packing} For every $i,j$, if $\norm{\p_i}{\p_j}<2$, then
$i=j$.
\item \case{i~norm} If $i\in I$, then $\norm{\p_i}{\p_{i+1}}=2\hm$.
\item \case{e~norm} For all $i$, $\norm{\p_i}{\p_{i+1}} \le 2\hm$.
\item \case{diagonal}  $\norm{\p_i}{\p_j} \ge 2\hm$ when $|i-j|>1$.
\item \case{card}  Let $s=\card(I)$ and $r=k-s$.  Then 
\begin{displaymath}0\le s \le 3,\quad\text{and}\quad3-s \le r \le 6
- 2s.\end{displaymath}
\item \case{angle} $\op{azim}(\orz,\p_i,\p_{i+1},\p_{i-1})\le \pi$ for
all $i$.
\item \case{wedge} $\p_j\in W(\orz,\p_i,\p_{i+1},\p_{i-1})$ for all $i,j$.
\end{itemize}
\end{definition}


\begin{lemma}[standard fan]\guid{CKQOWSA}\rating{1000}\label{lemma:std-fan} 
%older harder proof w/o computers was 1800.
% rating includes the Tarski calculations.
%\formalauthor{Roland Zumkeller: \case{four points} by Bernstein polynomials}
Let $V\subset \BB$ be a packing.  Set 
\begin{displaymath}E_{std} = \{\{\v,\w\}\subset V\mid 0 <
\norm{\v}{\w} \le 2\hm\}.\end{displaymath} Then $(V,E_{std})$ is a fan.
\end{lemma}
\indy{Notation}{E@$E_{std}$}%

\begin{proof}
The properties \case{cardinality}, \case{origin}, and \case{non-parallel} follow
by the methods of Remark~\ref{remark:fan-verify}.

\case{intersection}: Some geometrical reasoning is required to
establish the intersection property.  The case
\begin{displaymath}
C\{\u\}\cap C\{\v\} = \{\orz\}
\end{displaymath}
follows from the strict triangle inequality 
\begin{displaymath}
\normo{\u} \le 2\hm < 4 \le \normo{\v} + \norm{\u}{\v}.
\end{displaymath}
The other cases of the proof are based on the following two facts from
Tarski arithmetic.  The second fact is a consequence of the theory of
Cayley-Menger determinants, as described in Remark~\ref{rem:CM5}.
%\footformal{Further details about these calculations
%  can be found in
%  \url{http://www.math.pitt.edu/~thales/papers/Lemmas_Elementary_Geometry.pdf}
%  Lemma 58 (UQQVJON) and Lemma 60 (ZHBBLXP).  An alternative approach
%  to the proof of \case{four points} proceeds as follows. Assume a
%  counterexample to \case{four points} exists. By symmetry we may
%  assume without loss of generality that a counterexample satisfies the further
%incidence relation
%$\op{conv}\{\v_1,\v_3\}$   meets  $\op{conv}\{\v_0,\v_4,\v_2\}$, where
%we set $\v_0=\orz$.
%  The set of all counterexamples $\{\v_1,\ldots,\v_4\}$ satisfying this further
%incidence relation
%is a nonempty compact set $X_0$ of $\ring{R}^{12}$, because all of the defining conditions are closed conditions and $\BB$ is bounded.  Among all countexamples, consider the
%nonempty subset $X_1$ of those configurations in $X_0$ that maximize the continuously differentiable function $\normo{\v_4}+\normo{\v_2}+\norm{\v_4}{\v_2}$.  For configurations in $X_1$,
%the maximum of this continuously differentiable function is the {\it boundary value} $\norm{\v_i}{\v_j}=2.52$, with $i\ne j\in\{0,2,4\}$ because otherwise we are not
%at a critical point of the function.  Let $X_2$ be the compact subset of $X_1$
%that minimizes the continuously differentiable function $\norm{\v_1}{\v_3}$.  
%Similar arguments show that there is no interior point critical point, and the
%minimum appears at the boundary when $\norm{\v_i}{\v_j}=2$ for $i=1,3$ and $j=0,2,4$.
%The function $\mu_V$ (from
%the {\it Lemmas in Elementary Geometry} collection Lemma PFDFWWV) computes the
%value of $\norm{\v_1}{\v_3}$, which is larger than $2.52$.  This contradicts
%what is assumed.
%}
\begin{description}
\item \case{three points} Let $\{\v_1,\v_2,\v_3\}\subset \BB$ be a
  packing of three points.  Assume that $\norm{\v_1}{\v_2}\le 2\hm$.
  Then $\v_3\not\in\cap C\{\v_1,\v_2\}$.
\item \case{four points} Let $\{\v_1,\v_2,\v_3,\v_4\}\subset \BB$ be a
  packing of four points.  Assume that $\norm{\v_1}{\v_3}\le 2\hm$ and
  $\norm{\v_4}{\v_2}\le 2\hm$.  Then $C\{\v_1,\v_3\}\cap
  C\{\v_4,\v_2\} = \{\orz\}$.
\end{description}
We give a proof of \case{three points} by contradiction, assuming that
the configuration exists with $\v_3\in\cap C\{\v_1,\v_2\}$, and in the
more general setting where the constraint $2\le \norm{\v_1}{\v_2}$ is
dropped.  The point $\v_2$ can be moved in a circular arc centered at
$\orz$ until $\norm{\v_2}{\v_3}=2$.  (By the triangle inequality, this
constraint is met before $\{\orz,\v_2,\v_3\}$ become collinear.)
Similarly, we may assume that $\norm{\v_1}{\v_3}=2$.  The proof then
follows from a computer calculation that $\Delta(x_{ij})\ne0$, for
$x_{ij}=\norm{\v_i}{\v_j}^2$, which contradicts the established fact
that the polynomial $\Delta$ vanishes on planar arrangements of four
points $\{\orz,\v_1,\v_2,\v_3\}$.
%%XX computer calculation needs to be entered.

The proof of \case{four points} is also a proof by contradiction,
assuming as we may by symmetry that $\op{conv}\{\v_1,\v_3\}$ meets
$\op{conv}\{\v_0,\v_4,\v_2\}$, where we set $\v_0=\orz$.  We may relax
the problem by dropping the lower bounds on $\norm{\v_1}{\v_3}$.  Move
$\v_1$ along the circle with perpendicular central axis
$\op{aff}\{\v_0,\v_2\}$. By moving along the circle in the direction
of $\v_3$, the distance $\norm{\v_1}{\v_2}$ also decreases.  Sometime
before the points $\{\v_0,\v_1,\v_2,\v_3\}$ become planar, the
distance $\norm{\v_1}{\v_2}$ decreases to $2$.  The constraints of the
relaxed problem are all still satisfied for this new configuration.
By a symmetrical argument, carried out by relabeling subscripts, we
may also assume that
\[
\norm{\v_1}{\v_2}= \norm{\v_1}{\v_4}=\norm{\v_3}{\v_2}=\norm{\v_3}{\v_4}=2.
\]
A computer calculation now shows that under these additional
constraints the Cayley-Menger determinant of the five points
$\{\v_0,\ldots,\v_4\}$ is nonzero.  This contradicts the established
fact that the Cayley-Menger determinant of any five points in
$\ring{R}^3$ is zero Remark~\ref{rem:CM5}.

\end{proof}

\begin{lemma}[]\guid{VYNCGCO}\rating{ZZ}
If $\p$ is a fan datum of shape $(k,I)$, then
\begin{displaymath}
(V(\p),E(\p),F(\p),S(\p,I))
\end{displaymath}
is a special fan.  Moreover, every special fan is equal to
\begin{displaymath}
(V(\p),E(\p),F(\p),S(\p,I))
\end{displaymath}
for some fan datum $\p$ of some shape $(k,I)$.
\end{lemma}

\begin{proof}
\claim{Every special fan $(V,E,F,S)$ is equal to
$(V(\p),E(\p),F(\p),S(\p,I))$ for some fan datum $\p$ of some
shape $(k,I)$.}  Let $k=\card(F)$.  Fix any element $\u\in V$ and
set $\p_i = \rho^i \u$.  Let $I = \{i\mid (\p_i,\p_{i+1}) \in S\}$.
Clearly $(V,E,F,S)=(V(\p),E(\p),F(\p),S(\p,I))$, and $\p$ is a fan
datum.

Let $\p$ be any fan datum of shape $(k,I)$.
$(V(\p),E(\p))$
is a fan by Lemmas~\ref{lemma:std-fan} and \ref{lemma:subset-fan}.


\claim{$(V(\p),E(\p),F(\p))$ is a convex local fan.} The properties
\case{wedge} and \case{angle} have been built into the definition of a
fan datum.  The azmuth cycle $\sigma(\p_i)$ on $E(\p_i) =
\{\p_{i+1},\p_{i-1}\}$ interchanges $\p_{i+1}$ with $\p_{i-1}$.  From
this fact, the combinatorial properties \case{face} and
\case{dihedral} are easily determined with the help of
Lemma~\ref{lemma:dih-iso}.
%The bijection from the disjoint
%union of two copies of $Z_k$ onto the set of darts is given by the
%pair of maps
%\begin{displaymath}
%i\mapsto (\p_i,\p_{i+1}),\quad i\mapsto (\p_{i+1},\p_i).
%\end{displaymath}
%This bijection extends to an isomorphism of hypermap $\op{Dih}_{2k}$ onto
%$\op{hyp}(V(\p),E(\p))$.

\claim{$(V(\p),E(\p),F(\p),S(\p,I))$ is a special fan.} The properties
of a special fan all follow trivially from the corresponding
properties of a fan datum.
\end{proof}

The set of fan data of shape $(k,I)$ is a subspace of the compact metric
space $\BB^k$.


\begin{lemma}[compactness of fans]\guid{KFIIPLO}\rating{ZZ}\label{lemma:compact-fan}
The space of fan data of shape $(k,I)$ is a compact metric space.
Moreover,
\begin{displaymath}
\p \mapsto \tau(V(\p),E(\p),F(\p))
\end{displaymath}
is a continuous function on the space of fan data of shape $(k,I)$.
\end{lemma}

\begin{proof} $\BB^k$ is a compact metric space and the constraints
are all closed conditions.

The function $\tau$ is a polynomial in $\normo{\p_i}$ and
$\op{azim}(\orz,\p_i,\p_{i+1},\p_{i-1})$.  The norm and azimuth
angle are both continuous functions of $\p$.
\end{proof}




\subsection{internal blades}

This subsection continues with preparations for the main estimate.
The proof of the main estimate is a proof by contradiction.  We will
assign a natural number invariant to each special fan.  We will
examine a minimal counterexample in the sense of having the smallest
possible invariant.  To use the minimality in the proof of the main
estimate, we need to relate the minimal counterexample to a special
fan that has an even smaller numerical invariant. This is accomplished
by changing the fan of the minimal counterexample by adding one more
blade.  This subsection describes how to add a blade to a fan.



\begin{lemma}[]\guid{PGSQVBL}\rating{ZZ} Let $(V,E,F)$ be a convex local fan.
If $\v,\w\in V$ are non-parallel, then $C\{\v,\w\} \subset
\bWdart(x)$ for any dart $x\in F$.
\end{lemma}
\indy{Index}{fan!local}%

\begin{proof} This is an elementary consequence of the definitions,
the cone shape of $\bWdart(x)$, and the condition that $V\subset
\bWdart(x)$.
\end{proof}


\begin{lemma}[internal
blades]\guid{YOLCBTG}\rating{700} \label{lemma:internal}
Let $(V,E,F)$ be a convex local fan.  Let $\v,\w\in V$ be non-parallel.
Suppose that there exists $\v',\w'$ such that
$\angle(\v'),\angle(\w')<\pi$, where $\v,\v',\w,\w'$ are four
distinct elements of $V$ that appear in cyclic order.  Then
$C^0\{\v,\w\}\subset \Wdart(x)$ for all $x\in F$.
% Pick a dart $x=(\v_0,\v_1)\in V$.  Set $\v_j = \rho^j \v_0$.
% Assume that there are four darts $(y_1,y_2,y_3,y_4)$, $y_j =
% x_{j(j)}$, with $0\le j(1) < j(2) < j(3) < i(4)\le k-1$ such that
% $\op{azim}(y_j) < \pi$, for $j=2,4$.  Then
% $C^0\{\v_{i(1)},\v_{i(3)}\} \subset \Wdart(x)$, for all $x\in F$.
\end{lemma}
\indy{Index}{blade!internal}%
\indy{Index}{internal blade}%
\indy{Index}{cyclic order}%

To say that a sequence $\v_i$ of elements is in \newterm{cyclic
order} means that $\v_i = \rho^{j (i)}\v_0$, for some increasing
function $j$ with range $\{0,\ldots,k-1\}$.

\begin{proof} Abbreviate $C^0 = C^0\{\v,\w\}$.  The first case to
consider is $\nd(x)=\v$.  For all $\p\in C^0\cap \bWdart(x)$,
\begin{displaymath}
  0 \le \op{azim}(\orz,\v,\rho \v,\p) 
\le \op{azim}(\orz,\v,\rho \v,\rho^{-1} \v).
\end{displaymath}  
These inequalities are in fact strict.  If, for example $0 =
\op{azim}(\orz,\v,\rho \v,\p)$, then the set $\{\orz,\v,\rho \v,\w\}$
is coplanar.  Repeated application of Lemma~\ref{lemma:A} gives
\begin{displaymath}
\angle(\v) = \angle(\rho \v) = \cdots = \angle(\rho^{-1} \w) = \pi,
\end{displaymath}
which is contrary to $\angle(\v') = \pi$.  The strict inequalities
imply $\p\in \Wdart(x)$, as desired.  The case $\nd(x)=\w$ is similar.

Now assume that $\u=\nd(x)\ne \v,\w$.  By Lemma~\ref{lemma:A}, one may
assume that $\{\orz,\u,\v,\w\}$ is not coplanar.  (Otherwise, the
contradiction $\angle(\v')=\pi$ or $\angle(\w')=\pi$ is reached.) Then
\begin{displaymath}
  \op{aff}\{\orz,\u,\v\}\cap C^0 \subset \op{aff}\{\orz,\u,\v\}
  \cap \op{aff}\{\orz,\v,\w\} \cap C^0 
= \op{aff}\{\orz,\v\} \cap C^0 = \emptyset.
\end{displaymath}
Thus, $C^0$ is disjoint from $\op{aff}\{\orz,\u,\v\}$ and is also
disjoint from $\op{aff}\{\orz,\u,\w\}$.

We have the following facts:
\begin{displaymath}
\v,\w\in W(\orz,\u,\v,\w),\quad C^0\{\v,\w\} \subset W(\orz,\u,\v,\w).
\end{displaymath}
Also,
\begin{displaymath}
\begin{array}{rll}
  C^0 &= C^0\cap W(\orz,\u,\v,\w) \\
  &\subset C^0 \cap (W^0(\orz,\u,\v,\w) 
\cup \op{aff}\{\orz,\u,\v\} \cup \op{aff}\{\orz,\u,\w\})\\
  &\subset C^0 \cap W^0(\orz,\u,\v,\w)\\
  &\subset \Wdart(x).
  % \v,\w\in \barW &= \{\p \mid 0 
  %\le \op{azim}(\orz,\u,\v,\p) \le \op{azim}(\orz,\u,\v,\w)\}.\\
  % C^0 &\subset \barW\\
  % \bar W &\subset W(\orz,\u,\v,\w) 
  %\cup \op{aff}\{\orz,\u,\v\} \cup \op{aff}\{\orz,\u,\w\}\\
  % W(\orz,\u,\v,\w) & \subset \Wdart(x),
\end{array}
\end{displaymath}
%\begin{displaymath}
%\Wdart(x) = \bWdart(x)
%\setminus (\op{aff}\{\orz,\u,\v\}\cup\op{aff}\{\orz,\u,\w\}).
%\end{displaymath}
\end{proof}

\begin{lemma}[]\guid{TECOXBM}\rating{ZZ}\label{lemma:2hm-slice}
Let $(V,E,F,S)$ be a special fan.  Let $\v,\w\in V$ be distinct
elements such that $\norm{\v}{\w}=2\hm$ and such that
$\{\v,\w\}\not\in E$.  Then $\v$ and $\w$ are non-parallel, and
$C^0\{\v,\w\}\subset \Wdart(x)$ for all $x\in F$.
\end{lemma}

\begin{proof} By Remark~\ref{remark:fan-verify}, non-parallelism follows
from the strict triangle inequality
\begin{displaymath}
\norm{\v}{\w} \le 2\hm < 2 + 2 \le \normo{\v} + \normo{\w}.
\end{displaymath}
Assume for a contradiction that the conclusion of the lemma is false.
Then by Lemma~\ref{lemma:internal}, all the intermediate internal
angles between $\v$ and $\w$ are equal to $\pi$.  As a result, (after
interchanging $\v$ and $\w$ if necessary), the set
$\{\orz,\v,\rho\v,\rho^2\v,\ldots,\rho^r\v\}$ is planar, and
\begin{displaymath}
  \arc_V(\orz,\{\v,\rho^r\v\}) 
= \sum_{i=0}^{r-1} \arc_V(\orz,\{\rho^i\v,\rho^{i+1}\v\}),
\end{displaymath}
where $\rho^r \v = \w$ and $1 < r \le k-1$.
However, the left-hand side is at most
\begin{displaymath}
\arc(2,2,2\hm) < 1.5,
\end{displaymath}
while the right-hand side is at least
\begin{displaymath}
2\arc(2\hm,2\hm,2) > 1.5.
\end{displaymath}
This gives the desired contradiction.
\end{proof}

\subsection{slicing}

We stop to recall the current context of what we are trying to accomplish.
We are in the midst of preparations for the main estimate.
The proof of the main estimate is a proof by contradiction.  We will
assign a natural number invariant to each special fan.  We will
examine a minimal counterexample in the sense of having the smallest
possible invariant.  To use the minimality in the proof of the main
estimate, we need to relate the minimal counterexample to a special
fan that has an even smaller numerical invariant. This is accomplished
by changing the fan of the minimal counterexample by adding one more
blade.  This has the effect of slicing the fan in two.  This
subsection studies how the parameters of the minimal counterexample
relate to the parameters of the two fans produced by slicing.

\begin{definition}[slice]\guid{CNAQAAA}
 Let $(V,E,F)$ be a convex local fan.  Assume that
$\v,\w\in V$ are non-parallel and that $(V,E')=(V,E\cup
\{\{\v,\w\}\})$ is a fan.  Let $F'$ be the face of $\op{hyp}(V,E')$
containing the dart $(\w,\v)$.  Write
\begin{displaymath}(V[\v,\w],E[\v,\w],F[\v,\w])\end{displaymath}
for the localization of $(V,E')$ along $F'$, where
\begin{displaymath}
\begin{array}{lll}
  V[\v,\w] &= \{\v,\rho \v,\rho^2 \v,\ldots,\w\};\\
  E[\v,\w] &= \{\{\v,\rho \v\},\ldots,\{\rho^{-1}\w,\w\},\{\w,\v\}\};\\
  F[\v,\w] &= \{(\v,\rho \v),(\rho \v,\rho^2 \v),
 \ldots,(\rho^{-1}\w,\w),(\w,\v)\}.
\end{array}
\end{displaymath}
The triple $(V[\v,\w],E[\v,\w],F[\v,\w])$ is called the
\newterm{slice\/} of $(V,E,F)$ along $(\v,\w)$.
\end{definition}
\indy{Index}{fan!local}%
\indy{Index}{slice}
\indy{Notation}{1@$\cdot[\v,\w]$ (slicing a fan)}

To allow for more than one convex local fan $(V,E,F)$, we will extend
the notation, writing $\angle(H,\v)$ for $\angle(\v)$ in the hypermap
$H$.  Similarly, we write $\Wdart(H,\v)$ for $\Wdart(x)$, and so
forth.  \indy{Notation}{azimhv@$\op{azim}(H,\v)$}%
\indy{Notation}{wdart@$\Wdart$}%


\begin{lemma}[slicing]\guid{EJRCFJD}\rating{ZZ}\label{lemma:slice} Let
$(V,E,F)$ be a convex local fan with hypermap $H$.  Pick $\v,\w\in V$. For
each $\u\in \{\v,\w\}$, assume that $\u$ is not parallel with  any
element of $V\setminus\{\u\}$.  Assume that $C^0\{\v,\w\}\subset
\Wdart(x)$ for all darts $x\in F$.  Then
\begin{itemize}
\item $(V[\v,\w],E[\v,\w],F[\v,\w])$ and
$(V[\w,\v],E[\w,\v],F[\w,\v])$ are convex local fans.
\item Let $H[\v,\w]$ and $H[\w,\v]$ be the hypermaps of these two
  convex local fans, respectively.  Let $g:V\to\ring{R}$ be any
  function.  Then
\begin{displaymath}
  \sum_{\v\in V} g(\v)\angle(H,\v) 
  = \sum_{\v\in V[\v,\w]}g(\v)\angle(H[\v,\w],\v) 
  + \sum_{\v\in V[\w,\v]}g(\v)\angle(H[\w,\v],\v).
\end{displaymath}
\end{itemize}
\end{lemma}
\indy{Index}{slice}%
\indy{Index}{fan!local}%

\begin{proof} 
\claim{$(V,E')$ is a fan, where $E' = E\cup \{\{\v,\w\}\}$.}
Indeed, except for the intersection property, all of the properties
of a fan follow trivially from the fact that $(V,E)$ is a fan and
that $\v$ and $\w$ are non-parallel.  (Note the similarity with
Lemma~\ref{lemma:add-edge}.)  The intersection property also is
trivial except in the case $\ee=\{\v,\w\}$ and $\ee'\setminus \ee\ne
\emptyset$.  Pick $\u\in \ee'\setminus\ee$.  It follows from the
node partition of Lemma~\ref{lemma:disjoint} that
\begin{displaymath}
\begin{array}{lll}
C(\ee) \cap C(\ee') &= (C(\v) \cap C(\ee')) \cup (C(\w)\cap C(\ee')) \\
&= C(\{\v\}\cap \ee') \cup C(\{\w\}\cap \ee') \\
&= C(\{\v,\w\}\cap \ee').
\end{array}
\end{displaymath}
The intersection property thus holds and $(V,E')$ is a fan.

It follows by Lemma~\ref{lemma:localization} that
$(V[\v,\w],E[\v,\w],F[\v,\w])$ is a local fan.

The second conclusion of the lemma follows from the following identities.
If $\u\ne \v,\w$ with $\u\in V[\v,\w]$, then $\u\not\in V[\w,\v]$ and 
\begin{equation}
\Wdart(H,\u)=\Wdart(H[\v,\w],\u),\quad \angle(H,\u) = \angle(H[\v,\w],\u).
\end{equation}
If $\u\in\{\v,\w\}$, then 
$\angle(H,\u)=\angle(H[\v,\w],\u) +\angle(H[\w,\v],\u)$.

Finally, it remains to be shown that the local fan is convex.
%Lemma~\ref{lemma:localization} already shows that the hypermap is
%isomorphic to $\op{Dih}_{2k}$. and that $F[\v,\w]$ can be identified with a
%face.  
The conclusion $V[\v,\w]\subset \bWdart(x)$ follows from the
fact that the angles $\beta(i)$ are increasing in
Lemma~\ref{lemma:monotone}.
\end{proof}



The slicing procedure can also be applied to a special fan $(V,E,F,S)$.
Abbreviate the slices as
\begin{displaymath}
\begin{array}{lll}
(V',E',F')&=(V[\v,\w],E[\v,\w],F[\v,\w]),\quad\text{and}\\
(V'',E'',F'')&= (V[\w,\v],E[\w,\v],F[\w,\v]).
\end{array}
\end{displaymath}
Both edge sets $E'$ and $E''$ contain $\{\v,\w\}$.  The sets $G'$ and
$G''$ can be defined as
\begin{displaymath}
\begin{array}{lll}
G' &= \{\{\v,\w\}\} \cup (E'\cap S).\\
G'' &= \{\{\v,\w\}\} \cup (E''\cap S).\\
\end{array}
\end{displaymath} 

Let $(r',s')$ and $(r'',s'')$ be the parameters for $(V',E',F',G')$
and $(V'',E'',F'',G'')$, respectively.  Set $k=r+s$, $k'=r'+s'$, and
$k''=r''+s''$.

\begin{lemma}[slicing special fans]\guid{IXZYRSY}\rating{ZZ}\label{lemma:param-add}  
Let $(V,E,F,S)$ be a special fan (with parameters $s,r$).  Pick
distinct elements $\v,\w\in V$ such that $\{\v,\w\}\not\in E$.
Assume that $\norm{\v}{\w}=2\hm$.  For each $\u\in \{\v,\w\}$,
assume that $\u$ is not parallel with any element of 
$V\setminus\{\u\}$.
%Assume that $C^0\{\v,\w\}\subset \Wdart(x)$ for all darts $x\in F$. 
Then $(V',E',F',G')$ and $(V',E',F',G')$ are special fans.  Moreover,
the parameters satisfy the relations
\begin{displaymath}
k'+k'' = k + 2,\quad s'+s'' = s + 2,\quad r'+r''=r.
\end{displaymath}
Finally,
\begin{displaymath}
\tau(V,E,F)= \tau(V'',E'',F'') +\tau(V',E',F').
\end{displaymath}
\end{lemma}

By interchanging $\v$ and $\w$, the lemma also asserts that
$(V'',E'',F'',G'')$ is a special fan.

\begin{proof} The proof considers in turn each of the defining properties of a
  special fan.  By Lemma~\ref{lemma:2hm-slice}, if
  $\{v,w\}\not\in E$, then $C^0\{v,w\}\subset \Wdart(x)$ for all darts
  $x\in F$.

The properties \case{packing}, \case{annulus}, \case{diagonal},
\case{subset}, \case{s~norm}, and \case{e~norm} follow directly from
definitions and the corresonding properties for $(V,E,F,S)$.  The
property \case{local~fan} follows from Lemma~\ref{lemma:slice}.

The relations between the constants $k,s,r$ for the various fans
follows directly from the construction.  For example, there are two
more darts in $F''\cup F'$ than in $F$.

\claim{Property~\case{card} holds.}  Indeed, recall that each face
in the hypermap of a fan has at least three darts.  Hence $3\le k$,
$3\le k'$, and $3\le k''$.  The sets $G'$ and $G''$ contain
$\{\v,\w\}$.  Hence $1\le s'$ and $1\le s''$.  From $0\le r\le 6 -
2s$, it follows that $s\le 3$.  If $s=3$, then $r=0$, and $k=r+s=3$.
This means that $F$ is a triangle.  For some ordering of the pair,
$x=(\v,\w)$ is a dart in $F$.  Hence $C^0\{\v,\w\} \not\subset
\Wdart(x)$.  This is contrary to hypothesis.  Therefore, $s\le 2$.

Now for the verifications.
\begin{displaymath}0\le s' = s + 2 - s'' \le s+1\le 3.\end{displaymath}
\begin{displaymath}3-s'\le k'-s' = r'.\end{displaymath}
\begin{displaymath}k' = k + 2 - k'' \le k-1.\end{displaymath}
\begin{displaymath}
  r'= k'+s' - 2 s' \le (k-1) + (s+1) - 2s' 
  =k+s - 2s' = r + 2s -2s' \le 6 - 2s'.
\end{displaymath}
This completes the proof of property~\case{card}.

The additivity of $\tau$ follows directly from the azimuth angle
estimates in Lemma~\ref{lemma:slice}.
\end{proof}


\section{Minimality}

%\subsection{minimality}

This section continues to prepare for the proof of the main estimate.
The main estimate has the form
\[
\tau(V,E,F) \ge d(r,s),
\]
for all special fans $(V,E,F,S)$ with parameters $(r,s)$.  The proof
is by contradiction; we assume the existence of a counterexample.
Rather than analyze a generic counterexample, it is more convenient to
assume that a counterexample has various minimality properties.  This
section defines minimality precisely and develops the properties of a
minimal counterexample.  Eventually we show that a minimal
counterexample does not exist (Lemma~\ref{lemma:min-empty}).  This
gives the proof of main estimate (Lemma~\ref{lemma:empty-d}).


This definition captures the properties of a minimal counterexample.
This definition is transitory in the sense that eventually we show
that minimal fans do not exist.

\begin{definition}[minimal~fan,~$k_{min}$,~$\tau_{min}$]\guid{DJISMZB}
Let $k_{min}$ be the minimum of $r+s$ over
all special fans $(V,E,F,S)$ such that 
\begin{equation}\label{eqn:kmin}
\tau(V,E,F) < d (r,s),
\end{equation}
where $(r,s)$ are the parameters of the special fan.  If no special
fan exists that satisfies the inequality, then set $k_{min}=0$.  Let
$\tau_{min}$ be the infimum of $\tau(V,E,F)-d(r,s)$ over all special
fans $(V,E,F,S)$ whose parameters $(r,s)$ satisfy $r+s=k_{min}$.  If
no special fans exist with parameters $r+s=k_{min}$, then set
$\tau_{min}=0$.  Any special fan $(V,E,F,S)$ with parameter
$r+s=k_{min}$ such that
\begin{displaymath}
\tau_{min}= \tau(V,E,F)-d(r,s)
\end{displaymath}
is a \newterm{minimal} fan.
\end{definition}
\indy{Index}{fan!minimal}
\indy{Index}{minimal fan}


\begin{lemma}[]\guid{ADKOXQY}\rating{ZZ}\label{lemma:c-bound}
There is a constant $c>2\hm$ such that for every minimal fan
$(V,E,F,S)$ and every distinct $\v,\w\in V$, either $\{\v,\w\}\in E$
or $\norm{\v}{\w}\ge c$.
\end{lemma}

\begin{proof} 
  Pick a sequence of fan data $\p$ such that the associated fan is a
  minimal fan and such that
\begin{displaymath}
\min_{|i-j|>1} \norm{\p_i}{\p_j}
\end{displaymath}
is tending to the minimal value $c$.  By passing to s subsequence, we
may assume without loss of generality that every term in the sequence
has the same shape $(k_{min},I)$ for some $I$.  The sequence then lies
in a compact metric space.  Passing again to a subsequence, we may
assume without loss of generality that the sequence converges.  The
limiting value is a fan datum whose associated minimal fan $(V,E,F,S)$
satisfies
\begin{displaymath}
\norm{\v}{\w}=c,
\end{displaymath}
for some $\v,\w\in V$ such that $\{\v,\w\}\not\in E$.

Suppose for a contradiction that $c=2\hm$.  Then by
Lemma~\ref{lemma:2hm-slice}, $C^0\{\v,\w\}\subset \Wdart(x)$, for all
$x\in F$.  The fan can be sliced along the $\{\v,\w\}$.  The
parameters for the new pieces satisfy:
\begin{displaymath}
k' = k+2 - k'' \le k-1.
\end{displaymath}
The function $d$ is additive over slices.  For some constants $d_1$
and $d_2$,
\begin{equation}\label{eqn:drs}
\begin{array}{lll}
d(r,s) &= d_1 (2 - s) + d_2 (r + 2 s-4) \\
&= d_1 (2-s') + d_2 (r'+2 s'-4) + d_1 (2-s'') + d_2 (r''+2s''-4)\\
&= d(r',s') + d(r'',s''). \\
\end{array}
\end{equation}
By Lemma~\ref{lemma:param-add}, one of the two resulting fans
satisfies inequality~\ref{eqn:kmin}.  This is contrary to the choice
of $k_{min}$ in the definition of minimality.
\end{proof}





\subsection{genericity}

This subsection continues to prepare for the proof of the main
estimate.  The method of proof is to show that no minimal
counterexample can exist.  To that end, we proof various properties of
minimal fans throughout the section.  This subsection starts out with
the proof that minimal fans are generic.

\begin{lemma}[]\guid{RRAJQBH}\rating{ZZ}\label{lemma:circular-nonmin}
Every minimal fan is generic.
\end{lemma}

\begin{proof}
  By the classification of convex local fans, it is enough to show
  that the fan is not circular and not lunar.

\claim{In a minimal fan, $\op{azim}(x)<\pi$ for some $x\in F$.}
Otherwise, the result follows from the following estimates:
\begin{displaymath}
\begin{array}{lll}
  \tau(V,E,F) &=\sum_{x\in F} \op{azim}(x)
  \left(1 + \dfrac{\sol_0}{\pi}  \dfrac{\normo{v}-2}{2\hm-2}\right) 
+ \left(\pi+{\sol_0}\right) (2- k(F))\\
  &\ge \sum_{x\in F} \pi + \left(\pi+{\sol_0}\right) (2- k(F))\\
  &= 2\pi + 2\sol_0 - k(F) \sol_0\\
  &\ge 2\pi - 4\sol_0\\
  &> 0.7578\\
  &=0.103 (2) + 0.2759 (2)\\
  &\ge 0.103 (2-s) + 0.2759 (r+2s-4) \\ 
  &= d(r,s)\\
\end{array}
\end{displaymath}
This proves the claim.  In particular, a minimal fan is not circular.


Section~\ref{sec:deformation} develops a theory of deformations of
convex local fans $(V,E,F)$.  Recall that a deformation is continuous
function $\varphi:V\times I\to\ring{R}^3$.  A deformation determines
sets $V(t)$, $E(t)$, $F(t)$, and $S(t)$ for each $t\in I$.


\claim{A minimal fan is not lunar.}  Otherwise, suppose for a
contradiction, that $(V,E,F)$ is lunar, special, and minimal.  Let
$\v,\w\in V$ be parallel.  By the previous claim, we may assume
without loss of generality that $\angle(\v)=\angle(\w)<\pi$.
Example~\ref{example:lunar} describes a deformation of the lunar fan
$(V,E,F)$ over $I=\leftclosed0,1\rightopen$.  The deformation fixes
$\normo{\u}$ and is non-increasing in $\angle(\u(t))$, for $\u\in V$.
From the defining formula for $\tau$, it follows that the function
$\tau(V(t),E(t),F(t))$ is a decreasing function of $t$.

\claim{For sufficiently small positive $t$, the deformed value is a
special fan.}  The deformed value is a convex local fan for small positive
$t$ by Lemma~\ref{lemma:fan-open}.  The property~\case{diagonal} of
special fans holds by Lemma~\ref{lemma:c-bound}.  The other properties
of a special fan follow immediately from the construction.

However, the function $\tau$ attains its minimum at $(V,E,F)$ and
hence $\tau(V(t),E(t),F(t))$ cannot be decreasing in $t$.  This
contradiction proves the claim that a minimal fan is not lunar.
\end{proof}

\begin{lemma}[]\guid{QAGHDMN}\rating{ZZ}
Let $(V,E,F)$ be a convex local fan.  If $V$ is contained in a plane
through the origin, then $(V,E,F)$ is not generic.
\end{lemma}

\begin{proof}
\end{proof}

\begin{lemma}[]\guid{EAEKYHM}\rating{ZZ}\label{lemma:3-nonflat}
Every non-generic convex local fan $(V,E,F)$ has at least three nonflat
nodes in $V$.
\end{lemma}

\begin{proof}
If there is at most one nonflat nodes $\v\in V$, then there exists
a plane $A$ through the origin that contains $V$.  This is not
generic.

If there are exactly two nonflat nodes in $V$, then
there exist two half-planes through the origin whose union contains
$V$.  The two non-flat points necessarily lie along the intersection
of the two planes.  They are parallel nodes of a lunar fan.
This is not generic.
\end{proof}

\subsection{features}

The aim is to show that the set of minimal fans is empty.  The proof
is not a simple proof that works in one fell swoop.  Instead, we
proceed indirectly, by proving a long list of features that any
minimal fan must have.  Each new feature puts a small constraint on
the set of minimal fans.  By slowly amassing more and more features,
the set gradually becomes heavily constrained, until finally it is
seen to be empty.

\begin{definition}[extremal,~minimal]\guid{VLEKVIO}
 Let $(V,E,F,S)$ be a special fan
and let $\{\v,\w\}\in E$.  The edge $\{\v,\w\}\in E$ is
\newterm{S-extremal} if $\{\v,\w\}\in S$ or
\begin{displaymath}
\norm{\v}{\w}\in\{2,2\hm\}.
\end{displaymath}
The edge $\{\v,\w\}\in E$ is \newterm{S-minimal} if $\{\v,\w\}\in S$ or
\begin{displaymath}
\norm{\v}{\w}=2.
\end{displaymath}
\end{definition}
\indy{Index}{extremal ($S$-extremal edge)}
\indy{Index}{minimal ($S$-minimal edge)}

Here we make a summary of various sets of special fans. In a long
series of lemmas, it will be shown that the set of minimal fans is a
subset of every one of these sets.  To lighten the notation in this
subsection, we introduce the following convention.  $(V,E,F,S)$ is a
special fan.  If $v\in V$, then we introduce the abbreviation $\v_i =
\rho^i\v$, for all $i\in\ring{Z}$.  In particular, $\v_0=\v$,
$\v_1=\rho\v$, and so forth.

The following lemma has been phrased in a pecular way.  It gives a
list of features that are quite evidently inconsistent.  The lemma
asserts the inconsistency.  After the lemma, we will show in a series
of lemmas that a minimal fan has all of these features.  Because of
the inconsistency of the list, it follows that a minimal fan cannot
exist.  We name each feature for ease of reference.

\begin{lemma}[feature list]\guid{IDOTAPN}\label{lemma:feature}
No special fan $(V,E,F,S)$ has all of the following features.
%\newterm{irreducible} if the following properties hold.
\begin{itemize}
\item \case{extreme~edge} For every $\v\in V$ and every $\w\in E(\v)$, the edge
$\{\v,\w\}\in E$ is $S$-extremal.
\item \case{flat~exists} For every $\v\in V$, at least one of $\v_1,\v_2,\v_3,\v_4$ is flat.
%then $r+s\le 4$ and $\norm{\u}{\w}=2$ for all $\{u,w\}\in E$.
% $r+s=4$. %, and $\norm{\u}{\w}\in\{ 2,2\hm}$ for all $\{\u,\w\}\in E$.
\item \case{no~triple~flat} For every $\v\in V$, at least one of $\v_0$, $\v_1$, and $\v_2$
is not flat.
\item \case{balance} For every $\v\in V$, if $\v$ is flat and if $\{\u,\v\},\{\w,\v\}\in
E\setminus S$, then
\begin{displaymath}
\norm{\u}{\v} = \norm{\w}{\v}.
\end{displaymath}
\item \case{s~flat} For every $\v\in V$, if $\v_1$ and $\v_2$ are both flat, then 
\begin{displaymath}S\cap
\{\{\v_0,\v_1\},\{\v_1,\v_2\},\{\v_2,\v_3\}\} =
\emptyset.\end{displaymath}
\item \case{flat~middle} For every $\v\in V$, if $\v_1$ and $\v_2$ are both flat, then
\begin{displaymath}
\norm{\v_1}{\v_2} = 2.
\end{displaymath}
\item \case{minimal~node} For every $\v\in V$, either $\normo{\v}=2$
  or there exists $\w\in E(\v)$ such that $\{\v,\w\}$ is $S$-minimal.
\item \case{minimal~node~flat} For every $\v\in V$, if $\v$ is flat, and $\{\u,\v\}\in
E\setminus S$, then $\norm{\u}{\v}=2$ or $\normo{\v}=2$.
\item \case{flat~extremal} For every $\v\in V$, if $\v_1$ and $\v_2$
  are both flat, then either $\v_1$ or $\v_2$ has extremal norm:
\begin{displaymath}\normo{\v_1}\in \{2,2\hm\}\quad\text{ or }\quad
\normo{\v_2}\in \{2,2\hm\}.\end{displaymath}
% \item \case{extremal~node} If $\v_0$, $\v_1$, and $\v_2$ are not
%   flat, then
%\begin{displaymath}
%\normo{\v_1}\in \{2,2\hm\}.
%\end{displaymath}
% \item \case{flat~extremal~node} If $\v_1$ is flat, but none of
%   $\v_0,\v_2,\v_3$ is flat, then
%\begin{displaymath}
%\normo{\v_1}\in \{2,2\hm\}
%\quad\text{ or }\quad\normo{\v_2}\in \{2,2\hm\}.
%\end{displaymath}
% \item \case{flat~extremal~node sym} If $\v_2$ is flat, but none of
%   $\v_0,\v_1,\v_3$ is flat, then
%\begin{displaymath}
%\normo{\v_1}\in \{2,2\hm\}
%\quad\text{ or }\quad\normo{\v_2}\in \{2,2\hm\}.
%\end{displaymath}
\item \case{flat~count} There are at least $3$ nonflat nodes of $V$.
\item \case{card} %$k=\card(F)$,
Let      $s=\card(S)$ and $r=\card(E) - s = \card(F)-s$.  Then
\begin{displaymath}0\le s \le 3,\quad\text{and}\quad3-s \le r \le 6 -
2s.\end{displaymath}
\item \case{s~minimal} Either $r+2s = 6$, or every edge of $E$ is $S$-minimal.
\item \case{triangle~free} $\op{card}(F) \ne 3$.
\item \case{quadrilateral~free} $\op{card}(F) \ne 4$.
\item \case{pentagon~free} $\op{card}(F) \ne 5$.
\item \case{hexagon~free} $\op{card}(F) \ne 6$.
\end{itemize}
\end{lemma}

\begin{proof}
  By \case{card}, $3\le\op{card}(F)\le 6-s\le 6$.  However, by
  \case{triangle~free}, \case{quadrilateral~free},
  \case{pentagon~free}, and \case{hexagon~free}, we cannot have
  $3\le\op{card}(F)\le 6$.
\end{proof}


%\begin{lemma}[]\guid{KDKAGRS}\rating{ZZ}\label{lemma:min-irred}
%Every minimal fan $(V,E,F,S)$ is irreducible at every $\v\in V$.
%\end{lemma}

%The proof appears in the next subsection as a series of small
%verifications.

%\subsection{features}

If there are $i$ consecutive flat nodes $\v_1,\ldots,\v_i$, then there
are $i+2$ corresponding nodes that lie in a plane $A$ through the
origin: $\{\orz,\v_0,\ldots,\v_{i+1}\}\subset A$.

\begin{lemma}[]\guid{FPITROS}\rating{ZZ}
Every minimal fan $(V,E,F,S)$ has the feature \case{card}.
\end{lemma}

\begin{proof} This holds by the property \case{card} of special fans.
\end{proof}


\begin{lemma}[]\guid{TESVAFW}\rating{ZZ}
Every minimal fan $(V,E,F,S)$ has the feature \case{no~triple~flat}.
\end{lemma}

\begin{proof}  Three flat nodes produces an arc
\begin{displaymath}
  \arc_V(0,\{\v_0,\v_4\}) 
= \op{per}(V,E,F,\v_0,\v_4) \ge4\arc(2\hm,2\hm,2) > \pi,
\end{displaymath}
but the remaining edges  have combined length at most
\begin{displaymath}
\arc_V(0,\{\v_0,\v_4\})\le 2\arc(2,2,2\hm) < \pi.
\end{displaymath}
\end{proof}

\begin{lemma}[]\guid{GOKZLRP}\rating{ZZ}
Every minimal fan $(V,E,F,S)$ has the feature \case{flat~count}.
\end{lemma}

\begin{proof}  
This lemma is a consequence of Lemma~\ref{lemma:3-nonflat}.
\end{proof}



\begin{lemma}[]\guid{SDCCMGA}\rating{ZZ}
Every minimal fan $(V,E,F,S)$ has the feature \case{s~flat}.
\end{lemma}

\begin{proof} Argue by contradiction.  From the constraints on $r$ and
$s$, if $S\ne\emptyset$, then $s>0$ and $r+s\le 5$.
By property \case{flat~count}, it follows that $r+s\ge 5$.
Thus, $r+s=5$.  
% then $V = \{\v,\rho \v,\rho^2\v,\rho^3\v\}$.  A plane $A$ contains
% $V\cup\{\orz\}$.  This is not generic.

A plane $A$ contains
$\{\orz,\v_0,\v_1,\v_2,\v_3\}$.  We obtain a contradiction
by computing $\arc_V(\orz,\v_0,\v_3)$ in two ways.  On the
one hand, it is equal to the partial perimeter from $\v_0$ to
$\v_3$.  One the other hand, it is estimated by the triangle
inequality, applied to the other two edges:
\begin{displaymath}
\begin{array}{lll}
\arc_V(\v_0,\v_3)
&\le\arc_V(\v_3,\v_4) + \arc_V(\v_4,\v_0) \\
&\le\arc(2,2,2\hm)+\arc(y,2,2\hm)\\
&<\arc(y,2\hm,2)+\arc(2\hm,2\hm,2) +\arc(2\hm,2\hm,2\hm)\\
&\le\arc_V(\v_0,\v_3).
\end{array}
\end{displaymath}
where $y=\normo{\v_0}$.
This is a contradiction.
\end{proof}

\begin{lemma}[]\guid{BAWWPPB}\rating{ZZ}
Every minimal fan $(V,E,F,S)$ has the feature \case{extreme~edge}.
\end{lemma}

\begin{proof} 
By Lemma~\ref{lemma:3-nonflat}, there are at least three flat
nodes of $V$.

Assume for a contradiction that some edge $\{\u,\rho\u\}\in
E\setminus S$ of the minimal fan $(V,E,F,S)$ is not extremal.  
We can find $\v_0,\ldots,\v_n$ with the following properties:
\begin{itemize}
\item $\rho^i\v_0 =\v_i$.
\item $\v_0$, $\v_n$ and exactly one other point $\v_m$ (for some
  $0<m<n$) are not flat.
\item $\u=\v_k$ for some $0\le k<m$.
\end{itemize}

Select a deformation $\varphi$ that satisfies the following conditions.
\begin{itemize}
\item The deformation fixes every element of
$V\setminus\{\v_1,\ldots,\v_{n-1}\}$.
\item The set $\{\orz,\v_0(t),\ldots,\v_m(t)\}$ is coplanar.
\item The set $\{\orz,\v_m(t),\ldots,\v_n(t)\}$ is coplanar.
\item The deformation  fixes the norm $\normo{\v_i(t)}$ for all $0<i<n$.
\item The deformation fixes the norm $\norm{\v_i(t)}{\v_{i+1}(t)}$
  whenever $0\le i<n$ and $i\ne k$.
\item $\norm{\v_k(t)}{\v_{k+1}(t)} = \norm{\v_k}{\v_{k+1}}+t$.
\end{itemize}
There is a unique deformation that satisfies these properties.  The
deformation is a special fan for sufficiently small $t$.

\claim{The function $t\mapsto \tau(V(t),E(t),F(t))$ does not have a
local minimum at $t=0$.}  Indeed, by Definition~\ref{def:tau},
$\tau$ has the form
\begin{displaymath}
  g(s) = \dih(2,2,2,a+s,b,c) e_1 
+ \dih(2,2,2,b,c,a+s) e_2 + \dih(2,2,2,c,a+s,b) e_3,
\end{displaymath}
for some $e_i\in\leftclosed1,1+\sol_0/\pi\rightclosed$, some
reparametrization $s=s(t)$ of $t$ such that $s(0)=0$, and some
parameters $a,b,c$.  The parameters $a,b,c$ lie in
$\leftclosed2/\hm,4\rightclosed$ and satisfy $\Delta\ge0$, where
$\Delta = \Delta(4,4,4,a^2,b^2,c^2)$.

If $g$ has a local minimum at $s=0$, then
\begin{equation}\label{eqn:g''}
\Delta g'(0)^2 - 0.01\Delta^{3/2} g''(0) \le 0.
\end{equation}
Indeed, a local minimum gives $g'(0)=0$ and $g''(0)\ge0$.  However,
Calculation~\ref{calc:Lexell} shows that the left-hand side of
inequality~(\ref{eqn:g''}) is positive for all parameters
$a,b,c,e_1,e_2,e_3$.  Thus, $g$ does not have a local minimum at
$s=0$.  The claim contradicts that assumed minimality of $(V,E,F,S)$.
This completes the proof.
\end{proof}


\begin{lemma}[]\guid{PZFSEVR}\rating{ZZ}
Every minimal fan $(V,E,F,S)$ has the feature \case{minimal~node}.
\end{lemma}

\begin{proof} If the conclusion is false, then for sufficiently small
positive $a$, the deformation $\varphi$ over $[0,a]$ given by
\begin{displaymath}
\varphi(\u,t) =
\begin{cases}
\u, & \text{if }\u \ne \v,\\
(1-t) \v, & \text{if }\u = \v\\
\end{cases}
\end{displaymath}
decreases $\tau$.  It is a deformation of special fans.  This shows
that $(V,E,F,S)$ is not minimal.
\end{proof}

\begin{lemma}[]\guid{JLXMEJN}\rating{ZZ}   %case minimal node flat
Every minimal fan $(V,E,F,S)$ has the feature \case{minimal~node
flat}.
% \case{minimal~node~flat} If $v$ is flat, and $\{u,v\}\in
% E\setminus S$, then $\norm{\u}{\v}=2$ or $\normo{\v}=2$.
\end{lemma}

\begin{proof} 
Let $A$ be the plane containing $\{\orz,\rho^{-1}\v,\v,\rho\v\}$.
Assume that $\{\u,\v\}\in E\setminus S$.  Let $\u\ne \w\in E(\v)$.
Then $\u,\v,\w\in A$.

If the conclusion is false, then for sufficiently small positive
$a$, let $\varphi$ be the deformation over $[0,a]$ that fixes all
elements of $V$ other than $\v$ and moves $\v$ in the plane $A$
around the circle of radius $\norm{\w}{\v}$ with center $\w$.  The
direction of the circular motion can be chosen to decrease $\tau$, $\normo{\v}$
and $\norm{\v}{\u}$.
It is a deformation of special fans.  This shows that $(V,E,F,S)$ is
not minimal.
\end{proof}

\begin{lemma}[]\guid{TPKKQOL}\rating{ZZ}
Every minimal fan $(V,E,F,S)$ has the feature \case{balance}.
%For every edge $\{\v,\w\}\in E$ in a minimal fan $(V,E,F,S)$, 
%\begin{displaymath}\norm{\v}{\w}\in\{2,2\hm\}.\end{displaymath}
\end{lemma}

\begin{proof} Assume for a contradiction that the property fails.  In
view of properties \case{extreme~edge} and \case{minimal~node~flat},
after possibly swapping $\u$ and $\w$, we may assume without loss of
generality that
\begin{displaymath}
\normo{\v}=2,\quad \norm{\u}{\v}=2,\quad \norm{\w}{\v}=2\hm.
\end{displaymath}
For sufficiently small positive $a$, let $\varphi$ be the deformation
over $[0,a]$ that fixes all elements of $V$ other than $\v$ and moves
$\v$ in the plane of $\{\orz,\u,\v,\w\}$ around a circle of radius $2$
with center at the origin.  The direction of the deformation can be
chosen to be increasing in $\norm{\u}{\v}$.  The function $\tau$ is
constant along the deformation.  The deformation carries minimal fans
to minimal fans.  However, the deformed fan does not satisfy property
\case{extreme~edge}.  This is a contradiction.
\end{proof}




\begin{lemma}[]\guid{CFJSRQH} Every minimal fan $(V,E,F,S)$ has property
\case{flat~middle} at every $\v\in V$.
\end{lemma}

\begin{proof} %Write $\v_i = \rho^{i-2} \v$.  
Assume for a contradiction tht the conclusion is false.  There is a
plane $A$ that contains $\{\orz,\v_0,\v_1,\v_2,\v_3\}$.  By property
\case{s~flat}, none of the edges $\{\v_0,\v_1\}$, $\{\v_1,\v_2\}$,
$\{\v_2,\v_3\}$ lies in $S$.  Furthermore, by \case{extreme~edge}
and \case{balance}, the distances $\norm{\v_i}{\v_{i+1}}$ are equal
to one another and take values in $\{2,2\hm\}$, for $i=0,1,2$.  By
\case{minimal~node}, if these distances are $2\hm$, then the partial
perimeter from $\v_0$ to $\v_3$ is at least
\begin{equation}\label{eqn:3side}
\arc(y_1,2,2\hm)+\arc(2,2,2\hm)+\arc(2,y_4,2\hm),
\end{equation}
where $y_i=\normo{\v_i}$.  However, there are at most two other nodes
($\v_4$ and $\v_5$) and three other edges in the special fan.  By the
triangle inequality, the sum of these three lengths is at most
\eqn{eqn:3side}.  Thus, equality is obtained in the triangle
inequality.  This implies that $\v_4$ and $\v_5$ lie in the plane $A$.
Thus, $V\subset A$.  This is not generic.
\end{proof}

\begin{lemma}[]\guid{OUCPLRI} 
Every minimal fan $(V,E,F,S)$ has the feature \case{flat~extremal}.
\end{lemma}

\begin{proof}  The proof is by contradiction.
Assume there are two adjacent flat nodes $\v_1,\v_2$, so that
there is a plane $A$ that contains $\{\orz,\v_0,\v_1,\v_2,\v_3\}$.
By previously established properties of minimality,
\indy{Index}{angle!flat}%
$\norm{\v_i}{\v_{i+1}}=2$, for $i=0,1,2$.  Let $y_i = \normo{\v_i}$.
%

\claim{There exists a deformation that fixes all $\u\ne \v_1,\v_2$,
maintains flatness at $\v_1$ and $\v_2$, maintains the coplanarity
of $\{\orz,\v_0,\v_1(t),\v_2(t),\v_3\}$, and that keeps $y_1+y_2$,
$\norm{\v_0}{\v_1}$, $\norm{\v_2}{\v_3}$, and $\tau$ constant,
while increasing $\norm{\v_1}{\v_2}$.}  Indeed, the partial
perimeter $c$ is given by a sum of three terms:
\begin{displaymath}
c=\sum_{i=0}^2\arc(y_i,y_{i+1},2).
\end{displaymath}
Set
\begin{displaymath}
  g(t_1,t_2) = \arc(y_0,y_1+t_1,2) 
+ \arc(y_1+t_1,y_2-t_1,2+t_2) + \arc(y_2-t_1,y_3,2) - c.
\end{displaymath}
Then $g(0,0)=0$.  

Use subscript notation for partial derivatives of $g$.  If $g_1(0,0)
\ne 0$, then by the implicit function theorem there is a function $h$
locally near $0$ such that $h(0)=0$ and $g(h(t_2),t_2)=0$.  Define a
deformation that fixes $\u\ne\v_1,\v_2$ and moves $\v_1,\v_2$ in the
fixed plane $\op{aff}\{\orz,\v_1,\v_2\}$ subject to the constraints
$\norm{\v_1(t)}{\v_2(t)}=2+t$, $\normo{\v_1(t)} = 2+h(t)$, and
$\normo{\v_2(t)} = 2-h(t)$.  These conditions uniquely determine the
deformation.  The deformed value is a special fan for sufficiently
small positive $t$.  It keeps $\tau$ fixed.  Hence the deformed fan is
also minimal.  However, the existence of this deformation contradicts
property \case{flat~middle}.

On the other hand, if $ g_1(0,0) =0$, then a calculation gives
\begin{displaymath}g_2(0,0) = \dfrac{2}{y_1y_2\sqrt{\ldots}} >
0.\end{displaymath} Again by the implicit function theoreom there is
a function $h$ locally near $0$ such that $h(0)=0$ and
$g(t_1,h(t_1))=0$.  Define a deformation that fixes $\u\ne \v_1,\v_2$
and moves $\v_1$ and $\v_2$ within the fixed plane
$\op{aff}\{\orz,\v_1,\v_2\}$ by the conditions $\normo{\v_1(t)} = 2 +
t$, $\normo{\v_2(t)} = 2 - t$, and $\norm{\v_1(t)}{\v_2(t)}=2+h(t)$.
These conditions uniquely determine the deformation.  The deformed fan
is special for sufficiently small $t$.  It keeps $\tau$ fixed.  Hence
the deformed fan is also a minimal.

By implicit differentiation, $h'(0) = 0$ and $g_2(0,0) h''(0) =
-g_{11}(0,0)$.  Thus, $h''(0)$ and $g_{11}(0,0)$ have opposite signs.
The function $t_1\mapsto g(t_1,0)$ equals, up to a constant, the sum
of three terms of the form $\arc(\cdot,\cdot,2)$.
Calculation~\ref{calc:2der} shows that the second derivative of each
term  is negative at $t=0$:
\begin{equation}\label{eqn:arc''}
\def\vt{\vrule_{\relax{\raisebox{-0.55em}{\ensuremath{\displaystyle{~t=0}}}}}}
\frac{d^2 \arc(y_0,y_1+t,2)}{d t^2}\,\vt < 0;\quad
\dfrac{d^2 \arc(y_0+t,y_1-t,2)}{d t^2}\,\vt < 0.
\end{equation}  
Thus, $h''(0)>0$
and for sufficiently small nonzero $t$, $\norm{\v_1(t)}{\v_2(t)}>2$.
However, the existence of this deformation contradicts property
\case{flat~middle}.
\end{proof}


\begin{lemma}[]\guid{HJXLQOT}\rating{ZZ} 
Every minimal fan $(V,E,F,S)$ has the feature \case{s~minimal}.
\end{lemma}

\begin{proof} Assume for a contradiction that the minimal fan has the
  properties $r+2s < 6$, and some edge is not $S$-minimal.  We also
  have $s<3$, for otherwise $r+2s\ge 6$.  The constants $d(r,s)$
  satisfy $d(r-1,s+1) > d(r,s)$.  Thus, we may decrease
  $\tau(V,E,F)-d(r,s)$ by transferring one edge of $E\setminus S$ that
  is not $S$-minimal into the set $S$.  This changes the parameters
  $(r,s)\mapsto (r-1,s+1)$.
\end{proof}

\begin{lemma}[]\guid{OMKYNLT}\rating{ZZ}\label{lemma:triangle-free}
Every minimal fan $(V,E,F,S)$ has the feature \case{triangle~free}.
\end{lemma}

\begin{proof}
  For a contradiction, let $(V,E,F,S)$ be a minimal fan with
  $\card(F)=3$.  By property \case{flat~count}, every node is nonflat.
  By \case{s~minimal}, $r+2s=6$, or every edge of $E$ is $S$-minimal.
  However, if $r+2s=6$, then $S=E$, and so no matter what, every edge
  of $E$ is $S$-minimal.

There are four cases: $\card(S)\in\{0,1,2,3\}$.  Each case is a
nonlinear optimization problem in three variables $\normo{\v_i}$,
$i=1,2,3$.  The edge lengths $\norm{\v_i}{\v_{i+1}}$ are determined by
$S$-minimality.  Calculations~\ref{calc:XX-not-done} show that
$\tau(V,E,F)\ge d(r,s)$ in each case.
\end{proof}



%\begin{lemma}[]\guid{DWXPIHA}\rating{ZZ}
%  Every minimal fan $(V,E,F,S)$ has the feature \case{extremal
%    node} for every $\v\in V$.
%\end{lemma}
%
%\begin{proof} %\guid{DFSLRHA} 
%  Assume for a contradiction that the conclusion is false.  By
%  property~\case{extremal~edge},
%\begin{displaymath}
%\norm{\v_{-1}}{\v_0},~\norm{\v_0}{\v_1}\in \{2,2\hm\}.
%\end{displaymath}
%Consider the deformation $\varphi:V\times I\to V$ given by
%\begin{displaymath}
%\varphi(\u,t) =
%\begin{cases}
%\v(t), & \text{if } \u=\v\\
%\u, & \text{otherwise},\\
%\end{cases}
%\end{displaymath}
%where $\v(t)$ is the unique point in $\op{aff}_+(\{0,\rho^{-1}\v,\rho
% \v\},\v)$ determined by the conditions:
%\begin{displaymath}
%\norm{\v(t)}{\u} = \norm{\v}{\u}, \text{~when~} \{\u,\v\}\in E,\quad
%\text{~and~}
%\normo{\v(t)} = \normo{\v} + t.
%\end{displaymath}
%For sufficiently small $t$, the deformed value  is a special fan.
%The function $t\mapsto\tau(V(t),E(t),F(t))$ has the form
%\begin{equation}\label{eqn:tau}
%t\mapsto \tau_{tri}(y_1+t,y_2,y_3,y_4,y_5,y_6) + c
%\end{equation}
%for some parameters $y_1,\ldots,y_6$ and $c$.
%Calculation~\ref{calc:cc:d2a} shows that the function (\ref{eqn:tau})
%has no local minimum at $t=0$.
%This is contrary to the minimality of $(V,E,F,S)$.
%\end{proof}



%The same can be done when there is one flat node forming a linear
% series of length $2$, with compressed edges:

% \begin{lemma}[]\guid{DCEETTF}\rating{ZZ} Every minimal fan
%   $(V,E,F,S)$ satisfies properties \case{flat~extremal~node} and
%   \case{flat~extremal~node sym} for every $\v\in V$.
%\end{lemma}
%
%\begin{proof}
%  The property \case{flat~extremal~node sym} is obtained from
%  \case{flat~extremal~node} by symmetry $v_1\leftrightarrow v_2$, so
%  it is enough to prove \case{flat~extremal~node}.  Assume for a
%  contradiction that property \case{flat~extremal~node} fails.  Let
%  $(V,E,F,S)$ be a minimal fan that fails.
%
%  Define a deformation $\varphi$ of the minimal fan that fixes $\u\ne
%  \v_1,\v_2$ and that moves $\v_1,\v_2$ subject to the following
%  constraints.
%\begin{itemize}
%\item $\{\orz,\v_0,\v_1(t),\v_2(t)\}$ is coplanar.
%\item $\norm{\v_2(t)}{\v_3}$ is constant.
%\item $\norm{\v_0}{\v_1(t)}$, $\norm{\v_1(t)}{\v_2(t)}$, and
%  $\norm{\v_0(t)}{\v_2(t)}$ are constant.
%\item $\normo{\v_2(t)} = \normo{\v_2} + t$.
%\end{itemize}
%These constraints uniquely determine the deformation. The deformed
% value is a special fan for sufficiently small $t$.
%
% By Calculation~\ref{calc:cc:d2b}, the function does not have a local
% minimum at $t=0$.  This contradicts the assumption that $(V,E,F,S)$
% is minimal.
%\end{proof}



\begin{lemma}[]\guid{CMBZAOZ}\rating{ZZ}\label{lemma:flat-exists}
Every minimal fan $(V,E,F,S)$ has the feature \case{flat~exists}.
  That is, if none of $\v,\rho
\v,\rho^2 \v,\rho^3 \v\in V$ is flat, then $r+s\le 4$ and
$\norm{\u}{\w}=2$ for all $\{u,w\}\in E$.
%then $r+s=4$. %, and $\norm{\v}{\w}= 2$, for all $\{\v,\w\}\in E$.
\end{lemma}

\begin{proof} 
Assume for a contradicton that the property fails for $(V,E,F,S)$.
There are four
consecutive nodes $\v_0,\v_1,\v_2,\v_3$ that are not flat.  By
property \case{extreme~edge}, each edge $\norm{\v_i}{\v_{i+1}}$ is
$2$ or $2\hm$. Set $y_i = \normo{\v_i}$ and $y_{ij} =
\norm{\v_i}{\v_j}$.
%By \case{extremal~node}, $y_i$ is $2$ or $2\hm$, for $i=1,2$.   

Shifting $v$ to a different element of $V$ if necessary when
$r+s=4$, we may assume without loss of generality that $y_{03}\ge
2\hm$.

Define a deformation of $(V,E,F,S)$ by fixing $\u\ne \v_1,\v_2$ and
moving $\v_1,\v_2$ according to the following constraints:
\begin{itemize}
\item $\norm{\v_0}{\v_1(t)}$, $\norm{\v_1(t)}{\v_2(t)}$, and
$\norm{\v_2(t)}{\v_3}$ are constant.
\item $\normo{\v_1(t)}$ and $\normo{\v_2(t)}$ are constant.
\item $\norm{\v_0}{\v_2(t)} = y_{02} + t$.
\end{itemize}
These constraints uniquely determine the deformation. The deformed
value is a special fan for sufficiently small $t$.  In this context,
by Calculation~\ref{calc:cc:qua}, the function $\tau$ as a function of
$t$ (with parameters $y_{02},y_0,y_3,y_{03}$) does not have a local
minimum at $t=0$.  This is contrary to the assumed minimality of
$(V,E,F,S)$.
\end{proof}

\begin{lemma}[]\guid{JNTEFVP}\rating{ZZ}\label{lemma:quadrilateral-free}
Every minimal fan $(V,E,F,S)$ has the feature \case{quadrilateral~free}.
\end{lemma}

\begin{proof}
For a contradiction, let $(V,E,F,S)$ be a minimal fan with $\card(F)=4$.  


If no node is flat, then property \case{flat~exists} shows that
$S=\emptyset$ and every edge is $\emptyset$-minimal.  The inequality
$\tau(V,E,F) \ge d(4,0)$ is established by
Calculation~\ref{calc:XX-not-done}.  The space of configurations is
five-dimensional.

By property \case{flat~count}, one node is flat.  The parameter $s$
runs over three cases $s=0,1,2$.  Calculation~\ref{calc:XX-not-done}
gives $\tau(V,E,F)\ge d(4-s,s)$ in each case.  The space of
configurations is four-dimensional.
\end{proof}

\begin{lemma}[]\guid{PQFYWHW}\rating{ZZ}\label{lemma:pentagon-free}
Every minimal fan $(V,E,F,S)$ has the feature \case{pentagon~free}.
\end{lemma}

\begin{proof}
For a contradiction, let $(V,E,F,S)$ be a minimal fan with $\card(F)=5$.  

By a double application of \case{flat~exists}, there are at least two
flat nodes.  By \case{flat~count}, there are exactly two flat nodes.
The flat nodes may either be consecutive or not.  The parameter $s$
may be $0$ or $1$.  Calculation~\ref{calc:XX-not-done} in each of
these cases gives $\tau(V,E,F)\ge d(5-s,s)$.  The space of
configurations is five-dimensional when the flat nodes are not
adjacent.  Because of \case{flat~extremal}, the space of
configurations is only four-dimensional when the flat nodes are
adjacent.
\end{proof}

\begin{lemma}[]\guid{GYQVFXJ}\rating{ZZ}\label{lemma:hexagon-free}
Every minimal fan $(V,E,F,S)$ has the feature \case{hexagon~free}.
\end{lemma}

\begin{proof}
For a contradiction, let $(V,E,F,S)$ be a minimal fan with $\card(F)=6$.
From \case{card}, it follows that $s=0$, and $S=\emptyset$.

By \case{flat~exists}, there are at least two flat nodes.  By
\case{flat~count}, there are at most three flat nodes.  By the
property \case{no~triple~flat}, the possible arrangements of flat
nodes are as follows.
\begin{itemize}
\item two flat nodes, opposite to one another: $\v_0,\v_3$.
\item two flat nodes, not adjacent and not opposite: $\v_0,\v_2$.
\item three flat nodes, alternating arrangement: $\v_0,\v_2,\v_4$.
\item three flat nodes, $\v_0,\v_1,\v_i$, $i=3$ (or symmetrically $i=4$).
\end{itemize}
Calculations~\ref{calc:XX-not-done} in each case give the inequality
$\tau(V,E,F)\ge d(6,0)$.  The dimension of the configuration spaces
are seven, seven, six, and five respectively.
\end{proof}  

% The preceding series of lemmas completes the proof of
% Lemma~\ref{lemma:min-irred}, showing that every minimal fan is
% irreducible at each of its nodes.


\subsection{emptiness}


\begin{lemma}[]\guid{LPQUDGF}\rating{ZZ}\label{lemma:min-empty}  
The set of minimal fans is empty.
\end{lemma}

\begin{proof} Let $(V,E,F,S)$ be a minimal fan.  By the long sequence
  of lemmas in the previous section, it is a special fan that has all
  the features listed in Lemma~\ref{lemma:feature}.  However, that
  lemma states that no special fan has all these features.  Hence
  $(V,E,F,S)$ does not exist.
\end{proof}

\begin{lemma}[main estimate -- first form]\guid{JEJTVGB}\rating{ZZ}
\label{lemma:empty-d}
\begin{displaymath}
\tau(V,E,F) \ge d (r,s)
\end{displaymath}
for every special fan $(V,E,F,S)$, where $(r,s)$ are the parameters of
$(V,E,F,S)$.
\end{lemma}

\begin{proof} 
The function $\p\mapsto \tau(V(\p),E(\p),F(\p),S(\p,I))$ is
continuous on the space of fan data of shape $(k_{min},I)$

Assume for a contradiction that the conclusion is false.  Then
$k_{min}>0$, because it is the cardinality of some nonempty set $V$.
There exists a sequence of fan data $\p_i$ of varying shapes
$(k_{min},I_i)$ such that $\tau(V(\p_i),E(\p_i),F(\p_i))-d(r,s)$ tends
to $\tau_{min}$.  By passing to a subsequence, we may assume without
loss of generality that $I_i = I$ is independent of $i$.  By passing
again to a subsequence in the compact metric space of fan data of
shape $(k_{min},I)$, the sequence $\p_i$ converges to some fan datum
$(\p,I)$.  It follows that $\tau_{min} = \tau(V,E,F)-d(r,s)$, where
$(V,E,F,S)=(V(\p),E(\p),F(\p),S(\p,I))$.  This is a minimal fan.
However, the set of minimal fans is empty.
\end{proof}

 {\it Praise to
the emptiness that blanks out
existence.} %-- Rumi % The essential Rumi page 21.
\indy{Index}{Rumi}%


\section{Chapter Summary}

This chapter contains the main technical estimate of the book.  It
has been necessary to introduce substantial technical scaffolding to
prove the results.  Now that the estimate has been established, we
can eliminate the scaffolding and present the estimate in a form that
will be useful in the next chapter.

The next chapter will complete the proof of the Kepler conjecture.
The following definitions are the only ones needed in the next chapter
for that proof: convex local fan $(V,E,F)$, special fan $(V,E,F,S)$,
standard fan $(V,E_{std})$, the parameters $(r,s)$ of a special fan,
$d(r,s)$, $\tau(V,E,F)$, and perimeter.  Here $(V,E)$ is a fan, $F$ is
a face of the hypermap of $(V,E)$, and $S$ is a subset of $E$.  The
standard fan $(V,E_{std})$ associates a particular set of edges
$E_{std}$ with a packing $V$.  Also, $(r,s)$ is an ordered pair of
natural numbers, and $d(r,s)$ is given as a table of real numbers.
The function $\tau$ measures the looseness of a packing $V$ around a
face $F$.  The perimeter is a real number associated with a convex
local fan $(V,E,F)$.

Only two lemmas from this chapter are needed in the next chapter:
perimeter majorization (Lemma~\ref{lemma:convex-hyp}) and the main
estimate (Lemma~\ref{lemma:empty-d}).  Perimeter majorization gives
the upper bound of $2\pi$ for the perimeter of a geodesically convex
spherical polygon.  The main esimate is $\tau(V,E,F)\ge d(r,s)$ for
special fans $(V,E,F,S)$ with parameters $(r,s)$.
