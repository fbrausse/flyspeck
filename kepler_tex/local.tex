%
% total ratings 16400 as of 5/18/2010.

\chapter{Local Fan}\label{sec:local}


\begin{summary}
  The difficult technical estimates that we need for the proof of the
  Kepler conjecture are found in this chapter.  The standard form
of the main estimate
  (Lemma~\ref{lemma:empty-d}) takes the form
\begin{equation}\label{eqn:main}
\tau(V,E,F) \ge d(k),\quad k= \card(V)\in \{3,4,5,6\}.
\end{equation}
Here $(V,E)$ is a fan satisfying various technical conditions, and $F$
is a face of the hypermap of $(V,E)$.  The function $d$ is defined by
a table of real numbers.  Heuristically, the real-valued
function $\tau$ measures the looseness of a packing.  Large values of
$\tau$ indicate that the points of $V$ are loosely arranged around the
face and small values of $\tau$ indicate a tight packing.  The main
estimate gives limits to the tightness of a packing, with the eventual
aim of showing that no packing can have density greater than the FCC
packing.  

This chapter also proves the well-know result that the perimeter of a
geodesically convex spherical polygon is never greater than $2\pi$,
the length of a great circle.
\end{summary}


\section{Localization}

The \newterm{localization} of a fan along
a face discards everything but the part of the fan near the face.  
The localization is used to focus attention on a single face
in a fan.  
We also introduce a notion of convexity that is suitable for local fans.

\subsection{basics}

A local fan is to a fan what a polygon is to a plane graph.  


\begin{definition}[local fan]\guid{FTNGOGF} \label{def:convex-local}
\formaldef{local fan}{local\_fan}
\formaldef{nonreflexive local fan}{convex\_local\_fan}
A triple $(V,E,F)$ is a \newterm{local fan} if the following conditions hold:
\begin{enumerate} 
\item \case{fan} $(V,E)$ is a fan.
\item \case{face} $F$ is a face of $H = \op{hyp}(V,E)$.
\item \case{dihedral} $H$ is isomorphic to $\op{Dih}_{2k}$, where $k =
\card(F)$.
\end{enumerate}
A local fan $(V,E,F)$ is said to be \newterm{nonreflexive} if the following
additional conditions hold:
\begin{enumerate}
\setcounter{enumi}{3}
\item %\setcounter{enumi}{4} 
\case{angle} $\op{azim}(x)\le \pi$ for all darts $x\in F$.
\item \case{wedge} $V\subset \bWdart(x)$ for all $x\in F$.
% \item If $\{\v,\w\}\in E$, then $\{\orz,\v,\w\}$ is not
%   collinear. %% part of def of fan.
\end{enumerate}
\end{definition}
\indy{Index}{fan!local}%

In the proof of the Kepler conjecture in this chapter and the next,
all local fans are nonreflexive.  Local fans (that are reflexive) appear in
applications to other packing problems in Chapter~\ref{sec:further}.

\begin{remark}[visualization]\guid{PNCVUMY}
  The intersection of $X(V,E)$ with the unit sphere is a spherical
  polygon, gives a visual representation of a local fan $(V,E,F)$.
  The choice of $F$ distinguishes the interior of the polygon from its
  exterior.  If the local fan is nonreflexive, then the interior of the
  polygon is geodesically convex.
\end{remark}


\begin{lemma}[]\guid{WRGCVDR}\rz{0}%
\oldrating{300}
\formalauthor{Nguyen Quang Truong}
For any nonreflexive local fan $(V,E,F)$, there is a bijection from $F$ onto $V$
given by
\[ 
(\v,\w) \mapsto \v.
\] 
Moreover, write $\v\mapsto(\v,\rho\v)$ for the inverse map. 
Then $\rho:V\to V$ is a cyclic permutation.
\end{lemma}
\indy{Index}{cyclic permutation}%

\begin{proof} The map from a face to the set of nodes is a bijection
  for the dihedral hypermap $\op{Dih}_{2k}$. It is a also bijection
  for a fan isomorphic to $\op{Dih}_{2k}$.

For all $(\v,\rho \v)\in F$,
\[ 
f(\v,\rho \v) = (\rho\v,\rho^2\v),
\] 
so that the order of $\rho$ on $V$ is the order of $f$ on $F$, which
is $k=\card(F)$.  Thus, $\rho$ is a cyclic permutation of $V$ of order
$k=\card(V)$.
\end{proof}

\begin{definition}[$\rho$,~$\nd$]\guid{MFMPCVM} 
\formaldef{$\rho$}{rho\_node}
\formaldef{$\nd$}{FST}
For any nonreflexive local fan $(V,E,F)$, write
$\nd:F\to V$ and $\rho:V\to V$ for the bijections of the preceding
lemma.
\end{definition}
\indy{Notation}{ZZrho@$\rho$}%
\indy{Notation}{node@$\nd$}%

\begin{definition}[interior angle,~$\angle$,~$\bWdart$]\guid{PJRIMCV}
\formaldef{interior angle}{interior\_angle}
\formaldef{$\bWdart$}{cw\_node\_fan}
\formaldef{$\Wdart$}{w\_node\_fan}
For any nonreflexive local fan $(V,E,F)$,
write
\[ 
\angle(\v) = \op{azim}((\v,\rho\v)),
\] 
for all $\v\in V$.  This is the \newterm{interior} angle of the
nonreflexive local fan at $\v$.  Also, write
\[ 
  \Wdart(F,\v) = \Wdart((\v,\rho \v)),\quad 
\bWdart(F,\v) =\bWdart((\v,\rho \v)).
\] 
\indy{Notation}{1@$\angle$}%
\indy{Notation}{Wdart@$\Wdart(F,\v)$}%
\indy{Notation}{Wdart@$\bWdart(F,\v)$}%
%\indy{Notation}{node@$\nd(x)$}%
\end{definition}


\begin{definition}[localization]\guid{BIFQATK}
\formaldef{localization}{localization}
 Let $(V,E)$ be a fan and let $F$ be
a face of $\op{hyp}(V,E)$.  Let
\begin{align*}
V' &= \{\v\in V \mid \exists~\w\in V.~(\v,\w)\in F\}.\\
E' &= \{\{\v,\w\} \in E\mid (\v,\w)\in F\}.
\end{align*}
The triple $(V',E',F)$ is called the \newterm{localization} of $(V,E)$ along $F$.
\end{definition}
\indy{Index}{localization}%


\begin{lemma}[localization]\guid{LVDUCXU}\rz{0}%
\oldrating{450}
\formalauthor{Nguyen Quang Troung}
\label{lemma:localization}
Let $(V,E)$ be any fan and let $F$ be a face of its hypermap that is
simple and has cardinality at least three.  Then the localization
$(V',E',F)$ is a local fan.  Moreover, the angle $\op{azim}(x)$ and
the wedges $\bWdart(x)$ and $\Wdart(x)$ do not depend on whether they
computed relative to $\op{hyp}(V,E)$ or to $\op{hyp}(V',E')$ for all
$x\in F$.
%That is, 
%\[ 
%\begin{align}
%\op{azim}(V,E,x) &=\op{azim}(V',E',x)\\
%\Wdart(V,E,x) &=\Wdart(V',E',x)\\
%\end{align}
%\] 
\end{lemma}



\begin{proof}
The proof that $(V',E')$ is a fan consists of various simple
verifications based on the techniques of
Remark~\ref{remark:fan-verify}.  The details are left to the reader.

The dart set $D'$ of $\op{hyp}(V',E')$ is naturally identified with
the disjoint union $F\coprod F'$, where $F = \{(\v,\rho\v) \mid \v\in
V\}$ and $F'=\{(\v,\rho^{-1}\v) \mid \v\in V\}$.  Under this
identification, $F$ is a face of $\op{hyp}(V',E')$.  The face, node,
and edge permutations have orders $k$, $2$, and $2$, respectively.  By
Lemma~\ref{lemma:dih-iso}, this bijection extends to an isomorphism of
hypermaps $\op{Dih}_{2k}$ onto $\op{hyp}(V',E')$.

The proof that the $\op{azim}(x)$ and $\Wdart(x)$ do not depend on the
choice of fan is a consequence of their definitions:
\begin{align*}
\op{azim}(x) &= \op{azim}(\orz,\v,\w,\sigma(\v,\w)),\textand \\
\Wdart(x) &= \Wdart(\orz,\v,\w,\sigma(\v,\w)).\\
\end{align*}
where $x = (\v,\w)$.  It is enough to check that $\sigma(\v,\w)\in
E'(\v)$.  But $\{\sigma(\v,\w),\v\}\in F$, so this is indeed the case.
% In particular $\op{azim}(x)\le\pi$ follow by the assumptions of the
% lemma.
\end{proof}



\subsection{geometric type}\label{sec:types}

\begin{definition}[generic,~lunar,~circular]\guid{RTPRRJS}
\formaldef{generic}{is\_generic\_clf}
\formaldef{lunar}{is\_lunar\_clf}
\formaldef{circular}{is\_circular\_clf}
A nonreflexive local fan $(V,E,F)$ is \newterm{generic}  if for every $\{\v,\w\}\in E$
and every $\u\in V$, 
\[ 
C\{\v,\w\}\cap C^0_-\{\u\} = \emptyset.
\] 
A nonreflexive local fan is \newterm{circular}  if there exists $\u\in V$ and
$\{\v,\w\}\in E$ such that
\[ 
C^0\{\v,\w\}\cap C^0_-\{\u\}\ne \emptyset.
\] 
A nonreflexive local fan is \newterm{lunar} with pole $\{\v,\w\} \subset V$ if it is not
circular, if $\v\ne\w$, and if $\{\v,\w\}$ is a parallel set.
\end{definition}
\indy{Index}{generic}%
\indy{Index}{lunar}%
\indy{Index}{circular}%


\begin{lemma}[trichotomy]\guid{CIZMRRH}\rz{0}%
\oldrating{200} %
\formalauthor{Nguyen Quang Truong}
Every nonreflexive local fan is either
generic, lunar, or circular.  Moreover, these three properties are
mutually exclusive.
\end{lemma}
\indy{Index}{fan!local}%
\indy{Index}{nonreflexive local fan}%
\indy{Index}{generic}%
\indy{Index}{lunar}%
\indy{Index}{circular}%

\begin{proof} If $(V,E,F)$ is not generic,  select some $\{\v,\w\}\in E$
and some $\u\in V$ such that
\begin{equation}\label{eqn:nongeneric}
C\{\v,\w\}\cap C^0_-\{\u\} \ne \emptyset.
\end{equation}
Now $C\{\v,\w\} = C^0\{\v,\w\} \cup C\{\v\}\cup C\{\w\}$.  If, for
some such triple $(\u,\v,\w)$, the
intersection~(\ref{eqn:nongeneric}) meets $C^0\{\v,\w\}$, then the
nonreflexive local fan is circular.  Otherwise, the nonreflexive local fan is lunar.
\end{proof}

\begin{definition}[flat]\guid{YPSTLXA}
\formaldef{flat}{is\_flat}
 Let $(V,E,F)$ be a nonreflexive local fan.
If $\angle(\v)=\pi$, then $\v$ is \newterm{flat}.
\end{definition}
\indy{Index}{flat (node of a fan)}


\begin{lemma}[]\guid{LDURDPN}\rz{0}%
\oldrating{100}  \label{lemma:coplanar}%
\formalauthor{Nguyen Quang Truong}%
Assume that $\{\orz,\u,\w\}$ and $\{\orz,\u,\v\}$ are not collinear sets.
Then $\op{azim}(\orz,\u,\v,\w)=\pi$ if and only if
there exists a plane $A$ such that $\{\orz,\u,\v,\w\}\subset A$
and such that the line $\op{aff}\{\orz,\u\}$ separates $\v$ from
$\w$ in $A$.
\end{lemma}

\begin{proof} The given azimuth angle is $\pi$ if and only if
$\dih(\{\orz,\u\},\{\v,\w\})=\pi$.  This holds exactly when $\{\orz,\u,\v,\w\}$ is
coplanar, and the line $\op{aff}\{\orz,\u\}$ separates $\v$ from $\w$
in $A$.
\end{proof}

\begin{lemma}[]\guid{KOMWBWC}\rating{0}\label{lemma:kom}
\formalauthor{Nguyen Quang Truong}
\oldrating{400}
Let $(V,E,F)$ be a nonreflexive local fan.  Let $k=\card(F)$.  Assume that for
some $0<r\le k-1$ and some $\v\in V$, the set $U=\{\v,\rho
\v,\ldots,\rho^r \v\}$ is contained in a plane $A$ passing through
$\orz$.  Let $\e$ be the unit normal to $A$ in the direction
$\v\times \rho \v$.  Then the set $U$ is cyclic with respect to
$(\orz,\e)$, and the azimuth cycle $\sigma$ on $U$ is
\[ 
  \sigma \u = 
\begin{cases} 
\rho \u, & \u\ne \rho^r\v,\\ \v, & \u = \rho^r\v.
\end{cases}
\] 
Furthermore, for all $0\le i\le r-1$,
\[ 
(\rho^i \v\times\rho^{i+1}\v)\cdot \e > 0.
\] 
\end{lemma}

\begin{proof} 
We write $\v_i = \rho^i \v$ for $i=0,\ldots,r$.

\claim{We claim that $(\v_i \times \v_{i+1})\cdot \e > 0$ for all $i\le r-1$. }
Indeed, the base case $(\v_0\times \v_1)\cdot \e > 0$ of an induction
argument holds by assumption.  Assume for a contradiction that the
inequality holds for $i$, but not for $i+1$.  Then
\[ 
  \op{aff}^0_+(\{\orz,\v_{i+1}\},\v_i) 
= \op{aff}^0_+(\{\orz,\v_{i+1}\},\v_{i+2}).
\]  
This forces $C^0\{\v_i,\v_{i+1}\}$ to meet $C^0\{\v_{i+1},\v_{i+2}\}$,
which is contrary to the definition of a fan.  Thus, the claim holds.

The fact that $U$ is cyclic follows trivially from the fact that $U$
is contained in a plane $A$ through $\orz$ and that $\e$ is orthogonal
to $A$.

\claim{For all $0\le i \le r-1$,  $\sigma \v_i = \v_{i+1}$.}
Otherwise, there is some 
\[ 
  \u \in (U\setminus \{\v_i,\v_{i+1}\}) 
\cap W^0(\orz,\e,\v_i,\v_{i+1}) \cap A.
%  ~~\subset~~ 
\] 
However, by the claim, this intersection is a subset of $C^0\{\v_i,\v_{i+1}\}$, and
$\u\in C^0\{\v_i,\v_{i+1}\}$ is contrary to the property
\case{intersection} of fans.  The result follows.
% The membership of $\u$ in the rightmost term is contrary to the
% definition of fan.  Let $\e_3$ be the unit vector in the direction
% $\v\times \u$.  Let $\e_1$ be the unit vector in the direction $\u$.
% Let $\e_2 = \e_3 \times \e_1$.  The coordinates of $\u,\v,\w$ with
% respect to the frame $(\e_1,\e_2,\e_3)$ take the form
%\[ 
%\begin{align}
%\v &= \wild  \e_1 + a \e_2,  &\quad & a &< 0\\
%\u &= a' \e_1, &\quad & a' &>0\\
%\w&= \wild  \e_1 + a'' \e_2, &\quad & a'' &>0\\
%\end{align}
%\] 
%From this representation it is clear that $\v\times \u$ points in the
% same direction as $\u\times \w$.  The set $\{\v,\u,\w\}$ is clearly
% cyclic and the counterclockwise cycle $\sigma$ in the
% $\{\e_1,\e_2\}$ plane takes $\v$ to $\u$ and $\u$ to $\w$.
\end{proof}

\begin{lemma}[]\guid{OZQVSFF}\rating{0} \label{lemma:A}  
\formalauthor{Nguyen Quang Truong}
\oldrating{600}
Let $(V,E,F)$ be a nonreflexive local fan and let
  $\u,\v,\w\in V$ satisfy
\begin{enumerate}\wasitemize 
\item $\{\orz,\u,\v,\w\}$ is contained in a plane $A$. \vspace{3pt}
\item $\u,\w\not\in\op{aff}\{\orz,\v\}$. \vspace{3pt}
\item $\op{aff}^0_+(\{\orz,\v\},\u) \ne \op{aff}^0_+(\{\orz,\v\},\w)$.
\end{enumerate}\wasitemize 
Then $\v$ is flat.  Moreover, $\rho \v,\rho^{-1} \v\in A$.
\end{lemma}

\begin{proof} Let $x = (\v,\rho\v)\in F$.  
Order $\u$ and $\w$ so that
\[ 
\op{azim}(\orz,\v,\rho\v,\u) \le \op{azim}(\orz,\v,\rho\v,\w).
\] 
By the definition of nonreflexive local fan, by the conditions $\u,\w\in \bWdart(x)$, 
and by  Lemma~\ref{lemma:coplanar},
%By the assumptions, $\dih(\{\orz,\v\},\{\u,\w\})=\pi$.  Since
%$\u,\w\in \bWdart(x)$, it follows that
%\[ \pi = \dih(\{\orz,\v\},\{\u,\w\}) \le 
%\op{azim}(x) = \angle(\v) \le \pi.\] 
%The first conclusion follows.
\begin{align*}
%\begin{align}
0 &\le \op{azim}(\orz,\v,\rho\v,\u) \\
&= \op{azim}(\orz,\v,\rho\v,\w) - \op{azim}(\orz,\v,\u,\w)\\
&= \op{azim}(\orz,\v,\rho\v,\w)-\pi \\
&\le \op{azim}(\orz,\v,\rho\v,\rho^{-1}\v) - \pi \\
&=\op{azim}(x) - \pi \\
&=\angle(\v)-\pi\\
&\le 0. 
%\end{align}
\end{align*}
Hence, each inequality is equality.  In particular, $\v$ is flat.
In particular, $0 =\op{azim}(\orz,\v,\rho\v,\u)$, so that 
\[ 
\rho\v\in \op{aff}_+(\{\orz,\v\},\u) \subset A.
\] 
Similarly,
\[ 
\rho^{-1}\v\in \op{aff}_+(\{\orz,\v\},\w) \subset A.
\] 
\end{proof}

If Lemma~\ref{lemma:A} can be applied once to a set of vectors, then
it can often be applied repeatedly along a chain of vectors.  For
example, the conclusion of the lemma implies that $\rho^{-1} \v \in
A$.  In fact, by the definition of fan,
\[ 
  \rho^{-1}\v \in A \setminus \op{aff}\{\orz,\v\} 
= \op{aff}^0_+(\{\orz,\v\},\u) \cup \op{aff}^0_+(\{\orz,\v\},\w).
\] 
Suppose that $\rho^{-1} \v$ lies in the second term of the union.  If
$\w\ne \rho^{-1}\v$, then the assumptions of the lemma are met for
$\{\w,\rho^{-1} \v,\v\}$, giving the conclusions that $\rho^{-1} \v$
is flat and that $\rho^{-2}\v\in A$.  Repeating the argument
on a new set of vectors, we obtain a chain
\[ 
\pi=\angle(\v) = \angle(\rho^{-1} \v) = \cdots,
\] 
with $\v,\rho^{-1}\v,\ldots\in A$.  Another chain $\v,\rho \v,\ldots$
of vectors can be constructed in the other direction.  This process of
chaining gives the following lemma.

\begin{lemma}[circular geometry]\guid{KCHMAMG}\rating{0}
\label{lemma:circular}
\formalauthor{Nguyen Quang Truong}
\oldrating{300}
Let $(V,E,F)$ be a circular fan. Then
\begin{enumerate}\wasitemize 
\item $\v$ is flat for all $\v\in V$.
\item The set $V$ lies in a plane $A$ through $\orz$.
\item For some choice of unit vector $\e$ orthogonal to $A$, the set
$V$ is cyclic with respect to $(\orz,\e)$, and the azimuth cycle on
$V$ coincides with $\rho:V\to V$.  
\item 
$
\op{azim}(\orz,\e,\v,\rho\v) = \dih(\{\orz,\e\},\{\v,\rho\v\})
=\op{arc}_V(\orz,\{\v,\rho \v\}) <\pi
$.
\end{enumerate}\wasitemize 
\end{lemma}

\begin{proof} Let $\v, \u\in V$ be such that $C^0\{\u,\rho \u\}$ meets
$C^0_-\{\v\}$.  Apply Lemma~\ref{lemma:A} to $\{\u,\v,\rho \u\}$ to
conclude that $\v$ is flat, and some plane $A$ contains
$\{\orz,\u,\rho\u,\v,\rho \v,\rho^{-1} \v\}$.  If
%\[ 
%\{\orz,\u,\rho \u,\v,\rho \v,\rho^{-1} \v\} \subset V\cap A,
%\] 
%and 
$\w\in V\cap A$, then there exists $\w_1,\w_2\in (V\cap
A)\setminus\{\w\}$ for which the assumptions of Lemma~\ref{lemma:A}
hold for $\w_1,\w,\w_2$.  
Then $\w$ is flat, and $\rho \w \in V\cap A$.  The set $V\cap
A$ is therefore preserved by $\rho$.  By observing that $V$ is the
only nonempty subset of $V$ that is preserved by $\rho$, it follows
that $V\subset A$ and that $\w$ is flat for all $\w\in V$.


By Lemma~\ref{lemma:coplanar}, $V$ is cyclic with respect to a unit
vector $\e$ orthogonal to $A$.  The azimuth cycle on $V$ is $\v
\mapsto \rho \v$.

We turn to the final conclusion.  By the final conclusion of
Lemma~\ref{lemma:kom} and Lemma~\ref{lemma:sim}, the azimuth angle is
less than $\pi$.  Under this constraint, the azimuth angle equals the
dihedral angle by Lemma~\ref{lemma:dih-azim}.  By definition, the
dihedral angle is the angle $\arc_V(\orz,\wild )$ of an orthogonal
projection of $\{\v,\rho\v\}$ to a plane with normal $\e$.  But
$\{\v,\rho\v\}$ is already a subset of the plane $A$, so that the
projection is the identity map, and the dihedral angle is
$\arc_V(\orz,\{\v,\rho\v\})$.
%
%\[ 
%%  \op{azim}(\orz,\e,\v,\rho\v) = \dih(\{\orz,\e\},\{\v,\rho\v\}) =
%%  \op{arc}_V(\orz,\{\v,\rho\v\}),
%\] 
%because the azimuth angle is less than $\pi$ and
%the dihedral angle is defined as the angle obtained by orthogonal projection
%to the plane $A$.  Since the points $\v,\rho\in A$, orthogonal projection has no effect.
\end{proof}

\begin{lemma}[lunar geometry]\guid{HKIRPEP}\rating{0}
\formalauthor{Nguyen Quang Truong}
\oldrating{300}\label{lemma:lunar}
Let $(V,E,F)$ be a lunar fan with pole $\{\v,\w\}\subset V$.  
%Assume that $\rho^r \v = \w$ for some $0< r < k$.
Then
\begin{enumerate}\wasitemize 
\item $\u$ is flat for all $\u\in V\setminus \{\v,\w\}$. \vspace{3pt}
\item $0< \angle(\v) = \angle(\w)\le \pi$. \vspace{3pt}
\item $V\cap \op{aff}_+(\{\orz,\v\},\rho \v) = \{\v,\rho \v,\ldots,
\w\}$. \vspace{3pt}
\item $V \cap \op{aff}_+(\{\orz,\v\},\rho^{-1} \v) = \{\w,\rho
\w,\ldots \v\}$. \vspace{3pt}
\end{enumerate}\wasitemize 
\end{lemma}

\begin{proof} Set $V_1 = \{\v,\rho \v,\ldots,\w\}$ and $V_2 =
\{\w,\rho \w,\ldots,\v\}$.  Let $\u\in V\setminus\{\v,\w\}$ be
arbitrary.  Apply Lemma~\ref{lemma:A} to the set $\{\v,\u,\w\}$ to
find that $\u$ is flat and that $\{\orz,\u,\rho \u,\rho^{-1} \u\}$
belongs to a plane $A(\u)$.  Now $A(\u)$ and $A(\rho \u)$ are both
the unique plane containing $\{\orz,\u,\rho \u\}$; hence, $A(\u) =
A(\rho \u)$ when $\rho \u\not\in \{\v,\w\}$.  By induction, there
are planes $A_1, A_2$ such that $V_i\subset A_i$.  There is an
azimuth cycle $\sigma_i$ on $V_i$ such that $\sigma_i \u = \rho \u$,
when $\u \in A_i\setminus \{\v,\w\}$.

The angles $\angle(\v)$ and $\angle(\w)$ are both equal to the
dihedral angle between the half-planes
$\op{aff}_+(\{\orz,\v\},\rho^{\pm}\v)$.  In particular,
$0<\angle(\v)=\angle(\w)\le\pi$.
\end{proof}




\begin{lemma}[monotonicity]\guid{EGHNAVX}\rating{650} 
\label{lemma:monotone}
Let $(V,E,F)$ be a nonreflexive local fan and let $k$ be the cardinality
of $F$.  Fix $\v_0\in V$.  Assume that
$\{\orz,\v_0,\u\}$ is not collinear for any $\u\in
V\setminus\{\v_0\}$.  For all $i$, set $\v_i = \rho^i \v_0$ and
$\beta(i) = \op{azim}(\orz,\v_0,\v_1,\v_i)$.  Then
\[ 0=\beta(1)\le \beta(2)\le \cdots\le
\beta(k-1)\le\pi.\] 
Moreover, if $\beta(i)=0$ for some $1<i \le k-1$, then
\[ 
\angle(\v_1) = \cdots = \angle(\v_{i-1}) = \pi,
\] 
and $\{\v_1,\ldots,\v_i\} \subset \op{aff}^0_+(\{\orz,\v_0\},\v_1)$.
Finally, if $\beta(i)=\beta(k-1)$ for some $1\le i<k-1$, then 
\[ 
\angle(\v_{i+1}) = \cdots = \angle(\v_{k-1}) = \pi,
\textand  \{\v_i,\ldots,\v_{k-1}\} \subset
\op{aff}^0_+(\{\orz,\v_0\},\v_{k-1}).
\]
\end{lemma}

\begin{proof}  
  With respect to a frame, the points $\v_j$ can be represented in
  spherical coordinates $(r_j,\theta_j,\phi_j)$.  In an appropriate
  frame, $\phi_0=0$ and $\theta_j=\beta(j)$ for all $j$.  From
  $\v_j\in \bWdart(F,\v_0)$ and $\angle(\v_0)\le\pi$, it follows that
  $0\le\theta_j\le\theta_{k-1}\le\pi$ when $0\le j\le k-1$.

One may assume the induction hypothesis that $0\le \beta(1)\le\cdots\le
\beta(i)$.  The condition
\[ 
\v_0\in \bWdart(F,\v_i)
\] 
implies that
\[ 
  0 \le \op{azim}(\orz,\v_i,\v_{i+1},\v_0)
\le \op{azim}(\orz,\v_i,\v_{i+1},\v_{i-1})\le\pi.
\] 
By Lemma~\ref{lemma:sim}, the resulting inequality
\[ 
\sin(\op{azim}(\orz,\v_i,\v_{i+1},\v_0))\ge 0
\] 
reduces to a triple-product:
\[ 
(\v_0 \times \v_i)\cdot \v_{i+1}\ge 0.
\] 
In spherical coordinates, this inequality becomes
\[ 
r_0r_ir_{i+1}\sin\phi_i\sin\phi_{i+1}\sin(\theta_{i+1}-\theta_i)\ge0.
\] 
Under the noncollinearity assumption, $\sin\phi_i\sin\phi_{i+1}\ne0$
(when $0< i < k-1$).  Once we deal with the degenerate
case $\theta_{i+1}=0,\theta_i=\pi$, these inequalities give
$\theta_i\le \theta_{i+1}$, and the result follows by induction.

Turn to the degenerate case $\theta_{i+1}=0,\theta_i=\pi$.  In this case,
the set $\{\orz,\v_0,\v_i,\v_{i+1}\}$ is coplanar.  Let $C^0_+=C^0_+(\v_i,\v_{i+1})$. 
The values the angles $\theta_i$ and $\theta_{i+1}$
imply that $C^0_+$ meets
the line $\op{aff}(\orz,\v_0)$. In particular, $\epsilon \v_0\in C^0_+$
for some choice of sign $\epsilon\in\{\pm 1\}$.
 The definition of a fan implies
$\v_0\not\in C^0_+$. Hence $-\v_0\in C^0_+$.
By the definition of a circular fan, $(V,E,F)$ is circular.
The lemma follows in this case from the explicit description of circular
fans in Lemma~\ref{lemma:circular}.  This completes the proof of the first
statement of the lemma.

Assume that $\beta(i)=\theta_i=0$ for some $1<i\le k-1$.  Then by the first
conclusion, $\theta(j)=0$ for $0\le j\le i$.  That is, 
$\v_1,\ldots,\v_i$ all lie in the half-plane
$\op{aff}^0_+(\{\orz,\v_0\},\v_1)$.  In particular, they are coplanar.
A chaining argument based on Lemma~\ref{lemma:A} gives the result.

The final conclusion follows by a similar chaining argument.
\end{proof}


\section{Modification}

\subsection{deformation}\label{sec:deformation}


This subsection develop a theory of deformations of a nonreflexive local fan
$(V,E,F)$, including sufficient conditions for the deformation of a
nonreflexive local fan to remain a nonreflexive local fan.


\begin{definition}[deformation]\guid{YWNHMBP}
\formaldef{deformation}{is\_deformation\_clf}
A \newterm{deformation} 
of a nonreflexive local fan $(V,E,F)$ over an interval
$I\subset\ring{R}$ is a function $\varphi:V\times I
\to\ring{R}^3$ that is continuous $\varphi(v,\wild):I\to\ring{R}^3$
for each $v\in V$.
\end{definition}
\indy{Index}{deformation}%
\indy{Index}{fan!local}%

\begin{notation}
  Beware of the notational distinction between the zenith angle $\phi$
  and the deformation $\varphi$.  When a deformation $\varphi$ is
  given, write $\v(t)$ as an abbreviation of $\varphi(\v,t)$ for
  $t\in I$.  Also, set
\begin{align*}
V(t)&=\{\v(t) \mid \v\in V\},\\
E(t)&=\{\{\v(t),\w(t)\}\mid \{\v,\w\}\in E\},\\
F(t)&= \{(\v(t),\w(t)) \mid  (\v,\w)\in F\}.
\end{align*}
\indy{Notation}{vt@$\v(t)$ (deformation of $\v$)}
\end{notation}

A deformation does not require $(V(t),E(t),F(t))$ to be a nonreflexive local fan
for all $t\in I$, although this is often  the case. The
permutation $\rho:V\to V$ gives $\varphi(\rho \v,t)\in V(t)$ for
every $\v\in V$.



Next consider a deformation of a nonreflexive local fan.  The following lemma
gives a list of conditions that ensure that the deformed fan remains
nonreflexive local throughout the deformation.

%% XX need to generalize so that flat vertices can become unflat.

\begin{lemma}[]\guid{XRECQNS}\rating{700}\label{lemma:fan-open}
Let $(\varphi,V,I)$ be a deformation of a nonreflexive local fan $(V,E,F)$ over an
interval $I$.  Assume that $0\in I$ and that $\varphi(\v,0)=\v$ for
all $\v\in V$.  For all $\v\in V$ assume that if
$\v$ is flat, then $\v(t)$ is flat for all sufficiently small $t$.
%$\{\orz,\rho^{-1}\v,\v,\rho\v\}$ is coplanar, then
%\[ 
%\{\orz,\varphi(\rho^{-1}\v,t),\varphi(\v,t),\varphi(\rho\v,t)\}
%\]  is coplanar for all sufficiently small $t$.
%If the property \case{wedge} is maintained for sufficiently small $t$,
Then exists $\epsilon>0$ such that $(V(t),E(t),F(t))$ is a nonreflexive local fan
for all $t\in I\cap \leftopen-\epsilon,\epsilon\rightopen$.  
%Moreover,
%if $(V,E,F)$ is generic, then the property \case{wedge} is in fact
%maintained for sufficiently small $t$.  
Moreover,  if $(V,E,F)$ is
generic, then the deformed fan is also generic for sufficiently small
$t$.
\end{lemma}

\begin{proof} We examine in turn each of the defining properties of a generic fan.

\case{cardinality}: The set $V(t)$ is the image of $V$ and is
therefore finite and nonempty.

\case{origin} Since $\varphi$ is continuous and
$\orz\not\in V$, it follows that $\orz\not\in V(t)$ for sufficiently
small $t$.

\case{nonparallel}: If $\v,\w$ are nonparallel, then $\v(t)$ and
$\w(t)$ are nonparallel for sufficiently small $t$.

\case{intersection}: If $\ee \cap \ee'=\emptyset$, then $C(\ee)\cap
S^2$ has a positive distance from $C(\ee')\cap S^2$.  Hence, for
sufficiently small times, the deformation of these sets remain
disjoint.  If $\ee=\{\u,\v\}$ and $\ee'=\{\v,\w\}$, where $\u\ne\w$,
then again the deformations of $C(\ee)\cap C(\ee')$ is
$C(\{\v(t)\})$ for sufficiently small $t$.  The other cases are
similar.

\case{face},~\case{dihedral}: The azimuth cycle on $E(\v(t))$
is preserved; hence, the combinatorial properties of the hypermap do
not change when $t$ is sufficiently small.

\case{angle}: If $\op{azim}(x)<\pi$, then the inequality remains
strict for sufficiently small $t$.  If $\op{azim}(x)=\pi$, then the
flatness assumption of the lemma forces the equality to be
preserved for all sufficiently small $t$.

\case{wedge}: The property $\u\in \Wdart(x)$ is an open condition.
It holds for sufficiently small $t$. Consider the case $\u\in
\bWdart(x)\setminus \Wdart(x)$, where $x= (\v_0,\rho\v_0)$.  Write
$\v_i = \rho^i\v_0$ and pick $r\le k-1$ such that $\u = \v_r$.  The
wedge property holds trivially when $r\in\{0,1,k-1\}$. Assume that
$r\not\in\{0,1,k-1\}$.  We finish the argument in cases, according to 
the type of the nonreflexive local fan  $(V,E,F)$.

If the fan $(V,E,F)$ is circular, then some plane $A$ through
the origin  contains $V$.  By the flatness condition, the
deformation maintains a coplanarity condition $V(t)\subset A(t)$.  The
sets $\bWdart(x(t))$ are half-spaces bounded by $A(t)$.  Thus,
$V(t)\subset A(t)\subset \bWdart(x(t))$, as desired.

If the fan $(V,E,F)$ is lunar with pole $\{v,w\}$, the argument is
similar, but there are two half-planes $A_1(t)$ and $A_2(t)$ such that
$V(t)\subset A_1(t)\cup A_2(t)$.  Arguing as in the previous case,
each wedge $\bWdart(x(t))$ contains $A_1(t)\cup A_2(t))$.  The result
follows.

If the fan $(V,E,F)$ is generic, then
by the genericity assumption $\v_r$ and
$\v_0$ are nonparallel.  In the notation of
Lemma~\ref{lemma:monotone}, $\beta(r) = 0$ or $\beta(r) =
\beta(k-1)$.  This proof treats the case $\beta(r)=0$. (The case
$\beta(r)=\beta(k-1)$ is similar.)  By Lemma~\ref{lemma:monotone},
\[ 0=\beta(1)=\beta(2)=\cdots=\beta(r),\] 
and 
\[ 
\{\orz,\v_0,\v_1,\ldots,\v_r\} \subset \op{aff}^0_+(\{\orz,\v_0\},\v_1).
\] 
In particular, the set is coplanar.  By the flatness assumption of
the lemma, $\{\orz,\v_0(t),\v_1(t),\ldots,\v_r(t)\}$ is coplanar for
sufficiently small $t$.  In fact, the condition $\v_r(t)\not\in
\op{aff}_+(\{\orz,\v_0(t)\})$ is an open condition, so that
$\v_r(t)\in \op{aff}^0_+(\{\orz,\v_0(t)\},\v_1(t))$ for sufficiently
small $t$.  This half-plane is the bounding half-plane of
$\bWdart(F,\v_0(t))$.  Hence, $\u = \v_r\in \bWdart(F,\v_0(t))$ for
sufficiently small $t$.

\case{generic}: Genericity is stated as open conditions $\v\not\in
C\{\u,\w\}$.  These conditions continue to hold for sufficiently small
$t$.
\end{proof}


%%%%%%%%%%%


\subsection{internal blades}


\begin{lemma}[]\guid{PGSQVBL}\oldrating{80}\rating{0}
\formalauthor{Nguyen Quang Truong} 
Let $(V,E,F)$ be a nonreflexive local fan.
If $\v,\w\in V$ are nonparallel, then $C\{\v,\w\} \subset
\bWdart(x)$ for any dart $x\in F$.
\end{lemma}
\indy{Index}{fan!local}%

\begin{proof} This is an elementary consequence of
the cone shape of $\bWdart(x)$,  the condition that $V\subset
\bWdart(x)$, and definitions.
\end{proof}


\begin{lemma}[internal
blades]\guid{YOLCBTG}\rating{700} \label{lemma:internal}
Let $(V,E,F)$ be a nonreflexive local fan.  Let $\v,\w\in V$ be nonparallel.
Suppose that there exists $\v',\w'$ such that
$\angle(\v'),\angle(\w')<\pi$, where $\v,\v',\w,\w'$ are four
distinct elements of $V$ that appear in cyclic order.  Then
$C^0\{\v,\w\}\subset \Wdart(x)$ for all $x\in F$.
% Pick a dart $x=(\v_0,\v_1)\in V$.  Set $\v_j = \rho^j \v_0$.
% Assume that there are four darts $(y_1,y_2,y_3,y_4)$, $y_j =
% x_{j(j)}$, with $0\le j(1) < j(2) < j(3) < i(4)\le k-1$ such that
% $\op{azim}(y_j) < \pi$ for $j=2,4$.  Then
% $C^0\{\v_{i(1)},\v_{i(3)}\} \subset \Wdart(x)$ for all $x\in F$.
\end{lemma}
\indy{Index}{blade!internal}%
\indy{Index}{internal blade}%

To say that a sequence $\v_i$ of elements is in \newterm{cyclic
order} means that $\v_i = \rho^{j (i)}\v_0$ for some increasing
function $j$ with range $\{0,\ldots,k-1\}$.

\begin{proof} Abbreviate $C^0 = C^0\{\v,\w\}$.  The first case to
consider is $\nd(x)=\v$.  For all $\p\in C^0\cap \bWdart(x)$,
\[ 
  0 \le \op{azim}(\orz,\v,\rho \v,\p) 
\le \op{azim}(\orz,\v,\rho \v,\rho^{-1} \v).
\]   
These inequalities are in fact strict.  If, for example $0 =
\op{azim}(\orz,\v,\rho \v,\p)$, then the set $\{\orz,\v,\rho \v,\w\}$
is coplanar.  Repeated application of Lemma~\ref{lemma:A} gives
\[ 
\angle(\v) = \angle(\rho \v) = \cdots = \angle(\rho^{-1} \w) = \pi,
\] 
which is contrary to $\angle(\v') = \pi$.  The strict inequalities
imply $\p\in \Wdart(x)$, as desired.  The case $\nd(x)=\w$ is similar.

Now assume that $\u=\nd(x)\ne \v,\w$.  By Lemma~\ref{lemma:A}, we may
assume that $\{\orz,\u,\v,\w\}$ is not coplanar.  (Otherwise, the
contradiction $\angle(\v')=\pi$ or $\angle(\w')=\pi$ is reached.) Then
\[ 
  \op{aff}\{\orz,\u,\v\}\cap C^0 \subset \op{aff}\{\orz,\u,\v\}
  \cap \op{aff}\{\orz,\v,\w\} \cap C^0 
= \op{aff}\{\orz,\v\} \cap C^0 = \emptyset.
\] 
Thus, $C^0$ is disjoint from $\op{aff}\{\orz,\u,\v\}$ and is also
disjoint from $\op{aff}\{\orz,\u,\w\}$.

We have the following facts:
\[ 
\v,\w\in W(\orz,\u,\v,\w),\quad C^0\{\v,\w\} \subset W(\orz,\u,\v,\w).
\] 
Also,
\begin{align*}
  C^0 &= C^0\cap W(\orz,\u,\v,\w) \\
  &\subset C^0 \cap (W^0(\orz,\u,\v,\w) 
\cup \op{aff}\{\orz,\u,\v\} \cup \op{aff}\{\orz,\u,\w\})\\
  &\subset C^0 \cap W^0(\orz,\u,\v,\w)\\
  &\subset \Wdart(x).
  % \v,\w\in \barW &= \{\p \mid 0 
  %\le \op{azim}(\orz,\u,\v,\p) \le \op{azim}(\orz,\u,\v,\w)\}.\\
  % C^0 &\subset \barW\\
  % \bar W &\subset W(\orz,\u,\v,\w) 
  %\cup \op{aff}\{\orz,\u,\v\} \cup \op{aff}\{\orz,\u,\w\}\\
  % W(\orz,\u,\v,\w) & \subset \Wdart(x),
\end{align*}
%\[ 
%\Wdart(x) = \bWdart(x)
%\setminus (\op{aff}\{\orz,\u,\v\}\cup\op{aff}\{\orz,\u,\w\}).
%\] 
\end{proof}


\subsection{slicing}

This subsection shows that a nonreflexive local fan can be sliced
along a internal blade to divide it into two nonreflexive local fans.

\begin{definition}[slice]\guid{CNAQAAA}
 Let $(V,E,F)$ be a nonreflexive local fan.  Assume that
$\v,\w\in V$ are nonparallel and that $(V,E')=(V,E\cup
\{\{\v,\w\}\})$ is a fan.  Let $F'$ be the face of $\op{hyp}(V,E')$
containing the dart $(\w,\v)$.  Write
\[ (V[\v,\w],E[\v,\w],F[\v,\w])\] 
for the localization of $(V,E')$ along $F'$, where
\begin{align*}
  V[\v,\w] &= \{\v,\rho \v,\rho^2 \v,\ldots,\w\},\\
  E[\v,\w] &= \{\{\v,\rho \v\},\ldots,\{\rho^{-1}\w,\w\},\{\w,\v\}\},\\
  F[\v,\w] &= \{(\v,\rho \v),(\rho \v,\rho^2 \v),
 \ldots,(\rho^{-1}\w,\w),(\w,\v)\}.
\end{align*}
The triple $(V[\v,\w],E[\v,\w],F[\v,\w])$ is called the
\newterm{slice\/} of $(V,E,F)$ along $(\v,\w)$.
\end{definition}
\indy{Index}{fan!local}%
\indy{Index}{slice}
\indy{Notation}{1@$\wild[\v,\w]$ (slicing a fan)}

To allow for more than one nonreflexive local fan $(V,E,F)$, we  extend
the notation, writing $\angle(H,\v)$ for $\angle(\v)$ in the hypermap
$H$.  Similarly, we write $\Wdart(H,\v)$ for $\Wdart(x)$ and so
forth.  \indy{Notation}{azimhv@$\op{azim}(H,\v)$}%
\indy{Notation}{wdart@$\Wdart$}%


\begin{lemma}[slicing]\guid{EJRCFJD}\rating{400}\label{lemma:slice} Let
$(V,E,F)$ be a nonreflexive local fan with hypermap $H$.  Pick $\v,\w\in V$. For
each $\u\in \{\v,\w\}$, assume that $\u$ is not parallel with  any
element of $V\setminus\{\u\}$.  Assume that $C^0\{\v,\w\}\subset
\Wdart(x)$ for all darts $x\in F$.  Then
\begin{enumerate}\wasitemize 
\item $(V[\v,\w],E[\v,\w],F[\v,\w])$ and
$(V[\w,\v],E[\w,\v],F[\w,\v])$ are nonreflexive local fans.
\item Let $H[\v,\w]$ and $H[\w,\v]$ be the hypermaps of these two
  nonreflexive local fans, respectively.  Let $g:V\to\ring{R}$ be any
  function.  Then
\[ 
  \sum_{\v\in V} g(\v)\angle(H,\v) 
  = \sum_{\v\in V[\v,\w]}g(\v)\angle(H[\v,\w],\v) 
  + \sum_{\v\in V[\w,\v]}g(\v)\angle(H[\w,\v],\v).
\] 
\end{enumerate}\wasitemize 
\end{lemma}
\indy{Index}{slice}%
\indy{Index}{fan!local}%

\begin{proof} 
\claim{$(V,E')$ is a fan, where $E' = E\cup \{\{\v,\w\}\}$.}
Indeed, except for the intersection property, all of the properties
of a fan follow trivially from the fact that $(V,E)$ is a fan and
that $\v$ and $\w$ are nonparallel.  (Note the similarity with
Lemma~\ref{lemma:add-edge}.)  The intersection property also is
trivial except in the case $\ee=\{\v,\w\}$ and $\ee'\setminus \ee\ne
\emptyset$.  Pick $\u\in \ee'\setminus\ee$.  It follows from the
node partition of Lemma~\ref{lemma:disjoint} that
\begin{align*}
C(\ee) \cap C(\ee') &= (C(\v) \cap C(\ee')) \cup (C(\w)\cap C(\ee')) \\
&= C(\{\v\}\cap \ee') \cup C(\{\w\}\cap \ee') \\
&= C(\{\v,\w\}\cap \ee').
\end{align*}
The intersection property thus holds and $(V,E')$ is a fan.

It follows by Lemma~\ref{lemma:localization} that
$(V[\v,\w],E[\v,\w],F[\v,\w])$ is a local fan.

The second conclusion of the lemma follows from the following identities.
If $\u\ne \v,\w$ with $\u\in V[\v,\w]$, then $\u\not\in V[\w,\v]$ and 
\begin{equation}
\Wdart(H,\u)=\Wdart(H[\v,\w],\u),\quad \angle(H,\u) = \angle(H[\v,\w],\u).
\end{equation}
If $\u\in\{\v,\w\}$, then 
$\angle(H,\u)=\angle(H[\v,\w],\u) +\angle(H[\w,\v],\u)$.

Finally, it remains to be shown that the local fan is nonreflexive.
%Lemma~\ref{lemma:localization} already shows that the hypermap is
%isomorphic to $\op{Dih}_{2k}$. and that $F[\v,\w]$ can be identified with a
%face.  
The conclusion $V[\v,\w]\subset \bWdart(x)$ follows from the
fact that the angles $\beta(i)$ are increasing in
Lemma~\ref{lemma:monotone}.
\end{proof}





\section{Perimeter}

The perimeter bound of $2\pi$ for convex spherical polygons is
classical~\cite[p.~100]{vanderWaerden:1951}.  This result adapts
easily to a nonreflexive local fan, which also has a perimeter.  This
section proves the bound $2\pi$ on the perimeter of a nonreflexive
local fan (Lemma~\ref{lemma:convex-hyp}).  A great circle has
perimeter $2\pi$.  The proof follows the classical argument, which
deforms the nonreflexive local fan in a way that flattens out the
interior angles and increases the perimeter.  Eventually a circular
nonreflexive local fan or lunar nonreflexive local fan is reached.
Both of these extremal cases have perimeter $2\pi$.



%\subsection{perimeter}

\begin{definition}[perimeter]\guid{IQCPCGW}\label{lemma:perim}
Let $(V,E,F)$ be a nonreflexive local fan.    Set
\[ 
  \op{per}(V,E,F) 
= \sum_{i=0}^{k-1} \arc_V(\orz,\{\rho^i \v,\rho^{i+1} \v\}), 
\] 
where $k=\card(F)$.  The right-hand side of this formula is easily
seen to be independent of the choice of $\v\in V$.  Call $\op{per}$
the \newterm{perimeter} of the nonreflexive local fan.  If $\v,\w\in V$ are
distinct nodes, define the \newterm{partial perimeter}
\[ 
  \op{per}(V,E,F,\v,\w) 
= \sum_{i=0}^{r-1} \arc_V(\orz,\{\rho^i \v,\rho^{i+1} \v\}), 
\] 
where $r$ is chosen so that $\w=\rho^r \v$ and $0<r\le k-1$.
\end{definition}
\indy{Index}{perimeter!nonreflexive local fan}%
\indy{Index}{fan!local}%
\indy{Notation}{per@$\op{per}$ (perimeter)}%



\begin{lemma}[perimeter majorization]\guid{WSEWPCH}\rating{1200} %was 400
\label{lemma:convex-hyp}
The perimeter of every nonreflexive local fan is at most $2\pi$.  
\end{lemma}
\indy{Index}{fan!local}%
\indy{Index}{perimeter}%

\begin{proof} 
\claim{If the nonreflexive local fan is circular, then its perimeter is
$\op{per}(V,E,F) =2\pi$.}  Indeed, by Lemma~\ref{lemma:circular},
the arcs making up the perimeter all lie in a common plane.  The
azimuth cycle on $V$ coincides with $\rho:V\to V$.  The sum of the
terms in the formula defining the perimeter is the sum of the
azimuth angles in the azimuth cycle.  The sum is $2\pi$ by
Lemma~\ref{lemma:2pi-sum}.


\claim{If the nonreflexive local fan is lunar, then its perimeter is
$\op{per}(V,E,F) =2\pi$.}  Indeed, by Lemma~\ref{lemma:lunar}, the
set $V$ is contained in the union of two half-planes.  The perimeter
is the sum of arcs in a half-circle in the first half-plane plus the
sum of arcs in a half-circle in the second half-plane. This sum is
$2\pi$.

Finally, assume that the nonreflexive local fan is generic.  Suppose for a
contradiction that the lemma is false.  Consider all counterexamples
that minimize the cardinality of $V$.  
%Among all such
%counterexamples, pick a counterexample with the smallest number of
%darts $x\in F$ such that $\op{azim}(x) = \pi$.

A nonreflexive local fan $(V,E,F)$ is determined by $V$ and the cyclic
permutation $\rho:V\to V$: $E=\{\{\v,\rho \v\}\mid \v\in V\}$ and $F
= \{(\v,\rho \v)\mid \v\in V\}$.

In such a counterexample, if there is any flat dart $x=(\v,\w)\in F$,
then there is a new nonreflexive local fan $(V',E',F')$ with $V' =
V\setminus\{\v\}$ and $\rho':V'\to V'$ given by
\[ 
\rho'(\u) = \begin{cases}
\rho(\u), & \text{if } \rho(\u)\ne \v,\\
\rho(\v), & \text{if }\rho(\u) = \v.\\
\end{cases}
\] 
This is a nonreflexive local fan with the same perimeter, contrary to the presumed
minimality of the counterexample.  Thus, in the minimal counterexample
$\op{azim}(x) <\pi$ for all $x\in F$.

If $\card(V) <3$, then the nonreflexive local fan is circular or lunar.  The
circular and lunar cases have 
already been treated.  If $\card(V)=3$, then $V=\{\v_1,\v_2,\v_3\}$.
By the triangle inequality, $\arc_V(\orz,\{\v_2,\v_3\}) \le
\arc_V(\orz,\{\v_2,-\v_1\})+\arc_V(\orz,\{-\v_1,\v_3\})$.  Thus,
\begin{align*}
  \op{per} &=\arc_V(\orz,\{\v_1,\v_2\}) 
  + \arc_V(\orz,\{\v_2,\v_3\}) 
  + \arc_V(\orz,\{\v_1,\v_3\})\\
  &\le(\arc_V(\orz,\{\v_1,\v_2\})+\arc(\orz,\{\v_2,-\v_1\}))
  \\&\qquad\qquad+(\arc_V(\orz,\{\v_1,\v_3\})
+\arc_V(\orz,\{-\v_1,\v_3\})) \\
  &= \pi+\pi.
\end{align*}

Now assume that $\card(V)\ge 4$.  Select $\v\in V$.  Consider a
deformation of the nonreflexive local fan $\varphi:V\times I \to \ring{R}^3$ that
fixes $V\setminus\{\v\}$ and moves $\v$:
\[ 
\varphi(\v,t) = \cos(t) \v - \sin(t) \rho \v,\quad
 t \in I=\leftclosed0,\frac{\pi}{2}\rightclosed.
\] 
By the spherical triangle inequality, this deformation increases in the
perimeter.  For sufficiently small $t$, it remains a generic nonreflexive local fan
(Lemma~\ref{lemma:fan-open}).  For sufficiently small $t$, the
minimality gives $\angle(\varphi(\u,t))<\pi$.  Eventually, for some
smallest $t=t_0$, the deformed value is no longer a generic nonreflexive local fan.
% or for some $\u\in V$, $\angle(\varphi(\u,t_0))=\pi$.  In the latter
% case, the minimality condition fails, and the result follows.

When $t=\pi/2$, we have $\rho\v,-\rho\v\in V(t)$, which violates the
genericity condition.  This gives an upper bound on the time $t_0$ of
``first failure.''  Let $L$ be the line of intersection of two planes
through the origin:
\[ 
L= \op{aff}\{\orz,\v,\rho\v\}
\cap\op{aff}\{\orz,\rho^{-1}\v,\rho^{-2}\v\}.
\] 
These two planes are not equal by the nonflatness at $\v$ and
Lemma~\ref{lemma:A}.

\claim{The line $L$ meets the segment
  $\op{aff}^0_+(\emptyset,\{\v,-\rho\v\})$.}  Otherwise, it meets
$\op{conv}\{\v,\rho\v\}$.  The points $\v$ and $\rho\v$ lie in distinct
half-spaces bounded by $\op{aff}\{\orz,\rho^{-1}\v,\rho^{-2}\v\}$.  By
the nonflatness conditions, they even lie in different open
half-spaces.  Since $\bWdart(F,\rho^{-1}\v)$ is contained in one of
the two half-spaces, the condition $\v,\rho\v\in
\bWdart(F,\rho^{-1}\v)$ fails, and $(V,E,F)$ is not a nonreflexive local
fan.  This contradiction establishes the claim.

\claim{The time of $t_0$ of first failure satisfies $t_0 \le t_1
  <\pi/2$, where $t_1$ is the first time at which $\varphi(\v,t)$
  meets $L$.}  Indeed, by the previous claim, $t_1 <\pi/2$.  Also, at
time $t_1$, the set $\{\orz,\varphi(\v,t_1),\rho^{-1}\v,\rho^{-2}\v\}$
is coplanar, so that the dart of $F$ at $\rho^{-1}$ is flat, which is
contrary to the established flatness property.  This gives the claim.

\claim{The \case{nonparallel} property holds at $t=t_0$}.  Indeed, it
is enough to check nonparallelism when one of the points is
$\varphi(\v,t_0)$ and the other is $\rho\v$ or $\rho^{-1}\v$.  Begin
with $\rho^{-1}\v$.  If the set $\{\orz,\rho^{-1}\v,\varphi(\v,t)\}$
is collinear, then that line is contained in $L$, which is impossible
since $\rho^{-1}\v\not\in \op{aff}\{\orz,\v,\rho\v\}$.  Thus,
$\rho^{-1}\v$ and $\varphi(\v,t)$ are not parallel.  Also $\pi/2 >
t_0$ implies that $\varphi(\v,t_0)$ and $\rho\v$ are not
parallel. This gives nonparallelism.

\claim{The \case{intersection} property holds.}  If $\ee\subset V$,
write $\ee(t) = \{\varphi(\u,t) \mid \u\in \ee\}$.  Let $W
=W(\orz,\v,\rho\v,\rho^{-1}\v)$. The verification of the intersection
property is based on the following facts (when $t>0$):
\begin{enumerate}\wasitemize  
\item If $\v\not\in \ee$, then $C(\ee(t))=C(\ee)\subset W$.
\item $C^0\{\v(t)\} \cap W = \emptyset$.
\item $C^0\{\v(t),\rho^{-1}\v\}\cap W = \emptyset$.
\item $C\{\v(t),\rho\v\}\cap W = C\{\v,\rho\v\}.$
\item $C\{\v(t),\rho\v\}\cap C\{\v(t),\rho^{-1}\v\} = C\{\v(t)\}$.
\end{enumerate}\wasitemize 

\claim{At time $t=t_0$, the deformed value is a nonreflexive local fan.}
Otherwise, if the object fails to be a nonreflexive local fan at time
$t=t_0$, then the \case{nonparallel} property, \case{intersection}
property, or the \case{dihedral} property fails.  The first two
properties have already been checked, so that the deformed value must
be a fan.  Furthermore, the conditions for a fan to be nonreflexive local
are closed conditions, so they must also hold.

In summary, at $t=t_0$, the deformed fan is nonreflexive local but not
generic.  The perimeter bound follows from circular and lunar cases.
\end{proof}

Here is a second proof of the same lemma.  It is conceptually much
simpler but possibly more difficult to formalize.  The proof is based
on polar polygons (a generalization of polar triangles to spherical
polygons).

\begin{proof} A fan does not have any faces of cardinality less than
three.  Every blade of the fan has radian measure less than $\pi$.
\indy{Index}{polygon!polar}%

Consider the case of a spherical triangle with edges $a_i$
and polar triangle with angles $\beta_i$. Then $\beta_i=\pi-a_i$.
The perimeter is 
\[ a_1+a_2+a_3 = 2\pi - (\beta_1 -\beta_2 -
\beta_3-\pi)= 2\pi-\op{sol} < 2\pi\]  because the
solid angle $\op{sol}$ of the polar triangle is always strictly
positive.  \indy{Index}{triangle!spherical}%

Similarly, if the edges of the spherical polygon are
$a_i$, then the angles of the polar polygon are $\beta_i = \pi-a_i$.
The perimeter is
\[ 
a_1+\cdots+a_n  = 2\pi- \op{sol}< 2\pi,
\] 
where $\op{sol} = 2\pi-\sum a_i$ is the solid angle of the polar polygon.
%~\cite[\p.261]{williamson:2008}.
\indy{Notation}{solid@$\sol$ (solid angle)}%
\end{proof}


\section{Main Estimate}\label{sec:weight}  

Our aim becomes single-minded throughout the rest of the chapter; we
wish to give a proof of the main estimate (Lemma~\ref{lemma:empty-d}).
This,  the longest proof in the book,  requires substantial
preparation.  We  assume the existence of a counterexample to the
main estimate.  Assuming the existence of some counterexample, a
compactness argument gives the existence of a minimal counterexample.
The properties of minimal counterexamples are developed in a long
series of lemmas.  Eventually, enough properties of a minimal
counterexample are established to conclude that it cannot exist.

Although the main estimate is all that is needed for the text part of
the proof of the Kepler conjecture, the linear programming part of the
proof requires other estimates that are similar in nature to the
main estimate.  These secondary estimates will be given at the end of the chapter.

\subsection{statement of results}\label{sec:statement}

This subsection states the main results of the chapter.

\begin{definition}[$\hm$,~$\tau$,~$\dih_i$]\guid{CUFCNHB}\label{def:tau}
Let $(V,E,F)$ be a nonreflexive local fan.  Set $\hm = 1.26$.  Set
\[ 
  \tau(V,E,F) =\sum_{x\in F} \op{azim}(x)\left(1 + \left(\dfrac{\sol_0}{\pi}  \right)
    \dfrac{\normo{\nd(x)}-2}{2\hm-2}\right) 
+ \left(\pi+{\sol_0}\right) (2- k(F)),
\] 
where $\sol_0=3\arccos(1/3)-\pi\approx0.551$ is the solid angle of a
spherical equilateral triangle of side $\pi/3$, and $k(F)$ is the
cardinality of $F$.  
Let 
\begin{equation}\label{eqn:tautri}
  \tau_{tri}(y_1,y_2,y_3,y_4,y_5,y_6) =
  \sum_{i=1}^3 \dih_i(y_1,\ldots,y_6)
\left(1 + \left(\dfrac{\sol_0}{\pi}  \right)\dfrac{y_i -2}{2\hm-2}\right) 
- \left(\pi+{\sol_0}\right),
\end{equation}
where
\begin{align}\label{eqn:dihi}
\dih_1(y_1,y_2,y_3,y_4,y_5,y_6) &= \dih(y_1,y_2,y_3,y_4,y_5,y_6),\notag\\
\dih_2(y_1,y_2,y_3,y_4,y_5,y_6) &= \dih(y_2,y_3,y_1,y_5,y_6,y_4),\textand\notag \\
\dih_3(y_1,y_2,y_3,y_4,y_5,y_6) &= \dih(y_3,y_1,y_2,y_6,y_4,y_5).
\end{align}
\indy{Notation}{h0@$\hm$}
\indy{Notation}{zzt@$\tau$}
\indy{Notation}{sol@$\sol_0$}
\indy{Notation}{dih@$\dih_i$}
\end{definition}


\begin{definition}[standard,~superior,~diagonal] Let $(V,E)$ be a fan.  
We write $\normo{\ee}$ for $\norm{\v}{\w}$, when $\ee=\{\v,\w\}\subset V$.
We say that  $\ee$ is standard if
\[
2\le \normo{\ee}\le2\hm.
\]
We say that  $\ee$ is superior if
\[
2\hm\le \normo{\ee}\le\sqrt{8}.
\]
If $\v,\w\in V$ are distinct, and $\ee={\v,\w}$ is not an edge in $E$, then
we call $\ee$ a diagonal of the fan.
\end{definition}


\begin{theorem}[main~estimate]\guid{JEJTVGB}
Let $(V,E,F)$ be a nonreflexive local fan (Definition~\ref{def:convex-local}).
We make the following additional
assumptions on $(V,E,F)$:
\begin{enumerate}
\item \case{packing} $V$ is a packing.  That is, for every $\v,\w\in
V$, if $\norm{\v}{\w}<2$, then $\v=\w$.
\item \case{annulus} $V\subset \BB$.
%\item \case{local~fan} $(V,E,F)$ is a nonreflexive local fan 
%(Definition~\ref{def:convex-local}).
%\item \case{subset} $S\subset E$.
%\item \case{s~norm} If $\{\v,\w\}\in S$, then $\norm{\v}{\w}=2\hm$.
\item \case{diagonal} For all distinct elements $\v,\w\in V$, if
$\{\v,\w\}\not\in E$, then 
\[ 
\norm{\v}{\w}\ge 2\hm.
\] 
\item \case{card} %$k=\card(F)$,
Let   $k=\card(E)=\card(F)$.  Then $3\le k \le 6$.
%   $s=\card(S)$ and $r=\card(E) - s = \card(F)-s$.  Then
%\[ 0\le s \le 3,\textand 3-s \le r \le 6 -
%2s.\] 
\end{enumerate}
\label{lemma:empty-d}
In this context, we have the following conclusions.
\begin{enumerate}
\item Assume $k\ge 4$.  If  every edge of $E$ is standard, then
\[ 
\tau(V,E,F) \ge d (k), \text{ where } d(k) =
\begin{cases}
  0.206&k=4,\\
  0.4819&k=5,\\
  0.712&k=6.
\end{cases}
\] 
\item Assume $k=5$.  Assume that every edge of $E$ is standard.
Assume that every diagonal $\ee$ of the fan satisfies $\normo{\ee}\ge\sqrt{8}$.
Then 
\[
\tau(V,E,F)\ge 0.616.
\]
\item Assume $k=5$.  Assume there exists some superior edge in $E$ 
and that the other four are standard.  Then 
\[
\tau(V,E,F)\ge 0.616.
\]
\item Finally, assume that $k=4$.  Assume that there exists some superior
 edge in $E$ and that the other three are standard.  Then
\[
\tau(V,E,F)\ge 0.477.
\]
\end{enumerate}
\end{theorem}

There are two related inequalities that we will prove separately. For that reason,
we state them as a separate lemma.

\begin{lemma}\label{lemma:tau3}
Under the same hypotheses on $(V,E,F)$, 
\begin{enumerate}
\item Assume $k=3$. Then
\[\tau(V,E,F)\ge 0.\]
\item Assume $k=4$.  Assume that every edge of $E$ is standard.
Assume that both diagonals $\ee$ of the fan satisfy $\normo{\ee}\ge3$.
Then
\[
\tau(V,E,F)\ge 0.467.
\]
\end{enumerate}
\end{lemma}

The proof of the main estimate occuplies the rest of the chapter.
We refer to the first conclusion as the {\it standard main estimate}.  The 
other estimates will be referred to as the {\it ad hoc estimates}. 

The main estimate and Lemma~\ref{lemma:tau3} are obtained by computer
calculation, proving nonlinear inequalities by interval arithmetic.
Two  difficulties
arise in the proof of  the main estimate.  First, nonlinear optimization is
in general NP hard; and our calculations in particular rapidly become
more difficult to carry out as the dimension increases.  When $k=3$,
the set $V=\{\v_1,\v_2,\v_3\}$ is $6$ dimensional ($9$ spacial
coordinates minus a three-dimensional group of rotational symmetries).
These calculations in six dimensions are relatively simple.  However,
by the time $k=6$, the dimension of $V$ has reached $15$, which is far
beyond our computational capacity.  We are forced to prove a series of
lemmas, showing that any configuration $(V,E,F)$ that minimizes $\tau$
lies in an explicit low-dimensional subset of this set of local
nonreflexive fans, where low-dimensional means anything small enough
to be treated directly by a computer calculation.

The second source of difficulty comes from numerical instabilities.
For numerical stability, we insist on using analytic functions on
compact domains.  One of our favorite strategies is to slice along
internal blades to cut local fans into smaller fans, and inductively
build up the desired estimates from the smaller fans.  However, when
we slice along an internal blade, it is very difficult to avoid sets
$V$ that degenerate in the sense of lying in a plane through the
origin.  The functions defining $\tau$ are not analytic in a
neighborhood of degenerate $V$.  The functions behave as
$\sqrt{\Delta}$, with $\Delta$ tending to $0$ from above.  Concerns
such as these force us to use relatively short diagonals when we
slice.  The general heuristic we use is that degeneracies are avoided when
$\normo{\ee}< 3.106\ldots$, and calculations become stable
when $\normo{\ee}<3.01$  (see the proof of
Lemma~\ref{lemma:compact:bs}).

We do not present a complete proof of the main estimate in the text,
because much of it is done by computer.  In the rest of this chapter,
we describe how the local fans of large dimension (especially, the case $k=6$)
can be reduced to much lower dimension.  From there, the reader must
trust that the small calculations have been executed, or turn directly to
the computer implementation for details.

\subsection{constraints}

Let $(V,E,F)$ be a local nonreflexive fan that satisfies all the assumptions
of the main estimate.  The main estimate takes the form of a series of
bounds
\begin{equation}\label{eqn:td}
\tau(V,E,F) > d,
\end{equation}
assuming various length constraints on the edges and diagonals of the fan.
In building up these estimates inductively (by slicing into
smaller fans), we will need to consider further estimates of the same
general form (\ref{eqn:td}), under many different length constraints on
edges and diagonals.
With that in mind, we introduce a \newterm{constraint datum}.
\indy{Notation}{s@$s$ (constraint datum)}

\begin{definition}[constraint~datum]
A \newterm{constraint datum} $s$ consist of the following:
\begin{enumerate}
\item a natural number $k\in \{3,4,5,6\}$,
\item a real number $d$,
\item real constants $a_{i,j}$, $b_{i,j}$,  satisfying
   $a_{i,j} = a_{j,i}$, $b_{i,j}=b_{j,i}$, $a_{i,j}\le b_{i,j}$, for $i,j\in\ring{Z}/k\ring{Z}$.
\item a subset $I\subset \ring{Z}/k\ring{Z}$, such that $\card(I)+k\le 6$.
%\item real constants $h_{i,j}=h_{j,i}$, for $i,j\in\{1,\ldots,k\}$.
%  We assume that if $k>3$, we have $h_{i,j}=0$ for all $i,j$.
\end{enumerate}
\end{definition}

The constants $a_{i,j}$ and $b_{i,j}$ will give the lower and upper bounds
on the edges and diagonals of a fan $(V,E,F)$ with $V=\{\v_1,\ldots,\v_k\}$:
\[
a_{i,j}\le \norm{\v_i}{\v_j} \le b_{i,j}.
\]
The constant $d$ appears in (\ref{eqn:td}).    The set $I$ is used to
make minor adjustments to the estimates, and will be explained later.
In most cases, we can take $I=\emptyset$.
%The constants $h_{i,j}$
%will give minor correction factors to (\ref{eqn:td}) that will only
%been needed when $k=3$.  

\begin{example} The constants in the conclusions of the main estimate
  can be packaged into constraint data.  For example, the standard
  main estimate for $k=6$ gives the constraint datum $d=0.712$, $I=\emptyset$,
 and
\[
a_{i,j} = \begin{cases} 0, & i=j,\\
  2, & j\equiv i\pm1\op{mod}~k,\\
  2 \hm, & \text{otherwise}.
  \end{cases}
\qquad
b_{i,j}=\begin{cases}
 0, & i=j,\\
 2\hm, & j\equiv i\pm1\op{mod}~k,\\
 4\hm, & otherwise.
  \end{cases}
\]
The upper bound $4\hm$ on any diagonal comes from the triangle
inequality: $\norm{\v_i}{\v_j} \le \normo{\v_i}+\normo{\v_j} \le
4\hm$.   

In the cases of the main estimate with a superior edge, there is a
choice involved in assigning an index $i\in \ring{Z}/k\ring{Z}$ to the
superior edge.  We pick the index arbitrarily.
%In such cases, there are $k$ different constraint
%data, depending on the choice of index. 
We write $\smain$ for
the set of constraint data $s$, for all cases of the main estimate.
\end{example}
\indy{Notation}{Smain@$\smain$ (main estimate constraint data)}

\begin{example}[ear]  We have a constraint datum $s$ given by
$k=3$, $d=0.11$, 
\[
\hbox{}
\leftclosed a_{01},b_{01}\rightclosed=\leftclosed\sqrt8,\stab\rightclosed,\quad 
\leftclosed a_{12},b_{12}\rightclosed=
\leftclosed a_{23},b_{23}\rightclosed=
\leftclosed 2,2\hm\rightclosed.
\]
with $I=\{0\}$.  We call $s$ an \newterm{ear} (by analogy with an ear
in a triangulation of a polygon, which is a triangle that has two of
its edges in common with the polygon).
\end{example}

To prove the main estimate, we will use a finite set $S$ of constraint
data that includes $\smain$.  For each $s\in S$, we write $k(s)$,
$d(s)$, $a_{i,j}(s)$, and so forth for the associated constants.  We
extend all subscripts modulo $k(s)$, so that $a_{i+n k,j} = a_{i,j}$,
and so forth.  For any two subscripts, $i$ and $j$, we write $|i-j|_0$
for the minimum value of $|i - j + n k|$ as $n$ runs over integers.


To obtain the main estimate by induction by slicing fans
into pieces, the constraint data must be compatible. We impose the
following coherence conditions.  We use the constant $\stab=3.01$
to make the data numerically stable.  Its use will become apparent
in Lemma~\ref{lemma:compact:bs}.
\indy{Notation}{ccrit@$\stab=3.01$}

\begin{definition}[coherence]
A set $S$ of constraint data is \newterm{coherent}, if the following conditions
hold.
\begin{enumerate}
\item \case{bounds} $2\le a_{i,j}(s)\le \stab$ and $a_{i,j}(s)\le
  b_{i,j}(s)$, for all $s\in S$ and all $i\ne j\in
  \ring{Z}/k\ring{Z}$.  Also, $0 = a_{i,i}(s)\le b_{i,i}(s)$ and
  $b_{i,i+1}\le \stab$.  If $i\in I(s)$, then $\leftclosed
  a_{i,i+1},b_{i,i+1}\rightclosed=\leftclosed\sqrt{8},\stab\rightclosed$.
%\item \case{dihedral symmetry}
 % For every $s\in S$, there exists $s'\in S$ such that 
  %\[
  %k=k(s)=k(s'),\ d(s)=d(s'),\ f_{i,j}(s) = f_{-i,-j}(s'),\quad f = a, b, h.
  %\]
  %For every $s\in S$, there exists $s'\in S$ such that
  %\[
  %k=k(s)=k(s'),\ d(s)=d(s'),\ f_{i,j}(s) = f_{i+1,j+1}(s'),\quad f=a, b, h.
  %\]
\item \case{induction}
  If $s\in S$ where $k(s)>3$, and if  $\ell'<\ell''\in \{1,\ldots, k\}$, where
    $|\ell''-\ell'|_0 > 1$, then
 there exist $s',s''\in S$, $m'\in \ring{Z}/k'\ring{Z}$, $m''\in \ring{Z}/k''\ring{Z}$,
where $k'=k(s')$ and $k''=k(s'')$,
with the following
properties
\begin{enumerate}
\item  $k(s') = \ell' - \ell'' +1$,
  $k(s'') = (k(s)+\ell'')  - \ell' + 1$.
\item $d(s) \le d(s') + d(s'')$.
\item Define a one-to-one function from 
  \begin{equation}\label{eqn:injk'}
  \ring{Z}/k'\ring{Z}\to \ring{Z}/k\ring{Z}
   \end{equation}
 by $m'-i \mapsto \ell'-i$, for $i=0,\ldots,k'-1$.
If $m'\in I(s')$, then $s''$ is an ear.   If
$m'-i\in I(s')$ and if $1\le i\le k'-1$, then $\ell'-i\in I(s)$.
% if and only if
%If $m'-i\in I(s')$ if and only if
%\[
%\begin{cases}
%  \ell'-i \in I(s), & \text{if } 1\le i \le k'-1\\
%  s'' \text{ is an ear}, &\text{if } i=0.
%\end{cases}
%\]
\item $(a_{m'-i,m'-j}(s'),b_{m'-i,m'-j}(s'))=(a_{\ell'-i,\ell'-j}(s),b_{\ell'-i,\ell'-j}(s))$,
  for $i,j\in \{0,\ldots,k'-1\}$, provided $\{i,j\}\ne \{0,k'-1\}$.
\item $a_{\ell',\ell''}(s) \in \leftclosed a_{m',m'+1}(s'),b_{m',m'+1}(s')\rightclosed$.
\item The corresponding properties $(c'')$, $(d'')$, $(e'')$, 
under $\phantom{a}'\leftrightarrow \phantom{a}''$.
%\item $a_{i,j}(s') \le a_{i,j}(s)  \le b_{i,j}(s')$, for $i,j\in \{\ell',\ldots,\ell''\}$.
%\item $a_{i,j}(s'') \le a_{i,j}(s) \le b_{i,j}(s'')$, for $i,j\in \{\ell'',\ldots,k(s)+\ell'\}$.
\end{enumerate}
\end{enumerate}
\end{definition}
% XX need to fix d when there is an hij correction term.

%The second
%condition imposes dihedral symmetry.  The third condition gives compatibility
%with induction.  We warn that in the conditions relating $*_{i,j}(s')$ to $*_{i,j}(s)$
%the subscripts are reduced relative to two different moduli $k(s')$ and $k(s)$.
The induction corresponds to splitting a fan with $k$ vertices into
two smaller fans with $k(s')$ and $k(s'')$ vertices.  All of the edge
length constraints are to be preserved under the splitting (d), with a
mild compatibility condition on the new edge created by the split (e).
The indexing set $I$ keeps track of the ears that are split off.
Every time an ear is severed, an indexing set $I$ is increased by one
element.

The set $I$ is used to make a small correction $d(s,\v)$ to the
constants $d(s)$.  Set $\sigma(s) =1$ when $s$ is an ear;  $\sigma =
-1$, otherwise.  Let $V=\{\v_i\mid i\in \ring{Z}/k\ring{Z}\}$ 
be a set of points in $\ring{R}^3$.
Write
\begin{equation}
d(s,\v) = d(s) +  0.1\, \sigma\,\sum_{i\in I(s)} (\stab - \norm{\v_i}{\v_{i+1}}).
\end{equation}
The set $I$ is empty for $s\in \smain$, so this correction does not
directly affect the main estimates:
\[
d(s,\v) = d(s), \text{ for all } s \in \smain.
\]
If we have coherent data $s',s'',\ell',\ell'',m',m''$ we can use the injections 
(\ref{eqn:injk'}) to pick out  cyclically ordered subsets
$\{\v'_i\mid i\in \ring{Z}/k'\ring{Z}\}$
and $\{\v'_i\mid i\in \ring{Z}/k'\ring{Z}\}$ of $V$.
The coherence property $d(s)\le d(s') + d(s'')$ under splitting
implies a related coherence:
\begin{equation}
d(s,\v) \le d(s',\v') + d(s'',\v'').
\end{equation}
Indeed, this holds when two ears are created by the splitting, 
because both signs $\sigma$ are positive,
and the sums over the singleton sets $I(s')$ and $I(s'')$ are nonnegative.
Hence
\[
d(s,\v) = d(s) \le d(s') + d(s'') \le d(s',\v) + d(s'',\v).
\]
This hold when a single ear is created by the splitting, because the
terms $\sigma(s')$ and $\sigma(s'')$ have opposite signs and terms cancel.
When no new ear is created, then by the coherence conditions the indexing
set $I(s)$ is identified with the disjoint union of $I(s')$ and $I(s'')$ so that
the sum over $I(s)$ breaks into corresponding sums over
$I(s')$ and $I(s'')$.


\subsection{minimality}

Next we associate a set $\BB_s$ with each constraint datum $s$ in a coherent
family.
\indy{Notation}{BBs@$\BB_s$}

\begin{definition}[$\BB_s$]
  For every constraint datum $s$, and every function
  $\v:\{1,\ldots,k(s)\}\to \BB$, let $V_\v\subset \BB$ be the image of
  $\v$.  Let $E_\v$ be the image of $i\mapsto \{\v_i,\v_{i+1}\}$.  Let
  $F_v$ be the cyclic permutation $(\v_1,\v_2,\ldots,\v_k)$ of $V$.
 Let $\BB_s$ be
  the set of all functions $\v$ such that
\begin{enumerate}
\item $a_{ij}(s)\le\norm{\v_i}{\v_j}\le b_{ij}(s)$, for all $i,j$.
\item $(V_\v,E_\v,F_\v)$ is a local nonreflexive fan.
\end{enumerate}
\end{definition}

\begin{lemma}[]\label{lemma:compact:bs}
Let $S$ be a coherent set of constraint data.  Then
for every $s\in S$, the set $\BB_s$ is compact (as a subset of
$\BB^k \subset \ring{R}^{3k}$).
\end{lemma}

\begin{proof}  The set $\BB$ is defined as a closed subset of a closed
ball in $\ring{R}^3$.  It is compact.  By taking products of
a compact set, $\BB^k$ is compact. The set $\BB_s$ is defined by two
conditions.  The first enumerated condition in
the definition of $\BB_s$ is a closed constraint.  
It is enough to check that the condition the second condition is also
a closed constraint.  That is, it is enough to show that the set of functions $\v$
such that $(V_\v,E_\v,F_\v)$ is a local nonreflexive fan is closed in $\BB^k$.
 
For this, we run through each defining property of fan, local, and nonreflexive
in turn, and check that they are all closed conditions. 
For that purpose, consider a 
function
\[
\v : \{1,\ldots,k\}\to\BB.
\]
that lies in the closure of functions in $\BB_s$.
We must show that the limit $(V_\v,E_\v,F_\v)$ is also a  local nonreflexive fan.
By the coherence condition $2\le \norm{\v_i}{\v_j}$, when $i\ne j$, we see
that $\v$ is an injective function on the domain $\{1,\ldots,k\}$.
We find that $V_\v$ is a subset of $\BB$ of cardinality $k$.  In particular,
it is a nonempty finite set such that $\orz\not\in V_\v$.  This verifies the first
two defining properties of fan.

The condition \case{nonparallel} follows from the estimates based on coherence.
\[
2 \le \norm{\v_i}{\v_{i+1}} \le \stab, 
\]
If $\v_i$ and $\v_{i+1}$ are parallel, we get a contradiction:
\[
\norm{\v_i}{\v_{i+1}} = | \, \normo{\v_i} \pm \normo{\v_{i+1}} \,|
\]
which at least $4>\stab$ or no greater than $2\hm - 2 < 2$.

We turn to the condition \case{intersection}.  This is the most tedious part 
of the proof, because there are several cases involved in showing that 
for all $\ee ,\ee '\in E \cup \{\{\w\}\mid \w\in V\}$, 
\[ C(\ee )\cap C(\ee ') = C(\ee \cap \ee ').\] We leave most of these
routine verifications to the reader.  Two cases are noteworthy.  (1)
Suppose that $\ee$ and $\ee'$ are disjoint sets of cardinality two,
such that the data for $\v$ gives a nonempty intersection
$C^0(\ee)\cap C^0(\ee')\ne\emptyset$.  The intersection of these two
blades is an open condition, so that this failure to satisfy the fan
constraint is open, and satisfaction of the constraint is therefore
closed.  (2) Suppose that $\ee=\{\u\}$ and $\ee'\in E_\v$ is disjoint
from $\ee$.  Suppose for a contradiction that $C^0(\ee')$ meets
$C(\ee)$.  We obtain a planar quadrilateral with diagonals $\ee'$ and
$\{\orz,\u\}$.  By contracting the diagonal $\ee'$, we obtain a
rhombus of side $2$.  By vector geometry, the two diagonals $d_1$ and
$d_2$ of the rhombus satisfy
\begin{equation}\label{eqn:rhombus16}
d_1^2 = 16 - d_2^2.
\end{equation}
We have $d_2\le 2\hm$ because  $\u$ is an element of the annulus $\BB$,
 and $\ee'$ satisfies an upper bound coming from the
coherence conditions:
\begin{equation}\label{eqn:rhombus}
d_1^2 = \normo{\ee'}^2\le \stab< 3.106\ldots^2 = 
16-(2\hm)^2 \le 16-d_2^2 = d_1^2.
\end{equation}
%16 = d_1^2 + d_2^2 \le (2\hm)^2 + \stab^2 = 15.4105.
This is a contradiction. 

The defining properties of local fan are combinatorial, and depend only on 
$s$.
In the definition of nonreflexive, the condition \case{angle} is given
as a closed condition on the azimuth angle.  The condition
\case{wedge} is also given as a closed condition.
This completes the proof.
\end{proof}

The following lemma is based on the same methods as the previous
lemma.  It tells us that sufficiently short blades are necessarily
internal.

\begin{lemma}[]\guid{TECOXBM}\rating{0}\label{lemma:2hm-slice}
Let $S$ be a coherent family, let $s\in S$, and let $\v\in \BB_s$.
Let $\u,\w\in V_\v$ satisfy $2\le\norm{\u}{\w}\le \stab$ where
$\{\u,\w\}\not\in E_\v$.  Then $\u$ and $\w$ are nonparallel.
Moreover,
$C^0\{\u,\w\}\subset \Wdart(x)$ for all $x\in F$.
\end{lemma}
% can extend the norm out to \stab.

\begin{proof} 
The proof that $\u$ and $\w$ are nonparallel is identical to the
proof in the previous lemma that showed $\v_i$ and $\v_{i+1}$ are not parallel.

We turn to the second conclusion of the lemma.
Assume for a contradiction that the second conclusion of the lemma is false.
Then by Lemma~\ref{lemma:internal}, all the intermediate internal
angles between $\u$ and $\w$ are equal to $\pi$.  As a result, (after
interchanging $\u$ and $\w$ if necessary), the set
$\{\orz,\u,\rho\u,\rho^2\u,\ldots,\rho^r\u\}$ is planar.
We obtain a planar quadrilateral with diagonals
$\{\orz,\rho\u\}$ and $\{\u,\w\}$.  We obtain the same contradiction
as in the proof of case (2) of \case{intersection} in the previous lemma,
by deforming the quadrilateral to a rhombus.
\end{proof}

\begin{lemma}[continuity]\guid{XX}\rating{}\label{lemma:compact-fan}
Let $S$ be a coherent set, let $s\in S$.  Then $\tau^*:\BB_s\to\ring{R}$ defined
by
\[ 
(s,\v) \mapsto \tau(V_\v,E_\v,F_\v)-d(s,\v)
\] 
is a continuous function on $\BB_s$.  Moreover, it attains a minimum.
\end{lemma}

\begin{proof} 
The function $\tau$ is a polynomial in $\normo{\v_i}$ and
$\op{azim}(\orz,\v_i,\v_{i+1},\v_{i-1})$.  The norm and azimuth
angle are both continuous functions of $\v$.
Moreover, a continuous function on a compact space attains its minimum.
\end{proof}



The proof of the main estimate has the following structure.  We
construct an explicit finite coherent set $S$ of constraint data that
includes the set $\smain$ of constraint data appearing in the main
estimate.  We give a proof that for every $s\in S$ and every $\v\in
\BB_s$
\begin{equation}\label{eqn:main:sv}
\tau^*(s\v)> 0.
\end{equation}


\begin{definition}[minimal counterexample]
  Let $S$ be a coherent set of constraint data.  We say that
  $(s,\v)\in S\times\BB_s$ is a \newterm{minimal counterexample} to
  the constraints $S$ if the following conditions hold.
\begin{enumerate}
\item $\v$ minimizes the function $\v\mapsto
  \tau^*(s,\v)=\tau(V_\v,E_\v,F_\v)-d(s,\v)$ over $\BB_s$.
\item  $\tau^*(s,\v)\le 0$.
%If $s$ is the standard constraint datum for $k=3$, then the minimum value
%is negative.
\item If $s'\in S$ is any constraint datum such that $k(s')<k(s)$, then the
minimum value of $\tau^*$ on $\BB_s$ is positive.
\item For all $i,j$ such that $|i-j|_0>1$, we have $a_{i,j}(s) < \norm{\v_i}{\v_j}$.
\end{enumerate}
\end{definition}

\begin{lemma}
Let $S$ be a coherent set of constraint data.
If (\ref{eqn:main:sv}) fails to hold for some $s\in S$, 
then there exists a minimal counterexample for $S$.
\end{lemma}

\begin{proof}
  Pick some $s$ that minimizes $k(s)$, from the set $S_1\subset S$ of
  constraint data $s$, such that $\BB_s$ contains points violating
  (\ref{eqn:main:sv}).  Let $\v$ minimize $\tau^*$ on $\BB_s$.  By
  construction, it satisfies the first three defining properties of a
  minimal counterexample.  The fourth property is a consequence of the
  induction property of coherence: if $\norm{\v_i}{\v_j}$ reaches its
  minimum $a_{i,j}(s)$, then we can find induction data
  $s',s'',\ldots$ with $k(s')<k(s)$ and $s'\in S_1$, contrary to the
  minimality of $k(s)$.
\end{proof}

\begin{remark}[tightening inequalities]  
  One way to study minimal counterexamples is through
  \newterm{tightening} inequalities.  Let $(s,\v)$ be a minimal
  counterexample to a coherent set $S$ of constraints.  Assume that
  $s$ is not an ear.  Suppose that we can find $s'\in S$ such that
\begin{enumerate}
\item $k(s')=k(s)$, 
\item $\v\in \BB_{s'}$, 
\item $I(s')\subset I(s)$,
\item $d(s')\ge d(s)$.
%\item For all $i,j$ such that $|i-j|_0>1$, we have $a_{i,j}(s) < \norm{\v_i}{\v_j}$.
\end{enumerate}
It follows that $d(s',\v)\ge d(s,\v)$, and $\tau^*(s',\v)\le \tau^*(s,\v)\le0$.
Then by the minimality of $(s,\v)$, we find that $\BB_{s'}$ contains
a nonempty set of minimal counterexamples. 
That is, the existence of a minimal counterexample can be detected on $s'$.
We say that $s'$ is a tightening
of $s$.  
\end{remark}

\begin{lemma}
Let $S$ be a coherent set of constraint data.  Let $(s,\v)\in S\times \BB_s$
be a minimal counterexample to $S$.  
Suppose that
\[
d(s)\le 0.9.
\]
Then the solid angle of $U_{F_\v}$ is less than
$\pi$ in the local fan $(V_\v,E_\v,F_\v)$.
In particular, $(V_\v,E_\v,F_\v)$ is not a circular local fan.
\end{lemma}

The largest value that arises in our calculations will be
$d(s)=0.712$ (in the standard main estimate for $k=6$).  In particular,
we may assume in all that follows that the fan of a minimal counterexample
is not circular.

\begin{proof}
Assume for a contradiction that the area of $U_F$ is at least $\pi$.
\begin{align*}
\tau(V,E,F) &=\left(\pi+{\sol_0}\right) (2- k)+ \sum_{x\in F}\op{azim}(x)
\left(1 + \dfrac{\sol_0}{\pi}  \dfrac{\normo{v}-2}{2\hm-2}\right) \\
  &\ge\left(\pi+{\sol_0}\right) (2- k)+ \sum_{x\in F} \op{azim}(x) \\
  &=\sol(U_F) + (2-k)\sol_0\\
  &\ge \pi + (2-6)\sol_0 \\
  &> 0.92
%\\
  %&\ge 2\pi - 4\sol_0\\
 % &> 0.7578\\
  %&=0.103 (2) + 0.2759 (2)\\
 % &\ge 0.103 (2-s) + 0.2759 (r+2s-4) \\ 
  %&= d(r,s).
\end{align*}
\begin{align*}
d(s,\v) &= d(s) + 0.1\, \sigma \sum_I (\stab  - \norm{\v_i}{\v_{i+1}}) \\
   &\le d(s) + 0.1 (\stab - \sqrt{8}) \\
   &\le d(s) + 0.02\\
    &\le 0.92.
\end{align*}
The result follows.  The solid angle of $U_F$ is 
a hemisphere $2\pi$ when the fan is circular.
\end{proof}

\subsection{reducing dimension}

As we pointed out at the beginning of this section, one of the main
difficulties of the proof of the main estimate is the dimension of
$\BB_s$ is so large that we cannot minimize $\tau^*$ over $\BB_s$ directly
by computer.  The dimension of $\BB_s$ is $ 3 k- 3 \le 15$, and to
obtain reasonable performance, we prefer to restrict our computer
calculations to at most six dimensions.  This subsection gives a
series of lemmas that show that some minimal counterexample (to
suitable $S$) must lie in a subset of $\BB_s$ of small dimension.
This will allow us to use computers to complete the verifications of
the main estimate.

Throughout this subsection we let $S$ denote a coherent set of constraint
data, and let $s\in S$ with $k=k(s)$.
To study minimality, we consider (differentiable) curves
\[
\v:\leftopen\epsilon,\epsilon\rightopen\to \BB^{k(s)}.
\]
If we show that $\tau^*(s,\v(t))<\tau^*(s,\v(0))$ and $\v(t)\in\BB_s$,
whenever $t$ is positive and sufficiently small, then
 $(s,\v(0))$ is not a minimal counterexample.
For simplicity, we will start our study with curves that move a single point:
\begin{equation}\label{eqn:move1}
   \v_j(t)= \w_j\text{ if } j\ne i,\quad \v_i(0) = \w_i,
\end{equation}
for some subscript $i\in \ring{Z}/k\ring{Z}$ and some fixed $\w$.

\begin{lemma} If $(s,\w)$ is a minimal counterexample to the coherent constraint
set $S$, then for all $i$, one of the following constraints hold:
\begin{enumerate}
\item $\norm{\w_i}{\w_{i+1}}$ attains its lower bound $a_{i,i+1}(s)$.
\item $\norm{\w_i}{\w_{i-1}}$ attains its lower bound $a_{i,i-1}(s)$.
\item $\normo{\w_i}$ attains its lower bound $2$.
\end{enumerate}
\end{lemma}

\begin{proof} Fix $i$.  The function $\tau^*$ is decreasing along
the curve of the form (\ref{eqn:move1}) such that
$\v_i(t)=(1-t) \w_j$.
That is, we push the point $\w_i$ radially towards the origin.
Explictly, along this deformation, up to a constant, $\tau^*$ is equal
to a positive constant times $\normo{\v_i(t)}$.
If none of the constraints of the lemma are satisfied, then
$\v(t)\in \BB_s$ for all $t$ positive and sufficiently
small.
\end{proof}

Recall that we call $\w_i$ \newterm{flat} 
if $\angle(\w_i)=\pi$ in the local fan $(V_\v,E_\v,F_\v)$.

\begin{lemma}  If $(s,\w)$ is a minimal counterexample to the coherent constraint
set $S$, and if $\w_i$ is flat, then one of the following constraints hold:
\item $\norm{\w_i}{\w_{i+1}}$ attains its lower bound $a_{i,i+1}(s)$, and
 $\norm{\w_i}{\w_{i-1}}$ attains its lower bound $a_{i,i-1}(s)$.
\item $\normo{\w_i}$ attains its lower bound $2$.
\end{lemma}

\begin{proof}  The points $\{\orz,\w_{i-1},\w_i,\w_{i+1}\}$ line in a plane $A$.
Assume for a contradiction that neither constraint holds.  By the previous
lemma one of the norm constraints is satisfied, say
\[
\norm{\w_i}{\w_{i+1}}=a_{i,i+1}(s).
\]
We consider a  curve $\v$ of the form (\ref{eqn:move1}).
We let the curve $\v_i$ describes a circle through
 $\w_i$ with center $\w_{i+1}$ in the plane $A$.  Parameterize the curve
so that as $t$ increases, the norm $\normo{\v_i(t)}$ decreases.
The function $\tau^*(s,\v)$ is decreasing in $t$.  Explicitly, the
function again depends linearly on $\normo{\v_i(t)}$, because
the azimuth angles remain fixed.  The result follows.
\end{proof}

The following lemma allows us to propagate a lower bound constraint
along the edges meeting at a flat vertex.

\begin{lemma} Let $S$ be a coherent set of constraint data, and let $s\in S$.
Suppose that every minimal counterexample $(s,\w)$ has the
property that whenever $\w_i$ is flat $\norm{\w_i}{\w_{i+1}}$ attains its
lower bound $a_{i,i+1}(s)$.   Assume $a_{i,i+1}(s)<b_{i,i+1}(s)$.
Then every minimal counterexample $(s,\w)$
for which $\w_i$ is flat, also satisfies $\norm{\w_i}{\w_{i-1}}=a_{i,i-1}(s)$.
Moreover, the same result holds with $i+1$ and $i-1$ interchanged.
\end{lemma}

\begin{proof}
Let $(s,\w)$ be a minimal counterexample as described.
Assume for a countradiction that 
\[
\norm{\w_i}{\w_{i-1}}>a_{i,i-1}(s)
\]
We consider a curve $\v$ of the form (\ref{eqn:move1}) that moves $\v_i$
in a circular arc with center $\orz$ through the point $\w_i$ and in
the fixed plane $\{orz,\w_{i-1},\w_{i+1}$.  The function $\tau^*$ is
constant along this curve.  We orient the curve to be increasing
in $\norm{\w_i}{\w_{i+1}}$.  For sufficiently, small $t$, we find that
$\v(t)\BB_s$ is a minimal counterexample with $\v_i(t)$ such that
$\norm{\w_i}{\w_{i+1}} > a_{i,i+1}(s)$. This is a contradiction.
\end{proof}

\begin{remark}[lateral motion]\label{rem:contract}
  We continue to study curves $\v$ of the form (\ref{eqn:move1}).  We
  consider a curve $\v_i$ in $\ring{R}^3$ with parameter $t$ that
  describes the circle through $\w_i$ at fixed distance from $\orz$
  and $\v_{i-1}$ (or alternatively, at fixed distance from $\orz$ and
  $\v_{i+1}$).  Up to a constant, the function $\tau^*(s,\v)$ depends
  on $\w$ only through the three points $\w_{i-1}$, $\w_i$, and
  $\w_{i+1}$.  The function $\tau^*$ is invariant orthogonal
  transformations.  The dependence on $\w$ can be expressed through
  the function$\tau_{tri}$ in Definition~\ref{def:tau}, where $y_i$
  are the six edge lengths of the simplex
  $\{\orz,\w_{i-1},\w_i,\w_{i+1}$.  The derivative of $\tau^*$ along
  $\v$ is given by the partial derivative of $\tau_{tri}$ with respect
  to a single variable, say $y_4$.  Even when the dimension of $\BB_s$
  is large, the derivative calculation reduces to this function of six
  variables.  By a \cc{UPONLFY}{} we can show that under rather
  general conditions on $s$, the function $\tau_{tri}$ is increasing
  in $y_4$.  More generally, when the partial deriviative of
  $\tau_{tri}$ of $y_4$ vanishes, computer calculations of the second
  derivative show that the $\tau_{tri}$ has a local maximum (again
  under mild restrictions on the domain), so that there are no
  interior point local minima as a function of $y_4$.
\end{remark}

\begin{remark}[obtuse and flat nodes]
A complication occurs in lateral motions.  The curve $\v$ is required
to remain inside $\BB_s$ for $t$ sufficiently small and positive.  But
when $\w_i$ or $\w_{i+1}$ is flat, the curve does not generally remain
inside $\BB_s$.  We use several different strategies to get around
this complication.  First, sometimes a computer calculation shows that
for a given $s$, the azimuth angle of $\w_i$ is obtuse (and less than
$\pi$).  In this case, the curve decreases the azimuth angle at
$\w_{i+1}$ and the germ of the curve remains inside $\BB_s$.

Second, if $\w_{i-1}$ is flat but $\w_{i+1}$ is not, we can
interchange the role of $i-1$ and $i+1$, to move $\v_i$ in a circle at
fixed distance from $\orz$ from $\v_{i+1}$.

Finally, if $\w_{i-1},\w_{i-2},\ldots,\w_{i'}$ are flat but
$\w_{i'-1}$ is not, the points $\orz$ and
$W=\{\w_{i},\ldots,\w_{i'-1}\}$ lie in a common plane $A$.  We
consider a curve $\v$ that fixes all coordinates, except those in $W$,
and moves the points $W$ by rotating the plane $A$ about the line
through $\{\orz,\w_{i'-1}\}$.  (This is the one place where we
consider a curve $\v$ that moves more that one component $\v_i$ at
once.)  In this case, the dependence of $\tau^*$ on the curve factors
through the six edges of the simplex
$\{\orz,\w_i,\w_{i+1},\w_{i'-1}\}$.  The local minima of this function
can be studied by computer in the same way.
\end{remark}

As an illustration of these methods, we prove Lemma~\ref{lemma:tau3}.

\begin{proof}[computer proof]
  When $k=3$, the space of configuations has six dimensions.  The
  inequality $\tau(V,E,F)\ge0$ is a simple computer calculation.  When
  $k=4$, the assumptions of the lemma give that both diagonals have
  length at least $3$.  By a rigorous computer estimate of dihedral
  angles, all the nodes of a quadrilateral with diagonals at least $3$
  must have angle less than $\pi$.  We laterally contract edges (by
  Remark~\ref{rem:contract}) until both diagonals are precisely $3$ or
  all four edges reach the lower bound $2$.  However, the rhombus
  diagonal inequality (\ref{eqn:rhombus16}) shows that both diagonals
  become $3$ before the four edges reach the lower bound $2$.  Thus,
  it is enough to consider the case when both diagonals are $3$.  By
  adding these two constraints, we have reduced the dimension from $9$
  to $7$. This $7$-dimensional inequality is assessible to direct
  computer calculation.
\end{proof}

\begin{remark}[radial motion]\label{rem:radial}
We continue to study curves $\v$ of the form (\ref{eqn:move1}).  We
consider a curve $\v_i$ in $\BB\subset\ring{R}^3$ with parameter $t$ that
describes the circle through $\w_i$ at fixed distance from $\w_{i-1}$ and
$\w_{i+1}$. 
The function $\tau^*(s,\v)$ again reduces to
$\tau_{tri}$ and the derivative with respect to $t$ is given by the
partial derivative of $\tau_{tri}$ with respect to  $y_1=\normo{\v_i}$,
provided we use parameterization $t=y_1$.
 Whenever we use this curve, we  impose the preconditions
\begin{enumerate}
\item $\norm{\w_{i-1}}{\w_i}=\norm{\w_i}{\w_{i+1}}=2$, and
\item $\norm{\w_{i-1}}{\w_{i+1}} \ge \stab$.
\end{enumerate}
Under these conditions a computer calculation of the first and second
derivatives of $\tau_{tri}$ shows that it has no local minimum, provided
$\w_i$ is not flat.
To maintain nonreflexivity along the curve, we must also assume that
 neither of the nodes in $\{\w_{i-1},\w_{i+1}\}$ is flat,
Thus,  any minimal counterexample $(s,\w)$ that satisfies this and
the preconditions
must have an extremal norm:
\begin{equation}\label{eqn:extremal}
\normo{\w_i}=2,\text{ or } \normo{\w_i}=2\hm, \text{ or } \w_i
\text{ is flat.}
\end{equation}
\end{remark}

\subsection{computer proof of main estimate}

In this subsection, we sketch the computer proof of the main estimate.

\subsubsection{construction of $S$}

We sketch the construction of a coherent set $S$, starting with $\smain$.
We always wish to split along diagonals of length at most $\stab$. 
We saturate $S$ by adding constaint data for
all possible ways of  splitting data $s\in \smain$
along diagonals of length at most $\stab$.  (By Lemma~\ref{XX}, the
corresponding blades are necessarily internal, and can be used to
split the local fan.)  For example, when $k=5$, we need to add two
additional constraint data (up to symmetry), one corresponding to a
single diagonal (which partitions the pentagon into a quadrilateral
and triangle), and two diagonals (which triangulates the pentagon).
For these smaller constraint data, the constants $d$ have been
determined experimentally, and satisfy the coherence conditions.  We
do not list all the constants here.  They are available in the
computer code.  We define $I(s)$ to be as large as possible, subject
to the coherence conditions.  

For every $s$ in this large set $S$, 
we add a constraint datum $s'$ to $S$, where $s'$ has
the same parameters as $s$, except that for every $i,j$ such that $|i-j|_0>1$,
we set
\begin{equation}\label{eqn:s'}
a_{i,j}(s') = \max(a_{i,j}(s),\stab),\quad b_{i,j}(s')=\max(b_{i,j}(s),\stab).
\end{equation}
so that
\begin{equation}\label{eqn:stab}
\stab \le a_{i,j}(s')\le b_{i,j}(s').
\end{equation}
At this point,  a minimal counterexample
necessarily has diagonals greater than $\stab$.  By passing to a tightened
constraint, we may assume that if $(s',\v)$ is a minimal counterexample,
then the datum $s'$ satisfies (\ref{eqn:stab}).

In general, whenever we add any datum to $S$, we recursively add
further constraint data, corresponding to splitting along diagonals,
to restore the coherence.  We assume this is done in what follows, without
further mention.

When $s\in S$ satisfies $k(s)>3$, and $I(s)\ne\emptyset$, we add a
tightening $s'$ that is identical to $s$, except that
$I(s')=\emptyset$.

We consider a common tightening that combines all the cases with $k=5$
(with diagonals greater than $\stab$) into a single constraint datum
$s$.  For this we put the standard constraint on all edges but one
\[
(a_{i,i+1}(s),b_{i,i+1}(s))=
\begin{cases}
(2,2\hm),&\text{if } i\ne 0\\
(2,\stab),&\text{if } i=0.
\end{cases}
\]
and the usual tightened constraint on all diagonals:
\[
(a_{i,j},b_{i,j})=(\stab,4\hm),\quad \text{ if } |i-j|_0>1.
\]
For the constant $d(s)$ we take the tightest value: the maximum of
the constant $d(s)$ as $s$ runs over cases with $k(s)=5$.


\subsubsection{triangles and quadrilaterals}

The computer proof that there are no minimal counterexamples $(s,\v)$
with $k(s)=3$ goes as follows.  Up to rotational invariance, the function
$\tau^*$ can be expressed in terms of the function $\tau_{tri}$ of six
variables.  The rigorous nonlinear minimization of $\tau^*$ is easily done
by computer, and we find that for each $s\in S$, and every $\v\in \BB_s$,
we have $\tau^*(s,\v)>0$.

The computer proof that there are no minimal counterexamples $(s,\v)$
with $k(s)=4$ is not much more difficult.

\subsubsection{pentagons}

By taking a common tightening of constraint data, we have reduced the
calculation for pentagons ($k=5$) to a single datum $s$.   
By lateral motions  (Remark~\ref{rem:contract}), 
a computer assisted argument shows that every minimal
counterexample $(s,\w)$ satisfies
\[
\norm{\w_i}{\w_{i+1}} = 2,
\]
for all $i$.  We use the following lemma.

\begin{lemma}
Consider any skew pentagon in $\ring{R}^3$ whose five edges equal $2$.
Then there are two  diagonals  of the pentagon with a common
endpoint whose lengths  are at most
\[
1 + \sqrt{5} = 3.2607\ldots
\]
\end{lemma}

\begin{proof}
  Cut the pentagon into a triangle and skew quadrilateral along the
  shortest diagonal, of length $t$.  By the triangle inequality
  $t\le4$.  The shortest diagonal of the skew quadrilateral is
  maximized, when the quadrilateral is planar, with equal diagonals.
  The length of this diagonal is given by the largest root $u$ of
\[
\Delta(u^2,4,4,u^2,4,t^2)=0.
\]
Solve for $u$ in terms of $t$ to obtain
\[
u = \sqrt{4 + 2 t}.
\]
If $t>1+\sqrt{5}$, this gives a contradiction $u<t$.  
So $t\le 1+\sqrt{5}$.
Then also
\[
u = \sqrt{4 + 2t} \le 1+\sqrt{5}.
\]
\end{proof}


We assume that indexing is chosen so that $(s,\w)$ is a minimal
counterexample with two chosen diagonals.
\[
\stab < \norm{\w_0}{\w_i} \le 1+\sqrt{5},\quad i=2,3.
\]
Dihedral angle computer calculations show that under these
constraints, no vertex in $\{\w_0,\w_2,\w_3\}$ is flat.  This allows
us to apply radial motion (Remark~\ref{rem:radial}), to show that
$\w_1$ and $\w_2$ are extremal, in the sense of (\ref{eqn:extremal}).
This determines $\w_1$ and $\w_2$ (up to three cases each) as a
function of $\w_0$, $\w_2$, $\w_3$.  The calculations reduce in this
way to a single simplex $\{\orz,\w_0,\w_2,\w_3\}$, which have been
carried out by computer.

\subsubsection{hexagons}

Only one of the inequalites in the main estimate has $k=6$.  It
asserts that $\tau(V,E,F) > 0.712$.  By the construction of the
coherent set $S$, there is a tightening for which all diagonals
satisfy
\begin{equation}\label{eqn:astab}
a_{i,j}(s)=\stab.
\end{equation}  
We may assume that our constraint datum
has this property.
By lateral motions, we reduce to the case
\[
\norm{\w_i}{\w_{i+1}}=2,
\]
for all $i$.  We triangulate with three blades
$\{\w_{2i},\w_{2i+2}\}$, for $i=0,2,4$.

\begin{lemma}
The norms $\norm{\w_{2i}}{\w_{2i+2}}$, for $\w\in \BB_s$, 
are at least $\stab$ and at most $3.915$.
\end{lemma}

\begin{proof}
  The lower bound comes from (\ref{eqn:astab}).  By
  Lemma~\ref{lemma:delta-pos}, the squares of the edges $x_{ij}$ of
  the simplex $\{\orz,\w_{2i},\w_{2i+1},\w_{2i+2}\}$ gives a
  nonnegative value
\[
\Delta(x_{ij})\ge 0.
\]
However, this polynomial is  negative when
$\norm{\w_{2i}}{\w_{2i+2}} > 3.915$.
\end{proof}

By radial motion, we may assume that the (\ref{eqn:extremal}) holds at
each odd vertex $\w_{2i+1}$.  We warn that these contraction arguments
may produce reflex vertices in the local fan at some of the even
vertices $\v_{2i}$.  At this final stage, we abandon the nonreflexive
condition.  In fact, at this stage, we may abandon the geometry
altogether, and view $\tau^*$ analytically as a sum of four terms
$\tau_{tri}$, indexed by the four triangles in the triangulation of
the hexagon.  After radial motion, the points $\v_{2i+1}$ are rigidly
determined, up to three cases, in terms of the simplex
$\{\orz,\v_2,\v_4,\v_6\}$.  We have reduced the calculations for $k=6$
to a single simplex, which have been carried out by computer.

\subsubsection{instabilities}

We add a final lemma that we used to deal with the issue of numerical
instability in the calculations when one of the simplices (ears)
$\{\orz,\w_{2i},\w_{2i+1},\w_{2i+2}\}$ is close to being planar.

\begin{lemma}
Consider the function $\tau_{tri}$
on the domain
\[
y_1,y_2,y_3\in\leftclosed 2,2\hm\rightclosed,
\quad y_4\in\leftclosed \stab,3.915\rightclosed,
\quad y_5=y_6=2,\quad
\Delta(y_1^2,\ldots,y_6^2)\ge0.
\]
Then $\tau_{tri}$ has the following properties
\begin{enumerate}
\item If $y_1=2$, then $\tau_{tri}(y_1,y_2,y_3,y_4,y_5,y_6)\ge -\sol_0$.
\item If $y_1=2\hm$, then $\tau_{tri}(y_1,y_2,y_3,y_4,y_5,y_6)\ge 0$.
\item If $\dih_1(y_1,y_2,y_3,y_4,y_5,y_6) =\pi$, then 
  \[
  \tau_{tri}(y_1,y_2,y_3,y_4,y_5,y_6) = \sol_0\dfrac{y_1 -2\hm}{2\hm-2}.
  \]
\end{enumerate}
\end{lemma}

\begin{proof}
The first claim is the trivial lower bound that we obtain by replacing
${y_i -2}$
with zero, and the solid angle $-\pi+\sum_{i=1}^3 \dih_i$ with zero
in the Definition~\ref{def:tau} of $\tau_{tri}$.  It does not need the
assumption that $y_1=2$.

To establish the second claim, we write each of the three terms 
\[
\pi-\dih_1,\quad\dih_2,\quad \dih_3.
\]
in the form $f_i\sqrt{\Delta}$, for $i=1,2,3$.  The explicit formulas
for dihedral angles show that $f_i$ is an analytic function of
$y_1,\ldots,y_6$.  When $y_1=2\hm$, we obtain a formula for
$\tau_{tri}$ of the general form
\[
\tau_{tri}(y_1,\ldots,y_6)= f(y_1,\ldots,y_6)\sqrt{\Delta(y_1^2,\ldots,y_6^2)}
\]
for some analytic function $f$.  A computer calculation shows that $f\ge0$
on the domain given in the lemma. Hence $\tau_{tri}$ is also nonnegative.

We turn to the third claim.  If $\dih_1=\pi$, then $\dih_2=\dih_3=0$.
If we make these substitutions into the formula for $\tau_{tri}$, the
claim follows immediately.
\end{proof}

The lemma is used to avoid numerical instabilities as follows.  The
three statements of the lemma correspond to the three cases given by
(\ref{eqn:extremal}).  We may assume that the simplex
$\{\orz,\w_{2i},\w_{2i+1},\w_{2i+2}\}$ falls into one these cases.
When the simplex approaches a planar configuration, that is as
$\Delta$ approaches $0$, we replace the term $\tau_{tri}$ with the
lower bound given by the lemma, to avoid computing the a nonanalytic
term directly.  By doing this, all of the computer calculations go
through without trouble.
