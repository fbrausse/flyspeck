%
\chapter{Cyclic Fan}\label{sec:cyclic}


\begin{summary}
In various proofs, it is useful to focus attention on a single face
in a fan.  This leads to the notion of a localization of a fan along
a face.  The localization is again a fan.  Special properties of the
localized fan are captured in the notion of a cyclic fan.  A cyclic
fan has a perimeter.  One of the results of this chapter is a proof
of the upper bound $2\pi$ on the perimeter of a cyclic fan.

A theory of deformations of cyclic fans is developed.  Sufficient
conditions are stated for the deformation of a cyclic fan to remain
a cyclic fan.
\end{summary}


\section{Localization}


\subsection{basics}


\begin{definition}[cyclic fan]\guid{FTNGOGF} 
A triple $(V,E,F)$ is a \newterm{cyclic
fan} if the following conditions hold.
\begin{description} 
\item \case{fan} $(V,E)$ is a fan;
\item \case{face} $F$ is a face of $H = \op{hyp}(V,E)$;
\item \case{dihedral hypermap} $H$ is isomorphic to $\op{Dih}_{2k}$, where $k =
\card(F)$;
\item \case{angle} $\op{azim}(x)\le \pi$ for all darts $x\in F$; and
\item \case{wedge} $V\subset \bWdart(x)$ for all $x\in F$.
% \item If $\{\v,\w\}\in E$, then $\{\orz,\v,\w\}$ is not
%   collinear. %% part of def of fan.
\end{description}
\end{definition}
\indy{Index}{fan!cyclic}%

\begin{remark}[visualization]\guid{PNCVUMY}
The intersection of $X(V,E)$ with the unit sphere is a spherical
polygon, when $(V,E,F)$ is a cyclic fan.  The spherical polygon gives
a visual representation of the cyclic fan. The choice of $F$
distinguishes the ``interior'' of the polygon from its exterior.  The
final two conditions are convexity constraints.
\end{remark}

%% XX background cyclic permutation.

\begin{lemma}[]\guid{WRGCVDR}\rating{ZZ}  
For any cyclic fan $(V,E,F)$, there is a bijection from $F$ onto $V$
given by
\begin{displaymath}
(\v,\w) \mapsto \v
\end{displaymath}
Moreover, write $\v\mapsto(\v,\rho\v)$ for the inverse map. 
Then $\rho:V\to V$ is a cyclic permutation.
\end{lemma}
\indy{Index}{cyclic permutation}%

\begin{proof} The map from a face to the set of nodes is a bijection
for the dihedral hypermap $\op{Dih}_{2k}$. It is a also bijection for a fan isomorphic
to $\op{Dih}_{2k}$.

For all $(\v,\rho \v)\in F$,
\begin{displaymath}
f(\v,\rho \v) = (\rho\v,\rho^2\v),
\end{displaymath}
so that the order of $\rho$ on $V$ is the order of $f$ on $F$, which
is $k=\card(F)$.  Thus, $\rho$ is a cyclic permutation of $V$ of order
$k=\card(V)$.
\end{proof}

\begin{definition}[$\rho$,~$\v$]\guid{MFMPCVM} 
For any cyclic fan $(V,E,F)$, write
$\nd:F\to V$ and $\rho:V\to V$ for the bijections of the preceding
lemma.
\end{definition}
\indy{Notation}{ZZrho@$\rho$}%
\indy{Notation}{node@$\nd$}%

\begin{definition}[interior angle,~$\angle$,~$\bWdart$]\guid{PJRIMCV}
For any cyclic fan $(V,E,F)$,
write
\begin{displaymath}
\angle(\v) = \op{azim}((\v,\rho\v)),
\end{displaymath}
for all $\v\in V$.  This is the \newterm{interior} angle of the cyclic
fan at $\v$.  Also, write
\begin{displaymath}
  \Wdart(F,\v) = \Wdart((\v,\rho \v)),\quad 
\bWdart(F,\v) =\bWdart((\v,\rho \v)).
\end{displaymath}
\indy{Notation}{1@$\angle$}%
\indy{Notation}{Wdart@$\Wdart(F,\v)$}%
\indy{Notation}{Wdart@$\bWdart(F,\v)$}%
%\indy{Notation}{node@$\nd(x)$}%
\end{definition}


\begin{definition}[localization]\guid{BIFQATK}
 Let $(V,E)$ be a fan, and let $F$ be
a face of $\op{hyp}(V,E)$.  Let
\begin{displaymath}
\begin{array}{rll}
V' &= \{\v\in V \mid \exists~\w\in V.~(\v,\w)\in F\}.\\
E' &= \{\{\v,\w\} \in E\mid (\v,\w)\in F\}.
\end{array}
\end{displaymath}
The pair $(V',E')$ is called the \newterm{localization} of $(V,E)$
along $F$.
\end{definition}
\indy{Index}{localization}%


\begin{lemma}[localization]\guid{LVDUCXU}\rating{ZZ}
\label{lemma:localization}
Let $(V,E)$ be any fan, and let $F$ be a face of its hypermap.  Then
the localization $(V',E')$ is a fan, and $F$ is a face of
$\op{hyp}(V',E')$.  Moreover, the hypermap $\op{hyp}(V',E')$ is
isomorphic to $\op{Dih}_{2k}$, where $k= \card(F)$.  Finally, the angle
$\op{azim}(x)$ and the wedges $\bWdart(x)$ and $\Wdart(x)$ do not depend
whether they computed relative to $\op{hyp}(V,E)$ or to
$\op{hyp}(V',E')$, for all $x\in F$.
%That is, 
%\begin{displaymath}
%\begin{array}{lll}
%\op{azim}(V,E,x) &=\op{azim}(V',E',x)\\
%\Wdart(V,E,x) &=\Wdart(V',E',x)\\
%\end{array}
%\end{displaymath}
\end{lemma}



\begin{proof}
The proof that $(V',E')$ is a fan is a sequence of simple
verifications based on the techniques of
Remark~\ref{remark:fan-verify}.  The details are left to the reader.

The dart set $D'$ of $\op{hyp}(V',E')$ is naturally identified with
the disjoint union $F\coprod F'$, where $F = \{(\v,\w) \in F\}$ and
$F'=\{(\w,\v) \mid (\v,\w)\in F\}$.  Under this identification, $F$
is a face of $\op{hyp}(V',E')$.  The edge map $e(\v,\w)= (\w,\v)$ is
a bijection of $F$ onto $F'$.  Pick any $x\in F$ and define a
bijection from the disjoint union of two copies of the cyclic group
$Z_k$ onto $D'$ by the pair of maps
\begin{displaymath}
i \mapsto f^i x,\quad\text{ and } i'\mapsto e f^{i-1} x.
\end{displaymath}
This bijection extends to an isomorphism of hypermaps $\op{Dih}_{2k}$ onto
$\op{hyp}(V',E')$.

The proof that the $\op{azim}(x)$ and $\Wdart(x)$ do not depend on the
choice of fan is a consequence of their definitions:
\begin{displaymath}
\begin{array}{lll}
\op{azim}(x) &= \op{azim}(\orz,\v,\w,\sigma(\v,\w)),\\
\Wdart(x) &= \Wdart(\orz,\v,\w,\sigma(\v,\w)).\\
\end{array}
\end{displaymath}
where $x = (\v,\w)$.  It is enough to check that $\sigma(\v,\w)\in
E'(\v)$.  But $\{\sigma(\v,\w),\v\}\in F$, so this is indeed the case.
% In particular $\op{azim}(x)\le\pi$ follow by the assumptions of the
% lemma.
\end{proof}



\subsection{geometric types}\label{sec:types}

\begin{definition}[generic,~lunar,~circular]\guid{RTPRRJS}
A fan $(V,E)$ is \newterm{generic} if it is cyclic and if for every $\{\v,\w\}\in E$
and every $\u\in V$, 
\begin{displaymath}
C\{\v,\w\}\cap C^0_-\{\u\} = \emptyset.
\end{displaymath}
A fan is \newterm{circular} if it is cyclic and if there exists $\u\in V$ and
$\{\v,\w\}\in E$ such that
\begin{displaymath}
C^0\{\v,\w\}\cap C^0_-\{\u\}\ne \emptyset.
\end{displaymath}
A fan is \newterm{lunar} with pole $\{\v,\w\} \subset V$ if it is cyclic,  if it is not
circular, if $\v\ne\w$, and if $\{\v,\w\}$ is a parallel set.
\end{definition}
\indy{Index}{generic}%
\indy{Index}{lunar}%
\indy{Index}{circular}%


\begin{lemma}[]\guid{CIZMRRH}\rating{ZZ} Every cyclic fan is
generic, lunar, or circular.  Moreover, these three properties are
mutually exclusive.
\end{lemma}
\indy{Index}{fan!cyclic}%
\indy{Index}{cyclic fan}%
\indy{Index}{generic}%
\indy{Index}{lunar}%
\indy{Index}{circular}%

\begin{proof} If $(V,E,F)$ is not generic,  select some $\{\v,\w\}\in E$
and some $\u\in V$ such that
\begin{equation}\label{eqn:non-generic}
C\{\v,\w\}\cap C^0_-\{\u\} \ne \emptyset.
\end{equation}
Now $C\{\v,\w\} = C^0\{\v,\w\} \cup C\{\v\}\cup C\{\w\}$.  If, for
some such triple $(\u,\v,\w)$, the
intersection~(\ref{eqn:non-generic}) meets $C^0\{\v,\w\}$, then the
cyclic fan is circular.  Otherwise, the cyclic fan is lunar.
\end{proof}

\begin{definition}[flat]\guid{YPSTLXA}
 Let $(V,E,F)$ be a cyclic fan.
If $\angle(\v)=\pi$, then $\v$ is \newterm{flat}.
\end{definition}
\indy{Index}{flat (node of a fan)}


\begin{lemma}[]\guid{LDURDPN}\rating{ZZ}  \label{lemma:coplanar}
Assume that $\{\orz,\u,\w\}$ and $\{\orz,\u,\v\}$ are not collinear sets.
Then $\op{azim}(\orz,\u,\v,\w)=\pi$ if and only if
there exists a plane $A$ such that $S=\{\orz,\u,\v,\w\}\subset A$
and such that the line $\op{aff}\{\orz,\u\}$ separates $\v$ from
$\w$ in $A$.
\end{lemma}

\begin{proof} The given azimuth angle is $\pi$ if and only if
$\dih(\{\orz,\u\},\{\v,\w\})=\pi$.  This holds exactly when $S$ is
coplanar and the line $\op{aff}\{\orz,\u\}$ separates $\v$ from $\w$
in $A$.
\end{proof}

\begin{lemma}[]\guid{KOMWBWC}\rating{ZZ}\label{lemma:kom}
Let $(V,E,F)$ be a cyclic fan.  Let $k=\card(F)$.  Assume that for
some $0<r\le k-1$ and some $\v\in V$, the set $U=\{\v,\rho
\v,\ldots,\rho^r \v\}$ is contained in a plane $A$ passing through
$\orz$.  Let $\e$ be the unit normal to $A$ in the direction
$\v\times \rho \v$.  Then the set $U$ is cyclic with respect to
$(\orz,\e)$, and the azimuth cycle $\sigma$ on $U$ is
\begin{displaymath}
  \sigma \u = 
\begin{cases} 
\rho \u, & \u\ne \rho^r\v,\\ \v, & \u = \rho^r\v.
\end{cases}
\end{displaymath}
Furthermore, for all $0\le i\le r-1$,
\begin{displaymath}
(\rho^i \v\times\rho^{i+1}\v)\cdot \e > 0.
\end{displaymath}
\end{lemma}

\begin{proof} 
Write $\v_i = \rho^i \v$, for $i=0,\ldots,r$.

\claim{ $(\v_i \times \v_{i+1})\cdot \e > 0$, for all $i\le r-1$. }
Indeed, the base case $(\v_0\times \v_1)\cdot \e > 0$ of an induction 
holds by assumption.  Assume for a contradiction that the inequality
holds for $i$, but not for $i+1$.  Then
\begin{displaymath}
  \op{aff}^0_+(\{\orz,\v_{i+1}\},\v_i) 
= \op{aff}^0_+(\{\orz,\v_{i+1}\},\v_{i+2}).
\end{displaymath} 
This forces $C^0\{\v_i,\v_{i+1}\}$ to meet $C^0\{\v_{i+1},\v_{i+2}\}$,
which is contrary to the definition of a fan.  Thus, the claim holds.

The fact that $U$ is cyclic follows trivially from the fact that $U$
is contained in a plane $A$ through $\orz$ and that $\e$ is orthogonal
to $A$.

\claim{For all $0\le i \le r-1$,  $\sigma \v_i = \v_{i+1}$.}
Otherwise, there is some 
\begin{displaymath}
  \u \in (U\setminus \{\v_i,\v_{i+1}\}) 
\cap W^0(\orz,\e,\v_i,\v_{i+1}) \cap A.
%  ~~\subset~~ 
\end{displaymath}
However, by the claim, this intersection is a subset of $C^0\{\v_i,\v_{i+1}\}$, and
$\u\in C^0\{\v_i,\v_{i+1}\}$ is contrary to the property
\case{intersection} of fans.  The result follows.
% The membership of $\u$ in the rightmost term is contrary to the
% definition of fan.  Let $\e_3$ be the unit vector in the direction
% $\v\times \u$.  Let $\e_1$ be the unit vector in the direction $\u$.
% Let $\e_2 = \e_3 \times \e_1$.  The coordinates of $\u,\v,\w$ with
% respect to the frame $(\e_1,\e_2,\e_3)$ take the form
%\begin{displaymath}
%\begin{array}{rllrl}
%\v &= * \e_1 + a \e_2,  &\quad & a &< 0\\
%\u &= a' \e_1, &\quad & a' &>0\\
%\w&= * \e_1 + a'' \e_2, &\quad & a'' &>0\\
%\end{array}
%\end{displaymath}
%From this representation it is clear that $\v\times \u$ points in the
% same direction as $\u\times \w$.  The set $\{\v,\u,\w\}$ is clearly
% cyclic and the counterclockwise cycle $\sigma$ in the
% $\{\e_1,\e_2\}$ plane takes $\v$ to $\u$ and $\u$ to $\w$.
\end{proof}

\begin{lemma}[]\guid{OZQVSFF}\rating{ZZ} \label{lemma:A}  
Let $(V,E,F)$ be a cyclic fan and let
  $\u,\v,\w\in V$ satisfy
\begin{itemize}
\item $\{\orz,\u,\v,\w\}$ is contained in a plane $A$; \vspace{3pt}
\item $\u,\w\not\in\op{aff}\{\orz,\v\}$; and \vspace{3pt}
\item $\op{aff}^0_+(\{\orz,\v\},\u) \ne \op{aff}^0_+(\{\orz,\v\},\w)$.
\end{itemize}
Then $\v$ is flat.  Moreover, $\rho \v,\rho^{-1} \v\in A$.
\end{lemma}

\begin{proof} Let $x = (\v,\rho\v)\in F$.  
Order $\u$ and $\w$ so that
\begin{displaymath}
\op{azim}(\orz,\v,\rho\v,\u) \le \op{azim}(\orz,\v,\rho\v,\w).
\end{displaymath}
By the definition of cyclic fan, by the conditions $\u,\w\in \bWdart(x)$, 
and by  Lemma~\ref{lemma:coplanar},
%By the assumptions, $\dih(\{\orz,\v\},\{\u,\w\})=\pi$.  Since
%$\u,\w\in \bWdart(x)$, it follows that
%\begin{displaymath}\pi = \dih(\{\orz,\v\},\{\u,\w\}) \le 
%\op{azim}(x) = \angle(\v) \le \pi.\end{displaymath}
%The first conclusion follows.
\begin{eqnarray*}
%\begin{array}{rll}
0 &\le& \op{azim}(\orz,\v,\rho\v,\u) \\
&=& \op{azim}(\orz,\v,\rho\v,\w) - \op{azim}(\orz,\v,\u,\w)\\
&=& \op{azim}(\orz,\v,\rho\v,\w)-\pi \\
&\le& \op{azim}(\orz,\v,\rho\v,\rho^{-1}\v) - \pi \\
&=&\op{azim}(x) - \pi \\
&=&\angle(\v)-\pi\\
&\le& 0. 
%\end{array}
\end{eqnarray*}
Hence each inequality is equality.  In particular, $\v$ is flat.
In particular, $0 =\op{azim}(\orz,\v,\rho\v,\u)$, so that 
\begin{displaymath}
\rho\v\in \op{aff}_+(\{\orz,\v\},\u) \subset A.
\end{displaymath}
Similarly,
\begin{displaymath}
\rho^{-1}\v\in \op{aff}_+(\{\orz,\v\},\w) \subset A.
\end{displaymath}
\end{proof}

If Lemma~\ref{lemma:A} can be applied once, then it can often be
applied repeatedly along a chain.  For example, the conclusion of the
lemma implies that $\rho^{-1} \v \in A$.  In fact, by the definition
of fan,
\begin{displaymath}
  \rho^{-1}\v \in A \setminus \op{aff}\{\orz,\v\} 
= \op{aff}^0_+(\{\orz,\v\},\u) \cup \op{aff}^0_+(\{\orz,\v\},\w).
\end{displaymath}
Suppose that $\rho^{-1} \v$ lies in the second term of the union.  If
$\w\ne \rho^{-1}\v$, then the assumptions of the lemma are met for
$\{\w,\rho^{-1} \v,\v\}$, giving the conclusions that $\rho^{-1} \v$
is flat, and $\rho^{-2}\v\in A$.  Repeating, we obtain a chain
\begin{displaymath}
\pi=\angle(\v) = \angle(\rho^{-1} \v) = \cdots,
\end{displaymath}
with $\v,\rho^{-1}\v,\ldots\in A$.  Another chain can be constructed
in the other direction $\v,\rho \v,\ldots$.  This process of chaining
gives the following lemma.

\begin{lemma}[circular geometry]\guid{KCHMAMG}\rating{ZZ}
\label{lemma:circular}
Let $(V,E,F)$ be a circular fan. Then
\begin{itemize}
\item $\v$ is flat for all $\v\in V$.
\item The set $V$ lies in a plane $A$ through $\orz$.
\item For some choice of unit vector $\e$ orthogonal to $A$, the set
$V$ is cyclic with respect to $(\orz,\e)$, and the azimuth cycle on
$V$ coincides with $\rho:V\to V$.  
\item 
$
\op{azim}(\orz,\e,\v,\rho\v) = \dih(\{\orz,\e\},\{\v,\rho\v\})
=\op{arc}_V(\orz,\{\v,\rho \v\}) <\pi
$.
\end{itemize}
\end{lemma}

\begin{proof} Let $\v, \u\in V$ be such that $C^0\{\u,\rho \u\}$ meets
$C^0_-\{\v\}$.  Apply Lemma~\ref{lemma:A} to $\{\u,\v,\rho \u\}$ to
conclude that $\v$ is flat, and some plane $A$ contains
$\{\orz,\u,\rho\u,\v,\rho \v,\rho^{-1} \v\}$.  If
%\begin{displaymath}
%\{\orz,\u,\rho \u,\v,\rho \v,\rho^{-1} \v\} \subset V\cap A,
%\end{displaymath}
%and 
$\w\in V\cap A$, then there exists $\w_1,\w_2\in (V\cap
A)\setminus\{\w\}$ for which the assumptions of Lemma~\ref{lemma:A}
hold for $\w_1,\w,\w_2$.  
Then $\w$ is flat, and $\rho \w \in V\cap A$.  The set $V\cap
A$ is therefore preserved by $\rho$.  By observing that $V$ is the
only nonempty subset of $V$ that is preserved by $\rho$, it follows
that $V\subset A$ and that $\w$ is flat for all $\w\in V$.


By Lemma~\ref{lemma:coplanar}, $V$ is cyclic with respect to a unit
vector $\e$ orthogonal to $A$.  The azimuth cycle on $V$ is $\v
\mapsto \rho \v$.

We turn to the final conclusion.  By the final conclusion of Lemma~\ref{lemma:kom}, and Lemma~\ref{lemma:sim}, the azimuth angle is less than $\pi$.  Under this
constraint, the azimuth angle equals the dihedral angle by Lemma~\ref{lemma:dih-azim}.
By definition, the dihedral angle is the angle $\arc_V(\orz,*)$ of an orthogonal projection of $\{\v,\rho\v\}$
to a plane with normal $\e$.  But $\{\v,\rho\v\}$ is already a subset of the plane $A$, so that
the projection is the identity map, and the dihedral angle is $\arc_V(\orz,\{\v,\rho\v\})$.
%
%\begin{displaymath}
%%\op{azim}(\orz,\e,\v,\rho\v) = \dih(\{\orz,\e\},\{\v,\rho\v\}) = \op{arc}_V(\orz,\{\v,\rho\v\}),
%\end{displaymath}
%because the azimuth angle is less than $\pi$ and
%the dihedral angle is defined as the angle obtained by orthogonal projection
%to the plane $A$.  Since the points $\v,\rho\in A$, orthogonal projection has no effect.
\end{proof}

\begin{lemma}[lunar geometry]\guid{HKIRPEP}\rating{ZZ}\label{lemma:lunar}
Let $(V,E,F)$ be a lunar fan with pole $\{\v,\w\}\subset V$.  
%Assume that $\rho^r \v = \w$, for some $0< r < k$.
Then
\begin{itemize}
\item $\u$ is flat, for all $\u\in V\setminus \{\v,\w\}$; \vspace{3pt}
\item $0< \angle(\v) = \angle(\w)\le \pi$; \vspace{3pt}
\item $V\cap \op{aff}_+(\{\orz,\v\},\rho \v) = \{\v,\rho \v,\ldots,
\w\}$; \vspace{3pt}
\item $V \cap \op{aff}_+(\{\orz,\v\},\rho^{-1} \v) = \{\w,\rho
\w,\ldots \v\}$; \vspace{3pt}
\end{itemize}
\end{lemma}

\begin{proof} Set $V_1 = \{\v,\rho \v,\ldots,\w\}$ and $V_2 =
\{\w,\rho \w,\ldots,\v\}$.  Let $\u\in V\setminus\{\v,\w\}$ be
arbitrary.  Apply Lemma~\ref{lemma:A} to the set $\{\v,\u,\w\}$ to
find that $\u$ is flat, and that $\{\orz,\u,\rho \u,\rho^{-1} \u\}$
belongs to a plane $A(\u)$.  Now $A(\u)$ and $A(\rho \u)$ are both
the unique plane containing $\{\orz,\u,\rho \u\}$, hence $A(\u) =
A(\rho \u)$ when $\rho \u\not\in \{\v,\w\}$.  By induction, there
are planes $A_1, A_2$ such that $V_i\subset A_i$.  There is an
azimuth cycle $\sigma_i$ on $V_i$ such that $\sigma_i \u = \rho \u$,
when $\u \in A_i\setminus \{\v,\w\}$.

The angles $\angle(\v)$ and $\angle(\w)$ are both equal to the
dihedral angle between these two half-planes.  In particular,
$0<\angle(\v)=\angle(\w)\le\pi$.
\end{proof}




\begin{lemma}[monotonicity]\guid{EGHNAVX}\rating{ZZ} 
\label{lemma:monotone}
  Let $(V,E,F)$ be a cyclic fan. Fix $\v_0\in V$.  Assume that $\{\orz,\v_0,\u\}$
is not collinear for any $\u\in V\setminus\{\v_0\}$.  For
  all $i$, set $\v_i = \rho^i \v_0$ and $\beta(i) =
  \op{azim}(\orz,\v_0,\v_1,\v_i)$.  Then
\begin{displaymath}0=\beta(1)\le \beta(2)\le \cdots\le
\beta(k-1)\le\pi.\end{displaymath}
Moreover, if $\beta(i)=0$ for some $1<i \le k-1$, then
\begin{displaymath}
\angle(\v_1) = \cdots = \angle(\v_{i-1}) = \pi,
\end{displaymath}
and $\{\orz,\v_0,\ldots,\v_i\} \subset \op{aff}^0_+(\{\orz,\v_0\},\v_1)$.
Finally, if $\beta(i)=\beta(k-1)$ for some $1\le i<k-1$ then 
\begin{displaymath}
\angle(\v_{i+1}) = \cdots = \angle(\v_{k-1}) = \pi,
\end{displaymath}
and $\{\orz,\v_i,\ldots,\v_{k-1}\} \subset
\op{aff}^0_+(\{\orz,\v_0\},\v_{k-1})$.
\end{lemma}

\begin{proof}  
Pick a frame and write the points $\v_j$ in spherical coordinates
$(r_j,\theta_j,\phi_j)$.  In an appropriate frame, $\phi_0=0$, and
$\theta_j=\beta(j)$, for all $j$.  From $\v_j\in \bWdart(F,\v_0)$
and $\angle(\v_0)\le\pi$, it follows that
$0\le\theta_j\le\theta_{k-1}\le\pi$ when $0\le j\le k-1$.

One may assume by induction that $0\le \beta(1)\le\cdots\le
\beta(i)$.  The condition
\begin{displaymath}
\v_0\in \bWdart(F,\v_i)
\end{displaymath}
implies that
\begin{displaymath}
  0 \le \op{azim}(\orz,\v_i,\v_{i+1},\v_0)
\le \op{azim}(\orz,\v_i,\v_{i+1},\v_{i-1})\le\pi.
\end{displaymath}
By Lemma~\ref{lemma:sim}, the resulting inequality
\begin{displaymath}
\sin(\op{azim}(\orz,\v_i,\v_{i+1},\v_0))\ge 0
\end{displaymath}
reduces to a triple-product:
\begin{displaymath}
(\v_0 \times \v_i)\cdot \v_{i+1}\ge 0.
\end{displaymath}
In spherical coordinates, this inequality becomes
\begin{displaymath}
r_0r_ir_{i+1}\sin\phi_i\sin\phi_{i+1}\sin(\theta_{i+1}-\theta_i)\ge0.
\end{displaymath}
Under the non-collinearity assumption, $\sin\phi_i\sin\phi_{i+1}\ne0$
(when $0< i < k-1$).  These inequalities give
$\theta_i\le\theta_{i+1}$ (with a small extra argument to exclude the
degenerate case $\theta_{i+1}=0,\theta_i=\pi$).  The conclusion
follows by induction.

Assume that $\beta(i)=\theta_i=0$, for some $i>1$.  Then by the first
conclusion, $\theta(j)=0$, for $0\le j\le i$.  That is, 
$\v_1,\ldots,\v_i$ all lie in the half-plane
$\op{aff}^0_+(\{\orz,\v_0\},\v_1)$.  In particular, they are coplanar.
A chaining argument based on Lemma~\ref{lemma:A} gives the result.

The final conclusion is similar.
\end{proof}



\subsection{deformation}\label{sec:deformation}

This section considers deformations of a cyclic fan $(V,E,F)$.

\begin{definition}[deformation]\guid{YWNHMBP}
A \newterm{deformation} of a cyclic fan $(V,E,F)$ over an interval
$I\subset\ring{R}$ is a continuous function $\varphi:V\times I
\to\ring{R}^3$ (in the discrete topology on $V$ and the product
topology on $V\times I$).
\end{definition}
\indy{Index}{deformation}%
\indy{Index}{fan!cyclic}%

\begin{notation}
  Beware of the notational distinction between the zenith angle $\phi$
  and the deformation $\varphi$.  When a deformation $\varphi$ is
  given, write $\v(t)$ as an abbreviation of $\varphi(\v,t)$, for
  $t\in I$.  Also, set
\begin{displaymath}
\begin{array}{lll}
V(t)&=\{\v(t) \mid \v\in V\},\\
E(t)&=\{\{\v(t),\w(t)\}\mid \{\v,\w\}\in E\},\\
F(t)&= \{(\v(t),\w(t)) \mid  (\v,\w)\in F\}.
\end{array}
\end{displaymath}
\indy{Notation}{vt@$\v(t)$ (deformation of $\v$)}
\end{notation}

A deformation does not require $(V(t),E(t),F(t))$ to be a cyclic fan
for all $t\in I$, although this will often be the case. The
permutation $\rho:V\to V$ gives $\varphi(\rho \v,t)\in V(t)$, for
every $\v\in V$.


\begin{example}[lunar deformation]\label{example:lunar}
Consider a lunar  fan $(V,E,F)$ with pole $\{\v,\w\}\subset V$.
Pick a frame and spherical coordinates $(r,\theta,\phi)$ such that $\phi(\v)=0$,
$\phi(\w)=\pi$,  $\theta(\rho \v)=0$, and $\theta(\rho^{-1}
\v)=\theta_{k-1}\le\pi$.  Consider the deformation $\varphi$ over $I
= \{t \mid 0 \le t < 1\}$ such that radial $r$ and zenith $\phi$
coordinates of $\varphi(\u,t)$ are constant as functions of $t$, and
the azimuth angle $\theta$ of $\varphi(\u,t)$ equals

\begin{displaymath}
\begin{cases} 
  (1-t) \theta_{k-1} & \text{if } \u\in 
\{\rho \w,\rho^2 \w,\ldots, \rho^{-1} \v\};\\
  0 & \text{if } \u\in \{\rho \v,\rho^2 \v,\ldots,\rho^{-1} \w\}.\\
\end{cases}
\end{displaymath}
%This deformation $(V(t),E(t),F(t))$ is a cyclic fan for all $t\in I$.
%The cardinality of $V$ is independent of $t\in I$.
Note that $(V(0),E(0),F(0)) = (V,E,F)$.
\end{example}
\indy{Index}{lunar}%
\indy{Index}{spherical coordinates}%

\begin{lemma}[]\guid{HZIYFIZ}\rating{ZZ}\label{lemma:lunar-deform} 
Let $(V,E,F)$ be a lunar fan with pole $\{\v,\w\}\subset V$.  In
the deformation described above, the triple $(V(t),E(t),F(t))$ is a
lunar  fan for all $t\in I$.  The cardinality of $V$ is
independent of $t\in I$.
\end{lemma}

\begin{proof} The proof consists of checking the defining properties of a lunar fan, 
one by one.  The verification of the property \case{cardinality}
follows from the methods of Remark~\ref{remark:fan-verify}.

The map $\varphi(\cdot,t)$ is invertible, so that $V(t)$ is in
bijection with $V$.  In particular, the cardinality does not depend
on $t$.

\case{origin} The radial spherical coordinate is nonzero for every
element of $V(t)$.  Hence $V(t)$ does not contain $\orz$.

\case{non-parallel} For $\{\v,\rho \v\}\in E$, the angle
$\alpha=\op{arc}_V(\orz,\{\varphi(\v,t),\varphi(\rho \v,t)\})$ is
independent of $t$.  The non-parallel property is equivalent to
$\alpha\ne0,\pi$.  Thus, the  property for $E$ implies
the property for $E(t)$.

\case{intersection} The points $V(t)$ are contained in the union of
two half-planes $A_1,A_t$.  The deformation is the identity on $A_1$
and an isometry $A_0\to A_t$ on the second half-plane.  These
bijections preserve the incidence relations of blades of the cones
$C(\ee)$.

\case{dihedral hypermap}~\case{face} The combinatorial properties of
the hypermap do not depend on $t$.  In particular, the hypermap has
face $F(t)$ and the hypermap is isomorphic to $\op{Dih}_{2k}$.

\case{angle} The azimuth angle $\angle(\varphi(\u,t))$ is fixed when
$\u\ne \v,\w$ and is decreasing in $t$ when $\u\in \{\v,\w\}$.
Hence, the upper bound on the angle is preserved.

\case{wedge} $V(t)\subset A_0\cup A_t$, where $A_0$ and $A_t$ are
the half-planes described above.  Now $\bWdart(F,\varphi(\u,t))$ is
a half-space containing $A_0$ and $A_t$ when $\u\ne \v,\w$.  Also,
$\bWdart(F,\varphi(\u,t))$ is a wedge with bounding half-planes
$A_0$ and $A_t$ when $\u\in \{\v,\w\}$.  Hence $V(t)\subset A_0\cup
A_t\subset \bWdart(F,\varphi(\u,t))$ for all $\u\in V$.

\case{lunar} The points $\v,\w\in V(t)$ remain fixed and hence
remain parallel.  Thus, the cyclic fan remains lunar.
\end{proof}
\indy{Index}{angle!azimuth}%

Next consider a deformation of a cyclic fan.  The following lemma
gives a list of conditions that ensure that the deformed fan remains
cyclic throughout the deformation.


\begin{lemma}[]\guid{XRECQNS}\rating{ZZ}\label{lemma:fan-open}
Let $\varphi$ be a deformation of a cyclic fan $(V,E,F)$ over an
interval $I$.  Assume that $0\in I$ and that $\varphi(\v,0)=\v$ for
all $\v\in V$.  For all $\v\in V$ assume that if
$\v$ is flat, then $\v(t)$ is flat for all sufficiently small $t$.
%$\{\orz,\rho^{-1}\v,\v,\rho\v\}$ is coplanar, then
%\begin{displaymath}
%\{\orz,\varphi(\rho^{-1}\v,t),\varphi(\v,t),\varphi(\rho\v,t)\}
%\end{displaymath} is coplanar for all sufficiently small $t$.
%If the property \case{wedge} is maintained for sufficiently small $t$,
Then exists $\epsilon>0$ such that $(V(t),E(t),F(t))$ is a cyclic fan
for all $t\in I\cap \leftopen-\epsilon,\epsilon\rightopen$.  
%Moreover,
%if $(V,E,F)$ is generic, then the property \case{wedge} is in fact
%maintained for sufficiently small $t$.  
Moreover,  if $(V,E,F)$ is
generic, then the deformed fan is also generic for sufficiently small
$t$.
\end{lemma}

\begin{proof} Each of the defining properties of a generic  fan will be
examined in turn.

\case{cardinality}: The set $V(t)$ is the image of $V$ and is therefore finite
and nonempty.  

\case{origin} Since $\varphi$ is continuous and
$\orz\not\in V$, it follows that $\orz\not\in V(t)$ for sufficiently
small $t$.

\case{non-parallel}: If $\v,\w$ are non-parallel, then $\v(t)$ and
$\w(t)$ are non-parallel for sufficiently small $t$.

\case{intersection}: If $\ee \cap \ee'=\emptyset$, then $C(\ee)\cap
S^2$ has a positive distance from $C(\ee')\cap S^2$.  Hence for
sufficiently small times, the deformation of these sets remain
disjoint.  If $\ee=\{\u,\v\}$ and $\ee'=\{\v,\w\}$ where $\u\ne\w$,
then again the deformations of $C(\ee)\cap C(\ee')$ is
$C(\{\v(t)\})$ for sufficiently small $t$.  The other cases are
similar.

\case{face},~\case{dihedral hypermap}: The azimuth cycle on $E(\v(t))$
is preserved; hence the combinatorial properties of the hypermap do
not change when $t$ is sufficiently small.

\case{angle}: If $\op{azim}(x)<\pi$, then the inequality remains
strict for sufficiently small $t$.  If $\op{azim}(x)=\pi$, then the
flatness assumption of the lemma forces the equality to be
preserved for all sufficiently small $t$.

\case{wedge}: The property $\u\in \Wdart(x)$ is an open condition.
It holds for sufficiently small $t$. Consider the case $\u\in
\bWdart(x)\setminus \Wdart(x)$, where $x= (\v_0,\rho\v_0)$.  Write
$\v_i = \rho^i\v_0$ and pick $r\le k-1$ such that $\u = \v_r$.  The
wedge property holds trivially when $r\in\{0,1,k-1\}$. Assume that
$r\not\in\{0,1,k-1\}$.  We finish the argument in cases, according to 
the type of the cyclic fan  $(V,E,F)$.

If the  fan $(V,E,F)$ is circular, then there is a plane $A$ through the origin
that contains $V$.  By the flatness condition, the deformation maintains
a coplanarity condition $V(t)\subset A(t)$.  The sets $\Wdart(x(t))$ are half-spaces
bounded by $A(t)$.  Thus $V(t)\subset A(t)\subset \Wdart(x(t))$, as desired.

If the fan $(V,E,F)$ is lunar with pole $\{v,w\}$, 
the argument is similar, but there are two half-planes
$A_1(t)$ and $A_2(t)$  such that $V(t)\subset A_1(t)\cup A_2(t)$.  Arguing as
in the previous case, each wedge $\Wdart(x(t))$ 
contains $A_1(t)\cup A_2(t))$.  The result follows.

If the fan $(V,E,F)$ is generic, then
by the genericity assumption $\v_r$ and
$\v_0$ are non-parallel.  In the notation of
Lemma~\ref{lemma:monotone}, $\beta(r) = 0$ or $\beta(r) =
\beta(k-1)$.  This proof treats the case $\beta(r)=0$. (The case
$\beta(r)=\beta(k-1)$ is similar.)  By Lemma~\ref{lemma:monotone},
\begin{displaymath}0=\beta(1)=\beta(2)=\cdots=\beta(r),\end{displaymath}
and 
\begin{displaymath}
\{\orz,\v_0,\v_1,\ldots,\v_r\} \subset \op{aff}^0_+(\{\orz,\v_0\},\v_1).
\end{displaymath}
In particular, the set is coplanar.  By the flatness assumption of
the lemma, $\{\orz,\v_0(t),\v_1(t),\ldots,\v_r(t)\}$ is coplanar for
sufficiently small $t$.  In fact, the condition $\v_r(t)\not\in
\op{aff}_+(\{\orz,\v_0(t)\})$ is an open condition, so that
$\v_r(t)\in \op{aff}^0_+(\{\orz,\v_0(t)\},\v_1(t))$ for sufficiently
small $t$.  This half-plane is the bounding half-plane of
$\bWdart(F,\v_0(t))$.  Hence $\u = \v_r\in \bWdart(F,\v_0(t))$ for
sufficiently small $t$.

\case{generic}: Genericity is stated as open conditions $\v\not\in
C\{\u,\w\}$.  These conditions continue to hold for sufficiently small
$t$.
\end{proof}


%%%%%%%%%%%




\section{Perimeter}

%\subsection{perimeter}

\begin{definition}[perimeter]\guid{IQCPCGW}\label{lemma:perim}
Let $(V,E,F)$ be a cyclic fan.    Set
\begin{displaymath}
  \op{per}(V,E,F) 
= \sum_{i=0}^{k-1} \arc_V(\orz,\{\rho^i \v,\rho^{i+1} \v\}), 
\end{displaymath}
where $k=\card(F)$.  This is easily seen to be independent of the
choice of $\v\in V$.  Call $\op{per}$ the \newterm{perimeter} of the cyclic fan.
If $\v,\w\in V$ are distinct nodes, define the \newterm{partial perimeter}
\begin{displaymath}
  \op{per}(V,E,F,\v,\w) 
= \sum_{i=0}^{r-1} \arc_V(\orz,\{\rho^i \v,\rho^{i+1} \v\}), 
\end{displaymath}
where $r$ is chosen so that $\w=\rho^r \v$ and $0<r\le k-1$.
\end{definition}
\indy{Index}{perimeter!cyclic fan}%
\indy{Index}{fan!cyclic}%
\indy{Notation}{per@$\op{per}$ (perimeter)}%



\begin{lemma}[]\guid{WSEWPCH}\tlabel{lemma:convex-hyp}\rating{400}
The perimeter of every cyclic fan is at most $2\pi$.
\end{lemma}
\indy{Index}{fan!cyclic}%
\indy{Index}{perimeter}%

\begin{proof} 
\claim{If the cyclic fan is circular, then its perimeter is
$\op{per}(V,E,F) =2\pi$.}  Indeed, by Lemma~\ref{lemma:circular},
the arcs making up the perimeter all lie in a common plane.  The
azimuth cycle on $V$ coincides with $\rho:V\to V$.  The sum of the
terms in the formula defining the perimeter is the sum of the
azimuth angles in the azimuth cycle.  The sum is $2\pi$ by
Lemma~\ref{lemma:2pi-sum}.


\claim{if the cyclic fan is lunar, then its perimeter is
$\op{per}(V,E,F) =2\pi$.}  Indeed, by Lemma~\ref{lemma:lunar}, the
set $V$ is contained in the union of two half-planes.  The perimeter
is the sum of arcs in a half-circle in the first half-plane plus the
sum of arcs in a half-circle in the second half-plane. This sum is
$2\pi$.

Finally, assume that the cyclic fan is generic.  Suppose for a
contradiction that the lemma is false.  Consider all counterexamples
that minimize the cardinality of $V$.  Among all such
counterexamples, pick a counterexample with the smallest number of
darts $x\in F$ such that $\op{azim}(x) = \pi$.

A cyclic fan $(V,E,F)$ is determined by $V$ and the cyclic
permutation $\rho:V\to V$: $E=\{\{\v,\rho \v\}\mid \v\in V\}$ and $F
= \{(\v,\rho \v)\mid \v\in V\}$.

In this particular counterexample, if there is any dart
$x=(\v,\w)\in F$ with $\op{azim}(x)=\pi$, then there is a new cyclic
fan $(V',E',F')$ with $V' = V\setminus\{\v\}$ and $\rho':V'\to V'$
given by
\begin{displaymath}
\rho'(\u) = \begin{cases}
\rho(\u), & \text{if } \rho(\u)\ne \v;\\
\rho(\v), & \text{if }\rho(\u) = \v.\\
\end{cases}
\end{displaymath}
This is a cyclic fan with the same perimeter, contrary to the presumed
minimality of the counterexample.  Thus $\op{azim}(x) <\pi$, for all
$x\in F$.

If $\card(V) <3$, then the cyclic fan is circular or lunar, which has
aleady been treated.  If $\card(V)=3$, then $V=\{\v_1,\v_2,\v_3\}$.
By the triangle inequality $\arc_V(\orz,\{\v_2,\v_3\}) \le
\arc_V(\orz,\{\v_2,-\v_1\})+\arc_V(\orz,\{-\v_1,\v_3\})$.  Thus,
\begin{displaymath}
\begin{array}{rll}
  \op{per} &=\arc_V(\orz,\{\v_1,\v_2\}) 
  + \arc_V(\orz,\{\v_2,\v_3\}) 
  + \arc_V(\orz,\{\v_1,\v_3\})\\
  &\le(\arc_V(\orz,\{\v_1,\v_2\})+\arc(\orz,\{\v_2,-\v_1\}))
  \\&\qquad\qquad+(\arc_V(\orz,\{\v_1,\v_3\})
+\arc_V(\orz,\{-\v_1,\v_3\})) \\
  &= \pi+\pi.
\end{array}
\end{displaymath}

Now assume that $\card(V)\ge 4$.  Select $\v\in V$.  Consider a
deformation of the cyclic fan $\varphi:V\times I \to \ring{R}^3$ that
fixes $V\setminus\{\v\}$, and gives motion to $\v$:
\begin{displaymath}
\varphi(\v,t) = \cos(t) \v - \sin(t) \rho \v.
\end{displaymath}
This is increasing in the perimeter.  For sufficiently small $t$, it
remains a generic cyclic fan (Lemma~\ref{lemma:fan-open}).  For
sufficiently small $t$, the minimality gives
$\angle(\varphi(\u,t))<\pi$.  Eventually, for some smallest $t=t_0$
the deformed value is no longer a generic cyclic fan.
% or for some $\u\in V$, $\angle(\varphi(\u,t_0))=\pi$.  In the latter
% case, the minimality condition fails, and the result follows.




\claim{The \case{non-parallel} property holds at $t=t_0$}.  Indeed, it
is enough to check non-parallelism when one of the points is
$\varphi(\v,t_0)$ and the other is $\rho\v$ or $\rho^{-1}\v$.  The set
$\{\orz,\rho\v,\v,\rho^i\v\}$ is not coplanar for $i=-1,-2$, because
otherwise some interior angle is $\pi$, by Lemma~\ref{lemma:A}.  Thus,
the plane through $\{\orz,\v,\rho\v\}$ meets the plane through
$\{\orz,\rho^{-2}\v,\rho^{-1}\v\}$ along a line $L$ through the
origin.  If the set $\{\orz,\rho^{-1}\v,\varphi(\v,t)\}$ is collinear,
then that line is contained
in \begin{displaymath}
\op{aff}\{\orz,\v,\rho\v\}
\cap\op{aff}\{\orz,\rho^{-1}\v,\rho^{-2}\v\}=L,
\end{displaymath}
which is impossible since $\rho^{-1}\v\not\in
\op{aff}\{\orz,\v,\rho\v\}$.  Thus, $\rho^{-1}\v$ and $\varphi(\v,t)$
are non-parallel.

\claim{The line $L$ meets the segment $\op{aff}^0_+(\emptyset,\{\v,-\rho\v\})$.}
Otherwise, it meets $\op{conv}\{\v,\rho\v\}$.  Then $\v$ and $\rho\v$
lie in distinct half-spaces bounded by
$\op{aff}\{\orz,\rho^{-1}\v,\rho^{-2}\v\}$.  Since
$\bWdart(F,\rho^{-1}\v)$ is contained in one of the two half-spaces,
the condition $\v,\rho\v\in \bWdart(F,\rho^{-1}\v)$ fails, and
$(V,E,F)$ is not a cyclic fan.  This contradiction establishes the
claim.

There is a time $t_1$ at which $\{\orz,\rho\v,\varphi(\v,t)\}$ is
collinear, and a time $t_2$ at which $\varphi(\v,t)\in L$.  By the
previous claim, $t_2<t_1$.  At time $t_2$, some interior angle is
$\pi$.  If $t_1\le t_0$, then $t_2 < t_0$, and this contradicts the
observation that all interior angles of the fan are less than $\pi$.
So $t_1> t_0$. Thus, $\varphi(\v,t)$ and $\rho\v$ are
non-parallel. This gives non-parallelism.

\claim{The \case{intersection} property holds.}  If $\ee\subset V$,
write $\ee(t) = \{\varphi(\u,t) \mid \u\in \ee\}$.  Let $W
=W(\orz,\v,\rho\v,\rho^{-1}\v)$. The verification of the intersection
property is based on the following facts (when $t>0$):
\begin{itemize} 
\item If $\v\not\in \ee$, then $C(\ee(t))=C(\ee)\subset W$.
\item $C^0\{\v(t)\} \cap W = \emptyset$.
\item $C^0\{\v(t),\rho^{-1}\v\}\cap W = \emptyset$.
\item $C\{\v(t),\rho\v\}\cap W = C\{\v,\rho\v\}.$
\item $C\{\v(t),\rho\v\}\cap C\{\v(t),\rho^{-1}\v\} = C\{\v(t)\}$.
\end{itemize}

\claim{At time $t=t_0$, the deformed value is a cyclic fan.}
Otherwise, if the object fails to be a fan at time $t=t_0$, then the
\case{non-parallel} property, \case{intersection} property, or the
\case{cyclic} property fails.  The first two properties have already
been checked, so that the deformed value must be a fan.  Furthermore,
the conditions for a fan to be cyclic are closed conditions, so they
must also hold.

At $t=t_0$, the deformed fan is cyclic but not generic.  The perimeter
bound follows from previous cases.
\end{proof}

Here is a second proof of the same lemma.  It is conceptually much
simpler, but possibly more difficult to formalize.  It is based on polar
polygons (a generalization of polar triangles to spherical polygons).

\begin{proof} A fan does not have any faces of cardinality less than
three.  Every blade of the fan has radian measure less than $\pi$.
\indy{Index}{polygon!polar}%

Consider the case of a spherical triangle.  If the edges of the
the triangle are $a_i$ and the angles of the polar
triangle are $\beta_i$, then $\beta_i=\pi-a_i$.
The the perimeter is 
\begin{displaymath}a_1+a_2+a_3 = 2\pi - (\beta_1 -\beta_2 -
\beta_3-\pi)= 2\pi-\op{sol} < 2\pi,\end{displaymath} because the
solid angle $\op{sol}$ of the polar triangle is always strictly
positive.  \indy{Index}{triangle!spherical}%

Similarly, if the edges of the spherical polygon are
$a_i$, then the angles of the polar polygon are $\beta_i = \pi-a_i$.
The perimeter is
\begin{displaymath}
a_1+\cdots+a_n  = 2\pi- \op{sol}< 2\pi,
\end{displaymath}
where $\op{sol} = 2\pi-\sum a_i$ is the solid angle of the polar polygon.
%~\cite[\p.261]{williamson:2008}.
\indy{Notation}{sol $\op{sol}$ (solid angle)}%
\end{proof}


\section{Special Fan}\label{sec:weight}  






\subsection{definition}

\begin{definition}[special~fan,~$\hm$]\guid{FIJJSLP}
Let $\hm = 1.26$.
A \newterm{special fan} is a tuple $(V,E,F,G)$, where
\begin{description}
\item \case{packing} $V$ is a packing.  That is, for every $\v,\w\in
V$, if $\norm{\v}{\w}<2$, then $\v=\w$.
\item \case{annulus} $V\subset \BB$.
\item \case{cyclic fan} $(V,E,F)$ is a cyclic fan.
\item \case{subset} $G\subset E$.
\item \case{g norm} If $\{\v,\w\}\in G$, then $\norm{\v}{\w}=2\hm$.
\item \case{e norm} If $\{\v,\w\}\in E$, then $\norm{\v}{\w}\le 2\hm$.
\item \case{diagonal} For all distinct elements $\v,\w\in V$, if
$\{\v,\w\}\not\in E$, then \begin{displaymath}\norm{\v}{\w}\ge
2\hm.\end{displaymath}
\item \case{card} %$k=\card(F)$,
Let      $s=\card(G)$ and $r=\card(E) - s = \card(F)-s$.  Then
\begin{displaymath}0\le s \le 3,\quad\text{and}\quad3-s \le r \le 6 -
2s.\end{displaymath}
\end{description}
The constants $r,s$ are called the \newterm{parameters} of the special
fan.
\end{definition}
\indy{Index}{special fan}
\indy{Index}{parameters (of a special fan)}
\indy{Notation}{r (special fan parameter)}
\indy{Notation}{s (special fan parameter)}


\begin{definition}[d]\guid{EFWASJQ}
\begin{displaymath}d(r,s) = \begin{cases}
0.103 (2-s) + 0.2759 (r+2s-4), & r + 2s > 3\\
0, & r + 2s \le 3.\\
\end{cases}\end{displaymath}
\end{definition}
%d(3,0)=0, d(4,0)= 0.206; d(5,0)= 0.4819,   .7578
\indy{Notation}{d (lower bound for $\tau$)}

\begin{definition}[$\hm$,~$\tau$,~$\dih_i$]\guid{CUFCNHB}\label{def:tau}
Let $(V,E,F)$ be a cyclic fan.  Set $\hm = 1.26$.  Set
\begin{displaymath}
  \tau(V,E,F) =\sum_{x\in F} \op{azim}(x)\left(1 + \dfrac{\sol_0}{\pi}  
    \dfrac{\normo{\nd(x)}-2}{2\hm-2}\right) 
+ \left(\pi+{\sol_0}\right) (2- k(F)),
\end{displaymath}
where $\sol_0=3\arccos(1/3)-\pi\approx0.551$ is the solid angle of a
spherical equilateral triangle of side $\pi/3$, and $k(F)$ is the
cardinality of $F$.  The function $\tau$ is defined on special fans by
disregarding $G$:
\begin{displaymath}
\tau(V,E,F,G) = \tau(V,E,F).
\end{displaymath}
Let 
\begin{displaymath}
  \tau(y_1,y_2,y_3,y_4,y_5,y_6) =
  \sum_{i=1}^3 \dih_i(y_1,\ldots,y_6)
\left(1 + \dfrac{\sol_0}{\pi}  \dfrac{y_i -2}{2\hm-2}\right) 
- \left(\pi+{\sol_0}\right),
\end{displaymath}
where
\begin{displaymath}
\begin{array}{lll}
\dih_1(y_1,y_2,y_3,y_4,y_5,y_6) &= \dih(y_1,y_2,y_3,y_4,y_5,y_6),\\
\dih_2(y_1,y_2,y_3,y_4,y_5,y_6) &= \dih(y_2,y_3,y_1,y_5,y_6,y_4),\\
\dih_3(y_1,y_2,y_3,y_4,y_5,y_6) &= \dih(y_3,y_1,y_2,y_6,y_4,y_5).\\
\end{array}
\end{displaymath}
\indy{Notation}{h0@$\hm$}
\indy{Notation}{zzt@$\tau$}
\indy{Notation}{sol@$\sol_0$}
\indy{Notation}{dih@$\dih_i$}
\end{definition}



\subsection{compactness}

Let 
\begin{displaymath}
\begin{array}{lll}
V(\p) &= \{\p_{i}\mid i=0,\ldots,k-1\},\\
E(\p) &= \{\{\p_{i},\p_{i+1}\}\mid i=0,\ldots,k-1\},\\
F(\p) &= \{(\p_{i},\p_{i+1}) \mid i=0,\ldots,k-1\},\\
\end{array}
\end{displaymath}
where $\p:\{0,\ldots,k-1\}\to \ring{R}^3$ is any function, and $\p_k =
\p_0$.  (That is, for purposes of indexing, identify
$\{0,\ldots,k-1\}$ with the cyclic group $Z_k$.)  If
$I\subset\{0,\ldots,k-1\}$ then set
\begin{displaymath}G(\p,I) = \{(\p_i ,\p_{i+1}) \mid i\in
I\}\end{displaymath}

\begin{definition}[fan~datum]\guid{PJWMYDB}
 Let $k\in\ring{N}$ and $I\subset
\{0,\ldots,k-1\}$.  A fan datum of shape $(k,I)$ is a function
$\p:\{0,\ldots,k-1\}\to \BB$ such that
\begin{itemize}
\item \case{packing} For every $i,j$, if $\norm{\p_i}{\p_j}<2$, then
$i=j$.
\item \case{i norm} If $i\in I$, then $\norm{\p_i}{\p_{i+1}}=2\hm$.
\item \case{e norm} For all $i$, $\norm{\p_i}{\p_{i+1}} \le 2\hm$.
\item \case{diagonal}  $\norm{\p_i}{\p_j} \ge 2\hm$ when $|i-j|>1$.
\item \case{card}  Let $s=\card(I)$ and $r=k-s$.  Then 
\begin{displaymath}0\le s \le 3,\quad\text{and}\quad3-s \le r \le 6
- 2s.\end{displaymath}
\item \case{angle} $\op{azim}(\orz,\p_i,\p_{i+1},\p_{i-1})\le \pi$ for
all $i$.
\item \case{wedge} $\p_j\in W(\orz,\p_i,\p_{i+1},\p_{i-1})$ for all $i,j$.
\end{itemize}
\end{definition}

\begin{lemma}[]\guid{CKQOWSA}\rating{ZZ}\label{lemma:ctc-fan}
Let $V\subset \BB$ be a packing.  Set 
\begin{displaymath}E_{std} = \{\{\v,\w\}\subset V\mid 0 <
\norm{\v}{\w} \le 2\hm\}.\end{displaymath} Then $(V,E_{std})$ is a
fan.
\end{lemma}
\indy{Notation}{E@$E_{std}$}%

\begin{proof}
The properties \case{cardinality}, \case{origin}, and \case{non-parallel} follow
by the methods of Remark~\ref{remark:fan-verify}.

\case{intersection}: Some geometrical reasoning is required to
establish the intersection property.  The case
\begin{displaymath}
C\{\u\}\cap C\{\v\} = \{\orz\}
\end{displaymath}
follows from the strict triangle inequality 
\begin{displaymath}
\normo{\u} \le 2\hm < 4 \le \normo{\v} + \norm{\u}{\v}.
\end{displaymath}
The other cases of the proof are based on the following 
two facts from Tarski arithmetic.
\begin{itemize}
\item Let $\{\v_0,\v_1,\v_2\}\subset \BB$ be a packing of three points.
Assume that $\norm{\v_1}{\v_2}\le 2\hm$.  Then
$C\{\v_0\}\cap C\{\v_1,\v_2\} = \{\orz\}$.
\item Let $\{\v_0,\v_1,\v_2,\v_3\}\subset \BB$ be a packing of four
points.  Assume that $\norm{\v_1}{\v_3}\le 2\hm$ and
$\norm{\v_2}{\v_4}\le 2\hm$.  Then $C\{\v_1,\v_3\}\cap
C\{\v_2,\v_4\} = \{\orz\}$.
\end{itemize}
\end{proof}

\begin{lemma}[]\guid{VYNCGCO}\rating{ZZ}
If $\p$ is a fan datum of shape $(k,I)$, then
\begin{displaymath}
(V(\p),E(\p),F(\p),G(\p,I))
\end{displaymath}
is a special fan.  Moreover, every special fan is equal to
\begin{displaymath}
(V(\p),E(\p),F(\p),G(\p,I))
\end{displaymath}
for some fan datum $\p$ of some shape $(k,I)$.
\end{lemma}

\begin{proof}
\claim{Every special fan $(V,E,F,G)$ is equal to
$(V(\p),E(\p),F(\p),G(\p,I))$ for some fan datum $\p$ of some
shape $(k,I)$.}  Let $k=\card(F)$.  Fix any element $\u\in V$ and
set $\p_i = \rho^i \u$.  Let $I = \{i\mid (\p_i,\p_{i+1}) \in G\}$.
Clearly $(V,E,F,G)=(V(\p),E(\p),F(\p),G(\p,I))$, and $\p$ is a fan
datum.

Let $\p$ be any fan datum of shape $(k,I)$.
$(V(\p),E(\p))$
is a fan by Lemmas~\ref{lemma:ctc-fan} and \ref{lemma:subset-fan}.


\claim{$(V(\p),E(\p),F(\p))$ is a cyclic fan.} The properties
\case{wedge} and \case{angle} have been built into the definition of a
fan datum.  The azmuth cycle $\sigma(\p_i)$ on $E(\p_i) =
\{\p_{i+1},\p_{i-1}\}$ interchanges $\p_{i+1}$ with $\p_{i-1}$.  From
this fact, the combinatorial properties \case{face} and \case{cyclic
hypermap} are easily determined.  The bijection from the disjoint
union of two copies of $Z_k$ onto the set of darts is given by the
pair of maps
\begin{displaymath}
i\mapsto (\p_i,\p_{i+1}),\quad i\mapsto (\p_{i+1},\p_i).
\end{displaymath}
This bijection extends to an isomorphism of hypermap $\op{Dih}_{2k}$ onto
$\op{hyp}(V(\p),E(\p))$.

\claim{$(V(\p),E(\p),F(\p),G(\p,I))$ is a special fan.} The properties
of a special fan all follow trivially from the corresponding
properties of a fan datum.
\end{proof}

The set of fan data of shape $(k,I)$ is a subspace of the topological
space $\BB^k$.


\begin{lemma}[]\guid{KFIIPLO}\rating{ZZ}
The space of fan data of shape $(k,I)$ is a compact metric space.
Moreover,
\begin{displaymath}
\p \mapsto \tau(V(\p),E(\p),F(\p))
\end{displaymath}
is a continuous function on the space of fan data of shape $(k,I)$.
\end{lemma}

\begin{proof} $\BB^k$ is a compact metric space and the constraints
are all closed conditions.

The function $\tau$ is a polynomial in $\normo{\p_i}$ and
$\op{azim}(\orz,\p_i,\p_{i+1},\p_{i-1})$.  The norm and azimuth
angle are both continuous functions of $\p$.
\end{proof}




\subsection{internal blades}


\begin{lemma}[]\guid{PGSQVBL}\rating{ZZ} Let $(V,E,F)$ be a cyclic fan.
If $\v,\w\in V$ are non-parallel, then $C\{\v,\w\} \subset
\bWdart(x)$ for any dart $x\in F$.
\end{lemma}
\indy{Index}{fan!cyclic}%

\begin{proof} This is an elementary consequence of the definitions,
the cone shape of $\bWdart(x)$, and the condition $V\subset
\bWdart(x)$.
\end{proof}


\begin{lemma}[internal
blades]\guid{YOLCBTG}\rating{ZZ} \label{lemma:internal}
Let $(V,E,F)$ be a cyclic fan.  Let $\v,\w\in V$ be non-parallel.
Suppose that there exists $\v',\w'$ such that
$\angle(\v'),\angle(\w')<\pi$, where $\v,\v',\w,\w'$ are four
distinct elements of $V$ that appear in cyclic order.  Then
$C^0\{\v,\w\}\subset \Wdart(x)$ for all $x\in F$.
% Pick a dart $x=(\v_0,\v_1)\in V$.  Set $\v_j = \rho^j \v_0$.
% Assume that there are four darts $(y_1,y_2,y_3,y_4)$, $y_j =
% x_{j(j)}$, with $0\le j(1) < j(2) < j(3) < i(4)\le k-1$ such that
% $\op{azim}(y_j) < \pi$, for $j=2,4$.  Then
% $C^0\{\v_{i(1)},\v_{i(3)}\} \subset \Wdart(x)$, for all $x\in F$.
\end{lemma}
\indy{Index}{blade!internal}%
\indy{Index}{internal blade}%
\indy{Index}{cyclic order}%

(To say that a sequence $\v_i$ of elements is in \newterm{cyclic
order} means that $\v_i = \rho^{j (i)}\v_0$, for some increasing
function $j$ with range $\{0,\ldots,k-1\}$.)

\begin{proof} Abbreviate $C^0 = C^0\{\v,\w\}$.  The first case to
consider is $\nd(x)=\v$.  For all $\p\in C^0\cap \bWdart(x)$,
\begin{displaymath}
  0 \le \op{azim}(\orz,\v,\rho \v,\p) 
\le \op{azim}(\orz,\v,\rho \v,\rho^{-1} \v).
\end{displaymath}  
These inequalities are in fact strict.  If, for example $0 =
\op{azim}(\orz,\v,\rho \v,\p)$, then the set $\{\orz,\v,\rho \v,\w\}$
is coplanar.  Repeated application of Lemma~\ref{lemma:A} gives
\begin{displaymath}
\angle(\v) = \angle(\rho \v) = \cdots = \angle(\rho^{-1} \w) = \pi,
\end{displaymath}
which is contrary to $\angle(\v') = \pi$.  The strict inequalities
imply $\p\in \Wdart(x)$ as desired.  The case $\nd(x)=\w$ is similar.

Now assume that $\u=\nd(x)\ne \v,\w$.  By Lemma~\ref{lemma:A}, one may
assume that $\{\orz,\u,\v,\w\}$ is not coplanar.  (Otherwise, the
contradiction $\angle(\v')=\pi$ or $\angle(\w')=\pi$ is reached.) Then
\begin{displaymath}
  \op{aff}\{\orz,\u,\v\}\cap C^0 \subset \op{aff}\{\orz,\u,\v\}
  \cap \op{aff}\{\orz,\v,\w\} \cap C^0 
= \op{aff}\{\orz,\v\} \cap C^0 = \emptyset.
\end{displaymath}
Thus, $C^0$ is disjoint from $\op{aff}\{\orz,\u,\v\}$ and is similarly
disjoint from $\op{aff}\{\orz,\u,\w\}$.

We have the following facts:
\begin{displaymath}
\v,\w\in W(\orz,\u,\v,\w),\quad C^0\{\v,\w\} \subset W(\orz,\u,\v,\w).
\end{displaymath}
Also,
\begin{displaymath}
\begin{array}{rll}
  C^0 &= C^0\cap W(\orz,\u,\v,\w) \\
  &\subset C^0 \cap (W^0(\orz,\u,\v,\w) 
\cup \op{aff}\{\orz,\u,\v\} \cup \op{aff}\{\orz,\u,\w\})\\
  &\subset C^0 \cap W^0(\orz,\u,\v,\w)\\
  &\subset \Wdart(x).
  % \v,\w\in \barW &= \{\p \mid 0 
  %\le \op{azim}(\orz,\u,\v,\p) \le \op{azim}(\orz,\u,\v,\w)\}.\\
  % C^0 &\subset \barW\\
  % \bar W &\subset W(\orz,\u,\v,\w) 
  %\cup \op{aff}\{\orz,\u,\v\} \cup \op{aff}\{\orz,\u,\w\}\\
  % W(\orz,\u,\v,\w) & \subset \Wdart(x),
\end{array}
\end{displaymath}
%\begin{displaymath}
%\Wdart(x) = \bWdart(x)
%\setminus (\op{aff}\{\orz,\u,\v\}\cup\op{aff}\{\orz,\u,\w\}).
%\end{displaymath}
\end{proof}

\begin{lemma}[]\guid{TECOXBM}\rating{ZZ}\label{lemma:2hm-slice}
Let $(V,E,F,G)$ be a special fan.  Let $\v,\w\in V$ be distinct
elements such that $\norm{\v}{\w}=2\hm$ and such that
$\{\v,\w\}\not\in E$.  Then $\v$ and $\w$ are non-parallel, and
$C^0\{\v,\w\}\subset \Wdart(x)$ for all $x\in F$.
\end{lemma}

\begin{proof} By Remark~\ref{remark:fan-verify}, non-parallelism follows
from the strict triangle inequality
\begin{displaymath}
\norm{\v}{\w} \le 2\hm < 2 + 2 \le \normo{\v} + \normo{\w}.
\end{displaymath}
Assume for a contradiction that the conclusion of the lemma is false.
Then by Lemma~\ref{lemma:internal}, all the intermediate internal
angles between $\v$ and $\w$ are equal to $\pi$.  As a result, (after
interchanging $\v$ and $\w$ if necessary), the set
$\{\orz,\v,\rho\v,\rho^2\v,\ldots,\rho^r\v\}$ is planar, and
\begin{displaymath}
  \arc_V(\orz,\{\v,\rho^r\v\}) 
= \sum_{i=0}^{r-1} \arc_V(\orz,\{\rho^i\v,\rho^{i+1}\v\}),
\end{displaymath}
where $\rho^r \v = \w$ and $1 < r \le k-1$.
However, the left-hand side is at most
\begin{displaymath}
\arc(2,2,2\hm) < 1.5,
\end{displaymath}
while the right-hand side is at least
\begin{displaymath}
2\arc(2\hm,2\hm,2) > 1.5.
\end{displaymath}
This gives the desired contradiction.
\end{proof}

\subsection{slicing}

\begin{definition}[slice]\guid{CNAQAAA}
 Let $(V,E,F)$ be a cyclic fan.  Assume that
$\v,\w\in V$ are non-parallel and that $(V,E')=(V,E\cup
\{\{\v,\w\}\})$ is a fan.  Let $F'$ be the face of $\op{hyp}(V,E')$
containing the dart $(\w,\v)$.  Write
\begin{displaymath}(V[\v,\w],E[\v,\w],F[\v,\w])\end{displaymath}
for the localization of $(V,E')$ along $F'$.  Explicitly,
\begin{displaymath}
\begin{array}{lll}
  V[\v,\w] &= \{\v,\rho \v,\rho^2 \v,\ldots,\w\};\\
  E[\v,\w] &= \{\{\v,\rho \v\},\ldots,\{\rho^{-1}\w,\w\},\{\w,\v\}\};\\
  F[\v,\w] &= \{(\v,\rho \v),(\rho \v,\rho^2 \v),
 \ldots,(\rho^{-1}\w,\w),(\w,\v)\}.
\end{array}
\end{displaymath}
The triple $(V[\v,\w],E[\v,\w],F[\v,\w])$ is called the
\newterm{slice\/} of $(V,E,F)$ along $(\v,\w)$.
\end{definition}
\indy{Index}{fan!cyclic}%
\indy{Index}{slice}
\indy{Notation}{1@$\cdot[\v,\w]$ (slicing a fan)}

To allow for more than one cyclic fan $(V,E,F)$, expand the notation,
writing $\angle(H,\v)$ for $\angle(\v)$ in the hypermap $H$.
Similarly, write $\Wdart(H,\v)$ for $\Wdart(x)$, and so forth.
\indy{Notation}{azimhv@$\op{azim}(H,\v)$}%
\indy{Notation}{wdart@$\Wdart$}%


\begin{lemma}[slicing]\guid{EJRCFJD}\rating{ZZ}\label{lemma:slice} Let
$(V,E,F)$ be a cyclic fan with hypermap $H$.  Pick $\v,\w\in V$. For
each $\u\in \{\v,\w\}$, assume that $\u$ is not parallel with  any
element of $V\setminus\{\u\}$.  Assume that $C^0\{\v,\w\}\subset
\Wdart(x)$ for all darts $x\in F$.  Then
\begin{itemize}
\item $(V[\v,\w],E[\v,\w],F[\v,\w])$ and
$(V[\w,\v],E[\w,\v],F[\w,\v])$ are cyclic fans.
\item Let $H[\v,\w]$ and $H[\w,\v]$ be their hypermaps, respectively.
Let $g:V\to\ring{R}$ be any function.  Then
\begin{displaymath}
  \sum_{\v\in V} g(\v)\angle(H,\v) 
  = \sum_{\v\in V[\v,\w]}g(\v)\angle(H[\v,\w],\v) 
  + \sum_{\v\in V[\w,\v]}g(\v)\angle(H[\w,\v],\v).
\end{displaymath}
\end{itemize}
\end{lemma}
\indy{Index}{slice}%
\indy{Index}{fan!cyclic}%

\begin{proof} 
\claim{$(V,E')$ is a fan, where $E' = E\cup \{\{\v,\w\}\}$.}
Indeed, except for the intersection property, all of the properties
of a fan follow trivially from the fact that $(V,E)$ is a fan and
that $\v$ and $\w$ are non-parallel.  (Note the similarity with
Lemma~\ref{lemma:add-edge}.)  The intersection property also is
trivial except in the case $\ee=\{\v,\w\}$ and $\ee'\setminus \ee\ne
\emptyset$.  Pick $\u\in \ee'\setminus\ee$.  It follows from the
node partition of Lemma~\ref{lemma:disjoint} that
\begin{displaymath}
\begin{array}{lll}
C(\ee) \cap C(\ee') &= (C(\v) \cap C(\ee')) \cup (C(\w)\cap C(\ee')) \\
&= C(\{\v\}\cap \ee') \cup C(\{\w\}\cap \ee') \\
&= C(\{\v,\w\}\cap \ee').
\end{array}
\end{displaymath}
The intersection property thus holds and $(V,E')$ is a fan.

It follows by Lemma~\ref{lemma:localization} that
$(V[\v,\w],E[\v,\w])$ is a fan.

The second conclusion of the lemma follows from the following identities.
If $\u\ne \v,\w$ with $\u\in V[\v,\w]$, then $\u\not\in V[\w,\v]$ and 
\begin{equation}
\Wdart(H,\u)=\Wdart(H[\v,\w],\u),\quad \angle(H,\u) = \angle(H[\v,\w],\u).
\end{equation}
If $\u\in\{\v,\w\}$, then 
$\angle(H,\u)=\angle(H[\v,\w],\u) +\angle(H[\w,\v],\u)$.

Finally, it remains to be seen that the fan is cyclic.
Lemma~\ref{lemma:localization} already shows that the hypermap is
isomorphic to $\op{Dih}_{2k}$. and that $F[\v,\w]$ can be identified with a
face.  The condition $V[\v,\w]\subset \bWdart(x)$ follows from the
fact that the angles $\beta(i)$ are increasing in
Lemma~\ref{lemma:monotone}.
\end{proof}



The slicing procedure can also be applied to a special fan $(V,E,F,G)$.
Abbreviate the slices as
\begin{displaymath}
\begin{array}{lll}
(V',E',F')&=(V[\v,\w],E[\v,\w],F[\v,\w]),\quad\text{and}\\
(V'',E'',F'')&= (V[\w,\v],E[\w,\v],F[\w,\v]).
\end{array}
\end{displaymath}
Both edge sets $E'$ and $E''$ contain $\{\v,\w\}$.  The sets $G'$ and
$G''$ can be defined as
\begin{displaymath}
\begin{array}{lll}
G' &= \{\{\v,\w\}\} \cup (E'\cap G).\\
G'' &= \{\{\v,\w\}\} \cup (E''\cap G).\\
\end{array}
\end{displaymath} 

Let $(r',s')$ and $(r'',s'')$ be the parameters for $(V',E',F',G')$
and $(V'',E'',F'',G'')$, respectively.  Set $k=r+s$, $k'=r'+s'$, and
$k''=r''+s''$.

\begin{lemma}[]\guid{IXZYRSY}\rating{ZZ}\label{lemma:param-add}  
Let $(V,E,F,G)$ be a special fan (with parameters $s,r$).  Pick
distinct elements $\v,\w\in V$ such that $\{\v,\w\}\not\in E$.
Assume that $\norm{\v}{\w}=2\hm$.  For each $\u\in \{\v,\w\}$,
assume that $\u$ is not parallel with any element of 
$V\setminus\{\u\}$.
%Assume that $C^0\{\v,\w\}\subset \Wdart(x)$ for all darts $x\in F$. 
Then $(V',E',F',G')$ and $(V',E',F',G')$ are special fans.  Moreover,
the parameters satisfy the relations
\begin{displaymath}
k'+k'' = k + 2,\quad s'+s'' = s + 2,\quad r'+r''=r.
\end{displaymath}
Finally,
\begin{displaymath}
\tau(V,E,F)= \tau(V'',E'',F'') +\tau(V',E',F').
\end{displaymath}
\end{lemma}

By interchanging $\v$ and $\w$, the lemma also asserts that
$(V'',E'',F'',G'')$ is a special fan.

\begin{proof} Each of the defining properties of a special fan will be
considered in turn.  By Lemma~\ref{lemma:2hm-slice}, if
$\{v,w\}\not\in E$, then $C^0\{v,w\}\subset \Wdart(x)$ for all darts
$x\in F$.

The properties \case{packing}, \case{annulus}, \case{diagonal},
\case{subset}, \case{g norm}, and \case{e norm} follow directly from
definitions and the corresonding properties for $(V,E,F,G)$.  The
property \case{cyclic fan} follows from Lemma~\ref{lemma:slice}.

The relations between the constants $k,s,r$ for the various fans
follows directly from the construction.  For example, there are two
darts more in $F''\cup F'$ than in $F$.

\claim{Property~\case{card} holds.}  Indeed, recall that each face
in the hypermap of a fan has at least three darts.  Hence $3\le k$,
$3\le k'$, and $3\le k''$.  The sets $G'$ and $G''$ contain
$\{\v,\w\}$.  Hence $1\le s'$ and $1\le s''$.  From $0\le r\le 6 -
2s$, it follows that $s\le 3$.  If $s=3$, then $r=0$, and $k=r+s=3$.
This means that $F$ is a triangle.  For some ordering of the pair,
$x=(\v,\w)$ is a dart in $F$.  Hence $C^0\{\v,\w\} \not\subset
\Wdart(x)$.  This is contrary to hypothesis.  Therefore, $s\le 2$.

Now for the verifications.
\begin{displaymath}0\le s' = s + 2 - s'' \le s+1\le 3.\end{displaymath}
\begin{displaymath}3-s'\le k'-s' = r'.\end{displaymath}
\begin{displaymath}k' = k + 2 - k'' \le k-1.\end{displaymath}
\begin{displaymath}
  r'= k'+s' - 2 s' \le (k-1) + (s+1) - 2s' 
  =k+s - 2s' = r + 2s -2s' \le 6 - 2s'.
\end{displaymath}
This completes the proof of property~\case{card}.

The additivity of $\tau$ follows directly from the azimuth angle
estimates in Lemma~\ref{lemma:slice}.
\end{proof}


\section{Minimality}

%\subsection{minimality}


\begin{definition}[minimal~fan,~$k_{min}$,~$\tau^d_{min}$]\guid{DJISMZB}
Let $k_{min}$ be the minimum of $r+s$ over
all special fans $(V,E,F,G)$ such that 
\begin{equation}\label{eqn:kmin}
\tau(V,E,F) < d (r,s),
\end{equation}
where $(r,s)$ are the parameters of the special fan.  If no special
fan exists that satisfies the inequality, then set $k_{min}=0$.  Let
$\tau^d_{min}$ be the infimum of $\tau(V,E,F)-d(r,s)$ over all special
fans $(V,E,F,G)$ whose parameters $(r,s)$ satisfy $r+s=k_{min}$.  If
no special fans exist with parameters $r+s=k_{min}$, then set
$\tau^d_{min}=0$.  Any special fan $(V,E,F,G)$ with parameter
$r+s=k_{min}$ such that
\begin{displaymath}
\tau^d_{min}= \tau(V,E,F)-d(r,s)
\end{displaymath}
is a \newterm{minimal}  fan.
\end{definition}
\indy{Index}{fan!minimal}
\indy{Index}{minimal fan}


\begin{lemma}[]\guid{ADKOXQY}\rating{ZZ}\label{lemma:c-bound}
There is a constant $c>2\hm$ such that for every minimal fan
$(V,E,F,G)$ and every distinct $\v,\w\in V$, either $\{\v,\w\}\in E$
or $\norm{\v}{\w}\ge c$.
\end{lemma}

\begin{proof} Pick a sequence of fan data $\p$ such that the associated
fan is a minimal fan and such that 
\begin{displaymath}
\min_{|i-j|>1} \norm{\p_i}{\p_j}
\end{displaymath}
is tending to the minimal value $c$.  By passing to s subsequence, we
may assume without loss of generality that every term in the sequence
has the same shape $(k_{min},I)$ for some $I$.  The sequence then lies
in a compact metric space.  Passing again to a subsequence, we may
assume without loss of generality that the sequence converges.  The
limiting value is a fan datum whose associated minimal fan $(V,E,F,G)$
satisfies
\begin{displaymath}
\norm{\v}{\w}=c,
\end{displaymath}
for some $\v,\w\in V$ such that $\{\v,\w\}\not\in E$.

Suppose for a contradiction that $c=2\hm$.  Then by
Lemma~\ref{lemma:2hm-slice}, $C^0\{\v,\w\}\subset \Wdart(x)$, for all
$x\in F$.  The fan can be sliced along the $\{\v,\w\}$.  The
parameters for the new pieces satisfy:
\begin{displaymath}
k' = k+2 - k'' \le k-1.
\end{displaymath}
The function $d$ is additive over slices.  For some constants $d_1$
and $d_2$,
\begin{equation}\label{eqn:drs}
\begin{array}{lll}
d(r,s) &= d_1 (2 - s) + d_2 (r + 2 s-4) \\
&= d_1 (2-s') + d_2 (r'+2 s'-4) + d_1 (2-s'') + d_2 (r''+2s''-4)\\
&= d(r',s') + d(r'',s''). \\
\end{array}
\end{equation}
By Lemma~\ref{lemma:param-add}, one of the two resulting fans
satisfies inequality~\ref{eqn:kmin}.  This is contrary to the choice
of $k_{min}$ in the definition of minimality.
\end{proof}





\subsection{genericity}

\begin{lemma}[]\guid{RRAJQBH}\rating{ZZ}\label{lemma:circular-nonmin}
Every minimal fan is generic.
\end{lemma}

\begin{proof}
By the classification of cyclic fans, it is enough to show that the
fan is not circular and not lunar.

\claim{In a minimal fan, $\op{azim}(x)<\pi$ for some $x\in F$.}
Otherwise, the result follows from the following estimates:
\begin{displaymath}
\begin{array}{lll}
  \tau(V,E,F) &=\sum_{x\in F} \op{azim}(x)
  \left(1 + \dfrac{\sol_0}{\pi}  \dfrac{\normo{v}-2}{2\hm-2}\right) 
+ \left(\pi+{\sol_0}\right) (2- k(F))\\
  &\ge \sum_{x\in F} \pi + \left(\pi+{\sol_0}\right) (2- k(F))\\
  &= 2\pi + 2\sol_0 - k(F) \sol_0\\
  &\ge 2\pi - 4\sol_0\\
  &> 0.7578\\
  &=0.103 (2) + 0.2759 (2)\\
  &\ge 0.103 (2-s) + 0.2759 (r+2s-4) \\ 
  &= d(r,s)\\
\end{array}
\end{displaymath}
This proves the claim.


Section~\ref{sec:deformation} develops a theory of deformations of
cyclic fans $(V,E,F)$.  Recall that a deformation is continuous
function $\varphi:V\times I\to\ring{R}^3$.  A deformation determines
sets $V(t)$, $E(t)$, $F(t)$, and $G(t)$ for each $t\in I$.


\claim{A minimal fan is not lunar.}  Otherwise, suppose for a
contradiction, that $(V,E,F)$ is lunar, special, and minimal.  Let
$\v,\w\in V$ be parallel.  By the previous claim, we may assume
without loss of generality that $\angle(\v)=\angle(\w)<\pi$.
Example~\ref{example:lunar} describes a deformation of the lunar fan
$(V,E,F)$ over $I=\leftclosed0,1\rightopen$.  The deformation fixes
$\normo{\u}$ and is non-increasing in $\angle(\u(t))$, for $\u\in V$.
From the defining formula for $\tau$, it follows that the function
$\tau(V(t),E(t),F(t))$ is a decreasing function of $t$.

\claim{For sufficiently small positive $t$, the deformed value is a
special fan.}  The deformed value is a cyclic fan for small positive
$t$ by Lemma~\ref{lemma:fan-open}.  The property~\case{diagonal} of
special fans holds by Lemma~\ref{lemma:c-bound}.  The other properties
of a special fan follow immediately from the construction.

However, the function $\tau$ attains its minimum at $(V,E,F)$ and
hence $\tau(V(t),E(t),F(t))$ cannot be decreasing in $t$.  This
contradiction proves the claim that a minimal fan is not lunar.
\end{proof}

\begin{lemma}[]\guid{QAGHDMN}\rating{ZZ}
Let $(V,E,F)$ be a cyclic fan.  If $V$ is contained in a plane
through the origin, then $(V,E,F)$ is not generic.
\end{lemma}

\begin{lemma}[]\guid{EAEKYHM}\rating{ZZ}\label{lemma:3-nonflat}
Every non-generic cyclic fan $(V,E,F)$ has at least three nonflat
elements in $V$.
\end{lemma}

\begin{proof}
If there is at most one nonflat element $\v\in V$, then there exists
a plane $A$ through the origin that contains $V$.  This is not
generic.

If there are exactly two elements nonflat elements in $V$, then
there exist two half-planes through the origin whose union contains
$V$.  The two non-flat points necessarily lie along the intersection
of the two planes.  They are parallel elements of a lunar fan.
This is not generic.
\end{proof}

\subsection{irreducibility}

\begin{definition}[extremal,~minimal]\guid{VLEKVIO}
 Let $(V,E,F,G)$ be a special fan
and let $\{\v,\w\}\in E$.  The edge $\{\v,\w\}\in E$ is
\newterm{G-extremal} if $\{\v,\w\}\in G$ or
\begin{displaymath}
\norm{\v}{\w}\in\{2,2\hm\}.
\end{displaymath}
The edge $\{\v,\w\}\in E$ is \newterm{G-minimal} if $\{\v,\w\}\in G$ or
\begin{displaymath}
\norm{\v}{\w}=2.
\end{displaymath}
\end{definition}
\indy{Index}{extremal ($G$-extremal edge)}
\indy{Index}{minimal ($G$-minimal edge)}

\begin{definition}[irreducible]\guid{IDOTAPN}
 Let $(V,E,F,G)$ be a special fan and
let $\v\in V$.  Write $\v_i = \rho^i \v$.  The special fan is
\newterm{irreducible} at $\v$ if the following properties hold.
\begin{itemize}
\item \case{card} %$k=\card(F)$,
Let      $s=\card(G)$ and $r=\card(E) - s = \card(F)-s$.  Then
\begin{displaymath}0\le s \le 3,\quad\text{and}\quad3-s \le r \le 6 -
2s.\end{displaymath}
\item \case{extreme edge} For every $\w\in E(\v)$, the edge
$\{\v,\w\}\in E$ is $G$-extremal.
\item \case{flat exists} If none of $\v_1,\v_2,\v_3,\v_4$ is flat,
then $r+s\le 4$ and $\norm{\u}{\w}=2$ for all $\{u,w\}\in E$.
% $r+s=4$. %, and $\norm{\u}{\w}\in\{ 2,2\hm}$ for all $\{\u,\w\}\in E$.
\item \case{no triple flat} At least one of $\v_0$, $\v_1$, and $\v_2$
is not flat.
\item \case{balance} If $\v$ is flat and if $\{\u,\v\},\{\w,\v\}\in
E\setminus G$, then
\begin{displaymath}
\norm{\u}{\v} = \norm{\w}{\v}.
\end{displaymath}
\item \case{g flat} If $\v_1$ and $\v_2$ are both flat, then 
\begin{displaymath}G\cap
\{\{\v_0,\v_1\},\{\v_1,\v_2\},\{\v_2,\v_3\}\} =
\emptyset.\end{displaymath}
\item \case{flat middle} If $\v_1$ and $\v_2$ are both flat, then
\begin{displaymath}
\norm{\v_1}{\v_2} = 2.
\end{displaymath}
\item \case{minimal node} Either $\normo{\v}=2$ or there exists $\w\in
E(\v)$ such that $\{\v,\w\}$ is $G$-minimal.
\item \case{minimal node flat} If $\v$ is flat, and $\{\u,\v\}\in
E\setminus G$, then $\norm{\u}{\v}=2$ or $\normo{\v}=2$.
\item \case{flat extremal} If $\v_1$ and $\v_2$ are both flat, then
either $\v_1$ or $\v_2$ has extremal norm:
\begin{displaymath}\normo{\v_1}\in \{2,2\hm\}\quad\text{ or }\quad
\normo{\v_2}\in \{2,2\hm\}.\end{displaymath}
% \item \case{extremal node} If $\v_0$, $\v_1$, and $\v_2$ are not
%   flat, then
%\begin{displaymath}
%\normo{\v_1}\in \{2,2\hm\}.
%\end{displaymath}
% \item \case{flat extremal node} If $\v_1$ is flat, but none of
%   $\v_0,\v_2,\v_3$ is flat, then
%\begin{displaymath}
%\normo{\v_1}\in \{2,2\hm\}
%\quad\text{ or }\quad\normo{\v_2}\in \{2,2\hm\}.
%\end{displaymath}
% \item \case{flat extremal node sym} If $\v_2$ is flat, but none of
%   $\v_0,\v_1,\v_3$ is flat, then
%\begin{displaymath}
%\normo{\v_1}\in \{2,2\hm\}
%\quad\text{ or }\quad\normo{\v_2}\in \{2,2\hm\}.
%\end{displaymath}
\item \case{flat count} There are at least $3$ nonflat elements of $V$.
% XX add: \item \case{no triangle} $\op{card}(F) \ne 3$.
% XX add: \item \case{no quadrilateral} $\op{card}(F) \ne 4$.
% XX add: \item \case{no pentagon} $\op{card}(F) \ne 5$.
% XX add: \item \case{no hexagon} $\op{card}(F) \ne 6$.
\end{itemize}
\end{definition}
\indy{Index}{irreducible}



\begin{lemma}[]\guid{KDKAGRS}\rating{ZZ}\label{lemma:min-irred}
Every minimal fan $(V,E,F,G)$ is irreducible at every $\v\in V$.
\end{lemma}

The proof appears in the next subsection as a series of small
verifications.

\subsection{features}

If there are $i$ consecutive flat nodes $\v_1,\ldots,\v_i$, then there
are $i+2$ corresponding nodes that lie in a plane $A$ through the
origin: $\{\v_0,\ldots,\v_{i+1}\}\subset A$.

\begin{lemma}[]\guid{FPITROS}\rating{ZZ}
Every minimal fan $(V,E,F,G)$ satisfies property \case{card} of
irreducibility.
\end{lemma}

\begin{proof} This holds by the property \case{card} of special fans.
\end{proof}


\begin{lemma}[]\guid{TESVAFW}\rating{ZZ}
Every minimal fan $(V,E,F,G)$ satisfies property \case{no triple
flat} of irreducibility at every $\v\in V$.
\end{lemma}

\begin{proof}  Three flat nodes produces an arc
\begin{displaymath}
  \arc_V(0,\{\v_0,\v_4\}) 
= \op{per}(V,E,F,\v_0,\v_4) \ge4\arc(2\hm,2\hm,2) > \pi,
\end{displaymath}
but the remaining edges  have combined length at most
\begin{displaymath}
\arc_V(0,\{\v_0,\v_4\})\le 2\arc(2,2,2\hm) < \pi.
\end{displaymath}
\end{proof}

\begin{lemma}[]\guid{GOKZLRP}\rating{ZZ}
Every minimal fan $(V,E,F,G)$ satisfies property \case{flat count}
of irreducibility for every $\v\in V$.
\end{lemma}

\begin{proof}  
This lemma is a consequence of Lemma~\ref{lemma:3-nonflat}.
\end{proof}



\begin{lemma}[]\guid{SDCCMGA}\rating{ZZ}
Every minimal fan $(V,E,F,G)$ satisfies property \case{g flat} of
irreducibility at every $\v\in V$.
\end{lemma}

\begin{proof} Argue by contradiction.  From the constraints on $r$ and
$s$, if $G\ne\emptyset$, then $s>0$ and $r+s\le 5$.

By property \case{flat count}, it follows that $r+s> 4$.
% then $V = \{\v,\rho \v,\rho^2\v,\rho^3\v\}$.  A plane $A$ contains
% $V\cup\{\orz\}$.  This is not generic.

Thus, $r+s=5$.  A plane $A$ contains
$\{\orz,\rho^{-1}\v,\v,\rho\v,\rho^2\v\}$.  We obtain a contradiction
by computing $\arc_V(\orz,\rho^{-1}\v,\rho^2\v)$ in two ways.  On the
one hand, it is equal to the partial perimeter from $\rho^{-1}\v$ to
$\rho^2 \v$.  One the other hand, it is estimated by the triangle
inequality, applied to the other two edges:
\begin{displaymath}
\begin{array}{lll}
\arc_V(\rho^{-1}\v,\rho^2\v)
&\le\arc_V(\rho^2\v,\rho^3\v) + \arc_V(\rho^3\v,\rho^{-1}\v) \\
&\le\arc(2,2,2\hm)+\arc(y,2,2\hm)\\
&<\arc(y,2\hm,2)+\arc(2\hm,2\hm,2) +\arc(2\hm,2\hm,2\hm)\\
&\le\arc_V(\rho^{-1}\v,\rho^2\v).
\end{array}
\end{displaymath}
where $y=\normo{\rho^{-1} \v}$.
This is a contradiction.
\end{proof}

\begin{lemma}[]\guid{BAWWPPB}\rating{ZZ}
Every minimal fan $(V,E,F,G)$ satisfies property \case{extreme edge}
of irreducibility for every $\v\in V$.
\end{lemma}

\begin{proof} 
By Lemma~\ref{lemma:3-nonflat}, there are at least three flat
elements of $V$.

Assume for a contradiction that the edge $\{\v,\rho\v\}\in
E\setminus G$ of the minimal fan $(V,E,F,G)$ is not extremal.  Let
$\u = \rho^i\v$, where $i$ is the smallest positive index for which
$\rho^i\v$ is not flat.  Let $r$ be the largest negative index such
that $\rho^r\u$ is non-flat.  Let $s$ be the smallest positive index
such that $\rho^s\u$ is non-flat.

Define a deformation $\varphi$ that satisfies the following conditions.
\begin{itemize}
\item The deformation fixes every element of
$V\setminus\{\rho^{r+1}\u,\ldots,\rho^{s-1}\u\}$.
\item The deformation  fixes the norm $\normo{\w(t)}$ for all $\w\in V$.  
\item The deformation fixes the norms $\norm{\w(t)}{\varphi(\rho
\w,t)}$ for all $\w\in V\setminus\{\v\}$.
\item The set
$\{\orz,\rho^r\u,\varphi(\rho^{r+1}\u,t),\ldots,\varphi(\u,t)\}$ is
coplanar.
\item The set
$\{\orz,\varphi(\u,t),\varphi(\rho\u,t),\ldots,\rho^s\u\}$ is
coplanar.
\item $\norm{\varphi(\v,t)}{\varphi(\rho\v,t)} = \norm{\v}{\rho\v}+t$.
\end{itemize}
There is a unique deformation that satisfies these properties.  The
deformation is a special fan for sufficiently small $t$.

\claim{The function $t\mapsto \tau(V(t),E(t),F(t))$ does not have a
local minimum at $t=0$.}  Indeed, by Definition~\ref{def:tau},
$\tau$ has the form
\begin{displaymath}
  g(s) = \dih(2,2,2,a+s,b,c) e_1 
+ \dih(2,2,2,b,c,a+s) e_2 + \dih(2,2,2,c,a+s,b) e_3,
\end{displaymath}
for some $e_i\in\leftclosed1,1+\sol_0/\pi\rightclosed$, some
reparametrization $s=s(t)$ of $t$ such that $s(0)=0$, and some
parameters $a,b,c$.  The parameters $a,b,c$ lie in
$\leftclosed2/\hm,4\rightclosed$ and satisfy $\Delta\ge0$, where
$\Delta = \Delta(4,4,4,a^2,b^2,c^2)$.

If $g$ has a local minimum at $s=0$, then
\begin{equation}\label{eqn:g''}
\Delta g'(0)^2 - 0.01\Delta^{3/2} g''(0) \le 0.
\end{equation}
Indeed, the first derivative is zero and the second derivative is
non-negative.  However, Calculation~\ref{calc:Lexell} shows that the
left-hand side of inequality~(\ref{eqn:g''}) is positive for all
parameters $a,b,c,e_1,e_2,e_3$.  Thus, there is no local minimum.

The claim contradicts that assumed minimality of $(V,E,F,G)$.  This
completes the proof.
\end{proof}


\begin{lemma}[]\guid{PZFSEVR}\rating{ZZ}
Every minimal fan $(V,E,F,G)$ satisfies property \case{minimal node}
of irreducibility for every $\v\in V$.
\end{lemma}

\begin{proof} If the conclusion is false, then for sufficiently small
positive $a$, the deformation $\varphi$ over $[0,a]$ given by
\begin{displaymath}
\varphi(\u,t) =
\begin{cases}
\u, & \text{if }\u \ne \v,\\
(1-t) \v, & \text{if }\u = \v\\
\end{cases}
\end{displaymath}
decreases $\tau$.  It is a deformation of special fans.  This shows
that $(V,E,F,G)$ is not minimal.
\end{proof}

\begin{lemma}[]\guid{JLXMEJN}\rating{ZZ}   %case minimal node flat
Every minimal fan $(V,E,F,G)$ satisfies property \case{minimal node
flat} of irreducibility for every $\v\in V$.
% \case{minimal node flat} If $v$ is flat, and $\{u,v\}\in
% E\setminus G$, then $\norm{\u}{\v}=2$ or $\normo{\v}=2$.
\end{lemma}

\begin{proof} 
Let $A$ be the plane containing $\{\orz,\rho^{-1}\v,\v,\rho\v\}$.
Assume that $\{\u,\v\}\in E\setminus G$.  Let $\u\ne \w\in E(\v)$.
Then $\u,\v,\w\in A$.

If the conclusion is false, then for sufficiently small positive
$a$, let $\varphi$ be the deformation over $[0,a]$ that fixes all
elements of $V$ other than $\v$ and moves $\v$ in the plane $A$
around the circle of radius $\norm{\w}{\v}$ with center $\w$.  The
direction of the circular motion can be chosen to decrease $\tau$.
It is a deformation of special fans.  This shows that $(V,E,F,G)$ is
not minimal.
\end{proof}

\begin{lemma}[]\guid{TPKKQOL}\rating{ZZ}
Every minimal fan $(V,E,F,G)$ satisfies property \case{balance}
of irreducibility at every $\v\in V$.
%For every edge $\{\v,\w\}\in E$ in a minimal fan $(V,E,F,G)$, 
%\begin{displaymath}\norm{\v}{\w}\in\{2,2\hm\}.\end{displaymath}
\end{lemma}

\begin{proof} Assume for a contradiction that the property fails.  In
view of properties \case{extreme edge} and \case{minimal node flat},
after possibly swapping $\u$ and $\w$, we may assume without loss of
generality that
\begin{displaymath}
\normo{\v}=2,\quad \norm{\u}{\v}=2,\quad \norm{\w}{\v}=2\hm.
\end{displaymath}
For sufficiently small positive $a$, let $\varphi$ be the deformation
over $[0,a]$ that fixes all elements of $V$ other than $\v$ and moves
$\v$ in the plane of $\{\orz,\u,\v,\w\}$ around a circle of radius $2$
with center at the origin.  The direction of the deformation can be
chosen to be increasing in $\norm{\u}{\v}$.  The function $\tau$ is
constant along the deformation.  The deformation carries minimal fans
to minimal fans.  However, the deformed fan does not satisfy property
\case{extreme edge}.  This is a contradiction.
\end{proof}




\begin{lemma}[]\guid{CFJSRQH} Every minimal fan $(V,E,F,G)$ has property
\case{flat middle} at every $\v\in V$.
\end{lemma}

\begin{proof} %Write $\v_i = \rho^{i-2} \v$.  
Assume for a contradiction tht the conclusion is false.  There is a
plane $A$ that contains $\{\orz,\v_0,\v_1,\v_2,\v_3\}$.  By property
\case{g flat}, none of the edges $\{\v_0,\v_1\}$, $\{\v_1,\v_2\}$,
$\{\v_2,\v_3\}$ lies in $G$.  Furthermore, by \case{extreme edge}
and \case{balance}, the distances $\norm{\v_i}{\v_{i+1}}$ are equal
to one another and take values in $\{2,2\hm\}$, for $i=0,1,2$.  By
\case{minimal node}, if these distances are $2\hm$, then the partial
perimeter from $\v_1$ to $\v_4$ is at least
\begin{equation}\label{eqn:3side}
\arc(y_1,2,2\hm)+\arc(2,2,2\hm)+\arc(2,y_4,2\hm),
\end{equation}
where $\normo{\v_i}=y_i$.  However, there are at most two other nodes
($\v_4$ and $\v_5$) and three other edges in the special fan.  By the
triangle inequality, the sum of these three lengths is at most
\eqn{eqn:3side}.  Thus, equality is obtained in the triangle
inequality.  This implies that $\v_4$ and $\v_5$ lie in the plane $A$.
Thus, $V\subset A$.  This is not generic.
\end{proof}

\begin{lemma}[]\guid{OUCPLRI} 
Every minimal fan $(V,E,F,G)$ satisfies property \case{flat
extremal} of irreducibility at every $\v\in V$.
\end{lemma}

\begin{proof}
Assume there are two adjacent flat elements $\v_1,\v_2$, so that
there is a plane $A$ that contains $\{\orz,\v_0,\v_1,\v_2,\v_3\}$.
By previously established properties of irreducibility,
\indy{Index}{angle!flat}%
$\norm{\v_i}{\v_{i+1}}=2$, for $i=0,1,2$.  Let $y_i = \normo{\v_i}$.
%

\claim{There exists a deformation that fixes all $\u\ne \v_1,\v_2$,
maintains flatness at $\v_1$ and $\v_2$, maintains the coplanarity
of $\{\orz,\v_0,\v_1(t),\v_2(t),\v_3\}$, and that keeps $y_1+y_2$,
$\norm{\v_0}{\v_1}$, $\norm{\v_2}{\v_3}$, and $\tau$ constant,
while increasing $\norm{\v_1}{\v_2}$.}  Indeed, the partial
perimeter $c$ is given by a sum of three terms:
\begin{displaymath}
c=\sum_{i=0}^2\arc(y_i,y_{i+1},2).
\end{displaymath}
Set
\begin{displaymath}
  g(t_1,t_2) = \arc(y_0,y_1+t_1,2) 
+ \arc(y_1+t_1,y_2-t_1,2+t_2) + \arc(y_2-t_1,y_3,2) - c.
\end{displaymath}
Then $g(0,0)=0$.  

Use subscript notation for partial derivatives of $g$.  If $g_1(0,0)
\ne 0$, then by the implicit function theorem there is a function $h$
locally near $0$ such that $h(0)=0$ and $g(h(t_2),t_2)=0$.  Define a
deformation that fixes $\u\ne\v_1,\v_2$ and moves $\v_1,\v_2$ in the
fixed plane $\op{aff}\{\orz,\v_1,\v_2\}$ subject to the constraints
$\norm{\v_1(t)}{\v_2(t)}=2+t$, $\normo{\v_1(t)} = 2+h(t)$, and
$\normo{\v_1(t)} = 2-h(t)$.  These conditions uniquely determine the
deformation.  The deformed value is a special fan for sufficiently
small positive $t$.  It keeps $\tau$ fixed.  Hence the deformed fan is
also minimal.  However, the existence of this deformation contradicts
property \case{flat middle}.

On the other hand, if $ g_1(0,0) =0$, then a calculation gives
\begin{displaymath}g_2(0,0) = \dfrac{2}{y_1y_2\sqrt{\ldots}} >
0.\end{displaymath} Again by the implicit function theoreom there is
a function $h$ locally near $0$ such that $h(0)=0$ and
$g(t_1,h(t_1))=0$.  Define a deformation that fixes $\u\ne \v_1,\v_2$
and moves $\v_1$ and $\v_2$ within the fixed plane
$\op{aff}\{\orz,\v_1,\v_2\}$ by the conditions $\normo{\v_1(t)} = 2 +
t$, $\normo{\v_2(t)} = 2 - t$, and $\norm{\v_1(t)}{\v_2(t)}=2+h(t)$.
These conditions uniquely determine the deformation.  The deformed fan
is special for sufficiently small $t$.  It keeps $\tau$ fixed.  Hence
the deformed fan is also a minimal.

By implicit differentiation, $h'(0) = 0$ and $g_2(0,0) h''(0) =
-g_{11}(0,0)$.  Thus, $h''(0)$ and $g_{11}(0,0)$ have opposite signs.
The function $t_1\mapsto g(t_1,0)$ equals, up to a constant, the sum
of three terms of the form $\arc(\cdot,\cdot,2)$.
Calculation~\ref{calc:2der} shows that the second derivative of each
term with respect to $t_1$ is negative at $t_1=0$.  Thus, $h''(0)>0$
and for sufficiently small nonzero $t$, $\norm{\v_1(t)}{\v_2(t)}>2$.
However, the existence of this deformation contradicts property
\case{flat middle}.
\end{proof}







%\begin{lemma}[]\guid{DWXPIHA}\rating{ZZ}
%  Every minimal fan $(V,E,F,G)$ satisfies property \case{extremal
%    node} for every $\v\in V$.
%\end{lemma}
%
%\begin{proof} %\guid{DFSLRHA} 
%  Assume for a contradiction that the conclusion is false.  By
%  property~\case{extremal edge},
%\begin{displaymath}
%\norm{\v_{-1}}{\v_0},~\norm{\v_0}{\v_1}\in \{2,2\hm\}.
%\end{displaymath}
%Consider the deformation $\varphi:V\times I\to V$ given by
%\begin{displaymath}
%\varphi(\u,t) =
%\begin{cases}
%\v(t), & \text{if } \u=\v\\
%\u, & \text{otherwise},\\
%\end{cases}
%\end{displaymath}
%where $\v(t)$ is the unique point in $\op{aff}_+(\{0,\rho^{-1}\v,\rho
% \v\},\v)$ determined by the conditions:
%\begin{displaymath}
%\norm{\v(t)}{\u} = \norm{\v}{\u}, \text{~when~} \{\u,\v\}\in E,\quad
%\text{~and~}
%\normo{\v(t)} = \normo{\v} + t.
%\end{displaymath}
%For sufficiently small $t$, the deformed value  is a special fan.
%The function $t\mapsto\tau(V(t),E(t),F(t))$ has the form
%\begin{equation}\label{eqn:tau}
%t\mapsto \tau(y_1+t,y_2,y_3,y_4,y_5,y_6) + c
%\end{equation}
%for some parameters $y_1,\ldots,y_6$ and $c$.
%Calculation~\ref{calc:cc:d2a} shows that the function (\ref{eqn:tau})
%has no local minimum at $t=0$.
%This is contrary to the minimality of $(V,E,F,G)$.
%\end{proof}



%The same can be done when there is one flat node forming a linear
% series of length $2$, with compressed edges:

% \begin{lemma}[]\guid{DCEETTF}\rating{ZZ} Every minimal fan
%   $(V,E,F,G)$ satisfies properties \case{flat extremal node} and
%   \case{flat extremal node sym} for every $\v\in V$.
%\end{lemma}
%
%\begin{proof}
%  The property \case{flat extremal node sym} is obtained from
%  \case{flat extremal node} by symmetry $v_1\leftrightarrow v_2$, so
%  it is enough to prove \case{flat extremal node}.  Assume for a
%  contradiction that property \case{flat extremal node} fails.  Let
%  $(V,E,F,G)$ be a minimal fan that fails.
%
%  Define a deformation $\varphi$ of the minimal fan that fixes $\u\ne
%  \v_1,\v_2$ and that moves $\v_1,\v_2$ subject to the following
%  constraints.
%\begin{itemize}
%\item $\{\orz,\v_0,\v_1(t),\v_2(t)\}$ is coplanar.
%\item $\norm{\v_2(t)}{\v_3}$ is constant.
%\item $\norm{\v_0}{\v_1(t)}$, $\norm{\v_1(t)}{\v_2(t)}$, and
%  $\norm{\v_0(t)}{\v_2(t)}$ are constant.
%\item $\normo{\v_2(t)} = \normo{\v_2} + t$.
%\end{itemize}
%These constraints uniquely determine the deformation. The deformed
% value is a special fan for sufficiently small $t$.
%
% By Calculation~\ref{calc:cc:d2b}, the function does not have a local
% minimum at $t=0$.  This contradicts the assumption that $(V,E,F,G)$
% is minimal.
%\end{proof}



\begin{lemma}[]\guid{CMBZAOZ}\rating{ZZ}
Every minimal fan $(V,E,F,G)$ satisfies property \case{flat exists}
of irreducibility for every $\v\in V$.  That is, if none of $\v,\rho
\v,\rho^2 \v,\rho^3 \v\in V$ is flat, then $r+s\le 4$ and
$\norm{\u}{\w}=2$ for all $\{u,w\}\in E$.
%then $r+s=4$. %, and $\norm{\v}{\w}= 2$, for all $\{\v,\w\}\in E$.
\end{lemma}

\begin{proof} 
Assume for a contradicton that the property fails for $(V,E,F,G)$.
In particular, the parameters satisfy $r+s>4$.  There are four
consecutive nodes $\v_1,\v_2,\v_3,\v_4$ that are not flat.  By
property \case{extreme edge}, each edge $\norm{\v_i}{\v_{i+1}}$ is
$2$ or $2\hm$. Set $y_i = \normo{\v_i}$ and $y_{ij} =
\norm{\v_i}{\v_j}$.
%By \case{extremal node}, $y_i$ is $2$ or $2\hm$, for $i=2,3$.   

Shifting $v$ to a different element of $V$ if necessary when
$r+s=4$, we may assume without loss of generality that $y_{14}\ge
2\hm$.

Define a deformation of $(V,E,F,G)$ by fixing $\u\ne \v_2,\v_3$ and
moving $\v_2,\v_3$ according to the following constraints:
\begin{itemize}
\item $\norm{\v_1}{\v_2(t)}$, $\norm{\v_2(t)}{\v_3(t)}$, and
$\norm{\v_3(t)}{\v_4}$ are constant.
\item $\normo{\v_2(t)}$ and $\normo{\v_3(t)}$ are constant.
\item $\norm{\v_1}{\v_3(t)} = y_{13} + t$.
\end{itemize}
These constraints uniquely determine the deformation. The deformed
value is a special fan for sufficiently small $t$.  In this context,
by Calculation~\ref{calc:cc:qua}, the function $\tau$ as a function of
$t$ (with parameters $y_{13},y_1,y_4,y_{14}$) does not have a local
minimum at $t=0$.  This is contrary to the assumed minimality of
$(V,E,F,G)$.
\end{proof}


\subsection{emptiness}

\begin{lemma}[]\guid{LPQUDGF}\rating{ZZ}\label{lemma:min-empty}  
The set of minimal fans is empty.
\end{lemma}

\begin{proof}  
Let 
\begin{displaymath}X= \{(V,E,F,G,\v_0) \mid (V,E,F,G) \text{ is
minimal and } \v_0\in V\}.
\end{displaymath}  
The strategy is to partition $X$ into smaller sets and show that each
set is empty.

For each $(V,E,F,G,\v_0)\in X$, there is a unique fan datum $\v$ and
shape $(k_{min},I)$ such that $\v_i = \rho^i\v_0$ and $(V,E,F,G) =
(V(\v),E(\v),F(\v),G(\v,I))$.  The index $i$ ranges over the cyclic
group $Z_k$ where $k=k_{min}$.

For each $(V,E,F,G,\v_0)$ with fan datum $\v$, define a function
\begin{displaymath}
  \op{vlabel}:Z_k \to \{\op{low},\op{mid},\op{high}\} 
\times \{\op{flat},\op{nonflat}\}
\end{displaymath}
(where $\op{low}$, $\op{mid}$, $\op{high}$, $\op{flat}$, and
$\op{nonflat}$ are symbolic labels for elements in sets of cardinality
three and two, respectively).  Define
\begin{displaymath}
\op{vlabel}(i)_1 = \begin{cases}
\op{low}, &\text{if } \normo{\v_i} = 2,\\
\op{high}, &\text{if } \normo{\v_i} = 2\hm,\\
\op{mid}, &\text{otherwise}
\end{cases}
\end{displaymath}
and
\begin{displaymath}
\op{vlabel}(i)_2 = \begin{cases}
\op{flat}, &\text{if $\v_i$ is flat},\\
\op{nonflat}, &\text{otherwise}.
\end{cases}
\end{displaymath}

For each $(V,E,F,G,\v_0)$ with fan datum $\v$, define a function
\begin{displaymath}
  \op{elabel}:Z_k \to \{\op{low},\op{mid},\op{high}\} 
\times \{\op{g},\op{nong}\}
\end{displaymath}
(where $\op{low}$, $\op{mid}$, $\op{high}$, $\op{g}$, and $\op{nong}$
are symbolic labels for elements in sets of cardinality three and two,
respectively).  Define
\begin{displaymath}
\op{elabel}(i)_2 = \begin{cases}
\op{low}, &\text{if } \norm{\v_i}{\v_{i+1}} = 2,\\
\op{high}, &\text{if } \norm{\v_i}{\v_{i+1}} = 2\hm,\\
\op{mid}, &\text{otherwise}
\end{cases}
\end{displaymath}
and
\begin{displaymath}
\op{elabel}(i)_2 = \begin{cases}
\op{g}, &\text{if $i\in I$},\\
\op{nong}, &\text{otherwise}.
\end{cases}
\end{displaymath}

For given functions $\op{vlabel}$ and $\op{elabel}$, let
$X(\op{vlabel},\op{elabel})$ be the subset of $X$ with the given
functions.  There are finitely many labeling functions $\op{vlabel}$
and $\op{elabel}$.  Thus, we may enumerate all cases.

By Lemma~\ref{lemma:min-irred}, every minimal fan is irreducible at
every point in $V$.  The labels have been designed in such a way that
every fan in a given set $X(\op{vlabel},\op{elabel})$ is irreducible
or no fan in the set is irreducible.  Each set that does not contain
an irreducible fan is empty.  We may filter the set of labeling
functions, discarding those that do not contain an irreducible fan.

If $f:Z_k\to Y$ is any function, then define the shift $f[j]:Z_k\to Y$
by $f[j](i) = f(i+j)$.  It is clear that $X(\op{vlabel},\op{elabel})$
is empty if and only if $X(\op{vlabel}[j],\op{elabel}[j])$, because
$(V,E,F,G,\v_0)$ lies in the first set if and only if
$(V,E,F,G,\rho^j\v_0)$ lies in the second set.  Thus, we may reduce
the list of labeling functions so that they are listed up to a shift.

A short computer calculation shows that every irreducible set of
labels is equal to one of the following up to a shift:

\begin{note}%XX minimal_fan.ml calculation to be supplied.
This list will be supplied in a later version.  The code for the
calculation appears in a file \verb!minimal_fan.ml! in the flyspeck
svn repository.
\end{note}

The elements of a given set $X(\op{vlabel},\op{elabel})$ can be
parametrized by the lengths
\begin{displaymath}\normo{\v_i} \mid \op{vlabel}(i)_1 =
\op{mid},\end{displaymath}
\begin{displaymath}\norm{\v_i}{\v_{i+1}} \mid \op{elabel}(i)_1 =
  \op{mid},\end{displaymath}
and $k_{min}-3$ diagonals.  In every case, the number of such 
parameters is at most six.    Calculation~\ref{calc:irred} %cc:par
shows in each case that
\begin{displaymath}
\tau(V,E,F) \ge d(r,s),
\end{displaymath}
for every $(V,E,F,G,\v_0)\in X(\op{vlabel},\op{elabel})$.  This
inequality shows that none of the elements $(V,E,F,G,\v_0)$ is
minimal.  This completes the proof that $X$ is empty.  {\it Praise to
the emptiness that blanks out
existence.} %-- Rumi % The essential Rumi page 21.
\indy{Index}{Rumi}%
\end{proof}

\begin{corollary}\guid{JEJTVGB}\rating{ZZ}\label{lemma:empty-d}
\begin{displaymath}
\tau(V,E,F) \ge d (r,s)
\end{displaymath}
for every special fan $(V,E,F,G)$, where $(r,s)$ are the parameters of
$(V,E,F,G)$.
\end{corollary}

\begin{proof} 
The function $\p\mapsto \tau(V(\p),E(\p),F(\p),G(\p,I))$ is
continuous on the space of fan data of shape $(k_{min},I)$

Assume for a contradiction that the conclusion is false.  Then
$k_{min}>0$, because it is the cardinality of some nonempty set $V$.
There exists a sequence of fan data $\p_i$ of varying shapes
$(k_{min},I_i)$ such that $\tau(V(\p_i),E(\p_i),F(\p_i))-d(r,s)$
tends to $\tau^d_{min}$.  By passing to a subsequence, we may assume
without loss of generality that $I_i = I$ is independent of $i$.  By
passing again to a subsequence in the compact metric space of fan
data of shape $(k_{min},I)$, the sequence $\p_i$ converges to some
fan datum $(\p,I)$.  It follows that $\tau^d_{min} =
\tau(V,E,F)-d(r,s)$, where $(V,E,F,G)=(V(\p),E(\p),F(\p),G(\p,I))$.
This is a minimal fan.  However, the set of minimal fans is empty.
\end{proof}

