%% Cluster Inequality.


\section{cluster inequality}

Turn to the proof\footnote{This section is not to be included
in the published version of the book.  It gives additional details about the
main cluster inequality.} of Lemma~\ref{lemma:cluster}.  Let $Z$ be a cell
cluster and let $\ee$ be the critical edge shared by the cells in the
cluster.  Call this critical edge the \newterm{spine}.  Consider all
the faces along the spine $\{ \u_0, \u_1\}$, consisting of three
points $\{ \u_0, \u_1, \u_2\}$ in the packing with circumradius less
than $\sqrt2$.  Call such a face a \newterm{leaf}. 
\indy{Index}{edge!spine}%
\indy{Index}{edge!critical}%
\indy{Index}{leaf}%

Consecutive cells around the spine are \newterm{adjacent}.  A cell
adjacent to a $4$-cell is a $3$-cell or another $4$-cell.  A cell
adjacent to a $2$-cell is a $3$-cell.  There are no $0$ or $1$-cells
along a critical edge.  \indy{Index}{adjacent}%
\indy{Index}{quarter}%

Call a $4$-cell a \newterm{quarter}, when it has exactly one critical
edge and all other edges of the simplex have length at most $2 h_-$.
The weight of any quarter is $1$.


\subsection{two leaves}

The proof will be divided into cases, according to the number of
leaves along the critical edge.

\claim{It suffices to prove Lemma~\ref{lemma:cluster} when the cluster
contains at least one quarter and at least two leaves.} Indeed, by
Calculation~\ref{calc:cc:qtr}, if $X$ is any cell,
then % gammaL is nonneg on quarters. cc:qtr ~[GLFVCVK]
\begin{displaymath} 
\gamma(X,L) \op{wt}(X) + \beta(\ee,X)\ge 0
\end{displaymath} 
when $X$ is not a quarter.  This is the desired inequality.  Also, a
quarter is flanked by two leaves.

Consider a cluster with exactly two leaves.
Without loss of generality, the two leaves flank a quarter
$X$. 
The azimuth angle of what remains outside the quarter
is greater than $\pi$.  Thus, there must be a $3$-cell
along each leaf.  Let $Y$ be one of these $3$-cells.
The cell $Y$ has weight $1$.
Then 
\begin{equation}\label{eqn:cc2bl} 
\Gamma(Z)\ge \gamma(X,L)+\gamma(Y,L)\ge 0,
\end{equation}
by Calculation~\ref{calc:cc:2bl}.
%a calculation~\cite[FHBVYXZ]
% 2-leaf calculation, gammaL(fourcell)+gammaL(threecell) >=0. % cc:2bl:
\indy{Index}{azimuth}%
\indy{Index}{angle!azimuth}%

\subsection{five or more leaves}

Let $B$ be the set of cells in the cluster that lie between any two
consecutive leaves.  $B$ is either a singleton set containing a
$4$-cell, or a set of three cells: a $2$-cell and two adjacent
$3$-cells.  Write $\op{azim}(B)$ for the azimuth angle formed by the
two leaves.  By Calculation~\ref{calc:cc:5bl}, the cells between two
consecutive leaves satisfy an
inequality: % five or more leaves along a spine are found
% at~[ZTGIJCF].  % cc:5bl:

\begin{displaymath} 
\sum_{X\in B} \gamma(X,L)\op{wt}(X) + \beta(\ee,X) \ge a + b\,\op{azim}(B),
\end{displaymath}
where
\begin{displaymath} 
a= 0.0560305, \quad\text{and}\quad  b= -0.0445813.
\end{displaymath}
It follows that
\begin{displaymath} 
\Gamma(Z) \ge 5 a + b\, (2\pi) > 0.
\end{displaymath}

\subsection{three or four leaves}\label{sec:3or4}

In the case of three and four leaves, the proof relies to an even
greater extent on computer calculations.  
Lemma~\ref{lemma:cluster} asserts a nonlinear inequality.  The inequality
asserts that some continuous function $f$ is positive on a compact
domain $D$ in Euclidean space.  Rigorous methods of global nonlinear
optimization can be used to show that the function $f$ is indeed
positive.

Details about global nonlinear optimization appear in the appendix.
The method is linear relaxation, which can be briefly described as
follows.\footnote{Linear relaxations appear again later in the
solution of significantly more difficult nonlinear optimzation
problems.}  The domain $D$ is partitioned into finitely many subsets
$D_1,\ldots, D_r$.  The positivity of $f$ is established on each
subset $D_i$.  The graph $\{(f(x), x)\mid x\in D_i\}$ of $f$ on $D_i$
is a subset of a polyhedron $P_i$.  The inequalities defining the
polyhedron can be determined explicitly.  A linear program computes
the minimum of the first coordinate over $P_i$.  This minimum is
positive.  Since the graph is contained in $P_i$, the values of $f$ on
$D_i$ must also be positive.

This strategy has been implemented and gives the desired lower bound.
The linear programming has been implemented as a {\tt MathProg} model
in Calculation~\ref{calc:shorts}.\footnote{{\tt MathProg} implements a
subset of the {\it AMPL} modeling language.  The computer code
giving the model appears in {\tt shorts.mod} and the computer code
controlling the branching appears in {\tt shorts.ml}.}  This model
contains the number of leaves, variables representing the edges and
the azimuth angles between consecutive leaves.

The proof that each graph is contained in a explicit polyhedron $P_i$
relies on nonlinear inequalities.  In fact, the proof requires over
one hundred nonlinear inequalities that have been established by
computer.

A function on the set $D$ with finite range partitions $D$ into
finitely many subsets, according to the image of a point.  Several
simple boolean functions and functions with finite range were used to
partition $D$ into subsets $D_i$: Are there three leaves or four?  Is
there a $2$-cell in the cluster?  Is the $4$-cell a quarter?  Is the
azimuth angle of a given cell greater than $2.3$?  Is the value of
$\gamma(\cdot,L)$ on a given quarter negative? Does a particular leaf have a
nonspline edge of length greater than $2h_-$?  What is the weight of a
given $4$-cell?  The inequalities that define the polyhedra $P_i$ have
been designed specifically for the subdomains $D_i$, according to the
answers to these questions.

These linear relaxations and are sufficient to prove the bound in all
but one case.  This is the case of four leaves, three quarters, and
one $4$-cell that has two critical edges: the spine and the edge
opposite the spine.  The other four edges of the $4$-cell have length
at most $2h_-$.  In this case as well, the inequality is established
by linear relaxation, but things are more involved.

The method to prove this case by computer is as follows.  Number the
four simplices $j=1,2,3,4$, with $j=1$ representing the $4$-cell of
weight $1/2$.  Write $\gamma^j$ in abbreviation of
$\gamma(X,L)\op{wt}(X) + \beta(\ee,X)$.  The desired inequality is
\begin{equation}\label{eqn:gpos} 
\sum_{j=1}^4 \gamma^j > 0.
\end{equation}
The domain is a set of ordered four-tuples of simplices
$X_1,\ldots,X_4$ that fit together into an octahedron (the dihedral
angles along the spine sum to $2\pi$, the four simplices all have the
same spline length $y_1(X_j)$, and the shared edges are the same
length for the two simplices sharing the edge).  Instead of the single
inequality asserted by Lemma~\ref{lemma:cluster}, many inequalities of
the following form are established:
\begin{equation}\label{eqn:gpart} 
\gamma^j + a_i \dih^j + b_i^j y_1 + c_i^j (y_2+y_3+y_5+y_6) + d_i^j > 0, 
\end{equation}
for all $X \in I_i^j$, \quad $i \in I$, and $j\in \{1,2,3,4\}$.  Here,
$\gamma^j$, $\dih^j$, and $y_i$ are all functions of $X$.  The simplex
$X$ has spine of length $y_1$ and other edges with lengths $y_k$.
Also, $\dih^j$ is the dihedral angle of $X$ along the spine.  $I$ is a
finite indexing set.  Each domain $I_i^j$ is a product of intervals in
$\ring{R}^6$, under the parameterization of a simplex by the lengths
of its six edges.  The union over $i$ of the sets
\begin{displaymath} 
\{(X_1,\ldots,X_4)\mid~ X_j \in I_i^j,\text{ for } j=1,2,3,4.\}
\end{displaymath}
covers the entire domain of the desired inequality~\eqn{eqn:gpos}.   

\claim{A subset of this union already covers the domain.}  Indeed, the
dihedral angles of the four simplices along the spine sum to $2\pi$:
\begin{displaymath} 
\sum_{j=1}^4 \gamma^j = 2\pi.
\end{displaymath}
Furthermore, the spine lengths of the simplices must agree: $y_1(X_j)
= y_1(X_k)$ for all $j,k\in\{1,2,3,4\}$.  Finally, each leaf flanks
two different simplices; the edge lengths of the leaf must agree.
This gives a collection of inequalities of the form $y_i(X_j) =
y_{i'}(X_{j'})$.  These are the subset relations.

The coefficients of the inequalities~\eqn{eqn:gpart}
are chosen so that for each $i$, the following inequality holds:
\begin{equation}\label{eqn:glin} 
0 > a_i 2\pi + 
\sum_{j=1}^4 (b_i^j y_1 +  c_i^j (y_2(X_j)+y_3(X_j)+y_5(X_j)+y_6(X_j)) + d_i^j).
\end{equation}
(Notice that this inequality is linear in the variables $y_k(X_j)$ and
the domain is a product of intervals.  Hence this inequality is
particularly easy to check.)  The sum of the
inequality~\eqn{eqn:gpart} over $j$, the inequality~\ref{eqn:glin},
and the subset relation yield the desired inequality~\ref{eqn:gpos}.

A four-leafd cluster is an octahedron.  The set of octahedra is a
$13$-dimensional object, parameterized by $12$ external edges and one
diagonal (the spine).  By contrast, a simpliex in $\ring{R}^3$ is only
a $6$-dimensional object, parameterized by $6$ edges.  The four-leafed
cluster inequality is an inequality in $13$-dimensions.  The preceding
arguments reduce the $13$-dimensional inequality into a series of
$6$-dimensional inequalities.  The $6$-dimensional inequalities are
within within the reach of a computer.

The only question is how the magical coefficients
$a_i,b_i^j,c_i^j,d_i^j$ are obtained.  Clearly, the
inequalities~\eqn{eqn:glin} and \eqn{eqn:gpart} are linear in these
coefficients.  Thus, a they can be found by linear programming.  It is
somewhat troubling that there are infinitely many constraints, as each
point $X_j$ in the domain gives one constraint.  In practice, the
infinite number of constraints can be replaced by a finite collection.
Some information is lost in approximating the system with a finite
number of constraints.  However, the coefficients, once they are
guessed by approximations, can be checked independently by computer.
(An exact fit it not necessary; any coefficients that work will do.)
It was necessary to partition the domain into smaller pieces to
produce coefficients that work.  In other words, the indexing set $I$
contains more than one element.  After some experimentation, we found
a set $I$ of cardinality $23$ that works.  Through these methods,
coefficients were found.

