%% OLD REFERENCES



\begin{thebibliography}{CHMS94}

\bibitem[AZ98]{AZ98} M. Aigner and G. Ziegler, Proofs from the Book,
Springer, first edition, 1998.

\bibitem[AH83]{interval} G. Alefeld and J. Herzeberger, Introduction
    to Interval Computations, Academic Press, New York, 1983.

\bibitem[Ann06]{Ann06}  Statement by the Editors on Computer-Assisted Proofs,
Annals of Mathematics,
http://annals.princeton.edu/EditorsStatement.html, 2006.

\bibitem[Ans02]{Ans02}  K. Anstreicher, The Thirteen Spheres: A New Proof,
citeseer.ist.psu.edu/anstreicher02thirteen.html, 2002.

\bibitem[arXiv]{arXiv} http://xxx.lanl.gov.



\bibitem[BV06]{BV06}  C. Bachoc and F. Vallentin, New upper bounds
for kissing numbers from semidefinite programming, to appear in J. AMS,
arXiv:0608426v4 [math.MG].

\bibitem[Ban]{Ban} T. Banchoff, Historical Background,
http://www.math.brown.edu/~banchoff/STG/ma8/papers/anogelo/hist4dim.html.


\bibitem[Ben74]{Ben74}  Bender, C., Bestimmung der gr\"ossten Anzahl gleich
grosser Kugeln, welche sich auf eine Kugel von demselben Radius,
wie die \"ubrigen, auflegen lassen, {\it Archiv Math. Physik} 56
(1874), 302--306.

\bibitem[Bez90]{Bez90} A. Bezdek and W. Kuperberg, Maximum density space packing with
    congruent circular cylinders of infinite length,
    {\it Mathematica} 37 (1990), 74--80.

\bibitem[BKM91]{BKM91} A. Bezdek, W. Kuperberg, and E. Makai Jr., Maximum density
    space packing with parallel strings of balls,
    {\it DCG} 6 (1991), 227--283.

\bibitem[Bez94]{Bez94} A. Bezdek, A remark on the packing density in the 3-space
    in {\it Intuitive Geometry}, ed. K. B\"or\"oczky and G. Fejes
    T\'oth, {\it Colloquia Math. Soc. J\'anos Bolyai} 63, North-Holland
    (1994), 17--22.

\bibitem[Bez97]{Bez97} K. Bezdek, Isoperimetric inequalities and the dodecahedral
    conjecture, {\it Internat. J. Math.} 8, no. 6 (1997), 759--780.

\bibitem[Bez04]{Bez04} K. Bezdek, Sphere packings in $3$-space,
Proc. of the COE Workshop on Sphere Packings, Kyushu Univ. Press 2004, 32--49.

\bibitem[Bli19]{Bli19} H. F. Blichfeldt,
    Report on the theory of the geometry of numbers,
    {\it Bull. AMS}, 25 (1919), 449--453.

\bibitem[Bli29]{Bli29} H. F. Blichfeldt,
    The minimum value of quadratic forms and the closest
    packing of spheres, {\it Math. Annalen} 101 (1929), 605--608.

\bibitem[Bli35]{Bli35} H. F. Blichfeldt,
    The minimum values of positive quadratic forms in six,
    seven and eight variables, {\it Math. Zeit.} 39 (1935), 1--15.

\bibitem[Boe52]{Boe52} Boerdijk, A. H. Some remarks concerning close-packing
of equal Spheres, {\it Philips Res. Rep.} 7 (1952), 303--313.


\bibitem[Bor03]{Bor03}  K. B\"or\"oczky, 
The Newton-Gregory problem revisited. Proc. Discrete Geometry, A. Bezdek ed., Marcel Dekker, 2003. 103--110.

\bibitem[BoSz03]{BoSz03}  K. B\"or\"oczky, L. Szab\'o,
Arrangements of 13 points on a sphere, 
Discrete Geometry, ed. A. Bezdek, CRC, 2003.

\bibitem[Bou68]{Bo} N. Bourbaki, Elements of Sets, Addison-Wesley, 1968.

\bibitem[Cas00]{BC00} B. Casselman, Packing Pennies in the Plane,
An illustrated proof of Kepler's conjecture in 2D,  AMS Feature Column
Archive,
http://www.ams.org/featurecolumn/archive/cass1.html, December 2000.

\bibitem[Cas04]{BC04}  B. Casselman, The Difficulties of Kissing in
  Three Dimensions, Notices of the AMS,  Vol 51, Number 8, 2004, 884--885.

\bibitem[Cip00]{Cip00} B. Cipra, The Best of the 20th Century: Editors
Name Top 10 Algorithms, SIAM News, 33(4), May 16, 2000.

\bibitem[CK04]{CoKu} H. Cohn, A. Kumar, The densest lattice in twenty-four
dimensions, math.MG/0408174, (2004).

\bibitem[CHMS94]{CoHMS94} J. H. Conway, T. C. Hales, D. J. Muder, and N. J. A. Sloane,
    On the Kepler conjecture, {\it Math. Intelligencer} 16,
    no. 2 (1994), 5.

\bibitem[CS95]{CoSl95}  J. H. Conway,  N. J. A. Sloane, What are all the
best sphere packings in low dimensions? {\it DCG} 13 (1995),
383--403.

\bibitem[CS98]{CS} J. H. Conway and N. J. A. Sloane, Sphere packings, lattices
    and groups,  third edition, Springer-Verlag, New York, 1998.

\bibitem[BC04]{COQ} Y. Berto and P. Cast\'eran,
  Interactive Theorem Proving and Program Development Coq'Art: The Calculus
 of Inductive Constructions, Texts in Theoretical Compuer Science, Springer, 2004.

\bibitem[Dan91]{Dan91}  G. Dantzig, Linear Programming, 
History of Mathematical Programming:
A Collection of Personal Reminiscences, ed. J.K. Lenstra et al.
Elseveir, http://www2.informs.org/History/dantzig/LinearProgramming\_article.pdf, 1991.

\bibitem[Eul78]{Euler} L. Euler, Variae speculationes super area
Triangulorum Sphaericorum, 1778.

\bibitem[Fej93]{Fej93} G. Fejes T\'oth and W. Kuperberg, Recent results in the
    theory of packing and covering, in New trends in
    discrete and computational geometry, ed. J. Pach, Springer
    1993, 251--279.

\bibitem[Fej95]{Fej95} G. Fejes T\'oth, Review of [Hsi93], {\it Math. Review} 95g\#52032, 1995.

\bibitem[Fej95b]{Fej95b} G. Fejes T\'oth, Densest packings of typical convex sets
    are not lattice-like, {\it DCG}, 14 (1995), 1--8.

\bibitem[Fej97]{Fej97} G. Fejes T\'oth, Recent progress on packing and covering,
     Advances
in Discrete and Computational Geometry, (South Hadley, MA, 1996),
pp. 145-162. Contemp. Math. 223 (1999), AMS, Providence, RI, 1999.
MR 99g:52036

\bibitem[Fej72]{Fej72} L. Fejes T\'oth, {\it Lagerungen in der Ebene auf der
    Kugel und im Raum}, second edition,
    Springer-Verlag, Berlin New York, 1972.

\bibitem[Fej64]{Fej64} L. Fejes T\'oth, Regular figures, Pergamon Press,
    Oxford London New York, 1964.

\bibitem[Fej42]{Fej42} L. Fejes T\'oth,  \"Uber die dichteste Kugellagerung,
{\it Math. Zeit.} 48 (1942 1943), 676--684.

\bibitem[Fej50]{Fej50} L. Fejes T\'oth, Some packing and covering theorems,
    {\it Acta Scientiarum Mathematicarum (Szeded)} 12/A, 62--67.

\bibitem[Fej53]{Fej53} L. Fejes T\'oth, {\it Lagerungen in der Ebene auf
der Kugel und im Raum}, Springer, Berlin, first edition, 1953.

\bibitem[FH98]{Form} S. P. Ferguson and T. C. Hales, A formulation of
the Kepler Conjecture, preprint 1998.

\bibitem[Fer97]{Fer97} Samuel P. Ferguson, Sphere
Packings V, thesis, University of Michigan,
    1997;  arXiv math.MG/9811077; to appear in Discrete and Computational
    Geometry.

\bibitem[Fly]{flydis}  Flyspeck discussion group,
http://groups.google.com/group/flyspeck.

\bibitem[Gau31]{Gau31} C. F. Gauss, Untersuchungen \"uber die Eigenscahften der
positiven tern\"aren quadratischen Formen von Ludwig August Seber,
    {\it G\"ottingische gelehrte Anzeigen}, 1831 Juli 9,
also published in {\it J. reine angew. Math.} 20 (1840), 312--320,
and
    {\it Werke},  vol. 2,
    K\"onigliche Gesellschaft der Wissenschaften, G\"ottingen,
            1876, 188--196.

\bibitem[Gol]{Gol} D. Goldberg, What Every Computer Scientist Should
Know About Floating-Point Arithmetic, Computing Surveys, ACM, 1991.
http://docs.sun.com/source/806-3568/ncg\_goldberg.html.

\bibitem[Gon05]{Gon} G. Gonthier, A computer-checked proof of the Four
Colour Theorem, preprint 2005.

\bibitem[Goo97]{Goo97} J. E. Goodman and J. O'Rourke, Handbook of discrete and
    computational geometry, CRC, Boca Raton and New York, 1997.

\bibitem[Gun75]{Gun75} S. G\"unther, {\it Ein stereometrisches Problem},
{\it Archiv der Math. Physik} 57 (1875), 209--215.

\bibitem[Hal92]{spp} Thomas C. Hales, The Sphere Packing Problem, J. of Comp.
and App. Math. 44 (1992) 41--76.

\bibitem[Hal93]{remarks} Thomas C. Hales, Remarks on the Density of Sphere Packings,
        Combinatorica, 13 (2) (1993) 181-197.

\bibitem[Hal94]{Hal94} T. C. Hales, The status of the Kepler conjecture,
    {\it Math. Intelligencer} 16, no. 3, (1994), 47--58.

\bibitem[Hal96a]{reform} Thomas C. Hales, A reformulation of the
Kepler Conjecture, unpublished manuscript, Nov. 1996.

\bibitem[Hal96b]{Hal96} T. C. Hales,
    {http://www.pitt.edu/\~\relax thales/kepler98/holyoke.html}

\bibitem[Hal97a]{part1} Thomas C. Hales, Sphere Packings I,
    Discrete and Computational Geometry, 17 (1997), 1-51.

\bibitem[Hal97b]{part2} Thomas C. Hales, Sphere Packings II,
    Discrete and Computational Geometry, 18 (1997), 135-149.

\bibitem[Hal98b]{Hal98B} T. C. Hales, Sphere Packings III, math.MG/9811075.

\bibitem[Hal98c]{Hal98C} T. C. Hales, Sphere Packings IV, math.MG/9811076.

\bibitem[Hal98d]{Hal98D} T. C. Hales, The Kepler Conjecture, math.MG/9811078.

%\bibitem[Kep98]{KEP98} T. Hales, The Kepler Conjecture, arXiv, August 1998,
%preprint. (Part I--Part IV, Part VI).

\bibitem[Hal00]{CH} Thomas C. Hales, Cannonballs and Honeycombs,
Notices Amer. Math. Soc.  47  (2000),  no. 4, 440--449.

\bibitem[Hal01]{arbeitstagung} Thomas C. Hales, Sphere Packings in 3
Dimensions, Arbeitstagung, 2001.

\bibitem[Hal03]{algorithm} Thomas C. Hales, Some algorithms arising in
the proof of the Kepler Conjecture, Discrete and computational
geometry, 489--507, Algorithms Combin., 25, Springer, Berlin,
2003.

\bibitem[Hal03]{Fl}   T. Hales, The Flyspeck Fact Sheet, 2003; revised
2007. http://code.google.com/p/flyspeck/wiki/FlyspeckFactSheet.


\bibitem[Hal05a]{annals} Thomas C. Hales, A proof of the
Kepler conjecture, Annals of Mathematics, 162 (3), Nov. 2005.

\bibitem[Hal05b]{web} Thomas C. Hales, Computer resources for the Kepler conjecture, \hfill\break
    http://www.math.princeton.edu/\~\!annals/KeplerConjecture/ (2005 snapshot).
    http://code.google.com/p/flyspeck/ (regularly maintained).
    %\hfill{\it http://www.math.pitt.edu/\~%
    %\relax thales/kepler98.html} \hfil\break
    % (The source code, inequalities,
    %and other computer data relating to the solution are also found
    %at {\it http://xxx.lanl.gov/abs/math/9811078v1}.)
    %% {Hal98A} Needs to be updated to this.

\bibitem[Hal05c]{fly}  Thomas C. Hales, The Flyspeck Project, 
{\it Mathematics, Algorithms, Proofs}, Dagstuhl seminar proceedings, 
ed. T. Coquand, H. Lombardi, and M.-F. Roy,
IBFI 2005.

\bibitem[Hal06a]{DCG} Thomas C. Hales, A Proof of the
Kepler Conjecture (unabridged version), 
Discrete and Computational Geometry, 35:1, 2006.

\bibitem[Hal06b]{historical} Thomas C. Hales, An
Overview of the Kepler Conjecture, in \cite{DCG}.

\bibitem[Hal06c]{quad} Thomas C. Hales, Equidecomposable Quadratic Regions,
{\it Automated Deduction in Geometry}, ed. F. Botana and T. Recio,
LNCS, Springer, 2006.

\bibitem[Hal07a]{method} Thomas C. Hales,  Some Methods of Problem
Solving in Elementary Geometry, Logic in Computer Science, 2007.

\bibitem[Hal07b]{EZ} Thomas C. Hales, Easy Pieces in Geometry, preprint, 2007.

\bibitem[Hal08]{errata} Thomas C. Hales, Errata and Revisions of ``The Kepler Conjecture'',
   http://flyspeck.googlecode.com/svn/trunk/dcg\_errata/dcg\_errata.tex, 2008.

\bibitem[Har68]{hart} J. F. Hart et al., Computer Approximations,
John Wiley and Sons, 1968.

\bibitem[HHS95]{HHS95} J. Hass, M. Hutchings, R. Schlafly, The double bubble conjecture,
   Elect. Res. Ann.  AMS, 1 (3), 1995.

\bibitem[Hil01]{hilbert} D. Hilbert, Mathematische Probleme, {\it Archiv Math. Physik} 1 (1901),
    44--63, also in {\it Proc. Sym. Pure Math.} 28 (1976), 1--34.

\bibitem[Hop74]{Hop74} Hoppe R. {\it Bemerkung der Redaction}, Math. Physik 56
(1874), 307-312.

\bibitem[HPT95]{HoPT95} R. Horst, P.M. Pardalos, N.V. Thoai, {\it Introduction
    to Global Optimization}, Kluwer, 1995.

\bibitem[How81]{How81} D. Howarth, 1066: The Year of the Conquest, Penguin, 1981.

\bibitem[Hsi93a]{Hsi93} W.-Y. Hsiang, On the sphere packing problem and the proof
    of Kepler's conjecture, Internat. J. Math 93 (1993), 739-831.

\bibitem[Hsi93b]{Hsi93a} W.-Y. Hsiang, On the sphere packing problem and the
    proof of Kepler's conjecture, in {\it Differential geometry and
    topology} (Alghero, 1992), World Scientific, River Edge,
    NJ, 1993,  117--127.

\bibitem[Hsi93c]{Hsi93b} W.-Y. Hsiang, The geometry of spheres, in {\it Differential
    geometry} (Shanghai, 1991), World Scientific, River Edge, NJ,
    1993, 92-107.

\bibitem[Hsi95]{Hsi95} W.-Y. Hsiang, A rejoinder to T. C. Hales's article ``The status
    of the Kepler conjecture,'' {\it Math. Intelligencer} 17, no. 1, (1995),
    35--42.

\bibitem[Hsi02]{Hsi02} W.-Y. Hsiang, Least Action Principle of Crystal Formation
of Dense Packing Type and the Proof of Kepler's Conjecture, World
Scientific, 2002.

\bibitem[IEEE]{IEEE} IEEE Standard for Binary Floating-Point
Arithmetic, ANSI/IEEE Std. 754-1985, IEEE, New York.


\bibitem[Kar66]{Kar66} R. Kargon, Atomism in England from Hariot to Newton,
    Oxford, 1966.

\bibitem[Kar84]{Kar84} N. Karmarkar, A New Polynomial Time Algorithm
for Linear Programming, Combinatorica 4 (4) 1984, 373--395.

\bibitem[Kea96]{Kea96} R. B. Kearfott, Rigorous Global Search: Continuous
Problems, Kluwer 1996.

\bibitem[Kep66]{Kep66} J. Kepler, The Six-cornered snowflake, Oxford Clarendon Press,
    Oxford, 1966,  forward by L. L. Whyte.

\bibitem[Kha79]{Kha79} L. G. Khachiyan. A polynomial algorithm in linear programming. Doklady Akademii Nauk SSSR, 20(1):1093-1096, 1979. English translation in Soviet Mathematics Doklady, 20(1):191-194, 1979.

\bibitem[KZ73]{KoZ73} A. Korkine and  G. Zolotareff, Sur les formes quadratiques,
    {\it Math. Annalen} 6 (1873), 366--389.

\bibitem[KZ77]{KoZ77} A. Korkine and  G. Zolotareff, Sur les formes quadratiques
    positives, {\it Math. Annalen} 11 (1877), 242--292.

\bibitem[Lag73]{Lag73} J. L. Lagrange,  Recherches d'arithm\'etique, {\it Nov. Mem.
    Acad. Roy. Sc. Bell Lettres Berlin} 1773, in {\it \OE uvres}, vol. 3,
    693--758.

\bibitem[Law97]{cfsqp} C. Lawrence, J. L. Zhou, A. L. Tits,  User's Guide for CFSQP Version 2.5:
   A C Code for Solving (Large Scale) Constrained Nonlinear (Minimax) Optimization Problems, Generating
   Iterates Satisfying All Inequality Constraints, 
   http://www.aemdesign.com/download-cfsqp/cfsqp-manual.pdf, 
   Institute for Systems Research, University of Maryland, Technical Report TR-94-16r1, 1997.

\bibitem[Lee56]{Lee56} J. Leech, The Problem of the Thirteen Spheres,
{\it The Mathematical Gazette}, Feb 1956, 22--23.

\bibitem[Lin86]{Lin86} J. H. Lindsey II, Sphere packing in $R^3$, {\it Mathematika}
    33 (1986), 137--147.

\bibitem[Loo68]{Loomis} L. Loomis and S. Sternberg, Advanced Calculus, Adison-Wesley, 1968.

\bibitem[LP]{lpsolve} lp\_solve, http://groups.yahoo.com/group/lp\_solve/.

\bibitem[Mar07]{Mar07} M. Cs. Mark\'ot, Interval Methods for
Verifying Structural Optimality of Circle Packing Configurations
in the Unit Square,  J. Comp. and Appl. Math. (199),
2007, 353--357.

\bibitem[Mae01]{Mae01} H. Maehara, Isoperimetric theorem for spherical
polygons and the problem of 13 spheres, Ryukyu Math. J. 14 (2001), 41--57.

\bibitem[Mae07]{Mae07} H. Maehara, The problem of thirteen spheres - a proof for undergraduates, European Journal of Combinatorics, 28(6), 2007,
1770--1778.

\bibitem[Mas66]{Mas66} B. J. Mason, On the shapes of snow crystals, in \cite{Kep66}.

\bibitem[McL98]{McL98} S. McLaughlin, A proof of the dodecahedral conjecture,
    preprint, math.MG/9811079.

\bibitem[McL08]{McL08} S. McLaughlin, Standard ML port of the code
verifying the proof of the Kepler Conjecture.
http://code.google.com/p/kepler-code/.

\bibitem[Mel97]{Mel97} J. B. M. Melissen, Packing and covering with circles,
    Ph.D. dissertation, Univ. Utrecht, Dec. 1997.

\bibitem[Mil76]{Mil76} J. Milnor, Hilbert's problem 18: on crystallographic groups,
    fundamental domains, and on sphere packings, in
    Mathematical developments arising from Hilbert problems,
    {\it Proc. Symp. Pure Math.}, vol 28, 491--506, AMS, 1976.

\bibitem[MP93]{MP93} W. Moser, J. Pach, Research problems in discrete geometry,
    DIMACS Technical Report, 93032, 1993.

\bibitem[Mud88]{Mud88} D. J. Muder, Putting the best face on a Voronoi polyhedron,
    {\it Proc. London Math. Soc.} (3) 56 (1988), 329--348.


\bibitem[Mud93]{Mud93}  D. J. Muder A New Bound on the Local Density
of Sphere Packings, {\it Discrete and Comp. Geom.} 10 (1993),
351--375.

\bibitem[Mud97]{Mud97}  D. J. Muder, letter, in {\it Fermat's enigma}, by S. Singh,
        Walker, New York, 1997.

\bibitem[Mus06]{Mus06}  O. R. Musin: The Kissing Problem in Three Dimensions. Discrete and Computational Geometry 35(3): 375-384 (2006)

\bibitem[NB06]{BN}  T. Nipkow, G. Bauer, P. Schultz, Flyspeck I: Tame Graphs, in
{\it Lecture Notes in Computer Science}, 21--35, 4130 (2006). 

\bibitem[Ob08]{Ob}  S. Obua, Flyspeck II: The Basic Linear Programs,
thesis, TU Munich 2008.

\bibitem[Oes90]{Oes90} J. Oesterl\'e,  Empilements de sph\`eres,
    S\'eminaire Bourbaki, vol. 1989/90, Ast\'erisque (1990),
        No. 189--190 exp. no. 727, 375--397.

\bibitem[PA95]{PaA95} J. Pach, P.K. Agarwal, {\it Combinatorial geometry}, John Wiley,
    New York 1995.

\bibitem[Plo00]{Plo00}  K. Plofker, private communication, January 2000.

\bibitem[Ran47]{Ran47} R. A. Rankin, {\it Annals of Math.} 48 (1947), 228--229.


\bibitem[Rog58]{Rog58} C. A. Rogers, The packing of equal spheres, {\it Proc. London Math.
    Soc.} (3) 8 (1958), 609--620.

\bibitem[Rog64]{Rog64} C. A. Rogers, {\it Packing and covering}, Cambridge University Press,
    Cambridge, 1964.

\bibitem[SW53]{Sch53} K. Sch\"utte and B.L. van der Waerden, Das
Problem der dreizehn Kugeln, {\it Math. Annalen} 125, (1953),
325--334.

\bibitem[SM44]{SeM44} B. Segre and K. Mahler, On the densest packing of
    circles, {\it Amer. Math Monthly} (1944), 261--270.

\bibitem[Shi83]{Shi83} J. W. Shirley,
{\it Thomas Harriot: a biography}, Oxford, 1983.

\bibitem[SHDC95]{SHDC95} N. J. A. Sloane, R. H. Hardin, T. D. S. Duff, J. H. Conway,
    Minimal-energy clusters of hard spheres,
    {\it DCG} 14,  no. 3, (1995), 237--259.

\bibitem[Sma98]{Sma98} S. Smale, 
   Mathematical Problems for the Next Century, Math. Intelligencer 20, No. 2, 7-15, 1998.
   
\bibitem[Sza07]{Sza07} P. G. Szab\'o, M. D. Mark\'ot, T. Csendes,
E. Specht, L. G. Casado, New Approaches to Circle Packing in a Square,
Spring Optimization and Its Applications, Vol. 6, 2007.
% other refs at http://www.sztaki.hu/~markot/publ.pdf 

\bibitem[Szp02]{Szp02} G. G. Szpiro, Kepler's Conjecture, Wiley, 2002.

\bibitem[Thu92]{Thu92} A. Thue, Om nogle geometrisk taltheoretiske Theoremer,
    {\it Forandlingerneved de Skandinaviske Naturforskeres} 14 (1892), 352--353.

\bibitem[Thu10]{Thu10} A. Thue, \"Uber die dichteste Zusammenstellung von
    kongruenten Kreisen in der Ebene, {\it Christinia Vid. Selsk. Skr.} 1
    (1910), 1--9.

\bibitem[Tod02]{Tod02} M. J. Todd, The many facets of linear programming,
Math. Program., Ser. B 91:417--436(2002). 

\bibitem[Tuc02]{Tuc02}  W. Tucker, ``A Rigorous ODE Solver and Smale's 14th Problem.'' 
   Found. Comput. Math. 2, 53-117, 2002.

\bibitem[Tut84]{tutte} W. T. Tutte, Graph Theory, Addison-Wesley, 1984.

\bibitem[VS83]{oost} A. van Oosterom and J. Strackee, The solid
angle of a plane triangle, IEEE Trans Biomed Eng. 1983 Feb; 30(2):
125--6.

\bibitem[Wie05]{freek} F. Wiedijk, Formalizing 100
Theorems\hfil\break http://www.cs.ru.nl/\~{\hbox{}}freek/100/,

\bibitem[Why66]{Why66} L. L. Whyte, forward to \cite{Kep66}.

\bibitem[Wri05]{Wri05} M. H. Wright, The interior-point revolution in
optimization: History, recent developments, and lasting consequences,
Bull. AMS. 42 (2005), 39--56.


\bibitem[Zu06]{Zu}  R. Zumkeller,  Formal Global Optimisation with
 Taylor Models, {\it Automated Reasoning -- IJCAR 2006},
 ed. U. Furbach and N. Shankar, LNCS, 408--422, 4130/2006.

%%  FIX THESE REFS.  MERGE WITH STUFF ABOVE.
%\bibitem{LS}  Loomis and Sternberg,
%\bibitem{Har1} J. Harrison, real numbers thesis.
%\bibitem{Har2} J. Harrison, Kurzweil-Henstock integration.



\end{thebibliography}
