

The aim of the present book is to present the proof of the Kepler
Conjecture. Several years have passed since a proof was first
obtained. Why give a new presentation of the proof?

The original proof was long and complex.  The complexity was not
because of conceptual challenges.  In fact, the proof makes only
modest demands on the theoretical training of the reader.  It is
possible to read and understand the proof with a knowledge of a
limited body of mathematics, such as basic calculus and elementary
Euclidean geometry.

Nevertheless, the proof involves many  calculation that are routine
and yet sometimes long and tedious.  The proof relies on the results
of computer programs.  An error in any calculation or a bug in the
computer code has the potential of toppling the entire proof.

The referees were conscientious and checked many of the
calculations.  However, the computer code was never seriously
checked, and even the most conscientious auditor can make an
occasional slip.

After all is said and done, no proof is more reliable than the
reliability of the processes that are used to verify its
correctness.  These processes include the checking that the author
makes before releasing the proof for public scrutiny, the checking
of the referees, and the checking done by readers over time.

In recent years, I have been increasingly preoccupied by the
processes that we rely on to insure the correctness of complex
proofs. Researchers from Frege to G\"odel, who solved a problem of
rigor in mathematics, found a theoretical solution but did not
extinguish the burning fire at the foundations of mathematics,
because they omitted the practical implementation. Some, such as
Bourbaki, have even gone so far as to claim that ``formalized
mathematics cannot in practice be written down in full'' and call
such a project
``absolutely unrealizable'' \cite[p 10,11]{Bo}. % Theory of Sets, page 10,11.

While it is true that formal proofs may be too long to print,
computers -- which do not have the same limitations as paper -- have
become the natural host of formal mathematics. In recent decades,
logicians and computer scientists have reworked the foundations of
mathematics, putting them in an efficient form designed for real use
on real computers.

For the first time in history, it is possible to generate and verify
every single logical inference of major mathematical theorems.  This
has now been done for the four-color theorem, the prime number
theorem, the Jordan curve theorem, the Brouwer fixed point theorem,
the fundamental theorem of calculus, and many other theorems.  Freek
Wiedijk reports that 63\% of a list of 100 ``top'' theorems have now
been checked formally \cite{freek}.

Some mathematicians remain skeptical of the process because
computers have been used to generate and verify the logical
inferences.  Computers are notoriously imperfect, with flaws ranging
from software bugs to defective chips.  Even if a computer verifies
the inferences, who will verify the verifier, or then verify the
verifier of the verifier?  Indeed, it would be unscientific of us to
place an unmerited trust in computers.

As I said above, I believe that mathematical proofs are ultimately
no more reliable that the processes we use to verify them.  We have
two competing verification processes.  The first is the traditional
process of referees, which depends largely on the luck of the draw
-- some referees are meticulous, others are careless.   The second
process is formal computer verification. In this case, the process
is less dependent on the whims of a particular referee.

In my view, choosing between the conventional referee process and
computer verification is like choosing between choosing between an
hourglass and an atomic clock as the scientific standard of time. It
impedes science to hold to the hourglass because of imperfections in
the atomic clock.

My standard of proof has become the highest scientific standard
available by current technology.  Today the highest available
standard is formal verification by computer.  This standard will
continue to evolve with the advancement of technology.

My dream is to have some day a fully formally verified proof of the
Kepler conjecture. This has not been done, but progress is being
made.  This book is an attempt to rearrange the proof of the Kepler
conjecture in such a way to make formal verification easier to do.

The proof style of formal proofs is different from that of
conventional proofs.  It is better to have a large number of short
snappy proofs, rather than a few ingenious ones.  Humans enjoy
surprising new perspectives, but computers prefer standardization.
Despite these differences, I have worked to make this a proof that
will be valuable both to humans and to computers.
