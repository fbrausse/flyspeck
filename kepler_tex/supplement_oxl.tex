% File added Nov 20, 2012.

%\chapter{Supplementary Notes}\label{sec:supplement}

\def\bve{{\underline {\hat\u}}}
\def\ke{ k_{\mathbf e}}
\def\pe{{ {\mathbf p}}_{\mathbf e}}
%\def\sgn{\op{sgn}}
\def\cX{{\mathcal X}}
\def\gg{\check\gamma}
\def\QX{\op{QX}}
\def\QY{\op{QY}}
\def\QU{\op{QU}}
\def\eps{\epsilon}
%\def\sig{\sigma}

\newpage
\section{Appendix on OXLZLEZ}\label{sec:oxl}

This appendix was written in Nov 2012 to give details of the formalization of Lemma~{\tt OXLZLEZ}.

\subsection{Formulas for $\gamma$ and $\dih$}

The dihedral angle of a cell is given by a formula $\op{dihX}$, which is defined in terms
of $\op{dihV}$.  Various lemmas related $\op{dihV}$ to the function giving the dihedral
angle of a simplex as a function of its edge lengths.

The function $\gamma$ can also be expressed as a function of edge lengths.
In the following lemmas, we fix a saturated packing $V$ and take cells with respect to $V$.


\begin{lemma}\guid{YJBIAOE} Let $X$ be a $4$-cell.  Let $y_1,\ldots,y_6$ be its edge lengths.  
Then $\gamma(X,L)$  is given by 
\[
{\tt gamma4fgcy~y1~y2~y3~y4~y5~y6~lmfun}
\]
defined formally in the module Sphere.
\end{lemma}

\begin{lemma}\guid{XKYBPAI} Let $X$ be a $3$-cell.  Let $y_4,y_5,y_6$ be the lengths of the three edges
in $E(X)$.  Then $\gamma(X,L)$ is given by
\[
{\tt gamma3f~y4~y5~y6~sqrt2~lmfun}
\]
defined formally in the module Sphere.
\end{lemma}

\begin{lemma}\guid{KKHWUHM} Let $X$ be a $2$-cell.  Let $y$ be the length of the unique edge in $E(X)$.
Let $\alpha$ be the dihedral angle along that edge.
Then $\gamma(X,L)=\alpha\cdot t$, where $t$ is
\[
{\tt
gamma2\_x\_div\_azim~(h0cut~y)~(y*y)
}
\]
defined formally in the module Functional\_equation.
\end{lemma}

\begin{lemma}\guid{OWEWPJG} Let $X$ be a $1$-cell.  Let $\v$ be the unique element of $V(X)$.
Let $s=\sol(X,\v)$.  Then $\gamma(X,L)=s t$, where $t$ is
\[
\frac{8\pi \sqrt2}{3} - 8 m_1
\]
\end{lemma}

\begin{lemma}\guid{KPJNKIL} Let $X$ be a $0$-cell.  Then $\gamma(X,L)$ is equal to
the volume of $X$.
\end{lemma}


\subsection{Leaf and cell}

\begin{definition}[leaf] \guid{NIPHFIE}
Let $V$ be a saturated packing.  A \newterm{leaf} of $V$ is an element $\bu=[\u_0;\u_1;\u_2] \in \bV(2)$ such
that $h(\bu) < \sqrt2$.  The \newterm{stem} of the leaf is $\{\u_0,\u_1\}$.
\end{definition}





\begin{lemma}\guid{GBEWYFX} Let $V$ be a saturated packing, and let $\bu = [\u_0;\u_1;\u_2]$ be a leaf of $V$.
Then $S=\{\u_0,\u_1,\u_2\}$ is not collinear.
\end{lemma}

\begin{proof}  Part 1 of MHFTTZN states that $S$ has affine dimension $2$, hence the set is not
collinear.
\end{proof}





\begin{lemma}\guid{NWVRFMF}\label{lemma:facetv}  Let $V$ be a saturated packing, and let $\bu$ be a leaf of $V$.  Let
$\p\in\ring{R}^3$ be such that $\{\p\}$ is a facet of $\Omega(V,\bu)$.  Then
there exists $\bv\in \bV(3)$ such that $d_2\bv = \bu$ and $\omega_3(\v)=\p$.
\end{lemma}

\begin{proof} This follows directly from Lemma IDBEZAL and $\bu\in \bV(2)$.
\end{proof}

\begin{lemma}\guid{YBZFUPO}\label{lemma:p1p2} 
Let $V$ be a saturated packing with leaf $\bu$.  Then there exist distinct $\p_1$ and $\p_2$ such
that $\Omega(V,\bu)$ is the convex hull of $\{\p_1,\p_2\}$ and such that
$F$ is a facet of $\Omega(V,\bu)$ if and only if $F\in \{\{\p_1\},\{\p_2\}\}$.
\end{lemma}

\begin{proof}  By the definition of $\bV(2)$,  we have that $\bu\in \bV(2)$ implies that
the affine dimension of $\Omega(V,\bu)$ is one.  This is a bounded polyhedron of dimension one,
hence a segment given as a convex hull of distinct points $\p_1$ and $\p_2$.  The facets of a segment
are its extreme points as given.
\end{proof}

\begin{lemma}\guid{ZASUVOR}
Let $V$ be a saturated packing with leaf $[\u_0;\u_1;\u_2]$.  Then $[\u_1;\u_0;\u_2]$ is also a leaf
with the same stem.
\end{lemma}

\begin{proof} The stem is clearly the same, and the circumradius does not change upon reordering of
elements.  Let $\bv\in \bV(3)$ be an element constructed in the previous lemma such that
$d_2\bv=\bu$.  Let $\bv'$ be obtained by transposing the first two elements.  By YNHYJIT,
we have $\bv'\in \bV(3)$.    Then $d_2\bv' = [\u_1;\u_0;\u_2]\in \bV(2)$.  The result ensues.
\end{proof}



\begin{lemma}\guid{FUZBZGI}  Let $V$ be a saturated packing with leaf $\bu$.  let $\q$ be the circumcenter of $\bu$.
Then $\q\in\Omega(V,\bu)$, but is not an extreme point of $\Omega(V,\bu)$.
\end{lemma}

\begin{proof} The third part of Lemma MHFTTZN gives that $\q\in \Omega(V,\bu)$. 

Assume for a contradiction that
$\q$ is an extreme point, then Lemma~\ref{lemma:facetv} gives $\bv\in \bV(3)$ such that
$d_2\bv = \bu$ and $\omega_3(\v)=\q$.  The set $\Omega(V,\bv)$ is convex of affine dimension $0$,
and is therefore a singleton $\omega_3(\bv)$.   By Lemma MHFTTZN applied to $\bv$, we have that
$\q=\omega_3(\bv)$ is the circumcenter of $\bv$.  This contradicts the strict inequality of Lemma XYOFCGX.
\end{proof}

\begin{definition}[$\chi$]\guid{MSBKFLD} For any list $\bu = [\u_0;\u_1;\u_2]$ of  elements in $\ring{R}^3$, define 
$\chi(\bu,\p) = ((\u_1-\u_0)\times (\u_2-\u_0))\cdot(\p-\u_0)$.
\end{definition}




\begin{lemma}\guid{JDHAWAY} Let $V$ be a saturated packing with leaf $\bu$.  Let $\p_1$ and $\p_2$ be the
distinct points constructed in Lemma~\ref{lemma:p1p2}.  Then $\chi(\bu,\p_i)$ is not zero,
and $\chi(\bu,\p_1)$ and $\chi(\bu,\p_2)$ have opposite signs.
\end{lemma}

\begin{proof}
Let $\q$ be the circumcenter of $\bu$.
If $\chi(\bu,\p_i) = 0$, then $\p_i$ lies in the affine hull of $\bu$.   By MHFTTZN, this implies that
$\q = \p_i$, which is impossible by the previous lemma.  Hence $\chi(\bu,\p_i)\ne 0$.

By the previous lemma, the circumcenter $\q$ of $\bu$ has the form $\q=\p_1 t_1 + \p_2 t_2$,
for some $t_i$ such that $t_1+t_2=1$ and $t_i>0$.  Since $\q$ lies in the affine hull of $\bu$, we have
\[
0 = \chi(\bu,\q) = t_1 \chi(\bu,\p_1) + t_2 \chi(\bu,\p_2)
\]
Since $t_i>0$, this implies that $\chi(\bu,\p_1)$ and $\chi(\bu,\p_2)$ have opposite signs.
\end{proof}

\begin{remark}\guid{LITLFSC}
Recall from Lemma JBDNJJB that  $\chi(\bu,\p)$ has the same sign as
\[
\sin(\azim(\u_0,\u_1,\u_2,\p)).
\]
Also, $\chi([\u_0;\u_1;\u_2],\p) = -\chi([\u_1;\u_0;\u_2],\p)$.
\end{remark}

\begin{definition}[$\pe$,~$\bve$,~$\ke$,~$c$]\guid{AQEQEDX}  
Let $V$ be a saturated packing with leaf $\bu$.
We define $\pe(\bu)$, $\bve$, and $\ke$ as functions of $\bu$ as follows.
By the previous lemma, there exists a unique extreme point $\pe$ of $\Omega(V,\bu)$ such that
$\chi(\bu,\pe)  > 0$.  By an earlier lemma, there exists
$\bve\in \bV(3)$ (choose one)
where $d_2(\bve) = \bu$ and $\omega_3(\bve) = \pe$.  Finally, let $\ke\in\{3,4\}$
be given by 
\[
\ke = \begin{cases} 4, &h(\bve) < \sqrt2\\
    3,&\text{otherwise}.
\end{cases}
\]
We abbreviate $c(\bu)=\op{cell}(\bve,\ke(\bu))$.
\end{definition}

This construction uses a leaf $\bu=[\u_0;\u_1;\u_2]$ to select 
an extreme point of $\Omega(V,\bu)$
and an associated cell.
By earlier lemmas, the other extreme point of $\Omega(V,\bu)$ is determined by the other leaf
$[\u_1;\u_0;\u_2]$ with the same stem.  It reverses the sign of $\chi$.
Note that $[\u_1;\u_0;\u_2]$ is the unique nontrivial rearrangement of $\bu$ with the same stem.
If $\bu$ is a leaf, we write 
\[
A_+^0(\bu) = \op{aff}_+^0(\{\u_0;\u_1\},\u_2), \text{ and } A(\bu) = \op{aff}\{\u_0,\u_1,\u_2\}.
\]

\begin{lemma}\guid{NUNRRDS}  Let $V$ be a saturated packing with leaf $\bu$.
Then $c(\bu)$ meets $A_+^0(\bu)$.
Furthermore, for every $\q\in c(\bu)$, we have $\chi(\bu,\q)\ge0$.
%Moreover, $c(\bu)$ is contained in a closed half space bounded by $A(\bu)$.
%Also, let $\bu'$ be the unique nontrivial rearrangement of $\bu$ with the same stem.
%Then $c(\bu)$ and $c(\bu')$ lie in opposite half spaces.
\end{lemma}

\begin{proof} The cell is given as a convex hull of $\{\u_0,\u_1,\u_2,\p\}$, for some point $\p$.
Hence the cell contains $\u_2$ which lies in $A_+(\bu)$.   The convex hull clearly lies in a half
space bounded by $A(\bu)$.

We have $\chi(\bu,\p)>0$.  Since $\q$ is in the convex hull of four points with
$\chi\ge0$, we also have $\chi(\bu,\q)\ge0$.
\end{proof}


\subsection{Planarity}

\begin{lemma}\guid{RIJRIED}\label{lemma:em2}  Let $V$ be a saturated packing. 
Let $X$ be a cell with an edge.  Then $X$ is not coplanar.
\end{lemma}

\begin{proof}
By definition, the vertex set is empty, if the  cell is a null set.
\end{proof}

\begin{lemma}\guid{ZWVCBMN} Assume that $S=\{\u_0,\ldots,\u_3\}\subset\ring{R}^3$ is not coplanar.  Then
the convex hull of $S$ has positive measure.
\end{lemma}

\begin{proof} The volume of a tetrahedron has the form $\sqrt{\Delta}/12$, and the condition
of planarity is $\Delta=0$.
\end{proof}

\begin{lemma}\guid{ASVAYEW}\label{lemma:em34} 
Let $V$ be a saturated packing and let $X$ be a nonempty $3$ or $4$-cell.  Then
$X$ is not coplanar.
\end{lemma}

\begin{proof} Write the cell as $\op{cell}(\bu,k)$. 
A nonempty $3$ or $4$-cell has the form of a convex hull of four points,
 $\{\u_0,\u_1,\u_2,\p\}$, where $\p=\u_3$ in the case of a $4$-cell ($h(\bu)<\sqrt2$) or
$\p=\xi$ in the case of a $3$-cell ($h_2 <\sqrt2 \le h(\bu)$).  
In the case of a $4$-cell
the points are not coplanar by Lemma MHFTTZN.  

In the case of a $3$-cell, we have that  $A(\bu)$ has dimension $2$
by Lemma MHFTTZN.  It is enough to show that $\xi$ is not in this affine hull.   
The point $\xi$
is equidistant from $\u_0,\u_1$ and  $\u_2$.  If $\xi$ is in the affine hull, then it is the circumcenter,
and we arrive at a contradiction
\[
\sqrt2 = \norm{\xi}{\u_0} = h(d_2\bu) < \sqrt2.
\]
\end{proof}

\subsection{Classification of cells}


\begin{lemma}\guid{CFFONNL}\label{lemma:meet-halfplane}  
Let $V$ be a saturated packing with leaf $\bu = [\u_0;\u_1;\u_2]$.  
Let $X$ be a cell of $V$, and let $\{ \u_0,\u_1 \}\in E(X)$ be
an edge.  Suppose that 
\[
X \cap  A_+^0(\bu)\ne\emptyset.
\]
Then there exists a $2$-rearrangement $\bu'$ of $\bu$  such that
\[
X = c(\bu').
\]
\end{lemma}

\begin{proof} Since $X$ has an edge, it is at least a $2$-cell.  The cell $X$ 
is not a subset of $A(\bu)$.
Pick $\q\in X\setminus A$, and choose a $2$-rearrangement $\bu'$ of $\bu$
such that  $\chi(\bu',\q)>0$.  
The cell $X$ contains the convex hull of four points
\[
\u_0,\u_1,\p,\q
\]
where $\p\in A_+^0$ and $\chi(\bu',\q)>0$.  

Let 
$\bv=\bve'\in \bV(3)$ and $k = \ke(\bu')$.
Let $X'=\op{cell}(\bv,k)$.  
The cell $X'$ also contains the convex hull of
four points
\[
\u_0,\u_1,\p',\q'
\]
where $\p'\in A_+^0$ and $\chi(\bu',\q')>0$.  Any two such convex hulls meet in a set of
positive measure.  So $X\cap X'$ has positive measure.  Hence by Lemma AJRIPQN, we have
$X = X'$.  This gives the result.
\end{proof}


\begin{lemma}\guid{FUEIMOV}  Suppose that $V$ is a saturated packing with leaves $\bu$ and $\bu'$ 
with the same stem.   Let $k = \ke(\bu)$ and $k'=\ke(\bu')$.
Assume that $\bu\ne \bu'$.
and
$c(\bu) = c(\bu')$.  Then
 $k=k'=4$,  $\bve = [\v_0;\v_1;\v_2;\v_3]$, 
and $\bve' = [\v_1;\v_0;\v_3;\v_2]$.
\end{lemma}

\begin{proof}
If the cells are equal, then $k=k'\in \{3,4\}$ by the definiton of $\ke$ and Lemma AJRIPQN.

We consider two cases depending on the value of $k$.
Suppose that $k=4$.  The definition of $\ke$ gives $h(\bve) < \sqrt2$.  The set $V \cap \op{cell}(\bv,4)$ 
determines the parameter $\bv$ up to rearrangement.
The stem is fixed, giving only two possibilities $[\u_0;\u_1;\dots]$ and $[\u_1;\u_0;\dots]$ for the
first two elements.  The last two entries must also be equal or transpositions.  To preserve the
sign of $\chi$, the permutation must be even.  The only nontrivial possibility is the one given
in the lemma.

Suppose that $k=3$.    The
cell is the convex hull of its extreme points, three of which are elements of the vertex set $V(X)$.
The first two entries of $\bu$ are determined by the stem, up to transposition.  The third entry
is fixed by membership in $V(X)$.   The fourth entry must be the common extreme point $\xi$.
If $\bu=[\u_0;\u_1;\u_2]$ and $\bu'=[\u_1;\u_0;\u_2]$, then
\[
0 < \chi(\bu,\xi) = -\chi(\bu',\xi).
\]
This gives incompatible sign constraints.
\end{proof}

\begin{lemma}\guid{YSAKKTX}  Let $V$ be a saturaed packing.  Let $X$ be a cell with edge $\{\u_0,\u_1\}$.
Then $X = \op{cell}(\bu,k)$, for some $\bu$ of the form $[\u_0;\u_1;\ldots]$.
\end{lemma}

\begin{proof} This follows from RVFXZBU.
\end{proof}

\begin{lemma}\guid{RBUTTCS}  Let $V$ be a saturated packing.  Let $X$ be a $4$-cell with
edge $\{\u_0,\u_1\}$.  Then there exists a leaf $\bu$ 
with stem $\{\u_0,\u_1\}$ such that $X=c(\bu)$.
\end{lemma}

\begin{proof}  By YSAKKTX, we may write $X = \op{cell}(\bv,4)$,
where $\bv$ has the form $\bv = [\u_0;\u_1;\v_2;\v_3]$, for some $\v_2$ and $\v_3$.  
Set $\bu = d_2\bv$.
We have $h(\bu)<\sqrt2$, so that $\bu$ is a leaf.
The cell meets
$A_+^0(\bu)$  at $\v_2$.  Lemma~\ref{lemma:meet-halfplane} gives
 that $X = c(\bu')$ for some $2$-rearrangement of $\bu$.
\end{proof}

\begin{lemma}\guid{FCHKUGT}
Let $V$ be a saturated packing, and let 
 $\bu=[\u_0;\u_1;\u_2]$ and $\bu'=[\u_0;\u_1;\u_2']$ be two leaves with the same first two entries, 
such that $A_+^0(\bu) = A_+^0(\bu')$.
Then $\bu = \bu'$.
\end{lemma}

\begin{proof}  We have that $c(\bu)$ and $c(\bu')$ both meet $A_+^0$.  Hence,
there is a $2$-rearrangement $\bu''$ of $\bu'$ such that $c(\bu) = c(\bu'')$.
By FUEIMOV, $k=k'=4$, and $\bu''$ is a rearrangement of $\bu$.  If $\bu\ne\bu''$,
then $\bu''=[\u_1;\u_0;\u_3]$, where $c(\bu) = \op{conv}\{\u_0,\ldots,\u_3\}$.  But
this final possibility is impossible, since $\{\u_0,\ldots,\u_3\}$ is not coplanar.
\end{proof}

\begin{lemma}\guid{BDXKHTW} Let $V$ be a saturated packing.
Let $X$ be a cell of $V$ with edge $e = \{\u_0,\u_1\}$.   Let $\bu = [\u_0;\u_1;\u_2]$ and
$\bu' = [\u_0;\u_1;\u_2']$ be distinct leaves with stem $e$.  Assume that the intersection of $X$ with
$\op{wedge}^0 (\u_0,\u_1,\u_2,\u_2')$ is nonempty.  Then $X$ is a subset of $\op{wedge}(\u_0,\u_1,\u_2,\u_2')$.
\end{lemma}

\begin{proof}  We claim that $X$ is not coplanar.  Otherwise it is a null set, and by definition
a null set has no vertices or edges.

Since the leaves are distinct, by FCHKUGT, we have $A_+^0(\bu)\ne A_+^0(\bu')$.
This implies that the wedges are nondegenerate.

For a contradiction, 
pick $\p\in X\cap \op{wedge}^0 (\u_0,\u_1,\u_2,\u_2')$ and $\q\in X\cap \op{wedge}^0(\u_0,\u_1,\u_2',\u_2)$.
We may assume that $\{\u_0,\u_1,\p,\q\}$ is not coplanar.
The sets $\op{wedge}^0$ are open and disjoint.  By the connectedness of the unit interval,
the path $t\mapsto  (1-t) \p + t \q$ crosses $A_+^0(\bu)$ or $A_+^0(\bu')$ at $\q_t$, 
for some $0<t<1$.
We then have that $\{\u_0,\u_1,\q_t\}$ is not collinear.
This point of crossing, by an earlier lemma gives $X=c(\bu'')$ for some $2$-rearrangement of $\bu$ or $\bu'$, and say $A_+^0(\bu'') = A_+^0(\bu)$.
By an earlier lemma, 
\[
0 \le  (1-t)\chi(\bu'',\p) + t \chi(\bu'',\q) = \chi(\bu'',\q_t) = 0.
\]
This forces $\chi(\bu'',\p) = \chi(\bu'',\q)=0$, which is contrary to our assumption that
 $\{\p,\q,\u_0,\u_1\}$ is not coplanar.
\end{proof}

\begin{lemma}\guid{EWYBJUA} Let $V$ be a saturated packing.  Let $X$ be any cell with edge $\{\u_0,\u_1\}$.
Let $\bu=[\u_0;\u_1;\u_2]$ and $\bu'=[\u_0;\u_1;\u_2']$ be distinct leaves of $V$. Then
\[
X \subset \op{wedge}(\u_0,\u_1,\u_2,\u_2') \text{ or } X \subset\op{wedge}(\u_0,\u_1,\u_2',\u_2).
\]
\end{lemma}

\begin{proof}  
The cell $X$ is not a null set, so there exists a point in the intersection of $X$ with
\[
\op{wedge}^0(\u_0,\u_1,\u_2,\u_2') \text{ or } X \subset\op{wedge}^0(\u_0,\u_1,\u_2',\u_2).
\]
Then apply the previous lemma.
\end{proof}

\subsection{Angle Sums Revisited}

In the following lemma, the hypothesis \eqref{eqn:reu} always holds by preceding lemmas,
but we include it to make it independent of these lemmas.   The lemma is a  variant of Lemma GRUTOTI.


\begin{lemma}\guid{REUHADY}
  Let $V$ be a saturated packing.  Assume that $\u_0,\u_1\in V$
  satisfy $\norm{\u_0}{\u_1}<2\nsqrt2$.  Set $\ee=\{\u_0,\u_1\}$.  
Let $[\u_0;\u_1;\w]$ and $[\u_0;\u_1;\w']$ be two leaves (not necessarily distinct) such
that for every cell $X$ with edge $\{\u_0,\u_1\}$ we have
\begin{equation}\label{eqn:reu}
X\subset \op{wedge}(\u_0,\u_1,\w,\w')\quad\text{ or } \quad
X\subset \op{wedge}(\u_0,\u_1,\w',\w).
\end{equation}
Then 
\[
\sum_{X\in\cX} \op{dih}(X,\ee) = \azim(\u_0,\u_1,\w,\w').
\]
The sum runs over the set $\cX$ of
 cells $X$ such that $\ee\in E(X)$ and $X \subset \op{wedge}(\u_0,\u_1,\w,\w') $.
\end{lemma}

\begin{proof} 
  Consider the sets
\[
C=B(\u_0,r)\cap \op{rcone}^0(\u_0,\u_1,a),\text{ and } C' = C\cap \op{wedge}(\u_0,\u_1,\w,\w') ,
\]
 where
  $r$ and $a$ are small positive real numbers.  From the definition of
  $k$-cells, it follows that we can choose $r$ and $a$ sufficiently
  small so that if $X$ is a $k$ cell that meets $C'$ in a set of
  positive measure, then $k\ge 2$ and there exists $\bu\in \bV(3)$
  such that $X=\cell(\bu,k)$ and $d_1\bu=[\u_0;\u_1]$.  Moreover,
\[
C'\cap X = C\cap X = C\cap A, \quad A=\op{aff}_+(\{\u_0,\u_1\},\{\v,\w\}),
\]
where $A$ is the lune of Definition~\ref{def:lune} and $\v$, $\w$ are
chosen as in Definition~\ref{def:dihX}.  By
Lemma~\ref{lemma:wedge-sol} and Definition~\ref{def:dihX}, the volume
of this intersection is
\[
\op{vol}(C\cap A) = \op{vol}(C)\,
 {\op{dih}_V(\{\u_0,\u_1\},\{\v,\w\}) }/{(2\pi)} =
  \op{vol}(C)\, {\dih(X,\ee)}/{(2\pi)}.
\]
The set of cells meeting $C'$ in a set of positive measure gives a 
partition of $C'$ into finitely many measurable sets.
This gives
\begin{align*}
\op{vol}(C) \azim(\u_0,\u_1,\w,\w')/(2\pi) &= 
\op{vol}(C') \\
&= \sum_{X\in\cX} \op{vol}(C\cap X)  \\
&= \op{vol}(C)\sum_{X\in\cX} \dih(X,\ee)/(2\pi).
\end{align*}
The calculation of volumes in Chapter~\ref{chapter:volume} gives
$\op{vol}(C)>0$.  The conclusion follows by canceling $\op{vol}(C)$
from both sides of the equation.
\end{proof}


\subsection{Linear Programs}

Let $V$ be a saturated packing and let $\{\u_0,\u_1\}$ be a critical edge (of some cell).
We may order the set 
\[
\op{Leaf} = \{ \v\in V \mid [\u_0;\u_1;\v] \text{ is a leaf } \}
\]
(and the corresponding leaves)
by arbitrarily fixing one element $\v_0\in L$ and then ordering them by increasing
azimuth angle $\azim(\u_0,\u_1,\v_0,\v)$, for $\v\in \op{Leaf}$.  This partitions $\ring{R}^3$ into
finitely many wedges delimited by the leaves.  Each cell with edge $\{\u_0,\u_1\}$ lies in one of these
wedges.    

If $h(\hat\bu) < \sqrt2$, then $c(\bu)=c(\bu')$ is a $4$-cell in the wedge,
 and this is the only cell  (of positive measure or in fact the only cell at all)
contained in the wedge.

If $h(\hat\bu)\ge\sqrt2$, then $c(\bu)$ and $c(\bu')$ are distinct $3$-cells in the wedge,
and these are the only $3$ cells in the wedge.  The wedge may also include a finite number of 
$2$-cells, but it has no $4$-cells.  The $2$-cells within a given wedge are combined into a 
total $\gamma$ and total azimuth angle.

We have a number of nonlinear inequalities bounding the value of 
\[
\gamma(X,L) \op{wt}(X) + \beta(\{\u_0,\u_1\},X)
\]
 as a function
of its azimuth angle.  These inequalities are based on the partition of cells according to wedges.
Based on these inequalities,
we run linear programs giving lower bounds for $\sum \gamma(X,L)$, subject to the constraint
that the azimuth angles sum to $2\pi$.  In every case, we find that $\Gamma\ge0$ for each cluster.
We run a separate linear program depending on the number of leaves.  A generic case handles
the case of five or more leaves.
The following subsection gives details.


\subsection{Cell Cluster Inequality}

This section shows how to deduce Lemma OXLZLEZ from a collection of nonlinear inequalities.
The nonlinear inequalities are specified in an abbreviated way according to the table at the end of this section.
Some collections of  nonlinear inequalities have been merged into a single inequality.
Once the source of a nonlinear inequality has been cited, we assume that the reader has become
familiar with it, and it is not repeatedly cited.

\begin{remark}[preparation]\guid{XSBYGIQ}
\begin{enumerate}
\item If a leaf separates two $3$-cells, erase it, scoring it at $0.008\azim$.  This is justified by
a \cc{cell3}{}.
\item Score all $2$-cells at $0.008\azim$ by a \cc{grki}{}.
\item Score all $3$-cells with $\eta ^+$ at $0.008\azim$, but keep the leaf, by \cc{cell3}{}.
\end{enumerate}
\end{remark}

\begin{definition}[small~leaf,~subcritical,~supercritical]\guid{HYTORSD}
A \newterm{critical} edge has length $2h_-\le y\le 2h_+$.  A \newterm{subcritical} edge has length $2\le y< 2h_-$.
A \newterm{supercritical} edge has length $2h_+<y\le\sqrt8$.
A \newterm{small leaf} is a triangle along the critical edge $y_1$ with one critical edge $y_1$ and two
subcritical edges $y_2$, $y_6$.
A $\beta$-cell is one with two critical edges $y_1,y_4$ and the remaining four edges subcritical.
Set $\gg = \gamma(X)\op{wt} + \beta(X)$.
A $23$-cell is used to refer to the collection of $2$ and $3$-cells that lie between two consecutive nonerased leaves.  The term \newterm{cell} will now refer to a $4$-cell or a $23$-cell,
rather than a Marchal cell.

A cell along a fixed stem $[\u_0;\u_1]$ is call a quarter, if it
 is a $4$-cell with a critical edge along the stem and five subcritical edges.
We write $\QU$ for a quarter.
We write $\QX$ as an abbreviation for any $4$-cell that is not a quarter,
and $\QY$ for any $23$-cell. Write $\eps=0.0057$.
When we speak of a leaf, we restrict our usage to the two leaves along the common stem.
We write $\eta^+$ for the condition that the circumradius of a leaf $>1.34$ and
$\eta^-$ for the circumradius $\le 1.34$.
\end{definition}

In the following lemmas, we work in the context of OXLZLEZ.  That is, we have a number of cells sharing
a critical edge.  For a contradiction, we assume that we are working with a counterexample.  That is,
we assume as a hypothesis in each lemma that $\sum \gg_i < 0$.

\begin{lemma}\label{lemma:ox34}\guid{CHQSQEY}
There are at least three $4$-cells (in every counterexample).
\end{lemma}

\begin{proof} We consider cases, according to the number of $4$-cells.  If there are no quarters, then
$\gg_i\ge0$ for all $i$ by \cc{gamma\_qx}{}.
We may in fact assume that there is a quarter with $\gg<0$.

If there is one $4$-cell (a quarter), then $\dih < 2\pi$ by \cc{azim\_c4}, so there is another cell, a $23$-cell.
Then $\gg_4 + \gg_3 > 0$ by \cc{quqy}.
If there are two $4$-cells.  The total angle of the two cells is at most $2 (2.8) < 2\pi$, so there is at least one $23$-cell, and each $4$-cell is flanked by at least one $23$-cell.
Then use $\gg_4+\gg_3>0$.
\end{proof}

\begin{lemma}\label{lemma:ox2}\guid{MTMLSRF} 
There exists a quarter with $\gg<0$ and flanked on both sides by another $4$-cell.
\end{lemma}

\begin{proof}
Otherwise, we again have $\gg_4+\gg_3>0$.
\end{proof}

\begin{lemma}\label{lemma:ox44}\guid{LXDEYBO} There are at most four $4$-cells.
\end{lemma}

\begin{proof}
  Otherwise, by a \cc{ztg4}{}
\[
\sum_{(5)} \gg \ge 5 a_5 + b_5\alpha \ge 5 a_5 +  b_5 (2\pi) > 0.
\]
\end{proof}

\begin{lemma}\label{lemma:oxC}\guid{UNPNFVW} The $4$-cells are all contiguous.  That is, there is at most one $23$-cell.
\end{lemma}

\begin{proof}
The number of $4$-cells is three or four.
By Lemma~\ref{lemma:ox2}, there is a contiguous block of three or four $4$-cells.  Thus,
if they are not all contiguous, there is a block of three and a block of one,
with $23$-cells interspersed.  That is, there would be at least two $23$-cells,
each with angle at least $0.606$.  Then by a \cc{gckb ztg4 cell3 grki}{}
\[
\sum \gg \ge 2 (0.606) (0.008) + 4 a_5 + (2\pi-2 (0.606)) b_5 > 0.
\]
\end{proof}

\begin{lemma}\label{lemma:ox-cases}\guid{DHCVTVE} The configuration must be one of the four possibilities $C_{n,k}$, where $n$ is the number of leaves, $k$ is the number of $4$-cells, and $n-k$ is the number of $23$-cells:
\[
C_{3,3}\quad C_{4,4}\quad C_{4,3}\quad C_{5,4}.
\]
\end{lemma}

\begin{proof} Since there are three or four $4$-cells and zero or one $23$-cell,
there are four cases as given.
\end{proof}

\begin{lemma}\guid{PMZTATI} Let $\QX$ be a cell with one small leaf and the other not small.
Then $\gg > \eps$ or $\QX$ is a $\beta$-cell.  
\end{lemma}

\begin{proof} This is a \cc{gamma8}{}.
\end{proof}

\begin{lemma}\guid{RJSZKQX} 
Let $\QU$ be a cell with (at least) one leaf $\eta^+$.  Then $\gg > \eps$.
\end{lemma}

\begin{proof} This is a \cc{fhvb2}{}.
\end{proof}

\begin{lemma}\guid{IXPFBKA} If $\eta^-$, then the leaf is small.
\end{lemma}

\begin{proof} This is a \cc{jsp}{}.
\end{proof}

\begin{remark}[neutralize]\guid{XCLCXWG}
 $\gg_{\QU} > -\eps$ and $\gg_{\QX}\ge 0$ by a \cc{gamma\_qu gamma\_gx}.
If $\QX$ has one leaf that is small and another that is not small, 
it \newterm{neutralizes} a quarter in the
sense that it gives $\eps$ against the $-\eps$ of the quarter.
Notice that this condition propagates, forcing one cell after another to have
small leaves, if it doesn't neutralize.
If a $4$-cell is next to a $23$-cell, then we have by a \cc{gamma10\_gamma11 qu\_qy}{},
\[
\gg_4 + \gg_3 > \eps,
\]
and again we can neutralize a quarter.  Call this $234$-neutralization.
\end{remark}

\begin{lemma}\guid{IPVICGW}  All leaves along the stem are small.
\end{lemma}

\begin{proof} We consider many cases, breaking it down intially according to the cases $C_{n,k}$.

$C_{3,3}$:  If there are at least two quarters, every leaf lies along a quarter and is small.
If there is one quarter, we can neutralize it, unless all leaves are small.

$C_{4,4}$, $C_{4,3}$: If there are at least three quarters (or two that are nonadjacent), then every leaf lies along a quarter
and is small.  If there are two adjacent or one quarter, we can neutralize them unless every leaf is small.

$C_{5,4}$: If there are four quarters, then every leaf is along a quarter.
If there is one quarter, we can neutralize it, if it is not small.
There remain two cases: two or three quarters.

$C_{5,4}$: two or three quarters.  There are various combinatorial placements of the quarters among the four $4$-cells.  If two of them are not adjacent to the $23$-cell, then we can treat it with neutralization arguments.  The one case that is not easily neutralized is an arragnement with a single contiguous block of quarters, occupying the two slots that are not adjacent to the $23$-cell, and possibly one other slot.  Furthermore, there is exactly one leaf (along the $23$-cell) that is not small.   That nonsmall leaf has $\eta^+$.  We break this case into subcases.

\claim{Assume $\azim_{23}>1.074$}.  By a \cc{cell3 grki ztg4}{}, we have
\[
\sum \gg > 0.008 \azim_{23} + (4 a_5 + b_5 (2\pi-\azim_{23})) > 0.
\]

\claim{Assume $\azim_{23}\le 1.074$ 
and assume $\eta^-$ on the small leaf along the $23$-cell}.
By a \cc{pem ztg4}{}, the $23$-cell can be included in the estimate,
\[
\sum \gg > 5 a_5 + b_5 (2\pi) > 0.
\]

\claim{Assume $\azim_{23}\le 1.074$ 
and assume $\eta^+$ on the small leaf along the $23$-cell}.
In this case we can split the $4$-cells into two groups of two, each with
a quarter paired with a neutralizing $4$-cell by a \cc{gamma10\_gamma11 fhbv2}{}.
\end{proof}

\begin{lemma}\guid{RSIWAMP}  There are at most four leaves.  That is, $C_{5,4}$ does not occur.
\end{lemma}

\begin{proof}  Otherwise, we have reduced to the case of five leaves, four $4$-cells,
and one $23$-cell, and all leaves small.  Let $A$ and $B$ be the two leaves
along the $23$-cell.  We consider various cases.

\claim{Assume $\azim_{23}>1.074$.}  This case is the same as in the previous lemma.
Without generality we now assume that $\azim_{23}\le 1.074$.

\claim{Assume $\eta_A ^-$ or $\eta_b^-$.}  In this case,
by a \cc{pem tew}{}, we have
\[
\sum_{(5)} \gg > 5 a_5 + b_5 (2\pi) > 0.
\]

\claim{Assume $\eta_A ^+$ and $\eta_B > 1.3$.}  If further, some $4$-cell is not a quarter, then by a \cc{gaz9 gaz6 gckb}{},
\[
\sum_{(5)} \gg > 0.606 (0.008) + 0.21849 + 3 (0.161517) - (2\pi - 0.606) 0.119482 > 0.
\]
And if all $4$-cells are quarters, then we can easily neutralizes the two pairs of $4$-cells.
\end{proof}

\begin{lemma}\guid{BKLETJQ} If $\QX$ is adjacent to a $23$-cell, then
$\gg_4 + \gg_3 > \eps$.
\end{lemma}

\begin{proof} This is a \cc{gamma10\_gamma11}{}.
\end{proof}

\begin{lemma}\guid{UTEOITF} There is no $23$-cell.  That is, $C_{4,3}$ does not occur.
\end{lemma}

\begin{proof} Again, there are several cases.  Label the three consecutive $4$-cells
as $A$, $B$, $C$.  All leaves are small, $B$ is a quarter with $\gg_B<0$ and its two leaves have $\eta^-$.  If any leaf has $\eta^+$, then we can easily neutralize.  Assume now that $\eta^-$ for all leaves.  If any $4$-cell is not a quarter, we can neutralize.

Assume that all $4$-cells are quarters.  

\claim{Assume $\azim_{23} \ge 2.089$.}  By a \cc{gaz6}{},
\[
\sum\gg > \azim_{23} (0.008) + 3 (0.161517) + (2\pi - \azim_{23}) (-0.119482) > 0.
\]

\claim{Assume $\azim_{23} \le 1.946$.}  By a \cc{azim1}{},
\[
\sum\gg > \azim_{23} (0.008) + 3 (-0.0659) + (2\pi - \azim_{23}) 0.42 > 0.
\]

\claim{Assume $1.946\le\azim_{23}\le 2.089$.}  By a \cc{txq}{},
\[
\sum\gg > 3\eps - 3\eps\ge 0.
\]
\end{proof}

\begin{lemma}\guid{LUIKGMH} The case $C_{3,3}$ does not occur.
\end{lemma}

\begin{proof} By a \cc{azim\_nqu azim\_c4 g\_qxd}{}, we have
$\azim < 2.8$. If $\gg<\eps$, then $\azim < 2.3$.  If $\gg <0$, then $\azim < 1.65$.

If there are at least two negative quarters, then the total angle is
\[
2\pi \le 1.65 + 1.65 + 2.8 < 2.8.
\]
If there is one negative quarter, it will be neutralized unless both others have angle $< 2.3$.  Then
\[
2\pi \le 1.65 + 2 (2.3) < 2\pi.
\]
\end{proof}

\begin{lemma}\guid{GRHIDFA}
The case $C_{4,4}$ does not occur.
\end{lemma}

\begin{proof} Let $k$ be the number of quarters.  We have seen that all faces are small.

If $k\le 2$, then by a \cc{gaz6 gaz9}{},
\[
k\QU + (4-k)\QX:\quad \sum\gg > k 1.61517 + (4-k) 0.213849 - 2\pi (0.119482) > 0.
\]
if $k=4$, then by a \cc{azim1}{},
\[
\sum \gg > -4 (0.0659) + 0.42 (2\pi) > 0.
\]

Finally assume that $k=3$.  If the fourth cell is a $\beta$-cell, we are done by
a \cc{ox3q1h}.  Thus, the fourth edge of the fourth cell is supercritical.  This gives by a \cc{gaz4 azim2}{},
\[
3\QU + \QX:\quad 3(-0.0142852) + (0.00457511) + 2\pi (0.00609451) > 0.
\]
\end{proof}

\subsection{Table of inequalities}

Here is a table of the inequalities that were used.  The first column gives
the name referenced in the text.  Then it is indicated if the inequality
is in the module {\it Ineq} or {\it Module} and the name of the inequality
in the computer development. The final column gives a shorthand form of the
inequality.

\begin{table}
\centering
\begin{tabular}{|l|l|l|l}
\text{\bf general bounds} &&&\vspace{6pt}\\
jsp & ineq & JSPEVYT &$\eta^+$\\
gckb & ineq & GCKBQEA & $\azim > 0.606$\\
azim\_c4 & ineq & BIXPCGW 6652007036 a2 & $\QU\vee\QX\Rightarrow \azim < 2.8$\\
ox3q1h & merge & ox3q1h\_merge & $3\QU+1 B\Rightarrow\sum_4 \gg > 0$.\vspace{6pt}\\
%
\text{\bf quarters} &&&\vspace{6pt}\\
gamma\_qu & ineq & BIXPCGW 9455898160 & $\QU\Rightarrow \gg >-\eps$\\
fhbv2 & ineq & FHBVYXZv2 a &$\QU\  \eta^+ \Rightarrow \gg > \eps$\\
quqy & merge & g\_quqya\_g\_guqyb &$\QU+\QY\Rightarrow \gg_4 + \gg_3 >\eps$\\
ztg4 & merge & ztg4 & $\QU\vee\QX\Rightarrow (a_5,b_5)$\\
azim1 & ineq & QITNPEA 5653573305 & $\QU\Rightarrow (\gg,\azim)$ \\
gaz4 & ineq & QITNPEA 6206775865 &$\QU\Rightarrow (\gg,\azim)$ \\
gaz6 & ineq & QITNPEA 3848804089 & $\QU\Rightarrow (\gg,\azim)$\vspace{6pt}\\
%
\text{\bf nonquarter $4$-cells} &&&\vspace{6pt}\\
gamma\_qx & merge & gamma\_qx & $\QX\Rightarrow \gg > 0$\\
g\_qxd & merge & gamma\_qxd & $\op{QXD}\Rightarrow \gg > \eps$\\
gamma10\_gamma11 & merge & gamma10\_gamma11& $\QX^{ss}+\QY\Rightarrow \gg_4+\gg_3>\eps$ \\
% gl4 & ineq & GLFVCVK4 2477216213 y4... &$\QX\wedge \neg B\Rightarrow \gg >\eps$\\
gamma8 &merge&QITNPEA\_9063653052\_weak &$\QX^{1s}\Rightarrow \gg > \eps$\\
gaz9 & ineq & QITNPEA 2134082733 & $\QX^{ss}\Rightarrow (\gg,\azim)$\\
azim2 & ineq & QITNPEA 9939613598 & $\QX^{ss,super}\Rightarrow (\gg,\azim)$\vspace{6pt}\\
%
\text{\bf $23$-cells} &&&\vspace{6pt}\\
cell3 & merge & cell3\_008\_from\_ineq &$\gg_3 > 0.008\azim$\\
grki & ineq & GRKIBMP a &$\gg_2 > 0.008\azim$\\
pem & ineq & PEMKWKU & $\QY\ \eta^+\ \eta^-\ \azim^-\Rightarrow (a_5,b_5)$\\
tew & ineq & TEWNSCJ & $\QY,\ 2\eta^-\Rightarrow (a_5,b_5)$\\
txq & ineq & TXQTPVC, IXPOTPA & $\QY\ 2\eta^{-}\azim[]\Rightarrow \gg > 3\eps$\\
%
\end{tabular}
\caption{inequalities}
\label{tab:myfirsttable}
\end{table}