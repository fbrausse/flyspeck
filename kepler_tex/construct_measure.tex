%% %% gives the construction of measure.
%% should probably be left out of the published
%% version.

%% Inner Measure on Measurable --> Measure.

% I won't put this in kepmacros, but it is used nowhere else.
\def\aspan#1{{\langle #1\rangle}}
\def\vc{\op{vc}} % vertical component

\section{Definition of Measure}

We take a decidedly unsophisticated approach to measure, because in
the end we only need to take very simple kinds of volumes.

Recall that a vector is by definition any element of
$\ring{R}^\infty = \cup_n \ring{R}^n$, under the natural nesting
    $$\ring{R}^1 \subset \ring{R}^2 \subset \cdots.$$


\begin{definition}
    The $m$-cube  $\op{cube}(P,m,i_1,\ldots,i_n)$
    with respect to $P=(v_0,e_1,\ldots,e_n)$ is the
    set
        $$\{P(\frac{(i_1+x_1)}{2^m},\ldots\frac{(i_n+x_n)}{2^m}) \mid 0 \le
        x_j
        < 1,\ \text{ for } j=1,\ldots,n.\}.$$
Let $X^P_m(A_1,\ldots,A_k)$ be the set of $m$-cubes with respect to
$P$ that meet each of the sets $A_j$, for $j=1,\ldots,k$.  In
particular, $X^P_m$ (written without arguments) is the set of all
$m$-cubes without restriction with respect to $P$. Let $Y^P_m(A)$ be
the set of $m$-cubes with respect to $P$ that are contained in $A$.
\end{definition}

\begin{definition}
    Let $A$ be a set of vectors.  The measure of $A$ with respect to
        $P=(v_0,e_1,\ldots,e_n)$ is the limit
    \begin{equation}\label{eqn:measure}
    \lim_{m\mapsto\infty} 2^{-m n}\card(X^P_m(A)).
    \end{equation}
    Write $\op{meas} (P,A)$ for the measure.
\end{definition}

This is the Calculus 101 style of computing measure.  Count the
number of properly aligned cubes that meet the set, times the
measure of an $n$-dimensional cube.  Take the limit as the size of
the cubes tends to zero  (Figure~\ref{fig:cubes}).

\begin{figure}[htb]
  \centering
  \myincludegraphics{\ps/cubes.EPS}
  \caption{Volumes can be approximated in Calculus
  101 fashion by counting cubes
   that meet a given set, scaled according to the size of the cubes.}
  \label{fig:cubes}
\end{figure}




\begin{definition}
    Let $$B(v,r) = \{u\in \ring{R}^\infty\mid |u - v|< r\}$$ be
    the open ball centered at $v$ with radius $r$.
    The boundary of a set of vectors $A$ with respect to $P$ is the set of
    $v\in \aspan{P}$ such that
        $$\{j\in\ring{N}\mid A \cap B(v,2^{-j})\cap\aspan{P}\ne\emptyset\}$$
    and
        $$\{j\in\ring{N}\mid (B(v,2^{-j})\cap\aspan{P})\setminus A\ne\emptyset\}$$
    are both infinite sets.
    Write $\partial_P A$ for the
    boundary of $A$ with respect to $P$.
\end{definition}

\begin{definition} \label{def:measurable}
    $B$ is measurable with respect to $P=(v_0,e_1,\ldots,e_n)$
    if the following hold.
    \begin{enumerate}
    \item $B \subset \aspan{P}$,
    \item $B \subset B(v_0,r)$, for some $r$.
    \item The limit of Equation~\ref{eqn:measure} exists for $A = \partial_P B$,
    and that limit is $0$.
    \end{enumerate}
\end{definition}

We will say that $B$ is a $P$-set if the first condition holds, that
it is bounded if the second condition holds, and that its boundary
is a null-set if the third condition holds.   That is, a measurable
set with respect to $P$ is defined as a bounded $P$-set whose
boundary is a null set.


\section{Algebra of Measurable Sets}

\begin{lemma}\label{lemma:meas-le}
    Let $P$ be an orthonormal $n$-frame.
    If $A$ is a bounded $P$-set, then the limit of Equation~\ref{eqn:measure}
    exists.  If $A\subset B$ are bounded $P$-sets then
    $$\op{meas}(P,A)\le \op{meas}(P,B).$$
\end{lemma}

\begin{proof}  The number of properly aligned $0$-cubes that meet a
bounded set is finite.  So $X^P_0(A)$ is a finite set.  Each
$m$-cube is a disjoint union of exactly $2^n$ size $m+1$ cubes. If a
size $m+1$ cube meets $A$, then the containing $m$-cube meets $A$.
It follows that the terms in the sequence of
Equation~\ref{eqn:measure} are monotonic decreasing, and bounded
below by $0$.  Thus, the limit exists.

To get the inequality $\op{meas}(P,A)\le \op{meas}(P,B)$, it is
enough to note that any $m$-cube meeting $A$ is also an $m$-cube
meeting $B$, so that $X^P_m(A) \subset X^P_m(B)$.
\end{proof}

\begin{lemma}  Suppose that $A$ and $B$ are bounded $P$-sets with
measure $0$.  Then $A\cup B$ is also a bounded $P$-set with measure
$0$.
\end{lemma}

\begin{proof}  It is elementary that $A\cup B$ is a bounded $P$-set.
Also, $$X^P_m(A\cup B) = X^P_m(A)\cup X^P_m(B).$$  The result
follows.
\end{proof}

\begin{lemma}
    Let $P$ be an orthonormal $n$-frame.
    If $A$ and $B$ are measurable with respect to $P$, then so are
    $$A\cap B,\quad A\cup B,\quad\text{ and }A\setminus B.$$
\end{lemma}

\begin{proof}  The property of being  bounded is preserved under
these boolean operations.  The property of being a subset of a
larger set is preserved under these boolean operations.  It remains
to be shown that the null-set property of the boundary
(Def~\ref{def:measurable}) is preserved under these boolean
operations.

By Lemma~\ref{lemma:meas-le}, it is enough to show that
    $$\partial_P (A \star B) \subset \partial_P A \cup \partial_P
    B,$$
where $\star$ is any of the operations $(\cup)$, $(\cap)$, or
$(\setminus)$.  In each case, the argument is similar.  Pick a point
$v\in\partial_P(A\star B)$. Let $U_j = B(v,2^{-j})\cap\aspan{P}$. To
say that $v$ is in the boundary is to say that the sets
    $$
    \{j \mid (A\star B)\cap U_j\ne\emptyset\},\quad
    \{j \mid U_j\setminus (A\star B)\ne\emptyset\}
    $$
are infinite. Using well-known facts about how finite and infinite
sets behave under the boolean operation $(\star)$ (such as the union
of finite sets being finite), we get that
    $$
    \{j \mid A'\cap U_j\ne\emptyset\},\quad
    \{j \mid U_j\setminus A'\ne\emptyset\}
    $$
are infinite for either $A'=A$ or $A'=B$.  The result follows.
\end{proof}

\begin{lemma}\label{lemma:alg}  Let $P$ be an orthonormal $n$-frame.  Let $A$ and $B$
be disjoint sets that are measurable with respect to $P$.  Then
    $$\op{meas}(P,A\cup B) = \op{meas}(P,A) + \op{meas}(P,B).$$
\end{lemma}

\begin{proof}    The definition of $X^P_m$ gives
    $$
    \card(X^P_m(A\cup B)) = \card(X^P_m (A)) + \card(X^P_m(B)) -
    \card(X^P_m(A,B)).$$
This implies immediately that
    $$\op{meas}(P,A\cup B) \le \op{meas}(P,A) + \op{meas}(P,B).$$
To complete the proof, we need to bound $\card(X^P_m(A,B))$.  It is
enough to show that for any $\epsilon >0$, we can choose $m'$
sufficiently large so that
    $$\card(X^P_{m'}(A,B)) < 2^{m' n} \epsilon.$$

Since $\partial_P (A\cup B)$ is a null-set, we can pick $m>0$ large
enough so that $\card(X^P_m(\partial_P (A\cup B)))/2^{m n} <
\epsilon/2^n$. Recall that the cubes are chosen so that they include
their left-hand endpoints but not their right-hand endpoints.  We
can take $2^n$ $m$-cubes for each $m$-cube $C_m^P(i_1,\ldots,i_n)$
by taking all the cubes $C_m^P(i_1- \delta_1,\ldots,i_n-\delta_n)$
for all functions $\delta:\{1,\ldots,n\}\to \{0,1\}$.  Enlarging the
set of $m$-cubes this way, we get a collection $X$ of at most $2^n
\card(X^P_m(\partial_P (A\cup B))) < 2^{m n}\epsilon$ $m$-cubes with
the property that any $m$-cube whose topological closure meets
$\partial_P(A\cup B)$ must lie in this collection.

Let $C$ be one of the finitely many $m$-cubes that are not in $X$
but that meets both $A$ and $B$. Suppose there is no positive lower
bound on the distance between points in $A\cap C$ and $B\cap C$. By
the compactness of the closure of $C$, there are convergent
sequences $a_n\in A\cap C$ and $b_n\in B\cap C$ tending to the same
point $c$ in the closure of $C$.  This point $c$ must lie in
$\partial_P A$, and then by definition $C$ lies in $X$ (contrary to
hypothesis). Thus, there is a positive lower bound to the distance
between points in $A\cap C$ and $B\cap C$.  Hence for some
sufficiently large $m_C$, any $m_C$-cube in $C$ does not meet both
$A$ and $B$.


Because $A$ and $B$ are disjoint, a point in the closure of $C$ that
comes arbitrarily close to both $A$ and $B$ must be in the boundary
of both $A$ and $B$, and hence in $\partial_P(A\cup B)$.  However,
any cube whose closure contains such a point lies in $X$.

Now as $C$ varies over the finitely many choices, we can pick $m'$
larger than all such $m_C$.  Then any $m'$-cube that meets both $A$
and $B$ is contained in a cube that lies in $X$.  For this choice of
$m'$, we conclude that
    $$\card(X^P_{m'}(A,B)) < 2^{m' n}\epsilon.$$
\end{proof}

\begin{lemma}
    Let $P$ be an orthonormal $n$-frame.  Let $A$ and $B$ be
    measurable with respect to $P$.  Then
        $$\begin{array}{lll}
        \op{meas}(P,A) &= \op{meas}(P,A\setminus B) + \op{meas}(P,A\cap
        B),\\
        \op{meas}(P,A\cup B) &= \op{meas}(P,A) + \op{meas}(P,B) - \op{meas}(P,A\cap B).
        \end{array}$$
\end{lemma}

\begin{proof} Use the previous lemma and the fact that $A$ is the disjoint
union of $A\setminus B$ and $A\cap B$, and that $A\cup B$ is the
disjoint union of $A$ and $B\setminus A$.
\end{proof}

\begin{lemma} Let $P$ be an orthonormal $n$-frame.
Let $A$ and $B$ be measurable with respect to $P$. Then
    $$\op{meas}(P,A\cup B) \le \op{meas}(P,A) + \op{meas}(P,B).$$
Let $C$ be a finite collection of sets that are measurable with
respect to $P$. Then
    $$\op{meas}(P,\bigcup C) \le \sum_{c\in C} \op{meas}(P,c).$$
\end{lemma}

\begin{proof}  The first statement follows by dropping the term
$\op{meas}(P,A\cap B)$ in the previous lemma.  The second statement
is by induction on the cardinality of $C$.
\end{proof}

\begin{lemma} Let $P$ be an orthonormal $n$-frame. Let $A$ be a measurable $P$-set.
Let $Y^P_m(A)$ be the set of $m$-cubes that lie entirely in $A$.
Then the sequence $2^{-m n}\card(Y_m(A))$ is increasing with limit
$\op{meas}(P,A)$.
\end{lemma}

\begin{proof} Let $B$ be a large $P$-aligned cube containing
$A$ and having width a power of $2$.  Then the approximations $2^{-m
n}\card(X_m(B))$ are exact and equal $\op{meas}(P,B)$.  We have that
$X_m(B)$ is the disjoint union of $Y^P_m(A)$ and $X_m(B\setminus
A)$.  Since $2^{-m n} \card(X_m(B\setminus A))$ are decreasing to
$\op{meas}(P,B\setminus A)$, we must have that
  $$2^{-m n}\card(Y^P_m(A))$$
are increasing to
    $$-\op{meas}(P,B\setminus A) + \op{meas}(P,B) =
    \op{meas}(P,A).$$
\end{proof}

\begin{lemma}  Let $P$ be an orthonormal $n$-frame.  Let $A$ be a bounded $P$-set.
Suppose that $A$ has measure $0$ with respect to $P$.  Then $A$ is
measurable.
\end{lemma}

\begin{proof} Pick $0 <\epsilon$.
As in Lemma~\ref{lemma:alg}, we get an $m$ and a collection $X$ of
at most $2^n \card(X^P_m(A)) < 2^{m n}\epsilon$ $m$-cubes with the
property that any $m$-cube whose closure meets $A$ must lie in this
collection.

$\partial_P A$ is a bounded $P$-set, so has a measure.  It meets
finitely many $m$-cubes.  We must check that $\partial_P A$ has
measure $0$.

Let $C$ be one of the finitely many cubes that are not in the
collection $X$.  Each point $v$ of $C$ contains a ball $B(v,2^{-j})$
that does not meet $A$, so $C$ does not meet $\partial_P A$.

Thus, the $m$-cubes counted for the measure of $\partial_P A$ lie in
$X$.  It follows that $\partial_P A$ has measure less than
$\epsilon$.  Since $\epsilon$ was arbitrary, $\partial_P A$ has
measure zero.
\end{proof}




\begin{lemma}  Let $P$ be an orthonormal $n$-frame.
Let $E$ be a $P$-set of measure zero.  Let $A$ be a measurable set
with respect to $P$.  Let $B$ be a finite collection of measurable
sets. Assume that $\bigcup B \subset A \cup E$.  Assume further that
    $$b \cap b' \subset E$$
for all $b,b'\in B$.  Then
    $$\sum_{b\in B} \op{meas}(P,b) \le \op{meas}(A).$$
\end{lemma}

\begin{proof}
The sets $b\setminus E$ for $b\in B$ are pairwise disjoint and their
union is a subset of $A\setminus E$.  Thus we find that
    $$\sum_{b\in B}\op{meas}(P,b\setminus E) \le \op{meas}(A\setminus
    E).$$
Since $E$ has measure $0$, we can drop $E$ everywhere from this
identity to get the result.
\end{proof}

\section{Examples of Null Sets and Measurable Sets}

Write $(x_1,\ldots,\hat x_k,\ldots,x_n)$ for the $n-1$-tuple that
omits the $k^{th}$ coordinate. If $f:\ring{R}^{n-1}\to\ring{R}$ is
any function, let $\Gamma^k_f:\ring{R}^{n-1}\to\aspan{P}$ be given
by
    $$
    (x_1,\ldots,x_{n-1}) \mapsto
    P(x_1,\ldots,x_{k-1},f(x_1,\ldots,x_{n-1}),x_k,\ldots,x_{n-1}).$$
That is, insert the value of $f$ at the $k^{th}$ coordinate position
while maintaining the other coordinates in the same relative order.
In the special case when $f=0$, $\Gamma^k_0$ is the function that
maps $(x_1,\ldots,x_{n-1})$ to $\aspan{P}$ in such a way that the
coefficient of the $k^{th}$ unit vector $e_k$ is zero.

\begin{definition}  We say that $f:\ring{R}^{n-1} \to \ring{R}$
satisfies a Lipschitz condition on $A\subset \ring{R}^{n-1}$ if
there exists a constant $M$ such that
    $$|f(x) - f(y)|\le M |x - y|,$$
whenever $x,y\in A$.
\end{definition}

\begin{lemma}
Let $P = (v_0,e_1,\ldots,e_n)$ be an orthonormal $n$-frame. Fix
$e_k$ for some $1\le k\le n$.  Let $A$ be a bounded $P$-set that is
a subset of
    $$\{P(x_1,\ldots,x_n) \mid x_k = 0\}.$$
Let $A'$ be the preimage of $A$ under $\Gamma^k_0$ in
$\ring{R}^{n-1}$.   Assume $f,g,f',g'$ satisfy a Lipschitz condition
on $A'$. Assume that $f(x)\le g(x)$ and $f'(x)\le g'(x)$.  Assume
there is a real number $0\le r$ such that $g(x)-f(x) = r(g'(x) -
f'(x))$ for all $x\in A'$. Let
    $$\begin{array}{lll}
    B=\{P(x_1,\ldots,x_n) &\mid (x_1,\ldots,\hat x_k,\ldots,x_n) \in
    A'\\
    &f(x_1,\ldots,\hat x_k,\ldots,x_n) \le x_k \le
    g(x_1,\ldots,\hat x_k,\ldots,x_n)\}.
    \end{array}
    $$
and
    $$\begin{array}{lll}
    B'=\{P(x_1,\ldots,x_n) &\mid (x_1,\ldots,\hat x_k,\ldots,x_n) \in
    A'\\
    &f'(x_1,\ldots,\hat x_k,\ldots,x_n) \le x_k \le
    g'(x_1,\ldots,\hat x_k,\ldots,x_n)\}.
    \end{array}
    $$
Then $B$ and $B'$ are bounded $P$-sets and
    $$\op{meas}(P,B) = r\,\op{meas}(P,B').$$
\end{lemma}

\begin{proof}
In the proof, we refer to the plane $\{P(x_1,\ldots,x_n)\mid
x_k=0\}$ as the horizontal plane, and vectors with the same
projection onto the horizontal plane as lying in the same vertical
column.

There is a constant $N'$ such that any two points in the same
$m$-cube satisfy $|x-y|\le N' 2^{-m}$.  From the Lipschitz
conditions on $f,f',g,g'$, there is a constant $N$ such that for any
$m$, if there is an $m$-cube $C$ containing both $P(x)\in A$ and
$P(y)\in A$, then $|h(x)-h(y)|\le N 2^{-m}$, for $h=f,f',g,g'$.
Since $A$ lies in the horizontal plane, there is at most one
$m$-cube in each vertical column that meets $A$.  Thus the number of
$m$-cubes meeting $A$ is at most $2^{(n-1)m} c_0$ for some constant
$c_0$ independent of $m$.

Every $m$-cube meeting $B$ lies in a vertical column above an
$m$-cube meeting $A$.  Fix an $m$-cube $C$ meeting $A$. Fix $x_C\in
 A\cap C$.  The number of $m$-cubes meeting $B$ in the vertical column over
 $C$ has the form
    $$2^m (f(x_C) - g(x_C)) + \epsilon_{C,m}$$
for some $0\le \epsilon_{C,m}\le 2 + 2N$.  (The constant $2$ on the
right is a discretization error coming from the fact that $f(x_C)$
and $g(x_C)$ might not be precisely aligned to the bottom and top
edges of a cube.  The constant $2N$ comes from the fact that $f$ and
$g$ are not constant, but may each take values varying by as much as
$N 2^{-m}$ on the cube.)  We have a corresponding formula for $B'$
(with primes inserted as appropriate).

$B$ and $B'$ are bounded, so the limit defining the measures exist
for both of them.  There exists some $N''$ such that
    $$
    \begin{array}{lll}
    |2^{-m n}\card(X^P_m(B)) &- r 2^{-m n} \card(X^P_m(B')|
    \vspace{6pt}\\
    &= |\sum_{C\in X^P_m(A)} 2^{-m n} (\epsilon_{C,m} - r
    \epsilon'_{C,m})|
    \vspace{6pt}\\
    &\le (2+2N)2^{-m n}\card(X^P_m(A))
    \vspace{6pt}\\
    &\le (2+2N) 2^{-m n} 2^{(n-1)m} c_0 \le 2^{-m} N''.
    \end{array}
    $$
Now take the limit as $m$ tends to infinity to get the result.
\end{proof}

\begin{lemma}  Let $P = (v_0,e_1,\ldots,e_n)$ be an orthonormal $n$-frame. Fix
some $k$ in the range $1\le k\le n$. Let $A$ be a bounded $P$-set
that is a subset of
    $$\{P(x_1,\ldots,x_n) \mid x_k = 0\}.$$
Let $A'$ be the preimage of $A$ under $\Gamma^k_0$ in
$\ring{R}^{n-1}$.   Assume $f$ satisfies a Lipschitz condition on
$A'$. Let
    $$
    B=\{P(x_1,\ldots,x_n) \mid (x_1,\ldots,\hat x_k,\ldots,x_n) \in A'\quad
    x_k = f(x_1,\ldots,\hat x_k,\ldots,x_n)\}.
    $$
Then $B$ is a bounded $P$-set and
    $$\op{meas}(P,B) = 0.$$
\end{lemma}

\begin{proof}
Let $f=f$, $g=f$, $f'=0$, $g'=0$, $r=0$ in the previous lemma.
\end{proof}

This lemma allows us to construct measurable sets as bounded
$P$-sets whose boundaries are a finite unions of graphs of Lipschitz
functions.  It is useful for us to prove that various functions are
Lipschitz.

\begin{lemma}\label{lemma:lip}  \begin{itemize}
    \item The function $f(x_1,\ldots,x_{n-1}) = \beta + \alpha_1 x_1 +
\cdots \alpha_n x_n$ satisfies a Lipschitz condition on any $A$.
    \item Let $|c| < r$.  The function
    $$
    f(x_1,\ldots,x_{n-1}) = b +  b'\sqrt{r^2
    - ((x_1-a_1)^2+ \cdots (x_{n-1}-a_{n-1})^2)}
    $$
    satisfies a Lipschitz condition on
    $$
    \{(x_1,\ldots,x_n)\mid ((x_1-a_1)^2 +\cdots (x_{n-1}-a_{n-1})^2) \le c^2\}.
    $$
    \item The function
    $$
    f(x_1,\ldots,x_{n-1}) = b + r \sqrt{(x_1-a_1)^2 + \cdots (x_{n-1}-a_{n-1})^2}
    $$
    satisfies a Lipschitz condition on any domain $A$.
    \end{itemize}
\end{lemma}

As a consequence, any bounded $P$-set whose boundary can be built up
in a finite way from the graphs of these functions is measurable.
Here are a few examples.

\begin{lemma} Let $P$ be an orthonormal $n$-frame.  The sphere of
radius $r$
    $$
    \{P(x_1,\ldots,x_n) \mid \sum_j (x_j-a_j)^2 = r^2\}
    $$
is a $P$-set of measure zero.
\end{lemma}

\begin{proof}
Let $v$ be a point on the unit sphere, with coordinates
$(x_1,\ldots,x_n)$.    Pick the index $k$ for which this $|x_k-a_k|$
is the smallest.  We have $|x_k - a_k|\ge r/\sqrt{n}$.
    The unit sphere is then a union of sets $X^{\pm}_k$ on which
$\pm(x_k - a_k) \ge r/\sqrt{n}$.  Each such piece is the graph of a
Lipschitz function of the form of Lemma~\ref{lemma:lip}.
\end{proof}

\begin{lemma} Let $P$ be an orthonormal $n$-frame.  The closed ball
of radius $r$
    $$
    \{P(x_1,\ldots,x_n) \mid \sum_j (x_j-a_j)^2 \le r^2\}
    $$
is measurable.  Similarly, the open ball of radius $r$ is
measurable.
\end{lemma}

\begin{lemma}  Let $P$ be an orthonormal $n$-frame.  The suspension\FIXX{insert}
is measurable.
\end{lemma}

\begin{lemma}
Let $P,P'$ be an $n$-frame.  Assume $\aspan{P} = \aspan{P'}$. Let
$A$ be the $P'$-coordinate rectangle
    $$\{v_0 + \sum_j x_j e'_j \mid 0\le x_j < a_j\}.$$
Then $A$ is $P$-measurable.
\end{lemma}

\begin{lemma} Let $P$ be an $n$-frame.  The measure of an $m$-cube $C$ is
$2^{-m n}$.
\end{lemma}

\begin{proof} Each term $2^{-m n}X^P_m(C)$
in the limit defining the measure is exactly $2^{-m n}$.  Thus the
limit is also $2^{-m n}$.
\end{proof}

\begin{lemma}
Let $P$ be an $n$-frame.  Let $0 < r$.  Then the closed unit ball
$B_r$ of radius $r$ (and any center) has positive measure:
    $$0 < \op{meas}(P,B_r)$$
\end{lemma}

\begin{proof} For sufficiently large $m$, it contains an $m$-cube,
which is a measurable set of positive measure.
\end{proof}

\section{Invariance of Measure}

\begin{lemma}  Let $P$ and $P'$ be two $n$-frames.  Assume that
$\aspan{P} = \aspan{P'}$.    Assume that for all $m$, that
$\op{meas}(P,C'_m) = 2^{-m n}$, where $C'_m$ is an $m$-cube with
respect to $P'$.  Then for every set $A$ that is measurable with
respect to $P$, we have
\begin{itemize}
        \item $A$ is measurable with respect to $P'$.
        \item $\op{meas}(P,A) = \op{meas}(P',A)$.
    \end{itemize}
\end{lemma}

\begin{proof}
We prove the second statement, assuming only that $A$ is a bounded
$P$-set.  It is then also a bounded $P'$-set.  It is enough to show
for any $0<\epsilon$ that $\op{meas}(P,A)\le
\op{meas}(P',A)+\epsilon$, because the statement is symmetrical in
$P$ and $P'$. Cover $A$ with $m$-cubes with respect to $P'$ of total
measure at most $\op{meas}(P',A) + \epsilon$. The $m$-cubes (with
respect to $P'$) are $P$-measurable have the same measure with
respect to $P$, so the $m$-cubes that cover $A$ have measure at most
$\op{meas}(P',A)+\epsilon$ with respect to $P$. Thus,
$\op{meas}(P,A) \le \op{meas}(P',A) + \epsilon$. The result follows.

Now we show that $A$ is measurable implies that $A'$ is measurable.
We have $\partial_P A = \partial_{P'} A$ since this definition
depends only on $\aspan{P}$.  We have that $\partial_P A$ is a
bounded $P$-set and also a bounded $P$-set.  Furthermore, by the
part of the lemma already established,
 $$0=\op{meas}(P,\partial_P A) =
    \op{meas}(P',\partial_P A) =\op{meas}(P',\partial_{P'} A).$$
So $A$ is also $P'$-measurable.
\end{proof}

\begin{lemma} Let $P = (v_0,e_1,\ldots,e_n)$ be an $n$-frame.
Let $P' = (v'_0,e_1,\ldots,e_n)$ be an $n$-frame.  Assume that
$\aspan{P} = \aspan{P'}$.  Let $A$ be a measurable set with respect
to $P$.  Then
    \begin{itemize}
        \item $A$ is measurable with respect to $P'$.
        \item $\op{meas}(P,A) = \op{meas}(P',A)$.
    \end{itemize}
\end{lemma}

\begin{proof} By the previous lemma,
It is enough to check that $\op{meas}(P,C'_m) = 2^{-m n}$, where
$C_m'$ is an $m$-cube with respect to $P'$.  Furthermore, it is
enough to prove the lemma when $v'_0 = v_0 + s e_k$, for some $k$.
Then by applying this special case $n$ times, for $k=1,\ldots,n$, we
obtain the general result.

Let $C'$ be an $m$-cube with respect to $P'$.  By Lemma~\ref{XX}
applied in the $k$-coordinate direction to the top and bottom faces
of $C'$, we see that $C'$ has the same measure with respect to $P$
as it does with respect to $P'$.
\end{proof}

\begin{lemma}  Let $P = (v_0,e_1,\ldots,e_n)$ be an $n$-frame.
Let $P' = (v_0,e'_1,\ldots,e'_n)$ be an $n$-frame.  Assume that
$\aspan{P} = \aspan{P'}$.  Let $A$ be a measurable set with respect
to $P$.  Then
    \begin{itemize}
        \item $A$ is measurable with respect to $P'$.
        \item $\op{meas}(P,A) = \op{meas}(P',A)$.
    \end{itemize}
\end{lemma}

\begin{proof}
We have $\op{meas}(P,C'_m) = c_0 2^{-m n}$, for some $0 < c_0$,
where $C_m'$ is an $m$-cube with respect to $P'$. By the previous
lemma, it is enough to check that $c_0=1$.  Let $B\subset \aspan{P}$
be the closed unit $n$-dimensional ball of radius $1$ centered at
$v_0$.  We have
    $$2^{-m n} \card(Y^{P'}_m(B)) \le \op{meas}(P',B)
    \le 2^{- m n} \card(X^{P'}_m(B)).$$
The limits exist and are all $\op{meas}(P',B)$. At the same time
$Y^{P'}_m$ represents disjoint measurable sets whose union is
contained in $B$ so that the measure of their union (with respect to
$P$) is at most the measure of $B$ with respect to $P$.  This gives
$$2^{-m n} c_0 \card(Y^{P'}_m(B)) \le \op{meas}(P,B).$$
Similar reasoning gives
$$\op{meas}(P,B)
    \le 2^{- m n} c_0\card(X^{P'}_m(B)),$$ and then
$$2^{-m n} \card(Y^{P'}_m(B)) \le \frac{\op{meas}(P,B)}{c_0}
    \le 2^{- m n} \card(X^{P'}_m(B)).$$
Comparing limits, we have
  $$\op{meas}(P,B) = c_0 \op{meas}(P',B).$$

There is a bijection
    $$\op{cube}(P,m,i_1,\ldots,i_n)\mapsto
    \op{cube}(P',m,i_1,\ldots,i_n)
    $$
between $X^P_m(B)$ and $X^{P'}_m(B)$. The bijection shows that the
cardinalities of these sets are equal. This implies that
    $$\op{meas}(P,B) = \op{meas}(P',B)$$
and that $c_0 = 1$.
\end{proof}

\begin{lemma}  Let $P$ be an orthonormal $n$-frame.  Let $A$ be a
bounded $P$ set that lies in a hyperplane.  Explicitly, assume there
are constants $(\alpha_1,\ldots,\alpha_n)$ (not all zero) such that
    $$A \subset \{P(x_1,\ldots,x_n)\mid \sum \alpha_i x_i = 0\}$$
Then $A$ has measure zero.
\end{lemma}

\begin{proof} Pick an orthonormal $n$-frame $P'$ such that
$\aspan{P} = \aspan{P'}$ and such that the hyperplane has the form
$\{P(x_1,\ldots,x_n)\mid x_n = 0\}$.  Then Lemma~\ref{XX} gives the
result.
\end{proof}

\begin{lemma}  Let $P$ and $P'$ be $n$-frames.  Let $A$ be a
measurable set with respect to $P$.  Suppose that $A\subset
\aspan{P'}$.  Then $A$ is measurable with respect to $P'$ and
$\op{meas}(P,A) = \op{meas}(P',A)$.
\end{lemma}

\begin{proof}
If $\aspan{P} = \aspan{P'}$, then use the preceding lemmas. If
different, then $A$ lies in a hyperplane $\aspan{P}\cap \aspan{P'}$
and its measure is zero with respect to both $P$ and $P'$.
\end{proof}

\begin{definition}
We say that $A$ is $n$-measurable if it is measurable with respect
to some orthonormal $n$-frame.  We write $\op{meas}_n(A)$ for
$\op{meas}(P,A)$ for any $n$-frame $P$ with respect to which $A$ is
measurable.  By the preceding lemmas, this is independent of the
choice of $n$-frame $P$.
\end{definition}

\begin{definition} We say that $A$ and $A'$ are congruent (through the frames $(P,P')$)
if there are orthonormal $n$-frames such that $A\subset \aspan{P}$,
$A'\subset \aspan{P'}$ such that $$P(x_1,\ldots,x_n) \in A
\Leftrightarrow P'(x_1,\ldots,x_n)\in A'.$$
\end{definition}

\begin{lemma}
Let $P$ and $P'$ be orthonormal $n$-frames.  Suppose that $A$ and
$A'$ are $n$-measurable sets.  Suppose that $A$ and $A'$ are
congruent through the frames $(P,P')$. Then $$\op{meas}_n (A) =
\op{meas}_n (A').$$
\end{lemma}

\begin{proof}
Compute the measure of $A$ with respect to the frame $P$ and $A'$
with respect to the frame $P'$.  There is a natural bijection
between $m$-cubes with respect to $P$ meeting $A$ and $m$-cubes with
respect to $P'$ meeting $A'$.  Thus, the measures are also equal.
\end{proof}



\chapter{Primitive Areas and Volumes}

\section{Area}

We write $\op{area}$ for $\op{meas}_2$.  This section calculates the
areas of some well-known regions in the plane.

\begin{definition}  Let $V$ be a set of vectors.  The simplex
$\op{spx}(V)$ spanned by $V$ is the set
    $$\{ \sum_{v\in V} x_v v \mid \sum_{v\in V} x_v = 1,\ 0\le x_v \text{ all } v\in V,
    \ x \text{ has finite support}\}.$$
\end{definition}

\begin{definition}  Let $P$ be an orthonormal $2$-frame.
    \begin{itemize}
    \item A rectangle is a set
    $$R(a,a',b,b') = \{ P(x,y) \mid a \le x \le a',\quad b\le y\le b'\}.$$
    \item Let $\op{disk}(P,r)$
be a $2$-dimensional closed ball of radius $r$.  Explicitly, with
respect to some $2$-frame $P$, we have
    $$
    \op{disk}(P,r) = \{P(x,y) \mid x^2 + y^2 \le r^2\}.
    $$
    \end{itemize}
\end{definition}

\begin{lemma}
Rectangles are measurable and if $a\le a'$ and $b\le b'$, then
    $$\op{area}(R(a,a',b,b')) = (a'-a)(b'-b).$$
\end{lemma}

\begin{proof} Calculate the area with respect to $P$.  Scaling in
the coordinate directions by Lemma~\ref{XX}, we may assume that
$a=0$, $a'=1$, $b=0$, $b'=1$.  It is then enough to show that a
$0$-cube has area $1$.  But this was already established in
Lemma~\ref{XX}.
\end{proof}



\begin{lemma} Let $V$ be finite of cardinality $n>1$.  Then $\op{spx}(V)$ is
$n-1$-measurable.
\end{lemma}

\begin{proof}\FIXX{Insert proof.}
\end{proof}

\begin{lemma} Let $v_0,v_1,v_2$ be three points in $\ring{R}^3$.
Then 
    $$
    \op{area}\op{conv}\{v_0,v_1,v_2\} = 
    |\det(v_1-v_0,v_2-v_0)|/2.
    $$
\end{lemma}



\begin{lemma}[wedges]\label{lemma:wedges}  Let $P=(v_0,e_1,\ldots,e_n)$ be an $n$-frame.
Suppose that $A$ is a measurable $P$-set that is cylindrically
symmetric in the sense that for all $r,\theta,\theta'$, we have
    $$P(r\cos\theta,r\sin\theta,x_3,\ldots,x_n)\in A
    \Leftrightarrow
    P(r\cos\theta',r\sin\theta',x_3,\ldots,x_n)\in A.$$
Let $W(P,A,\theta,\theta')$ be the set
    $$\{P(r\cos\theta'',r\sin\theta'',x_3,\ldots,x_n)\in
    A\mid \theta\le\theta''\le\theta'\}.$$
Assume $\theta\le\theta'\le 2\pi+\theta$. Then
$W(P,A,\theta,\theta')$ is measurable and
    $$\op{meas}_n(W(P,A,\theta,\theta'))=
    \frac{(\theta'-\theta)}{2\pi}\op{vol}(A).$$
\end{lemma}

\begin{proof}  The wedge $W(P,A,\theta,\theta')$ is measurable because it
is the intersection of $A$ with measurable sets defined by
hyperplanes.

Assume first that  $\theta'-\theta = 2\pi b/c$, for some natural
numbers $b$ and $c$.  We can slice the wedge into $b$ congruent
copies of $W(P,A,0,2\pi/c)$, and we can slice $A$ into $c$ congruent
copies of $W(P,A,0,2\pi/c)$, so the result easily holds in this
case.

If $\theta'-\theta$ is general, we have
    $$W(P,A,\theta,\theta + 2 r \pi)\subset W(P,A,\theta,\theta')
    \subset W(P,A,\theta,\theta + 2 r'\pi),$$
where $r$ and $r'$ are rational numbers chosen so that
    $0\le 2 r \pi \le \theta'-\theta \le 2 r'\pi$.
Taking limits, as $r,r'\mapsto (\theta'-\theta)/(2\pi)$ we get the
result.
\end{proof}

\begin{lemma} Let $P$ be a $2$-frame.
Assume that $0 < r$. Then
    $$\op{area}(\op{disk}(P,r)) = \pi r^2.$$
\end{lemma}

\begin{proof} We can use the classical method, due to Archimedes.

For each $n>2$, let
    $$\op{disk}(P,r,n)= \{P(r\cos\theta,r\sin\theta)\in \op{disk}(P,r,n)\mid
        \theta\in\leftclosed-\pi/n,\pi/n\rightclosed\}.$$
By Lemma[wedges]~\ref{lemma:wedges}, this set is measurable and we
have
    $$n\op{area}(\op{disk}(P,r,n)) = \op{area}(\op{disk}(P,r)).$$
On the other hand, we have containments of measurable sets:
    $$
    \begin{array}{lll}
    &\op{spx}(P(0,0),P(r\cos(\pi/n),r\sin(\pi/n)),P(r\cos(\pi/n),r\sin(\pi/n)))
    \vspace{6pt}\\
    &\quad\subset \op{disk}(P,r,n)
    \vspace{6pt}\\
    &\quad\subset \op{spx}(P(0,0),P(r,\tan(\pi/n)),P(r,-\tan(\pi/n))).
    \end{array}
    $$
This gives inequalities
    $$
    r^2 n \cos(\pi/n)\sin(\pi/n)\le \op{disk}(P,r) \le r^2 n\tan(\pi/n).
    $$
The limit of the middle term is sandwiched between the limits of the
two extremes, which are both $\pi r^2$.
\end{proof}




\section{Volume}

Write $\op{vol}$ for $\op{meas}_3$.

\begin{lemma}
Let $P$ be an orthonormal $3$-frame.  A rectangle is a set
    $$R(a,a',b,b',c,c') = \{ P(x,y,z) \mid a \le x \le a'\quad b\le y\le b'\quad
    c\le z \le c'\}.$$
Rectangles are $3$-measurable. If $a\le a'$  $b\le b'$, and $c\le
c'$ then
    $$\op{vol}(R(a,a',b,b',c,c')) = (a'-a)(b'-b)(c'-c).$$
\end{lemma}

\begin{proof} Calculate the area with respect to $P$.  Scaling in
the coordinate directions by Lemma~\ref{XX}, we may assume that
$a=0$, $a'=1$, $b=0$, $b'=1$, $c=0$, $c'=1$.  It is then enough to
show that a $0$-cube has area $1$.  But this was already established
in Lemma~\ref{XX}.
\end{proof}


\begin{lemma} Suppose that $v_0,v_1,v_2$ lie in an orthonormal $2$-frame $P'$.
The volume of $S=\op{spx}\{v_0,v_1,v_2,v_3\}$ is the same as the
volume of $S'=\op{spx}\{v_0,v_1,v_2,v_0+ v'_3\}$, where $v'_3$ is
the vertical component of $v_3$ with respect to $P$.
\end{lemma}


\begin{proof} Pick an orthonormal $3$-frame $P = (0,e_1,e_2,e_3)$ such that
$v_3 - v'_3$ is orthogonal to $e_2$ and $e_3$ and so that
$$\aspan{P'} = \aspan{v_0,e_1,e_2}.$$
Let $A$ be the projection of $S$ to
    $$\{P(0,y,z)\}.$$
It equals the projection of $S'$.  The vertical fiber over each
$a\in A$ in $S$ has the same length as each vertical fiber over
$a\in A$ in $S'$.  By Lemma~\ref{XX}, the volumes are equal.
\end{proof}




\begin{lemma} Let $$(x_1,\ldots,x_6) = \op{simplex\_coord}^2(v_0,v_1,v_2,v_3).$$
Then
    $$\op{vol}(\op{spx}\{v_0,v_1,v_2,v_3\}) =
    |\det(v_1-v_0,v_2-v_0,v_3-v_0)|/6.$$
\end{lemma}

\begin{proof} By
translation, we may assume that $v_0=0$. 
Assume first that $\{v_1,v_2,v_3\}$ are linearly dependent. Then
both sides are zero (Lemma~\ref{XX}).

Assume that the four vectors are linearly independent.   The previous lemma tells
us that we can move a vector $v_3$ in a way that does not change the
vertical component while preserving volume. The projections
$v_2\mapsto v'_2$ in\FIXX{Give reference.} fit this bill.  These projections also
preserve the determinant. So we may assume that $v_1-v_0$ and $v_2-v_0$ are
orthogonal to $v_3-v_0$. Applying such a transformation one more
time, we obtain a tetrahedron with $v_1-v_0$, $v_2-v_0$ and
$v_3-v_0$ all orthogonal and with the same volume and the same determinant
as the original. Scaling in the coordinate directions, we may assume
that $(v_0,\{v_1-v_0,v_2-v_0,v_3-v_0\})$ is an orthonormal frame.
For this tetrahedron,
the given determinant is $1$.

Thus, we show that this tetrahedron $A$ has volume $1/6$.  The unit
cube $C_0$ is a union of six tetrahedra congruent to $A$.
Explicitly, they are $\op{spx}(0,e_i,e_i+e_j,e_1+e_2+e_3)$ for all
choices of distinct indices $i,j$. The intersection of any two such
tetrahedra lies in a plane, and hence has measure $0$.  Thus,
    $$1 = \op{vol}(C_0) = 6\,\op{vol}(A).$$
The result follows.
\end{proof}

\begin{lemma}[Rogers simplex]  Let
$P=(v_0,e_1,e_2,e_3)$ be an orthonormal $3$-frame.   Fix $0\le a\le
b\le c$.
   $$\op{rog}(P,a,b,c) = \op{spx}\left\{0,a\, e_1,a\, e_1 + \sqrt{b^2-a^2}\, e_2,
     a\, e_1 + \sqrt{b^2-a^2}\, e_2 + \sqrt{c^2-b^2}\, e_3\right\}.$$
Then
    $$\op{vol}(R(a,b,c)) = \frac{1}{6} a
    \sqrt{b^2-a^2}\sqrt{c^2-b^2}.$$
\end{lemma}

\begin{proof}
Compute the determinant from the previous
lemma.  In fact, the matrix is diagonal with diagonal entries
	$$a,\quad \sqrt{b^2-a^2},\quad
	\sqrt{c^2-b^2}.$$
\end{proof}



\begin{lemma} Suppose
$\op{spx}\{v_0,v_1,v_2\}\subset \aspan{P'}$, where $P'$ is an
orthonormal $2$-frame. Let $h = |\vc(P',v_3)|$ be the length of
the vertical component of $v_3$ with respect to $P'$.  Then
    $$\op{vol}(\op{spx}\{v_0,v_1,v_2,v_3\}) = \frac{1}{3} |\vc(P',v_3)|
     \,
    \op{area}(\op{spx}\{v_0,v_1,v_2\}),
    $$

\end{lemma}

\begin{proof} By Lemma~\ref{XX}, we may assume that $v_3 = \vc(P',v_3) +
v_0$.  In this case, the projections of Lemma~\ref{XX} are the identity map.
The result then follows from Lemmas~\ref{XX} and \ref{XX}.\FIXX{Interp as
$|v_1||v_2|\sin\gamma$.}
\end{proof}

\begin{lemma}[suspension] Suppose
that $A$ is measurable with respect to the orthonormal $2$-frame
$P'=(v_0,e_1,e_2)$.  Let $\op{susp}(w,A)$ be the {\it suspension\/}
of $A$ over $w$:
    $$
    \op{susp}(w,A) = \{ t w + (1-t) v \mid 0\le t\le 1\quad v\in
    A\}.
    $$
Then $\op{susp}(w,A)$ is $3$-measurable and
    $$
    \op{vol}(C(w,A)) = \frac{1}{3} |\vc(P',w)|\, \op{area}(A).
    $$
\end{lemma}

\begin{proof}
Take inner and outer approximations by suspensions over squares to
get that the volume formula holds.

To check that the suspension is measurable, note that the boundary
of $\op{susp}(w,A)$ is contained in the union of $A$ and
$\op{susp}(w,\partial_P A)$. The first is easily shown to have
volume $0$. For the last, apply the volume formula for suspensions
to get that its area is
$$\op{area}(\partial_{P'} A) |\vc(P',w)| /3 = 0.$$
\end{proof}

\begin{lemma}  $\op{susp}^0(v,W(P',A,\theta,\theta')) =
W(P,\op{susp}^0(v,A),\theta,\theta')$.
\end{lemma}

\begin{proof} \FIXX{Insert}.
\end{proof}

\begin{lemma}[cylinder] Let $P=(v_0,e_1,\ldots,e_n)$ be an orthonormal $n$-frame.  Suppose
that $A$ is measurable with respect to the $n-1$-frame
$P'=(v_0,e_1,\ldots,e_{n-1})$.  Let
    $$
    \op{cyl}(a,b,A) = \{ P(x_1,\ldots,x_n) \mid P(x_1,\ldots,x_{n-1},0)\in A,
    \quad a\le x_n\le b\},
    $$
be the cylinder between levels $a$ and $b$.   Assume that $a\le b$.
Then $\op{cyl}(a,b,A)$ is $n$-measurable and
    $$
    \op{meas}_n(\op{cyl}(a,b,A)) = (b-a)\,\op{meas}_{n-1}(A).
    $$
\end{lemma}

\begin{proof}
By scaling by $a,b$ we may assume that $a=0,b=1$.  Take inner and
outer approximations by cylinders over squares to get that the
volume formula holds.

To check that the suspension is measurable, note that the boundary
of is contained in the union of the two faces $x_n=0$ and $x_n=h$,
and $\op{cyl}(a,b,\partial_P A)$. The first two are easily shown to
have volume $0$.  For the last, apply the measure formula for
cylinders to get that its measure is
$$\op{meas}_{n-1}(\partial_{P'} A)(b-a) = 0.$$
\end{proof}


\begin{lemma}[cylindrical slices]  Let $P$ be an $n$-frame.
Let $f:\leftclosed a,b \rightclosed\to \ring{R}^2$ be a continuous
and monotonic increasing function.  Let $A$ be a bounded $P$-set.
For $z\in\ring{R}$, let $A_z$ be the horizontal projection:
    $$A_z = \{P(x_1,\ldots,x_{n-1},0)\mid P(x_1,\ldots,x_{n-1},z)\in A\}.$$
Suppose that $A$ is contained in $\op{cyl}(a,b,A_{b})$,  Suppose
that for each $a\le a'\le b'\le b$, we have
    $$\op{cyl}(a',b',A_{a'})\subset \op{cyl}(a',b',A_{b})\cap A
    \subset \op{cyl}(a',b',A_{b'}).$$
Assume that each $A_z$, for $a\le z\le b$, is $n-1$-measurable with
measure $f(z)$.  Then the limit
    $$\lim_{n\mapsto\infty}
    \sum_{i=0}^{n-1}\frac{f(a+(b-a)i/n)}{n}$$
exists and is equal to
    $$\op{meas}_n(A).$$
\end{lemma}

\begin{proof}
Fix $i,a,b$.  Let $t_i = a + (b-a)i/n$.  Let $c(i,B)$ be an
abbreviation for
    $$
    \op{cyl}(t_i,t_{i+1}).
    $$
Each cylinder is measurable, because it is a cylinder over a
measurable set $A_z$. The intersection of two adjacent cylindrical
slices $\op{cyl}'(i,A_z)$ and $\op{cyl}'(i+1,A_{z'})$ lies in a
hyperplane, which has measure zero. By Lemma~\ref{XX},
    $$\op{meas}_n(\op{cyl}'(i,B)) = \op{meas}_{n-1}(B)/n,$$
which is independent of $i$.

We can use stacked cylindrical slices that are contained in $A$ to
give a lower bound on measure.  Thus
$$\sum_{i=0}^{n-1}\op{meas}_n(\op{cyl}'(i,A_{a+(b-a)(i/n)})) \le
\op{meas}_n(A)\le
\sum_{i=0}^{n-1}\op{meas}_n(\op{cyl}'(i,A_{a+(b-a)((i+1)/n)})).$$
The difference of the left and the right hand terms is a telescoping
series with sum
    $$\frac{1}{n} f(b) -
    \frac{1}{n}f(a) .$$
This difference tends to zero as $n$ tends to infinity. Thus, the
limit given in the statement of the lemma exists and is equal to the
measure.
\end{proof}

\begin{lemma} Let $P$ be an orthonormal $3$-frame.
Let $f(x) = \pi(r^2-x^2)$.  The volume of
    $$A=\{P(x,y,z) \mid z^2\le x^2 + y^2\le r^2,\quad r t\le z\le r\}$$
is $$\lim_{n\mapsto\infty} \frac{f(t + (1-t)i/n)}n.$$ The volume is
also equal to
    $$
    (r-r t)\pi r^2 - \frac{\pi r^2 - \pi r^2 t^2}3 =
    \frac{\pi}3 r^3 (t-1)^2 (t+2).
    $$
\end{lemma}

\begin{proof} For the first formula, apply cylindrical shells,
noting that each horizontal slice $A_z$ is of the form $B\setminus
C$, for two concentric disks $B$ and $C$.

For the second formula, note that there are suspensions $B$ and
$B'$, with $B'\subset B$, and $A\cap (B\setminus B')$ measure $0$
with $A\cup (B\setminus B')$ a cylinder of height $r-r t$ over a
disk of radius $r$.  The terms in the second formula represent the
terms of
    $$\op{vol}(A) = \op{vol}(A\cup (B\setminus B')) - (\op{vol}(B)-\op{vol}(B')).$$
\end{proof}

\begin{lemma} Let $B(r,t)$ be the part of the $3$-dimensional closed ball of
radius $r$ between heights $r t$ and $r$.  Explicitly, with respect
to some $3$-frame $P$, we have
    $$
    B(r) = \{P(x,y,z) \mid x^2 + y^2 + z^2\le r^2\quad r t \le z \le r\}.
    $$
Assume that $0\le t\le r$ and $0 < r$. Then
    $$\op{volume}(B(r,t)) = \frac{\pi}{3} r^3 (t-1)^2(t+2).$$
\end{lemma}

\begin{proof}  Use the method of cylindrical slices and the function
$f(x) = \pi(r^2-x^2)$, to get the the volume is
 $$\lim_{n\mapsto\infty} \frac{f(t + (1-t)i/n)}n.$$
This limit is calculated in the previous lemma, and is as given.
\end{proof}

In the case, $t=0$ we get that a hemisphere of the ball has volume
$2\pi r^3/3$.  Since the two hemispheres are congruent, we get that
a ball has volume $4\pi r^3/3$.

\begin{lemma} Let $P$ be a $3$-frame. Let
    $$
    C(r,\phi)=
    \{P(r\cos\theta\sin\phi',r\sin\theta\sin\phi',r\cos\phi')\mid
        0\le\phi'\le\phi\}.
    $$
Then $\op{vol}(C(r,\phi)) = {2\pi r^3(1-\cos\phi)}/3$.
\end{lemma}

\begin{proof} Let $t=\cos\phi$.  The set is a union of a suspension
of base area $\pi(r^2 - t^2)$ and height $r t$ (so volume $\pi
(r^2-t^2) r t/3$), and the chunk of the ball $B(r,t)$ computed in
the previous lemma. The intersection of these two sets has measure
$0$.  The desired volume is then the sum of these two volumes.  The
result follows easily.
\end{proof}



\begin{lemma}[spherical triangle]
Let $v_0,v_1,v_2,v_3$ be vectors in $\ring{R}^3$.  Write $w_i = v_i
- v_0$.  Assume that $w_1\cdot (w_2 \times w_3) \ne 0$.  Let
$\epsilon(i)\in\{1,2,3\}$ be the congruence class modulo three of
$i$. Let $\alpha_i$ be the dihedral angle formed by
$w_{\epsilon(i+1)}$ and $w_{\epsilon(i+2)}$ at $w_i$, for $i=1,2,3$.
Let $A$ be the set
    $$\{ v_0 + \sum_{i=1}^3 x_i w_i \mid \sum_i x_i^2\le 1\quad
    x_i\ge 0,\ i=1,2,3\}$$
Then $A$ is measurable and
    $$
    \op{vol}(A) = {(-\pi+\sum_{i=1}^3\alpha_i)}/3.
    $$
\end{lemma}

\begin{proof} For each function $\sigma:\{1,2,3\}\to\{\pm 1\}$, let
    $$
    Q_\sigma =
    \{ v_0 + \sum_{i=1}^3 x_i w_i \mid \sum_i x_i^2\le 1\quad
    x_i\sigma_i \ge 0,\ i=1,2,3\}.
    $$
The intersection of $Q_\sigma$ with $Q_{\sigma'}$ for
$\sigma\ne\sigma'$ lies in a hyperplane, so has measure $0$.  The
union of all $Q_\sigma$ is the unit ball at $v_0$ of volume
$4\pi/3$.

The sets $Q_\sigma$ are clearly measurable, being the intersection
of a unit ball with half-spaces.

If $\tau$ is the function always taking value $+1$, then $Q_{\tau}$
is $A$.  In particular, $A$ is measurable. Let $a$ be its volume.
If $-\tau$ always takes value $-1$, then $Q_{-\tau}$ is congruent to
$A$ under the antipodal map $v_0 + w\mapsto v_0-w$.

Any other $\sigma$ has exactly two positive values or two negative
values. If $\sigma$ has two positive values, and is negative for
argument $i$,  then $Q_\sigma\cup Q_{\tau}$ is a wedge of angle
$\alpha_i$ of the unit ball (called a lune). By
Lemma~\ref{lemma:wedges} this has volume $2\alpha_i/3$. Thus
$Q_\sigma$ has volume $2\alpha_i/3 - a$. Similarly $Q_{-\sigma}$ has
volume $2\alpha_i/3 - a$.

The unit ball is the sum of the volumes $Q_\sigma$, which gives
    $$\frac{4\pi}3 = 2a + \sum_{\sigma\ne\tau,-\tau}
    \op{vol}(Q_\sigma) =
    2 a + 2 (\sum_{i=1}^3 (\frac{2\alpha_i}{3} - a)).$$
Solving for $a$ gives the result.
\end{proof}

\section{Solid Angle}

\begin{definition}
Let $P=(v_0,e_1,e_2,e_3)$ be an orthonormal $3$-frame.    Let
$B(v_0)$ be the closed $3$-ball in $\aspan{P}$ of radius $1$. Let
$X$ be a set of vectors.  The solid angle of $X$ with respect to $P$
is the real number
    $$\op{sol}(P,X) = 3\,\op{vol}(B(v_0)\cap\op{cone}(P,X)).$$  We
    say that $X$ is spherically measurable if
    $B(v_0)\cap\op{cone}(P,X)$ is measurable.
\end{definition}

\begin{remark} Although we have not defined area for nonplanar
surfaces, if $X$ is a subset of the unit sphere centered at $X$, it
is possible to interpret $\op{sol}(P,X)$ as the area of $X$.  In
fact, this could serve as a definition of area for subsets of the
unit sphere.  For example, the solid angle of the full unit sphere
is $4\pi$.
\end{remark}

\begin{remark} Notice that there is a $3$ in the denominator of the
spherical triangle and Euler triangle volume formulas.  Interpreted
as a solid angle, the formula states that the solid angle of a
spherical triangle is $$\alpha + \beta + \gamma - \pi$$ where
$\alpha,\beta,\gamma$ are the angles of the spherical triangle.
\end{remark}








