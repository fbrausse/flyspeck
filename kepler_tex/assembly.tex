%\chapter{Assembly Listing}




\section{Listing}

\subsection{triangle}


\begin{lemma} \label{lemma:1pt}
%\proclaim{Lemma 3.13}
A quasi-regular tetrahedron $S$ satisfies $\sigma(S)\le 1\,\pt$.
Equality occurs if and only if the quasi-regular tetrahedron is
regular of edge length $2$.\index{Index}{quasi-regular!tetrahedron}
%
\end{lemma}

\begin{proof}
This is \calc{586468779}.
\end{proof}

\subsection{triangle-quad}

%% DCG 10.2, p100

%If more than $\squander$ are squandered at a vertex of a given type,
%then that type of vertex cannot be part of a centered packing scoring
%more than $8\,\pt$.  These relations between scores and vertex types
%will allow us to reduce the feasible planar maps to an explicit finite
%list. For each of the planar maps on this list, we calculate a second,
%more refined linear programming bound on the score. Often, the refined
%linear programming bound is less than $8\,\pt$.

This section derives the bounds on the scores of the clusters
around a given vertex as a function of the type of the vertex.
Define constants $\tlp(\pqr{(p,q,0)})/\pt$ by Table~\ref{eqn:old5.1}.  The
entries marked with an asterisk will not be needed.

\bigskip
% Table eqn:old5.1 of constants.


% page 246 of TeXBook
%\def\pt{\hbox{\it pt}}

\begin{equation}
\vbox{\offinterlineskip \hrule
\halign{&\vrule#&\strut\ \hfil#\hfil\ \cr   % "\ " was quad
height 7pt&\omit&&\omit&&\omit&&\omit&&\omit&&\omit&&\omit&\cr
&\hfil $\tlp(\pqr{(p,q,0)})/\pt$\hfil
        &&\hfil $q=0$\hfil
        &&\hfil1\hfil
        &&\hfil2\hfil
        &&\hfil3\hfil
        &&\hfil4\hfil
        &&\hfil5\hfil&
\cr height 7pt&\omit&&\omit&&\omit&&\omit&&\omit&&\omit&&\omit&\cr
\noalign{\hrule}
height7pt&\omit&&\omit&&\omit&&\omit&&\omit&&\omit&&\omit&\cr
&$p=0$&& *&& *&& 15.18&& 7.135&& 10.6497&& 22.27&\cr &1&&    *&&
*&&  6.95&& 7.135&&17.62  && 32.3&\cr &2&&    *&&
8.5&&4.756&&12.9814&&*&&*&\cr &3&& *&&
3.6426&&8.334&&20.9&&*&&*&\cr
&4&&4.1396&&3.7812&&16.11&&*&&*&&*&\cr
&5&&0.55&&11.22&&*&&*&&*&&*&\cr &6&&6.339&&*&&*&&*&&*&&*&\cr
&7&&14.76&&*&&*&&*&&*&&*&\cr
height7pt&\omit&&\omit&&\omit&&\omit&&\omit&&\omit&&\omit&\cr}
\hrule }
    %oldtag 5.1
    \label{eqn:old5.1}
\end{equation}
% based on sp in more.m



\begin{lemma}
    \label{lemma:pq}
    %{Proposition 5.2}
Let $S_1,\ldots,S_p$ and $R_1,\ldots,R_q$ be the tetrahedra and quad
clusters around a vertex of type $\pqr{(p,q,0)}$. Consider the constants of
Table~\ref{eqn:old5.1}.               Now,
    $$
    \begin{array}{lll}
    &\sum^p\tau(S_i) + \sum^q\tau_{R_i}(\Lambda,v) \ge \tlp(\pqr{(p,q,0)}),\\
    \end{array}
    $$
\end{lemma}

\begin{proof} Set
    $$
    (d_i^0,t_i^0)=(\dih(S_i),\tau(S_i)),\qquad
    (d_i^1,t_i^1)=(\dih(R_i),\tau(R_i)).
    $$
The linear combination $\sum^p\tau(S_i)+\sum^q\tau_{R_i}(\Lambda,v)$ is at
least the minimum of $\sum^p t_i^0+\sum^q t_i^1$ subject to
$\sum^p d_i^0+\sum^q d_i^1 = 2\pi$ and to the system of linear
inequalities \calc{830854305} and the system of linear
inequalities \calc{940884472} (obtained by replacing $\tau$ and
dihedral angles by $t_i^j$ and $d_i^j$). The constant $\tlp(\pqr{(p,q,0)})$
was chosen to be slightly smaller than the actual minimum of this
linear programming problem.

The entry $\tlp(\pqr{(5,0,0)})$ is based on Lemma~\ref{lemma:0.55}, $k=1$.
\end{proof}

\subsection{five triangles around a node}

\begin{lemma}
    \label{lemma:0.55}
    %proclaim{Lemma 5.3}
Let $v_1,\ldots, v_k$, for some $k\le 4$, be distinct vertices of
a centered packing of type $\pqr{(5,0,0)}$.  Let $S_1,\ldots, S_r$ be
quasi-regular tetrahedra around the edges $\{v_0,v_i\}$, for $i\le
k$. Then
    $$\sum_{i=1}^r \tau(S_i)> 0.55k\,\pt,$$
and
    $$\sum_{i=1}^r \sigma(S_i) < r\,\pt - 0.48k\,\pt.$$
\end{lemma}


\begin{proof}
We have $\tau(S)\ge 0$, for any quasi-regular tetrahedron $S$.  We
refer to the edges $y_4,y_5,y_6$ of a simplex indexed by the
usual edge conventions 
as its top edges. Set $\xi=2.1773$.

The proof of the first inequalities relies on seven
calculations\footnote{\calc{636208429}}. Throughout the proof, we
will refer to these inequalities simply as Inequality~$i$, for
$i=1,\ldots,7$.

We claim (Claim~1) that if $S_1,\ldots,S_5$ are quasi-regular
tetrahedra around an edge $\{v_0,v\}$ and if 
$y_5(S_1)\ge\xi$ is the length of a top edge $e$ on $S_1$ shared
with $S_2$, then $\sum_1^5\tau(S_i) > 3(0.55)\,\pt$.  This claim
follows from Inequalities~1 and~2 if some other top edge in this
group of quasi-regular tetrahedra has length greater than $\xi$.
Assuming all the top edges other than $e$ have length at most
$\xi$, the estimate follows from $\sum_1^5\dih(S_i)=2\pi$ and
Inequalities~3, ~4.

Now let $S_1,\ldots,S_8$ be the eight quasi-regular tetrahedra
around two edges $\{v_0,v_1\}$, $\{v_0,v_2\}$ of type $\pqr{(5,0,0)}$. Let $S_1$
and $S_2$ be the simplices along the face $\{v_0,v_1,v_2\}$. Suppose
that the top edge $\{v_1,v_2\}$ has length at least $\xi$. We claim
(Claim 2) that $\sum_1^8\tau(S_i)> 4(0.55)\,\pt$.  If there is a
top edge of length at least $\xi$ that does not lie on $S_1$ or
$S_2$, then this claim reduces to Inequality~1 and Claim 1. If any
of the top edges of $S_1$ or $S_2$ other than $\{v_1,v_2\}$ has
length at least $\xi$, then the claim follows from Inequalities~1
and ~2. We assume all top edges other than $\{v_1,v_2\}$ have length
at most $\xi$. The claim now follows from Inequalities~3 and ~5,
since the dihedral angles around each vertex sum to $2\pi$.

We prove the bounds for $\tau$.  The proof for $\sigma$ is
entirely similar, but uses the constant $\xi=2.177303$ and seven
new calculations\footnote{\calc{129662166}} rather than the seven
given above. Claims analogous to Claims~1 and 2 hold for the
$\sigma$ bound by this new group of seven inequalities.


Consider $\tau$ for $k=1$.  If a top edge has length at least
$\xi$, this is Inequality~1.  If all top edges have length less
than $\xi$, this is Inequality~3, since dihedral angles sum to
$2\pi$.

We say that a top edge lies around a vertex $v$ if it is an edge
of a quasi-regular tetrahedron with vertex $v$. We do not require
$v$ to be the endpoint of the edge.

Take $k=2$. If there is an edge of length at least $\xi$ that lies
around only one of $v_1$ and $v_2$, then Inequality~1 reduces us
to the case $k=1$.  Any other edge of length at least $\xi$ is
covered by Claim 1.  So we may assume that all top edges have
length less than $\xi$.  And then the result follows easily from
Inequalities~3 and ~6.

Take $k=3$. If there is an edge of length at least $\xi$ lying
around only one of the $v_i$, then Inequality~1 reduces us to the
case $k=2$. If an edge of length at least $\xi$ lies around
exactly two of the $v_i$, then it is an edge of two of the
quasi-regular tetrahedra. These quasi-regular tetrahedra give
$2(0.55)\,\pt$, and the quasi-regular tetrahedra around the third
vertex $v_i$ give $0.55\,\pt$ more. If a top edge of length at
least $\xi$ lies around all three vertices, then one of the
endpoints of the edge lies in $\{v_1,v_2,v_3\}$, so the result
follows from Claim 1. Finally, if all top edges have length at
most $\xi$, we use Inequalities~3, ~6, ~7.

Take $k=4$.  Suppose there is a top edge $e$ of length at least
$\xi$. If $e$ lies around only one of the $v_i$, we reduce to the
case $k=3$. If it lies around two of them, then the two
quasi-regular tetrahedra along this edge give $2(0.55)\,\pt$ and
the quasi-regular tetrahedra around the other two vertices $v_i$
give another $2(0.55)\,\pt$.  If both endpoints of $e$ are among
the vertices $v_i$, the result follows from Claim 2.  This happens
in particular if $e$ lies around four vertices.  If $e$ lies
around only three vertices, one of its endpoints is one of the
vertices $v_i$, say $v_1$.  Assume $e$ is not around $v_2$. If
$v_2$ is not adjacent to $v_1$, then Claim 1 gives the result. So
taking $v_1$ adjacent to $v_2$, we adapt Claim 1, by using all
seven Inequalities, to show that the eight quasi-regular
tetrahedra around $v_1$ and $v_2$ give $4(0.55)\,\pt$. Finally, if
all top edges have length at most $\xi$, we use Inequalities~3,
~6, ~7.
\end{proof}

In a special case, the constant of Lemma~\ref{lemma:0.55} can be
improved by a small amount.

\begin{lemma}
    \label{lemma:0.55A}
    %proclaim{Lemma 5.3}
Let $v$ be a vertex of a centered packing of type $\pqr{(5,0,0)}$.  Let
$S_1,\ldots, S_5$ be quasi-regular tetrahedra around the edge
$\{v_0,v\}$. Then
    $$\sum_{i=1}^5 \sigma(S_i) < 4.52\,\pt - 10^{-8}.$$
\end{lemma}

\begin{proof}
If any of the top edges has length greater than $\xi$, we use a
slightly improved calculation\footnote{\calc{241241504-1}} that
yields this constant. Otherwise, the same
calculation\footnote{\calc{82950290}} that was used in the previous
lemma gives the desired estimate
  $$
  \sum\sigma < 5(0.31023815) - 2\pi(0.207045) < 4.52\,\pt - 10^{-8}
  $$
\end{proof}

\subsection{limitations on type}%DCG 10.3, p103
    %\heads{6. Limitations on types}

Recall that a vertex of a planar map has type $\pqr{(p,q,0)}$ if it is the
vertex of exactly $p$ triangles and $q$ quadrilaterals. This
section restricts the possible types that appear in a centered
packing.

Let $t_4$ denote the constant $0.1317\approx 2.37838774\,\pt$.

\begin{lemma}\label{lemma:0.1317} If $R$ is a quad cluster, then
   $$\tau_R(\Lambda,v) \ge t_4.$$
\end{lemma}

\begin{proof}
A calculation\footnote{\calc{996268658}} asserts precisely this.
\end{proof}

\begin{lemma} \label{lemma:pq-impossible}
    %\proclaim{Lemma 6.1}
The following eight types $\pqr{(p,q,0)}$ are impossible:
    (1)  $p\ge 8$,
    (2)  $p\ge 6$ and $q\ge 1$,
    (3)  $p \ge 5$ and $q\ge 2$,
    (4)  $p \ge 4$ and $q\ge 3$,
    (5)  $p \ge 2$ and $q\ge 4$,
    (6)  $p \ge 0$ and $q\ge 6$,
    (7)  $p \le 3$ and $q=0$,
    (8) $p \le 1$ and $q=1$.
\end{lemma}

\begin{proof}
Calculations\footnote{\calc{657406669}, \calc{208809199},
\calc{984463800}, and \calc{277330628}} give a lower bound on the
dihedral angle of $p$ simplices and $q$ quadrilaterals at
$0.8638p+1.153 q$ and an upper bound of $1.874445 p + 3.247 q$. If
the type exists, these constants must straddle $2\pi$. One readily
verifies in Cases 1--8 that these constants do not straddle
$2\pi$.
\end{proof}

\begin{lemma}
    \label{lemma:pq-types}
    %\proclaim{Lemma 6.2}
If the type of any vertex of a centered packing is one of
$\pqr{(4,2,0)}$, $\pqr{(3,3,0)}$, $\pqr{(1,4,0)}$, $\pqr{(1,5,0)}$, $\pqr{(0,5,0)}$, $\pqr{(0,2,0)}$, %$\pqr{(7,0,0)}$,
then the centered packing does not contravene.
\end{lemma}


\begin{proof}  According to Table~\ref{eqn:old5.1},
we have $\tlp\pqr{(p,q,0)}> \squander$, for $\pqr{(p,q,0)} = \pqr{(4,2,0)}$, $\pqr{(3,3,0)}$,
$\pqr{(1,4,0)}$, $\pqr{(1,5,0)}$, $\pqr{(0,5,0)}$, or $\pqr{(0,2,0)}$. By
Lemma~\ref{lemma:sigma-tau}, the result follows in these cases.
\end{proof}



\begin{remark} \label{rem:pq-list}
In summary of the preceding two lemmas, we find that we may
restrict our attention to the following types of vertices.
    $$
    \begin{matrix}
   \pqr{(7,0,0)}&      &       &       &       \\
   \pqr{(6,0,0)}&      &       &       &       \\
   \pqr{(5,0,0)}&\pqr{(5,1,0)} &       &       &       \\
   \pqr{(4,0,0)}&\pqr{(4,1,0)} &       &       &       \\
        &\pqr{(3,1,0)} &\pqr{(3,2,0)}  &       &       \\
        &\pqr{(2,1,0)} &\pqr{(2,2,0)}  &\pqr{(2,3,0)}  &       \\
        &      &\pqr{(1,2,0)}  &\pqr{(1,3,0)}  &       \\
        &      &       &\pqr{(0,3,0)}  &\pqr{(0,4,0)}  \\
    \end{matrix}
    $$
It will be shown in Lemma~\ref{lemma:70}, that the type $\pqr{(7,0,0)}$
does not occur in a contravening centered packing.
\end{remark}

\subsection{expunge upright diagonals }

\begin{lemma}\label{lemma:ex-xtazimge}
Let $(\Lambda,v_0)$ be a centered packing.  Let $d$ be an upright
system diagonal with context $\xtazimge$.  Then $d$ can be expunged.
\end{lemma}


\begin{proof}
To expunge,
we need a new inequality for upright quarters of compression type:
   $$
   \Gamma(S) < \op{octavor}_0(S) + 0.5\,\op{dih}(S) - 0.54125.
   \qquad \text{(HYPOTHESIS I 9467217686)}
   $$
Case 1: Assume first that there is no masked quarter.
When all three upright quarters have compression type, we use
$\sum_{(3)} \op{dih}(S) \le \pi$ to get
   $$\sum_{(3)}(\sigma-\op{vor}_0) < 
      [0.5\pi - 3 (0.54125)] < 0.$$
When one of the three uprights has Voronoi type, we use
  $$\sum_{(3)}(\sigma - \op{vor}_0 ) <
    (0.5(\pi - 0.956_{\text{dih min}}) - 2(0.54125)) 
    - 0.02274_{\kappa}
    + \xi_V < 0.$$
When two or three uprights have Voronoi type, then we use
   $$\sum_{(3)}(\sigma - \op{vor}_0 ) <
   \xi_\Gamma + 2\xi_V -0.02274 < 0.$$
Case 2: Assume that there is masked flat quarter.  Then there
is exactly one and the upright diagonal is not enclosed over it.
The proof of Lemma~11.23 (page 121) applies to this situation
to show that we can expunge the $\xtazimge$ upright diagonal.
\end{proof}

\begin{lemma}\label{lemma:ex-x31}
Let $(\Lambda,v_0)$ be a centered packing.  Let $d=\{v_0,v_1\}$ be an upright
system diagonal.  Suppose that $d$ is enclosed over  
a flat quarter $Q=\{v_0,\ldots\}$ (which is necessarily masked).  
Then either
\begin{itemize}
  \item $d$ is a loop, having context $\x{(3,1)}$, and such that
    $|v_0-v_1| < 2.91$; or
  \item $d$ can be expunged.
\end{itemize}
\end{lemma}



\begin{proof}
  (Since one of the cases of
the function $\op{mask}(Q)$ is $\op{sv}_0$, when
$\eta_{456}\ge\sqrt2$, in this situation there is no difference between erasing and expunging.) Let $\{0,v\}$ be the upright
diagonal.  Let $\{0,v_1,v_2,v_3\}$ be the flat quarter, with
diagonal $\{v_1,v_3\}$.  If $|v-v_2|>2t_0$, then geometric
considerations show that the upright diagonal is not in the
$Q$-system.  (It is part of an isolated quarter, or inside
a quad.)  So we have $|v-v_2|\le 2t_0$.  We are in the situation of
$\op{bound}_{456}$ ($\eta_{456}\ge0$).  
Lemma~11.27 (page 124) and the estimates
of Inequality~13.1 (page 147) and~13.2 (page 148)
prove the the two upright quarters with anchor $v_2$ can
be erased.  In the case that there are three anchors to the upright diagonal, the proof is now complete.  In the remaining case, by Corollary~11.25, 11.17, there are four anchors, forming a loop (with four slices).  Moreover, at least three of the slices are upright quarters.  We use the following
calculation: if an slice has $y_4\in [2.91,3.2]$, then
\begin{equation}\label{eqn:0201}
 \kappa < -0.0201\qquad\text{(HYPOTHESIS I 1427782443)}.
\end{equation}
If an slice
in the loop has $y_4> 2.91$, then we expunge the two upright quarters with anchor $v_2$ as described above. The third quarter and
slice give
  $\xi_V - 0.0201_\kappa < 0$.
Also, the ``masked''
slice $S=\{0,v,v_1,v_3\}$ satisfies $\op{sv}_0(S) <0$
(because there are no quoins and $\phi(1,t_0)<0$).
Hence the upright diagonal can be expunged.
\end{proof}



\subsection{two darts at an upright node}


\begin{lemma}
Let $(\Lambda,v_0)$ be a centered packing, with $(\Lambda,\CalQ(\Lambda,v_0))$-compatible fan $H$. Let $N$ be an upright node
of cardinality $2$.  Then exactly one dart at $N$ is a quarter.
\end{lemma}

\begin{proof}
An upright node has at least one dart that is a quarter.
The dihedral angle of a quarter is
less than\footnote{\calc{971555266}} $\pi$, so it is
impossible for both darts to be quarters.
\end{proof}

\begin{lemma}\label{a:context11} %\label{eqn:4.9}
Let $(\Lambda,v_0)$ be a centered packing, 
with $(\Lambda,\CalQ(\Lambda,v_0))$-compatible fan $H$.  
Let $N$ be an upright node
of cardinality $2$.  Assume that exactly one dart $\alpha$ 
at $N$ is a quarter.  Let $\alpha'$ be the other dart at $N$.
Then
 $$
 \optt{mu}(\alpha) + \optt{kappa}(\alpha') < \optt{vor0}(\alpha).
 $$
\end{lemma}

\begin{proof}
This follows from
calculations\footnote{\calc{906566422}, \calc{703457064}, and
\calc{175514843}}.
\end{proof}

\subsection{three darts at an upright node}
%\section{Three anchors} %DCG 11.3,p.114
    \oldlabel{3.4}


\begin{lemma}\dcg{Lemma~11.2}{114}
    \oldlabel{3.4.1}
Let $(\Lambda,v_0)$ be a centered packing, 
with $(\Lambda,\CalQ(\Lambda,v_0))$-compatible fan $H$.
Assume that $N$ is a node
of cardinality three and that a unique dart $\alpha_0$  of $N$
is an upright quarter.  Let $\alpha_1,\alpha_2$ be the other
two darts of $N$.
Let 
  $$\optt{s}(\alpha)=\begin{cases}
  \optt{kappa}(\alpha),&\text{if $\alpha$ fitted crown}\\
  \optt{vor\_anal}(\alpha)-\optt{vor0}(\alpha),&\text{if $\alpha$ type C}\\
  0,&\text{otherwise}
  \end{cases}
  $$
Then 
  $$
  s(\alpha_1) + s(\alpha_2) + \optt{nu}(\alpha_0) < \optt{vor0}(\alpha_0).
  $$
% The upright diagonal can be erased in the context $\x{(1,2)}$.
\end{lemma}


\begin{proof}
Let $v_1$ and $v_2$ be the two anchors of the upright diagonal $\{v_0,v\}$
along the quarter. Let the third anchor be $v_3$.

Assume first that $|v-v_0|\ge 2.696$. If $Q$ is compressed,
then\footnote{\calc{73974037}} %A10  by $\A_{10}$,
the score is dominated by the truncated
function $\op{sv}_0$.  Assume $Q$ is decompressed. If $|v_1-v_0|$,
$|v_2-v_0|\le 2.45$, then a calculation\footnote{\calc{764978100}} %A11
gives the result. Take $|v_2-v_0|\ge
2.45$.  By symmetry, $|v-v_1|$ or $|v-v_2|\ge 2.45$. The case
$|v-v_1|\ge2.45$ is treated by another calculation.%
\footnote{\calc{764978100}} %A11
We take
$|v-v_2|\ge2.45$. Let $S=\{v_0,v,v_2,v_3\}$. If $S$ is of type $\SC$,
the result follows.\footnote{\calc{764978100}} %A11
$S$ is of type $\SC$, if and only if $y_4\le 2.77$, (because
by Lemma~\ref{tarski:eta245}, $\eta_{456}>\sqrt2$).
If $S$ is  not of type $\SC$, then by Lemma~\ref{tarski:eta696} and
Lemma~\ref{tarski:eta-rad},
we have $\rad_V(S) \ge\eta(|v-v_0|,2.45,2.45)\ge \eta(2,2.51,|v-v_0|)$.
This justifies the use of $\kappa$ (see Section~\ref{x-2.3}
Case (2)). That the truncated function dominates the score now
follows from a calculation.\footnote{\calc{618205535}} %A9

Now assume that $|v-v_0|\le 2.696$. If the simplices $\{v_0,v,v_1,v_3\}$
and $\{v_0,v,v_2,v_3\}$ are of type $\SC$, the bound follows from a
calculation.\footnote{\calc{73974037}} %A10
\footnote{\calc{764978100}} %A11
%$\A_{10},\A_{11}$.
If say  $S=\{v_0,v,v_2,v_3\}$ is not of type $\SC$,
then
    $$\rad_V(S)\ge\sqrt2>  \eta(2,2.51,2.696)\ge\eta(2,2h,2.51),$$
justifying the use of $\kappa$. The bound follows from further
calculations.\footnote{\calc{618205535}} %A9
\footnote{\calc{73974037}} %A10
\footnote{\calc{764978100}} %A11
%    $\A_9,\A_{10},\A_{11}$.
($\Gamma+\kappa <\octavor_0$,
etc.)
\end{proof}


\begin{lemma}\dcg{Lemma~11.3}{115}
    \oldlabel{3.4.2}
    \label{lemma:unerased}
Let $(\Lambda,v_0)$ be a centered packing, 
with $(\Lambda,\CalQ(\Lambda,v_0))$-compatible fan $H$.
Let $N$ be a node
of cardinality three.  Assume that exactly two darts at $N$
are quarters.  Assume that the width of the third dart $\alpha_0$
is at least $2\sqrt2$.\FIXX{(By the rules of Definition~\ref{def:q-system},
this is equivalent to saying that the node is not enclosed over
a masked flat quarter.)}
Then 
% The upright diagonal can be erased in the context $\x{(2,1)}$, provided
% the three anchors do not form a flat quarter at the origin.
\end{lemma}

\begin{proof}
In the absence of a flat quarter, truncate, score, and remove the
vertex $v$ as in the context $\x{(2,1)}$ of
Lemma~\ref{lemma:mixed-vor0}. 
\end{proof}

\subsection{six darts at an upright node}
%\section{Six anchors} %DCG 11.4, p.115
    \oldlabel{3.5}
% fill in dcg page number.
\begin{lemma}\dcg{Lemma~11.4}{XX}\label{lemma:context6-erase}{115}  
Let $(\Lambda,v_0)$ be a centered packing, 
with $(\Lambda,\CalQ(\Lambda,v_0))$-compatible fan $H$.
Let $N$ be a node of cardinality at least six.
Let the darts at $N$ that are quarters be
$\alpha_1,\ldots,\alpha_k$.
Then 
  $$
  \sum_{i=1}^k\optt{tau\_nu}(\alpha_i) > \squander.
  $$
%An upright diagonal has at most five anchors.
\end{lemma}

\begin{proof}
The proof relies on constants and inequalities from two
calculations.\footnote{\calc{729988292}} %A3
\footnote{\calc{83777706}} %A8
%$\A_3$ and $\A_8$.
If between two anchors there is a quarter, then the angle is
greater than $0.956$, but if there is not,  the angle is greater than
$1.23$.  So if there are $k$ quarters and at least six anchors, they
squander more than
    $$ k (1.01104) - [2\pi-(6-k)1.23]0.78701 > \squander,$$
for $k\ge0$.
\end{proof}

\subsection{five darts at an upright node}
\label{sec:5updart}

\begin{lemma}\dcg{Sec~11.7,intro}{118}\label{a:5dart:concave}  
Let $H=(D,n,e,f)$ be a geometric
hypermap.  Let $N$ 
be a node of $H$ of cardinality 5.    Assume that for all $\alpha\in N$
  $$
  \optt{upright}(\alpha).
  $$
Then $\optt{azim}(\alpha)\slt \pi$ for all $\alpha\in N$.
Hence $\optt{gap}(\alpha)$ or $\optt{slice}(\alpha)$.
\end{lemma}

\begin{proof}  The angle is at most $2\pi - 4(0.956) < \pi$.
\end{proof}

\begin{lemma}\dcg{Sec~11.7,Rem~11.3}{118}  An upright diagonal with
five darts has at most $2$ gaps.  More precisely, let $H$ be a geometric
hypermap.  Let $N$ 
be a node of $H$ of cardinality $5$.   Assume that $\optt{upright}(\alpha)$
for $\alpha\in N$.  There is a set $S\subset N$ of cardinality at most
$2$ such that $\optt{gap}(\alpha) \Rightarrow \alpha\in S$.
\end{lemma}

\begin{proof}
There are at most
two gaps by the calculation\footnote{\calc{83777706}} %A8
    $$3(1.65)+2(0.956)>2\pi.$$
\end{proof}

\begin{lemma}\label{a:5dart:3q}\dcg{Sec~11.7,in Lemma~11.14}{119}
Let $(\Lambda,v_0)$ be a centered packing, 
with $(\Lambda,\CalQ(\Lambda,v_0))$-compatible fan $H$.
Let $N$ be a node of $H$ of  of cardinality $5$.  Assume that
two of the darts at $N$ are gaps.  Then the other three darts
at $N$ are quarters.
\end{lemma}

\begin{proof}
By a calculation,\footnote{\calc{83777706}} %A8 $\A_8$,
the slices are all quarters,
    $1.23+2(1.65)+2(0.956)>2\pi$.
\end{proof}

\begin{lemma}\dcg{Lemma~11.14}{119}  Let $H$ be a geometric
hypermap.  Let $N$ be a node of $H$ of cardinality $5$.  Assume
that $\optt{upright}(\alpha)$ for $\alpha\in N$.  Assume that there
is a set $S\subset N$ of cardinality $3$ such that
  $$\optt{quarter}(\alpha)\Leftrightarrow \alpha\in S.$$
Then 
  $$
  \sum_{\alpha\in S} \optt{tau\_nu}(\alpha) \sgt \squander.
  $$
\end{lemma}

\begin{proof}
The dihedral angle of the quarters combined is less than $2\pi-2(1.65)$.  
The linear programming
bound based on various inequalities\footnote{\calc{729988292}} %A3 $\A_3$
is greater than $0.859>\squander$.
\end{proof}



\begin{lemma}\label{lemma:4-crowdedq}\dcg{Lemma~11.18}{119}
    %\oldlabel{3.8.3}
Let $(\Lambda,v_0)$ be a centered packing, 
with $(\Lambda,\CalQ(\Lambda,v_0))$-compatible fan $H$.
Suppose a node of cardinality five is an upright diagonal.
Suppose that exactly one of the five darts is a gap.
 If
any of the four slices is not an upright quarter then
the centered packing does not contravene.
\end{lemma}

\begin{proof}
We use a series of inequalities.\footnote{\calc{628964355}} %A5
\footnote{\calc{187932932}} %A7
\end{proof}

\begin{lemma}
\label{lemma:4-crowded}\dcg{Lemma~11.18}{119}
Let $(\Lambda,v_0)$ be a centered packing, 
with $(\Lambda,\CalQ(\Lambda,v_0))$-compatible fan $H$.
Suppose a node of cardinality five is an upright diagonal.
Suppose that exactly one of the five darts is a gap.
The sum of $\optt{nu}$ over the four slices is at most
$-0.25$. The sum of $\optt{tau\_nu}$ over the
four slices is at least $0.4$.
\end{lemma}

\begin{proof}
A list of inequalities\footnote{\calc{815492935}} %A2 $\A_2$
together with\footnote{\calc{83777706}} %A8
$\dih>1.65$ give the bound $-0.25$.
Further inequalities \footnote{\calc{729988292}} %A3 $\A_3$
give the bound $0.4$.  
\end{proof}


\begin{lemma}\dcg{Cor~11.9}{120}  
Let $(\Lambda,v_0)$ be a centered packing, 
with $(\Lambda,\CalQ(\Lambda,v_0))$-compatible fan $H$.
Suupose that the node $N$ is a $\xfgap$ upright
diagonal.  Let $\alpha_1,\ldots,\alpha_4$ be the quarters
and let $\alpha_0$ be a gap at the node $N$.  Then
  $$
  \optt{kappa}(\alpha_0) + \sum_{i=1}^4 \optt{tau\_nu}(\alpha_i)
  > 0.42274.
  $$
\end{lemma}

\begin{proof}  The crown along the gap,
with the bound of Lemma~\ref{lemma:4-crowded}, 
gives\footnote{\calc{618205535}} %A9
    $0.4-\kappa \ge 0.4+0.02274$
squandered by the upright quarters around a $\xfgap$ upright
diagonal.
\end{proof}


\begin{lemma} \label{a:min0-vor} 
Let $(\Lambda,v_0)$ be a centered packing, 
with $(\Lambda,\CalQ(\Lambda,v_0))$-compatible fan $H$.
Let $N$ be a node of degree three.
Assume that darts $\alpha_1$ and $\alpha_2$ are quarters,
and that $\alpha_0$ is a dart with azimuth angle at most $\pi$ 
and width less than $2\sqrt2$.
Then $$\optt{vu}(\alpha_1) + \optt{nu}(\alpha_2) <
     \optt{vor0}(\alpha_1) + \optt{vor0}(\alpha_2).$$
\end{lemma}

\begin{proof}
By a calculation\footnote{\calc{855677395}}, if $|v-v_0|\ge 2.69$,
then the upright quarters satisfy
    $$\nu < \op{sv}_0 + 0.01 (\pi/2-\dih)$$
so the upright quarters can be erased.  Thus we assume without
loss of generality that $|v-v_0|\le 2.69$.

By Lemma~\ref{tarski:E:part4:2}, we have $|v-v_0|>2.6$.
If $|v_1-v_2|\le 2.1$,  or $|v_1-v_3|\le 2.1$, then
Lemma~\ref{tarski:E:part4:3}, gives $|v-v_0|>2.72$, 
 contrary to assumption.  So take $|v_1-v_2|\ge 2.1$ and
$|v_1-v_3|\ge2.1$. Under these conditions we have the interval
calculation\footnote{\calc{148776243}} %A13
  $\nu(Q) < \op{sv}_0(Q)$ where $Q$ is the upright quarter.
\end{proof}


\subsection{four darts at an upright node}


\begin{lemma}\dcg{Remark~11.28}{125}
\label{remark:3rd-quarter} 
Let $(\Lambda,v_0)$ be a centered packing, 
with $(\Lambda,\CalQ(\Lambda,v_0))$-compatible fan $H$.
Let $N$ be an upright node of degree four at which there are
exactly three darts $\alpha_1$, $\alpha_2$, $\alpha_3$
that are upright quarters.  Assume that the node is enclosed
over a masked flat quarter.  
Then
 $$
 \sum_{i=1}^3\optt{vu}(\alpha_i) <
     \sum_{i=1}^3\optt{vor0}(\alpha_i) + \xiV.
 $$
\end{lemma}

\begin{proof}
 If we have an upright diagonal enclosed
over a masked flat quarter in the context $\x{(3,1)}$, then there are
three upright quarters.  By the same argument as in Lemma~\ref{a:min0-vor}, 
the two quarters over the masked flat quarter score $\le\op{sv}_0$. The
third quarter is dominated by $\op{sv}_0 + \xiV$.
\end{proof}


\begin{lemma}[Erasing four darts, no masked]
\dcg{Lemma~11.21}{120}
\oldlabel{3.9.1}
Let $(\Lambda,v_0)$ be a centered packing, 
with $(\Lambda,\CalQ(\Lambda,v_0))$-compatible fan $H$.
Let $N$ be a node
whose darts are upright.  Assume that $N$ has cardinality four.
Assume that there are at least as many non-slices as quarters at $N$.
Let $f(\alpha)$ be given as $\optt{nu}$ if $\alpha$ is an upright
quarter, $\optt{vor0}$ if it is another slice, and
$\optt{kappa}$ at darts that are not slices.  Then
  $$
  \sum_{\alpha\in N} f(\alpha) < \sum_{\alpha\in N} \optt{vor0}(\alpha).
  $$
\end{lemma}

\begin{proof}
By assumption, there are at least as non-slices as upright quarters. Each
non-slice drops us by $\xik$ and each quarter lifts us by at most%
\footnote{\calc{618205535}} %A9
\footnote{\calc{73974037}} %A10
\footnote{\calc{764978100}} %A11
$\xiG$. We have $\xikG<0$.
\end{proof}

\begin{lemma}[Erasing four darts, masked]\dcg{After Rem~11.22}{120}
Let $(\Lambda,v_0)$ be a centered packing, 
with $(\Lambda,\CalQ(\Lambda,v_0))$-compatible fan $H$.
Let $N$ be a node
whose darts are upright.  Assume that $N$ has cardinality four.
Assume that ``$N$ is enclosed over a flat quarter'' at dart $\beta$.
Assume that there are at least as many non-slices as quarters at $N$.
Let $f(\alpha)$ be given as $\optt{nu}$ if $\alpha$ is an upright
quarter, $\optt{vor0}$ if it is another slice, and
$\optt{kappa}$ at darts that are not slices.  Then
  $$
  \sum_{\alpha\in N} f(\alpha) < -0.0114 + 
  \sum_{\alpha\in N} \optt{vor0}(\alpha).
  $$
\end{lemma}

\begin{proof}
The azimuth angle at a non-slice is $>1.65$. 
We have
$0.0114< -2\xikG$.
\FIXX{This proof seems incomplete.  Don't we need $\optt{nu}(\alpha) <
\optt{vor0}$ based on something like DCG-Remark~11.22?}
\end{proof}
\FIXX{We remark that the preceding lemma is designed for use with 
the scoring of DCG-page123-2(c).}




\begin{lemma}\dcg{Lemma~11.23}{121}
    \oldlabel{3.9.2}
    \label{lemma:0.008}
Let $(\Lambda,v_0)$ be a centered packing, 
with $(\Lambda,\CalQ(\Lambda,v_0))$-compatible fan $H$.
Let
Let $N$ be a node that is an upright diagonal with four darts.  
Assume that one of the darts $\alpha_0$ is a gap and that the other
three $\alpha_1,\alpha_2,\alpha_3$ are slices.  
\begin{itemize}
\item If all of the slices are upright quarters, then
  $$
  \optt{kappa}(\alpha_0) + \sum_{i=1}^3 \optt{nu}(\alpha_i) <
  \sum_{i=1}^3 \optt{vor0}(\alpha_i) + 0.008.
  $$
%The slices can be erased with penalty  $\pi_0=0.008$. 
\item Assume that one of the slices is not an upright quarter.
Let 
$\optt{s}(\alpha)$ equal $\optt{nu}$ at slices that
are quarters and $\optt{vor0}$ at slices that are not.
Then
  $$
  \optt{kappa}(\alpha_0) + \sum_{i=1}^3 \optt{s}(\alpha) <
  \sum_{i=1}^3 \optt{vor0}(\alpha_i) + 0.00222.
  $$
% we can erase with penalty $\pi_0=0.00222$.
\end{itemize}
\end{lemma}


\begin{proof}
The constants and inequalities used in this proof can be found in
a series of calculations.%
\footnote{\calc{618205535}} %A9
\footnote{\calc{73974037}} %A10
\footnote{\calc{764978100}} %A11


First we establish the penalty $0.008$.   
By these inequalities, the result follows
if the diagonal satisfies $y_1\ge 2.57$.

Take $y_1\le 2.57$. If any of the upright quarters are decompressed, 
the result follows from $(\xikG+\xiG<0.008)$. If the edges
along the gap are less than $2.25$, the result follows from
$(-0.03883+3\xiG = 0.008)$. If all but one edge along the
gap are  less than 2.25, the result follows from $(-0.0325 + 2\xiG
+ 0.00928 = 0.008)$.

If there are at least two edges along the gap of length at
least $2.25$, we consider two cases according to whether they lie
on a common face of an upright quarter.  The same group of
inequalities gives the result. The bound $0.008$ is now fully
established.

\smallskip
Next we prove the bound involving $0.00222$, when one
of the slices is not a quarter.  If $|v-v_0|\ge2.57$, then
we use
    $$2\xiG + \xiV + \xik \le 0.00935+0.003521 -0.2274\le 0.$$
If $|v-v_0|\le2.57$, we use
    $$2(0.01561)-0.029 \le 0.00222.$$
\end{proof}


\begin{lemma}\dcg{Lemma~11.23}{121}
Let $(\Lambda,v_0)$ be a centered packing, 
with $(\Lambda,\CalQ(\Lambda,v_0))$-compatible fan $H$.
Let
Let $N$ be a node that is an upright diagonal with four darts.  
Assume that one of the darts $\alpha_0$ is a gap and that the other
three $\alpha_1,\alpha_2,\alpha_3$ are slices.  
Let 
$\optt{s}(\alpha)$ equal $\optt{nu}$ at slices that
are quarters and $\optt{vor0}$ at slices that are not.
Assume 
some upright quarter along this diagonal masks a flat quarter.
Then (1) or (2) holds.
   \begin{enumerate}
    \item 
  $$
  \optt{kappa}(\alpha_0) + \sum_{i=1}^3 \optt{s}(\alpha) <
  \sum_{i=1}^3 \optt{vor0}(\alpha_i) -0.0063.
  $$
  The diagonal of the flat is at least $2.6$, and the edge
    opposite the diagonal is at least $2.2$.
    \item 
    $$
  \optt{kappa}(\alpha_0) + \sum_{i=1}^3 \optt{s}(\alpha) <
  \sum_{i=1}^3 \optt{vor0}(\alpha_i) -0.0114.
  $$
   The diagonal of the flat is at least $2.7$, and the edge
    opposite the diagonal is at most $2.2$.
    \end{enumerate}
\end{lemma}



\begin{proof}
\smallskip
Let $v_1\ldots,v_4$ be the consecutive anchors of
the upright diagonal $\{v_0,v\}$ with $\{v_1,v_4\}$ the gap.
Suppose $|v_1-v_3|\le 2\sqrt{2}$.

By Lemma~\ref{tarski:dcg-p122}, 
the upright diagonal $\{v_0,v\}$ is not enclosed over
$\{v_0,v_1,v_2,v_3\}$.   
Thus, $\op{conv}^0\{v_1,v_3\}$ meets $\op{conv}\{v_0,v,v_2\}$ so that the
simplices $\{v_0,v,v_1,v_2\}$
and $\{v_0,v,v_2,v_3\}$ are decompressed.

To complete the proof of the lemma, we show that when
some upright quarter along this diagonal masks a flat quarter, 
 either (1) or (2) holds.
Suppose we mask a flat quarter $Q'=\{v_0,v_1,v_2,v_3\}$.
We have established that $\op{conv}^0\{v_1,v_3\}$ meets 
$\op{conv}\{v_0,v,v_2\}$.
To establish (1) assume that $|v_2-v_0|\ge 2.2$.  Lemma~\ref{tarski:last:E} 
gives
    $$|v_1-v_3|>2.6.$$
The bound $0.0063$ comes from
    $$\xikG + 2\xiV < -0.0063$$

To establish (2) assume that $|v_2-v_0|\le 2.2$. Lemma~\ref{tarski:last:E} gives
    $$|v_1-v_3|>2.7.$$
  If the simplex
$\{v_0,v,v_3,v_4\}$ is decompressed, then $$\xik + 3\xiV  < -0.0114$$
Assume that $\{v_0,v,v_3,v_4\}$ is compressed. We have
    $$-0.004131 +\xikG + \xiV \le -0.0114.$$
\end{proof}

\subsection{flat quarters} %



\begin{lemma}\dcg{Lemma~11.29}{125}
    \oldlabel{3.11.3}
$\mu < \op{sv}_0 +0.0268$ for all flat quarters. If the central
vertex has height $\le2.17$, then $\mu<\op{sv}_0+0.02$.
\FIXX{The lemma is a direct application of interval arithmetic.
Why repeat it here?}
\end{lemma}

\begin{proof}
This is an interval calculation.\footnote{\calc{148776243}} %A13
\end{proof}




\begin{lemma}\dcg{Lemma~11.30}{125}\label{lemma:1.32}
    \oldlabel{3.11.4}
Let $(\Lambda,v_0)$ be a centered packing, 
with $(\Lambda,\CalQ(\Lambda,v_0))$-compatible fan $H$.
Let $\alpha$ be a dart that
is standard\FIXX{meaning not upright and edges of length $2$ to $2.51$}.
Then the width of $\alpha$ is less than $2\sqrt2$.
\end{lemma}


\begin{proof} Let $S$ be the simplex inside the exceptional
cluster centered at $v_1$, with edges labeled by the usual edge
conventions, with $y_1=|v_1-v_0|$. The inequality $\dih\le 1.32$
gives the interval calculation\FIXX{source?} 
$y_4< 2\sqrt{2}$., so $S$ is a quarter.
\end{proof}



\begin{lemma}
Let $v$ be a corner of a flat quarter at which the
dihedral angle is at most $1.32$. 
Then $\hat\tau(Q)>3.07\,\pt$. Moreover, if $\hat\sigma=\op{sv}_0$ and if
$\eta_{456}\ge\sqrt2$, 
we may use the stronger constant
$\tau_0(Q)> 3.07\,\pt+\xi_V+2\xiG'$.
\FIXX{The lemma is a direct application of an interval
arithmetic calculation.  Why repeat it here?}
\end{lemma}


\begin{proof}
The result follows by
interval arithmetic.\footnote{\calc{148776243}} %A13
\end{proof}

\subsection{tame plane hypermap}


\begin{lemma}\label{a:6}\dcg{Lemma~21.4}{223} 
Formally contravening hypermaps satisfy Property
\ref{definition:tame:degree} of tameness: The cardinality of every
node is at least $2$ and at most $6$.
\end{lemma}

\begin{proof}
Let the type of the node be $(p,q,r)$.  If $r=0$, then the
impossibility of a node of cardinality $7$ or more is found in the
table entry $b\pqr{(7,0,0)}$ (Lemma~\ref{lemma:pq-types}). If $r\ge1$,
then Lemma~\ref{lemma:0.8638} shows that the azimuth angles of the
darts at the node cannot sum to $2\pi$:
    $$6 (0.8638) + 1.153 > 2\pi.$$
\end{proof}




\subsection{computer calculation}
\label{sec:ccc}

We have the following linear program. There are many different
choices of objective function and constraint {\it Csum} depending on
the particular constants $\sLP$, $\tauLP$, or $\tlp/\pt$ that need
to be computed.  In the linear program the constants $\pi$ and $\pt$
are replaced by numerical approximations.  Section~\ref{XX} explains
how the output from the numerical routines can be adjusted to yield
perfectly rigorous results.  The listing is in a format that can be
read by the program {\it LPSolve}.  See \cite{lpsolve}.

The origin of these inequalities is interval arithmetic.  They are
listed in nonlinear form at \cite{web}.  The numeric labels of the
equations here is consistent with the labels in that archive.

The correspondence between linear program variables in the program
listing and the variables in use elsewhere in the book is the
following.  Here $F$ is a face, and $x$ is a dart in that face.
    $$
    \begin{array}{lll}
    \card(F)=3 &\Rightarrow\  \azim(x)=\azim_3,\ \tau(F)=\op{tau}_3,\ \sigma(F)=\op{sigma}_3\\
    \card(F)=4 &\Rightarrow\  \azim(x)=\azim_4,\ \tau(F)=\op{tau}_4,\ \sigma(F)=\op{sigma}_4\\
    \end{array}
    $$

{ \obeylines\tt
  \hbox{}\parindent=4pt

 /* Change  "min/max" and "Csum", according to the objective */
 \ \hbox{}
 /* This example computes b(2,2) */
 // min: 2 tau3\_s + 2 tau4\_s;
 // Csum: 2 azim3 + 2 azim4 - twopi = 0;
 \ \hbox{}
 /* This example computes tauLP(2,2,5.0) */
 //min: 2 tau3 + 2 tau4;
 //Csum: 2 azim3 + 2 tau4 <= 5.0;
 \ \hbox{}
 /* This example computes sigmaLP(5,0,2pi-1.153) */
 max: 5 sigma3 + 0 sigma4;
 Csum: 5 azim3 + 0 sigma4 - twopi <= -1.153;
 \ \hbox{}
 /* Variable bounds */
 twopi: twopi =  6.2831853071795862;
 \ \hbox{}
 // pt = 0.055373645668464144;
 Ctaup: 0.055373645668464144 tau3\_s - tau3 = 0;
 Ctauq: 0.055373645668464144 tau4\_s - tau4 = 0;
 \ \hbox{}
 /* assumed conditions: */
 /* triangle tau */
 J927432550: 0.3897 azim3 + tau3 > 0.4666;
 J221945658: 0.2993 azim3 + tau3 > 0.3683;
 J53415898:  tau3 > 0.0;
 J106537269: -0.1689 azim3 + tau3 > -0.208;
 J254527291: -0.2529 azim3 + tau3 > -0.3442;
 \ \hbox{}
 /* triangle sigma */
 J539256862: sigma3 - 0.37898 azim3 < -0.4111;
 J864218323: sigma3 + 0.142 azim3 < 0.23021;
 Jsigma\_1pt: -Infinity <= sigma3 <= 1.0;
 J776305271: sigma3 + 0.3302 azim3 < 0.5353;
 \ \hbox{}
 /* quad  tau */
 J539320075: 4.49461 azim4 + tau4 > 5.81446 ;
 J122375455: 2.1406 azim4 + tau4 > 2.955;
 J408478278: 0.316 azim4 + tau4 > 0.6438;
 J996268658: tau4 > 0.1317;
 J393682353: -0.2365 azim4 + tau4 > -0.3825;
 J775642319: -0.4747 azim4 + tau4 > -1.071;
 \ \hbox{}
 /* quad sigma */
 J310151857: sigma4 - 4.56766 azim4 < -5.7906;
 J655029773: sigma4 - 1.5094 azim4 < -2.0749;
 J\_73283761:  sigma4 - 0.5301 azim4 < -0.8341;
 JLemm14\_11: -Infinity <= sigma4 <= 0;
 J\_15141595:  sigma4 - 0.3878 azim4 < -0.6284;
 J574391221: sigma4 + 0.1897 azim4 < 0.4124;
 J396281725: sigma4 + 0.5905 azim4 < 1.5707;
 \ \hbox{}

 /* all vars have lower bound 0, except sigma3, sigma4  */


}

\bigskip

We let $\tauLP(p,q,\alpha)$ denote the solution to this linear
program with objective $$\min: p\, \tau_3 + q\, \tau_4$$ and
constraint
$$\op{Csum}: p\, \azim_3 + q\,\azim_4 \le d.$$

We let $\tlp\pqr{(p,q,0)}$ denote the solution to this linear program with
objective $$\min: p \tau_3 + q \tau_4$$ and constraint
$$\op{Csum}: p\, \azim_3 + q\,\azim_4 =2\pi.$$  The constants $b\pqr{(p,q,0)}$
are computed as lower bounds satisfying $\tlp\pqr{(p,q,0)} > b\pqr{(p,q,0)}\,\pt$,
which the exception of the constants $b\pqr{(5,0,0)}$ and $b\pqr{(7,0,0)}$, which
are slight improvements on the linear programs.

We let $\sLP(p,q,\alpha)$ denote the solution to this linear program
with objective $$\max: p\, \sigma_3 + q\, \sigma_4$$ and constraint
$$\op{Csum}: p\, \azim_3 + q\,\azim_4 \le d.$$



\begin{lemma} We have the following estimates:
    $$
    \begin{array}{lll}
    &s_5+\sLP(5,0,2\pi-1.153)< c(8)\,\pt\\
    &s_6+\sLP(5,0,2\pi-1.153) < s_9\\
    &s_5+\sLP(5,0,2\pi-1.153)<s_8\\
    &(9-2(0.48))\,\pt+s_5+\sLP(2,2,2\pi-1.153)<8\,\pt\\
    &2t_5+\tauLP(4,0,2\pi-2(1.153))>\squander\\
    \end{array}
    $$
\end{lemma}

\begin{proof} Run the linear programs and see what you get.
\end{proof}

These are just a few of a long list of inequalities such as these
that will appear in the pages that follow.  They all come from the
same basic linear program with varying objective function and angle
sum constraint.



\begin{lemma}  If $v$  is a node of an exceptional face,
and if there are $6$ faces meeting at $v$, then the exceptional face
is a pentagon and the other $5$ faces are triangles.  In particular,
the node has type $(5,0,1)$.
\end{lemma}

\begin{proof}  Let $(p,q,r)$ be the type of the node.  We consider
several cases, according to the value of $p$.

{\bf($p\le2$)} If there are at least four non-triangular components at
the node, then the sum of azimuth angles around the node is at least
$4(1.153)+2(0.8638)>2\pi$, which is impossible.  (See
Lemma~\ref{lemma:0.8638}.)

{\bf($p=3$)} If there are three non-triangular components at the node,
then $\tau^*(H)$ is at least
$2t_4+t_5+\tauLP(3,0,2\pi-3(1.153))>\squander$.

{\bf($p=4$)} If there are two exceptional components at the node, then
$\tau^*(H)$ is at least $2t_5+\tauLP(4,0,2\pi-2(1.153))>\squander$.

If there are two non-triangular components at the node, then
$\tau^*(H)$ is at least  $t_5+\tauLP(4,1,2\pi-1.153)>\squander$.

{\bf($p=5$)} We are left with the case of five triangles and one
exceptional face.

When there is an exceptional face at a node of cardinality six, we
claim that the exceptional face must be a pentagon. If the face is a
heptagon or more, then $\tau^*(H)$ is at least
$t_7+\tauLP(5,0,2\pi-1.153) > \squander$.

If the face is a hexagon, then $\tau^*(H)$ is at least $t_6 +
\tauLP(5,0,2\pi-1.153) > t_9$. Also, $s_6+\sLP(5,0,2\pi-1.153) <
s_9$. The contour loop around the six faces has at most $9$ face
steps. Lemma~\ref{lemma:s9-t9} gives the bound of $8\,\pt$.
\end{proof}




\begin{lemma}
    \label{lemma:aggregate6}
    Let $(H,\azim,\flat,\sigma)$ be formally contravening.
    \begin{enumerate}
    \item The aggregate $F$ of the six faces at a node of type
    $(5,0,1)$ satisfies
            $$
            \begin{array}{lll}
            \sigma(F) < s_8,\\
            \tau(F) > t_8.
            \end{array}
            $$
    \item There are at most two nodes of type $(5,0,1)$.  If
        there are two, then they are non-adjacent vertices on a
        pentagon, as shown in Figure \ref{fig:doubledegree6}.  (The
        pentagon has a node of type $(1,0,1)$.)
    \end{enumerate}
\end{lemma}
\begin{figure}[htb]
  \centering
  \myincludegraphics{\ps/doubledegree6.eps}
  \caption{Non-adjacent nodes of cardinality $6$ on a pentagon}
  \label{fig:doubledegree6}
\end{figure}

\begin{proof}
We begin with the first part of the lemma. The sum  $\tau(F)$ over
these six standard components is at least
    $$t_5+\tauLP(5,0,2\pi-1.153)> t_8.$$
Similarly,
    $$s_5+\sLP(5,0,2\pi-1.153)<s_8.$$
%
We note that there can be at most one exceptional face with a node
of cardinality six.  Indeed, if there are two, then they must both
be nodes of the same pentagon:
    $$t_8+t_5>\squander.$$
Such a second node on the octagonal aggregate leads to one of the
follow constants greater than $\squander$.  These same constants
show that such a second node on a hexagonal aggregate must share two
triangular faces with the first node of cardinality six.
$$\begin{array}{lll}
    t_8 &+\tauLP(4,0,2\pi-1.32-0.8638),\quad\text{or}\\
    t_8 &+1.47\,\pt+\tauLP(4,0,2\pi-1.153-0.8638),\quad\text{or}\\
    t_8 &+\tauLP(5,0,2\pi-1.153) .
\end{array}
$$
(The relevant constants are found at Lemma~\ref{lemma:1.47} and
Lemma~\ref{lemma:0.8638}.)
\end{proof}

\begin{lemma}\label{a:311}  % Used in separation props of tame graphs.
Assume the node has type $(3,1,1)$.
 Assume the azimuth angle of the dart on the exceptional face is
least $1.32$, then
    \begin{equation}
    \tauLP(3,1,2\pi-1.32)>1.4\,\pt + t_4.
    \label{eqn:tau1.32}
    \end{equation}
This gives the bound in the sense of Lemma~\ref{lemma:split} at such
a node. 
\end{lemma}

\begin{lemma}\label{a:no-ef}  Let $v$ be a node of type
$(p,q,r)=(4,0,1)$, $(3,1,1)$, or $(3,0,2)$.  Assume that at $v$ the
exceptional darts are not flat.  Then
    $$\tauLP(p,q,\alpha) > ( p d(3) + q d(4) + a(p))\,\pt.$$
\end{lemma}

\begin{proof}
By Lemma~\ref{lemma:1.32},
the azimuth angles of the
exceptional components at $v$ are at least $1.32$.   The conclusion%
\footnote{\calc{551665569}, \calc{824762926}, and
\calc{325738864}}
%% K.C.-2002-version: 17.20 Group 20, 17.21 Group 21. (page 49).
follows.
\end{proof}

\section{Extracted from Elsewhere}

\subsection{quarters score nonpositive}

The following lemma appears as Lemma~\ref{lemma:quarter0}.
It has been implemented as assembly problem \assembly{JPOMPNK}.

\begin{lemma} %\label{lemma:quarter0}\dcg{Lemma 8.12}
Let $Q$ be a quarter in the $Q$-system (either flat or upright).
Then $\sigma(Q)\le 0$. 
\end{lemma}


\begin{proof}  
We make use
of the definition of $\sigma$ on quarters from
Definition~\ref{def:sigma}. The general context (that is, contexts
other than $\x{(1,1)}$ and $\x{(4,0)}$) of upright quarters is established
by the inequalities\footnote{\calc{522528841} and
\calc{892806084}} that hold for all upright quarters $Q$ with
distinguished vertex $v$:
    $$
    \begin{array}{lll}
    &2\Gamma(Q) + \op{sv}_0(v,Q) - \op{sv}_0(\hat v,Q,\lambda_{goct}) \le 0\\
    &\op{svan}(v,Q) + \op{svan}(\hat v,Q) 
  +\op{sv}_0(v,Q) - \op{sv}_0(\hat v,Q)\le0.
    \end{array}
    $$
For the remaining upright quarters (that is, contexts $\x{(1,1)}$ and $\x{(4,0)}$)
and for all flat quarters,
it is enough to show that $\Gamma(Q)\le0$, if $\eta^+\le\sqrt2$ and
$\op{svan}(Q,v)\le0$, if $\eta^+\ge\sqrt2$.

Consider the case $\eta^+\le\sqrt2$.  If $Q$ is a quarter such that
every face has circumradius at most $\sqrt2$,
then\footnote{\calc{346093004}} $\Gamma(Q)\le0$.  
Because of this, we may assume that the circumradius of $Q$ is
greater than $\sqr2$. 
Since
(Definition~\ref{def:svor})
    $$4\Gamma(Q)=\sum_{i=1}^4 \op{svan}(Q,v_i),$$
it is enough to show that $\op{svan}(Q)<0$.  Since $\eta^+\le\sqrt2$ 
and the circumradius is greater than
$\sqrt2$, $\op{sv}(Q,\sqrt2)$ is a strict truncation of the $V$-cell
in $Q$, so that
    $$\op{svan}(Q)<\op{svan}(Q,\sqrt2).$$
We show the right hand side is nonpositive.  Let $v$ be the
distinguished vertex of $Q$.  Let $A$ be $1/3$ the solid angle of
$Q$ at $v$ . By the definition of $\op{svan}(Q,\sqrt2)$, it is
nonpositive if and only if
    \begin{equation}
        A\le \doct \,\op{vol}(\op{VC}(Q,v)\cap B(v,\sqrt2)).
        \label{eqn:Adoct}
    \end{equation}
($\op{VC}(Q,v_0)$ is defined in Section~\ref{sec:rules}.) The
intersection $\op{VC}(Q,v)\cap B(v,\sqrt2)$ consists of six Rogers
simplices $R(a,b,\sqrt2)$, three conic wedges (extending out to
$\sqrt2$), and the intersection of $B(v,\sqrt2)$ with a cone over
$v$. By Lemmas~\ref{lemma:rog-doct}, \ref{lemma:wedge}, and
\ref{lemma:cone}, these three types of solids give inequalities like
that of Equation~\ref{eqn:Adoct}. Summing the inequalities from
these lemmas, we get Equation~\ref{eqn:Adoct}.

Consider the case $\eta^+\ge\sqrt2$ and $\sigma=\op{svan}$. If the
quarter is upright, then\footnote{\calc{40003553}} $\op{svan}(Q)\le0$.
Thus, we may assume the quarter is flat.  
The
analytic continuation defining $\op{svan}(Q)$ is the same
\footnote{This claim is justified by \calc{5901405}, which
shows that $\op{svan}(Q)\le0$ when the two functions differ.} as
    $$4(-\doct\op{vol}(X) + \sol(X)/3),$$
where $X$ is the subset of the cone at $v$ over $Q$ consisting of
points in that cone closer to $v$ than to any other vertex of $Q$.
The extreme point of $X$ has distance at least $\sqrt2$ from $v$
(since $\eta^+$ and hence the circumradius of $Q$ are at least
$\sqrt2$).  Thus,
    $$\op{svan}(Q) \le \op{svan}(Q,\sqrt2).$$
We have $\op{svan}(Q,\sqrt2)\le0$ as in the previous paragraph, by
Lemma~\ref{lemma:rog-doct}, \ref{lemma:wedge}, and
\ref{lemma:cone}.
\end{proof}



\subsection{a particular 4-circuit} %DCG 14.2, p158
\tlabel{sec:4circ}

This subsection bounds the score of a particular $4$-circuit on a
contravening planar hypermap.  The interior of the circuit
consists of two faces: a triangle and a pentagon.  The circuit and
its enclosed vertex are show in Figure \ref{fig:no4circuit} with
vertices marked $p_1,\ldots,p_5$.  The vertex $p_1$ is the
enclosed vertex, the triangle is $(p_1,p_2,p_5)$ and the pentagon
is $(p_1,\ldots,p_5)$.

\begin{figure}[htb]
  \centering
  \myincludegraphics{\ps/no4circuit.eps}
  \caption{A $4$-circuit}
  \label{fig:no4circuit}
\end{figure}

Suppose that $(\Lambda,v)$ is a centered packing whose associated hypermap
contains such triangular and pentagonal standard components. Recall
that $(\Lambda,v)$ determines a set $\Lambda(v_0,2t_0)$ of vertices in Euclidean
$3$-space of distance at most $2t_0$ from $v_0$, and that
each vertex $p_i$ can be realized geometrically as a point on the
unit sphere at $v_0$, obtained as the radial projection of
some $v_i\in \Lambda(v,2t_0)$.

\begin{lemma}\dcg{Lemma~14.3}{158}  
One of the edges $\{v_1,v_3\}$, $\{v_1,v_4\}$ has
length less than $2\sqrt{2}$.  Both of the them have lengths less
than $3.02$. Also, $|v_1-v_0|\ge2.3$.
\end{lemma}

\begin{proof} This is Lemma~\ref{tarski:4circuit}.
\end{proof}


There are restrictive bounds on the dihedral angles of the
simplices $\{v_0,v_1,v_i,v_j\}$ along the edge $\{v_0,v_1\}$. The
quasi-regular tetrahedron has a dihedral angle of at most%
\footnote{\calc{984463800}} $1.875$.  The dihedral angles of the
simplices $\{v_0,v_1,v_2,v_3\}$, $\{v_0,v_1,v_5,v_4\}$
adjacent to it are at most%
\footnote{\calc{821707685}}  $1.63$. The dihedral angle of the
remaining simplex $\{v_0,v_1,v_3,v_4\}$ is at most%
\footnote{\calc{115383627}} $1.51$.   This leads to lower bounds
as well. The quasi-regular tetrahedron has a dihedral angle that
is at least $2\pi - 2(1.63)-1.51 > 1.51$.  The dihedral angles
adjacent to the quasi-regular tetrahedron is at least $2\pi-
1.63-1.51-1.875> 1.26$. The remaining dihedral angle is at least
$2\pi-1.875-2(1.63) > 1.14$.

A centered packing $(\Lambda,v)$ determines a set of vertices $\Lambda(v,2t_0)$ that
are of distance at most $2t_0$ from $v$.  Three
consecutive vertices $p_1$, $p_2$, and $p_3$ of a standard component
are determined as the projections to the unit sphere of three
corners $v_1$, $v_2$, and $v_3$, respectively in $\Lambda(v,2t_0)$. By
Lemma~\ref{lemma:1.32}, if the interior angle of the standard
component is less than $1.32$, then $|v_1-v_3|\le\sqrt{8}$.

\begin{lemma}\dcg{Lemma~14.4}{159} \tlabel{lemma:11.16}
These two standard components $F=\{R_1,R_2\}$ give
    $\tau_F(\Lambda,v) \ge 11.16\,\pt$.
\end{lemma}

\begin{proof}
Let $\dih$ denote the dihedral angle of a simplex along a given
edge. Let $S_{ij}$ be the simplex $\{v_0,v_1,v_i,v_j\}$, for
$(i,j)=(2,3),(3,4), (4,5),(2,5)$. We have $\sum_{(4)}\dih(S_{ij})
= 2\pi$. Suppose one of the edges $\{v_1,v_3\}$ or $\{v_1,v_4\}$ has
length $\ge2\sqrt2$. Say $\{v_1,v_3\}$.

We have\footnote{\calc{572068135}, \calc{723700608},
\calc{560470084}, and \calc{535502975}}
    $$
    \begin{array}{lll}
    \tau(S_{25}) &- 0.2529\dih(S_{25}) > -0.3442,\\
    \tau_0(S_{23}) &- 0.2529\dih(S_{23}) > -0.1787,\\
    \hat\tau(S_{45}) &- 0.2529\dih(S_{45}) > -0.2137,\\
    \tau_0(S_{34}) &- 0.2529\dih(S_{34}) > -0.1371.\\
    \end{array}
    $$
We have a penalty $\xiG$ for erasing, so that
    $$
    \begin{array}{lll}
        \tau(\Lambda,v) &\ge \sum_{(4)}\tau_x(S_{ij}) - 5\xiG\\
                &>2\pi(0.2529)-0.3442\\
                &\qquad -0.1787-0.2137-0.1371-5\xiG\\
                &>11.16\,\pt,
    \end{array}
    $$
where $\tau_x=\tau,\hat\tau,\tau_0$ as appropriate.

Now suppose $\{v_1,v_3\}$ and $\{v_1,v_4\}$ have length $\le2\sqrt2$.
If there is an upright diagonal that is not enclosed over either
flat quarter, the penalty is at most $3\xiG+2\xiV$. Otherwise, the
penalty is smaller: $4\xiG'+\xiV$. We have
    $$
    \begin{array}{lll}
    \tau(\Lambda,v)
    &\ge \sum_{(4)}\tau(S_{ij})-(3\xiG+2\xiV)\\
    &>2\pi(0.2529)-0.3442\\
    &\qquad -2(0.2137)-0.1371 -(3\xiG+2\xiV)\\
    &>11.16\,\pt.\\
    \end{array}
    $$
\end{proof}


\subsection{a mixed quad bound}%DCG 10.5, p107

%In Definition~\ref{def:delta-e}, we found a region $\delta(v)$
%that lies outside the ball of radius $t_0$ at $0$ but inside
%$\op{VC}(v_0)$.  A formula for its volume is developed
%in Section~\ref{sec:anc}.  It introduces two functions
%$\cro$ and $\anc$.


\smallskip
If $(P,D)$ is a mixed quad cluster, let $(P,D')$ be the new quad
cluster obtained by removing all the enclosed vertices.  We define
a $V$-cell $V(P,D')$ of $(P,D')$ and the truncation of $V(P,D')$
at $t_0$. We take its score $\op{vor}_{0,P}(D')$  as we do for
standard clusters.  $(P,D')$ does not contain any quarters.

\begin{lemma} \label{lemma:mixed-vor0}
%\proclaim{Proposition 4.7}
If $(P,D)$ is a mixed quad cluster, $\sigma_P(D') <
\op{svR}(v,P,\Lambda,t_0)$.  
% Moreover, we can erase any number of the enclosed 
% vertices over the mixed quad cluster.
\end{lemma}

\begin{proof}
%
Suppose there exists an enclosed vertex that has context
$\x{(1,1)}$; that is, there is a single upright quarter
$Q$ and no additional anchors.  Let its edges be
$(y_1,\ldots,y_6)$, indexed by the usual edge conventions. In this
context $\sigma(Q)=\mu(Q)$. Let $v$ be the enclosed vertex.  To
compare $\sigma_P(\Lambda,v)$ with $\op{svR}(P,D')$, consider the $V$-cell
near $Q$. The quarter $Q$ cuts a wedge of angle $\dih(Q)$ from the
crown at $v$. There is an anchor term for the two anchors of $v$
along the faces of $Q$. Let $V_P^v$ be the truncation at height
$t_0$ of $V_P$ near $v$ and under the four Rogers simplices
stemming from the two anchors.
(Figure~\ref{fig:anchor-quarter-bis} shades the truncated parts of
the quad cluster.) As a consequence
\smallskip
    \begin{equation}
        \op{sovo}(V_P,\lambda_{oct}) <(1-\dih(Q)/(2\pi))\cro(y_1/2)+\anc(y_1,y_2,y_6)
        +\anc(y_1,y_3,y_5) +\op{sovo}(V_P^v,\lambda_{oct}).
    \label{eqn:4.8}
    \end{equation}
Combining this inequality with Lemma~\ref{a:context11}, we get the
result.


\begin{figure}[htb]
  \centering
  \myincludegraphics{noimage.eps}
  \caption{}
  \label{fig:anchor-quarter-bis}
\end{figure}



Now suppose there is an enclosed vertex $v$ with context
$\x{(2,1)}$. Let the quad cluster have corners $v_1$, $v_2$, $v_3$,
$v_4$, ordered consecutively.  Suppose the two quarters along $v$
are $Q_1=\{v_0,v,v_1,v_2\}$ and $Q_2=\{v_0,v,v_2,v_3\}$.  We consider
two cases.

\noindent Case 1:  $\dih(Q_1)+\dih(Q_2)<\pi$ or
$\rad_V(v_0,v,v_1,v_3)\ge\eta(|v-v_0|,2,2t_0)$. In this case, the use of
correction terms to the crown are legitimate as in
Definition~\ref{def:crown-tuple}. Proceeding as in context $\x{(1,1)}$, we
find that
\smallskip
    \begin{equation}
    \op{sovo}(V_P,\lambda_{oct}) < (1-(\dih(Q_1)+\dih(Q_2))/(2\pi))\cro(|v-v_0|/2)
    +\anc(F_1) +\anc(F_2) +\op{sovo}(V_P^v,\lambda_{oct}).
    \label{eqn:4.10}
    \end{equation}
Here $V_P^v$ is defined by the truncation at height $t_0$ under the
$V$-face determined by $v$ and under the Rogers simplices stemming
from the side of $F_i$ that occur in the definition of $\anc$. Also,
$\anc(F_i)=\anc(y_i,y_j,y_k)$ for a face $F_i$ with edges $y_i$
along an upright quarter. By a
calculation\footnote{\calc{554253147}} applied to both $Q_1$ and
$Q_2$, we have
    \begin{equation}
    \op{sovo}(V_P,\lambda_{oct}) +\sum_{i=1}^2\sigma(Q_i)
    < \op{sovo}(V_P^v,\lambda_{oct}) + \sum_{i=1}^2 \op{sv}_0(Q_i).
    \label{eqn:4.11}
    \end{equation}
That is, by truncating near $v$, and changing the scoring of the
quarters to $\op{sv}_0$, we obtain an upper bound on the score.

\noindent Case 2:  $\dih(Q_1)+\dih(Q_2)\ge\pi$ and
    $\rad_V(v_0,v,v_1,v_3)\le \eta_{V0}(v,v_0)$.
 In the mixed case,
$\sqr8<|v_1-v_3|$, so
$$\sqr2<{\frac{1}{2}}|v_1-v_3|\le\rad_V \le \eta_{V0}(v,v_0),$$
and this implies $|v-v_0|\ge 2.696$. We
have\footnote{\calc{855677395}}
$$\sum_{i=1}^2 \sigma(Q_i) < \sum_{i=1}^2 \op{sv}_0(Q_i) +
\sum_{i=1}^2 0.01(\pi/2-\dih(Q_i))< \sum_{i=1}^2 \op{sv}_0(Q_i).$$
Inequality~\ref{eqn:4.11} holds, for $V_P^v=V_P$.

In the general case, we run over all enclosed vertices $v$ and
truncate around each vertex.  For each vertex we obtain
Inequality~\ref{a:context11} or \ref{eqn:4.11}. These inequalities can
be coherently combined over multiple enclosed vertices because the
$V$-faces were associated with different vertices $v$ and none of
the Rogers simplices used in the terms $\anc()$ overlap. More
precisely, if $Z$ is a set of enclosed vertices, set $V_P^Z =
\cap_{v\in Z} V_P^v$, and $V_P^{v,Z} = V_P^Z\cap V_P^v$. Coherence
means that we obtain valid inequalities by adding the superscript
$Z$ to $V_P$ and $V_P^v$ in Inequalities~\ref{a:context11} and
\ref{eqn:4.11}, if $v\not\in Z$. In sum,
    $\sigma_P(\Lambda,v) < \op{svR}(v,P,\Lambda,t_0)$.
%
\end{proof}



\subsection{mixed bound} %DCG 10.14, p105 (moved -1.04 bound)




% Rewritten.
\begin{lemma}\dcg{Lemma~10.14}{105}\tlabel{lemma:1.04}
%\proclaim{Proposition 4.1}
Let $(\Lambda,v_0)$ be a centered packing with mixed quad cluster
$C=(v_1,v_2,v_3,v_4)$, with component $U_C$.
Then 
   $$
   \sigma(\Lambda,v_0,U_C) < -1.04\,\pt.
   $$
\end{lemma}

\begin{proof}
In a mixed quad cluster there is at least one enclosed vertex
$v \in U_C\cap \Lambda(v_0,\sqrt8)$.
Any enclosed vertex in a quad cluster has height at least $2t_0$
by Lemma~\ref{tarski:enclosed}. In particular, the anchors of an
enclosed vertex are corners of the quad cluster. There are no flat
quarters.  Both diagonals $\{v_1,v_3\}$, $\{v_2,v_4\}$ have length
at least $\sqrt8$.

We may assume that the height of each corner is at least $2.3$.
This is seen as follows.  An upper bound on the mixed quad cluster
is $\op{sv}_0(\cdot)$ by Lemma~\ref{XX}.  If any height is at least
$2.38$, then by a calculation\footnote{\newcalc{7710172071}},
the value is less than $-1.04\,\pt$.

We may assume that each top edge $|v_i-v_{i+1}|$ is at most $2.43$.
To show this we again use the upper bound $\op{sv}_0(\cdot)$.
This is a calculation\footnote{\newcalc{8227268739}} that uses\FIXX{DCG
Lemma 12.10 and Lemma 13.1  (update reference to this book)}
to reduce the dimension.


We erase all of the enclosed vertices except for one vertex $v$, which
we can do with the estimate of Lemma~\ref{lemma:mixed-vor0}.
The enclosed vertex has between zero and three anchors.
The upright
quarters around that vertex are scored with the function appropriate
to its context.
The rest of the quad cluster is estimated by the function $\op{svR}(\cdot,t_0)$.

\bigskip
We describe some deformations of the quad cluster.  Whenever a deformation
creates a new anchor for $v$, the length of that anchor remains fixed at $2t_0$,
and all further deformations are required to preserve that constraint.

If the enclosed vertex has zero  or one anchor, or two anchors at opposite corners,
then the entire quad
cluster satisfies $$\sigma(\Lambda,v,U)\le \op{sv}_0(\Lambda,v,U).$$
The function $\op{sv}_0$ is independent of the enclosed
vertex $v$.  In particular, we can move $v$ until 
two consecutive  corners of the quad cluster are anchors of $\{v_0,v\}$
and one of the distances $|v-v_i|=2.51$ for one of those two anchors,
and $|v-v_j|\le 2.51$ for the other.

We have that $|v-v_i|\le 3.15$ for each corner $v_i$.
In fact, if $|v-v_i| > 3.15$,
then the sum of the dihedral angles around $\{v_0,v\}$ satisfies%
\footnote{\newcalc{94010277298}, \newcalc{8713619400}.  
These are new interval calculations 
  that need to be typed in.  They appear in blueprint.cc, but not in kep\_ineq\_bis.ml}
%For an upright quarter in
%  $[2.51,\sqrt8][2,2.51]^2[2,2.42][2,2.51]^2$, $\dih < 2.059$.
%For a simplex in $[2.51,\sqrt8][2,2.51]^2 [2] [2.96,++][2,++]$,
%$\dih < 1.081$.
 $$
 2\pi = \sum_{i=1}^4 \dih_V(\{v_0,v\},\{v_i,v_{i+1}\}) < 2(2.059+1.081) < 2\pi.
 $$

Assume first that there is at most one upright quarter.
Let $Q$ be the simplex $\{v_0,v,v_i,v_j\}$ just constructed when
the original quad cluster has no upright quarters, and let $Q$
be the upright quarter when there is one.  Let $\{v_0,v\}$ be the
diagonal of $Q$.  

We then have the following contributions to the score: the fitted
crown, and the four simplices formed by $\{v_0,v\}$ and the four
corners $v_i$.  The fitted crown gives at most $-0.019$
by an interval calculation\footnote{\newcalc{XX}}.
The  edges $|v-v_i|$  have length at most $3.15$ by another interval
calculation\footnote{\newcalc{XX}}.  The upper bound on $Q$ is $0$.
The upper bound on the simplex that is not adjacent to $Q$ is $-0.05$,
and the other two have upper bound $0.005$, by more interval calculations\footnote{\newcalc{XX}}.  This gives
$$
  -0.019 + 0  - 0.05 +  2 (0.005) < -1.04\,\pt.
$$

Now assume that there are exactly two upright quarters along the
diagonal $\{v_0,v\}$.  They
are necessarily adjacent.  Let $v_1,v_2,v_3,v_4$ be the corners of
the quad cluster, 
indexed consecutively, with $v_2$ the anchor shared between the
quarters.  Deform the opposite vertex $v_4$ until one of three things
occurs.  
\begin{itemize}
\item $v_4$ becomes an anchor of $v$ and
  $$
  \op{azim}(v_0,v,v_3,v_4) + \op{azim}(v_0,v,v_4,v_1) > 2.9.
  $$
\item  $v_4$ becomes an anchor of $v$,
  $$
  \op{azim}(v_0,v,v_3,v_4) + \op{azim}(v_0,v,v_4,v_1)\le 2.9,
  $$
and $|v_3-v_1|\ge \sqrt8$.
\item $|v_3-v_4|=|v_4-v_1|=2$, $2t_0 \le |v-v_4| \le 3.15$, 
and $|v_4-v_0|=2$ or $2.51$.
\end{itemize}
One of these three situations must occur.

In the first case, the upper bound on the fitted crown is
$-0.019$ by an interval calculation\footnote{\newcalc{XX}, 2.9 case.}.
The upper bound on the two quarters is $0$, and on the two remaining
simplices combined\footnote{\newcalc{XX}. Was $-0.04$} is $-0.039$.
This gives
  $$
  -0.019 - 0.039 < -1.04\,\pt.
  $$

In the second case, the fitted crown gives $-0.019$, the two
remaining simplices combined\footnote{\newcalc{XX}. } give $-0.035$.
The two quarters combined\footnote{\newcalc{XX}} give $-0.0036$.
This gives
   $$
   -0.019 - 0.035 - 0.0036 < -1.04\,\pt.
   $$

In the third case, suppose first that $|v_4-v_0|=2.51$.  The
fitted crown gives $-0.019$, the two quarters give $0$, and the
remaining two give $-0.02$ each\footnote{\newcalc{XX}}  The
total is
$$
  -0.019 - 2 (0.02) < -1.04\,\pt.
$$

In the third case, suppose that $|v_4-v_0| =2$.  A calculation\footnote{\newcalc{XX}} shows that $\azim(v_0,v,w_1,w_2) < \pi/2$, for $(w_1,w_2)=(v_3,v_4),\ (v_4,v_1)$.  We also have $|v_1-v_3|\ge \sqrt8$ and $2t_0\le |v-v_2|\le 3.15$.
The two upright quarters give $0$.  The fitted crown, and two other
simplices combined give\footnote{\newcalc{XX}} less than $-1.04\,\pt$.
This completes the proof.
\end{proof}

%
%% BEGGNING OF OLD PROOF;
%%
%
%
%We may assume that if any top edge $|v_i-v_{i+1}|$ is at least $2.25$,
%then the other top edges $|v_j-v_{j+1}|$ are at most $2.22$.
%This argument uses the upper bound $\op{svR}(\cdot,t_0)$ and
%a calculation\footnote{\newcalc{6337649845} deleted from kep_ineq_bis.ml on March 21, 2008. Deprecated.} in the same way.
%%
%
%
%
%A calculation shows%
%\footnote{\calc{XX}. This is a new interval calculation that
%needs to be verified: If $(R,D)$ is any quad cluster with both
%diagonals greater than $\sqrt8$, and some distance $|v_i-v_{i+1}|>2.38$,
%then $\op{svR}(\Lambda,v,R,t_0) < -1.04\,\pt$.  XX This isn't quite right.
%{\tt (VorVc[2, 2, 2, Sqrt[8], 2, 2] + VorVc[2, 2, 2, Sqrt[8], 2.42, 2])/pt.}
%gives $-0.888822$.  I was leaving out half the quad.  Perhaps this reduction
%won't be needed.
%} 
%that if $\sigma_R(\Lambda,v) \ge -1.04\,\pt$, then 
%$|v_i-v_{i+1}|\le 2.38$ for $i=1,2,3,4$.  We assume these
%constraints.
%
%We may use the deformation of Lemma~\ref{x-4.9.2}
%at each vertex $v_i$ that is not an anchor of $\{v_0,v\}$ so either
%it becomes an anchor (with $|v-v_i|=2t_0$), or it satisfies
%  $$|v_i-v_{i+1}|=|v_i-v_{i-1}|=2;\quad |v_i-v_0|\in\{2,2t_0\}.$$ 
%
%In the course of deformation (say of corner $v_1$),  
%the diagonal $\{v_1,v_3\}$ may reach length $\sqrt8$.  
%(Note that when this happens, the enclosed vertex $v$ must have anchors at two opposite corners
%by Lemma~\ref{XX}, which implies that there are at least three anchors.)
%In this case, we stop
%the deformation at that corner and continue with another
%corner (say $v_2$, if another non-anchor exists)
%until its deformation
%is complete.  By tarski\tarf{XX}, we cannot have both diagonals $\{v_1,v_3\}$ and $\{v_2,v_4\}$
%drop to $\sqrt8$, unless $\{v_0,v\}$ has four
%anchors.  The result, after the deformations are complete, may
%have a diagonal of length less than $\sqrt8$.  At this point,
%that constraint is no longer needed.  Its only purpose was to
%fulfill a hypothesis of Lemma~\ref{x-4.9.2}.
%
%We summarize the conditions that hold after deformation.
%There are two, three, or four anchors.  Each
%corner that is not an anchor has distance $2$ from the two adjacent corners
%and has height $2$ or $2.3$.  There are at most two upright quarters in the
%$Q$-system.  (If there are three, then they have four anchors, hence four
%quarters.  If there are four, then it is not a mixed quad cluster.)  If there
%are two upright quarters in the $Q$-system, they are adjacent to each other.
%Every edge $\{v_i,v_{i+1}\}$ that is not on an upright quarter has length $2$.
%Every edge $\{v_i,v_{i+1}\}$ that is on an upright quarter has length at most $2.42$,
%and if it is at least $2.25$ then every other edge $\{v_j,v_{j+1}\}$ has length at most $2.22$.
%
%
%We consider the lengths of the edges $\{v,v_i\}$.  Each upright quarter that is not in 
%the $Q$-system was obtained by deformation, hence has at least one anchor $v_i$
%such that $|v-v_i|=2t_0$.
%
%
%We claim that if corner $v_i$ is not an anchor of $\{v_0,v\}$,
%then $2t_0 < |v-v_i| \le 3.15$.  
%
%
%
%
%We write the upper bound on the mixed quad cluster
%as a sum of contributions from the four simplices
%$\{v_0,v,v_i,v_{i+1}\}$.  
%Each upright quarter in the $Q$-system is scored by $\sigma(v_0,S,c)$ according to context $c$.
%The rest of the quad cluster is scored by $\op{svR}(\cdot,t_0)$.  
%
%\FIXX{There is a problem starting from here.
%   Things don't really check in cfsqp.  (See blueprint.cc).
%   The arguments that follow are based on an old bound of 2.38 rather than 2.42.
%   Redo all this with different constants.}
%
%
%If $\{v_0,v\}$ has just two anchors (say at adjacent corners $v_1,v_4$),
%then the contribution from $\{v_0,v,v_1,v_4\}$ is at most $0$.  In
%fact, if the simplex is an upright quarter in the $Q$-system, this
%follows from Lemma~\ref{XX}.  If the simplex is scored by $\op{svR}(\cdot,t_0)$,
%then (say) $|v-v_1|=2t_0$ and the result is a calculation.%
%\footnote{\calc{XX}.  This is a new calculation.  
%It needs to be verified.  If an upright
%simplex satisfies $[2.51,\sqrt8][2,2.51]^2[2,2.38][2.51][2,2.51]$,
%then $\op{svR}(v,S,t_0) < 0$.}  Moreover, the contributions from the other
%three simplices give%
%\footnote{\calc{XX}. These are new calculations.  They need
%to be verified.  On
%$[2.51,\sqrt8]\{2,2.51\}^2[2][2.51,2.96]^2$, $\op{svR}(\cdot,t_0) < -0.0475$; and
%on $[2.51,\sqrt8]\{2,2.51\}[2,2.51][2][2.51,2.96][2,2.51]$,
%   $\op{svR}(\cdot,t_0) < -0.0055$.
%}
% $$\sum_{i=1}^3\op{svR}_{0,R,V}(v_0,\{v,v_i,v_{i+1}\}) <
%   -0.0055 - 0.0475 - 0.0055 < -1.04\,\pt.$$
%
%If $\{v_0,v\}$ has three anchors (say $v_1,v_2,v_4$), then
%writing $\sigma'_i$ for the upper bound on the contribution
%from $\{v_0,v,v_i,v_{i+1}\}$, we have%
%\footnote{\calc{XX}.  These are new calculations that need to
%be verified.  For an upright quad with $y_4\in[2,2.38]$, we
%have $\sigma(Q) < \epsilon_1 -0.08(\dih(Q)-\pi/2)$.  I added
%this small $\epsilon_1$ to make it more likely that it will follow
%by a linear program for the other inequalities for $\sigma(Q)$.
%Note that there are various cases, according to the context; we
%haven't erased anything here.  We have the similar
%  $\op{svR}_0 < \epsilon_1 -0.08(\dih(Q)-\pi/2)$,
%when $[2.51,\sqrt8][2,2.51]^2[2,2.38][2,2.51][2.51]$.  Then we
%have $\op{svR}_0 < \epsilon_2  -0.08(\dih(Q)-\pi/2)$,
%when $[2.51,\sqrt8][2,2.51]^2[2][2,2.51][2.51,2.96]$.
%}
%  $$
%  \sum_{i=1}^4\sigma'_i < 
%  \sum (\epsilon_i -0.08 (\dih_V(\{v_0,v\},\{v_i,v_{i+1}\})-\pi/2))
%  = \sum\epsilon_i = -1.04\,\pt.
%  $$
%where $\epsilon_2=\epsilon_3 = -0.54\,\pt$ and $\epsilon_1=\epsilon_4 =
%0.02\,\pt$.


%% OLD PROOF. 
%
%We generally truncate the $V$-cell at $\sqr2$ as in the proof of
%Theorem~\ref{lemma:quad0}.  By that lemma, it breaks the $V$-cell
%into pieces whose score is nonpositive. Thus, if we identify
%certain pieces that score less than $-1.04\,\pt$, the result
%follows. Nevertheless, a few simplices will be left untruncated in
%the following argument. We will leave a simplex untruncated only
%if we are certain that this is justified.
%%% Avoid mention of orien-tation.
%% Each of its faces has positive orien-tation
%% and that the simplices sharing a face $F$ with $S$ either lie in
%% the $Q$-system or have positive orien-tation along $F$.  
%If so, we may use\footnote{\calc{185703487},
%\calc{69785808}, and \calc{104677697}} the function $\op{svan}$ on $S$
%rather than truncation $\op{sv}_0$.
%
%In this proof, by enclosed vertex, we mean one of height at most
%$2\sqrt2$. Let $v$ be an enclosed vertex with the fewest anchors.
%If there are no anchors, the right circular cone $C(h,\eta(2,2h,2.51))$
%(aligned along $\{v_0,v\}$; see Definition~\ref{def:cone}) belongs
%to $\op{VC}(v_0)$, where $\eta(2,2h,2.51)=\eta(2h,2,2t_0)$ as in
%Definition~\ref{def:eta0} and $|v-v_0|=2h$. In fact, if such a point
%lies in $\op{VC}(u)$, with $u \ne v$, then $u$ must be a corner of
%the quad cluster or an enclosed vertex of height at least $2t_0$.
%In either case, the right circular cone belongs to $\op{VC} (v_0)$.
%By Formula~\ref{lemma:sovoFR}, the score of this cone is
%$2\pi(1-h/\eta(2,2h,2t_0))\phi(h,\eta(2,2h,2.51))$. An optimization in one
%variable gives an upper bound of $-4.52\,\pt$, for $t_0\le h\le
%\sqr2$.   This gives the bound of $-1.04\,\pt$ in this case.
%
%If there is one anchor,  we cut the cone in half along the plane
%through $\{v_0,v\}$ perpendicular to the plane containing the anchor
%and $\{v_0,v\}$. The half of the cone on the far side of the anchor
%lies under the face at $v$ of the $V$-cell.  We get a bound of
%$-4.52\,\pt/2 < -1.04\,\pt$.
%
%In the remaining cases, each enclosed vertex has at least two
%anchors.  Each anchor is a corner of the quad cluster.  Fix an
%enclosed vertex $v$. Suppose that $v_1$, a corner, is an anchor of
%$v$. Assume that the face $\{v_0,v,v_1\}$ bounds at most one upright
%quarter. We sweep around the edge $\{v_0,v_1\}$, away from the
%upright quarter if there is one,  until we come to another
%enclosed vertex $v'$ such that $\{v_0,v_1,v'\}$ has circumradius
%less than $\sqr2$ or such that $v_1$ is an anchor of $\{v_0,v'\}$.
%If such a vertex $v'$ does not exist, we sweep all the way to
%$v_2$ a corner of the quad cluster adjacent to $v_1$.
%
%Section~\ref{sec:K} defines a function $K$ that we use in
%this proof.
%
%If $v'$ exists, then various
%calculations\footnote{\calc{104677697}, \calc{69785808},
%\calc{586706757}, and \calc{87690094}} give the bound
%$-1.04\,\pt$, depending on the size of the circumradius of
%$\{v_0,v,v'\}$. This allows us to assume that we do not encounter
%such an enclosed vertex $v'$ whenever we sweep away, as above,
%from the face formed by an anchor.
%
%Now consider the simplex $S=\{v_0,v_1,v_2,v\}$, where $v_1$ is an
%anchor of $\{v_0,v\}$.  We assume that it is not an upright quarter.
%There are three alternatives. The first is that $S$ decreases the
%score of the quarter by at least $v_0.52\,\pt$.
%Calculations\footnote{\calc{185703487} and \calc{441195992}} show
%that this occurs if the circumradius of the face $\{v_0,v,v_2\}$ is
%less than $\sqr2$, or if the circumradius of the face is greater
%than $\sqr2$, provided that the length of $\{v,v_1\}$ is at most
%$2.2$. The second alternative\footnote{\calc{848147403},
%\calc{969320489}, and \calc{975496332}.} is that the face
%$\{v_0,v,v_1\}$ of $S$ is shared with a quarter $Q$ and that $S$ and
%$Q$ taken together bring the score down by $0.52\,\pt$. In fact,
%if there are two such simplices $S$ and $S'$ along $Q$, then the
%three simplices $Q$, $S$, and $S'$ pull the
%score\footnote{\calc{766771911}} below $-1.04\,\pt$. The third
%alternative is that there is a simplex $S'=\{v_0,v,v,v_3\}$ sharing
%the face $\{v_0,v,v_1\}$, which, like $S$, scores less than
%$-0.31\,\pt$.  In each case, $S$ and the adjacent simplex through
%$\{v_0,v,v_1\}$ score less than $-0.52\,\pt$. Since $v$ has at least
%two anchors, the quad cluster scores less than $2(-0.52)\,\pt
%=-1.04\,\pt$.
%%
%
%



\subsection{a particular 5-circuit} %DCG 14.3, p160

\begin{lemma}\dcg{Lemma~14.5}{160}\label{lemma:6079}  
Assume that $R$ is a pentagonal standard component
    with an enclosed vertex $v$ of height at most $2t_0$.
    %(See Figure~\ref{fig:pent-tri1}.)
    Assume further that
    \begin{itemize}
        \item $|v_i-v_0|\le 2.168$ for each of the five corners.
        \item Each interior angle of the pentagon is at most
        $2.89$.
        \item If $v_1$, $v_2$, $v_3$ are consecutive corners over
        the pentagonal component, then $$|v_1-v_2|+|v_2-v_3|<4.804.$$
        \item $\sum_5 |v_i-v_{i+1}|\le 11.407.$
    \end{itemize}
    Then $\sigma_R(\Lambda,v)< -0.2345$ or $\tau_R(\Lambda,v) > 0.6079.$
\end{lemma}

\begin{proof}
Since $-0.4339$ is less than this the lower bound, a $\xtazimge$
upright diagonal does not occur. Similarly, since $-0.25$ is less
than the lower bound, a $\xfgap$ upright diagonal does not
occur (Lemma~\ref{lemma:4-crowded} and Lemma~\ref{x-3.8}).

Suppose that there is a loop in context $\y{(n,k)}=\y{(4,2)}$. Again by
Lemma~\ref{lemma:loop} (with $n(R)=7$),
$$\sigma_R(\Lambda,v)  < -0.2345.$$
%The constants come from
%Table~\ref{x-5.11} and  Theorem~\ref{thm:the-main-theorem}.

%If we branch and bound on the triangular faces, this LP-derived
%inequality can be improved to
%    $$\tau[F] < 0.6079.$$

%If there is a loop other than $\y{(4,2)}$ and $\y{(4,1)}$, the linear
%program becomes infeasible:
%    $$\tau[F] < 0.644 < t_7 + \dloop\y{(n,k)} < \tau[F].$$
We conclude that all loops have context $\y{(n,k)}=\y{(4,1)}$.


{\bf Case 1.}  {\it The vertex $v=v_{12}$ has distance at least
$2t_0$ from the five corners of $\Lambda(v,2t_0)$ over the pentagon.}

%The interval calculations relevant in Case 1 appear in
%~\ref{A.3.8}.

The penalty to switch the pentagon to a pure $\op{svR}_0$ score is at
most $5\xiG$ (see Section~\ref{sec:prep-cluster}).  There cannot
be two flat quarters because then Lemma~\ref{tarski:E:part4:5} gives
$$|v_{12}-v_0|>2t_0.$$


{\bf (Case 1-a)} Suppose there is one flat quarter,
$|v_1-v_4|\le2\sqrt2$. There is a lower bound of 1.2 on the
dihedral angles of the simplices $\{v_0,v_{12},v_i,v_{i+1}\}$.  This
is obtained as follows.  The proof relies on the convexity of the
quadrilateral component.  We leave it to the reader to verify that
the following pivots can be made to preserve convexity.  Disregard
all vertices except $v_1,v_2,v_3,v_4,v_{12}$.  We give the
argument that $\dih(v_0,v_{12},v_1,v_4)>1.2$.  The others are
similar. Disregard the length $|v_1-v_4|$.  We show that
    $$
    \begin{array}{lll}
        sd &:=\dih(v_0,v_{12},v_1,v_2)+\dih(v_0,v_{12},v_2,v_3)\\
           &+\dih(v_0,v_{12},v_3,v_4) < 2\pi-1.2.
    \end{array}
    $$
Lift $v_{12}$ so $|v_{12}-v_0|=2t_0$. Maximize $sd$ by taking
$|v_1-v_2|=|v_2-v_3|=|v_3-v_4|=2t_0$.  Fixing $v_3$ and $v_4$,
pivot $v_1$ around $\{v_0,v_{12}\}$ toward $v_4$, dragging $v_2$
toward $v_{12}$ until $|v_2-v_{12}|=2t_0$.  Similarly, we obtain
$|v_3-v_{12}|=2t_0$. We now have $sd\le 3(1.63)< 2\pi-1.2$, by a
calculation.\footnote{\calc{821707685}}

Return to the original figure and move $v_{12}$ without increasing
$|v_{12}-v_0|$ until each simplex $\{v_0,v_{12},v_i,v_{i+1}\}$ has an edge
$(v_{12},v_j)$ of length $2t_0$. Interval
calculations\footnote{\calc{467530297} and \calc{135427691}} show
that the four simplices around $v_{12}$ squander
    $$2\pi(0.2529)-3(0.1376)-0.12 > \squander + 5\xiG.$$

{\bf (Case 1-b)} Assume there are no flat quarters. By hypothesis,
the perimeter satisfies $$\sum|v_i-v_{i+1}|\le 11.407.$$ We have
$\arc(2,2,x)'' = 2x/(16-x^2)^{3/2} >0$. The arclength of the
perimeter is therefore at most
$$2\arc(2,2,2t_0) + 2\arc(2,2,2) + \arc(2,2,2.387) <  2\pi.$$
There is a well-defined interior of the spherical pentagon, a
component of area $<2\pi$.  If we deform by decreasing the
perimeter, the component of area $<2\pi$ does not get swapped with
the other component.

Disregard all vertices but $v_1,\ldots,v_5,v_{12}$.  If a vertex
$v_i$ satisfies  $|v_i-v_{12}|>2t_0$, deform $v_i$ as in
Section~\ref{x-4.9} until $|v_{i-1}-v_{i}|=|v_i-v_{i+1}|=2$, or
$|v_i-v_{12}|=2t_0$. If at any time, four of the edges realize the
bound $|v_i-v_{i+1}|=2$, we have reached an impossible situation,
because it leads to the contradiction\footnote{\calc{115383627}
and \calc{603145528}}
    $$2\pi = \sum^{(5)}\dih < 1.51 + 4 (1.16) < 2\pi.$$
(This inequality relies on the observation, which we leave to the
reader, that in any such assembly, pivots can by applied to bring
$|v_{12}-v_i|=2t_0$ for at least one edge of each of the five
simplices.)



The vertex $v_{12}$ may be moved without increasing $|v_{12}-v_0|$ so
that eventually by these deformations (and reindexing if
necessary) we have $|v_{12}-v_i|=2t_0$, $i=1,3,4$. (If we have
$i=1,2,3$, the two dihedral angles along $\{v_0,v_2\}$
satisfy\footnote{\calc{115383627}} $<2(1.51)<\pi$, so the
deformations can continue.)



There are two cases. In both cases $|v_i-v_{12}|=2t_0$, for
$i=1,3,4$.
$$
\begin{array}{lll}
(i)\quad &|v_{12}-v_2|=|v_{12}-v_5|=2t_0,\\
(ii)\quad &|v_{12}-v_2|=2t_0,\quad |v_4-v_5|=|v_5-v_1|=2,\\
\end{array}
$$
Case (i) follows from interval
calculations\footnote{\calc{312132053}}
$$
\sum\tau_0 \ge 2\pi(0.2529) - 5 (0.1453) > 0.644+7\xiG.
$$
In case (ii), we have again
    $$2\pi(0.2529)-5 (0.1453).$$
In this interval calculation we have assumed that
$|v_{12}-v_5|<3.488$. Otherwise, setting $S=(v_{12},v_4,v_5,v_1)$, Lemma~\ref{tarski:3488}
shows the simplex does not exist.
($|v_4-v_1|\ge2\sqrt2$ because
there are no flat quarters.)
This completes Case 1.

\medskip

{\bf Case 2.} {\it The vertex $v_{12}$ has distance at most $2t_0$
from the vertex $v_1$ and distance at least $2t_0$ from the
others.}

Let $\{v_0,v_{13}\}$ be the upright diagonal of a loop with context $\x{(3,1)}$.  The
vertices of the loop are not $\{v_2,v_3,v_4,v_5\}$ with $v_{12}$
enclosed over $\{v_0,v_2,v_5,v_{13}\}$ by
Lemma~\ref{lemma:anc-simplex-not-enc}. The vertices of the loop
are not $\{v_2,v_3,v_4,v_5\}$ with $v_{12}$ enclosed over
$\{v_0,v_1,v_2,v_5\}$ because this and Lemmas~\ref{tarski:E:part4:6}
and \ref{tarski:E:part4:7} would lead to a contradiction
$y_{12}>2t_0$. 
We get a contradiction for the same reasons
 unless $\{v_1,v_{12}\}$ is an edge of some
upright quarter of every loop of context $\x{(4,1)}$.

We consider two cases.  (2-a) There is a flat quarter along an
edge other than $\{v_1,v_{12}\}$.  That is, the central vertex is
$v_2$, $v_3$, $v_4$, or $v_5$.  (Recall that the {\it central
vertex} of a flat quarter is the vertex other than $v_0$ that
is not an endpoint of the diagonal.) (2-b) Every flat quarter has
central vertex $v_1$.

{\bf Case 2-a.}  We erase all upright quarters including those in
loops, taking penalties as required. There cannot be two flat
quarters because then Lemmas~\ref{tarski:E:part4:8} and
\ref{tarski:E:part4:9} would imply $|v_{12}-v_0|>2t_0$.

The penalty is at most $7\xiG$.  We show that the component (with
upright quarters erased) squanders $>7\xiG+0.644$.  We assume that
the central vertex is $v_2$ (case 2-a-i) or $v_3$ (case 2-a-ii).
In case 2-a-i, we have three types of simplices around $v_{12}$,
characterized by the bounds on their edge lengths.  Let
$\{v_0,v_{12},v_1,v_5\}$ have type A, $\{v_0,v_{12},v_5,v_4\}$ and
$\{v_0,v_{12},v_4,v_3\}$ have type B, and let $\{v_0,v_{12},v_3,v_1\}$
have type C.  In case 2-a-ii there are also three types.  Let
$\{v_0,v_{12},v_1,v_2\}$ and $\{v_0,v_{12},v_1,v_5\}$ have type A,
$\{v_0,v_{12},v_5,v_4\}$ type B, and $\{v_0,v_{12},v_2,v_4\}$ type D.
(There is no relation here between these types and the types of
simplices $A$, $B$, $C$ defined in Chapter~\ref{sec:fine}.) Upper
bounds on the dihedral angles along the edge $\{v_0,v_{12}\}$ are
given as calculations\footnote{\calc{821707685}, \calc{115383627},
\calc{576221766}, and \calc{122081309}}. These upper bounds come
as a result of a pivot argument similar to that establishing the
bound 1.2 in Case 1-a.

These upper bounds imply the following lower bounds.  In case
2-a-i,
$$
\begin{array}{lll}
\dih &> 1.33 \quad(A),\\
\dih &> 1.21 \quad(B),\\
\dih &> 1.63 \quad(C),\\
\end{array}
$$
and in case 2-a-ii,
$$
\begin{array}{lll}
\dih &> 1.37 \quad(A),\\
\dih &> 1.25 \quad(B),\\
\dih &> 1.51 \quad(\Lambda,v),\\
\end{array}
$$
In every case the dihedral angle is at least $1.21$. In case
2-a-i, the inequalities give a lower bound on what is squandered
by the four simplices around $\{v_0,v_{12}\}$. Again, we move $v_{12}$
without decreasing the score until each simplex
$\{v_0,v_{12},v_i,v_{i+1}\}$ has an edge satisfying
$|v_{12}-v_j|\le2t_0$. Interval
calculations\footnote{\calc{644534985}, \calc{467530297}, and
\calc{603910880}} give
    $$
    \begin{array}{lll}
    \sum_{(4)}\tau_0 &> 2\pi (0.2529) - 0.2391-2(0.1376)-0.266\\
        &>0.808.
    \end{array}
    $$
In case 2-a-ii, we have\footnote{\calc{135427691}}
    $$
    \begin{array}{lll}
    \sum_{(4)}\tau_0 &> 2\pi (0.2529) - 2(0.2391)-0.1376-0.12\\
        &>0.853.
    \end{array}
    $$
So we squander more than $7\xiG+0.644$, as claimed.

{\bf Case 2-b.}  We now assume that there are no flat quarters
with central vertex $v_2,\ldots,v_5$. We claim
 that $v_{12}$ is not enclosed over $\{v_0,v_1,v_2,v_3\}$ or
$\{v_0,v_1,v_5,v_4\}$. In fact, if $v_{12}$ is enclosed over
$\{v_0,v_1,v_2,v_3\}$, then we reach the
contradiction\footnote{\calc{821707685} and \calc{115383627}}
    $$
    \begin{array}{lll}
    \pi&<\dih(v_0,v_{12},v_1,v_2)+\dih(v_0,v_{12},v_2,v_3)\\
        &< 1.63+1.51 < \pi.
    \end{array}
    $$

We claim
 that $v_{12}$ is not enclosed over $\{v_0,v_5,v_1,v_2\}$.
Let $S_1=\{v_0,v_{12},v_1,v_2\}$, and $S_2=\{v_0,v_{12},v_1,v_5\}$.  We
have by hypothesis,
$$y_4(S_1)+y_4(S_2) = |v_1-v_2|+|v_1-v_5|< 4.804.$$
An interval calculation\footnote{\calc{69064028}} gives
    $$
    \begin{array}{lll}
    \sum_{(2)}\dih(S_i) &\le \sum_{(2)}
    \left(\dih(S_i)+0.5(0.4804/2-y_4(S_i))\right)\\
    &<\pi.
    \end{array}
    $$
So $v_{12}$ is not enclosed over $\{v_0,v_1,v_2,v_5\}$.

Erase all upright quarters, taking penalties as required.  Replace
all flat quarters with $\op{sv}_0$-scoring taking penalties as
required. (Any flat quarter has $v_1$ as its central vertex.) We
move $v_{12}$ keeping $|v_{12}-v_0|$ fixed and not decreasing
$|v_{12}-v_1|$.  The only effect this has on the score comes
through the quoins along $\{v_0,v_1,v_{12}\}$. Stretching
$|v_{12}-v_1|$ shrinks the quoins and increases the score. (The
sign of the derivative of the quoin with respect to the top edge
is computed in the proof of Lemma~\ref{x-4.9.1}.)

If we stretch $|v_{12}-v_1|$ to length $2t_0$, we are done by case
1 and case 2-a. (If deformations produce a flat quarter, use case
2-a, otherwise use case 1.) By the claims, we can eventually
arrange (reindexing if necessary) so that
$$
\begin{array}{lll}
(i)&\quad |v_{12}-v_3|=|v_{12}-v_4|=2t_0,\quad\text{or}\\
(ii)&\quad |v_{12}-v_3|=|v_{12}-v_5|=2t_0.
\end{array}
$$
We combine this with the deformations of Section~\ref{x-4.9} so
that in case (i) we may also assume that if $|v_5-v_{12}|>2t_0$,
then $|v_4-v_5|=|v_5-v_1|=2$ and that if $|v_2-v_{12}|>2t_0$, then
$|v_1-v_2|=|v_2-v_3|=2$. In case (ii) we may also assume that if
$|v_4-v_{12}|>2t_0$, then $|v_3-v_4|=|v_4-v_5|=2$ and that if
$|v_2-v_{12}|>2t_0$, then $|v_1-v_2|=|v_2-v_3|=2$.

Break the pentagon into substandard components by cutting along the edges
$(v_{12},v_i)$ that satisfy $|v_{12}-v_i|\le2t_0$. So for example
in case (i), we cut along $(v_{12},v_3)$, $(v_{12},v_4)$,
$(v_{12},v_1)$, and possibly along $(v_{12},v_2)$ and
$(v_{12},v_5)$.  This breaks the pentagon into triangular and
quadrilateral components.

In case (ii), if $|v_4-v_{12}|>2t_0$, then the argument used in
Case 1 to show that $|v_4-v_{12}|<3.488$ applies here as well.
%% Keep comment: DCG p164.  I commented out to avoid mention of Delta.
%% But it is used eventually in the argument.
%% Deelta doesn't explicitly get mentioned in the two footnoted calcs,
%% so what I'm commented out isn't that essential.
% In
%case (i) or (ii), if $|v_{12}-v_2|>2t_0$, then for similar
%reasons, we may assume
%    $$\Deelta(|v_{12}-v_2|^2,4,4,8,(2t_0)^2,|v_{12}-v_1|^2)\ge0.$$
We use
calculations\footnote{\calc{312132053} and \calc{644534985}} to
conclude that
    $$\sum\tau_0 \ge 2\pi (0.2529) -3 (0.1453) -2 (0.2391) > 0.6749.$$
If the penalty is less than $0.067=0.6749-0.6079$, we are done.

We have ruled out the existence of all loops except $\y{(4,1)}$. Note
that a flat quarter with central vertex $v_1$ gives penalty at
most $0.02$ by Lemma~\ref{x-3.11.3}.
  If there is at most one
such a flat quarter and at most one loop, we are done:
$$3\xiG + 0.02 < 0.067.$$
Assume there are two loops of context $\y{(n,k)}=\y{(4,1)}$.  They both
lie along the edge $\{v_1,v_{12}\}$, which precludes any unmasked
flat quarters. If one of the upright diagonals has height
$\ge2.696$, then the penalty is at most $3\xiG+3\xiV< 0.067$.
Assume both heights are at most $2.696$. The total interior angle
of the exceptional face at $v_1$ is at least four times the
dihedral angle of one of the flat quarters along $\{v_0,v_1\}$, or
$4(0.74)$ by an interval calculation\footnote{\calc{751442360}}. This is
contrary to the hypothesis of an interior angle $<2.89$.   This
completes Case 2. This shows that heptagons with pentagonal hulls
do not occur.
\end{proof}



\subsection{truncated corner cell}

In this section, we calculate several different bounds
on the function $\op{sovo}$ on truncated corner cells for
various choices of the parameters $t$, $\mu$, and $\lambda$.

The dependence of $\op{sovo}$ on the azimuth angle
is linear with coefficient $(1-\cos\psi)\phi(t,t,\lambda)$.
For fixed, $\psi$, $t$, and $\lambda$, this coefficient
has fixed sign.  Also, if $\op{azim}<\pi$, the azimuth
angle depends monotonically on $|w_1-w_2|$.  We thus get
an bound on $\op{sovo}$ 
when $|w_1-w_2|$ is chosen to be as small as possible.

\medskip

\begin{lemma}\tlabel{lemma:tcc-est} 
Let $TCC=TCC(v_0,v_1,w_1,w_2,t,\mu)$ 
be a truncated corner cell with
parameters $t=1.255$, $\mu=1.6$ and azimuth angle at least $\pi$. 
Assume that $|v_1-w_i|\ge 3.07$ for $i=1,2$.
Let $y=|v_1-v_0|$.  Assume that $2\le y\le 2t$.
Assume that $2\le |v_1-w_i|\le 2t$, for $i=1,2$.
Then $\op{sovo}(v_0,TCC,\lambda_{sq}) > 0.297$.
\end{lemma}

\begin{proof}
Let $z_i = |v_1-w_i|$, for $i=1,2$.
Our estimate is based on Lemma~\ref{lemma:tcc}.  From that lemma,
we have
  $$
  \begin{array}{lll}
  \op{sovo}(v_0,TCC,\lambda) &= T_0 + T_1 + T_2 + T_3 \\
  T_0 &= \op{sovo}(v_0,Q_1,\lambda) +  \op{sovo}(v_0,Q_2,\lambda)\\
  T_1  &=  - s_1\phi(t,t,\lambda) \\
  T_2  &= - s_2\phi(t,t,\lambda) \\
  T_3 &= 
  \op{azim}(v_0,v_1,w_1,w_2) \left((1-\cos\psi)\phi(t,t,\lambda)+
    A(y/2,t,\lambda)\right) \\
  \end{array}
  $$
The condition $z_i\ge3.07$
forces the circumradius of $\{v_0,v_1,w_i\}$ to be greater
than $t$.  This implies that $Q_i=\emptyset$.  The corresponding
term $T_0$ is zero.

The coefficient of $\op{azim}$ in the term $T_3$
is an explicit function of a single variable $y\in[2,2t]$.
It is minimized and takes a positive value
when $y=2t$.
In particular,
the coefficient of $\op{azim}$ is
positive. We obtain a lower bound on the $T_3$ by taking
$\op{azim}(v_0,v_1,w_1,w_2)\ge \pi$ and $y=2t$.
This gives $T_3 > 0.32$.

For the given choices of $t,\lambda$, we have
$0 < \phi(t,t,\lambda) < 0.6671$.  The term $s_i$ is
maximized when $y_3=2t_0$, $y_5=3.07$,
so that $s_i < 0.017$.  (This was checked with interval arithmetic in
Mathematica.) Thus,
    $$\op{sovo}(v_0,TCC,\lambda)  = T_0 + T_1 + T_2 + T_3 >
   0 - 2 (0.0017)(0.6671) +0.32 > 0.297.$$
\end{proof}



\begin{lemma}\label{lemma:CC815}  
Let $CC=CC(v_0,v_1,w_1,w_2,t,\mu)$
be an untruncated corner
cell with parameters $\mu=1.815$, $t=1.255$,
$y=|v_1-v_0|$, with $2\le y\le 2.2$, 
and azimuth angle at least $\pi$.  Then
 $$\op{sovo}(v_0,CC,\lambda_{sq}) > 0.8862.$$ %WW check: was \squander +\maxpi.$$
\end{lemma}

% WW Recheck proof

\begin{proof}  According to Lemma~\ref{lemma:sovo:CC},
the function $\op{sovo}(CC)$ has the form
$\op{azim}(v_0,v_1,w_1,w_2) f(y)$, for
some explicit rational function $f$ of the
variable $y$. 
The minimum, which occurs at $y=2.2$, is positive.
A lower bound is then $\pi f(2.2)$. 
We evaluate this constant to get the result.
\end{proof}


%% TO HERE.

\subsection{slices do not overlap} %DCG 11.6, p.116
    \oldlabel{3.7}


%% Lemma lemma:anchor-no-overlap has been moved to fine.tex.
%% It relies on the following two lemmas.


\begin{lemma}\label{lemma:slice-quarter}
Around a $\xtazimge$ upright diagonal, all of the slices
are quarters.
\end{lemma}

\begin{proof}  The proof makes use of constants and inequalities from
several different calculations.\footnote{\calc{815492935}} %A2
\footnote{\calc{83777706}} %A8
\footnote{\calc{855294746}} %A12
%$\A_2$, $\A_8$, and $\A_{12}$.
The dihedral angles are at most $\pi-
2(0.956) < 1.23$. This forces $y_4\le 2t_0$, for each simplex $S$.
So they are all quarters.
\end{proof}

\begin{lemma}\oldlabel{3.7.1}\label{lemma:3-crowded}
If there is $\xtazimge$ upright diagonal, then the three 
slices squander more than $0.5606$ and score at most $-0.4339$.
\end{lemma}


\begin{proof}  The proof makes use of constants and inequalities from
several different calculations.\footnote{\calc{815492935}} %A2
\footnote{\calc{83777706}} %A8
\footnote{\calc{855294746}} %A12
%$\A_2$, $\A_8$, and $\A_{12}$.
The three slices squander at
least
    $$
    3 (1.01104) - \pi (0.78701) > 0.5606.
    $$
The bound on score follows similarly from $\nu<-0.9871+0.80449\dih$.
\end{proof}

\begin{lemma}
    \oldlabel{3.7.2}
If a simplex at a $\xtazimge$ upright diagonal meets at an
interior point with a slice, the centered packing does
not contravene.
\end{lemma}

\begin{proof}
Suppose that $\{v_0,v,v_1,v_2\}$ is a slice that another
slice overlaps, with $\{v_0,v\}$ the upright diagonal.  Let
$\{v_0,w\}$ be a $\xtazimge$ upright diagonal. We score the two
simplices $S'_i = \{v_0,v,w,v_i\}$ by truncation at $\sqrt{2}$.
Truncation at $\sqrt{2}$ is justified by tarski\tarf{tarski:old372}.
A calculation\footnote{\calc{855294746}} %A12
gives
%$\A_{12}$,
    $$\tau_V(S'_1,\sqrt{2})+\tau_V(S'_2,\sqrt{2})\ge 2(0.13) +
        0.2(\dih(S'_1)+\dih(S'_2)-\pi) > 0.26.
    $$
Together with the three simplices around the $\xtazimge$ upright
diagonal that squander at least $0.5606$, we obtain the stated
bound.
\end{proof}

\subsection{four and five darts} %DCG 11.7, p.118 % Was "Five Anchors"
    \oldlabel{3.8}
    \label{sec:five-anchors}
    %\section{Four darts} %DCG 11.8, p. 120 % Was "Four anchors"
    \oldlabel{3.9}



\begin{remark}\label{rem:5dart}
The situation of five darts at an upright diagonal is
described in Section~\ref{sec:5updart}.
%    \oldlabel{3.8.1}
\end{remark}


\begin{lemma}\dcg{Cor~11.25}{122}
If there are four anchors and if the upright diagonal is enclosed over a
flat quarter, then there are four slices and at least three
quarters around the upright diagonal.
\end{lemma}

\begin{proof}
This follows by tarski\tarf{tarski:dcg-p122}.
\end{proof}


\subsection{penalty constant}

\begin{definition}[$\xiG$,~$\xiV$,~$\xiG'$,~$\xik$,~$\xikG$]\index{Greek}{zzxiG@$\xiG$}\index{Greek}{zzxiV@$\xiV$}
We set $\xiG = 0.01561$, $\xiV = 0.003521$, $\xiG'=0.00935$,
$\xik=-0.029$, $\xikG = \xik+\xiG = -0.01339$.
\end{definition}

The first two constants appear in calculations%
%$\A_{10}$ and $\A_{11}$ as
\footnote{\calc{73974037}} %A10
\footnote{\calc{764978100}} %A11
as penalties for erasing upright quarters that are compressed, and
decompressed, respectively. $\xiG'$ is an improved bound on the
penalty for erasing when the upright diagonal is at least $2.57$.
Also, $\xik$ is an upper bound\footnote{\calc{618205535}} %A9
 on $\kappa$, when the
upright diagonal is at most $2.57$.  If the upright diagonal is at
least $2.57$, then we still obtain the bound%
\footnote{\calc{618205535}} %A9
$\xikG =-0.02274+\xiG'$ on the sum of $\kappa$ with the
penalty from erasing an upright quarter.

Recall that $\xiV=0.003521$, $\xiG=0.01561$, $\xiG'=0.00935$. They are
the penalties that result from erasing a 
decompressed upright quarter, a comprssed upright quarter, 
and a comprssed upright quarter
with diagonal $\ge2.57$. (See calculations.%
\footnote{\calc{73974037}} %A10
\footnote{\calc{764978100}} %A11)


\subsection{upright summary} %DCG 11.9 p 122
    \oldlabel{3.10}
    \label{sec:upright-summary}

Here is a summary of the cases of upright quarters that have
been treated.
% in Section~\ref{sec:upright}. If the number of anchors is
%the number of slices (no gaps), the results appear in
%Section~\ref{x-5.11}. Every other possibility has been treated.
%
%    \begin{itemize}
%    \item 0,1,2 anchors\hfill Sec.~\ref{x-3.3}
%    \item $3$ anchors \hfill Sec.~\ref{x-3.4}
%        \begin{itemize}
%        \item context $\x{(3,0)}$
%        \item context $\x{(2,1)}$
%        \item context $\x{(1,2)}$
%        \item context $\x{(0,3)}$
%        \end{itemize}
%    \item $4$ anchors \hfill Sec.~\ref{x-3.9}
%        \begin{itemize}
%        \item $0$ gaps (Section~\ref{x-5.11})
%        \item $1$ gap
%        \item $2$ or more gaps
%        \end{itemize}
%    \item $5$ anchors \hfill Sec.~\ref{x-3.8}
%        \begin{itemize}
%        \item $0$ gaps (Section~\ref{x-5.11})
%        \item $1$ gap ($\xfgap$)
%        \item $2$ or more gaps
%        \end{itemize}
%    \item $6$ or more anchors \hfill Sec.~\ref{x-3.5}
%    \end{itemize}
%
%
%\smallskip
%By truncation and various comparison lemmas, we have entirely eliminated
%upright diagonals except when there are between three and five anchors.
%We may assume that there is at most one gap around the upright
%diagonal.
%
%\smallskip
%1.  Consider a slice $Q$ around a remaining upright
%diagonal. The score of is $\nu(Q)$ if $Q$ is a quarter, the
%analytic function $\op{svan}(Q)$ if the simplex is of type $\SC$
%(Section~\ref{x-2.5}), and the truncated function $\op{sv}_0(Q)$
%otherwise.
%
%\smallskip
%2.  Consider a flat quarter $Q$ in an exceptional cluster. An
%upper bound on the score is obtained by taking the maximum of all
%of the following functions that satisfy the stated conditions on
%$Q$.  Let $y_4$ denote the length of the diagonal and $y_1$ be the
%length of the opposite edge.
%
%(a)  The function $\mu(Q)$.
%
%(b)  $\op{sv}_0(Q) - 0.0063$, if $y_4\ge 2.6$ and $y_1\ge
%2.2$.\hfill
%    (Lemma~\ref{lemma:0.008})
%
%(c)  $\op{sv}_0(Q) - 0.0114$, if $y_4\ge 2.7$ and $y_1\le 2.2$.
%    \hfill (Lemma~\ref{lemma:0.008})
%
%(d)  $\nu(Q_1)+\nu(Q_2)+\op{sv}_x(S)$, if there is an enclosed
%vertex
%    $v$ over $Q$ of height between $2t_0$ and $2\sqrt{2}$ that
%    partitions the convex hull of $(Q,v)$ into two upright quarters
%    $Q_1$, $Q_2$ and a third simplex $S$. Here $\op{sv}_x=\op{svan}$
%    if $S$ is of type $\SC$, and $\op{sv}_x=\op{sv}_0$ otherwise.
%    \hfill (Lemma~\ref{lemma:unerased})
%
%(e)  $\op{sv}(Q,1.385)$ if the simplex is of type $\SB$
%(Section~\ref{x-2.5}).
%
%(f) $\op{sv}_0(Q)$ if the simplex is an isolated quarter with
%    $\max(y_2,y_3)\ge2.23$, $y_4\ge2.77$,
%    and $\eta_{456}\ge\sqrt2$.
%
%\smallskip
%3.   If $S$ is a simplex is of type $\SA$, its score is
%$\op{svan}(S)$. (Section~\ref{x-2.5}.)
%
%\smallskip
%
%    Formula~\ref{eqn:3.7} is used on these remaining pieces.
%    On top of what is obtained for the standard cluster by summing all
%these terms, there is a penalty $\pi_0=0.008$ each time a
%$\xtazimlt$ upright diagonal is erased.
%
%\smallskip
%5.  The remaining upright diagonals that are not completely
%surrounded by slices are $\xtazimlt$, $\xtazimge$,
%or $\xfgap$ from Section~\ref{x-3.7}, \ref{x-3.8},  and
%\ref{x-3.9}.
%


\subsection{flat quarter} %DCG 11.10, p124
    \oldlabel{3.11}
    \label{sec:some-flat}




In the next lemma, we score a flat quarter by any of the functions
on the given domains
     $$\hat\sigma=
        \begin{cases}
            \Gamma,& \eta_{234},\eta_{456}\le\sqrt2,\\
             \op{svan}, &\eta_{234}\ge\sqrt2,\\
            \op{sv}_0, & y_4\ge 2.6, y_1\ge2.2,\\
            \op{sv}_0, & y_4\ge 2.7,\\
            \op{sv}_0,& \eta_{456}\ge\sqrt2.
        \end{cases}
    $$

\begin{lemma}
    \oldlabel{3.11.1}
    \label{lemma:hatsigma}
$\hat\sigma$ is an upper bound on the functions in
Section~\ref{x-3.10}.2(a)--(f). That is, each function in
Section~\ref{x-3.10}.2 is dominated by some choice of $\hat\sigma$.
\end{lemma}

\begin{proof}  The only case in doubt is the function of 3.10(d):
$$\nu(Q_1)+\nu(Q_2)+\op{sv}_x(S).$$ This is established by the
following lemma.
\end{proof}


We consider the context $\x{(2,1)}$ that occurs when two upright
quarters in the $Q$-system lie over a flat quarter. Let $\{v_0,v\}$ be
the upright diagonal, and assume that $\{v_0,v_1,v_2,v_3\}$ is the
flat quarter, with diagonal $\{v_2,v_3\}$. Let $\sigma$ denote the
score of the upright quarters and other slice lying
over the flat quarter.

\begin{lemma}\label{lemma:min0-svor}
    \oldlabel{3.11.2}
    $\sigma\le \min(0,\op{sv}_0)$.
\end{lemma}

\begin{proof}
The bound of $0$ is established in Theorem~\ref{lemma:quad0}.
The bound of $\op{sv}_0$ is established in Lemma~\ref{a:min0-vor}.
\end{proof}



%\chapter{Further Bounds}%DCG Sec. 14, p. 157
%    \oldlabel{5.12}
%    \label{sec:fb}



\subsection{penalty} %DCG 13.6, p148
\label{sec:4.2} \label{sec:penalty}

Erasing a compressed upright quarter gives a penalty of
at most $\xiG$ and a decompressed one gives at most $\xiV$. We
take the worst possible penalty.  It is at most $n\xiG$ in an
$n$-gon. If there is a masked flat quarter, the penalty is at most
$2\xi_V$ from the two upright quarters along the flat quarter.  We
note in this connection that both edges of a polygon along a flat
quarter lie on upright quarters, or neither does.

If an upright diagonal appears enclosed over a flat quarter, the
flat quarter is part of a loop with context $\x{(3,1)}$, for a
penalty at most $2\xi'_\Gamma+\xi_V$.  This is smaller than the
bound on the penalty obtained from a loop with context
$\x{(3,1)}$, when the upright diagonal is not enclosed over the
flat quarter:
    $$\xi_\Gamma + 2\xi_V.$$
So we calculate the worst-case penalties under the assumption that
the upright diagonals are not enclosed over flat quarters.

A loop of context $\x{(3,1)}$ gives $\xi_\Gamma+2\xi_V$ or
$3\xi_\Gamma$.  A loop of context $\x{(2,2)}$ gives
$2\xi_\Gamma$ or $2\xi_V$.

If we erase an  upright diagonal of context $\xtazimlt$, there is a penalty
of $0.008$ (or 0 if it masks a flat quarter.) This is dominated by
the penalty $3\xi_\Gamma$ of context $\x{(3,1)}$.

Suppose we have an octagonal standard component.  We claim that a loop
does not occur in context $\x{(2,2)}$. If there are at most three
vertices that are not corners of the octagon, then there are at most
twelve quasi-regular tetrahedra, and the score is at most
$$s_8 + 12\,\pt<8\,\pt.$$
Assume there are more than three vertices that are not corners
over the octagon. We squander
$$t_8+ \dloop(\y{(4,2)})+4\tlp(\pqr{(5,0,0)}) > \squander.$$
As a consequence, context $\x{(2,2)}$ does not occur.

So there are at most two upright diagonals and at most six quarters,
and the penalty is at most $6\xi_\Gamma$. Let $f$ be the number of
flat quarters This leads to
    $$
    \piF = \begin{cases} 6\xiG, & f=0,1,\\
                   4\xiG+2\xiV, & f=2,\\
                    2\xiG+4\xiV, & f=3,\\
                    0, & f=4.
            \end{cases}
    $$
The 0 is justified by a parity argument.  Each upright quarter
occurs in a pair at each masked flat quarter.  But there is an odd
number of quarters along the upright diagonal, so no penalty at
all can occur.

Suppose we have a heptagonal standard component.  Three loops are a
geometric impossibility. Assume there are at most two upright
diagonals.
 If there is no context $\x{(2,2)}$,
 then we have the following bounds on the penalty
    $$
    \piF = \begin{cases} 6\xiG, & f=0,\\
                 4\xiG+2\xiV, & f=1,\\
                3\xiG, & f=2,\\
                \xiG+2\xiV, & f=3.
            \end{cases}
    $$
If an upright diagonal has context $\x{(2,2)}$, then
    $$
    \piF = \begin{cases} 5\xiG, & f=0,1,\\
                3\xiG + 2\xiV, & f=2,\\
                \xiG + 4 \xiV, &f = 3.\\
            \end{cases}
    $$
This gives the bounds used in the diagrams of cases.

% WW Estimates moved to Assembly Listing.







\section{Triangle}


\subsection{positivity}%DCG 9.10, p98 %% Delta(v) stuff
    \oldlabel{2.11}
    \label{sec:pos}

The function $\op{sovo}$ depends on a parameter $\lambda$.
A frequently used choice of parameter $\lambda$ is
$$
 %\lambda_{oct}=(\lambda_v,\lambda_s)=(-4\doct,1/3).
 \lambda_{sq} =(\lambda_v,\lambda_s)=(4\doct,\zeta\,\pt-1/3).
$$
\index{Index}{sovo}\index{Greek}{ZZlambda@$\lambda$}
%\index{Greek}{ZZlambdaoct@$\lambda_{oct}$}
\index{Greek}{ZZlambdasq@$\lambda_{sq}$}


\begin{lemma}\label{lemma:rog-squ0}
Let $R=\op{rog}^0(v_0,v_1,v_2,v_3,t_0)$ be a Rogers simplex
with $|v_i-v_j|\ge 2$ for $i,j=0,1,2$, $i\ne j$.
Then $\op{sovo}(v_0,R,\lambda_{sq})\ge 0$.
\end{lemma}

\begin{proof}
The $abc$-parameters are $1 \le |v_i-v_j|/2 = a$,
$2/\sqrt{3}\le b = \eta_V(v_0,v_1,v_2)$ (tarski\tarf{tarski:XX}),
and $c = t_0 > \sqrt6/2$.  By Lemma~\ref{lemma:rog-tet},
$$
0 = \op{sovoR}(1,2/\sqrt{3},\sqrt{6}/2,\lambda_{sq}) 
   \le \op{sovoR}(a,b,c,\lambda_{sq}).
$$
\end{proof}



\begin{lemma}\label{lemma:phi-sq0}
For $1\le h\le t_0$, we have
$$
\phi(h,t_0,\lambda_{sq})\ge 0.
$$
\FIXX{Move following to primitive vol}
\end{lemma}

\begin{proof}
This is a polynomial in $h$, which is easy to evaluate over the
given range.
\end{proof}


\begin{lemma}\label{lemma:pl-sq0}
Let $PL(v_0,v_1,w_1,w_2,t_0)$ be a plate with
vertices $v_i,w_i$ in a packing $\Lambda$ and $|w_1-w_2|\ge 2 t_0$.
Then
$\op{sovo}(v_0,PL(v_0,v_1,w_1,w_2,t_0),\lambda_{sq}) \ge 0$. 
\FIXX{Move to primitive volume.}
\end{lemma}

\begin{proof} In the notation of Definition~\ref{def:plate},
a plate is a composite of Rogers simplices and
a frustum $W''\cap FR(v_0,v_1,h,h/t)$,
where $2h = |v_0-v_1|$.  (The assumption $|w_1-w_2|\ge 2 t_0$
is needed for the Assumption~\ref{eqn:q1q2}.)
It is enough to show non-negativity for each piece in the composite.
The non-negativity of the Rogers simplex follows from Lemma~\ref{lemma:rog-squ0}.
For the frustum, we have by Lemma~\ref{lemma:sovoFR},
$$
\begin{array}{lll}
\op{sovo}(v_0,W''\cap FR(v_0,v_1,h,h/t_0),\lambda_{sq}) &=
  \sol(W''\cap FR(v_0,v_1,h,h/t_0) \phi(t_0,t_0,\lambda_{sq})\\
\end{array}
$$
Hence, the result follows from Lemma~\ref{lemma:phi-sq0}.
\end{proof}


\begin{lemma}
    \label{lemma:tau-positive}
    Let $R$ be a standard component that is not a triangle in a
    centered packing $(\Lambda,v_0)$.
    $\tau_{0}(\Lambda,v_0,R)\ge 0$.
\FIXX{Might want to generalize to more than standard components.}
\end{lemma}

\begin{proof}
The function $\tau_0$ is expressed on a standard component as a sum
of terms
  $$
  \op{sovo}(v_0,PL,\lambda_{sq}),
  $$
for various plates $PL$ and
   $$
   \op{sovo}(v_0,B,\lambda_{sq}),
   $$
for some $t_0$-radial measurable set.\FIXX{Reference this decomposition.}
The plates give a non-negative contribution by Lemma~\ref{XX},
and the $t_0$-radial measurable set gives a contribution
   $$
   \sol(v_0,B)\phi(t_0,t_0,\lambda_{sq}),
   $$
which is positive by Lemma~\ref{lemma:phi-sq0}.
\end{proof}


\begin{lemma}\label{lemma:roger0}
    %proclaim{Lemma 3.1}
    %\oldlabel{part3.3.1}
    $\tau(\Lambda,v,R)\ge 0$, for all standard components $R$.\FIXX{Doesn't this proof need quarter estimates and exceptional component
estimates?}
\end{lemma}


\begin{proof}
If $R$ is not a quasi-regular tetrahedron, then $\sigma(\Lambda,v,R)\le0$
by Theorem~\ref{lemma:quad0} and $\sol(R)> 0$, so that the result
is immediate. If $R$ is a quasi-regular tetrahedron, the result
appears in the archive of inequalities \calc{53415898}.
\end{proof}



\begin{lemma}
        \label{lemma:no-enclosed-tri}
        A triangular standard component does not contain any enclosed
        vertices.
\end{lemma}

\begin{proof}
    This fact is proved in tarski\tarf{tarski:2t0-doesnt-pass-through}.
\end{proof}



\section{Bounds in Quadrilateral Region}%DCG 10.4, p104
    \label{sec:bounds}



\subsection{pure bound}%DCG 8.2, p73



\begin{lemma} \label{lemma:wedge} Consider the wedge of a cone
    $$
    W =W(\alpha,z_0) =
    \{ t\, x : 0\le t \le 1, x\in P(\alpha,z_0)\}\subset\ring{R}^3,
    $$
where $P(\alpha,z_0)$ has the form
    $$
    P = \{(x_1,x_2,x_3) :
    x_3 = z_0,\   x_1^2+x_2^2+x_3^2\le 2,\ 0\le x_2\le \alpha x_1\},
    $$
with $z_0\ge1$.  Let $A$ be the volume of the intersection of the
wedge with $B(0,1)$. Then
    $$A\le\doct\,\op{vol}(W).$$
Equality is attained if and only if $W$ has zero volume.\index{Index}{cone}
\end{lemma}

\begin{proof} This is calculated in \cite[Sec. 4]{part2}.  See the
second frame of Figure~\ref{fig:doct}.
\end{proof}

\begin{lemma} \label{lemma:cone}
Let $C$ be the cone at $v_0$ over a set $P$, where $P$ is
measurable and every point of $P$ has distance at least $1.18$
from $v_0$.  Let $A$ be the volume of the intersection of $C$
with $B(v_0,1)$. Then
    $$A\le\doct\,\op{vol}(C).$$
Equality is attained if and only if $C$ has zero volume.
\end{lemma}

\begin{proof} The ratio $A/\op{vol}(C)$ is at most $1/1.18^3 < \doct$.   See the
first frame of Figure~\ref{fig:doct}.
\end{proof}

\begin{figure}[htb]
  \centering
  \myincludegraphics{\ps/haII42.ps}
  \caption{Some sets of low density.}
  \label{fig:doct}
\end{figure}

\begin{lemma}\label{lemma:pure0}
Let $(R,D)$ be a pure quad cluster.  Then
  $\sigma_R(\Lambda,v)\le 0$.
\end{lemma}

\begin{proof}  The pure quad cluster breaks into the types
of regions of low density described by Lemmas~\ref{lemma:cone},
\ref{lemma:wedge}, and \ref{lemma:rog-doct}.\FIXX{Fill in details.}
\end{proof}




\subsection{quad cluster bound}

\begin{lemma} \label{lemma:quarter0}\dcg{Lemma 8.12}
Let $Q$ be a quarter in the $Q$-system (either flat or upright).
Then $\sigma(Q)\le 0$. 
\end{lemma}

\begin{proof} This is an assembly problem.\footnote{JPOMPNK}
\end{proof}


The following theorem is also one of the main results of this
chapter. It is a key part of the proof of local optimality.


\begin{theorem}\label{lemma:quad0} Let $(R,D)$ be a quad cluster.
Then $\sigma_R(\Lambda,v)\le 0$.
\end{theorem}\index{Index}{cluster!quad}

\begin{proof}
The types of quad clusers have been classified in Lemma~\ref{lemma:quad-classify}.
We prove the bound for each type.
If it consists of two flat quarters, then the result is
Lemma~\ref{lemma:quarter0}.  If it is a quartered octahedron with
four upright quarters, then the result again follows from
Lemma~\ref{lemma:quarter0}.  If it is a mixed quad cluster,
the result follows from Lemma~\ref{lemma:1.04}.  Finally,
if it is a pure quad cluster, then an upper bound on the score
is given by $\sigma_R(D,\sqrt2)$ by Lemma~\ref{lemma:pure0}.  
\end{proof}







\subsection{local optimality}%DCG 8.1, p72
\label{sec:local-opt}

\begin{lemma}  %=claim\label{claim-F}
Contravening centered packings $(\Lambda,v)$ exist such that
$\sigma(\Lambda,v)=8\pt$. If $(\Lambda,v)$ is a contravening centered packing, and
if the hypermap of $(\Lambda,v)$ is isomorphic to $G_{fcc}$ or $G_{hcp}$,
then $\sigma(\Lambda,v) \le 8\,\pt$.
\end{lemma} %\label{lemma:local-optimality} in local_opt.tex

\begin{proof}
In each of these two hypermaps there are $8$ triangles and
$6$ quadrilaterals.  In the corresponding centered packings,
there are  eight quasi-regular tetrahedra and six quad clusters.
In each triangular component $\sigma_R(\Lambda,v)\le 1,\pt$ by Lemma~\ref{lemma:1pt}.
In each quad cluser $\sigma_R(\Lambda,v)\le 0$ by Lemma~\ref{lemma:quad0}.  
Thus, the total is
at most $8\,\pt$.
\end{proof}












%% INTRO SPIV

\section{Quarter} %DCG 11.
    \label{sec:upright}
    \oldlabel{3}


%\section{Erasing Upright Quarters} %DCG 11.1,p.112 (deleted)
    \oldlabel{3.1}

% Fix an exceptional cluster $R$. 

%\section{Truncation} (deleted)
    \oldlabel{3.2}

\FIXX{Clean this up. What if something is masked. Get rid of 
"erasing" It is so messy. Use expunge}




To understand how the interiors of slices meet, we
need a bound satisfied by vertices enclosed over a slice.


\begin{lemma}
    \label{lemma:anc-simplex-not-enc}
A vertex $w$ of height between 2 and $2\sqrt{2}$, enclosed in the cone
over a slice $\{v_0,v,v_1,v_2\}$ with diagonal $\{v_0,v\}$ satisfies
$|w-v|\le 2t_0$. In particular, if $|w-v_0|\le 2t_0$, then $w$ is an anchor.
\end{lemma}

\begin{proof}
This appears as tarski\tarf{tarski:anc-simplex-not-enc}.
\end{proof}


\begin{corollary}
A vertex of height at most $2t_0$ is never enclosed over a slice.
\end{corollary}

\begin{proof}  If so, it would be an anchor to the upright diagonal, contrary to
the assumption that the slice is formed by consecutive
anchors.
\end{proof}


\section{Main Estimate}

\subsection{constant} %DCG 13.7, p149
    \oldlabel{5.6}

Theorem~\ref{thm:the-main-theorem} now results from the calculation of a
host of constants. Perhaps there are simpler ways to do it, but it was a
routine matter to run through the long list of constants by computer.
What must be checked is that the Inequalities~\ref{eqn:tau>D(n,k)}
and~\ref{eqn:sigma<Z(n,k)} of Section~\ref{x-5.5} hold for all possible
convex substandard components. Call these inequalities the {\it D} and {\it Z}
inequalities.  This section describes in detail the constants to check.

We begin with a substandard component given as a convex $n$-gon, with at least
four sides.   The heights of the corners and the lengths of edges
between adjacent edges have been reduced by deformation to a finite
number of possibilities (lengths $2$, $2t_0$, or lengths $2$,
$2t_0$, $2\sqrt{2}$, respectively). By Lemma~\ref{lemma:7-sides}, we
may take $n=4,5,6,7$. Not all possible assignments of lengths
correspond to a geometrically viable configuration. One constraint
that eliminates many possibilities, especially heptagons, is that of
Section~\ref{x-5.1}: the perimeter of the convex polygon is at most
a great circle.  Eliminate all length-combinations that do not
satisfy this condition.  When there is a special simplex it can be
broken
from the substandard component and scored\footnote{\calc{148776243}} %A13
separately unless the two heights along the diagonal are $2$.
We assume in all that follows that all specials that can be
broken off have been. There is a second condition related to special
simplices.  By Lemma~\ref{tarski:311}, if
$(y_1,y_2,y_3,y_5,y_6)= (2t_0,2,2,2,2)$, then
the simplex must be special
($y_4\in[2\sqrt{2},3.2]$).


The easiest cases to check are those with no special simplices over the
polygon.  In other words, these are substandard components for which the distances
between nonadjacent corners are at least $3.2$.  In this case we
approximate the score (and what is squandered) by tccs at the corners.
We use monotonicity to bring the fourth edge to length $3.2$. We
calculate the tcc constant bounding the score, checking that it is less
than the constant
    $ Z(k_0+k_1+k_2,k_1+k_2) - \pi_\sigma$,
from the Z inequalities. The D inequalities  are verified in the same
way.

When $n=5,6,7,$ and there is one special simplex, the situation is not
much more difficult.  By our deformations,  we decrease the lengths of
edges $2,3,5,6$ of the special to 2. We remove the special by cutting
along its fourth edge $e$ (the diagonal).  We score the special with
weak bounds.\footnote{\calc{148776243}} %A13  found in $\A_{13}$.
Along the edge $e$, we then apply deformations to the $(n-1)$-gon
that remains. If this deformation brings $e$ to length
$2\sqrt{2}$, then the $(n-1)$-gon may be scored with tccs as in
the previous paragraph.  But there are other possibilities. Before
$e$ drops to $2\sqrt{2}$, a new distinguished edge of length $3.2$
may form between two corners (one of the corners will be a chosen
endpoint of $e$).  The substandard component breaks in two. By deformations,
we eventually arrive at $e=2\sqrt2$ and a substandard component with diagonals
of length at least $3.2$.  (There is one case that may fail to be
deformable to $e=2\sqrt2$, a pentagonal cases discussed further in
Section~\ref{x-5.9}.) The process terminates because the number of
sides to the polygon drops at every step. A simple recursive
computer procedure runs through all possible ways the substandard component
might break into pieces and checks that the tcc-bound gives the
$D$ and $Z$ inequalities. The same argument works if there is a
special simplex that meets at an interior point with each of the
other special simplices in the subcluster.

When $n=6,7$ and there are two nonoverlapping special simplices, a
similar argument can be applied. Remove both specials by cutting along
the diagonals. Then deform both diagonals to length $2\sqrt{2}$, taking
into account the possible ways that the substandard component can break into pieces
in the process.  In every case the $D$ and $Z$ inequalities are
satisfied.

There are a number of situations that arise that escape this generic
argument and were analyzed individually. These include the cases
involving more than two special simplices over a given substandard component, two
special simplices over a pentagon, or a special simplex over a
quadrilateral.  Also, the deformation lemmas are insufficient to bring
all of the edges between adjacent corners to one of the three standard
lengths $2,2t_0,2\sqrt{2}$ for certain triangular and quadrilateral
components.  These are treated individually.

The next few sections describe the cases treated individually. The cases
not mentioned in the sections that follow fall within the generic
procedure just described.

\subsection{triangle} %DCG 13.8, p151
    \oldlabel{5.7}

With triangular substandard components, there is no need to use any of the
deformation arguments because the dimension is already sufficiently
small to apply interval arithmetic directly to obtain our bounds.
There is no need for the tcc-bound approximations.

Flat quarters and simplices of type $\SA$ are treated by a computer
calculation.\footnote{\calc{163548682}} %A16
Other simplices are scored by the truncated function
$\op{sovo}_0(\cdot,\lambda_{oct})$. We break the edges between corners into the cases
    %((
    $[2,2t_0)$, $[2t_0,2\sqrt{2})$, $[2\sqrt{2},3.2]$.
    %]]
Let $k_0$, $k_1$, and $k_2$, with $k_0+k_1+k_2=3$, be the number
of edges  in the respective intervals.

If $k_2=0$, we can improve the penalties,
    $$\pi_\tau = \pi_\sigma=0.$$
To see this, first we observe that there can be no $\xtazimge$ or
$\xfgap$ upright diagonals. By placing $\ge3$ quarters around
an upright diagonal, if the substandard component is triangular, the upright
diagonal becomes surrounded by slices, a case deferred
until Section~\ref{x-5.11}.

If $k_0=k_1=k_2=1$, we can take $\pi'_\tau=
\xiG+2\xiV+0.0114=0.034052$. A few cases are needed to justify
this constant. If there are no $\xtazimge$ upright diagonals,
$\pi'_\tau$ is at most
    $$
    \begin{array}{lll}
    &[\xiG + 2 \xiV +\xikG ]3/4 < 0.0254,\\
    \hbox{or\quad }&[\xiG+2\xiV+\xikG]2/4 + 0.008/3 < 0.0254
    \end{array}
    $$
If there are at most two edges in the substandard component coming from an
$\xtazimge$ upright diagonal,
    $$(\xiG+2\xiV+0.0114)2/3 + 0.008/3 < 0.0254.$$
If three edges come from the simplices of a $\xtazimge$ upright
diagonal, we get $0.034052$. To get somewhat sharper bounds, we
consider how the edge $k_2$ was formed.  If it is obtained by
deformation from an edge in the standard component of length
$\ge3.2$, then it becomes a distinguished edge when the length
drops to $3.2$.  If the edge in the standard component already has
length $\le3.2$, then it is distinguished before the deformation
process begins, so that the substandard component can be treated in isolation
from the other substandard components. We conclude that when
$\pi'_\tau=0.034052$ we can take $y_4\ge2.6$ or $y_5=3.2$
(Lemma~\ref{tarski:last:E}).

The $D$ and $Z$ inequalities now follow.%
\footnote{\calc{852270725}} %A17
\footnote{\calc{819209129}} %A18
% from $\A_{17}$ and $\A_{18}$.

\subsection{quadrilateral} %DCG 13.9, p152
    \oldlabel{5.8}

We introduce some notation for the heights and edge lengths of a convex
polygon.  The heights will generally be $2$ or $2t_0$, the edge lengths
between consecutive corners will generally be $2$, $2t_0$, or
$2\sqrt{2}$.  We represent the edge lengths by a vector
    $$(a_1,b_1,a_2,b_2,\ldots,a_n,b_n),$$
if the corners of an $n$-gon, ordered cyclically have heights $a_i$ and
if the edge length between corner $i$ and $i+1$ is $b_i$.  We say two
vectors are equivalent if they are related by a different cyclic
ordering on the corners of the polygon, that is, by the action of the
dihedral group.

The vector of a polygon with a special simplex is equivalent to one of
the form
    $$(2,2,a_2,2,2,\ldots).$$  If $a_2=2t_0$, then what we have is
necessarily special (Section~\ref{x-5.6}). However, if $a_2=2$, it is
possible for the edge opposite $a_2$ to have length greater than $3.2$.


Turning to quadrilateral components, we use tcc scoring if both diagonals
are greater than $3.2$.   Suppose that both diagonals are between
$[2\sqrt{2},3.2]$, creating a pair of overlapping special simplices. The
deformation lemma requires a diagonal longer than $3.2$, so although we
can bring the quadrilateral to the form
    $$(a_1,2,2,2,2,2,a_4,b_4),$$
the edges $a_1,a_4,b_4$ and the diagonal vary%
\footnote{\calc{148776243}} %A13  (see $\A_{13}$).
continuously.
We have bounds\footnote{\calc{128523606}} %A19
 on the score
    $$
    \begin{array}{lll}
    \tau_0 &> 0.235, \quad \op{svR}_0 < -0.075,
                \hbox{ if } b_4\in[2t_0,2\sqrt{2}],\\
    \tau_0 &> 0.3109, \quad \op{svR}_0 < -0.137,
                \hbox{ if } b_4\in[2\sqrt{2},3.2],\\
    \end{array}
    $$
We have $D(\y{(4,1)})=0.2052$, $Z(\y{(4,1)})=-0.05705$. When
$b_4\in[2t_0,2\sqrt{2}]$, we can take $\pi_\tau=\pi_\sigma=0$. (We are
excluding loops here.) When $b_4\in[2\sqrt{2},3.2]$, we can take
    $$
    \begin{array}{lll}
    \pi_\tau &= \maxpi+ 0.0066, \\
    \pi_\sigma &= 0.008 (5/3)+ 0.009. \\
    \end{array}
    $$
It follows that the $D$ and $Z$ Inequalities are satisfied.

Suppose that one diagonal has length $[2\sqrt{2},3.2]$ and the other has
length at least $3.2$.  The quadrilateral is represented by the vector
    $$(2,2,a_2,2,2,b_3,a_4,b_4).$$
The hypotheses of the deformation lemma hold, so that $a_i\in\{2,2t_0\}$
and $b_j\in\{2,2t_0,2\sqrt2\}$. To avoid quad clusters, we assume
$b_4\ge\max(b_3,2t_0)$. These are one-dimensional with a diagonal of
length $[2\sqrt{2},3.2]$ as parameter.
 The required verifications\footnote{\calc{874876755}} %A20
have been made by interval arithmetic.
%appear in $\A_{20}$.


\subsection{pentagon} %DCG 13.10, p153
    \oldlabel{5.9}

Some extra comments are needed when there is a special simplex. The
general argument outlined above removes the special, leaving a
quadrilateral.  The quadrilateral is deformed, bringing the edge that
was the diagonal of the special to $2\sqrt{2}$. This section discusses
how this argument might break down.

Suppose first that there is a special and that both diagonals on the
resulting quadrilateral are at least $3.2$.  We can deform using either
diagonal, keeping both diagonals at least $3.2$. The argument breaks
down if both diagonals drop to $3.2$ before the edge of the special
reaches $2\sqrt{2}$ and both diagonals of the quadrilateral lie on
specials. When this happens, the quadrilateral has the form
    $$(2,2,2,2,2,2,2,b_4),$$
where $b_4$ is the edge originally on the special simplex.  If both
diagonals are $3.2$, this is rigid, with $b_4= 3.12$. We find its score
to be
    $$
    \begin{array}{lll}
    &\op{sovo}_0(S(2,2,2,b_4,3.2,2),\lambda_{oct})+\op{sovo}_0(S(2,2,2,3.2,2,2),\lambda_{oct})+0.0461<-0.205,\\
    &\op{sovo}_0(S(2,2,2,b_4,3.2,2),\lambda_{sq})+\op{sovo}_0(S(2,2,2,3.2,2,2),\lambda_{sq})2> 0.4645,\\
    \end{array}
    $$
where $S(y_1,\ldots,y_6)$ is an arbitrary simplex with edges $y_i$,
indexed by the usual edge conventions.
So the $D$ and $Z$ Inequalities hold easily.

If there is a special and there is a diagonal on resulting quadrilateral
$\le3.2$, we have two nonoverlapping specials.  It has the form
    $$(2,2,a_2,2,2,2,a_4,2,2,b_5).$$
The edges $a_2$ and $a_4$ lie on the special.  If $b_5>2$, cut
away one of the special simplices.  What is left can be reduced to
a triangle, or a quadrilateral case and then treated%
\footnote{\calc{874876755}} %A20
by computer. % in $\A_{20}$.
Assume
$b_5=2$.  We have a pentagonal standard component. We may assume that
there is no $\xtazimge$ or $\xfgap$ upright diagonal, for
otherwise Theorem~\ref{thm:the-main-theorem} follows trivially
from the bounds in Section~\ref{x-2}. A pentagon can then have at
most an upright diagonal of context $\xtazimlt$ for a penalty of $0.008$.

If $a_2=2t_0$ or $a_4=2t_0$, we again remove a special simplex and
produce triangles, quadrilaterals, or the special cases treated
by computer.%
\footnote{\calc{874876755}} %A20
We may impose the condition $a_2=a_4=b_5=2$. We score this full pentagonal
arrangement by computer,\footnote{\calc{692155251}} %A21 $\A_{21}$,
using the edge lengths of the two diagonals of
the specials as variables. The inequalities follow.

\subsection{hexagon and heptagon} %DCG 13.11, p154
    \oldlabel{5.10}

We turn to hexagons. There may be three specials whose diagonals do not
cross.  Such a subcluster is represented by the vector
    $$(2,2,a_2,2,2,2,a_4,2,2,2,a_6,2).$$
The heights $a_{2i}$ are $2$ or $2t_0$.  Draw the diagonals between
corners $1$, $3$, and $5$.  This is a three-dimensional configuration,
determined by the lengths of the three diagonals, which is treated
by computer.\footnote{\calc{692155251}} %A21
%The required bound follows from $\A_{21}$.

There is one case with a special simplex that
did not satisfy the generic computer-checked inequalities for
what is to be squandered.  Its vector is
    $$(a_1,2,2,2,2,2,2,b_4,2,2,2,2),$$
with $a_1=b_4=2t_0$. A vertex of the special simplex has height
$a_1=2t_0$ and all other corners have height $2$.  The substandard component
is a hexagon with one edge longer than $2$.  We have $D(\y{(6,1)})=
0.48414$. This is certainly obtained if the substandard component contains a
$\xtazimge$ upright diagonal, squandering $0.5606$. But if this
configuration does not appear, we can decrease $\pi_\tau$ to
    $0.03344 + (2/3) 0.008$,
a constant coming from $\xfgap$ upright diagonals in
Section~\ref{x-4.7}. With this smaller penalty the inequality is
satisfied.

Now turn to heptagons.The bound $2\pi$  on the perimeter of the polygon,
eliminates all but one equivalence class of vectors associated with a
polygon that has two or more potentially specials simplices. The vector
is
    $$(2,2,a_2,2,2,2,a_4,2,2,2,a_6,2,a_7,2),$$
$a_2=a_4=a_6=a_7=2t_0$. In other words, the edges between adjacent
corners are $2$ and four heights are $2t_0$. There are two specials.
This case is treated by the procedure outlined for substandard components with two
specials whose diagonals do not cross.

\subsection{loop} %DCG 13.12, p154
    \oldlabel{5.11}
    \label{sec:loops}

We now return to a collection of slices that surround
the upright diagonal.  This is the last case needed to complete
the proof of Theorem~\ref{x-4.3}. There are four or five 
slices around the upright diagonal.
There are linear inequalities%
\footnote{\calc{815492935}} %A2
\footnote{\calc{729988292}} %A3
\footnote{\calc{531888597}} %A4
\footnote{\calc{628964355}} %A5
\footnote{\calc{934150983}} %A6
\footnote{\calc{187932932}} %A7
%$\A_2$--$\A_7$ give a list
%of linear inequalities
satisfied by the slices, broken
up according to type: upright, type $\SC$, opposite edge $>3.2$,
etc. The slices are related by the constraint that the
sum of the dihedral angles around the upright diagonal is $2\pi$.
We run a linear program in each case based on these linear
inequalities, subject to this constraint to obtain bounds on the
score and what is squandered by the slices.

When the edge opposite the diagonal of a slice has length
$\in[2\sqrt{2},3.2]$ and the simplex adjacent to the slice
across that edge is a special simplex, we use inequalities%
\footnote{\calc{485049042}} %A22
\footnote{\calc{209361863}} %A23
% $\A_{22}$ and $\A_{23}$
that run parallel to the similar system\footnote{\calc{531888597}} %A4
\footnote{\calc{628964355}} %A5
%$\A_4$ and $\A_5$.
It is not
necessary to run separate linear programs for these.  It is enough to
observe that the constants for what is squandered improve on those from
the similar system\footnote{\calc{531888597}} %A4
%$\A_4$ by at least $0.06445$
and that the constants for the score in one system%
\footnote{\calc{485049042}} %A22
differ with those of the other%
\footnote{\calc{531888597}} %A4
by no more than $0.009$.

When the dihedral angle of a slice is greater than $2.46$,
the simplex is dropped, and the remaining slices are subject
to the constraint that their dihedral angles sum to at most $2\pi-2.46$.
There can not be a slice with dihedral angle greater than
$2.46$ when there are five darts: $2.46+4 (0.956)>2\pi$. There cannot%
\footnote{\calc{83777706}} %A8
be two slices with dihedral angle greater than $2.46$:
$2(2.46+0.956)>2\pi$.

The following table summarizes the linear programming results.

$$
\begin{matrix}
\y{(n,k)}   &   \DLP(\y{(n,k)}) & D(\y{(n,k)})      &\ZLP(\y{(n,k)})  &Z(\y{(n,k)})\\
\y{(4,0)}   &   0.1362  &   0.1317  &   0   &   0\\
\y{(4,1)}   &   0.208   &   0.20528 &-0.0536&   -0.05709\\
\y{(4,2)}   &   0.3992  &   0.27886 &-0.2   &   -0.11418\\
\y{(4,3)}   &  0.6467   &   0.35244 &-0.424 &   -0.17127\\
\y{(5,0)}   &   0.3665  &   0.27113 &-0.157 &   -0.05704\\
\y{(5,1)}   &  0.5941   &   0.34471 &-0.376 &   -0.11413\\
(5,\ge2)&  0.9706   &  \squander    &*          &   *
\end{matrix}
$$

The bound for $D(\y{(4,0)})$ comes from Lemma~\ref{lemma:0.1317}. A few
more comments are needed for $Z(\y{(4,1)})$.  Let $S$
be the slice that is not a quarter, with edges $(y_1,\ldots,y_6)$,
indexed by the usual edge conventions.  If $y_4\ge2\sqrt2$
or $\dih(S)\ge 2.2$, the linear programming bound is $<Z(\y{(4,1)})$.
With this, if $y_1\le 2.75$, we have\footnote{\calc{855294746}} %A12
    $\sigma(S) < Z(\y{(4,1)})$.
But if $y_1\ge2.75$, the three upright quarters along the upright
diagonal satisfy
    $$\nu< -0.3429+0.24573\dih.$$
With this stronger inequality, the linear programming bound becomes
$<Z(\y{(4,1)})$. This completes the proof of
Theorem~\ref{thm:the-main-theorem}.

\begin{lemma}\label{lemma:loop}
Consider an upright diagonal that is a loop.  Let $R$ be the
standard component that contains the upright diagonal and its
surrounding simplices.   Then the following contexts $\y{(m,k)}$ are the
only ones possible.  Moreover, the constants that appear in the
columns marked $\sigma$ and $\tau$ are upper and lower bounds
respectively for $\tau_R(\Lambda,v)$ when $R$ contains one loop of that
context.
    $$
    \begin{array}{llll}
        n=n(R)&\y{(m,k)} &\sigma &\tau \\
        &&&\\
        4& & &\\
        &\y{(4,0)} &-0.0536 & 0.1362 \\
        5 & & &\\
        &\y{(4,1)} &s_5 &0.27385\\
        &\y{(5,0)} &-0.157   &0.3665\\
        6 & & &\\
        &\y{(4,1)} &s_6 &0.41328\\
        &\y{(4,2)} &-0.1999  &0.5309\\
        &\y{(5,1)} &-0.37595 &0.65995\\
        7 & & &\\
        &\y{(4,1)} &s_7 &0.55271\\
        &\y{(4,2)} &-0.25694 &0.67033\\
        8 & & &\\
        &\y{(4,1)} &s_8 &0.60722\\
        &\y{(4,2)} &-0.31398 &0.72484
    \end{array}
    $$
\end{lemma}

\begin{proof} In context $\y{(m,k)}$, and if $n=n(R)$, we have
    $$
    \sigma_R(\Lambda,v) < s_{n} + \ZLP(\y{(m,k)})-Z(\y{(m,k)})\quad
    \tau_R(\Lambda,v)> t_{n}+\DLP(\y{(m,k)})-D(\y{(m,k)}).
    $$
The result follows.
\end{proof}

In the context $\y{(n,k)}=\y{(4,3)}$, the standard component $R$ must have at
least seven sides $n(R)\ge7$.   Then
    $$
    \begin{array}{lll}
    \tau(\Lambda,v)&\ge t_7+\dloop(\y{(4,3)})\\
            &>\squander.
    \end{array}
    $$
Thus, we may assume that this context does not occur.

If the context $\y{(5,1)}$ appears in an octagon, we have
    $$\tau(\Lambda,v) >\dloop(\y{(5,1)})+t_8>\squander.$$
If this appears in a heptagon, we have
$$\tau(\Lambda,v) >\dloop(\y{(5,1)}) + t_7+ 0.55\,\pt > \squander,$$
because there must be a vertex that is not a corner of the
heptagon. It cannot appear on a pentagon.








\section{Pentahedral Prism}

Recall that $\epsilon_0 = 10^{-12}$.\index{Greek}{ZZepsilon@$\epsilon_0$}

\begin{definition}[type,~$(p,q,r)$]\tlabel{def:face-type}
We say that a set $A$ of faces of a hypermap has face type $(p,q,r)$ if
the hypermap has a node of type $(p,q,r)$ such that $A$ is precisely the
set of faces that meets the node.
\end{definition}


\begin{definition}[pentahedral~prism]\tlabel{def:pentprism}
Let $(\Lambda,v_0)$ be a centered packing with aggregate fan $(v_0,V,E)$,
and standard hypermap $\op{hyper}(v_0,V,E)$.
We say that $(\Lambda,v_0)$ is a pentahedral prism if the following conditions hold 
  \begin{itemize}
   \item The hypermap has exactly $15$ faces, 
   \item  $10$ of the faces are triangles and $5$ are quadrilaterals.
   \item The ten triangles appear in two disjoint sets of face type $(5,0,0)$.
  \end{itemize}
\end{definition}

\begin{lemma}\tlabel{lemma:pentprism-aggregate}
Let $(\Lambda,v_0)$ be a pentahedral prism.  Suppose that its aggregate
fan is not equal to its standard fan.  Then 
    $\sigma(\Lambda,v_0) < 8\,\pt - \epsilon_0$.
\end{lemma}

\begin{proof} There is only one aggregate face type that is a triangle or quadrilateral.  It is the aggregate of Section~\ref{sec:4circ}.   Suppose that $q_1\ge 1$ of the quadrilaterals are aggregates and $5-q_1$ are not. 
By Lemma~\ref{lemma:11.16}, we have
   $$
   \tau(\Lambda,v_0) > q_1 (11.16)\,\pt + (5-q_1)\,(4.14)\,\pt
    > (\trgt).
   $$
\end{proof}

\begin{lemma}\tlabel{lemma:pentprism-lowtet}
Let $(\Lambda,v_0)$ be a pentahedral prism.  Suppose that for some triangular face $F$ of the standard hypermap, we have
  $$
  \sigma(\Lambda,v_0,F) < -0.52\,\pt.
  $$
Then $(\Lambda,v_0)$ does not contravene.
\end{lemma}

\begin{proof}
Break the set of faces of the hypermap into four sets:
the set $\CalF_1$ of quadrilaterals, a set $\CalF_2$ of face type $(5,0,0)$,
the singleton $\CalF_3=\{F\}$, and the remaining set $\CalF_4$ of four
triangles.
  $$
  \begin{array}{llll}
  \sigma(\Lambda,v_0) &= \sum_{i=1}^4\sum_{\CalF_i}\sigma(\Lambda,v_0,F)\\
    &< 0 + ((4.52)\,\pt - \epsilon_0) - (0.52)\,\pt + (4)\,\pt
    &= 8\,\pt - \epsilon_0.
  \end{array}
  $$
\end{proof}

\begin{lemma}\tlabel{lemma:pentprism-lowquad}
Let $(\Lambda,v_0)$ be a pentahedral prism.  Suppose that for some 
quadrilateral face $F$ of the standard hypermap, we have
  $$
  \sigma(\Lambda,v_0,F) < -1.04\,\pt.
  $$
Then $(\Lambda,v_0)$ does not contravene.
\end{lemma}

\begin{proof}
Break the set of faces of the hypermap into four sets:
the set $\CalF_1=\{F\}$ a singleton, two sets $\CalF_2,\CalF_3$ of
face type $(5,0,0)$, and the four remaining quadrilaterals $\CalF_4$.
  $$
  \begin{array}{llll}
  \sigma(\Lambda,v_0) &= \sum_{i=1}^4\sum_{\CalF_i}\sigma(\Lambda,v_0,F)\\
    &< -1.04\,\pt + 2((4.52)\,\pt - \epsilon_0) + 0
    &< 8\,\pt - \epsilon_0.
  \end{array}
  $$
\end{proof}

We remark that if $F$ is a quadrilateral face of the standard hypermap
that tags a mixed quad cluster, then the hypothesis of the lemma
is satisfied (Lemma~\ref{lemma:1.04}).  This means that in a contravening
centered packing, each quadrilateral tags a pure, flat, or octahedral
quad cluster.


\begin{lemma}\tlabel{lemma:pentprism-lowcap}
Let $(\Lambda,v_0)$ be a pentahedral prism.  Suppose that for some 
set $\CalF_1$ of face type $(5,0,0)$,  we have
  $$
  \sum_{F\in\CalF_1}\sigma(\Lambda,v_0,F) < 3.48\,\pt.
  $$
Then $(\Lambda,v_0)$ does not contravene.
\end{lemma}

\begin{proof}
Break the set of faces of the hypermap into three sets:
the set $\CalF_1,\CalF_2$ of face type $(5,0,0)$, and $\CalF_3$ the set
of quadrilateral faces.
  $$
  \begin{array}{llll}
  \sigma(\Lambda,v_0) &= \sum_{i=1}^3\sum_{\CalF_i}\sigma(\Lambda,v_0,F)\\
    &< 3.48\,\pt + (4.52\,\pt - \epsilon_0) + 0.
    &= 8\,\pt - \epsilon_0.
  \end{array}
  $$
\end{proof}

\begin{definition}[proper~pentahedral~prism]\tlabel{def:proper-pp}
Let $(\Lambda,v_0)$ be a pentahedral prism.
It is proper if 
\begin{enumerate}
  \item  Its standard fan is the same as its aggregate fan.
  \item $\sigma(\Lambda,v_0,F) \ge -0.52\,\pt$ for each triangular hypermap face $F$.
  \item $\sigma(\Lambda,v_0,F) \ge -1.04\,\pt$ for each quadrilateral hypermap face $F$.
  \item $\sum_{F\in\CalF} \sigma(\Lambda,v_0,F) \ge 3.48\,\pt$ for each set $\CalF$ of face type $(5,0,0)$.
\end{enumerate}
\end{definition}

The preceding lemmas show that every contravening pentahedral prism is proper.

\begin{lemma}\tlabel{lemma:proper-pp-dih}\rating{20}
Let $(\Lambda,v_0)$ be a proper pentahedral prism.  
Let $x$ be a dart in the standard hypermap at a node of type $(5,0,0)$.
Then $$\op{azim}(x) < 1.4674.$$
\end{lemma}

\begin{proof}
The follows directly from the definition of proper pentahedral prism
and two calculations\footnote{\calc{XX} \calc{XX}.  DCG-V-17.4.1.1. DCG-V-17.4.1.2}
\end{proof}

\begin{lemma}\tlabel{lemma:pp-flat}\rating{50}
Let $(\Lambda,v_0)$ be a proper pentahedral prism.
Let $(v_0,V,E) = (v_0,V_{std}(\CalQ),E_{std}(\CalQ)$ be the $Q$-system
extended fan.  Let $F$ be a face of the
hypermap that
tags a
flat quarter.  Then
   $$\sigma(\Lambda,v_0,F) < -0.3621 \sol(F) + 0.49246/2.$$
%% m = 0.3621; b = 0.49246
\end{lemma}

\begin{proof} If $F$ tags a flat quarter $Q$, then
$$
\begin{array}{lll}
   \sigma(\Lambda,v_0,F) &= \mu(v_0,Q)  \\
   \sol(F) &= \sol(v_0,Q).
\end{array}
$$
The result now follows from an assembly of interval calculations%
\footnote{\calc{XX}. DCG-V 17.4.2.1 -- 17.4.2.6}
\end{proof}

\begin{lemma}\tlabel{lemma:pp-ht}\rating{50}
Let $(\Lambda,v_0)$ be a proper pentahedral prism.  
Let $(v_0,V,E) = (v_0,V_{std}(\CalQ),E_{std}(\CalQ)$ be the $Q$-system
extended fan.
Let $\CalF$ be the set
of four faces of its hypermap that correspond to a quartered octahedron
$(v_0,w,v_1,v_2,v_3,v_4)$.  For every $i=1,2,3,4$ and
every $u\in\{v_0,w\}$, we have $|u-v_i|\le 2.2$.  
\end{lemma}

\begin{proof}
Suppose that $|u-v_i|>2.2$ for some $i=1,2,3,4$ and $u\in\{v_0,w\}$.
The edge $\{u,v_i\}$ is shared by two of the upright quarters
$Q_i = \{v_0,w,v_i,v_{i+1}\}$.  Let $F_1,\ldots,F_4$ be the faces
of the hypermap of the quad-structured fan that tag these quarters.
For each of these two faces $F$, a
calculation\footnote{\calc{XX} DCG-V-17.4.3.1} gives 
  $$\sigma(\Lambda,v_0,F) < -0.52\,\pt.$$
The remaining two faces give $$\sigma(\Lambda,v_0,F)\le 0.$$
Then $$\sum_{i=1}^4\sigma(\Lambda,v_0,F_i) < -2(0.52)\,\pt = -1.04\,\pt.$$
In the standard hypermap, there is a quadrilateral face $F'$ such that 
  $$\sigma(\Lambda,v_0,F') = \sum^4_{i=0}\sigma(\Lambda,v_0,F_i) <
   -1.04\,\pt.$$
Thus, the pentahedral prism is not proper.\FIXX{Need to discuss relation between the sigmas for refinements of fans. 
Notation of sigma should include the ambient fan.}
\end{proof}

\begin{lemma}\tlabel{lemma:pp-oct}\rating{100}
Let $(\Lambda,v_0)$ be a proper pentahedral prism.
Let $(v_0,V,E) = (v_0,V_{std}(\CalQ),E_{std}(\CalQ)$ be the $Q$-system
extended fan.
Let $\CalF$ be the set
of four faces of its hypermap that correspond to a quartered octahedron.
Then
   $$\sum_{F\in\CalF}(\sigma(\Lambda,v_0,F) + 0.3621 \sol(F)) > 0.49246.$$
\end{lemma}

\begin{proof}
Write the quartered octahedron as $(v_0,w,v_1,v_2,v_3,v_4)$.  We
have $\sol(F_i) = \sol(v_0,S_i)$, with $S_i=\{v_0,w,v_i,v_{i+1}\}$ for
$F_i\in \CalF$.  Also, by the rules for scoring
  $$\sigma(\Lambda,v_0,F_i) = (\mu(v_0,S_i) + \mu(w,S_i))/2.$$
This is an assembly of interval calculations%
\footnote{\calc{XX}. DCG-V 17.4.3.1 -- 17.4.3.6}
\end{proof}

\begin{lemma}\tlabel{lemma:acute-pp}
Let $(\Lambda,v_0)$ be a proper pentahedral prism.
Let $(v_0,V,E) = (v_0,V_{std}(\CalQ),E_{std}(\CalQ)$ be the $Q$-system
extended fan.
Let $F$ be a face
of its hypermap that tags a pure quad cluster $(v_0,v_1,v_2,v_3,v_4)$.
Assume the acuteness condition:
  $$
  |v_1-v_3|^2 \le |v_1-v_0|^2 + |v_3-v_0|^2
  $$
Then
  $$
  \sigma(\Lambda,v_0,F) + m \sol(F) < b, 
  $$
where $m = 0.3621$; $b = 0.49246$.
\end{lemma}

\begin{proof}
Set $E' = E\cup \{v_1,v_2\}$.  By tarski\tarf{tarski:XX}, it follows
that $(v_0,V,E')$ is a fan.  There are two faces  $F_1$ and $F_2$
that correspond to $F$ in the  hypermap of the $Q$-system extended fan.
We have
 $$
  \begin{array}{lll}
  \sigma(\Lambda,v_0,F) &= \sigma(\Lambda,v_0,F_1) + \sigma(\Lambda,v_0,F_2)\\
   \sol(F) &=\sol(F_1) + \sol(F_2)\\
 \end{array}
 $$
A calculation\footnote{\calc{XX} New one $\sigma < 0.04\,\pt$.}
gives 
  $$
  \sigma(\Lambda,v_0,F_i) < 0.02\,\pt.
  $$
Thus, by properness, we may assume that
  $$
  \sigma(\Lambda,v_0,F_i) \ge -1.06\,\pt.
  $$
A calculation\footnote{\calc{XX}. DCG-V-17.4.4.1--17.4.4.3.  
Use the constant $-1.06\,\pt$, rather than $-1.04\,\pt$, as is done there.}
\end{proof}


\begin{lemma}\tlabel{lemma:obtuse-pp}\rating{400}
Let $(\Lambda,v_0)$ be a proper pentahedral prism.
Let $(v_0,V,E) = (v_0,V_{std}(\CalQ),E_{std}(\CalQ)$ be the $Q$-system
extended fan.
Let $F$ be a face
of its hypermap that tags a pure quad cluster $(v_0,v_1,v_2,v_3,v_4)$.
Assume the obtuseness conditions:
  $$
  \begin{array}{lll}
  |v_1-v_3|^2 &\ge |v_1-v_0|^2 + |v_3-v_0|^2\\
  |v_2-v_4|^2 &\ge |v_2-v_0|^2 + |v_4-v_0|^2\\
  \end{array}
  $$
Then
  $$
  \sigma(\Lambda,v_0,F) + m \sol(F) < b, 
  $$
where $m = 0.3621$; $b = 0.49246$.
\end{lemma}

\begin{proof}
We push all four vertices $v_1,\ldots,v_4$ towards $v_0$.
That is, let $w_i = 2 v_i'/|v_i'|$ where $v_i = v_i-v_0$.
The solid angle is unchanged by pushing the four vertices.

The obtuseness condition scales to
   $$|w_1-w_3|\ge \sqrt8,\quad |w_2-w_4|\ge \sqrt8.$$
We discuss some deformations that preserve these bounds on diagonals.
Under scaling, we have
   $$2/t_0 \le |w_i - w_{i+1}|\le 2t_0.$$
We will require the deformations to preserve these bounds as well.
By an interval calculation\footnote{\calc{XX}. New one:
$\dih(2,2,2,2t_0,y_5,y_6) < \pi/2$, if $(y_5,y_6)\in[\sqrt8,4][2/t_0,2t_0]$}
we see that each angle $\dih(\{v_0,w_i\},\{w_{i-1},w_{i+1}\}) < \pi$, so
that each dart is convex.  If a deformation preserves the given
constraints on edge lengths and diagonals, then it preserves this
convexity property too.  This implies in particular that
$$\{v_0,w_{i-1},w_i,w_{i+1}\}$$
is not coplanar.

Let 
$$
  \begin{array}{lll}
  f(v_0,w_1,w_2,w_3,w_4) &= \op{sv}(v_0,\{v_0,w_1,w_2,w_3\},\sqrt2) +
    \op{sv}(v_0,\{v_0,w_1,w_4,w_3\},\sqrt2) \\
  &= \op{sv}(v_0,\{v_0,w_2,w_3,w_4\},\sqrt2) +
    \op{sv}(v_0,\{v_0,w_2,w_1,w_4\},\sqrt2) \\
  \end{array}
$$
The function $f$ can be expressed as a continuous function
of the four variables $|w_i-w_{i+1}|$ together with either diagonal
$|w_i-w_{i+1}|$.
For pure quad clusters satisfying the obtuseness conditions, 
we have 
$$\sigma(\Lambda,v_0,F) \le f(v_0,w_1,w_2,w_3,w_4).$$

Let
$$
  \begin{array}{lll}
  \sol(v_0,w_1,w_2,w_3,w_4) &= \sol(v_0,\{v_0,w_1,w_2,w_3\}) +
    \sol(v_0,\{v_0,w_1,w_4,w_3\}) \\
  &= \sol(v_0,\{v_0,w_2,w_3,w_4\}) +
    \sol(v_0,\{v_0,w_2,w_1,w_4\}) \\
  \end{array}
$$
Pushing vertices does not change solid angle.  Thus,
$$
\sol(v_0,w_1,\ldots,w_4) = \sol(F).
$$
The function $\sol$ is continuous on the domain.

Thus, the lemma follows if we show that
$$f(v_0,w_1,\ldots,w_4) + m \sol(v_0,w_1,\ldots,w_4) < b.$$
Suppose that we have a counterexample and that the counterexample
has solid angle $c$.  By the continuity of $f$, and the compactness
of the domain, there is a counterexample that maximizes $f$ constrained
to have solid angle $c$.  

Suppose that in this maximal counterexample, one of the diagonals, say 
$|w_1-w_3|$ is exactly $\sqrt8$.  
By Lemma~\ref{lemma:quoin-equilize}, maximality implies
$$
|w_1-w_i|=|w_i-w_3|,\quad i=2,4.
$$
The result now follows from an interval calculation%
\footnote{\calc{XX}. DCG-V-17.4.4.5}

If, on the other hand, both diagonals are greater than $\sqrt8$, then
Lemma~\ref{lemma:quoin-equilize} and maximality implies that
$|w_i-w_{i+1}|$ is independent of $i$.  Another interval
calculation%
\footnote{\calc{XX}. DCG-V-17.4.4.4}
treats this case.
\end{proof}

\begin{lemma}\tlabel{lemma:pp-contravene}
Let $(\Lambda,v_0)$ be a pentahedral prism.  Then it does not
contravene.
\end{lemma}

\begin{proof} We have already seen that it must be proper to
controvene.  Each quadrilateral face $F$ of the standard hypermap
gives
   $$\sigma(\Lambda,v_0,F) + m\sol(F) < b,$$
where $m,b$ are as given in Lemmas~\ref{lemma:acute-pp} and 
\ref{lemma:obtuse-pp}.  Each set $\CalF$ of face type $(5,0,0)$ gives
   $$
   \sum_{F\in\CalF}(\sigma(\Lambda,v_0,F) + m\sol(F)) < 5 b',
   $$
where $b'= 0.253095$ by another calculation.%
\footnote{\calc{XX}.  DCG-V-17.4.1.3.}
The pentahedral prism has five quadrilateral faces and two sets
of faces of face type $(5,0,0)$.  The sum of all solid angles is $4\pi$.
Thus we obtain
   $$
   \begin{array}{lll}
   \sigma(\Lambda,v_0) &\le 10 b' + 5 b - 4\pi m < 8\,\pt -\epsilon_0.
   \end{array}
   $$
This completes the proof.
\end{proof}

\section{Assembly Theory} \label{linear}


\FIXX{This section is severely out of date.   For now, it should be
ignored.}


In this section we define a class of nonlinear optimization
problems that we call {\it linear assembly problems}.

Assume given a topological space $X$, and a finite collection of
topological spaces, called {\it local domains}.  For each local
domain $D$ there is a map $\pi_D:X\to D$.  There are functions
$u_i$, $i=1,\ldots,N$, each defined on some local domain $D_i =
\op{dom}(u_i)$, and we let $x_i$ denote the composite $x_i =
\pi_{D_i}\circ u_i$.

On each local domain $D$, the functions $u_i$ are related by a
finite set of nonlinear relations
\begin{equation}\phi(u_i : \op{dom}(u_i) = D) \ge0, \quad \phi \in \Phi_D.
    \label{phi}
\end{equation}

We use vector notation $x = (x_1,\ldots,x_N)$, with constant
vectors $c$, $b$, and matrix $A$ given.

The problem is to maximize $c\cdot x$ subject to the constraints
    \begin{equation}\label{Ax}A\, x \le b,
    \end{equation}
and to the nonlinear relations~\ref{phi}.  A problem of this form
is called a linear assembly problem.  (Intuitively, there are a
number of nonlinear objects $D$, that form the pieces of a jigsaw
puzzle that fit together according to the linear
conditions~\ref{Ax}.)

\begin{example}
Assume a single local domain $D$, and let $\pi_D:X=D$ be the
identity map.  The function $f = c\cdot x $ is nonlinear. The
problem is to maximize $f$ over $D$ subject to the nonlinear
relations $\Phi_D$.  This is a general constrained nonlinear
optimization problem.
\end{example}

\begin{example}  Assume that each $u_i$ has a distinct local domain $D_i = \ring{R}$.
Let $X = \ring{R}^N$, let $\pi_D$ be the projection onto the $i$th
coordinate, and let $x_i$ be the $i$th coordinate function on
$\ring{R}^n$. Assume that $\Phi_D$ is empty for each $D$.  The
problem becomes the general linear programming problem
    $$\max c\cdot x$$
such that $A x\le b$.
\end{example}

These two examples give the nonlinear and linear extremes in
linear assembly problems. The more interesting cases are the mixed
cases which combine nonlinear and linear programming.
%Example~\ref{pr:third} gives one such case.

\begin{figure}[htb]
  \centering
  \myincludegraphics{\ps/vor.eps} %% CORRECT GRAPHIC?
  \caption{A truncated Voronoi cell and a subset of the cell lying in a sector}
  \label{voronoi}
\end{figure}

\begin{example} (2D Voronoi cell minimization). Take a packing of disks of
radius $1$ in the plane.  Let $\Lambda$ be the set of centers of
the disks.  Assume that the origin $0\in\Lambda$ is one of the
centers. The truncated Voronoi cell at $0$ is the set of all
$x\in\ring{R}^2$ such that $|x|\le t$, and $x$ is closer to the
origin than to any other center in $\Lambda$.  We assume
$t\in(1,\sqrt2)$.

Only the centers of distance at most $2t$ affect the shape and
area of the truncated Voronoi cell.  For each $n=0,1,2,\ldots$, we
have a topological space of all truncated Voronoi cells with $n$
nonzero disk centers $v_i$ at distance at most $2t$.  Fix $n$, and
let $X$ be the topological space.

Let $D=D_i$, $i=1,\ldots,n$,  be the sectors lying between
consecutive segments $(0,v_i)$.  Each sector is characterized by
its angle $\alpha$ and the lengths $y_a$ and $y_b$ of the two
segments $(0,v_i)$, $(0,v_j)$ between which the sector lies.  The
part $A$ in $D$ of the area of the truncated Voronoi cell is a
function of the variables $\alpha$, $y_a$, $y_b$.  A nonlinear
implicit equation $\phi=0$ relates $A$, $\alpha$, $y_a$, and $y_b$
on $D$. The variables $u_i$ of the linear assembly problem for the
local domain $D$ are $A$, $y_a$, $y_b$, $\alpha$.


We have a linear assembly problem.  The function $c\cdot x$ is the
area of the truncated Voronoi cell, viewed as a sum of variables
$A$, for each sector $D$ (or rather, their pullbacks to $X$ under
the natural projections $X\to D$).

The assembly constraints are all linear. One linear relation
imposes that the angles of the $n$ different sectors must sum to
$2\pi$. Other linear relations impose that the variable $y_a$ on
$D$ equals the variable $y_b$ on $D'$ if the two variables
represent the length of the same segment $(0,v_i)$ in $X$.
\end{example}


\subsection{solving linear assembly}

In this section we describe how various linear assembly problems
are solved for the sphere packing problem in terms
sufficiently general to apply to other linear assembly problems as
well.

Let us introduce some general notation.  Let $x_D = (x_i:
\op{dom}(u_i)=D)$ be the vector of variables with local domain
$D$. Write $c\cdot x$ in the form $\sum_D c_D\cdot x_D$ and the
assembly conditions as
$$A \,x =\sum_D A_D x_D,$$
according to the local domain of the variable.


\subsection{linear relaxation}
The first general technique is {\it linear relaxation}. We replace
the nonlinear relations $\phi(x_D)\ge0, \phi\in\Phi_D$ with a
collection of linear inequalities that are true whenever the
constraints $\Phi_D$ are satisfied: $A'_D x_D \le b_D$.  A linear
program is obtained by replacing the nonlinear constraints
$\Phi_D$ with the linear constraints. Its solution dominates the
nonlinear optimization problem.  In this way, the nonlinear
maximization problem can be bounded from above.

Let us review some constructions that insure rigor in linear
programming solutions. We assume general familiarity with the
basic theory and terminology of linear programming. It is
well-known that the primal has a feasible solution iff the dual is
bounded.  We will formulate our linear programs in such a way that
both the primal and the dual problems are feasible and bounded.

We use vector notation to formulate a primal problem as
    \begin{equation}
        \max\, c\cdot x
        \label{cx}
    \end{equation}
such that $A x \le b$, where $x$ is a column vector of free
variables (no positivity constraints), $A$ is a matrix, $c$ is a
row vector, and $b$ is a column vector.

We can insure that this primal problem is bounded by bounding each
of the variables $x_i$.  (This is easily achieved considering the
geometric origins of our problem, which provides interpretations
of variables as particular dihedral angles, edge lengths, and
volumes.) We assume that these bounds form part of the constraints
$A x\le b$.

The linear programs we consider have the property that if the
maximum is less than a constant $K$, the solution does not
interest us.  (For instance, in the dodecahedral conjecture,
Voronoi cell volumes are of interest only if the volume is less
than the volume of the regular dodecahedron.) This observation
allows us to replace the primal problem with one having an
additional variable $t$:
    %%


\subsection{nonlinear duality}
The second general technique is nonlinear duality.  Suppose that
we wish to show that the maximum of the primal problem~\ref{cx} is
at most $M$.

Let $x^* = (x^*_D)$ be a guess of the solution to the problem,
obtained for example, by numerical nonlinear optimization. We
relax the nonlinear optimization by dropping from the matrix $A$
and the vector $b$ those inequalities that are not binding at
$x^*$. With this modification, we may assume that $A\,x^*=b$.  Let
$m$ be the size of the vector $b$, that is, the number of binding
linear conditions. Let $d$ be the number of local domains $D$.

We introduce a linear dual problem with real variables $t$,
$r_\phi: \phi\in\Phi_D$, and $w\in\ring{R}^m$. The variables
$r_\phi$ and $w$ are constrained to be non-negative.

We consider the linear problem of maximizing $t$ such that
    \begin{equation}
        M + d\, t - c\cdot x^* \ge 0
        \label{Mx}
    \end{equation}
and such that for each $x_D$ in each $D$ the linear inequality
    \begin{equation}
        c_D\cdot (x_D-x^*_D) + \sum_{\Phi_D} r_\phi \phi(x) +
                    w A_D (x^*_D-x_D) + t
            < 0
        \label{xD}
    \end{equation}
is satisfied.

There is no guarantee that a feasible solution exists to this
system of inequalities.  However, any feasible solution gives an
upper bound $M$. Indeed, let $x=(x_D)$ be any feasible argument to
the primal, and let $t,r_\phi,w$ be a feasible solution to the
dual. Taking the sum of the linear inequalities~\ref{xD}, over $D$
at $x$, we have (recall $\phi\ge0$ and $A x\le b$):
$$
\begin{array}{lll}
M &\ge M + c\cdot (x-x^*) + \sum_D\sum_{\Phi_D} r_\phi \phi(x)
    + w A (x^*-x) + d\, t,\\
    &\ge c\cdot x + (M + d\, t - c\cdot x^*) + w (b-A x),\\
    &\ge c\cdot x.
\end{array}
$$

Since the dual problem has infinitely many constraints (because of
constraints for each $x\in D$), we solve the dual problem in two
stages. First, we approximate each $D$ by a finite set of test
points, and solve the finitely constrained linear programming
problem for $t, r_\phi$, and $w$.

We replace $t$ with $t_0 = (-M +c\cdot x^*)/d$ (to make the
constraint \ref{Mx} bind).  It follows from the feasibility of $t$
that $t\ge t_0$, and that $t_0,r_\phi,w$ is also feasible on the
finitely constrained problem. To show that $t_0,r_\phi,w$
satisfies all the inequalities~\ref{xD} (under the substitution
$t\mapsto t_0$), we use interval arithmetic to show that each of
these inequalities hold. (To make these interval arithmetic
verifications as easy as possible, we have chosen the solution
$t_0,r,w$ to make the closest inequality hold by as large a margin
$t-t_0$ as possible. This is the meaning of the maximization over
$t$ in the dual problem.)  The next section will give further
details about interval arithmetic verifications.

\subsection{branch and bound}
The third technique is branch and bound.  When no feasible
solution is found in step (2), it may still be possible to
partition $X$ into finitely many sets $X = \coprod X_i$, on which
feasible solutions to the dual may be found.  Although this is an
essential part of the solution, the rules for branching in the
sphere packing problem follow the structure of that problem, and we do
not give a general branching algorithm.


\subsection{definition and Interpretation}

\begin{definition}[quarter]
By $\optt{quarter}(\alpha)$ we mean that at dart $\alpha$, we have
an upright quarter.
\end{definition}

\begin{definition}[slice]
By $\optt{slice}(\alpha)$ we mean that at dart $\alpha$ we have an upright
diagonal and a slice: $\optt{azim}\slt \pi$ and
the opposite edge length is at most $3.2$ and at least $2$.
\end{definition}

\begin{definition}[gap]
By $\optt{gap}(\alpha)$ we mean that at dart $\alpha$, we have an
upright diagonal $\optt{azim}\slt \pi$ and the opposite edge length
is greater than $3.2$.
\end{definition}

\begin{definition}[upright]
By $\optt{upright}(\alpha)$ we mean that at $\alpha$ there is an upright
diagonal.
\end{definition}


If $\optt{f}$ is any of the
functions
    $$\optt{vor0},\optt{gamma}, \optt{nu},$$
we set $\optt{tau0}$, $\optt{tau\_gamma}$,
$\optt{tau\_nu}$, respectively,
to
    $$\optt{tau\_*} = -f(\alpha) +\optt{sol}(\alpha)\zeta\pt.$$
We set
    $$
    \optt{tau}(\alpha,t) = -
    \optt{sovo}(S,t,\lambda_{sq})+\optt{sol}(\alpha)\zeta\pt.
    $$
We say that $\alpha$ is compressed or decompressed 
according to the scoring of $\optt{mu}(\alpha)$.  (See
Section~\ref{sec:rules}.)

We  measure what is squandered by a flat quarter by $\hat\tau =
\sol\zeta\pt - \hat\sigma$.
\FIXX{Define width $y_4$ of a (geometric) dart,
types fitted crown, type C, enclosed masking dart, masking dart, etc. }


\subsection{basic relation}

\begin{lemma} If $\optt{upright}(\alpha)$ then $\optt{gap}(\alpha)$ or
$\optt{slice}(\alpha)$ or $\optt{azim}(\alpha) \sge \pi$.
\end{lemma}


\begin{lemma}  If $N$ is a node that is upright, then at least
one dart at $N$ is a quarter.
\end{lemma}


