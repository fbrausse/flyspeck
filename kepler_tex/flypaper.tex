% Author Thomas C. Hales
% Copyright Thomas C. Hales
% Format latex.
% 

%!TEX TS-program = latex    
%% This line is for TexShop. 

% Revision history. See svn.
% Document created Dec 6, 2002
% Revision started Jan 2007, from published DCG.

\documentclass[cup9a]{cupbook}
%% Cambridge University Press Macros from
%% https://authornet.cambridge.org/information/productionguide/laTex_files/

%

\usepackage{graphicx}
\usepackage{verbatim}
\usepackage{latexsym}
\usepackage{amsfonts}
%\usepackage{amsthm}
\usepackage{amsmath}
%\usepackage{mathidx}
\usepackage{makeidx}
\usepackage{multicol}
\usepackage{crop}
\usepackage{txfonts}
%\usepackage{pdfsync}  %for TexShop sync.
\usepackage[letterpaper,colorlinks=true,
            ps2pdf,hyperindex=true]{hyperref}
%\usepackage{mparhack} %http://www.tex.ac.uk/cgi-bin/texfaq2html?label=marginparside
\usepackage{multind}
\usepackage{url}






% This file contains local settings and system dependencies



% Auxiliary directories
\def\dsp{/Users/thomashales/Pictures/mathFigures/DenseSpherePackings}  % flypaper graphics
%\def\pdf{/Users/thomashales/Pictures/collect_geom} % tarski graphics
\def\pdfp{/Users/thomashales/Pictures/mathFigures/collect_geom} % kepler graphics

\def\showgraphics{t}  
% t: display graphics (there are none to show yet)
% f (default): print a "no graphics logo" where graphics would normally go.


\def\displayallproof{t} 
% t (default): display all proofs.
% f: print documents without the proofs-- theorem statements only

\def\displayrating{f}
% t (default): display all ratings (verbose is also true)
% f : don't show them.

\def\verbose{t}
% f (default): do not display debugging information,
% t : display debug information and information about the formalization.

     
%-%
% --Repository--
%-%
% generate revision number by
% svn propset svn:keywords "LastChangedRevision" kepmacros.tex
\def\svninfo{%
  Document TeXed on \today. \hfill\break
  Repository Root: https://flyspeck.googlecode.com/svn \hfill\break
  SVN $LastChangedRevision$
  }

%-%
% --Fonts--
%-%
\font\twrm=cmr8

%-%
% --Graphics--
%-%
%set \showgraphics option in flag.tex
% flypaper graphics
 \def\szincludegraphics[#1]#2{%
      \if\showgraphics t{\includegraphics[#1]{#2}}%
      \else{\includegraphics{noimage.eps}}\fi}

% % kepler graphics
% \def\pdffigtemplatex[#1]#2#3#4{%   
% % usage: \pdffigtemplatex[width=80mm]{file.eps}{labelname}{caption}
% \begin{figure}[htb]%
%   \centering
%  \szincludegraphics[#1]{\pdfp/#2}
%  \caption{#4}
%  \label{fig:#3}%
% \end{figure}%
% }

\def\tikzfig#1#2#3{%
\begin{figure}[htb]%
  \centering
\begin{tikzpicture}#3
\end{tikzpicture}
  \caption{#2}
  \label{fig:#1}%
\end{figure}%
}

\def\tikzwrap#1#2#3#4{%
\begin{wrapfigure}{r}{#4\textwidth}
  \begin{center}
\begin{tikzpicture}#3
\end{tikzpicture}
\end{center}
  \caption{#2}
  \label{fig:#1}%
\end{wrapfigure}%
}


%\def\pdfg#1#2#3#4{\if\showgraphics t{\pdffigtemplatex[#1]{#2}{#3}{#4}}\else{}\fi}
%\def\myincludegraphics#1{%
%      \if\showgraphics t{\includegraphics{#1}}%
%      \else{\includegraphics{noimage.eps}}\fi}


%-%
% --Footnotes and Endnotes--
%-%
% http://help-csli.stanford.edu/tex/latex-footnotes.shtml
%\long\def\symbolfootnote[#1]#2{\begingroup%
%\def\thefootnote{\ensuremath{\fnsymbol{footnote}}}\footnote[#1]{#2}\endgroup}

%-%
% --Special Formatting--
%-%
% http://en.wikibooks.org/wiki/LaTeX/Formatting#List_Structures
%\renewcommand{\labelitemii}{$\star$}
\renewcommand{\labelitemii}{$\circ$}
\renewcommand{\labelenumii}{\alph{enumii}}
\newenvironment{summary}
  {\begingroup\bigskip\narrower\noindent{\bf Summary.~}\it}
%  {~\ding{98}\par\phantom{!}\endgroup\bigskip}
  {~\par\phantom{!}\endgroup\bigskip}
\newenvironment{tidbit}{\smallskip\begingroup}{\endgroup\smallskip}
%\newenvironment{enumerate}
%  {\renewcommand{\labelitemi}{}\begin{itemize}}
%  {\end{itemize}}
\def\wasitemize{\relax}
\def\uncase#1{{\sc #1}}
\def\case#1{{\sc (#1)}}
\def\claim#1{{\it  #1}}
\def\calcentry#1#2#3#4{{\smallskip{\bf #1}\quad{\tt [#2]}\quad{(#3)}\quad {#4}}} % computer calc entry
\def\id#1{\ensuremath{\text{\tt #1}}}



%-%
% --Indexing, References, Citations--
%-%
\def\indy#1#2{\index{index/#1}{#2}}
%\def\eqn#1{{\bf (\ref{#1})}}   % deprecated, use \eqref.
\def\newterm#1{\indy{Index}{#1}{\it #1}\relax}
\def\oldterm#1{\indy{Index}{#1}{#1}\relax}
\def\cc#1#2{%
  \indy{Index}{computer calculation!{#1}}{\it computer calculation}%
  \ifverbose{\footnote{\guid{#1}  #2}}~\cite{website:FlyspeckProject}} % arg dropped.


%-%
% --Endnotes
%-%
%\renewcommand{\maketextnotes}{\global\textnotesontrue
%  \newwrite\textnotes
%  \immediate\openout\textnotes=\jobname.ent
% \literaltextnote{
%\notesheadername={\the\textnotesheadername}
%%\pagestyle{endnotesstyle}
%\mark{3}
%\label{textualnotes}
%\normalfont \backmattertextfont}
%}
%\newcommand{\shipnotes}{
%   \iftextnoteson
%   \theendnotes
%   \immediate\closeout\textnotes
%   \input \jobname.ent
%   \else
%   \relax
%   \fi
%}

%-%
% --Proof Display--
%-%
% set with \displayallproof in flag_fly. If f, then proofs are swallowed.
%% "proved" environment. toggle with \displayallproof
%
\def\hide#1{}
\def\swallowed{\relax}
\def\swallow#1\swallowed{}
\newenvironment{iproved}{}{}
\newenvironment{proved}{\resetproved\begin{iproved}}{\end{iproved}}
\def\hideproof{\renewenvironment{iproved}{%
   \centerline{\it -- Proof Proofed --}
  \renewenvironment{itemize}{}{}
  \renewenvironment{enumerate}{}{}
  \def\item{\relax}
  \catcode13=12
  \swallow
}{}}
\def\showproof{\renewenvironment{iproved}{\begin{proof}}{\end{proof}}}
\def\resetproved{\if\displayallproof t\showproof\else\hideproof\fi}



%-%
% --Debugging Information--
%-%
%% verbose:
\def\rating#1{\if\displayrating t%
  {{\textsc {[rating={\ensuremath {#1}}].\ }}}\else{}\fi}
\def\rz#1{\rating{#1}}
\def\cutrate{}
\def\oldrating#1{\if\displayrating t%
  {{\textsc {[former rating={\ensuremath {#1}}].\ }}}\else{}\fi}

\def\formalauthor#1{\if\verbose t{{\tt [formal proof by #1].\ }}\else{}\fi}
\def\dcg#1#2{{\if\verbose t%
  {{\tt{[DCG-#1]}}\indy{References}{ZC{#2 #1}@{DCG-#1}|page{#2}}}\else{}\fi}}
\def\tlabel#1{\label{#1}\if\verbose t{{\tt [#1].\ }%
   \indy{References}{#1|itt}}\else{}\fi}
\def\ifcverbose#1#2{\if\verbose t{{#1}}\else{#2}\fi}
\def\ifverbose#1{\ifcverbose{#1}{}}  %\verbose t{{#1}}\else{}\fi}
%\def\formal#1{\ifverbose{[{#1}]}}
\def\formal#1{\relax }
\def\formaldef#1#2{\ifverbose{\texttt{[{#1} $\leftrightsquigarrow$ {#2}]}}}
\def\footformal#1{\if\verbose t{\footnote{#1}}\else{}\fi}
\def\guid#1{{\tt[#1].\ }\indy{References}{ZA{#1}@{#1}|itt}}
\def\guid#1{\ifverbose{{\tt [#1]}}}
\def\guid#1{{{\tt [#1]}}}
\def\ineq#1{{{\tt  [#1]}}}
%\def\guid#1{{{\tt [#1]}}}

%\def\calc#1{{\textsc{calc-#1}}\indy{Interval}{{#1}@{#1}}}
%\def\xfootnote#1{\if\verbose t{\endnote{#1}}\else{\footnote{#1}}\fi}
%\def\xfootnote#1{\footnote{#1}}
%\def\xendnote#1{\if\verbose t{\endnote{#1}}\else{}\fi}


% margin notes
\setlength{\marginparwidth}{1.2in}
\def\mar#1{}
 %\ifverbose{\marginpar{\text{\raggedright\footnotesize #1}}}}
%\def\hypermark[#1]#2{\ifcverbose{\hyperref[#1]{#2}}{#2}}


%-%
%--Formatting--
%-%
\def\dfrac#1#2{\frac{\displaystyle #1}{\displaystyle #2}}
\def\textand{\text{ \ and \ }}  % for math eqns.

%-%
%--Redefining--
%-%
\def\emptyset{\varnothing}
\def\ups{\upsilonup} % Needs txfonts; else use \upsilon


%-%
% --Symbols--
%-%
% norm and brackets
\def\|{{\hskip0.1em|\hskip-0.15em|\hskip0.1em}}
\def\mid{\ :\ }
\def\tc{\hbox{:}}
\def\cooln{:\hskip-0.02em:}
\def\norm#1#2{\hbox{\ensuremath{\|#1\unskip-\unskip{#2}\|}}}
\def\normo#1{{\|#1\|}}
\def\sland{\ \land\ }
\def\abs#1{|#1|}
% brackets
\def\leftopen{(}
\def\leftclosed{[}
\def\rightopen{)}
\def\rightclosed{]}
\def\lp#1{{\llbracket{#1}\rrbracket}} 
%\def\comp#1{\llbracket #1 \rrbracket}
\def\comp#1{[#1]}
\def\tangle#1{\langle #1\rangle}
\def\ceil#1{\lceil #1\rceil}
\def\floor#1{\lfloor #1\rfloor}
%accents:
\def\=#1{\accent"16 #1}
\def\ast{\ensuremath{{}^*}}

% mathcal
\def\CalV{{\mathcal V}}
\def\CalL{{\mathcal L}}
\def\BB{{\mathcal B}}
\def\powerset{{\mathscr P}}

% mathbb
\newcommand{\ring}[1]{\mathbb{#1}}
%\def\N{{\mathbb N}}
%\def\Rp{\ring{R}^{3\,\prime}}
%\def\A{{\mathbf A}}
\def\F{{\mathbf F}} % map on faces H to H/{cal L}

% vector notation
\def\v{{\mathbf v}}
\def\u{{\mathbf u}}
\def\w{{\mathbf w}}
\def\e{{\mathbf e} }  
\def\p{{\mathbf p}}
\def\q{{\mathbf q}}

% operatorname
\def\op#1{{\operatorname{#1}}}
\def\optt#1{{\operatorname{{\texttt{#1}}}}}

%\def\opat{{\op{@}}}
\def\atn{\op{arctan\ensuremath{_2}}}
\def\azim{\op{azim}}
\def\nd{\op{node}}
\def\sol{\operatorname{sol}}
\def\vol{\op{vol}}
\def\dih{\operatorname{dih}}
\def\Adih{\operatorname{Adih}}
\def\arc{\operatorname{arc}}
\def\rad{\operatorname{rad}}
\def\bool{\operatorname{bool}}
\def\true{\op{true}}
\def\false{\op{false}}


%\def\orz{\varthetaup} % center of packing
\def\orz{{\mathbf 0}} % center of packing
\def\Wdarto{W^0_{\text{dart}}}
\def\Wdart{W_{\text{dart}}}
%\def\Wedge{W_{\text{edge}}}
\def\cell{\operatorname{cell}}
\def\dimaff{\operatorname{dim\,aff}}
\def\aff{\operatorname{aff}}
\def\card{\op{card}}

\def\del{\partial}
\def\doct{\delta_{oct}}
\def\dtet{\delta_{tet}}
\def\hm{{h_0}} % 1.26
\def\stab{c_{{\scriptstyle \text{stab}}}} % 3.01
\def\tgt{\operatorname{\it{target}}}
\def\pqr#1{#1} % marks type (p,q,r).
\def\trunc#1#2{#1\hbox{\ensuremath{[:\hskip-0.25em plus 0em minus 0em{#2}]}}}
\def\trunc#1#2{#1[\text{:}\hskip0em plus 0em minus 0em{#2}]}
\def\trunc#1#2{d_{#2}{#1}}
%\def\trunc#1#2{#1{[\le\hskip-0.25em{ #2}]}}

%% HYPERMAP macros:
% avoid e for both hypermap edge and edge {v,w}
\def\ee{\varepsilonup}
\def\ocirc{}
\def\wild{{*}}  % wildcard char.

%% PACKNG macros:
%\def\lam{\lambda}
%\def\Lam{\Lambda}
\def\bu{{\underline{\u}}}
\def\bv{{\underline{\v}}}
\def\bw{{\underline{\w}}}
\def\bV{{\underline{V}}}
%\def\arcs#1#2#3{{\arcV(#1,\{#2,#3\})}}

%% LOCAL FAN macros:
\def\smain{S_{\scriptstyle\text{main}}} 
 
 

 \crop
%\makeindex

% CUP BOOK CLASS SPECIFIC
\newtheorem{lemma}{Lemma}[chapter]
\newtheorem{definition}{Definition}[chapter]
\newtheorem{remark}{Remark}[chapter]
\newtheorem{theorem}{Theorem}[chapter]
\newtheorem{corollary}{Corollary}[chapter]
\newtheorem{example}{Example}[chapter]
\newtheorem{claim}{Claim}[chapter]
\newtheorem{notation}{Notation}[chapter]
\newtheorem{assumption}{Assumption}[chapter]
\newtheorem{interpretation}{Interpretation}[chapter]
\newtheorem{conjecture}{Conjecture}[chapter]
%%%%%%%%%%%%%%%%%%%%%%%%%%%%%%%%%%%%%%%%%%%%%%%%%%%%

\begin{document}
\raggedbottom  % for now.
%\raggedright  % don't worry for now.

    \title{Flyspeck :
      %\\ \phantom{Hello}
      \\  A Blueprint for the Formal Proof of the Kepler Conjecture
      \\ (Marchal Version)}
    \author{Thomas C. Hales}
    \maketitle
    \frontmatter
    \tableofcontents
    \thanks

\mainmatter

\noindent



\bigskip




\begin{note}%XX
This manuscript is not for ready general distribution.  Please do not circulate it.
\begin{enumerate}\wasitemize 
\item The computer calculations that back up various claims have not
  been completed.  In particular, various nonlinear inequalities
  remain to be proved.  The linear programs for one of the hypermaps
  have still not terminated.
\item The figures are missing.
\end{enumerate}\wasitemize 
\end{note}

\bigskip\noindent %
This research has been supported by the National Science Foundation
under Grants 0503447 and 0804189 as well as a grant from the Benter
Foundation.

\bigskip\noindent\svninfo 

\newpage



%%%%%%%%%%%%%%%%%%%%%%%%%%%%%%%%%%%%%%%%%%%%%%%%%%%%


%%  \label{part:intro}
   \newpage

   \setcounter{chapter}{-1}
   \chapter{Preface}
   \label{XX}  % for missing references. XX.
   %%------------------------------------------------------------
% Author: Thomas C. Hales
% Format: LaTeX
% Book Chapter: Dense Sphere Packings
%------------------------------------------------------------

\chapter*{Preface}

% Believe not everything, but only what is proven: the former is foolish, the latter the act of a sensible man. -- Democritus.

%{
%
%\narrower
%
%{\it ``A personality which has been veiled by a formal method
%  throughout many chapters is suddenly seen face to face in the
%  Preface.'' }
%% Introductory note, p3, Famous Prefaces, Harvard Classics, Vol
%% 39. P.F. Collier & Son, 1910.
%
%}

%\bigskip

%\centerline{\it ``Those who justify themselves do not convince.''
%  --Lao-Tzu}
%% quoted in A. Watts's essay in "Modern Buddhism" ed. Donald Lopez,
%% page 160


{

\narrower

{\it

  ``I think there's a revolution in mathematics around the corner. I
  think that $\ldots$ %in later times
  people will look back on the fin-de-siecle of the twentieth century
  and say `then is when it happened' (just like we look back at the
  Greeks for inventing the concept of proof and at the nineteenth
  century for making analysis rigorous). I really believe that. And it
  amazes me that no one seems to notice $\ldots$

  ``Never before have the platonic mathematical world and the physical
  world been this similar, this close. Is it strange that I expect
  leakage between these two worlds? That I think the proof strings
  will find their way to the computer memories?$\ldots$

  ``What I expect is that some kind of computer system will be
  created, a proof checker, that all mathematicians will start using
  to check their work, their proofs, their mathematics. I have no idea
  what shape such a system will $\ldots$ take. But I expect some
  system to come into being that is past some threshold so that it is
  practical enough for real work, and then quite suddenly some kind of
  `phase transition' will occur and everyone will be using that
  system.''

{\hfill--Freek Wiedijk \cite{FWR}} % http://www.cs.ru.nl/~freek/jordan/index.html

}

}

\newpage

{

\narrower\parindent=0pt
\parskip=0.4\baselineskip

{\it

Alecos: Christos has a problem with the `foundational quest'!

Christos:  Wrong!  I have two problems with your  {\rm {version}} of it!  One, it
didn't fail and, two, it wasn't a tragedy!  Granted, there are some tragic
parts!  But the ending is happy, as in the `Oresteia'!  

Apostolos:  Happy for whom?  Cantor, going insane?  G\"odel starving himself to
death out of paranoia? Hilbert or Russell and their psychotic sons? Or Frege with--

Christos: `The meaning
is in the ending!' you said so yourself!  So, follow the quest for ten more years and
you get a brand-new triumphant finale with the creation of the computer, which is the
quest's real hero!   Your problem is, simply, that you see it as a story of people!

Apostolos: Well, stories do tend to be about people!

Christos:  So, choose the right people!  And show what they really did!  All we we learn
of the great von Neumann is he said `It's over'  when he heard G\"odel!

Alecos: But it was over in a sense, wasn't it?  Pop went Hilbert's `no ignorabimus'!

Christos:  But then came the quest's jeune premier, its parsifal $\ldots$ Alan Turing!
He said `Ok, we can't prove everything! So, let's see what we can prove!' and to define
proof, he invented, in 1936, a theoretical machine which contains all the ideas of the
computer! $\ldots$ which, after the war, he and von Neumann, the quest's proudest sons,
brought to full life!

{\hfill--Logicomix} % page 303.

}

}




\newpage

{

\narrower\parindent=0pt
\parskip=0.4\baselineskip

{\it

``Ever hear of the Kepler Conjecture?''

``Nope.''

I laid the notebook on the table and flipped through the pages. ``It was first stated
in 1611 by Johannes Kepler,'' I said.  ``Kepler becomae interested in the problem while
he was corresponding with an Englishman named Thomas Harriot, who was trying
to help his friend Sir Walter Raleigh figure out the best way to stack cannonballs on ship
decks.  The goal was to find the densest possible spherical arrangements, $\ldots$ basically,
the way grocers stack oranges''

``Okay,'' he said, nodding.

``Kepler's conjecture {\rm{seems}} perfectly sound,'' I said.

``That is does,'' Ben said.

``But here's the thing. $\ldots$  I looked it up and discovered that, in 1998, a proof
had finally been put forward by an American mathematician named Thomas Hales.  In 2003,
a committee that had been assigned to verify Hales's work confirmed that they were ninety-nine
percent certain of the proof's correctness.  But that one percent was key.  The mathematical
world is still waiting for the publication of the data that will prove the Kepler Conjecture definitively.''

``Sucks for Thomas Hales,'' Ben said.

``I agree.  But it makes sense that they have to be certain, doesn't it?''


{\hfill--Michelle Richmond, No One You Know} % page 177.

}

}

%%DD Figure of cannonballs.

\bigskip

{

\narrower\parindent=0pt
\parskip=0.4\baselineskip

{\it

Sometimes fixing a $1$ percent defect takes $500$ percent effort.

{\hfill-- Joel Spolsky, Joel on Software} % page 122

}

}

\bigskip

{

\narrower\parindent=0pt
\parskip=0.4\baselineskip

{\it

Every one fully persuaded is a fool.

{\hfill-- Balthasar Graci\'an, the Art of Worldly Wisdon} % p110

}

}



\newpage

\section*{Blueprint for Formal Proofs}

In 1611, Kepler wrote a booklet in which he asserted that the familiar
cannonball arrangement of congruent balls in space achieves the
highest possible density.  No other arrangement fills a larger
fraction of space.  This assertion is the Kepler conjecture.  In 1900,
Hilbert made this conjecture part of his eighteenth problem.  This
book presents a proof of this assertion.

This assertion has become a test of the capability of computers to
deliver a reliable mathematical proof.  The original proof by Sam
Ferguson and me involved many long computer calculations that
exhausted the efforts of a team of referees.  This book represents my
efforts to redesign the proof in a way that makes the correctness of
the computer proof as transparent as possible.

After all is said and done, a proof is only as reliable as the
processes that are used to verify its correctness.  The ultimate
standard of proof is a formal proof.  A formal proof is nothing other
than an unbroken chain of logical inferences from an explicit set of
axioms.  While this may be the mathematical ideal of proof, actual
mathematical practice generally deviates significantly from the ideal.



%More than ten years have passed since a proof was first
%obtained. Why give a new presentation of the proof?
%
%The original proof was not widely understood.  The complexity was not
%because of conceptual challenges.  In fact, the proof makes only
%modest demands on the theoretical training of the reader.  It is
%possible to read and understand the proof with a knowledge of a
%limited body of mathematics, such as basic calculus and elementary
%Euclidean geometry.
%
%Nevertheless, the proof involves many long calculations. Even worse,
%it it relies on computer calculations.  An error in any calculation or
%a bug in the computer code has the potential to topple the entire
%proof.
%
%The referees were conscientious and checked many of the calculations.
%However, for the most part, the computer code lay beyond the scope of
%referee review, and even careful quality control can let a
%occasional bug slip through undetected.
%
%After all is said and done, no proof is more reliable than the
%reliability of the processes that are used to verify its
%correctness.  These processes include the checking that the author
%makes before releasing the proof for public scrutiny, the checking
%of the referees, and the checking done by readers after publication.

In recent years, as part of this project, I have been increasingly preoccupied by the
processes that mathematicians rely on to insure the correctness of complex
proofs. Researchers from Frege to G\"odel, who solved a problem of
rigor in mathematics, found a theoretical solution but did not
extinguish the burning fire at the foundations of mathematics
because they omitted the practical implementation. Some, such as
Bourbaki, have even gone so far as to claim that ``formalized
mathematics cannot in practice be written down in full'' and call
such a project
``absolutely unrealizable'' \cite[p 10,11]{Bour:68:Sets}. % Theory of
                                                          % Sets, page
                                                          % 10,11.

While it is true that formal proofs may be too long to print,
computers -- which do not have the same limitations as paper -- have
become the natural host of formal mathematics. In recent decades,
logicians and computer scientists have reworked the foundations of
mathematics, putting them in an efficient form designed for real use
on real computers.

For the first time in history, it is possible to generate and verify
every single logical inference of a major mathematical theorem.  This
has now been done for the four-color theorem, the prime number
theorem, the Jordan curve theorem, the Brouwer fixed point theorem,
and the fundamental theorem of calculus, among others.  Freek Wiedijk
reports that 82\% of a list of 100 famous theorems have now been
checked formally \cite{wiedijk:100}.  The list of 18 remaining
theorems contains two particular challenges: the independence of the
Continuum Hypothesis and Fermat's Last theorem.

Some mathematicians remain skeptical of the process because computers
have been used to generate and verify the logical inferences.
Computers are notoriously imperfect, with flaws ranging from software
bugs to defective chips.  Even if a computer verifies the inferences,
who will verify the verifier, or then verify the verifier of the
verifier?  Indeed, it would be unscientific of us to place an
unmerited trust in computers.

The choice comes down to two competing verification processes.  The
first is the traditional process of referees, which depends largely on
the luck of the draw -- some referees are meticulous, others are
careless.  The second process is formal computer verification, which
is less dependent on the whims of a particular referee.  In my view,
the choice between the conventional referee process and computer
verification is as evident as the choice between a sundial and an atomic
clock in contemporary science.

The boundary that separates an ``easy'' proof from a ``difficult''
proof shifts with current technology.  The introduction of steel in
architecture is not a mere reinforcement of wood and stone, it changes
the architect's world of possibilities.  There will no longer be any
reason to limit ourselves to ten-thousand-page proofs when our
technology supports million-page proofs.

The standard of proof I have adopted is the highest scientific standard
available by current technology.  That 
standard is formal verification by computer.  This standard
continues to evolve with the advancement of technology.

I dream of a fully formally verified solution to the
packing problem.  This project is still unfinished, but significant
progress is being made.  In this book, I rearrange the proof with
formal verification in mind .  The book is {\it a blueprint for formal
  proofs} because it gives the design of the formal proof to be
constructed.  My decisions about what to include in this book has been
shaped by the list of theorems already available in the library the
proof assistant {\tt HOL Light}.  For example, this book assumes basic
point-set topology and measure theory, which have been formalized by
John Harrison~\cite{HOLL}.

The style of formal proofs is different from that of conventional
proofs.  It is better to have a large number of short snappy proofs,
rather than a few intricate ones.  Humans enjoy surprising new
perspectives, but computers benefit from repetition and
standardization.  Despite these differences, I have worked to make
proofs that will bring pleasure to the human reader while providing
precise instructions for the implementation in silicon.



\section*{Structure of this Book}

The book is divided into parts.
The introductory part describe the major ideas, methods, and
organization of the proof.  

%There is an essay on each major computer
%component of the proof. The purpose is to provide a panoramic view of
%proof, to provide intuition about proof strategies.  After reading
%this part of the book, the reading should understand what the proof is
%all about, without yet dipping into technical details.
The part on foundations provides background material about
constructions in discrete geometry.    The first
of these chapters
covers trigonometric identities and basic vector geometry.  The second
treats volume from an elementary point of view.  The third chapter
covers planar graph theory from a purely combinatorial point of view.
The fourth chapter continues with planar graphs, now from a 
geometric point of view.

The next part of the book gives the solution to the packing problem.
The first chapter in this part gives a top-level overview of the major
steps of the proof.  It describes how the problem can be reduced from
a problem in infinitely many variables to a problem in finitely many
variables.  The remaining chapters in this part flesh out that
skeleton.

The final part of the book resolves some other longstanding conjectures in
discrete geometry: K. Bezdek's strong dodecahedral conjecture and Fejes
T\'oth's full contact conjecture.

Many simplifications of the original proof have been found over the past
several years.  The simplified proof is published here for the first time.
G. Gonthier expresses his formal proof of the four-color
theorem in terms of hypermaps.  He reworks the proofs of the
four-color theorem to avoid the use of the Jordan curve theorem, using
instead the much simpler notion of M\"obius contour.  I have followed
Gonthier's lead in these respects and also avoid the use of the Jordan curve theorem.

The optimality of the face-centered cubic packing is an assertion
about infinite space-filling packings.  For computational purposes, it is
useful to reduce the sphere packing problem to finite packings.  A
{\it correction term} is associated with each different reduction from
infinite packings to finite packings.  S. Ferguson and I considered a
large number of different correction terms.  We searched for one that
would simplify the computations as much as possible.  In a discussion
of the solution of the packing problem, I wrote that ``correction
terms are extremely flexible and easy to construct, and soon Samuel
Ferguson and I realized that every time we encountered difficulties in
solving the minimization problem, we could adjust $f$ [the correction
term] to skirt the difficulty. $\ldots$ If I were to revise the proof
to produce a simpler one, the first thing I would do would be to
change the correction term once again.  It is the key to a simpler
proof.''  C. Marchal has recently found a very simple 
correction term, that is, a simple way  to make the reduction from infinite packings
to finite packings.~\cite{marchal:2009}.  We use his reduction in this book.

There are many other improvements of the proof that are not visible in
the book, because they are implemented in computer code.  We have been
able to reduce the number of lines of computer code from over 187,000
to well under 10,000.  Needless to say, this significantly simplifies the 
formalization project.





\bigskip
\hbox{}



\bigskip
\hbox{}

{
\parindent=0pt
\obeylines

Thomas C. Hales
Pittsburgh, PA
May 2010

}







 %proofed
   
   

%%%%%%%%%%%%%%%%%%%%%%%%%%%%%%%%%%%%%%%%%%%%%%%%%%%%
  
    
    \chapter{Trigonometry}
\label{part:trig}
\indy{Index}{trigonometry}%

\begin{summary}
  This part of this book, which is the first of the four foundational
  chapters, presents a systematic development of trigonometry, volume,
  hypermap, and fan.  There is a separate chapter on each of these
  topics.  The purpose of the this material is to build a bridge
  between the foundations of mathematics, as presented in formal
  theorem proving systems such as HOL Light, and the solution to the
  packing problem.  

  In this chapter, trigonometry is developed analytically.  The basic
  trigonometric functions are defined by their power series
  representations, and calculus of a single real variable is used to
  develop the basic properties of these functions.  Basic vector
  geometry is presented.
\end{summary}


\section{Background Knowledge}

\subsection{formal proof}

We repeat that our purpose is to give a
blueprint of the formal proof of Kepler's conjecture that no packing of
congruent balls in three-dimensional Euclidean space has density
greater than the familiar cannonball packing.  The blueprint of a
formal proof is not the same as a formal proof, which is a 
fleeting pattern of bits in a computer.  The book describes to the
reader\footnote{``The words will be minced into atomized search-engine
  keywords~\dots{} copied millions of times by algorithms~\dots{}
  scanned, rehashed, and misrepresented by crowds\dots.  And yet
  it is you, the person, the rarity among my readers, I hope to
  reach.'' --Jaron Lanier \cite{Lanier}}  how to construct the
computer code that  produces and then reliably reproduces that
pattern of bits.

A more traditional book might take as its starting point the imagined
mathematical background of a typical reader.  The blueprint of a
formal proof starts instead with the current mathematical background
of a formal proof assistant.  I surveyed the knowledge of my formal
proof assistant and compared it with what is needed in the
construction of our formal proof.  It turns out that the proof
assistant already has an adequate background in real analysis, basic
topology, and plane trigonometry, including the trigonometric addition
laws, and formulas for derivatives.  Since the proof assistant already
has a significant library of theorems in real analysis and point-set topology, we
 use background facts in these areas wherever they help.


However, when this project began, the proof assistant lacked the
background in some of the less frequently used trigonometric
identities and has had nothing at all about spherical trigonometry.
While it had adequate command of general concepts of vector geometry
in $n$-dimensional Euclidean space, its knowledge of three-dimensional
analytic geometry was spotty.  For example, dihedral angles and
cylindrical and spherical coordinates were missing from the system.
%This foundational chapter supplies all of this necessary
%background in trigonometry and three-dimensional analytic geometry.

I imagine the typical reader to have a much stronger background in
trigonometry and analytic geometry than the proof assistant, which,
after all, is still in its youth.  The mathematician might want to
jump directly to the definition~\ref{def:aff} of the subsets
$\op{aff}_\pm$ of affine space.  This definition gives a compact
notation that encompasses many of the standard polyhedra (points,
lines, planes, rays, half-planes, half-spaces, convex hulls, affine
hulls) that appear throughout the book.  From there, the reader can
consult the definition of two important polynomials $\Delta$ and
$\ups$, make a note of the unorthodox notation $\arc(a,b,c)$ for the
angle opposite $c$ of a triangle with sides of lengths $a,b,c$, stop a
moment to admire Euler's formula for the solid angle of a spherical
triangle; and then jump directly to the final section, which
introduces polar cycle.

Polar cycle is a familiar concept, wrapped in an unfamiliar way
for the sake of the proof assistant: take a finite set of points in
the plane, order them by increasing angle, and then take the cyclic
permutation on the points induced by this order.  The azimuth cycle
is the corresponding permutation in three dimensions, ordering points by
increasing azimuth angle (longitude) in spherical coordinates.
Although intuitively clear, our proof assistant demands extra
assistance at this point.
\indy{Index}{longitude}%



\subsection{real analysis}
\label{back:analysis}  
This chapter assumes general facts about
real analysis at the level of a typical
undergraduate textbook.  In particular, it assumes a general working
knowledge of set theory and basic properties of the set of natural
numbers and the field of real numbers.  In real analysis, it assumes
basic properties of convergence, absolute convergence, limits, and
differentiation.  In this chapter, the term \newterm{real analysis} is
to be interpreted broadly to include even the most elementary facts of
real arithmetic, including results that do not involve limits.
\indy{Index}{real analysis}%
\indy{Index}{real arithmetic}%


\subsection{Tarski arithmetic}

\label{back:tarski}
  Certain sentences in real arithmetic can be expressed with nothing
  more than the usual logical operations (the connectives {\it and},
  {\it or}, {\it implies}, {\it logical negation}); the ring
  operations (addition, subtraction, and multiplication) for the real
  numbers; comparison ($(=)$ and $(>)$) of real numbers; the constants
  $0$ and $1$; real-valued variables; and quantifiers (universal and
  existential) over the real numbers.  Such sentences are said to
  belong to the Tarski arithmetic.  For example, the sentence
\begin{equation}\label{eqn:tarski}
\exists x.~x^7 - 4 x - 3 = 0 ~~\land~~ x > 0.
\end{equation}
falls within the Tarski arithmetic (after expanding the exponent $x^7$
as $x\cdot x\cdot x\cdot x\cdot x\cdot x\cdot x$ and the constants
$4=1+1+1+1$ and $3=1+1+1$).  Starting with Tarski, researchers have
developed algorithms to decide the truth of any sentence in the Tarski
arithmetic \cite{tarski-decision},~\cite{Mishra:1997}.  
Although these algorithms are generally too slow to be of practical
use, it is useful to identify such sentences.  To follow the details of proofs, 
reader should have the skill to solve particularly simple
problems in the Tarski arithmetic such as determining that the
sentence \eqref{eqn:tarski} is true.
\indy{Index}{Tarski arithmetic}%


\section{Trig Identities}


\subsection{sine and cosine}

The cosine and sine functions are defined\footnote{This is how the
  trigonometric functions were originally defined in the proof
  assistant HOL Light.  More recently, complex analysis has been
  developed in HOL Light sufficient for the analytic proof of
  the prime number theorem \cite{harrison:2009:pnt}.  The cosine and
  sine are now defined in the system as the real an complex parts of
  the exponential function $e^{i x}$.  To simplify the exposition, this section
  presents the original definitions.} by their infinite series:%
%\footformal{sin,\ cos,\ SIN\_0\, COS\_0}%




% from pgfmanual.pdf page 27, sec. 2.12
\begin{equation}\label{eqn:cos-def}\cos(x) = 1 - x^2/2! + x^4/4! \cdots,\qquad
  \sin(x) = x - x^3/3! + x^5/5! \cdots.
  \indy{Notation}{cos}%
  \indy{Index}{cosine}%
  \indy{Notation}{sin}%
  \indy{Index}{sine}%
  \indy{Index}{cosine!series definition}%
  \indy{Index}{sine!series definition}%
\end{equation}
%
\figODPCVGH % fig:trig

\mar{\guid{FOYTTIX} Eq.~\ref{eqn:cos-def}}
By real analysis, convergence is absolute
for every real number $x$.  Each series can be evaluated at $0$:
\begin{equation}\label{eqn:cos0}
  \cos(0) = 1,\qquad \sin(0) = 0.
\end{equation}
\mar{\guid{YIXJNJQ} Eq.~\ref{eqn:cos0}}


These series may be differentiated term by term to establish the
identities: \indy{Index}{cosine!derivative}%
\begin{equation}\label{eqn:cos'}
\frac{d\phantom{~}} {dx}\cos(x) 
= -\sin(x),\qquad \frac{ d\phantom{~} }{dx}\sin(x) = \cos(x).
\end{equation}
\mar{\guid{COHWECZ} Eq.~\ref{eqn:cos'}}%
The powers $(\cos(x))^n$ and $(\sin(x))^n$ are conventionally written
$\cos^n(x)$ and $\sin^n(x)$.

%Trigonometric identities follow easily from these definitions.    
If two functions are the {\it unique} solution of the same ordinary
linear differential equation with given initial conditions, then the
two functions are necessarily equal.  This observation gives
 a method to prove many functional identities,
including trigonometric identities.  
%This method can be developed
%further to give fully automated proofs of functional identities.  The
%intereseted reader may consult the mathematical literature of
%holonomic $D$-modules \cite{coutinho}, \cite{huishi-li},
%\cite{chyzak}.  % Chyzak % Huishi Li % Coutinho. p. 185.
The next two lemmas take this approach, by
%We do not strive to give a fully automated proof.  
 certifying a trigonometric identity with a function $f$ that
satisfies the ordinary differential equation $f' = 0$ with initial
condition $f(0)=0$.  
\indy{Index}{trigonometry!identities}%

\begin{lemma}[]\guid{WPMXVYZ}
\label{lemma:circle}\formal{SIN\_CIRCLE} 
\[ 
\sin^2(x) + \cos^2(x) = 1.
\] 
\end{lemma}
\indy{Index}{trigonometry!circle identity}%

% DEPRECATED HGMTQFG


\begin{proved}
  By real analysis and~\eqref{eqn:cos'}, the
  derivative of  $f(x) = \cos^2(x) +\sin^2(x)$ is
  identically zero, so the function itself is constant.
  From~\eqref{eqn:cos0}, it follows that $f(x)=f(0)=1$.
  \swallowed\end{proved}

%\footnote{
%  Incidentally, this trigonometric identity recently tried to crash
%  through the gates of physics.  Two robotics experts, Schmidt and
%  Lipson, wrote a computer program that automatically discovers
%  Hamiltons and Lagrangians from raw experimental data.  Discover
%  magazine reported that this program can discover the same laws in
%  hours that Newton took decades to find~\cite{discover-2009}.
%  However, one of the primary challenges of their project was to keep
%  out purely mathematical identities such as $\sin^2(x)+\cos^2(x)=1$,
%  which may try to pass as a conservation law with physical
%  significance~\cite{lipson}.
%}


\begin{lemma}[]\guid{WNYVJPE}\label{lemma:sin-add}
\formal{SIN\_ADD,\ COS\_ADD}
\begin{align*}
\sin(x+y) &= \sin(x)\cos(y) + \cos(x)\sin(y)\\
\cos(x+y)  &= \cos(x)\cos(y) - \sin(x)\sin(y).
\end{align*}
\end{lemma}
\indy{Index}{trigonometry!addition formula}%

%\figRUESSGQ % fig:cosadd  cos(x+y).

\begin{proved}
The proof is an exercise in real analysis.
Fix $y$.  Let
\begin{align*}
f(x) &=(\cos(x+y) - \cos(x)\cos(y) +
\sin(x)\sin(y))^2 \\ 
  &\quad+ (\sin(x+y) -\sin(x)\cos(y) -\cos(x)\sin(y))^2.
\end{align*}
The derivative of $f$ is identically zero.  The function is therefore
constant.  Also, $f(0)=0$.  Thus, $f$ is
identically zero.  If a sum of real squares is zero, the individual
terms are zero. The identities follow.  \swallowed\end{proved}

\begin{lemma}[]\guid{KGLLRQT}\label{lemma:cos-neg}
\formal{COS\_NEG,\ SIN\_NEG}
  The cosine is an even function.  The sine is an odd function.  That
  is,
\[ 
\cos(-x) = \cos(x),\quad\sin(-x) =
    -\sin(x).
\] 
\end{lemma}


\begin{proved}
The result can be checked directly from the definition of the trigonometric functions
as power series.  A second proof can be given by differentiation, as follows.
By real analysis, the derivative of
\[ 
(\cos(-x) - \cos(x))^2 + (\sin(-x)
  +\sin(x))^2
\] 
is identically zero.  Complete the proof as in the proof of
Lemma~\ref{lemma:sin-add}.  \swallowed\end{proved}

\subsection{periodicity}
\label{sec:pi}
\indy{Index}{periodicity}%

It is known that the cosine function has a unique root between $0$
and $2$. The constant $\pi$ is defined to be twice that root.  Thus, by
definition 
\begin{align}\label{eqn:cospi2}
\cos(\pi/2) &= 0,\nonumber\\
\cos(x) &>0,\quad \text{when } 0<x<\pi/2
\end{align}
\mar{\guid{CFXEKKP} Eq.~\ref{eqn:cospi2}}
The $\cos$ function is in fact
nonnegative on the interval $\leftclosed 0,\pi/2\rightclosed$:
\begin{equation}\label{eqn:cospos}
\cos(x)\ge 0,   \quad 0\le x \le \pi/2.
\end{equation}
\mar{\guid{ZSKECZV} Eq.~\ref{eqn:cospos}}
\indy{Index}{cosine!roots}%

\begin{lemma}[]\guid{CPIREMF}\label{lemma:sin-pi2}
\formal{SIN\_PI2}
$\sin$ is nonnegative on $[0,\pi/2]$ and  $\sin (\pi/2) = 1.$
\end{lemma}

\begin{proved}
  The proof is an exercise in real analysis.
  The derivative of $\sin$ is
  nonnegative between $0$ and $\pi/2$.  The
  value of $\sin$ at $0$ is $0$.  It follows that
  $\sin$ is nonnegative on $[0,\pi/2]$.  It is enough to check that
   $\sin^2(\pi/2)$ equals $1$.  Then $\sin^2(\pi/2)
  = {1-\cos^2(\pi/2)}
  = 1$.  \swallowed\end{proved}

\begin{lemma}[]\guid{SCEZKRH}\label{lemma:cos-sin}
\begin{align*}
\sin(\pi/2 - x)&=\cos(x),\\
\cos(\pi/2 - x)&=\sin(x).
\end{align*}
\end{lemma}

\begin{proved}
Apply the addition law  for the sine function (Lemma~\ref{lemma:sin-add}),
\[ 
\sin(\pi/2 - x) = \sin(\pi/2)\cos(-x) + \cos(\pi/2)\sin(-x)
\] 
and use $\sin(\pi/2) = 1$ and
$\cos(\pi/2) = 0$.  Then use that $\cos$ is an
even function.  The second identity is
similar.  \swallowed\end{proved}

Similarly,~%
%\footformal{SIN\_COS,\ SIN\_PERIODIC\_PI,\ COS\_PERIODIC\_PI, 
%SIN\_PERIODIC,\ COS\_PERIODIC}%
$\cos(\pi/2 + x) =
-\sin(x)$, $\sin(\pi/2 + x) = \cos(x)$.  Further,
\begin{alignat}{2}
\label{eqn:periodic}
\sin(\pi + x) &= \phantom{-}\cos(\pi/2 + x) &= -\sin(x),\nonumber\\
\cos(\pi + x) &= -\sin(\pi/2 + x) &= -\cos(x),\nonumber\\
\sin(2\pi + x) &= -\sin(\pi + x) &= \phantom{-}\sin(x),\\
\cos(2\pi + x) &= -\cos(\pi + x) &= \phantom{-}\cos(x)\nonumber.
\end{alignat}
\mar{\guid{LLOYXRK} Eq.~\ref{eqn:periodic}}%
\indy{Index}{trigonometry!periodicity}%

\begin{lemma}[]\guid{WIBGJRR}\label{lemma:sin-pos}
$\sin$ is nonnegative on $[0,\pi]$.
\end{lemma}

\begin{proof} By Lemma~\ref{lemma:sin-pi2}, $\sin$ is nonnegative on
  $[0,\pi/2]$.  Furthermore, for $x\in[\pi/2,\pi]$,
\[ 
  \sin(x) = -\sin(-x)   =  \sin(\pi-x) \ge 0.
\] 
\end{proof}



\subsection{tangent}
\label{sec:tangent}

\begin{definition}[tangent]\guid{BIRXGXP}\label{def:tan}
\formaldef{$\tan$}{tan}
Let $\tan(x) = \sin(x)/\cos(x)$, defined when $\cos(x)\ne0$.
%(See Figure~\ref{fig:trig}.)
\indy{Index}{tangent}%
\indy{Notation}{tan@$\tan$}%
\end{definition}



\begin{lemma}[]\guid{KWYPRWZ}
\label{lemma:tan-add}\formal{TAN\_ADD}
If $\cos(x)\ne 0$, $\cos(y)\ne 0$, and $\cos(x+y)\ne0$ then
\[ \tan(x+y) = \frac{\tan(x) + \tan(y) }{ 1 -
    \tan(x)\tan(y)}\] 
\end{lemma}
\indy{Index}{trigonometry!tangent}%

\begin{proved}
  Divide the first line of Lemma~\ref{lemma:sin-add} by the second
  line of the same lemma.  Then use the definition
  of $\tan$.  \swallowed\end{proved}

\begin{lemma}[]\guid{KSQDZSF}\label{lemma:tan-pi4}\formal{TAN\_PI4}
\[ \tan(\pi/4) = 1.\] 
\end{lemma}

\begin{proved}  
\[ 
\tan(\pi/4) = \sin(\pi/2-\pi/4)/\cos(\pi/4) 
  =
  \cos(\pi/4)/\cos(\pi/4) = 1.
\] 
\swallowed\end{proved}

\begin{lemma}[]\guid{UTNKIAC}\label{lemma:tan-monotone}
The function $\tan$ is strictly increasing and one-to-one on the domain
$\leftopen-\pi/2,\pi/2\rightopen$.
\end{lemma}

\begin{proof} By a derivative test, the function $\tan$ is strictly
  increasing on $\leftopen-\pi/2,\pi/2\rightopen$.  By
  real arithmetic, a strictly increasing
  function is one-to-one.
\end{proof}

\subsection{arctangent}

This section reviews the properties of the arctangent function.  

\begin{definition}[arctangent]\guid{RIQVMHH}\label{def:arctan}
\formaldef{$\arctan$}{atn}
\formal{atn,\ ATN,\ ATN\_TAN,\ ATN\_BOUNDS,\ TAN\_ATN}
  By the inverse function theorem of real
    analysis and properties of $\tan$,
  there is a unique function $\arctan:\ring{R}\to\ring{R}$ with image
  $(-\pi/2,\pi/2)$ such that
\begin{equation}\label{eqn:tanarctan}\tan(\arctan x) =x.\end{equation}
(See Figure~\ref{fig:trig}.)
\mar{\guid{EWITKLU} Eq.~\ref{eqn:tanarctan}}
\indy{Index}{arctangent}%
\end{definition}
%


Additional properties of the arctangent function are exercises in
real analysis.  If $-\pi/2 < x < \pi/2$,
then also $\arctan(\tan(x)) = x$. In particular,%
%\footformal{ATN\_1}
\begin{equation}\label{eqn:arctan-1}\
\arctan(1) = \arctan(\tan(\pi/4)) = \pi/4.
\end{equation}
\mar{\guid{YTXYLRB} Eq.~\ref{eqn:arctan-1}}  % X->Y


The function $\arctan$ is differentiable with derivative%
%\footformal{\ ATN\_MONO\_LT,\ ATN\_MONO\_LT\_EQ}
\begin{equation}\label{eqn:deriv-tan}\frac{d\phantom{~}} {dx} \arctan(x) = \frac{1}{1 +
    x^2}.\end{equation}
\mar{\guid{OKENMAM} Eq.~\ref{eqn:deriv-tan}}
The derivative is everywhere positive, and the function $\arctan$ is
strictly increasing.   \mar{\guid{LQCXGZX} increasing}
\indy{Index}{arctangent!derivative}%
Proofs in this book often need to use $\arctan(y/x)$ as  $x$ approaches $0$.
For this, the following variant of $\arctan$ is preferable because it clears the denominator.


\begin{definition}[$\atn$]\guid{GYKGARD}\label{def:atn}
\formaldef{$\atn$}{atn2}
\[ 
\atn: \ring{R}^2 \to \leftopen-\pi,\pi\rightclosed.
\] 
\[ 
\atn(x,y) = \begin{cases}
\arctan(y/x), & x > 0\\
\pi/2- \arctan(x/y), & y > 0 \\
\pi + \arctan(y/x), & x< 0,\  y\ge 0\\
-\pi/2- \arctan(x/y), & y< 0 \\
\pi, & x= y=0.\\
\end{cases}
\] 
\end{definition}
\indy{Notation}{arctan2@$\atn$}%
\indy{Notation}{arctan@$\arctan$}%
\indy{Index}{arctangent!atn@$\atn$}%
%
\figYOXQFUB % fig:atn-polar


There is some overlap between cases. Nevertheless, trig identities
similar to those already established show that this function is
well-defined.  For example, to check the equality of the first two
cases, we compute the tangent of both sides, which is sufficient,
since both sides lie between $\leftopen-\pi/2,\pi/2\rightopen$ and
$\tan$ is one-to-one:
\[ 
  \tan(\arctan(y/x)) = y/x = 
  1/\tan(\arctan(x/y)) = \tan(\pi/2 - \arctan(x/y)).
\] 
We can give a more intuitive description of
the function $\atn$:    the polar angle of $(x,y)$ with the
branch cut along the negative axis.  That is, $x = r\cos\theta$ and
$y=r\sin\theta$ for some $r\ge0$, where $\theta=\atn(x,y)$.  This definition avoids all
the case distinctions of Definition~\ref{def:atn}.

The ANSI C programming language implements this function as {\it
  arctan2}.  Note that some programming languages implement this
function with the two arguments in reverse: $(y,x)$.
\indy{Index}{arctangent!near 0}%
%\indy{Notation}{r@$r$ (coordinate)}%
\indy{Notation}{r@$r$ (polar, cylindrical, and spherical radius)}%
\indy{Notation}{xy@$(x,y)$ (Cartesian point)}%
%\indy{Notation}{zzh@$\theta$ (coordinate)}%
\indy{Notation}{zzh@$\theta$ (polar, cylindrical, and spherical angle)}%


\subsection{inverse trig}
\indy{Index}{trigonometry!inverse}%

We prefer the arctangent over other inverse trigonometric functions
because its domain is the entire field of real numbers, its range is
bounded, and its derivative is a rational function.  Wherever angles
appear in this book, the arctangent is apt to appear as well.  Other
inverse trigonometric functions are generally  reduced to the
arctangent.  This section defines the $\arccos$ function and shows how
it can be expressed in terms of $\atn$.

\begin{definition}[arccos]\guid{QZTBJMH}
\formaldef{$\arccos$}{acs}
\label{def:arccos}\formal{acs,\ ACS\_COS,\ COS\_ACS}
  By the inverse function theorem of real
    analysis, there exists a unique function $\arccos y$ on the
  interval $[-1,1]$, which takes values in $[0,\pi]$ and which is the
  inverse function of $\cos$:
\begin{align*}
y\in [-1,1] &\Rightarrow \cos(\arccos y) = y\\
x\in[0,\pi] &\Rightarrow \arccos(\cos x) = x
\end{align*}
\indy{Index}{arccosine}%
\indy{Notation}{arccos}%
%(See Figure~\ref{fig:trig}.)
\end{definition}


\begin{lemma}[]\guid{FMGMALU}\label{lemma:sin-arccos}
\formal{sin\_acs\_t} 
  If $y\in[-1,1]$, then
\[ \sin(\arccos(y)) = \sqrt{1-y^2}.\] 
\end{lemma}

\begin{proved}
  The range of $\arccos(y)$ is $[0,\pi]$.  On
  this interval, $\sin$ is nonnegative.  By
  real analysis, it is enough to check that
  the squares of the two nonnegative numbers are equal.  It then an
  arithmetic consequence of the circle identity
  (Lemma~\ref{lemma:circle}) and Definition~\ref{def:arccos}.
  \swallowed\end{proved}

The following lemma shows how to rewrite any occurrence of the $\arccos$ function
in terms of  $\atn$.   
%Our preference is to remove the $\arccos$ function whenever
%possible, by replacing it with the $\atn$ function through the
%following identity.  


\begin{lemma}[]\guid{OUIJTWY}\label{lemma:arccos-arctan}
\formal{acs\_atn2\_t}  
  If $y\in [-1,1]$, then
  \[ \arccos(y) +  \atn({
      \sqrt{1-y^2}},{y}) = \pi/2.\] 
\end{lemma}
\indy{Index}{trigonometry!arccos}%
\indy{Index}{trigonometry!arctan}%
\indy{Notation}{arccos}%
\indy{Notation}{arctan@$\arctan$}%
\mar{FIGURE:The two acute angles of a right triangle have sum $\pi/2$.}

\begin{proved}
The brief justification is simply that 
$\arccos(y/z)$ gives one acute angle of a right triangle with
hypotenuse $z$ and sides $x$ and $y$, and $\atn(x,y)$ gives the other acute angle.
The two acute angles of a right triangle have sum $\pi/2$.

A bit more detail is needed for an argument that can be turned into a formal proof.
  The endpoints $y=\pm1$ can be checked directly from definitions.  If
  $y\in (-1,1)$, $\beta = \arccos(y)$, and \[ \alpha =
    \arctan(y/\sqrt{1-y^2}) =
    \atn({\sqrt{1-y^2}},{y}),\]  then arithmetic gives
  $-\pi/2 < \pi/2 - \beta < \pi/2$, and $-\pi/2 < \alpha
    < \pi/2$.  By the injectivity of
  the function $\tan$, it is therefore enough to check that
  $\tan(\pi/2 - \beta) = \tan(\alpha)$.  But
\[ 
\tan(\pi/2-\beta)=
\frac{\cos(\beta)}{\sin(\beta)} =
\frac{y}{        \sin(\arccos(y))} 
=\frac{y}{ \sqrt{1-y^2}} 
=\tan(\alpha).
\] 
\swallowed\end{proved}



\section{Vector Geometry}

This section reviews vector geometry in $\ring{R}^N$, including
products (scalar and dot), inequalities (triangle and Cauchy-Schwarz),
and hulls (convex and affine).

\subsection{Euclidean space}

\begin{definition}[$\ring{R}^N$,~vector]\guid{KRZJIAD}
\formaldef{$\ring{R}^N$}{:real\textasciicircum N}
  For any finite set $N$, define $\ring{R}^N$ as the set of functions
  $\v:N\to\ring{R}$. Write $v_i$ for the value of the function $\v$ at
  $i\in N$. 
  \indy{Notation}{reals@$\ring{R}^N$}%
  A function in $\ring{R}^N$ is called a \newterm{vector}.  The zero
  vector $\orz$ is the function that is identically zero.
  \indy{Index}{vector}%
\end{definition}
\indy{Index}{vector!zero}%
Vectors are written in a bold face: $\u$, $\v$, $\w$, $\p$, $\q$, and
so forth.  As a general notational practice, there is a general
tendency to use $\u$, $\v$, and $\w$ to denote vectors that are
constrained to lie in some previously determined subset $V\subset
\ring{R}^N$ and to use $\p$ and $\q$ to denote vectors that run
without restriction over all of $\ring{R}^N$.

No distinction is made between vectors and points in $\ring{R}^N$, and
none is made between $\ring{R}^N$ and Euclidean space.  Write
$\ring{R}^n$ as an alias of $\ring{R}^N$ when $n\in\ring{N}$ and
$N=\{0,\ldots,n-1\}$.  
\indy{Index}{Euclidean space}%

\begin{definition}[vector addition,~scalar multiplication]\guid{WHIAXYC} % X->Y
\formaldef{vector addition}{(+)}
\formaldef{scalar multiplication}{(\%)}
  Two standard arithmetic operations, addition and scalar
  multiplication, are defined on the set $\ring{R}^N$.  These
  operations are the pointwise addition and scalar multiplication of
  functions:
\begin{align}
(\u + \v)_i &= u_i + v_i.\nonumber\\
(t \u)_i &= t u_i,\quad t\in\ring{R}.
\end{align}
\indy{Index}{vector!addition}%
\indy{Index}{vector!scalar multiplication}%
Define the difference of two vectors to be $\u - \v = \u + (-1) \v$.
\indy{Index}{vector space} %
\indy{Index}{vector!subtraction}%
\end{definition}
The operations on $\ring{R}^N$ 
satisfy the axioms of a vector space. 
In particular, addition is commutative and associative.


\begin{definition}[dot product]\guid{VFPCZBI}
\label{def:dot}
\formaldef{dot product}{(dot)}
The  \newterm{dot product} $(\,\cdot\,)$ is the
 bilinear binary operation on $\ring{R}^N$
%\[ 
%(\cdot):\ring{R}^N\to\ring{R}^N\to\ring{R}
%\] 
defined by
\[ 
\u\cdot \v = \sum_{i\in N} u_i v_i.
\] 
\indy{Index}{vector!dot product}%
\indy{Notation}{4@$\cdot $ (dot product)}%
\end{definition}


The dot product satisfies the following
properties:
\begin{align}\label{eqn:dot}
\u \cdot (\v + \w) &= \u \cdot \v + \u \cdot \w\nonumber\\
(\u + \v)\cdot \w &= \u \cdot \w + \v \cdot \w\nonumber\\
(t \u)\cdot \w &= t(\u \cdot \w) = \u \cdot (t \w)\\
0 &\le \u\cdot \u\nonumber
\end{align}


\begin{definition}[norm]\guid{XHVXJVB}
\label{def:norm}
\formaldef{norm}{vector\_norm}
The \newterm{norm} of a vector $\u\in\ring{R}^N$ is
\[ \normo{\u} = \sqrt{\u\cdot \u}.\] 
\indy{Index}{vector!norm}%
\end{definition}
%\indy{Notation}{norm@\hbox{$\normo{\u}$} (vector norm)}%

By  real arithmetic,
$\normo{\u}=0$  if and only if $\u=\orz$.  Moreover,
$\normo{ t \u } = |t| \, \normo{\u}$.   

% The distance function $d(\u,\v) = \norm{ \u }{ \v}$ makes
% $\ring{R}^N$ into a metric space.  \indy{Index}{metric space}%
% The proof that $d$ is indeed a metric depends on the Cauchy-Schwarz
% inequality:


\begin{lemma}[Cauchy-Schwarz~inequality]\guid{JJKJALK}
\formal{Jordan/metric\_spaces.ml:cauchy\_schwartz}
  \[ |\u \cdot \v| \le
    \normo{\u}\,\normo{\v}.\]  Furthermore, the case
  $\pm \u\cdot \v = \normo{\u}\,\normo{\v}$ of equality holds exactly
  when $\normo{\v} \u = \pm\normo{\u} \v$ (with matching signs).
\end{lemma}
\indy{Index}{Cauchy-Schwarz inequality}%

\begin{proved}
  This is an exercise in real arithmetic.  Let $\w = \normo{\v} \u \pm
  \normo{\u} \v$.  The expansion of $\w\cdot \w$ gives
  \[ 0\le \w\cdot \w = 2\normo{\u}^2\normo{\v}^2 \pm
    2\normo{\u}\, \normo{\v} (\u\cdot \v) = 2\normo{\u}\, \normo{\v}
    (\normo{\u}\, \normo{\v} \pm (\u \cdot \v)).\]  If
  $2\normo{\u} \,\normo{\v} = 0$, then $\u$ or $\v$ is zero, and the
  result easily ensues.  Otherwise divide both sides of the
  inequality by the positive quantity $2 \normo{\u} \,\normo{\v}$ to
  get the result.  \swallowed\end{proved}

\begin{lemma}[triangle~inequality]\guid{OIPLPTM}
\formal{Jordan/metric\_spaces.ml:norm\_triangle}
\label{lemma:triangle-ineq}
\[ 
\normo{\u + \v} \le \normo{\u} + \normo{\v }.
\] 
Equality holds exactly when $\normo{\v}\u = \normo{\u}\v$.
\end{lemma}
\indy{Index}{triangle inequality}%

\begin{proved}  This is an exercise in real arithmetic.
Both sides are nonnegative; it is enough to compare the squares of
both sides.  By the Cauchy-Schwarz inequality,
\[ \normo{\u + \v}^2 = \u\cdot \u + 2 \u\cdot \v + \v\cdot \v \le
  \u\cdot \u + 2 \normo{ \u}\,\normo{\v} + \v\cdot \v = (\normo{\u}+\normo{\v})^2.
\] 
The case of equality follows from the case of equality in the
Cauchy-Schwarz inequality.
\swallowed\end{proved}



\subsection{affine geometry}




Most of the following definitions apply to
  $n$-dimensional Euclidean space; however, this book uses them only
  in two and three dimensions.  The first definition gives the affine
  span of a finite set.  For example, the affine span of two distinct
  points is a line; the affine span of three independent points is a
  plane.  By placing additional positivity constraints on the linear
  combinations, the definitions extend to a large assortment of other
  geometric objects such as rays, half-planes, convex hulls, and
  cones.  Each of these comes in two versions: an open version defined
  by strict inequality and a closed version defined by weak
  inequality.  For example, the closed half-plane includes a bounding
  line and the open half-plane does not.  In this chapter, open and
  closed are not topological notions; rather, they indicate the
  semialgebraic conditions of strict and weak inequality.


%% No notation is introduced for a general affine hull!
\begin{definition}[affine hull]\guid{KVLZSAQ}
\formaldef{$\op{aff}$}{(hull) affine}
A set $A\subset\ring{R}^N$ is \newterm{affine}, if for
every finite nonempty subset $S\subset A$ and every function $t:S\to\ring{R}$ such that $\sum _{\v\in S} t(\v)=1$, we have
% $\v,\w\in A$ and every $t \in \ring{R}$, 
\[ 
 \sum_{\v\in S} t(\v) \v \in A.  % t \v + (1-t) \w \in A.
\] 
The \fullterm{affine hull}{affine!hull}, $\op{aff}(S)$, of a set $S\subset\ring{R}^N$ is the smallest affine set
containing $S$. 
That is, the affine hull of $S$ is the intersection of all affine
sets containing $S$. 
\end{definition}
\indy{Index}{hull!affine}%



\begin{definition}[affine]\guid{BYFLKYM}\label{def:aff} 
\formaldef{$\op{aff}_\pm$}{aff\_ge, aff\_le}
\formaldef{$\op{aff}^0_\pm$}{aff\_gt, aff\_lt}
  If $V = \{\v_1,\v_2,\ldots,\v_k\}$ and $V'=\{\v_{k+1},\ldots,\v_n\}$
  are finite subsets of $\ring{R}^N$, then set
	\begin{align*}
\op{aff}_{\pm} (V,V') &= \{t_1 \v_1 +\cdots t_n \v_n \mid
	t_1 +\cdots+t_n = 1, \pm t_j \ge 0, \text{ for } j>k\},\\
\op{aff}^0_{\pm} (V,V') &= \{t_1 \v_1 +\cdots t_n \v_n \mid
	t_1 +\cdots+t_n = 1, \pm t_j > 0, \text{ for } j>k\}.
%\op{aff}\, V &= \op{aff}_\pm(V,\emptyset).\\
		\end{align*}
To lighten the notation for singleton sets, abbreviate
$\op{aff}_\pm(\{\v\},V')$ to $\op{aff}_\pm(\v,V')$.
\indy{Notation}{aff2@$\op{aff}_{\pm}$, $\op{aff}^0_{\pm}$}%
\indy{Index}{affine}%
\indy{Notation}{V@$V\subset\ring{R}^n$}%
\end{definition}

\figITGCYIF % fig:affset. 

\begin{remark}
  When $n+1=\card(V)+\card(V')$, the generic set $\op{aff}_+(V,V')$ is
  an $n$-dimensional polyhedron bounded by $\card(V')$ hyperplanes.
  For example, $n=1$, gives a segment, a ray, or a line
  (Figure~\ref{fig:affset}).  When $n=2$, the set is a $2$-simplex, a
  planar wedge bounded by two lines, a half-plane, or a plane.  When
  $n=3$, the set is a $3$-simplex; an unbounded connected region in
  space bounded by one, two, or three intersecting planes; or all of
  $\ring{R}^3$.
\end{remark}

\begin{definition}[convex hull]\guid{OWECYNV}
\formaldef{$\op{conv}$}{(hull) convex}
A subset $C\subset\ring{R}^N$ is \newterm{convex}, if for
every $\v,\w\in C$ and every $t \in \leftclosed0,1\rightclosed]$,
\[ 
t \v + (1-t) \w \in C.
\] 
If $S\subset\ring{R}^N$, then let $\op{conv}(S)$ be the smallest convex set
(or equivalently, the intersection of all convex sets)
containing $S$.  It is called the \newterm{convex hull}.
\end{definition}

When the set is finite, the convex hull takes the following form.

\begin{lemma}[]\guid{GDCZMLO}
If $V = \{\v_1,\v_2,\ldots,\v_n\}\subset\ring{R}^N$, then
	\[ 
\op{conv}\, V = \op{aff}_+\, (\emptyset,V)
\] 
\indy{Notation}{conv}%
\indy{Index}{convex hull}%
\end{lemma}

\begin{lemma}[]\guid{UIVNNRR}
If $V\subset\ring{R}^N$ is finite, then
$\op{aff}_\pm (V,\emptyset) = \op{aff}^0_\pm(V,\emptyset)$
is the affine hull of $V$.
\end{lemma}

\begin{proof}  Both proofs are left as  exercises for the reader.
\end{proof}

In the following definition of a cone, the point $\v$ serves as apex,
and $V$ is a generating set for the positive directions.  In the
special case that $V$ is a singleton $\{\w\}$, the cone gives a ray
originating at $\v$ and passing through $\w$.  Later chapters call
a set of the form $\op{aff}_+(\v,\{\u_1,\u_2\})$ a \newterm{blade}.
Blades are planar sets bounded by two rays originating at $\v$.
\indy{Notation}{v7@$\v\in\ring{R}^3$}%
\indy{Index}{blade}%

% Cone deprecated Jan 17, 2012.
%\begin{definition}[cone]\guid{LLOUBAX}
%\formaldef{$\op{cone}$}{cone}
%Let $V$ be a finite subset of
%$\ring{R}^N$ and let $\v\in\ring{R}^N$. Set
%\[ 
%\op{cone}(\v,V) = \op{aff}_+(\{\v\},V)
%\op{cone}^0(\v,V) &= \op{aff}^0_+(\{\v\},V)\\
%\] 
%\indy{Index}{cone}%
%\indy{Notation}{cone@$\op{cone}$}%
%\end{definition}

% The Voronoi cell is one of the fundamental geometric objects in this
% book.  Earlier chapters have already discussed it at great length.
% Some authors use a weak inequality in the definition, others strict.
% The definition takes strict inequalities.


%\begin{definition}[Voronoi cell $\Omega$]\guid{VOWVHBW}
%Let $V$ be a finite set of points in 
%$\ring{R}^3$.  Let $\v\in\ring{R}^3$. Set
% \[ 
%\Omega(\v,V) = 
%\{x \mid \norm{\v}{x} \le \norm{\w}{x} forall \w\in V\setminus\{\v\}\}
%\] 
%\end{definition}

%% Changed to weak inequality May 14, 2009. -tchales.
	
\begin{definition}[line,~collinear,~parallel]\guid{SWKFLBJ}
\formaldef{line}{line}
\formaldef{collinear}{collinear}
\formaldef{parallel}{parallel}
  Any set of the form $\op{aff}\{\v,\w\}$ is a \newterm{line} when
  $\v\ne \w$.  A set that is contained in some $\aff\{\v,\w\}$ is
  \newterm{collinear}.  If $\{\orz,\v,\w\}$ is collinear, then
  $\v$ and $\w$ are said to be \newterm{parallel}. Also, $\{\v,\w\}$
  is said to be a parallel set.
\end{definition}
\indy{Index}{line}%
\indy{Index}{collinear}%

\begin{definition}[plane, half plane, coplanar]\guid{JLWZFBH}\label{def:plane}
\formaldef{plane}{plane}
\formaldef{half-plane}{closed\_half\_plane, open\_half\_plane}
\formaldef{coplanar}{coplanar}	
  An affine hull $A=\op{aff}\{\u,\v,\w\}$ is a \newterm{plane} when
  $\{\u,\v,\w\}$ is not collinear.  A set $\op{aff}_\pm(\{\u,\v\},\{\w\})$
  is a \newterm{half-plane} when $\{\u,\v,\w\}$ is not collinear. A
  set that is contained in some $\aff\{\u,\v,\w\}$ is \newterm{coplanar}.
\end{definition}
\indy{Index}{plane}%
\indy{Index}{half-plane}%
\indy{Notation}{A@$A$ (plane)}%
\indy{Index}{coplanar}%


\begin{definition}[half space]\guid{OAUVFPS} 
\formaldef{half space}{closed\_half\_space, open\_half\_space}
A set
  $\op{aff}_{\pm}(\{\u,\v,\w\},\{\v'\})$ is a \newterm{half-space},
  when $\{\u,\v,\w,\v'\}$ is not coplanar.  Under the substitution of
  $\op{aff}_{\pm}$ for $\op{aff}_{\pm}^0$, it is called an
  \newterm{open half-space}.
\end{definition}
\indy{Index}{half-space}%
\indy{Index}{half-space!open}%

\subsection{parallelepiped}\label{sec:piped}
\indy{Index}{parallelepiped}%



The following polynomial, $\Delta$, appears in many different
functions related to the geometry of three dimensions.  The formula
following the definition shows that it is closely related to the
square of the volume of a parallelepiped.  The interpretation as
volume is not relevant until the next chapter, but  its nonnegativity is
immediately relevant.  

%% WW repeated DEF.
\begin{definition}[$\Delta$]\guid{AVWKGNB}\label{def:delta}
\formaldef{$\Delta$}{delta\_x}
\formal{definitions\_kepler.ml:delta\_x}
  Let
\begin{align*}
\Delta(x_1,\ldots,x_6) &= x_1 x_4 (- x_1+x_2+x_3- x_4+x_5+x_6)\\
&\qquad+x_2 x_5 (x_1- x_2+x_3+x_4- x_5+x_6)\\
&\qquad+x_3 x_6 (x_1+x_2- x_3+x_4+x_5- x_6)\\
&\qquad- x_2 x_3 x_4- x_1 x_3 x_5- x_1 x_2 x_6- x_4 x_5 x_6.
\end{align*}
\end{definition}
\indy{Notation}{zzD@$\Delta$}%
\indy{Index}{Cayley-Menger!determinant}%
\indy{Index}{determinant!Cayley-Menger}%


\begin{remark}[Cayley-Menger determinant]\guid{DQGHCSH}\label{rem:cayley}
  The polynomial $\Delta$ first appears in the following context.
  Cayley and Menger found a formula for the square of the determinant
  $D$ of the matrix with rows $\v_1-\v_0$, $\ldots,$ $\v_n-\v_0$ for
  arbitrary vectors $\v_i\in\ring{R}^n$.  Set
\begin{equation}\label{eqn:xij}
x_{ij} = \norm{\v_i}{\v_j}^2,
\end{equation}
arranged as entries of a matrix $[x_{ij}]$.
Write $\underbar 1$ for a row vector of length $n$ 
with entries that are all equal to $1\in\ring{R}$.
They found that elementary matrix manipulations give an identity
of determinants:
\begin{align}\label{eqn:cmd}
D^2 &= \frac{(-1)^{n-1}}{2^n}
\left|\begin{matrix}[x_{ij}]& {}^t{\underbar 1}\\ {\underbar 1}& 0
\end{matrix}\right|.
\end{align}
The right-hand side is a polynomial in the squares of the edge lengths.
% The special case $n=2$ gives the polynomial $\ups$
% (Definition~\ref{def:ups}).
\indy{Notation}{D@$D$ (determinant)}%

A calculation of the determinant on the right when $n=3$ yields 
the polynomial $\Delta$.
\[ 
4 D^2 = \Delta(x_{01},x_{02},x_{03},x_{23},x_{13},x_{12}).
\] 
The left-hand side is evidently a square and the polynomial on the
right is nonnegative, whenever the variables $x_{ij}$
satisfy~\eqref{eqn:xij} for some vectors
$\v_0,\ldots,\v_3\in\ring{R}^3$.  Moreover, $D$ and hence also
$\Delta$ is positive when the set of four vectors is not
coplanar.
\end{remark}
\indy{Index}{edge!length}%
\indy{Notation}{zzu@$\ups$ (polynomial)}%
\indy{Notation}{zzD@$\Delta$}%
%\indy{Notation}{xij@$x_{ij}=\norm{\v_i}{\v_j}$}% % doesn't parse
%Write $\Delta_j$ for the $j$th partial derivative of $\Delta$. 
%Let $D = \det(\v_2-\v_1,\v_3-\v_1,\v_4-\v_1)$.
\indy{Index}{determinant}%

\begin{background}[matrix theory]\label{back:matrix}
  Very little matrix theory is required in this book.  The next lemma
  is a rare exception.  Its proof requires various very basic facts
  about $3\times 3$ matrices and determinants.  The determinant of a
  product of two matrices is the product of determinants.  The
  transpose of a matrix $A$ has the same determinant as $A$.  The
  determinant of a matrix $A$ is zero if and only if there exists a
  (row) vector $\u$ such that $\u\, A = \orz$.
\end{background}

\begin{lemma}[]\guid{CTCZHMR}\label{lemma:delta-pos}
  Let $V=\{\v_0,\v_1,\v_2,\v_3\}\subset\ring{R}^3$.  Let $x_{ij} =
  \norm{\v_i}{\v_j}^2$.  Then $\Delta(x_{ij})\ge 0$.  Moreover, the
  set $V$ is coplanar if and only if $\Delta(x_{ij}) = 0$.%
\mar{\guid{LBEVAKV} $\ge$ LEG}  % delta >=0, 
\mar{\guid{POLFLZY} $=$ LEG} % coplanar <=> delta=0.
\mar{more on $\Delta$ in LEG}
\end{lemma}

\begin{proof} The proof is an exercise in
  matrix theory and
  real arithmetic.  (The statement also
  falls within the scope of Tarski
  arithmetic.)  This lemma can be proved directly as follows, without
  recourse to the general Cayley-Menger theorem.

  Let $A$ be the $3\times 3$ matrix with rows  $\v_i - \v_0$.  Then
  $D^2 = \det(A)^2 = \det(A \,\hbox{}^t A)$.  Each entry of the
  product $A\,\hbox{}^tA$ is a dot product $(\v_i-\v_0)\cdot
  (\v_j-\v_0)$, which can be expressed in terms of the constants
  $x_{ij}$ by the following identity:
\begin{align}\label{eqn:dot-law}
  2 (\v_i-\v_0)\cdot (\v_j-\v_0) 
&= (\v_i-\v_0)\cdot(\v_i-\v_0) + (\v_j-\v_0)\cdot (\v_j-\v_0)\nonumber \\
  & \qquad - (\v_i-\v_j)\cdot (\v_i-\v_j)\nonumber\\
&= x_{i0} + x_{j0} - x_{ij}.
\end{align}
%which equals $x_{i0} + x_{j0} - x_{ij}$.
A computation of the determinant then gives $4D^2=\Delta$.
Thus, $D^2\ge0$ implies $\Delta\ge 0$.

Also $\Delta=0$ if and only if $D=0$, which holds if and only if $\u\,
A = \orz$ for some vector $\u$.  By the
definition of coplanar, this holds if and only
if $V$ is coplanar.
\end{proof}

\begin{remark}\guid{SZIHGLO}
  The calculation of the general Cayley-Menger formula~\eqref{eqn:cmd}
  for $n+1$ points in $\ring{R}^n$ is based on the same method as the
  $3\times 3$ case; the identity~\eqref{eqn:dot-law} gives a rewrite
  rule for each matrix entry of $\det(A\,\hbox{}^t A)$ as a linear
  combination of the variables $x_{ij}$.  Row and column operations
  then put the matrix in a form in which each matrix entry is a single
  variable $x_{ij}$.
\end{remark}

\begin{remark}[]\guid{KZVHHBG}\label{rem:CM5}
  The volume of a $4$-simplex in $\ring{R}^3$ is zero.  This implies
  that Cayley-Menger determinant for $\v_0,\ldots,\v_4\in\ring{R}^3$
  is zero.  %
  \mar{\guid{NUHSVLM} LEG}%
  \mar{\guid{RPFVZDI} LEG}%
  \mar{\guid{GJWYYPS} LEG}%
  \mar{\guid{GDLRUZB} LEG}% 
  \mar{\guid{LTCTBAN} LEG}%
  This gives a polynomial relation between the $10 = \tbinom{5}{2}$
  squared edge lengths $x_{ij}$.  The relation $a x^2 + b x + c=0$ is
  quadratic in the tenth edge, say $x=x_{04}$, in terms of the other
  nine. The leading coefficient $a$ is nonzero if $\{\v_1,\v_2,\v_3\}$
  is not collinear.
\end{remark}

\section{Angle}\label{sec:angle}

Until now, the discussion of trigonometric functions has been purely
analytic.  This section interprets them geometrically.  It covers
fundamental identities in both Euclidean and spherical trigonometry,
including the law of cosines, the law of sines, the spherical law of
cosines, and a beautiful formula that Euler and Lagrange gave for the area of a spherical triangle.

If $\v,\w$ are nonzero vectors, then by the
Cauchy-Schwarz inequality,
\[ -1 \le \frac{\v\cdot \w}{\normo{\v}\,\normo{\w}}
  \le 1.\]  The middle term  lies in the
domain of the function $\arccos$. The value of this function is the angle in  the following
definition.  \indy{Index}{Cauchy-Schwarz inequality}%
\indy{Notation}{v7@$\v\in\ring{R}^3$}%
\indy{Notation}{wi@$\w\in\ring{R}^3$}%

\begin{definition}[angle,\ arclength]\guid{WZYUXVC}\label{def:angle}
\formaldef{$\arc_V$}{arcV}
Let $\u,\v,\w$ be vectors with $\u\ne \v,\w$.
Define 
\[ 
  \arc_V(\u,\{\v,\w\}) = \arccos\left(\frac{(\v-\u)\cdot 
(\w-\u)}{\norm{\v}{\u}\,\norm{\w}{\u}}\right).
\] 
The value of this function is the \newterm{angle} at $\u$ formed by
$\v$ and $\w$.  
\indy{Index}{angle}%
\indy{Index}{arc}%
\indy{Index}{arclength}%
\indy{Notation}{wi@$\w\in\ring{R}^3$}%
\indy{Notation}{arcv@$\arc_V$}%
\end{definition}

\figLIKEURF % fig:arcV

\begin{remark}\guid{OAXCOPU}
  According to the formalist,  a definition requires no
  justification.  As Hilbert
  famously said, ``One must be able to say at all times -- instead of
  points, straight lines, and planes -- tables, beer mugs, and
  chairs.''  The intuitive geometer asks for more: angles should be
  invariant under isometries of $\ring{R}^n$, and when $\u,\v,\w$ are
  mapped by an isometry into a fixed plane, this definition of angle
  in $\ring{R}^n$ should agree with the accepted definition in the
  plane.  This definition meets such a standard.  The norm
  (Definition~\ref{def:norm}) also extends the intuition of the plane,
  obtained as it is by successive applications of the Pythagorean
  theorem in two dimensions.
\[
\normo{\u}^2 = u_0^2 + v_0^2,\quad v_0^2 = u_1^2 + v_1^2,\ldots
\quad v_{n-2}^2 = u_{n-1}^2 + v_{n-1}^2,\qquad v_{n-1}=0.
\]
\end{remark}

By the relation between $\arccos$ and $\atn$
(Lemma~\ref{lemma:arccos-arctan}), %if $|\u\cdot \v|\ne
                                   %\normo{\u}\,\normo{\v}$,
%then 
\begin{equation}\label{eqn:angle}
  \arc_V(\orz,\{\v,\w\}) = \frac{\pi}2 - \atn\left ({\sqrt{(\normo{\v}^2\normo{\w}^2 -
        (\v\cdot \w)^2)}}, {\v\cdot \w}\right).
\end{equation}
\mar{\guid{ACNBFRL} Eq.~\ref{eqn:angle}}
\indy{Index}{arclength}%

The notation $\arc_V$ for angle comes from its interpretation as the
length of a geodesic arc on a unit sphere
centered at $\u$ from point $\v$ to $\w$.
\indy{Index}{arc!geodesic}%
The subscript $V$ is a reminder that
the function arguments are vectors.  The function
$\arc$, without the subscript,  gives the angle as a function
of the three edge lengths of a triangle.
\indy{Notation}{V@$V$ (subscript marking vector functions)}%
\indy{Notation}{arc@$\arc$}%

% \begin{definition}[arc length]\guid{WHDSIZT} 
%The arclength of a geodesic arc on a
%   unit sphere centered at $\v_0$ from point $\v_1$ to $\v_2$ is the
%   angle formed by $\v_1$ and $\v_2$ at $\v_0$.
%\end{definition}

\begin{definition}[arc]\guid{PQQDENV}\label{def:arc}
\formaldef{arc}{arclength}
Define
\[ \arc(a,b,c) = \arccos(\frac{a^2 + b^2 - c^2}{2 a
    b}).\] 
\indy{Index}{arc}%
\end{definition}

If the triangle inequalities hold:
\[ 
a + b \ge c,\quad b + c \ge a, \quad c+a \ge b
\] 
and if $a,b >0$, then
\[ 
  2 a b = (\mp a+b+c)(a \mp b \pm c) \pm (a^2 + b^2 - c^2) 
\ge \pm  (a^2 + b^2 - c^2)
\] 
and the argument of $\arccos$ in the definition of $\arc$ falls within
its domain.

\begin{lemma}[law of cosines]\guid{HQTBPCM}\label{lemma:loc}
Let $\u,\v,\w$ be vectors with $\v\ne \u$, $\w\ne \u$.  Let $a
= \norm{\w }{ \u}$, $b = \norm{\v }{ \u}$, and $c = \norm{\v }{ \w}$.
Let $\gamma=\arc_V(\u,\{\v,\w\})$.    Then
\[ c^2 = a^2 + b^2 - 2 a b \cos\gamma.\] 
Also,
\[ 
\arc_V(\u,\{\v,\w\})= \arc(a,b,c).
\] 
%if $\v$, $\w$, and $\u$ are not collinear then

\end{lemma}
\indy{Notation}{zzc@$\gamma$ (angle)}%
\indy{Index}{arc}%
\indy{Index}{law of cosines} %
\indy{Index}{trigonometry!law of cosines}%
\indy{Index}{cosine!law of cosines}%

\begin{proved}
By the definition of $\arc_V$, the
definition of $\arccos$, and~\eqref{eqn:dot-law},
\[ 
2 a b \cos \gamma = 2 (\w - \u)\cdot (\v - \u) = a^2 + b^2 - c^2.
\] 
This identity can be solved for $\gamma$ and
 gives the final statement of the lemma.  \swallowed\end{proved}


\begin{definition}[$\ups$]\guid{OBPIOXD}\label{def:ups}
\formaldef{$\ups$}{ups\_x}
Let $\ups$ (the symbol is a Greek upsilon, which is written with a
wider stroke than a roman vee) be the polynomial
\[ \ups(x,y,z) = -x^2 - y^2 - z^2 + 2 x y + 2 y z + 2
  z x.\] 
\indy{Notation}{zzu@$\ups$ (polynomial)}%
\end{definition}



%% WW Repeated def (tarski.tex)
This polynomial is nonnegative under conditions described by the
following lemma. 


\begin{lemma}[]\guid{QRAAWFS}\label{lemma:ups} Let
  $V=\{\v_0,\v_1,\v_2\}\subset\ring{R}^3$.  Let $x_{ij} =
  \norm{\v_i}{\v_j}^2$.  Then $\ups(x_{01},x_{12},x_{02})\ge 0$.
  Moreover, the set $V$ is collinear if and only if $\ups(x_{ij}) =
  0$.%
\mar{\guid{FHFMKIY} $=$ LEG}
\end{lemma}

\begin{proof}
The polynomial factors
\begin{equation}\label{eqn:ups}
\ups(a^2,b^2,c^2) = 16 s (s-a) (s-b)
  (s-c),
\end{equation}
\mar{\guid{IHIQXLM} Eq.~\ref{eqn:ups}}
where $s = (a+b+c)/2$.  If $a,b,c$ are the sides of a triangle, then
$a,b,c\ge0$ and the triangle inequality (Lemma~\ref{lemma:triangle-ineq})
holds for all orderings of sides: $(b+c-a)\ge 0$ and so forth.
Non-negativity $0\le \ups(a^2,b^2,c^2)$ follows from the triangle
inequality applied to each factor in the factorization of $\ups$:
$2(s-a) = (b+c-a) \ge0$ and so forth.  The case of equality in the lemma is the
case of equality in the triangle inequality.
\indy{Index}{Cauchy-Schwarz inequality}%
\indy{Index}{triangle inequality}%
\end{proof}

An alternative way to view nonnegativity is
that $\ups$, like $\Delta$, is the square of a Cayley-Menger determinant
\eqref{eqn:cmd}.  
\[
0\le (2D)^2 = -
\left|\begin{matrix} 0 & a^2 & b^2 & 1\\ a^2 & 0 & c^2 & 1\\  b^2 & c^2 & 0 & 1\\
1 & 1 & 1& 1
\end{matrix}\right| = \ups(a^2,b^2,c^2).
\]
Section~\ref{sec:cross}  further identifies
the determinant $D$ as the norm of a cross product.
Volume and area are the topics of the next chapter, but it
is appropriate at this point to consider a formula for the area of a
triangle.  By means of formula \eqref{eqn:ups} for $\ups$, Heron's classical formula
for the area of a triangle with sides $a,b,c$ can be put in the form
\[ \sqrt{\ups(a^2,b^2,c^2)}/4.\] 
\indy{Index}{Heron's formula}%


\begin{lemma}[law of sines]\guid{UKBAHKV}\label{lemma:los}
Assume that $a,b>0$ and $a+b\ge c$, $b+c\ge a$, and $c+a\ge b$.
Let $\gamma=\arc(a,b,c)$.  Then
\[ 2 a b \sin\gamma =
  \sqrt{\ups(a^2,b^2,c^2)}.\] 
\end{lemma}
\indy{Index}{trigonometry!law of sines}%
\indy{Index}{law of sines}%
\indy{Index}{sine!law of sines}%
\begin{proved}
  Both sides are nonnegative, so it is
  enough to check that their squares are equal.  By the definition of
  $\arc$, we have
\[ 
4 a^2 b^2 \sin^2\gamma 
= 4 a^2 b^2 (1-\cos^2\gamma) 
= (4 a^2 b^2 - (a^2 + b^2 -
c^2)^2) 
= \ups(a^2,b^2,c^2).\] 
% checked 4/4/2008
\swallowed\end{proved}

Another useful relation writes $\arc$ in terms of $\atn$.
\begin{equation}\label{eqn:arc-atn}
\arc(a,b,c) = 
\pi/2 - \atn({\sqrt{\ups(a^2,b^2,c^2)}},{ a^2 + b^2 - c^2}).
\end{equation}
\mar{\guid{GVWTZKY} Eq.~\ref{eqn:arc-atn}}
This follows directly from Lemma~\ref{lemma:arccos-arctan} and the
definitions of $\arc$ and
  $\ups$.



\subsection{cross product} \label{sec:cross}

This book makes infrequent use of the cross product.
A definition and the most basic properties  suffice.

\begin{definition}[cross product]\guid{FCUAGAJ}\label{def:cross}
\formaldef{cross product}{(cross)}   
Let $\v =(x,y,z)$ and $\w = (x',y',z')$.  
Let the cross product be defined
by
\[ 
\v \times \w = (y z' - y' z, z x' - x z', x y' - y x').
\] 
\indy{Index}{cross product}%
\indy{Index}{vector!cross product}%
\indy{Notation}{5@$\times$ (cross product)}%
\end{definition}

\begin{lemma}[]\guid{KVVWPNA}  
\label{lemma:los-cross}
Any two vectors $\v,\w\in \ring{R}^3$ satisfy
\[ \normo{\v \times \w} =
  \normo{\v}\,\normo{\w}\sin\gamma,\] 
where $\gamma=\arc_V(\orz,\{\v,\w\})$.
Also, $\v \cdot (\v\times \w) = \w\cdot (\v\times \w) = \orz$.
\end{lemma}

\begin{proved} This proof is an exercise in
  real arithmetic and basic trigonometry.
  Both the left and
  right sides are nonnegative, so it is
  enough to compare the squares of both sides.  The square of the
  left-hand side is
\begin{align}\label{eqn:cross-dot}
  \normo{\v \times \w}^2 &= (y z'- y'z)^2 + (z x' - x z')^2 + (x y' - y x')^2 \notag\\
  &= %\qquad\qquad=
  (x^2 + y^2 + z^2)(x'^2 + y'^2 + z'^2) - (x x' + y y' + z z')^2
 \notag \\&= %\qquad\qquad =\, 
  \normo{\v}^2\normo{\w}^2 - (\v\cdot \w)^2\\
  &= %\qquad\qquad=\, 
  \normo{\v}^2\normo{\w}^2 ( 1 - \cos^2\gamma)\notag\\
  &= %\qquad\qquad=\, 
\normo{\v}^2\normo{\w}^2 \sin^2\gamma.\notag
\end{align}
The second assertion of the lemma follows by arithmetic directly from
the definitions of the dot and cross products.  \swallowed\end{proved}

\begin{lemma}[]\guid{GZPIJUR}\label{lemma:cross-collinear}
For any $\v,\w\in\ring{R}^3$,
the set $\{\orz,\v,\w\}$ is collinear if and only if $\v\times\w=\orz$.
\end{lemma}

\begin{proof}
By Equation~\eqref{eqn:cross-dot},
\[
\v\times\w = \orz \quad\text{ if and only if }\quad \normo{\v}\normo{\w} = | (\v\cdot \w) |.
\]
This is the case of equality in the Cauchy-Schwartz inequality, which is given as
%
%
%By Lemma~\ref{lemma:los-cross}, $\v\times\w=\orz$ if and only if $\v=\orz$, $\w=\orz$,
%or $\arc_V(\orz,\{\v,\w\})$ is $0$ or $\pi$.  By the definition of $\arc_V$, it equals $0$ or $\pi$
%exactly in the case of equality of the Cauchy-Schwartz inequality:
\[
\normo{\v} \,\w = \pm \normo{\w}\,\v.
\]
This is equivalent to the collinearity of $\{\orz,\v,\w\}$.
\end{proof}

\begin{lemma}[]\guid{BKMUSOX}\label{lemma:cross-id}
\[ 
\u\times \v = -\v\times \u,\quad
(\u\times \v)\cdot \w = (\v\times \w)\cdot \u,\quad
(\u\times  \v)\times \w = (\u\cdot \w)\, \v - (\v\cdot\w) \,\u.
\] 
\end{lemma}

\begin{proved}
These are arithmetic consequences of the definition of cross product.
\swallowed\end{proved}



\subsection{dihedral angle}

A dihedral angle of a tetrahedron is the angle formed between two of
its faces. In general, the dihedral angle refers to the angle formed
by two half-planes delimited by a common line.  The dihedral angle is
determined by a pair $\{\v_0,\v_1\}$ of points on the delimiting line
and another pair $\{\v_2,\v_3\}$ of two points on the respective
half-planes.  \indy{Index}{angle!dihedral}%
\indy{Index}{tetrahedron}%
\indy{Notation}{dihv@$\dih_V$}%
\indy{Index}{vector!projection}%
\indy{Index}{orthogonality} %

\begin{definition}[dihedral angle]\guid{YMHELNF}\label{def:dih}
\formaldef{$\dih_V$}{dihV}
 When $\v_0\ne \v_1$,
  write $\dih_V(\{\v_0,\v_1\},\{\v_2,\v_3\})$ for the angle
  $\gamma\in[0,\pi]$ formed by
\[ 
\bar \w_2 = (\w_1\cdot \w_1) \w_2 - (\w_1\cdot \w_2) \w_1\textand  \bar \w_3 =
(\w_1\cdot \w_1) \w_3 - (\w_1\cdot \w_3) \w_1,
\] 
where $\w_i=\v_i-\v_0$.  We call it
the dihedral angle formed by $\v_2$ and $\v_3$ along $\{\v_0,\v_1\}$.
\indy{Notation}{dih}%
\indy{Index}{angle!dihedral}%
\end{definition}
The subscript $V$ is a reminder 
that the dihedral angle takes vector arguments.
Later, a second version, without the subscript, 
computes the angle as a function of the lengths of edges of a 
tetrahedron.
\indy{Index}{edge!length}%
\indy{Notation}{V@$V$ (subscript marking vector functions)}%
As the notation suggests, the dihedral angle depends only
on the unordered pairs $\{\v_0,\v_1\}$, $\{\v_2,\v_3\}$.

\figGJRSLPT % fig:dih


The dihedral angle can be interpreted as the planar angle between two rays, obtained by
projection of the two half-planes to a plane orthogonal to both of
them.  Up to positive scalars, $\bar \w_2$ and $\bar \w_3$ are the
projections of $\w_2$ and $\w_3$ to the plane through the origin
orthogonal to the vector $\w_1$.  The dihedral angle is the angle
between the projections $\bar \w_2$ and $\bar \w_3$ at $\orz$.

\begin{remark}\label{rem:dih}\guid{LHYCNII}
  The dihedral angle is unchanged if $\w_1$ is replaced with $t \w_1$ with
  $t\ne0$. The dihedral angle is unchanged if $\w_2$ is replaced with
  $t_2 \w_2 + t_1 \w_1$ with $0 < t_2$ and $t_1$ arbitrary because
  such points project along the same ray.  It is unchanged if $\w_3$ is
  replaced with $t_3 \w_3 + t_1 \w_1$ with $0 < t_3$ and $t_1$
  arbitrary, because such points project along the same ray.  In
  particular, the dihedral angle formed by $\w_2$ and $\w_3$ along
  $\{\orz,\w_1\}$ is the same as that formed by $\w_2/\normo{\w_2}$ and
  $\w_3/\normo{\w_3}$ along $\w_1/\normo{\w_1}$.
\end{remark}

The dihedral angle is degenerate and is not be used when $\w_1 =
\orz$, $\bar \w_2 = \orz$, or $\bar \w_3 = \orz$.  Equivalently, degeneracy
occurs when $\{\v_0,\v_1,\v_2\}$ or $\{\v_0,\v_1,\v_3\}$ is a collinear set.

\begin{lemma}\guid{HVIHVEC}\label{lemma:dih-cross}
  Let $\v_0,\ldots,\v_3\in\ring{R}^3$ be given with $\v_0\ne \v_1$.
  Then $\dih_V(\{\v_0,\v_1\},\{\v_2,\v_3\})$ is the angle formed by
  $\w_1\times \w_2$ and $\w_1\times \w_3$, where $\w_i= \v_i-\v_0$.
\end{lemma}

\begin{proof}  For any  $\u,\v,~\w\in\ring{R}^3$ with $\u\cdot\w=\v\cdot\w=0$, 
we use Lemma~\ref{lemma:cross-id}
to compute
\begin{align*}
  (\u\times \w)\cdot (\v\times \w) &= - (\u\times \w)\cdot (\w\times \v) \\
  &= -\v\cdot ((\u\times \w)\times \w) \\
  &= -\v\cdot (-(\w\cdot \w) \u)\\
  &= \,\,(\u\cdot \v) (\w\cdot \w).
\end{align*}
That is, $\wild\times \w$ preserves dot products, up to a scalar
$(\w\cdot\w)$.  Thus, if $\w\ne\orz$, the angle formed by $\u$ and
$\v$ is equal to the angle formed by $\u\times \w$ and $\v \times \w$.

The dihedral angle is the angle formed by
\begin{align*}
  \bar \w_2 & = (\w_1\cdot \w_1) \w_2 - (\w_1\cdot \w_2) \w_1 = (\w_1\times \w_2)\times \w_1\\
  \bar \w_3 & = (\w_1\cdot \w_1) \w_3 - (\w_1\cdot \w_3) \w_1 = (\w_1\times \w_3)\times \w_1
\end{align*}
Let $\u=\w_1\times \w_2$, $\v=\w_1\times \w_3$, and $\w=\w_1$.  The
preceding calculation shows that the angle formed by $\bar \w_2 =
\u\times\w$ and $\bar \w_3 = \v\times \w$ is equal to the angle formed
by $\u$ and $\v$.  The lemma ensues.
\end{proof}

\begin{lemma}[spherical law of cosines]\guid{RLXWSTK}\label{lemma:sloc}
   \formalauthor{Nguyen Quang Truong} Let $\gamma$ be the
  dihedral angle formed by $\v_2$ and $\v_3$ along $\{\v_0,\v_1\}$.  Let
  $a$, $b$, and $c$ be the angle at $\v_0$ between $\v_3$ and $\v_1$, $\v_2$
  and $\v_1$, and $\v_2$ and $\v_3$, respectively. %Assume $\v_1\ne \v_0$.
  Assume that $\{\v_0,\v_1,\v_2\}$ and $\{\v_0,\v_1,\v_3\}$ are not collinear.
  Then
  \[ \cos\gamma = \frac{\cos c - \cos a \cos b}{\sin
      a\sin b}.\] 
\end{lemma}
\indy{Index}{cosine!spherical law of cosines}%
\indy{Index}{spherical!law of cosines}%
\indy{Index}{trigonometry!spherical}%

\begin{remark}
  The spherical law of cosines is the most fundamental identity of
  spherical trigonometry.  A \fullterm{spherical
    triangle}{spherical!triangle} is a figure formed by three points
  on a unit sphere, together with three minimal geodesic arcs on the
  sphere that connect each pair of points.  In the lemma, $a$, $b$,
  and $c$ are the arclengths of the sides of a spherical triangle with
  vertices $\v_2/\normo{\v_2}$, $\v_3/\normo{\v_3}$, and
  $\v_1/\normo{\v_1}$, when $\v_0=\orz$. See Figure~\ref{fig:sloc}.
  Also, $\gamma$ measures the angle of the spherical triangle opposite
  the side $c$.  
  \indy{Index}{triangle!spherical}%
\end{remark}

\figNUPFYMD % spherical law of cosines.


\begin{proof} The proof is an exercise based on previously established
  trigonometric identities.  Let $\w_i = \v_i-\v_0$.  An earlier
  remark states that the dihedral angle is
  unchanged if $\w_2$, $\w_3$, and $\w_1$ are replaced by
  $\w_2/\normo{\w_2}$, $\w_3/\normo{\w_3}$, $\w_1/\normo{\w_1}$,
  respectively.  Hence, we may assume without loss of generality that
  $\normo{\w_2}=\normo{\w_3}=\normo{\w_1}=1$.

Let $\bar \w_2$ and $\bar \w_3$ be the vectors in Definition~\ref{def:dih}.
The law of cosines gives
\[ \cos\gamma = \frac{\bar \w_2\cdot \bar \w_3}
{\normo{\bar \w_2}\,\normo{\bar \w_3}}.
\] 
The unit normalizations of $\w_3,\w_2,\w_1$ give
\[ 
\normo{\bar \w_2}^2 = \bar \w_2\cdot \bar \w_2 =
(\w_2 - (\w_1\cdot \w_2)\w_1)\cdot (\w_2 - (\w_1\cdot \w_2) \w_1) =
1 - (\w_1\cdot \w_2)^2 = \sin^2 b.
\] 
So $\normo{\bar \w_2} =\sin b$. Similarly, $\normo{\bar \w_3} = \sin a$.
These calculations give the denominator in the spherical law of cosines.  An
expansion of the dot product gives the numerator:
\begin{align*}
\bar \w_2\cdot \bar \w_3 &=
 (\w_2 - (\w_1\cdot \w_2) \w_1)\cdot (\w_3 - (\w_1\cdot \w_3) \w_1)\\
&= (\w_2\cdot \w_3) - (\w_1\cdot \w_2) (\w_1\cdot \w_3) \\
&= \cos c - \cos a \cos b.
\end{align*}
The identity ensues.
\end{proof}

The spherical law of cosines gives the angles of a spherical triangle
as a function of its sides.  In spherical geometry, a
duality%
\footnote{In three-dimensional Euclidean space, the orthogonal
  complement of a plane through the origin is a line through the
  origin, giving a duality between planes and lines through the
  origin.  The intersection of each plane and line with a unit sphere
  at the origin yields a duality between great circles and antipodal
  pairs of points (the poles of the great circle).  The three edges of
  a spherical triangle $ABC$ lie on three great circles that
  determine three antipodal pairs of points.  From each of the three
  pairs, a coherent choice can be made between the two poles (with the preferred pole closer to the opposite vertex of $ABC$).  These
  three poles are the vertices of the polar triangle $A'B'C'$.  Each
  statement about the triangle $ABC$ can be dualized to a statement
  about $A'B'C'$.  In particular, the edges $a,b,c$ and angles
  $\alpha,\beta,\gamma$ of $ABC$ are related to those $a',b',\ldots$
  of $A'B'C'$ by
\[ 
a + \alpha' = \pi,\quad a' + \alpha= \pi,
\] 
and so forth.}%
\indy{Notation}{A@$ABC$ (triangle)}%
\indy{Index}{polar!triangle}%
exists between angles and sides of a triangle.  As a result, formulas
in spherical trigonometry tend to come in pairs.  The spherical law of
cosines gives the angle of a spherical triangle as a function of its
edge lengths.  The polar form of the formula gives the edge length of
a spherical triangle as a function of its angles.  Up to signs, the
polar formula has the same form as the law of cosines.
\indy{Index}{great circle}%

\begin{lemma}[spherical law of cosines - polar form]\guid{NLVWBBW}
\label{lemma:sloc2} \formalauthor{Nguyen Quang Truong} Consider 
  $\{\v_0,\v_1,\v_2,\v_3,\}\subset\ring{R}^3$.  Let
  $\alpha,\beta,\gamma$ be the dihedral angles:
\begin{align*}
\alpha &= \dih_V(\{\v_0,\v_2\},\{\v_3,\v_1\})\\
\beta &= \dih_V(\{\v_0,\v_3\},\{\v_2,\v_1\})\\
\gamma&= \dih_V(\{\v_0,\v_1\},\{\v_3,\v_2\})\\
\end{align*}
Let $c$ be the
angle between $\v_2$ and $\v_3$ at $\v_0$. 
Assume that $\{\v_0,\v_2,\v_1\}$, $\{\v_0,\v_2,\v_3\}$, and $\{\v_0,\v_3,\v_1\}$ are not collinear.
Then
\[ 
\cos c = \frac{\cos \gamma + \cos \alpha \cos \beta}
{\sin \alpha\sin \beta}.
\] 
\end{lemma}
\indy{Index}{cosine!spherical law of cosines}%
\indy{Index}{spherical!law of cosines}%

\begin{proof}  
  What follows is a direct computational proof that avoids polarity and
is an
  application of established trigonometric identities.  Let $a$ be the
  angle between $\v_1$ and $\v_3$, and let $b$ be the angle between $\v_1$
  and $\v_2$ at $\v_0$.  Let $A=\cos a$, $B=\cos b$, $C=\cos c$,
  $A'=\sin a$, $B'=\sin b$, $C'=\sin c$.  The spherical
  law of cosines gives
\[ \sin^2\beta = 1-\left(\frac{B-A C}{A' C'}\right)^2
  = \frac{p}{A'^2 C'^2},\] 
where $p=1-A^2 - B^2 - C^2 + 2 A B C$.
In particular, $p\ge 0$.
\indy{Notation}{p@$p$ (trigonometric expression)}%
A computation of $\sin^2\alpha$ and the remaining terms in the same way gives
\begin{align*}
  \sin\alpha\sin\beta &= \frac{\displaystyle p}{\displaystyle A' B' C'^2}\\ 
  \\
  \cos\gamma + \cos\alpha \cos\beta &=
  \frac{\displaystyle C - A B}{\displaystyle A' B'} + 
\frac{\displaystyle A - B C}{\displaystyle B' C'} 
\frac{\displaystyle B - A C}{\displaystyle A' C'}
  = \frac{\displaystyle p C}{\displaystyle A' B' C'^2}.
% C &= \frac{\cos\gamma + \cos\alpha \cos\beta}{\sin\alpha \sin\beta}\\
\end{align*}
The result follows by real arithmetic.
\end{proof}

The following lemma gives a formula for the dihedral angle
of a tetrahedron along an edge in terms of its edge lengths.  The
familiar polynomials $\ups$ and $\Delta$ appear once again.
\indy{Notation}{zzu@$\ups$ (polynomial)}%
\indy{Notation}{zzD@$\Delta$}%


\begin{lemma}[]\guid{OJEKOJF}\label{lemma:dihform}
\formalauthor{Nguyen Quang Truong}
Let $\v_0,\v_1,\v_2,\v_3$ 
be vectors with $\{\v_0,\v_1,\v_2\}$ not collinear, 
and $\{\v_0,\v_1,\v_3\}$ not
collinear. 
Let $\gamma$ be the dihedral angle formed
by $\v_2$ and $\v_3$ along $\{\v_0,\v_1\}$. Let
\[ (x_1,\ldots,x_6) = 
(x_{01},x_{02},x_{03},x_{23},x_{13},x_{12}),
\text{ where } x_{ij}=\norm{\v_i}{\v_j}^2.\] 
Let $\Delta_4$ be the partial derivative of $\Delta(x_1,\ldots,x_6)$ with
respect to $x_4$.
The dihedral angle $\gamma=\dih_V(\{\v_0,\v_1\},\{\v_2,\v_3\})$
is given by
\[ 
\gamma=\arccos(\frac{\Delta_4(x_1,\ldots,x_6)}{\sqrt{
\ups(x_1,x_2,x_6)\ups(x_1,x_3,x_5)}}).
\] 
%Assuming that $\gamma\ne 0,\pi$, 
It is also given by
\[ 
\gamma=\frac{\pi}{2} - \atn
({\sqrt{4 x_1 \Delta(x_1,\ldots,x_6)}},{\Delta_4(x_1,\ldots,x_6)}).
\] 
\end{lemma}
%% pi/ 2. -  arctan(  deltax4/ (sqrt (4. * x1 * delta)))
\indy{Index}{angle!dihedral}%

\indy{Notation}{zzb@$\beta$ (angle)}%
\begin{proof}  We use the notation $\w_i, \bar \w_i$ established in Definition~\ref{def:dih}.
  Let $\beta = \arc_V(\v_0,\{\v_1,\v_2\})$.  The assumptions give
  $\bar \w_2\ne \orz$ and $\bar \w_3 \ne \orz$.  
  By expanding definitions and dot products and by the
  law of sines,
\[ 
  \bar \w_2\cdot \bar \w_2 = (\w_1\cdot \w_1) ((\w_1\cdot \w_1)(\w_2\cdot \w_2) -
  (\w_1\cdot \w_2)^2) =  x_1^2 x_2 \sin^2 \beta = \frac{1}{4}
  x_1
  \ups(x_1,x_2,x_6).
\] 
Similarly,
\[ \bar \w_3 \cdot \bar \w_3 = \frac{1}{4} x_1
  \ups(x_1,x_3,x_5)\] 
%Let $y_i = \sqrt{x_i}$. 
and by dot product formula \eqref{eqn:dot-law},
\begin{align*}
    \bar \w_2\cdot \bar \w_3 &= (\w_1\cdot \w_1)((\w_1\cdot \w_1)(\w_2\cdot \w_3) -
    (\w_1\cdot \w_2)(\w_1\cdot \w_3) ) \vspace{6pt} \\  
    &= x_1 \left(\frac{\displaystyle x_1 (x_2 + x_3 -
        x_1)}{2} - \frac{\displaystyle (x_1 + x_2 - x_6)(x_1 + x_3 -
        x_5)}{4} \right)\vspace{6pt}\\
%= \frac{x_1}{4} (2 x_1 (x_2+x_3-x_4) -
%(x_1+x_2-x_6)(x_1+x_3-x_5)) \vspace{6pt}\\
&= {x_1\Delta_4(x_1,\ldots,x_6)}/{4}.
\end{align*}
The result follows in terms of $\arccos$.

The translation to $\atn$ uses the $\arccos$-$\atn$
identity (Lemma~\ref{lemma:arccos-arctan}) and the following polynomial
identity
\[ 
  % \frac{16}{x_1^2}(\normo{\bar \w_2}^2 \normo{\bar \w_3}^2 - (\bar
  % \w_2\cdot \bar \w_3)^2) =
\ups(x_1,x_2,x_6)\ups(x_1,x_3,x_5) - \Delta_4(x_1,\ldots,x_6)^2
= 4 x_1 \Delta(x_1,\ldots,x_6).
\] 
\end{proof}

\subsection{Euler triangle}

The expression $\alpha_1+\alpha_2+\alpha_3-\pi$ is \newterm{Girard's
  formula} (known first to Harriot) for the area of a spherical
triangle with angles $\alpha_1$, $\alpha_2$, $\alpha_3$.  We return to
this formula in the next chapter~\eqref{eqn:girard}, when area and
volume are treated.  Although the statement and proof do
not explicitly mention area, the following lemma can be interpreted
as an alternative formula discovered by Euler and Lagrange for the
area of a spherical triangle.
% \indy{Index}{Girard, A.}%
% Albert Girard's Book on trigonometry was published in 1626. Harriot
% lived 1560 - 1621.
\indy{Index}{Girard's formula}%
\indy{Index}{triangle!spherical}%
\indy{Notation}{zza@$\alpha$ (angle)}%
\indy{Index}{Harriot, T.}%

\begin{lemma}[Euler triangle]\guid{JLPSDHF}\label{lemma:euler} %was 600
Let $\v_0,\v_1,\v_2,\v_3$ be points in $\ring{R}^3$. 
Let 
\[ (y_1,\ldots,y_6)
  =(y_{01},y_{02},y_{03},y_{23},y_{13},y_{12}), \text{ where }
  y_{ij}=\norm{\v_i}{\v_j}.\]  Set $x_i = y_i^2$.  and
\[ 
p = y_1 y_2 y_3 + y_1 (\w_2\cdot \w_3) + y_2 (\w_1\cdot \w_3) + y_3
(\w_1\cdot \w_2).
\] 
\indy{Notation}{p@$p$ (Euler solid angle numerator)}%
where $\w_i = \v_i- \v_0$.  Let \[ \alpha_i
  =\dih_V(\{\v_0,\v_i\},\{\v_j,\v_k\})\] 
where $\{i,j,k\}=\{1,2,3\}$.
Assume that $\Delta(x_1,\ldots,x_6)>0$. 
Then
\[ 
\alpha_1+\alpha_2+\alpha_3 - \pi
= {\pi} - 2\,\atn({\Delta(x_1,\ldots,x_6)^{1/2}},{2 p}).
\] 
\end{lemma}
\indy{Index}{triangle!Euler}%


Before we jump into the details of the proof, it helps to understand
why a formula of this general form should exist.  
Each angle $\alpha_i$ equals a single arctangent (Lemma~\ref{lemma:dihform}).
The addition law for arctangent, which is obtained by inverting the additional law for
the tangent (Lemma~\ref{lemma:tan-add}),
rewrites the sum $\alpha_1+\alpha_2+\alpha_3$
of arctangents as a single arctangent, or as twice a single arctangent if
the double angle formula is invoked.  Euler's formula is a precise formula for the
sum of arctangents in the form $2\atn(\cdots)$.

In practice, it is easier to carry out the details of the proof by a
slightly different strategy.  We can check that the derivatives of the
two sides of the identity are equal as rational functions.  The domain
is connected, and from this it follows that the two sides differ by at
most a constant.  By calculating a particular test value, we see that
the two sides are precisely equal.



\begin{proof}
%% I checked all the details of this proof in 
%% Math'ca on May12,2007
  This proof is an exercise in real analysis
  and established trigonometric identities.  According to an earlier
  remark, the dihedral angles are unchanged if the
  vectors $\w_i$ are rescaled so that $\normo{\w_i}=1$.  By
  inspection, the given formula is also unchanged under rescalings:
  the factor $a$ is homogeneous of degree $3$ under a change $\w_i
  \mapsto t \w_i$ for $t>0$, and so is $\sqrt{\Delta}$ by the formula
  for $\Delta$.  Thus, without loss of generality, $\normo{\w_i}=1$ for $i=1,2,3$.  Consequently, $y_1=y_2=y_3=1$.  It is convenient to
  use different notation $a=x_4$, $b=x_5$, $c=x_6$ for the other
  variables. The expansion of the dot products in $p$ by the dot
  product law gives
\[ 2 p = 8 - (a+b+c).\] 
Also, the definitions of $\Delta$ and $\ups$ give
\[ \Delta(x_1,\ldots,x_6) = \Delta(1,1,1,a,b,c) =
\ups(a,b,c) - a b c.\] 
Since $\Delta>0$ by assumption, the arctangent formula
in Lemma~\ref{lemma:dihform} 
applies for the dihedral angles $\alpha_i$.  After
this substitution (and clearing a factor of $3$),  %and clearing the
                                                   %$3$ from the
                                                   %denominator,
the desired identity takes the form $f(a,b,c)=0$, where
\[ 
f(a,b,c)= -\pi/2 - \sum_{i=1}^3\arctan(u_i/\sqrt{\Delta}) +
2\arctan(2 p/\sqrt{\Delta}),
\] 
for some rational functions $u_i$ of $a,b,c$.  The aim is to prove
this trig identity holds whenever $\Delta>0$.

To see that the function $f$ does not depend on $a$, 
we fix $(b,c)$ and differentiate $f$ with respect to $a$.  The partial
derivative $\partial f/\partial a$ has the form
$g(a,b,c)/\sqrt{\Delta}$ for some rational function $g$ of $a,b,c$.
The denominator of $g$ has no real zero.  Algebraic simplification of
this rational function shows that the polynomial numerator of
$g(a,b,c)$ is identically $0$.  (Euler himself did not shun brute
force~\cite{Euler}.)

By real analysis, the derivative of $f$ is zero, and the function $f$
is constant along any segment in $\ring{R}^3$ along which $\Delta$ is
positive.  The remaining part of the proof constructs two segments
along which $\Delta$ is positive.
%\footnote{In the formal proof, Vu
%  Khac Ky uses three segments: the first segment runs from $(a,b,c)$
%  to $(b+c-b c/2,b,c)$, the second continues to $(2c - c^2/2,c,c)$,
%  and the final segment terminates at $(2,2,2)$.}   
The
first connects $f(a,b,c)$ to $f(a,2,2)$, provided the variables are
ordered appropriately.  The second connects $(a,2,2)$ to $(2,2,2)$.
From this construction it follows that $f(a,b,c)=f(2,2,2)$.  The last
step is to evaluate the constant $f(2,2,2)$.  Arithmetic gives
$\Delta=4$, $2p= 2$, $u_1=u_2=u_3 =0$, when $a=b=c=2$.  Finally,
\[ f(a,b,c)= f(2,2,2) = -\pi/2 + 2\arctan(1)
  =0.\] 


Let us return to the construction of the two segments.  By the
triangle inequality, $a =\norm{\v_2}{\v_3}^2 \le
(\norm{\v_2}{\v_0}+\norm{\v_3}{\v_0})^2 = 4$.  If equality holds, then
$\{\v_0,\v_2,\v_3\}$ is collinear and $\{\v_0,\ldots,\v_4\}$ is
coplanar.  From this it follows that $\Delta=0$, which is contrary to
assumption.  Similarly, $a=0$ implies that $\Delta=0$.  Hence $0<a<4$.
Similarly, $0<b<4$ and $0<c<4$.  By the  {\it pigeonhole}
principle, two of the real numbers $a,b,c$ must lie in the same
subinterval $[0,2]$ or $[2,4]$.  To fix notation, assume that $b$ and
$c$ lie in the same subinterval.

\claim{The polynomial $\Delta$ is positive\footnote{This paragraph follows the book's
    general convention of typesetting claims in italic.} 
  on the linear segment from $(a,b,c)$ to
  $(a,2,2)$.}  Indeed, for $0\le t \le 1$, Tarski
arithmetic gives
\begin{align*}
\Delta(1,1,1,a, &\,b(1-t)+2t,c(1-t)+2t)  \\
&= \Delta(1,1,1,a,b,c) + 
t (2-t) (a (b-2)(c-2) + (b-c)^2)\\
&\ge \Delta(1,1,1,a,b,c)\\
&> 0.
\end{align*}

\claim{The polynomial $\Delta$ is positive on the linear segment from $(a,2,2)$ to
  $(2,2,2)$.}  Indeed,
\[ \Delta(1,1,1,a,2,2) = a(4-a)>0.\]   
The rest of the proof has been sketched above.
\end{proof}
\indy{Notation}{zzD@$\Delta$}%






%\subsection{Lexell's theorem}
%
%\begin{lemma}[Lexell]\guid{UWIPRDV}
%% was 1000 with old proof including lemma ZHH
% Fix two points $\v_1,\v_2$ on a unit sphere that are not antipodal.
% Let $\u,\u'$ be two other points the sphere in the same open
% hemisphere determined by the great circle through $\v_1,\v_2$.  Then
% the two spherical triangles $\{\v_1,\v_2,\u\}$ and
% $\{\v_1,\v_2,\u'\}$ have the same area if and only if the four
% points $\u$, $\u'$, $\v^*_1$, $\v^*_2$ are concircular, where
% $\v^*_i$ is the point antipodal to $\v_i$.
%\end{lemma}
%\indy{Index}{Lexell's Theorem}%
%
%
%
%\begin{proof} By the polarity of triangles mentioned above, it is
%  enough to prove the polar statement.  By Girard's formula, fixing
%  the area fixes the sum of the angles.  The polar triangle has fixed
%  perimeter.  By polarity, Lexell's theorem is a consequence of the
%  following lemma.
%\end{proof}
%\indy{Index}{Girard's formula}%
%
%\begin{lemma}[]\guid{ZHHSGTF} Fix one point $\v$ on the unit
%  sphere, with antipodal point $\v^*$.  Consider two great
%  half-circles $D_i$, $i=1,2$ between $\v$ and $\v^*$ that are not
%  coplanar.  Two great circles $A$ and $B$ cut equi-perimeter
%  triangles with vertex $\v$ along $D_i$ if and only if the great
%  circles $D_i$, $A$, and $B$ are tangent to a common circle $C$.
%\end{lemma}
%\indy{Index}{great circle}%
%
%\begin{proof} The two tangents to a circle through a given point have
%  the same length.  If $C$ exists, then this fact implies that a
%  great circle $A$ that is tangent to $C$ cuts a triangle with vertex
%  $\v$ along $D_i$ with a perimeter that is equal to the sum of the
%  distances from $\v$ to the two points of tangency $C\cap D_i$.
%  This is independent of $A$. 
%
%Conversely, for $A$ any great circle there is a unique $C$ that
% inscribes the great circles $D_i$, and $A$.  The perimeter of the
% triangle is the sum of the distances from $\v$ to the points $C\cap
% D_i$.  If a second $A$ gives a triangle with the same perimeter, its
% circle $C'$ must satisfy $C'\cap D_i = C\cap D_i$.  This forces
% $C=C'$.
%\end{proof}
%

\section{Coordinates}

This section establishes the existence and basic properties of the
standard coordinate systems: polar coordinates, spherical coordinates,
and cylindrical coordinates.  
%
\indy{Index}{azimuth}%
\indy{Index}{angle!azimuth}%
\indy{Index}{azimuth cycle}%

\subsection{azimuth angle}

\label{sec:polar}
\indy{Index}{polar!coordinates}%
\indy{Index}{coordinate systems!polar coordinates}%


For every pair of real numbers $x$ and $y$,  there are real numbers
$r$ and $\theta$ such that
\begin{equation}\label{eqn:polar}
x = r\cos\theta,\quad y = r\sin\theta.
\end{equation}
\mar{\guid{FEVNANL} Eq.~\ref{eqn:polar}} If $x$ and $y$ are both zero,
then take $r=0$, and~\eqref{eqn:polar} holds for all choices of
$\theta$. If $x$ and $y$ are not both zero, then take $0<r$, and
$\theta$ is uniquely determined (up to multiples of $2\pi$).  By
convention, we take $0\le\theta < 2\pi$.  
\indy{Notation}{r@$r$ (polar, cylindrical, and spherical radius)}%
%\indy{Notation}{zzh@$\theta$ (polar, cylindrical, and spherical coordinate)}%
\indy{Notation}{zzh@$\theta$ (polar, cylindrical, and spherical angle)}%




%%  e3 is defined in terms of v1. The indexing confuses.  % fixed 3/21/2010


\begin{definition}[frame,~positive,~adapted]\guid{AXBTGQX}
\formaldef{frame}{orthonormal}
A tuple $(\e_1,\e_2,\e_3)$ of vectors in $\ring{R}^3$ is a 
\newterm{frame} if $\e_i\cdot \e_j$ and $\normo{\e_i}=1$ 
for all $i$ and $j$.
A tuple $(\e_1,\e_2,\e_3)$ is positive if $(\e_1\times \e_2)\cdot\e_3=1$.
% Let $\{\v_0,\v_1,\v_2\}\subset\ring{R}^3$ be a set that is not
% collinear.
A tuple $(\e_1,\e_2,\e_3)$ is \newterm{adapted} to $(\v_0,\v_1,\v_2)$ if
$\e_1 = (\v_1-\v_0)/\norm{\v_0}{\v_1}$ and
$\e_2\in\op{aff}_+^0(\{\v_0,\v_1\},\v_2)$.
% such that $\normo{\e_1}=1$ and $\e_1\cdot\e_3=0$; $\e_2 = \e_3\times
% \e_1$.  The tuple $E=(\e_1,\e_2,\e_3)$ is \newterm{adapted} to
% $(\v_0,\v_1,\v_2)$.
\end{definition}
\indy{Index}{frame}%
\indy{Index}{adapted}%
\indy{Notation}{E4@$E$ (frame)}%

\begin{lemma}[orthonormalization]\guid{QAUQIEC}
\label{lemma:frame}
  Assume that $\{\v_0,\v_1,\v_2\}\subset\ring{R}^3$ is not collinear.
  Then the unique positive frame adapted to 
  $\{\v_0,\v_1,\v_2\}$ is $(\e_1,\e_2,\e_3)$, where
\begin{align*}
\e_1 &= \w_1/\normo{\w_1},\\
\e_2 &= \bar{\w_2}/\normo{\bar{\w}_2},\quad \bar{\w}_2 = \w_2 - (\e_1\cdot \w_2) \e_1,\\
\e_3 &= \e_1 \times \e_2,
\end{align*}
and where $\w_i = \v_i - \v_0$.
\end{lemma}

\begin{proof} It follows by basic vector arithmetic that
  $(\e_1,\e_2,\e_3)$ is a positive frame adapted to
  $\{\v_0,\v_1,\v_2\}$.  The choices of vectors $\e_1$ and $\e_2$ are
  dictated by the definition of adapted frame.  The choice of $\e_3$
  is dictated by the definition of positive frame.
\end{proof}

\begin{lemma}[cylindrical coordinates]\guid{EYFCXPP}
\formalauthor{Nguyen Quang Truong}
Let $\v_0$ and $\v_1$ be distinct points in 
$\ring{R}^3$.  Let $(\e_1,\e_2,\e_3)$ be a positive frame 
where $\e_1 = (\v_1-\v_0)/\norm{\v_1}{\v_0}$.
Then every
$\p\in\ring{R}^3$ that is not in the line $\op{aff}(\v_0,\v_1)$
can be uniquely expressed in the form
\[ 
\p = \v_0 + r\cos\psi\, \e_2 + r\sin\psi\, \e_3 + h (\v_1-\v_0),
\] 
\indy{Notation}{h@$h$ (cylindrical coordinate)}%
\indy{Notation}{ez@$\e_i$ (orthonormal vectors)}%
for some $0< r$, $0\le \psi < 2\pi$, $h\in\ring{R}$.
Furthermore,
assume that $\p_1$ and $\p_2$ do
not lie in the line $\op{aff}(\v_0,\v_1)$.
Then there exist unique $\psi,\theta,r_1,r_2,h_1,h_2$
such
that $0\le\psi<2\pi$, $0\le\theta < 2\pi$, $0 < r_1$, $0 < r_2$, and
\begin{align*}
\p_1 &= \v_0 + r_1\cos\psi\, \e_2 + r_1\sin\psi\, \e_3 + h_1(\v_1-\v_0),\\
\p_2 &= \v_0 + r_2\cos(\psi+\theta)\, \e_2 + r_2\sin(\psi+\theta)\, \e_3 
+ h_2(\v_1-\v_0).
\end{align*}
Finally, the angle $\theta$ is independent of the choice of $\e_2,\e_3$
giving the positive frame.
\end{lemma}
\indy{Index}{coordinate systems!cylindrical coordinates}%
\indy{Index}{cylindrical coordinates}%
\indy{Notation}{zzv@$\psi$}%
%\indy{Notation}{zzh@$\theta$ (coordinate)}%
\indy{Notation}{zzh@$\theta$ (polar, cylindrical, and spherical angle)}%
%\indy{Notation}{r@$r$ (coordinate)}%
\indy{Notation}{r@$r$ (polar, cylindrical, and spherical radius)}%
\indy{Notation}{h@$h$ (cylindrical coordinate)}%
%
The degenerate point $\p\in\op{aff}\{\v_0,\v_1\}$ is excluded from the
lemma.  Nevertheless, it too has a cylindrical coordinate
representation of the form $\p = \v_0 + h(\v_1-\v_0)$ (with $r=0$).
Only uniqueness fails, because every $\theta$ gives the same
representation.

\begin{remark}
The reader should carefully note the indexing of the vectors in the
orthonormal frame as it appears in the cylindrical coordinate system.
This book breaks with tradition by making $h$ the coefficient of the
frame vector $\e_1$ (rather than $\e_3$) and makes a corresponding
change in spherical coordinates.  This nontraditional order is better
suited to the definition of dihedral angle, the arguments of which 
 are grouped in pairs $\dih_V(\{\v_0,\v_1\},\{\v_2,\v_3\})$
to emphasize the symmetries
$\v_0\leftrightarrow\v_1$ and $\v_2\leftrightarrow\v_3$.  Under this pairing
of arguments, the axis of the dihedral angle is the line
$\op{aff}\{\v_0,\v_1\}$, which gives the direction $\v_1-\v_0$ of the cylinder.
%
\indy{Index}{coordinate systems}%
\end{remark}

\begin{definition}[azim]\guid{UJBHGUX}\label{def:azim}
\formaldef{$\op{azim}$}{azim}
  Define $\op{azim}(\v_0,\v_1,\v_2,\v_3)$, the \newterm{azimuth} angle
  (or \newterm{longitude}), to be the uniquely determined angle
  $\theta$  given by the previous lemma for the points $\p_1=\v_2$ and $\p_2=\v_3$.
  By convention, let the azimuth angle be $0$ in the degenerate cases
  where $\{\v_0,\v_1,\v_2\}$ or $\{\v_0,\v_1,\v_3\}$ is collinear.
  \indy{Notation}{azim}%
  \indy{Index}{azimuth}%
  \indy{Index}{angle!azimuth}%
\end{definition}
\indy{Notation}{azim}%

%The azimuth angle is a polar coordinate of the projection 
%$\p -\v_0-r\cos\phi\,\e_1 \in\op{aff}\{\e_2,\e_3\}$:
%    \[ 
%    (x,y) = (r'\cos\theta,r'\sin\theta), \quad r' = r\sin\phi.
%    \] 

The azimuth and dihedral angles are closely related~(Figure~\ref{fig:dih}).  The azimuth
angle
 takes values between $0$ and $2\pi$, but the dihedral angle is
never greater than $\pi$.  The following lemma reveals that the
azimuth angle is an oriented extension of the dihedral angle and is always
equal to $\dih$ or $2\pi - \dih$.  \indy{Index}{angle!azimuth}%
\indy{Index}{angle!dihedral}%
\indy{Notation}{dih}%
\indy{Index}{azimuth}%

\begin{lemma}[]\guid{QQZKTXU}\label{lemma:dih-azim}
    \formalauthor{Nguyen Quang Truong} Let
  $\v_1\ne \v_0$ be a nonzero vectors in $\ring{R}^3$.  Assume that
  $\v_2$ and $\v_3$ do not lie in the line $\op{aff}(\v_0,\v_1)$.  Let
\[ 
\gamma = \dih_V(\{\v_0,\v_1\},\{\v_2,\v_3\}).
\] 
Then
\[ 
\cos(\op{azim}(\v_0,\v_1,\v_2,\v_3)) = \cos\gamma.
\] 
\end{lemma}

\begin{proof} For simplicity, take $\w_i = \v_i-\v_0$.  Let
  $\bar{\w_i} = (\w_1\cdot \w_1) \w_i - (\w_1\cdot \w_i) \w_1$.  From
  the assumptions of the lemma, $\bar{\w_2}\ne 0$.  Set $\e_2 =
  \bar{\w_2}/\normo{\bar{\w_2}}$.  Choose a unit vector $\e_3$ so that
  $(\e_2\times \e_3)\cdot\w_1>0$ and $\e_2\cdot \e_3 = \w_1\cdot
  \e_3=0$.  Write $\w_i$ in cylindrical coordinates as
\begin{alignat*}{3}
\w_2 &= r_1 \e_2 &    &+h_1 \w_1\\
\w_3 &= r_2 \cos\theta\, \e_2 &+ r_2 \sin\theta\, \e_3 &+ h_2 \w_1.
\end{alignat*}
The definition of $\op{azim}$ gives
$\op{azim}(\w_0,\w_1,\w_2,\w_3)=\theta$.  By definition, $\cos\gamma$
is the angle between $\bar{\w_2}$ and $\bar{\w_3}$.  We compute
\begin{align*}
\bar{\w_2} &= \normo{\bar{\w_2}} \e_2 \\
\bar{\w_3} &= (\w_1\cdot \w_1) r_2 \cos\theta\, \e_2 
+ (\w_1\cdot \w_1) r_2 \sin\theta\, \e_3 \\
\end{align*}
The result $\cos\theta=\cos\gamma$ 
is now a result of the definition of angle 
(Definition~\ref{def:angle}).
\end{proof}
\indy{Notation}{zzc@$\gamma$ (angle)}%
%\indy{Notation}{zzh@$\theta$ (angle)}%
\indy{Notation}{zzh@$\theta$ (polar, cylindrical, and spherical angle)}%

The previous lemma identifies the cosine of the azimuth angle.  The final
lemma of this subsection determines the sign of its sine.

\begin{lemma}[]\guid{JBDNJJB}\label{lemma:sim}
\formalauthor{Nguyen Quang Truong}
% 
Write $x\sim y$ when there exists $t>0$ such that $x= t y$. 
Then 
\[ 
\sin(\op{azim}(\orz,\v_1,\v_2,\v_3))\sim (\v_1
  \times \v_2)\cdot \v_3.
\] 
\end{lemma}
\indy{Notation}{9a@$\sim$ (equal up to positive scalar)}%

\begin{proof}
  The relation $(\sim)$ is an equivalence relation.  We may assume that
  $\{\orz,\v_1,\v_2\}$ and $\{\orz,\v_1,\v_3\}$ are not collinear
  sets, because otherwise both sides are zero.  Let $(\e_1,\e_2,\e_3)$ be
  the positive frame adapted to $(\orz,\v_1,\v_2)$.
%\[ 
%\begin{align}
%   \e_1 &= \v_1/\normo{\v_1}\\
%   \v_2' &= \v_2 - (\e_1\cdot \v_2) \e_1\\
%   \e_2 &= \v_2'/\normo{\v_2'}\\
%   \e_3 &= \e_1 \times \e_2 \\
%\end{align}
%\] 
Write $\v_3= r\cos\theta\, \e_2 + r\sin\theta \, \e_3 + h\, \e_1$ in
cylindrical coordinates, where $\theta =
\op{azim}(\orz,\v_1,\v_2,\v_3)$.  Then by the explicit formulas for
the positive frame,
\begin{align*}
(\v_1\times \v_2)\cdot \v_3 &\sim (\e_1\times \v_2)\cdot \v_3\\
%&= (\e_1\times \v_2')\cdot \v_3\\
&\sim (\e_1\times \e_2)\cdot \v_3\\
&= \e_3 \cdot \v_3\\
&= r\sin\theta \\
&\sim \sin\theta.
\end{align*}
\end{proof}


\subsection{zenith angle}
\label{sec:spherical}

\indy{Index}{angle!zenith}%
\indy{Index}{zenith}%
\indy{Index}{latitude}%
\indy{Notation}{zzv@$\phi$ (zenith)}%

%
%\begin{definition}[spherical coordinates]\guid{IESAXWY}
%Let $x,y,z$ be any real numbers.  A
%triple $(r,\theta,\phi)$ such that
%    \begin{equation}
%    \label{eqn:spherical}
%    x = r\cos\theta\sin\phi,\quad y = r\sin\theta\sin\phi,\quad
%    z = r\cos\phi
%    \end{equation}
%with $0\le r$, $0\le\theta<2\pi$, and $0\le\phi\le\pi$ are called
%spherical coordinates of $(x,y,z)$. 


%\begin{definition}[azimuth]\guid{OSPVIBZ}\label{def:azimuth}


The following lemma identifies the \newterm{zenith} angle $\phi$.  Because it is
easily expressed in terms of the more basic function $\arc_V$, there
is little need to refer to it directly.
\indy{Index}{orthogonal frame}%

\begin{lemma}[zenith]\guid{QAFHJNM}
    \formalauthor{Nguyen Quang Truong} Let
  $(\v_0,\v_1)$ be an ordered pair of distinct points in $\ring{R}^3$.
  Let $\v_2\ne \v_0$.  Set $\phi =
  \arc_V(\v_0,\{\v_2,\v_1\})\in[0,\pi]$.  Let $\e_1$ be the unit
  vector $(\v_1-\v_0)/\norm{\v_1}{\v_0}$.  Let $r =
  \norm{\v_2}{\v_0}$.  Then $\v_2$ can be expressed in the form
\[ 
\v_2 = \v_0 + \bar{\v}_2 +
r\cos\phi\, \e_1,
\] 
where $\bar{\v}_2\cdot \e_1 = 0$.  The angle $\phi$ is called the
\newterm{zenith} angle (or \newterm{latitude}) of $\v_2$ along
$(\v_0,\v_1)$.  \indy{Index}{zenith}%
\indy{Index}{angle!zenith}%
\end{lemma}

\begin{proof} The lemma is a direct consequence of the definition of $\arc_V$:
\[ (\v_2-\v_0)\cdot \e_1 = r\cos\phi.\] 
\end{proof}

\begin{lemma}[spherical coordinates]\guid{XPHCPNY}\label{lemma:sph}
\formal{SPHERICAL\_COORDINATES}
  Assume that
  $\{\v_0,\v_1,\v_2\}\subset\ring{R}^3$ % and $\{\v_0,\v_1,\p\}$
  is not a collinear set.  Let $(\e_1,\e_2,\e_3)$ be the positive
  frame adapted to $(\v_0,\v_1,\v_2)$.  Then for any $\p$,
\begin{equation}
\p = \v_0 + r \cos\theta \sin\phi\, \e_2 + r \sin\theta\sin\phi\, \e_3 +
r\cos\phi\,\e_1,
\label{eqn:sph}
\end{equation}
where%
  \footnote{This book follows the variable naming conventions
    $(\theta,\phi)$ of American calculus textbooks, which reverses the
    international scientific notation.} 
\begin{align*}
r &= \norm{\v_0 }{ \p}\\
\phi &= \text{zenith angle of } \p \text{ along } (\v_0,\v_1)\\
\theta &=\op{azim}(\v_0,\v_1,\v_2,\p).
\end{align*}
\end{lemma}
%\indy{Notation}{zzh@$\theta$ (azimuth)}%
\indy{Notation}{zzh@$\theta$ (polar, cylindrical, and spherical angle)}%
\indy{Index}{coordinate systems}%
\indy{Index}{coordinate systems!spherical coordinates}%
\indy{Index}{spherical!coordinates}%
%\indy{Notation}{r@$r$ (polar, cylindrical, spherical coordinate)}%
\indy{Notation}{r@$r$ (polar, cylindrical, and spherical radius)}%

 
\begin{proof}
Cylindrical coordinates give
\[ 
\p = \v_0 + r'\cos\theta\,\e_2 + r'\sin\theta\,\e_3 + h\, \e_1,
\] 
for some $h$ and $r'=\normo{\p-\v_0-h\,\e_1}\ge0$.  The zenith angle
puts $\p$ in the form
\[ 
\p = \v_0 + r'\cos\theta\,\e_2 + r'\sin\theta\,\e_3 + r\cos\phi\, \e_1,
\] 
where
\begin{align*}
r^2 &= \norm{\p}{\v_0}^2\\ 
&= \normo{\p-\v_0-h\,\e_1}^2 + \normo{h\,\e_1}^2\\
&= (r')^2 + r^2 \cos^2\phi,
\end{align*}
Since $\sin\phi$, $r$, and $r'$ are nonnegative, it follows that $r'=r\sin\phi$, as
desired.
\end{proof}

\begin{definition}[spherical coordinates]\guid{LVDJVFD}\label{def:sph}
  \formaldef{spherical coordinates}{SPHERICAL\_COORDINATES}
  Equation~\eqref{eqn:sph} is called the spherical coordinate representation of $\p$ with
  respect to $(\v_0,\v_1,\v_2)$.
\end{definition}



% Any triple $(x,y,z)$ has spherical coordinates.  The radial
% component is $r = \sqrt{x^2+y^2+z^2}$.  In the degenerate case when
% $r=0$, Equations~\eqref{eqn:spherical} becomes independent of
% $\theta$ and $\phi$. In the degenerate case when $\phi = 0$ or $\phi
% = \pi$, the equations become independent of $\theta$. If $0<r$ and
% $\phi\ne 0,\pi$, then $\theta$ is uniquely determined by $x,y,z$. If
% $0<r$, then $\theta$ is uniquely determined.
%


%The following gives the existence of polar coordinates on any
% oriented plane in three dimensions, with a general point $\v$ on the
% plane serving as the origin.  A normal vector $n$ orients the plane,
% then polar coordinates appear as the restriction of the spherical
% coordinates $(r,\theta,\phi)$ to the plane.  The following lemma
% shows that the value of $\phi$ is fixed, so that it may be dropped
% from the notation.  \indy{Index}{polar!coordinates}%
% \indy{Notation}{n@$n$ (normal vector)}%
%
%\begin{lemma}[]\guid{YBXRVTS}\label{lemma:polar-gen}
%   \formalauthor{Nguyen Quang Truong} Let $\{\v,\w,\u\}$ be
%  a set of three points in $\ring{R}^3$ that is not collinear.  Let
%  $n = (\w-\v) \times (\u-\v)$.  Then the zenith angle of any $\p\ne
%  \v$ in the plane $\op{aff}\{\v,\w,\u\}$, computed with respect to
%  $(\v,\v+n)$, is $\pi/2$.
%\end{lemma}
%\indy{Index}{zenith}%
%\indy{Index}{angle!zenith}%
%
%\begin{definition}[polar coordinate]\guid{WNLMGUV}\label{def:polar}
%Call  the two remaining coordinates, $(r,\theta)$, 
%the polar coordinates of $\p\in\op{aff}\{\v,\w,\u\}$ with
%respect to $(\v,\w,\u)$.
%\end{definition}
%\indy{Index}{coordinate systems}%
%\indy{Index}{coordinate systems!polar coordinates}%
%\indy{Index}{polar coordinates}%
%\indy{Notation}{r@$r$ (coordinate)}%
%\indy{Notation}{zzh@$\theta$ (coordinate)}%
%
%In the special case that $\op{aff}\{\v,\w,\u\}=\ring{R}^2\subset
% \ring{R}^3$, this construction agrees with the previously defined
% polar coordinates of a point in the plane.


%\subsection{Lexell without polarity}
%
%Here is a second proof of Lexell's theorem that does not depend on
% polar triangles.
%
%\begin{proof} Select coordinates so that the Lexell circle (through
%  $\u,\v^*_1,\v^*_2$) has constant zenith angle $\phi$.  Without loss
%  of generality, an appropriate coordinate system gives
%\[ 
%\begin{align}
%\v_1 &= \{\cos\theta\sin\phi,+\sin\theta\sin\phi,-\cos\phi\}\\
%\v_2 &= \{\cos\theta\sin\phi,-\sin\theta\sin\phi,-\cos\phi\}\\
%\u &= \{\cos\alpha\sin\phi,\sin\alpha\sin\phi,\cos\phi\}\\
%\end{align}
%\] 
%The area of a triangle is given by Euler's formula
% (Lemma~\ref{lemma:euler}).  If these coordinates are used in Euler's
% formula, then a calculation gives the area $\pi-2\atn(t,1)$, when
%\[ 
%t=\cos\phi \tan\theta.
%\] 
%This is independent of $\alpha$, proving that every point on the
% Lexell circle (except for the degenerate points $\u=
% \v^*_1,\u=\v^*_2$ with $\Delta=0$) gives the same solid angle.
%
% To check that points on different Lexell circles give different
% solid angles, any convenient point on the circle will do.  For
% example, there is an isosceles triangle $b=c$.  An easy derivative
% calculation shows that the function is increasing.  Hence different
% Lexell circles give different values.
%\end{proof}






\section{Cycle}


The azimuth angle of the spherical coordinate system
determines a cyclic permutation, called the azimuth cycle, on a finite
set of points in $\ring{R}^3$, ordered according to increasing azimuth
angle.  The basic properties of that permutation are developed.
\indy{Index}{cyclic!permutation}%
\indy{Index}{angle}%
\indy{Index}{coordinate systems}%


\subsection{polar cycle}

Let $V=\{\v_1,\ldots,\v_k\}$ be a finite set of nonzero points in the
plane, with polar coordinates $\v_i =
(r_i\cos\theta_i,r_i\sin\theta_i)$.  It is useful to order the set of
points according to increasing angle.  To deal with degenerate cases
when some points have exactly the same angle, order the points with
the lexicographic order on their polar coordinates.  We write $\v_i \prec
\v_j$ for the total lexicographical order on points: 
$\theta_i < \theta_j$ or both $\theta_i=\theta_j$ and $r_i<r_j$.
(The degenerate case of two equal angles does not occur in this book, but by
defining a total order, there is no need to revisit the issue.)
See Figure~\ref{fig:polarcycle}
\indy{Index}{order!total}%

\figROHSJRP % {fig:polarcycle}

\begin{definition}[polar cycle]\guid{TNZQDCX}
\formaldef{polar cycle}{polar\_cycle}
A cyclic permutation $\sigma:V\to V$ sends $\v\in V$ to
the next larger element with respect to this order or back to the
first element if $\v$ is the largest.  We call $\sigma$ the
\fullterm{polar cycle}{polar!cycle} of the set $V$.
\end{definition}
\indy{Index}{order!lexicographic}%
\indy{Index}{cyclic!permutation}%
\indy{Notation}{zzs@$\sigma$ (azimuth and polar cycle)}%




For $\psi\in\ring{R}$, let $T:\ring{R}^2\to\ring{R}^2$ be the
rotation of the plane:
\begin{equation}
\label{eqn:rotate}
(x,y) \mapsto  (x\cos\psi + y\sin\psi,-x\sin\psi+y\cos\psi).
\indy{Index}{rotation}%
\end{equation}
Let $\sigma'$ be the polar cycle for $T(V)$.  Then it easy to verify
that
\[ 
\sigma'(T \v) = T (\sigma \v),\quad \text{ for } \v\in V. 
\] 
\indy{Notation}{zzv@$\psi$}%
\indy{Notation}{T@$T$ (rotation)}%

\begin{lemma}[]\guid{PDPFQUK}\label{lemma:polar2}
    \formalauthor{Nguyen Quang Truong}
  \formal{thetaij\_t} Let $\theta_i$ be real numbers such that $0\le
  \theta_i < 2\pi$ for $i=1,2$.  Let \[  \theta_{ji}
    = \theta_i - \theta_j + 2\pi k_{ji},
\] 
where integers $k_{ij}$ satisfy $0\le \theta_{ji}< 2\pi$.
Then 
\[ 
\theta_{12} + \theta_{21} = \begin{cases}
2\pi, & \text{ if }\theta_i\ne\theta_j\\
0,    & \text{ if }\theta_i=\theta_j.
\end{cases}
\] 
\end{lemma}
%\indy{Notation}{zzh@$\theta$}%
\indy{Notation}{zzh@$\theta$ (polar, cylindrical, and spherical angle)}%

\begin{proof} The proof is elementary.
\end{proof}

The next lemma gives a precise form to the observation
that given a finite number of rays emanating from the origin
in the plane, the sum of the included angles is $2\pi$.
In precise form, the polar cycle is used to place
a cyclic order on the rays.  There is a degenerate case
when there is at most one ray.


\begin{lemma}[]\guid{ISRTTNZ}\label{lemma:polar-sum}
\formalauthor{Nguyen Quang Truong}
%\formal{thetapq\_wind\_t}
  Let $V\subset\ring{R}^2$ be a finite set of cardinality $n$ that
  does not contain $0$.  Let $\sigma$ be the polar cycle on $V$.  In
  polar coordinates,
\[ 
\v=\left(\, r(\v)\cos\theta(\v),\, r(\v)\sin\theta(\v)\,\right),
\]  
for $\v\in V$, with
$0\le\theta(\v)<2\pi$.
Write
\[ 
\theta(\v,\w) = \theta(\w) - \theta(\v) + 2\pi k_{pq},
\] 
for some integers $k_{pq}$ that satisfy $0\le \theta(\v,\w) < 2\pi$.
Then for all $\v\in V$
and all $0\le i \le j < n$,
\[ 
\theta(\v,\sigma^i(\v)) +\theta(\sigma^i(\v),\sigma^j(\v)) =
\theta(\v,\sigma^j(\v)).
\] 
Moreover, if there exist $\v,\w\in V$ such that $\theta(\v)\ne\theta(\w)$,
\[ 
\sum_{i=0}^{n-1} \theta(\sigma^{i}\v,\sigma^{i+1} \v) = 2\pi.
\] 
(If $\theta(\v)=\theta(\w)$ for all $\v,\w\in V$, then all the
summands are zero.)
\end{lemma}
%\indy{Notation}{zzs@$\sigma$ (permutation)}%
\indy{Notation}{zzs@$\sigma$ (azimuth and polar cycle)}%

\begin{proof}
Fix $\v\in V$.
For $0\le i<n$, define $\theta_i$ by
$\theta_0=\theta(\v)$ and 
\[ \theta_i = \theta(\sigma^i(\v)) + 2\pi \ell_i,
\] 
where  $\ell_i$ satisfies $\theta_0\le \theta_i < \theta_0+2\pi$.
It follows from the definition of the polar cycle that
$\theta_i \le \theta_j$ for $0\le i\le j < n$.  Then
$\theta(\sigma^i \v ,\sigma^j \v) = \theta_j - \theta_i$.
The first conclusion of the lemma reduces to
\[ 
(\theta_i-\theta_0) + (\theta_j-\theta_i) = (\theta_j-\theta_0),
\] 
which is certainly true.
The second conclusion reduces to
\[ 
\sum_{i=0}^{n-2} (\theta_{i+1}-\theta_i) + \theta(\sigma^{n-1}\v,\v)
= \theta(\v,\sigma^{n-1}\v) + \theta(\sigma^{n-1}\v,\v).
\] 
By the previous lemma, this is $0$ or $2\pi$.
\end{proof}


\subsection{azimuth cycle}

As already defined, the polar cycle is a cyclic permutation on a set
of vectors in the plane that traverses them in order of increasing
angle.  What follows is the corresponding construction in three
dimensional space.  There is a cyclic permutation, called the
\newterm{azimuth cycle}, on a set $V$ of vectors in space that
traverses them in order of increasing azimuth angle.  Most of the work
for this construction has already been done in the subsection on polar
cycle, because the azimuth cycle may be constructed as the polar cycle
on the projection of $V$ to a plane.  However, a nondegeneracy
condition must be imposed on $V$ to ensure that the projection to the
plane is one-to-one.  The following definition captures this
nondegeneracy condition.  \indy{Index}{azimuth cycle}%
\indy{Index}{azimuth}%
\indy{Index}{cyclic!permutation}%
\indy{Index}{vector!projection}%


\begin{definition}[cyclic set]\guid{KFKHLWK}
\formaldef{cyclic set}{cyclic\_set} 
Let $(\v_0,\v_1)$ be an ordered pair of
  distinct points in $\ring{R}^3$.  Let $V$ be a finite set of points
  in $\ring{R}^3$.  We say that $V$ is \fullterm{cyclic}{cyclic!set} with respect to
  $(\v_0,\v_1)$ if the following two conditions hold:
\begin{enumerate}
\item If $\u = \w + h (\v_1-\v_0)$, with $\u,\w\in V$ and $h\in \ring{R}$,
then $\u=\w$.  
\item  The line through $\v_0$ and $\v_1$ does not meet $V$.
\end{enumerate}
\end{definition}

A cyclic set $V$ has a well-defined azimuth cycle (Figure~\ref{fig:azimuthcycle}).

\figHOUNZSY % {fig:azimuthcycle}

\begin{definition}[azimuth cycle]\guid{YESEEWW}
\formaldef{$\sigma$}{azim\_cycle}
  Let $\v_0$ and $\v_1$ be distinct points in $\ring{R}^3$.  Let $V$
  be a finite set of points in $\ring{R}^3$ that is cyclic with
  respect to $(\v_0,\v_1)$.  Pick $\p\in\ring{R}^3$ such that
  $\{\v_0,\v_1,\p\}$ is not collinear and let $\{\e_1,\e_2,\e_3\}$ be
  the corresponding positive, adapted, frame.  Let $f$ be the
  projection map:
\[ \v_0 + x\, \e_2 + y\, \e_3 + z\, \e_1 \mapsto
(x,y).\] 
Let $\sigma'$ be the polar cycle on $f(V)$. We define
$\sigma:V\to V$ by $f\sigma(\u) =\sigma'f(\u)$
and call $\sigma$ the \newterm{azimuth cycle}
on $V$ with respect to $(\v_0,\v_1)$.
\indy{Index}{azimuth cycle} %
\indy{Index}{frame}%
\indy{Index}{polar!cycle}%
\indy{Notation}{f@$f$ (function name)}%
\indy{Notation}{zzs@$\sigma$ (azimuth and polar cycle)}%
%\indy{Notation}{zzs@$\sigma$ (polar cycle)}%
%\indy{Notation}{zzs@$\sigma$ (azimuth cycle)}%
\end{definition}

Because facts about the polar cycle lift to facts about the azimuth cycle,
the next few lemmas follow naturally.


\begin{lemma}[]\guid{NLOFMTR} The azimuth cycle $\sigma:V\to V$ on
  a cyclic set $V$ with respect to $(\v_0,\v_1)$ does not depend on
  the choice of $\p\in\ring{R}^3$ such that $\{\v_0,\v_1,\p\}$ is
  noncollinear.
\end{lemma}
\indy{Index}{azimuth cycle}%
\indy{Index}{cyclic!set}%

\begin{proof} The lemma follows from independence of $\sigma\,'$ from
rotations in the $\{\e_2,\e_3\}$ plane  in~\eqref{eqn:rotate}.
\end{proof}


\begin{lemma}[]\guid{YVREJIS} 
\formalauthor{Nguyen Quang Truong}
Let $(\v_0,\v_1)$ be an ordered pair of points in $\ring{R}^3$,
with $\v_0\ne \v_1$.  Assume that $\{\v_2,\v_3\}$ is cyclic
with respect to $(\v_0,\v_1)$.  Then
\[ 
\op{azim}(\v_0,\v_1,\v_2,\v_3) + \op{azim}(\v_0,\v_1,\v_3,\v_2) 
= \begin{cases} 2\pi, & \text{if }\op{azim}(\v_0,\v_1,\v_2,\v_3)\ne 0,\\
0, & \text{if }\op{azim}(\v_0,\v_1,\v_2,\v_3)=0.
\end{cases}
\] 
\end{lemma}


\begin{proof} The lemma follows immediately from Lemma~\ref{lemma:polar2}.
\end{proof}

\begin{lemma}[]\guid{ULEKUUB} \label{lemma:2pi-sum}
Let $(\v_0,\v_1)$ be an ordered pair of points in $\ring{R}^3$,
with $\v_0\ne \v_1$.  Let $V$ be a finite set in $\ring{R}^3$ of
cardinality $n$ that
is cyclic with respect to $(\v_0,\v_1)$,
with azimuth cycle $\sigma$.
Then for all $\u\in V$,
and all $0\le i \le j < n$,
\[ 
\op{azim}(\v_0,\v_1,\u,\sigma^i(\u)) +
\op{azim}(\v_0,\v_1,\sigma^i(\u),\sigma^j(\u)) =
\op{azim}(\v_0,\v_1,\u,\sigma^j(\u)).
\] 
Moreover, if there exists $\w\in V$ such that 
$\op{azim}(\v_0,\v_1,\u,\w)\ne0$,
then
\[ 
\sum_{i=0}^{n-1} \op{azim}(\v_0,\v_1,\sigma^i\u,\sigma^{i+1}\u) = 2\pi.
\] 
(If $\op{azim}(\v_0,\v_1,\u,\w)=0$ for all $\w\in V$, then all the
summands are zero.)
\end{lemma}
\indy{Notation}{azim}%
\indy{Index}{azimuth}%
\indy{Index}{azimuth cycle}%
%\indy{Notation}{zzs@$\sigma$}%
\indy{Notation}{zzs@$\sigma$ (azimuth and polar cycle)}%
\indy{Notation}{n@$n$ (integer variable)}%

\begin{proof} This follows immediately from 
Lemma~\ref{lemma:polar-sum}.
\end{proof}


\subsection{spherical triangle inequality} %%
\indy{Index}{triangle!spherical}%
\indy{Index}{spherical!triangle inequality}%

The geodesic length between two points
$\u,\v$ on a unit sphere centered at $\v_0$ is $\arc_V(\v_0,\{\u,\v\})$.
The following lemma is part of the verification that
the function $d(\u,\v) = \arc_V(\v_0,\{\u,\v\})$ is a metric
on the unit sphere.  The lemma excludes the degenerate case when
points on the sphere are antipodal.
\indy{Notation}{d@$d(\u,\v)$ (metric on $\ring{R}^3$)}%

\begin{lemma}[]\guid{KEITDWB}\label{lemma:sph-tri-ineq}

\formalauthor{Nguyen Quang Truong}
Let $\{\v_0,\v_1,\v_2,\v_3\}$ be a set of four points in $\ring{R}^3$.
Assume that $\v_0$ is not collinear with any of pair of the other points.
Then
\[ 
  \arc_V(\v_0,\{\v_1,\v_3\}) \le \arc_V(\v_0,\{\v_1,\v_2\}) + \arc_V(\v_0,\{\v_2,\v_3\}).
\] 
Equality occurs if and only if $\v_2\in\op{aff}_+(\v_0,\{\v_1,\v_3\})$.
\end{lemma}

\begin{proof} Let $\v_2'$ be the projection of $\v_2$ to the plane
$\op{aff}\{\v_0,\v_1,\v_3\}$.  
By the spherical law of cosines, when the triangle is right
\[ 
\cos\psi = \cos\beta\cos\alpha \le \cos\beta,
\] 
where $\psi = \arc_V(\v_0,\{\v_1,\v_2\})$, $\beta =
\arc_V(\v_0,\{\v_1,\v_2'\})$, $\alpha=\arc_V(\v_0,\{\v_2,\v_2'\})$.  Thus,
$\arc_V(\v_0,\{\v_1,\v_2'\})=\beta\le \psi=\arc_V(\v_0,\{\v_1,\v_2\})$.
Similarly, $\arc_V(\v_0,\{\v_2',\v_3\}) \le \arc_V(\v_0,\{\v_2,\v_3\})$.
Thus, it is enough to show that
\[ 
  \arc_V(\v_0,\{\v_1,\v_3\}) \le \arc_V(\v_0,\{\v_1,\v_2'\}) + \arc_V(\v_0,\{\v_2',\v_3\}).
\] 
The points $\v_0,\v_1,\v_3,\v_2'$ are coplanar.
By the additivity of planar angle (Lemma~\ref{lemma:polar-sum}), if 
$\v_2'\in \op{aff}_+(\v_0,\{\v_1,\v_3\})$, then
\[ 
  \arc_V(\v_0,\{\v_1,\v_3\}) = \arc_V(\v_0,\{\v_1,\v_2'\}) + \arc_V(\v_0,\{\v_2',\v_3\}),   
\] 
and otherwise,
\[ 
  \arc_V(\v_0,\{\v_1,\v_3\}) = \norm{\arc_V(\v_0,\{\v_1,\v_2'\}) }{ \arc_V(\v_0,\{\v_2',\v_3\})}.
\] 
The inequality ensues.

A trace of the argument shows that equality occurs exactly when
$\alpha=0$ and $\v_2'\in \op{aff}_+(\v_0,\{\v_1,\v_3\})$.  Equivalently,
$\v_2'=\v_2\in\op{aff}_+(\v_0,\{\v_1,\v_3\})$.
\end{proof}

\begin{lemma}[]\guid{FGNMPAV}
\formalauthor{Nguyen Quang Truong}
\label{lemma:sph-tri-multi}
Let $\{\v_0,\u_0,\u_1,\u_2,\ldots,\u_r\}$ be a set of points in
$\ring{R}^3$.  Assume that no triple $\{\v_0,\u_i,\u_{i+1}\}$ is
collinear.  Assume that $\{\v_0,\u_0,\u_r\}$ is not collinear.  Then
\[ 
  \arc_V(\v_0,\{\u_0,\u_r\}) \le \sum_{i=0}^{r-1} \arc_V(\v_0,\{\u_i,\u_{i+1}\}).
\] 
\end{lemma}

\begin{proof} The proof is an easy induction on $r$ with base case given by
  Lemma~\ref{lemma:sph-tri-ineq}.
\end{proof}


\section{Chapter Summary}

%\subsection{formal proof}

%Formal proofs of all of the major results in this chapter have already
%been constructed.  In fact, many of them are part of standard
%distribution of the HOL Light proof assistant.  In 2008, Jason Rute
%constructed the formal proof of some of the theorems in this chapter.
%Most of the remaining work was carried out by Nguyen Quang Truong in
%2009.  He also constructed the formal proofs of a large collection of
%lemmas in elementary geometry, including basic facts about the
%functions $\ups$ and $\Delta$.  Finally in 2010, Euler's theorem was
%formalized by Vu Khac Ky and Trieu Thi Diep.
%

%\subsection{summary of notation}

We give a brief chapter summary.  The trigonometric functions $\cos$,
$\sin$, $\arctan$, $\arccos$ are defined in the standard way.  The
function $\atn(x,y)$ is an extension of $\arctan(y/x)$ to every point
$(x,y)$ in the plane.  It is the polar coordinate angle of $(x,y)$.

 $\ring{R}^N$ is the vector space of functions from the finite set $N$ to $\ring{R}$.  If
$n\in\ring{R}$,   then by convention, $\ring{R}^n = \ring{R}^N$, where $N=\{0,\ldots,n-1\}$.
A bold face $\u,\v,\p,\q$ is used for points in $\ring{R}^N$.   Vector space operations, the
dot product $\u\cdot \v$, and the norm $\normo{\u}$ are defined in the standard way.

We write $\op{aff}(S)$ for the affine hull of a set and $\op{conv}(S)$ for the convex hull of a set.
The notation is extended to allow inequality constraints:
	\begin{align*}
\op{aff}_{\pm} (V,V') &= \{t_1 \v_1 +\cdots t_n \v_n \mid
	t_1 +\cdots+t_n = 1, \pm t_j \ge 0, \text{ for } j>k\},\\
\op{aff}^0_{\pm} (V,V') &= \{t_1 \v_1 +\cdots t_n \v_n \mid
	t_1 +\cdots+t_n = 1, \pm t_j > 0, \text{ for } j>k\}.
%\op{aff}\, V &= \op{aff}_\pm(V,\emptyset).\\
\end{align*}
Lines, planes, rays, cones, half-planes, half-spaces, and convex hulls
can all be represented compactly in this notation.

The polynomials $\ups$ and $\Delta$, which appear in
formulas for angle, area, and volume, depend on three and six
variables, respectively.  The function $\arc_V(\u,\{\v,\w\})$ gives
the angle at point $\u$ of a triangle with vertices $\u,\v,\w$.  The
function $\arc(a,b,c)$ is the angle opposite $c$ of a triangle with
sides $a,b,c$.  The function $\dih_V$ is the dihedral angle of a
simplex, expressed as a function of its four vertices.  The function
$\dih$ is the dihedral angle of a simplex, expressed as a function of
its six edges.

The cylindrical coordinates of a point in $\ring{R}^3$ are
$(r,\theta,h)$.  The spherical coordinates are $(r,\theta,\phi)$.  The
angle $\theta$ is called the azimuth angle and is determined by four
points $\v_0,\v_1,\v_2,\v_3$.  The angle $\phi$ is the zenith angle.
This book follows a nonstandard convention for the labeling of the
coordinate axes in cylindrical and spherical coordinates: the central
line of the cylinder and the line through the poles of the coordinate
sphere lie in the direction of the first unit vector $\e_1$.

The cyclic permutation of a finite set of points in the plane, ordered
by increasing angle in polar coordinates is called the polar cycle.
The cyclic permutation $\sigma$ of a finite set of points in
three-dimensional space, ordered by increasing azimuth angle is called
the azimuth cycle.


  
    %% Quadratic Volumes
%% File Created 3/22/07.

\section{Properties of Measure}

Nowhere do we need a notion
of integration.  Measure alone suffices.  (However, there are a few
volumes described below that I do not see how to calculate without
first writing them as an integral.)

We need a concepts of null set, measurable set, and volume in
three dimensions.  For our purposes, we can take the
the three dimensional Lebesgue measure.   
The null sets can be defined
to be the sets of zero Lebesgue measure. The measurable sets can
be defined as the bounded Lebesgue measurable sets.  The volume of
a measurable set can be defined as its Lebesgue measure.
As we will see in a moment, we need considerably less than Lebesgue measure.



\subsection{properties of null sets}

We assume a notion of null set with the following
properties.

\begin{enumerate}%[Null Set]
\item A finite union of null sets is a null set.\\
 \item A plane is a null set.\\
 \item A sphere is a null set.\\
 \item A circular cone is a null set; that is, a union of all
  lines through a fixed point $P$ and forming fixed
 forming fixed angle with a line through $P$.
\tlabel{enum:null}
\end{enumerate}

We write $A\equiv B$ if sets $A$ and $B$ are equal up to a null set.
That is, there exists a null set $E$ such that
   $(A\setminus B) \cup (B\setminus A) \subset E$.
\index{null set}\index{ZZZequiv@$\equiv$}

\subsection{properties of measure}

We assume a notion of measurability that has the following properties.

\begin{enumerate}%[Measurable set]
 \item The union of two measurable sets is measurable.\\
 \item The intersection of two measurable sets is measurable.\\
 \item The difference of two measurable sets is measurable.
\tlabel{enum:measure}
\end{enumerate}

\subsection{properties of volume}

We assume a notion of volume that has the following properties.

\begin{enumerate}%[Volume]
 \item The volume is defined for every measurable set.  It is
    a non-negative real number.
 \item If $X$ and $Y$ are  measurable, and if
 the symmetric difference of
 $X$ and $Y$ is contained in a null set, then 
    $X$ and $Y$ have the same volume.\\
 \item If $X$ and $Y$ are measurable sets, and if $X\cap
 Y$ is contained in a null set, then
    $$
    \op{vol}(X\cup Y) = \op{vol}(X) + \op{vol}(Y).
    $$
  \item (linear stretch) If $X\subset \ring{R}^3$, $t\in\ring{R}$, 
    $i=1,2,3$, and $e_i\in\ring{R}^3$ is the $i$th standard basis vector,
    set 
      $$T_i(X,t) = \{ u + (t-1) u_i e_i \mid u\in X\}.
      $$
    If $X$ is measurable, then $X'=T_i(X,t)$ is as well,
    and $\op{vol}(X') = |t|\op{vol}(X)$.
  \item (translation) If $X\subset \ring{R}^3$ and $v\in\ring{R}^3$, then let
    $X+v = \{x + v\mid x\in X\}$.  If $X$ is measurable, then $X+v$ is
    as well, and $\op{vol}(X) = \op{vol}(X+v)$.
\tlabel{enum:volume}
\end{enumerate}

In particular, if $X$ is contained in a null set, we may take
$X=Y$ in the preceding to deduce that $\op{vol}(X)=0$.

In addition to these properties, we will also need specific
volume calculations of primitive regions as described in
Lemmas~\ref{lemma:prim-volume} and~\ref{lemma:wedge-vol}.
%% NB Don't need lemma:wedge-sol because solid of a FR is a SC.
%% Don't need lemma:prim-sol, because solds are SC or ST.

\subsection{radial sets and solid angle}\label{sec:solid}


Surface integrals are not required in this book.  Although
the `solid angle' is traditionally defined as a surface integral,
we give an alternative definition based on volume.


\begin{definition}
    A set $C$ is $r$-radial at center $x$ if  $C\subset B(x,r)$
    and if
        $x + u \in C$ implies
        $x + t u \in C$ for all $t$ satisfying $0\le |u| t < r$.
A set $C$ is eventually radial at center $x$ if $C\cap B(x,r)$ is
$r$-radial at center $x$, for some $r>0$.
\end{definition}

\begin{lemma}\tlabel{lemma:r-r'}
Assume that $C$ is measurable and $r$-radial at $x$.  Let $0\le r'<r$,
then $C\cap B(x,r')$ is measurable and
$\op{vol}(C\cap B(x,r')) = \op{vol}(C) (r'/r)^3$.
\end{lemma}

\begin{proof}  We can transform $C$ into $C\cap B(x,r')$ by
a series of translations and stretch transformations.
\end{proof}


\begin{definition}\tlabel{def:sol}
If $C$ is measurable and eventually radial at center $x$, then we
define the solid angle of $C$ at $x$ to be
    $$
    \op{sol}(x,C) = 3 \op{vol}(C\cap B(x,r))/r^3,
    $$
where $r$ is as in the definition of eventually radial. 
By Lemma~\ref{lemma:r-r'}, this
definition is independent of any such $r$.  When the center $x$ is
clear from the context, we write $\op{sol}(C)$ for
$\op{sol}(x,C)$.
\end{definition}



The following properties follow immediately from the definitions.
If $C$ is $r$-radial for some $r>0$ then it is eventually radial.
If $C$ is measurable and $r$-radial, then the volume of $C$
satisfies
    $$
    \op{vol}(C) = \op{sol}(C) r^3/3.
    $$
If $C$ is bounded away from $x$, then $C$ is eventually radial at
$x$, and $\op{sol}(C) = 0$.

\begin{lemma}  If $C$ and $C'$ are  $r$-radial
at $v_0$, then $C\cap C'$ is also $r$-radial at
$x$.
\end{lemma}






\section{Primitive Volumes}

We accept 
certain elementary volume calculations as axioms.  
These regions will be called primitive volumes.  There are only
a few primitive volumes.
All further
volumes calculations will be obtained from these through the basic
properties of measure.   
Our treatment of volume 
is hardly about measure at all.  The focus is rather on
the geometry of the various regions and how to decompose them into
primitives.

We prefer to take the volume of open sets whenever that can be
arranged.  We begin with a description of some of the primitive
regions.






\subsection{ball}

\begin{definition}  The open ball $B(x,r)$ with center $x$ and
radius $r$ is the set
    $$
    \{ y\in\ring{R}^3 \mid |x-y| < r.\}
    $$
\end{definition}



\subsection{wedge}


The set $\op{aff}^0_+(\{v_0,v_1\},\{v_2,v_3\})$ was defined
in Definition~\ref{def:aff}.  We call it a lune.  It is the intersection
of two open half-spaces
    $$
    \op{aff}^0_+(\{v_0,v_1\},\{v_2,v_3\})
    =\op{aff}^0_+(\{v_0,v_1,v_2\},v_3)\cap
    \op{aff}^0_+(\{v_0,v_1,v_3\},v_2)
    $$


A lune has a dihedral angle $\dih(\{v_0,v_1\},\{v_2,v_3\})$ between
$0$ and $\pi$.   For angles that are larger than $\pi$,  we use a wedge
$W(v_0,v_1,w_1,w_2)$.  Assume that $v_0\ne v_1$ and that
$w_1$ and $w_2$ do not lie on
the line $\op{aff}\{v_0,v_1\}$.  Set
$$
W(v_0,v_1,w_1,w_2) = 
  \{x\not\in\op{aff}\{v_0,v_1\} \mid 
  0< \op{azim}(v_0,v_1,w_1,x) < \op{azim}(v_0,v_1,w_1,w_2)\}.
$$
When the angle is less than $\pi$, there is no difference between
a wedge and a lune:

\begin{lemma} Let $\{v_0,v_1,v_2,v_3\}$ be a set of four points
in $\ring{R}^3$.  Assume that the set is not coplanar.
Assume that $\op{azim}(v_0,v_1,v_2,v_3)<\pi$.
Then,
   $$W(v_0,v_1,v_2,v_3) = \op{aff}^0_-(\{v_0,v_1\},\{v_2,v_3\}).$$
\end{lemma}


\subsection{solid triangle}

\begin{definition} The solid triangle $ST(v_0,\{v_1,v_2,v_3\},r)$ is
specified by four points $v_i\in\ring{R}^3$, and a radius $r\ge0$. 
    $$
    ST(v_0,\{v_1,v_2,v_3\},r) = 
    B(v_0,r)\cap \op{cone}(v_0,\{v_1,v_2,v_3\}).
    $$
\end{definition}



\subsection{conic cap}

% renamed from spherical cap.

\begin{definition}
The conic cap $SC(v_0,v_1,r,a)$ is specified by an apex
$v_0\in\ring{R}^3$, a radius $r\ge0$, a non-zero vector $v_1-v_0$ giving
direction, and constant $a$.  The conic cap is the intersection of
the ball $B(v_0,r)$ with a solid right-circular cone:
    $$
    SC(v_0,v_1,r,a)=\{y \in B(v_0,r) \mid (y-v_0)\cdot (v_1-v_0) > |y-v_0|\, |v_1-v_0|\, a\}.
    $$
\end{definition}

\subsection{frustum}

\begin{definition}\tlabel{def:p:rcone}
%\begin{definition}\tlabel{def:rcone} % from tarski..
\index{rcone}\index{right-circular cone}
We define the following collection of right-circular cones.
If $v$ and $w$ are points in $\ring{R}^3$, and
  $h\in\ring{R}$, then set
  $$\begin{array}{lll}
    \op{rcone}(v,w,h) &= \{x\mid (x-v)\cdot (w-v) \ge |x-v|\,|w-v| h\},\\
    \op{rcone}^0(v,w,h) &= \{x\mid (x-v)\cdot (w-v) > |x-v|\,|w-v| h\}.\\
    \op{rcone}_-^0(v,w,h) &= \{x\mid (x-v)\cdot (w-v) < |x-v|\,|w-v| h\}.\\
    \partial\op{rcone}(v,w,h) &= \{x\mid (x-v)\cdot (w-v) = |x-v|\,|w-v| h\}.\\
    \end{array}
    $$
\end{definition}



\begin{definition} The frustum
$FR(v_0,v_1,h',h,a)$ is specified by an apex $v_0\in\ring{R}^3$, heights
$0\le h'\le h$, a vector $v_1-v_0$ giving its direction, and
$a\in[0,1]$. The set $FR$ is given as
    $$
    \{ y \in\op{rcone}^0(v_0,v_1,a) \mid \ 
       h'|v_1-v_0| < (y-v_0) \cdot (v_1-v_0) < h|v_1-v_0| \}.
    $$
\end{definition}

That is, the frustum is the part of a right-circular cone between two
parallel planes that cut the axis of the cone at a right angle.
When $h'=0$, the frustum extends to the apex of the cone, and
we write $FR(v_0,v_1,h,a)=FR(v_0,v_1,h',h,a)$.

\subsection{tetrahedron}

\begin{definition} A tetrahedron is a set of the form
$$\op{conv}^0\{v_1,v_2,v_3,v_4\}.$$
\end{definition}

By Lemma~\ref{tarski:hedra-tope}, this set can also be described
as the intersection of four open half-spaces, with each bounding
plane defined by three of the four points.
Taking this into account, we note that
the sets in this section have all been defined by linear and quadratic
constraints.

\subsection{primitive}

\begin{definition} A primitive region is any of the following.

\begin{enumerate}%%[Primitive Volumes]
 \item A solid triangle $ST$.
 \item A tetrahedron $S$.
 \item A wedge of a frustum (with $h'=0$); 
that is, the intersection of a frustum with
 a wedge:
    $$
     FR(v_0,v_1,h,a) \cap W(v_0,v_1,v_2,v_3).
    $$
\item A wedge of a conic cap; that is, the intersection of a conic cap
with
    a wedge:
    $$
    SC(v_0,v_1,r,c) \cap W(v_0,v_1,v_2,v_3).
    $$
\tlabel{enum:volume-prim}
\end{enumerate}

\end{definition}

\subsection{primitive volume calculation}

\begin{lemma}\tlabel{lemma:prim-volume} 
\begin{enumerate} 
 \item Let $v_1,v_2,v_3$ be unit vectors.
   A solid triangle $ST(v_0,\{v_1,v_2,v_3\},r)$ has volume
   $$
   (\alpha_{123}+\alpha_{231}+\alpha_{312}-\pi)r^3/3,
   $$
   where $\alpha_{ijk} = \dih_V(\{v_0,v_i\},\{v_j,v_k\})$.
  \item The conic cap $SC(v_0,v_1,r,a)$ has volume:
   $$
    2\pi(1-a) r^3/3,
   $$
 \item A frustum $FR(v_0,v_1,h,a)$ has volume:
   $$
   \pi (t^2-h^2) h/3,\quad h = t a.
   $$
 \item A tetrahedron $\op{conv}^0(\{v_1,v_2,v_3,v_4\})$ has volume:
   $$
   \sqrt{\Delta(x_{12},x_{13},x_{14},x_{34},x_{24},x_{23})}/12,
   $$
   where $x_{ij} = |v_i-v_j|^2$.
\end{enumerate}
\end{lemma}

Euler's formula (Lemma~\ref{lemma:euler}) gives an
equivalent expression for $(\alpha_{123}+\alpha_{231}+\alpha_{312}-\pi)$.
Euler's formula will often be used instead of this formula.

\begin{proof}
The formula for the volume of a solid triangle is $r^3/3$ times
its solid angle.  The formula 
   $$\alpha_{123}+\alpha_{231}+\alpha_{312}-\pi$$
for the area of a spherical triangle is classical.    
The conic cap volume is
$r^3/3$ times its solid angle.  
The volume of a right-circular cone is $1/3$ its base times height.
The volume of a tetrahedron is
   $$|\det(v_2-v_1,v_3-v_1,v_4-v_1)|/6.$$
By Lemma~\ref{tarski:cm4}, 
the square of this determinant is given by a formula
$\op{CM}_4(x_{ij})$, which by Lemma~\ref{tarski:cm4} is
$\Delta/4$, with $\Delta\ge0$.  The result follows.
\end{proof}



\subsection{wedge}

If the region is realized by revolution along an axis $\op{aff}\{v_0,v_1\}$, 
then
we can also give the volume of the intersection of the region
with a wedge $W=W(v_0,v_1,v_2,v_3)$.
  In the following
let $\theta = \azim(v_0,v_1,v_2,v_3)$.

\begin{lemma}\tlabel{lemma:wedge-vol}  Let $C$ be either $SC(v_0,v_1,r,a)$ or
   $FR(v_0,v_1,h,a)$.  Let $m$ be the volume of $C$.  
   Then $C\cap W$ has volume $m\,\theta/(2\pi)$.   
\end{lemma}

\begin{lemma}\tlabel{lemma:wedge-sol}  Let $C$ be either $SC(v_0,v_1,r,a)$ or
   $FR(v_0,v_1,h,a)$.  Then $C$ and $C\cap W$ are eventually 
radial at $v_0$. Furthermore,
    $C\cap W$ solid angle 
  $s\,\theta/(2\pi)$, where $s$ is the solid angle of $C$.
\end{lemma}


\begin{proof}
These are elementary integrals.
\end{proof}


\subsection{solid angle of primitives}

All of the primitive sets are eventually radial at the natural
base point $v_0$, so we may take their
solid angle.  By Lemma~\ref{lemma:wedge-sol}, it is enough to compute
the solid angle before intersecting with a wedge.

\begin{lemma} \tlabel{lemma:prim-sol}
\begin{enumerate}
    \item  $ST(v_0,\{v_1,v_2,v_3\})$ is eventually radial at $v_0$
     with solid angle 
     $$
     (\alpha_{123}+\alpha_{231}+\alpha_{312}-\pi),\quad
     \alpha_{ijk}=\dih_V(\{v_0,v_i\},\{v_j,v_k\}).
     $$
    \item $\op{conv}^0\{v_0,v_1,v_2,v_3\}$ is eventually radial at $v_0$
      with solid angle
           $$
     (\alpha_{123}+\alpha_{231}+\alpha_{312}-\pi),\quad
     \alpha_{ijk}=\dih_V(\{v_0,v_i\},\{v_j,v_k\}).
     $$
    \item $SC(v_0,v_1,r,a)$ is eventually radial at $v_0$ with solid
      angle 
      $2\pi(1-a)$.
    \item $FR(v_0,v_1,h,a)$ is eventually radial at $v_0$ with solid
      angle
        $$
        2\pi (1-a).
        $$
\end{enumerate}
\end{lemma}

\begin{proof} In every case, the intersection of 
  the region with $B(v_0,r')$, for $r'>0$ sufficiently small, is
  a conic cap or a solid triangle.  These two volumes have
  already been calculated.  This gives the results as stated.
\end{proof}

\subsection{combining solid angle and volume}

It is often convenient to consider various linear combinations
of the solid angle and volume of eventually radial sets.  

\begin{definition}\label{def:sovo}
With
that in mind, we define the function
  $$
  \op{sovo}(v_0,V,\lambda) = \lambda_v \op{vol}(V) + \lambda_s
  \op{sol}(v_0,V),
  $$
where $V$ is a measurable set that is eventually radial at $v_0$
and $\lambda=(\lambda_v,\lambda_s)$ is a pair of real numbers
determining the linear combination.
\end{definition}

We define some auxiliary functions that will help us express
the value of $\op{sovo}$ on primitive regions.

\begin{definition}\tlabel{def:A}\tlabel{def:phi}
Define the function
 $$
 \phi(h,t,\lambda)=
   \lambda_v  t h (t+h)/6 + \lambda_s 
 $$
Define the function
 $$A(h,t,\lambda) = (1-h/t) (\phi(h,t,\lambda) - \phi(t,t,\lambda)).$$
\index{A}
\index{phi}
\end{definition}

\begin{lemma} If $V$ is measurable and $t$-radial at $v_0$,
then $$\op{sovo}(v_0,V,\lambda) = \op{sol}(v_0,V)\phi(t,t,\lambda).$$
\end{lemma}

\begin{proof} We have $$\vol(V) = \op{sol}(v_0,V)t^3/3,$$
so $$\op{sovo}(v_0,V,\lambda) = 
  \op{sol}(v_0,V)(\lambda_v t^3/3 + \lambda_s) = 
   \op{sol}(v_0,V)\phi(t,t,\lambda).$$
\end{proof}

\begin{lemma}\tlabel{lemma:sovoFR} Let $0 < h < t$.
Set $F  = FR(v_0,v_1,h,h/t)$ and $s = \sol(v_0,F)$.
  We have
  $$
  \begin{array}{lll}
  \op{sovo}(v_0,F,\lambda) 
   &= s
  \phi(h,t,\lambda) \\
   &= 2\pi A(h,t,\lambda) + s\phi(t,t,\lambda).
  \end{array}
  $$
\end{lemma}

\begin{proof}    From the primitive volume calculations,
we have 
  $$
  \begin{array}{lll}
  s &= 2\pi(t-h)/t,\\
  \op{vol}(F) &= \pi(t^2-h^2)h/3\\
      &= s t h (t+h)/6,\\
  \op{sovo}(v_0,F) &= 
     \lambda_v s t h (t+h)/6 + \lambda_s s\\
   &= s \phi(h,t,\lambda).
  \end{array}
  $$
Also,
  $$
  \begin{array}{lll}
  s\phi(h,t,\lambda) &= 2\pi (1-h/t)\phi(h,t,\lambda)\\
  &= 2\pi A(h,t,\lambda) + 2\pi (1-h/t)\phi(t,t,\lambda)\\
  &= 2\pi A(h,t,\lambda) + s \phi(t,t,\lambda).
  \end{array}
  $$
\end{proof}

\section{Scissors and Volumes}
\tlabel{sec:measure-second}

There are many other volumes that can be computed from the
primitive ones enumerated in Definition~\ref{enum:volume-prim}.

\subsection{lune}  

To give a simple example of a derived volume, we consider the
lune $A=\op{aff}_+^0(\{v_0,v_1\},\{v_2,v_3\})$.  It is eventually
radial at $v_0$, so we may compute its solid angle.

\begin{lemma}  $\sol(v_0,A) = 2\dih_V(\{v_0,v_1\},\{v_2,v_3\})$.
\end{lemma}

\begin{proof}
Let $B_{\pm} = \op{aff}_\pm(\{v_0,v_2,v_3\},v_1)$.  $B_- \cap B_+$
is a null set.  The intersections $A\cap B_{\pm}\cap B(v_0,r)$ 
are solid triangles.  This gives the solid angle of $A$ as
follows:
   $$\begin{array}{lll}
   \sol(v_0,A) &= \sol(v_0,A\cap B_+)+\sol(v_0,A\cap B_-) \\
   &= 
   \sol(ST(v_0,\{v_1,v_2,v_3\})) + \sol(ST(v_0,\{v_1',v_2,v_3\})) \\
   &=
   2\dih_V(\{v_0,v_1\},\{v_2,v_3\}).
   \end{array}
   $$
Here, we have used $v_1'= 2 v_0 - v_1$, the reflection of $v_1$
through $v_0$.  The usual calculation of the volume of a solid triangle
inverts this proof, 
and derives the volume from the solid angle of a lune.
\end{proof}



\begin{lemma}\tlabel{lemma:wedge:sol} 
Assume that the sets $\{v_0,v_1,w_1\}$ and
$\{v_0,v_1,w_w\}$ are not collinear. 
$$\sol(v_0,W(v_0,v_1,w_1,w_2)) = 2\op{azim}(v_0,v_1,w_1,w_2).$$
\end{lemma}    

\begin{proof} Every wedge is a union of two lunes, up to a null set.
\end{proof}

\begin{lemma}  
   $$
   \begin{array}{lll}
    \op{sol}(v_0,B(v_0,t)) &= 4\pi\\
    \op{vol}(v_0,B(v_0,t)) &= 4\pi t^3/3\\
    \op{sovo}(v_0,B(v_0,t),\lambda) &= 4 \pi \phi(t,t,\lambda).
   \end{array}
   $$
\end{lemma}

\begin{proof}
The ball is $t$-radial at $v_0$, so the volume is given by
Definition~\ref{def:sol} in terms of solid angle.  It is enough
to check that a hemisphere has solid angle $2\pi$.  This follows
from Lemma~\ref{lemma:wedge:sol}.
\end{proof}  




\subsection{Rogers simplex}

\begin{definition} \tlabel{def:ortho}
An {\it orthosimplex} is a tetrahedron
    $$\op{conv}^0(x,x+v_1,x+v_1+v_2,x+v_1+v_2+v_3),$$
where $v_i\cdot v_j=0$, for $1\le i<j\le 3$.   We write
$\op{orth}^0(x,v_1,v_2,v_3)$ for this orthosimplex.
 \index{orthosimplex}
\end{definition}

\begin{figure}[htb]
  \centering
  \myincludegraphics{\ps/rogers.eps}
  \caption{The Rogers simplex is an orthosimplex.}
  \tlabel{fig:rogers}
\end{figure}


\begin{definition} \tlabel{def:rog}
Let $\{v_0,v_1,v_2,v_3\}$ be a set of four points in $\ring{R}^3$.
Assume that they are not coplanar.  Let $p$ be the circumcenter
of $\{v_0,v_1,v_2\}$ and $r$ its circumradius (see Definition~\ref{def:circumrad2}).  Let $c\ge r$.
By Lemma~\ref{tarski:rog-exist}, there exists a unique
point $p'$ in $A=\op{aff}_+(\{v_0,v_1,v_2\},v_3)$ at equal distance $c$
from $v_0,v_1,v_2$.
Let $$
    \op{rog}^0(v_0,v_1,v_2,v_3,c) = 
    \op{ortho}^0(v_0,w_1,w_2,w_3),
    \quad w_1=(v_0+v_1)/2,\quad w_1+w_2=p,\quad w_1+w_2+w_3=p'.
    $$
(We define $\op{rog}(v_0,\ldots,v_3,c)$ similarly, where we use
$\op{conv}$ instead of $\op{conv}^0$.)
We take $\op{rog}^0$ to be the empty set, if $c< r$.
 \index{rogers simplex}
\end{definition}

\begin{lemma} The vectors $w_1,w_2,w_3$ of Definition~\ref{def:ortho}
are indeed mutually orthogonal.
\end{lemma}

\begin{proof} The orthogonality of $w_1$ and $w_2$ is found in
Lemma~\ref{tarski:eta-ortho}.  The orthogonality of $w_3$ with the
others is found in Lemma~\ref{tarski:rog-ortho}.
\end{proof}

\begin{definition}
Let $\eta(x,y,z)$ be the circumradius of a triangle with sides
$x,y,z$, and let $\eta_V(v_0,v_1,v_2) = \eta(|v_0-v_1|,|v_0-v_2|,|v_1-v_2|)$.
\index{circumradius}
\index{ZZeta@$\eta$}
\end{definition}

\begin{definition}\tlabel{def:abc}
We associate with $\op{rog}^0(v_0,v_1,v_2,v_3,c)$ the constants
$a=|v_1-v_0|/2$, $b=\eta_V(v_0,v_1,v_2)$, and $c$.
We call these the $abc$ parameters of $\op{rog}^0$. 
\end{definition}

The Rogers simplex is a tetrahedron.  Hence it is one of our
primitive regions.  It is eventually radial at $v_0$, hence
it has a solid angle at $v_0$.  When we mention its dihedral
angle, it is understood that it refers to 
   $$
   \dih_V(\{v_0,v_1\},\{v_2,p'\})=\dih_V(\{v_0,(v_0+v_1)/2\},\{p,p'\}),
   $$
where $p$ and $p'$ are the points 
constructed in Definition~\ref{def:rog}.

The squares of the edge lengths of the tetrahedron are
   $$
   (a^2,b^2,c^2,c^2-b^2,c^2-a^2,b^2-a^2).
   $$
Define the functions
   $$
   \begin{array}{lll}
     \op{volR}(a,b,c) &= \begin{cases}
       a\sqrt{(b^2-a^2)(c^2-b^2)}/6& 0 < a < b < c,\\
       0,&\text{otherwise}
       \end{cases}\\
     \op{solR}(a,b,c) &= \begin{cases}
      2 \arctan\left(\sqrt{\frac{(b-a) (c-b)}{(a+b)
   (b+c)}}\right).& 0 < a < b < c,\\
      0,&\text{otherwise}
     \end{cases}\\
     \op{dihR}(a,b,c) &= \begin{cases}
      \arctan\left(\sqrt{\left((c^2 - b^2)/(b^2 - a^2)\right)}\right)
      & 0 < a < b < c,\\
      0,&\text{otherwise}
     \end{cases}
     \end{array}
   $$
Specializing the formulas for dihedral angle, volume, and solid angle to this
setting we get the following expressions for volume and solid angle.
(The calculation of the
 solid angle formula is based on Euler's formula in 
Lemma~\ref{lemma:euler}.)

\begin{lemma}\label{lemma:rog:abc} 
Let $R=\op{rog}^0(v_0,v_1,v_2,v_3,c)$ and let $a$, $b$,
$c$ be the $abc$-parameters of $R$.  Let $\op{dih}(R)$ be the dihedral
angle of $R$ along the edge extending along $\op{aff}\{v_0,v_1\}$.  Then
$$
\begin{array}{lll}
\op{vol}(R) &= \op{volR}(a,b,c)\\
\op{sol}(v_0,R) &= \op{solR}(a,b,c)\\
\op{dih}(R) &= \op{dihR}(a,b,c)\\
\end{array}
$$
\end{lemma}





\begin{remark}
The volume of a unit cube aligned along the coordinate axes is $1$.  
If we want to insist on deriving all volumes from the primitive
volumes, then we can derive the volume of the cube by partitioning
it into six Rogers simplices,
each of volume $\op{vorR}(1,\sqrt2,\sqrt3) = 1/6$, for a total
of $1$, as desired.  Figure~\ref{fig:rogers} shows one of the six
Rogers simplices.
\end{remark}



\subsection{Rogers's lemma}


The following lemma is the key step in the proof of Rogers's
bound on the density of sphere packings \cite{Rog58}.

\begin{lemma} \tlabel{lemma:rogers}
Suppose that $a,b,c$ and $a',b',c'$
are real numbers that satisfy $0 <a \le b \le c$, $0 \le a'\le b'\le c'$,
$a \le a'$, $b \le b'$, $c \le c'$. Then
  $$
  \op{solR}(a',b',c')\op{volR}(a,b,c) \le \op{solR}(a,b,c)\op{volR}(a',b',c').
  $$
\end{lemma}

\begin{proof} If any of the equalities hold: $a=b$, $b=c$, $a'=b'$,
$b'=c'$, then both sides are zero.  We assume $a<b<c$ and $a'<b'<c'$.
Let $w_1=a\,e_1,w_2=\sqrt{b^2-a^2}\, e_2,w_3=\sqrt{c^2-b^2}\, e_3,$
for the standard basis $\{e_1,e_2,e_3\}$.  Each point of the orthosimplex
$S = \op{ortho}^0(0,w_1,w_2,w_3)$ has
the form
   $$s(t_1,t_2,t_3) = t_1 w_1 + t_2 w_2 + t_3 w_3$$
where $t_i>0$ and $t_1+t_2+t_3< 1$.  Similarly,
we define $w'_i$, $S'$, and $s'(t_1,t_2,t_3)$ for the primed objects.

The inequality of the lemma is equivalent to
  $$
  \frac{\sol(0,S')}{\op{vol}(S')} \le \frac{\sol(0,S)}{\op{vol}(S)}.
  $$
By the scaling properties of the measure, $\op{solR}$ and $\op{volR}$ both
scale by the same factor under linear stretching along coordinate axes.
By such a transformation, $S'$ can be transformed to $S$.  The
transformation $T$ is given by 
   $$
   T s'(t_1,t_2,t_3) = s(t_1,t_2,t_3).
   $$
Under this transformation $T$, the volumes become equal.
The desired inequality follows from
   $$
   \sol(0,T S') \le \sol(0,S).
   $$
This follows if the transformation $T$ satisfies
   $T(S'\cap B(0,r))\subset S\cap B(0,r)$.
By Lemma~\ref{tarski:rog-lemma}, we have that 
   $$|s(t_1,t_2,t_3)|\le |s'(t_1,t_2,t_3)|.$$
This means that $T$ carries each point of $S'$ to a point closer to
the origin.  In particular,
  $T(S'\cap B(0,r))\subset S\cap B(0,r)$.
\end{proof}

%% WW Repeated parts of def.
\begin{definition}  Define 
  $$
  \begin{array}{lll}
  \dtet &= \sqrt8 \arctan(\sqrt2/5)\\
  \doct &= \pi/\sqrt8 - \sqrt2 \arctan(\sqrt2/5)\\
  \delta(a,b,c)&= \op{solR}(a,b,c)/(3\op{volR}(a,b,c)),
  \end{array}
  $$
for $a<b<c$.  
\index{ZZdeltatet@$\dtet$}
\index{ZZdeltaoct@$\doct$}
\index{ZZdelta@$\delta$}
\end{definition}

\begin{lemma}\tlabel{lemma:doct-calc}
  $\delta(1,2/\sqrt{3},\sqrt2)=\doct$.
\end{lemma}

\begin{proof}  In this calculation, we do not use Euler's formula
for the solid angle (Lemma~\ref{lemma:euler}).  
Use the dihedral angle formula instead.
A calculation gives
  $$
  \doct(1,2/\sqrt{3},\sqrt2)=
  3 \sqrt{2} \left(\frac{\pi }{4}-\arctan
   \left(\frac{1}{\sqrt{2}}\right)\right).$$
To complete the proof, we need the trig identity
  $$\arctan(\sqrt2/5)  = 3\arctan(\sqrt2/2)-\pi/2.$$
Both sides are between $0$ and $\pi/2$.  Thus, we can prove this
by taking the tangent of both sides. By the addition formula
(Lemma~\ref{lemma:tan-add}),
if $x=\arctan(\sqrt2/2)$, then
   $$\tan(3 x) = \frac{\tan^3(x) - 3\tan(x)}{1-3 \tan^2(x)} = -5/\sqrt2.$$
The result follows.
\end{proof}

\begin{lemma}\tlabel{lemma:dtet-cal}
  $\delta(1,2/\sqrt3,\sqrt6/2)=\dtet$.
\end{lemma}

\begin{proof} Calculating as in Lemma~\ref{lemma:doct-calc}, and using
the same trig identity, we get
$$\begin{array}{lll}
  \delta &=
2\sqrt{2} \left(3\arctan\left(\sqrt2/2\right) - \pi/2)\right),\\
  &=2\sqrt{2}\left(\arctan\left(\sqrt2/5\right)\right),\\
  &=\dtet
\end{array}
$$
\end{proof}

\begin{lemma}\tlabel{lemma:rog-doct}
Suppose $1\le a\le  b\le c$,  $2/\sqrt{3}\le b$, and $\sqrt2\le c$.  Then
$$
\delta(a,b,c) \le \doct.
$$
\end{lemma}

\begin{proof} This follows from Lemma~\ref{lemma:rogers} and
Lemma~\ref{lemma:doct-calc}.
\end{proof}

\begin{lemma}\tlabel{lemma:rog-tet}
Let $1\le a \le b \le c$, $2/\sqrt{3}\le b$ and $\sqrt6/2\le c$.
Then $\delta(a,b,c) \le  \dtet$.
\end{lemma}

\begin{proof}  This follows from Lemma~\ref{lemma:rogers} and
Lemma~\ref{lemma:dtet-cal}.
\end{proof}

By Lemma~\ref{tarski:eta-root3}, the circumradius of a triangle
with sides at least $2$ is always at least $\eta(2,2,2)=2/\sqrt3$.



\subsection{quoin}

Define the function $\op{quovol}$ when $0<a<b<c$ by
    \begin{equation}
    \begin{array}{lll}
    6\,\op{quovol}(a,b,c) &= (a+2c)  %
    % -(a^2+ac-2c^2)
    (c-a)^2\arctan(e)
        +a(b^2-a^2)e\\&-4c^3\arctan(e(b-a)/(b+c)),
    \tlabel{eqn:3.3}
    \end{array}
    \end{equation}
where $e\ge0$ is given by $e^2(b^2-a^2)=(c^2-b^2)$.
Extend the function by $0$, when the condition $0<a<b<c$ fails.


\begin{definition}\label{def:quoin}
Let $\{v_0,v_1,v_2,v_3\}$ be a set of four points in $\ring{R}^3$.
Let $c>  \eta_V(v_0,v_1,v_2)$.  Let $p$ be the circumcenter
of $\{v_0,v_1,v_2\}$.  Let $p'$ be the point 
in $\op{aff}_+(\{v_0,v_1,v_2\},v_3)$ at equal distance $c$
from $v_0,v_1,v_2$ (given by Lemma~\ref{tarski:mk-point}).
We define
$\op{quo}(v_0,v_1,v_2,v_3,c)$ to be the following set 
(Figure~\ref{fig:quoin}):
   $$
   B(v_0,c) \cap \op{aff}_+^0(\{v_0,v_1,v_2\},v_3)
   \cap \op{aff}_-^0(\{v_0,p,p'\},v_1) \cap
   \op{aff}_-^0(\{v_1,p,p'\},v_0).
   $$
We associate with this set the $abc$-parameters, defined
by $a = |v_0-v_1|/2$, $b=|p-v_0|$, $c$ (as given).
\end{definition}

\begin{figure}[htb]
  \centering
  \myincludegraphics{\ps/quoin.eps}
  \caption{The quoin above a Rogers simplex is the part of the
  shaded solid outside
   the illustrated box.  It is bounded by the two
  shaded planes, the plane through
   the front face of the box, and a sphere
   centered at the origin passing through the opposite corner of the box.}
  \tlabel{fig:quoin}
\end{figure}



\begin{lemma}\tlabel{lemma:quo-vol}
Let $Q=\op{quo}(v_0,v_1,v_2,v_3,c)$. Let $(a,b,c)$ be the
$abc$-parameters of $Q$.  
Then $$\op{vol}(Q) = \op{quovol}(a,b,c).$$
%
 \index{quoin}
\end{lemma}

\begin{proof} We give the proof in some detail, because it illustrates
our method of calculating derived volumes from primitive volumes.  Moreover,
this is a key calculation used in several other identities.
The calculations are essentially formal.

Let $\chi_X$ be the characteristic
function of $X$.  If $P = \chi_X$, write $\bar P$ for the characteristic
function of the complement of $X$.  We consider characteristic functions
only up to a null set, and this means that we can ignore issues such
as whether we take open half-spaces or closed half-spaces and so forth.

Set
$$
\begin{array}{llll}
  A &= \chi_X,\quad X = B(v_0,c)&\text{(ball)}\\
  B &= \chi_X,\quad X = \op{aff}_+^0(\{v_0,v_1,v_2\},v_3)&\text{(front)}\\
  C &= \chi_X,\quad X = \op{aff}_-^0(\{(v_0+v_1)/2,p,p'\},v_0)
    &\text{(top)}\\
  D &= \chi_X,\quad X = \op{aff}_-^0(\{v_0,p,p'\},v_1)&\text{(diag)}\\
  E &= \chi_X,\quad X = \op{aff}_+^0(\{v_0,v_1,p'\},p)&\text{(diag)}\\
  F &= \chi_X,\quad X = \op{rcone}^0(v_0,v_1,a/c)\\
\end{array}
$$

We see from Figure~\ref{fig:quoin} that we have the following implications
$(f(x)=1)\implies (g(x)=1)$ when $f,g$ are any of the following characteristic
functions
  $$
   (f,g) = (B\bar C \bar D E,A),\quad
   (B\bar C E F,A),\quad (A B C D, E),\quad (A B C E,F).
  $$
(These implications are justified without pictures in Lemmas~\ref{tarski:BCDE},\ref{tarski:BCEF},\ref{tarski:ABCD}, and~\ref{tarski:ABCE}.)
We recognize $ABEF$ as the characteristic function $[SC]$ of a
conic cap, $B\bar C E F$ as the characteristic function $[WFR]$ of a
wedge of a frustum,  $AB\bar D E$ as the characteristic function $[ST]$
of a solid triangle, $B\bar C\bar D E$ as the characteristic function $[R]$
of a Rogers simplex.  The characteristic function $[Q]$
of the quoin is given
by $A B C D$.  We let $[X]$ be the characteristic function of
$A B C \bar D E$.

We then have formally that
$$
\begin{array}{lllll} \,[SC] &=  A B E F \\
     &= A B \bar C E F &+ A B C E F\\
     &= B \bar C E F &+ A B C D E F &+ A B C \bar D E F\\
     &= [WFR] &+ A B C D &+ A B C \bar D E\\
     &= [WFR] &+ [Q] &+ [X]\\ 
     \,[ST] &= A B \bar D E\\
     &= A B C \bar D E &+ A B \bar C \bar D E\\
     &= [X] &+ B \bar C \bar D E\\
     &= [X] &+ [R].
\end{array}
$$
Solving for $[Q]$, we get
\begin{equation}\tlabel{eqn:qr}
  [Q] = [SC] - [WFR] + [R] - [ST].
\end{equation}
Thus, the volume of a quoin is expressed in terms of primitive volumes.
Substituting the given formulas for the volumes of primitives, we obtain
the result.  (It is necessary to use the Euler formula for solid
angle.)
\end{proof}

\begin{remark}  This type of analysis can be turned into an algorithm
for computing regions described by quadratic constraints in terms
of primitive volumes \cite{quad}.  All the volumes that arise in \cite{DCG}
can be computed by this algorithm.
In particular, the proof of Lemma~\ref{lemma:quo-vol} follows from
the algorithm.
\end{remark}

\begin{remark}\tlabel{rem:RQ}  
We can rewrite Equation~\ref{eqn:qr} as
$$
  [ST]-[R] = ([SC]-[WFR]) - [Q].
$$
That is, as we can see from Figure~\ref{fig:quoin}, the region
in a ball above and outside a Rogers simplex is the same as the
region in a ball above and outside a frustum and outside the quoin.
\end{remark}

\begin{lemma}\tlabel{lemma:solquo}  The quoin
$\quo(v_0,v_1,v_2,v_3,c)$ is eventually radial at $v_0$ and
has solid angle $0$.
\end{lemma}


\begin{proof}  This is trivial, because the quoin is bounded
away from $v_0$.
\end{proof} 

Consider the function
$\op{sovo}(v_0,\quo(v_0,v_1,v_2,v_3,c),\lambda)$, expressed
as a function of the $abc$-parameters of the quoin.  By
Lemma~\ref{lemma:solquo}, the contribution from the solid angle
is zero, so that
$$
\op{sovo}(v_0,\quo(v_0,v_1,v_2,v_3,c),\lambda) = 
 \lambda_v \op{quovol}(a,b,c).
$$


\begin{lemma}\tlabel{lemma:sovo:rog}  Let $R = \op{rog}^0(v_0,
v_1,v_2,v_3,c)$.  Let $(a,b,c)$ be the $abc$-parameters of $R$.
Let $Q=\quo(v_0,v_1,v_2,v_3,c)$.  Let $\op{dih}(R)$ be the
dihedral angle of $R$ along $\{v_0,v_1\}$.
Then
  $$
  \begin{array}{lll}
  \op{sovo}(v_0,R,\lambda) = \sol(v_0,R)\phi(c,c,\lambda) +
  \op{sovo}(v_0,Q) + \op{dih}(R)\,\, A(a,c,\lambda).
  \end{array}
  $$
\end{lemma}

\begin{proof}
By the identity of Remark~\ref{rem:RQ}, we have
  $$
  \op{sovo}(v_0,R,\lambda) - \op{sovo}(v_0,Q) = 
  \op{sovo}(v_0,ST,\lambda) - \op{sovo}(v_0,SC,\lambda) + 
  \op{sovo}(v_0,WFR,\lambda),
  $$
for regions $ST$, $SC$, $WFR$ described in that remark.
We have from previous calculations that
  $$
  \begin{array}{lll}
  \op{sovo}(v_0,ST,\lambda) &= \sol(v_0,ST)\phi(c,c,\lambda) \\ &= 
  \sol(v_0,R)\phi(c,c,\lambda)\\
  \op{sovo}(v_0,WFR,\lambda) &= \op{sol}(v_0,FR)\,\frac{\dih(R)}{2\pi}
   \phi(a,c,\lambda)\\
  \op{sovo}(v_0,SC,\lambda) &= \op{sol}(v_0,FR)\,\frac{\dih(R)}{2\pi}
   \phi(c,c,\lambda)\\ 
  \op{sol}(v_0,FR) (\phi(a,c,\lambda) -\phi(c,c,\lambda)) &=
   2\pi A(a,c,\lambda).
  \end{array}
  $$
The result follows from these equations.
\end{proof}



%\subsection{caps}
%
%\begin{lemma}\tlabel{lemma:cap-rogers}
%Let $B(0,t)$ be a ball of radius $t$ centered at the origin.  Let
%$v_1$ and $v_2$ be vertices.  Assume that $|v_1|< 2t$ and $|v_2|<2
%t$.  Truncate the ball by cutting away the caps
%   $$\op{cap}_i = \{x\in B(0,t) :  |x- v_i| < |x|\}.$$
%Assume that the circumradius of the triangle $\{0,v_1,v_2\}$ is
%less than $t$. Then the intersection of the caps, $\op{cap}_1\cap
%\op{cap}_2$, is the union of four quoins.
%\end{lemma}
%
%\begin{proof} This is true by inspection.  See Figure~\ref{fig:capriquoin}.
%Slice the intersection $\op{cap}_1\cap\op{cap}_2$ into four pieces
%by two perpendicular planes: the plane through $\{0,v_1,v_2\}$,
%and the plane perpendicular to the first and passing through $0$
%and the circumcenter of $\{0,v_1,v_2\}$.  Each of the four pieces
%is a quoin.
%\end{proof}
%
%\begin{figure}[htb]
%  \centering
%  \myincludegraphics{\ps/capriquoin.eps}
%  \caption{The intersection of two caps on the unit ball can
%   be partitioned into four quoins (shaded).}
%  \tlabel{fig:capriquoin}
%\end{figure}
%

\subsection{truncating Rogers}

\begin{lemma}\tlabel{lemma:sovo:truncRog}
Let $R_c=\op{rog}^0(v_0,v_1,v_2,v_3,c)$.  Let $r$ be the
circumradius of $\{v_0,v_1,v_2\}$.  Assume that $r$, $t$,
and $c$ are positive real numbers that satisfy $r, t\le c$.
Let $d$ be the dihedral angle of $R_c$ along the edge $\{v_0,v_1\}$.
Then
   $$
   \op{sovo}(v_0,R_c\cap B(v_0,t),\lambda) = 
   d [(1-\cos\psi)\phi(t,t,\lambda)+A(h,t,\lambda)]
   -s \phi(t,t,\lambda) + \lambda_v \op{quovol}(a,b,t),
   $$
where $s = d (1-\cos\psi) - \sol(v_0,R_c)$,   $h=|v_0-v_1|/2$,
and $\cos\psi = h/c$.
\end{lemma}

\begin{proof}  Let $R_t= \op{rog}^0(v_0,v_1,v_2,v_3,t)$.   Let
$q_t$ (resp. $q_c$) be the point at equidistance $t$ 
(resp. $c$) from $v_0,v_1,v_2$ in
$\op{aff}_+(\{v_0,v_1,v_2\},v_3)$.
If $t\ge r$, then the unique existence of $q_t$ is given by
Lemma~\ref{tarski:rog-exist}.  If $t< r$, set $R_t=\emptyset$ and
$q_t=v_2$.
When $t>r$, the point $q_t$ (resp. $q_c$) is an extreme point of $R_t$ (resp. $R_c$).

Set
  $$
  \begin{array}{lll}
  W_1 &= W(v_0,v_1,v_2,q_t),\\  
  W_2 &= W(v_0,v_1,q_t,q_c).\\
  A &= \op{rcone}^0(v_0,v_1,h/t),\\
  \bar A &= \op{rcone}^0_-(v_0,v_1,h/t).\\
  \end{array}
  $$
We have the following relations by Lemmas~\ref{tarski:wedge-union},
\ref{tarski:rogers-ball}, \ref{tarski:rogers2}, \ref{tarski:rcone-ball}, \ref{tarski:rogers-FR}:
  $$
  \begin{array}{rll}
  R_c \cap (W_1\cup W_2) &\equiv R_c,\\
  W_1\cap W_2 &= \emptyset\\
  R_c \cap B \cap W_1 &= R_t \cap B = R_t\\
  R_c\cap B \cap W_2 \cap A &= FR(v_0,v_1,h,h/t)\cap W_2\\
  \end{array}
  $$
Also, by Lemma~\ref{tarski:rogers-rad},
  $$R_c\cap B\cap W_2\cap \bar A = W_2\cap B\cap \bar A \cap
   \op{aff}_+^0(\{v_0,q_t,q_c\},v_1).$$ 
This is measurable and $t$-radial
at $v_0$.
For simplicity, surpress the parameters $v_0$ and $\lambda$ from
$\op{sovo}$.  It follows from these identities and the
Lemmas~\ref{lemma:sovo:rog} 
and~\ref{lemma:sovoFR} that
  $$
  \begin{array}{lll}
  \op{sovo}(R_c\cap B) &= \op{sovo}(R_c\cap B\cap W_1) + 
  \op{sovo}(R_c\cap B\cap W_2)\\
  &= \op{sovo}(R_t)+ \op{sovo}(R_c\cap B\cap W_2\cap A) +
  \op{sovo}(R_c\cap B\cap W_2\cap \bar A).\\
  \op{sovo}(R_t) &=\sol(R_t)\phi(t,t,\lambda) + \op{sovo}(Q) +
     \dih(R_t) A(h,t,\lambda).\\
  \op{sovo}(R_c\cap B\cap W_2\cap A) &= \op{sovo}(FR(v_0,v_1,h,h/t)\cap W_2)\\
   &= \dih(R_c) A(h,t,\lambda) + \sol(W_2\cap R_c\cap B\cap A)\phi(t,t,\lambda).\\
   \op{sovo}(R_c\cap B\cap W_2\cap \bar A) &= 
        \sol(R_c\cap B\cap W_2\cap \bar A) \phi(t,t,\lambda)\\
   &= (\sol(R_c)-\sol(R_c\cap B\cap W_2\cap A) -\sol(R_t))\phi(t,t,\lambda).
  \end{array}
  $$
Combining these equations, we get the result.
\end{proof}

%\begin{lemma}\label{lemma:truncRog:noQ}
%Let $R=\op{ortho}^0(w_0,v_1,v_2,v_3)$ and $B=B(w_0,t)$.
%Assume that $|v_1|< t < |v_1+v_2|$.  Then
%$$
%  \begin{array}{lll}
%  \op{sovo}(w_0,B\cap R,\lambda) &=
%  \sol(R)\phi(t,t,\lambda) + \dih(R)(1-y/(2t))(\phi(y/2,t,\lambda)
%  -(\phi(t,t,\lambda)))\\
%  &=\sol(R)\phi(t,t,\lambda) + \dih(R) A(y/2,t,\lambda)\\
%  \end{array}
%$$
%\end{lemma}
%
%We note that this formula is that obtained from Lemma~\ref{lemma:sovo:truncRog} by setting the quoin term to zero.
%
%\begin{proof}
%Set
%  $$
%  \begin{array}{lll}
%  A &= \{x \mid (x-v_0)\cdot (v_1-v_0) > |x-v_0| |v_1-v_0| (y/2t)\},\\
%  \bar A &= \{x \mid (x-v_0)\cdot (v_1-v_0) < |x-v_0| |v_1-v_0| (y/2t)\},\\
%  W &= W(w_0,w_0+v_1,w_0+v_1+v_2,w+0+v_1+v_2+v_3\\ 
%  \end{array}
%  $$
%We have that $R\cap (A\cap \bar A)\equiv R$ and $A\cap \bar A=\emptyset$.
%Also, by Lemma~\ref{tarski},
% Put these in Tarski in the Volume section.
%  $$
%  \begin{array}{lll}
%  R\cap B\cap A &= FR(v_0,v_1,h,y/(2t))\cap W\\
%  \end{array}
%  $$
%Also, $R\cap B\cap\bar A$ is measurable and $t$-radial
%at $v_0$.
%
%We have
%  $$
%  \begin{array}{lll}
%  \op{sovo}(B\cap R) &= \op{sovo}(A\cap B\cap R) + \op{sovo}(\bar A\cap B\cap R) \\
%   \op{sovo}(\bar A\cap B\cap R) &= \sol(\bar A\cap B\cap R) \phi(t,t,\lambda)\\
%    &= (\sol(B\cap R)-\sol(A\cap B\cap R))\phi(t,t,\lambda)\\
%    &= (\sol(B\cap R)-\sol(FR\cap W))\phi(t,t,\lambda)\\
%  \op{sovo}(A\cap B\cap R) &= \op{sovo}(FR\cap W)\\
%     &=\sol(FR\cap W)\phi(y/2,t,\lambda)\\
%   \sol(FR\cap W) &= \dih(R)\sol(FR)/(2\pi)\\
%  \sol(FR) = 2\pi (1-y/(2t))\\
%  \end{array}
%  $$
%These identities combine to give the proof.
%\end{proof}
%

\section{Composite Regions}\tlabel{sec:tcc}
    %\oldlabel{4.10}

We will consider in this section several different types
of regions that are composites of pieces that have already been
considered.  The various regions share certain features.

The constructions will depend on points
 $v_0,v_1,w_1,w_2\in\ring{R}^3$. Assume that $\{v_0,v_1,w_1,w_2\}$
is not coplanar.  We also choose constants $c_1,c_2>0$.
Let
$W=W(v_0,v_1,w_1,w_2)$ be the corresponding wedge.  
Let $p_i$ be the circumcenter of
$\{v_0,v_1,w_i\}$ and $b_i$ its circumradius.
Let $u_i$ be the normal to $\{v_0,v_1,w_i\}$, directed so that
$(-1)^i\azim(v_0,v_1,w_i,w_i+u_i) < 0$.  That is, $u_i$ points
from the half-plane $\op{aff}_+^0(\{v_0,v_1\},w_i)$ into the wedge $W$.
For $c_i > b_i$, let $q_i = q(c_i) = p_i + s_i u_i$ (with $s_i>0$) be the
unique point that is equidistant $c_i$ from $v_0,v_1,w_i$.
(The unique existence of $q_i$ is established by Lemma~\showref{tarski:rog-exist}.)
If $c_i\le b_i$, we set $c_i=w_i$.
By the choice of normals $u_i$, we have that
 $$
 0\le \op{azim}(v_0,v_1,w_1,q_1), \text{ and }
 \op{azim}(v_0,v_1,w_1,q_2) \le \op{azim}(v_0,v_1,w_1,w_2).
 $$
Equality occurs exactly when $c_1\le b_1$ (resp. $c_2\le b_2$).
We make the assumption that
\begin{equation}\tlabel{eqn:q1q2}
\op{azim}(v_0,v_1,w_1,q_1) \le \op{azim}(v_0,v_1,w_1,q_2).
\end{equation}
(This assumption will be briefly lifted in Section~\showref{sec:inverted}.)

We define the wedges 
$$
   \begin{array}{lll}
   W &= W(v_0,v_1,w_1,w_2)\\
   W_1 &= W(v_0,v_1,w_1,q_1)\\
   W_2 &= W(v_0,v_1,q_2,w_2)\\
   W'' &= W(v_0,v_1,q_1,q_2)\\
   \end{array}
$$
Under Assumption~\ref{eqn:q1q2}, we have that
$W_1,W_2,W''$ are mutually disjoint, and that
   $$
   W \equiv W_1 \cup W_2 \cup W''.
   $$
Define Rogers simpices for $i=1,2$:
  $$
  \begin{array}{lll}
  R_i &= \op{ortho}^0(v_0,v_1,w_i,w_i+u_i,c_i)\\
  R_i'&= \op{ortho}^0(v_0,w_i,v_1,v_1+u_i,c_i)\\
  \end{array}
  $$
The $abc$-parameters of $R_i$ are $a=|v_1-v_0|/2$, $b_i$, $c_i$.
Those of $R_i$ are $a'_i=|w_i-v_0|/2$, $b_i$, $c_i$.
The dihedral angle of $R_i$ along the edge $\{v_0,v_1\}$ is
equal to the dihedral angle (and azimuth angle) of $W_i$.
We have $R_i,R_i'\subset W_i$.

The composites that we consider will be constructed in various
ways from $R_i,R'_i$ and regions $W''\cap X$, for various regions
$X$ that are rotationally symmetric along the line $\op{aff}\{v_0,v_1\}$.
In the following subsections, we specialize this general
context to various specific composites.

\subsection{plates}

The first composite that we consider will be called a plate.

\begin{definition}\tlabel{def:plate}
Let $v_0,v_1,w_1,w_2$ be points in $\ring{R}^3$ that are not
coplanar.  Let $t > 0$.  Choose $c_1=c_2=t$ and construct
the regions $R_1,R_2,W'',\ldots$ for the parameters 
$v_0,v_1,w_1,w_2,c_1,c_2$.
Define the plate in terms of the regions $R_1,R_2,W''$ as follows
  $$
  PL(v_0,v_1,w_1,w_2,t) = 
  R_1 \cup (W''\cap FR(v_0,v_1,h,h/t))\cup R_2,
  $$
where $h = |v_1-v_0|/2$.
\end{definition}

Under Assumption~\ref{eqn:q1q2}, the regions $R_1,R_2,(W''\cap FR)$
are disjoint and have the appearance of Figure~\ref{fig:plate}.
According to construction, if $t$ is less than the circumradius
of $\{v_0,v_1,w_i\}$, then the corresponding Rogers simplex $R_i$
is empty, the wedge $W_i$ is empty, and that piece can be dropped
from the description of the plate.

% Not yet sketched.
\begin{figure}[htb]
  \centering
  \myincludegraphics{noimage.eps}
  \caption{A plate}
  \tlabel{fig:plate}
\end{figure}


\begin{lemma}
Let $v_0,v_1,w_1,w_2$, $h,t$, be given as in the definition of
the plate. Let $q_i$ be the point constructed above. 
Assume that $h\le t$.  
Let $PL=PL(v_0,v_1,w_1,w_2,t)$.
Let $Q_i=\quo^0(v_0,v_1,w_i,q_i,t)$.
Then for all $\lambda$, we have
  $$
  \op{sovo}(v_0,PL,\lambda) = 
  \sol(v_0,PL)\phi(t,t,\lambda) + 
  \sum_{i=1}^2\op{sovo}(v_0,Q_i,\lambda) +
  \op{azim}(v_0,v_1,w_1,w_2) A(h,t,\lambda).
  $$
\end{lemma}

\begin{proof} We have already calculated the function $\op{sovo}$
on the individual pieces $R_i$ and $W''\cap FR$.  The result
follows by assembling the pieces into the composite and 
Lemmas~\ref{lemma:sovo:rog} and \ref{lemma:sovoFR}:
  $$
  \begin{array}{lll}
  \op{sovo}(PL) &= \op{sovo}(R_1) + \op{sovo}(R_2) + \op{sovo}(W''\cap FR) \\
  \op{sovo}(R_i) &= \op{sol}(R_i) \phi(t,t,\lambda) +\op{sovo}(Q_i)
   + \dih(R_i) A(h,t,\lambda)\\
  \op{sovo}(W''\cap FR) &= \op{sol}(W''\cap FR)\phi(t,t,\lambda) +
   \op{azim}(W'') A(h,t,\lambda)\\
  \sol(PL) &=\sol(R_1)+\sol(R_2)+\sol(W''\cap FR)\\
  \op{azim}(v_0,v_1,w_1,w_2) &= \dih(R_1)+\dih(R_2)+\op{azim}(W'').
  \end{array}
  $$
The result follows immediately from these equations.
\end{proof}



\subsection{corner cells}

Let $\{v_0,v_1,w_1,w_2\}$ be a set of four points in $\ring{R}^3$.  
Assume that $\{v_0,v_1,w_1,w_2\}$ is not coplanar.
We
attach a {\it corner cell} $CC(v_0,v_1,w_1,w_2,t,\mu)$
to these four points and positive
real parameters $t,\mu$.  Let $2h=|v_0-v_1|$
 %$b=\eta(2h,t,\mu)$, 
and $\psi=\arc(2h,t,\mu)$.

Construct the cone $C=\op{rcone}^0(v_0,v_1,\cos\psi)$.
Let $P$ be the half-space containing $v_0$ bounded by
the perpendicular bisector of  $\{v_0,v_1\}$.  Set
$$
  \begin{array}{lll}
  CC_1 &= C\cap P \cap B(v_0,t)\\
  W &=W(v_0,v_1,w_1,w_2) \\
  CC(v_0,v_1,w_1,w_2,t,\mu) &= CC_1 \cap W
  \end{array}
$$

\begin{lemma}\tlabel{lemma:sovo:CC} 
Suppose that $h/t \ge \cos\psi$.  Let the other notation
be as above.   We have
  $$
  \op{sovo}(v_0,CC(v_0,v_1,w_1,w_2,t,\mu),\lambda)=
  \op{azim}(v_0,v_1,w_1,w_2) \left((1-\cos\psi)\phi(t,t,\lambda)+
    A(h,t,\lambda)\right).
  $$
\end{lemma}

\begin{proof}
Set
$$
 \begin{array}{lll}
 A &= \op{rcone}^0(v_0,v_1,h/t),\\
  \bar A &= \op{rcone}^0_-(v_0,v_1,h/t),\\
 W &= W(v_0,v_1,w_1,w_2)\\
 \end{array}
$$
It follows from the definitions and from the
fact that $\partial\op{rcone}(v_0,v_1,h/t)$
is a null set that
we have $A\cap \bar A = \emptyset$ and 
$A\cup \bar A \equiv \ring{R}^3$.
By Lemma~\ref{tarski:CC} and Lemma~\ref{tarski:CCbar}, we have 
$$
  \begin{array}{lll}
    A\cap CC &= W \cap FR(v_0,v_1,h,h/t)\\
    \bar A \cap CC &= \bar A \cap C \cap B(v_0,t).
  \end{array}
$$
This last set is
is $t$-radial and measurable.
It follows from Lemma~\ref{lemma:sovoFR} that
$$
  \begin{array}{lll}
  \op{sovo}(CC) &= \op{sovo}(A\cap CC) + \op{sovo}(\bar A\cap CC)\\
  \op{sovo}(A\cap CC) &= \op{sovo}(W\cap FR) \\
     &= \op{azim}(W) A(h,t,\lambda) + \sol(W\cap FR(v_0,v_1,h,h/t))\phi(t,t,\lambda)\\
  &=\op{azim}(W) A(h,t,\lambda) + (\sol(CC)-\sol(\bar A \cap CC))\phi(t,t,\lambda)\\
  \op{sovo}(\bar A \cap CC) &= \sol(\bar A \cap CC)\phi(t,t,\lambda)\\
  \sol(CC) &= \op{azim}(W) \sol(SC(v_0,v_1,t,\cos\psi))/(2\pi) \\
         &=\op{azim}(W) (1-\cos\psi)\\
  \end{array}
$$
The result follows immediately from these equations.
\end{proof}

\subsection{truncated corner cells}
%\subsection{Formulas for Truncated corner cells}
\tlabel{sec:ftcc}
    %\oldlabel{4.11}



%Starting from the corner cell $CC(v_0,v_1,w_1,w_2,t,\mu)$, 
We define a subset $TCC(v_0,v_1,w_1,w_2,t,\mu)$ of the corner cell
called
the {\it truncated corner cell}.  
We use the construction of $b_i$, $u_i$, $p_i$, $q_i$, $R_i$,
$W_1$, $W_2$, $W''$, $\ldots$ from Section~\ref{sec:tcc},
associated with the parameters $v_0,v_1,w_1,w_2$ and
$c_1=c_2 = h/\cos\psi$, where $h = |v_1-v_0|/2$.  
Assumption~\ref{eqn:q1q2} remains
in force.  Let $B  = B(v_0,t)$.

\begin{definition} We define the truncated corner cell
to be
$$
TCC(v_0,v_1,w_1,w_2,t,\mu) =
(R_1\cap B)\cup (R_2\cap B) \cup (W''\cap CC(v_0,v_1,w_1,w_2,t,\mu)).
$$
\end{definition}

%Let $y=y_1=2h$.
The points $p_i$ and
 $q_i$ are construction in Section~\ref{sec:tcc}.
We also need the Rogers simplices 
$R''_i= \op{rog}^0(v_0,v_1,w_i,q_i,t)$ based on the parameter
$t$, rather than $c_i$.  The Rogers simplex $R''_i$
has $abc$-parameters
$(h,b_i,t)$.  

Let $s(y_1,y_2,y_3,t,\lambda)$ be given by the following formula.
Let $b = \eta(y_1,y_2,y_3)$, $\psi = \arc(y_1,t,\lambda)$,
and $c = y_1/(2\cos\psi)$ in
  $$
  s(y_1,y_2,y_3) = \op{dihR}(h,b,c) (1-\cos\psi) - \op{solR}(h,b,c).
  $$


\begin{lemma}\label{lemma:tcc}  
Let $TCC=TCC(v_0,v_1,w_1,w_2,t,\mu)$ be as constructed.
Let $p_i$ be the circumcenter constructed in Section~\ref{sec:tcc}.
%Let $q_i(t)$  X
As usual, assume Condition~\ref{eqn:q1q2}.  
Let $y = 2h = |v_1-v_0|$.  
Let $s_i = s(y,|w_i-v_0|,|v_1-w_i|,t,\lambda)$, for $i=1,2$.
Assume that
$h/t \ge \cos\psi$.  
Let $Q_i = \op{quo}^0(v_0,v_1,w_i,q_i,t)$.  Then
  \begin{equation}\label{eqn:tcc}
  \begin{array}{lll}
  \op{sovo}(v_0,TCC,\lambda) &= 
  \op{azim}(v_0,v_1,w_1,w_2) \left((1-\cos\psi)\phi(t,t,\lambda)+
    A(y/2,t,\lambda)\right) \\
    &\quad + \op{sovo}(v_0,Q_1,\lambda) - s_1\phi(t,t,\lambda) \\
    &\quad + \op{sovo}(v_0,Q_2,\lambda) - s_2\phi(t,t,\lambda) \\
  \end{array}
  \end{equation}
\end{lemma}

\begin{proof}  The result follows by combining the function
$\op{sovo}$ on each of the pieces of the composite $TCC$.
By Lemmas~\ref{lemma:sovo:CC} and~\ref{lemma:sovo:truncRog},  
% noQ
we have
$$
\begin{array}{lll}
  \op{sovo}(TCC) &= \op{sovo}(W''\cap TCC) + \op{sovo}(W_1\cap TCC)
  + \op{sovo}(W_2\cap TCC)\\
  &=\op{sovo}(W''\cap TCC) + \op{sovo}(R_1\cap B)
  + \op{sovo}(R_2\cap B)\\
  &= \op{azim}(W'') \left((1-\cos\psi)\phi(t,t,\lambda)+
    A(h,t,\lambda)\right) 
   + \op{sovo}(R_1\cap B)
  + \op{sovo}(R_2\cap B)\\
  \op{azim}(W'') &= \op{azim}(v_0,v_1,w_1,w_2)-\dih(R_1)-\dih(R_2)\\
  \op{sovo}(R_i\cap B) &= \dih(R_i) [(1-\cos\psi)\phi(t,t,\lambda)+A(h,t,\lambda)]\\
   &\quad -(\dih(R_i) (1-\cos\psi) - \sol(R_i)) \phi(t,t,\lambda) + 
   \op{sovo}(v_0,Q_i,\lambda).
\end{array}
$$
The result follows from these equations and the definition of
the function $s$.
\end{proof}



We give a second expession for the truncated corner cell.
Let $A_i^\pm$ be the half-spaces
$$
  A_i^\pm = \op{aff}_\pm^0(\{v_0,p_i,q_i\},v_0).
$$
Let $L_i^\pm = CC(v_0,v_1,w_1,w_2,t,\lambda)\cap A_i^\pm$.

\begin{lemma}\label{lemma:LL}
We have
  $$
  TCC(v_0,v_1,w_1,w_2,t,\lambda) = L_1^+\cap L_2^+.
  $$
Moreover, 
$L_1^-\cap L_2^- =\emptyset$.
\end{lemma}

\begin{proof}
The second statement follows from the containment
   $L_i^- \subset W_i$, because, as we have seen,
$W_1$
from $W_2$.
Consider the first statement.  
We have
 $$
 W'' \cap TCC = W'' \cap CC = W'' \cap L_1^+ \cap L_1^-.
 $$
The first equality holds by construction of the truncated corner
cell.  The second equality holds because the parts $L_i^-$ excised
from $CC$ to form $TCC$ are contained in $W_i$.
We also have
  $$
  W_i \cap TCC = R_i \cap B(v_0,t) = W_i  \cap L_i^+.
  $$
%% XX IS THIS A NEW TARSKI??
The result follows.
\end{proof}




\subsection{inverted truncated corner cells}\tlabel{sec:inverted}
%\subsection{Analytic continuation} %DCG 13.3, p144
\oldlabel{5.3}

In this section, we develop a formula for $\op{sovo}$ on 
a truncated corner cell that remains valid if the 
Assumption~\ref{eqn:q1q2} is not in force.
When Assumption~\ref{eqn:q1q2} does not necessarily hold, we
define $TCC(v_0,v_1,w_1,w_2,t,\lambda) = L_1^+ \cap L_2^+$.
By Lemma~\ref{lemma:LL}, 
this is a compatible extension of the definition
of truncated corner cells, up to a null set.
The proof that $L_1^+\cap L_2^- = \emptyset$ 
relies on Assumption~\ref{eqn:q1q2}.


\begin{lemma} Let $\op{sovo}^g(v_0,TCC,\lambda)$ be given
by the right-hand side of Equation~\ref{eqn:tcc}.
If $\phi(t,t,\lambda) > 0$, then
   $$
   \op{sovo}(v_0,TCC,\lambda) > \op{sovo}^g(v_0,TCC,\lambda).
   $$
Furthermore, the sign is reversed if $\phi(t,t,\lambda) < 0$.
\end{lemma}

\begin{proof}
We have by inclusion-exclusion, the formula
$$
\begin{array}{lll}
\op{sovo}(v_0,TCC,\lambda) &=
\op{sovo}(v_0,CC,\lambda) -
\op{sovo}(v_0, L_1^-,\lambda) -
\op{sovo}(v_0, L_2^-,\lambda) \\
 &\quad +
\op{sovo}(v_0, L_1^- \cap L_2^-,\lambda).
\end{array}
$$

If we compare this formula with the calculations for 
$\op{sovo}(v_0,TCC,\lambda)$ in Lemma~\ref{lemma:tcc}, 
we find that, in the notation of that lemma:
$$
\op{sovo}(v_0,CC\cap L_i^-) = \op{sovo}(v_0,Q_i,\lambda) - s_i.
$$

Thus, the formula for $\op{sovo}(TCC)$ in the general case,
differs from the formula under Assumption~\ref{eqn:q1q2} through
the term $\op{sovo}(v_0,L_1^-\cap L_2^-,\lambda)$.

We note that $L_1^-\cap L_2^-$ is $t$-radial 
(Lemma~\ref{tarski:rCCinvert-rad}).  Thus,
$$
\op{sovo}(v_0,L_1^-\cap L_2^-,\lambda) =
\sol(v_0, L_1^-\cap L_2^-)\phi(t,t,\lambda)
$$
and
$$
\op{sovo}(v_0,TCC,\lambda) = \op{sovo}^g(v_0,TCC,\lambda) +
   \sol(v_0,L_1^-\cap L_2^-) \phi(t,t,\lambda).
$$
In particular, the direction of the inequality between
$\op{sovo}$ and $\op{sovo}^g$ is determined by the sign
of $\phi(t,t,\lambda)$.
\end{proof}



\subsection{overlapping truncated corner cells}
%\subsection{More on Truncated Corner Cells}





\begin{lemma}\tlabel{lemma:2tcc} 
For $i=1,2$, let $$CC_i =CC(v_0,v_i,w_i,u_i,t,\mu)$$ and 
be untruncated corner cells, both
with azimuth angle at least $\pi$ and parameters $t=1.255$ and 
$\mu=1.945$.
Assume $|v_1-v_2|\ge 3.2$.  Let $y_i =|v_i-v_0|$ and
Suppose that
$2\le y_i\le 2t$.  Then
$\op{sovo}(v_0,CC_1\cup CC_2,\lambda_{sq}) > 0.8862$ 
\end{lemma}

%% WW Recheck proof. Constant was squander + pimax

\begin{proof}
%Suppose first that $CC_1$ and $CC_2$ are disjoint.
%Following the proof of Lemma~\ref{lemma:CC815}, 
%but with parameter $\mu=1.945$, a lower bound on
%$\op{sovo}$ is obtained when  $y=2t$,
%$\op{azim}=\pi$. The explicit formulas give 
%  $$\op{sovo}(v_0,C,\lambda) > 0.734$$
%for $C=CC,CC'$.
%The result follows in this case.
%
%Suppose that $CC$ meets $CC'$. 
We have
 $$
 \op{sovo}(CC_1\cup CC_2)=
 \op{sovo}(CC_1)+\op{sovo}(CC_2)-\op{sovo}(CC_1\cap CC_2).
 $$
It follows from Lemma~\ref{tarski:2CCrad} that
$CC_1\cap CC_2$ is $t$-radial at $v_0$.  Thus,
$$\op{sovo}(v_0,CC_1\cap CC_2,\lambda) =
  \sol(v_0,CC_1\cap CC_2)\phi(t,t,\lambda).$$

Let $q$ and $q'$ be the two points defined by distances
$t$ from $v_0$, $\mu$ from $v_1$, and $\mu$ from $v_2$.
The existence of such points is given by Lemma~\ref{tarski:mk-point}.
Let $A=\op{aff}_+(v_0,\{q,v_1,v_2\})$ and
$A'=\op{aff}_+(v_0,\{q',v_1,v_2\})$.
Let $rc_i = \op{rcone}^0(v_0,v_i,y_i/(2\cos\psi_i))$.
By Lemma~\ref{tarski:AA'}, we have the relations
$$
\begin{array}{lll}
CC_1\cap CC_2 &\equiv (A\cap CC_1\cap CC_2) \cup (A'\cap CC_1\cap CC_2).\\
\sol(v_0,A\cap CC_1\cap CC_2) &= \sol(v_0,A\cap CC_1) + \sol(v_0,A\cap CC_2)
  -\sol(v_0,A)\\
    &\le \sol(v_0,A\cap rc) + \sol(v_0,A\cap rc') - \sol(v_0,A)\\
    &= \sol(v_0, A \cap rc \cap rc').
\end{array}
$$
The constant $\phi(t,t,\lambda)$ is positive.  Thus, we
get a lower bound on $\op{sovo}(CC_1\cup CC_2)$ by taking 
the intersection $rc \cap rc'$ to be as large as possible.
By Lemma~\ref{tarski:rcone2}, 
this happens when $v_1$ is as close to $v_2$ as possible:
$|v_1-v_2|=\ell=3.2$.  Assume this.

The bound is now easily estimated in terms of primitive
regions.  Adding the similar
term for $A'$, we get
a function $f(y_1,y_2)$ that gives a lower bound on 
$\op{sovo}(CC_1\cup CC_2)$.  Recall that $\lambda=\lambda_{sq}$.
    $$
    \begin{array}{lll}
    \alpha_1 &= \dih(y_1,t,y_2,\mu,\ell,\mu),\\
    \alpha_2 &= \dih(y_2,t,y_1,\mu,\ell,\mu),\\
    \sol &= \sol(y_2,t,y_1,\mu,\ell,\mu),\\
    \phi_i &= \phi(y_i/2,t,\lambda),\quad i=1,2,\\
    f(y_1,y_2)&=
    2\phi(t,t,\lambda_{sq})\sol+
    2\sum_1^2 \alpha_i(1-y_i/(2t))(\phi(t,t,\lambda)-\phi_i)\\
        &\quad +
       \sum_1^2 \op{sovo}(v_0,CC_1(y_i,\pi-2\alpha_i,t,\mu),\lambda).
    \end{array}
    $$
Here $CC(y,\beta,t,\mu)$ is any corner cell
$$CC(v_0,v_1,w_1,w_2,t,\mu)$$ with $|v_1-v_0|=y$ and
$\op{azim}(v_0,v_1,w_1,w_2)=\beta$.
An interval calculation\footnote{\calc{984628285}} %A14
gives $f(y_1,y_2)>0.8862$, for $y_1,y_2\in[2,2t]$.
\end{proof}




%\section{Scores of Simplices and Cones}


%\begin{remark}\tlabel{remark:vor}\index{vor}\index{c-vor}\index{score}
% Deleted function {c-vor} that has been replaced by sovo

%\label{eqn:3.2} deleted. It should be replaced by sovo(FR) formula ref.

%\section{The Function K}
%\tlabel{sec:K} %DCG p105-106.
%% No longer used.  Proof of -1.04 lemma was rewritten.
%%
%
%We define a function $K(S)$ on
%certain simplices $S$ with circumradius at least $\sqr2$. Let
%$S=S(y_1,y_2,\ldots,y_6)$.  Let $R(a,b,c)$ denote a Rogers
%simplex. Set
%    \begin{equation}
%    K(S) = K_0(y_1,y_2,y_6)+K_0(y_1,y_3,y_5)
%    + \dih(S)(1-y_1/\sqr8) \phi(y_1/2,\sqr2),
%    \tlabel{eqn:KS}
%    \end{equation}
%where
%    $$
%    $$
%(If the given Rogers simplices do not exist because the condition
%$0<a<b<c$ is violated, we set the corresponding terms in these
%expressions to 0.) The function $K(S)$ represents the part of the
%score coming from the four Rogers simplices along two of the faces
%of $S$, and the conic region extending out to $\sqr2$ between the
%two Rogers simplices along the edge $y_1$ (Figure~\ref{fig:KS}).
%This region is closely related to the regions $\BigD(v,W)$ of
%Definition~\ref{def:delta-e}, with the difference that the regions
%$\BigD$ lie in a ball of radius $\eta_0(|v|/2)$, but the regions
%here are truncated at $\sqrt2$.
%
%\begin{figure}[htb]
%  \centering
%  \myincludegraphics{\ps/diag43.ps}
%  \caption{The set measured by the function $K(S)$.}
%  \tlabel{fig:KS}
%\end{figure}
%
%

\subsection{crowns}
%\section{The Function anc}
\tlabel{sec:anc} %DCG p 107.

This subsection considers
one final composite region.
Let $\eta(x,y,z)$ be the circumradius of a triangle with
sides $x,y,z$ and let $\eta_0(h,t) = \eta(2,2h,2t)$.

We return to the context established at the beginning
of Section~\ref{sec:tcc}.  We use the parameters $v_0,v_1,w_1,w_2$
and $c=c_1=c_2=\eta_0(h,t)$, where $h =|v_0-v_1|/2 \le t$. 
Assumption~\ref{eqn:q1q2} remains in force.
Let $W_1,W_2,W'',W,R_i,R_i',p_i,q_i$ be as given at the
beginning of Section~\ref{sec:tcc}.
%
Let 
  $$\bar B = \bar B(v_0,t) = \{x \mid |x-v_0| > t\},$$
the complement of a closed ball of radius $t$ at $v_0$.
%
We define the fitted crown to be
$$
FCR(v_0,v_1,w_1,w_2,t) =
  \left((W'' \cap FR(v_0,v_1,h,h/c)) \cup
  R_1 \cup R_2  \cup
  R_1' \cup R_2'\right) \cap \bar B.
$$
As described at the beginning of the section, the 
sets $R_i\cap \bar B$ and $R_i'\cap \bar B$ are empty
(and also $W_i=\emptyset$) when the circumradius of 
$\{v_0,v_1,w_i\}$ is greater than $c=c_i$.

We define some functions that will be used in a formula
for the value of $\op{sovo}$ on a fitted crown.
Define
\begin{equation}\cro(h,t,\lambda) =
2\pi(1-h/\eta_0(h,t))(\phi(h,\eta_0(h,t),\lambda)-\phi(t,t,\lambda)). 
\end{equation} 

\begin{lemma}\label{lemma:sovo:CR} 
Let $t$ and $h$ be real numbers satisfying 
$0 < t \le h$.
Let $b=h/\eta_0(h,t)$.
Let $CR=FR(v_0,v_1,h,b) \setminus B(v_0,t)$.
  Then
$$\op{sovo}(v_0,CR,\lambda) = \cro(h,t,\lambda).$$
\end{lemma}

\begin{proof}  Let $RC=\op{rcone}^0(v_0,v_1,h,b)$.
Then by Lemma~\ref{tarski:RCFR}, we have
$B' = B(v_0,t)\cap RC = FR\cap B(v_0,t)$.
Furthermore, $B'$ is $t$-radial with solid angle equal to that
of $FR$.  Thus, by Lemma~\ref{lemma:sovoFR},
$$
\begin{array}{lll}
\op{sovo}(CR) &= \op{sovo}(FR) - \op{sovo}(B')\\
 &= \op{sol}(FR) (\phi(h,b,\lambda) - \phi(t,t,\lambda))\\
 &= \cro(h,t,\lambda).
\end{array}
$$
\end{proof}

Similarly, if $WCR = W(v_0,v_1,w_1,w_2) \cap CR$, then
$$
\op{sovo}(v_0,WCR,\lambda) = \op{azim}(v_0,v_1,w_1,w_2)\cro(h,t,\lambda)/(2\pi),
$$
because $CR$ 
is rotationally symmetric about the axis $\op{aff}\{v_0,v_1\}$.


%\begin{figure}[htb]
%  \centering
%  \myincludegraphics{\ps/diag44.ps}
%  \caption{An illustration of the terms $\anc$.}
%  \tlabel{fig:anchor}
%\end{figure}

Let $\op{dihR}(a,b,c)$ be the dihedral angle along the edge
$\{v_0,v_1\}$ of a
Rogers simplex $\op{rog}^0(v_0,v_1,v_2,v_3,c)$ with $abc$-parameters
$(a,b,c)$.  Similarly, let $\op{solR}(a,b,c)$ (resp. $\op{sovoR}(a,b,c,\lambda)$)
be the solid angle (resp. value of $\op{sovo}$)
at $v_0$ of such a Rogers simplex.  By Lemma~\ref{lemma:rog:abc},
these values depend only on the $abc$-parameters.
Set
    \begin{equation}
    \begin{array}{lll}
    \anc(y_1,y_2,y_6,t,\lambda) &= 
     -\op{dihR}_1\cro(y_1/2,t,\lambda)/(2\pi)
       %
    -\op{dihR}_2\, A(y_2/2,t,\lambda) \\
        %(1-y_2/(2t))(\phi(y_2/2,t,\lambda)-\phi(t,t,\lambda))
      &+\sum_{i=1}^2 (\op{sovoR}_i - \op{solR}_i \phi(t,t,\lambda))\\
       %-\op{solR}_1\phi(t,t,\lambda)+\op{sovoR}_1\\
       % -\op{solR}_2\phi(t,t,\lambda) + \op{sovoR}_2,
    \tlabel{eqn:4.5}
    \end{array}
    \end{equation}\index{anc@$\anc$}
where $\op{dihR}_i$, $\op{solR}_i$, $\op{sovoR}_i$ are the values
of $\op{dihR}$, $\op{solR}$, and $\op{sovoR}(\cdot,\lambda)$
at $(a_i,b,c) = (y_i/2,\eta(y_1,y_2,y_6),\eta_0(y_1/2,t))$, for $i=1,2$.
Recall that the terms are defined as zero if the inequalities
$0 < a_i \le b\le c$ are violated.  Hence, the function $\anc$ is
zero, except at points in a certain domain.

%% Moved from DCG 11.2 Contexts.
Set
    $$\kappa(y_1,y_2,y_3,y_5,y_6,\alpha,t,\lambda) =
   \alpha\,\cro(y_1/2,t,\lambda)/(2\pi) +
        \anc(y_1,y_2,y_6,t,\lambda)+\anc(y_1,y_3,y_5,t,\lambda).
    $$
    \index{zzkappa@$\kappa$}
We are finally ready to state the main result about fitted crowns.
Assumption~\ref{eqn:q1q2} remains in force.

\begin{lemma}\label{lemma:sovo:FCR}
Let $\{v_0,v_1,w_1,w_2\}$ be a set of four points in $\ring{R}^3$.
Assume the set is not planar.
Let $0 < t < h$, where $h = |v_1-v_0|/2$.
Set $\alpha = \op{azim}(v_0,v_1,w_1,w_2)$.
Let 
 $$(y_1,y_2,z_2,z_1) =
   (|w_1-v_0|,|w_2-v_0|,|w_2-v_1|,|w_1-v_1|).
 $$
Then
$$
\op{sovo}(v_0,FCR(v_0,v_1,w_1,w_2,t),\lambda) =
 \kappa(2h,y_1,y_2,z_2,z_1,\alpha,t,\lambda).
$$
\end{lemma}

\begin{proof}
Let $B = B(v_0,t)$.  We have
$$R\equiv (\bar B\cap R) \cup (B\cap R),\quad B\cap \bar B = \emptyset,
$$
for $R=R_i,R'_i$.  Moreover, $B\cap R_i$ is $t$-radial at $v_0$.
Thus, 
 $$
\begin{array}{lll}
 \op{sovo}(\bar B\cap R_i) &= \op{sovo}(R_i) - \op{sovo}(B\cap R_i)\\
 &= \op{sovo}(R_i) - \sol(R_i)\phi(t,t,\lambda) \\
 &= \op{anc}(2h,y_i,z_i) \\
    &\quad + \dih(R_i)\cro(h,t,\lambda)/(2\pi) + \dih(R'_i) A(y_i/2,t,\lambda)\\
  &\quad -(\op{sovo}(R_i') -\sol(R_i')\phi(t,t,\lambda)). \\
\end{array}
 $$
We have by Lemma~\ref{lemma:sovo:truncRog}, 
$$
\begin{array}{lll}
\op{sovo}(\bar B\cap R_i') &= \op{sovo}(R'_i) - \op{sovo}(B\cap R'_i)\\
\op{sovo}(B\cap R'_i) &= \sol(R'_i)\phi(t,t,\lambda) + \dih(R'_i) A(y_i/2,t,\lambda)\\
\end{array}
$$
We have by Lemma~\ref{lemma:sovo:CR} that
$$
\begin{array}{lll}
\op{sovo}(FCR) &= \op{sovo}(W''\cap FCR) + \op{sovo}(W_1\cap FCR)
 +\op{sovo}(W_2\cap FCR).\\
\op{sovo}(W_i\cap FCR) &= \op{sovo}(\bar B\cap R_i) + \op{sovo}(\bar B\cap R'_i).\\
 \op{sovo}(W''\cap FCR) &= \op{azim}(W'')\cro(h,t,\lambda)/(2\pi),\\
\op{azim}(v_0,v_1,w_1,w_2)&= \op{azim}(W'')+\dih(R_1)+\dih(R_2).\\
\end{array}
$$
These equations give the lemma.
\end{proof}

\begin{lemma}  The solid angle of $FCR(v_0,v_1,w_1,w_2,t,\lambda)$
is zero at $v_0$.
\end{lemma}

\begin{proof}  The region $FCR$ is contained in the complement
of the ball of radius $t>0$ at $v_0$.  It is bounded away from
zero.  Hence, its solid angle is zero.
\end{proof}

It follows that $\op{sovo}(v_0,FCR,\lambda) = \lambda_v\op{vol}(FCR)$,
where $\lambda=(\lambda_v,\lambda_s)$.



\section{Finiteness and Volume}

We have now developed all of the volume calculations that will
be needed in this book.   We finish this chapter with some 
elementary estimates based on the volumes of  cubes and balls.

\begin{lemma}\tlabel{lemma:Zcount}
    For all $p\in\ring{R}^3$ and all $r\ge 0$, the set
    $\ring{Z}^3\cap B(p,r)$ is finite of cardinality at most
    $4\pi (r+\sqrt3)^3/3$.
\end{lemma}

\begin{proof}  If $v\in\ring{Z}^3\cap B(p,r)$, then the ith
coordinate $v_i$ of $v$ must lie in the finite range
    $$
    p_i - r \le v_i \le p_i + r.
    $$
Hence there are only finitely many possibilities for $v$.


Place an open unit cube at each point of $\ring{Z}^3\cap B(p,r)$.
The cubes are measurable, disjoint, and contained in
$B(p,r+\sqrt3)$.  Thus, the combined volume of the cubes, which is
$|\ring{Z}^3\cap B(p,r)|$,  is no greater than the volume of the
containing ball.  The result follows.
\end{proof}

\begin{lemma}\tlabel{lemma:Zlow-count}
  For all $p\in\ring{R}^3$ and all $r\ge\sqrt3$, the set
    $\ring{Z}^3\cap B(p,r)$ is finite of cardinality at least
    $4\pi (r-\sqrt3)^3/3$.
\end{lemma}

\begin{proof} We have already established finiteness in
Lemma~\ref{lemma:Zcount}.  Place a closed unit cube at each point
of $\ring{Z}^3\cap B(p,r)$.  The cubes are measurable and cover
$B(p,r-\sqrt3)$.  Thus, the combined volume of the cubes is at
least the volume of the covered ball.  The result follows.
\end{proof}

\begin{lemma}\tlabel{lemma:Zr2}
For all $p\in\ring{R}^3$, and $k,k'>0$, there exists a $C$ such
that for all $r\ge k'$, we have
    $$
    \ring{Z}^3 \cap (B(p,r+k) \setminus B(p,r-k')) \le C r^2.
    $$
\end{lemma}

\begin{proof}  When $r \ge k'+\sqrt3$, the previous two lemmas show
that the cardinality is at most $4\pi/3$ times
    $$(r + +k + \sqrt3)^3 - (r - k' - \sqrt3)^3 \le C' r^2$$
for some $C'$.  Similarly, if $k'\le r\le k'+\sqrt3$, the
cardinality is at most some fixed constant $C''$.  The result
easily follows.
\end{proof}


    %\chapter{Hypermap}\label{chap:hypermap}
\indy{Index}{hypermap}%

\section{Background on Permutations}

\begin{definition}[permutation]\guid{IFPQAWD}
A \newterm{permutation} $f$ on a set
  $D$ is a bijection $f:D\to D$.
\end{definition}

For example, the identity map $I_D$ on a set $D$,
\begin{displaymath}
I_D(x)=x \text{ for all } x \in D,
\end{displaymath}
 is a permutation.
If $f:D\to D$ is a permutation then there is an inverse function $f^{-1}:D\to D$
that is also a permuation.  
It satisfies
\begin{displaymath}
f f^{-1} = f^{-1} f = I_D.
\end{displaymath}
(This chapter uses product notation $f g$ for the composition of maps
$f\circ g$.)
If $D$ is a finite set, and two maps
$f,g:D\to D$ satisfy $f g = I_D$ on $D$, then $f$ and $g$ are permutations and are
inverses of one another:
\begin{displaymath}
f g = g f = I_D.
\end{displaymath}

Natural number powers  $f^k$ of a permutation $f:D\to D$ are defined
recursively by
\begin{displaymath}
f^0 = I_D,\quad\text{ and } f^{k+1} = f f^k.
\end{displaymath}
Integer powers $f^m$ of a permutation are defined as
$$f^m = f^i (f^{-1})^j,$$ where $m = i -j$.  This is well-defined.
The usual rule of exponents holds:
\begin{displaymath}
f^{a+b} = f^a f^b.
\end{displaymath}

If $f:D\to D$ is a permutation on a finite set $D$, then there is a smallest
positive integer $k$ such that $f^k=I_D$.  The integer $k$ is the \newterm{order}
of the permutation $f$.  If $f^m=I_D$ for any some $m$, then $m = k i$ for some
integer $i$, where $k$ is the order of $f$. The inverse $f^{-1} = f^k f^{-1} = f^{k-1}$ can be written as a
non-negative power of $f$.

A permutation $f$ of a finite set $D$ is \newterm{cyclic}, if the order of $f$ is the cardinality
of $D$.  A permutation $f$ is cyclic if and only if for every $x,y\in D$, there exists an integer $i$
such that $f^i x = y$.

The set of all permutations of the set $\{0,1,2,\ldots,k-1\}$ is written $\op{Sym}(k)$.
The set $\op{Sym}(k)$ is finite and has cardinality $k!$.



\section{Definitions}



\begin{definition}[hypermap,~dart]\guid{ZIHYYRA}\label{def:hypermap}  
  A hypermap is a finite set $D$, together with three functions
  $e,n,f:D\to D$ whose composition is the identity:
  \begin{displaymath}
e\ocirc n\ocirc f = I_D.
\end{displaymath} The
elements of $D$ are called \newterm{darts}.  The functions $e,n$ and
$f$ are called the \newterm{edge map}, the \newterm{node map}, and
the \newterm{face map}, respectively.  \indy{Index}{hypermap}%
\indy{Index}{dart}%
\indy{Index}{edge!map}%
\indy{Index}{node!map}%
\indy{Index}{face!map}%
\indy{Notation}{edgemapz@$e$ (edge map)}%
\indy{Notation}{nodemap@$n$ (node map)}%
\indy{Notation}{facemap@$f$ (face map)}%
\indy{Notation}{D@$D$ (dart)}%
\end{definition}

%\pdf{dart.pdf}{dart}{The arrowhead represents a dart.}
\begin{figure}[htb]
\centering
\szincludegraphics[width=2mm]{\pdfp/dart.eps}
\caption{This symbol represents a dart.}
\label{fig:dart}
\end{figure}

\begin{remark}[planar graphs as hypermaps]\guid{IVPJYAG}\tlabel{rem:hypermap} A hypermap is an abstraction of
the concept of 
planar graph.  Place a dart at each angle of a planar graph.
One function, $f$, 
cycles counterclockwise around the angles of each face.  
Another function, $n$, 
rotates counterclockwise around the angles at each
node.  A third function, $e$, pairs angles at opposite ends of
each edge  (Figure~\ref{fig:hypermap_ex}).   The hypermap extracts
the data $(D,e,n,f)$ from the planar graph and discards the rest.
\indy{Index}{planar graph}%
\end{remark}

\begin{figure}[htb]
\centering
\szincludegraphics[width=80mm]{\pdfp/hypermap-ex.eps}
\caption{Darts mark the angles of a planar graph.  Darts may
be permuted about faces, nodes, and edges.}
\label{fig:hypermap_ex}
\end{figure}

By the background on permutations, $e,n,f$ are all permutations on $D$.
A hypermap satisfies 
\begin{equation}\tlabel{eqn:triality}
e \ocirc n\ocirc f = n\ocirc f\ocirc e = f\ocirc e\ocirc n = I_D.
\end{equation}
Inverted, this triality becomes
\begin{displaymath}
n^{-1} \ocirc e^{-1} \ocirc f^{-1} = (f \ocirc e \ocirc n)^{-1} = I_D.
\end{displaymath}
This inversion is the abstract form of the the duality between nodes
and faces in a planar graph.  Because of these symmetries in the
defining relation, there will be multiple versions of theorems about
hypermaps, all obtained from one proof by symmetry.


\begin{definition}[path,~list,~sublist,~visit,~dart~set]\guid{RRQWGAY} 
Let $D$ be a set (of darts), and let $S$ be a set of permutations of $D$.
A \newterm{path} with \newterm{steps} in $S$
from $x_0$ to $x_{k-1}$ is a \newterm{list}\footnote{The empty path $[]$ seems
to have an ancient origin: ``This is the path made known to me
when I had learned to remove all darts.'' --The Dhammapada} of
darts $[x_0;x_1;\ldots;x_{k-1}]$ such that for each $i$, $x_{i+1} = h_i x_i$,
for some $h_i \in S$.   A \newterm{sublist} of a list is a consecutive
subsequence  $[x_i;x_{i+1};x_{i+1};\ldots;x_j]$, with $0\le i\le j\le k-1$.
A \newterm{unit list} is a list of the form $[x]$.  A
path is \newterm{injective} if $x_i=x_j$ implies $i=j$. 
The \newterm{dart set} of $L$ is $\{x_0,x_1,\ldots,x_{k-1}\}$.  A path \newterm{visits}
a dart $x$, if $x$ is an element of the dart set of $L$.  A set of paths visits a
dart $x$, if some path in the set visits the dart.
\end{definition}

\begin{notation}[$\cooln$]
%Write $P[x_i:x_j]$ for $i<j$ for the sublist $[x_{i+1};\ldots;x_j]$ of
%$P=[x_0;\ldots;x_{k-1}]$.  (The notation is ambiguous when the path is
%not injective.)  
The infix operator $\cooln$ prepends an element $x$ to a list $[x_0;\ldots]$:
\begin{displaymath}
x\cooln[x_0;\ldots] = [x;x_0;\ldots].
\end{displaymath}
%The infix operator $\opat$  \newterm{concatenates} 
%lists:
%\begin{displaymath}
%[a;\ldots;b] \opat [c;\ldots;d]  = [a;\ldots;b;c;\ldots;d].
%\end{displaymath}
\end{notation}
\indy{Notation}{1@$\cooln$ (list operation)}%
\indy{Notation}{P@$P$ (dart path)}%
%\indy{Notation}{1@$[-:-]$ (dart sublist)}%
%\indy{Notation}{Z@$\opat$~(concatenation)}%


\begin{definition}[$\sim_S$]\guid{IENSLJP}
Let $D$ be a set, and let $S$ be a 
set of permutations on $D$.
Define a relation on the set of darts by $x\sim_S y$ when there is a
path from $x$ to $y$ with steps in $S$.
\end{definition}

\begin{lemma}[equal equivalences]\guid{YBGABWW}\rating{50}\label{lemma:er} %\guid{QLPBIKV}
% wording changed by thales Jan 7, 2010.
Let $(D,e,n,f)$ be a hypermap and let $S$ be a  set of permutations.
Then for each $h_1,h_2\in S$, 
the relation $\sim_S$ is the same as the relation $\sim_T$, where
\begin{displaymath}
T = S \cup \{h_1h_2\}.
\end{displaymath}
Moreover, for each $h\in S$, 
the relation $\sim_S$ is the same as the relation $\sim_T$, where
\begin{displaymath}
T = S \cup \{h^{-1}\}.
\end{displaymath}
Also,  the relation $\sim_S$ (that is, $\sim_T$) is an equivalence relation.  
\indy{Index}{equivalence relation}%
\end{lemma}

\begin{proof} If $x\sim_S y$ then clearly $x\sim_T y$.  Conversely,
if $x\sim_T y$, where $T = S\cup\{h_1,h_2\}$, pick a path $P$ from $x$ to $y$ with steps
in $T$ that contains the fewest $h_1h_2$-steps.  

\claim{$P$ does not contain any $h_1h_2$-steps}.  Otherwise, a sublist $[\ldots;u;h_1h_2u;\ldots]$
of $P$ can be expanded to a path $[\ldots;u;h_2u;h_1u;\ldots]$ that contradicts the minimal
property of $P$.

This proves the first conclusion of the lemma.  Fix $h$ in a set of permutations $R$.
By an induction that uses the first conclusion,  for all $i$, $\sim_R$ equals the relation $\sim_{R(h,i)}$,
where $R(h,i) = R \cup \{h,h^2,\ldots,h^i\}$.  If $h\in S$ is an element of order $k$, 
and $T = S\cup\{h^{-1}\}$, then
the second conclusion follows because the following sets give the same relation:
\begin{displaymath}
S,\quad S(h,k-1) = T(h,k-2),\quad T.
\end{displaymath}

By repeated action of the previous conclusion, $\sim_S=\sim_T$, where 
$T = S\cup S^{-1}\cup \{I_D\}$, and where $S^{-1} = \{h^{-1}\mid h\in S\}$.
The unit path $[x]$ yields reflexivity of $\sim_T$.  Also, $T^{-1} = T$ gives the symmetry.  Finally, concatenation of paths gives transitivity.  Thus, $\sim_T$ (i.e., $\sim_S$) is an equivalence relation.
\end{proof}

\begin{definition}[combinatorial~component,~connected]\guid{JVTRXQR}
A \newterm{combinatorial component} of a hypermap $(D,e,n,f)$ is an 
equivalence class of the relation $\sim_S$, where
$S=\{e,f,n\}$. 
(See Lemma~\ref{lemma:er} for other sets that define the same equivalence classes.)  
Write $\#c$ for the
number of combinatorial components.  The hypermap is \newterm{connected} if
$\#c=1$.  \indy{Index}{Dhammapada}%
\indy{Index}{path}%
\indy{Index}{connected}%
\indy{Index}{component!combinatorial}%
\indy{Notation}{1@$\#c$~ (number of components)}%
\end{definition}





\begin{definition}[orbit,~node,~face,~edge]\guid{JIOUCMV}
The \newterm{orbit} of $x\in D$ under a permutation $h$ on
a set $D$ is a set of the form $\{h^i x\mid i\in\ring{N}\}$.  A \newterm{node}
of a hypermap $(D,e,n,f)$ is the orbit of a dart $x\in D$ under $n$.  
A \newterm{face} is an orbit under $f$.  
An \newterm{edge} is an
orbit under $e$.  \indy{Index}{node}%
\indy{Index}{face}%
\indy{Index}{edge}%
\end{definition}

Write $\#h$ for the
number of orbits of a permutation $h$ on $D$.  
\indy{Notation}{h@$h$ (permutation)}%
\indy{Notation}{1@$\#h$~(number of orbits)}%


\begin{lemma}[orbit relation]\guid{PKRQTKP}
Let $D$ be a finite set.  The orbit of $x\in D$ of a permutation $h:D\to D$
is the equivalence class of $x$ under the relation $\sim_S$, when $S=\{h\}$.
\end{lemma}

\begin{definition}[plain]\guid{HFRNMIU}
A hypermap $(D,e,n,f)$ is \newterm{plain}
  (carefully note\footnote{This deliberate play on the homophonous
    {\it plane} privileges writing over speech.  Every plane (hyper)graph may
be planar, but not all plain hypermaps are planar.}
  the  spelling!) when $e$ is an involution on $D$ (that is, $e^2 = I_D$).
  \indy{Index}{planar}%
\end{definition}




\begin{definition}[degenerate]\guid{MKSZLRM}
 A dart in a hypermap $(D,e,n,f)$ 
is degenerate if it is a
fixed point of one of the maps $e,n,f$; otherwise it is nondegenerate.  
%%It is nondegenerate otherwise.
\indy{Index}{dart!degenerate}%
\indy{Index}{dart!nondegenerate}%
\end{definition}

\begin{definition}[simple]\guid{KMHUQNS} 
A hypermap is \newterm{simple} if the intersection of each face with
each node contains at most one dart.  \indy{Index}{simple}%
\end{definition}


% Moved from cup05_tame.tex section on tame plane graphs. 9/5/07:
\begin{lemma}[nodal fixed point]\guid{ZHQCZLX}\rating{50}\tlabel{lemma:nondegen} 
Let $(D,e,n,f)$ be a simple plain hypermap such that every face has
at least three darts.
Then $n$ has no fixed point.
\indy{Index}{fixed point}%
\end{lemma}

\begin{proof} For a contradiction, let $x$ be a fixed point of
$n$. 

\claim{The darts $e x$ and $f x$ lie in the same node and face, so are
equal in the simple hypermap.}  Indeed, they lie in the same node
because $n(f x) = e^{-1} x = e x$. They lie in the same face because
\begin{displaymath}f^2 (e x) = f (f e n x) = f x.\end{displaymath}
So $e x = f x$.

Thus, $f^2 (e x) = f x = e x$, and $e x$ lies on a
face with at most two darts.  This contradicts what is given.
\end{proof}




\section{Walkup}

To focus on a dart $x$ in a
hypermap, it can be useful to draw a hexagon around $x$ and place
the six darts $e x$,
$f x$, $e^{-1} x$, $n x$,  $f^{-1} x$, $e x$, $n^{-1} x$  at its corners
in Figure~\ref{fig:dart+}.  Some of these seven darts may be
equal to one another, even if the figure draws them apart.
Figure~\ref{fig:dart-fix} shows the layout of a degenerate dart.
\indy{Notation}{x@$x$ (dart)}%

\begin{figure}[htb]
\centering
\szincludegraphics[width=40mm]{\pdfp/dart+.eps}
\caption{A dart $x$ and its entourage}
\label{fig:dart+}
\end{figure}

\begin{figure}[htb]
\centering
\szincludegraphics[width=60mm]{\pdfp/dart-fix.eps}
\caption{A dart fixed under a face map.}
\label{fig:dart-fix}
\end{figure}

\subsection{single}

A \newterm{walkup} deletes
a dart from a hypermap and reforms the edge, node, and face
maps to produce a hypermap on the reduced set of darts.  Walkups
come in three flavors: edge walkups, face walkups,
and node walkups.

\begin{definition}[walkup,~degenerate]\guid{DAIZNHD}
The edge \newterm{walkup}
$W_e$ at  a dart $x\in D$ of a hypermap $(D,e,n,f)$ is the hypermap
$(D',e',n',f')$, where $D' = D\setminus\{x\}$ and the
the maps skip over $x$:
\begin{displaymath}
\begin{array}{lll}
f' y &= \text{ if } (f y =  x) \text{ then } f x \text{ else
} f y\\
n' y &= \text{ if } (n y = x) \text{ then } n x \text{ else
} n y\\
e' &= (n'\ocirc f')^{-1}
\end{array}
\end{displaymath}
A walkup at $x$ is said to be \newterm{degenerate} if the dart $x$ is
degenerate.  
\indy{Index}{walkup}%
\indy{Index}{edge!walkup}%
\indy{Index}{face!walkup}%
\indy{Index}{node!walkup}%
\indy{Notation}{Wh@$W_h$ (walkup)}%
\end{definition}

Figure~\ref{fig:walk} shows
the result of an edge walkup on the hexagon around a dart $x$.
The triality symmetry~\ref{eqn:triality}, applied to the definition
of edge walkups, yields the definition of
face walkup $W_f$ and node walkup $W_n$.  
% Figure~\ref{fig:walkfn} shows the result of the face and node
% walkups on the hexagon around a dart $x$.

At a degenerate dart $x$, all three walkups are equal:
$W=W_e=W_n=W_f$ (Figure~\ref{fig:walkdegen}).
\indy{Index}{walkup!degenerate}%
\indy{Notation}{x@$x$ (dart)}%

\begin{figure}[htb]
\centering
\szincludegraphics[width=80mm]{\pdfp/walk.eps}
\caption{The effect of a walkup at $x$}
\label{fig:walk}
\end{figure}


\begin{figure}[htb]
\centering
\szincludegraphics[width=80mm]{\pdfp/walkdegen.eps}
\caption{The effect of a walkup at a degenerate dart}
\label{fig:walkdegen}
\end{figure}


\begin{definition}[merge,~split]\guid{KJIOZBJ}\tlabel{def:merge-split} 
Let $(D,e,n,f)$ be a hypermap, and let $h=n,e$, or $f$.  Let $\op{orbit}(h,x)$
denote the orbit of $x\in D$ under $h$.  Let $(D',e',n',f')$ be the hypermap obtained
from $(D,e,n,f)$ by the walkup $W_h$ at $x\in D$.
Let $h'=e',n',f'$, respectively, according to the choice of $h$.
The walkup $W_h$ at $x$ \newterm{merges} when the walkup joins the
orbit of $h$ through $x$ with another orbit.  That is, there is an orbit $O$ of some
$y\in D'$ under $h':D'\to D'$ of the form
\begin{displaymath}
O \cup\{x\} = \op{orbit}(h,x) \cup \op{orbit}(h,y),
\end{displaymath}
where $y\not\in \op{orbit}(h,x)$.
It \newterm{splits}
when the walkup splits the orbit at $x$ into two orbits.  That is, there are 
distinct orbits $O_1,O_2$ under $h'$ in the hypermap $(D',e',n',f')$ such that
\begin{displaymath}
\{x\}\cup O_1\cup O_2 = \op{orbit}(h,x).
\end{displaymath}
\indy{Index}{split}%
\indy{Index}{merge}%
\indy{Index}{orbit}%
\end{definition}

\begin{lemma}[merge-split]\guid{ZMFKZNH}\rating{150}\tlabel{lemma:merge-split} 
  Let $(D,e,n,f)$ be a hypermap and let $W_h$ be a nondegenerate
  walkup at a dart $x$.  Then $W_h$ merges or splits. Moreover, it merges if
  and only if $x$ and $y$ lie in distinct $h$-orbits, where
  $(h,y)=(f,e x)$,  $(e,n x)$, or $(n,f x)$.
\end{lemma}

\begin{proof} The walkup $W_f$ splits if and only if $f x$ 
(or $x$)
and $e x$ lie in the same $f$-orbit before the split. 
Figure~\ref{fig:split} makes this clear.
The other cases $h=e,n$ hold by triality.
\end{proof}


\begin{figure}[htb]
\centering
\szincludegraphics[height=90mm]{\pdfp/split.eps}
\caption{The face walkup at $x$ mixes $f$-orbits.  If it mixes a
single orbit, the orbit splits. If it mixes two separate orbits, the
orbits merge. }
\label{fig:split}
\end{figure}

The following is a useful way to tell if a walkup merges.


\begin{lemma}[merge criterion]\guid{FKSNTKR}\rating{80}\tlabel{lemma:ng-merge}  
Suppose, in a simple plain hypermap $(D,e,n,f)$, that an edge $\{x,y\}$ consists
of two nondegenerate darts.  Then the walkup $W_f$ 
% (resp. $W_n$)  removed Jan 10, 2009.  Needed? Is it even true?
at $x$ merges.
\end{lemma}
\indy{Index}{merge}%

\begin{proof} 
The darts $f x$ and $e x$ lie in the same node: $n (f x) = e^{-1} x
= e x$. If they are also in the same face of a simple hypermap, then
$f x = e x = y$. So
\begin{displaymath}n y  = n f x = n f e y = y,\end{displaymath}
and $y$ is a fixed
point of $n$, hence degenerate, contrary to assumption.  
Thus, $f x$ and $e x$ are in different faces, and the walkup merges
by Lemma~\ref{lemma:merge-split}.  
\end{proof}


\subsection{double}
\indy{Index}{walkup!double}%

A double walkup is the composite of two walkups of the same type.  The
two darts for the two walkups are to be the members of an orbit of
size two (under $n$, $e$, or $f$).
%%XX?The first walkup is to be chosen so that it merges.  
By choosing the type of the walkups to be different from the type of
the orbit, the first walkup reduces the orbit to a singleton, forcing
the second walkup to be degenerate.

Here are some examples.
\begin{itemize}
\item A double $W_n$ along an edge deletes the edge and 
merges the two endpoints into
a single node (Figure~\ref{fig:doublenode}). 
\item A double $W_f$ along an edge 
deletes the edge and merges the two faces along the edge into
one (Figure~\ref{fig:doubleface}).
\item A double $W_e$ at a node of degree two
deletes the node and merges the two edges at the node into
one (Figure~\ref{fig:doubleedge}).
\end{itemize}


\begin{figure}[htb]
\centering
\szincludegraphics[width=90mm]{\pdfp/double-node-walkup.eps}
\caption{The double node walkup applied to an edge}
\label{fig:doublenode}
\end{figure}


\begin{figure}[htb]
\centering
\szincludegraphics[width=90mm]{\pdfp/double-face-walkup.eps}
\caption{The double face walkup applied to an edge}
\label{fig:doubleface}
\end{figure}


\begin{figure}[htb]
\centering
\szincludegraphics[width=80mm]{\pdfp/double-edge-walkup.eps}
\caption{The double edge walkup applied to a node}
\label{fig:doubleedge}
\end{figure}

\begin{figure}[htb]
\centering
\szincludegraphics[width=80mm]{\pdfp/double_edge.eps}
\caption{The double edge walkup preserves plainness.}
\label{fig:doubleplain}
\end{figure}


\begin{lemma}[plain walkup]\guid{HOZKXVW}\rating{150}\tlabel{lemma:dwalk-planar}  
The three preceding double walkups carry plain
hypermaps into plain hypermaps.
\end{lemma}
\indy{Index}{hypermap!plain}%

\begin{proof} The walkups $W_n$ and $W_f$ preserve the orbit structure
of edges, except for dropping one dart.  By dropping both darts from
the same edge, one edge is lost and all others edges remain
unchanged.

Figure~\ref{fig:doubleplain} illustrates the double $W_e$.  The two
edges $\{x,e x\}$, $\{y, e y\}$ meeting the node are fused by the
double walkup into $\{e x, e y\}$, which is still an edge of size
two.
\end{proof}

\begin{remark}[reverse double walkup]\guid{KPRURND}\label{rem:reverse-double-walkup}
Double walkup transformations can be run in reverse.
Let $H'=(D',e',n',f')$ be
a hypermap and let $D\supset D'$ be a set that contains two additional elements
$x,y$.  Fix distinct elements $x',y'\in D'$.  Define $n,e:D\to D$ as follows.
\begin{displaymath}
\begin{cases} 
e x = y, &\\
e y = x,&\\
e z = e' z,&\text{otherwise.}\\
\end{cases}
\qquad\qquad
\begin{cases} 
n x' = x, &\\
n x =  n' x',&\\
n y' = y,&\\
n y =  n' y',&\\
n z = n' z, &\text{otherwise.}\\
\end{cases}
\end{displaymath}
Define $f$ by forcing the hypermap identity $e n f = I_D$.  
The edge $\{x,y\}$ has been inserted by a reverse double walkup.  The insertion points of the edge
into the hypermap
depend on the data $\{(x',x),(y',y)\}$.   

Reverse double walkup transformations that
insert a node $\{x,y\}$ or a face $\{x,y\}$ into a hypermap are obtained similarly  by triality symmetry.  For example, to insert a node onto an edge $\{x',y'\}$ of cardinality $2$, use
\begin{displaymath}
\begin{cases} 
n x = y, &\\
n y = x,&\\
n z = n' z,&\text{otherwise.}\\
\end{cases}
\qquad\qquad
\begin{cases} 
f x' = x, &\\
f x =  f' x',&\\
\text{etc.}&\\%f
%f y' = y,&\\
%f y =  f' y',&\\
%f z = f' z, &\text{otherwise.}\\
\end{cases}
\end{displaymath}
\indy{Index}{reverse double walkup}%
\end{remark}

\begin{remark}[dart universe]\guid{SCYVYJW}\label{rem:dart-universe}
For reverse double walkups, we need a set from which to draw new darts $x,y$.
We will use a well-ordered set $\Omega$ from which we draw, as needed,
the minimal element of the complement in $\Omega$ of the set of darts
already in play.  We assume that darts can be supplied from $\Omega$,
without mentioning it explicitly.  

For example, if we insert an edge into $(D',e',n',f')$
using the
ordered pair $(x',y')$, we use the data $\{(x',x),(y',y)\}$, where $x$ is the
least element of $\Omega\setminus D'$, and $y$ is the least element of
$\Omega\setminus (D'\cup \{x\})$.  To insert a node into an edge of cardinality
two, it is enough to specify one dart $x'$ in the edge.  Then let $y'$ be the other
dart in the edge, and choose $x,y\in \Omega$ as above.
\indy{Notation}{zzZ@$\Omega$ (well-ordered universe of darts)}
\end{remark}

\begin{definition}[RDW]\guid{TPEZAAM}\label{def:R}  
  Let $H'=(D',e',n',f')$ be a hypermap and let $x'\in D'$.  
%  Let $r$ be the
%  cardinality of the face of $x$, and let $m,p,q$ be integers that
%  satisfy $0\le p$, $0\le m < q < r$, and $m+1 <
%  p+q$.
Let $m,q,p$ be natural numbers.  Assume that $y' =(f')^{m+1}x'$ is not
equal to $z'=(f')^{q+1}x'$.
Construct a hypermap $RDW(H',x',m,p,q)$  as follows.  
First
  add an edge into $H$ using the ordered pair $(y',z')$
(by the reverse double walkup of
  Remark~\ref{rem:reverse-double-walkup}).  Then insert $p$ new nodes
  (of degree $2$) again by reverse double
  walkup transformations, each time at the edge containing $y'$.  
This is $RDW(H,x',m,p,q)$.
\end{definition}
\indy{Notation}{RDW@$RDW$ (reverse double walkup)}


\section{Planarity}
\indy{Index}{walkup}%
\indy{Index}{planarity}%

\begin{definition}[planar]\guid{QVATKMJ}
A hypermap is \newterm{planar} (note the
spelling!) when the Euler relation holds:
\begin{displaymath}\# n + \# e + \# f = \# D + 2\, \#c.\end{displaymath}
\indy{Index}{planar}%
\end{definition}


\begin{remark}[Eulerian relation]\guid{YPVCMHI}\label{rem:Euler}   
The Euler relation for planar graphs can be translated into the
language of hypermaps.  Consider a connected planar graph that
satisfies the Euler relation for the alternating sum of Betti
numbers:
\begin{displaymath}b_0 - b_1 + b_2 = 2\end{displaymath} where $b_0$
is the number of vertices, $b_1$ the number of edges, and $b_2$ the
number of faces (including an unbounded face) of the planar
graph. The hypermap $(D,e,n,f)$, made from the planar graph in
Remark~\ref{rem:hypermap}, is plain, and the involution $e$ has no fixed points.  
Thus, $\# D = 2\#e$, according to the partition of $D$ into edges.  Moreover,
\begin{displaymath}\begin{array}{lll}
b_0 &= \# n\\
b_1 &= \# e\\
b_2 &= \# f\\
2b_1 &= \# D\\
1 &= \#c\\
b_0 - b_1 + b_2  &= \# n + (\#e - \#D) + \# f = 2\,\# c.
\end{array}
\end{displaymath}
Thus, the hypermap is also planar.
\indy{Index}{Euler relation} %
\end{remark}


\begin{lemma}[dart bound]\guid{TGJISOK}\rating{80}\label{lemma:dart-upper} 
Let $H$ be a connected plain planar hypermap such that every edge
has cardinality two.  Assume that there are at least three darts in
every node.  Then
\begin{displaymath}
\# D \le (6\, \#f - 12).
\end{displaymath}
\end{lemma}
\indy{Notation}{H@$H$ (hypermap)}%

\begin{proof}  In a plain planar hypermap, the Euler relation becomes
\begin{displaymath}6\, \#f - 12 = 3\,\#D - 6\,\#n,\end{displaymath}
so it is enough to show that
\begin{displaymath}
\# D \ge 3\,\#n.
\end{displaymath}
This follows directly by assumption: the set of darts can be
partitioned into nodes, with at least three darts per node.
\end{proof}


\begin{definition}[planar~index]\guid{ICAWSNK}
The planar index of a hypermap is
\begin{displaymath}\iota = \# f + \# e + \# n - \# D - 2\,\#
c.\end{displaymath}
(A hypermap with null index is planar.)
\indy{Index}{hypermap!planar index}%
\indy{Notation}{ZZiota@$\iota$ (planar index)}%
\end{definition}

\begin{lemma}[walkup index]\guid{IUCLZYI}\rating{400}\tlabel{lemma:index} 
Let $x$ be a nondegenerate dart of a hypermap $(D,e,n,f)$. Let
$(D',e',n',f')$ be the result of the face walkup $W$ at $x$.  The
walkup changes the size of some orbits.
\begin{displaymath}
\begin{array}{lll}
%\text{\bf Non-degenerate dart $x$: }&\\
\# f' &=\# f +\op{split}_f  \\  
\# e'&=\# e \\
\# n'&=\# n \\
\# D'&=\# D - 1 \\
\#c'&=\# c + \op{split}_c\\
\iota' &= \iota + 1+\op{split}_f - 2\op{split}_c,\\
\end{array}
\end{displaymath}
where
\begin{displaymath}
\op{split}_f = \begin{cases}
1,&\text{if $W$ splits }\\
-1,&\text{if $W$ merges}\\
\end{cases}
\end{displaymath}
and $\op{split}_c=1$ if $e x$ and $f^{-1} x$ belong to different
combinatorial components after the walkup $W$, and $\op{split}_c=0$
otherwise. Moreover, a walkup at a degenerate dart preserves the
planar index.  \indy{Notation}{splitc@$\op{split}_c$}%
\indy{Notation}{splitf@$\op{split}_f$}%
\indy{Notation}{W@$W$ (walkup)}%
\end{lemma}

\begin{proof} The figures make this clear.
\end{proof}

\begin{lemma}[index inequality]\guid{BISHKQW}\rating{100}\tlabel{lemma:planar-index2}
Let $\iota$ be the index of a hypermap $(D,e,n,f)$, and let $\iota'$
be the index after a walkup $W_h$ at a dart $x$.  Then $\iota \le
\iota'$.
\end{lemma} 


\begin{proof} Without loss of generality, by triality symmetry, the
walkup is a face walkup.  If $\op{split}_c=0$, the inequality is
immediate by Lemma~\ref{lemma:index}.  If $\op{split}_c=1$, 
then $e x$ and $f^{-1} x$ lie in
different components after the walkup, hence also in different
faces.  Thus, the walkup splits by Lemma~\ref{lemma:merge-split}.
Hence  $\op{split}_f = 1$.  The result
follows by Lemma~\ref{lemma:index}.
\end{proof}


\begin{lemma}[non-positive index]\guid{FOAGLPA}\rating{50}
\tlabel{lemma:planar-nonpos}  
The planar index
of a hypermap is never positive.
\end{lemma}

\begin{proof}  An face walkup never decreases the index.  A sequence
of face walkups leads to the empty hypermap, which has
index zero.
\end{proof}


\begin{lemma}[planar walkup]\guid{SGCOSXK}\rating{50}
\tlabel{lemma:walkup-planar}
Walkups take planar hypermaps to planar
hypermaps.
\end{lemma}

\begin{proof}  
A planar hypermap has maximum index.  The walkup
can only increase the index, but never beyond its maximum.  
Thus, the index remains at its maximum value.
\end{proof}





\section{Path}

\subsection{contour}

\begin{definition}[cyclic~list]\guid{MYJNYCZ}
A \newterm{cyclic list} $\lp{x_0;\ldots;x_{k-1}}$ is an equivalence class of lists under the transitive closure of the relation:
\begin{displaymath}
[x_0;x_1;x_2\ldots;x_{k-1}] \sim [x_1;x_2;\ldots;x_{k-1};x_0].
\end{displaymath}
A sublist of a cyclic list is a sublist of some representative of the equivalence class.
\end{definition}

\begin{definition}[contour~path,cyclic~list,~contour~loop]\guid{AUIDQRN}
 A \newterm{contour path} from
$x_0$ to $x_{k-1}$ is a path $[x_0;x_1;\ldots;x_{k-1}]$ such that
$x_{i+1} = n^{-1} x_i$ or $f x_i$ for each $i<k$.  (That is, each
step in the path is clockwise step around a node or a
counterclockwise step around a face.)  
A \newterm{contour loop} is an injective cyclic list
$\lp{x_0;x_1;\ldots;x_{k-1}}$ such that
for every $i$, there exists $h_i\in \{f,n^{-1}\}$ such that $x_{i+1} = h_i x_i$, 
where the subscripts are
read modulo $k$.
%$[x_1;\ldots;x_{k-1}]$ is injective and $x_0 = x_{k-1}$, then it is
%a \newterm{contour loop}.  
%A sublist of a contour loop $[x_0;\ldots;x_{k-1}]$ is a path
%$[y_0;
\indy{Index}{contour!path}%
\indy{Index}{contour!loop}%
\indy{Index}{loop}%
\end{definition}



\begin{remark}[contour path illustration]\guid{AWRGIPA}
 Figure~\ref{fig:hypermap_ex}
  constructs a hypermap from a planar graph by drawing darts next to
  each angle.  In this representation, the darts along a contour path
  lie to the left of the corresponding planar graph edges.  For that
  reason, a shaded region to the left of a curve depicts a contour
  path.
\end{remark}

\begin{figure}[htb]
\centering
\szincludegraphics[width=80mm]{\pdfp/shade_dart.eps}
\caption{A contour path as a sequence as dart is represented as a
shaded path.}
\label{fig:shade-dart}
\end{figure}

\begin{lemma}[injective path]\guid{QZTPGJV}\rating{50} 
An injective contour path from
  $x$ to $y$ can be constructed from an arbitrary contour path from
  $x$ to $y$ by dropping some darts from the path.
\end{lemma}

\begin{proof} Repeatedly replace $[\ldots;a;b;\ldots;b;c;\ldots]$ with
$[\ldots;a;b;c;\ldots]$.
\end{proof}





\begin{lemma}[contours-components]\guid{KDAEDEX}\rating{100}\tlabel{lemma:connect-contour}  
Let $H$ be a hypermap.
If $x$ and $y$ are darts in the same combinatorial component of $H$ if and only if
there exists a contour path from $x$ to $y$.
\end{lemma}

\begin{proof} 
Combinatorial components are defined by an equivalence relation $\sim_S$, where
$S = \{e,n,f\}$.  By Lemma~\ref{lemma:er}, this is the same equivalence relation as
$\sim_T$, where $T = \{n^{-1},f\}$.  By the definition of the equivalence relation $T$,
$x\sim_T y$ if and only if there is a contour path from $x$ to $y$.
\end{proof}
\indy{Index}{component!combinatorial}%

\begin{definition}[complement]\guid{GCACAFP} 
Let $(D,e,n,f)$ be a plain hypermap.
Let $P=\lp{x;y;\ldots}$ be a contour loop that does not visit any node
twice in a plain hypermap.   (That is, the dart set of $P$ intersected with a node
is the dart set of a maximal sublist $[z;n^{-1}z;\ldots;n^{-k}z]$ of $n^{-1}$ steps.)
 Replace each maximal sublist of
$n^{-1}$-steps
\begin{displaymath}
[z;n^{-1} z; \ldots; n^{-k} z]
\end{displaymath}
with the sublist
\begin{displaymath}
[n^{-(k+1)} z;n^{-(k+2)} z;\ldots; n z]
\end{displaymath}
Concatenate these new sublists in reverse order.  By the relation $n f = f^{-1} n^{-1}$,
the transitions between the new sublists are $f$-steps.
The resulting contour loop $P^c$
is the \newterm{complement}. 
\end{definition}
\indy{Notation}{1@$*^c$ (complement)}

\begin{figure}[htb]
\centering
\szincludegraphics[width=70mm]{\pdfp/complement.eps}
\caption{The complement contour traces the remaining darts
at the same nodes as the original contour loop. }
\label{fig:contour-comp}
\end{figure}


\subsection{M\"obius}

\begin{definition}[M\"obius~contour]\guid{MBYIEQP}
 A M\"obius contour in a hypermap
$(D,e,n,f)$ is an
injective contour path $P=[x_0;\ldots]$ that satisfies
\begin{equation}
\tlabel{eqn:mobius}
x_j = n x_0\quad x_k = n x_i
\end{equation}
for some $0 < i\le j< k$ (Figure~\ref{fig:mobius}).
\indy{Index}{contour!M\"obius}%
\end{definition}


\begin{remark}[Four-Color theorem]\guid{ROIPZSU}
G. Gonthier devised the notion of M\"obius contour as a way to prove
the Four-Color theorem without appeal to topology.  (The Appel-Haken
proof of the Four-Color theorem relies on the Jordan curve theorem.)
This chapter uses a significant amount of material from ~\cite{Gonthier:2005:FourColor}.
\end{remark}

\begin{figure}[htb]
\centering
\szincludegraphics[width=50mm]{\pdfp/mobius.eps}
\caption{A M\"obius contour}
\label{fig:mobius}
\end{figure}

\begin{figure}[htb]
\centering
\szincludegraphics[width=30mm]{\pdfp/3m.eps}
\caption{The face map on this hypermap gives a M\"obius contour with
three darts}
\label{fig:3m}
\end{figure}

\begin{remark}[M\"obius strip]\guid{NGALZAC}
 Heuristically, a M\"obius contour is a 
combinatorial M\"obius strip that
twists to make 
its left-hand side into
its right-hand side.  A planar hypermap has no such contour.  
Figure~\ref{fig:violate-jct}
redraws a violation of the Jordan curve theorem
as a M\"obius contour.   
\end{remark}

\begin{figure}[htb]
\centering
\szincludegraphics[width=80mm]{\pdfp/violate-jct2.eps}
\caption{A path that tunnels from the interior to the exterior
of a simple closed curve
is analogous to a M\"obius contour.}
\label{fig:violate-jct}
\end{figure}

\begin{figure}[htb]
\centering
\szincludegraphics[width=80mm]{\pdfp/mobius_contour.eps}
\caption{Some M\"obius contours}
\label{fig:mobius-contour}
\end{figure}






\begin{lemma}[planar-non-M\"obius]\guid{LIPYTUI}\rating{300}\tlabel{lemma:no-mobius}
A planar hypermap does not have a M\"obius contour.
\end{lemma}
\indy{Index}{hypermap!planar}%

\begin{proof} For a contradiction, assume that there exist planar
hypermaps with M\"obius contours.  An edge walkup carries
planar hypermaps into planar hypermaps. An edge walkup
at a dart that is not on the M\"obius contour carries the
M\"obius contour to a M\"obius contour 
and reduces the number of darts.  
In the M\"obius Condition~\ref{eqn:mobius},
a walkup at a dart that is not at position $0$, $i$, $j$, $k$
along the contour carries the M\"obius contour to a M\"obius contour
and reduces the number of darts. Thus, a counterexample with
the smallest possible number of darts contains no
darts except those on the M\"obius contour, and its only darts
are at positions $0$, $i=j=1$, $k=2$.

This is a three darted hypermap (Figure~\ref{fig:3m}.)  
The M\"obius condition, the
definition of contours, together with $e\ocirc n\ocirc f=I_D$ force
$e=n=f$, all permutations of order three.  This hypermap is not planar:
\begin{displaymath}\# e + \# n + \# f = 3~~\ne~~ 5 = \# D + 2\,
\#c.\end{displaymath}
\end{proof}



%\subsection{interior}
%\indy{Index}{interior}%
%
%\begin{definition}[interior]\guid{NQXQALU}\label{def:interior} 
%A dart $y$ lies in the \newterm{interior} of a contour
%loop $L$ if there is a an injective contour path
%$x_0,x_1,\ldots,x_k=y$ such that $x_1 = f x_0$ (or $k=0$), and
%such that $x_i$ lies on the loop $L$ if and only if $i=0$.
%\indy{Index}{interior!contour loop}%
%\indy{Notation}{L@$L$ (contour loop)}%
%\indy{Notation}{y@$y$ (dart)}%
%\end{definition}
%

\begin{lemma}[step coherence]\guid{ILTXRQD}\rating{100}\tlabel{lemma:contour-path-type}
Suppose that a hypermap has no M\"obius contours. Let $L$ be a
contour loop.  Let $P$ be any injective contour path with at least
$3$ darts, that starts and ends on $L$, but visits no other darts of
$L$.  Then the first and last steps of $P$ are both of the same type
($n^{-1}$ or $f$).
\end{lemma}
\indy{Notation}{P@$P$ (contour path)}%

\begin{figure}[htb]
\centering
\szincludegraphics[width=80mm]{\pdfp/interior_nf.eps}
\caption{A path must enter and depart from a contour loop with the
same type of step.}
\label{fig:interior_nf}
\end{figure}


\begin{proof} The proof shows the contrapositive.  Suppose $P=[n x;f n
x;\ldots;n y;y]$.  The successor of $n x$ on $L$ is $x$.  Starting
at $x$, follow $L$ to $y$, and on to $n x$.  Follow $P$ back to $n
y$.
%\begin{displaymath}
%x\cooln L[x:n x] \opat P[ n x;ny].
%\end{displaymath}  
This is a M\"obius contour $x\ldots y\ldots n x\ldots n y$.

Suppose $P=[n x;x;\ldots;f^{-1} y;y]$.  Starting at $x$, follow $P$ to
$y$, then follow $L$ to $n x$, and on to $n y$.  This is a M\"obius
contour.\footnote{The second statement can also be deduced from the first statement
by the duality $(D,e,n,f)\leftrightarrow (D,e^{-1},f^{-1},n^{-1})$ that swaps
$f$-steps with $n^{-1}$-steps in a path.}
\end{proof}

%\begin{lemma}[]\guid{UMYSGDB}\rating{80}\tlabel{lemma:dart-interior}
%  Let $L$ be a contour loop on a plain hypermap without M\"obius
%  contours.  Assume a dart $x$ lies in the interior of the loop $L$.
%  Then every dart in its $f$-orbit lies in the interior of the loop.
%  Moreover, if the dart $x$ does not lie on the same node as any dart
%  in $L$, then every dart in the $n$-orbit of $x$ lies in the
%  interior of $L$.
%\end{lemma}

%\begin{proof} Let $P= [x_0;\ldots;x]$ be an injective path that
%  certifies that $x$ lies in the interior of $L$.  If $f x$ lies
%  along this path already or if it lies on $L$, then it is clearly
%  interior.  Otherwise, $[x_0;\ldots,x;f x]$ is a certifying path for
%  $f x$.  Similarly, use the certifying path $[x_0;\ldots;x;n^{-1}
%  x]$ for $n^{-1} x$.
%\end{proof}
%
%%
%\begin{definition}[interior~face,~node]\guid{JUXKJTU}
% A face or a node is interior
%  to a loop in a hypermap if all of its darts are interior.
%  \indy{Index}{interior!face}%
%  \indy{Index}{interior!node}%
%\end{definition}


\begin{lemma}[loop separation]\guid{ICJHAOQ}\rating{180}\tlabel{lemma:contour-f}
Suppose that a hypermap has no M\"obius contours.  Let $L$ be a
contour loop.  Then there does not exist a contour path
$[x_0;\ldots;x_k]$, for $k\ge 1$ with the following properties:
\begin{enumerate}
\item $x_i$ lies on $L$ if and only if $i=0$.
\item $x_1 = f x_0$.
\item $x_0$ and $x_k$ lie in different nodes.
\item Some dart of $L$ is at the node of $x_k$.
\end{enumerate}
\end{lemma}

%\begin{figure}[htb]
%  \centering
%  \szincludegraphics[width=40mm]{\pdfp/no_node_path.eps}
%  \caption{No path exists from a node of $L$ to the interior.}
%  \label{fig:no-node-path}
%\end{figure}

\begin{proof} Assume for a contradiction that the path $P$ exists.
Some sublist is injective and satisfies the same conditions.  Again,
without loss of generality, shrinking the path if needed, $k$ is the
smallest index for which the last two conditions are met.  Append
$n^{-1}$-steps to $P$ to reach a dart of $L$.  This is contrary to
Lemma~\ref{lemma:contour-path-type}.
\end{proof}

\begin{lemma}[three darts]\guid{EUXPBPO}\rating{ZZ}\label{lemma:3dart}  
Assume that each face of a hypermap  has at least three darts.
Then every contour loop that meets at least two nodes has at least
three darts.
\end{lemma}

\begin{proof} Let $P=\lp{x;y}$ be a contour loop meeting two nodes.  Then
$y = f x$ and $x = f y$, so that the face has size two.
\end{proof}


\section{Quotient}
\indy{Index}{quotient}%

\subsection{definition}

\begin{definition}[isomorphism]\guid{GUDUERI}
 Two hypermaps $(D,e,n,f)$ and
$(D',e',n',f')$ are \newterm{isomorphic} when there is a bijection
$G:D\to D'$ such that
\begin{displaymath}h'\circ G = G\circ h\end{displaymath}
for $(h,h')=(e,e'), (f,f'), (n,n')$.
\indy{Index}{isomorphic hypermaps}%
\indy{Notation}{G@$G$ (morphism of hypermaps)}%
\end{definition}


\begin{definition}[normal family]\guid{RQSVFLE}
Let $(D,e,n,f)$ be a hypermap. 
%Assume that 
%there are no darts fixed by $e$ 
%(so that $f x \ne n^{-1} x$ at each dart). 
Let $\cal L$ be a family of contour
loops.  The family $\cal L$ is  normal if the following
conditions hold of its loops. \begin{enumerate}
\item  No dart is visited by two different loops.
\item  Every loop visits at least two nodes.
\item  If a loop visits a node, then every dart at that node is visited
by some loop.
\end{enumerate}
\indy{Index}{normal family}%
\end{definition}

A normal family determines a new hypermap.  A dart in the new set $D'$
of darts is a maximal sublist $[x;n^{-1} x; n^{-2} x;\ldots;n^{-k}
x]$ of $n^{-1}$ steps appearing in some loop in $\cal L$. The map $f'$
takes the maximal path $[x;n^{-1}x;\ldots;y]$ to the maximal path (in
the same contour loop) starting at $f y$. The map ${n'}^{-1}$ takes
the maximal path $[\ldots;y]$ to the maximal sequence (in some other
contour loop) starting $[n^{-1}y;\ldots]$. Equivalently, $n'$ takes
the maximal path $[x;\ldots]$ to the maximal path ending $[\ldots;n
x]$. The map $e'$ is defined by $e'\ocirc n'\ocirc f' = I_{D'}$.
\indy{Index}{path!maximal} %
\indy{Notation}{D@$D$ (dart)}%

\begin{definition}[quotient]\guid{AJENHSB}
 The hypermap $(D',e',n',f')$
constructed from the normal
family $\cal L$ of $H=(D,e,n,f)$ 
is called the \newterm{quotient} of $H$ by $\cal L$, and is denoted
$H/{\cal L}$.  If $x$ is a dart visited by some loop in $\cal L$, then
the maximal path $[\ldots;x;\ldots]$ is called the \newterm{quotient dart} of $x$.
%The hypermap $H$ is said to be a \newterm{cover} of $H/{\cal L}$.  
\indy{Index}{quotient}%
\end{definition}
\indy{Notation}{L@$\cal L$}%
\indy{Notation}{H@$H/{\cal L}$}%
%\indy{Index}{cover}%
\indy{Index}{hypermap!quotient}%
\indy{Index}{normal family}%

Intuitively, the quotient hypermap is represented as a graph whose
cycles under $f'$ are precisely the contour loops in the normal family
(Figure~\ref{fig:quot}).


\begin{figure}[htb]
\centering
\szincludegraphics[width=70mm]{\pdfp/quot.eps}
\caption{The contour loops in a normal family become faces in the
quotient}
\label{fig:quot}
\end{figure}

\subsection{properties}

This subsection explores some of the properties of a quotient hypermap.
The first two lemmas describe the faces and the nodes of the quotient
in terms of the combinatorics of the normal family.

\begin{lemma}[quotient face,~$\F$]\guid{WIRLCNL}\label{lemma:quotient-bijection}
  Let ${\cal L}$ be a normal family of the hypermap $H$.  Then ${\cal
    L}$ is in natural bijection with the set of faces of the quotient
  $H/{\cal L}$.  If $x'=[x;\ldots;n^{-k}x]$ is a maximal path of
  $n^{-1}$ steps in the contour loop $L\in{\cal L}$, then the
  corresponding face $\F(L)$ of $H/{\cal L}$ is the
  one containing the quotient dart $x'$.
\end{lemma}
\indy{Notation}{F@$\F$ (quotient bijection)}

\begin{proof}  This is left as an exercise to the reader.
\end{proof}


\begin{lemma}[quotient node]\guid{UDJNSHH}\label{lemma:quotient-node}
Let $H$ be a hypermap and let ${\cal L}$ be a normal family of $H$.
Then there is a natural bijection between  the set of nodes of
$H/{\cal L}$ and the set of nodes of $H$ that
are visited by some contour loop in ${\cal L}$.   
The bijective function sends the node in $H/{\cal L}$ of 
the dart $x' = [x;n^{-1} x;\ldots;n^{-k}x]$ to the node of $x$ in $H$.
\end{lemma}

\begin{proof}  The proof is an elementary verification.
Let $H/{\cal L} = (D',e',n',f')$.
\claim{This function is well-defined.}  Indeed, 
 \begin{displaymath}
(n')^{-1} x' = [n^{-(k+1)} x;\ldots]
\end{displaymath}
 is also sent to the node of $x$ in $H$.

\claim{This function is onto.}  Indeed,
If $L\in {\cal L}$ visits $x$,  and $x'$ is the image of $x$ in $D'$.
Then the node of $x'$ clearly maps to the node of $x$.  

\claim{Finally, the function is one-to-one.}  Indeed, if the nodes of
two quotient darts $x'$, $y'$ map to the same node of $H$, then
$x'=[x;\ldots]$ and $y'=[y;\ldots]$, where $n^j x = y$ for some $j$.
It follows by the definition of the node map on the quotient, that
$x'$ and $y'$ belong to the same node.
\end{proof}

The next two lemmas look at properties of the quotient that are
inherited from the orignal hypermap.

\begin{lemma}[plain quotient]\guid{JMKRXLA}\rating{280}\tlabel{lemma:quotient-plain}
Let $H$ be a plain hypermap, and let $\cal L$ be a
normal family.  Then $H/{\cal L}$ is a plain hypermap.
\end{lemma}

\begin{proof}  Write $H=(D,e,n,f)$ and $H/{\cal L} = (D',e',n',f')$.  
Write $[\ldots; x]$ for the node in
the quotient ending in dart $x\in D$ and $[x;\ldots]$ for the node
in the quotient starting with dart $x\in D$.  Plainness gives $e^2 x
= x$, so that for any dart $[\ldots x]$ in the quotient:
\begin{displaymath}\begin{array}{lll}
{e'}^{-2} [\ldots; x] &= n' f' n' f' [\ldots; x] = n' f' n' [f x; \ldots] \\&=
n' f' [\ldots; n f x] = n' [f n f; \ldots] = [\ldots; n f n f x]\\ &=
[\ldots; e^{-2} x] = [\ldots; x].
\end{array}\end{displaymath}
Thus, $e'$ has order $2$ on the quotient.
\end{proof}




\begin{definition}[no double joins]\guid{EDUYIEA}
A hypermap $H$ has no \newterm{double joins}, if for every two nodes
in $H$, there is at most one edge that meets both of them.
\end{definition}


\begin{lemma}[quotient-no-double-joins]\guid{KSRDPTZ}
Let $H$ be a plain hypermap with no double joins and let ${\cal L}$ be a normal
family of $H$.  Then $H/{\cal L}$ has no double joins.
\end{lemma}

\begin{proof} By Lemma~\ref{lemma:quotient-plain}, the quotient
  $H/{\cal L}$ is plain.  Let $\{x',e'x'\}$ and $\{y',e'y'\}$ be edges
  with the property that $x'$ and $y'$ lie at one node of $H/{\cal L}$
  and $e'x'$ and $e'y'$ lie at a second (different) node.  Write $x' =
  [\ldots;x]$ and $y' = [\ldots;y]$.  Then $e'x' = [\ldots;e x]$ and
  $e'y' = [\ldots;e y]$.  According to
  Lemma~\ref{lemma:quotient-node}, there is an injective map from
  nodes of $H/{\cal L}$ to nodes of $H$.  It follows that $x$ and $y$
  belong to the same node and that $e x$ and $e y$ belong to the same
  (different) node.  By the assumption that $H$ has no double joins,
  it follows that $x=y$.  Hence also $x' = y'$, and $H/{\cal L}$ has
  no double joins.
\end{proof}


\begin{lemma}[nodal fixed point]\guid{PYOVATA}\label{lemma:nfp}
Let $H=(D,e,n,f)$ be a hypermap in which the edge map has no fixed points.
Let ${\cal  L}$ be a normal family of $H$, with quotient $H/{\cal L} = (D',e',n',f')$.  
Then the following are equivalent conditions:
\begin{itemize}
\item $n'$ has a fixed point in $D'$.
\item The dart set of some $L\in {\cal L}$ contains a node.
\end{itemize}
\end{lemma}

\begin{proof}
If $x'=[x;n^{-1} x;\ldots;n^{-k} x]$ is a dart in $D'$, then $(n')^{-1}x'$ is
$[n^{-(k+1)} x;\ldots]$.  The dart $x'$ is a fixed point if and only if
$x = n^{-(k+1)} x$.  This holds if and only if the dart set of $x'$ is an entire node.
\end{proof}

\subsection{example}

\begin{example}[maximal normal family]\label{ex:Hall} 
  Assume that $H=(D,e,n,f)$ is a
  hypermap. % with no fixed points under $e$.
  Assume that every face meets at least two nodes. Then the set of all
  faces defines a normal family of contour loops: follow $f$ around
  each face $[x;f x;\ldots]$.  If $e$ acts without fixed points, then
  each dart of the quotient is just a unit path consisting of a single
  dart of $H$, and the quotient is isomorphic to $H$ itself.
\end{example}

\begin{example}[minimal normal family]\label{ex:H2} 
  Assume that $H=(D,e,n,f)$ is a plain hypermap.  Let $F = \{x,f x,\ldots\}$ be a face
  that visits at least three nodes and that meets each node in at most
  one dart.  Let $\cal L$ be the family with two contour loops: $\lp{x;f x;\ldots}$ 
and its complement $L^c = \lp{n^{-1} x;\ldots}$.
%\begin{displaymath}
%  [n^{-1} x;n^{-2} x;\ldots;n x;f n x = y;n^{-1} y; n^{-2} y;\ldots; n y; f ny;\ldots]
%\end{displaymath}
The family $\cal L$ is normal. The quotient hypermap $H/{\cal L}$ has
two faces: $F$ and a back side $F'$ of the same cardinality $k$.
\indy{Notation}{F@$F$ (hypermap face)}%
\end{example}

\begin{example}[cyclic]\label{ex:H2k} 
There is a hypermap $H_{2k}$ with two face.  The set of darts is the
disjoint union of two copies of $Z_k$, a cyclic group of order $k$
with generator $1$.  Each cyclic group is a face.  Use the variable
$i$ to index the first cyclic group and $i'$ to index the second.
The face map is $i\mapsto i+1$ and $i'\mapsto (i-1)'$.  The node map
is the involution $i\leftrightarrow i'$.  The edge map is the
involution $i\leftrightarrow (i+1)'$.  The relation $e\ocirc n\ocirc
f = I_D$ is verified:
\begin{displaymath}
\begin{array}{llllllll}
enf(i) &= e n(i+1) &= e(i+1)' &= i\\
e n f (i') & e n (i-1)' & e (i-1) &= i'.\\
\end{array}
\end{displaymath}
If a hypermap is isomorphic to $H_{2k}$ for
some $k$, then it is \newterm{cyclic}.  In particular,
the hypermap constructed in the previous example is cyclic.
\indy{Index}{hypermap!cyclic}%
\indy{Notation}{Z@$Z_k$ (cyclic group)}%
\end{example}


%\begin{lemma}[three darts]\guid{BPAKIYP}
%Let $H$ be a plain hypermap with no double joins in which every
%face has at least three darts.  Let ${\cal L}$ be a node-free normal family of $H$.
%Then every face of $H/{\cal L}$ has at least three darts.
%\end{lemma}
%
%\begin{proof}  By definition, every contour loop in a normal family meets
%at least two nodes.  It follows that every face of $H/{\cal L}$ has at least two
%darts.
%
%Suppose for a contradiction that a face of $H'=H/{\cal L}$ only has two darts $x'$
%and $y'$, exchanged by the face map $f'$ of $H'$:
%\begin{displaymath}
%\begin{array}{lllll}
%y' &= [y;n^{-1} y;\ldots; n^{-k} y] &= f' x' &= [f n^{-\ell} x;\ldots;f x],\\
%x' &= [x;n^{-1}x;\ldots; n^{-\ell} x] &= f' y' &= [f n^{-k} y;\ldots; f y].\\
%\end{array}
%\end{displaymath}
%Using the relation $e^{-1} = n f$, it follows that $\{n^{-\ell} x, n y\}$
%and $\{n^{-k} y, n x\}$ are edges.  Both meet the node of $x$ and the node of $y$.
%In a hypermap with no mutiple joins, this implies that the edges are equal:
%\begin{displaymath}
%n y = n^{-k} y,\qquad n x = n^{-\ell} x.
%\end{displaymath}
%This shows that the dart set of $x'$ is the entire node.  This contradicts
%the assumption that ${\cal L}$ is node-free.
%\end{proof}
%

%\begin{lemma}[]\guid{ZOKKAOI}\tlabel{lemma:quotient-planar}
%Let $H$ be a plain planar hypermap, and let $\cal L$
%be a normal family.  Then $H/{\cal L}$ is a plain planar hypermap.
%\end{lemma}
%
%\begin{proof} Suppose $H/{\cal L}$ is not planar.
%Let $P$ be a M\"obius contour on $H/{\cal L}$.  It lifts uniquely to
%a contour on $H$ with the property that the darts visited on $H$ are
%precisely the darts that belong to a dart in the quotient.  This is
%compatible with the node map $n$.  So the contour path lifts to a
%M\"obius contour on $H$.  Thus, $H$ is not planar.
%\end{proof}

\section{Generation}
\indy{Index}{generation}%


The final section of this chapter presents an algorithm that generates all
simple, plain, planar hypermaps satisfying certain general conditions
(Definition~\ref{def:restricted}).  The algorithm proceeds by adding more and
more
edges and nodes to a cyclic hypermap by a sequence of reverse double walkup
transformations.

\begin{definition}[restricted]\guid{INCRVQC}\label{def:restricted}
A restricted hypermap $H = (D,e,n,f)$ is one with the following
properties.
\begin{enumerate}
\item The hypermap $H$ has no double joins, and is nonempty, planar,
  connected, plain and simple.
\item The edge map $e$ has no fixed points.  % Needed in Lemma:[flag quotient]
\item The node map $n$ has no fixed points.
\item The size of every face is at least $3$.
%%  (All hypoth. Needed?)
\end{enumerate}
\indy{Index}{hypermap!restricted}%
\indy{Notation}{H@$H$ (hypermap)}%
\end{definition}

\begin{remark}[step type]\guid{BMZYWKV}
The assumption that $e x \ne x$ implies that $f x \ne n^{-1} x$ so that $f$-steps of a 
path can be distinguished from $n^{-1}$-steps.
\end{remark}


\subsection{boolean value}
\indy{Index}{flag}%

The algorithm  marks certain faces as `true.'
Roughly, this  means that the the face cannot be modified
at any later stage of the algorithm.   When all of its faces
are true, the hypermap stands in final form.
The function that marks each face as true or false is a
\newterm{flag}.  For the algorithm to work properly, it is necessary
to impose some constraints, as captured in Definition~\ref{def:flag}.
\indy{Index}{hypermap!algorithm}%

%% XX \hat doesn't get used.

Under the bijection $\F$ between a normal family ${\cal L}$ and the set of
faces of a quotient $H/{\cal L}$, any function $\hat\varphi$ on ${\cal L}$
can be identified with a function $\check\varphi$ on the set of faces of $H/{\cal L}$.
(The {\it hat} points up to $H$, and the {\it check} points down to the quotient.)
This identification of functions will be used frequently in this section.

\begin{definition}[canonical function]\guid{CRUDEHU}
 Let $H$ be a hypermap with
  normal family $\cal L$.  The \newterm{canonical function}
  $\hat\varphi_{can}$ is the boolean-valued function on ${\cal L}$
  that is true on $L$ exactly when the dart set of $L$ maps
  bijectively to the face $\F(L)$ of $H/{\cal L}$, under $x\mapsto [x]$.  The
  corresponding function $\check\varphi_{can}$ is also called the
  canonical function.  A contour loop $L$ (or face $\F(L)$)
  is said to be canonically true or false, according to the value of the canonical
  function.  
\indy{Index}{function!canonical boolean}%
\indy{Notation}{zzP@$\check\varphi_{can}$}
\indy{Notation}{zzP@$\hat\varphi_{can}$}
\end{definition}

In other words, the face in the quotient is canonically true, exactly
when the corresponding contour loop $L\in {\cal L}$ has no $n^{-1}$
steps.  The dart set of such a contour loop $L$ is a face of $H$.  
%Based on this observation, we make the following definition.


\begin{definition}[flag]\guid{HFTAHWB}\label{def:flag} 
  Let $S$ be a set of darts in a hypermap $H$.  An $S$-\newterm{flag}
  on $H$ is a boolean-valued function $\check\varphi$ on the set of faces that
  satisfies the following two constraints.
\begin{enumerate}
\item If darts $x,y$ belong to true faces,
then there is a contour path from $x$ to $y$ that remains
in true faces.
\item Each edge of the hypermap meets a true face or $S$.
\end{enumerate}
An $\emptyset$-flag is simply called a flag.
%An isomorphism of flagged hypermaps is an isomorphism of
%hypermaps that respects the flags.
\indy{Index}{flag}%
\indy{Notation}{S@$S$ (set ofdarts)}%
\end{definition}



\begin{example}[cyclic hypermap flag] 
The cyclic hypermap of Example~\ref{ex:H2k}, carries a
flag that marks one face true and the other false.
\end{example}

\begin{example}[maximal quotient flag]\label{ex:Hall-flag} 
Let $H$ be a connected hypermap, and let $\cal L$ be the example of
Example~\ref{ex:Hall}, then the canonical map takes value
$\op{true}$ on every face.  This is a flag. In fact,
Lemma~\ref{lemma:connect-contour} provides the contour paths that
are required in the definition of flag.
\end{example}


%\begin{definition}[canonically true]\guid{PYAOBOS}
%  A contour loop $L$ in a hypermap is \newterm{canonically true} 
%if its
%  dart set is a face of $H$.
%\end{definition}


%\begin{lemma}[quotient isomorphism criterion]\guid{STKBEPH}\rating{100}
%\tlabel{lemma:all-dart}  
%  Let $H$ be a hypermap in which $e$ acts without fixed points, and
%  let ${\cal L}$ be a normal family of $H$. If the canonical boolean
%  function on the set of faces of $H/{\cal L}$ has at least as many
%  true values as there are faces of $H$, then $\cal L$ is the normal
%  family in Example~\ref{ex:Hall}. In particular, $H/{\cal L}$ is
%  isomorphic to $H$.
%\end{lemma}
%
%\begin{proof} If a face of  $H/{\cal L}$ is $\op{true}$,  then
%its darts are unit paths, and the face of $H/{\cal L}$ is in natural
%bijection with a face in $H$.  This is an injective map from the
%set of true faces of $H/{\cal L}$ to the set of faces of $H$.  The
%hypothesis of the lemma implies that this injective map is
%bijective. All of the darts of $H$ are accounted for under this
%bijection. Thus, the quotient has no false faces.  The result
%follows.
%\end{proof}
%


There is a standard way of constructing the sets $S$ of darts that
will be used in $S$-flags.  


\begin{definition}[S]\guid{FDRMSZG}
Let $H$ be a hypermap, $L$ a contour loop of the hypermap,
and $x$ an element of the dart set of $L$.
If  $L$ is  canonically true, then let $S=\emptyset$.
Otherwise,
let $m\ge0$ to be the largest $m$ 
such that 
\begin{displaymath}
[x;f x; f^2 x;\ldots;f^{m+1} x]
\end{displaymath}  
is a sublist of $L$, and
%Set $y = f^{m+1} x$
set $S(H,L,x) = \{f^i x \mid 1 \le i\le m\}$.
\end{definition}
\indy{Notation}{S@$S(H,L,x)$ (flag set)}

\begin{lemma}[flag quotient]\guid{KHGAQRG}\label{lemma:flag-set-quotient}
Let $H$ be a hypermap in which $e$ acts without fixed points, 
$L$ a contour loop, and $x$ and element of the dart set of $L$.
Let ${\cal L}$ be a normal family of $H$ that contains $L$.
Then $S(H,L,x)$ maps bijectively to a set $S'$ of darts in the quotient $H/{\cal L}$.
\end{lemma}

\begin{proof} The darts of the quotient are maximal sublists
  $[y;n^{-1} y;\ldots;n^{-k} y]$ of contour loops $L'\in {\cal L}$
  made entirely of $n^{-1}$ steps.  Each $y\in S(H,L,x)$ is preceded
  by an $f$-step and is followed by an $f$-step in $L$.  Hence the
  maximal sublist of $L$ containing $y$ is a unit path $[y]$.  The
  bijection follows.
\end{proof}


\subsection{markup}\label{sec:face-insert}
\indy{Index}{extension}%


%This section describes the transition from $H/{\cal L}$ to $H/{\cal
%M}$.  The algorithm carries along various auxiliary data satisfying
%various assumptions, enumerated as follows:

%\begin{definition}[node free]\guid{RZNSLOS}
%A  family ${\cal L}$ of contour loops in a hypermap $H$  
%is \newterm{node free}, if no node of $H$ is contained the dart set of any 
%$L\in {\cal L}$.
%\end{definition}



\begin{definition}[marked hypermap]\guid{TUFOKWK}\label{def:marked}
Let $(H,{\cal L},L,x)$ be a tuple, consisting of 
\begin{itemize}
\item a hypermap $H=(D,e,n,f)$ with no M\"obius contours, and in which
  $e$ acts without fixed points.  % Mobius needed for HQY...
\item a normal family ${\cal L}$, 
\item a contour loop $L\in{\cal L}$, and
\item a dart $x$ visited by $L$.
%and 
%\item the canonical boolean-valued function $\varphi'$ on $H/{\cal L}$
%  (identified with a function $\varphi$ on ${\cal L}$ by
%  Lemma~\ref{lemma:quotient-bijection}).
\end{itemize}
Such a tuple is a \newterm{marked hypermap} if
the following conditions hold.
\begin{enumerate}
\item The quotient $H'=H/{\cal L} = (D',e',n',f')$ is simple.  
\item $n'$ has no fixed points on $D'$.
\item $x$ is followed by an $f$-step in the loop: $L = \lp{x;fx;\ldots}$.
%\item $\varphi$ coincides with the natural boolean function on ${\cal L}\setminus \{L\}$.
%\item $\varphi$ is false on the loop $L$.
\item The contour loop $L'\in {\cal L}$ that visits%
\footnote{$L$ visits  $f x$.  By the definition of normality, some contour loop in
${\cal L}$ visits the dart $n f x$ at the same node.} 
$n f x$ is canonically true.
\item 
  $\check\varphi_{can}$ is an $S'$-flag on $H'$, where $S'$ is the image of 
  $S(H,L,x)$ in $D'$.  %Lemma~\ref{lemma:flag-set-quotient}.
  %under the identification of $\varphi$ with a boolean-valued
  %function on $H'$ ().
\end{enumerate}
\end{definition}

%\item $H$ is a hypermap.
%\item $\cal L$ is a normal family of $H$.
%\item $L$ is a contour loop in ${\cal L}$.
%\item $x$ belongs to the dart set of $L$.
%\item $\varphi$ is a boolean function on the faces of $H'$.
%\item The set $S(H,L,x)$ maps bijectively to a set $S'$
%of darts in the quotient.






\begin{example}[illustration]\label{ex:graph-gen}  
  This example illustrates the markup (Figure~\ref{fig:graph-gen}).
  In the figure, the hypermap $H$ is represented as a planar graph.
  The contour loops are represented by left-side shadings of the edges
  of the planar graph.  The shaded edges give the edges of a planar
  graph representing the quotient.  The polygons that are fully shaded
  are true.  Two polygons in the quotient are false.  A dart of $H$ in
  a false contour loop $L$ in $H'$ is marked $x$.  By inspection,
  $S(H,L,x)=\{f x,f^2 x,f^3 x\}$.  By inspection,
  $\check\varphi_{can}$ is an $S'$-flag.  (In fact, the darts in true
  faces form a connected region.  Every edge in the quotient except
  $\{f' x', e' f' x'\}$ meets a true face, and this one edge meets
  $S'$.)
%% XX recheck
\end{example}

\begin{figure}[htb]
\centering
\szincludegraphics[width=90mm]{\pdfp/graph_gen.eps}
\caption{An example of the current situation.}
\label{fig:graph-gen}
\end{figure}

\begin{definition}[$m$,$p$,$q$,$y$,$z$]\guid{BVUFRRE}\label{def:yz}
Let $(H,{\cal L},L,x)$ be a marked hypermap.
Several lemmas use the following natural numbers $m,p,q$ and darts $y,z$.
Set  $y = f^{m+1} x$, where $m = \card(S(H,L,x))$.
  Set
$z=f^{p+1} y$, where $p$ is the smallest natural number 
%Then there is a smallest $p\ge0$,
such that some contour loop in ${\cal L}$ visits $f^{p+1} y$.
Let $x'$ and $z'$ be the images of $x$ and $z$ respectively in $H/{\cal L} = (D',e',n',f')$.
Let $q$ be the smallest natural number such that $z' = (f')^{q+1} x'$.  
\end{definition}
\indy{Notation}{m@$m$ (face map exponent)}
\indy{Notation}{p@$p$ (face map exponent)}
\indy{Notation}{y@$y$ (dart)}
\indy{Notation}{z@$z$ (dart)}

The existence of $q$ follows from the following lemma showing that $x'$ and $z'$
lie in the same face $\F(L)$ of $H/{\cal L}$.  (The existence of $q$ is trivial
when $L$ is canonically true.)

\begin{lemma}[loop confinement]\guid{HQYMRTX}\rating{200} \label{lemma:yz}
Let $(H,{\cal L},L,x)$ be a marked hypermap.
Assume that  $L$ is canonically false. % canonically true assumption not needed.
Let the natural number $m,p$ and darts $y,z$ be given by Definition~\ref{def:yz}.
Then, $L$ visits $z$, and $z\ne f^k x$ when $0 < k \le {m+1}$.
\end{lemma}

\begin{proof} 
%We break the proof into two cases, depending on whether $L$ is canonically true.
%
%\claim{[$L$ is canonically true.]}  In this case $S=\emptyset$,  $m=0$, $p=0$, $k=1$,
%$y = f x$, and $z = f y = f^2 x$.  Also, $L = \lp{x;f x;f^2 x;\ldots}$, and it clearly
%visits $z$.  If $z=f^k x = y$, then $x=y=z$ is a fixed point of $f$.
%
%This completes the first case.
%
%\claim{[$L$ is canonically false.]} In this case,
  For a contradiction, suppose $f^{p+1} y = f^k x$, for some $0<k\le
  m+1$.  Then also, $f^p y = f^{k-1} x$.  If $p>0$, then this
  contradicts the minimality of $p$.   (Note that $L$ visits $f^{k-1}x$ by the
definition of $S$ and $m$.)  So $p=0$, and $y=f^{k-1} x =
  f^{m+1} x$.  Also, $0\le k-1 < {m+1}$, which implies that the face of $x$ has size at
most $m+1$.  This  forces $L$ to be canonically true, which is contrary to assumption.  This proves the second
  conclusion of the lemma.  In particular, $z\not\in S$, where $S =
  S(H,L,x)$.

Let $L'$ be the contour loop of ${\cal L}$ that visits $z$.  For a contradiction,
assume that $L'\ne L$.

\claim{$L'$ is false.}  Otherwise, $L'$ is true with respect to the
canonical flag and it therefore a loop consisting entirely of
$f$-steps.  In particular, $L'$ visits $z,x$, and $y$.  This is
contrary, to the assumption that the contour loop containing $x$ is
false.

Let $H' = (D',e',n',f') = H/{\cal L}$, and let $S'$ be the image of $S$ in $D'$.
Let $z' = [\ldots;u]\in D'$ be the
image of $z$ in $D'$.    
As $z\not\in S$, we also have $z'\not\in S'$.
By the definition of
$S'$-flag, the dart $e'z'$ lies in a true face or $e'z'\in S'$.  
This disjunction splits
splits the proof into two cases.
\begin{enumerate}
\item\claim{[$e'z'$ lies in a true face.]}  In this case, 
\begin{displaymath}
e'z' = f^{-1} n^{-1} [\ldots,u] = f^{-1} [n^{-1}u;\ldots],
\end{displaymath}
so that $n^{-1} u$ is visited by a true contour loop.
Consider the following
path in $H$:
\begin{displaymath}
[y;fy;\ldots;z] @ [n^{-1}z;\ldots;u] @ [n^{-1} u;\ldots;n^{-1} x].
\end{displaymath}
The first segment consists of $f$-steps; the second of $n^{-1}$-steps;
and the third segment exists within true contour loops of ${\cal L}$
by the connectedness of true faces (by properties of flags).  This
path satisfies all the assumptions of Lemma~\ref{lemma:contour-f}.  (In particular,
the node of $n f x$ (or of $f x$)  is distinct from the node of $x$ by the
assumed simplicity of the quotient $H/{\cal L}$.)
The lemma asserts that the path does not exist.
\item 
\claim {[$e'z'\in S$.]}  In this case,  
$f^{-1}n^{-1}u \in S$ and $L$ visits $n^{-1} u$ at the node of $z$.
Consider the following path of $f$-steps in $H$:
\begin{displaymath}
[y;f y;\ldots;z].
\end{displaymath}
This path satisfies all the enumerated conditions of
Lemma~\ref{lemma:contour-f}.  (In particular, $y$ and $z$ are at
different nodes by the assumed simplicity of the quotient $H/{\cal
  L}$.)  The lemma asserts that the path does not exist.
\end{enumerate}
\end{proof}

\begin{lemma}[parameters]\guid{QRDYXYJ}\label{lemma:parameters}
Let $(H,{\cal L},L,x)$ be a
marked hypermap, where $H$ is restricted. Assume that $L$ is canonically false.
Let $m,p,q$ be the natural numbers and let $x,y,z$ be the darts given by
Definition~\ref{def:yz}.  Let $r = \op{card}(\F(L))$.  Then
\begin{displaymath}
0\le p,\quad 0\le m < q < r,\quad m+1 < p+q.
\end{displaymath}
Furthermore, the darts $x$, $y$ belong to different nodes; the darts
$y$ and $z$ belong to different nodes.
\end{lemma}

\begin{proof}
Let $H/{\cal L}=(D',e',n',f')$.  Let $y'$ and $z'$ be the images of $y$ and $z$
in $D'$, respectively.  Let $S'$ be the image of $S(L,x)$ in $D'$.
By definition $m = \op{card}(S(L,x))$.  Both $m$ and $p$ are natural numbers,
so $0\le p$ and $0\le m$. 

\claim{The darts $x$, $y$ belong to different nodes; the darts $y$ and
  $z$ belong to different nodes.}  Indeed, $x$, $y$, and $z$ belong to
the same face.  By the simplicity of $H$, if two of these darts belong
to the same node, then they are equal to each other.  However, $x\ne
y$, for otherwise the subpath $P=[x;f x;\ldots;f^{m+1}x]$ gives a canonically true contour loop, which is contrary to the assumption that $L$ is
canonically false.  Also, $y\ne z$, for otherwise the face of $y$ is
equal to $\{y,f y,f^2 y,\ldots,f^p y\}$.  It follows that $x = f^k y$
is visited by $L$, for some $1<k\le p$.  This contradicts the defining
minimality property of $p$.

\claim{$q<r$.}  Indeed,
by definition, $z' = (f')^{q+1} x'$ and no smaller natural
number has this property.  Also, $x'$ and $y'$ belong to the face $\F(L)$.
If $q\ge r$, then $(f')^{q+1} = (f')^{q-r+1}$, which contradicts the minimality of $q$.

\claim{$m<q$.} Indeed, if $0\le k< m$, then
\begin{displaymath}
(f')^{k+1} x' = [f^{k+1} x]\in S', \quad z' \not\in S',
\end{displaymath}
by Lemma~\ref{lemma:flag-set-quotient} and Lemma~\ref{lemma:yz}.
Thus, $0\le q< m$.  Also, $q\ne m$, for otherwise $z' = (f')^{m+1} x'
= y'$.  This implies that $z$ and $y$ lie in the same node, which has
been proved impossible.  This completes the proof that $m<q$.
 
\claim{$m+1 < p+q$.}  Indeed, the inequalities $0\le p$ and $m<q$ imply
that either $m+1 < p+q$ or $p=0~\land~m+1=q$.  The second disjunct
cannot hold, for otherwise $z' = (f')^{q+1} x' = f' y'$.  Write $y' = [y;\ldots;u]$.
This is not a unit path by the definitions of $S(L,x)$ and $m$, so $y\ne u$, but
they lie at the same node.  Also from $p=0$ it follows that $z= f^{p+1} y = f y$.
So $e y$ and $e u$ both lie at the node of $z'$.  The existence of
 two edges, $\{y, e y\}$ and
$\{u, e u\}$, between two nodes contradicts the hypothesis
on $H$ of no double joins.  This proves the claim and the lemma.
\end{proof}

\subsection{transform}

\begin{definition}[transform]\guid{YQANQNF}
  From one marked hypermap $(H,{\cal L},L,x)$ in which $L$ is
  canonically false, we construct a new tuple
\begin{displaymath}
T(H,{\cal L},L,x) = (H,{\cal M},L_1,x),
\end{displaymath}
 called the \newterm{transform} of
  $(H,{\cal{L}},L,x)$.  
As the notation indicates, the hypermap $H$ and the dart $x$ are the same for both
tuples.  The data ${\cal M}$ and $L_1$ are specified in
the following paragraphs.
\end{definition}

Let $m$, $p$, $y$, and $z$ be  given by
Definition~\ref{def:yz}.
Let $L_1$ be 
%With improvised notation, write
%\begin{displaymath}
%L_1 = \lp{L[x:y] \opat P[y:z] \opat L[z:x]},
%\end{displaymath}
the contour loop in $H$ that follows $L$ from $x$ to $y$, then takes
$f$-steps from $y$ to $z$, then continues along $L$ back to $x$.  
Let $L_2$ be
%\begin{displaymath}
%L_2 = \lp{L[n^{-1}y:n z] \opat P^c[n z:n^{-1}y]},
%\end{displaymath}
the contour loop in $H$ that follows $L$ from $n^{-1} y$ to $n z$,
then complements the path of $L_1$ from $y$ to $z$, traveling instead
from $n z$ to $n^{-1} y$. 


Set
\begin{displaymath}{\cal M} = ({\cal L}\setminus \{L\}) \cup
\{L_1,L_2\}.\end{displaymath}

\begin{remark}[canonical compatibility]\guid{HBLIYVM}
There is a canonical boolean function on ${\cal L}$ and one on ${\cal M}$.
The canonical boolean functions agree on the intersection ${\cal L}\cap {\cal M}$.
This means, there is a well defined boolean-valued function on 
${\cal L}\cup {\cal M}$.  There is no ambiguity.  
\end{remark}
\indy{Index}{function!boolean}%

%  Under the bijection between the set of faces of a quotient $H/{\cal
%    L}$ and the family ${\cal L}$ (of
%  Lemma~\ref{lemma:quotient-bijection}), an $S$-flag on a quotient
%  $H/{\cal L}$ can be identified with a boolean function on ${\cal L}$
%  satisfying appropriate properties.  With this in mind, 
%\begin{displaymath}
%\begin{cases}
%\psi(L) = \varphi(L), &\text{if } L\ne L_1,L_2,\\
%\psi(L_2) = \op{true}, &\text{if $L_2$ is canonically true},\\
%\psi(L) = \op{false}, &\text{otherwise}.\\
%\end{cases}
%\end{displaymath}




\begin{figure}[htb]
\centering
\szincludegraphics[width=80mm]{\pdfp/L1L2.eps}
\caption{The loop $L$ is replaced with two loops $L_1, L_2$.}
\label{fig:L1L2}
\end{figure}



%Each of the two
%loops $L_i$ meets at least two nodes (those
%of $y$ and $z$) and has length at least three by
%Lemma~\ref{lemma:3dart}.  
% $S(H,L_1,x)$ contains the proper subset $S(H,L,x)$.


\begin{lemma}[markup transform]\guid{AQIUNPP}\rating{600}\tlabel{lemma:flag} 
Let $H$ be a restricted hypermap.
If $(H,{\cal L},L,x)$ is a marked hypermap such that $L$
is canonically false,  then the transform
$(H,{\cal M},L_1,x)$ 
is also a marked hypermap.
\end{lemma}

\begin{proof} Let 
\begin{displaymath}
H=(D,e,n,f),\   H' =(D',e',n',f') = H/{\cal
    L},\  H'' = (D'',e'',n'',f'') = H/{\cal M}.   
\end{displaymath}
Let $S'$ be the image of $S(H,L,x)$ in $D'$.  Let $y,z$ be the darts
constructed in Definition~\ref{def:yz} from the marked hypermap $(H,{\cal
  L},L,x)$.  The dart $z$ is not at the same node as $y$ (by
the simplicity of $H$).

  The proof can be organized into independent parts, according to the separate
  properties of a marked hypermap.  The first part of the proof
  establishes that ${\cal M}$ is a normal family.


\case{normal-1} \claim{No dart is visited by two different loops.}
Indeed by construction, the sets of darts of $L_1$ and $L_2$
are disjoint from each other and disjont from the sets of darts of $L'\in
{\cal L}\setminus \{L\}$.  The result now follows from the normality of  ${\cal L}$.

\case{normal-2} \claim{Every loop visits at least two nodes.}  Indeed, this
is true for $L_1$ and $L_2$ because they visit the nodes of $y$ and
$z$.  It is true of the other loops because they belong to the
normal family ${\cal L}$. 

\case{normal-3} \claim{If a loop visits a node, then every dart at
that node is visited by some loop.} 
% Indeed, the new nodes visited on the
%sublist of $L_1$ from $y$ to $z$ do not contain any darts in any
%loop in ${\cal L}$.  (By the definition of normal family, if a loop
%visits a node, then every dart at that node is visited by some loop
%of ${\cal L}$.)  These new nodes contain two darts, with one dart
%along $L_1$ and the other along $L_2$.  The paths $L_1$ and $L_2$
%visit every dart visited by $L$.  They visit every dart at the nodes
%of $y,\ldots,z$.
Indeed, the nodes that are visited by some loop in ${\cal M}$ are
precisely those visited by some loop in ${\cal L}$, together with the
``new'' nodes; that is, the nodes of $f y,\ldots,f^p y$.  The set of
darts that are visited by some loop of ${\cal M}$ is the union of the
set visited by loops in ${\cal L}$, together with two darts at the
new nodes.  As each new nodes has only two darts, and as ${\cal L}$ itself
it normal, the result follows. It follows that ${\cal M}$ is normal.  


\case{simple} To prove the simplicity of the quotient, it is enough to show that
none of the contour
loops in ${\cal M}$  ever return to a node after leaving it.
 (More precisely, the dart set of any $L'\in{\cal M}$ intersected with a node
is the dart set of a maximal sublist $[z;n^{-1}z;\ldots;n^{-k}z]$ of $n^{-1}$ steps.)
This is true of $L'\in{\cal L}\setminus\{L\}$ by assumption and true of
$L_1$ and $L_2$ by construction.  Simplicity follows.

\case{fixed-point free} By Lemma~\ref{lemma:nfp}, to prove that $n''$
does not have a fixed-point, it is enought to show that no loop in
${\cal M}$ has a dart set containing a node.  It is sufficient to
consider the loops $L_1$ and $L_2$.  The set of darts of $L_1$ and
$L_2$ at the old nodes (that is, those not meeting $\{f y,\ldots,f^p
y\}$) are subsets of the set of darts of $L$ at those nodes.  As the
dart set of $L$ does not contain an old node, neither do $L_1$ and
$L_2$.  At the new nodes, $L_1$ and $L_2$ both have at least one dart,
so neither contains the entire node.  It follows that $n''$ is
fixed-point free.

\claim{$e'y'\not\in S'$, where $y'$ is the image of $y$ in $H'$. }
Otherwise, write $y' = [y;n^{-1}y;\ldots;u]$, a sublist of $L$, and
pick $k$ such that $e'y' = [f^k x]\in S'$.  Then
\begin{displaymath}
(n')^{-1} y' = f'e'y' = f'[f^k x] = [f^{k+1}x;\ldots].
\end{displaymath}
By the construction of $S(H,L,x)$, we know that $L$ visits $f^{k+1}x$.
Hence $y'$ and $(n')^{-1}y'$ both lie in the same node and in the same
face $\F(L)$.  By the simplicity of $H'$, it follows that $y' =
(n')^{-1} y'$.  That is, $y'$ is a fixed point of $n'$.  This is contrary to
assumption. The claim follows.


\claim{$e'y'$ lies in a true face of $H'$.}  Indeed,
 since $\check\varphi_{can}$ is
an $S'$-flag on $H'$, the edge $\{y',e'y'\}$ meets a true face or
$S'$.  However, $y'\not\in S'\subset \F(L)$, by the simplicity of $H$.  Also,
$e'y'\not\in S'$, by the previous paragraph.
The dart $y'$ lies in the false face $\F(L)$.  The only
remaining possibility is that $e'y'$ lies in a true face.  



\case{flag-1} \claim{The true faces of $H''$ are
  connected.}  Indeed,  $L_1$ is connected to a true face
by the contour path $[x;n^{-1} x]$, because $n^{-1} x$ lies
in the same face as $n f x$, which is a true face by assumption.
If $L'\in{\cal M}\setminus\{L_1,L_2\}$ is a true, then $L'\in  {\cal L}$, and
it connects with the true faces of ${\cal L}$ as before.
If $L_2$ is true, then the proof requires more
argument.   Write $y'=[y;\ldots;u]$ as above.
The dart $u''=[n^{-1}y;\ldots;u]$ of $H''$ lies in the face $\F(L_2)$.
Also, 
\begin{displaymath}
(n'')^{-1} u''= [n^{-1} u;\ldots] = [f e u;\ldots] = f'' [\ldots; e u]
  = f'' e' y'
\end{displaymath}
Thus, we have a contour path from $u''$ to $e'' y'$, which lies in a
true face, by an earlier claim.  (The dart $e'y'$ is naturally
identified with the dart $e'' y'$ on in $H''$, because the faces are
true.)  Hence a path exists from the true face $\F(L_2)$ into another
true face, and from there any true face may be reached.

\case{flag-2} \claim{Each edge of $H''$ meets a true face or $S''$, where $S''$ is
the image of $S(H,L_1,x)$ in $D''$.}
Indeed, the function $\check\varphi_{can}$ is an $S'$-flag on $H'$.  The edges of
$H'$ can be identified with a subset of the edges of $H''$.  For this
subset, the flag condition on edges is immediate.  Consider the
new  edges  (that is, edges of $H''$ that are not in $H'$).
They all meet $\{y,f y,\ldots,f^p y\}$.  This set is either contained
ini $S(H,L_1,x)$ (when $F_1$ is false), or is contained in the true face
$F_1$ (when $F_1$ is true).  Hence each new  edge meets
a true face or $S''$.


The other verifications are routine.
\end{proof}


\subsection{digraph}

The aim is to prove that every restricted hypermap with a given bound
on the size of the dart set is generated by a particular algorithm.
The proof is a long induction argument.  The proof starts by showing
how to go from one partially constructed hypermap to another more
fully constructed hypermap.  The hypermap $H$ represents the fully
constructed hypermap and two quotients $H/{\cal L}$ and $H/{\cal M}$
represent the partially constructed hypermaps.  The algorithm involves
the transition from the hypermap $H/{\cal L}$ to $H/{\cal M}$.  The
transition from one quotient to another is given by the transform of
marked hypermaps.  The transform thus represents one step of the
algorithm.  \indy{Notation}{M@$\cal M$ (normal family)}%


\begin{definition}[digraph,~vertex,~edge,~head,~tail,~sink,~path]\guid{AVXKIRW}
  A \newterm{digraph} (directed graph) is an ordered pair $(V,E)$
  where $V$ is any set, and $E$ is a set of ordered pairs of vertices.
  An element of $V$ is called a \newterm{vertex}.  An element of $E$
  is called a \newterm{directed edge}.  If $(v,w)\in E$, then $v$ is
  the \newterm{head} and $w$ is the \newterm{tail} of the directed
  edge.  A vertex $v$ is a \newterm{sink} if it is not the head of any
  directed edge.  A path $P=[v_0;v_1;\ldots;v_{k-1}]$ in a digraph is
  a list of vertices, such that $(v_i,v_{i+1})\in E$ for all $i<k-1$.
\end{definition}

\begin{definition}[digraph of a hypermap]\guid{QQHIKFL}
Let $H$ be a restricted hypermap.  Form a directed graph as follows.
The vertex set $V$ of the digraph is the set of all marked hypermaps 
${\cal H}=(H,{\cal L},L,x)$ such that $L$ is canonically false.  
Write $T{\cal H} = (H,{\cal M},L_1,x)$, where $T$ is the transform on $V$.  
The set of tails of directed edges with head ${\cal H}$ is as follows:
\begin{itemize}
\item If every contour loop of ${\cal M}$ is canonically true, then ${\cal H}$ is a sink.
\item If $M$ is canonically false, 
then there is a single tail $T{\cal H}$.
\item If $M$ is canonically true, but not every loop in ${\cal M}$ is canonically true,
then the tails are $(H,{\cal M},M,y)$, where $M$ is a 
canonically false loop in ${\cal M}$ and $y$ is a dart visited by $M$ such
that $y$ is followed by an $f$-step:  $M = \lp{y;f y;\ldots}$.
\end{itemize}
\end{definition}

\begin{lemma}[digraph sink]\guid{XCOXWYJ}\label{lemma:digraph-sink}
Let $(V,E)$ be the digraph of a restricted hypermap $H=(D,e,n,f)$.  Then every path
in $(V,E)$ reaches a sink after at most $\#D$ steps.  Moreover, if ${\cal H}$ is
a sink, and if $T{\cal H} =(H,{\cal M},L_1,x)$ is its transform, then
$H$ is naturally isomorphic to $H/{\cal M}$, under the map that sends
a dart $y$ of $H$ to the unit path $[y]$.
\end{lemma}

\begin{proof} Each step in the path makes one transform.  Each
  transform increases the number of darts visited by the normal
  family.  The number of darts visited by a normal family is bounded
  by $\#D$.  This bounds the length of a path.

  The condition on a sink ${\cal H}$ is that every contour loop of
  ${\cal M}$ is canonically true. 

\claim{${\cal M}$ visits every dart.}  Indeed, let $y$ be any dart of
$H$. Since $H$ is assumed connected, there exists some contour path
$P=[x;\ldots;y]$.  Let $u$ be the last dart on the path that is
visited by ${\cal M}$.  Let $M\in{\cal M}$ be the contour loop that
visits $u$.  Then $u$ is not followed by an $f$-step, because every contour
loop is canonically true: $M = \lp{u;f u;\ldots}$.
Nor is $u$ followed by a $n^{-1}$ because in a normal family every dart
at the node of $u$ is visited by a loop in ${\cal M}$.  Thus, $u$ is the final
dart in the path $P$, which means that $y$ is visited by ${\cal M}$.

It follows that is a bijection between the dart set of $H$ and the
dart set of $H/{\cal M}$, sending a dart $y$ to the unit path $[y]$.
This bijection induces an isomorphism of hypermaps.
\end{proof}


%In the next lemma there are a number of choices to be made.  Let
%$\op{ch}$ be any function on the set of normal families ${\cal L}$ of $H$ containing
%at least one canonically false loop, returning a dart $x'$ of $H/{\cal L}$
%in a false face (with respect to the canonical boolean function).
%Given a normal family ${\cal L}$ (with a canonically false loop),  write
% $x' = [\ldots;x]$ and let $L$ be the the canonically false contour loop of ${\cal L}$ that
%visits $x'$.  Write ${\cal L}\rightsquigarrow (L,x)$.   The dart $x$ of $H$ has the property
%that $L = \lp{x;f x;\ldots}$; that is, $x$ is followed by an $f$-step in $L$.
%
%
%\begin{lemma}[sequence]\guid{YCFWVQJ}\label{lemma:sequence}  
%  Let $H$ be a restricted hypermap.  Fix a choice function $\op{ch}$ as
%  above, and let $F$ be any face of $H$. Then associated with $(\op{ch},F)$, there is a sequence of
%  marked hypermaps ${\cal H}_i = (H,{\cal L}_i,L_i,x_i)$ for
%  $i=0,\ldots,k-1$ such that
%\begin{itemize}
%\item $\varphi'_i$ is the canonical boolean function on $H/{\cal L}_i$, and
%$\varphi_i$ is its lift to ${\cal L}_i$.
%\item ${\cal L}_0$ is the normal family associated with the given face $F$,
%described in Example~\ref{ex:H2}, and ${\cal L}_0 \rightsquigarrow (L_0,x_0)$.
%\item If the contour loop $L$ in the transform $T{\cal H}_i=(H,{\cal M},L,\ldots)$ 
%is canonically false, then ${\cal H}_{i+1} = T{\cal H}_i$.
%\item If the contour loop $L$ in the transform $T{\cal H}_i=(H,{\cal
%    M},L,\ldots)$ is canonically true, and there exists some other
%  contour loop of ${\cal M}$ that is canonically false, then ${\cal
%    H}_{i+1}=(H,{\cal M},M,y,\varphi_{i+1})$, where ${\cal M}\rightsquigarrow (M,y)$. 
%\item The transform $(H,{\cal L}_k,\ldots)$ of $\,{\cal H}_{k-1}$ has
%  quotient $H/{\cal L}_k$ that is naturally isomorphic to $H$.
%\end{itemize}
%\end{lemma}
%
%\begin{proof}
%  Let ${\cal H}_0=(H,{\cal L}_0,L_0,x_0,\varphi_0)$ be given as
%  follows.  Pick any face $F$ of $H$, and construct the normal family
%  ${\cal L}$ of Example~\ref{ex:H2}.
%
%  The construction of ${\cal H}_{i+1}$ depends on the structure of the
%  transform $(H,{\cal M},L,x)$ of ${\cal H}_i$.  We consider
%  three cases.
%\begin{nomerate}
%\item 
%\claim{[Every contour loop of ${\cal M}$ is canonically true.]}   In this case,  by
%Lemma~\ref{lemma:all-dart}, the quotient map $H\to H/{\cal M}$ is an
%isomorphism.  Set $k= i+1$. The sequence terminates.
%\item \claim{[The contour loop $L$ is canonically false.]}  In this
%  case, from the inductive hypothesis that $\varphi'_i$ is the
%  canonical function and the definition of $\psi$, it follows that
%  $\psi'$ is the canonical function.  By Lemma~\ref{lemma:flag}, the
%  transform is a marked hypermap.  Let ${\cal H}_{i+1}$ equal the
%  transform of ${\cal H}_i$.
%\item \claim{[The contour loop $L$ is canonically true, but not all contour loops in
%     ${\cal M}$ are canonically true.]}  In this
%  case, the canonical function on $H/{\cal M}$ is a flag (with
%  $S=\emptyset$).   
%${\cal H}_{i+1}$, as defined, is a marked
%  hypermap.
%\end{nomerate}
%
%The sequence must terminate eventually, because each step
%constructs a quotient of $H$ with more darts than the previous one and
%the number of steps is bounded by the size of the dart set of
%$H$.
%\end{proof}

Next, we wish to describe the directed edges in a way that relies to a
lesser degree on the structural details of the marked hypermaps.
(These details will not be available to us in the algorithm of the
next subsection.)  The next lemma uses reverse doble walkup
transformations to construct a new hypermap from a given hypermap $H'$
that does not require us to represent it first as a quotient $H' =
H/{\cal L}$ for some normal family.
We can immediately relate the to the digraph we
have constructed.   

\begin{lemma}[walkup-digraph]\guid{ISMLATS}\label{lemma:RDW}
Let ${\cal H} =(H,{\cal L},L,x)$ be a marked hypermap,
  with $H$ restricted.  Assume that $L$ is canonically false, and let 
$(H,{\cal{M}},M,x)$ be 
  the transform of $\cal H$.   
Let $m$, $p$, $q$, $y$, and $z$ be
  the constants of Definition~\ref{def:yz}.  Let $x'$ be the
  image of the dart $x$ in $H/{\cal L} = (D',e',n',f')$.
%Then $m,p,q$ satisfy the constraints of
%  Definition~\ref{def:R}, and 
Then $H/{\cal M}$ is isomorphic to $RDW(H/{\cal{L}},x',m,p,q)$.
\end{lemma}


\begin{proof}
%The constraints on $(m,p,q)$ of Definition~\ref{def:R} are satisfied 
%by Lemma~\ref{lemma:parameters}.
%
  By construction, the passage from $H/{\cal M}$ to $H/{\cal L}$
  consists of a double walkups to eliminate the nodes (of size two) at
  $f y$, $f^2 y, \ldots, f^p y$, and then a double walkup to eliminate
  the edge that runs from the the node of $y$ to the node of
  $z$.  %The passage in the other direction from $H/{\cal L}$ to
%$H/{\cal M}$ comes by adding an edge from the node of $y$ to
%the node of $z$ and inserting $p$ new nodes (of degree two)
%along it.
If we play these double walkups in reverse,
one may also pass from $H/{\cal L}$ to $H/{\cal M}$.  
%If
%$m,p,q$ are chosen as above, 
Then $RDW(H/{\cal L},x',m,p,q)$ is isomorphic to
$H/{\cal M}$.  
\end{proof}


\begin{figure}[htb]
\centering
\szincludegraphics[width=80mm]{\pdfp/L1L2dart.eps}
\caption{$H/{\cal L}$ is obtained from $H/{\cal M}$ by double walkup
transformations.}
\label{fig:L1L2dart}
\end{figure}



\subsection{algorithm}

This final section puts the algorithm a precise form, based on
the Knaster-Tarski fixed point theorem.  The Knaster-Tarski fixed
point theorem is a common way to give precise mathematical form to
an algorithm.

\begin{lemma}[Knaster-Tarski]\guid{EAOGWLE}\rating{ZZ}   
Let $X$ be a set.  Let $f:\powerset(X)\to \powerset(X)$ be a
function from the powerset of $X$ to itself.  Assume that $f$ is
monotonic in the sense that whenever $Y\subset Z\subset X$, it
follows that $f(Y) \subset f(Z)$.  Then $f$ has a least fixed point.
That is there exists a set $\op{fix}(f,X)\subset X$ such that
$f(\op{fix}(f,X)) = \op{fix}(f,X)$ and such that the following
minimality condition holds: if $Y\subset X$ is any set such that
$f(Y) \subset Y$, then $Y\subset \op{fix}(f,X)$.
\end{lemma}
\indy{Notation}{X@$X$ (set)}%
\indy{Notation}{f@$f$ (function on powerset)}%
\indy{Notation}{fix@$\op{fix}$~(Knaster-Tarski fixed point)}%
\indy{Notation}{P@$\powerset(\cdot)$ (powerset)}%
\indy{Index}{Knaster-Tarski fixed point theorem}%

\begin{proof} Let $\op{fix}(f,X)$ be the intersection of all subsets
$Y$ of $X$ such that $f(Y)\subset Y$.  It is easily verified that
this set has the required properties.
\end{proof}

Various data are needed for the application of the Knaster-Tarski
fixed-point theorem to the construction of restricted hypermaps.  This
data is presented in a series of definitions.  The first definition
gives a domain $\Omega$ that will contain all the darts of all the
hypermaps that will be constructed by the algorithm.  In practice, we
take $\Omega$ to be a finite subset of the set of natural numbers.

\begin{definition}[$\Omega$,~$\op{ch}$,~$d$]\guid{LVXTTSP}
  Let $\Omega$ be any fixed finite set.  Fix a choice function
  $\op{ch}:\powerset(\Omega)\to \Omega$ that picks an element from
  each nonempty subset:
\begin{displaymath}
X\ne\emptyset\quad  \Rightarrow \quad  \op{ch}(X)\in X.
\end{displaymath}
For example, when
$\Omega$ is a well-ordered set, let $\op{ch}$ choose the least element of a subset
of $\Omega$.  An $(\Omega,d)$-hypermap is a hypermap whose dart set is a
subset of $\Omega$ and such that $d$ is the maximum face size.  A
$(\card(\Omega),d)$-hypermap is one isomorphic to an $(\Omega,d)$-hypermap.
(In applications later in the book, $\card(\Omega)\le14$ and $3\le d\le 6$.) 
\end{definition}
\indy{Notation}{d3@$d$ (upper bound)}%
\indy{Notation}{zzZ@$\Omega$ (set of darts)}%
\indy{Notation}{ch@$\op{ch}$~(choice)}%


The following constructions depend on $\Omega$ and $d$, although the
notation does not reflect this.  One may think of $X_1$ in the next
definition as holding the output of the algorithm and $X_2$ as the
workspace that holds partially constructed hypermaps.

\begin{definition}[$X$,~$X_1$,~$X_2$]\guid{KFPEPWO}
Define a set $X$ as the disjoint union of $X_1$ and $X_2$ as follows.
Let $X_1$ be the set of all $(\Omega,d)$-hypermaps.
Let $X_2$ be the set of tuples $(H,m,\check\varphi,x)$, where 
\begin{itemize}
\item $H$ is an $(\Omega,d)$-hypermap,
\item $\check\varphi$ is an $S$-flag on $H$,
\item  $x$ is a dart in a false face of $H$ (with respect to $\check\varphi$),
\item $m\in\{0,\ldots,d-1\}$, 
and
\item $S = \{f^i x\mid 1 \le i \le m\}$.
\end{itemize}
\end{definition}


The set $A$ in the following definition gives the initialization of the algorithm.

\begin{definition}[A]\guid{JBUOJMF}
Let $H$ be a fixed hypermap isomorphic to $H_{2d}$, with darts in $\Omega$.
Let $\check\varphi$ be the flag on $H$ (with one true face and one
false face).  Let $x$ be the value of the choice function on the false
face.  Set
\begin{displaymath}
A = \{(H,m,\check\varphi,x) \mid 0\le m \le d-1\} \subset X_2.
\end{displaymath}
\end{definition}

The following set gives the indexing set for the iteration of the algorithm.

\begin{definition}[C]\guid{IDDKWYX}
Let $C$ be the set of of $4$-tuples $(m,p,q,r)$ that satisfy the following
constraints:
\begin{itemize}
\item $0\le m < q < r$.
\item $0\le p$.
\item $m+1 < p+q$.
\item $m+p+2 \le d$.
\item if $q+1< r$, then $m+p+3\le d$.
\end{itemize}
\end{definition}

\begin{definition}[extension]\guid{ZMVBANY}  
  Let $(H,m,\check\varphi,x)\in X_2$.  Choose $p,q$ such that $(m,p,q,r)\in
  C$, where $r$ is the cardinality of the face of $x$.  Let $F$ be the
  face of $x$ in $H$.  It follows by construction, that every face
  $F'\ne F$ of $H$ is naturally identified with a face of
  $RDW(H,x,m,p,q)$.  Say that a boolean-valued function $\check\psi$ on the
  set of faces of $RDW(H,x,m,p,q)$ is an \newterm{extension} of
  $\check\varphi$ on $H$ if $\check\psi(F') =\check\varphi(F')$, when $F'\ne F$.
  \indy{Index}{extension}%
\end{definition}


The functions $f$ and $g$ give one iteration of the algorithm.  The
powerset-valued function $g$ is the heart of the algorithm.  It takes
one partially constructed hypermap and modifies it in various ways to
construct further partially constructed hypermaps.  When the flag
$\check\varphi$ permits, some hypermaps are also fully constructed.


\begin{definition}[g]\guid{DMAMRYR}
 Let $g:X_2 \mapsto \powerset(X)$ be given as
  follows.  Let $r$ be the cardinality of the face $F$ of $x$.  The
  subset $g(H,m,\check\varphi,x)\subset X$ is presented as a union of two
  sets:
\begin{displaymath}
   Y_i = X_i \cap g(H,m,\check\varphi,x).
\end{displaymath}
 $H'\in Y_1$ if and only 
\begin{itemize}
\item there exists
$p,q$ with $(m,p,q,r)\in C$ such that $H'=RDW(H,x,m,p,q)$,
and 
\item $\check\varphi(F')$ is true for all $F'\ne F$.
\end{itemize}
 $(H',m',\check\psi,x')\in Y_2$ if and only if
\begin{itemize}
\item there exists $p,q$ with $(m,p,q,r)\in C$, such that $H'=RDW(H,x,m,p,q)$,
\item  $\check\psi$ is an extension of
$\check\varphi$, and 
\item Let $F_1$ be the face of $x$ in $H'$.  One of the following two
  conditions hold:
\begin{itemize}
\item $\check\psi(F_1)$ is false;  $x' = x$; and  $p+m+1 \le m' < r$.
\item $\check\psi(F_1)$ is true; there exists a false face in $H'$; $x'$ is
  the value of the choice function on the union of false faces of
  $H'$; and $0 \le m' < r$.
\end{itemize}
\end{itemize}
\end{definition}


\begin{definition}[f]\guid{YSJTEDX}
Given the function 
$g:X_2 \to \powerset(X)$, set 
\begin{displaymath}f(S) = A \cup (\bigcup \{g(s) \mid s\in S\cap
X_2\}).\end{displaymath}
\indy{Notation}{g@$g$(function)}%
\end{definition}

Any function $f :\powerset(X)\to \powerset(X)$ of this form is
monotonic.  Thus, we have a Knaster-Tarski fixed point set
$\op{fix}(f,X)$.  The main result of this chapter is that a fixed
point construction generates all restricted hypermaps:

\begin{theorem}[hypermap algorithm]\guid{BRGEFNH}\rating{2000}  
\label{lemma:algorithm}
Define $f $ and $X$ as above (depending on $d\ge 3$ and $\Omega\ne
\emptyset$) .    Then every restricted hypermap with at
most $\card(\Omega)$ darts and such that the largest face has size  $d$
is isomorphic to a hypermap in $\op{fix}(f,X)\cap X_1$.
\end{theorem}


In informal terms, by starting with the \newterm{seed} hypermaps in $A$
one may find all restricted hypermaps (for given $\Omega$ and $d$) by
applying the function $f$ repeatedly:
\begin{displaymath}
A_0 = A = f(\emptyset),\quad A_1 = f(A_0),\quad A_2 = f(A_1),\ldots
\end{displaymath}
and by looking at the output $A_i \cap X_1$.
\indy{Index}{seed}%

The proof will be presented below.  The proof is a matter of correlating the Knaster-Tarski fixed point set with the digraph of a restricted hypermap $H$.
Write $\op{fix}_i$ for $\op{fix}(f,X)\cap X_i$.


\begin{definition}[correlation]\guid{SFBFNVW}
  A marked hypermap $(H,{\cal L},L,x)$ is said to be
  \newterm{correlated} to an element $(H',m',\check\varphi,x')\in X_2$ if
  the following conditions hold:
\begin{itemize}
\item $H/{\cal L}$ is isomorphic to $H'$ by some isomorphism $G$.
\item The image of $x$ in $H/{\cal L}$ maps to $x'$ under $G$.
\item The pull back of $\check\varphi$ under $G$ is the canonical function
$\check\varphi_{can}$ on $H/{\cal L}$.
\item $m = \card(S(H,L,x))$.
\end{itemize}
\end{definition}

\begin{lemma}[correlated seed]\guid{NRDWGYQ}\label{lemma:correlated-seed}
  Let $H$ be a restricted $(\card(\Omega),d)$-hypermap with digraph
  $(V,E)$.  There exist a marked hypermap in
  $V$ and an element in $\op{fix}_2$ that are
  correlated.
\end{lemma}

\begin{proof}  From the definition of $f$, it follows
that $A\subset \op{fix}_2$.  Thus, it suffices to correlate a marked
hypermap with an element of $A$.  Let $(H',\cdot,\check\varphi,x')\in A$.

Let $F$ be a face of $H$ of cardinality $d$.  Form the normal family ${\cal L}$ of example~\ref{ex:H2}.  The quotient $H/{\cal L}$ is isomorphic to $H_{2d}$, hence also isomorphic to $H'$.  The isomorphism can be chosen so that
$\check\varphi$ pulls back to $\check\varphi_{can}$ on the faces of $H/{\cal L}$, and so that $x'$ is the image of a dart $x$ in the canonically false 
face of $H/{\cal L}$.  

Let $m = \card(S(H,L,x))$.  Then $(H,{\cal L},L,x)$ is a marked hypermap that
is correlated with $(H',m,\check\varphi,x')\in A$.
\end{proof}

\begin{lemma}[correlated edge]\guid{FDQZOSJ}\label{lemma:correlated-edge}
  Let $H$ be a restricted $(\card(\Omega),d)$-hypermap with digraph $(V,E)$. 
  Assume that ${\cal H}\in V$ is not a
  sink and is correlated with an element in $\op{fix}_2$.
  Then there exists a directed edge  $({\cal H},{\cal H}')\in E$, such that
 $\cal H'$ is correlated with an element of  $\op{fix}_2$.
\end{lemma}

\begin{proof}  Assume that the marked hypermap ${\cal H}=(H,{\cal L},L,x)$
is correlated with the tuple $(H',m,\check\varphi,x')\in \op{fix}_2$,
by means of an isomorphism
\begin{displaymath}
G: H/{\cal L} \to H'.
\end{displaymath}
Let $T{\cal H} = (H,{\cal M},L_1,x)$ be the transform.  Let $m,p,q,x,y,z$
be the parameters of Definition~\ref{def:yz}.  Let $r$ be the cardinality
of $\F(L)$, which is equal to the cardinality of the face of $x'$ in $H'$.  
Then $(m,p,q,r)\in C$ by Lemma~\ref{lemma:yz}.

Let $H'' = RDW(H',x',m,p,q)$.  The isomorphism $G$ and Lemma~\ref{lemma:RDW} combine to give an isomorphism $G':H/{\cal M} \mapsto H''$.
Push the canonical function on the faces of $H/{\cal M}$ to a function
$\check\psi$ on the faces of $H''$.

If $L_1$ is false, then $({\cal H},T{\cal H})$ is a directed edge of the digraph, and
$T{\cal H}$ is correlated with $(H'',m',\check\psi,x')\in g(H',m,\check\varphi,x')\subset \op{fix}_2$, where $m'=\card(S(L_1,x))$.  

If $L_1$ is true, then $H''$ has a false face, and the choice function $\op{ch}$ picks a dart $x''$ in the union of the false faces of $H''$.  Transport this by $G$ to a dart $y'\in H/{\cal M}$ in a false face.  Write $y'=[\ldots;y]$
with $y$ a dart visited by a false contour loop $M$ of ${\cal M}$.  Then
$(H,{\cal L},L,x),(H,{\cal M},M,y))$ is a directed edge of the digraph,
whose tail is correlated with $(H'',m',\check\psi,x'')\in g(H',m,\check\varphi,x')\subset \op{fix}_2$, where $m'=\card(S(M,y))$.
\end{proof}

\begin{lemma}[correlated sink]\guid{RIZGJVS}\label{lemma:correlated-sink}
Let $H$ be a restricted $(\card(\Omega),d)$-hypermap with digraph $(V,E)$.
Then some sink in the digraph is correlated with some element of
$\op{fix}_2$.
\end{lemma}

\begin{proof} Start with any  correlated pair $({\cal
    H}_0,{\cal K}_0)$ with ${\cal H}_0\in V$ and ${\cal K}_0\in \op{fix}_2$ 
  (Lemma~\ref{lemma:correlated-seed}).  Use
  Lemma~\ref{lemma:correlated-edge}, to produce a sequence $({\cal
    H}_i,{\cal K}_i)$ of correlated pairs, where $[{\cal H}_0;{\cal
    H}_1;\ldots]$ is a path in the vertex set $V$.  By Lemma~\ref{lemma:digraph-sink},
  the path reaches a sink within $\#D$ steps.  The final marked
  hypermap ${\cal H}_k$ in the path is a sink that is correlated with
  ${\cal K}_k\in X_2$.
\end{proof}

\begin{proof} Turn to the proof of Theorem~\ref{lemma:algorithm}.  Let
  $(H,{\cal L},L,x)$ be a sink that is correlated with some tuple
  $(H',m,\check\varphi',y')\in \op{fix}_2$
  (Lemma~\ref{lemma:correlated-sink}).  By
  Lemma~\ref{lemma:digraph-sink}, $H$ is isomorphic to the quotient
  $H/{\cal M}$, where $(H,{\cal M},\ldots)$ is the transform of the
  sink.  This quotient is isomorphic to $\op{RDW}(H/{\cal
    L},x',m,p,q)$, where $x'$, $m$, $p$, $q$ are given by
  Definition~\ref{def:yz}.  By the correlatedness of the sink, $H$ is
  isomorphic to $H''=\op{RDW}(H',x',m,p,q)$.  By
  Lemma~\ref{lemma:parameters}, $(m,p,q,r)\in C$, where
  $r=\op{card}(\F(L))$.  Under the isomorphism, $r$ is the cardinality
  of the face of $y$ in $H'$.  By the definition of $f$ and $g$,
  $H''\in \op{fix}_1$.
\end{proof}





%
%
%\begin{proof} Let $H$ be a restricted hypermap whose dart set belongs
%  to $D$ and whose greatest face size is $d$.  It has a quotient
%  $H/{\cal L}_0$ isomorphic to $H_{2d}$.  Let ${\cal L}_i = (H,{\cal
%    L}_i,L_i,x_i,\check\varphi_i)$ be the sequence of marked hypermaps
%  constructed in Lemma~\ref{lemma:sequence}.  XX.
%\end{proof}
%
%\begin{proof} Let $H$ be a restricted hypermap whose dart set belongs
%to $D$ and whose greatest face size is $d$.  It has a quotient
%$H/{\cal L}_0$ isomorphic to $H_{2d}$.  By repeating the
%construction of Section~\ref{sec:face-insert}, one obtains a
%sequence of hypermaps $H_i = H/{\cal L}_i$, $i=0,\ldots,N$,
%terminating with a hypermap $H/{\cal L}_N$, which is isomorphic to
%$H$.  The data $m_i,\check\varphi_i,x_i$ is also obtained for each $H_i$.
%
%The tuple $(H_0,m_0,\check\varphi_0,x_0)$ is isomorphic to an element of
%$A$.  By construction, $A\subset \op{fix}(f,X)$.  If the lemma is
%false, there is a smallest $i>0$ for which
%$(H_i,m_i,\check\varphi_i,x_i)\not\in \op{fix}(f,X)$ (or if $i=N$, for which
%$H\not\in \op{fix}(f,X)$).  There are isomorphisms
%$RDW(H_i,x_i,m_i,p_i,q_i) \simeq H_{i+1}$ for appropriate choices of
%$p_i,q_i$.  By construction, when the data belongs to $\op{fix}(f,X)$
%for $i-1$, the data belongs to $\op{fix}(f,X)$ for $i$.  Thus, $H$
%belongs to $\op{fix}(f,X)\cap X_1$.
%\end{proof}
%
%\subsection{old algorithm}
%
%If a hypermap is restricted and $x$ is any dart, then $x$ and $n x$
% lie on different faces.  In particular, a restricted hypermap has at
% least two faces.  To begin the process, take ${\cal L}$ to be the
% normal family of Example~\ref{ex:H2} with two contour loops, whose
% quotient hypermap is a polygon $H_{2d}$.  When the the initial
% contour loop is chosen on a face of maximal size, the natural number
% $d$ is an upper bound on size of a face.
%
% In summary, a process starts with a single polygon and then adds
% edges and nodes of degree two along the inserted edges, to obtain a
% restricted hypermap $H$.
%
% A modification of the process avoids explicit reference to the
% hypermap $H$ and to the normal family ${\cal L}$.  Let $D$ be a
% finite set that contains all the darts for all of the restricted
% hypermaps to be constructed.  For each $d=3,\ldots,\# D$, the
% process generates all restricted hypermaps with greatest face-size
% $d$, with darts in $D$.
%
% The algorithms performs the following initialization.  The initial
% hypermap is the polygonal hypermap $H_{2d}$.  A flag $\varphi$ marks
% one face true and the other false.  A distinguished dart $x$ is
% selected on the false face.  For each $m<d$, set $S=S_m = \{f^i
% x\mid 1\le i\le m\}$.
%
% Each iteration processes a finite list ${\cal H}$ of quadruples
% $(H,m,\varphi,x)$, where $H$ is a simple hypermap, $\varphi$ is an
% $S$-flag, $x$ lies in a false face, and $S = \{f^i x\mid 1\le i\le
% m\}$ for some $m$.  The algorithm terminates when every face of
% every hypermap in ${\cal H}$ is true.  At every step of the
% algorithm, one quadruple with some false face is removed from ${\cal
%   H}$ and finitely many quadruples are returned to ${\cal H}$.
%
% At each iteration the chosen $(H,m,\varphi,x)$ is modified in the
% following ways and each modification is placed back in the list
% ${\cal H}$.  As the natural number $m$ depends on the unknown normal
% family ${\cal L}$, all possible $m < d$ are used.  Similarly, the
% natural number $p < d$ depends on ${\cal L}$, and all possible $p$
% are used.  Any quadruple that is isomorphic to one previously
% considered is discarded as redundant.
%
% In summary, the algorithm constructs of hypermaps.  The process must
% terminate, because the set $D$ is finite, so there are only finitely
% many quadruples (or quadruples up to isomorphism) that construct
% their dart sets from $D$.
%
%\begin{lemma}[]\guid{BRGEFNH}\rating{2000}  Fix $d\ge 3$ and $D\ne \emptyset$.
%  This process constructs in a finite number of steps all restricted
%  hypermaps, up to isomorphism, such that the dart set belongs to the
%  finite set $D$, and whose greatest face size is $d$.
%\end{lemma}




%%%%%%%%%%%%%%%%%


    
    \chapter{Hypermap}
    Use the electrical dual charge proof%
    \footnote{\url{http://www.ics.uci.edu/~eppstein/junkyard/euler/all.html}} of the Euler formula for convex polyhedra (a combinatorial Morse function).  This does not seem to need an explicit invocation of the Jordan curve theorem, since the faces of the polyhedron are convex.
    
    
      
    The local index of a vertex is $2$ - number of min/max edges + number of min/max faces. The local index at the absolute max and min = 2 each.  The local index at any other vertex is $0$ (two of the local face elements are not a max or min).  Sum of local indices $=4$.
    
    The sum of the vertex contributions is $2V$, edge contributions $-2E$, and face contributions $2F$ (by convexity each face has a unique max and min). 
	Thus, the graph is planar: $2 = V-E+F$.

	Without loss of generality, the spherical net of edges formed by quasi-regular tetrahedra is 2-connected.  
    
    Conjecture: We can show that the quasi-regular tetrahedra unit sphere arcs belong to the set of edges of the convex hull of the projection of the points to the unit sphere.  
    
    Removing an edge that does not disconnect the graph does not change the Euler characteristic, so we can work our way from the planar graph formed by the convex hull to the planar graph formed by the quasi-regular tetrahedra.


%%%%%%%%%%%%%%%%%%%%%%%%%%%%%%%%%%%%%%%%%%%%%%%%%%%%
    % file started March 22, 2009
% Marchal objective function

\def\lam{\lambda}
\def\Lam{\Lambda}
\def\bl{{\underline{\lam}}}
\def\bm{{\underline{\mu}}}
\def\norm#1{{\|#1\|}}
\def\angle#1#2#3{{\op{angle}(#1,{#2,#3})}}

\chapter{Formulation}

\section{Rogers's partition}

\begin{definition}[saturated packing]
A packing $\Lam\subset \ring{R}^3$ is a set such that
$$\forall \lam~\mu\in \Lam.~  \norm{\lam-\mu} < 2 \Rightarrow (\lam=\mu).$$
The packing is saturated if for every $v\in\ring{R}^3$,   there exists $\lam\in\Lam$
such that $\norm{\lam-v}\le 2$.
\end{definition}

We think of $\Lam$ as the set of centers of a  packing of congruent balls of radius $1$.
To be saturated means that there is no room for further balls to be added to the packing.
There is no loss in generality in assuming that the packing is saturated, when searching
for the greatest possible density of a packing.

Given a packing $\Lam$, Rogers gives a partition of Euclidean space into
simplices with vertices at $\Lam$ \cite{Rog58}.   This section describes his partition.

Let $\Omega(\lam)$  be the closed Voronoi cell centered at $\lam$:
$$
  \Omega(\lam) = \{x \mid  \norm{x-\lam} \le \norm{x-\mu},\ \lam\ne\mu\in\Lam\}.
$$
If $S\subset\Lam$, let $\Omega(S)$ be the intersection of the family:
$$\Omega(S)  = \bigcap \{\Omega(\lam)\mid \lam\in S \}.$$
We call $\Omega(S)$ the $S$-face of $\Omega(\lam)$.

For $X\subset\ring{R}^3$, we let $\dim(X)$ be the dimension of the affine hull
of $X$: $\dim(X) = \dim(\op{aff}(X))$.

If $\bl=(\lam_0,\ldots,\lam_k)$ and $j\le k$ write $\bl[j] = (\lam_0,\ldots,\lam_j)$.
Let $\Lam(k)$ be the set of $k+1$-tuples $\bl=(\lam_0,\ldots,\lam_k)$ such
that 
\begin{equation}\label{eqn:omega-dim}
\dim(\Omega(\bl[j]))+1 = \dim\Omega(\bl[j-1])
\end{equation}
for all $0<j\le k$.
In particular, $\Lam(0)=\Lam$.  Also, $\Lam(1)$ is the
set of all pairs $(\lam,\mu)$ such that the Voronoi cells at $\lam$ and $\mu$ meet along
a face of codimension $1$, and
so forth.


Define $v(\bl)$ recursively on $\Lam(k)$ by
$$v(\lam[0]) = \lam[0]$$
and $v(\lam[k+1])$ is the closest point on $\Omega\{\lam[k+1]\}$ to $v(\lam[k])$.  The point is defined whenever $\Omega\{\lam[k+1]\}$ is nonempty.
The set $\Omega\{\lam[k+1]\}$ is convex and compact.  Thus, the point $v(\lam[k+1])$ exists
uniquely.

For $\bl\in\Lam(k)$, let 
$$R(j,\bl) = \op{conv}\{v(\lam[j]), v(\lam[j+1]),\ldots,v(\lam[k])\}.$$  Set $R(\bl)=R(0,\bl)$.


\begin{lemma} 
For any saturated packing $\Lambda$, we have
$$\ring{R}^3 = \bigcup \{ R(\bl) \mid \bl\in \Lam(3)\}.$$
\end{lemma}

\begin{proof}
We have $$\ring{R}^3 = \bigcup \{\Omega(\lam)\mid \lam\in \Lam(0)\}.$$
So it is enough to show that
$$\Omega(\bl[0]) = \bigcup \{ R(\bl') \mid \bl'\in \Lam(3),~\bl[0]=\bl'[0]\}.$$
We have
$$\Omega(\bl[0]) = \bigcup \op{conv}(c(\bl[0]),\{\Omega(\bl[1])\mid (\bl[1])\in \Lam(1)\}).$$
So it is enough to show that
$$\Omega(\bl[1]) = \bigcup \{ R(1,\bl') \mid \bl'[1]=\bl[1],~\bl'\in \Lam(3)\}.$$
Repeating, it is enough to show that
$$\Omega(\bl[3]) = \bigcup\{R(3,\bl')\mid\bl'[3]=\bl[3],~\bl'\in\Lambda(3)\}.$$
The right-hand side is the singleton $\{v(\bl[3])\}$.  The left-hand side
contains this point and is contained in a properly decreasing chain of affine spaces:
$\ring{R}^3$, the bisector of $\lam_0$ and $\lam_1$, and so forth.  This determines a point,
so the two sides are equal.
\end{proof}

\begin{lemma}  The intersection of any two distinct $R(\bl)$ and $R(\bm)$ is
contained in a plane (hence has measure zero).
\end{lemma}

Combined with the previous lemma, we see that the simplices $R(\bl)$ partition Euclidean
space.

\begin{proof}  Let $\bl = (\lambda_0,\ldots)$ and $\mu = (\mu_0,\ldots)$.
If $\lambda_0\ne\mu_0$, the intersection of the simplices lies in $\Omega\{\lam_0,\mu_0\}$
and we are done.  Let $k+1$ the first index such that
$$\op{conv}(v(\lam_0),\ldots,v(\lam_0,\ldots,\lam_{k+1}))\ne
\op{conv}(v(\mu_0),\ldots,v(\mu_0,\ldots,\mu_{k+1})).
$$
By induction $\Omega\{\lam_0,\ldots,\lam_k\} = \Omega\{\mu_0,\ldots,\mu_k\}$
but
$$\Omega\{\lam_0,\ldots,\lam_{k+1}\} \ne \Omega\{\mu_0,\ldots,\mu_{k+1}\}.$$
The intersection lies in the convex hull $C$ of
$\{v(\lam_0),\ldots,v(\lam_0,\ldots,\lam_k)\}$ and
$V=\Omega\{\lam_0,\ldots,\lam_{k+1},\mu_{k+1}\}$.  The set $V$ lies in an affine space of 
codimension $k+2$
and $C$ has codimension $(k+2) - (k+1) = 1$.
\end{proof}

\begin{lemma}\label{lemma:v2} Let $\bl=(\lam_0,\ldots,\lam_3)\in\Lambda(3)$ and fix $0\le j\le 3$.  Assume that
$$
\norm{v(\bl[j]) - v(\bl[0])} < \sqrt2.
$$
Then $v(\bl[j])$ is the circumcenter of the points $S_j =\{\lam_0,\ldots,\lam_j\}$ and
lies in the convex hull of $S_j$.  The points $v(\bl[j])$, for $j=0,1,2,3$, are distinct.
\end{lemma}

\begin{proof} By definition, an element $\bl\in\Lambda(3)$ gives dimensions: 
$$\Omega\{\lam_0,\ldots,\lam_j\} = 3-j.$$  The case $j=0$ of the lemma is trivially
satisfied.  Assume by induction the result holds for numbers less than $j$.

Now $j>0$.
The circumcenter $p=p_{j}$ of the set $S_{j}$ is the point on the plane $\op{aff}(S_{j})$
closest to $p_{j-1}=v(\bl[j-1])$.  We claim that $p$ lies in $\Omega(S_{j})$.  Otherwise
there is a point $\mu\in\Lambda$ closer to $p$ than any $\lambda\in S_{j}$.  
The angles $\angle{p}{\mu}{\lam_i}$ are obtuse for $i\le j$.  We make a case-by-case
argument for each $j$.

If $j=1$, the points $p,\lam_0,\lam_1$ are collinear and cannot give two obtuse angles.

If $j=2$, $\mu$ lies in the intersection of half spaces with bounding plane $\mu$
and normal $-\lambda_i$.  Let $\mu'$ be the projection of $\mu$ to the plane containing
$p,\lam_0,\lam_1,\lam_2$. The four points $\mu',\lam_0,\lam_1,\lam_2$ can be arranged
cyclically around $p$, each forming an obtuse angle with the next.  A circle around $p$
cannot give four obtuse angles.


If $j=3$, assume that $\lam_0,\ldots,\lam_4$ are labeled in cyclic order around the line
$\op{aff}\{p,\mu\}$.  Consider the dihedral angle 
  $$
  \gamma_i=\gamma=\dih(\{p,\mu\},\{\lam_i,\lam_{i+1}\}).
  $$
By the law of cosines the angle $\gamma$ is given in terms of the edges of the spherical
triangle $a,b,c$ by
$$
  \cos c - \cos a \cos b = \sin a \sin b \cos \gamma.
$$
All the terms on the left-hand side are negative, so $\gamma =\gamma_i > \pi/2$.
This is impossible, as the sum of the four dihedral angles $\gamma_i$ is $2\pi$.
This completes the proof that $v(\bl[j])$ is the circumcenter.

Next we show that this point lies in the convex hull of $S_j$.
If $j=0$, there is nothing to show.  If $j=1$, the point is the midpoint of the convex hull.
If $j=2$, the point is the circumcenter of an acute triangle.  If $j=3$, the point is the
circumcenter of a simplex such that every face has positive orientation (see XX).  Thus,
in every case the point lies in the convex hull.

The distances $\norm{v(\bl[j])-v(\bl[0])}$ are increasing, so it is enough to show
that $v(\bl[j])\ne v(\bl[j+1])$.  If we had equality, then the circumcenter $v(\bl[j])$
would have an equally close packing point $\lambda_{j+1}\in\Lambda$, which is impossible
the the first part of the proof.
\end{proof}

\begin{lemma}   Let $\bl\in\Lambda(k)$.  Assume that $\norm{v(\bl)-v(\bl[0])}<\sqrt2$.
Let $\bm$ be any permutation of the components of $\bl$.  Then $\bm\in\Lambda(k)$ and
 $v(\bl) = v(\bm)$.
\end{lemma}

\begin{proof} 
Since the sets $\Omega(v(\bl[j]))$ satisfy (\ref{eqn:omega-dim}), we have that
$\Omega(v(\bl))$ is a point consisting of the singleton $\{v(\bl)\}$, which is the
circumcenter of the simplex with vertices $\{\lam_0,\ldots,\lam_k\}$.  This describes
the point $v(\bl)$ in a way that is independent of the permutation.

We check that the condition (\ref{eqn:omega-dim}) holds for the permutation $\mu$.
The proof of Lemma~\ref{lemma:v2} shows that the midpoint of $\op{conv}(\mu_0,\mu_1)$
lies in $\Omega(\bm[1])$ and since the inequalities are strict, some neighborhood
of this midpoint in the bisecting plane of $\{mu_0,\mu_1\}$ lies in $\Omega(\bm[1])$.
Thus, $\dim(\Omega(\bm[1]))=2$.  We continue in this fashion, showing that each dimension
$\dim(\Omega(\bm[j]))$ drops by exactly one.
\end{proof}

\begin{lemma}\label{lemma:Rconv} Let $\bl = (\lam_0,\ldots,\lam_k)\in\Lambda(k)$.  Assume that $\norm{v(\bl)-v(\bl[0])}<\sqrt2$.
For any permutation $\pi$, let $\pi(\bl)\in\Lam(k)$ be the permuted entry.  Let
$S(k)$ be the group of all permutations on $k+1$ letters.   Then
$$
\op{conv}\{\lam_0,\ldots,\lam_k\} = \bigcup \{ R(\pi(\bl)) : \pi\in S(k)\}.
$$
\end{lemma}

\begin{proof} Let $L = \{\lam_0,\ldots,\lam_k\}$.  We prove it by induction on $k$.
When $k=0$, the result is trivial.  Now assume $k>0$.

The circumcenter $v(\bl)$ of $L$ lies in the convex hull of these
point (see the proof of Lemma~\ref{lemma:v2}).  Thus, the left-hand side is the union
of the cones over the (k+1)-faces:
$$
\op{conv}(S) = \bigcup\{ \op{conv}(v(\bl),L\setminus \{\lam_i\}\mid i=0,\ldots,k) \}.
$$
The sets $L\setminus \{\lam_i\}$ can be identified with  cosets of $S(k)/S(k-1)$.
By induction $L\setminus \{\lam_i\}$ is the union of $R(\bm)$ as $\bm$ runs
over all permutations $S(k-1)$ of $L\setminus \{\lam_i\}$.
The result follows by induction.
\end{proof}

We remark that the Rogers's simplices $R(\bl)$ are compatible with the Voronoi
decomposition of space (by construction).  Also, by Lemma~\ref{lemma:Rconv}, under
mild restrictions on the circumradius, they can also be reassembled into simplices
with vertices at the centers of the packing (the Delaunay simplices).  

\section{Marchal's partition}

C, Marchal has introduced a new approach to the Keper conjecture that is manifestly superior to the one in \cite{DCG}.  If his approach can be pushed
to completion, it would cut years off the Flyspeck project.

His articles claim to give
a {\it demonstration} of the Kepler conjecture \cite{Mar08}, \cite{Marc07}.  However, the
mathematically rigorous part of the article only gives a reduction
of the problem to an optimization problem in a finite number of
variables.  The method of gradient descent is then used to explore
the local minima of the optimization problem in finitely many variables.

Marchal's partition of space is a variant of Rogers's partition into simplices
$R(\bl)$.  The main part of construction is 
the decomposition obtained by truncating the Voronoi cells
by a ball of radius $\sqrt2$.  In a few carefully chosen situations he assembles the simplices
into a larger convex hull along the lines of Lemma~\ref{lemma:Rconv}.

Let $r(\bl[i]) = \norm{v(\bl[i])-\lam_0}$.  The numbers $r(\bl[i])$ are increasing with $i$.

\begin{definition} Let $R=R(\bl)$ be a Rogers simplex with $\bl=(\lam_0,\ldots,\lam_3)$.
\hfill\break\smallskip  
{\bf The $0$-cell} of $R$ is
$$
R(\bl)_0 = \{x\in R(\bl) \mid \norm{v(\lam_0)-x} > \sqrt2.
$$
\bigskip
{\bf The $1$-cell} of $R$ is 
$$
R(\bl)_1 = \op{conv}(v(\lam_0),T_1),\quad T_1 = \{x \in R(\bl) \mid \norm{v(\lam_0)-x}= \sqrt2\}.
$$
\bigskip
{\bf The $2$-cell} of $R$ is
$$
\begin{array}{rll}
R(\bl)_2 &= \op{conv}(v(\bl[0]),v(\bl[1]),T_2),\quad \\
  T_2 &= \{x \in R(\bl)\cap \Omega(\bl[1]) \mid \norm{\lam_0-x}=\norm{\lam_1-x} =\sqrt2\}.
\end{array}
$$
\bigskip
{\bf The $3$-cell} of $R$ is defined to be empty unless 
$$
r(\bl[2]) <\sqrt2 \le r(\bl[3]).
$$
When this inequality holds,
there is a a unique point $p(\bl[2])$ in
$\op{conv}(\bl[2],\bl[3])$ at distance exactly $\sqrt2$ from $\lam_0$.  
Define the $3$ cell to be
$$
R(\bl)_3 = \op{conv}\{p(\bl[2]),\lam_0,\lam_1,\lam_2\}.
$$
\bigskip
{\bf The $4$-cell} of $R$ is defined to be empty unless
$$
r(\bl[3]) <\sqrt2.
$$
When this inequality holds, define the $4$ cell to be
$$
R(\bl)_4 = \op{conv}\{\lam_0,\lam_1,\lam_2,\lam_3\}.
$$
\end{definition}

Note the the $0$, $1$, and $2$-cells are always subsets of $R$.  However, the $3$ and
$4$-cells are finite unions of Rogers simplices.

\begin{lemma}  The $0,\ldots,4$-cells of all Rogers simplices $R(\bl)$, for $\bl\in \Lam(3)$
give a partition of $\ring{R}^3$.  An $i$-cell is never equal to a $j$-cell when $i\ne j$.
Two $3$-cells $R(\bl)_3 = R(\bm)_3$ are equal exactly when
$(\lam_0,\lam_1,\lam_2)$ is a permutation of $(\mu_0,\mu_1,\mu_2)$.  Two $4$-cells are
equal exactly when $\bl$ is a permutation of $\bm$.
\end{lemma}

\begin{proof}  First we will prove that cells are either disjoint or equal.  At the same
time, we determine when $4$-cells or $3$-cells are equal to one another.
By Lemma~XX, the $4$-cell $R(\bl)_4$ is a union of the Rogers simplices
$R(\bm)$, as $\bm$ runs over permutations of $\bl$. 
 Two $4$-cells
that meet are equal. 
Every $4$-cell comes from a simplex
$R(\bl)$ with $r(\bl)<\sqrt2$.  This condition gives $R(\bl)_i=\emptyset$, for $i=0,1,2,3$.

Similarly, the $3$-cell is a union
of the convex hulls $\op{conv}(p(\bm[2]),R(\bm[2]))$ as $\bm[2]$ runs over permutations of $\bl[2]$.  Note that the point $p(\bm[2])=p(\bl[2])$ is independent of the permutation, since
it is determined as the point at distance $\sqrt2$ from $\lam_0$ along the line through
from $v(\bm[2])=v(\bl[2])$ perpendicular to the plane $\op{aff}\{\lam_0,\lam_1,\lam_2\}$
(in the half-space of $\bl[3]$). Two $3$-cells that meet are equal and come from parameters that are permutations of
one another as described.   The intersection $R(\bl)_3$ cannot meet $R(\bl)_i$, for $i<3$, in
a set of full dimension, because the plane $P(\bl[2])=\op{aff}\{\lam_0,\lam_1,p\}$ separates them.

The $2$-cell $R(\bl)_2$ is separated from the $0$ and $1$-cells $R(\bl)_i$
by the cone $P(\bl[1])$ with apex $\lam_0$
passing through $T_2$. The $1$-cell is separated from $0$-cells the the sphere $P(\bl[0])$
of radius
$\sqrt2$, centered at $\lam_0$.

Finally, we show that every point $x$ in $\ring{R}^3$ belongs to a cell.  By the Rogers
partition, the point $x$ belongs to a Rogers simplex $R(\bl)$.  If $r(\bl)<\sqrt2$, then
$x$ belongs to a $4$-cell.  Otherwise $x\in R(\bl)$ belongs to $R(\bl)_i$ according to
which side of the plane $P(\bl[2])$, cone $P(\bl[1])$, and sphere $P(\bl[0])$ the point
lies, as described above.
\end{proof}


\section{Kissing number estimates}

This section shows how Conjecture~\ref{conj:m1} would follow from an explicit estimate
related to the Gregory-Newton problem


We define the following constants and functions.
$$
\begin{array}{lll}
\alpha=\alpha_{cm} &= \arccos(1/3)\\
K=K_{cm} &= (3\alpha-\pi)\sqrt2/(12\pi - 30\alpha)\\
M = M_{cm} &= (18\alpha-7\pi)\sqrt2/(144\pi-360\alpha)\\
f(r) = f_{cm}(r) &=
\begin{cases}
 (\sqrt2-r) (r-1.3254) (9r^2 - 17 r + 3)/(1.627 (\sqrt2-1))& r\le\sqrt2\\
 0 & r >\sqrt2.
\end{cases}
\\
\end{array}
$$
We have 
\begin{equation}\label{eqn:km}K - 12M = \sqrt{1/2}\end{equation}
and
\begin{equation}f(1) = 1,\quad f(\sqrt2) =0\end{equation}


\begin{conjecture}[Marchal-2]\label{conj:m1} For any saturated packing $\Lam$, and
any $\lam_0\in\Lam$, we have
$$
\sum_{\bl=(\lam_0,\lam_1)\in\Lam(1)} f(r(\bl)) \le 12.
$$
\end{conjecture}

\begin{theorem}\label{theorem:mk}
Conjecture~\ref{conj:m1} implies the Kepler conjecture.
\end{theorem}

\begin{proof} 
We show that $A(\lam_0)  = 8 K - \sum 8 M f(r(\bl)) -\op{vol}(\Omega(\lam_0))$ is fcc-compatible and negligeable.  The
result then follows from \cite[Lemma~3.3]{DCG}.  This is fcc-compatible directly
by equation~(\ref{eqn:km})
and Conjecture~\ref{conj:m1}.  The issue is to prove it negligeable.  Explicitly, we need
to show there exists a constant  $C$ such that or all $r\ge 1$ and all $x\in\ring{R}^3$:
\begin{equation}\label{eqn:neg}
\sum 8K - \sum (\sum 8 M f(r)) - \sum \op{vol}(\Omega(\lam))) \le C r^2,
\end{equation}
where the outer sum runs over $\lam\in \Lam(x,r)$.


Define the {\it total solid angle} and the {\it basic dihedral angles} of $i$-cells as follows for
$i>0$.  The total solid angle is the sum of the solid angles of the $i$-cell, summed
over all the extreme points of the cell that belong to $\Lambda$.  For a $0$-cell
the total solid angle is zero. For $1$ and $2$-cells,
it is just the solid angle of the cell at $\lam_0$.  For a $3$-cell, it is the sum of the
solid angles at $\lam_0,\lam_1,\lam_2$.  And for a $4$-cell, it is the sum of the solid
angles at $\lam_0,\ldots,\lam_3$. Write $\op{tsol}(X)$ for the total solid angle of a cell $X$.


The  basic dihedral angles of a cell $X = R(\bl)_i$ are indexed by the set $D(X)$ of
edges $\op{conv}(\lam_0,v(\bl[1]))$ that lie on the boundary of the cell.  The value
$\dih(X,d)$ of
of the basic dihedral angle indexed by $d$ is just the dihedral angle along that edge
of the boundary.  The indexing set $X$ is empty for  $0$ and $1$-cells.
The (single) basic  dihedral angle of a $2$-cell is
the same as the radian angle subtended by the arc $T_2$ in the construction of $2$-cells.
The basic dihedral angles of a $3$-cell are the dihedral angles along the
six directed edges $(\lam_i,\lam_j)$, $i,j=0,1,2$.  The basic dihedral angles of a $4$-cell are
the dihedral angles along any directed edge.  Each element $d\in D(X)$ also determines
the real number $r(d) = \norm{v(\bl[1])-\lam_0}$.

We have the Marchal's fundamental estimate for any cell $X$:
\begin{equation}\label{eqn:mfe}
\op{vol}(X) \ge \left(\frac{2K}{\pi}\right) \op{tsol}(X) - \left(\frac{4M}{\pi}\right)
\sum_{d\in D(X)} \dih(X,d) f(r(d)).
\end{equation}
We write $\op{vol}_e(X)$ for the volume estimate appearing on the 
right-hand-side, to write this in the form:
\begin{equation}\label{eqn:vole}
\op{vol}(X) \ge \op{vol}_e(X).
\end{equation}
(Note that this is an inequality in at most six variables; the most difficult case to prove
is that of a $4$-cell.)  The volumes and solid angles and so forth are given by the 
explicit formulas.

Sum the inequality~(\ref{eqn:mfe}) over all cells in a large ball $B(x,r)$ to get an
inequality of the form $0\ge T_1 + T_2 + T_3$ for three sums $T_i = T_i(r)$.  We compare this term-by-term
with the three terms $T'_i(r)$ of the desired equation~(\ref{eqn:neg}).  The solid angles around each point sum to
$4\pi$, and the dihedral angles around each directed edge sum to $2\pi$, so we see that
the three terms satisfy
$$T'_i(r) \le T_i(r) + C r^2,$$
for some constant $C$ and error term $C r^2$ coming from the boundary effects of the cells $X$ that meet the boundary of $B(x,r)$.  The result follows.
\end{proof}


\section{More kissing number estimates}

This section shows how to improve on the estimates of the previous section
by combining various Marchal cells into {\it supercells}.

\begin{definition}
Set
$$
\begin{array}{lll}
  h_0  &= 1.26\\
  h_+  &= 1.3254\\
\end{array}
$$
Let $L:[1,\sqrt{2}]\to\ring{R}^2$ be the linear function interpolating the
values:
$$
L(1) = 1\quad L(h_0) = 0.
$$
Let $h_- = 1.23175\ldots$ be the unique root of the quartic polynomial
$f(r)-L(r)$ lying in the interval $[1.2,1.3]$.
Define the function $f_2:[1,\sqrt{2}]\to\ring{R}$ by
$$
f_2(r) = \max(0,\min(f(r),L(r))) = 
\begin{cases}
  f(r) & r \in [1,h_-]\\
  L(r) & r \in [h_-,h_0]\\
  0 & r \in [h_0,\sqrt2]\\
\end{cases}
$$
\end{definition}

The function $f_2$ is continuous and piecewise smooth.  We have
$f_2(r)\le f(r)$ except when $r\in [h_-,h_+]$.  The aim of this section is to prove a variant of Theorem~\ref{theorem:mk} that uses the function $f_2$ rather than $f$.  For this, we need to combine cells into larger groups, called supercells.

\begin{definition}
Let $X$ be a $k$-cell, with $k\ge 2$.  Let $e$ be an edge of $X$ that
has an endpoint at a vertex of $\Lambda$.  The edge is determined by
a pair of point $\lambda_1,\lambda_2\in\Lambda$.  We call
$|\lambda_1-\lambda_2|/2$ the {\it reduced length} of the edge.
(For a $2$-cell this is the actual length of the edge, but for a $3$-cell
or $4$-cell this is half the length of the edge.)
We say the edge is {\it critical} if the reduced length lies in the
range $[h_-,h_+]$.
Let $X$ be a $k$-cell with a critical edge.  The weight of $X$  
is $1/m$, where $m:1..6$ is the
number of critical edges on $X$.
(A $2$-cell with a critical edge has weight $1$; a $3$-cell can have
weights $1$, $1/2$, $1/3$; a $4$-cell has possible weights $1/m$, $1\le m\le 6$.)
\end{definition}

\begin{definition}
Let $e$ be a critical edge of a $k$-cell for some $k\ge 1$.
A supercell is a formal linear combination
$$
\sum_X [X] w(X),
$$
where $w(X)$ is the weight of $X$ and the sum runs over all cells with an edge along $e$.  If $Z = \sum_X [X] w(X)$ is a supercell, define
$$
\Gamma(Z) = \sum_S (\op{vol}(X)-\op{vol}_e(X)) w(X).
$$
\end{definition}

\begin{theorem}\label{lemma:superineq} 
Let $Z$ be any supercell.  Then $\Gamma(Z)\ge 0$.
\end{theorem}

The proof of this theorem will occupy the rest of the section.  Before giving the proof, we give its relation to the Kepler conjecture.


\begin{conjecture}\label{conj:m2} For any saturated packing $\Lam$, and
any $\lam_0\in\Lam$, we have
\begin{equation}\label{eqn:f12}
\sum_{\bl=(\lam_0,\lam_1)\in\Lam(1)} f_2(r(\bl)) \le 12.
\end{equation}
\end{conjecture}

\begin{theorem}\label{theorem:mk2}
Conjecture~\ref{conj:m1} implies the Kepler conjecture.
\end{theorem}

\begin{proof}  For cells $X$ that do not form part of a supercell,
the proof is just as in the proof of Theorem~\ref{theorem:mk2}.
Each cell $X$ that has a critical edge belongs to $1/w(X)$ different
supercells, each with weight $w(X)$.  In supercells, we replace inequality
(\ref{eqn:vole}) with Theorem~\ref{lemma:superineq}.
\end{proof}


\section{bounding the number of spheres}

The purpose of the remainder of this work is to prove the inequality~(\ref{eqn:f12}).   Let $L_2(r) = \max(f_2(r),L(r))$.  We will actually aim for the 
following stronger
inequality, although the weaker inequality would suffice for the proof of the Kepler conjecture.  This stronger inequality has the advantage of being piecewise linear.

\begin{conjecture}
\begin{equation}\label{eqn:L12}
\sum_{\bl=(\lam_0,\lam_1)\in\Lam(1)} L_2(r(\bl)) \le 12.
\end{equation}
\end{conjecture}
Since $L_2(r) = f_2(r) = 0$, for $r\ge h_0$, we may discard all summands for which $r(\bl)\ge h_0$.  Since $L_2(r)\le 1$, it is clear that the inequality holds whenever the number of summands is at most $12$. The following is a variant of a lemma of Marchal.


\begin{lemma}  If the number of nonzero summands is $15$ or greater, then inequality~\ref{eqn:L12} holds.
\end{lemma}

\begin{proof} 
Consider a configuration with $15$ or more nonzero summands, $\lambda_1,\ldots,\lambda_N$. 
We fix the origin $\lambda_0=0$.  Set $r_i = \|\lambda_i\|/2$.  Set
$$
a(r) = \arccos(r/2) - \pi/6.
$$
On the unit sphere,  consider the disks $D_i$ of radii $a(r_i)$, centered at $\lambda_i/\|\lambda_i\|$.  These disks do not overlap; this follows from the easy inequality 
$$
a(r_i) + a(r_j) \le \op{arc}(2r_i,2r_j,2).
$$
Let $P_i$ be the half-space containing the origin, bounded by the plane through the circular boundary of $D_i$.  The intersection of these half-spaces is a convex polytope with faces $F_i$.  The radial projection of $F_i$ to the unit sphere is a spherical polygon $R_i$ containing $D_i$.  Let $n_i$ be the number of sides to the polygon $R_i$.  The area of $R_i$ is at least the area $A(a(r_i),n_i)$ of the smallest $n_i$-gon containing $D_i$.  The smallest spherical $n$-gon containing a disk of radius $a$ is regular.  This leads to the formula
$$
A(a,n) = 2\pi - 2 n (\arcsin(\cos(a)\sin(\pi/n))).
$$
We can verify directly that
$$
A(a(r),n) \ge 0.6327 - 0.0333 n + 0.4754 L_2(r),\quad
n = 3,4,\ldots,\quad 1\le r\le h_0.
$$
The number of faces, edges, and vertices of the convex
polyhedron satisfy the Euler relation $\sum_i n_i \le (6N-12)$.
Summing over $i=1..N$, for $N\ge 15$, we get the follow
estimate on $\bar L_2 = \sum_i L_2(r_i)$:
$$
\begin{array}{lll}
4\pi &= \sum_i\op{area}(R_i)\\
     &\ge \sum_i A(a(r_i),n_i) \\
     &\ge 0.6327 N - 0.0333\sum_i n_i + 0.4754 \bar L_2\\
     &\ge 0.6327 N - 0.0333 (6N-12) + 0.4754 \bar L_2\\
     &\ge 0.6327 (15) - 0.0333 (78) + 0.4754 \bar L_2\\
11.94 &\ge \bar L_2. 
\end{array}
$$
\end{proof} 


    %% Kepler Conjecture.
% Thomas C. Hales
% Starting with Chapter on Tame Hypermaps


%% XX Notation: A vs. v for nodes.
%% XX Notation sigma' for aggregates, sigma'' for the full hypermap.
%% Allow the pentagon-triangle into the definition of tame graph.
%%%% Show there is at most one.  Let it be the seed.
%%




\label{sec:tame}


This chapter defines a class of hypermaps.  Hypermaps in this class
are said to be {\it tame}.  In the next chapter, we give a complete
classification of all tame hypermaps.  This classification of tame
hypermaps was carried out by computer.   This classification is a
major step of the proof of the Kepler Conjecture.

\section{Definition and Classification}


\begin{definition}
Faces of cardinality $3$ are called {\it triangles}, those of
cardinality $4$ are called {\it quadrilaterals}, and so forth. Let
$p_v$ be the number of triangles incident with a node $v$. A face of
cardinality at least $5$ is called an {\it exceptional\/} face.
 %
 \index{triangle}
 \index{exceptional}
 \index{quadrilateral}
 \index{exceptional!face}
 \index{pZ@$p_v$}
\end{definition}

\begin{definition}\label{definition:type}
A face of a hypermap is said to be exceptional if it has at least
five darts.  The {\it type\/} of a node is defined to be a triple of
non-negative integers $(p,q,r)$, where $p$ is the number of
triangles containing the node, $q$ is the number of quadrilaterals
containing it, and $r$ is the number of exceptional faces.
%
 \index{type (of a node)}
\end{definition}


\subsection{weight assignment}\label{sec:wtassign}

We call the constant $\op{tgt}=14.8$, which arises repeatedly in
this section, the {\it target}.  (This constant arises as an
approximation to $4\pi\zeta -8\approx 14.7947$, where $\zeta =
1/(2\arctan(\sqrt{2}/5)$.)
%
 \index{target}\index{tgt@$\op{tgt}=14.8$}
 \index{ZZdzeta@$\zeta= 1/(2\arctan(\sqrt{2}/5))$}

\begin{definition}
  Define $a:\ring{N}\to \ring{R}$ by
  $$a(n) = \begin{cases}
    14.8 &n=0,1,2,\\
    1.4 & n=3,\\
    1.5 & n=4,\\
    0 & \text{otherwise.}
  \end{cases}
  $$
\end{definition}

\begin{definition}
  Define $b:\ring{N}\times \ring{N}\to \ring{R}$ by $b(p,q)=14.8$,
  except for the values in the following table
  (with  $\op{tgt}=14.8$):
  {
  \def\tx{\op{tgt}}
  $$\begin{matrix}  &q=0&1&2&3&4\\
           p=0&\tx&\tx&\tx&7.135&10.649\\
           1&\tx&\tx&6.95&7.135&\tx\\
           2&\tx&8.5&4.756&12.981&\tx\\
           3&\tx&3.642&8.334&\tx&\tx\\
           4&4.139&3.781&\tx&\tx&\tx\\
           5&0.55&11.22&\tx&\tx&\tx\\
           6&6.339&\tx&\tx&\tx&\tx
   \end{matrix}
   $$
   }
\end{definition}

\begin{definition}
  Define $c:\ring{N}\to \ring{R}$ by
  $$c(n) = \begin{cases}
    1 & n=3,\\
    0 & n=4,\\
    -1.03 &n=5,\\
    -2.06 &n=6,\\
    -3.03 &\text{otherwise.}
    \end{cases}
    $$
\end{definition}

\begin{definition}
    Define $d:\ring{N}\to \ring{R}$ by
  $$d(n) = \begin{cases}
    0 & n=3, \\
    2.378 & n=4, \\
    4.896 & n=5, \\
    7.414 & n=6, \\
    9.932 & n=7, \\
    10.916 & n=8,\\
    \op{tgt}=14.8 & \text{otherwise}.
  \end{cases}
  $$
\end{definition}

\begin{definition}
A set $V$ of nodes in a hypermap is called a {\it separated\/} set
of nodes if the following four conditions hold.
%
 \index{separated set}
    \begin{enumerate}
      \item Every node in $V$ is incident with an exceptional face.
      \item No two
        nodes in $V$ are adjacent.  (That is, no edge is incident
        with two different nodes in $V$.)
      \item No quadrilateral in $V$ is incident with two different nodes
        in $V$.
      \item Each node in $V$ has cardinality 5.
    \end{enumerate}
\end{definition}

\begin{definition}
%
A {\it weight assignment\/} of a hypermap $H$ is a function $w$ on
the set of faces of $H$, taking values in the set of non-negative
real numbers. A weight assignment is {\it admissible} if the
following properties hold:
%
 \index{weight assignment}
 \index{admissible (weight assignment)}
\begin{enumerate}
  \item If the face $F$ has cardinality $n$, then
        $w(F) \ge d(n)$
  \item If a node $v$ has type $(p,q,0)$, then
        $$\sum_{F:\,v\cap F\ne\emptyset} w(F) \ge b(p,q).$$
        \label{admissible:b}
  \item Let $V$ be any set of nodes of type $(5,0,0)$, and let $A =\bigcup V$ be
        the set of darts in these nodes.
        If the cardinality of $V$ is $k\le 4$, then
        then
        $$\sum_{F:\,F\cap A\ne\emptyset} w(F) \ge 0.55 k.$$
  \item Let $V$ be any separated set of nodes, and let $A =\bigcup V$ be
        the set of darts in these nodes.
        Then
        $$\sum_{F:\,F\cap A\ne\emptyset} (w(F) -d(\card(F)))
            \ge \sum_{v\in V} a(p_v).$$
        \label{definition:admissible:excess}
\end{enumerate}
The sum $\sum_F w(F)$ is called the {\it total weight} of $w$.
\index{total weight}
\end{definition}





\subsection{hypermap properties}
\label{sec:graphproperty}

We say that a hypermap is {\it tame\/} if it satisfies the following
conditions.
%
 \index{tame}

\begin{enumerate}
    \label{definition:tame}
    %1
    \item The hypermap is plain, planar, and connected.
    \item The edge map $e$ has no fixed points.
    \item The two darts of each edge lie in different nodes.
    \item The set of edges meeting any two given nodes has cardinality at most $1$.
    \item There are at least $2$ faces.
    \item Every face meets every node in at most one
        dart.
    \item There are never two nodes of type $(4,0,0)$ that are
    adjacent to each other.
    \label{definition:tame:40}
    \item The cardinality of each face is at least $3$ and at most $8$.
    \label{definition:tame:length}

    \item If $L$ is a contour loop with $3$ face steps, and if there exists a node in
    the exterior of $L$, then $L$ is a face of the hypermap.
    \label{definition:tame:3-circuit}

    \item If $L$ is a contour loop  $4$ face steps, and there are at least two nodes
    in the exterior of $L$, then the interior of $L$ takes one of the forms
    illustrated in Figure
    \ref{fig:fourcircuit}.  [XX make this more precise.]
    \label{definition:tame:4-circuit}
    \begin{figure}[htb]
        \centering
        \myincludegraphics{\ps/tame4circuit.eps}
        \caption{Tame $4$-circuits}
        \label{fig:fourcircuit}
    \end{figure}

    \item The cardinality of every node is at least $2$ and at most
    $6$.
    \label{definition:tame:degree}

    \item If a node is incident with an exceptional face,
        then the cardinality of the node is at most $5$.
    \label{definition:tame:degreeE}

    \item $$\sum_F c(\card(F)) \ge 8,$$
    \label{definition:tame:score}


    \item There exists an admissible weight assignment
        of total weight less than the target, $\op{tgt}=14.8$.
    \label{definition:tame:squander}



\end{enumerate}
%
Property \ref{definition:tame:score} implies that the hypermap has
at least eight triangles.


\subsection{classification of tame hypermaps}
    \label{sec:proof-classification}

%\section{Statement of the Theorem}
\label{sec:classification}

A list of several thousand hypermaps appears at \cite{web}. The
following theorem is listed as one of the central claims in the
proof in Section~\ref{sec:logic}.

\begin{definition} The opposite of a hypermap $(D,e,n,f)$ is the
hypermap $(D,f n,n^{-1},f^{-1})$.
\end{definition}

\begin{lemma} If a hypermap has properties XXX, then so does its
opposite.
\end{lemma}

\begin{theorem}
\label{theorem:classification} Every tame hypermap is isomorphic to
a hypermap in this list, or is isomorphic to the opposite of a
hypermap in this list.
\end{theorem}

The results of this section are not needed except in the proof of
Theorem \ref{theorem:classification}.

\smallskip

Computers are used to generate a list of all hypermaps and to check
them against the archive of tame hypermaps.  The computer program is
based on the face-insertion construction of Lemma~XX.  There it is
proved that all sufficiently nice hypermaps can be generated by an
elementary face-insertion process.  Tame hypermaps satisfy all the
hypotheses of that lemma.





\section{Contravention is Tame}
    \label{sec:contraproof}

Let $(\Lambda,v_0)$ be a centered packing with
aggregated fan $P=(v_0,V,E)$.  Let  hypermap $H=(D,e,n,f)$
be the planar hypermap attached to $P$.
The hypermap $H$ is connected (Lemma~\ref{XX}).  Each of its
faces is simple (Lemma~\ref{XX}).

The connected components of $Y(v_0,V,E)$ are in bijection with
faces of $H$.  
The fan gives a azimuth angle function
$$
\op{azim} : D \to (0,2\pi).
$$
For each face of $H$, the corresponding component $R$
is eventually radial with solid
angle
  $$
  2\pi + \sum_{x\in F} (\op{azim}(x) -\pi).
  $$
We write $\sol(F)$ for the solid angle of the connected component
of $Y(v_0,V,E)$ associated with a face $F$ of the hypermap.
We have
    $$\sum_{F} \sol(F) = 4\pi.$$


For each face, there is a
real number $\tau(\Lambda,v_0,F)$ such that
$$
  \sum_{F : \text{face}}\tau(\Lambda,v_0,F) = \tau(\Lambda,v_0).
$$
We define a weight function $w(F)$ on the faces of the hypermap
by $w(F) = \sigma(\Lambda,v_0,F)/pt$.  In this way, we attach
a pair $(H,w)$ to each contravening centered packing $(\Lambda,v_0)$.


Let $D_f\subset D$ be the set of darts 
   $x = (v_0,v,u,w)$
such that the face of $x$ is exceptional and $|u-w|<\sqrt8$.
In this case, $\{v_0,v,u,w\}$ are the vertices of a flat quarter,
which is not necessarily in the $Q$-system.

\begin{theorem} \label{theorem:contravene}
Let $(\Lambda,v_0)$ be a contravening centered packing.  Let $(H,w)$ be
the hypermap and function on its faces attached to $(\Lambda,v_0)$ as above.
Then $H$ is a tame hypermap with admissible weight function $w$.
\end{theorem}

\subsection{hypermap is not empty}

%% Proof that the hypermap is not empty.



\begin{lemma}
\label{prop:nonempty} The construction of Section
\ref{sec:stargraph} associates a nonempty hypermap with at least
two faces to every centered packing $(\Lambda,v_0)$ with $\sigma(\Lambda,v_0)>0$.
In particular, the hypermap of a contravening centered packing is not empty.
\end{lemma}

\begin{proof}
First we show that centered packings with $\sigma(D)>0$ have
nonempty vertex sets $U$. (Recall that $U$ is the set of vertices
of distance at most $2t_0$ from the center).  The vertices of $U$
are used in \Chaps~\ref{sec:construction} and \ref{sec:vcells} to
create all of the structural features of the centered packing:
quasi-regular tetrahedra, quarters, and so forth. If $U$ is empty,
the $V$-cell is a solid containing the ball $B(t_0)$ of radius
$t_0$, and $\sigma(D)$ satisfies
    $$
    \begin{array}{lll}
    \sigma(\Lambda,v) &= \op{sovo}(v,VC(\Lambda,v),\lambda_{oct})\\
              &< \op{sovo}(v,B(v,t_0),\lambda_{oct})\\
              &= \sol(B(v,t_0))\phi(t,t,\lambda_{oct})\\
              &< 0.
    \end{array}
    $$
By hypothesis, $\sigma(D)>0$.  So $U$ is not empty.

XX The proof of making each standard region a simple polygon, assumes a
certain amount of nondegeneracy that isn't covered here.

Equation~\ref{eqn:sig-all} shows that the function $\sigma$ can be
expressed as a sum of terms $\sigma_R$ indexed by the standard
regions $R$. It is proved in Theorem~\ref{lemma:quad0} that
$\sigma_R\le0$, unless $R$ is a triangle. Thus, a centered packing
with positive $\sigma(D)$ must have at least one triangle. Its
complement contains a second standard region. Even after we form
aggregates of distinct standard regions to form the simplified
hypermap (Remarks \ref{remark:tri-pent} and \ref{remark:degree6}),
there certainly remain at least two faces.
\end{proof}


\subsection{first properties of hypermaps}
    \label{sec:startame}


Recall that we say that a node $v$ has {\it type\/} $(p,q,r)$ if
there are exactly $p+q+r$ faces that meet the node, of which exactly
$p$ triangles and $q$ quadrilateral faces (see
Definition~\ref{definition:type}).  We write $(p_v,q_v,r_v)$ for the
type of a node $v$.

\begin{lemma} The hypermap $H$ satisfies Conditions XX-XX of tameness.
Explicitly, it is a plain, planar, and connected. The edge map $e$
has no fixed points. There are at least two faces. Every face meets
every node in at most one dart.  There are never two nodes of type
$(4,0,0)$ that are adjacent to each other.  Every face has
cardinality at least $3$ and at most $8$.  If $L$ is a contour loop
with $3$ face steps, and if there exists a node in the exterior of
$L$, then $L$ is a face of the hypermap.
\end{lemma}


\begin{lemma}\dcg{Lemma~21.4}{223} 
Formally contravening hypermaps satisfy Property
\ref{definition:tame:degree} of tameness: The cardinality of every
node is at least $2$ and at most $6$.
\end{lemma}

\begin{proof}  There is no node of cardinality one by
Lemma~\ref{lemma:nodegen}.  There is no node of degree
greater than $6$ by Lemma~\ref{a:6}.
\end{proof}


\subsection{contravening hypermaps}


\begin{lemma} \label{lemma:0.55:bis} %proclaim{Lemma 5.3}
Let $(H,\azim,\flat,\sigma)$ be a formally contravening hypermap.
Let $v_1,\ldots, v_k$, for some $k\le 4$, be distinct nodes of type
$(5,0,0)$.  Let $F_1,\ldots, F_r$ be all the triangles around the
nodes $v_i$, for $i\le k$. Then
    $$
    \sum_{i=1}^r \tau(F_i)> 0.55k\,\pt,
    $$
and
    $$\sum_{i=1}^r \sigma(F_i) < r\,\pt - 0.48k\,\pt.$$
\end{lemma}


\begin{lemma}\label{lemma:no-2}
Let $(H,\azim,\flat,\sigma)$ be a formally contravening hypermap.
Suppose that $L$ is a contour loop with at most four face steps.
Suppose that there are at least two nodes in the exterior of $L$.
Then there at most one node interior to $L$.
\end{lemma}


\begin{lemma} \label{lemma:0.8638}
Let $(H,\azim,\flat,\sigma)$ be a formally contravening
hypermap. For every dart $x$,
    $$0.8638\le \azim(x).$$
For every dart $x$ whose face is not a triangle, we have
    $$1.153\le\azim(x).$$
\end{lemma}
 %
 \index{ZZZZ1.153@$1.153$}
 \index{ZZZZ0.8638@$0.8638$}

\begin{lemma} \label{lemma:excess-1:bis}
Let $(H,\azim,\flat,\sigma)$ be a formally contravening hypermap.
Let $F$ be an exceptional face.  Let $V$ be a set of nodes of $F$.
Let $x(F,v)$ be the dart of $F$ at a node $v$.  Let $(p_v,q_v,r_v)$
be the type of $v\in V$.   Let $a:\ring{N}\to\ring{R}$ be the
function of Section XX. Assume that $V$ has the following
properties:
    \begin{enumerate}
        \item The set $V$ is separated.
        \item If $v\in V$, then there are exactly five faces at
        $v$.
        \item If $v\in V$, then $\flat(x(F,v))$.
        \item If $v\in V$, then $p_v\ge 3$.  That is, at least
        three of the five faces at $v$ are triangles.
        \item If there are two exceptional regions $F$ and $F'$ at
        $v$, then
            $$\azim(x(F,v)) > 1.32 \Rightarrow \azim(x(F',v)) > 1.32.$$
        \item If $(p_v,q_v,r_v)=(3,1,0)$, then $\azim(x(F,v))\le 1.32$.
    \end{enumerate}
Let $A$ be the union of the singleton $\{F\}$, the set of all
triangles with a dart in some $v\in V$, and the set of all
quadrilaterals with a dart in some $v\in V$. Then
    $$\sum_{F\in A}\tau(F) > \sum_{v\in V} (p_v d(3) + q_v d(4) + a
    (p_v))\,\pt.$$
\end{lemma}


\begin{lemma}\label{lemma:nobad4}
Let $(H,\azim,\flat,\sigma)$ be a formally contravening hypermap.
Let $v$ be a node of type $(1,0,1)$ with precisely one triangle and
one pentagon, as show in Figure~\ref{fig:no4circuit:bis}. Let $L$ be
the perimeter contour loop with four face steps having the node $v$
in its interior.  At each of the four nodes $w$ visited by $L$, let
$\azim(w)$ be the sum of the terms $\azim(x)$, with the sum running
over the darts $x$ at the node visited by $L$.  Then
    $\azim(w) > 1.32$
for each of the four nodes $w$ visited by $L$.
\end{lemma}

\begin{lemma} Let $(H,\azim,\flat,\sigma)$ be a formally contravening
hypermap.  Let $v$ be a node of $H$ of type $(1,0,1)$, such that the
exceptional region is a pentagon.  Let $W$ be the set of four nodes
of the pentagon other than $v$.  If there are four triangles
$F_1,\ldots,F_4$ at some node $W$ that do not meet $v$, then
    $$\sum_{i=1}^p \tau(F_i) > a(4)\,\pt.$$
\end{lemma}

\begin{lemma}
Let $(H,\azim,\flat,\sigma)$ be a formally contravening hypermap.
Let $X$ be the set of nodes $v$ with the following properties.
    \begin{enumerate}
    \item The node has type $(5,0,1)$.
    \item The exceptional face at the node is pentagonal.
    \item That pentagonal face has no nodes of type $(1,0,1)$.
    \end{enumerate}
Then $\card(X)\ne 1$.
\end{lemma}


\begin{lemma}  Let $(H,\azim,\flat,\sigma)$ be a formally contravening
hypermap. Assume that $v$ is a node of $H$ whose type is
$(p,q,r)=(3,0,2)$ or $(4,0,1)$.  Assume that $\neg\flat(x)$ for
every dart of $v$.  Let $\tau(F_1),\ldots,\tau(F_p)$ be the
triangles at the node $v$.  Then
    $$
    \sum_{i=1}^p \tau(F_i) > a(p)\,\pt.
    $$
\end{lemma}




\subsection{linear programs} %subsection
\label{sec:2.2}  To continue with the proof that formally
contravening hypermaps are tame, we need to introduce some more
notation and methods.

\begin{lemma} \label{lemma:deg5}
Every formally contravening hypermap satisfies Property
\ref{definition:tame:degreeE} of tameness: If a node meets an
exceptional face, then the cardinality of the node is at most $5$.
\end{lemma}

\begin{proof} Every node of type $(5,0,1)$ meets a face that is a pentagon.
If there are two or more such nodes, then it must be that of Lemma
XX.  However, this has a node of type $(1,0,1)$, which has been
made an aggregate.  Thus, there is at most one node of type $(5,0,1)$.
This arrangement does not appear on a formally contravening hypermap
by Lemma~\ref{lemma:nobad4}.
\end{proof}

\subsection{possible four-circuits}

Every contour loop partitions the faces into the interior and
exterior.  Every contour loop partitions the nodes that do not meet
the loop into exterior and interior nodes.
%
 \index{interior node}

Lemma~\ref{lemma:no-2} asserts that either the interior or the
exterior has at most $1$ enclosed vertex.   When choosing which
aggregate is to be called the interior, we may make our choice so
that the interior has area at most $2\pi$, and hence contains at
most $1$ node. With this choice, we have the following lemma.

\begin{lemma}
Let $(H,\azim,\flat,\sigma)$ be a formally contravening hypermap. If
$L$ is a contour loop with $4$ face steps, and there are at least
two nodes in the exterior of $L$, then the interior of $L$ takes one
of the forms illustrated in Figure XX in Property
    \ref{definition:tame:4-circuit} of tameness.
\end{lemma}

\begin{proof}
By Lemma~XX, the interior of $L$ contains at most one node.

$H$ is a connected plane planar map.  We form a normal family of
contour loops ${\cal L}$ by taking the contour loop $L^{-1}$
reversing $L$ (XX explain) and all the faces in the interior of $L$.
(Check this is a normal family.)  The quotient $H' = H/{\cal L}$ is
a plane planar map.  There is a further quotient of $H'$ with normal
family $\{L,L^{-1}\}$, which is isomorphic to $P_4$ with the natural
flag coming from $H'$.  The niceness conditions of LemmaXX are
satisfied, so we can recover $H'$ from $P_4$ by a sequence of
face-insertions.  Since the interior of $L$ contains at most one
node, this gives restrictions on the partitions that can be used in
face-insertion.

If there are no enclosed vertices, then the only possibilities are
for it to be a single quadrilateral face or a pair of adjacent
triangles.

Assume there is one enclosed vertex $v$.  If $v$ is connected to $3$
or $4$ nodes of the quadrilateral, then that possibility is listed
as part of the conclusion.

If $v$ is connected to $2$ opposite nodes in the $4$-cycle, then the
node $v$ has type $(0,2,0)$ and the bounds of
Lemma~\ref{lemma:pq:bis} show that the hypermap cannot be formally
contravening.

If $v$ is connected to $2$ adjacent nodes in the $4$-cycle, then we
appeal to Lemma~\ref{lemma:nobad4} to conclude that the hypermap
does not contravene.

If $v$ is connected to $0$ or $1$ nodes, then we appeal to
Lemma~\ref{lemma:enclosed:bis}.  This completes the proof.
\end{proof}

\subsection{weight assignments}
    \label{sec:weight}

The purpose of this section is to prove the existence of a good
admissible weight assignment for formally contravening hypermaps.
This will complete the proof that all formally contravening
hypermaps are tame.

\begin{theorem}  Every formally contravening hypermap has an admissible
weight assignment of total weight less than $\op{tgt}=14.8$.
\end{theorem}

Given a formally contravening hypermap $(H,\azim,\flat,\sigma)$, we
define a weight assignment $w$ by
    $$F \mapsto w(F) = \tau(F)/\pt.$$
Since the hypermap is formally contravening,
    $$
    \begin{array}{lll}
    \sum_F w(F) &= \sum_F \tau(F)/\pt \\
            &= \tau^*(H)/\pt\,\le\,\squander/\pt \\
        &< \op{tgt}=14.8.
    \end{array}
    $$
The challenge of the theorem will be to prove that $w$, when
defined by this formula, is admissible.

\subsection{admissibility}
\label{sec:admissibility}

The next three lemmas establish that this definition of $w(F)$ for
formally contravening hypermaps satisfies the first three defining
properties of an admissible weight assignment.

\begin{lemma}  Let $F$ be a face of cardinality $n$ in a formally contravening hypermap.
Define $w(F)$ as above. Then
        $w(F) \ge d(n)$.
\end{lemma}

\begin{proof} This is Lemma~\ref{proposition:wttau}.
\end{proof}

\begin{lemma} Let $v$ be a node of type $(p,q,0)$ in a
formally contravening hypermap.  Define $w(F)$ as above. Then
        $$\sum_{v\in F} w(F) \ge b(p,q).$$
\end{lemma}


\begin{proof} This is Lemma~\ref{lemma:pq:bis}.
\end{proof}

\begin{lemma} Let $V$ be any set of nodes of type $(5,0,0)$ in a
formally contravening hypermap.  Define $w(F)$ as above.
        If the cardinality of $V$ is $k\le 4$,
        then
        $$\sum_{V\cap F\ne\emptyset} w(F) \ge 0.55 k.$$
\end{lemma}

\begin{proof} This is Lemma~\ref{lemma:0.55:bis}.
\end{proof}

The following theorem establishes the final property that $w(F)$
must satisfy to make it admissible.  {\it Separated sets\/} are
defined in Section~\ref{sec:wtassign}.

\begin{theorem}
        \label{proposition:excess}
        Let $V$ be any separated set of nodes in a formally contravening hypermap.
        Define $w(F)$ as above.
        Then
        $$\sum_{V\cap F\ne\emptyset} (w(F) -d(\card(F)))
            \ge \sum_{v\in V} a(p_v),$$
        where $p_v$ denotes the number of triangles containing
        the node $v$.
\end{theorem}

The proof will occupy the rest of this \chap. Since the cardinality
of each node is five, and there is at least one face that is not a
triangle at the node, the only constants $p_v$ that arise are
    $$p_v \in\{0,\ldots,4\}$$
We will prove that in a formally contravening hypermap that the
Properties (1) and (4) of a separated set are incompatible with
$p_v\le 2$.  This will allows us to assume that
$$p_v\in\{3,4\},$$ for all $v\in V$.  These cases will be treated in
Section~\ref{sec:tri34}.

%\section{Proof that $p_v>2$}
%%subsection
%\label{sec:2.4} \label{sec:tri2}
%
%In this subsection $(H,\azim,\flat,\sigma)$ is a formally
%contravening hypermap.  Let $V$ be a separated set of nodes in $H$.
%
%\begin{lemma}  Under these conditions, for every $v\in V$,
%$p_v>1$.
%\end{lemma}
%
%\begin{proof}
%If there are $p$ triangles, $q$ quadrilaterals, and $r$ other
%faces, then
%    $$
%    \begin{array}{lll}
%    \tau^*(H) &\ge\sum_{v\in F}\tau(F)\\
%        &\ge r\, t_5 + \tauLP(p,q,2\pi-r(1.153)).
%    \end{array}
%    $$ If there is a node $w$ that is
%not on any of the faces containing $v$, then the sum of $\tau(F)$
%over the faces containing $w$ yield an additional $0.55\,\pt$ by
%Lemma~\ref{lemma:0.55:bis}. We calculate these constants for each
%$(p,q,r)$ and find that the bound is always greater than
%$\squander$. This implies that $H$ cannot be formally contravening.
%$$\begin{array}{llll}
%    (p,q,r)&\hbox{\it lower bound }&\hbox{\it justification}\\
%    &\\
%    (0,5,0)&22.27\,\pt&\text{Lemma~\ref{lemma:pq:bis}}\\
%    (0,q,r\ge1)& t_5+4 t_4\approx 14.41\,\pt +0.55\,\pt& \\
%    (1,4,0) &17.62\,\pt &\text{Lemma~\ref{lemma:pq:bis}}\\
%    (1,3,1) &t_5 + 12.58\,\pt &(\tauLP)\\
%    (1,2,2) &2t_5 + 7.53\,\pt &(\tauLP)\\
%    (1,q,r\ge3)& 3 t_5 + t_4& \\
%\end{array}
%$$
%\end{proof}
%
%
%\begin{lemma} Under these same conditions, for every $v\in V$,
%$p_v>2$.
%\end{lemma}
%
%\begin{proof}
%Assume that $p_v=2$.  We will show that this implies that $H$ does
%not contravene.  Let $r=r_v$ be the number of exceptional faces at
%$v$. We have $r+p_v\le5$.  We consider various cases, according to
%the value of $r$.
%
%The constants $0.55\,\pt$ and $0.48\,\pt$ used throughout the
%proof come from Lemma~\ref{lemma:0.55:bis}. The constants $t_n$
%comes from Lemma~\ref{lemma:sn-tn}.
%
%($(p,q,r)=(2,0,3)$): First, assume that there are three exceptional
%faces around node $v$. They must all be pentagons
%($2t_5+t_6>\squander$). The aggregate of the five faces is an
%$m$-gon (some $m\le11$).  If there is a node not on this aggregate,
%use $3t_5+0.55\,\pt>\squander$. So there are at most nine triangles
%away from the aggregate, and the Euler relation gives
%    $$
%    \sigma^*(H) \le 9\,\pt + (3 s_5+2\,\pt) < 8\,\pt.
%    $$
%
%($(p,q,r)=(2,1,2)$): The argument if there is a quad, pentagon, and
%hexagon is the same $(t_4+t_6=2t_5,s_4+s_6=2s_5)$.
%
%Assume next that there are two pentagons and a quadrilateral around
%the node. The contour loop around the two pentagons, quadrilateral,
%and two triangles is has $m$ face steps (some $m\le10$). There must
%be a node exterior to this loop, for otherwise the Euler relation
%gives
%    $$
%    \sigma^*(H) \le 8\,\pt+(2s_5+2\,\pt)<8\,\pt.
%    $$
%
%The azimuth angle of one of the pentagons is at most $1.32$.  For
%otherwise, $\tauLP(2,1,2\pi-2(1.32))+2t_5+0.55\,\pt>\squander$.
%
%Lemma~\ref{lemma:1.47} shows that any pentagon $F$ with an azimuth
%angle less than $1.32$ yields $\tau(F)\ge t_5+ (1.47\,\pt)$. If both
%pentagons have an azimuth angle $<1.32$ the lemma follows easily
%from this calculation:
%    $2(t_5+1.47\,\pt)\,\pt+\tauLP(2,1,2\pi-2(1.153))+0.55\,\pt>\squander$.
%If there is one pentagon with angle $>1.32$, we then have
%    $t_5+(1.47\,\pt)+\tauLP(2,1,2\pi-1.153-1.32)+t_5+0.55\,\pt>\squander$.
%
%
%($(p,q,r)=(2,2,1)$): Assume finally that there is one exceptional
%face at the node. If it is a hexagon (or more), we are done
%$t_6+\tauLP(2,2,2\pi-1.153)>\squander$. Assume it is a pentagon. The
%contour loop around the five faces at the node has $m$ face steps
%(some $m\le9$). If there are no more than $9$ triangles exterior to
%the contour loop, then $\sigma^*(H)$ is at most
%$(9-2(0.48))\,\pt+s_5+\sLP(2,2,2\pi-1.153)<8\,\pt$
%(Lemma~\ref{lemma:0.55:bis}). So by the Euler relation, we may
%assume that there are at least three nodes exterior to the contour
%loop.
%
%If the azimuth angle of the dart on the pentagon is greater than
%$1.32$, we have
%  $$\tau^*(H)\ge\tauLP(2,2,2\pi-1.32) +3(0.55)\,\pt +t_5 > \squander;$$
%and if it is less than $1.32$, we have by Lemma~\ref{lemma:1.47}
%    $$
%    \begin{array}{lll}
%        \tau^*(H)\ge\tauLP(2,2,2\pi-1.153)&+3(0.55)\pt+1.47\,\pt+t_5 \\
%            &> \squander.
%    \end{array}
%    $$
%\end{proof}
%

\subsection{separated sets} %when p=3,4 %subsection
\label{sec:2.7} \label{sec:tri34}

In this subsection $(H,\azim,\flat,\sigma)$ is a formally
contravening hypermap.  Let $V$ be a separated set of nodes.  We
assume that there are three or four triangles meeting $v$, for every
$v\in V$.

To prove the Inequality \ref{definition:admissible:excess} in the
definition of admissible weight assignments, we will rely on the
following reductions. Define an equivalence relation on exceptional
faces by $F\sim F'$ if $F=F'$ or if there is a sequence
$F=F_0,\ldots, F_r=F'$ of exceptional faces such that consecutive
faces share a node of type $(3,0,2)$. Let ${\cal F}$ be an
equivalence class of faces.

%% XX GIVE FIGURE HERE with lots of exceptionals.

\begin{lemma} Let $V$ be a separated set of nodes.  For every
equivalence class of exceptional faces $\cal F$, let $V({\cal F})$
be the subset of $V$ whose nodes meet a face in ${\cal F}$. Suppose
that for every equivalence class $\cal F$, the Inequality
\ref{definition:admissible:excess} (in the definition of admissible
weight assignments) holds for $V({\cal F})$. Then the Inequality
holds for $V$.
\end{lemma}

\begin{proof}
By construction, each node in $V$ lies in some $F$, for an
exceptional face.  Moreover, the separating property of $V$ insures
that the triangles and quadrilaterals in the inequality are
associated with a well-defined  ${\cal F}$. Thus, the inequality for
$V$ is a sum of the inequalities for each $V({\cal F})$.
\end{proof}


\begin{lemma}
\label{lemma:split}
 Let $v$ be a node of type $(p,q,r)$ in a separated set $V$.  Suppose that
for some $p'\le p$ and $q'\le q$, we have a lower bound of the form
    $$( p' d(3) + q' d(4) + a(p))\,\pt$$
for what is squandered by $p'$ triangles and $q'$ quadrilaterals at 
a vertex $v$.  Suppose further that the
Inequality~\ref{definition:admissible:excess} (in REFXX) holds for
the separated set $V' = V\setminus \{v\}$. Then the inequality holds
for $V$.
\end{lemma}

\begin{proof}  Let $F_1,\ldots,F_m$, $m={p'+q'}$, be faces corresponding
to the triangles and quadrilaterals in the lemma.  The hypotheses
of the lemma imply that
    $$\sum_{1}^{m} (w(F_i) - d(\card(F_i))) \ge a(p).$$
Clearly, the Inequality for $V$ is the sum of this inequality, the
inequality for $V'$, and $w(F)- d(F)\ge0$.
\end{proof}


\begin{lemma}  Property \ref{definition:admissible:excess}  of
admissibility holds.  That is, let $V$ be any separated set of
nodes. Then
        $$\sum_{F:\,V\cap F\ne\emptyset} (w(F) -d(\card(F)))
            \ge \sum_{v\in V} a(p_v).$$
\end{lemma}

\begin{proof}  Let $V$ be a separated set of nodes.
The results of Section~\ref{sec:tri2} reduce the lemma to the case
where $p_v\in\{3,4\}$ for every node $v\in V$.


One case is easy to deal with.  A node of type $(3,1,1)$ such
that the dart on the exceptional face is at least $1.32$ has a
bound of type Lemma~\ref{lemma:split} by Lemma~\ref{a:311}.
For the rest of the proof, assume that the azimuth angle on
the exceptional face $F$ is less than $1.32$ at nodes of type
$(p,q,r)=(3,1,1)$. This implies in particular by
Lemma~\ref{lemma:1.32:bis} that the dart $x(F,v)$ is flat.

Another case is easy to deal with.  Lemma~\ref{a:no-ef}
shows that a node with no exceptional flat darts also
falls into the situation of Lemma~\ref{lemma:split}.
Thus, we may assume that at each $v\in V$, there is an exceptional
flat dart.

Pick a function $f$ from the set $V$ to the set of exceptional
standard regions as follows. Let $X$ be the set of exceptional faces
$F$ at $v$ for which $x(F,v)$ is flat.  From $X$, let $f(v)$ be the
one with smallest $\azim(x(F,v))$.  We see by construction and
Lemma~\ref{lemma:1.32} that $F = f(v)$ has the properties:
    \begin{itemize}
        \item $\flat(x(F,v))$
        \item $\azim(x(F,v)) > 1.32\ \Rightarrow\ \azim(x(F',v)) >
        1.32$, for any exceptional face $F'$ meeting $v$.
    \end{itemize}

For each exceptional face $F$, let
    $$V_F = \{ v\in V : f(v) = F\}.$$  This set may be empty for
some $F$.  Let $A_F$ be the union of $\{F\}$, and the set of
triangles and quadrilaterals with a dart in some $v\in V_F$.  If
$V_F$ is empty, then $A_F =\{F\}$.  The indexing set $A$ of Property
\ref{definition:admissible:excess} of admissibility is the disjoint
union of $A_F$.  The set $V$ is the disjoint union of the $V_F$ (or
at least of the nonempty ones). So the result follows from
Lemma~\ref{XX} for all the faces.
\end{proof}



The proof that formally contravening hypermaps are tame is complete.

\subsection{more about tame hypermaps}
%% CUT FROM TAME GOOD STUFF.

We have seen that a system of points and arcs on the unit sphere
can be associated with a centered packing $D$.  The points are the
radial projections of the nodes of $U(D)$ (those at distance at
most $2t_0=2.51$ from the origin).  The arcs are the radial
projections of edges between $v,w\in U(D)$, where $|v-w|\le2t_0$.
If we consider this collection of arcs combinatorially as a
hypermap, then it is not always true that these arcs form a
hypermap in the restrictive sense of
\Chap~\ref{sec:def-and-class}.

The purpose of this section is to show that if the original
centered packing contravenes, then minor modifications can be made
to the system of arcs hypermap so that the resulting combinatorial
hypermap has the structure of a hypermap in the sense of
\Chap~\ref{sec:def-and-class}. These hypermaps are called
contravening hypermaps.

A natural number $n(R)$ is associated with each standard region. If
the boundary of that region is a simple polygon, then $n(R)$ is the
number of sides.   If the boundary consists of $k$ disjoint simple
polygons, with $n_1,\ldots,n_k$ sides then
    $$n(R) = n_1+\cdots+n_k + 2(k-1).$$


\begin{lemma}\label{lemma:enclosed:bis} % {Lemma 2.2}
A quadrilateral region does not enclose any vertices of height at
most $2t_0$.
\end{lemma}

%% Summation convention:

%If $F$ is a face of $H$, let
%    $$\sigma_F(D) = \sum \sigma(F),$$
%where the sum runs over the set of standard regions associated with
%$F$.  This sum reduces to a single term unless $F$ is an aggregate
%in the sense of Remarks~\ref{remark:degree6} and
%\ref{remark:tri-pent}.


%% Here is stuff for after the definition of formally contravening hypermap.

%\begin{assumption}  $H$ is a planar hypermap.  $e$ is an
%involution that acts without fixed points.  Every face meets every
%node in at most one dart.  Every face has cardinality at least $3$
%and at most $8$.
%\end{assumption}


%\chapter{The Aggregate Cases}
%    \label{sec:aggregate}

\subsection{weight assignments for aggregates}

\begin{lemma} The bound $tri(v)>2$ holds if $v$ is a node
of an aggregate face.
\end{lemma}

\begin{proof}
The exceptional region enters into the preceding two proofs in a
purely formal way.  Pentagons enter through the bounds
    $$t_5,\ s_5,\ 1.47\,\pt$$
and angles $1.153$, $1.32$.  Hexagons enter through the bounds
    $$t_6,\ s_6$$
and so forth.  These bounds hold for the aggregate faces.  Hence the
proofs hold for aggregates as well.
\end{proof}

\begin{lemma}
Consider a separated set of nodes $V$ on an aggregated face $F$ as
in Remark \ref{remark:tri-pent}.  The Inequality
\ref{definition:admissible:excess} holds (in the definition of
admissible weight assignments):
    $$\sum_{V\cap F\ne\emptyset} (w(F) -d(\card(F)))
            \ge \sum_{v\in V} a(tri(v)).$$
\end{lemma}

\begin{proof}
We may assume that $tri(v)\in\{3,4\}$.

First consider the aggregate of Remark \ref{remark:tri-pent} of a
triangle and eight-sided region, with pentagonal hull $F$. There
is no other exceptional region in a contravening centered packing
with this aggregate:
    $$t_8 + t_5 > \squander.$$
A separated set of nodes $V$ on $F$ has cardinality at most $2$.
This gives the desired bound $$t_8 > t_5 + 2 (1.5)\,\pt.$$

Next, consider the aggregate of a hexagonal hull with an enclosed
node.  Again, there is no other exceptional face. If there are at
most $k\le 2$ nodes in a separated set, then the result follows from
    $$t_8 > t_6 + k (1.5)\,\pt.$$
There are at most three nodes in $V$ on a hexagon, by the
non-adjacency conditions defining $V$. A node $v$ can be removed
from $V$ if it is not the central node of a flat quarter (Lemma
\ref{lemma:split} and Inequalities~\ref{eqn:tau1.32} and
Lemma~\ref{a:no-ef}). If there is an enclosed node $w$, it is
impossible for there to be three nonadjacent nodes, each the central
node of a flat quarter.  In fact, by Lemma~\ref{tarski:node},
any enclosed node must have height greater than $2t_0$.



Finally consider the aggregate of a pentagonal hull with an enclosed
node.  There are at most $k\le2$ nodes in a separated set in $F$.
There is no other exceptional region:
    $$t_7 + t_5 > \squander.$$
The result follows from
    $$t_7 > t_5 + 2(1.5)\,\pt.$$
\end{proof}

\begin{lemma}
Consider a separated set of nodes $V$ on an aggregate face of a
contravening hypermap as in Remark~\ref{remark:degree6}.  The
Inequality~\ref{definition:admissible:excess} holds in the
definition of admissible weight assignments.
\end{lemma}

\begin{proof}
There is at most one exceptional face in the hypermap:
    $$t_8 + t_5 > \squander.$$
Assume first that aggregate face is an octagon (Figure
\ref{fig:degree6}). At each of the nodes of the face that lies on a
triangular standard region in the aggregate, we can remove the node
from $V$ using Lemma \ref {lemma:split} and the estimate
    $$\tauLP(4,0,2\pi-2 (0.8638)) > 1.5\,\pt.$$
This leaves at most one node in $V$, and it lies on a node of $F$
which is ``not aggregated,'' so that there are five standard
regions of the associated centered packing at that node, and one
of those regions is pentagonal.  The value $a(4)=1.5\,\pt$ can be
estimated at this node in the same way it is done for a
non-aggregated case in Section~\ref{sec:tri34}.

Now consider the case of an aggregate face that is a hexagon (Figure
\ref{fig:degree6}).  The argument is the same: we reduce to $V$
containing a single node, and argue that this node can be treated as
in Section~\ref{sec:tri34}.  (Alternatively, use the fact that the
pentagon-triangle combination in this aggregate has been eliminated
by Lemma~\ref{lemma:nobad4}.)
\end{proof}


%% STUFF ABOUT CENTRAL VERTICES, QUARTERS AND SO FORTH.

Recall that the central vertex of a flat quarter is defined to be
the one that does not lie on the triangle formed by the origin and
the diagonal.
%
 \index{central}

%%

XX?  We will say that there is a flat quarter centered at $v$, if
the corner $v'$ over $v$ is the central node of a flat quarter and
that flat quarter lies in the cone over an exceptional region.

%%


%% Table simplified...  The entry (7,0) with 14.76 is relevant here.
%% It needs to be treated.

Define constants $\tlp(p,q)/\pt$ by Table~\ref{eqn:old5.1:bis}. The
entries marked with an asterisk will not be needed.
%
 \index{type (of a node)}
 \index{ZZtauLP@$\tlp(p,q)$}

\begin{equation}
\vbox{\offinterlineskip \hrule
\halign{&\vrule#&\strut\ \hfil#\hfil\ \cr   % "\ " was quad
height 7pt&\omit&&\omit&&\omit&&\omit&&\omit&&\omit&&\omit&\cr
&\hfil $\tlp(p,q)/\pt$\hfil
        &&\hfil $q=0$\hfil
        &&\hfil1\hfil
        &&\hfil2\hfil
        &&\hfil3\hfil
        &&\hfil4\hfil
        &&\hfil5\hfil&
\cr height 7pt&\omit&&\omit&&\omit&&\omit&&\omit&&\omit&&\omit&\cr
\noalign{\hrule}
height7pt&\omit&&\omit&&\omit&&\omit&&\omit&&\omit&&\omit&\cr
&$p=0$&& *&& *&& 15.18&& 7.135&& 10.6497&& 22.27&\cr &1&&    *&& *&&
6.95&& 7.135&&17.62  && 32.3&\cr &2&&    *&&
8.5&&4.756&&12.9814&&*&&*&\cr &3&& *&& 3.6426&&8.334&&20.9&&*&&*&\cr
&4&&4.1396&&3.7812&&16.11&&*&&*&&*&\cr
&5&&0.55&&11.22&&*&&*&&*&&*&\cr &6&&6.339&&*&&*&&*&&*&&*&\cr
&7&&14.76&&*&&*&&*&&*&&*&\cr
height7pt&\omit&&\omit&&\omit&&\omit&&\omit&&\omit&&\omit&\cr}
\hrule }
    %oldtag 5.1
    \label{eqn:old5.1:bis}
\end{equation}


%%  (1,0,1)

\subsection{a non-contravening four-circuit}
\label{sec:impossible-circuit}

This subsection rules out the existence of a particular four-circuit
on a contravening hypermap.  The interior of the circuit consists of
two faces: a triangle and a pentagon.  The circuit and its interior
node are show in Figure~\ref{fig:no4circuit:bis} with nodes marked
$p_1,\ldots,p_5$. The node $p_1$ is the interior node, the triangle
is $(p_1,p_2,p_5)$ and the pentagon is $(p_1,\ldots,p_5)$.


Let $v_1,\ldots,v_4,v_5$ be the corresponding vertices of $U(D)$.
XX?.

The diagonals $\{v_5,v_3\}$ and $\{v_2,v_4\}$ have length at least
$2\sqrt2$ by Lemma~\ref{tarski:2t0-doesnt-pass-through}.  If an
azimuth angle of the  quadrilateral is less than $1.32$, then by
Lemma~\ref{lemma:1.32:bis},  $|v_1-v_3|\le\sqrt{8}$.  Thus, we
assume in the following lemma, that all azimuth angles of the
quadrilateral aggregate are at least $1.32$.

%%

\begin{remark}
We have now fully justified the claim made in
Remark~\ref{remark:degree6}: there is at most one node on six
standard regions, and it is part of an aggregate in such a way that
it does not appear as the node of $H$.
\end{remark}

%%%%%%%%%%%%%%%%%%%%%%

%% AGGREGATE STUFF IN THE PROOF OF \ref{definition:tame:score}
%% Property of Tameness.

We consider three cases for Inequality \ref{eqn:sigma}. In the
first case, assume that the face $F$ corresponds to exactly one
standard region in the centered packing.  XX? In this case,
Inequality \ref{eqn:sigma} follows directly from the bounds of
Lemma~\ref{lemma:sn-tn}:
    $$\sigma(F)\le s_n \le c(n)\,\pt.$$

In the second case, assume we are in the context of a pentagon $F$
formed in Remark~\ref{remark:tri-pent}.  Then, again by
Theorem~\ref{lemma:sn-tn}, we have
$$\sigma(F) \le s_3+s_8\le (c(3)+c(8))\,\pt \le c(5)\,\pt.$$
(Just examine the constants $c(k)$.)

In the third case, we consider the situation of Remark
\ref{remark:degree6}.  The six faces give
$$\sigma(F)\le s_5+\sLP(5,0,2\pi-1.153)< c(8)\,\pt.$$
The constant $1.153$ comes from Lemma~\ref{lemma:0.8638}.


%%%%%%%%%%%%%%%%%%%%%%%%%%%%%


%%%%%%%%%%%%%%%%%%%%%%%%%%%%%%%%%%%%%%%%%%%%%%%%%%%%

    %
\chapter{Linear Programs}
    \label{sec:lp}


\section{Introduction}

Until now, the most poorly redacted part of the proof of the
packing problem has been the linear programming.  The proof
involves approximately 100,000 linear programs, each involving
some 200 variables and 2000 constraints.  The data for these
linear programs fills about three gigabytes of storage.

The purpose of this document is to give a specification of the
linear programs. This specification has never been produced,
except in highly abbreviated form in the original 1998 proof. The
linear programs are left so vague that it would be extremely
difficult from that description alone to give independent
corroboration of the results.

The original proof appeared as a series of papers named {\it
Sphere Packings I} through {\it Sphere Packings VI}.  This paper
may be viewed as a detailed version of the second half of 
{\it Sphere Packings VI}.

\subsubsection{flyspeck}

An important part of the motivation for the work comes from the
Flyspeck project, which aims to give a formal proof of the packing
problem.  The word flyspeck has the form F*P*K, which is an
acronym for the Formal Proof of the Kepler conjecture.

Steven Obua has made important progress toward the formal
verification of linear programs, as needed for the Flyspeck
project.  A specification
of the linear programs in a form suitable for formal verification
is required to complete that project.
This chapter is the first step toward bridging that gap.

\subsubsection{geometry}

This chapter eliminates all geometry.  The problem is transformed
into a purely combinatorial and algebraic problem.
A hypermap (which is just a finite set
with two permutations) encodes most of the combinatorial
structure.  The rest is encoded as discrete collection of flags.
The linear programs can be specified directly from the hypermap
and flags.  There is a database of a few thousand hypermaps that
are used in the construction of the systems of inequalities.  The
flags are abstractions of various geometric conditions, and yet
it is not necessary to understand anything about these geometric
conditions to work with the flags.

\subsubsection{formal inequality}

There is a tight relationship between the linear inequalities that
appear in the linear programs and certain nonlinear inequalities
that appear in a database.  A simple example is illustrated in
Figure~\ref{XX}.  The three angles $\alpha_i$ of the triangle
satisfy
    \begin{equation}
    \label{eqn:2pi}
    \alpha_1 + \alpha_2 + \alpha_3 = 2\pi.
    \end{equation}
This equality can be viewed two ways.  It can be viewed linearly
as the equation of an affine plane in $\ring{R}^3$.  Or, it can be
viewed nonlinearly, where each $\alpha_i$ is a nonlinear function
of the edge lengths of the triangles, and the equation asserts a
nonlinear relation among edge lengths.

For humans it is a simple matter to switch back and forth between
these points of view.  To understand the 1998 solution to the 
packing problem, it is necessary to jump continually back and forth
between these two views of an equation.

For computers, it is not such a simple matter to jump from one
point of view to another.  To assist in the full automation of the
proof, we create a new object called {\it formal inequalities}
whose purpose is to provide the connection between the two types
of inequalities.  A formal inequality can be {\it deformalized}
into an ordinary linear inequality.  It can also admits a
nonlinear {\it interpretion}.  A formal inequality thus appears as
a structure that unifies a linear inequality with a nonlinear
inequality (Figure~\ref{fig:formal}). With this new structure, we
have the framework that would allow us to completely automate the
process of retrieving linear programs from a database of nonlinear
inequalities.

\begin{figure}[htb]
  \centering
  %% WW \myincludegraphics{\ps/formal.eps}
  \caption{Unification of nonlinear and linear inequalities}
  \label{fig:formal}
\end{figure}

\subsubsection{infeasibility}

To solve the packing problem, it is enough to show that no
counterexamples exist.  In technical terms, it is enough to show
that no contravening centered packings exist (except for those
attached to the face-centered cubic and hexagonal close packings).

The strategy for proving the nonexistence of counterexamples is
remarkably naive. In simple terms, it goes as follows.  First, put
all the counterexamples in a box. Then measure the width of the
box. If the width of the box is negative, the box contains no
counterexamples.

This basic idea can be embellished in various ways.  The box can
be replaced by more general polyhedra defined by hyperplane
inequalities.  Instead of showing that the box has negative width,
we can show that the system of hyperplane inequalities admits no
feasible solutions.

A further refinement of this idea allows us to cover the set of
counterexamples with a finite collection of polyhedra and then
show that each polyhedron is empty.  We can do this in with an
iterative refinement.  If a given cover of the space of
counterexamples contains a polyhedron that is not empty, we can
replace it with two others that give a finer cover of the space of
counterexamples.  If by a series of  refinements we eventually
arrive at a cover by empty polyhedra, our goal (of eliminating all
counterexamples) is accomplished.

In the original 1998 proof, the series of refinements was guided
by detailed geometric information about the space of centered
packings.  In this revision of the proof, the series of
refinements is purely algebraic and combinatorial.

\subsubsection{linear program}

This idea of covering the space of counterexamples with a finite
collection of empty polyhedron can be expressed somewhat
differently in terms of linear programming.  In fact, it is
possible to describe the problem in such a way that no reference
is made to the space of counterexamples.

The basic aim of the linear program is to show the infeasibility
of a certain system.  However, the primary linear programs are not
infeasible in the traditional sense.  It is necessary to weaken
the definition of infeasibility to encompass systems that
eventually become infeasible after certain branch and bound
operations.  These operations are somewhat subtle, because the
underlying combinatorial structure of the linear program changes
as branching operations progress.   In Section~\ref{sec:weak}, we
will say that a system is weakly infeasible, if a combinatorial
tree of systems can be constructed starting with the given system
as the root, and infeasible systems as the outermost leaves.  The
hypermap and its flags encode the possible branchings of the tree.
The main theorem of this chapter will then assert that each
primary system is weakly infeasible.

This weak infeasibility result is one of the main steps of the
solution to the packing problem.  Another chapter will show that
if there is a counterexample, then it can
be used to construct a strongly feasible solution to a primary
system.  The existence of a strongly feasible solution is
incompatible with weak infeasibility.  Thus, weak infeasibility
leads to a proof that there are no counterexamples.


\subsubsection{functional language}

Our aim has been to give a description of the linear programming
part of the packing problem in a way that is conducive to
implementation in a functional programming language.  We have
allowed ourselves to drift somewhat from conventional mathematical
syntax toward expressions that have a bit more the look and feel
of a functional programming language.  We note that there are
scarcely any theorems in this chapter.  This is actually a highly
desirable thing, from the point of view of formalization.

\subsubsection{modular design}

In a long and complex proof, it is desirable to modularize and
encapsulate the various parts of the proof to the greatest extent
possible.  We have been able to make this chapter almost entirely
self contained: it can be read with almost no reference to other
chapters in this book.  To achieve this, we have repeated a few
results from earlier chapters, such as material on hypermaps.
There are a few outside facts that we make reference to, such as
the existence of a database of nonlinear inequalities, and the
existence of an archive of planar hypermaps.  For the purposes of
this chapter, is not necessary to understand the origin of this
database or this archive. They can simply be accepted as part of
the givens of our situation.

The main theorem (Theorem~\ref{thm:lpbound}) of this chapter solves
one of the main parts of the packing problem.
Although the theorem seems at first glance to be rather simple,
this apparent simplicity is deceptive, because the definitions on
which the theorem rests, are extremely long and cumbersome.  This
was the price of encapsulation.

\subsubsection{warning and hope}

There are some minor incompatibilities between how things are set
up here and how things were set up in the original 1998 solution.
These incompatibilities were introduced to
simplify matters.  My hope is that they have not broken anything.
However, the only way to be sure of this is to generate all the
linear programs and check that they all hold.  I have not carried
this out, so the reader is encouraged to be a bit skeptical of
everything that is written here, at least until those
verifications have been made.

As part of the Flyspeck project, it is our hope that one day there
will be a formal proof of the main theorem of this chapter.  S.
Obua can provide details about the current status of this project.

\section{Hypermap Review}

The combinatorial structure of the packing problem is expressed
through various hypermaps.   This chapter takes
the connected sums of hypermaps, as described in Section~\ref{sec:csum}.

%
%\subsection{basic Definitions}
%
%\begin{definition}[hypermap]  A {\it hypermap} $H=(D,e,n,f)$ is a finite set $D$
%together with three permutations $e,n,f:D\to D$ satisfying the
%identity $e\circ n\circ f = I$.  The elements of $D$ are called
%{\it darts}.  The permutations $e,n,f$ are called the {\it edge},
%{\it node}, and {\it face} permutations, respectively.
%\end{definition}
%

\begin{definition}[face,~$H/f$]  A {\it face} of a hypermap $H=(D,e,n,f)$ is an orbit of $D$
under $f$.  We write $\op{face}(\alpha)$ for the face of
$\alpha\in D$.  We write $H/f$ for the set of faces.
\end{definition}

\begin{definition}[edge,~$H/e$]  An {\it edge} of a hypermap $H=(D,e,n,f)$ is an orbit of $D$
under $e$.  We write $\op{edge}(\alpha)$ for the edge of
$\alpha\in D$.  We write $H/e$ for the set of edges.
\end{definition}

\begin{definition}[node,~$H/n$]  A {\it node} of a hypermap $H=(D,e,n,f)$ is an orbit of $D$
under $n$.  We write $\op{node}(\alpha)$ for the node of
$\alpha\in D$.  We write $H/n$ for the set of nodes.
\end{definition}


\section{Linear Programming}



\subsection{abstract theory}

We assume a finite indexing set $I$ that will be used  to index
all the variables for a linear program. Let $\ring{R}^I$ be the
vector space of real-valued functions on $I$. We define the
standard dot product $(\cdot)$ on $\ring{R}^I$ by
    $$v\cdot w = \sum_{i\in I} v_i w_i.$$

\begin{definition}[formal~inequality]
A formal inequality (on the indexing set $I$) is a triple
    $$(f:I\to\ring{R},s\in\{(\le),(=),(\ge)\},c\in\ring{R}).$$
The second coordinate $s$ is a binary relation on $\ring{R}$.
\end{definition}

Let $\op{FOR}_I$ be the set of formal inequalities.  We drop the
subscript $I$ when the indexing set $I$ is fixed.  There is a
function $\op{def}:\op{FOR}\to \ring{R}^I \to \bool$ (short for
deformalize) given by
    $$\op{def}:\ (f,s,c)\ v \mapsto s(f\cdot v,c).$$

\begin{remark}
We use the shorthand of writing $(f,\op{sgn},c)$ as
   $$f\,\,\op{{\bf sgn}},\, c,$$
where $\op{{\bf sign}}$ is the curly version $(\seq,\sle,\sge)$ of
the relation symbol $(=,\le,\ge)$. For example, the formal
inequality
    $(\optt{y}(\alpha),(\ge),0)$ is written simply as $\optt{y}(\alpha)\sge 0$.
The curly relation symbol reminds us that the inequality is
formal.  Without the modified relation symbol, this shorthand
would be dangerously ambiguous, because a formula such as
    $$\optt{y}(\alpha) = 0$$
could be read either as a formal identity
$(\optt{y}(\alpha),(=),0)$, or as an ordinary vector equality
between the vector $\optt{y}(\alpha)$ and the zero vector $0$.

We generalize the notation to allow statements such as
    $$v \seq w$$
for the formal inequality $(v-w,(=),0)$.
\end{remark}

%The image of $\op{def}$ on a formal inequalities is called an
%inequality (or equality if we wish to stress that it comes from an
%equality constraint). Let $\op{INEQ}$ be the set of inequalities.

\begin{definition}[formal~disjunction]
A formal disjunction $A$ is a finite set of finite sets of
$\op{FOR}_I$. (Intuitively, the set is to be thought of as
representing a formula in disjunctive normal form, whose atomic
formulas are formal inequalities.) Let $\op{LFOR}_I$ be the set of
formal disjunctions. Again, we drop the subscript $I$, when the
indexing set $I$ is fixed.  We may consider a formal inequality as
a formal disjunction under the map
    $$a \mapsto \{\{a\}\}.$$
\end{definition}

\begin{remark}
The formal disjunction
    $$A = \{\{a_{11},a_{12},\ldots,\},\ldots,\{a_{k1},\ldots\}\}$$
will be written more suggestively as
    $$A = (a_{11}\wedge a_{12}\wedge\cdots) \vee \cdots \vee (a_{k1}\wedge
    \cdots)$$
as a formula in disjunctive normal form.  For example, the formal
disjunction
    $$A =  \{\{(y,(\le),0)\},\{y,(\ge),0\}\}$$
will be written
    $$A = (y\sle 0) \vee (y \sge 0).$$
If $A$ is a singleton: $A = \{\{\{a_{11},a_{12},\ldots,\}\}$ then
we write it as a conjunction:
    $$
    A = (a_{11}\wedge a_{12}\wedge\cdots)
    $$
\end{remark}

We extend $\op{def}$ to a function $\op{ldef}$ on $\op{LFOR}$ by
    $$\op{ldef}\ A \ v  =
    (a'_{11}\wedge a'_{12}\wedge\cdots) \vee \cdots \vee (a'_{k1}\wedge
    \cdots).$$
where $a'_{ij} = \op{def}(a_{ij}) v$ and $A\in \op{LFOR}$ has the
form
    $$
    A = \{\{a_{11},a_{12},\ldots,\},\ldots,\{a_{k1},\ldots\}\}$$

If $S$ is a set of formal disjunctions, let $\op{single}(S)$ be
the union of singleton sets (that is, conjunctions):
    $$
    \op{single}(S) = \bigcup \{a \mid \{a\} \in S\}
    $$

\begin{definition}[infeasible]
Let $S$ be a set of formal disjunctions (on an indexing set $I$).
We say that $S$ is {\it infeasible} and write
    $$\op{INFEAS}(S)$$
if $\op{single}(S)$ is an infeasible system of linear
inequalities.
\end{definition}


\subsection{inequalities from a database}
\label{sec:lookup}

There is a large database of nonlinear inequalities in a small
number of variables that has been proved correct by means of
interval arithmetic.  This subsection indicates how to extract
linear inequalities from that database to be used for linear
programming.

%Suppose that we are given a set of  configurations $C$ and wish to
%generate inequalities that are valid on a subset $S\subset C$.

We assume that the indexing set $I$ contains a subset $I'$ (called
the set of nonlinear indices). Assume a nonlinearization function
    $$
    \op{interp}:\ I' \to \ring{R}^{I} \to \ring{R}.
    $$
Extend it to
    $$
    \begin{array}{rll}
    \op{interp\_vec}:&\ \ring{R}^{I} \to \ring{R}^I \to
    \ring{R}\\
    u\ v \mapsto& \sum_{i\in I'} v_i (\op{interp}_i u) +
        \sum_{j\in I\setminus I'} v_j u_j.
    \end{array}
    $$
Extend it to
    $$
    \begin{array}{rll}
    \op{interp\_for}:&\ \op{FOR}\to\ring{R}^I\to\bool\\
    (f,s,c)\ u\mapsto& s(\op{interp\_vec} f u,c)\\
    \end{array}
    $$
Finally, extend it to
    $$
    \begin{array}{rll}
    \op{interp\_lfor}:\ \op{LFOR}\to&\ring{R}^I\to\bool\\
    ((a_{11}\wedge a_{12}\wedge\cdots) \vee \cdots \vee (a_{k1}\wedge
    \cdots)\ u\mapsto& (a'_{11}\wedge a'_{12}\wedge\cdots) \vee \cdots \vee (a'_{k1}\wedge
    \cdots)\\
    \end{array}
    $$
where $a'_{ij} = \op{interp\_for} (a_{ij}) u$.


Let $\op{nonlin}(C,L)$ be the formula
    $$
    \forall y\in\ring{R}^\ring{N}.\ (\forall c\in C.\
    c(y)) \Rightarrow L(y).
    $$
Suppose that we have a database $B$ of such nonlinear
inequalities.  Given a map $\psi:\ring{N}\to I$, there is a dual
map
    $$
    \pi_\psi: \ring{R}^I \to \ring{N}\to \ring{R},\quad v\ n\mapsto
    v_{\psi n}.
    $$


\begin{definition}[lookup]
    We say that the formal disjunction $A$ is a {\it lookup} in
    database $B$ with domain constraints in $S$ (a set of formal disjunctions),
    if there exists
        $$
        \op{nonlin}(C,L)\in B
        $$
    and there exist $\psi:\ring{N}\to I$ and
    $r:C\to\op{FOR}_I$
    such that the following two conditions hold
    \begin{itemize}
    \item The system $S$ implies the domain restrictions of
    $\op{nonlin}(C,L)$.  That is, for every $c \in C$,
        $$
        \forall v\in\ring{R}^I.\ \op{interp\_for} (r(c)) v
        \Rightarrow (c(\pi_\psi v))
        $$
    \item The predicate $L$ comes by interpretation of $A$.  That
    is,
        $$
        \forall v\in \ring{R}^I.\ (\forall c\in C.\ \op{interp\_for}(r(c)) v)
        \Rightarrow (L(\pi_\psi v) =
        \op{interp\_lfor}(A) v).
        $$
    \end{itemize}
\end{definition}


In brief, a formal disjunction $A$ is a lookup of
$\op{nonlin}(C,L)$ if the interpretation of $A$ coincides with the
predicate $L$ on a suitably restricted domain, and if those domain
constraints are in the system $S$.

It should be possible to write a completely automated tool that
finds the lookups from a database $B$ whose domain constraints lie
in $S$.

\begin{example}  We give a somewhat contrived example of lookup.
Later, nonlinear lookup will be performed on a massive database of
inequalities.  Suppose that our database contains the nonlinear
inequality:
    $$\forall y_1,\ldots,y_6.\ (2\le y_i \le 2.1,\ i=1,\ldots
    6) \Rightarrow (\Gamma(y_1,\ldots,y_6) \le \pt),
    $$
    for some nonlinear function $\Gamma$ and constant $\pt$.
We can put this in the form $N = \op{nonlin}(C,L)$ with
    $$
    \begin{array}{llll}
    %J &=\{1,\ldots,6\}\\
    C &= \{\\
        &y&\mapsto (y_1\ge 2)\\
        &y&\mapsto (y_1\le 2.1)\\
        &&\cdots\\
        &y&\mapsto (y_6\le 2.1)\}\\
    L &= (y&\mapsto (\Gamma(y_1,\ldots,y_6)\le \pt)).
    \end{array}
    $$
Suppose, further, for the sake of an example that our indexing
sets contain elements $g\in I'$ and $i_1,\ldots,i_6\in I\setminus
I'$ and that
    $$
    \op{interp}(g) = (v \mapsto \Gamma(v_{i_1},\ldots,v_{i_6})).
    $$
Let $\optt{gamma} = e_g \in\ring{R}^{I}$. Then we see that the
formal disjunction (in this case a single formal inequality)
$\optt{gamma} \sle \pt$ is a lookup in the database $B = \{N\}$
with domain constraints in
    $$
    S = \{e_{i_1}\sge 2,\ldots,e_{i_6}\sge 2,
          e_{i_1}\sle 2.1,\ldots, e_{i_6}\sle 2.1\},
    $$
\end{example}


\section{Hypermap Systems}

The previous sections have described the general theory of linear
programming without reference to the specifics of the packing
problem.  It is now time to develop linear programming in the
specific context of  hypermaps.

\subsection{statement of the main lp bound}

Let $\mathcal H$ be the set of  hypermaps that are connected,
plain, planar, and simple. For each $H\in \mathcal H$, there is a
set of combinatorial flags $\op{fl}$.  This set is defined in
Section~\ref{sec:flag}.  Section~\ref{sec:var-index} will specify
an indexing set $I$, a subset of nonlinear indices $I'\subset I$,
and an interpretation
  $$\op{interp}:\ I'\to \ring{R}^I\to\ring{R}$$
for $I'$.  The data $I$, $I'$, and $\op{interp}$ depend on the
parameters $(H,\op{fl})$: $I = I(H,\op{fl})$, etc.

\begin{definition}[hypermap~system]
A {\it hypermap system} is a triple $(H,\op{fl},S)$ where
    $H\in {\mathcal H}$,
    $\op{fl}$ is a
    combinatorial flag,
    and $S\subset \op{LFOR}_I$, with $I = I(H,\op{fl})$.
    Let $\op{HS}$ be the set of hypermap systems.
\end{definition}

Section~\ref{sec:PTHS} defines a subset $\op{PTHS}$ of $\op{HS}$
(the set of primary tame hypermap systems).  There is a database
of nonlinear inequalities (B) that is described in
Section~\ref{XX}. Section~\ref{sec:basic} will describe the
systems $S$ for $(H,\op{fl},S)\in\op{PTHS}$.  A predicate
$\op{weak\_infeas}$ (weak infeasibility) on hypermap systems
$\op{HS}$ will be described in Section~\ref{sec:weak}.

\begin{theorem}[Main LP Bound]\label{thm:lpbound} Assume the nonlinear
inequalities (B).  Then for every $\rho\in \op{PTHS}$, we have
$\op{weak\_infeas}(\rho)$.
\end{theorem}

Briefly put, every primary tame hypermap system is weakly
infeasible.  The result is obtained by linear relaxation and
linear programming.\footnote{The warning that appears at the
beginning of the chapter should be repeated.  There are some mild
incompatibilities with the linear programming verifications done
in 1998, and the new linear programs have not yet been executed.
Thus, it is possible that some adjustments may have to be made
before claiming the Main LP Bound as a theorem.}

\begin{remark}  Each of the nonlinear inequalities in the database
(B) was proved by interval arithmetic as part of the 1998 proof.
Thus, we can state the Main LP Bound with this hypothesis.  The
hypothesis is included simply to keep interval arithmetic out of
the proof.
\end{remark}

\subsection{flags}
\label{sec:flag}

The second component of a  hypermap system $(H,\op{fl},S)$ is a
combinatorial flag.  This subsection describes the flag and some
auxiliary functions.

\begin{definition}[flag]
 A flag is a tuple
    $$
    \op{fl}=(\op{qrn},\op{qre},\op{hex\_loop\_diag},
    \op{typ}(4),\op{sum\_typ}(4),\ldots,
    \op{typ}(8),\op{sum\_typ}(8))
    $$
 with flag components $\op{qrn},\ldots,\op{sum\_typ}(8)$ to be listed below.
 \end{definition}

We give a brief description of each of the flag components.  The
interpretations of the flag components are irrelevant to the proof
of the Main LP Bound.  The interpretations are needed to relate the Main LP Bound to bounds on the score of a centered packing.  Beyond that, 
interpretations are merely provided as a
guide to the intuition.  All that is relevant is the domain and
range of each function.  For instance, in the following
definition, the comment that $\op{qrn}$ ``detects quasi-regular
nodes'' is irrelevant to the Main LP Bound.  Any function $\op{qrn}:H/n\to\bool$
qualifies as a flag component.

 \begin{definition}[quasi-regular~node~and~edge,~qrn,~qre]
 The flag components $\op{qrn}$ and $\op{qre}$ detect {\it quasi-regular
 nodes and edges}.
        $$\op{qrn}:H \to \bool$$
        $$\op{qre}:H \to \bool$$
%These functions satisfy the constraint that for every dart $x$, if
%  $\op{qre}(x)$ then $\op{qrn}(x)$.
 The function $\op{qrn}$ is constant on orbits of $n$, so that it
 descends to a function on $H/n$.  The function $\op{qre}$ is
 constant on the orbits of $e$, so that it descends to a function
 on $H/e$.
\end{definition}

\begin{interpretation}[qrn,~qre] 
Let $(\Lambda,v_0)$ be a centered packing.  Let
$H$ be the hypermap of a $(\Lambda,\CalQ)$-compatible fan $(v_0,V,E)$.  
If $x=(v_0,v,\ldots)$ is a dart in that hypermap, we
set $\op{qrn}(x)$ to be true exactly when $|v_0-v|\le 2t_0$.  Each 
edge $z=\{x,ex\}$ of the hypermap corresponds with some edge $\{v_1,v_2\}\in E$.
We set $\op{qre}(x) = \op{qre}(e x)$ to be true exactly when
$|v_1-v_2|\le 2t_0$. %, $|v_0-v_1|\le 2t_0$, and $|v_0-v_2|\le 2t_0$.
\end{interpretation}

\begin{definition}[standard~face,~std] The function $\op{std}$ detects standard faces
in a hypermap.
    $$\op{std}:H/f \to \bool$$
$$\op{std}(F) = \forall\alpha\in F.\
\op{qrn}(\alpha)\wedge \op{qre}(\alpha)
$$
\end{definition}

\begin{notation}
We write $(H/f)_X$ to denote the subset of $H/f$ described by the
satisfying the given flags in $X$.  If the subscript is a natural
number, we let $(H/f)_n$ be the set of faces in $H$ of cardinality
$n$.  For example, $(H/f)_{4,\op{std}}$ is the set
    $$
    \{F \in H/f \mid \card(F) = 4 \wedge \op{std}(F)\}.
    $$
\end{notation}

%\begin{definition}[pri]
%The flag component $\op{pri}:(H/f)_{\op{std}}\to\bool$ tells
%whether a standard face is primary.
%\end{definition}

%\begin{definition}[is-refined]
%The flag component $\op{is\_refined}$ keeps tracks of changes to
%an initial configuration hypermap.
%   $$\op{is\_refined}:H/f\to \bool$$
%\end{definition}

In the following definitions, we give symbolic names to elements
of various sets.  For example, in the following definition
    $$\{\op{patchable},\op{puremix}\}$$
is to be understood as a finite set of cardinality two, with
elements $\op{patchable}$ and $\op{puremix}$.  Also, some of the
values depend on parameters.  For example, in the following
definition, if $F=\{\alpha_1,\ldots,\alpha_4\}$, then
    $$
    \op{sum\_typ}(4)(F)
    $$
can take on five distinct values:
    $$
    \op{upright}, \op{flat}(\alpha_1), \ldots,
    \op{flat}(\alpha_4).
    $$
This is indicated in abbreviated form in the following
definitions.

\begin{definition}[$\op{typ}$,~$\op{sum\_typ}$]
The flag components $\op{typ}(4)$ and $\op{sum\_typ}(4)$
classify the type of a standard quad face.
    $$
    \begin{array}{rll}
    \op{typ}(4):& F\in (H/f)_{4,\op{std}}
    \to
    \{\op{patchable},\op{puremix}\}\\
    \op{sum\_typ}(4):& F\in (H/f)_{4,\op{patchable}}
    \to
    \{\op{upright},\op{flat}(\alpha:F)\}.
    \end{array}
    $$
\end{definition}

\begin{remark}
The data for $\op{flat}(\alpha)$ and $\op{flat}(f^2\alpha)$ are
equivalence in a sense that I won't spell out in detail here.
Roughly, they both correspond to a connected sum obtained by drawing the
diagonal of a quadrilateral. In the 1998 proof, these two cases
were combined into a single proof. This creates fewer cases, but
leads to a proof obligation of proving their equivalence.  (This
is geometrically obvious, but it is not obvious in formal
verification.) Following Nipkow's work on the formalization of the
classification of tame plane graphs, it is my hunch that the
formal proof will go more smoothly if the proof obligation is
dropped, and the two are treated as separate cases.
\end{remark}

\begin{definition}[$op{typ}$,~$\op{sum\_typ}$]
The flag components $\op{typ}(5)$ and $\op{sum\_typ}(5)$
classify the type of a standard pentagonal face.
    $$
    \begin{array}{rll}
    \op{typ}(5):& F\in (H/f)_{5,\op{std}}
    \to
    \{\op{patchable},\op{trunc},\op{aggA},\op{aggB},\op{crowd3},\op{notpri}\}\\
    \op{sum\_typ}(5):& F\in (H/f)_{5,\op{patchable}}
    \to
    \{\op{loop5},\op{flat}(\alpha:F),\op{flat2}(\alpha:F),\op{loop4}\}.
    \end{array}
    $$
\end{definition}
%% not primary is standard but occurring in a connected sum.

\begin{definition}[$op{typ}$,~$\op{sum\_typ}$]
The flag components $\op{typ}(6)$ and $\op{sum\_typ}(6)$
classify the type of a standard hexagonal face.
    $$
    \begin{array}{rll}
    \op{typ}(6):& F\in (H/f)_{6,\op{std}}
    \to
    \{\op{patchable},\op{trunc},\op{agg},\op{crowd3},\op{crowd4},\op{loop51}\}\\
    \op{sum\_typ}(6):& F\in (H/f)_{6,\op{patchable}}
    \to
    \{\text{list connected sumes}\ldots\}.
    \end{array}
    $$
\end{definition}
\FIXX{list connected sums 6,7,8}

\begin{definition}[$op{typ}$,~$\op{sum\_typ}$]
The flag components $\op{typ}(7)$ and $\op{sum\_typ}(7)$
classify the type of a standard heptagonal face.
    $$
    \begin{array}{rll}
    \op{typ}(7):& F\in (H/f)_{7,\op{std}}
    \to
    \{\op{patchable},\op{trunc},\op{crowd3},\op{crowd4}\}\\
    \op{sum\_typ}(7):& F\in (H/f)_{7,\op{patchable}}
    \to
    \{\text{list connected sumes}\ldots\}.
    \end{array}
    $$
\end{definition}

\begin{definition}[$op{typ}$,~$\op{sum\_typ}$]
The flag components $\op{typ}(8)$ and $\op{sum\_typ}(8)$
classify the type of a standard octagonal face.
    $$
    \begin{array}{rll}
    \op{typ}(8):& F\in (H/f)_{8,\op{std}}
    \to
    \{\op{patchable},\op{trunc},\op{crowd3},\op{crowd4}\}\\
    \op{sum\_typ}(8):& F\in (H/f)_{8,\op{patchable}}
    \to
    \{\text{list connected sumes}\ldots,\op{agg51}\}.
    \end{array}
    $$
\end{definition}


\begin{definition}[upright~quarter,~upq] The function $\op{upq}$ detects upright
quarters in a hypermap.
    $$\op{upq}:\op{dart}(H) \to \bool$$
    $$\op{upq}(\beta) =  (\card(\op{face}\beta)=3)\wedge
      \forall\alpha\in \op{face}(\beta).\
         \op{qre}(\alpha)\wedge
         (\op{qrn}(\alpha)\Leftrightarrow
         (\beta\ne\alpha))$$
\end{definition}

\begin{definition}[flat~quarter,~flatq]
The function $\op{flatq}$ detects flat quarters in a hypermap.
     $$\op{flatq}:H \to \bool$$
    $$\op{flatq}(\beta) = (\card(\op{face}\beta)=3)\wedge
       \forall\alpha\in \op{face}(\beta).\
         \op{qrn}(\alpha)\wedge
         (\op{qre}(\alpha) \Leftrightarrow
         (\beta\ne\alpha)).$$
\end{definition}
(Note: the dart $\beta$ on which $\op{flatq}$ is true is the one
for which $\op{edge}(\beta)$ is not quasi-regular; that is, the
diagonal.)


\begin{definition}[inquad]
The function
    $$\op{inquad}:(H/n)\to \bool$$
gives whether an upright diagonal lies on upright quarter of type
$\x{(4,0)}$ inside a quad cluster.
    $$\op{inquad}(N) =
    (\card(N)=4) \wedge (\forall \alpha\in N.\
        \op{upq}(\alpha))
    $$
\end{definition}

\begin{definition}[$\op{quad\_diag}$]
The function
    $$\op{quad\_diag}:\op{dart}(H)_{\neg\op{qre}} \to \bool$$
tells whether an edge that is not quasi-regular is the diagonal of
a quadrilateral.  The function $\op{quad\_diag}$ is constant on
orbits of $e$, so it descends to a function on a subset of $H/e$.
$$
    \op{quad\_diag}(\alpha) = \op{flatq}(\alpha)\wedge
    \op{flatq}(e\alpha).
$$
\end{definition}

\begin{definition}[$\op{hex\_loop\_diag}$]
The flag component
    $$
    \op{hex\_loop\_diag}:\op{dart}(H)_{\neg\op{qre}}\to\bool
    $$
tells whether an edge that is not quasi-regular is the diagonal of
a hexagon, along a slice in a loop.  The function
$\op{hex\_loop\_diag}$ is constant on orbits of $e$, so it
descends to a function on a subset of $H/e$.
\end{definition}


\begin{definition}[$\op{fqex}$]
Let $\op{fqex}$ be the predicate that picks out a dart in flat
quarters that are not part of a quadrilateral pair.
\begin{equation}
    \begin{array}{llll}
    \op{fqex}(\beta) &= \exists F\in(H/f)_{3,\neg\op{std}}.\ \op{flatq}(\beta) &\wedge\\
        \beta\in F &\wedge
        \ \neg\op{quad\_diag}(\beta).
    \end{array}
\end{equation}
\end{definition}




\subsection{bifurcation}



\begin{definition}[bifurcation,~bif]
We define a bifurcation relation $\op{bif}(\rho,\rho',\rho'')$ on
triples of  hypermap systems.  Let $\rho=(H,\op{fl},S)$, and
similarly for $\rho'$ and $\rho''$ with appropriately primed
notation.  We say that $\rho$ {\it bifurcates} into $\rho'$ and
$\rho''$, and write $\op{bif}(\rho,\rho',\rho'')$ if there exists
a formal disjunction $A\in S$, and formal disjunctions $A',A''$
such that
    $$
    \begin{array}{rlll}
    H &= H' &= H''\\
    \op{fl} &= \op{fl}' &= \op{fl}''\\
        A &= A'&\cup\ A''\\
    S'\setminus  \{A'\} &\subset S &\cup\ \op{single}(S)\\
    S''\setminus  \{A''\} &\subset S&\cup\ \op{single}(S)\\
    \end{array}
    $$
\end{definition}
The simplest way to bifurcate is to break a disjunction into two
disjuncts, putting one disjunct into each of the two branches of
the bifurcation.



\subsection{weak infeasibility}
\label{sec:weak}

We define a notion of infeasibility of  hypermap systems that is
broad enough to encompass branch and bound.  We describe the
predicate $\op{weak\_infeas}$ on the set of hypermap systems.

We define a tree structure that keeps track of the
branch-and-bound operations used in the search for infeasible
hypermap systems.  We define a type of an $\op{tree}$ by
    $$
    \begin{array}{lll}
    |\ \op{LEAF}&\\
    |\ \op{DATA} &\text{of } \alpha * \op{tree}\\
    |\ \op{NODE} &\text{of } \op{tree} * \op{tree}\\
    \end{array}
    $$
%
We define a predicate $\op{is\_data}$ on trees which is true
exactly when it has the form $\op{DATA}(\cdot,\cdot)$. We define a
relation $\op{head}:(\op{tree},\alpha)\to \bool$ by
    $$
    \begin{array}{lll}
    |\ \op{DATA}(\rho,s),\rho &\to \true\\
    |\  \op{\_} &\to \false
    \end{array}
    $$
Section~\ref{sec:refine} defines a binary relation
$\op{ref}(\rho,\rho')$ (called refinement) on hypermap systems. We
define a recursive predicate $\op{wft}$ (well-formed trees) on the
trees by
    $$
    \begin{array}{lll}
    |\ \op{LEAF} &\to\true\\
    |\ \op{NODE}(s,t) &\to \op{wft}(s) \wedge \op{wft}(t)
    \wedge \op{is\_data}(s) \wedge \op{is\_data}(t)\\
    |\ \op{DATA}(\rho,s) &\to \op{wft}(s) \wedge\\
        &\quad \text{case}\ s\ \text{of}\\
        &\qquad|\op{LEAF}\to\true\\
        &\qquad|\op{DATA}(\rho',s')\to \ \op{ref}(\rho,\rho')\\
        &\qquad|\op{NODE}(\op{DATA}(\rho',s'),\op{DATA}(\rho'',s''))\to
        \ \op{bif}(\rho,\rho',\rho'')\\
        &\qquad|\op{\_} \to \false\\
    \end{array}
    $$
In brief, in a well-formed tree, a child must be obtained from the
parent by bifurcating or refinement.
%
By abuse of notation, we write $\op{INFEAS}(H,\op{fl},S)$ for
$\op{INFEAS}(S)$.

We define a predicate $\op{infeas}$ (infeasible trees) on trees by
    $$
    \begin{array}{lll}
    |\ \op{LEAF} &\to \false\\
    |\ \op{DATA}(\rho,\op{LEAF}) &\to \op{INFEAS}(\rho)\\
    |\ \op{NODE}(s,s') &\to \op{infeas}(s) \wedge \op{infeas}(s')\\
    |\ \op{DATA}(\rho,s) &\to \op{infeas}(s)\\
    |\ \op{\_} &\to \false\\
    \end{array}
    $$
In brief, the data closest to the leaves must be infeasible
systems of equations.

\begin{definition}[$\op{weak\_infeas}$]
    Let $\op{weak\_infeas}$ be the predicate on hypermap systems given by
        $$
        \op{weak\_infeas}(\rho) = \exists t.\
        \op{head}(t,\rho) \wedge \op{wft}(t) \wedge
        \op{infeas}(t).
        $$
\end{definition}

In the actual linear programming, the trees are not used directly.
Rather, all the needed properties of the function
$\op{weak\_infeas}$ are given by the following three lemmas.

\begin{lemma}\guid{ZGVVZXL}
If $\op{bif}(\rho,\rho',\rho'',L)$, $\op{weak\_infeas}(\rho')$,
and
    $\op{weak\_infeas}(\rho'')$ then $\op{weak\_infeas}(\rho)$.
\end{lemma}

\begin{proof} If $t'$ and $t''$ are the trees for $\rho'$ and
$\rho''$, then use the tree $\op{DATA}(\rho,\op{NODE}(t',t''))$
for $\rho$.
\end{proof}

\begin{lemma}\guid{SPHYLTX}
If $\op{INFEAS}(\rho)$, then $\op{weak\_infeas}(\rho)$.
\end{lemma}

\begin{proof} Use the tree $\op{DATA}(\rho,\op{LEAF})$.
\end{proof}

\begin{lemma}\guid{AVPPMYF}
If $\op{ref}(\rho,\rho')$ and $\op{weak\_infeas}(\rho')$, then
  $\op{weak\_infeas}(\rho)$.
\end{lemma}

\begin{proof} If $t'$ is a tree for $\rho'$, then for $\rho$ use
    $$
    \op{DATA}(\rho,t')
    $$
\end{proof}


\subsection{variable indexing}
\label{sec:var-index}

We describe the finite set $I = I(H,\op{fl})$ of variable indices.
Extend notation by setting $I(\rho) = I(H,\op{fl})$, for $\rho =
(H,\op{fl},S)$.  If $e_i\in \ring{R}^I$ is a standard basis
vector, we let $\iota(e_i)=i\in I$ be the corresponding index.  We
generally give names to the basis vectors, and refer to the
corresponding indices by means of the function $\iota$.





\begin{definition}[indexing~set,~$I$]
Let $D$ be the set of darts of hypermap $H$.  The set $I$ (for
$(H,\op{fl})$ is defined as the disjoint union of the following
indexing sets:\FIXX{Finish List}
    $$\begin{array}{llll}
        \text{index} &\text{basis vector name}&\text{namesake?}
        \\
        %
        \alpha\in D, &\optt{azim}(\alpha), &\text{yes}
        \\
        %
        \alpha\in D, &\optt{yn} (\alpha)
        \\
        %%
        \alpha\in D,
        &\optt{ye} (\alpha),
        \\
        %%
        F\in H/f,
        &\optt{sol} (F), &\text{yes}
        \\
        %%
        F\in H/f
        &\optt{sc} (F)
        \\
        %%
        F\in H/f
        &{\optt{tau\_sc}} (F)
        &\text{ }\\
        %%
        F\in (H/f)_{3,flq}
        &\optt{sigmahat} (F)
        &  \text{yes}\\
        %%
        F\in (H/f)_{3,flq}
        &\optt{tauhat} (F)
        &\text{yes}\\
        %%
        F\in (H/f)_{3,\neg\op{std}}
        &\optt{eta}(F)
        &\text{yes}\\
        %%
        \alpha\in H
        &\optt{eta}'(\alpha)
        &\text{yes}\\
        %%
        F\in (H/f)_{3,std}
        &\optt{sigma\_qrtet\_x} (F)
        &\text{yes} \\
        %%
        F\in (H/f)_{3,std}
        &\optt{tau\_sigma\_x} (F)
        &\text{yes} \\
        %%
        F\in (H/f)_{\ge4,\neg\op{std}}
        &\optt{vor0}(F)
        &\text{yes}\\
        %%
        F\in (H/f)_{\ge4,\neg\op{std}}
        &\optt{tau0}(F)
        &\text{yes}\\
        %%
        \alpha\in D
        &\optt{Adih}(\alpha)
        &\text{yes}\\
        %%
        \alpha\in D
        &\optt{quo}(\alpha)
        &\text{yes}\\
        %%
        \alpha\in D
        &\optt{quob}(\alpha)
        &\\
        %%
        \alpha \in \op{flatq}
        &\optt{mu}(\alpha)
        &\text{yes}\\
        %%
        \alpha\in \op{upq}
        &\optt{nu}(\alpha)
        &\text{yes}\\
        %%
    \end{array}
    $$
\end{definition}

\subsection{interpretation}

There are a number of interpretations of variables that have the
general form
    $$\op{interp}(\iota\optt{mango})\ v\mapsto
    \op{mango}(u_0\cdot v,u_2\cdot v,\ldots,u_k\cdot v)$$
for some nonlinear function $\op{mango}$, and vectors $u_i\in
\ring{R}^I$.  We typeset the name ($\optt{mango}$) of the basis
vector in $\optt{tt}$ font and typeset name of the nonlinear
interpreting function $\op{mango}$ as an operator. Greek symbols
are expanded as text to make the correspondence with function
names in the file
    {\it definitions\_kepler.ml} explicit. (Thus, the nonlinear
    function $\sigma$ appears
    as $\optt{sigma\_qrtet\_x}$ and so forth.)
When the correspondence between a basis vector and nonlinear
interpreting function occurs on a name-by-name basis in this way,
we will omit details.  We will simply say that it is a {\it
namesake} interpretation.

In namesake interpretations, the vectors $u_i$ are general tightly
related to the index $\iota\optt{func}\in I$.  Generally, if
$\alpha$ is a dart, $F=\op{face}(\alpha)$,  and $\card(F)=3$, and
if $\optt{func}(\alpha)$ has nonlinear namesake
$\op{func}(y_1\cdot v,\ldots,v_{i_6})$, then $u_i$ follow the
standard convention:
    $$
    \begin{array}{llll}
    u_i &= \optt{yn}_i(f^i\alpha), &i &=0,1,2\\
    u_i &= \optt{ye}_i(f\alpha), &i &=3,4,5
    \end{array}
    $$
This indexing convention is followed with the nonlinear functions:
    $$
    \optt{gamma},\ \optt{nu},\ \optt{mu},\ \text{list.}
    %
    $$
(N.B. This list is incomplete.)\FIXX{Complete and correct.}


\subsection{penalty}
\label{sec:pc}


We define a function $\xi:(H/f)_{\neg\op{std}}\to\ring{R}$ as
follows.

If $\op{qrn}(\alpha)$ for all $\alpha\in F$, and $n=\card(F)\ge
4$, then we define $\xi$ as follows.  Let $k$ be the number of
edges of $F$ such that $\neg\op{qre}(\alpha)$.  Set
    $$
    \begin{array}{lll}
    \xi(F) & \y{(n,k)}\\
    6 \xi_\Gamma & \y{(8,0)}\\
    6 \xi_\Gamma & \y{(7,1)}\\
    4 \xi_\Gamma + 2 \xi_V & \y{(6,2)}\\
    2 \xi_\Gamma + 4 \xi_V & \y{(5,3)}\\
    0 & \y{(4,4)}\\
    6 \xi_\Gamma & \y{(7,0)}\\
    5 \xi_\Gamma & \y{(6,1)}\\
    3\xi_\Gamma + 2 \xi_V & \y{(5,2)}\\
    \xi_\Gamma + 4\xi_V & \y{(4,3)}\\
    2(0.008) & \y{(6,0)} \\
    0.008 & \y{(5,1)} \\
    0 & \y{(4,2)} \\
    0.008 & \y{(5,0)} \\
    0 & \y{(4,1)} \\
    0 & \y{(4,0)}
    \end{array}
    $$
%
\begin{itemize}
\item Set
    $\xi(F) = 3\xi_\Gamma$, if $F$ has the properties $\card(F)=5$
    and
    $$\forall\alpha\in F.\
    \op{qre}(\alpha)$$
    %
    $$
    \exists\beta\in F.\forall\alpha\in F.\
    \op{qrn}(\alpha)\Leftrightarrow
(\beta\ne\alpha).$$
%
%
\item Set $\xi(F) = \xi_\Gamma + 2\xi_V$, if $F$ has the
properties
    $\card(F)=4$ and
    $$\exists\beta\in F.\forall\alpha\in F.\
    \op{qre}(\alpha)\Leftrightarrow (\beta\ne\alpha)
    $$
    $$\exists\beta\in F.\ \op{qrn}(\alpha)\Leftrightarrow
    (\beta\ne\alpha).$$
\item Set
    $$
    \begin{array}{lll}
    \xi(F) & \card(F)\\
    3 \xi_\Gamma & 4\\
     \xi_\Gamma + 2\xi_V & 3\\
    \end{array}
    $$
if $F$ has the properties
    $$
    \exists\alpha\in F.\ \op{hex\_loop\_diag}(\alpha)
    $$
    $$
    \forall\alpha\in F.\ \op{qrn}(\alpha).
    $$
\item In all other cases, set $\xi(F)=0$.
\end{itemize}


\section{Inequalities}
%
\label{sec:basic}

This section gives a set $S=\op{basic}(H,\op{fl})$ of formal
disjunctions on the indexing set $I=I(H,\op{fl})$. Elements of
this set are call {\it basic formulas}.

The following are basic formulas.  The first formal inequality is
key.

\begin{equation}
    \sum_{F\in (H/f)} \optt{sc}(F) \sge 8\,\pt.
    \label{eqn:8pt}
\end{equation}

The next family of formal disjunctions allows quite arbitrary
bifurcations. For every $f\in\ring{R}^I$ and $c\in \ring{R}$, we
have basic formulas:
\begin{equation}
    (f \sle c) \vee (f \sge c).
\end{equation}

Some of the variables are constant on orbits:

\begin{equation}
    \begin{array}{lll}
  \optt{yn}(\alpha) &\seq \optt{yn}(n\alpha),\\
  \optt{ye}(\alpha) &\seq \optt{ye}(e\alpha),\\
  \optt{eta}'(\alpha) &\seq \optt{eta}'(e\alpha),\\
  \end{array}
\end{equation}


\begin{equation}
\optt{tau\_sc}(F) = \optt{sol}(F)\,\zeta\,\pt - \optt{sc}(F).
\end{equation}

Many of the basic formulas hold subject to various restrictions on
the indexing set.  These restrictions are indicated in a line
following the basic formula.  For example, the following basic
formula holds under the restriction that $F$ is a non-standard
face of cardinality at least four:
\begin{equation}
    \label{eqn:xi}
    \begin{array}{lll}
    \optt{sc}(F) \sle \optt{vor0} + \xi(F),\\
    &F\in (H/f)_{\ge4,\neg\op{std}}
    \end{array}
\end{equation}

\begin{equation}  % Type potA along a hex loop.
    \begin{array}{lll}
    \optt{sc}(F) &\sle \optt{vor0} + \xi(F),\\
    &&F\in (H/f)_{3,\exists\alpha\in F.\ \op{hex\_loop\_diag}(\alpha)}
    \end{array}
\end{equation}

\begin{equation}  % DCG 25.6.1 page 250, one flat quarter.
    \begin{array}{lll}
    \optt{sc}(F) & \sle (s_{1+\card(F)} - Z\y{(3,1)}) + \xi_\Gamma + 2 \xi_V,\\
    &&F\in (H/f)_{\ge6}\\
    &&\exists!\alpha\in F.\ \neg\op{qre}(\alpha)
    \end{array}
\end{equation}

\begin{equation}  % DCG 25.6.1 page 251, one flat quarter.
    \begin{array}{llll}
    \optt{yn}(\beta)&\sge 2.2&\vee\\
    \optt{ye}(\alpha)&\sge 2.7&\vee\\
    \optt{sc}(F) & \sle (s_{1+\card(F)} - Z\y{(3,1)}),\\
    &&&\neg\op{qre}(\alpha),\\
    &&&\beta=f^2e\alpha,\\
    &&&\alpha\in F\in (H/f)_{\ge6}\\
    &&&\exists!\alpha\in F.\ \neg\op{qre}(\alpha)
    \end{array}
\end{equation}

%% Two flat quarters, DCG 25.6.2, page 252.
\begin{equation}
    \begin{array}{lllll}
    \optt{yn}(\beta')&\sge 2.2&&\vee\\
    \optt{yn}(\beta')&\sge 2.2&&\vee\\
    \optt{ye}(\alpha')&\sge 2.7&&\vee\\
        \optt{ye}(\alpha')&\sge 2.7&&\vee\\
    (\optt{sc}(F) & \sle (s_{2+\card(F)} - 2 Z\y{(3,1)})&\wedge \\
    \optt{tau\_sc}(F) & \sge (t_{2+\card(F)} - 2 D\y{(3,1)})&\wedge \\
    \optt{sc}(F) & \sle \optt{vor0} + 3\xi_\Gamma), \\
    &&&&\neg\op{qre}(\alpha),\\
    &&&&\beta=f^2e\alpha,\ \beta'=f^2e\alpha',\\
    &&&&\alpha\ne\alpha'\in F\in (H/f)_{\ge5}\\
    &&&&\forall\gamma\in F.\ \op{qre}(\gamma)\Leftrightarrow (\gamma\ne\alpha,\alpha')\\
    \end{array}
\end{equation}

%% DCG 25.6.2, page 252, loop of type \y{(n,k)} = \y{(4,2)}, or not.
%% Constants from table 25.1.
\begin{equation}
    \begin{array}{lllll}
    \optt{sc}(F) + \optt{sc}(\op{face}(\beta)) +
    \optt{sc}(\op{face}(\beta')) &\sle -0.1999&\wedge \\
    \optt{tau\_sc}(F) + \optt{tau\_sc}(\op{face}(\beta)) +
    \optt{tau\_sc}(\op{face}(\beta')) &\sge 0.5309 &\wedge \\
    \optt{sc}(F) & \sle s_{2+\card(F)} - 2 Z\y{(3,1)} + 2(\xi_\Gamma + 2\xi_V), \\
    &&\neg\op{qre}(\alpha),\\
    &&\beta=f^2e\alpha,\ \beta'=f^2e\alpha',\\
    &&\alpha\ne\alpha'\in F\in (H/f)_{\ge5}\\
    &&\forall\gamma\in F.\ \op{qre}(\gamma)\Leftrightarrow (\gamma\ne\alpha,\alpha')\\
    \end{array}
\end{equation}



% Page 239 DCG
Let $(H,\op{fl},S)\in\op{HS}$.   Let $F\in H/f$ and $N\in H/n$. We
have the following formal equalities:
\begin{equation}
    \begin{array}{lll}
    \tau_{sc}(F) &\seq \sol(F)\zeta\pt - \op{sc}(F)\\
    %%
    \sum_{\alpha\in N}\op{azim}(\alpha) &\seq
    2\pi\\
    %%
    \op{sol}(F) &\seq \sum_{\alpha\in F}\op{azim}(\alpha) - (\card(F)
    - 2)\pi\\
    %%
    \end{array}
    \label{eqn:tau-sc}
\end{equation}


\begin{remark}  For every  primary tame hypermap system $(H,\op{fl},S)$, the
hypermap $H$ is connected, plane, and planar.  Thus, the Euler
relation holds in the form:
    $$
    \card(H/f) - \card(H/e) + \card(H/n) = 2.
    $$
From this and the previous formal inequality, it follows that
    \begin{equation}
    \sum_{F\in H/f} \optt{sol}(F) \seq 4\pi.
    \end{equation}
As a linear consequence of other equations program, it is
redundant.  Thus, it does not need to be inserted explicitly into
the linear programs.
\end{remark}



%%%%%%%%%%%%%%%%%%%%%%%%%%%%%%%%%%%%%%%%%%%%%%%%%%%%%%%%%%%%%%%%%
\subsection{variable bound}
%% Page 239 DCG.
\begin{equation}
    \begin{array}{lll}
        \optt{yn}(\alpha) &\sge0\\
        \optt{yn}(\alpha) &\sle 2\sqrt2\\
        \optt{yn}(\alpha) &\sle 2t_0 &\text{if }
        \op{qrn}(\alpha)\\
        \optt{ye}(\alpha)&\sge0\\
        \optt{ye}(\alpha)&\sle 2\sqrt2\\
        \optt{ye}(\alpha)&\sle 2t_0&\text{if }
        \op{qre}(\alpha)\\
        \optt{azim}(F) &\sge 0\\
        \optt{azim}(F) &\sle 2\pi\\
        \optt{sol}(F) &\sge 0\\
        \optt{sol}(F) &\sle 4\pi\\
    \end{array}
\end{equation}





\begin{remark}
The way the infeasibility certificates are set up in
Inequality~\ref{eqn:lpsys}, it is necessary to have lower and
upper bounds on all the variables, not just those shown here.\FIXX{Add bounds}
\end{remark}




%%%%%%%%%%%%%%%%%%%%%%%%%%%%%%%%%%%%%%%%%%%%%%%%%%%%%%%%%%%%%%%%%
\subsection{triangle}

%
\begin{equation}
    \optt{sc}(F) = \optt{sigma\_qrtet\_x}(F),\quad\text{if } F\in
    (H/f)_{3,\op{std}}.
\end{equation}
%
\begin{equation}
        \optt{tau\_sc}(F) =
        \optt{tau\_sigma\_x}(F),\quad\text{if } F\in
        (H/f)_{3,\op{std}}.
\end{equation}
%
\begin{equation}
    \begin{array}{lll}
    \optt{sc}(\op{face}(\alpha)) &\sle \optt{sigmahat}(\alpha),\\
    \optt{tau\_sc}(\op{face}(\alpha)) &\sge \optt{tauhat}(\alpha),\\
    &&\alpha\in \op{fqex}
    \end{array}
\end{equation}
%
\begin{equation}
    \begin{array}{lll}
    \optt{sc}(\op{face}(\alpha)) &\sle \optt{mu}(\alpha),\\
    \optt{tau\_sc}(\op{face}(\alpha)) &\sge \optt{tau\_mu}(\alpha),\\
    &&\alpha\in \op{quad\_diag}
    \end{array}
\end{equation}



%%25.7 page 252, Branching on Upright Quarters (triangles with a potC).
\FIXX{Put in as a condition that the long flat edge is not a hex-loop-diag.}



\subsection{quadrilateral}

\begin{equation}
 \optt{sc}(F) = \optt{sigma\_quad}(F),\quad\text{if } F\in
    (H/f)_{4,\op{std}}.
\end{equation}
%
\begin{equation}
 \optt{tau\_sc}(F) = \optt{tau\_quad}(F),\quad\text{if } F\in
    (H/f)_{4,\op{std}}.
\end{equation}
%

There is a list of nonlinear inequalities that appears in the
database for quadrilaterals.  The first of these reads



\begin{equation}
\label{eqn:exdih}
\begin{array}{lll} \op{let}\ \op{J\_310151857} &= (`\forall\ v_0\ v_1\ v_2\ v_3\ v_4.\
  (\op{is\_quad\_cluster\_v}\ v_0\ v_1\ v_2\ v_3\ v_4\ \ \Rightarrow\\
        &\ \ ((\op{sigma\_quad\_approx1}\ v_0\ v_1\ v_2\ v_3\ v_4) &<
        -5.7906\\
                + &&\ 4.56766*(\op{dih\_or\_v}\ v_0\ v_1\ v_2\ v_4)))`);;
\end{array}
\end{equation}

Let $C$ be a constant less than the right hand sides of all the
inequalities.  The function $\op{dih\_or\_v}$ takes non-negative
values.  So for instance, from Inequality~\ref{eqn:exdih}, we have
the constraint
    $$C < -5.7906.$$
Then let $\op{squad}(\op{sol},\op{azim}_1,\op{azim}_2)$ be the
maximum of $\op{sigma\_quad\_approx1}$ subject to the constraints
that the simplex formed by $v_0,v_1,v_2,v_3,v_4$ has solid angle
$\op{sol}$, and azimuth angles $\op{azim}_i$.  Or let
$\op{squad}=C$, if no such simplices exist. More precisely, set
    $$\begin{array}{rll}
    \op{squad}(s,d_1,d_2) =
        \sup \{\sigma&\mid \exists v_0\cdots v_4.\ \\
           \sigma &= \op{sig\_quad\_approx1}\ v_0\cdots v_4 \\
           &\quad \op{is\_quad\_cluster}\ v_0\cdots v_4\\
           s &= \op{solid\_v}\ v_0\cdots v_4\\
           d_1 &= \op{azim}\\
           d_2 &= \op{azim}
           \}\cup \{C\}
    \end{array}
    $$




With this definition, we can rewrite the quadrilateral
inequalities and then write them as formal inequalities that take
the following shape:
    \begin{equation}
    \label{eqn:quad}
    \op{QU}(\optt{squad},\optt{sol},\optt{azim}_1,\optt{azim}_2,i) :
    (\optt{squad} \sle a_i \optt{azim}_1 + b_i \optt{azim}_2 + c_i \optt{sol}
    + d_i),
    \end{equation}
for constants $a_i,b_i,c_i,d_i$, $i=1,\ldots,N$.  The formal
inequalities $\op{QU}$ (\ref{eqn:quad}) hold for all indices $i$
and the following parameter values:


\begin{equation}
  \begin{array}{rll}
    \optt{squad} &= \optt{sc}(F),\\
    \op{sol} &= \optt{sol}(F),\\
    \op{azim}_1 &= \optt{azim}(\alpha),\\
    \op{azim}_2 &= \optt{azim}(f\alpha),\\
    &&\alpha\in F\in (H/f)_{4,\op{std}}
   \end{array}
\end{equation}

\begin{equation}
\begin{array}{rll}
    \optt{squad} &=\sum_{i=0}^3\optt{sc}(\op{face}(n^i\alpha)),\\
    \op{sol} &=
          \sum_{i=0}^3\optt{sol}(\op{face}(n^i\alpha)),\\
    \op{azim}_1 &=\optt{azim}(f\alpha)+\optt{azim}(n f\alpha),\\
     \op{azim}_2 &= \optt{azim}(f^{-1}\alpha)+\optt{azim}(n^{-1}f^{-1}\alpha),
         \\
    &&\alpha\in N\in (H/n)_{\op{inquad}},\\
\end{array}
\end{equation}

\begin{equation}
\begin{array}{rll}
    \optt{squad} &=\optt{sc}(\op{face}(\alpha))+\optt{sc}(\op{face}(\beta)),\\
    \op{sol} &= \optt{sol}(\op{face}(\alpha))+\optt{sol}(\op{face}(\beta)),\\
    \op{azim}_1 &=\optt{azim}(\alpha)+\optt{azim}(f\beta),\\
     \op{azim}_2 &= \optt{azim}(f^{-1}\beta),
         \\
    &&\alpha,\beta \in (H)_{\op{quad\_diag}},\ \beta=e\alpha.\\
\end{array}
\end{equation}

\begin{equation}
\begin{array}{rll}
    \optt{squad} &=\optt{sc}(\op{face}(\alpha))+\optt{sc}(\op{face}(\beta)),\\
    \op{sol} &= \optt{sol}(\op{face}(\alpha))+\optt{sol}(\op{face}(\beta)),\\
    \op{azim}_1 &=\optt{azim}(f^{-1}\beta),\\
     \op{azim}_2 &= \optt{azim}(\beta)+\optt{azim}(f\alpha),
         \\
    &\alpha,\beta \in (H)_{\op{quad\_diag}},\ \beta=e\alpha\\
\end{array}
\end{equation}

\begin{remark}  There is a small conflict between nonlinear and
linear variables here that is easy to resolve.  The variables
$\optt{azim}$ are nonlinear and are interpreted as the azimuth
function.  However, in the quadrilateral inequalities give above,
they are independent variables to the function $\optt{squad}$, and
should be treated as linear variables.  To overcome these
conflicting constraints, we introduce two copies of each variable,
say $\optt{azim}$ and $\optt{azim}'$, related by the equation
    $$
    \optt{azim}(\alpha) = \optt{azim}(\alpha)'.
    $$
We then use the nonlinear copy everywhere but in the quadrilateral
inequalities, we switch to the linear copy.
\end{remark}

\subsubsection{specialized  constraint}



\begin{equation}
    \begin{array}{lll}
        \optt{sc}(F) &\sle s_8 + \pt\\
        \optt{tau\_sc}(F) &\sge t_8\\
        &&F \in (H/f)_{5,\op{aggA}}.
    \end{array}
\end{equation}

\begin{equation}
    \begin{array}{lll}
        \optt{sc}(F) &\sle s_7\\
        \optt{tau\_sc}(F) &\sge t_7\\
        &&F \in (H/f)_{5,\op{aggB}}.
    \end{array}
\end{equation}

\begin{equation}
    \begin{array}{lllll}
        \optt{yn}(\alpha) &\sge 2.168,&\text{some }\alpha\in F
        &\vee\\
        \optt{azim}(\alpha) &\sge 2.89,&\text{some }\alpha\in
        F&\vee\\
        \optt{ye}(\alpha)+\optt{ye}(f\alpha)&\sge
        4.804,&\text{some }\alpha\in F&\vee\\
        \sum_{\alpha\in F} \optt{ye}(\alpha) &\sge
        11.4707&&\vee\\
        \optt{sc}(F) &\sle -0.2345&&\vee\\
        \optt{tau\_sc}(F) &\sge 0.6079,\\
        &&&&F\in (H/f)_{5,\op{aggB}}.
    \end{array}
\end{equation}

\begin{equation}
    \begin{array}{llll}
        \optt{sc}(F) &\sle -0.4339&\vee\\
        \optt{sc}(F) &\sge 0.5606\\
        &&&F\in (H/f)_{n,\op{crowd3}},\quad n\ge 5.
    \end{array}
\end{equation}

\begin{equation}
    \begin{array}{llll}
        \optt{sc}(F) &\sle -0.25\\
        \optt{tau\_sc}(F) &\sge 0.4\\
        &&F\in (H/f)_{n,\op{crowd4}},\quad n\ge 6.
    \end{array}
\end{equation}

\begin{equation}
    \begin{array}{lllll}
        \optt{sc}(F) &\sle -0.31547&&\vee\\
        \optt{yn}(\alpha) + \optt{yn}(f^4\alpha) &\sge 4.6 &\text{some
        }\alpha\in F,\\
        &&&F\in (H/f)_{n,\op{crowd4}},\quad n\ge 6.
    \end{array}
\end{equation}

\begin{equation}
    \begin{array}{llll}
        \optt{sc}(F) &\sle s_8\\
        \optt{tau\_sc}(F) &\sge t_8\\
        &&F\in (H/f)_{6,\op{agg}}.
    \end{array}
\end{equation}

\begin{equation}
    \begin{array}{llll}
        \optt{sc}(F) &\sle -0.37595\\
        \optt{tau\_sc}(F) &\sge 0.65995\\
        &&F\in (H/f)_{6,\op{loop51}}.
    \end{array}
\end{equation}





\begin{equation}
    \begin{array}{lllll}
        \optt{yn}(\alpha) &\sge 2.14 &\text{some }\alpha\in F&\vee\\
        \optt{ye}(\alpha) &\sge 2.77 &\text{some }\alpha\in
        F&\vee\\
        (\optt{sc}(F) &\sle \optt{gamma}(F) &\wedge\\
         \optt{yn}(\alpha) &\sle 2.14 \text{ all }\alpha\in
         F&\wedge\\
        \optt{eta}(F) &\sle \sqrt2) &&\vee\\
        (\optt{ye}(\alpha) &\sle 2.7 &\wedge\\
        \optt{yn}(\alpha) &\sle 2.14 \text{ all }\alpha\in
         F&\wedge\\
        \optt{eta}(F) &\sge \sqrt2 &\wedge\\
        \optt{sc}(F) &\sle \optt{svor0}(F)) &&\vee\\
        (\optt{ye}(\alpha) &\sge 2.7 &\wedge\\
        \optt{yn}(\alpha) &\sle 2.14 \text{ all }\alpha\in
         F&\wedge\\
        \optt{sc}(F) &\sle \optt{svor0}(F)),\\
        &&&&\alpha\in F\in (H/f)_{3,\op{fqex}}.
    \end{array}
\end{equation}

 Let $\op{potA}$ be the predicate on $(H/f)_{3,\neg\op{std}}$
that picks out potentially type A.  That is, $\op{qrn}(\alpha)$
holds for all three darts, $\op{qre}(\alpha)$ holds for exactly
one of the three darts, and no dart is in a hex diagonal.

\begin{equation}
    \begin{array}{lll}
    \op{potA}(F) &= \exists\beta\in F.\ \forall\alpha\in F. \\
        &\op{qrn}(\alpha)\\
        &\op{qre}(\alpha) \Leftrightarrow (\alpha=\beta)\\
        &\neg\op{hex\_loop\_diag}(\alpha).
    \end{array}
\end{equation}

\begin{equation}
 \begin{array}{lllll}
    \optt{yn}(\alpha) &\sge 2.77 &&\vee\\
      \optt{yn}(\beta) &\sge 2.77 &&\vee\\
    (\optt{yn}(\alpha) &\sle 2.77 &\wedge\\
      \optt{yn}(\beta) &\sle 2.77 &\wedge\\
      \optt{sc}(F) &\sle \optt{svoran}) &&\vee\\
   (\optt{yn}(\alpha) &\sle 2.77 &\wedge\\
      \optt{yn}(\beta) &\sle 2.77 &\wedge\\
      \optt{eta}(F) &\sge \sqrt2) &&\vee\\
     &&&&\alpha\ne\beta,\gamma\in F\in (H/f)_{3,potA}\\
     &&&&\neg\op{qre}(\alpha),\neg\op{qre}(\beta),\op{qre}(\gamma).
 \end{array}
\end{equation}

Let $\op{potC}$ be the set of triangles that are potentially of
type $C$. That is, for $F\in (H/f)_{3,\neg\op{std}}$:
\begin{equation}
    \begin{array}{lll}
    \op{potC}(\beta,F) &= \beta\in F \wedge\ \forall\alpha\in F. \\
        &\op{qrn}(\alpha)\Leftrightarrow (\alpha\ne\beta)&\wedge \\
        &\op{qre}(\alpha) \Leftrightarrow (\alpha\ne f\beta).
    \end{array}
\end{equation}













\section{Final Specs}


\subsection{primary tame hypermap system}
\label{sec:PTHS}

This subsection gives a specification of primary tame hypermap
systems ($\op{PTHS}$).

Define $\op{type}(N):H/n\to\ring{N}^3$ by $\op{type}(N) = (p,q,r)$
if there are $p$ triangles, $q$ quadrilaterals, and $r$
exceptional faces that meet the node $N$.  (An exceptional face is
defined to be one with at least $5$ darts.)


\begin{definition}[primary~flag]
Let $H$ be a hypermap.  A flag $\op{fl} = (\op{qrn},\ldots)$ is a
{\it primary flag} for $H$ if it satisfies the following
conditions:
    \begin{enumerate}
        \item $\op{qrn}(\alpha)$, for every dart $\alpha$.
        \item $\op{qre}(\alpha)$, for every dart $\alpha$.
        \item $\op{typ}(5)\ne\op{notpri}$.  (This is a double negation: the flag is
        {\it not not\/} primary.)
        \item $\exists\alpha\in F.\ \op{type}(\op{node}(\alpha)) \ne
    (3,0,1),\quad F\in (F/f)_{5,\op{aggA}}$.
        \item Except for the preceding two restrictions on $\op{typ}(n)$, the flags
        $\op{typ}(n)$ and $\op{sum\_typ}(n)$ are arbitrary, for $n=4,\ldots,8$.
    \end{enumerate}
\end{definition}

\begin{remark} By the first two properties, every face is
standard.  Furthermore, the domains of $\op{quad\_diag}$ and
$\op{hex\_loop\_diag}$ are empty, and so they leave nothing to
specify.
\end{remark}

\begin{definition}[primary~tame~hypermap~system]
A triple $\rho=(H,\op{fl},S)$ is a primary tame hypermap system if
    \begin{enumerate}
    \item $H$ is a tame hypermap in the archive.
    \item $\op{fl}$ is a primary flag for $H$.
    \item $S=\op{basic}(H,\op{fl})$ is the set of basic formulas for $I=I(H,\op{fl})$.
    \end{enumerate}
\end{definition}


\subsection{refinement -- lookup}
%
\label{sec:refine}

The refinement predicate $\op{ref}(\rho,\rho')$ is a disjunction
    \begin{equation}
    \op{ref}(\rho,\rho') =
    \op{lookup\_database}(\rho,\rho')\vee
    \op{sum\_hyper}(\rho,\rho').
    \end{equation}
This subsection describes the predicate $\op{lookup\_database}$,
and the following subsections  will describe the more complicated
predicate $\op{path\_hyper}$.

As the name suggests, $\op{lookup\_database}(\rho,\rho')$ holds
when $\rho'$ is obtained from $\rho=(H,\op{fl},S)$ by adding
lookups from a database.

\begin{definition}[derived~lookup]
We say that $\rho'$ is derived from $\rho=(H,\op{fl},S)$ by lookup
in the database $B$ and write
    $$\op{lookup\_database}(\rho,\rho')$$
if $\rho' = (H,\op{fl},S \cup S')$ where $S'$ is a set of lookups
in $B$ with domain constraints in $\op{single}(S)$. (The database
$B$ is fixed once and for all and hence is omitted from notation.)
\end{definition}

\subsection{connected sum -- hypermap}

The second mode of refinement is the connected sum of a hypermap.  This
subsection gives the details of this construction.

Let $\rho=(H,\op{fl},S)$ and $\rho'=(H',\op{fl}',S')$.   The rough
idea is that $\op{sum\_hyper}(\rho,\rho')$ holds if we can
obtain $H'$ from $H$ by a connected sum. That is,
    $$H' = H\,\#_\phi\, B.
    $$
The actual
construction of the predicate $\op{sum\_hyper}$ is somewhat more
involved because it involves a change in hypermap and
corresponding flags. This means that the indexing sets for the
system of formal disjunctions also change.  Part of the connected sum
operation involves relating the formal disjunctions $S$ on
indexing set $I(H,\op{fl})$ to formal disjunctions $S'$ on the
indexing set $I(H',\op{fl'})$.

\subsubsection{sum kit}.


A connected sum of hypermaps $H\,\#\,B$  requires an
orientation reversing bijection $\phi:F\to F_B$ between a face of
$H$ and a face of $B$.  Some of the flags have been designed to
give such bijections.  In fact,  if $n=\card(F)$, every flag
$\op{sum\_typ}(n)(F)$ of a patchable face $F\in (H/f)_{n}$
uniquely determines a hypermap $B$ and an orientation reversing
bijection $\phi:F\to F_B$, for some face $F_B$ of $F$.

The data of $B,\phi$ and some auxiliary information about flags is
what we call a {\it connected sum kit}.  The connected sum kits are described by
various figures.  We list the figures (Figure~\ref{XX}) and then
say how to interpret them.

(N.B. This section is incomplete.  It is necessary to give all the
figures, labeled according to the interpretive key.)\FIXX{Do this.}

The following comments provide an interpretive key to the
figures.
\begin{itemize}
    \item $B$ is the connected, plain, planar
    hypermap obtained by inserting darts at each angle of the
    planar graph, in the usual way, as illustrated in
    Figure~\ref{XX}.%% Not yet drawn.
    (N.B. Figure is to be inserted later.)
    \item The face $F_B$ is the outermost face; that is, the one
    bounding the unbounded region.
    \item The face $F_B$ has a distinguished dart $x$ marked on it,
    indicated as a small black triangle.
    \item In cases when the hypermap $B$ is not isomorphic to the
    mirror-image of $B$, there is a second hypermap that is
    relevant. In these cases, the figure for mirror 
    connected sum is obtained as the
    mirror image of the figure.  Thus, we draw a single figure
    and give it a second mirror name that is used to refer to the
    mirror data.
    \item The orientation-reversing bijection $\phi:F\to F_B$ is
    determined by choosing $\alpha\in F$ and specifying that
    $\phi(\alpha) =x$.  The map $\phi$ extends by the rule
        $$f_B^i\phi(f^i\alpha ) = x.$$
    \item Each figure has a name that we use as the name of the connected sum kit.
\end{itemize}

\begin{remark} As just mentioned, the orientation-reversing bijection $\phi:F\to F_B$
is determined by $\alpha\in F$.  The $\alpha$ and $f^i\alpha$ give
isomorphic connected sums if there is an automorphism of the hypermap
carrying $\alpha$ to $f^i\alpha$.  In this case, the infeasibility
for the parameter $f^i\alpha$ can be deduced from the
infeasibility for the parameter $\alpha$.  Thus, the two cases can
be combined into one.  From a mathematical point of view this
reduces the amount of verification to be done.  However, Nipkow's
experience with the formal verification of tame graphs suggest
that case mergings generate additional proof obligations, and
sometimes actually increase the amount of work required for a
formal proof.
\end{remark}








\subsubsection{flag compatibility}
\label{sec:com}

Let $H'= H \,\#_\phi\, B$.   There is a natural inclusion of
darts.
    $$
    \iota: H\setminus F \to H'.
    $$
There is a natural inclusion of faces, which we will express by
the same symbol:
    $$
    \iota: H/f \setminus \{F\}\to H'/f.
    $$

\begin{definition}[compatible~flag]
We say that a flag $\op{fl}'$ is compatible with
$(H,\op{fl},B,\phi)$  if the following conditions hold:
    \begin{enumerate}
    \item $\op{fl}'$ is a flag on $H'= H\,\#_\phi\, B$.
    \item The domain $F$ of $\phi$ is a patchable face.  That is,
    $\op{type}(\card(F))(F) = \op{patchable}$.
    \item The connected sum kit $(B,F_B,\phi,\ldots)$ of $F$ is given by
    $\op{sum\_typ}(\card(F))(F)$.
    \item We have $\op{qrn}(\alpha) = \op{qrn}'(\iota\alpha)$.
    (We recall also $\op{qrn}'$ is required to be constant on each
    node.)
    \item The node $N$ of $H'/n'$ (there is at most one) that does not lie in the image of
    $\iota$ has
        $\op{qrn}'(\beta)$ for $\beta\in N$ exactly when that node
        is {\it not} marked $\otimes$ in the Figure~\ref{XX}.  (That is, in
        every case except Figure~\ref{XX}, darts at a new node satisfy
        $\neg\op{qrn}'(\beta)$.)
    \item We have $\op{qre}(\alpha)= \op{qre}'(\iota\alpha)$.
    (We recall that $\op{qre}'$ is required to be constant on each
    edge.)
    \item The edges $E$ of $H'/e'$ that do not lie in the image of
    $\iota$ satisfy $\op{qre}'(\beta)$ for $\beta\in E$ exactly
    when that edge is marked as a single line in Figure~\ref{XX}.
    (That is, double lines give $\neg\op{qre}'(\beta)$.)
    \item We have $\op{hex\_loop\_diag}(\alpha) =
    \op{hex\_loop\_diag}(\iota\alpha)$.
    \item The edges $E$ of $H'/e'$ that do not lie in the image of
    $\iota$ satisfy $\op{hex\_loop\_diag}(\alpha)$ for $\alpha\in E$
    exactly when the edge is explicitly marked as such in the
    figures.  (The figures also shed some light on the terminology: such
    edges always run between opposite corners of a hexagonal face
    and run alongside a loop of four triangles.)
    \item The flags $\op{typ}(i)$ and
    $\op{sum\_typ}(i)$ are preserved on faces:
        $$
        \begin{array}{lll}
        \op{typ}(\card(F))(F) &= \op{typ}'(\card(F))(\iota F),\\
        \op{sum\_typ}(\card(F))(F) &= \op{sum\_typ}'(\card(F))(\iota F),\\
        \end{array}
        $$
    \item Usually, a face that is not in the image of $\iota$ is
    not standard, and hence lies outside the domain of the flags
    $\op{typ}(i)$ and $\op{sum\_typ}(i)$.  In this case there is
    no compatibility condition.
    \item If $\card(F)=8$, $\op{typ}(8)(F)=\op{patchable}$,
    $\op{sum\_typ}(8)(F) = \op{agg51}$, then there is a
    pentagonal face $F'$ in $B$ that carries over to $H'$.  In this case, we
    give the compatibility condition
        $$\op{typ}'(5)(F') = \op{notpri}.$$
    (This is, in fact, the only way to obtain the flag value
    $\op{notpri}$.)
    \end{enumerate}
\end{definition}


\begin{remark}
We note that if $H'= H\,\#_\phi\, B$ and $\op{fl}$ are given,
then there is at most one flag $\op{fl}'$ that gives
compatibility.
\end{remark}

\begin{remark}
We also note that if $H' = H\,\#_\phi\, B$, then by the
compatibility conditions, the faces coming from $B$ never have
type $\op{patchable}$. This means that we can never apply one
connected sum upon another.
\end{remark}

\subsubsection{indexing}

We assume that $H' = H\,\#_\phi\, B$.  As in the previous
subsection (\ref{sec:com}) we write $\iota$ for the natural
inclusion  on darts and faces.

Under the natural inclusion
    $$\iota:H\setminus F \to H'$$
there is a correspondence between a subset of $I=I(H,\op{fl})$ and
$I'=I(H',\op{fl}')$.  We make this correspondence more explicit
here.



All the indexing sets for $I=I(H,\op{fl})$ run over a subset of
darts (for $\op{azim}$, $\op{yn}$, $\op{ye}$, and so forth) or
over a subset of faces (for $\op{sc}$, $\op{sol}$, and so forth).
Let $I^F\subset I$ be the subset of the indexing set that does not
involve any of the darts of $F$, and does not involve the face $F$
itself.

We then have a one-to-one map $I^F \to I'$.  This gives a
one-to-one map between the subset of the set of formal
inequalities on the indexing set $I$ (those involving only the
indices $I'$) and the set of formal inequalities on the indexing
set $I'$. Extending to formal disjunctions, we get a one-to-one
map $\op{LFOR}_{I^F}\to \op{LFOR}_{I'}$.

\begin{definition}[inheritance]  We say that a formal disjunction on the
indexing set $I'$ is {\it inherited} from one of $I$, if it lies
in the image of this one-to-one map with domain
    $$\op{LFOR}_{I^F}\subset \op{LFOR}_I.$$
\end{definition}



\begin{definition}[connected~sum]
We say that $\rho'=(H',\op{fl}',S')$ is obtained by a hypermap
connected sum $\rho =(H,\op{fl},S)$  and write
$\op{sum\_hyper}(\rho,\rho')$ if there exists a patchable face
$F\in H/f$ such that
    \begin{enumerate}
    \item $(B,\phi)$ is the data determined by $\op{sum\_type}(\card(F))(F)$.
    \item $H' = H\,\#_\phi\, B$.
    \item The flag $\op{fl}'$ is compatible with $(H,\op{fl},B,\phi)$.
    \item Every formal disjunction $A'$ in $S'$ is either a basic
    formula for $(H,\op{fl})$ or is inherited from $S$.
    \end{enumerate}
\end{definition}

\subsection{constants}

This chapter has used a number of constants: $\xi_\Gamma$,
$\xi_V$, $\pt$, $s_i$, $t_i$. $D\y{(n,k)}$, $Z\y{(n,k)}$, $t_0$, $\phi_0$,
$\delta_\op{oct}$.
%
These constants are all the same as in \cite{DCG}.  They are
listed in its index.


\section{Implementing the Proof of Weak Infeasibility}

By the definition of weak infeasibility, there exists a tree that
describes the branches followed in the search for infeasibility.
Ultimately the proof of the main LP bound is constructive, in the
sense that the proof gives all of the data needed to construct the
search tree.

Of course, many different infeasibility trees can be used as
certificate of weak infeasibility.  Making less than optimal
decisions about how to construct the tree may increase the amount
of computation needed, but if a reasonable strategy is followed,
the exact choices are not too important.  What we give in this
section are some search strategies that seem to be helpful, and
that were in fact helpful in the original 1998 solution.
  We wish to emphasize that it may very well be
possible to devise much more efficient search strategies than
those described here.

At every stage of the search there are many possible options about
how to proceed.  Some of the options are to
    \begin{itemize}
    \item Obtain a new hypermap by a connected sum of the current one.
    \item Bifurcate on a disjunctive formula.
    \item Enhancing the system of formal disjunctions by adding
    lookups.
    \end{itemize}
The simple act of bifurcating on a formula has infinitely many
possibilities.

To construct the infeasibility trees, it is necessary to be guided
by various heuristics.  We are not looking for a precise
algorithm, merely heuristics that can guide a computer program to
construct the appropriate trees. What actions should have high
priority and what actions should be used as a last resort?



Of course, it may be the the original system is infeasible.  The
first thing to try is the infeasibility of the system obtained
without including any inequalities that depend on the value of the
flags.  (It is often best to ignore the inequalities depending on
flags in the early stages.)  If this system is infeasible, it
implies in one fell swoop the weak infeasibility of all primary
tame hypermap systems $(H,\op{fl},S)$ with a given hypermap $H$.
In fact, thousands of hypermaps can be quickly eliminated by this
one simple strategy.  We now confine our remarks to primary tame
hypermap systems for which this simple strategy fails.

Sometimes the heuristic can be guided by geometry.  The nodes of
the hypermap represent the centers of a finite cluster of balls
surrounding the origin.
 The variables $\optt{yn}(\alpha)$ represent the distances of
 these ball centers from the origin.  The idea is move the
 system in the direction of a cluster of balls in which all of
 the distances $\optt{yn}(\alpha)$ are constrained to be about
 2.0:
    $$2 \sle \optt{yn}(\alpha) \sle 2+\epsilon.$$
 Once these variables are heavily constrained, the focus moves to the
 variables $\optt{ye}(\alpha)$.  The strategy is then to constrain
 these variables as tightly as possible between an upper and lower
 bound.  Once strong constraints are obtained on these variables,
 the weak infeasibility often follows quite easily.

Another rough guide is to keep the number of branches under
control.  For example, we could easily get exponential growth in
the tree if we repeatedly bifurcate all active branches of the
tree.  Large exponential growths should be avoided.  Thus, it is
often good to bifurcate in a way that one of two possibilities is
infeasible.  For example, we can choose a constant $c$ (obtained
by linear programming) for which
    $$\optt{yn}(\alpha)\sge c$$ is infeasible.  Then a bifurcation
    on
$$(\optt{yn}(\alpha) \sge c) \vee (\optt{yn}(\alpha) \sle c)$$
has a single continuation.  This allows us to add
    $$\optt{yn}(\alpha)\sle c$$
to the system.  Adding such an inequality has further
consequences, because there are many nonlinear lookups whose
domain constraints take precisely this shape.

One rough guide is that the most useful inequalities come from
faces of the smallest size.  In particular, the inequalities for
triangles are by far the most useful.  Thus, one strategy is to
make triangles the focus of the bifurcation for as far into the
search tree as possible, bringing in large faces only as necessary
late in the game.

Since all triangles in a primary tame hypermap system are
standard, the opening moves should concentrate on standard
triangles.  A good opening strategy might be to narrow the
feasible range on the variables $\optt{yn}(\alpha)$.

\section{Conclusion}

This chapter is still sketchy in many places.  However, the
general framework of the linear programming has been placed on
solid foundation.  There are other references that may be
consulted for further details.  These references include the
published solution to the packing problem in \cite{DCG} (Section
23, pages 236ff). There is the database of nonlinear inequalities
at Hales's web page. There is the archive of tame planar graphs.

\subsection{Mathematica}

There are Mathematica files that were used to generate the linear
programs.  The most important of these is {\it MathToCplex.m}
which appears with the linear programming materials in the
original 1998 proof.  This file is important because it generates
all of the basic linear programs.  There are also supplementary
files {\it MathToCplexExcept.m} and {\it MathToCplexPent.m} that
handle the more sophisticated branch and bound operations.  All
the linear programs were generated by these files.

Interactive Mathematica sessions were used to run and analyze the
linear programs.  The Mathematica programs generate text files in
cplex format and stored as external files. The Cplex linear
programming package was run on these files, by issuing a command
from Mathematica to the operating system.  The results from the
linear programs were stored as text files, then parsed back into
the Mathematica session for analysis.  The human controller
(Hales) made interactive decisions about how to branch the linear
programs, based on heuristics and intuition about how to establish
infeasibily in the quickest way possible.  Various Mathematica
procedures were written to automate the branch and bound process.

These interactive sessions were summarized in the form of log
files, to make it possible to reconstruct the moves that led to
infeasibility. These log files consist of notes written by Hales.
Mixed in with the notes are snippets of the Mathematica code that
was used to direct the branch and bound operations.  Thus, in
principle, the log files contain all that is necessary to
reconstruct the sequence of linear programs.  These are all
available from Hales's web page.

The files terminating in .m are Mathematica files.  Those in .exec
are files that control the execution of Cplex.  Those in .lp are
cplex format linear programs.  Those in .log are the output from
Cplex sessions that get parsed back into a Mathematica session.
Those in .html contain my notes and snippets of Mathematical code
that describe the branch and bound for the linear programs.

It would be clearly very valuable to rework this, so that the
process is automated from beginning to end.  The Flyspeck project
hopes to carry this even further, to give a formal proof of the
main theorem of this paper.

\subsection{log file}

The following is the file from Hales's webpage
(/SHORT/OCT/oct.zip/OCT/index.html) that describes how the
hypermaps containing an octahedral face were handled.  It is a
mixture of plain text and Mathematica code.  A few comments have
been added.  Otherwise, it is identical to the 1998 log.  The
hypermaps with octahedral faces would be a good place to start in
a second generation treatment of the linear programs for hypermaps
with an exceptional face.

\bigskip

{\obeylines\tt

(* How to do the octagonal cases.  This first block of code
generates all the linear programs for the octahedra.  The files
are stored in the SHORT/OCT/LP directory, and the file names all
start with cplexE.lp.  *) \

\

\

f[i\_] := (
        exceptStem = "SHORT/OCT/LP/cplexE.lp";
        Initialize[8,i];
        Array[(Print[{8,i, \#1}];
        WRITEOUTexcept[8,i, Length[GBLregions], \#1]) \& , Length[AllOct]]
        );
Map[f,locthi];

\

\

(* The following block of code generates code for an interactive
cplex session.  It tells cplex to read each of the files and run
the linear programming optimization in the file. *)

\

\

optText[n\_,i\_]:= Module[{},
Initialize[n,i];
        Array[{"read ",
            fileText["cplexE.lp",n,i,Length[GBLregions],\#], " lp$\backslash$n",
            "opt$\backslash$n"
            }\&, Length[AllOct]
            ]//StringJoin
        ];

\

\

(* This stores the cplex instructions to an external file called
cplexE.exec *)

\

\

Module[{vv,stream},
    vv = Map[optText[8,\#]\&,locthi]//StringJoin;
    stream=OpenWrite["SHORT/OCT/cplexE.exec"];
    WriteString[stream,vv];
    Close[stream]
    ];

\

\


(* After running the cplex program, the results are parsed back
into Mathematica as an array called values.  The following command
finds that there is just one case that didn't pass on this first
attempt.  The command is everything up to the symbol ==.
Everything following the symbol "==" is the output from the
command.  The constant 0.4429 is approximately 8 pt.  *)

\

\

hh= Select[values,\#[[1]]>0.4429\&]== {{0.459297, 8, 14, 17, 21}};

\

\

(* The rest gives a few hints about how to eliminate this final
case.  It is based on the inequalities from DCG page 252, section
25.6.2. We are able to get a smaller value of the penalty from the
final lines of DCG section 25.6.2.  What follows is a mixture of
plain text and Mathematica.  *)

\

\

LPmax[bf,"-sigma{17}"] == 0.299186; if there is a Z(4,2):   s8 +
ZLP(4,2)-Z(4,2) = -0.31398. So no (4,2)s.  Since the two flats are
opposite, the only penalties
    are from (4,1)s that mask a flat, so pen17 < 2(xiG+2xiV) = 0.045304.
    Edit the file so pen17 = 0.045304.
bf = "SHORT/OCT/LP/cplexE.lp8.14.F17.C21pen"

\

\

(* With this smaller penalty in place, we branch and bound on
eight of the triangular faces, with the usual 6.25 cutoff, as
described in DCG, Sec 24.1, page 241. We find that in every case
the bound comes out less than 8 pt. Again, everything before the
"==" is the Mathematica command, and everything following it is
the output from the interactive session.  Since this constant is
less than 8 pt, we are done. *)

\

\

ww=( Initialize[8,14]; Array[
LPdisplay["SHORT/OCT/LP/cplexE.lp8.14.F17.C21pen",
    Branch[{6,7,9,10,11,12,13,14},\#] ]\&,2\^8])//Max ==  0.441675

\

\

OCTs are done.

}




\subsubsection{second generation lp}

A couple of years after the proof was completed, an independent
check was made of the linear programs for tame plane graphs that
contain only triangles and quadrilaterals (2000, unpublished, but
available on Hales's web page). This is an entirely new and
independent implementation of the linear programming, but it was
not done in sufficient generality to handle hypermaps with
exceptional faces. This code is written in Java.  The results of
the linear programs are checked with a Java-based implementation
of interval arithmetic. This code has the advantage that the
branch and bound operations are entirely automated.  This code is
available from Hales's website.

The reason that this code is unable to treat hypermaps with
exceptional faces is that no implementation has been given of the
connected sum operation (called faced refinements in the original 1998
proof).


\subsection {what is still missing from this chapter}

This chapter is starting to get quite long, and it still is far
from complete.  Based on what is here, it now looks as if a
detailed specification of the linear programming part of the proof
may run to some 100 pages. My hope at this point is that it
contains enough about the structure of the task, to make the other
sources of information comprehensible.

This final subsection has been written as a work in progress,
giving lists of topics that must eventually be carefully
specified, but that have not yet been.

\subsubsection{inequality}

This chapter does not contain a complete list of the inequalities
that are used in the linear programming infeasibility.  Consult
the Mathematica files and \cite{DCG} for the list.  There are
various important inequalities that need to be added.

\begin{itemize}
    \item There is an important identity for $\optt{vor0}$ in terms of the
    variables $\optt{quo}$, $\optt{Adih}$, etc.
    \begin{equation}
    \optt{vor0}(F) = \phi_0 \optt{sol}(F) +\sum_{\alpha\in F}
    (\optt{Adih}(\alpha) - 4\delta_\op{oct}(\optt{quo}(\alpha)
    +\optt{quob}(\alpha))).
    \end{equation}
    (Compare \cite{DCG}, Sec 25.12, page 257.)
%
    \item There are various inequalities for $\optt{quo}$ obtained
    by nonlinear lookup.
%
    \item We have
        \begin{equation}
        \optt{quob}(\alpha) = \optt{quo}(e\alpha),\quad
        \forall\alpha.
        \end{equation}
    The nonlinear interpretation of $\optt{quo}(\alpha)$ is
    the function $\op{quo}(v,w)$ of \cite{DCG}, and that of
    $\optt{quob}(\alpha)$ is $\op{quo}(w,v)$.
%
    \item There are various inequalities for $\optt{Adih}(\alpha)$
    obtained by nonlinear lookup.  These are rather
    unconventional, because they require the input of lower and
    upper bounds on the variable $\optt{azim}(\alpha)$.  There is
    a note about this at the end of Sec 25.12 of \cite{DCG}, page 257.
%
    \item When $\optt{eta}'(\alpha)\sge t_0$, then $\optt{quo}(\alpha)
    = \optt{quob}(\alpha)=0$.
    \item There are lower and upper bounds on $\optt{azim}(\alpha)$ for all
    darts $\alpha$, depending on the structure of the face to
    which $\alpha$ belongs.  These are given in a table in
    \cite[VI,p.54]{Hal98D}.  The same bounds appear in the database as
    inequality 853728973.
    \item There are the hexagonal inequalities of \cite[25.13]{DCG}.
  There are some problems -- still unresolved by this chapter --
  about the interpretation of the variables $\sigma_R$,
  $\sigma_R^+$, $\tau_R$, $\tau_R^+$ of that paper in terms of the variables
  of this chapter.
  \item There is the ``$3.0$'' quadrilateral inequality of
  \cite[Sec.25.10]{DCG}.
  \item There are the $t(n)$ and $s(n)$ inequalities when there
  are no aggregates, and the weaker version of the same
  inequalities that hold over aggregates.
\end{itemize}

There are several important inequalities that hold over groups of
faces.

\begin{itemize}
  \item There are the ``$0.55$'' and ``$0.48$'' bounds found in
  \cite[Lemma 10.6,p.101]{DCG}.
  \item There are the ``$1.4$'' and ``$1.5$'' bounds found in
  \cite[Lemma~22.12,p.233]{DCG}.  There is a remark at the end of
  Section~4.10 of \cite[VI,p.37]{Hal98D} that shows how these
  inequalities can sometimes be improved by removing certain faces
  from the sums of those inequalities.
\end{itemize}


There are special inequalities on simplices that are potentially
of type C or C'.  See the note below in
Subsection~\ref{sec:wrong}.





\subsubsection{connected sum}

This chapter does not yet draw all of the connected sum possibilities.
Again, this is found in \cite[Sec.25.5--25.6,pp.249--251]{DCG}.
When drawing these connected sum kits, it is necessary to add a mirror
case to each one that does not have reflectional symmetry.  When
drawing these connected sum kits, it is necessary to add markings as
described in the interpretive key in this chapter.

The penalties from \cite{DCG} have been changed in this chapter
so that they adhere to a specific face, rather than to the entire
connected sum kit.  Doing so, cleans up the scoring rule of
Equation~\ref{eqn:xi}.




\subsubsection{lookup}

Detailed information about which inequalities were pulled from the
nonlinear database by lookup can be determined by consulting the
Mathematica files.





\subsubsection{variable}

Complete the list of variable names.  Check the names of variables
on such things as $\optt{sigmaflat}$ $\optt{tauflat}$,
$\optt{svoran}$, $\optt{nu}$, $\optt{vorx}$.

\subsubsection{extension}

There are other linear programs, besides the Main LP Bound that
arise in the proof of the packing problem.  These are the {\it
small} linear programs, so-called because they involve a small
number of variables and constraints.  An example of small linear
programs are those appearing in the proof of Lemma~10.5 in
\cite{DCG}.  There are similar linear programs scattered
throughout the paper.

The linear programs for the proof of the dodecahedral conjecture
should be developed along similar lines as this.

This chapter separates the linear programming infeasibility
results from the geometry.  To build a bridge between the Main LP
Bound and the proof of the packing problem, it is necessary to
prove that any counterexample  leads to a
strong solution to a tame hypermap system.  The basic ingredients
to prove this are already in \cite{DCG}, but it will be necessary
to write this up, to show exactly how it follows from results in
those papers.




\subsubsection{things that may be wrong or incomplete}
\label{sec:wrong}

The treatment of linear programming variables related to the score
is very muddy in \cite{DCG}.   I am not aware of any explicit
errors, but it is extremely difficult to piece together precise
linear programming inequalities, based on what is written there.
Section 25.12 of \cite{DCG} is particularly confusing, and should
be avoided whenever possible.  The description of penalties in
Section~\ref{sec:pc} and Equation~\ref{eqn:xi} are intended to be
clarified versions of \cite{DCG}.

There is one important case of the scoring function that has not
been entirely specified in this chapter.  This concerns the
inequalities for $\optt{sc}(F)$ for triangular faces $F$ with one
edge that is not quasi-regular and one node that is not
quasi-regular.  These include the cases of simplices of type C and
type C' from \cite[Sec.9.4,pp.94--99]{DCG}, which have special
scoring rules.  On such simplices there will be an inequality that
takes the general form
    $$
    \optt{sc}(F)\sle \optt{vorx}(F).
    $$
There are a number of lookup inequalities for $\optt{vorx}$.  What
I still need to confirm is that nothing funny is going on with the
switches to type C, C', etc.

All the linear programs use weak inequalities $(\le),(\ge)$.  When
we branch over a disjunction
    \begin{equation}
    \label{eqn:dis}
    (\optt{A}\sle C)\vee (\optt{A}\sge C),
    \end{equation}
there is overlap between the two cases when equality occurs.
Normally, this is not a problem; but in a couple of special
situations it creates a problem.  (In the 1998 proof this was
handled carefully at a conceptual level, but it does not find its
way into the code.  For the fully automated proof, it has to find
its way into the code.)

Even though this issue may seem minor, I think that it may be
necessary to add some components to the flags to resolve the
problems this creates.  What I am hoping to avoid is a
modification of the flag that makes it so that in a connected sum, there
can be more than one compatible flag $\op{fl}'$.  I prefer to
maintain uniqueness for $\op{fl}'$.

One place where it creates a problem is when there is explicit
branches between scoring systems.   For example, sometimes it is
necessary to consider two cases of scoring (compressed and 
decompressed) on quarters.  Since the scoring type cannot be
determined by weak inequalities (in the ambiguous case when a face
circumradius is exactly $\sqrt2$), weak inequalities are
inadequate in such cases.  The branching should really run over a
flag giving the scoring type, rather than over weak inequalities
for the circumradius of a face.


There is a trick that can often be used to avoid the case of
equality in the Formula~\ref{eqn:dis}.   One helpful strategy is
to transform disjunctions $A\vee (\optt{x} < B)$ into implications
    \begin{equation}
    (\optt{x} \ge B) \Rightarrow A.
    \end{equation}
(where there is no weakening), rather than into weak inequalities
    \begin{equation}
    A \vee (\optt{x}\le B)
    \end{equation}
Implications can be handled with nonlinear lookups,
$\op{nonlin}(C,L)$, where the antecedent is placed in $C$ and the
consequent is placed in $L$.

\begin{example}  For example, consider the branching over edge
lengths at the cutoff $2.45$ as discussed in \cite[Sec.25.7]{DCG}.
The inequalities $\optt{ye}(\alpha)\sle 2.45$ should transformed
as antecedents in the nonlinear lookup.\FIXX{Check this. Does it really work?}
\end{example}
 
 
%%%%%%%%%%%%%%%%%%%%%%%%%%%%%%%%%%%%%%%%%%%%%%%%%%%%
    %
%% XX issues.

\label{XX}\label{tarski:XX}

We mark warnings with the letters WW in 
the TeX file.

We mark serious issues with the letters XX.

\section{Notation}

We use the notation $]x,y[$ in trig.tex, then move away from it.

With functions $\sol$, $\op{sovo}$, $\op{sv}$, I am not
consistent about the order of the arguments.
The first argument should be $v$, then the simplex $S$,
then others such as truncation $t_0$ and $\lambda$.

\section{Spelling, Terminology}

Hyphenation on  quasi-, half-, semi-, non-, etc. needs to be made
consistent.  Two words or one, hyphenated or not?

$$
\begin{array}{lll}
 \text{geometrical} &\to& \text{geometric}\\
\end{array}
$$


 1. Give exact citation on algorithms.

 2. Put archival material at arXiv, with better title, etc.



 non conventions:
 non-Euclidean, non-triangular, non-strict, non-star, non-aggregated
 rest omit the hyphen: nonoverlapping nonlinear, nonzero, nonpositive, nonnegative,
   nonempty, nonadjacent, nondegenerate




\section{Definitions}

Some definitions are repeated ($\Delta$, $\dtet$, etc.)
Mark these in some way.

Is the notation $(p_x,p_y,p_z)$ for coordinates 
of $p\in R^3$ consistent throughout?

Use the word simplex combinatorially for four vertices.
Use the word tetrahedron geometrically for the convex hull.

barrier should be closed. Intersecting segment should be open ended.
(Open ends prevent meeting at a vertex.)
Everything will be done up to a null set
anyway for measure related things.

Definition of clusters has changed.  The word should be deleted
in favor of standard component.  

The definition of corner has been deleted.


\section{Define before use}

Make sure circle, great circle are defined before use.
A good place would be in vector geoemtry chapter.

Define basic notions of $\op{aff}$, $\op{conv}$, $\op{cone}$
before first use, as in vector geometry chapter.


\section{Policy Things}

Here we list some things that have been changed.  Make sure that
all occurences have been eliminated.

There should be no mention of enumerated packings.

Indexing: Things before the ``AT'' symbol are used for ordering entries.
Things after used for display in the actual index.
Subentry is!  After vertical slash stuff gets passed to page
number.  \\index{policy|bold}  puts the page in bold.
\\index{policy|(} starts a subrange \\index{policy|)} ends a page range.

Make all Deltas take same args in same order (to simplify formal proofs).
$\chi$, $\Delta$, $\ups$ should appear only in tarski.tex. 

\section{Incomplete Things in Tarski}

Eventually, I want to separate out the Tarski stuff, quoting
only the results that are needed.  Quote by footnote. Index every use.

Need an introduction to Tarski. What it is all about.

Replace $y_i$ by $y_{jk}$ globally.


\section{Change in Tame Graph Definition}

We add a new type of aggregate: the pentagon-triangle quadrilateral.
It needs to be checked that all of the properties of tameness go
through for this quadrilateral.  This affects the linear programs.
There needs to be a new flag (pent-tri) on quadrilateral faces.
None of the other inequalities for quads are valid in this case.
It has to be run as a separate initial case.

We add a new condition on separated set.  We require that 
at each node of the separated set, we have $p_v\in\{3,4\}$.
This allows us to eliminate a couple of pages of analysis
that rule out $p_v=0,1,2$.  However, this is an incompatible
change in the tame graph spec.  The graph generator has to
be rerun.  It is possible that new tame graphs will arise.
This affects the Bauer-Nipkow formal proof.

\section{Things to do:}

\subsection{hypermap}
Describe the localization of a hypermap.  (Throw away everything
except for one contour loop.)

Describe the components of $Y(v_0,V,E)$ that lie interior to
a contour loop.  


\subsection{Ferguson's thesis}

Add this.




    % backmatter

%\renewcommand\partname{Back Matter}
%\renewcommand\chaptername{Appendix}

\backmatter

\theendnotes
%% OLD REFERENCES



\begin{thebibliography}{CHMS94}


\bibitem[AH83]{interval} G. Alefeld and J. Herzeberger, Introduction
    to Interval Computations, Academic Press, New York, 1983.

\bibitem[arXiv]{arXiv} http://xxx.lanl.gov.

\bibitem[Ben74]{Ben74}  Bender, C., Bestimmung der gr\"ossten Anzahl gleich
grosser Kugeln, welche sich auf eine Kugel von demselben Radius,
wie die \"ubrigen, auflegen lassen, {\it Archiv Math. Physik} 56
(1874), 302--306.

\bibitem[Bez90]{Bez90} A. Bezdek and W. Kuperberg, Maximum density space packing with
    congruent circular cylinders of infinite length,
    {\it Mathematica} 37 (1990), 74--80.

\bibitem[BKM91]{BKM91} A. Bezdek, W. Kuperberg, and E. Makai Jr., Maximum density
    space packing with parallel strings of balls,
    {\it DCG} 6 (1991), 227--283.

\bibitem[Bez94]{Bez94} A. Bezdek, A remark on the packing density in the 3-space
    in {\it Intuitive Geometry}, ed. K. B\"or\"oczky and G. Fejes
    T\'oth, {\it Colloquia Math. Soc. J\'anos Bolyai} 63, North-Holland
    (1994), 17--22.

\bibitem[Bez97]{Bez97} K. Bezdek, Isoperimetric inequalities and the dodecahedral
    conjecture, {\it Internat. J. Math.} 8, no. 6 (1997), 759--780.

\bibitem[Bli19]{Bli19} H. F. Blichfeldt,
    Report on the theory of the geometry of numbers,
    {\it Bull. AMS}, 25 (1919), 449--453.

\bibitem[Bli29]{Bli29} H. F. Blichfeldt,
    The minimum value of quadratic forms and the closest
    packing of spheres, {\it Math. Annalen} 101 (1929), 605--608.

\bibitem[Bli35]{Bli35} H. F. Blichfeldt,
    The minimum values of positive quadratic forms in six,
    seven and eight variables, {\it Math. Zeit.} 39 (1935), 1--15.

\bibitem[Boe52]{Boe52} Boerdijk, A. H. Some remarks concerning close-packing
of equal Spheres, {\it Philips Res. Rep.} 7 (1952), 303--313.

\bibitem[Bou68]{Bo} N. Bourbaki, Elements of Sets, Addison-Wesley, 1968.

\bibitem[CK04]{CoKu} H. Cohn, A. Kumar, The densest lattice in twenty-four
dimensions, math.MG/0408174, (2004).

\bibitem[CHMS94]{CoHMS94} J. H. Conway, T. C. Hales, D. J. Muder, and N. J. A. Sloane,
    On the Kepler conjecture, {\it Math. Intelligencer} 16,
    no. 2 (1994), 5.

\bibitem[CS95]{CoSl95}  J. H. Conway,  N. J. A. Sloane, What are all the
best sphere packings in low dimensions? {\it DCG} 13 (1995),
383--403.

\bibitem[CS98]{CS} J. H. Conway and N. J. A. Sloane, Sphere packings, lattices
    and groups,  third edition, Springer-Verlag, New York, 1998.

\bibitem[Eul78]{Euler} L. Euler, Variae speculationes super area
Triangulorum Sphaericorum, 1778.

\bibitem[Fej93]{Fej93} G. Fejes T\'oth and W. Kuperberg, Recent results in the
    theory of packing and covering, in New trends in
    discrete and computational geometry, ed. J. Pach, Springer
    1993, 251--279.

\bibitem[Fej95]{Fej95} G. Fejes T\'oth, Review of [Hsi93], {\it Math. Review} 95g\#52032, 1995.

\bibitem[Fej95b]{Fej95b} G. Fejes T\'oth, Densest packings of typical convex sets
    are not lattice-like, {\it DCG}, 14 (1995), 1--8.

\bibitem[Fej97]{Fej97} G. Fejes T\'oth, Recent progress on packing and covering,
     Advances
in Discrete and Computational Geometry, (South Hadley, MA, 1996),
pp. 145-162. Contemp. Math. 223 (1999), AMS, Providence, RI, 1999.
MR 99g:52036

\bibitem[Fej72]{Fej72} L. Fejes T\'oth, {\it Lagerungen in der Ebene auf der
    Kugel und im Raum}, second edition,
    Springer-Verlag, Berlin New York, 1972.

\bibitem[Fej64]{Fej64} L. Fejes T\'oth, Regular figures, Pergamon Press,
    Oxford London New York, 1964.

\bibitem[Fej42]{Fej42} L. Fejes T\'oth,  \"Uber die dichteste Kugellagerung,
{\it Math. Zeit.} 48 (1942 1943), 676--684.

\bibitem[Fej50]{Fej50} L. Fejes T\'oth, Some packing and covering theorems,
    {\it Acta Scientiarum Mathematicarum (Szeded)} 12/A, 62--67.

\bibitem[Fej53]{Fej53} L. Fejes T\'oth, {\it Lagerungen in der Ebene auf
der Kugel und im Raum}, Springer, Berlin, first edition, 1953.

\bibitem[FH98]{Form} S. P. Ferguson and T. C. Hales, A formulation of
the Kepler Conjecture, preprint 1998.

\bibitem[Fer97]{Fer97} Samuel P. Ferguson, Sphere
Packings V, thesis, University of Michigan,
    1997;  arXiv math.MG/9811077; to appear in Discrete and Computational
    Geometry.

\bibitem[Gau31]{Gau31} C. F. Gauss, Untersuchungen \"uber die Eigenscahften der
positiven tern\"aren quadratischen Formen von Ludwig August Seber,
    {\it G\"ottingische gelehrte Anzeigen}, 1831 Juli 9,
also published in {\it J. reine angew. Math.} 20 (1840), 312--320,
and
    {\it Werke},  vol. 2,
    K\"onigliche Gesellschaft der Wissenschaften, G\"ottingen,
            1876, 188--196.

\bibitem[Gon05]{Gon} G. Gonthier, A computer-checked proof of the Four
Colour Theorem, preprint 2005.

\bibitem[Goo97]{Goo97} J. E. Goodman and J. O'Rourke, Handbook of discrete and
    computational geometry, CRC, Boca Raton and New York, 1997.

\bibitem[Gun75]{Gun75} S. G\"unther, {\it Ein stereometrisches Problem},
{\it Archiv der Math. Physik} 57 (1875), 209--215.

\bibitem[Hal92]{spp} Thomas C. Hales, The Sphere Packing Problem, J. of Comp.
and App. Math. 44 (1992) 41--76.

\bibitem[Hal93]{remarks} Thomas C. Hales, Remarks on the Density of Sphere Packings,
        Combinatorica, 13 (2) (1993) 181-197.

\bibitem[Hal94]{Hal94} T. C. Hales, The status of the Kepler conjecture,
    {\it Math. Intelligencer} 16, no. 3, (1994), 47--58.

\bibitem[Hal96a]{reform} Thomas C. Hales, A reformulation of the
Kepler Conjecture, unpublished manuscript, Nov. 1996.

\bibitem[Hal96b]{Hal96} T. C. Hales,
    {http://www.pitt.edu/\~\relax thales/kepler98/holyoke.html}

\bibitem[Hal97a]{part1} Thomas C. Hales, Sphere Packings I,
    Discrete and Computational Geometry, 17 (1997), 1-51.

\bibitem[Hal97b]{part2} Thomas C. Hales, Sphere Packings II,
    Discrete and Computational Geometry, 18 (1997), 135-149.

\bibitem[Hal98b]{Hal98B} T. C. Hales, Sphere Packings III, math.MG/9811075.

\bibitem[Hal98c]{Hal98C} T. C. Hales, Sphere Packings IV, math.MG/9811076.

\bibitem[Hal98d]{Hal98D} T. C. Hales, The Kepler Conjecture, math.MG/9811078.

\bibitem[Hal00]{CH} Thomas C. Hales, Cannonballs and Honeycombs,
Notices Amer. Math. Soc.  47  (2000),  no. 4, 440--449.

\bibitem[Hal01]{arbeitstagung} Thomas C. Hales, Sphere Packings in 3
Dimensions, Arbeitstagung, 2001.

\bibitem[Hal03]{algorithm} Thomas C. Hales, Some algorithms arising in
the proof of the Kepler Conjecture, Discrete and computational
geometry, 489--507, Algorithms Combin., 25, Springer, Berlin,
2003.

\bibitem[Hal06a]{DCG} Thomas C. Hales, A Proof of the
Kepler Conjecture (unabridged version), 
Discrete and Computational Geometry, 35:1, 2006.

\bibitem[Hal05a]{annals} Thomas C. Hales, A proof of the
Kepler conjecture, Annals of Mathematics, 162 (3), Nov. 2005.

\bibitem[Hal05b]{web} Thomas C. Hales, Computer resources for the Kepler conjecture, \hfill\break
    http://www.math.princeton.edu/\~\!annals/KeplerConjecture/.
    %\hfill{\it http://www.math.pitt.edu/\~%
    %\relax thales/kepler98.html} \hfil\break
    % (The source code, inequalities,
    %and other computer data relating to the solution are also found
    %at {\it http://xxx.lanl.gov/abs/math/9811078v1}.)
    %% {Hal98A} Needs to be updated to this.

\bibitem[Hal06b]{historical} Thomas C. Hales, An
Overview of the Kepler Conjecture, in \cite{DCG}.




\bibitem[Har68]{hart} J. F. Hart et al., Computer Approximations,
John Wiley and Sons, 1968.


\bibitem[Hil01]{hilbert} D. Hilbert, Mathematische Probleme, {\it Archiv Math. Physik} 1 (1901),
    44--63, also in {\it Proc. Sym. Pure Math.} 28 (1976), 1--34.

\bibitem[Hop74]{Hop74} Hoppe R. {\it Bemerkung der Redaction}, Math. Physik 56
(1874), 307-312.

\bibitem[HPT95]{HoPT95} R. Horst, P.M. Pardalos, N.V. Thoai, {\it Introduction
    to Global Optimization}, Kluwer, 1995.

\bibitem[Hsi93a]{Hsi93} W.-Y. Hsiang, On the sphere packing problem and the proof
    of Kepler's conjecture, Internat. J. Math 93 (1993), 739-831.

\bibitem[Hsi93b]{Hsi93a} W.-Y. Hsiang, On the sphere packing problem and the
    proof of Kepler's conjecture, in {\it Differential geometry and
    topology} (Alghero, 1992), World Scientific, River Edge,
    NJ, 1993,  117--127.

\bibitem[Hsi93c]{Hsi93b} W.-Y. Hsiang, The geometry of spheres, in {\it Differential
    geometry} (Shanghai, 1991), World Scientific, River Edge, NJ,
    1993, 92-107.

\bibitem[Hsi95]{Hsi95} W.-Y. Hsiang, A rejoinder to T. C. Hales's article ``The status
    of the Kepler conjecture,'' {\it Math. Intelligencer} 17, no. 1, (1995),
    35--42.

\bibitem[Hsi02]{Hsi02} W.-Y. Hsiang, Least Action Principle of Crystal Formation
of Dense Packing Type and the Proof of Kepler's Conjecture, World
Scientific, 2002.

\bibitem[IEEE]{IEEE} IEEE Standard for Binary Floating-Point
Arithmetic, ANSI/IEEE Std. 754-1985, IEEE, New York.


\bibitem[Kar66]{Kar66} R. Kargon, Atomism in England from Hariot to Newton,
    Oxford, 1966.

\bibitem[Kep66]{Kep66} J. Kepler, The Six-cornered snowflake, Oxford Clarendon Press,
    Oxford, 1966,  forward by L. L. Whyte.

\bibitem[KZ73]{KoZ73} A. Korkine and  G. Zolotareff, Sur les formes quadratiques,
    {\it Math. Annalen} 6 (1873), 366--389.

\bibitem[KZ77]{KoZ77} A. Korkine and  G. Zolotareff, Sur les formes quadratiques
    positives, {\it Math. Annalen} 11 (1877), 242--292.

\bibitem[Lag73]{Lag73} J. L. Lagrange,  Recherches d'arithm\'etique, {\it Nov. Mem.
    Acad. Roy. Sc. Bell Lettres Berlin} 1773, in {\it \OE uvres}, vol. 3,
    693--758.

\bibitem[Lee56]{Lee56} J. Leech, The Problem of the Thirteen Spheres,
{\it The Mathematical Gazette}, Feb 1956, 22--23.

\bibitem[Lin86]{Lin86} J. H. Lindsey II, Sphere packing in $R^3$, {\it Mathematika}
    33 (1986), 137--147.

\bibitem[Loo68]{Loomis} L. Loomis and S. Sternberg, Advanced Calculus, Adison-Wesley, 1968.

\bibitem[LP]{lpsolve} lp\_solve, http://groups.yahoo.com/group/lp\_solve/.

\bibitem[Mas66]{Mas66} B. J. Mason, On the shapes of snow crystals, in \cite{Kep66}.

\bibitem[McL98]{McL98} S. McLaughlin, A proof of the dodecahedral conjecture,
    preprint, math.MG/9811079.

\bibitem[Mel97]{Mel97} J. B. M. Melissen, Packing and covering with circles,
    Ph.D. dissertation, Univ. Utrecht, Dec. 1997.

\bibitem[Mil76]{Mil76} J. Milnor, Hilbert's problem 18: on crystallographic groups,
    fundamental domains, and on sphere packings, in
    Mathematical developments arising from Hilbert problems,
    {\it Proc. Symp. Pure Math.}, vol 28, 491--506, AMS, 1976.

\bibitem[MP93]{MP93} W. Moser, J. Pach, Research problems in discrete geometry,
    DIMACS Technical Report, 93032, 1993.

\bibitem[Mud88]{Mud88} D. J. Muder, Putting the best face on a Voronoi polyhedron,
    {\it Proc. London Math. Soc.} (3) 56 (1988), 329--348.


\bibitem[Mud93]{Mud93}  D. J. Muder A New Bound on the Local Density
of Sphere Packings, {\it Discrete and Comp. Geom.} 10 (1993),
351--375.

\bibitem[Mud97]{Mud97}  D. J. Muder, letter, in {\it Fermat's enigma}, by S. Singh,
        Walker, New York, 1997.

\bibitem[Oes90]{Oes90} J. Oesterl\'e,  Empilements de sph\`eres,
    S\'eminaire Bourbaki, vol. 1989/90, Ast\'erisque (1990),
        No. 189--190 exp. no. 727, 375--397.

\bibitem[PA95]{PaA95} J. Pach, P.K. Agarwal, {\it Combinatorial geometry}, John Wiley,
    New York 1995.

\bibitem[Plo00]{Plo00}  K. Plofker, private communication, January 2000.

\bibitem[Ran47]{Ran47} R. A. Rankin, {\it Annals of Math.} 48 (1947), 228--229.

\bibitem[Rog58]{Rog58} C. A. Rogers, The packing of equal spheres, {\it Proc. London Math.
    Soc.} (3) 8 (1958), 609--620.

\bibitem[Rog64]{Rog64} C. A. Rogers, {\it Packing and covering}, Cambridge University Press,
    Cambridge, 1964.

\bibitem[SW53]{Sch53} K. Sch\"utte and B.L. van der Waerden, Das
Problem der dreizehn Kugeln, {\it Math. Annalen} 125, (1953),
325--334.

\bibitem[SM44]{SeM44} B. Segre and K. Mahler, On the densest packing of
    circles, {\it Amer. Math Monthly} (1944), 261--270.

\bibitem[Shi83]{Shi83} J. W. Shirley,
{\it Thomas Harriot: a biography}, Oxford, 1983.

\bibitem[SHDC95]{SHDC95} N. J. A. Sloane, R. H. Hardin, T. D. S. Duff, J. H. Conway,
    Minimal-energy clusters of hard spheres,
    {\it DCG} 14,  no. 3, (1995), 237--259.

\bibitem[Szp02]{Szp02} G. G. Szpiro, Kepler's Conjecture, Wiley, 2002.

\bibitem[Thu92]{Thu92} A. Thue, Om nogle geometrisk taltheoretiske Theoremer,
    {\it Forandlingerneved de Skandinaviske Naturforskeres} 14 (1892), 352--353.

\bibitem[Thu10]{Thu10} A. Thue, \"Uber die dichteste Zusammenstellung von
    kongruenten Kreisen in der Ebene, {\it Christinia Vid. Selsk. Skr.} 1
    (1910), 1--9.

\bibitem[Tut84]{tutte} W. T. Tutte, Graph Theory, Addison-Wesley, 1984.

\bibitem[VS83]{oost} A. van Oosterom and J. Strackee, The solid
angle of a plane triangle, IEEE Trans Biomed Eng. 1983 Feb; 30(2):
125--6.

\bibitem[Wie05]{freek} F. Wiedijk, Formalizing 100
Theorems\hfil\break http://www.cs.ru.nl/\~{\hbox{}}freek/100/,

\bibitem[Why66]{Why66} L. L. Whyte, forward to \cite{Kep66}.

%% XX FIX THESE REFS.  MERGE WITH STUFF ABOVE.

\bibitem{BN}  XX G. Bauer, T. Nipkow, Flyspeck I: Tame Plane Graphs.

\bibitem{Fl}  XX T. Hales, The Flyspeck Fact Sheet, 2003; revised
2007.

\bibitem{LS}  XX Loomis and Sternberg,

\bibitem{Ob} XX S. Obua,

\bibitem{Sz} XX G. Szpiro,  The Kepler Conjecture.

\bibitem{Zu} XX R. Zumkeller,

\bibitem{Har1} J. Harrison, real numbers thesis.

\bibitem{Har2} J. Harrison, Kurzweil-Henstock integration.

\bibitem[KEP98]{KEP98} The Kepler Conjecture, arXiv, August 1998,
preprint. (Part I-- Part VI).

\bibitem[Hal07a]{quad} Equidecomposable Quadratic Regions,
preprint, 2007.

\bibitem[Hal05]{fly}  The Flyspeck Project, Dagstuhl, 2005.

\bibitem[Hal07b]{EZ} Easy Pieces in Geometry, preprint, 2007.

\end{thebibliography}


%\printindex

% to generate indices:
%% mkflydex:
%% makeindex Index.idx, etc.
\setcounter{chapter}{0}
\twocolumn{\chapter{Index}}
\input{Index.ind}

%\chapter{Formal Specs}
%This index cross references constants in the formal specification with notation in this book.  The formal spec gives the authoritative definition.
%\bigskip

%\input{Spec.ind}
%\twocolumn{\chapter{Greek Symbols}}
%\input{Greek.ind}
%\twocolumn{\chapter{Interval Calculations}}
%\input{Interval.ind}
%\twocolumn{\chapter{Cross References}}
%\input{References.ind}
%\twocolumn{\chapter{Proper Names}}
%\input{Names.ind}


    %%


\end{document}
