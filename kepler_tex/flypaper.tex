
%%
% Author Thomas C. Hales
% LaTeX Format


%!TEX TS-program = latex    
%% This line is for TexShop. 

% Revision history. See svn.
% Document created Dec 6, 2002
% Revision started Jan 2007, from published DCG.

% Revisions Sept 2009: dependency on Tarski removed. 2 enclosed over quad removed. hypermap algorithm rewritten.  

%% XX unable to make the \part command work.

\documentclass[spanningrule]{cambridge7A}
%% Cambridge University Press Macros from
%% https://authornet.cambridge.org/information/productionguide/laTex_files/
\usepackage{natbib}
\usepackage{rotating}
\usepackage{floatpag}
 \rotfloatpagestyle{empty}
\usepackage{amsthm}
\usepackage{graphicx}
\usepackage{multind}\ProvidesPackage{multind}
\usepackage{times}

%
%\usepackage[paper size={90mm, 120mm},left=2mm,right=2mm,top=2mm,bottom=2mm,nohead]{geometry}
\usepackage{verbatim}
\usepackage{latexsym}
\usepackage{amsfonts}
%\usepackage{amsthm}
\usepackage{amsmath}
%\usepackage{mathidx}
%%\usepackage{makeidx}
\usepackage{multicol}
\usepackage{crop}
\usepackage{txfonts}
%\usepackage{pdfsync}  %for TexShop sync.
\usepackage[letterpaper,colorlinks=true,%
  citecolor=red,%
  %breaklinks=true,%
  %pdftex,
  ps2pdf,%
  hyperindex=true]{hyperref}
%\usepackage{mparhack} %http://www.tex.ac.uk/cgi-bin/texfaq2html?label=marginparside
%%\usepackage{multind}
\usepackage{url}
%\usepackage{MnSymbol} % powerset.
\usepackage[mathscr]{euscript} % powerset.
\usepackage{pifont} %ding
\usepackage[displaymath]{lineno}

%\usepackage{zref}
%\usepackage{xkeyval}
%\usepackage{ifpdf}
%\usepackage{ifthen}
%\usepackage{calc}
%\usepackage{marginnote}
% http://tug.ctan.org/tex-archive/macros/latex/contrib/pdfcomment/doc/pdfcomment.pdf
\usepackage{pdfcomment}


% This file contains local settings and system dependencies



% Auxiliary directories
\def\dsp{/Users/thomashales/Pictures/mathFigures/DenseSpherePackings}  % flypaper graphics
%\def\pdf{/Users/thomashales/Pictures/collect_geom} % tarski graphics
\def\pdfp{/Users/thomashales/Pictures/mathFigures/collect_geom} % kepler graphics

\def\showgraphics{t}  
% t: display graphics (there are none to show yet)
% f (default): print a "no graphics logo" where graphics would normally go.


\def\displayallproof{t} 
% t (default): display all proofs.
% f: print documents without the proofs-- theorem statements only

\def\displayrating{f}
% t (default): display all ratings (verbose is also true)
% f : don't show them.

\def\verbose{t}
% f (default): do not display debugging information,
% t : display debug information and information about the formalization.

     
%-%
% --Repository--
%-%
% generate revision number by
% svn propset svn:keywords "LastChangedRevision" kepmacros.tex
\def\svninfo{%
  Document TeXed on \today. \hfill\break
  Repository Root: https://flyspeck.googlecode.com/svn \hfill\break
  SVN $LastChangedRevision$
  }

%-%
% --Fonts--
%-%
\font\twrm=cmr8

%-%
% --Graphics--
%-%
%set \showgraphics option in flag.tex
% flypaper graphics
 \def\szincludegraphics[#1]#2{%
      \if\showgraphics t{\includegraphics[#1]{#2}}%
      \else{\includegraphics{noimage.eps}}\fi}

% % kepler graphics
% \def\pdffigtemplatex[#1]#2#3#4{%   
% % usage: \pdffigtemplatex[width=80mm]{file.eps}{labelname}{caption}
% \begin{figure}[htb]%
%   \centering
%  \szincludegraphics[#1]{\pdfp/#2}
%  \caption{#4}
%  \label{fig:#3}%
% \end{figure}%
% }

\def\tikzfig#1#2#3{%
\begin{figure}[htb]%
  \centering
\begin{tikzpicture}#3
\end{tikzpicture}
  \caption{#2}
  \label{fig:#1}%
\end{figure}%
}

\def\tikzwrap#1#2#3#4{%
\begin{wrapfigure}{r}{#4\textwidth}
  \begin{center}
\begin{tikzpicture}#3
\end{tikzpicture}
\end{center}
  \caption{#2}
  \label{fig:#1}%
\end{wrapfigure}%
}


%\def\pdfg#1#2#3#4{\if\showgraphics t{\pdffigtemplatex[#1]{#2}{#3}{#4}}\else{}\fi}
%\def\myincludegraphics#1{%
%      \if\showgraphics t{\includegraphics{#1}}%
%      \else{\includegraphics{noimage.eps}}\fi}


%-%
% --Footnotes and Endnotes--
%-%
% http://help-csli.stanford.edu/tex/latex-footnotes.shtml
%\long\def\symbolfootnote[#1]#2{\begingroup%
%\def\thefootnote{\ensuremath{\fnsymbol{footnote}}}\footnote[#1]{#2}\endgroup}

%-%
% --Special Formatting--
%-%
% http://en.wikibooks.org/wiki/LaTeX/Formatting#List_Structures
%\renewcommand{\labelitemii}{$\star$}
\renewcommand{\labelitemii}{$\circ$}
\renewcommand{\labelenumii}{\alph{enumii}}
\newenvironment{summary}
  {\begingroup\bigskip\narrower\noindent{\bf Summary.~}\it}
%  {~\ding{98}\par\phantom{!}\endgroup\bigskip}
  {~\par\phantom{!}\endgroup\bigskip}
\newenvironment{tidbit}{\smallskip\begingroup}{\endgroup\smallskip}
%\newenvironment{enumerate}
%  {\renewcommand{\labelitemi}{}\begin{itemize}}
%  {\end{itemize}}
\def\wasitemize{\relax}
\def\uncase#1{{\sc #1}}
\def\case#1{{\sc (#1)}}
\def\claim#1{{\it  #1}}
\def\calcentry#1#2#3#4{{\smallskip{\bf #1}\quad{\tt [#2]}\quad{(#3)}\quad {#4}}} % computer calc entry
\def\id#1{\ensuremath{\text{\tt #1}}}



%-%
% --Indexing, References, Citations--
%-%
\def\indy#1#2{\index{index/#1}{#2}}
%\def\eqn#1{{\bf (\ref{#1})}}   % deprecated, use \eqref.
\def\newterm#1{\indy{Index}{#1}{\it #1}\relax}
\def\oldterm#1{\indy{Index}{#1}{#1}\relax}
\def\cc#1#2{%
  \indy{Index}{computer calculation!{#1}}{\it computer calculation}%
  \ifverbose{\footnote{\guid{#1}  #2}}~\cite{website:FlyspeckProject}} % arg dropped.


%-%
% --Endnotes
%-%
%\renewcommand{\maketextnotes}{\global\textnotesontrue
%  \newwrite\textnotes
%  \immediate\openout\textnotes=\jobname.ent
% \literaltextnote{
%\notesheadername={\the\textnotesheadername}
%%\pagestyle{endnotesstyle}
%\mark{3}
%\label{textualnotes}
%\normalfont \backmattertextfont}
%}
%\newcommand{\shipnotes}{
%   \iftextnoteson
%   \theendnotes
%   \immediate\closeout\textnotes
%   \input \jobname.ent
%   \else
%   \relax
%   \fi
%}

%-%
% --Proof Display--
%-%
% set with \displayallproof in flag_fly. If f, then proofs are swallowed.
%% "proved" environment. toggle with \displayallproof
%
\def\hide#1{}
\def\swallowed{\relax}
\def\swallow#1\swallowed{}
\newenvironment{iproved}{}{}
\newenvironment{proved}{\resetproved\begin{iproved}}{\end{iproved}}
\def\hideproof{\renewenvironment{iproved}{%
   \centerline{\it -- Proof Proofed --}
  \renewenvironment{itemize}{}{}
  \renewenvironment{enumerate}{}{}
  \def\item{\relax}
  \catcode13=12
  \swallow
}{}}
\def\showproof{\renewenvironment{iproved}{\begin{proof}}{\end{proof}}}
\def\resetproved{\if\displayallproof t\showproof\else\hideproof\fi}



%-%
% --Debugging Information--
%-%
%% verbose:
\def\rating#1{\if\displayrating t%
  {{\textsc {[rating={\ensuremath {#1}}].\ }}}\else{}\fi}
\def\rz#1{\rating{#1}}
\def\cutrate{}
\def\oldrating#1{\if\displayrating t%
  {{\textsc {[former rating={\ensuremath {#1}}].\ }}}\else{}\fi}

\def\formalauthor#1{\if\verbose t{{\tt [formal proof by #1].\ }}\else{}\fi}
\def\dcg#1#2{{\if\verbose t%
  {{\tt{[DCG-#1]}}\indy{References}{ZC{#2 #1}@{DCG-#1}|page{#2}}}\else{}\fi}}
\def\tlabel#1{\label{#1}\if\verbose t{{\tt [#1].\ }%
   \indy{References}{#1|itt}}\else{}\fi}
\def\ifcverbose#1#2{\if\verbose t{{#1}}\else{#2}\fi}
\def\ifverbose#1{\ifcverbose{#1}{}}  %\verbose t{{#1}}\else{}\fi}
%\def\formal#1{\ifverbose{[{#1}]}}
\def\formal#1{\relax }
\def\formaldef#1#2{\ifverbose{\texttt{[{#1} $\leftrightsquigarrow$ {#2}]}}}
\def\footformal#1{\if\verbose t{\footnote{#1}}\else{}\fi}
\def\guid#1{{\tt[#1].\ }\indy{References}{ZA{#1}@{#1}|itt}}
\def\guid#1{\ifverbose{{\tt [#1]}}}
\def\guid#1{{{\tt [#1]}}}
\def\ineq#1{{{\tt  [#1]}}}
%\def\guid#1{{{\tt [#1]}}}

%\def\calc#1{{\textsc{calc-#1}}\indy{Interval}{{#1}@{#1}}}
%\def\xfootnote#1{\if\verbose t{\endnote{#1}}\else{\footnote{#1}}\fi}
%\def\xfootnote#1{\footnote{#1}}
%\def\xendnote#1{\if\verbose t{\endnote{#1}}\else{}\fi}


% margin notes
\setlength{\marginparwidth}{1.2in}
\def\mar#1{}
 %\ifverbose{\marginpar{\text{\raggedright\footnotesize #1}}}}
%\def\hypermark[#1]#2{\ifcverbose{\hyperref[#1]{#2}}{#2}}


%-%
%--Formatting--
%-%
\def\dfrac#1#2{\frac{\displaystyle #1}{\displaystyle #2}}
\def\textand{\text{ \ and \ }}  % for math eqns.

%-%
%--Redefining--
%-%
\def\emptyset{\varnothing}
\def\ups{\upsilonup} % Needs txfonts; else use \upsilon


%-%
% --Symbols--
%-%
% norm and brackets
\def\|{{\hskip0.1em|\hskip-0.15em|\hskip0.1em}}
\def\mid{\ :\ }
\def\tc{\hbox{:}}
\def\cooln{:\hskip-0.02em:}
\def\norm#1#2{\hbox{\ensuremath{\|#1\unskip-\unskip{#2}\|}}}
\def\normo#1{{\|#1\|}}
\def\sland{\ \land\ }
\def\abs#1{|#1|}
% brackets
\def\leftopen{(}
\def\leftclosed{[}
\def\rightopen{)}
\def\rightclosed{]}
\def\lp#1{{\llbracket{#1}\rrbracket}} 
%\def\comp#1{\llbracket #1 \rrbracket}
\def\comp#1{[#1]}
\def\tangle#1{\langle #1\rangle}
\def\ceil#1{\lceil #1\rceil}
\def\floor#1{\lfloor #1\rfloor}
%accents:
\def\=#1{\accent"16 #1}
\def\ast{\ensuremath{{}^*}}

% mathcal
\def\CalV{{\mathcal V}}
\def\CalL{{\mathcal L}}
\def\BB{{\mathcal B}}
\def\powerset{{\mathscr P}}

% mathbb
\newcommand{\ring}[1]{\mathbb{#1}}
%\def\N{{\mathbb N}}
%\def\Rp{\ring{R}^{3\,\prime}}
%\def\A{{\mathbf A}}
\def\F{{\mathbf F}} % map on faces H to H/{cal L}

% vector notation
\def\v{{\mathbf v}}
\def\u{{\mathbf u}}
\def\w{{\mathbf w}}
\def\e{{\mathbf e} }  
\def\p{{\mathbf p}}
\def\q{{\mathbf q}}

% operatorname
\def\op#1{{\operatorname{#1}}}
\def\optt#1{{\operatorname{{\texttt{#1}}}}}

%\def\opat{{\op{@}}}
\def\atn{\op{arctan\ensuremath{_2}}}
\def\azim{\op{azim}}
\def\nd{\op{node}}
\def\sol{\operatorname{sol}}
\def\vol{\op{vol}}
\def\dih{\operatorname{dih}}
\def\Adih{\operatorname{Adih}}
\def\arc{\operatorname{arc}}
\def\rad{\operatorname{rad}}
\def\bool{\operatorname{bool}}
\def\true{\op{true}}
\def\false{\op{false}}


%\def\orz{\varthetaup} % center of packing
\def\orz{{\mathbf 0}} % center of packing
\def\Wdarto{W^0_{\text{dart}}}
\def\Wdart{W_{\text{dart}}}
%\def\Wedge{W_{\text{edge}}}
\def\cell{\operatorname{cell}}
\def\dimaff{\operatorname{dim\,aff}}
\def\aff{\operatorname{aff}}
\def\card{\op{card}}

\def\del{\partial}
\def\doct{\delta_{oct}}
\def\dtet{\delta_{tet}}
\def\hm{{h_0}} % 1.26
\def\stab{c_{{\scriptstyle \text{stab}}}} % 3.01
\def\tgt{\operatorname{\it{target}}}
\def\pqr#1{#1} % marks type (p,q,r).
\def\trunc#1#2{#1\hbox{\ensuremath{[:\hskip-0.25em plus 0em minus 0em{#2}]}}}
\def\trunc#1#2{#1[\text{:}\hskip0em plus 0em minus 0em{#2}]}
\def\trunc#1#2{d_{#2}{#1}}
%\def\trunc#1#2{#1{[\le\hskip-0.25em{ #2}]}}

%% HYPERMAP macros:
% avoid e for both hypermap edge and edge {v,w}
\def\ee{\varepsilonup}
\def\ocirc{}
\def\wild{{*}}  % wildcard char.

%% PACKNG macros:
%\def\lam{\lambda}
%\def\Lam{\Lambda}
\def\bu{{\underline{\u}}}
\def\bv{{\underline{\v}}}
\def\bw{{\underline{\w}}}
\def\bV{{\underline{V}}}
%\def\arcs#1#2#3{{\arcV(#1,\{#2,#3\})}}

%% LOCAL FAN macros:
\def\smain{S_{\scriptstyle\text{main}}} 
 
 

\setlength{\marginparwidth}{1.2in}
\def\mar#1{\marginpar{\raggedright\footnotesize #1}}

% line numbers
\def\lll{\resetlinenumber[1]}
\def\linenumberfont{\normalfont\small\sffamily}


 \crop
%\makeindex
\makeindex{index/Notation}
\makeindex{index/Index}

% CUP BOOK CLASS SPECIFIC
\theoremstyle{plain}
\newtheorem{theorem}{Theorem}[chapter]
\newtheorem{lemma}[theorem]{Lemma}


\newtheorem{background}[theorem]{Background}
\newtheorem{corollary}[theorem]{Corollary}
\newtheorem{example}[theorem]{Example}
\newtheorem{assumption}[theorem]{Assumption}
\newtheorem{interpretation}[theorem]{Interpretation}
\newtheorem{conjecture}[theorem]{Conjecture}

\theoremstyle{definition}
\newtheorem{definition}[theorem]{Definition}


\theoremstyle{remark}
\newtheorem{remark}[theorem]{Remark}
\newtheorem{notation}[theorem]{Notation}
\newtheorem{note}[theorem]{Author's Note}
\newtheorem{calculation}[theorem]{Calculation}
%%%%%%%%%%%%%%%%%%%%%%%%%%%%%%%%%%%%%%%%%%%%%%%%%%%%

\def\linput#1{\lll\input{#1}}

\begin{document}
\raggedbottom  % for now.
%\raggedright  % don't worry for now.

    \title[a blueprint for formal proofs]
    {%Flyspeck :
      Dense Sphere Packings}
    \author{Thomas C. Hales}
    
  
  %%%%%%%%%%%%%%%%%%%%%%%%%%%%%%%%%%%%%%%%%%%%%%%%%%%%  
%%% FRONT
    \frontmatter
    \maketitle
    \tableofcontents
    \thanks

%%%%%%%%%%%%%%%%%%%%%%%%%%%%%%%%%%%%%%%%%%%%%%%%%%%%


\noindent



\bigskip




\begin{note}%XX
This manuscript is not for ready general distribution.  Please do not circulate it.
\begin{enumerate}\wasitemize 
\item The computer calculations that back up various claims have not
  been completed.  In particular, various nonlinear inequalities
  remain to be proved.  The linear programs for one of the hypermaps
  have still not terminated.
\item The figures are missing.
\end{enumerate}\wasitemize 
\end{note}

\bigskip\noindent %
This research has been supported by the National Science Foundation
under Grants 0503447 and 0804189 as well as a grant from the Benter
Foundation.

\bigskip\noindent\svninfo 

\newpage


  \newpage
   %\setcounter{chapter}{-1}
   %%------------------------------------------------------------
% Author: Thomas C. Hales
% Format: LaTeX
% Book Chapter: Dense Sphere Packings
%------------------------------------------------------------

\chapter*{Preface}

% Believe not everything, but only what is proven: the former is foolish, the latter the act of a sensible man. -- Democritus.

%{
%
%\narrower
%
%{\it ``A personality which has been veiled by a formal method
%  throughout many chapters is suddenly seen face to face in the
%  Preface.'' }
%% Introductory note, p3, Famous Prefaces, Harvard Classics, Vol
%% 39. P.F. Collier & Son, 1910.
%
%}

%\bigskip

%\centerline{\it ``Those who justify themselves do not convince.''
%  --Lao-Tzu}
%% quoted in A. Watts's essay in "Modern Buddhism" ed. Donald Lopez,
%% page 160


{

\narrower

{\it

  ``I think there's a revolution in mathematics around the corner. I
  think that $\ldots$ %in later times
  people will look back on the fin-de-siecle of the twentieth century
  and say `then is when it happened' (just like we look back at the
  Greeks for inventing the concept of proof and at the nineteenth
  century for making analysis rigorous). I really believe that. And it
  amazes me that no one seems to notice $\ldots$

  ``Never before have the platonic mathematical world and the physical
  world been this similar, this close. Is it strange that I expect
  leakage between these two worlds? That I think the proof strings
  will find their way to the computer memories?$\ldots$

  ``What I expect is that some kind of computer system will be
  created, a proof checker, that all mathematicians will start using
  to check their work, their proofs, their mathematics. I have no idea
  what shape such a system will $\ldots$ take. But I expect some
  system to come into being that is past some threshold so that it is
  practical enough for real work, and then quite suddenly some kind of
  `phase transition' will occur and everyone will be using that
  system.''

{\hfill--Freek Wiedijk \cite{FWR}} % http://www.cs.ru.nl/~freek/jordan/index.html

}

}

\newpage

{

\narrower\parindent=0pt
\parskip=0.4\baselineskip

{\it

Alecos: Christos has a problem with the `foundational quest'!

Christos:  Wrong!  I have two problems with your  {\rm {version}} of it!  One, it
didn't fail and, two, it wasn't a tragedy!  Granted, there are some tragic
parts!  But the ending is happy, as in the `Oresteia'!  

Apostolos:  Happy for whom?  Cantor, going insane?  G\"odel starving himself to
death out of paranoia? Hilbert or Russell and their psychotic sons? Or Frege with--

Christos: `The meaning
is in the ending!' you said so yourself!  So, follow the quest for ten more years and
you get a brand-new triumphant finale with the creation of the computer, which is the
quest's real hero!   Your problem is, simply, that you see it as a story of people!

Apostolos: Well, stories do tend to be about people!

Christos:  So, choose the right people!  And show what they really did!  All we we learn
of the great von Neumann is he said `It's over'  when he heard G\"odel!

Alecos: But it was over in a sense, wasn't it?  Pop went Hilbert's `no ignorabimus'!

Christos:  But then came the quest's jeune premier, its parsifal $\ldots$ Alan Turing!
He said `Ok, we can't prove everything! So, let's see what we can prove!' and to define
proof, he invented, in 1936, a theoretical machine which contains all the ideas of the
computer! $\ldots$ which, after the war, he and von Neumann, the quest's proudest sons,
brought to full life!

{\hfill--Logicomix} % page 303.

}

}




\newpage

{

\narrower\parindent=0pt
\parskip=0.4\baselineskip

{\it

``Ever hear of the Kepler Conjecture?''

``Nope.''

I laid the notebook on the table and flipped through the pages. ``It was first stated
in 1611 by Johannes Kepler,'' I said.  ``Kepler becomae interested in the problem while
he was corresponding with an Englishman named Thomas Harriot, who was trying
to help his friend Sir Walter Raleigh figure out the best way to stack cannonballs on ship
decks.  The goal was to find the densest possible spherical arrangements, $\ldots$ basically,
the way grocers stack oranges''

``Okay,'' he said, nodding.

``Kepler's conjecture {\rm{seems}} perfectly sound,'' I said.

``That is does,'' Ben said.

``But here's the thing. $\ldots$  I looked it up and discovered that, in 1998, a proof
had finally been put forward by an American mathematician named Thomas Hales.  In 2003,
a committee that had been assigned to verify Hales's work confirmed that they were ninety-nine
percent certain of the proof's correctness.  But that one percent was key.  The mathematical
world is still waiting for the publication of the data that will prove the Kepler Conjecture definitively.''

``Sucks for Thomas Hales,'' Ben said.

``I agree.  But it makes sense that they have to be certain, doesn't it?''


{\hfill--Michelle Richmond, No One You Know} % page 177.

}

}

%%DD Figure of cannonballs.

\bigskip

{

\narrower\parindent=0pt
\parskip=0.4\baselineskip

{\it

Sometimes fixing a $1$ percent defect takes $500$ percent effort.

{\hfill-- Joel Spolsky, Joel on Software} % page 122

}

}

\bigskip

{

\narrower\parindent=0pt
\parskip=0.4\baselineskip

{\it

Every one fully persuaded is a fool.

{\hfill-- Balthasar Graci\'an, the Art of Worldly Wisdon} % p110

}

}



\newpage

\section*{Blueprint for Formal Proofs}

In 1611, Kepler wrote a booklet in which he asserted that the familiar
cannonball arrangement of congruent balls in space achieves the
highest possible density.  No other arrangement fills a larger
fraction of space.  This assertion is the Kepler conjecture.  In 1900,
Hilbert made this conjecture part of his eighteenth problem.  This
book presents a proof of this assertion.

This assertion has become a test of the capability of computers to
deliver a reliable mathematical proof.  The original proof by Sam
Ferguson and me involved many long computer calculations that
exhausted the efforts of a team of referees.  This book represents my
efforts to redesign the proof in a way that makes the correctness of
the computer proof as transparent as possible.

After all is said and done, a proof is only as reliable as the
processes that are used to verify its correctness.  The ultimate
standard of proof is a formal proof.  A formal proof is nothing other
than an unbroken chain of logical inferences from an explicit set of
axioms.  While this may be the mathematical ideal of proof, actual
mathematical practice generally deviates significantly from the ideal.



%More than ten years have passed since a proof was first
%obtained. Why give a new presentation of the proof?
%
%The original proof was not widely understood.  The complexity was not
%because of conceptual challenges.  In fact, the proof makes only
%modest demands on the theoretical training of the reader.  It is
%possible to read and understand the proof with a knowledge of a
%limited body of mathematics, such as basic calculus and elementary
%Euclidean geometry.
%
%Nevertheless, the proof involves many long calculations. Even worse,
%it it relies on computer calculations.  An error in any calculation or
%a bug in the computer code has the potential to topple the entire
%proof.
%
%The referees were conscientious and checked many of the calculations.
%However, for the most part, the computer code lay beyond the scope of
%referee review, and even careful quality control can let a
%occasional bug slip through undetected.
%
%After all is said and done, no proof is more reliable than the
%reliability of the processes that are used to verify its
%correctness.  These processes include the checking that the author
%makes before releasing the proof for public scrutiny, the checking
%of the referees, and the checking done by readers after publication.

In recent years, as part of this project, I have been increasingly preoccupied by the
processes that mathematicians rely on to insure the correctness of complex
proofs. Researchers from Frege to G\"odel, who solved a problem of
rigor in mathematics, found a theoretical solution but did not
extinguish the burning fire at the foundations of mathematics
because they omitted the practical implementation. Some, such as
Bourbaki, have even gone so far as to claim that ``formalized
mathematics cannot in practice be written down in full'' and call
such a project
``absolutely unrealizable'' \cite[p 10,11]{Bour:68:Sets}. % Theory of
                                                          % Sets, page
                                                          % 10,11.

While it is true that formal proofs may be too long to print,
computers -- which do not have the same limitations as paper -- have
become the natural host of formal mathematics. In recent decades,
logicians and computer scientists have reworked the foundations of
mathematics, putting them in an efficient form designed for real use
on real computers.

For the first time in history, it is possible to generate and verify
every single logical inference of a major mathematical theorem.  This
has now been done for the four-color theorem, the prime number
theorem, the Jordan curve theorem, the Brouwer fixed point theorem,
and the fundamental theorem of calculus, among others.  Freek Wiedijk
reports that 82\% of a list of 100 famous theorems have now been
checked formally \cite{wiedijk:100}.  The list of 18 remaining
theorems contains two particular challenges: the independence of the
Continuum Hypothesis and Fermat's Last theorem.

Some mathematicians remain skeptical of the process because computers
have been used to generate and verify the logical inferences.
Computers are notoriously imperfect, with flaws ranging from software
bugs to defective chips.  Even if a computer verifies the inferences,
who will verify the verifier, or then verify the verifier of the
verifier?  Indeed, it would be unscientific of us to place an
unmerited trust in computers.

The choice comes down to two competing verification processes.  The
first is the traditional process of referees, which depends largely on
the luck of the draw -- some referees are meticulous, others are
careless.  The second process is formal computer verification, which
is less dependent on the whims of a particular referee.  In my view,
the choice between the conventional referee process and computer
verification is as evident as the choice between a sundial and an atomic
clock in contemporary science.

The boundary that separates an ``easy'' proof from a ``difficult''
proof shifts with current technology.  The introduction of steel in
architecture is not a mere reinforcement of wood and stone, it changes
the architect's world of possibilities.  There will no longer be any
reason to limit ourselves to ten-thousand-page proofs when our
technology supports million-page proofs.

The standard of proof I have adopted is the highest scientific standard
available by current technology.  That 
standard is formal verification by computer.  This standard
continues to evolve with the advancement of technology.

I dream of a fully formally verified solution to the
packing problem.  This project is still unfinished, but significant
progress is being made.  In this book, I rearrange the proof with
formal verification in mind .  The book is {\it a blueprint for formal
  proofs} because it gives the design of the formal proof to be
constructed.  My decisions about what to include in this book has been
shaped by the list of theorems already available in the library the
proof assistant {\tt HOL Light}.  For example, this book assumes basic
point-set topology and measure theory, which have been formalized by
John Harrison~\cite{HOLL}.

The style of formal proofs is different from that of conventional
proofs.  It is better to have a large number of short snappy proofs,
rather than a few intricate ones.  Humans enjoy surprising new
perspectives, but computers benefit from repetition and
standardization.  Despite these differences, I have worked to make
proofs that will bring pleasure to the human reader while providing
precise instructions for the implementation in silicon.



\section*{Structure of this Book}

The book is divided into parts.
The introductory part describe the major ideas, methods, and
organization of the proof.  

%There is an essay on each major computer
%component of the proof. The purpose is to provide a panoramic view of
%proof, to provide intuition about proof strategies.  After reading
%this part of the book, the reading should understand what the proof is
%all about, without yet dipping into technical details.
The part on foundations provides background material about
constructions in discrete geometry.    The first
of these chapters
covers trigonometric identities and basic vector geometry.  The second
treats volume from an elementary point of view.  The third chapter
covers planar graph theory from a purely combinatorial point of view.
The fourth chapter continues with planar graphs, now from a 
geometric point of view.

The next part of the book gives the solution to the packing problem.
The first chapter in this part gives a top-level overview of the major
steps of the proof.  It describes how the problem can be reduced from
a problem in infinitely many variables to a problem in finitely many
variables.  The remaining chapters in this part flesh out that
skeleton.

The final part of the book resolves some other longstanding conjectures in
discrete geometry: K. Bezdek's strong dodecahedral conjecture and Fejes
T\'oth's full contact conjecture.

Many simplifications of the original proof have been found over the past
several years.  The simplified proof is published here for the first time.
G. Gonthier expresses his formal proof of the four-color
theorem in terms of hypermaps.  He reworks the proofs of the
four-color theorem to avoid the use of the Jordan curve theorem, using
instead the much simpler notion of M\"obius contour.  I have followed
Gonthier's lead in these respects and also avoid the use of the Jordan curve theorem.

The optimality of the face-centered cubic packing is an assertion
about infinite space-filling packings.  For computational purposes, it is
useful to reduce the sphere packing problem to finite packings.  A
{\it correction term} is associated with each different reduction from
infinite packings to finite packings.  S. Ferguson and I considered a
large number of different correction terms.  We searched for one that
would simplify the computations as much as possible.  In a discussion
of the solution of the packing problem, I wrote that ``correction
terms are extremely flexible and easy to construct, and soon Samuel
Ferguson and I realized that every time we encountered difficulties in
solving the minimization problem, we could adjust $f$ [the correction
term] to skirt the difficulty. $\ldots$ If I were to revise the proof
to produce a simpler one, the first thing I would do would be to
change the correction term once again.  It is the key to a simpler
proof.''  C. Marchal has recently found a very simple 
correction term, that is, a simple way  to make the reduction from infinite packings
to finite packings.~\cite{marchal:2009}.  We use his reduction in this book.

There are many other improvements of the proof that are not visible in
the book, because they are implemented in computer code.  We have been
able to reduce the number of lines of computer code from over 187,000
to well under 10,000.  Needless to say, this significantly simplifies the 
formalization project.





\bigskip
\hbox{}



\bigskip
\hbox{}

{
\parindent=0pt
\obeylines

Thomas C. Hales
Pittsburgh, PA
May 2010

}








 



%%%%%%%%%%%%%%%%%%%%%%%%%%%%%%%%%%%%%%%%%%%%%%%%%%%%
  \mainmatter  
    %\part[Lectures]{Lectures}
    %%------------------------------------------------------------
% Author: Thomas C. Hales
% Format: LaTeX
% Book Chapter: Dense Sphere Packings
%------------------------------------------------------------


\chapter{Close Packing}\label{sec:close}

\section{History}\label{sec:history}

This section gives a brief history of the study of dense sphere
packings.  Further details appear at \cite{Szpiro} and
\cite{Hales:2006:overview}.
The early history of sphere packings is concerned with the
face-centered cubic (FCC) packing, a familiar pyramid arrangement
of congruent balls used to stack cannonballs at war memorials and
oranges at fruit stands (Figure~\ref{fig:fcc-packing}).

\figDHQRILO % fig:fcc-packing


\subsection{Sanskrit sources}



The study of the mathematical properties of the FCC
packing can be traced\footnote{I am obliged to Plofker~\cite{Plo00}.} to a Sanskrit work (the \=Aryabha\d t\={\i}ya
 of \=Aryabha\d ta) composed around 499 CE.  The following passage gives
the formula for the number of balls in a pyramid pile with triangular base as
a function of the number of balls along an edge of the pyramid~\cite{Ary}:

% {\it \=Aryabha\d t\={\i}ya}, Ga\d nitap\=ada 21:

\bigskip

{\narrower\it\font\ninerm=cmr9

For a series [lit. ``heap''] with a common difference and
  first term of 1, the product of three [terms successively] increased
  by 1 from the total, or else the cube of [the total] plus 1
  diminished by [its] root, divided by 6, is the total of the pile
  [lit. ``solid heap''].  

}

\bigskip

 In modern notation, the passage gives two formulas for the number of
 balls in a pyramid with $n$ balls along an edge (Figure~\ref{fig:sanskrit}):
\begin{equation}\label{eqn:sanskrit}
\dfrac{n(n+1)(n+2)}{6} =  \dfrac{(n+1)^3 - (n+1)}{6}
\end{equation}

\figKSOEMIZ % fig:sanskrit

\subsection{Harriot and Kepler}

The modern mathematical study of spheres and their close packings can
be traced to Harriot.  His work -- unpublished, unedited, and largely
undated -- shows a preoccupation with sphere packings.  He seems to
have first taken an interest in packings at the prompting of Sir
Walter Raleigh.  At the time, Harriot was Raleigh's mathematical
assistant, and Raleigh gave him the problem of determining formulas
for the number of cannonballs in regularly stacked piles.  Harriot
interpreted the number of balls in a pyramid as an entry in Pascal's
triangle\footnote{Harriot was well-versed in Pascal's triangle long
  before Pascal.} (Figure~\ref{fig:pascal}). Through his study of
triangular and pyramidal numbers, Harriot later discovered finite
difference interpolation~\cite{BeS08}.  Shirley, Harriot's biographer,
writes that it was his study of cannonball arrangements in the late
sixteenth century that ``led him inevitably to the corpuscular or
atomic theory of matter originally deriving from Lucretius and
Epicurus'' \cite[p.~242]{Shi83}.

\figBDCABIA % fig:pascal

Kepler became involved in sphere packings through his correspondence
with Harriot around 1606--1607 on the topic of optics.
Harriot, the atomist, attempted to understand reflection and refraction
of light in atomic terms.  Kepler favored a more classical explanation of
reflection and refraction in terms of what Kargon describes as ``the union of two opposing
qualities -- transparence and opacity''~\cite[p.26]{Kar66}.  
Harriot was stunned that
Kepler would be satisfied by such reasons.

Despite Kepler's initial reluctance to adopt an atomic
theory, he was eventually swayed and  published an essay in  1611
that explores the consequences of a theory of matter composed of small
spherical particles. 
Kepler's essay describes the FCC packing and asserts
that ``the packing will be the tightest possible, so that in no other
arrangement could more pellets be stuffed into the same
container''~\cite{Kep66}.  This assertion has come to be known as the
Kepler conjecture.  This book
gives a proof of this conjecture.

\subsection{Newton and Gregory}

The next episode in the history of this problem,  a debate between
Isaac Newton and David Gregory,  centered on the
question of how many congruent balls  can be arranged to touch
a given ball.  The analogous question in two dimensions is readily answered;
six pennies, but no more, can be arranged
to touch a central penny.  In three dimensions, Newton said that the maximum was
twelve balls, but Gregory claimed that thirteen might be possible.

The Newton-Gregory problem was not solved until centuries later
(Figure~\ref{fig:musin}).  The first proper proof was obtained by van
der Waerden and Sch\"utte in 1953 \cite{Sch53}.  An elementary proof
appears in Leech \cite{Leech:1956:MG}.  Although a connection between
the Newton-Gregory problem and Kepler's problem is not obvious, Fejes
T\'oth successfully linked the problems in 1953~\cite{Fej53}.

\figPTFTWZM % fig:musin

\section{Face-Centered Cubic}



The FCC packing is the familiar pyramid arrangement of
balls on a square base as well as a pyramid arrangement on a
triangular base.  The two packings
differ only in their orientation in space.
Figure~\ref{fig:tri-square} shows how the triangular base
packing fits between the peaks of two adjacent square based pyramids.

\figNTNKMGO % fig:tri-square

Density, defined as a ratio of volumes, is insensitive to changes of
scale.  For convenience, it is sufficient to consider balls of unit
radius. This means that the distance between centers of balls in a
packing is always  at least $2$.  We identify a packing with its set $V$
of centers.   For our purposes, a packing is just a set of points
in $\ring{R}^3$ in which the elements are separated by distances of at least
$2$.



The density of a packing is the ratio of the volume occupied by the
balls to the volume of a large container.  The
purpose of a finite container is to prevent the volumes from becoming
infinite.  To eliminate the distortion of the packing caused by the
shape of the its boundary, we take the limit of the densities within an increasing
sequence of spherically shaped containers, as the diameter tends to infinity.

The FCC packing is obtained from a cubic lattice, by inserting a ball
at each of the eight extreme points of each cube and then inserting a
another ball at the center of each of the six facets of each cube
(Figure~\ref{fig:face-centered-cubic}).  The name
\fullterm{face-centered cubic}{FCC} comes from this construction.  The edge
of each cube is $\sqrt8$, and the diagonal of each facet is $4$.  The
density of the packing as a whole is equal to the density within a
single cube.  The cube has volume $\sqrt8^3$ and contains a total of
four balls: half a ball along each of six facets and one eighth a ball
at each of eight corners.  Thus, the density within one cube is
   \[ 
   \frac{   4 (4\pi/3)}{\sqrt8^3} = \frac{\pi}{\sqrt{18}}.
   \] 


\figTCFVGTS % fig:face-centered-cubic


%The tiling of regular tetrahedra and octahedra can be
%superimposed on the picture of the cube.  Each tetrahedron has an extreme point
%in common with the cube and three other extreme points at centers of facets
%of the cube.   One octahedron is concentric with the cube and has an extreme
%point at the center of each facet.  There is an
%additional quarter of an octahedron along each edge of the cube, extending to the
%midpoints of the two adjacent facets, making a total of eight
%tetrahedra and four octahedra.  As each octahedron has the volume of
%four tetrahedra, exactly $1/3$ of the cube is filled with tetrahedra,
%the other $2/3$ with octahedra.  This decomposition shows that the
%volume of a tetrahedron is $2\sqrt2/3$.
%%(pretend ignorance). The volume $16\sqrt2$ 
%%of the cube equals $24$ tetrahedra \dots, giving each a volume
%%of 
%%$2\sqrt{2}/3$.

The density $\pi/\sqrt{18}$ of the packing is the ratio of the volume
$4\pi/3$ of a ball to the volume of a fundamental domain of the FCC
lattice.  The volume of the fundamental domain is therefore
$4\sqrt{2}$.  A fundamental domain of the FCC lattice is a
parallelepiped that can be dissected into two regular tetrahedra and
one regular octahedron (Figure~\ref{fig:fcc-fun-domain}).  The FCC
packing is then an alternating tiling by tetrahedra and octahedra in
2:1 ratio.  A tetrahedron scaled by a factor of two consists of one
tetrahedron at each extreme point and one octahedron in the center
(Figure~\ref{fig:tet-oct-ratio}). By similarity, the total volume is
$8 = 2^3$ times the volume of each smaller tetrahedron. This
dissection exhibits the volume of a regular octahedron as exactly four
times the volume of a regular tetrahedron of the same edge length.  As
a result, the volume of a regular tetrahedron of side $2$ is $1/6$ the
volume of the fundamental domain, or $2\sqrt{2}/3$.

\figSEYIMIE % fig:fcc-fun-domain

\figAZGXQWC % fig:tet-oct-ratio

The density of the FCC packing is the weighted density
of the densities of the tetrahedron and octahedron.  Write $\dtet$ and
$\doct$ for these densities.  Explicitly, $\dtet$ is the ratio of the
volume of the part within the tetrahedron of the unit balls (at the
four extreme points) to the full volume of the tetrahedron.  As tetrahedra fill
$1/3$ of volume of the fundamental domain and an octahedron fills
the other $2/3$,
\[ 
  \frac{\pi}{\sqrt{18}} = \frac{1}{3}\dtet + \frac{2}{3}\doct.
\] 

As above, we identify a packing with the set $V$ of centers of the
balls.  The \fullterm{Voronoi cell}{decomposition!Voronoi} 
of a point $v$ in a packing $V$ is
defined as the set of all points in $\ring{R}^3$ (or more generally in
$\ring{R}^n$) that are at least as close to $v$ as to any other point
of $V$ (Figure~\ref{fig:voronoix}).  Each Voronoi cell of the FCC
packing is a rhombic dodecahedron
(Figure~\ref{fig:rhombic-dodec}),
which is constructed from an inscribed cube by placing a square based pyramid
(with height half as great as an edge of its square base) on each of
the six facets.

\indy{Index}{Voronoi|see{decomposition}}%

\figEVIAIQPx % fig:voronoix

\figPQJIJGE % fig:rhombic-dodec

%Recall that the cubes under discussion  have an edge length
%$\sqrt{2}$.
Rhombic dodecahedra, being the Voronoi cells of the FCC packing, tile space.
In each rhombic dodecahedron, we 
may color the inscribed cube black and the six square-based pyramids
white.  In the tiling, 
the black cubes fill the black spaces of an infinite three-dimensional
checkerboard, and the white pyramids fill the white spaces.

A Voronoi cell contains an inscribed black cube of side $\sqrt2$ and a total
of one white cube, for a total volume of $4\sqrt2$, which is
again the volume of the fundamental domain.  The density of the
FCC packing is the ratio of the volume of a ball to the volume
of its Voronoi cell, which gives $\pi/\sqrt{18}$ yet again.



\section{Hexagonal-Close Packing}\label{sec:hcp}

There is a popular and persistent misconception that the FCC
 packing is the only packing with density $\pi/\sqrt{18}$.
The hexagonal-closed packing (HCP) has the same density.
\indy{Index}{HCP}%
\indy{Index}{FCC}%


In the FCC packing, each ball is tangent to twelve others in the same
fixed arrangement.  We call it the \fullterm{FCC
  pattern}{FCC!pattern}.  Likewise, in the HCP, each ball is tangent
to twelve others in the same arrangement
(Figure~\ref{fig:fcc-hcp-pattern}).  We call it the \fullterm{HCP
  pattern}{HCP!pattern}.  The FCC pattern and HCP patterns are
different from each other.  In the FCC pattern, four different planes
through the center give a regular hexagonal cross section, while the
HCP pattern has only one such plane.

\figSGIWBEN % fcc-hcp-pattern

There are, in fact, uncountably many packings of density
$\pi/\sqrt{18}$ in which the tangent arrangement around each ball is
either the FCC pattern or the HCP pattern.

A \newterm{hexagonal layer} is a translate of the
two-dimensional hexagonal lattice (also known as the triangular
lattice). That is, it is a translate of the planar lattice generated
by two vectors of length $2$ and angle $2\pi/3$.  The FCC
 packing is an example of a packing built from hexagonal layers.

 If $L$ is a hexagonal layer, then a second hexagonal layer $L'$ can be
 placed parallel to the first so that each lattice point of $L'$ has
 distance $2$ from three different points of $L$,
%When the second
% layer is placed in this manner, 
 which is the smallest possible distance from first layer.  A choice
 of a unit normal vector $\e$ to the plane of $L$ determines an upward
 direction.  There are two different positions in which $L'$ can be
 closely placed above $L$
% and two different positions in which
% $L'$ can be placed closely below $L$ 
(Figure~\ref{fig:hex-layers}).  Each successive layer 
  ($L$, $L'$, $L''$, and so
 forth) offers two further choices  for the placement of
 that layer. Running through different 
 sequences of choices gives uncountably many packings.  In each of
 these packings the tangent arrangement around each ball is the FCC or HCP arrangement.

\figCCQCYWU % fig:hex-layers

As a packing is constructed, each layer may be labeled
$A$, $B$, or $C$ depending on three possible orthogonal projections to a fixed plane
with normal vector $\e$.
Each layer carries a different label from the layers immediately
above and below it.  In the FCC packing, the successive layers are
$A,B,C,A,B,C$, and so forth.  In the HCP packing, the successive layers are
$A,B,A,B$, and so forth.  If the vertices of a triangle are labeled $A$, $B$, and $C$,
then the succession of labels is a
walk along the vertices of the triangle, and inequivalent walks through the
triangle describe different packings.


The different walks through a triangle give all possible packings of
infinitely many congruent balls in which each tangent arrangement is
either the FCC pattern or the HCP pattern~\cite{CoSl95}.  To see that
there are no other possibilities, we first assume that every ball of
$V$ is surrounded by the FCC pattern.  Adjacent FCC patterns interlock
in a unique way that forces $V$ itself to crystallize into the FCC
packing.  This completes the proof in this case.

Now we assume that a packing $V$ contains some ball (centered at $\u$)
in the HCP pattern. Its uniquely determined
plane of reflectional symmetry contains $\u$ and the centers of six
others arranged in a regular hexagon. If $\v$ is the center of one of
the six other balls in the plane of symmetry, its  tangent arrangement
of twelve balls must include $\u$ and an additional four of the
twelve balls around $\u$. These five centers around $\v$ are not a
subset of the FCC pattern, but  extend uniquely to
a HCP pattern.   Around $\u$ and $\v$, the HCP patterns  have the same
plane of symmetry. In this way, as
soon as some center has the HCP pattern, the pattern
propagates along the plane of symmetry to create a hexagonal layer
$L$.

Once a packing $V$ contains a single hexagonal layer, the condition
that each ball be tangent to twelve others forces a hexagonal layer
$L'$ above $L$ and another hexagonal layer below $L$.  Thus, a single
hexagonal layer forces an infinite sequence of close-packed hexagonal
layers.  The position of each layer over the previous layer is described 
by the labels $A$, $B$, and $C$ of the triangle.
This completes the proof that the different walks through a triangle give
all possibilities.



\section{Gauss}

Gauss proved that the FCC packing has the greatest density
of any lattice packing in three-dimensional Euclidean space.  There is a
short proof that does not require any calculations.

\begin{proof}
Start with an arbitrary lattice $V$ in which every point has distance 
at least $2$ from every other.  Center a unit ball at each point in
the lattice.  In a lattice of greatest density, some pair of balls
touch.  The lattice property then forces the balls into parallel
infinite linear strings like beads on a string.  Two of these infinite
parallel strings touch if the lattice is  optimal.  The
lattice property then constrains the strings in parallel sheets.  On
each sheet the touching parallel strings form a rhombic tiling.  Each
parallel sheet sits as snugly as possible on the sheet below in an optimal
lattice.  In such an arrangement, a ball (centered at
$\v_0$) of one sheet touches three balls (centered at
$\v_1,\v_2,\v_3$) on a the next layer down (Figure~\ref{fig:rhombus}).

\figAFRJFRK % fig:rhombus


As the balls on each sheet form a rhombic tile, two of the distances
between $\v_1,\v_2,\v_3$, corresponding to two edges of the rhombus, are
equal to $2$.  This means that $\v_0$ together with two of
$\v_1,\v_2,\v_3$ form an equilateral triangle.  

From the perspective of the plane containing this equilateral
triangle, the lattice property forces this entire plane, as well as
parallel planes, to be tiled with equilateral triangles.  From the
earlier argument, each of these planes sits as snugly as possible on
the sheet below.  A ball of one sheet touches the three balls in an
equilateral triangle on the layer below.  These four balls form a
regular tetrahedron, which uniquely identifies the lattice as the FCC.
\end{proof}






\section{Thue}\label{sec:thue}


As mentioned in the preface, Thue solved the packing problem for
congruent disks in the plane.  The optimal packing is the hexagonal
packing (Figure~\ref{fig:2D-hex}).  The density of this packing is
$\pi/\sqrt{12}$, that is, the ratio of the area of a unit disk to the
area of a hexagon of inradius one.  Thue's theorem admits an
elementary proof that we sketch.  Casselman has an 
interactive demo of this solution \cite{casselman:pennies}.

\figOCULYIA % fig:2D-hex

\begin{proof}
Let $V$ be the set of centers of a collection of unit disks in
$\ring{R}^2$.  Take the Voronoi cell around each
disk.\footnote{Voronoi cells of packings in any dimension $\ring{R}^n$
  are defined by the same rule as we gave above for $\ring{R}^3$.}   It
is enough to show that each Voronoi cell has density at most
$\pi/\sqrt{12}$ because the limiting density of the packing in the entire plane cannot exceed
a bound on the density within a Voronoi cell.  


Truncate the Voronoi cell by intersecting it with a disk of radius
$r=2/\sqrt3$.   The density increases as the volume of the cell is made smaller,
so if the truncated Voronoi cell
has density at most $\pi/\sqrt{12}$, then so does the untruncated Voronoi cell.

There is not a point $\w$ in the plane that has distance  less than $r$
from three disk centers $\v_1,\v_2,\v_3$.  Otherwise, one of the three
angles $\gamma$ at $\w$ formed by pairs $(\v_i,\v_j)$ of points
 is at most $2\pi/3$, and $\cos\gamma\ge -0.5$.
The \newterm{law of cosines} applied to the triangle $\w,\v_i,\v_j$ with angle
$\gamma$ and sides $a$, $b$, and $c$ gives the contradiction
   \[ 
   4 \le c^2 = a^2 + b^2 - 2 a b \cos\gamma 
   \le a^2 + b^2 + a b < 3r^2 = 4.
   \] 
Thus, the boundary of the truncated Voronoi cell consists of circular
arcs and chords of the circle of radius $r$, as shown in Figure~\ref{fig:2D-proof}.

\figSENQMWT % fig:2D-proof

The parts of the Voronoi cell that lie within a circular sector have
density $1/r^2 = 3/4 < \pi/\sqrt{12}$.  A simple calculation shows
that the part of a Voronoi cell that lies within a triangle has
density
   \begin{equation}\label{eqn:rog2d}
   \frac{\theta}{r^2 \cos\theta\sin\theta}
   \end{equation}
% checked 3/31/2008.
for some $0 \le \theta\le \pi/6$.  An easy optimization gives the maximum
at $\theta=\pi/6$ with value $\pi/\sqrt{12}$.  This completes the proof of Thue's theorem.
\end{proof}

In some ways it it unfortunate that the problem in two dimensions is
so elementary.  It gives only meager hints about how to solve the problem
in three dimensions such as the value of Voronoi cells
and the usefulness of truncation.  The optimization problem on
triangles in Equation~\ref{eqn:rog2d} generalizes to $n$-dimensions.
But beyond these simple observations,  little from the proof of Thue's
theorem prepares us for higher dimensions.

%\subsection{two dimensions}

\bigskip

\indy{Index}{decomposition!Delaunay}%

There are other proofs of Thue's theorem, including one by Fejes
T\'oth that uses the \fullterm{Delaunay triangulation}{decomposition!Delaunay} 
of a packing $V$
in the plane (or in $n$-dimensions).  A Delaunay triangulation of $V$
is a triangulation of Euclidean space into simplices with extreme
points in $V$ such that no point of $V$ lies in the interior of any
circumscribing circle of any of the simplices (Figure~\ref{fig:delaunay}).  
If $V$ is
\newterm{saturated},\footnote{A packing $V$ is saturated if it is not
  a proper subset of any other packing $V'$.  To maximize density, it
  is useful to increase the density by saturating the packing with
  additional points.} then a Delaunay triangulation of
$V$ exists.  Each Delaunay triangle in a saturated packing $V$ has
circumradius at most $2$ because otherwise an additional point can be
placed at the center of the circumscribing circle, contrary to saturation.

\figANNTKZP % fig:delaunay

\begin{proof}
  By admitting the existence of a Delaunay triangulation, the proof of
  the packing problem for saturated packings $V$ in two dimensions becomes
  elementary.  Each Delaunay triangle contains a portion of a disk at each of
  its three vertices.  The three interior angles of a triangle sum to
  $\pi$, giving half a disk per triangle.  If we show that each triangle has area
  at least $\sqrt{3}$, then it follows that the density of the packing is at most
  $(\pi/2)/\sqrt{3} = \pi/\sqrt{12}$.  The problem thus reduces to
  an area minimization problem.  To decrease the area of a triangle
  $\{\v_0,\v_1,\v_2\}$, we first replace it with a smaller similar
  triangle with shortest edge (say $\v_1\v_2$) of length $2$.  The
  third vertex $\v_0$ is constrained to have distance at least $2$
  from $\v_1$ and $\v_2$, and to have circumradius at most $2$.  The
  constraints on $\v_0$ form three circular arcs as shown in
  Figure~\ref{fig:delaunay-proof}.

\figCCKQLLH  % fig:delaunay-proof


The minimizing triangle is determined by the point $\v_0$ closest to
the line through $\v_1$ and $\v_2$.  There are three such triangles,
each with area exactly $\sqrt3$.  This completes the proof.
\end{proof}

\section{Dense Packings in a Nutshell}

This section describes the proof of the Kepler conjecture in general,
without
getting embroiled in detail.  The entire book
is a blueprint with all the electrical schematics, plumbing, and
ventilation systems.  This section is the tourist brochure.

The Kepler conjecture asserts that no packing of congruent balls in
three-dimensional Euclidean space has density greater than the density
$\pi/\sqrt{18} \approx 0.74048$ of the FCC packing.  For a contradiction, we suppose that an
explicit counterexample exists to the Kepler conjecture in the form of a
packing of balls of radius $1$ with density  greater than
$\pi/\sqrt{18}$.  Additional balls may be added to this packing until
saturation is reached.  The saturation of a counterexample may push its
density even higher.

We present the proof in four stages.  Undefined terms are clarified in the
discussion that follows.

\begin{enumerate}
\item A geometric partition of space, adapted to a saturated
  counterexample $V$, reduces the problem to finite packing $W$ that
  gives a counterexample to a particular inequality.  In notation
  established below, the particular inequality is $\CalL(W,\orz)\le
  12$ for every finite packing $W\subset B(\orz,2.52)$.  The
  counterexample satisfies $\CalL(W,\orz)>12$.
\item The finite packing $W$ is transformed into another finite packing that violates the same
inequality and that has a few additional properties that make it a \newterm{contravening} packing.
\item The combinatorial structure of $W$ is encoded as a hypermap.  A list is
  made of the purely combinatorial properties of $W$.  A hypermap with
  these properties is said to be \newterm{tame}.
\item A computer generates an explicit list, enumerating
  tame hypermaps up to isomorphism.  Linear programs, which are
  adapted to each tame hypermap in the enumeration, certify that
  none of the combinatorial possibilities can be  realized geometrically as a finite packing
  $W\subset \ring{R}^3$.  
\end{enumerate}
\indy{Notation}{L1@$\CalL(V)$ (estimation of a packing)}%

From the nonexistence of a counterexample $W$, 
it follows that there is no saturated
counterexample $V$ to the Kepler conjecture.



\subsection{geometric partition}

The first stage of the proof defines a geometric partition of space
and uses it to reduce the Kepler conjecture to an optimization problem
in a finite number of variables.

We recall that a saturated packing is
identified with the discrete set $V$ of centers of the
congruent balls.  Also, as above, the Voronoi cell $\Omega(V,\v)$
associated with $\v\in V$ is the polyhedron formed by all points of
$\ring{R}^3$ that are at least as close to $\v$ as to any other $\w\in
V$.   

The Voronoi cell at $\v$ can be further partitioned into Rogers
simplices, each of which is determined by a facet of the Voronoi cell, an edge of
the facet, and an extreme point of the edge.  The Rogers simplex is defined to be the
convex hull of four points: $\v\in V$, the closest point $\v_1$ to $\v$ on the given facet, the closest point $\v_2$ to $\v_1$ on the edge, and the
extreme point $\v_3$ of the edge (Figure~\ref{fig:rogers-intro}).

\figORQISJR % fig:rogers-intro

We dissect and combine the Rogers simplices somewhat further to make
them into \fullterm{Marchal cells}{Marchal cell} (Figure~\ref{fig:marchal-intro}).  
The exact rules for the
construction of Marchal cells do not concern us here.  The rules
depend on which of the points $\v_1,\ldots,\v_3$ have distance less
than $\sqrt2$ from $\v$.
\indy{Index}{decomposition!Marchal}%

\figODGBUWK % fig:marchal-intro

The function
 $\CalL(V,\v)$ is defined as
\begin{equation}\label{eqn:LV0}
\CalL(V,\v) = \sum_{\w\in V} L(\norm{\w}{\v}/2),
\end{equation}
where $L$ is the piecewise linear function that has a linear graph from
$(x,y)=(1,1)$ to $(0,1.26)$ and is equal to zero for $x\ge 1.26$.  (The
constants $1.26$ and $2.52=2(1.26)$ appear throughout the proof as
parameters used in truncation.)   
The sum in the definition of $\CalL$ is actually finite for every packing $V$ because only finitely many terms
lie in the support of $L$. 

Next, a function $G:V\to \ring{R}$ is defined geometrically in terms
of the volumes, solid angles, and dihedral angles of Marchal cells.
We do not give the definition here because it is rather
complex.  The function $G$ has the following two fundamental
properties:
\begin{enumerate}
\item If $\CalL(V,\v)\le 12$, then 
\[
4\sqrt{2}\le \Omega(V,\v) +G(\v).
\]
\item There exists $C>0$ such that the points of $V$ in a ball $B(\orz,r)$
of radius $r\ge 1$ satisfy
\[
\sum_{\v\in V \cap B(\orz,r)} G(\v) < C r^2.
\]
\end{enumerate}
The constant $4\sqrt{2}$ is the volume of the Voronoi cell of the FCC packing.

From these fundamental properties and from the assumption that $V$ is a saturated counterexample,
it follows that $\CalL(V,\v)>12$ for some $\v\in V$.  Indeed, if $\CalL(V,\v)\le 12$ for all
$\v\in V$, then the fundamental properties
imply that on average the Voronoi cells of $V$ have volume at least that of the FCC packing, up to a negligible error term $C r^2$.  From this, it follows that the density
of the packing $V$ is at most that of the FCC packing.



Returning to the counterexample $V$, we pick $\v\in V$ such that
$\CalL(V,\v)>12$.  By the translational invariance of the problem, we
may assume that $\v=\orz$.  Then
\begin{equation}\label{eqn:LW}
\CalL(W,\orz) = \sum_{\w\in W} L(\normo{\w}/2)  > 12,
\end{equation}
where $W$ is the finite set  $\{\w\in V\mid 0 < \normo{\w} \le 2.52\}$.

This completes the first stage of the proof.
%, by reducing the Kepler conjecture to an optimization
%problem $\max_W \CalL(W,\orz) \le 12$ in a finite number of variables $W\subset B(\orz,2.52)$.
The counterexample $V$ to the Kepler conjecture leads to a finite packing $W$
that satisfies~\eqref{eqn:LW}.

\subsection{contravening packing}

We assume that $V$ is a counterexample to the Kepler conjecture and
that $W\subset V$ is a finite subset that satisfies \eqref{eqn:LW}.
The second stage of the proof shows that the finite packing $W$ can be
enhanced in various ways.  The result of the enhancement is a new
finite packing that is a \fullterm{contravening
  packing}{contravening}.  At this stage, we also make $W$ into a
graph by defining a set of edges $E$ with nodes in $W$.

For example, the value of $\CalL$
depends only on the norms $\normo{\w}$, and $L$ is a decreasing
function, so that any rearrangement of the points of $W$ that does not
increase the norms strengthens the inequality \eqref{eqn:LW}.

The finite packing $W$ determines a graph $(W,E)$ with node set $W$.  The set
of edges is defined by $\{\v,\w\}\in E$ if 
\[2\le\norm{\v}{\w}\le 2.52.\] This graph is called the \newterm{standard
  fan} of $W$.

We can get a crude idea about what $W$ must look like by studying the
set of normalized points $\w/\normo{\w}$ in the unit sphere.  These
points can be used to partition the unit sphere into spherical
polygons.  As we know that the sum of the areas of the polygons equals
the area $4\pi$ of the sphere, we can extract bits of information
about $W$ from estimates of the areas of the polygons.  Analysis along
these lines leads to the conclusion that some finite packing $W$
has the following 
properties:
\begin{enumerate}\wasitemize 
\item $W\subset B(\orz,2.52)$.
\item $\CalL(W) > 12$.
\item The cardinality of $W$ is thirteen, fourteen, or fifteen.
\item $W$ maximizes the function $\CalL$.
\item Join points $\v/\normo{\v}$ and $\w/\normo{\w}$ with a geodesic arc on the
unit sphere if $\{\v,\w\}\in E$.  Then the arcs do not meet except at the endpoints and
give a planar graph.  Moreover, the angle between each pair of consecutive arcs at a vertex is less
that $\pi$.  In particular, the spherical polygons cut out by the arcs are convex.
\end{enumerate}\wasitemize 
A finite packing $W$ with these properties is called a \newterm{contravening} packing.


\subsection{tame hypermap}

The starting point of the third stage of the proof is a contravening
packing $W$ and the corresponding planar graph $(W,E)$.  The result of
this stage is a \fullterm{tame hypermap}{tame!hypermap} (described below).

By definition, a \fullterm{planar graph}{planar!graph} is a graph that admits a
\newterm{planar} embedding.  On the other hand, a graph, endowed with
a fixed embedding into the plane, is a \fullterm{plane graph}{plane!graph}.  A
planar graph has too little structure for our purposes because it
does not single out a particular embedding and the plane graph has too
much structure because it gives a topological object where combinatorics
alone should suffice.  A hypermap gives just the right amount of
structure.  It is a purely combinatorial notion, yet encodes the
relations among nodes, edges, and faces determined by the embedding.
An entire chapter of this book is about hypermaps.


The graph $(W,E)$ of a contravening packing $W$ determines a planar hypermap
$\op{hyp}(W,E)$. We study the following question: what  purely
combinatorial properties of the hypermap $\op{hyp}(W,E)$ can be derived from
the assumption  that
$W$ is a contravening packing?  For example, the cardinality of a
contravening packing $W$ is thirteen, fourteen, or fifteen.  Hence, the hypermap
has thirteen, fourteen, or fifteen nodes.  Much of the later chapters of the book
revolve around the question of the combinatorial properties of the
hypermap.

The final chapter of the proof compiles all of these combinatorial
properties into a long list.  
Although the exact details of the list are not significant,
%The list of properties is every bit as
%artificial as a top ten list of world wonders or unsolved mysteries.
%This The idea is to produce -- with minimal effort -- any list of
the list of combinatorial properties severely constrains the set of possible
hypermaps.  

Any hypermap satisfying all of these properties is said to be
\newterm{tame}.  This list of properties appears in
Definition~\ref{def:tame}.

%There is little to be gained by extra efforts at this stage because
%the final stage gives more efficient means to constrain the set of
%possibilities.

\subsection{linear programming}

The fourth and final stage completes the proof the nonexistence of
the contravening packing $W$.  At the beginning of this stage,
$\op{hyp}(W,E)$ is a tame hypermap.  The list of defining properties
of a tame hypermap are sufficiently restrictive that an explicit finite
list can be generated of every tame hypermap, up to isomorphism.  
This list is generated by computer.  The details of the algorithm are
described in the chapter on hypermaps.

Equipped with an explicit list of possible combinatorial structures,
we move to the proof's end game.  At this stage, because of the computer
generated list of tame hypermaps, the cardinality and
combinatorial structure of $W$ are explicit.

A list is made of the properties of $W$ (and its associated hypermap)
that can be described by linear inequalities.  For each tame hypermap,
a computer solves one or more linear programs that test for feasible
solutions to the system of linear inequalities.  In each case, the
computer produces a certificate that shows that no feasible solution
exists.  It follows that no tame hypermap can be realized in the form
$\op{hyp}(W,E)$.  Each tame hypermap, which represents a
combinatorially feasible arrangement, is geometrical infeasible.  It
follows that $W$, and hence also $V$, do not exist.

As no counterexample exists, the proof of the Kepler conjecture ensues.

%\section{Gallery}

%This section explores the history of packings and coverings through a
%series of figures.

%Harriot (Pascal's triangle) -- Marchal 2D -- Marchal 3D -- Rogers's
%proof (2D) -- Roger's (3D) -- Fejes T\'oth's proof 2D -- Fejes
%T\'oth's proposal 3D -- Hsiang 3D -- dodecahedral conjecture --
%Delaunay simplices 3D (conjectured best) -- Hales 3D (superposition)
%-- Hales 3D hybrid -- Beth Chen (tetrahedra) -- covering problem 2D --
%covering 3D -- heptagons (Kuperberg) -- atom packings -- circle
%packings -- Tammes problem -- van der Waerden 13 (Musin)

    \begin{runninglinenumbers*}
    %\part{Foundations}
    %\linput{trig}  
    %\linput{volume}
    %\linput{hypermap}
    %\linput{fan}
    

%%%%%%%%%%%%%%%%%%%%%%%%%%%%%%%%%%%%%%%%%%%%%%%%%%%%
    
    %\part{Body}
    \linput{packing}
    %\linput{cyclic}
    %\linput{tame}
    %\linput{further}
    \end{runninglinenumbers*}
    \appendix
    %\chapter{Appendix}

\begin{note}%XX
The appendix is not for publication.  It contains some further notes about the
proof that may be of interest to a few experts.
\end{note}

Beyond the text, the proof relies on three separate external bodies of
computer code.  These are described in much greater detail at various
places in the book.  These external bodies of code are called
\begin{enumerate}
\item Tame Hypermap Generation,
\item Interval Arithmetic,
\item Linear Programming.
\end{enumerate}
The tame hypermap generation is an stand-alone program that is run
once at a specific point of the proof.  It carries out a combinatorial
classification of planar graphs satisfying a certain restrictive list
of properties.  The reader can safely ignore this computer program
until reaching the relevant point in the proof.  By contrast, the
interval arithmetic is a collection of nearly one thousand
inequalities that have been proved by computer.  These inequalities
are spread throughout the proof and appear throughout this book.  On
first reading, the reader is encouraged to accept these inequalities
as axiomatically given facts.  Detailed documentation about these
inequalities is available for those who wish to follow up later on the
computer-generated proofs of these inequalities.  The final stage of
the proof consists entirely of linear programming.  There are also
several small linear programs that appear in scattered places in the
book.

\section{Computation}

As this project  progressed, the computer  replaced conventional
mathematical arguments more and more, until now
nearly every aspect of the proof relies on
computer verifications.  Many assertions in this book
are results of computer calculations.
To make the solution more accessible, I have
posted extensive resources \cite{web}.

There are three major pieces of computer code that enter into the proof.

\begin{itemize}
\item {\it Combinatorics}.  A computer program classifies all of the
  planar hypermaps that are relevant to the packing problem.
\item {\it  Interval analysis}.  ``Sphere Packings
I'' describes a method of proving various inequalities in a small number
of variables by computer by interval arithmetic.
\item {\it  Linear programming}.  Many of the nonlinear optimization
problems for the scores of sphere packings are replaced by linear
problems that dominate the original score.  They are solved
by linear programming methods by computer.  A typical problem has
between 100 and 200 variables and 1000 and 2000 constraints.  Nearly
100000
such problems enter into the proof.
\end{itemize}

Computers are used in various other ways.  


\begin{itemize}
\item {\it Formal proof}.
Various parts of the solution of the packing problem have now been
formally verified.
\item  {\it Numerical optimization}.  The exploration of the problem
has been substantially
aided by nonlinear optimization and symbolic math packages.
\item {\it Branch and bound methods}.  When linear programming methods
  do not give sufficiently good bounds, they have been combined with
  branch and bound methods from global optimization.
\item {\it Computer Algebra Systems} Mathematica was used for many
  minor calculations, such as the calculation of exact explicit
  formulas for derivatives.
\item {\it Organization of output}.
The organization of the few gigabytes of code and data that
enter into the proof is in itself a nontrivial undertaking.
\end{itemize}



\section{Formal Proof}


\section{Tame Hypermaps}

\clearpage

\section{Interval Analysis}%DCG 8.3, p75
\label{sec:bounds-simplex}

Interval analysis is a method to obtain trustworthy results from a
computer.  This essay gives a basic introduction to this method.

\subsection{deliberate error}

Interval arithmetic can trace its origins to the method of deliberate
error.  This is an ancient method of navigation with imperfect
instruments.  When William the Conqueror crossed the English Channel
in 1066, he deliberately steered to the north of Hastings.

\begin{quote}
  % ``There is no direct evidence of how a twelfth-century pilot found
  % his way across the English Channel $\ldots$
  ``Any pilot even now who had to make that crossing without a chart
  or compass, as William's pilots did, would use the ancient method of
  Deliberate Error: he would not steer directly towards his objective
  but to one side of it, so that when he saw the coast he would know
  which way to turn.'' \cite[p81]{How81}
  % 1066: The Year of the Conquest, Penguin paperback edition
  % 1981. page 148.
\end{quote}

Deliberate error implements ``better safe than sorry.''
To find an address on a one-way street,  a  driver enters the
street too early, to point in the right direction.  
When searching for a familiar landmark
on a two-way street, it is more efficient  
to start the search safely to one side of the landmark, 
rather than search in ever
expanding zigzags in both directions.  
Deliberate error 
is the idea of trapping a target inside a large enough net, in the
way that
Khachiyan traps the optimal solutions of a linear program inside
an ellipsoid.


The method of deliberate error cushions the adverse effects of
imperfect technology.  The method does not aim to minimize the
imperfections.  It works with a faulty chart and compass.
% The method of deliberate error seeks not to minimize the error of
% imperfect technolo

\subsection{arithmetic}

The method of deliberate error, implemented to control for round-off
errors on a computer, leads to interval arithmetic.  To approximate
the circumference of a circle, Archimedes inscribes a polygon and
circumscribes a second polygon, trapping the unknown within a known
interval.  This is interval arithmetic.


Fixed precision floating point numbers on a computer exactly represent
a finite number of rational numbers.  The remaining uncountable set of
real numbers cannot be precisely represented.  For example, 64-bit
floating point numbers can encode at most $2^{64}$ different real
numbers.  This finite set of representable numbers is not closed under
addition, multiplication, subtraction, or division.  For example, the
input $2.0 + 2.0$ returns $4.0$, because all the numbers involved are
precisely representable.  However, through round-off error, my
computer returns
\begin{verbatim}0.0
\end{verbatim} 
in response to
\begin{verbatim}
-1.0 + (1.0 + 1.0e-16)
\end{verbatim}
and 
\begin{verbatim}1.0e-16
\end{verbatim} 
in response to 
\begin{verbatim}
(-1.0 + 1.0) + 1.0e-16.
\end{verbatim}
As this example shows, machine addition is not even associative.  One
is quickly led to absurd conclusions, if one tries to reason about
floating point operations as if they formed a group, a ring, or a
field.  To reason correctly about floating point, one must accept
these operations on their own terms.  To distinguish the imprecise
machine (floating point) operations from standard operations on the
field of real numbers, in the rest of this section (except in computer
code listings), place a dot over the floating point operations $(\dot
+)$, $(\dot -)$, and so forth.

The precise behavior of floating point operations on a computer is
governed by the IEEE-754 floating point standard \cite{Gol}.  By
giving a precise specification of properties of floating point
arithmetic, the standard makes it possible to reason about the
behavior of floating point.  It is possible to prove theorems about
the behavior of IEEE floating point.  For example, it is possible to
prove that a computer that correctly implements the standard should
return the values in reponse to the inputs given above (in the nearest
rounding mode).

Let $F$ be the set of machine-representable floating point numbers.
Assume that $F$ contains two special symbols $\pm\infty$.  The total
order on the real numbers is extended to $\ring{R}\cup\{\pm\infty\}$,
with $-\infty < x < \infty$ for all $x\in\ring{R}$.  Because of these
two special symbols, the floating point floor $x\to\floor{x}_F\in F$
and ceiling $x\mapsto\ceil{x}_F\in F$ functions, with domain
$\ring{R}$ can be defined.  In this section, we drop the subscript $F$
on the floor and ceiling functions.  We map the set of real numbers to
$F^2$, by sending $x$ to $[a,b]$, where $a = \floor{x}$ and
$b=\ceil{x}$.  If $x$ is Hastings, then $b$ is a point on the shore
north of Hastings.  It is the deliberate error to one side of the
target.





On most modern processors, the rounding mode can be set to directed
rounding, as described in the standard.  When the rounding mode is set
upward, the result of any basic arithmetic floating-point operation
$(\dot +)$, $(\dot -)$, $(\dot *)$, $(\dot /)$ applied to two floating
point numbers $(x,y)\mapsto x\dot\diamond y$ is defined by standard to
be the $\ceil{x\diamond y}$.  That is, make the calculation in the
field of real numbers and round up to the next floating point.  When
the rounding mode is set downward, set $x\dot\diamond y =
\floor{x\diamond y}$.  The notation $\dot\diamond$ does not show
rounding mode, but it should, because it is mode dependent.


Interval arithmetic, like the method of deliberate error, does not
seek to eliminate the sources of floating point round off error.
Rather it brings it under scientific control.


Let $I_F = \{[a,b] \in F^2 \mid a \le b\}$ be the set of floating
point intervals.  Basic machine operations extend to intervals.  The
sum $[a_3,b_3]$ of $[a_1,b_1]$ and $[a_2,b_2]$, is defined as
$a_3=\floor{a_1+a_2}$ and $b_3 = \ceil{b_1+b_2}$.  Write $[a_3,b_3] =
[a_1,b_1] \dot+ [a_2,b_2]$.  This addition of intervals is not
associative.  However, addition of intervals satisfy a crucial
inclusion property.  If $x\in[a_1,b_1]$ and $y\in [a_2,b_2]$, and $z =
x+y$, then $z\in [a_3,b_3]$.  The other arithmetic operations can be
extended to machine interval operations in a similar way, so as to
satisfy the corresponding inclusion properties.  Division requires
special treatment when $0$ belongs to an interval denominator.  We
will not go into those details here.

These operations are easily implemented in code.  For example, here is
the actual snippet of {\tt C++} code that implements the addition of
intervals.  The functions {\tt interMath::up()} and {\tt
  interMath::down()} set the rounding modes on the computer.
\begin{verbatim}
inline interval interval::operator+(interval t2) const
{
interval t;
interMath::up(); t.hi = hi+t2.hi;
interMath::down(); t.lo = lo+ t2.lo;
return t;
}
\end{verbatim}

By implementing interval operations on a computer, we can develop a
procedure that takes as input an arbitrary arithmetic expression over
the rational numbers and returns an interval $[a,b]$ with floating
point endpoints (augmented by $\pm\infty$ as usual) that contains the
rational value of that expression.  The true answer, still unknown,
lies trapped between the interval endpoints.  Since the associative
and distributive laws fail, the returned interval $[a,b]$ depends on
the actual syntax of the expresssion, on the placement of parentheses,
and so forth.



\subsection{analysis}

This essay is not a treatise on interval arithmetic.  Its purpose is
to give an introduction, to demonstrate the possibility of rigorously
bounding the error of floating point performed according to IEEE
standards.  We become increasingly sketchy.

The theory of interval analysis progresses step by step, starting with
basic arithmetic and developing through higher levels of analysis.  If
$p$ is a polynomial with rational coefficents
$p\in\ring{Q}[x_1,\ldots,x_n]$, then it has an interval extension
$\bar p : I_F^n \to I_F$.  Just as in the case of rational numbers,
the interval extension $\bar p$ depends on the syntax used to
expressed to polynomial $p$.  The polynomial extension is {\it
  monotonic}: if $z_i\in [a_i,b_i]$ for all $i$, then
$p(z_1,\ldots,z_n) \in \bar p([a_1,b_1],\ldots,[a_n,b_n])$.  Interval
extensions of rational functions is similar.

Consider a function $f:[a,b]\to\ring{R}$ that
has a rational function approximation $r$ with known
error bound:  $|f(x) - r(x)|<\epsilon$ for $x\in[a_1,b_1]$,
with $\pm\epsilon\in F$.
Define a monotonic interval extension $\bar f$ of $f$ by
$\bar f([c,d]) = \bar r([c,d]) \dot + [-\epsilon,\epsilon]$, for
$[c,d]\subset [a,b]$.  
Multivariate functions are similar.   
%
A large class of analytic functions fall within this framework.
We have interval extensions of trigonometric functions, logs, and
exponentials.  Interval extensions of real-valued functions can be
added, multiplied and composed.  

To prove that a function $f$ is positive on a rectangular domain
$[a_1,b_1]\cdots[a_n,b_n]$, compute the
value of an interval extension $[c,d]=\bar f([a_1,b_1]\cdots[a_n,b_n])$.
By the monotonic property of interval extensions, if $c>0$, then $f$
is positive on the domain.  If this is done naively, with the
first interval extension $\bar f$ that comes to mind, the image interval
$[c,d]$ may be so large that no worthwhile information results.
If we naively 
compute $x+(-x)$ by interval arithmetic for $x$ in the interval $[-1,1]$,
we find that $x+(-x)$ lies in the interval sum
$[-1,1]\dot + [-1,1] = [-2,2]$.  Useless!
There is a large mathematical literature describing various 
efficient algorithms
to compute image intervals $[c,d]$ with accuracy.
Kearfott's book is a recommended starting point, because it comes
close to describing the types of algorithms implemented for the
solution to the packing problem \cite{Kea96}. 



\subsection{archive}

Although this book is long, it represents only a fraction of the
solution of the packing problem.  The other resources that are needed
to understand the full solution are available on the internet.

There is an archive of several hundred inequalities that have been
proved by computer.  The list of inequalities can be found at
\cite{web}.\footnote{The archive of interval arithmetic inequalities
  appears at \url{http://flyspeck.googlecode.com/svn/trunk/}} The list
of inequalities are in computer readable form in the rigorous
mathematical syntax of HOL-Light (for Higher Order Logic).

For example, one of the inequalities from that file reads as follows.
\begin{verbatim}
let I_572068135=
all_forall `ineq 
[((square (#2.3)), x1, (#6.3001));
((#4.0), x2, (#6.3001));
((#4.0), x3, (#6.3001));
((#4.0), x4, (#6.3001));
((#4.0), x5, (#6.3001));
((#4.0), x6, (#6.3001))
]
((((tau_sigma_x x1 x2 x3 x4 x5 x6) -.  ((#0.2529) *.  
(dih_x x1 x2 x3 x4 x5 x6))) >. (--. (#0.3442))) \/ 
((dih_x x1 x2 x3 x4 x5 x6) <.  (#1.51)))`;;
\end{verbatim}
The first part of this snippet of code gives lower and upper bounds on
each of the six variables $x_1,\ldots,x_6$.  The final part of this
code gives an inequality (or rather a disjunction of two inequalities)
of nonlinear real-valued functions that holds on the given domain.
The $\#$ symbol marks exact decimal constants.  The functions {\tt
  dih\_x}, {\tt tau\_sigma\_x} are defined rigorously in a separate
file.  They correspond to the functions $\dih$ and $\tau$ in this
book.  The symbols for arithmetic operations are followed by periods
($*.$ and so forth) to distinguish them from the corresponding
operations on natural numbers.

The inequality carries a nine-digit identifier {\tt 572068135}.  This
number is a tracking number that can be used in a search engine to
locate everything known about a given inequality.  For example, if one
googles this number, the search engine returns nine matches related to
this inequality, including a preprint on the arXiv, the website of the
Annals of Mathematics, a Springer Link to the relevant issue of the
journal Discrete and Computational Geometry, a flyspeck discussion
group, the {\it C++} computer code proving the inequality, and the
output file from running that code.  A search on the pdf file of this
book links this inequality to Lemma~\ref{lemma:11.16}.  Every interval
arithmetic calculation in this book carries a nine-digit identifier,
for easy tracking.

Currently, the most convenient way to access the code that proves the
inequality is through the Google's Code Search, a custom search engine
for computer code.\footnote{The interval arithmetic code that proves
  the inequalities is available at
  \url{http://code.google.com/p/flyspeck/}.}


Further information about proving these inequalities by interval
arithmetic can be found in \cite{algorithm} and \cite{part1}.
%\indy{Index}{calc@\calc{123456789}}%

\subsection{code verification}

S. Ferguson and I put considerable effort into developing trustworthy
code.  The two of us made entirely independent implementations of
the code. %automated inequality proving by interval arithmetic. 
(We shared algorithms
but not source code.)  This allowed us to check our answers against each
other to ensure mutual consistency, 
and to eliminate certain sources of bugs.  We also made independent
implementations in Mathematica and {\tt C} of traditional floating point
versions of the functions used for the nonlinear
optimizations.  By requiring these different implementations to give
compatible answers, we eliminated further sources of bugs.

Each nonlinear inequality was also checked independently with the
nonlinear optimization package {\tt cfsqp}.  This is a collection of
{\tt C} routines that searches for the minimum of a smooth function on
a domain described by a system of constraints.  As is the case with
many nonlinear optimization packages, there is no guarantee that the
search will converge to the true global minimum of the function.  By
repeating the search with a large collection of initial values for the
search, it becomes more probable that the true global minimum will be
found.  In practice it works remarkably well.

This package was not used in any proofs.  The numerical testing with
{\tt cfsqp} was used to discover false inequalities before they were
shipped to the interval arithmetic prover for verification.  Those
that failed never shipped.  This extra level of testing adds an extra
level of robustness to this part of the proof.  Testing gives us
(nonrigorous) reasons to believe the inequalities, even if a bug
should appear in our interval arithmetic code.  This gives us some
hope that an undetected bug would be unlikely to affect the overall
design of the solution to the packing problem.


There are certain types of bugs that would be very difficult to
detect.  For example, since floating point arithmetic is not
associative, a misplaced pair of parentheses might throw a calculation
off by a machine epsilon.  This potential source of bugs is evidence
that the entire project was not sufficiently automated: the
parentheses should all have been worked out automatically.  To make
the calculations more robust, we have tried to design the collection
of inequalities so that they hold with a considerable margin of error,
rather than just squeeze by.  (There are only a few inequalities that
are sharp, and they are sharp for clear mathematical reasons related
to the theory of packings.)  In a bug-free environment, such
precautions would not be necessary.  Nonetheless, we take precautions.

As far as I know, no comprehensive efforts were made by the referees
and editors to check the correctness of the computer code before the
publication of the 1998 solution to the packing problem.  The editors'
preface to \cite{DCG} states that during the review process ``some
computer experiments were done in a detailed check.''


The flyspeck project is a long-term project intended to make the
solution to the packing problems one of the most thoroughly checked
computer proofs of all times.  Part of this project calls for a formal
proof of correctness of the computer code used in the interval
verification of inequalities.

Flyspeck is still far from completion in 2008.  Nevertheless, there
are various ongoing projects related to this second-generation
verification of the interval code.  S. McLaughlin has an independent
implementation of the interval arithmetic code used in the packing
problem~\cite{McL08}.  His work has exposed some data-entry bugs.
They are reported in the comprehensive errata to the 1998 proof, which
is maintained at \cite{errata} with additional discussion
at~\cite{flydis}.  Fortunately, no bugs have surfaced over the past
decade in the underlying interval arithmetic inequality proving
algorithms.  The reported errors have been at the data-entry level: a
mismatch between the data as typed into the preprint and data in
computer code.

One of the formal proof assistants under most active development is
the COQ system~\cite{COQ}.  R. Zumkeller has implemented automated
inequality proving with interval arithmetic inside the theorem prover
COQ, with the flyspeck project in mind; although (as of 2008) the
inequalities that are used in this book have not yet been checked in
this way \cite{Zu}.


\subsection{interval analysis and proof}


The editors of the Annals of Mathematics have posted a statement on
computer-assisted proof.  At first, the editors planned to make a
disclaimer directed at the computer solution of the packing problem.
Eventually, they formulated a general policy on computer-assisted
proofs.  The policy mentions interval arithmetic as one way to control
sources of computer error.


\begin{quote}
%Statement by the Editors on Computer-Assisted Proofs

  ``Computer-assisted proofs of exceptionally important mathematical
  theorems will be considered by the Annals.

  ``The human part of the proof, which reduces the original
  mathematical problem to one tractable by the computer, will be
  refereed for correctness in the traditional manner. The computer
  part may not be checked line-by-line, but will be examined for the
  methods by which the authors have eliminated or minimized possible
  sources of error: (e.g., round-off error eliminated by interval
  artihmetic, programming error minimized by transparent surveyable
  code and consistency checks, computer error minimized by redundant
  calculations, etc. [Surveyable means that an interested person can
  readily check that the code is essentially operating as claimed]).

  ``We will print the human part of the paper in an issue of the
  Annals. The authors will provide the computer code, documentation
  necessary to understand it, and the computer output, all of which
  will be maintained on the Annals of Mathematics website online.''
  \cite{Ann06}

%http://annals.princeton.edu/EditorsStatement.html
\end{quote}

A number of proofs in pure and applied mathematics have been based on
interval analysis.  W. Tucker implemented a rigorous ODE solver with
interval arithmetic and used it to prove that the Lorenz equations
have a strange attractor \cite{Tuc02}. The existence of strange
attractors is problem 14 on Smale's list of 18 Centennial Problems
\cite{Sma98}.  Another prominent problem solved by interval methods is
the double bubble conjecture, a generalization of the isoperimetric
problem in three dimensional Euclidean space.  A sphere gives the
solution to the classical isoperimetric problem.  The work of J. Hass,
M. Hutchings, and R. Schlafly shows that the surface area minimizing
way to enclose two regions of equal volume is the double bubble, which
consists of two partial spheres, separated by a flat circular disk
\cite{HHS95}.

Interval arithmetic has also yielded a number of new results on the
problem of packing circles in a square. M. Cs. Mark\'ot and T. Csendes
have obtained optimality proofs for packings of $28$, $29$, and $30$
circles in a square.  See Figure~\ref{fig:optimal-circles}.  This is
an area of active research. See, for example, \cite{Sza07} and
\cite{Mark07}.

%% WW not yet done.
\begin{figure}[htb]
\centering
\myincludegraphics{noimage.eps}
\caption{Optimal circle packings in a square}
\label{fig:optimal-circles}
\end{figure}



\subsection{note}

There are those who have tried to downplay the role of computers and
interval methods in the solution to the packing problem.  In fact,
they play an absolutely central role.  To segregate the computation
destroys the proof.  After writing the paper {\it The Sphere Packing
  Problem}, I had all but given up on solving the problem.  I had an
extremely difficult nonlinear optimization problem on my hands and no
rigorous mathematical method to solve it.  In the summer 1993, I
happened upon book on Pascal-XSC (a language extension of Pascal for
interval analysis) at the Seminary Coop Bookstore next to the
University of Chicago.  This book described the method I had lacked.
With fresh hope, in January 1994, I set aside all else and devoted
full effort to the packing problem.  The interval code was the most
difficult part of the computer code to implement because its speed was
crucial.  Thanks to the improvements of S. Ferguson, eventually the
code could run from beginning to end in about three months.  The
interval verifications were the last part of the proof to be completed
in August 1998.


\clearpage

\section{Inequality Listings}

%% XX MOVE ALL THIS ELSEWHERE.

\subsection{packings, general inequalities}

This appendix gives a summary of the nonlinear inequalities that have
been cited in the chapter on packings.  Information about the computer
verifications can be found at \cite{hales:2009:nonlinear}.


\begin{note}%XX
This book contains a number of nonlinear inequalities that have been
established by interval-arithmetic calculations by computer.  Some
of these interval arithmetic calculations are still in the process
of being verified.  The approach to the proof of the Kepler
conjecture described here is still work in progress.  A description
of the inequalities and their current status can be found
at~\cite{hales:2009:nonlinear}.
\end{note}


(Note that the following is an inequality in at most six variables; the most
difficult case to prove is that of a $4$-cell.)  Formulas for the
volumes and solid angles appear in Chapter~\ref{chapter:volume}.  An
explicit formula for the dihedral angle appears in
Chapter~\ref{part:trig}.


\begin{calculation}\label{calc:marchal}\guid{WJDLOCM}\guid{1025009205}\guid{3564312720}\rating{ZZ}
%% cc:mar are the k-cell estimates for non-cell clusters.
Define the function $M$ by equations (\ref{eqn:M}) and
(\ref{eqn:m-def}).  Define the function $\gamma(X,M)$ by equation
(\ref{eqn:gamma-def}).  If $X$ is a $0$, $1$, $2$, $3$, or $4$-cell,
then
\begin{displaymath}
\gamma(X,M)\ge 0.
\end{displaymath}
\end{calculation}

\begin{calculation}\label{calc:cc:qtr}\guid{GLFVCVK}\guid{4869905472}\guid{2477216213}\guid{8328676778}\rating{ZZ}
Let $\gamma_L$ be given by Definition~\ref{def:gammaL}, $\op{wt}$ by
Definition~\ref{def:wt}, and $\beta$ by Definition~\ref{def:beta}.
If $X$ is any $k$-cell that is not a quater with $k\in\{2,3,4\}$,
then % gammaL is nonneg on quarters. cc:qtr
\begin{displaymath}
\gamma_L(X) \op{wt}(X) + \beta(e,X)\ge 0.
\end{displaymath} 
\end{calculation}

\begin{calculation}\label{calc:cc:2bl}\guid{FHBVYXZ}\guid{1118115412}\rating{ZZ}
Let $\gamma_L$ be given by Definition~\ref{def:gammaL}.  Let $X$ be
any quarter.  Let $Y$ be a $3$-cell that flanks it.  Then
\begin{displaymath}
\gamma_L(X)+\gamma_L(Y)\ge 0,
\end{displaymath}
% 2-leaf calculation, gammaL(fourcell)+gammaL(threecell) >=0. % cc:2bl:
\end{calculation}

\begin{calculation}\label{calc:cc:5bl}\guid{ZTGIJCF}\rating{ZZ}
Let
\begin{displaymath}
a= 0.0560305, \quad\text{and}\quad  b= -0.0445813.
\end{displaymath}
\begin{itemize}
\item \case{1821661595} A $4$-cell $X$ along a spine $e$ satisfies
\begin{displaymath}
\gamma_L(X)\op{wt}(X) + \beta(e,X) \ge a + b\,\op{azim}(X),
\end{displaymath}
\item \case{7907792228} The $2$-cell $X_2$ and two $3$-cells $X_1,X_3$
that flank it along a spine $e$ satisfy
\begin{displaymath}
\sum_{i=1}^3 \left(\gamma_L(X_i)\op{wt}(X_i) + \beta(e,X_i)\right)\ge a + b\,\sum_{i=1}^3\op{azim}(X_i).
\end{displaymath}
\end{itemize}
\end{calculation}

\begin{calculation}\label{calc:cc:disks}\guid{8550443271}\rating{ZZ}
Let
\begin{displaymath}
g(h) = \arccos(h/2) - \pi/6.
\end{displaymath}
If $h_1,h_2\in [1,\hm]$, then
\begin{displaymath}
\op{arc}(2h_1,2h_2,2) - g(h_1) - g(h_2)\ge 0.
\end{displaymath}
\end{calculation}

\begin{calculation}\label{calc:cc:alin}\guid{7991525482}\rating{ZZ}
Let $L$ be given by Definition~\ref{def:L}.
Let
\begin{displaymath}
g(h) = \arccos(h/2) - \pi/6.
\end{displaymath}
Let
\begin{displaymath}
\op{reg}(a,k) = 2\pi - 2 k (\arcsin(\cos(a)\sin(\pi/k))).
\end{displaymath}
Then
\begin{displaymath}
\op{reg}(g(h),k) \ge c_0 + c_1 k + c_2 L(h),\quad
k = 3,4,\ldots,\quad 1\le h\le \hm,
\end{displaymath}
where
\begin{displaymath}c_0 = 0.6327,\quad c_1 = -0.0333,\quad c_2 =
0.4754.\end{displaymath}
\end{calculation}

\begin{calculation}\label{calc:cc:alin2}\guid{8540377696}\rating{ZZ}
Let $L$ be given by Definition~\ref{def:L}.
Let
\begin{displaymath}
g(h) = \arccos(h/2) - \pi/6.
\end{displaymath}
Let
\begin{displaymath}
\op{reg}(a,k) = 2\pi - 2 k (\arcsin(\cos(a)\sin(\pi/k))).
\end{displaymath}
Let
\begin{displaymath}a'=\arc(2,2,2\hm)-g(\hm) \approx
0.797.\end{displaymath} Then for $k=3,4,\ldots$,
\begin{displaymath}\op{reg}(a',k) \ge c_0 + c_1 k + c_2 L(1) +
c_3\end{displaymath}
where 
\begin{displaymath}c_0 = 0.6327,\quad c_1 = -0.0333,\quad c_2 =
0.4754,\quad c_3 = 0.85.\end{displaymath}
\end{calculation}

\begin{calculation}\label{calc:shorts}\rating{ZZ}
The following calculations involve many cases that are enumerated by
computer code.
\begin{itemize}
\item \case{BIXPCGW} Let $Z$ be any cell-cluster along a spine $e$
with three leaves.  Then
\begin{displaymath}
\Gamma(Z)> 0.
\end{displaymath}
\item \case{QITNPEA} Let $Z$ be any cell-cluster along a spine $e$
with four leaves.  Then
\begin{displaymath}
\Gamma(Z)> 0.
\end{displaymath}
\end{itemize}
\end{calculation}




\subsection{local fan: listing}

\begin{calculation}\guid{2065952723}\rating{ZZ}\label{calc:Lexell}
%See Mathematica numerical calculation.
Let
\begin{displaymath}
g(s;a,b,c,e_1,e_2,e_3) = \sum_{i=1}^3 \dih_i(2,2,2,a+s,b,c) e_i,
\end{displaymath}
where $\dih_i$ is given by Definition~\ref{def:tau}.
Let $\Delta = \Delta(4,4,4,a^2,b^2,c^2)$.
Let primes denote derivatives with respect to the variable $s$.
Assume that
$e_i\in\leftclosed1,1+\sol_0/\pi\rightclosed$,  that
$a,b,c\in\leftclosed2/\hm,4\rightclosed$.
Then
\begin{equation}\label{eqn:calc:Lexell}
  \Delta (g'(0;a,b,c,e_1,e_2,e_3))^2 
- 0.01\Delta^{3/2}g''(0;a,b,c,e_1,e_2,e_3) > 0.
\end{equation}
(The factors of $\Delta$ clear the denominator in
(\ref{eqn:calc:Lexell}) to simplify the inequality to be proved.)
\end{calculation}

\begin{calculation}\guid{2158872499}\rating{ZZ}\label{calc:2der}
%% checked in Mathematica NMaximize
Let $y_1,y_2\in \leftclosed 2,2\hm\rightclosed$.  
\begin{itemize}
\item 
Let $g(t) = \arc(y_1,y_2+t,2)$.  Then $g''(0) < 0$.
Explicitly,
\begin{displaymath}
  g''(0) = \dfrac{
    -64 + 48y_1^2 - 12 y_1^4 + y_1^6 
  + 80 y_2^2 - 8 y_1^2 y_2^2 - 3 y_1^4 y_2^2
    - 12 y_2 ^4 + 3 y_1^2 y_2^4 - y_2^6
  }{y_2^2 \sqrt{\ups(y_1^2,y_2^2,4)}^3}
\end{displaymath}
and the polynomial in the numerator takes negative values on the given
domain.
\item
Let $g(t) = \arc(y_1+t,y_2-t,2)$.  Then $g''(0) < 0$.
Explicitly,
\begin{displaymath}
  g''(0) = \dfrac{\sqrt{\ups(y_1^2,y_2^2,4)} \left(
      -4 y_1^2 + y_1^4 - 4y_1^3 y_2 - 4y_2^2 
   + 6 y_1^2 y_2^2 - 4 y_1 y_2^3 +y_2^4
    \right)}{y_1^2 y_2^2 (2+y_1-y_2)^2 (2+y_2-y_1)^2}
\end{displaymath}
and the polynomial in the numerator takes negative values on the given
domain.
\end{itemize}
\end{calculation}

\begin{calculation}\guid{2986512815}\rating{ZZ}\label{calc:cc:qua}  %m11
Let $y_1y_2,y_3,y_7\in \leftclosed 2,2\hm\rightclosed$,
$y_5,y_8,y_9\in \{2,2\hm\}$, $y_4,y_6\ge 2\hm$.
Let $x_i = y_i^2$.
Assume that
\begin{displaymath}
\Delta(x_1,x_2,x_3,x_4,x_5,x_6)>0,\quad{ and }
\Delta(x_3,x_2,x_7,x_9,x_8,x_4)>0.
\end{displaymath}
Assume that
\begin{displaymath}
\dih(y_3,y_1,y_2,y_6,y_4,y_5)+\dih(y_3,y_2,y_7,y_9,y_8,y_4) < \pi
\end{displaymath}
and\footnote{If $\{\v_1,\ldots,\v_4\}$ is a set of vectors such that
$y_i = \normo{\v_i}$ and $y_{ij} = \norm{\v_i}{\v_j}$, then
$\op{cross}(y_4,\ldots,y_{12}) = \norm{\v_2}{\v_4}$.}
\begin{displaymath}
\op{cross}(y_1,y_2,y_3,y_4,\ldots,y_9) \ge y_4.
\end{displaymath}
Let 
\begin{displaymath}g(t;y_1,\ldots,y_9) =
  \tau_{tri}(y_1,y_2,y_3,y_4+t,y_5,y_6)+\tau_{tri}(y_3,y_2,y_7,y_9,y_8,y_4+t).
\end{displaymath}
Then \begin{displaymath}g'(0)^2 - 0.01 g''(0) > 0.\end{displaymath}
\end{calculation}


\begin{calculation}\guid{EFJSUSK}\rating{ZZ}\label{calc:irred} %cc:tau
%%cc:par
Let $(V,E,F,G)$ be an irreducible minimal fan with parameters
$(r,s)$.  Then $\tau(V,E,F) \ge d(r,s)$.  (A separate calculation
has been made for each of the cases in the list given above.)
% Interval arithmetic
% calculations~ %% cc:par partition cases for tau[r,s].
\end{calculation}










%\begin{calculation}\guid{5779862781}\rating{ZZ}\label{calc:cc:d2a}
%  Let $y_5,y_6\in \{ 2,2\hm\}$, $y_1,y_2,y_3\in \leftclosed
%  2,2\hm\rightclosed$, and $y_4\in \leftclosed 2,4\hm\rightclosed$.
%  Let $g(t) = \tau_{tri}(y_1+t,y_2,y_3,y_4,y_5,y_6)$.  If
%  $\Delta(y_1^2,y_2^2,\ldots,y_6^2)> 0$ and if $ g'(0)=0, $ then
%  $g''(0)<0$.\footnote{The function $g(t)$ may fail to be
%    differentiable at $t=0$ for parameters $y_1,\ldots,y_6$ for which
%    $\Delta(y_1^2,\ldots,y_6^2)=0$.  Thus, it is necessary to work on
%    the noncompact domain $\Delta>0$ for this inequality.}
%%  (The proof is an interval arithmetic calculation over a
%%  four-dimensional space~cc:d2a.  %%cc:d2a
%%  The calculation verifies that the second derivative is negative
%%  whenever the derivative is zero.)
%\end{calculation} 
%

%\begin{calculation}\guid{6645853705}\rating{ZZ}\label{calc:cc:d2b}
%  Let $y_5\in \{2,2\hm\}$, $y_1,y_2,y_3\in \leftclosed
%  2,2\hm\rightclosed$, and $y_4\in \leftclosed 2,4\hm\rightclosed$.
%  Let $g(t) = \tau_{tri}(y_1+t,y_2,y_3,y_4,y_5,y_6)$.  If
%  $\Delta(y_1^2,y_2^2,\ldots,y_6^2)> 0$, if
%\begin{displaymath}
%\arc(y_1,2\hm,2) + \arc(y_2,2\hm,2) 
%\le y_6 \le \arc(y_1,2,2\hm)+\arc(y_2,2,2\hm)
%\end{displaymath} 
%and if
%$
%g'(0)=0,
%$
%then $g''(0)<0$.
%\end{calculation}
%
%\begin{calculation}\guid{5606476569}\rating{ZZ}\label{calc:cc:qua}
%Let $y_{12},y_{23},y_{34},y_2,y_3\in\{2,2\hm\}$.
%Let $y_1,y_4\in \leftclosed 2,2\hm\rightclosed$.
%Let $y_{14}\in\leftclosed 2\hm,y_1+y_4\rightclosed$.
%Let $y_{13}\in\leftclosed 2\hm,y_1+y_3\rightclosed$.
%Let $g(t) = \tau_{tri}(y_1+t,y_2,y_3,y_4,y_5,y_6)$.
%If 
%\begin{displaymath}
%\Delta(y_1^2,y_2^2,y_3^2,y_{23}^2,y_{13}^2,y_{12}^2)> 0 \text{ and }
%\Delta(y_1^2,y_4^2,y_3^2,y_{43}^2,y_{13}^2,y_{14}^2)> 0,
%\end{displaymath} 
%if\footnote{If $\{\v_1,\ldots,\v_4\}$ is a set of vectors such that
%  $y_i = \normo{\v_i}$ and $y_{ij} = \norm{\v_i}{\v_j}$, then
%  $\op{cross}(y_4,\ldots,y_{12}) = \norm{\v_2}{\v_4}$.}
%\begin{displaymath}
%\op{cross}(y_4,y_1,y_3,y_{13},y_{34},y_{14},y_2,y_{23},y_{12})\ge y_{13},
%\end{displaymath}
%and if
%$
%g'(0)=0,
%$
%then $g''(0)<0$.
%\end{calculation}
%

%%%%%%%%%%%%%%%%%%%%%%%%%%%%%%%%%%%%%%%%%%%%%%%%%%%%

\bibliographystyle{plainnat}
\bibliography{../tex/bibliography/all}

% shell:>makeindex index/Index
% shell:>makeindex index/Notation
%\printindex{index/Index}{General index}
%\printindex{index/Notation}{Notation index}

\end{document}
