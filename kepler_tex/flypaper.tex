
%%
% Author Thomas C. Hales
% LaTeX Format


%!TEX TS-program = latex    
%% This line is for TexShop. 

% Revision history. See svn.
% Document created Dec 6, 2002
% Revision started Jan 2007, from published DCG.

% Revisions Sept 2009: dependency on Tarski removed. 2 enclosed over quad removed. hypermap algorithm rewritten.  

\documentclass[cup9a]{cupbook}
%% Cambridge University Press Macros from
%% https://authornet.cambridge.org/information/productionguide/laTex_files/

%
%\usepackage[paper size={90mm, 120mm},left=2mm,right=2mm,top=2mm,bottom=2mm,nohead]{geometry}
\usepackage{graphicx}
\usepackage{verbatim}
\usepackage{latexsym}
\usepackage{amsfonts}
%\usepackage{amsthm}
\usepackage{amsmath}
%\usepackage{mathidx}
\usepackage{makeidx}
\usepackage{multicol}
\usepackage{crop}
\usepackage{txfonts}
%\usepackage{pdfsync}  %for TexShop sync.
\usepackage[letterpaper,colorlinks=true,%
  citecolor=red,%
  %breaklinks=true,%
  %pdftex,
  ps2pdf,%
  hyperindex=true]{hyperref}
%\usepackage{mparhack} %http://www.tex.ac.uk/cgi-bin/texfaq2html?label=marginparside
\usepackage{multind}
\usepackage{url}
%\usepackage{MnSymbol} % powerset.
\usepackage[mathscr]{euscript} % powerset.
\usepackage{pifont} %ding
\usepackage[displaymath]{lineno}

%\usepackage{zref}
%\usepackage{xkeyval}
%\usepackage{ifpdf}
%\usepackage{ifthen}
%\usepackage{calc}
%\usepackage{marginnote}
% http://tug.ctan.org/tex-archive/macros/latex/contrib/pdfcomment/doc/pdfcomment.pdf
\usepackage{pdfcomment}


% This file contains local settings and system dependencies



% Auxiliary directories
\def\dsp{/Users/thomashales/Pictures/mathFigures/DenseSpherePackings}  % flypaper graphics
%\def\pdf{/Users/thomashales/Pictures/collect_geom} % tarski graphics
\def\pdfp{/Users/thomashales/Pictures/mathFigures/collect_geom} % kepler graphics

\def\showgraphics{t}  
% t: display graphics (there are none to show yet)
% f (default): print a "no graphics logo" where graphics would normally go.


\def\displayallproof{t} 
% t (default): display all proofs.
% f: print documents without the proofs-- theorem statements only

\def\displayrating{f}
% t (default): display all ratings (verbose is also true)
% f : don't show them.

\def\verbose{t}
% f (default): do not display debugging information,
% t : display debug information and information about the formalization.

     
%-%
% --Repository--
%-%
% generate revision number by
% svn propset svn:keywords "LastChangedRevision" kepmacros.tex
\def\svninfo{%
  Document TeXed on \today. \hfill\break
  Repository Root: https://flyspeck.googlecode.com/svn \hfill\break
  SVN $LastChangedRevision$
  }

%-%
% --Fonts--
%-%
\font\twrm=cmr8

%-%
% --Graphics--
%-%
%set \showgraphics option in flag.tex
% flypaper graphics
 \def\szincludegraphics[#1]#2{%
      \if\showgraphics t{\includegraphics[#1]{#2}}%
      \else{\includegraphics{noimage.eps}}\fi}

% % kepler graphics
% \def\pdffigtemplatex[#1]#2#3#4{%   
% % usage: \pdffigtemplatex[width=80mm]{file.eps}{labelname}{caption}
% \begin{figure}[htb]%
%   \centering
%  \szincludegraphics[#1]{\pdfp/#2}
%  \caption{#4}
%  \label{fig:#3}%
% \end{figure}%
% }

\def\tikzfig#1#2#3{%
\begin{figure}[htb]%
  \centering
\begin{tikzpicture}#3
\end{tikzpicture}
  \caption{#2}
  \label{fig:#1}%
\end{figure}%
}

\def\tikzwrap#1#2#3#4{%
\begin{wrapfigure}{r}{#4\textwidth}
  \begin{center}
\begin{tikzpicture}#3
\end{tikzpicture}
\end{center}
  \caption{#2}
  \label{fig:#1}%
\end{wrapfigure}%
}


%\def\pdfg#1#2#3#4{\if\showgraphics t{\pdffigtemplatex[#1]{#2}{#3}{#4}}\else{}\fi}
%\def\myincludegraphics#1{%
%      \if\showgraphics t{\includegraphics{#1}}%
%      \else{\includegraphics{noimage.eps}}\fi}


%-%
% --Footnotes and Endnotes--
%-%
% http://help-csli.stanford.edu/tex/latex-footnotes.shtml
%\long\def\symbolfootnote[#1]#2{\begingroup%
%\def\thefootnote{\ensuremath{\fnsymbol{footnote}}}\footnote[#1]{#2}\endgroup}

%-%
% --Special Formatting--
%-%
% http://en.wikibooks.org/wiki/LaTeX/Formatting#List_Structures
%\renewcommand{\labelitemii}{$\star$}
\renewcommand{\labelitemii}{$\circ$}
\renewcommand{\labelenumii}{\alph{enumii}}
\newenvironment{summary}
  {\begingroup\bigskip\narrower\noindent{\bf Summary.~}\it}
%  {~\ding{98}\par\phantom{!}\endgroup\bigskip}
  {~\par\phantom{!}\endgroup\bigskip}
\newenvironment{tidbit}{\smallskip\begingroup}{\endgroup\smallskip}
%\newenvironment{enumerate}
%  {\renewcommand{\labelitemi}{}\begin{itemize}}
%  {\end{itemize}}
\def\wasitemize{\relax}
\def\uncase#1{{\sc #1}}
\def\case#1{{\sc (#1)}}
\def\claim#1{{\it  #1}}
\def\calcentry#1#2#3#4{{\smallskip{\bf #1}\quad{\tt [#2]}\quad{(#3)}\quad {#4}}} % computer calc entry
\def\id#1{\ensuremath{\text{\tt #1}}}



%-%
% --Indexing, References, Citations--
%-%
\def\indy#1#2{\index{index/#1}{#2}}
%\def\eqn#1{{\bf (\ref{#1})}}   % deprecated, use \eqref.
\def\newterm#1{\indy{Index}{#1}{\it #1}\relax}
\def\oldterm#1{\indy{Index}{#1}{#1}\relax}
\def\cc#1#2{%
  \indy{Index}{computer calculation!{#1}}{\it computer calculation}%
  \ifverbose{\footnote{\guid{#1}  #2}}~\cite{website:FlyspeckProject}} % arg dropped.


%-%
% --Endnotes
%-%
%\renewcommand{\maketextnotes}{\global\textnotesontrue
%  \newwrite\textnotes
%  \immediate\openout\textnotes=\jobname.ent
% \literaltextnote{
%\notesheadername={\the\textnotesheadername}
%%\pagestyle{endnotesstyle}
%\mark{3}
%\label{textualnotes}
%\normalfont \backmattertextfont}
%}
%\newcommand{\shipnotes}{
%   \iftextnoteson
%   \theendnotes
%   \immediate\closeout\textnotes
%   \input \jobname.ent
%   \else
%   \relax
%   \fi
%}

%-%
% --Proof Display--
%-%
% set with \displayallproof in flag_fly. If f, then proofs are swallowed.
%% "proved" environment. toggle with \displayallproof
%
\def\hide#1{}
\def\swallowed{\relax}
\def\swallow#1\swallowed{}
\newenvironment{iproved}{}{}
\newenvironment{proved}{\resetproved\begin{iproved}}{\end{iproved}}
\def\hideproof{\renewenvironment{iproved}{%
   \centerline{\it -- Proof Proofed --}
  \renewenvironment{itemize}{}{}
  \renewenvironment{enumerate}{}{}
  \def\item{\relax}
  \catcode13=12
  \swallow
}{}}
\def\showproof{\renewenvironment{iproved}{\begin{proof}}{\end{proof}}}
\def\resetproved{\if\displayallproof t\showproof\else\hideproof\fi}



%-%
% --Debugging Information--
%-%
%% verbose:
\def\rating#1{\if\displayrating t%
  {{\textsc {[rating={\ensuremath {#1}}].\ }}}\else{}\fi}
\def\rz#1{\rating{#1}}
\def\cutrate{}
\def\oldrating#1{\if\displayrating t%
  {{\textsc {[former rating={\ensuremath {#1}}].\ }}}\else{}\fi}

\def\formalauthor#1{\if\verbose t{{\tt [formal proof by #1].\ }}\else{}\fi}
\def\dcg#1#2{{\if\verbose t%
  {{\tt{[DCG-#1]}}\indy{References}{ZC{#2 #1}@{DCG-#1}|page{#2}}}\else{}\fi}}
\def\tlabel#1{\label{#1}\if\verbose t{{\tt [#1].\ }%
   \indy{References}{#1|itt}}\else{}\fi}
\def\ifcverbose#1#2{\if\verbose t{{#1}}\else{#2}\fi}
\def\ifverbose#1{\ifcverbose{#1}{}}  %\verbose t{{#1}}\else{}\fi}
%\def\formal#1{\ifverbose{[{#1}]}}
\def\formal#1{\relax }
\def\formaldef#1#2{\ifverbose{\texttt{[{#1} $\leftrightsquigarrow$ {#2}]}}}
\def\footformal#1{\if\verbose t{\footnote{#1}}\else{}\fi}
\def\guid#1{{\tt[#1].\ }\indy{References}{ZA{#1}@{#1}|itt}}
\def\guid#1{\ifverbose{{\tt [#1]}}}
\def\guid#1{{{\tt [#1]}}}
\def\ineq#1{{{\tt  [#1]}}}
%\def\guid#1{{{\tt [#1]}}}

%\def\calc#1{{\textsc{calc-#1}}\indy{Interval}{{#1}@{#1}}}
%\def\xfootnote#1{\if\verbose t{\endnote{#1}}\else{\footnote{#1}}\fi}
%\def\xfootnote#1{\footnote{#1}}
%\def\xendnote#1{\if\verbose t{\endnote{#1}}\else{}\fi}


% margin notes
\setlength{\marginparwidth}{1.2in}
\def\mar#1{}
 %\ifverbose{\marginpar{\text{\raggedright\footnotesize #1}}}}
%\def\hypermark[#1]#2{\ifcverbose{\hyperref[#1]{#2}}{#2}}


%-%
%--Formatting--
%-%
\def\dfrac#1#2{\frac{\displaystyle #1}{\displaystyle #2}}
\def\textand{\text{ \ and \ }}  % for math eqns.

%-%
%--Redefining--
%-%
\def\emptyset{\varnothing}
\def\ups{\upsilonup} % Needs txfonts; else use \upsilon


%-%
% --Symbols--
%-%
% norm and brackets
\def\|{{\hskip0.1em|\hskip-0.15em|\hskip0.1em}}
\def\mid{\ :\ }
\def\tc{\hbox{:}}
\def\cooln{:\hskip-0.02em:}
\def\norm#1#2{\hbox{\ensuremath{\|#1\unskip-\unskip{#2}\|}}}
\def\normo#1{{\|#1\|}}
\def\sland{\ \land\ }
\def\abs#1{|#1|}
% brackets
\def\leftopen{(}
\def\leftclosed{[}
\def\rightopen{)}
\def\rightclosed{]}
\def\lp#1{{\llbracket{#1}\rrbracket}} 
%\def\comp#1{\llbracket #1 \rrbracket}
\def\comp#1{[#1]}
\def\tangle#1{\langle #1\rangle}
\def\ceil#1{\lceil #1\rceil}
\def\floor#1{\lfloor #1\rfloor}
%accents:
\def\=#1{\accent"16 #1}
\def\ast{\ensuremath{{}^*}}

% mathcal
\def\CalV{{\mathcal V}}
\def\CalL{{\mathcal L}}
\def\BB{{\mathcal B}}
\def\powerset{{\mathscr P}}

% mathbb
\newcommand{\ring}[1]{\mathbb{#1}}
%\def\N{{\mathbb N}}
%\def\Rp{\ring{R}^{3\,\prime}}
%\def\A{{\mathbf A}}
\def\F{{\mathbf F}} % map on faces H to H/{cal L}

% vector notation
\def\v{{\mathbf v}}
\def\u{{\mathbf u}}
\def\w{{\mathbf w}}
\def\e{{\mathbf e} }  
\def\p{{\mathbf p}}
\def\q{{\mathbf q}}

% operatorname
\def\op#1{{\operatorname{#1}}}
\def\optt#1{{\operatorname{{\texttt{#1}}}}}

%\def\opat{{\op{@}}}
\def\atn{\op{arctan\ensuremath{_2}}}
\def\azim{\op{azim}}
\def\nd{\op{node}}
\def\sol{\operatorname{sol}}
\def\vol{\op{vol}}
\def\dih{\operatorname{dih}}
\def\Adih{\operatorname{Adih}}
\def\arc{\operatorname{arc}}
\def\rad{\operatorname{rad}}
\def\bool{\operatorname{bool}}
\def\true{\op{true}}
\def\false{\op{false}}


%\def\orz{\varthetaup} % center of packing
\def\orz{{\mathbf 0}} % center of packing
\def\Wdarto{W^0_{\text{dart}}}
\def\Wdart{W_{\text{dart}}}
%\def\Wedge{W_{\text{edge}}}
\def\cell{\operatorname{cell}}
\def\dimaff{\operatorname{dim\,aff}}
\def\aff{\operatorname{aff}}
\def\card{\op{card}}

\def\del{\partial}
\def\doct{\delta_{oct}}
\def\dtet{\delta_{tet}}
\def\hm{{h_0}} % 1.26
\def\stab{c_{{\scriptstyle \text{stab}}}} % 3.01
\def\tgt{\operatorname{\it{target}}}
\def\pqr#1{#1} % marks type (p,q,r).
\def\trunc#1#2{#1\hbox{\ensuremath{[:\hskip-0.25em plus 0em minus 0em{#2}]}}}
\def\trunc#1#2{#1[\text{:}\hskip0em plus 0em minus 0em{#2}]}
\def\trunc#1#2{d_{#2}{#1}}
%\def\trunc#1#2{#1{[\le\hskip-0.25em{ #2}]}}

%% HYPERMAP macros:
% avoid e for both hypermap edge and edge {v,w}
\def\ee{\varepsilonup}
\def\ocirc{}
\def\wild{{*}}  % wildcard char.

%% PACKNG macros:
%\def\lam{\lambda}
%\def\Lam{\Lambda}
\def\bu{{\underline{\u}}}
\def\bv{{\underline{\v}}}
\def\bw{{\underline{\w}}}
\def\bV{{\underline{V}}}
%\def\arcs#1#2#3{{\arcV(#1,\{#2,#3\})}}

%% LOCAL FAN macros:
\def\smain{S_{\scriptstyle\text{main}}} 
 
 

\setlength{\marginparwidth}{1.2in}
\def\mar#1{\marginpar{\raggedright\footnotesize #1}}

% line numbers
\def\lll{\resetlinenumber[1]}
\def\linenumberfont{\normalfont\small\sffamily}


 \crop
%\makeindex
\makeindex{index/Notation}
\makeindex{index/Index}

% CUP BOOK CLASS SPECIFIC
\newtheorem{lemma}{Lemma}[chapter]
\newtheorem{calculation}{Calculation}[chapter]
\newtheorem{definition}{Definition}[chapter]
\newtheorem{remark}{Remark}[chapter]
\newtheorem{theorem}{Theorem}[chapter]
\newtheorem{background}{Background}[chapter]
\newtheorem{corollary}{Corollary}[chapter]
\newtheorem{example}{Example}[chapter]
%\newtheorem{claim}{Claim}[chapter]
\newtheorem{notation}{Notation}[chapter]
\newtheorem{assumption}{Assumption}[chapter]
\newtheorem{interpretation}{Interpretation}[chapter]
\newtheorem{conjecture}{Conjecture}[chapter]
\newtheorem{note}{Author's Note}[chapter]
%%%%%%%%%%%%%%%%%%%%%%%%%%%%%%%%%%%%%%%%%%%%%%%%%%%%

\begin{document}
\raggedbottom  % for now.
%\raggedright  % don't worry for now.

    \title[a blueprint for formal proofs]
    {%Flyspeck :
      Dense Sphere Packings}
    \author{Thomas C. Hales}
    \maketitle
    \frontmatter
    \tableofcontents
    \thanks

\mainmatter

\noindent



\bigskip




\begin{note}%XX
This manuscript is not for ready general distribution.  Please do not circulate it.
\begin{enumerate}\wasitemize 
\item The computer calculations that back up various claims have not
  been completed.  In particular, various nonlinear inequalities
  remain to be proved.  The linear programs for one of the hypermaps
  have still not terminated.
\item The figures are missing.
\end{enumerate}\wasitemize 
\end{note}

\bigskip\noindent %
This research has been supported by the National Science Foundation
under Grants 0503447 and 0804189 as well as a grant from the Benter
Foundation.

\bigskip\noindent\svninfo 

\newpage





%%%%%%%%%%%%%%%%%%%%%%%%%%%%%%%%%%%%%%%%%%%%%%%%%%%%
   \newpage

   %\setcounter{chapter}{-1}
   %%------------------------------------------------------------
% Author: Thomas C. Hales
% Format: LaTeX
% Book Chapter: Dense Sphere Packings
%------------------------------------------------------------

\chapter*{Preface}

% Believe not everything, but only what is proven: the former is foolish, the latter the act of a sensible man. -- Democritus.

%{
%
%\narrower
%
%{\it ``A personality which has been veiled by a formal method
%  throughout many chapters is suddenly seen face to face in the
%  Preface.'' }
%% Introductory note, p3, Famous Prefaces, Harvard Classics, Vol
%% 39. P.F. Collier & Son, 1910.
%
%}

%\bigskip

%\centerline{\it ``Those who justify themselves do not convince.''
%  --Lao-Tzu}
%% quoted in A. Watts's essay in "Modern Buddhism" ed. Donald Lopez,
%% page 160


{

\narrower

{\it

  ``I think there's a revolution in mathematics around the corner. I
  think that $\ldots$ %in later times
  people will look back on the fin-de-siecle of the twentieth century
  and say `then is when it happened' (just like we look back at the
  Greeks for inventing the concept of proof and at the nineteenth
  century for making analysis rigorous). I really believe that. And it
  amazes me that no one seems to notice $\ldots$

  ``Never before have the platonic mathematical world and the physical
  world been this similar, this close. Is it strange that I expect
  leakage between these two worlds? That I think the proof strings
  will find their way to the computer memories?$\ldots$

  ``What I expect is that some kind of computer system will be
  created, a proof checker, that all mathematicians will start using
  to check their work, their proofs, their mathematics. I have no idea
  what shape such a system will $\ldots$ take. But I expect some
  system to come into being that is past some threshold so that it is
  practical enough for real work, and then quite suddenly some kind of
  `phase transition' will occur and everyone will be using that
  system.''

{\hfill--Freek Wiedijk \cite{FWR}} % http://www.cs.ru.nl/~freek/jordan/index.html

}

}

\newpage

{

\narrower\parindent=0pt
\parskip=0.4\baselineskip

{\it

Alecos: Christos has a problem with the `foundational quest'!

Christos:  Wrong!  I have two problems with your  {\rm {version}} of it!  One, it
didn't fail and, two, it wasn't a tragedy!  Granted, there are some tragic
parts!  But the ending is happy, as in the `Oresteia'!  

Apostolos:  Happy for whom?  Cantor, going insane?  G\"odel starving himself to
death out of paranoia? Hilbert or Russell and their psychotic sons? Or Frege with--

Christos: `The meaning
is in the ending!' you said so yourself!  So, follow the quest for ten more years and
you get a brand-new triumphant finale with the creation of the computer, which is the
quest's real hero!   Your problem is, simply, that you see it as a story of people!

Apostolos: Well, stories do tend to be about people!

Christos:  So, choose the right people!  And show what they really did!  All we we learn
of the great von Neumann is he said `It's over'  when he heard G\"odel!

Alecos: But it was over in a sense, wasn't it?  Pop went Hilbert's `no ignorabimus'!

Christos:  But then came the quest's jeune premier, its parsifal $\ldots$ Alan Turing!
He said `Ok, we can't prove everything! So, let's see what we can prove!' and to define
proof, he invented, in 1936, a theoretical machine which contains all the ideas of the
computer! $\ldots$ which, after the war, he and von Neumann, the quest's proudest sons,
brought to full life!

{\hfill--Logicomix} % page 303.

}

}




\newpage

{

\narrower\parindent=0pt
\parskip=0.4\baselineskip

{\it

``Ever hear of the Kepler Conjecture?''

``Nope.''

I laid the notebook on the table and flipped through the pages. ``It was first stated
in 1611 by Johannes Kepler,'' I said.  ``Kepler becomae interested in the problem while
he was corresponding with an Englishman named Thomas Harriot, who was trying
to help his friend Sir Walter Raleigh figure out the best way to stack cannonballs on ship
decks.  The goal was to find the densest possible spherical arrangements, $\ldots$ basically,
the way grocers stack oranges''

``Okay,'' he said, nodding.

``Kepler's conjecture {\rm{seems}} perfectly sound,'' I said.

``That is does,'' Ben said.

``But here's the thing. $\ldots$  I looked it up and discovered that, in 1998, a proof
had finally been put forward by an American mathematician named Thomas Hales.  In 2003,
a committee that had been assigned to verify Hales's work confirmed that they were ninety-nine
percent certain of the proof's correctness.  But that one percent was key.  The mathematical
world is still waiting for the publication of the data that will prove the Kepler Conjecture definitively.''

``Sucks for Thomas Hales,'' Ben said.

``I agree.  But it makes sense that they have to be certain, doesn't it?''


{\hfill--Michelle Richmond, No One You Know} % page 177.

}

}

%%DD Figure of cannonballs.

\bigskip

{

\narrower\parindent=0pt
\parskip=0.4\baselineskip

{\it

Sometimes fixing a $1$ percent defect takes $500$ percent effort.

{\hfill-- Joel Spolsky, Joel on Software} % page 122

}

}

\bigskip

{

\narrower\parindent=0pt
\parskip=0.4\baselineskip

{\it

Every one fully persuaded is a fool.

{\hfill-- Balthasar Graci\'an, the Art of Worldly Wisdon} % p110

}

}



\newpage

\section*{Blueprint for Formal Proofs}

In 1611, Kepler wrote a booklet in which he asserted that the familiar
cannonball arrangement of congruent balls in space achieves the
highest possible density.  No other arrangement fills a larger
fraction of space.  This assertion is the Kepler conjecture.  In 1900,
Hilbert made this conjecture part of his eighteenth problem.  This
book presents a proof of this assertion.

This assertion has become a test of the capability of computers to
deliver a reliable mathematical proof.  The original proof by Sam
Ferguson and me involved many long computer calculations that
exhausted the efforts of a team of referees.  This book represents my
efforts to redesign the proof in a way that makes the correctness of
the computer proof as transparent as possible.

After all is said and done, a proof is only as reliable as the
processes that are used to verify its correctness.  The ultimate
standard of proof is a formal proof.  A formal proof is nothing other
than an unbroken chain of logical inferences from an explicit set of
axioms.  While this may be the mathematical ideal of proof, actual
mathematical practice generally deviates significantly from the ideal.



%More than ten years have passed since a proof was first
%obtained. Why give a new presentation of the proof?
%
%The original proof was not widely understood.  The complexity was not
%because of conceptual challenges.  In fact, the proof makes only
%modest demands on the theoretical training of the reader.  It is
%possible to read and understand the proof with a knowledge of a
%limited body of mathematics, such as basic calculus and elementary
%Euclidean geometry.
%
%Nevertheless, the proof involves many long calculations. Even worse,
%it it relies on computer calculations.  An error in any calculation or
%a bug in the computer code has the potential to topple the entire
%proof.
%
%The referees were conscientious and checked many of the calculations.
%However, for the most part, the computer code lay beyond the scope of
%referee review, and even careful quality control can let a
%occasional bug slip through undetected.
%
%After all is said and done, no proof is more reliable than the
%reliability of the processes that are used to verify its
%correctness.  These processes include the checking that the author
%makes before releasing the proof for public scrutiny, the checking
%of the referees, and the checking done by readers after publication.

In recent years, as part of this project, I have been increasingly preoccupied by the
processes that mathematicians rely on to insure the correctness of complex
proofs. Researchers from Frege to G\"odel, who solved a problem of
rigor in mathematics, found a theoretical solution but did not
extinguish the burning fire at the foundations of mathematics
because they omitted the practical implementation. Some, such as
Bourbaki, have even gone so far as to claim that ``formalized
mathematics cannot in practice be written down in full'' and call
such a project
``absolutely unrealizable'' \cite[p 10,11]{Bour:68:Sets}. % Theory of
                                                          % Sets, page
                                                          % 10,11.

While it is true that formal proofs may be too long to print,
computers -- which do not have the same limitations as paper -- have
become the natural host of formal mathematics. In recent decades,
logicians and computer scientists have reworked the foundations of
mathematics, putting them in an efficient form designed for real use
on real computers.

For the first time in history, it is possible to generate and verify
every single logical inference of a major mathematical theorem.  This
has now been done for the four-color theorem, the prime number
theorem, the Jordan curve theorem, the Brouwer fixed point theorem,
and the fundamental theorem of calculus, among others.  Freek Wiedijk
reports that 82\% of a list of 100 famous theorems have now been
checked formally \cite{wiedijk:100}.  The list of 18 remaining
theorems contains two particular challenges: the independence of the
Continuum Hypothesis and Fermat's Last theorem.

Some mathematicians remain skeptical of the process because computers
have been used to generate and verify the logical inferences.
Computers are notoriously imperfect, with flaws ranging from software
bugs to defective chips.  Even if a computer verifies the inferences,
who will verify the verifier, or then verify the verifier of the
verifier?  Indeed, it would be unscientific of us to place an
unmerited trust in computers.

The choice comes down to two competing verification processes.  The
first is the traditional process of referees, which depends largely on
the luck of the draw -- some referees are meticulous, others are
careless.  The second process is formal computer verification, which
is less dependent on the whims of a particular referee.  In my view,
the choice between the conventional referee process and computer
verification is as evident as the choice between a sundial and an atomic
clock in contemporary science.

The boundary that separates an ``easy'' proof from a ``difficult''
proof shifts with current technology.  The introduction of steel in
architecture is not a mere reinforcement of wood and stone, it changes
the architect's world of possibilities.  There will no longer be any
reason to limit ourselves to ten-thousand-page proofs when our
technology supports million-page proofs.

The standard of proof I have adopted is the highest scientific standard
available by current technology.  That 
standard is formal verification by computer.  This standard
continues to evolve with the advancement of technology.

I dream of a fully formally verified solution to the
packing problem.  This project is still unfinished, but significant
progress is being made.  In this book, I rearrange the proof with
formal verification in mind .  The book is {\it a blueprint for formal
  proofs} because it gives the design of the formal proof to be
constructed.  My decisions about what to include in this book has been
shaped by the list of theorems already available in the library the
proof assistant {\tt HOL Light}.  For example, this book assumes basic
point-set topology and measure theory, which have been formalized by
John Harrison~\cite{HOLL}.

The style of formal proofs is different from that of conventional
proofs.  It is better to have a large number of short snappy proofs,
rather than a few intricate ones.  Humans enjoy surprising new
perspectives, but computers benefit from repetition and
standardization.  Despite these differences, I have worked to make
proofs that will bring pleasure to the human reader while providing
precise instructions for the implementation in silicon.



\section*{Structure of this Book}

The book is divided into parts.
The introductory part describe the major ideas, methods, and
organization of the proof.  

%There is an essay on each major computer
%component of the proof. The purpose is to provide a panoramic view of
%proof, to provide intuition about proof strategies.  After reading
%this part of the book, the reading should understand what the proof is
%all about, without yet dipping into technical details.
The part on foundations provides background material about
constructions in discrete geometry.    The first
of these chapters
covers trigonometric identities and basic vector geometry.  The second
treats volume from an elementary point of view.  The third chapter
covers planar graph theory from a purely combinatorial point of view.
The fourth chapter continues with planar graphs, now from a 
geometric point of view.

The next part of the book gives the solution to the packing problem.
The first chapter in this part gives a top-level overview of the major
steps of the proof.  It describes how the problem can be reduced from
a problem in infinitely many variables to a problem in finitely many
variables.  The remaining chapters in this part flesh out that
skeleton.

The final part of the book resolves some other longstanding conjectures in
discrete geometry: K. Bezdek's strong dodecahedral conjecture and Fejes
T\'oth's full contact conjecture.

Many simplifications of the original proof have been found over the past
several years.  The simplified proof is published here for the first time.
G. Gonthier expresses his formal proof of the four-color
theorem in terms of hypermaps.  He reworks the proofs of the
four-color theorem to avoid the use of the Jordan curve theorem, using
instead the much simpler notion of M\"obius contour.  I have followed
Gonthier's lead in these respects and also avoid the use of the Jordan curve theorem.

The optimality of the face-centered cubic packing is an assertion
about infinite space-filling packings.  For computational purposes, it is
useful to reduce the sphere packing problem to finite packings.  A
{\it correction term} is associated with each different reduction from
infinite packings to finite packings.  S. Ferguson and I considered a
large number of different correction terms.  We searched for one that
would simplify the computations as much as possible.  In a discussion
of the solution of the packing problem, I wrote that ``correction
terms are extremely flexible and easy to construct, and soon Samuel
Ferguson and I realized that every time we encountered difficulties in
solving the minimization problem, we could adjust $f$ [the correction
term] to skirt the difficulty. $\ldots$ If I were to revise the proof
to produce a simpler one, the first thing I would do would be to
change the correction term once again.  It is the key to a simpler
proof.''  C. Marchal has recently found a very simple 
correction term, that is, a simple way  to make the reduction from infinite packings
to finite packings.~\cite{marchal:2009}.  We use his reduction in this book.

There are many other improvements of the proof that are not visible in
the book, because they are implemented in computer code.  We have been
able to reduce the number of lines of computer code from over 187,000
to well under 10,000.  Needless to say, this significantly simplifies the 
formalization project.





\bigskip
\hbox{}



\bigskip
\hbox{}

{
\parindent=0pt
\obeylines

Thomas C. Hales
Pittsburgh, PA
May 2010

}









%%%%%%%%%%%%%%%%%%%%%%%%%%%%%%%%%%%%%%%%%%%%%%%%%%%%
  
    %%------------------------------------------------------------
% Author: Thomas C. Hales
% Format: LaTeX
% Book Chapter: Dense Sphere Packings
%------------------------------------------------------------


\chapter{Close Packing}\label{sec:close}

\section{History}\label{sec:history}

This section gives a brief history of the study of dense sphere
packings.  Further details appear at \cite{Szpiro} and
\cite{Hales:2006:overview}.
The early history of sphere packings is concerned with the
face-centered cubic (FCC) packing, a familiar pyramid arrangement
of congruent balls used to stack cannonballs at war memorials and
oranges at fruit stands (Figure~\ref{fig:fcc-packing}).

\figDHQRILO % fig:fcc-packing


\subsection{Sanskrit sources}



The study of the mathematical properties of the FCC
packing can be traced\footnote{I am obliged to Plofker~\cite{Plo00}.} to a Sanskrit work (the \=Aryabha\d t\={\i}ya
 of \=Aryabha\d ta) composed around 499 CE.  The following passage gives
the formula for the number of balls in a pyramid pile with triangular base as
a function of the number of balls along an edge of the pyramid~\cite{Ary}:

% {\it \=Aryabha\d t\={\i}ya}, Ga\d nitap\=ada 21:

\bigskip

{\narrower\it\font\ninerm=cmr9

For a series [lit. ``heap''] with a common difference and
  first term of 1, the product of three [terms successively] increased
  by 1 from the total, or else the cube of [the total] plus 1
  diminished by [its] root, divided by 6, is the total of the pile
  [lit. ``solid heap''].  

}

\bigskip

 In modern notation, the passage gives two formulas for the number of
 balls in a pyramid with $n$ balls along an edge (Figure~\ref{fig:sanskrit}):
\begin{equation}\label{eqn:sanskrit}
\dfrac{n(n+1)(n+2)}{6} =  \dfrac{(n+1)^3 - (n+1)}{6}
\end{equation}

\figKSOEMIZ % fig:sanskrit

\subsection{Harriot and Kepler}

The modern mathematical study of spheres and their close packings can
be traced to Harriot.  His work -- unpublished, unedited, and largely
undated -- shows a preoccupation with sphere packings.  He seems to
have first taken an interest in packings at the prompting of Sir
Walter Raleigh.  At the time, Harriot was Raleigh's mathematical
assistant, and Raleigh gave him the problem of determining formulas
for the number of cannonballs in regularly stacked piles.  Harriot
interpreted the number of balls in a pyramid as an entry in Pascal's
triangle\footnote{Harriot was well-versed in Pascal's triangle long
  before Pascal.} (Figure~\ref{fig:pascal}). Through his study of
triangular and pyramidal numbers, Harriot later discovered finite
difference interpolation~\cite{BeS08}.  Shirley, Harriot's biographer,
writes that it was his study of cannonball arrangements in the late
sixteenth century that ``led him inevitably to the corpuscular or
atomic theory of matter originally deriving from Lucretius and
Epicurus'' \cite[p.~242]{Shi83}.

\figBDCABIA % fig:pascal

Kepler became involved in sphere packings through his correspondence
with Harriot around 1606--1607 on the topic of optics.
Harriot, the atomist, attempted to understand reflection and refraction
of light in atomic terms.  Kepler favored a more classical explanation of
reflection and refraction in terms of what Kargon describes as ``the union of two opposing
qualities -- transparence and opacity''~\cite[p.26]{Kar66}.  
Harriot was stunned that
Kepler would be satisfied by such reasons.

Despite Kepler's initial reluctance to adopt an atomic
theory, he was eventually swayed and  published an essay in  1611
that explores the consequences of a theory of matter composed of small
spherical particles. 
Kepler's essay describes the FCC packing and asserts
that ``the packing will be the tightest possible, so that in no other
arrangement could more pellets be stuffed into the same
container''~\cite{Kep66}.  This assertion has come to be known as the
Kepler conjecture.  This book
gives a proof of this conjecture.

\subsection{Newton and Gregory}

The next episode in the history of this problem,  a debate between
Isaac Newton and David Gregory,  centered on the
question of how many congruent balls  can be arranged to touch
a given ball.  The analogous question in two dimensions is readily answered;
six pennies, but no more, can be arranged
to touch a central penny.  In three dimensions, Newton said that the maximum was
twelve balls, but Gregory claimed that thirteen might be possible.

The Newton-Gregory problem was not solved until centuries later
(Figure~\ref{fig:musin}).  The first proper proof was obtained by van
der Waerden and Sch\"utte in 1953 \cite{Sch53}.  An elementary proof
appears in Leech \cite{Leech:1956:MG}.  Although a connection between
the Newton-Gregory problem and Kepler's problem is not obvious, Fejes
T\'oth successfully linked the problems in 1953~\cite{Fej53}.

\figPTFTWZM % fig:musin

\section{Face-Centered Cubic}



The FCC packing is the familiar pyramid arrangement of
balls on a square base as well as a pyramid arrangement on a
triangular base.  The two packings
differ only in their orientation in space.
Figure~\ref{fig:tri-square} shows how the triangular base
packing fits between the peaks of two adjacent square based pyramids.

\figNTNKMGO % fig:tri-square

Density, defined as a ratio of volumes, is insensitive to changes of
scale.  For convenience, it is sufficient to consider balls of unit
radius. This means that the distance between centers of balls in a
packing is always  at least $2$.  We identify a packing with its set $V$
of centers.   For our purposes, a packing is just a set of points
in $\ring{R}^3$ in which the elements are separated by distances of at least
$2$.



The density of a packing is the ratio of the volume occupied by the
balls to the volume of a large container.  The
purpose of a finite container is to prevent the volumes from becoming
infinite.  To eliminate the distortion of the packing caused by the
shape of the its boundary, we take the limit of the densities within an increasing
sequence of spherically shaped containers, as the diameter tends to infinity.

The FCC packing is obtained from a cubic lattice, by inserting a ball
at each of the eight extreme points of each cube and then inserting a
another ball at the center of each of the six facets of each cube
(Figure~\ref{fig:face-centered-cubic}).  The name
\fullterm{face-centered cubic}{FCC} comes from this construction.  The edge
of each cube is $\sqrt8$, and the diagonal of each facet is $4$.  The
density of the packing as a whole is equal to the density within a
single cube.  The cube has volume $\sqrt8^3$ and contains a total of
four balls: half a ball along each of six facets and one eighth a ball
at each of eight corners.  Thus, the density within one cube is
   \[ 
   \frac{   4 (4\pi/3)}{\sqrt8^3} = \frac{\pi}{\sqrt{18}}.
   \] 


\figTCFVGTS % fig:face-centered-cubic


%The tiling of regular tetrahedra and octahedra can be
%superimposed on the picture of the cube.  Each tetrahedron has an extreme point
%in common with the cube and three other extreme points at centers of facets
%of the cube.   One octahedron is concentric with the cube and has an extreme
%point at the center of each facet.  There is an
%additional quarter of an octahedron along each edge of the cube, extending to the
%midpoints of the two adjacent facets, making a total of eight
%tetrahedra and four octahedra.  As each octahedron has the volume of
%four tetrahedra, exactly $1/3$ of the cube is filled with tetrahedra,
%the other $2/3$ with octahedra.  This decomposition shows that the
%volume of a tetrahedron is $2\sqrt2/3$.
%%(pretend ignorance). The volume $16\sqrt2$ 
%%of the cube equals $24$ tetrahedra \dots, giving each a volume
%%of 
%%$2\sqrt{2}/3$.

The density $\pi/\sqrt{18}$ of the packing is the ratio of the volume
$4\pi/3$ of a ball to the volume of a fundamental domain of the FCC
lattice.  The volume of the fundamental domain is therefore
$4\sqrt{2}$.  A fundamental domain of the FCC lattice is a
parallelepiped that can be dissected into two regular tetrahedra and
one regular octahedron (Figure~\ref{fig:fcc-fun-domain}).  The FCC
packing is then an alternating tiling by tetrahedra and octahedra in
2:1 ratio.  A tetrahedron scaled by a factor of two consists of one
tetrahedron at each extreme point and one octahedron in the center
(Figure~\ref{fig:tet-oct-ratio}). By similarity, the total volume is
$8 = 2^3$ times the volume of each smaller tetrahedron. This
dissection exhibits the volume of a regular octahedron as exactly four
times the volume of a regular tetrahedron of the same edge length.  As
a result, the volume of a regular tetrahedron of side $2$ is $1/6$ the
volume of the fundamental domain, or $2\sqrt{2}/3$.

\figSEYIMIE % fig:fcc-fun-domain

\figAZGXQWC % fig:tet-oct-ratio

The density of the FCC packing is the weighted density
of the densities of the tetrahedron and octahedron.  Write $\dtet$ and
$\doct$ for these densities.  Explicitly, $\dtet$ is the ratio of the
volume of the part within the tetrahedron of the unit balls (at the
four extreme points) to the full volume of the tetrahedron.  As tetrahedra fill
$1/3$ of volume of the fundamental domain and an octahedron fills
the other $2/3$,
\[ 
  \frac{\pi}{\sqrt{18}} = \frac{1}{3}\dtet + \frac{2}{3}\doct.
\] 

As above, we identify a packing with the set $V$ of centers of the
balls.  The \fullterm{Voronoi cell}{decomposition!Voronoi} 
of a point $v$ in a packing $V$ is
defined as the set of all points in $\ring{R}^3$ (or more generally in
$\ring{R}^n$) that are at least as close to $v$ as to any other point
of $V$ (Figure~\ref{fig:voronoix}).  Each Voronoi cell of the FCC
packing is a rhombic dodecahedron
(Figure~\ref{fig:rhombic-dodec}),
which is constructed from an inscribed cube by placing a square based pyramid
(with height half as great as an edge of its square base) on each of
the six facets.

\indy{Index}{Voronoi|see{decomposition}}%

\figEVIAIQPx % fig:voronoix

\figPQJIJGE % fig:rhombic-dodec

%Recall that the cubes under discussion  have an edge length
%$\sqrt{2}$.
Rhombic dodecahedra, being the Voronoi cells of the FCC packing, tile space.
In each rhombic dodecahedron, we 
may color the inscribed cube black and the six square-based pyramids
white.  In the tiling, 
the black cubes fill the black spaces of an infinite three-dimensional
checkerboard, and the white pyramids fill the white spaces.

A Voronoi cell contains an inscribed black cube of side $\sqrt2$ and a total
of one white cube, for a total volume of $4\sqrt2$, which is
again the volume of the fundamental domain.  The density of the
FCC packing is the ratio of the volume of a ball to the volume
of its Voronoi cell, which gives $\pi/\sqrt{18}$ yet again.



\section{Hexagonal-Close Packing}\label{sec:hcp}

There is a popular and persistent misconception that the FCC
 packing is the only packing with density $\pi/\sqrt{18}$.
The hexagonal-closed packing (HCP) has the same density.
\indy{Index}{HCP}%
\indy{Index}{FCC}%


In the FCC packing, each ball is tangent to twelve others in the same
fixed arrangement.  We call it the \fullterm{FCC
  pattern}{FCC!pattern}.  Likewise, in the HCP, each ball is tangent
to twelve others in the same arrangement
(Figure~\ref{fig:fcc-hcp-pattern}).  We call it the \fullterm{HCP
  pattern}{HCP!pattern}.  The FCC pattern and HCP patterns are
different from each other.  In the FCC pattern, four different planes
through the center give a regular hexagonal cross section, while the
HCP pattern has only one such plane.

\figSGIWBEN % fcc-hcp-pattern

There are, in fact, uncountably many packings of density
$\pi/\sqrt{18}$ in which the tangent arrangement around each ball is
either the FCC pattern or the HCP pattern.

A \newterm{hexagonal layer} is a translate of the
two-dimensional hexagonal lattice (also known as the triangular
lattice). That is, it is a translate of the planar lattice generated
by two vectors of length $2$ and angle $2\pi/3$.  The FCC
 packing is an example of a packing built from hexagonal layers.

 If $L$ is a hexagonal layer, then a second hexagonal layer $L'$ can be
 placed parallel to the first so that each lattice point of $L'$ has
 distance $2$ from three different points of $L$,
%When the second
% layer is placed in this manner, 
 which is the smallest possible distance from first layer.  A choice
 of a unit normal vector $\e$ to the plane of $L$ determines an upward
 direction.  There are two different positions in which $L'$ can be
 closely placed above $L$
% and two different positions in which
% $L'$ can be placed closely below $L$ 
(Figure~\ref{fig:hex-layers}).  Each successive layer 
  ($L$, $L'$, $L''$, and so
 forth) offers two further choices  for the placement of
 that layer. Running through different 
 sequences of choices gives uncountably many packings.  In each of
 these packings the tangent arrangement around each ball is the FCC or HCP arrangement.

\figCCQCYWU % fig:hex-layers

As a packing is constructed, each layer may be labeled
$A$, $B$, or $C$ depending on three possible orthogonal projections to a fixed plane
with normal vector $\e$.
Each layer carries a different label from the layers immediately
above and below it.  In the FCC packing, the successive layers are
$A,B,C,A,B,C$, and so forth.  In the HCP packing, the successive layers are
$A,B,A,B$, and so forth.  If the vertices of a triangle are labeled $A$, $B$, and $C$,
then the succession of labels is a
walk along the vertices of the triangle, and inequivalent walks through the
triangle describe different packings.


The different walks through a triangle give all possible packings of
infinitely many congruent balls in which each tangent arrangement is
either the FCC pattern or the HCP pattern~\cite{CoSl95}.  To see that
there are no other possibilities, we first assume that every ball of
$V$ is surrounded by the FCC pattern.  Adjacent FCC patterns interlock
in a unique way that forces $V$ itself to crystallize into the FCC
packing.  This completes the proof in this case.

Now we assume that a packing $V$ contains some ball (centered at $\u$)
in the HCP pattern. Its uniquely determined
plane of reflectional symmetry contains $\u$ and the centers of six
others arranged in a regular hexagon. If $\v$ is the center of one of
the six other balls in the plane of symmetry, its  tangent arrangement
of twelve balls must include $\u$ and an additional four of the
twelve balls around $\u$. These five centers around $\v$ are not a
subset of the FCC pattern, but  extend uniquely to
a HCP pattern.   Around $\u$ and $\v$, the HCP patterns  have the same
plane of symmetry. In this way, as
soon as some center has the HCP pattern, the pattern
propagates along the plane of symmetry to create a hexagonal layer
$L$.

Once a packing $V$ contains a single hexagonal layer, the condition
that each ball be tangent to twelve others forces a hexagonal layer
$L'$ above $L$ and another hexagonal layer below $L$.  Thus, a single
hexagonal layer forces an infinite sequence of close-packed hexagonal
layers.  The position of each layer over the previous layer is described 
by the labels $A$, $B$, and $C$ of the triangle.
This completes the proof that the different walks through a triangle give
all possibilities.



\section{Gauss}

Gauss proved that the FCC packing has the greatest density
of any lattice packing in three-dimensional Euclidean space.  There is a
short proof that does not require any calculations.

\begin{proof}
Start with an arbitrary lattice $V$ in which every point has distance 
at least $2$ from every other.  Center a unit ball at each point in
the lattice.  In a lattice of greatest density, some pair of balls
touch.  The lattice property then forces the balls into parallel
infinite linear strings like beads on a string.  Two of these infinite
parallel strings touch if the lattice is  optimal.  The
lattice property then constrains the strings in parallel sheets.  On
each sheet the touching parallel strings form a rhombic tiling.  Each
parallel sheet sits as snugly as possible on the sheet below in an optimal
lattice.  In such an arrangement, a ball (centered at
$\v_0$) of one sheet touches three balls (centered at
$\v_1,\v_2,\v_3$) on a the next layer down (Figure~\ref{fig:rhombus}).

\figAFRJFRK % fig:rhombus


As the balls on each sheet form a rhombic tile, two of the distances
between $\v_1,\v_2,\v_3$, corresponding to two edges of the rhombus, are
equal to $2$.  This means that $\v_0$ together with two of
$\v_1,\v_2,\v_3$ form an equilateral triangle.  

From the perspective of the plane containing this equilateral
triangle, the lattice property forces this entire plane, as well as
parallel planes, to be tiled with equilateral triangles.  From the
earlier argument, each of these planes sits as snugly as possible on
the sheet below.  A ball of one sheet touches the three balls in an
equilateral triangle on the layer below.  These four balls form a
regular tetrahedron, which uniquely identifies the lattice as the FCC.
\end{proof}






\section{Thue}\label{sec:thue}


As mentioned in the preface, Thue solved the packing problem for
congruent disks in the plane.  The optimal packing is the hexagonal
packing (Figure~\ref{fig:2D-hex}).  The density of this packing is
$\pi/\sqrt{12}$, that is, the ratio of the area of a unit disk to the
area of a hexagon of inradius one.  Thue's theorem admits an
elementary proof that we sketch.  Casselman has an 
interactive demo of this solution \cite{casselman:pennies}.

\figOCULYIA % fig:2D-hex

\begin{proof}
Let $V$ be the set of centers of a collection of unit disks in
$\ring{R}^2$.  Take the Voronoi cell around each
disk.\footnote{Voronoi cells of packings in any dimension $\ring{R}^n$
  are defined by the same rule as we gave above for $\ring{R}^3$.}   It
is enough to show that each Voronoi cell has density at most
$\pi/\sqrt{12}$ because the limiting density of the packing in the entire plane cannot exceed
a bound on the density within a Voronoi cell.  


Truncate the Voronoi cell by intersecting it with a disk of radius
$r=2/\sqrt3$.   The density increases as the volume of the cell is made smaller,
so if the truncated Voronoi cell
has density at most $\pi/\sqrt{12}$, then so does the untruncated Voronoi cell.

There is not a point $\w$ in the plane that has distance  less than $r$
from three disk centers $\v_1,\v_2,\v_3$.  Otherwise, one of the three
angles $\gamma$ at $\w$ formed by pairs $(\v_i,\v_j)$ of points
 is at most $2\pi/3$, and $\cos\gamma\ge -0.5$.
The \newterm{law of cosines} applied to the triangle $\w,\v_i,\v_j$ with angle
$\gamma$ and sides $a$, $b$, and $c$ gives the contradiction
   \[ 
   4 \le c^2 = a^2 + b^2 - 2 a b \cos\gamma 
   \le a^2 + b^2 + a b < 3r^2 = 4.
   \] 
Thus, the boundary of the truncated Voronoi cell consists of circular
arcs and chords of the circle of radius $r$, as shown in Figure~\ref{fig:2D-proof}.

\figSENQMWT % fig:2D-proof

The parts of the Voronoi cell that lie within a circular sector have
density $1/r^2 = 3/4 < \pi/\sqrt{12}$.  A simple calculation shows
that the part of a Voronoi cell that lies within a triangle has
density
   \begin{equation}\label{eqn:rog2d}
   \frac{\theta}{r^2 \cos\theta\sin\theta}
   \end{equation}
% checked 3/31/2008.
for some $0 \le \theta\le \pi/6$.  An easy optimization gives the maximum
at $\theta=\pi/6$ with value $\pi/\sqrt{12}$.  This completes the proof of Thue's theorem.
\end{proof}

In some ways it it unfortunate that the problem in two dimensions is
so elementary.  It gives only meager hints about how to solve the problem
in three dimensions such as the value of Voronoi cells
and the usefulness of truncation.  The optimization problem on
triangles in Equation~\ref{eqn:rog2d} generalizes to $n$-dimensions.
But beyond these simple observations,  little from the proof of Thue's
theorem prepares us for higher dimensions.

%\subsection{two dimensions}

\bigskip

\indy{Index}{decomposition!Delaunay}%

There are other proofs of Thue's theorem, including one by Fejes
T\'oth that uses the \fullterm{Delaunay triangulation}{decomposition!Delaunay} 
of a packing $V$
in the plane (or in $n$-dimensions).  A Delaunay triangulation of $V$
is a triangulation of Euclidean space into simplices with extreme
points in $V$ such that no point of $V$ lies in the interior of any
circumscribing circle of any of the simplices (Figure~\ref{fig:delaunay}).  
If $V$ is
\newterm{saturated},\footnote{A packing $V$ is saturated if it is not
  a proper subset of any other packing $V'$.  To maximize density, it
  is useful to increase the density by saturating the packing with
  additional points.} then a Delaunay triangulation of
$V$ exists.  Each Delaunay triangle in a saturated packing $V$ has
circumradius at most $2$ because otherwise an additional point can be
placed at the center of the circumscribing circle, contrary to saturation.

\figANNTKZP % fig:delaunay

\begin{proof}
  By admitting the existence of a Delaunay triangulation, the proof of
  the packing problem for saturated packings $V$ in two dimensions becomes
  elementary.  Each Delaunay triangle contains a portion of a disk at each of
  its three vertices.  The three interior angles of a triangle sum to
  $\pi$, giving half a disk per triangle.  If we show that each triangle has area
  at least $\sqrt{3}$, then it follows that the density of the packing is at most
  $(\pi/2)/\sqrt{3} = \pi/\sqrt{12}$.  The problem thus reduces to
  an area minimization problem.  To decrease the area of a triangle
  $\{\v_0,\v_1,\v_2\}$, we first replace it with a smaller similar
  triangle with shortest edge (say $\v_1\v_2$) of length $2$.  The
  third vertex $\v_0$ is constrained to have distance at least $2$
  from $\v_1$ and $\v_2$, and to have circumradius at most $2$.  The
  constraints on $\v_0$ form three circular arcs as shown in
  Figure~\ref{fig:delaunay-proof}.

\figCCKQLLH  % fig:delaunay-proof


The minimizing triangle is determined by the point $\v_0$ closest to
the line through $\v_1$ and $\v_2$.  There are three such triangles,
each with area exactly $\sqrt3$.  This completes the proof.
\end{proof}

\section{Dense Packings in a Nutshell}

This section describes the proof of the Kepler conjecture in general,
without
getting embroiled in detail.  The entire book
is a blueprint with all the electrical schematics, plumbing, and
ventilation systems.  This section is the tourist brochure.

The Kepler conjecture asserts that no packing of congruent balls in
three-dimensional Euclidean space has density greater than the density
$\pi/\sqrt{18} \approx 0.74048$ of the FCC packing.  For a contradiction, we suppose that an
explicit counterexample exists to the Kepler conjecture in the form of a
packing of balls of radius $1$ with density  greater than
$\pi/\sqrt{18}$.  Additional balls may be added to this packing until
saturation is reached.  The saturation of a counterexample may push its
density even higher.

We present the proof in four stages.  Undefined terms are clarified in the
discussion that follows.

\begin{enumerate}
\item A geometric partition of space, adapted to a saturated
  counterexample $V$, reduces the problem to finite packing $W$ that
  gives a counterexample to a particular inequality.  In notation
  established below, the particular inequality is $\CalL(W,\orz)\le
  12$ for every finite packing $W\subset B(\orz,2.52)$.  The
  counterexample satisfies $\CalL(W,\orz)>12$.
\item The finite packing $W$ is transformed into another finite packing that violates the same
inequality and that has a few additional properties that make it a \newterm{contravening} packing.
\item The combinatorial structure of $W$ is encoded as a hypermap.  A list is
  made of the purely combinatorial properties of $W$.  A hypermap with
  these properties is said to be \newterm{tame}.
\item A computer generates an explicit list, enumerating
  tame hypermaps up to isomorphism.  Linear programs, which are
  adapted to each tame hypermap in the enumeration, certify that
  none of the combinatorial possibilities can be  realized geometrically as a finite packing
  $W\subset \ring{R}^3$.  
\end{enumerate}
\indy{Notation}{L1@$\CalL(V)$ (estimation of a packing)}%

From the nonexistence of a counterexample $W$, 
it follows that there is no saturated
counterexample $V$ to the Kepler conjecture.



\subsection{geometric partition}

The first stage of the proof defines a geometric partition of space
and uses it to reduce the Kepler conjecture to an optimization problem
in a finite number of variables.

We recall that a saturated packing is
identified with the discrete set $V$ of centers of the
congruent balls.  Also, as above, the Voronoi cell $\Omega(V,\v)$
associated with $\v\in V$ is the polyhedron formed by all points of
$\ring{R}^3$ that are at least as close to $\v$ as to any other $\w\in
V$.   

The Voronoi cell at $\v$ can be further partitioned into Rogers
simplices, each of which is determined by a facet of the Voronoi cell, an edge of
the facet, and an extreme point of the edge.  The Rogers simplex is defined to be the
convex hull of four points: $\v\in V$, the closest point $\v_1$ to $\v$ on the given facet, the closest point $\v_2$ to $\v_1$ on the edge, and the
extreme point $\v_3$ of the edge (Figure~\ref{fig:rogers-intro}).

\figORQISJR % fig:rogers-intro

We dissect and combine the Rogers simplices somewhat further to make
them into \fullterm{Marchal cells}{Marchal cell} (Figure~\ref{fig:marchal-intro}).  
The exact rules for the
construction of Marchal cells do not concern us here.  The rules
depend on which of the points $\v_1,\ldots,\v_3$ have distance less
than $\sqrt2$ from $\v$.
\indy{Index}{decomposition!Marchal}%

\figODGBUWK % fig:marchal-intro

The function
 $\CalL(V,\v)$ is defined as
\begin{equation}\label{eqn:LV0}
\CalL(V,\v) = \sum_{\w\in V} L(\norm{\w}{\v}/2),
\end{equation}
where $L$ is the piecewise linear function that has a linear graph from
$(x,y)=(1,1)$ to $(0,1.26)$ and is equal to zero for $x\ge 1.26$.  (The
constants $1.26$ and $2.52=2(1.26)$ appear throughout the proof as
parameters used in truncation.)   
The sum in the definition of $\CalL$ is actually finite for every packing $V$ because only finitely many terms
lie in the support of $L$. 

Next, a function $G:V\to \ring{R}$ is defined geometrically in terms
of the volumes, solid angles, and dihedral angles of Marchal cells.
We do not give the definition here because it is rather
complex.  The function $G$ has the following two fundamental
properties:
\begin{enumerate}
\item If $\CalL(V,\v)\le 12$, then 
\[
4\sqrt{2}\le \Omega(V,\v) +G(\v).
\]
\item There exists $C>0$ such that the points of $V$ in a ball $B(\orz,r)$
of radius $r\ge 1$ satisfy
\[
\sum_{\v\in V \cap B(\orz,r)} G(\v) < C r^2.
\]
\end{enumerate}
The constant $4\sqrt{2}$ is the volume of the Voronoi cell of the FCC packing.

From these fundamental properties and from the assumption that $V$ is a saturated counterexample,
it follows that $\CalL(V,\v)>12$ for some $\v\in V$.  Indeed, if $\CalL(V,\v)\le 12$ for all
$\v\in V$, then the fundamental properties
imply that on average the Voronoi cells of $V$ have volume at least that of the FCC packing, up to a negligible error term $C r^2$.  From this, it follows that the density
of the packing $V$ is at most that of the FCC packing.



Returning to the counterexample $V$, we pick $\v\in V$ such that
$\CalL(V,\v)>12$.  By the translational invariance of the problem, we
may assume that $\v=\orz$.  Then
\begin{equation}\label{eqn:LW}
\CalL(W,\orz) = \sum_{\w\in W} L(\normo{\w}/2)  > 12,
\end{equation}
where $W$ is the finite set  $\{\w\in V\mid 0 < \normo{\w} \le 2.52\}$.

This completes the first stage of the proof.
%, by reducing the Kepler conjecture to an optimization
%problem $\max_W \CalL(W,\orz) \le 12$ in a finite number of variables $W\subset B(\orz,2.52)$.
The counterexample $V$ to the Kepler conjecture leads to a finite packing $W$
that satisfies~\eqref{eqn:LW}.

\subsection{contravening packing}

We assume that $V$ is a counterexample to the Kepler conjecture and
that $W\subset V$ is a finite subset that satisfies \eqref{eqn:LW}.
The second stage of the proof shows that the finite packing $W$ can be
enhanced in various ways.  The result of the enhancement is a new
finite packing that is a \fullterm{contravening
  packing}{contravening}.  At this stage, we also make $W$ into a
graph by defining a set of edges $E$ with nodes in $W$.

For example, the value of $\CalL$
depends only on the norms $\normo{\w}$, and $L$ is a decreasing
function, so that any rearrangement of the points of $W$ that does not
increase the norms strengthens the inequality \eqref{eqn:LW}.

The finite packing $W$ determines a graph $(W,E)$ with node set $W$.  The set
of edges is defined by $\{\v,\w\}\in E$ if 
\[2\le\norm{\v}{\w}\le 2.52.\] This graph is called the \newterm{standard
  fan} of $W$.

We can get a crude idea about what $W$ must look like by studying the
set of normalized points $\w/\normo{\w}$ in the unit sphere.  These
points can be used to partition the unit sphere into spherical
polygons.  As we know that the sum of the areas of the polygons equals
the area $4\pi$ of the sphere, we can extract bits of information
about $W$ from estimates of the areas of the polygons.  Analysis along
these lines leads to the conclusion that some finite packing $W$
has the following 
properties:
\begin{enumerate}\wasitemize 
\item $W\subset B(\orz,2.52)$.
\item $\CalL(W) > 12$.
\item The cardinality of $W$ is thirteen, fourteen, or fifteen.
\item $W$ maximizes the function $\CalL$.
\item Join points $\v/\normo{\v}$ and $\w/\normo{\w}$ with a geodesic arc on the
unit sphere if $\{\v,\w\}\in E$.  Then the arcs do not meet except at the endpoints and
give a planar graph.  Moreover, the angle between each pair of consecutive arcs at a vertex is less
that $\pi$.  In particular, the spherical polygons cut out by the arcs are convex.
\end{enumerate}\wasitemize 
A finite packing $W$ with these properties is called a \newterm{contravening} packing.


\subsection{tame hypermap}

The starting point of the third stage of the proof is a contravening
packing $W$ and the corresponding planar graph $(W,E)$.  The result of
this stage is a \fullterm{tame hypermap}{tame!hypermap} (described below).

By definition, a \fullterm{planar graph}{planar!graph} is a graph that admits a
\newterm{planar} embedding.  On the other hand, a graph, endowed with
a fixed embedding into the plane, is a \fullterm{plane graph}{plane!graph}.  A
planar graph has too little structure for our purposes because it
does not single out a particular embedding and the plane graph has too
much structure because it gives a topological object where combinatorics
alone should suffice.  A hypermap gives just the right amount of
structure.  It is a purely combinatorial notion, yet encodes the
relations among nodes, edges, and faces determined by the embedding.
An entire chapter of this book is about hypermaps.


The graph $(W,E)$ of a contravening packing $W$ determines a planar hypermap
$\op{hyp}(W,E)$. We study the following question: what  purely
combinatorial properties of the hypermap $\op{hyp}(W,E)$ can be derived from
the assumption  that
$W$ is a contravening packing?  For example, the cardinality of a
contravening packing $W$ is thirteen, fourteen, or fifteen.  Hence, the hypermap
has thirteen, fourteen, or fifteen nodes.  Much of the later chapters of the book
revolve around the question of the combinatorial properties of the
hypermap.

The final chapter of the proof compiles all of these combinatorial
properties into a long list.  
Although the exact details of the list are not significant,
%The list of properties is every bit as
%artificial as a top ten list of world wonders or unsolved mysteries.
%This The idea is to produce -- with minimal effort -- any list of
the list of combinatorial properties severely constrains the set of possible
hypermaps.  

Any hypermap satisfying all of these properties is said to be
\newterm{tame}.  This list of properties appears in
Definition~\ref{def:tame}.

%There is little to be gained by extra efforts at this stage because
%the final stage gives more efficient means to constrain the set of
%possibilities.

\subsection{linear programming}

The fourth and final stage completes the proof the nonexistence of
the contravening packing $W$.  At the beginning of this stage,
$\op{hyp}(W,E)$ is a tame hypermap.  The list of defining properties
of a tame hypermap are sufficiently restrictive that an explicit finite
list can be generated of every tame hypermap, up to isomorphism.  
This list is generated by computer.  The details of the algorithm are
described in the chapter on hypermaps.

Equipped with an explicit list of possible combinatorial structures,
we move to the proof's end game.  At this stage, because of the computer
generated list of tame hypermaps, the cardinality and
combinatorial structure of $W$ are explicit.

A list is made of the properties of $W$ (and its associated hypermap)
that can be described by linear inequalities.  For each tame hypermap,
a computer solves one or more linear programs that test for feasible
solutions to the system of linear inequalities.  In each case, the
computer produces a certificate that shows that no feasible solution
exists.  It follows that no tame hypermap can be realized in the form
$\op{hyp}(W,E)$.  Each tame hypermap, which represents a
combinatorially feasible arrangement, is geometrical infeasible.  It
follows that $W$, and hence also $V$, do not exist.

As no counterexample exists, the proof of the Kepler conjecture ensues.

%\section{Gallery}

%This section explores the history of packings and coverings through a
%series of figures.

%Harriot (Pascal's triangle) -- Marchal 2D -- Marchal 3D -- Rogers's
%proof (2D) -- Roger's (3D) -- Fejes T\'oth's proof 2D -- Fejes
%T\'oth's proposal 3D -- Hsiang 3D -- dodecahedral conjecture --
%Delaunay simplices 3D (conjectured best) -- Hales 3D (superposition)
%-- Hales 3D hybrid -- Beth Chen (tetrahedra) -- covering problem 2D --
%covering 3D -- heptagons (Kuperberg) -- atom packings -- circle
%packings -- Tammes problem -- van der Waerden 13 (Musin)

    \begin{runninglinenumbers*}
    %\chapter{Trigonometry}
\label{part:trig}
\indy{Index}{trigonometry}%

\begin{summary}
  This part of this book, which is the first of the four foundational
  chapters, presents a systematic development of trigonometry, volume,
  hypermap, and fan.  There is a separate chapter on each of these
  topics.  The purpose of the this material is to build a bridge
  between the foundations of mathematics, as presented in formal
  theorem proving systems such as HOL Light, and the solution to the
  packing problem.  

  In this chapter, trigonometry is developed analytically.  The basic
  trigonometric functions are defined by their power series
  representations, and calculus of a single real variable is used to
  develop the basic properties of these functions.  Basic vector
  geometry is presented.
\end{summary}


\section{Background Knowledge}

\subsection{formal proof}

We repeat that our purpose is to give a
blueprint of the formal proof of Kepler's conjecture that no packing of
congruent balls in three-dimensional Euclidean space has density
greater than the familiar cannonball packing.  The blueprint of a
formal proof is not the same as a formal proof, which is a 
fleeting pattern of bits in a computer.  The book describes to the
reader\footnote{``The words will be minced into atomized search-engine
  keywords~\dots{} copied millions of times by algorithms~\dots{}
  scanned, rehashed, and misrepresented by crowds\dots.  And yet
  it is you, the person, the rarity among my readers, I hope to
  reach.'' --Jaron Lanier \cite{Lanier}}  how to construct the
computer code that  produces and then reliably reproduces that
pattern of bits.

A more traditional book might take as its starting point the imagined
mathematical background of a typical reader.  The blueprint of a
formal proof starts instead with the current mathematical background
of a formal proof assistant.  I surveyed the knowledge of my formal
proof assistant and compared it with what is needed in the
construction of our formal proof.  It turns out that the proof
assistant already has an adequate background in real analysis, basic
topology, and plane trigonometry, including the trigonometric addition
laws, and formulas for derivatives.  Since the proof assistant already
has a significant library of theorems in real analysis and point-set topology, we
 use background facts in these areas wherever they help.


However, when this project began, the proof assistant lacked the
background in some of the less frequently used trigonometric
identities and has had nothing at all about spherical trigonometry.
While it had adequate command of general concepts of vector geometry
in $n$-dimensional Euclidean space, its knowledge of three-dimensional
analytic geometry was spotty.  For example, dihedral angles and
cylindrical and spherical coordinates were missing from the system.
%This foundational chapter supplies all of this necessary
%background in trigonometry and three-dimensional analytic geometry.

I imagine the typical reader to have a much stronger background in
trigonometry and analytic geometry than the proof assistant, which,
after all, is still in its youth.  The mathematician might want to
jump directly to the definition~\ref{def:aff} of the subsets
$\op{aff}_\pm$ of affine space.  This definition gives a compact
notation that encompasses many of the standard polyhedra (points,
lines, planes, rays, half-planes, half-spaces, convex hulls, affine
hulls) that appear throughout the book.  From there, the reader can
consult the definition of two important polynomials $\Delta$ and
$\ups$, make a note of the unorthodox notation $\arc(a,b,c)$ for the
angle opposite $c$ of a triangle with sides of lengths $a,b,c$, stop a
moment to admire Euler's formula for the solid angle of a spherical
triangle; and then jump directly to the final section, which
introduces polar cycle.

Polar cycle is a familiar concept, wrapped in an unfamiliar way
for the sake of the proof assistant: take a finite set of points in
the plane, order them by increasing angle, and then take the cyclic
permutation on the points induced by this order.  The azimuth cycle
is the corresponding permutation in three dimensions, ordering points by
increasing azimuth angle (longitude) in spherical coordinates.
Although intuitively clear, our proof assistant demands extra
assistance at this point.
\indy{Index}{longitude}%



\subsection{real analysis}
\label{back:analysis}  
This chapter assumes general facts about
real analysis at the level of a typical
undergraduate textbook.  In particular, it assumes a general working
knowledge of set theory and basic properties of the set of natural
numbers and the field of real numbers.  In real analysis, it assumes
basic properties of convergence, absolute convergence, limits, and
differentiation.  In this chapter, the term \newterm{real analysis} is
to be interpreted broadly to include even the most elementary facts of
real arithmetic, including results that do not involve limits.
\indy{Index}{real analysis}%
\indy{Index}{real arithmetic}%


\subsection{Tarski arithmetic}

\label{back:tarski}
  Certain sentences in real arithmetic can be expressed with nothing
  more than the usual logical operations (the connectives {\it and},
  {\it or}, {\it implies}, {\it logical negation}); the ring
  operations (addition, subtraction, and multiplication) for the real
  numbers; comparison ($(=)$ and $(>)$) of real numbers; the constants
  $0$ and $1$; real-valued variables; and quantifiers (universal and
  existential) over the real numbers.  Such sentences are said to
  belong to the Tarski arithmetic.  For example, the sentence
\begin{equation}\label{eqn:tarski}
\exists x.~x^7 - 4 x - 3 = 0 ~~\land~~ x > 0.
\end{equation}
falls within the Tarski arithmetic (after expanding the exponent $x^7$
as $x\cdot x\cdot x\cdot x\cdot x\cdot x\cdot x$ and the constants
$4=1+1+1+1$ and $3=1+1+1$).  Starting with Tarski, researchers have
developed algorithms to decide the truth of any sentence in the Tarski
arithmetic \cite{tarski-decision},~\cite{Mishra:1997}.  
Although these algorithms are generally too slow to be of practical
use, it is useful to identify such sentences.  To follow the details of proofs, 
reader should have the skill to solve particularly simple
problems in the Tarski arithmetic such as determining that the
sentence \eqref{eqn:tarski} is true.
\indy{Index}{Tarski arithmetic}%


\section{Trig Identities}


\subsection{sine and cosine}

The cosine and sine functions are defined\footnote{This is how the
  trigonometric functions were originally defined in the proof
  assistant HOL Light.  More recently, complex analysis has been
  developed in HOL Light sufficient for the analytic proof of
  the prime number theorem \cite{harrison:2009:pnt}.  The cosine and
  sine are now defined in the system as the real an complex parts of
  the exponential function $e^{i x}$.  To simplify the exposition, this section
  presents the original definitions.} by their infinite series:%
%\footformal{sin,\ cos,\ SIN\_0\, COS\_0}%




% from pgfmanual.pdf page 27, sec. 2.12
\begin{equation}\label{eqn:cos-def}\cos(x) = 1 - x^2/2! + x^4/4! \cdots,\qquad
  \sin(x) = x - x^3/3! + x^5/5! \cdots.
  \indy{Notation}{cos}%
  \indy{Index}{cosine}%
  \indy{Notation}{sin}%
  \indy{Index}{sine}%
  \indy{Index}{cosine!series definition}%
  \indy{Index}{sine!series definition}%
\end{equation}
%
\figODPCVGH % fig:trig

\mar{\guid{FOYTTIX} Eq.~\ref{eqn:cos-def}}
By real analysis, convergence is absolute
for every real number $x$.  Each series can be evaluated at $0$:
\begin{equation}\label{eqn:cos0}
  \cos(0) = 1,\qquad \sin(0) = 0.
\end{equation}
\mar{\guid{YIXJNJQ} Eq.~\ref{eqn:cos0}}


These series may be differentiated term by term to establish the
identities: \indy{Index}{cosine!derivative}%
\begin{equation}\label{eqn:cos'}
\frac{d\phantom{~}} {dx}\cos(x) 
= -\sin(x),\qquad \frac{ d\phantom{~} }{dx}\sin(x) = \cos(x).
\end{equation}
\mar{\guid{COHWECZ} Eq.~\ref{eqn:cos'}}%
The powers $(\cos(x))^n$ and $(\sin(x))^n$ are conventionally written
$\cos^n(x)$ and $\sin^n(x)$.

%Trigonometric identities follow easily from these definitions.    
If two functions are the {\it unique} solution of the same ordinary
linear differential equation with given initial conditions, then the
two functions are necessarily equal.  This observation gives
 a method to prove many functional identities,
including trigonometric identities.  
%This method can be developed
%further to give fully automated proofs of functional identities.  The
%intereseted reader may consult the mathematical literature of
%holonomic $D$-modules \cite{coutinho}, \cite{huishi-li},
%\cite{chyzak}.  % Chyzak % Huishi Li % Coutinho. p. 185.
The next two lemmas take this approach, by
%We do not strive to give a fully automated proof.  
 certifying a trigonometric identity with a function $f$ that
satisfies the ordinary differential equation $f' = 0$ with initial
condition $f(0)=0$.  
\indy{Index}{trigonometry!identities}%

\begin{lemma}[]\guid{WPMXVYZ}
\label{lemma:circle}\formal{SIN\_CIRCLE} 
\[ 
\sin^2(x) + \cos^2(x) = 1.
\] 
\end{lemma}
\indy{Index}{trigonometry!circle identity}%

% DEPRECATED HGMTQFG


\begin{proved}
  By real analysis and~\eqref{eqn:cos'}, the
  derivative of  $f(x) = \cos^2(x) +\sin^2(x)$ is
  identically zero, so the function itself is constant.
  From~\eqref{eqn:cos0}, it follows that $f(x)=f(0)=1$.
  \swallowed\end{proved}

%\footnote{
%  Incidentally, this trigonometric identity recently tried to crash
%  through the gates of physics.  Two robotics experts, Schmidt and
%  Lipson, wrote a computer program that automatically discovers
%  Hamiltons and Lagrangians from raw experimental data.  Discover
%  magazine reported that this program can discover the same laws in
%  hours that Newton took decades to find~\cite{discover-2009}.
%  However, one of the primary challenges of their project was to keep
%  out purely mathematical identities such as $\sin^2(x)+\cos^2(x)=1$,
%  which may try to pass as a conservation law with physical
%  significance~\cite{lipson}.
%}


\begin{lemma}[]\guid{WNYVJPE}\label{lemma:sin-add}
\formal{SIN\_ADD,\ COS\_ADD}
\begin{align*}
\sin(x+y) &= \sin(x)\cos(y) + \cos(x)\sin(y)\\
\cos(x+y)  &= \cos(x)\cos(y) - \sin(x)\sin(y).
\end{align*}
\end{lemma}
\indy{Index}{trigonometry!addition formula}%

%\figRUESSGQ % fig:cosadd  cos(x+y).

\begin{proved}
The proof is an exercise in real analysis.
Fix $y$.  Let
\begin{align*}
f(x) &=(\cos(x+y) - \cos(x)\cos(y) +
\sin(x)\sin(y))^2 \\ 
  &\quad+ (\sin(x+y) -\sin(x)\cos(y) -\cos(x)\sin(y))^2.
\end{align*}
The derivative of $f$ is identically zero.  The function is therefore
constant.  Also, $f(0)=0$.  Thus, $f$ is
identically zero.  If a sum of real squares is zero, the individual
terms are zero. The identities follow.  \swallowed\end{proved}

\begin{lemma}[]\guid{KGLLRQT}\label{lemma:cos-neg}
\formal{COS\_NEG,\ SIN\_NEG}
  The cosine is an even function.  The sine is an odd function.  That
  is,
\[ 
\cos(-x) = \cos(x),\quad\sin(-x) =
    -\sin(x).
\] 
\end{lemma}


\begin{proved}
The result can be checked directly from the definition of the trigonometric functions
as power series.  A second proof can be given by differentiation, as follows.
By real analysis, the derivative of
\[ 
(\cos(-x) - \cos(x))^2 + (\sin(-x)
  +\sin(x))^2
\] 
is identically zero.  Complete the proof as in the proof of
Lemma~\ref{lemma:sin-add}.  \swallowed\end{proved}

\subsection{periodicity}
\label{sec:pi}
\indy{Index}{periodicity}%

It is known that the cosine function has a unique root between $0$
and $2$. The constant $\pi$ is defined to be twice that root.  Thus, by
definition 
\begin{align}\label{eqn:cospi2}
\cos(\pi/2) &= 0,\nonumber\\
\cos(x) &>0,\quad \text{when } 0<x<\pi/2
\end{align}
\mar{\guid{CFXEKKP} Eq.~\ref{eqn:cospi2}}
The $\cos$ function is in fact
nonnegative on the interval $\leftclosed 0,\pi/2\rightclosed$:
\begin{equation}\label{eqn:cospos}
\cos(x)\ge 0,   \quad 0\le x \le \pi/2.
\end{equation}
\mar{\guid{ZSKECZV} Eq.~\ref{eqn:cospos}}
\indy{Index}{cosine!roots}%

\begin{lemma}[]\guid{CPIREMF}\label{lemma:sin-pi2}
\formal{SIN\_PI2}
$\sin$ is nonnegative on $[0,\pi/2]$ and  $\sin (\pi/2) = 1.$
\end{lemma}

\begin{proved}
  The proof is an exercise in real analysis.
  The derivative of $\sin$ is
  nonnegative between $0$ and $\pi/2$.  The
  value of $\sin$ at $0$ is $0$.  It follows that
  $\sin$ is nonnegative on $[0,\pi/2]$.  It is enough to check that
   $\sin^2(\pi/2)$ equals $1$.  Then $\sin^2(\pi/2)
  = {1-\cos^2(\pi/2)}
  = 1$.  \swallowed\end{proved}

\begin{lemma}[]\guid{SCEZKRH}\label{lemma:cos-sin}
\begin{align*}
\sin(\pi/2 - x)&=\cos(x),\\
\cos(\pi/2 - x)&=\sin(x).
\end{align*}
\end{lemma}

\begin{proved}
Apply the addition law  for the sine function (Lemma~\ref{lemma:sin-add}),
\[ 
\sin(\pi/2 - x) = \sin(\pi/2)\cos(-x) + \cos(\pi/2)\sin(-x)
\] 
and use $\sin(\pi/2) = 1$ and
$\cos(\pi/2) = 0$.  Then use that $\cos$ is an
even function.  The second identity is
similar.  \swallowed\end{proved}

Similarly,~%
%\footformal{SIN\_COS,\ SIN\_PERIODIC\_PI,\ COS\_PERIODIC\_PI, 
%SIN\_PERIODIC,\ COS\_PERIODIC}%
$\cos(\pi/2 + x) =
-\sin(x)$, $\sin(\pi/2 + x) = \cos(x)$.  Further,
\begin{alignat}{2}
\label{eqn:periodic}
\sin(\pi + x) &= \phantom{-}\cos(\pi/2 + x) &= -\sin(x),\nonumber\\
\cos(\pi + x) &= -\sin(\pi/2 + x) &= -\cos(x),\nonumber\\
\sin(2\pi + x) &= -\sin(\pi + x) &= \phantom{-}\sin(x),\\
\cos(2\pi + x) &= -\cos(\pi + x) &= \phantom{-}\cos(x)\nonumber.
\end{alignat}
\mar{\guid{LLOYXRK} Eq.~\ref{eqn:periodic}}%
\indy{Index}{trigonometry!periodicity}%

\begin{lemma}[]\guid{WIBGJRR}\label{lemma:sin-pos}
$\sin$ is nonnegative on $[0,\pi]$.
\end{lemma}

\begin{proof} By Lemma~\ref{lemma:sin-pi2}, $\sin$ is nonnegative on
  $[0,\pi/2]$.  Furthermore, for $x\in[\pi/2,\pi]$,
\[ 
  \sin(x) = -\sin(-x)   =  \sin(\pi-x) \ge 0.
\] 
\end{proof}



\subsection{tangent}
\label{sec:tangent}

\begin{definition}[tangent]\guid{BIRXGXP}\label{def:tan}
\formaldef{$\tan$}{tan}
Let $\tan(x) = \sin(x)/\cos(x)$, defined when $\cos(x)\ne0$.
%(See Figure~\ref{fig:trig}.)
\indy{Index}{tangent}%
\indy{Notation}{tan@$\tan$}%
\end{definition}



\begin{lemma}[]\guid{KWYPRWZ}
\label{lemma:tan-add}\formal{TAN\_ADD}
If $\cos(x)\ne 0$, $\cos(y)\ne 0$, and $\cos(x+y)\ne0$ then
\[ \tan(x+y) = \frac{\tan(x) + \tan(y) }{ 1 -
    \tan(x)\tan(y)}\] 
\end{lemma}
\indy{Index}{trigonometry!tangent}%

\begin{proved}
  Divide the first line of Lemma~\ref{lemma:sin-add} by the second
  line of the same lemma.  Then use the definition
  of $\tan$.  \swallowed\end{proved}

\begin{lemma}[]\guid{KSQDZSF}\label{lemma:tan-pi4}\formal{TAN\_PI4}
\[ \tan(\pi/4) = 1.\] 
\end{lemma}

\begin{proved}  
\[ 
\tan(\pi/4) = \sin(\pi/2-\pi/4)/\cos(\pi/4) 
  =
  \cos(\pi/4)/\cos(\pi/4) = 1.
\] 
\swallowed\end{proved}

\begin{lemma}[]\guid{UTNKIAC}\label{lemma:tan-monotone}
The function $\tan$ is strictly increasing and one-to-one on the domain
$\leftopen-\pi/2,\pi/2\rightopen$.
\end{lemma}

\begin{proof} By a derivative test, the function $\tan$ is strictly
  increasing on $\leftopen-\pi/2,\pi/2\rightopen$.  By
  real arithmetic, a strictly increasing
  function is one-to-one.
\end{proof}

\subsection{arctangent}

This section reviews the properties of the arctangent function.  

\begin{definition}[arctangent]\guid{RIQVMHH}\label{def:arctan}
\formaldef{$\arctan$}{atn}
\formal{atn,\ ATN,\ ATN\_TAN,\ ATN\_BOUNDS,\ TAN\_ATN}
  By the inverse function theorem of real
    analysis and properties of $\tan$,
  there is a unique function $\arctan:\ring{R}\to\ring{R}$ with image
  $(-\pi/2,\pi/2)$ such that
\begin{equation}\label{eqn:tanarctan}\tan(\arctan x) =x.\end{equation}
(See Figure~\ref{fig:trig}.)
\mar{\guid{EWITKLU} Eq.~\ref{eqn:tanarctan}}
\indy{Index}{arctangent}%
\end{definition}
%


Additional properties of the arctangent function are exercises in
real analysis.  If $-\pi/2 < x < \pi/2$,
then also $\arctan(\tan(x)) = x$. In particular,%
%\footformal{ATN\_1}
\begin{equation}\label{eqn:arctan-1}\
\arctan(1) = \arctan(\tan(\pi/4)) = \pi/4.
\end{equation}
\mar{\guid{YTXYLRB} Eq.~\ref{eqn:arctan-1}}  % X->Y


The function $\arctan$ is differentiable with derivative%
%\footformal{\ ATN\_MONO\_LT,\ ATN\_MONO\_LT\_EQ}
\begin{equation}\label{eqn:deriv-tan}\frac{d\phantom{~}} {dx} \arctan(x) = \frac{1}{1 +
    x^2}.\end{equation}
\mar{\guid{OKENMAM} Eq.~\ref{eqn:deriv-tan}}
The derivative is everywhere positive, and the function $\arctan$ is
strictly increasing.   \mar{\guid{LQCXGZX} increasing}
\indy{Index}{arctangent!derivative}%
Proofs in this book often need to use $\arctan(y/x)$ as  $x$ approaches $0$.
For this, the following variant of $\arctan$ is preferable because it clears the denominator.


\begin{definition}[$\atn$]\guid{GYKGARD}\label{def:atn}
\formaldef{$\atn$}{atn2}
\[ 
\atn: \ring{R}^2 \to \leftopen-\pi,\pi\rightclosed.
\] 
\[ 
\atn(x,y) = \begin{cases}
\arctan(y/x), & x > 0\\
\pi/2- \arctan(x/y), & y > 0 \\
\pi + \arctan(y/x), & x< 0,\  y\ge 0\\
-\pi/2- \arctan(x/y), & y< 0 \\
\pi, & x= y=0.\\
\end{cases}
\] 
\end{definition}
\indy{Notation}{arctan2@$\atn$}%
\indy{Notation}{arctan@$\arctan$}%
\indy{Index}{arctangent!atn@$\atn$}%
%
\figYOXQFUB % fig:atn-polar


There is some overlap between cases. Nevertheless, trig identities
similar to those already established show that this function is
well-defined.  For example, to check the equality of the first two
cases, we compute the tangent of both sides, which is sufficient,
since both sides lie between $\leftopen-\pi/2,\pi/2\rightopen$ and
$\tan$ is one-to-one:
\[ 
  \tan(\arctan(y/x)) = y/x = 
  1/\tan(\arctan(x/y)) = \tan(\pi/2 - \arctan(x/y)).
\] 
We can give a more intuitive description of
the function $\atn$:    the polar angle of $(x,y)$ with the
branch cut along the negative axis.  That is, $x = r\cos\theta$ and
$y=r\sin\theta$ for some $r\ge0$, where $\theta=\atn(x,y)$.  This definition avoids all
the case distinctions of Definition~\ref{def:atn}.

The ANSI C programming language implements this function as {\it
  arctan2}.  Note that some programming languages implement this
function with the two arguments in reverse: $(y,x)$.
\indy{Index}{arctangent!near 0}%
%\indy{Notation}{r@$r$ (coordinate)}%
\indy{Notation}{r@$r$ (polar, cylindrical, and spherical radius)}%
\indy{Notation}{xy@$(x,y)$ (Cartesian point)}%
%\indy{Notation}{zzh@$\theta$ (coordinate)}%
\indy{Notation}{zzh@$\theta$ (polar, cylindrical, and spherical angle)}%


\subsection{inverse trig}
\indy{Index}{trigonometry!inverse}%

We prefer the arctangent over other inverse trigonometric functions
because its domain is the entire field of real numbers, its range is
bounded, and its derivative is a rational function.  Wherever angles
appear in this book, the arctangent is apt to appear as well.  Other
inverse trigonometric functions are generally  reduced to the
arctangent.  This section defines the $\arccos$ function and shows how
it can be expressed in terms of $\atn$.

\begin{definition}[arccos]\guid{QZTBJMH}
\formaldef{$\arccos$}{acs}
\label{def:arccos}\formal{acs,\ ACS\_COS,\ COS\_ACS}
  By the inverse function theorem of real
    analysis, there exists a unique function $\arccos y$ on the
  interval $[-1,1]$, which takes values in $[0,\pi]$ and which is the
  inverse function of $\cos$:
\begin{align*}
y\in [-1,1] &\Rightarrow \cos(\arccos y) = y\\
x\in[0,\pi] &\Rightarrow \arccos(\cos x) = x
\end{align*}
\indy{Index}{arccosine}%
\indy{Notation}{arccos}%
%(See Figure~\ref{fig:trig}.)
\end{definition}


\begin{lemma}[]\guid{FMGMALU}\label{lemma:sin-arccos}
\formal{sin\_acs\_t} 
  If $y\in[-1,1]$, then
\[ \sin(\arccos(y)) = \sqrt{1-y^2}.\] 
\end{lemma}

\begin{proved}
  The range of $\arccos(y)$ is $[0,\pi]$.  On
  this interval, $\sin$ is nonnegative.  By
  real analysis, it is enough to check that
  the squares of the two nonnegative numbers are equal.  It then an
  arithmetic consequence of the circle identity
  (Lemma~\ref{lemma:circle}) and Definition~\ref{def:arccos}.
  \swallowed\end{proved}

The following lemma shows how to rewrite any occurrence of the $\arccos$ function
in terms of  $\atn$.   
%Our preference is to remove the $\arccos$ function whenever
%possible, by replacing it with the $\atn$ function through the
%following identity.  


\begin{lemma}[]\guid{OUIJTWY}\label{lemma:arccos-arctan}
\formal{acs\_atn2\_t}  
  If $y\in [-1,1]$, then
  \[ \arccos(y) +  \atn({
      \sqrt{1-y^2}},{y}) = \pi/2.\] 
\end{lemma}
\indy{Index}{trigonometry!arccos}%
\indy{Index}{trigonometry!arctan}%
\indy{Notation}{arccos}%
\indy{Notation}{arctan@$\arctan$}%
\mar{FIGURE:The two acute angles of a right triangle have sum $\pi/2$.}

\begin{proved}
The brief justification is simply that 
$\arccos(y/z)$ gives one acute angle of a right triangle with
hypotenuse $z$ and sides $x$ and $y$, and $\atn(x,y)$ gives the other acute angle.
The two acute angles of a right triangle have sum $\pi/2$.

A bit more detail is needed for an argument that can be turned into a formal proof.
  The endpoints $y=\pm1$ can be checked directly from definitions.  If
  $y\in (-1,1)$, $\beta = \arccos(y)$, and \[ \alpha =
    \arctan(y/\sqrt{1-y^2}) =
    \atn({\sqrt{1-y^2}},{y}),\]  then arithmetic gives
  $-\pi/2 < \pi/2 - \beta < \pi/2$, and $-\pi/2 < \alpha
    < \pi/2$.  By the injectivity of
  the function $\tan$, it is therefore enough to check that
  $\tan(\pi/2 - \beta) = \tan(\alpha)$.  But
\[ 
\tan(\pi/2-\beta)=
\frac{\cos(\beta)}{\sin(\beta)} =
\frac{y}{        \sin(\arccos(y))} 
=\frac{y}{ \sqrt{1-y^2}} 
=\tan(\alpha).
\] 
\swallowed\end{proved}



\section{Vector Geometry}

This section reviews vector geometry in $\ring{R}^N$, including
products (scalar and dot), inequalities (triangle and Cauchy-Schwarz),
and hulls (convex and affine).

\subsection{Euclidean space}

\begin{definition}[$\ring{R}^N$,~vector]\guid{KRZJIAD}
\formaldef{$\ring{R}^N$}{:real\textasciicircum N}
  For any finite set $N$, define $\ring{R}^N$ as the set of functions
  $\v:N\to\ring{R}$. Write $v_i$ for the value of the function $\v$ at
  $i\in N$. 
  \indy{Notation}{reals@$\ring{R}^N$}%
  A function in $\ring{R}^N$ is called a \newterm{vector}.  The zero
  vector $\orz$ is the function that is identically zero.
  \indy{Index}{vector}%
\end{definition}
\indy{Index}{vector!zero}%
Vectors are written in a bold face: $\u$, $\v$, $\w$, $\p$, $\q$, and
so forth.  As a general notational practice, there is a general
tendency to use $\u$, $\v$, and $\w$ to denote vectors that are
constrained to lie in some previously determined subset $V\subset
\ring{R}^N$ and to use $\p$ and $\q$ to denote vectors that run
without restriction over all of $\ring{R}^N$.

No distinction is made between vectors and points in $\ring{R}^N$, and
none is made between $\ring{R}^N$ and Euclidean space.  Write
$\ring{R}^n$ as an alias of $\ring{R}^N$ when $n\in\ring{N}$ and
$N=\{0,\ldots,n-1\}$.  
\indy{Index}{Euclidean space}%

\begin{definition}[vector addition,~scalar multiplication]\guid{WHIAXYC} % X->Y
\formaldef{vector addition}{(+)}
\formaldef{scalar multiplication}{(\%)}
  Two standard arithmetic operations, addition and scalar
  multiplication, are defined on the set $\ring{R}^N$.  These
  operations are the pointwise addition and scalar multiplication of
  functions:
\begin{align}
(\u + \v)_i &= u_i + v_i.\nonumber\\
(t \u)_i &= t u_i,\quad t\in\ring{R}.
\end{align}
\indy{Index}{vector!addition}%
\indy{Index}{vector!scalar multiplication}%
Define the difference of two vectors to be $\u - \v = \u + (-1) \v$.
\indy{Index}{vector space} %
\indy{Index}{vector!subtraction}%
\end{definition}
The operations on $\ring{R}^N$ 
satisfy the axioms of a vector space. 
In particular, addition is commutative and associative.


\begin{definition}[dot product]\guid{VFPCZBI}
\label{def:dot}
\formaldef{dot product}{(dot)}
The  \newterm{dot product} $(\,\cdot\,)$ is the
 bilinear binary operation on $\ring{R}^N$
%\[ 
%(\cdot):\ring{R}^N\to\ring{R}^N\to\ring{R}
%\] 
defined by
\[ 
\u\cdot \v = \sum_{i\in N} u_i v_i.
\] 
\indy{Index}{vector!dot product}%
\indy{Notation}{4@$\cdot $ (dot product)}%
\end{definition}


The dot product satisfies the following
properties:
\begin{align}\label{eqn:dot}
\u \cdot (\v + \w) &= \u \cdot \v + \u \cdot \w\nonumber\\
(\u + \v)\cdot \w &= \u \cdot \w + \v \cdot \w\nonumber\\
(t \u)\cdot \w &= t(\u \cdot \w) = \u \cdot (t \w)\\
0 &\le \u\cdot \u\nonumber
\end{align}


\begin{definition}[norm]\guid{XHVXJVB}
\label{def:norm}
\formaldef{norm}{vector\_norm}
The \newterm{norm} of a vector $\u\in\ring{R}^N$ is
\[ \normo{\u} = \sqrt{\u\cdot \u}.\] 
\indy{Index}{vector!norm}%
\end{definition}
%\indy{Notation}{norm@\hbox{$\normo{\u}$} (vector norm)}%

By  real arithmetic,
$\normo{\u}=0$  if and only if $\u=\orz$.  Moreover,
$\normo{ t \u } = |t| \, \normo{\u}$.   

% The distance function $d(\u,\v) = \norm{ \u }{ \v}$ makes
% $\ring{R}^N$ into a metric space.  \indy{Index}{metric space}%
% The proof that $d$ is indeed a metric depends on the Cauchy-Schwarz
% inequality:


\begin{lemma}[Cauchy-Schwarz~inequality]\guid{JJKJALK}
\formal{Jordan/metric\_spaces.ml:cauchy\_schwartz}
  \[ |\u \cdot \v| \le
    \normo{\u}\,\normo{\v}.\]  Furthermore, the case
  $\pm \u\cdot \v = \normo{\u}\,\normo{\v}$ of equality holds exactly
  when $\normo{\v} \u = \pm\normo{\u} \v$ (with matching signs).
\end{lemma}
\indy{Index}{Cauchy-Schwarz inequality}%

\begin{proved}
  This is an exercise in real arithmetic.  Let $\w = \normo{\v} \u \pm
  \normo{\u} \v$.  The expansion of $\w\cdot \w$ gives
  \[ 0\le \w\cdot \w = 2\normo{\u}^2\normo{\v}^2 \pm
    2\normo{\u}\, \normo{\v} (\u\cdot \v) = 2\normo{\u}\, \normo{\v}
    (\normo{\u}\, \normo{\v} \pm (\u \cdot \v)).\]  If
  $2\normo{\u} \,\normo{\v} = 0$, then $\u$ or $\v$ is zero, and the
  result easily ensues.  Otherwise divide both sides of the
  inequality by the positive quantity $2 \normo{\u} \,\normo{\v}$ to
  get the result.  \swallowed\end{proved}

\begin{lemma}[triangle~inequality]\guid{OIPLPTM}
\formal{Jordan/metric\_spaces.ml:norm\_triangle}
\label{lemma:triangle-ineq}
\[ 
\normo{\u + \v} \le \normo{\u} + \normo{\v }.
\] 
Equality holds exactly when $\normo{\v}\u = \normo{\u}\v$.
\end{lemma}
\indy{Index}{triangle inequality}%

\begin{proved}  This is an exercise in real arithmetic.
Both sides are nonnegative; it is enough to compare the squares of
both sides.  By the Cauchy-Schwarz inequality,
\[ \normo{\u + \v}^2 = \u\cdot \u + 2 \u\cdot \v + \v\cdot \v \le
  \u\cdot \u + 2 \normo{ \u}\,\normo{\v} + \v\cdot \v = (\normo{\u}+\normo{\v})^2.
\] 
The case of equality follows from the case of equality in the
Cauchy-Schwarz inequality.
\swallowed\end{proved}



\subsection{affine geometry}




Most of the following definitions apply to
  $n$-dimensional Euclidean space; however, this book uses them only
  in two and three dimensions.  The first definition gives the affine
  span of a finite set.  For example, the affine span of two distinct
  points is a line; the affine span of three independent points is a
  plane.  By placing additional positivity constraints on the linear
  combinations, the definitions extend to a large assortment of other
  geometric objects such as rays, half-planes, convex hulls, and
  cones.  Each of these comes in two versions: an open version defined
  by strict inequality and a closed version defined by weak
  inequality.  For example, the closed half-plane includes a bounding
  line and the open half-plane does not.  In this chapter, open and
  closed are not topological notions; rather, they indicate the
  semialgebraic conditions of strict and weak inequality.


%% No notation is introduced for a general affine hull!
\begin{definition}[affine hull]\guid{KVLZSAQ}
\formaldef{$\op{aff}$}{(hull) affine}
A set $A\subset\ring{R}^N$ is \newterm{affine}, if for
every finite nonempty subset $S\subset A$ and every function $t:S\to\ring{R}$ such that $\sum _{\v\in S} t(\v)=1$, we have
% $\v,\w\in A$ and every $t \in \ring{R}$, 
\[ 
 \sum_{\v\in S} t(\v) \v \in A.  % t \v + (1-t) \w \in A.
\] 
The \fullterm{affine hull}{affine!hull}, $\op{aff}(S)$, of a set $S\subset\ring{R}^N$ is the smallest affine set
containing $S$. 
That is, the affine hull of $S$ is the intersection of all affine
sets containing $S$. 
\end{definition}
\indy{Index}{hull!affine}%



\begin{definition}[affine]\guid{BYFLKYM}\label{def:aff} 
\formaldef{$\op{aff}_\pm$}{aff\_ge, aff\_le}
\formaldef{$\op{aff}^0_\pm$}{aff\_gt, aff\_lt}
  If $V = \{\v_1,\v_2,\ldots,\v_k\}$ and $V'=\{\v_{k+1},\ldots,\v_n\}$
  are finite subsets of $\ring{R}^N$, then set
	\begin{align*}
\op{aff}_{\pm} (V,V') &= \{t_1 \v_1 +\cdots t_n \v_n \mid
	t_1 +\cdots+t_n = 1, \pm t_j \ge 0, \text{ for } j>k\},\\
\op{aff}^0_{\pm} (V,V') &= \{t_1 \v_1 +\cdots t_n \v_n \mid
	t_1 +\cdots+t_n = 1, \pm t_j > 0, \text{ for } j>k\}.
%\op{aff}\, V &= \op{aff}_\pm(V,\emptyset).\\
		\end{align*}
To lighten the notation for singleton sets, abbreviate
$\op{aff}_\pm(\{\v\},V')$ to $\op{aff}_\pm(\v,V')$.
\indy{Notation}{aff2@$\op{aff}_{\pm}$, $\op{aff}^0_{\pm}$}%
\indy{Index}{affine}%
\indy{Notation}{V@$V\subset\ring{R}^n$}%
\end{definition}

\figITGCYIF % fig:affset. 

\begin{remark}
  When $n+1=\card(V)+\card(V')$, the generic set $\op{aff}_+(V,V')$ is
  an $n$-dimensional polyhedron bounded by $\card(V')$ hyperplanes.
  For example, $n=1$, gives a segment, a ray, or a line
  (Figure~\ref{fig:affset}).  When $n=2$, the set is a $2$-simplex, a
  planar wedge bounded by two lines, a half-plane, or a plane.  When
  $n=3$, the set is a $3$-simplex; an unbounded connected region in
  space bounded by one, two, or three intersecting planes; or all of
  $\ring{R}^3$.
\end{remark}

\begin{definition}[convex hull]\guid{OWECYNV}
\formaldef{$\op{conv}$}{(hull) convex}
A subset $C\subset\ring{R}^N$ is \newterm{convex}, if for
every $\v,\w\in C$ and every $t \in \leftclosed0,1\rightclosed]$,
\[ 
t \v + (1-t) \w \in C.
\] 
If $S\subset\ring{R}^N$, then let $\op{conv}(S)$ be the smallest convex set
(or equivalently, the intersection of all convex sets)
containing $S$.  It is called the \newterm{convex hull}.
\end{definition}

When the set is finite, the convex hull takes the following form.

\begin{lemma}[]\guid{GDCZMLO}
If $V = \{\v_1,\v_2,\ldots,\v_n\}\subset\ring{R}^N$, then
	\[ 
\op{conv}\, V = \op{aff}_+\, (\emptyset,V)
\] 
\indy{Notation}{conv}%
\indy{Index}{convex hull}%
\end{lemma}

\begin{lemma}[]\guid{UIVNNRR}
If $V\subset\ring{R}^N$ is finite, then
$\op{aff}_\pm (V,\emptyset) = \op{aff}^0_\pm(V,\emptyset)$
is the affine hull of $V$.
\end{lemma}

\begin{proof}  Both proofs are left as  exercises for the reader.
\end{proof}

In the following definition of a cone, the point $\v$ serves as apex,
and $V$ is a generating set for the positive directions.  In the
special case that $V$ is a singleton $\{\w\}$, the cone gives a ray
originating at $\v$ and passing through $\w$.  Later chapters call
a set of the form $\op{aff}_+(\v,\{\u_1,\u_2\})$ a \newterm{blade}.
Blades are planar sets bounded by two rays originating at $\v$.
\indy{Notation}{v7@$\v\in\ring{R}^3$}%
\indy{Index}{blade}%

% Cone deprecated Jan 17, 2012.
%\begin{definition}[cone]\guid{LLOUBAX}
%\formaldef{$\op{cone}$}{cone}
%Let $V$ be a finite subset of
%$\ring{R}^N$ and let $\v\in\ring{R}^N$. Set
%\[ 
%\op{cone}(\v,V) = \op{aff}_+(\{\v\},V)
%\op{cone}^0(\v,V) &= \op{aff}^0_+(\{\v\},V)\\
%\] 
%\indy{Index}{cone}%
%\indy{Notation}{cone@$\op{cone}$}%
%\end{definition}

% The Voronoi cell is one of the fundamental geometric objects in this
% book.  Earlier chapters have already discussed it at great length.
% Some authors use a weak inequality in the definition, others strict.
% The definition takes strict inequalities.


%\begin{definition}[Voronoi cell $\Omega$]\guid{VOWVHBW}
%Let $V$ be a finite set of points in 
%$\ring{R}^3$.  Let $\v\in\ring{R}^3$. Set
% \[ 
%\Omega(\v,V) = 
%\{x \mid \norm{\v}{x} \le \norm{\w}{x} forall \w\in V\setminus\{\v\}\}
%\] 
%\end{definition}

%% Changed to weak inequality May 14, 2009. -tchales.
	
\begin{definition}[line,~collinear,~parallel]\guid{SWKFLBJ}
\formaldef{line}{line}
\formaldef{collinear}{collinear}
\formaldef{parallel}{parallel}
  Any set of the form $\op{aff}\{\v,\w\}$ is a \newterm{line} when
  $\v\ne \w$.  A set that is contained in some $\aff\{\v,\w\}$ is
  \newterm{collinear}.  If $\{\orz,\v,\w\}$ is collinear, then
  $\v$ and $\w$ are said to be \newterm{parallel}. Also, $\{\v,\w\}$
  is said to be a parallel set.
\end{definition}
\indy{Index}{line}%
\indy{Index}{collinear}%

\begin{definition}[plane, half plane, coplanar]\guid{JLWZFBH}\label{def:plane}
\formaldef{plane}{plane}
\formaldef{half-plane}{closed\_half\_plane, open\_half\_plane}
\formaldef{coplanar}{coplanar}	
  An affine hull $A=\op{aff}\{\u,\v,\w\}$ is a \newterm{plane} when
  $\{\u,\v,\w\}$ is not collinear.  A set $\op{aff}_\pm(\{\u,\v\},\{\w\})$
  is a \newterm{half-plane} when $\{\u,\v,\w\}$ is not collinear. A
  set that is contained in some $\aff\{\u,\v,\w\}$ is \newterm{coplanar}.
\end{definition}
\indy{Index}{plane}%
\indy{Index}{half-plane}%
\indy{Notation}{A@$A$ (plane)}%
\indy{Index}{coplanar}%


\begin{definition}[half space]\guid{OAUVFPS} 
\formaldef{half space}{closed\_half\_space, open\_half\_space}
A set
  $\op{aff}_{\pm}(\{\u,\v,\w\},\{\v'\})$ is a \newterm{half-space},
  when $\{\u,\v,\w,\v'\}$ is not coplanar.  Under the substitution of
  $\op{aff}_{\pm}$ for $\op{aff}_{\pm}^0$, it is called an
  \newterm{open half-space}.
\end{definition}
\indy{Index}{half-space}%
\indy{Index}{half-space!open}%

\subsection{parallelepiped}\label{sec:piped}
\indy{Index}{parallelepiped}%



The following polynomial, $\Delta$, appears in many different
functions related to the geometry of three dimensions.  The formula
following the definition shows that it is closely related to the
square of the volume of a parallelepiped.  The interpretation as
volume is not relevant until the next chapter, but  its nonnegativity is
immediately relevant.  

%% WW repeated DEF.
\begin{definition}[$\Delta$]\guid{AVWKGNB}\label{def:delta}
\formaldef{$\Delta$}{delta\_x}
\formal{definitions\_kepler.ml:delta\_x}
  Let
\begin{align*}
\Delta(x_1,\ldots,x_6) &= x_1 x_4 (- x_1+x_2+x_3- x_4+x_5+x_6)\\
&\qquad+x_2 x_5 (x_1- x_2+x_3+x_4- x_5+x_6)\\
&\qquad+x_3 x_6 (x_1+x_2- x_3+x_4+x_5- x_6)\\
&\qquad- x_2 x_3 x_4- x_1 x_3 x_5- x_1 x_2 x_6- x_4 x_5 x_6.
\end{align*}
\end{definition}
\indy{Notation}{zzD@$\Delta$}%
\indy{Index}{Cayley-Menger!determinant}%
\indy{Index}{determinant!Cayley-Menger}%


\begin{remark}[Cayley-Menger determinant]\guid{DQGHCSH}\label{rem:cayley}
  The polynomial $\Delta$ first appears in the following context.
  Cayley and Menger found a formula for the square of the determinant
  $D$ of the matrix with rows $\v_1-\v_0$, $\ldots,$ $\v_n-\v_0$ for
  arbitrary vectors $\v_i\in\ring{R}^n$.  Set
\begin{equation}\label{eqn:xij}
x_{ij} = \norm{\v_i}{\v_j}^2,
\end{equation}
arranged as entries of a matrix $[x_{ij}]$.
Write $\underbar 1$ for a row vector of length $n$ 
with entries that are all equal to $1\in\ring{R}$.
They found that elementary matrix manipulations give an identity
of determinants:
\begin{align}\label{eqn:cmd}
D^2 &= \frac{(-1)^{n-1}}{2^n}
\left|\begin{matrix}[x_{ij}]& {}^t{\underbar 1}\\ {\underbar 1}& 0
\end{matrix}\right|.
\end{align}
The right-hand side is a polynomial in the squares of the edge lengths.
% The special case $n=2$ gives the polynomial $\ups$
% (Definition~\ref{def:ups}).
\indy{Notation}{D@$D$ (determinant)}%

A calculation of the determinant on the right when $n=3$ yields 
the polynomial $\Delta$.
\[ 
4 D^2 = \Delta(x_{01},x_{02},x_{03},x_{23},x_{13},x_{12}).
\] 
The left-hand side is evidently a square and the polynomial on the
right is nonnegative, whenever the variables $x_{ij}$
satisfy~\eqref{eqn:xij} for some vectors
$\v_0,\ldots,\v_3\in\ring{R}^3$.  Moreover, $D$ and hence also
$\Delta$ is positive when the set of four vectors is not
coplanar.
\end{remark}
\indy{Index}{edge!length}%
\indy{Notation}{zzu@$\ups$ (polynomial)}%
\indy{Notation}{zzD@$\Delta$}%
%\indy{Notation}{xij@$x_{ij}=\norm{\v_i}{\v_j}$}% % doesn't parse
%Write $\Delta_j$ for the $j$th partial derivative of $\Delta$. 
%Let $D = \det(\v_2-\v_1,\v_3-\v_1,\v_4-\v_1)$.
\indy{Index}{determinant}%

\begin{background}[matrix theory]\label{back:matrix}
  Very little matrix theory is required in this book.  The next lemma
  is a rare exception.  Its proof requires various very basic facts
  about $3\times 3$ matrices and determinants.  The determinant of a
  product of two matrices is the product of determinants.  The
  transpose of a matrix $A$ has the same determinant as $A$.  The
  determinant of a matrix $A$ is zero if and only if there exists a
  (row) vector $\u$ such that $\u\, A = \orz$.
\end{background}

\begin{lemma}[]\guid{CTCZHMR}\label{lemma:delta-pos}
  Let $V=\{\v_0,\v_1,\v_2,\v_3\}\subset\ring{R}^3$.  Let $x_{ij} =
  \norm{\v_i}{\v_j}^2$.  Then $\Delta(x_{ij})\ge 0$.  Moreover, the
  set $V$ is coplanar if and only if $\Delta(x_{ij}) = 0$.%
\mar{\guid{LBEVAKV} $\ge$ LEG}  % delta >=0, 
\mar{\guid{POLFLZY} $=$ LEG} % coplanar <=> delta=0.
\mar{more on $\Delta$ in LEG}
\end{lemma}

\begin{proof} The proof is an exercise in
  matrix theory and
  real arithmetic.  (The statement also
  falls within the scope of Tarski
  arithmetic.)  This lemma can be proved directly as follows, without
  recourse to the general Cayley-Menger theorem.

  Let $A$ be the $3\times 3$ matrix with rows  $\v_i - \v_0$.  Then
  $D^2 = \det(A)^2 = \det(A \,\hbox{}^t A)$.  Each entry of the
  product $A\,\hbox{}^tA$ is a dot product $(\v_i-\v_0)\cdot
  (\v_j-\v_0)$, which can be expressed in terms of the constants
  $x_{ij}$ by the following identity:
\begin{align}\label{eqn:dot-law}
  2 (\v_i-\v_0)\cdot (\v_j-\v_0) 
&= (\v_i-\v_0)\cdot(\v_i-\v_0) + (\v_j-\v_0)\cdot (\v_j-\v_0)\nonumber \\
  & \qquad - (\v_i-\v_j)\cdot (\v_i-\v_j)\nonumber\\
&= x_{i0} + x_{j0} - x_{ij}.
\end{align}
%which equals $x_{i0} + x_{j0} - x_{ij}$.
A computation of the determinant then gives $4D^2=\Delta$.
Thus, $D^2\ge0$ implies $\Delta\ge 0$.

Also $\Delta=0$ if and only if $D=0$, which holds if and only if $\u\,
A = \orz$ for some vector $\u$.  By the
definition of coplanar, this holds if and only
if $V$ is coplanar.
\end{proof}

\begin{remark}\guid{SZIHGLO}
  The calculation of the general Cayley-Menger formula~\eqref{eqn:cmd}
  for $n+1$ points in $\ring{R}^n$ is based on the same method as the
  $3\times 3$ case; the identity~\eqref{eqn:dot-law} gives a rewrite
  rule for each matrix entry of $\det(A\,\hbox{}^t A)$ as a linear
  combination of the variables $x_{ij}$.  Row and column operations
  then put the matrix in a form in which each matrix entry is a single
  variable $x_{ij}$.
\end{remark}

\begin{remark}[]\guid{KZVHHBG}\label{rem:CM5}
  The volume of a $4$-simplex in $\ring{R}^3$ is zero.  This implies
  that Cayley-Menger determinant for $\v_0,\ldots,\v_4\in\ring{R}^3$
  is zero.  %
  \mar{\guid{NUHSVLM} LEG}%
  \mar{\guid{RPFVZDI} LEG}%
  \mar{\guid{GJWYYPS} LEG}%
  \mar{\guid{GDLRUZB} LEG}% 
  \mar{\guid{LTCTBAN} LEG}%
  This gives a polynomial relation between the $10 = \tbinom{5}{2}$
  squared edge lengths $x_{ij}$.  The relation $a x^2 + b x + c=0$ is
  quadratic in the tenth edge, say $x=x_{04}$, in terms of the other
  nine. The leading coefficient $a$ is nonzero if $\{\v_1,\v_2,\v_3\}$
  is not collinear.
\end{remark}

\section{Angle}\label{sec:angle}

Until now, the discussion of trigonometric functions has been purely
analytic.  This section interprets them geometrically.  It covers
fundamental identities in both Euclidean and spherical trigonometry,
including the law of cosines, the law of sines, the spherical law of
cosines, and a beautiful formula that Euler and Lagrange gave for the area of a spherical triangle.

If $\v,\w$ are nonzero vectors, then by the
Cauchy-Schwarz inequality,
\[ -1 \le \frac{\v\cdot \w}{\normo{\v}\,\normo{\w}}
  \le 1.\]  The middle term  lies in the
domain of the function $\arccos$. The value of this function is the angle in  the following
definition.  \indy{Index}{Cauchy-Schwarz inequality}%
\indy{Notation}{v7@$\v\in\ring{R}^3$}%
\indy{Notation}{wi@$\w\in\ring{R}^3$}%

\begin{definition}[angle,\ arclength]\guid{WZYUXVC}\label{def:angle}
\formaldef{$\arc_V$}{arcV}
Let $\u,\v,\w$ be vectors with $\u\ne \v,\w$.
Define 
\[ 
  \arc_V(\u,\{\v,\w\}) = \arccos\left(\frac{(\v-\u)\cdot 
(\w-\u)}{\norm{\v}{\u}\,\norm{\w}{\u}}\right).
\] 
The value of this function is the \newterm{angle} at $\u$ formed by
$\v$ and $\w$.  
\indy{Index}{angle}%
\indy{Index}{arc}%
\indy{Index}{arclength}%
\indy{Notation}{wi@$\w\in\ring{R}^3$}%
\indy{Notation}{arcv@$\arc_V$}%
\end{definition}

\figLIKEURF % fig:arcV

\begin{remark}\guid{OAXCOPU}
  According to the formalist,  a definition requires no
  justification.  As Hilbert
  famously said, ``One must be able to say at all times -- instead of
  points, straight lines, and planes -- tables, beer mugs, and
  chairs.''  The intuitive geometer asks for more: angles should be
  invariant under isometries of $\ring{R}^n$, and when $\u,\v,\w$ are
  mapped by an isometry into a fixed plane, this definition of angle
  in $\ring{R}^n$ should agree with the accepted definition in the
  plane.  This definition meets such a standard.  The norm
  (Definition~\ref{def:norm}) also extends the intuition of the plane,
  obtained as it is by successive applications of the Pythagorean
  theorem in two dimensions.
\[
\normo{\u}^2 = u_0^2 + v_0^2,\quad v_0^2 = u_1^2 + v_1^2,\ldots
\quad v_{n-2}^2 = u_{n-1}^2 + v_{n-1}^2,\qquad v_{n-1}=0.
\]
\end{remark}

By the relation between $\arccos$ and $\atn$
(Lemma~\ref{lemma:arccos-arctan}), %if $|\u\cdot \v|\ne
                                   %\normo{\u}\,\normo{\v}$,
%then 
\begin{equation}\label{eqn:angle}
  \arc_V(\orz,\{\v,\w\}) = \frac{\pi}2 - \atn\left ({\sqrt{(\normo{\v}^2\normo{\w}^2 -
        (\v\cdot \w)^2)}}, {\v\cdot \w}\right).
\end{equation}
\mar{\guid{ACNBFRL} Eq.~\ref{eqn:angle}}
\indy{Index}{arclength}%

The notation $\arc_V$ for angle comes from its interpretation as the
length of a geodesic arc on a unit sphere
centered at $\u$ from point $\v$ to $\w$.
\indy{Index}{arc!geodesic}%
The subscript $V$ is a reminder that
the function arguments are vectors.  The function
$\arc$, without the subscript,  gives the angle as a function
of the three edge lengths of a triangle.
\indy{Notation}{V@$V$ (subscript marking vector functions)}%
\indy{Notation}{arc@$\arc$}%

% \begin{definition}[arc length]\guid{WHDSIZT} 
%The arclength of a geodesic arc on a
%   unit sphere centered at $\v_0$ from point $\v_1$ to $\v_2$ is the
%   angle formed by $\v_1$ and $\v_2$ at $\v_0$.
%\end{definition}

\begin{definition}[arc]\guid{PQQDENV}\label{def:arc}
\formaldef{arc}{arclength}
Define
\[ \arc(a,b,c) = \arccos(\frac{a^2 + b^2 - c^2}{2 a
    b}).\] 
\indy{Index}{arc}%
\end{definition}

If the triangle inequalities hold:
\[ 
a + b \ge c,\quad b + c \ge a, \quad c+a \ge b
\] 
and if $a,b >0$, then
\[ 
  2 a b = (\mp a+b+c)(a \mp b \pm c) \pm (a^2 + b^2 - c^2) 
\ge \pm  (a^2 + b^2 - c^2)
\] 
and the argument of $\arccos$ in the definition of $\arc$ falls within
its domain.

\begin{lemma}[law of cosines]\guid{HQTBPCM}\label{lemma:loc}
Let $\u,\v,\w$ be vectors with $\v\ne \u$, $\w\ne \u$.  Let $a
= \norm{\w }{ \u}$, $b = \norm{\v }{ \u}$, and $c = \norm{\v }{ \w}$.
Let $\gamma=\arc_V(\u,\{\v,\w\})$.    Then
\[ c^2 = a^2 + b^2 - 2 a b \cos\gamma.\] 
Also,
\[ 
\arc_V(\u,\{\v,\w\})= \arc(a,b,c).
\] 
%if $\v$, $\w$, and $\u$ are not collinear then

\end{lemma}
\indy{Notation}{zzc@$\gamma$ (angle)}%
\indy{Index}{arc}%
\indy{Index}{law of cosines} %
\indy{Index}{trigonometry!law of cosines}%
\indy{Index}{cosine!law of cosines}%

\begin{proved}
By the definition of $\arc_V$, the
definition of $\arccos$, and~\eqref{eqn:dot-law},
\[ 
2 a b \cos \gamma = 2 (\w - \u)\cdot (\v - \u) = a^2 + b^2 - c^2.
\] 
This identity can be solved for $\gamma$ and
 gives the final statement of the lemma.  \swallowed\end{proved}


\begin{definition}[$\ups$]\guid{OBPIOXD}\label{def:ups}
\formaldef{$\ups$}{ups\_x}
Let $\ups$ (the symbol is a Greek upsilon, which is written with a
wider stroke than a roman vee) be the polynomial
\[ \ups(x,y,z) = -x^2 - y^2 - z^2 + 2 x y + 2 y z + 2
  z x.\] 
\indy{Notation}{zzu@$\ups$ (polynomial)}%
\end{definition}



%% WW Repeated def (tarski.tex)
This polynomial is nonnegative under conditions described by the
following lemma. 


\begin{lemma}[]\guid{QRAAWFS}\label{lemma:ups} Let
  $V=\{\v_0,\v_1,\v_2\}\subset\ring{R}^3$.  Let $x_{ij} =
  \norm{\v_i}{\v_j}^2$.  Then $\ups(x_{01},x_{12},x_{02})\ge 0$.
  Moreover, the set $V$ is collinear if and only if $\ups(x_{ij}) =
  0$.%
\mar{\guid{FHFMKIY} $=$ LEG}
\end{lemma}

\begin{proof}
The polynomial factors
\begin{equation}\label{eqn:ups}
\ups(a^2,b^2,c^2) = 16 s (s-a) (s-b)
  (s-c),
\end{equation}
\mar{\guid{IHIQXLM} Eq.~\ref{eqn:ups}}
where $s = (a+b+c)/2$.  If $a,b,c$ are the sides of a triangle, then
$a,b,c\ge0$ and the triangle inequality (Lemma~\ref{lemma:triangle-ineq})
holds for all orderings of sides: $(b+c-a)\ge 0$ and so forth.
Non-negativity $0\le \ups(a^2,b^2,c^2)$ follows from the triangle
inequality applied to each factor in the factorization of $\ups$:
$2(s-a) = (b+c-a) \ge0$ and so forth.  The case of equality in the lemma is the
case of equality in the triangle inequality.
\indy{Index}{Cauchy-Schwarz inequality}%
\indy{Index}{triangle inequality}%
\end{proof}

An alternative way to view nonnegativity is
that $\ups$, like $\Delta$, is the square of a Cayley-Menger determinant
\eqref{eqn:cmd}.  
\[
0\le (2D)^2 = -
\left|\begin{matrix} 0 & a^2 & b^2 & 1\\ a^2 & 0 & c^2 & 1\\  b^2 & c^2 & 0 & 1\\
1 & 1 & 1& 1
\end{matrix}\right| = \ups(a^2,b^2,c^2).
\]
Section~\ref{sec:cross}  further identifies
the determinant $D$ as the norm of a cross product.
Volume and area are the topics of the next chapter, but it
is appropriate at this point to consider a formula for the area of a
triangle.  By means of formula \eqref{eqn:ups} for $\ups$, Heron's classical formula
for the area of a triangle with sides $a,b,c$ can be put in the form
\[ \sqrt{\ups(a^2,b^2,c^2)}/4.\] 
\indy{Index}{Heron's formula}%


\begin{lemma}[law of sines]\guid{UKBAHKV}\label{lemma:los}
Assume that $a,b>0$ and $a+b\ge c$, $b+c\ge a$, and $c+a\ge b$.
Let $\gamma=\arc(a,b,c)$.  Then
\[ 2 a b \sin\gamma =
  \sqrt{\ups(a^2,b^2,c^2)}.\] 
\end{lemma}
\indy{Index}{trigonometry!law of sines}%
\indy{Index}{law of sines}%
\indy{Index}{sine!law of sines}%
\begin{proved}
  Both sides are nonnegative, so it is
  enough to check that their squares are equal.  By the definition of
  $\arc$, we have
\[ 
4 a^2 b^2 \sin^2\gamma 
= 4 a^2 b^2 (1-\cos^2\gamma) 
= (4 a^2 b^2 - (a^2 + b^2 -
c^2)^2) 
= \ups(a^2,b^2,c^2).\] 
% checked 4/4/2008
\swallowed\end{proved}

Another useful relation writes $\arc$ in terms of $\atn$.
\begin{equation}\label{eqn:arc-atn}
\arc(a,b,c) = 
\pi/2 - \atn({\sqrt{\ups(a^2,b^2,c^2)}},{ a^2 + b^2 - c^2}).
\end{equation}
\mar{\guid{GVWTZKY} Eq.~\ref{eqn:arc-atn}}
This follows directly from Lemma~\ref{lemma:arccos-arctan} and the
definitions of $\arc$ and
  $\ups$.



\subsection{cross product} \label{sec:cross}

This book makes infrequent use of the cross product.
A definition and the most basic properties  suffice.

\begin{definition}[cross product]\guid{FCUAGAJ}\label{def:cross}
\formaldef{cross product}{(cross)}   
Let $\v =(x,y,z)$ and $\w = (x',y',z')$.  
Let the cross product be defined
by
\[ 
\v \times \w = (y z' - y' z, z x' - x z', x y' - y x').
\] 
\indy{Index}{cross product}%
\indy{Index}{vector!cross product}%
\indy{Notation}{5@$\times$ (cross product)}%
\end{definition}

\begin{lemma}[]\guid{KVVWPNA}  
\label{lemma:los-cross}
Any two vectors $\v,\w\in \ring{R}^3$ satisfy
\[ \normo{\v \times \w} =
  \normo{\v}\,\normo{\w}\sin\gamma,\] 
where $\gamma=\arc_V(\orz,\{\v,\w\})$.
Also, $\v \cdot (\v\times \w) = \w\cdot (\v\times \w) = \orz$.
\end{lemma}

\begin{proved} This proof is an exercise in
  real arithmetic and basic trigonometry.
  Both the left and
  right sides are nonnegative, so it is
  enough to compare the squares of both sides.  The square of the
  left-hand side is
\begin{align}\label{eqn:cross-dot}
  \normo{\v \times \w}^2 &= (y z'- y'z)^2 + (z x' - x z')^2 + (x y' - y x')^2 \notag\\
  &= %\qquad\qquad=
  (x^2 + y^2 + z^2)(x'^2 + y'^2 + z'^2) - (x x' + y y' + z z')^2
 \notag \\&= %\qquad\qquad =\, 
  \normo{\v}^2\normo{\w}^2 - (\v\cdot \w)^2\\
  &= %\qquad\qquad=\, 
  \normo{\v}^2\normo{\w}^2 ( 1 - \cos^2\gamma)\notag\\
  &= %\qquad\qquad=\, 
\normo{\v}^2\normo{\w}^2 \sin^2\gamma.\notag
\end{align}
The second assertion of the lemma follows by arithmetic directly from
the definitions of the dot and cross products.  \swallowed\end{proved}

\begin{lemma}[]\guid{GZPIJUR}\label{lemma:cross-collinear}
For any $\v,\w\in\ring{R}^3$,
the set $\{\orz,\v,\w\}$ is collinear if and only if $\v\times\w=\orz$.
\end{lemma}

\begin{proof}
By Equation~\eqref{eqn:cross-dot},
\[
\v\times\w = \orz \quad\text{ if and only if }\quad \normo{\v}\normo{\w} = | (\v\cdot \w) |.
\]
This is the case of equality in the Cauchy-Schwartz inequality, which is given as
%
%
%By Lemma~\ref{lemma:los-cross}, $\v\times\w=\orz$ if and only if $\v=\orz$, $\w=\orz$,
%or $\arc_V(\orz,\{\v,\w\})$ is $0$ or $\pi$.  By the definition of $\arc_V$, it equals $0$ or $\pi$
%exactly in the case of equality of the Cauchy-Schwartz inequality:
\[
\normo{\v} \,\w = \pm \normo{\w}\,\v.
\]
This is equivalent to the collinearity of $\{\orz,\v,\w\}$.
\end{proof}

\begin{lemma}[]\guid{BKMUSOX}\label{lemma:cross-id}
\[ 
\u\times \v = -\v\times \u,\quad
(\u\times \v)\cdot \w = (\v\times \w)\cdot \u,\quad
(\u\times  \v)\times \w = (\u\cdot \w)\, \v - (\v\cdot\w) \,\u.
\] 
\end{lemma}

\begin{proved}
These are arithmetic consequences of the definition of cross product.
\swallowed\end{proved}



\subsection{dihedral angle}

A dihedral angle of a tetrahedron is the angle formed between two of
its faces. In general, the dihedral angle refers to the angle formed
by two half-planes delimited by a common line.  The dihedral angle is
determined by a pair $\{\v_0,\v_1\}$ of points on the delimiting line
and another pair $\{\v_2,\v_3\}$ of two points on the respective
half-planes.  \indy{Index}{angle!dihedral}%
\indy{Index}{tetrahedron}%
\indy{Notation}{dihv@$\dih_V$}%
\indy{Index}{vector!projection}%
\indy{Index}{orthogonality} %

\begin{definition}[dihedral angle]\guid{YMHELNF}\label{def:dih}
\formaldef{$\dih_V$}{dihV}
 When $\v_0\ne \v_1$,
  write $\dih_V(\{\v_0,\v_1\},\{\v_2,\v_3\})$ for the angle
  $\gamma\in[0,\pi]$ formed by
\[ 
\bar \w_2 = (\w_1\cdot \w_1) \w_2 - (\w_1\cdot \w_2) \w_1\textand  \bar \w_3 =
(\w_1\cdot \w_1) \w_3 - (\w_1\cdot \w_3) \w_1,
\] 
where $\w_i=\v_i-\v_0$.  We call it
the dihedral angle formed by $\v_2$ and $\v_3$ along $\{\v_0,\v_1\}$.
\indy{Notation}{dih}%
\indy{Index}{angle!dihedral}%
\end{definition}
The subscript $V$ is a reminder 
that the dihedral angle takes vector arguments.
Later, a second version, without the subscript, 
computes the angle as a function of the lengths of edges of a 
tetrahedron.
\indy{Index}{edge!length}%
\indy{Notation}{V@$V$ (subscript marking vector functions)}%
As the notation suggests, the dihedral angle depends only
on the unordered pairs $\{\v_0,\v_1\}$, $\{\v_2,\v_3\}$.

\figGJRSLPT % fig:dih


The dihedral angle can be interpreted as the planar angle between two rays, obtained by
projection of the two half-planes to a plane orthogonal to both of
them.  Up to positive scalars, $\bar \w_2$ and $\bar \w_3$ are the
projections of $\w_2$ and $\w_3$ to the plane through the origin
orthogonal to the vector $\w_1$.  The dihedral angle is the angle
between the projections $\bar \w_2$ and $\bar \w_3$ at $\orz$.

\begin{remark}\label{rem:dih}\guid{LHYCNII}
  The dihedral angle is unchanged if $\w_1$ is replaced with $t \w_1$ with
  $t\ne0$. The dihedral angle is unchanged if $\w_2$ is replaced with
  $t_2 \w_2 + t_1 \w_1$ with $0 < t_2$ and $t_1$ arbitrary because
  such points project along the same ray.  It is unchanged if $\w_3$ is
  replaced with $t_3 \w_3 + t_1 \w_1$ with $0 < t_3$ and $t_1$
  arbitrary, because such points project along the same ray.  In
  particular, the dihedral angle formed by $\w_2$ and $\w_3$ along
  $\{\orz,\w_1\}$ is the same as that formed by $\w_2/\normo{\w_2}$ and
  $\w_3/\normo{\w_3}$ along $\w_1/\normo{\w_1}$.
\end{remark}

The dihedral angle is degenerate and is not be used when $\w_1 =
\orz$, $\bar \w_2 = \orz$, or $\bar \w_3 = \orz$.  Equivalently, degeneracy
occurs when $\{\v_0,\v_1,\v_2\}$ or $\{\v_0,\v_1,\v_3\}$ is a collinear set.

\begin{lemma}\guid{HVIHVEC}\label{lemma:dih-cross}
  Let $\v_0,\ldots,\v_3\in\ring{R}^3$ be given with $\v_0\ne \v_1$.
  Then $\dih_V(\{\v_0,\v_1\},\{\v_2,\v_3\})$ is the angle formed by
  $\w_1\times \w_2$ and $\w_1\times \w_3$, where $\w_i= \v_i-\v_0$.
\end{lemma}

\begin{proof}  For any  $\u,\v,~\w\in\ring{R}^3$ with $\u\cdot\w=\v\cdot\w=0$, 
we use Lemma~\ref{lemma:cross-id}
to compute
\begin{align*}
  (\u\times \w)\cdot (\v\times \w) &= - (\u\times \w)\cdot (\w\times \v) \\
  &= -\v\cdot ((\u\times \w)\times \w) \\
  &= -\v\cdot (-(\w\cdot \w) \u)\\
  &= \,\,(\u\cdot \v) (\w\cdot \w).
\end{align*}
That is, $\wild\times \w$ preserves dot products, up to a scalar
$(\w\cdot\w)$.  Thus, if $\w\ne\orz$, the angle formed by $\u$ and
$\v$ is equal to the angle formed by $\u\times \w$ and $\v \times \w$.

The dihedral angle is the angle formed by
\begin{align*}
  \bar \w_2 & = (\w_1\cdot \w_1) \w_2 - (\w_1\cdot \w_2) \w_1 = (\w_1\times \w_2)\times \w_1\\
  \bar \w_3 & = (\w_1\cdot \w_1) \w_3 - (\w_1\cdot \w_3) \w_1 = (\w_1\times \w_3)\times \w_1
\end{align*}
Let $\u=\w_1\times \w_2$, $\v=\w_1\times \w_3$, and $\w=\w_1$.  The
preceding calculation shows that the angle formed by $\bar \w_2 =
\u\times\w$ and $\bar \w_3 = \v\times \w$ is equal to the angle formed
by $\u$ and $\v$.  The lemma ensues.
\end{proof}

\begin{lemma}[spherical law of cosines]\guid{RLXWSTK}\label{lemma:sloc}
   \formalauthor{Nguyen Quang Truong} Let $\gamma$ be the
  dihedral angle formed by $\v_2$ and $\v_3$ along $\{\v_0,\v_1\}$.  Let
  $a$, $b$, and $c$ be the angle at $\v_0$ between $\v_3$ and $\v_1$, $\v_2$
  and $\v_1$, and $\v_2$ and $\v_3$, respectively. %Assume $\v_1\ne \v_0$.
  Assume that $\{\v_0,\v_1,\v_2\}$ and $\{\v_0,\v_1,\v_3\}$ are not collinear.
  Then
  \[ \cos\gamma = \frac{\cos c - \cos a \cos b}{\sin
      a\sin b}.\] 
\end{lemma}
\indy{Index}{cosine!spherical law of cosines}%
\indy{Index}{spherical!law of cosines}%
\indy{Index}{trigonometry!spherical}%

\begin{remark}
  The spherical law of cosines is the most fundamental identity of
  spherical trigonometry.  A \fullterm{spherical
    triangle}{spherical!triangle} is a figure formed by three points
  on a unit sphere, together with three minimal geodesic arcs on the
  sphere that connect each pair of points.  In the lemma, $a$, $b$,
  and $c$ are the arclengths of the sides of a spherical triangle with
  vertices $\v_2/\normo{\v_2}$, $\v_3/\normo{\v_3}$, and
  $\v_1/\normo{\v_1}$, when $\v_0=\orz$. See Figure~\ref{fig:sloc}.
  Also, $\gamma$ measures the angle of the spherical triangle opposite
  the side $c$.  
  \indy{Index}{triangle!spherical}%
\end{remark}

\figNUPFYMD % spherical law of cosines.


\begin{proof} The proof is an exercise based on previously established
  trigonometric identities.  Let $\w_i = \v_i-\v_0$.  An earlier
  remark states that the dihedral angle is
  unchanged if $\w_2$, $\w_3$, and $\w_1$ are replaced by
  $\w_2/\normo{\w_2}$, $\w_3/\normo{\w_3}$, $\w_1/\normo{\w_1}$,
  respectively.  Hence, we may assume without loss of generality that
  $\normo{\w_2}=\normo{\w_3}=\normo{\w_1}=1$.

Let $\bar \w_2$ and $\bar \w_3$ be the vectors in Definition~\ref{def:dih}.
The law of cosines gives
\[ \cos\gamma = \frac{\bar \w_2\cdot \bar \w_3}
{\normo{\bar \w_2}\,\normo{\bar \w_3}}.
\] 
The unit normalizations of $\w_3,\w_2,\w_1$ give
\[ 
\normo{\bar \w_2}^2 = \bar \w_2\cdot \bar \w_2 =
(\w_2 - (\w_1\cdot \w_2)\w_1)\cdot (\w_2 - (\w_1\cdot \w_2) \w_1) =
1 - (\w_1\cdot \w_2)^2 = \sin^2 b.
\] 
So $\normo{\bar \w_2} =\sin b$. Similarly, $\normo{\bar \w_3} = \sin a$.
These calculations give the denominator in the spherical law of cosines.  An
expansion of the dot product gives the numerator:
\begin{align*}
\bar \w_2\cdot \bar \w_3 &=
 (\w_2 - (\w_1\cdot \w_2) \w_1)\cdot (\w_3 - (\w_1\cdot \w_3) \w_1)\\
&= (\w_2\cdot \w_3) - (\w_1\cdot \w_2) (\w_1\cdot \w_3) \\
&= \cos c - \cos a \cos b.
\end{align*}
The identity ensues.
\end{proof}

The spherical law of cosines gives the angles of a spherical triangle
as a function of its sides.  In spherical geometry, a
duality%
\footnote{In three-dimensional Euclidean space, the orthogonal
  complement of a plane through the origin is a line through the
  origin, giving a duality between planes and lines through the
  origin.  The intersection of each plane and line with a unit sphere
  at the origin yields a duality between great circles and antipodal
  pairs of points (the poles of the great circle).  The three edges of
  a spherical triangle $ABC$ lie on three great circles that
  determine three antipodal pairs of points.  From each of the three
  pairs, a coherent choice can be made between the two poles (with the preferred pole closer to the opposite vertex of $ABC$).  These
  three poles are the vertices of the polar triangle $A'B'C'$.  Each
  statement about the triangle $ABC$ can be dualized to a statement
  about $A'B'C'$.  In particular, the edges $a,b,c$ and angles
  $\alpha,\beta,\gamma$ of $ABC$ are related to those $a',b',\ldots$
  of $A'B'C'$ by
\[ 
a + \alpha' = \pi,\quad a' + \alpha= \pi,
\] 
and so forth.}%
\indy{Notation}{A@$ABC$ (triangle)}%
\indy{Index}{polar!triangle}%
exists between angles and sides of a triangle.  As a result, formulas
in spherical trigonometry tend to come in pairs.  The spherical law of
cosines gives the angle of a spherical triangle as a function of its
edge lengths.  The polar form of the formula gives the edge length of
a spherical triangle as a function of its angles.  Up to signs, the
polar formula has the same form as the law of cosines.
\indy{Index}{great circle}%

\begin{lemma}[spherical law of cosines - polar form]\guid{NLVWBBW}
\label{lemma:sloc2} \formalauthor{Nguyen Quang Truong} Consider 
  $\{\v_0,\v_1,\v_2,\v_3,\}\subset\ring{R}^3$.  Let
  $\alpha,\beta,\gamma$ be the dihedral angles:
\begin{align*}
\alpha &= \dih_V(\{\v_0,\v_2\},\{\v_3,\v_1\})\\
\beta &= \dih_V(\{\v_0,\v_3\},\{\v_2,\v_1\})\\
\gamma&= \dih_V(\{\v_0,\v_1\},\{\v_3,\v_2\})\\
\end{align*}
Let $c$ be the
angle between $\v_2$ and $\v_3$ at $\v_0$. 
Assume that $\{\v_0,\v_2,\v_1\}$, $\{\v_0,\v_2,\v_3\}$, and $\{\v_0,\v_3,\v_1\}$ are not collinear.
Then
\[ 
\cos c = \frac{\cos \gamma + \cos \alpha \cos \beta}
{\sin \alpha\sin \beta}.
\] 
\end{lemma}
\indy{Index}{cosine!spherical law of cosines}%
\indy{Index}{spherical!law of cosines}%

\begin{proof}  
  What follows is a direct computational proof that avoids polarity and
is an
  application of established trigonometric identities.  Let $a$ be the
  angle between $\v_1$ and $\v_3$, and let $b$ be the angle between $\v_1$
  and $\v_2$ at $\v_0$.  Let $A=\cos a$, $B=\cos b$, $C=\cos c$,
  $A'=\sin a$, $B'=\sin b$, $C'=\sin c$.  The spherical
  law of cosines gives
\[ \sin^2\beta = 1-\left(\frac{B-A C}{A' C'}\right)^2
  = \frac{p}{A'^2 C'^2},\] 
where $p=1-A^2 - B^2 - C^2 + 2 A B C$.
In particular, $p\ge 0$.
\indy{Notation}{p@$p$ (trigonometric expression)}%
A computation of $\sin^2\alpha$ and the remaining terms in the same way gives
\begin{align*}
  \sin\alpha\sin\beta &= \frac{\displaystyle p}{\displaystyle A' B' C'^2}\\ 
  \\
  \cos\gamma + \cos\alpha \cos\beta &=
  \frac{\displaystyle C - A B}{\displaystyle A' B'} + 
\frac{\displaystyle A - B C}{\displaystyle B' C'} 
\frac{\displaystyle B - A C}{\displaystyle A' C'}
  = \frac{\displaystyle p C}{\displaystyle A' B' C'^2}.
% C &= \frac{\cos\gamma + \cos\alpha \cos\beta}{\sin\alpha \sin\beta}\\
\end{align*}
The result follows by real arithmetic.
\end{proof}

The following lemma gives a formula for the dihedral angle
of a tetrahedron along an edge in terms of its edge lengths.  The
familiar polynomials $\ups$ and $\Delta$ appear once again.
\indy{Notation}{zzu@$\ups$ (polynomial)}%
\indy{Notation}{zzD@$\Delta$}%


\begin{lemma}[]\guid{OJEKOJF}\label{lemma:dihform}
\formalauthor{Nguyen Quang Truong}
Let $\v_0,\v_1,\v_2,\v_3$ 
be vectors with $\{\v_0,\v_1,\v_2\}$ not collinear, 
and $\{\v_0,\v_1,\v_3\}$ not
collinear. 
Let $\gamma$ be the dihedral angle formed
by $\v_2$ and $\v_3$ along $\{\v_0,\v_1\}$. Let
\[ (x_1,\ldots,x_6) = 
(x_{01},x_{02},x_{03},x_{23},x_{13},x_{12}),
\text{ where } x_{ij}=\norm{\v_i}{\v_j}^2.\] 
Let $\Delta_4$ be the partial derivative of $\Delta(x_1,\ldots,x_6)$ with
respect to $x_4$.
The dihedral angle $\gamma=\dih_V(\{\v_0,\v_1\},\{\v_2,\v_3\})$
is given by
\[ 
\gamma=\arccos(\frac{\Delta_4(x_1,\ldots,x_6)}{\sqrt{
\ups(x_1,x_2,x_6)\ups(x_1,x_3,x_5)}}).
\] 
%Assuming that $\gamma\ne 0,\pi$, 
It is also given by
\[ 
\gamma=\frac{\pi}{2} - \atn
({\sqrt{4 x_1 \Delta(x_1,\ldots,x_6)}},{\Delta_4(x_1,\ldots,x_6)}).
\] 
\end{lemma}
%% pi/ 2. -  arctan(  deltax4/ (sqrt (4. * x1 * delta)))
\indy{Index}{angle!dihedral}%

\indy{Notation}{zzb@$\beta$ (angle)}%
\begin{proof}  We use the notation $\w_i, \bar \w_i$ established in Definition~\ref{def:dih}.
  Let $\beta = \arc_V(\v_0,\{\v_1,\v_2\})$.  The assumptions give
  $\bar \w_2\ne \orz$ and $\bar \w_3 \ne \orz$.  
  By expanding definitions and dot products and by the
  law of sines,
\[ 
  \bar \w_2\cdot \bar \w_2 = (\w_1\cdot \w_1) ((\w_1\cdot \w_1)(\w_2\cdot \w_2) -
  (\w_1\cdot \w_2)^2) =  x_1^2 x_2 \sin^2 \beta = \frac{1}{4}
  x_1
  \ups(x_1,x_2,x_6).
\] 
Similarly,
\[ \bar \w_3 \cdot \bar \w_3 = \frac{1}{4} x_1
  \ups(x_1,x_3,x_5)\] 
%Let $y_i = \sqrt{x_i}$. 
and by dot product formula \eqref{eqn:dot-law},
\begin{align*}
    \bar \w_2\cdot \bar \w_3 &= (\w_1\cdot \w_1)((\w_1\cdot \w_1)(\w_2\cdot \w_3) -
    (\w_1\cdot \w_2)(\w_1\cdot \w_3) ) \vspace{6pt} \\  
    &= x_1 \left(\frac{\displaystyle x_1 (x_2 + x_3 -
        x_1)}{2} - \frac{\displaystyle (x_1 + x_2 - x_6)(x_1 + x_3 -
        x_5)}{4} \right)\vspace{6pt}\\
%= \frac{x_1}{4} (2 x_1 (x_2+x_3-x_4) -
%(x_1+x_2-x_6)(x_1+x_3-x_5)) \vspace{6pt}\\
&= {x_1\Delta_4(x_1,\ldots,x_6)}/{4}.
\end{align*}
The result follows in terms of $\arccos$.

The translation to $\atn$ uses the $\arccos$-$\atn$
identity (Lemma~\ref{lemma:arccos-arctan}) and the following polynomial
identity
\[ 
  % \frac{16}{x_1^2}(\normo{\bar \w_2}^2 \normo{\bar \w_3}^2 - (\bar
  % \w_2\cdot \bar \w_3)^2) =
\ups(x_1,x_2,x_6)\ups(x_1,x_3,x_5) - \Delta_4(x_1,\ldots,x_6)^2
= 4 x_1 \Delta(x_1,\ldots,x_6).
\] 
\end{proof}

\subsection{Euler triangle}

The expression $\alpha_1+\alpha_2+\alpha_3-\pi$ is \newterm{Girard's
  formula} (known first to Harriot) for the area of a spherical
triangle with angles $\alpha_1$, $\alpha_2$, $\alpha_3$.  We return to
this formula in the next chapter~\eqref{eqn:girard}, when area and
volume are treated.  Although the statement and proof do
not explicitly mention area, the following lemma can be interpreted
as an alternative formula discovered by Euler and Lagrange for the
area of a spherical triangle.
% \indy{Index}{Girard, A.}%
% Albert Girard's Book on trigonometry was published in 1626. Harriot
% lived 1560 - 1621.
\indy{Index}{Girard's formula}%
\indy{Index}{triangle!spherical}%
\indy{Notation}{zza@$\alpha$ (angle)}%
\indy{Index}{Harriot, T.}%

\begin{lemma}[Euler triangle]\guid{JLPSDHF}\label{lemma:euler} %was 600
Let $\v_0,\v_1,\v_2,\v_3$ be points in $\ring{R}^3$. 
Let 
\[ (y_1,\ldots,y_6)
  =(y_{01},y_{02},y_{03},y_{23},y_{13},y_{12}), \text{ where }
  y_{ij}=\norm{\v_i}{\v_j}.\]  Set $x_i = y_i^2$.  and
\[ 
p = y_1 y_2 y_3 + y_1 (\w_2\cdot \w_3) + y_2 (\w_1\cdot \w_3) + y_3
(\w_1\cdot \w_2).
\] 
\indy{Notation}{p@$p$ (Euler solid angle numerator)}%
where $\w_i = \v_i- \v_0$.  Let \[ \alpha_i
  =\dih_V(\{\v_0,\v_i\},\{\v_j,\v_k\})\] 
where $\{i,j,k\}=\{1,2,3\}$.
Assume that $\Delta(x_1,\ldots,x_6)>0$. 
Then
\[ 
\alpha_1+\alpha_2+\alpha_3 - \pi
= {\pi} - 2\,\atn({\Delta(x_1,\ldots,x_6)^{1/2}},{2 p}).
\] 
\end{lemma}
\indy{Index}{triangle!Euler}%


Before we jump into the details of the proof, it helps to understand
why a formula of this general form should exist.  
Each angle $\alpha_i$ equals a single arctangent (Lemma~\ref{lemma:dihform}).
The addition law for arctangent, which is obtained by inverting the additional law for
the tangent (Lemma~\ref{lemma:tan-add}),
rewrites the sum $\alpha_1+\alpha_2+\alpha_3$
of arctangents as a single arctangent, or as twice a single arctangent if
the double angle formula is invoked.  Euler's formula is a precise formula for the
sum of arctangents in the form $2\atn(\cdots)$.

In practice, it is easier to carry out the details of the proof by a
slightly different strategy.  We can check that the derivatives of the
two sides of the identity are equal as rational functions.  The domain
is connected, and from this it follows that the two sides differ by at
most a constant.  By calculating a particular test value, we see that
the two sides are precisely equal.



\begin{proof}
%% I checked all the details of this proof in 
%% Math'ca on May12,2007
  This proof is an exercise in real analysis
  and established trigonometric identities.  According to an earlier
  remark, the dihedral angles are unchanged if the
  vectors $\w_i$ are rescaled so that $\normo{\w_i}=1$.  By
  inspection, the given formula is also unchanged under rescalings:
  the factor $a$ is homogeneous of degree $3$ under a change $\w_i
  \mapsto t \w_i$ for $t>0$, and so is $\sqrt{\Delta}$ by the formula
  for $\Delta$.  Thus, without loss of generality, $\normo{\w_i}=1$ for $i=1,2,3$.  Consequently, $y_1=y_2=y_3=1$.  It is convenient to
  use different notation $a=x_4$, $b=x_5$, $c=x_6$ for the other
  variables. The expansion of the dot products in $p$ by the dot
  product law gives
\[ 2 p = 8 - (a+b+c).\] 
Also, the definitions of $\Delta$ and $\ups$ give
\[ \Delta(x_1,\ldots,x_6) = \Delta(1,1,1,a,b,c) =
\ups(a,b,c) - a b c.\] 
Since $\Delta>0$ by assumption, the arctangent formula
in Lemma~\ref{lemma:dihform} 
applies for the dihedral angles $\alpha_i$.  After
this substitution (and clearing a factor of $3$),  %and clearing the
                                                   %$3$ from the
                                                   %denominator,
the desired identity takes the form $f(a,b,c)=0$, where
\[ 
f(a,b,c)= -\pi/2 - \sum_{i=1}^3\arctan(u_i/\sqrt{\Delta}) +
2\arctan(2 p/\sqrt{\Delta}),
\] 
for some rational functions $u_i$ of $a,b,c$.  The aim is to prove
this trig identity holds whenever $\Delta>0$.

To see that the function $f$ does not depend on $a$, 
we fix $(b,c)$ and differentiate $f$ with respect to $a$.  The partial
derivative $\partial f/\partial a$ has the form
$g(a,b,c)/\sqrt{\Delta}$ for some rational function $g$ of $a,b,c$.
The denominator of $g$ has no real zero.  Algebraic simplification of
this rational function shows that the polynomial numerator of
$g(a,b,c)$ is identically $0$.  (Euler himself did not shun brute
force~\cite{Euler}.)

By real analysis, the derivative of $f$ is zero, and the function $f$
is constant along any segment in $\ring{R}^3$ along which $\Delta$ is
positive.  The remaining part of the proof constructs two segments
along which $\Delta$ is positive.
%\footnote{In the formal proof, Vu
%  Khac Ky uses three segments: the first segment runs from $(a,b,c)$
%  to $(b+c-b c/2,b,c)$, the second continues to $(2c - c^2/2,c,c)$,
%  and the final segment terminates at $(2,2,2)$.}   
The
first connects $f(a,b,c)$ to $f(a,2,2)$, provided the variables are
ordered appropriately.  The second connects $(a,2,2)$ to $(2,2,2)$.
From this construction it follows that $f(a,b,c)=f(2,2,2)$.  The last
step is to evaluate the constant $f(2,2,2)$.  Arithmetic gives
$\Delta=4$, $2p= 2$, $u_1=u_2=u_3 =0$, when $a=b=c=2$.  Finally,
\[ f(a,b,c)= f(2,2,2) = -\pi/2 + 2\arctan(1)
  =0.\] 


Let us return to the construction of the two segments.  By the
triangle inequality, $a =\norm{\v_2}{\v_3}^2 \le
(\norm{\v_2}{\v_0}+\norm{\v_3}{\v_0})^2 = 4$.  If equality holds, then
$\{\v_0,\v_2,\v_3\}$ is collinear and $\{\v_0,\ldots,\v_4\}$ is
coplanar.  From this it follows that $\Delta=0$, which is contrary to
assumption.  Similarly, $a=0$ implies that $\Delta=0$.  Hence $0<a<4$.
Similarly, $0<b<4$ and $0<c<4$.  By the  {\it pigeonhole}
principle, two of the real numbers $a,b,c$ must lie in the same
subinterval $[0,2]$ or $[2,4]$.  To fix notation, assume that $b$ and
$c$ lie in the same subinterval.

\claim{The polynomial $\Delta$ is positive\footnote{This paragraph follows the book's
    general convention of typesetting claims in italic.} 
  on the linear segment from $(a,b,c)$ to
  $(a,2,2)$.}  Indeed, for $0\le t \le 1$, Tarski
arithmetic gives
\begin{align*}
\Delta(1,1,1,a, &\,b(1-t)+2t,c(1-t)+2t)  \\
&= \Delta(1,1,1,a,b,c) + 
t (2-t) (a (b-2)(c-2) + (b-c)^2)\\
&\ge \Delta(1,1,1,a,b,c)\\
&> 0.
\end{align*}

\claim{The polynomial $\Delta$ is positive on the linear segment from $(a,2,2)$ to
  $(2,2,2)$.}  Indeed,
\[ \Delta(1,1,1,a,2,2) = a(4-a)>0.\]   
The rest of the proof has been sketched above.
\end{proof}
\indy{Notation}{zzD@$\Delta$}%






%\subsection{Lexell's theorem}
%
%\begin{lemma}[Lexell]\guid{UWIPRDV}
%% was 1000 with old proof including lemma ZHH
% Fix two points $\v_1,\v_2$ on a unit sphere that are not antipodal.
% Let $\u,\u'$ be two other points the sphere in the same open
% hemisphere determined by the great circle through $\v_1,\v_2$.  Then
% the two spherical triangles $\{\v_1,\v_2,\u\}$ and
% $\{\v_1,\v_2,\u'\}$ have the same area if and only if the four
% points $\u$, $\u'$, $\v^*_1$, $\v^*_2$ are concircular, where
% $\v^*_i$ is the point antipodal to $\v_i$.
%\end{lemma}
%\indy{Index}{Lexell's Theorem}%
%
%
%
%\begin{proof} By the polarity of triangles mentioned above, it is
%  enough to prove the polar statement.  By Girard's formula, fixing
%  the area fixes the sum of the angles.  The polar triangle has fixed
%  perimeter.  By polarity, Lexell's theorem is a consequence of the
%  following lemma.
%\end{proof}
%\indy{Index}{Girard's formula}%
%
%\begin{lemma}[]\guid{ZHHSGTF} Fix one point $\v$ on the unit
%  sphere, with antipodal point $\v^*$.  Consider two great
%  half-circles $D_i$, $i=1,2$ between $\v$ and $\v^*$ that are not
%  coplanar.  Two great circles $A$ and $B$ cut equi-perimeter
%  triangles with vertex $\v$ along $D_i$ if and only if the great
%  circles $D_i$, $A$, and $B$ are tangent to a common circle $C$.
%\end{lemma}
%\indy{Index}{great circle}%
%
%\begin{proof} The two tangents to a circle through a given point have
%  the same length.  If $C$ exists, then this fact implies that a
%  great circle $A$ that is tangent to $C$ cuts a triangle with vertex
%  $\v$ along $D_i$ with a perimeter that is equal to the sum of the
%  distances from $\v$ to the two points of tangency $C\cap D_i$.
%  This is independent of $A$. 
%
%Conversely, for $A$ any great circle there is a unique $C$ that
% inscribes the great circles $D_i$, and $A$.  The perimeter of the
% triangle is the sum of the distances from $\v$ to the points $C\cap
% D_i$.  If a second $A$ gives a triangle with the same perimeter, its
% circle $C'$ must satisfy $C'\cap D_i = C\cap D_i$.  This forces
% $C=C'$.
%\end{proof}
%

\section{Coordinates}

This section establishes the existence and basic properties of the
standard coordinate systems: polar coordinates, spherical coordinates,
and cylindrical coordinates.  
%
\indy{Index}{azimuth}%
\indy{Index}{angle!azimuth}%
\indy{Index}{azimuth cycle}%

\subsection{azimuth angle}

\label{sec:polar}
\indy{Index}{polar!coordinates}%
\indy{Index}{coordinate systems!polar coordinates}%


For every pair of real numbers $x$ and $y$,  there are real numbers
$r$ and $\theta$ such that
\begin{equation}\label{eqn:polar}
x = r\cos\theta,\quad y = r\sin\theta.
\end{equation}
\mar{\guid{FEVNANL} Eq.~\ref{eqn:polar}} If $x$ and $y$ are both zero,
then take $r=0$, and~\eqref{eqn:polar} holds for all choices of
$\theta$. If $x$ and $y$ are not both zero, then take $0<r$, and
$\theta$ is uniquely determined (up to multiples of $2\pi$).  By
convention, we take $0\le\theta < 2\pi$.  
\indy{Notation}{r@$r$ (polar, cylindrical, and spherical radius)}%
%\indy{Notation}{zzh@$\theta$ (polar, cylindrical, and spherical coordinate)}%
\indy{Notation}{zzh@$\theta$ (polar, cylindrical, and spherical angle)}%




%%  e3 is defined in terms of v1. The indexing confuses.  % fixed 3/21/2010


\begin{definition}[frame,~positive,~adapted]\guid{AXBTGQX}
\formaldef{frame}{orthonormal}
A tuple $(\e_1,\e_2,\e_3)$ of vectors in $\ring{R}^3$ is a 
\newterm{frame} if $\e_i\cdot \e_j$ and $\normo{\e_i}=1$ 
for all $i$ and $j$.
A tuple $(\e_1,\e_2,\e_3)$ is positive if $(\e_1\times \e_2)\cdot\e_3=1$.
% Let $\{\v_0,\v_1,\v_2\}\subset\ring{R}^3$ be a set that is not
% collinear.
A tuple $(\e_1,\e_2,\e_3)$ is \newterm{adapted} to $(\v_0,\v_1,\v_2)$ if
$\e_1 = (\v_1-\v_0)/\norm{\v_0}{\v_1}$ and
$\e_2\in\op{aff}_+^0(\{\v_0,\v_1\},\v_2)$.
% such that $\normo{\e_1}=1$ and $\e_1\cdot\e_3=0$; $\e_2 = \e_3\times
% \e_1$.  The tuple $E=(\e_1,\e_2,\e_3)$ is \newterm{adapted} to
% $(\v_0,\v_1,\v_2)$.
\end{definition}
\indy{Index}{frame}%
\indy{Index}{adapted}%
\indy{Notation}{E4@$E$ (frame)}%

\begin{lemma}[orthonormalization]\guid{QAUQIEC}
\label{lemma:frame}
  Assume that $\{\v_0,\v_1,\v_2\}\subset\ring{R}^3$ is not collinear.
  Then the unique positive frame adapted to 
  $\{\v_0,\v_1,\v_2\}$ is $(\e_1,\e_2,\e_3)$, where
\begin{align*}
\e_1 &= \w_1/\normo{\w_1},\\
\e_2 &= \bar{\w_2}/\normo{\bar{\w}_2},\quad \bar{\w}_2 = \w_2 - (\e_1\cdot \w_2) \e_1,\\
\e_3 &= \e_1 \times \e_2,
\end{align*}
and where $\w_i = \v_i - \v_0$.
\end{lemma}

\begin{proof} It follows by basic vector arithmetic that
  $(\e_1,\e_2,\e_3)$ is a positive frame adapted to
  $\{\v_0,\v_1,\v_2\}$.  The choices of vectors $\e_1$ and $\e_2$ are
  dictated by the definition of adapted frame.  The choice of $\e_3$
  is dictated by the definition of positive frame.
\end{proof}

\begin{lemma}[cylindrical coordinates]\guid{EYFCXPP}
\formalauthor{Nguyen Quang Truong}
Let $\v_0$ and $\v_1$ be distinct points in 
$\ring{R}^3$.  Let $(\e_1,\e_2,\e_3)$ be a positive frame 
where $\e_1 = (\v_1-\v_0)/\norm{\v_1}{\v_0}$.
Then every
$\p\in\ring{R}^3$ that is not in the line $\op{aff}(\v_0,\v_1)$
can be uniquely expressed in the form
\[ 
\p = \v_0 + r\cos\psi\, \e_2 + r\sin\psi\, \e_3 + h (\v_1-\v_0),
\] 
\indy{Notation}{h@$h$ (cylindrical coordinate)}%
\indy{Notation}{ez@$\e_i$ (orthonormal vectors)}%
for some $0< r$, $0\le \psi < 2\pi$, $h\in\ring{R}$.
Furthermore,
assume that $\p_1$ and $\p_2$ do
not lie in the line $\op{aff}(\v_0,\v_1)$.
Then there exist unique $\psi,\theta,r_1,r_2,h_1,h_2$
such
that $0\le\psi<2\pi$, $0\le\theta < 2\pi$, $0 < r_1$, $0 < r_2$, and
\begin{align*}
\p_1 &= \v_0 + r_1\cos\psi\, \e_2 + r_1\sin\psi\, \e_3 + h_1(\v_1-\v_0),\\
\p_2 &= \v_0 + r_2\cos(\psi+\theta)\, \e_2 + r_2\sin(\psi+\theta)\, \e_3 
+ h_2(\v_1-\v_0).
\end{align*}
Finally, the angle $\theta$ is independent of the choice of $\e_2,\e_3$
giving the positive frame.
\end{lemma}
\indy{Index}{coordinate systems!cylindrical coordinates}%
\indy{Index}{cylindrical coordinates}%
\indy{Notation}{zzv@$\psi$}%
%\indy{Notation}{zzh@$\theta$ (coordinate)}%
\indy{Notation}{zzh@$\theta$ (polar, cylindrical, and spherical angle)}%
%\indy{Notation}{r@$r$ (coordinate)}%
\indy{Notation}{r@$r$ (polar, cylindrical, and spherical radius)}%
\indy{Notation}{h@$h$ (cylindrical coordinate)}%
%
The degenerate point $\p\in\op{aff}\{\v_0,\v_1\}$ is excluded from the
lemma.  Nevertheless, it too has a cylindrical coordinate
representation of the form $\p = \v_0 + h(\v_1-\v_0)$ (with $r=0$).
Only uniqueness fails, because every $\theta$ gives the same
representation.

\begin{remark}
The reader should carefully note the indexing of the vectors in the
orthonormal frame as it appears in the cylindrical coordinate system.
This book breaks with tradition by making $h$ the coefficient of the
frame vector $\e_1$ (rather than $\e_3$) and makes a corresponding
change in spherical coordinates.  This nontraditional order is better
suited to the definition of dihedral angle, the arguments of which 
 are grouped in pairs $\dih_V(\{\v_0,\v_1\},\{\v_2,\v_3\})$
to emphasize the symmetries
$\v_0\leftrightarrow\v_1$ and $\v_2\leftrightarrow\v_3$.  Under this pairing
of arguments, the axis of the dihedral angle is the line
$\op{aff}\{\v_0,\v_1\}$, which gives the direction $\v_1-\v_0$ of the cylinder.
%
\indy{Index}{coordinate systems}%
\end{remark}

\begin{definition}[azim]\guid{UJBHGUX}\label{def:azim}
\formaldef{$\op{azim}$}{azim}
  Define $\op{azim}(\v_0,\v_1,\v_2,\v_3)$, the \newterm{azimuth} angle
  (or \newterm{longitude}), to be the uniquely determined angle
  $\theta$  given by the previous lemma for the points $\p_1=\v_2$ and $\p_2=\v_3$.
  By convention, let the azimuth angle be $0$ in the degenerate cases
  where $\{\v_0,\v_1,\v_2\}$ or $\{\v_0,\v_1,\v_3\}$ is collinear.
  \indy{Notation}{azim}%
  \indy{Index}{azimuth}%
  \indy{Index}{angle!azimuth}%
\end{definition}
\indy{Notation}{azim}%

%The azimuth angle is a polar coordinate of the projection 
%$\p -\v_0-r\cos\phi\,\e_1 \in\op{aff}\{\e_2,\e_3\}$:
%    \[ 
%    (x,y) = (r'\cos\theta,r'\sin\theta), \quad r' = r\sin\phi.
%    \] 

The azimuth and dihedral angles are closely related~(Figure~\ref{fig:dih}).  The azimuth
angle
 takes values between $0$ and $2\pi$, but the dihedral angle is
never greater than $\pi$.  The following lemma reveals that the
azimuth angle is an oriented extension of the dihedral angle and is always
equal to $\dih$ or $2\pi - \dih$.  \indy{Index}{angle!azimuth}%
\indy{Index}{angle!dihedral}%
\indy{Notation}{dih}%
\indy{Index}{azimuth}%

\begin{lemma}[]\guid{QQZKTXU}\label{lemma:dih-azim}
    \formalauthor{Nguyen Quang Truong} Let
  $\v_1\ne \v_0$ be a nonzero vectors in $\ring{R}^3$.  Assume that
  $\v_2$ and $\v_3$ do not lie in the line $\op{aff}(\v_0,\v_1)$.  Let
\[ 
\gamma = \dih_V(\{\v_0,\v_1\},\{\v_2,\v_3\}).
\] 
Then
\[ 
\cos(\op{azim}(\v_0,\v_1,\v_2,\v_3)) = \cos\gamma.
\] 
\end{lemma}

\begin{proof} For simplicity, take $\w_i = \v_i-\v_0$.  Let
  $\bar{\w_i} = (\w_1\cdot \w_1) \w_i - (\w_1\cdot \w_i) \w_1$.  From
  the assumptions of the lemma, $\bar{\w_2}\ne 0$.  Set $\e_2 =
  \bar{\w_2}/\normo{\bar{\w_2}}$.  Choose a unit vector $\e_3$ so that
  $(\e_2\times \e_3)\cdot\w_1>0$ and $\e_2\cdot \e_3 = \w_1\cdot
  \e_3=0$.  Write $\w_i$ in cylindrical coordinates as
\begin{alignat*}{3}
\w_2 &= r_1 \e_2 &    &+h_1 \w_1\\
\w_3 &= r_2 \cos\theta\, \e_2 &+ r_2 \sin\theta\, \e_3 &+ h_2 \w_1.
\end{alignat*}
The definition of $\op{azim}$ gives
$\op{azim}(\w_0,\w_1,\w_2,\w_3)=\theta$.  By definition, $\cos\gamma$
is the angle between $\bar{\w_2}$ and $\bar{\w_3}$.  We compute
\begin{align*}
\bar{\w_2} &= \normo{\bar{\w_2}} \e_2 \\
\bar{\w_3} &= (\w_1\cdot \w_1) r_2 \cos\theta\, \e_2 
+ (\w_1\cdot \w_1) r_2 \sin\theta\, \e_3 \\
\end{align*}
The result $\cos\theta=\cos\gamma$ 
is now a result of the definition of angle 
(Definition~\ref{def:angle}).
\end{proof}
\indy{Notation}{zzc@$\gamma$ (angle)}%
%\indy{Notation}{zzh@$\theta$ (angle)}%
\indy{Notation}{zzh@$\theta$ (polar, cylindrical, and spherical angle)}%

The previous lemma identifies the cosine of the azimuth angle.  The final
lemma of this subsection determines the sign of its sine.

\begin{lemma}[]\guid{JBDNJJB}\label{lemma:sim}
\formalauthor{Nguyen Quang Truong}
% 
Write $x\sim y$ when there exists $t>0$ such that $x= t y$. 
Then 
\[ 
\sin(\op{azim}(\orz,\v_1,\v_2,\v_3))\sim (\v_1
  \times \v_2)\cdot \v_3.
\] 
\end{lemma}
\indy{Notation}{9a@$\sim$ (equal up to positive scalar)}%

\begin{proof}
  The relation $(\sim)$ is an equivalence relation.  We may assume that
  $\{\orz,\v_1,\v_2\}$ and $\{\orz,\v_1,\v_3\}$ are not collinear
  sets, because otherwise both sides are zero.  Let $(\e_1,\e_2,\e_3)$ be
  the positive frame adapted to $(\orz,\v_1,\v_2)$.
%\[ 
%\begin{align}
%   \e_1 &= \v_1/\normo{\v_1}\\
%   \v_2' &= \v_2 - (\e_1\cdot \v_2) \e_1\\
%   \e_2 &= \v_2'/\normo{\v_2'}\\
%   \e_3 &= \e_1 \times \e_2 \\
%\end{align}
%\] 
Write $\v_3= r\cos\theta\, \e_2 + r\sin\theta \, \e_3 + h\, \e_1$ in
cylindrical coordinates, where $\theta =
\op{azim}(\orz,\v_1,\v_2,\v_3)$.  Then by the explicit formulas for
the positive frame,
\begin{align*}
(\v_1\times \v_2)\cdot \v_3 &\sim (\e_1\times \v_2)\cdot \v_3\\
%&= (\e_1\times \v_2')\cdot \v_3\\
&\sim (\e_1\times \e_2)\cdot \v_3\\
&= \e_3 \cdot \v_3\\
&= r\sin\theta \\
&\sim \sin\theta.
\end{align*}
\end{proof}


\subsection{zenith angle}
\label{sec:spherical}

\indy{Index}{angle!zenith}%
\indy{Index}{zenith}%
\indy{Index}{latitude}%
\indy{Notation}{zzv@$\phi$ (zenith)}%

%
%\begin{definition}[spherical coordinates]\guid{IESAXWY}
%Let $x,y,z$ be any real numbers.  A
%triple $(r,\theta,\phi)$ such that
%    \begin{equation}
%    \label{eqn:spherical}
%    x = r\cos\theta\sin\phi,\quad y = r\sin\theta\sin\phi,\quad
%    z = r\cos\phi
%    \end{equation}
%with $0\le r$, $0\le\theta<2\pi$, and $0\le\phi\le\pi$ are called
%spherical coordinates of $(x,y,z)$. 


%\begin{definition}[azimuth]\guid{OSPVIBZ}\label{def:azimuth}


The following lemma identifies the \newterm{zenith} angle $\phi$.  Because it is
easily expressed in terms of the more basic function $\arc_V$, there
is little need to refer to it directly.
\indy{Index}{orthogonal frame}%

\begin{lemma}[zenith]\guid{QAFHJNM}
    \formalauthor{Nguyen Quang Truong} Let
  $(\v_0,\v_1)$ be an ordered pair of distinct points in $\ring{R}^3$.
  Let $\v_2\ne \v_0$.  Set $\phi =
  \arc_V(\v_0,\{\v_2,\v_1\})\in[0,\pi]$.  Let $\e_1$ be the unit
  vector $(\v_1-\v_0)/\norm{\v_1}{\v_0}$.  Let $r =
  \norm{\v_2}{\v_0}$.  Then $\v_2$ can be expressed in the form
\[ 
\v_2 = \v_0 + \bar{\v}_2 +
r\cos\phi\, \e_1,
\] 
where $\bar{\v}_2\cdot \e_1 = 0$.  The angle $\phi$ is called the
\newterm{zenith} angle (or \newterm{latitude}) of $\v_2$ along
$(\v_0,\v_1)$.  \indy{Index}{zenith}%
\indy{Index}{angle!zenith}%
\end{lemma}

\begin{proof} The lemma is a direct consequence of the definition of $\arc_V$:
\[ (\v_2-\v_0)\cdot \e_1 = r\cos\phi.\] 
\end{proof}

\begin{lemma}[spherical coordinates]\guid{XPHCPNY}\label{lemma:sph}
\formal{SPHERICAL\_COORDINATES}
  Assume that
  $\{\v_0,\v_1,\v_2\}\subset\ring{R}^3$ % and $\{\v_0,\v_1,\p\}$
  is not a collinear set.  Let $(\e_1,\e_2,\e_3)$ be the positive
  frame adapted to $(\v_0,\v_1,\v_2)$.  Then for any $\p$,
\begin{equation}
\p = \v_0 + r \cos\theta \sin\phi\, \e_2 + r \sin\theta\sin\phi\, \e_3 +
r\cos\phi\,\e_1,
\label{eqn:sph}
\end{equation}
where%
  \footnote{This book follows the variable naming conventions
    $(\theta,\phi)$ of American calculus textbooks, which reverses the
    international scientific notation.} 
\begin{align*}
r &= \norm{\v_0 }{ \p}\\
\phi &= \text{zenith angle of } \p \text{ along } (\v_0,\v_1)\\
\theta &=\op{azim}(\v_0,\v_1,\v_2,\p).
\end{align*}
\end{lemma}
%\indy{Notation}{zzh@$\theta$ (azimuth)}%
\indy{Notation}{zzh@$\theta$ (polar, cylindrical, and spherical angle)}%
\indy{Index}{coordinate systems}%
\indy{Index}{coordinate systems!spherical coordinates}%
\indy{Index}{spherical!coordinates}%
%\indy{Notation}{r@$r$ (polar, cylindrical, spherical coordinate)}%
\indy{Notation}{r@$r$ (polar, cylindrical, and spherical radius)}%

 
\begin{proof}
Cylindrical coordinates give
\[ 
\p = \v_0 + r'\cos\theta\,\e_2 + r'\sin\theta\,\e_3 + h\, \e_1,
\] 
for some $h$ and $r'=\normo{\p-\v_0-h\,\e_1}\ge0$.  The zenith angle
puts $\p$ in the form
\[ 
\p = \v_0 + r'\cos\theta\,\e_2 + r'\sin\theta\,\e_3 + r\cos\phi\, \e_1,
\] 
where
\begin{align*}
r^2 &= \norm{\p}{\v_0}^2\\ 
&= \normo{\p-\v_0-h\,\e_1}^2 + \normo{h\,\e_1}^2\\
&= (r')^2 + r^2 \cos^2\phi,
\end{align*}
Since $\sin\phi$, $r$, and $r'$ are nonnegative, it follows that $r'=r\sin\phi$, as
desired.
\end{proof}

\begin{definition}[spherical coordinates]\guid{LVDJVFD}\label{def:sph}
  \formaldef{spherical coordinates}{SPHERICAL\_COORDINATES}
  Equation~\eqref{eqn:sph} is called the spherical coordinate representation of $\p$ with
  respect to $(\v_0,\v_1,\v_2)$.
\end{definition}



% Any triple $(x,y,z)$ has spherical coordinates.  The radial
% component is $r = \sqrt{x^2+y^2+z^2}$.  In the degenerate case when
% $r=0$, Equations~\eqref{eqn:spherical} becomes independent of
% $\theta$ and $\phi$. In the degenerate case when $\phi = 0$ or $\phi
% = \pi$, the equations become independent of $\theta$. If $0<r$ and
% $\phi\ne 0,\pi$, then $\theta$ is uniquely determined by $x,y,z$. If
% $0<r$, then $\theta$ is uniquely determined.
%


%The following gives the existence of polar coordinates on any
% oriented plane in three dimensions, with a general point $\v$ on the
% plane serving as the origin.  A normal vector $n$ orients the plane,
% then polar coordinates appear as the restriction of the spherical
% coordinates $(r,\theta,\phi)$ to the plane.  The following lemma
% shows that the value of $\phi$ is fixed, so that it may be dropped
% from the notation.  \indy{Index}{polar!coordinates}%
% \indy{Notation}{n@$n$ (normal vector)}%
%
%\begin{lemma}[]\guid{YBXRVTS}\label{lemma:polar-gen}
%   \formalauthor{Nguyen Quang Truong} Let $\{\v,\w,\u\}$ be
%  a set of three points in $\ring{R}^3$ that is not collinear.  Let
%  $n = (\w-\v) \times (\u-\v)$.  Then the zenith angle of any $\p\ne
%  \v$ in the plane $\op{aff}\{\v,\w,\u\}$, computed with respect to
%  $(\v,\v+n)$, is $\pi/2$.
%\end{lemma}
%\indy{Index}{zenith}%
%\indy{Index}{angle!zenith}%
%
%\begin{definition}[polar coordinate]\guid{WNLMGUV}\label{def:polar}
%Call  the two remaining coordinates, $(r,\theta)$, 
%the polar coordinates of $\p\in\op{aff}\{\v,\w,\u\}$ with
%respect to $(\v,\w,\u)$.
%\end{definition}
%\indy{Index}{coordinate systems}%
%\indy{Index}{coordinate systems!polar coordinates}%
%\indy{Index}{polar coordinates}%
%\indy{Notation}{r@$r$ (coordinate)}%
%\indy{Notation}{zzh@$\theta$ (coordinate)}%
%
%In the special case that $\op{aff}\{\v,\w,\u\}=\ring{R}^2\subset
% \ring{R}^3$, this construction agrees with the previously defined
% polar coordinates of a point in the plane.


%\subsection{Lexell without polarity}
%
%Here is a second proof of Lexell's theorem that does not depend on
% polar triangles.
%
%\begin{proof} Select coordinates so that the Lexell circle (through
%  $\u,\v^*_1,\v^*_2$) has constant zenith angle $\phi$.  Without loss
%  of generality, an appropriate coordinate system gives
%\[ 
%\begin{align}
%\v_1 &= \{\cos\theta\sin\phi,+\sin\theta\sin\phi,-\cos\phi\}\\
%\v_2 &= \{\cos\theta\sin\phi,-\sin\theta\sin\phi,-\cos\phi\}\\
%\u &= \{\cos\alpha\sin\phi,\sin\alpha\sin\phi,\cos\phi\}\\
%\end{align}
%\] 
%The area of a triangle is given by Euler's formula
% (Lemma~\ref{lemma:euler}).  If these coordinates are used in Euler's
% formula, then a calculation gives the area $\pi-2\atn(t,1)$, when
%\[ 
%t=\cos\phi \tan\theta.
%\] 
%This is independent of $\alpha$, proving that every point on the
% Lexell circle (except for the degenerate points $\u=
% \v^*_1,\u=\v^*_2$ with $\Delta=0$) gives the same solid angle.
%
% To check that points on different Lexell circles give different
% solid angles, any convenient point on the circle will do.  For
% example, there is an isosceles triangle $b=c$.  An easy derivative
% calculation shows that the function is increasing.  Hence different
% Lexell circles give different values.
%\end{proof}






\section{Cycle}


The azimuth angle of the spherical coordinate system
determines a cyclic permutation, called the azimuth cycle, on a finite
set of points in $\ring{R}^3$, ordered according to increasing azimuth
angle.  The basic properties of that permutation are developed.
\indy{Index}{cyclic!permutation}%
\indy{Index}{angle}%
\indy{Index}{coordinate systems}%


\subsection{polar cycle}

Let $V=\{\v_1,\ldots,\v_k\}$ be a finite set of nonzero points in the
plane, with polar coordinates $\v_i =
(r_i\cos\theta_i,r_i\sin\theta_i)$.  It is useful to order the set of
points according to increasing angle.  To deal with degenerate cases
when some points have exactly the same angle, order the points with
the lexicographic order on their polar coordinates.  We write $\v_i \prec
\v_j$ for the total lexicographical order on points: 
$\theta_i < \theta_j$ or both $\theta_i=\theta_j$ and $r_i<r_j$.
(The degenerate case of two equal angles does not occur in this book, but by
defining a total order, there is no need to revisit the issue.)
See Figure~\ref{fig:polarcycle}
\indy{Index}{order!total}%

\figROHSJRP % {fig:polarcycle}

\begin{definition}[polar cycle]\guid{TNZQDCX}
\formaldef{polar cycle}{polar\_cycle}
A cyclic permutation $\sigma:V\to V$ sends $\v\in V$ to
the next larger element with respect to this order or back to the
first element if $\v$ is the largest.  We call $\sigma$ the
\fullterm{polar cycle}{polar!cycle} of the set $V$.
\end{definition}
\indy{Index}{order!lexicographic}%
\indy{Index}{cyclic!permutation}%
\indy{Notation}{zzs@$\sigma$ (azimuth and polar cycle)}%




For $\psi\in\ring{R}$, let $T:\ring{R}^2\to\ring{R}^2$ be the
rotation of the plane:
\begin{equation}
\label{eqn:rotate}
(x,y) \mapsto  (x\cos\psi + y\sin\psi,-x\sin\psi+y\cos\psi).
\indy{Index}{rotation}%
\end{equation}
Let $\sigma'$ be the polar cycle for $T(V)$.  Then it easy to verify
that
\[ 
\sigma'(T \v) = T (\sigma \v),\quad \text{ for } \v\in V. 
\] 
\indy{Notation}{zzv@$\psi$}%
\indy{Notation}{T@$T$ (rotation)}%

\begin{lemma}[]\guid{PDPFQUK}\label{lemma:polar2}
    \formalauthor{Nguyen Quang Truong}
  \formal{thetaij\_t} Let $\theta_i$ be real numbers such that $0\le
  \theta_i < 2\pi$ for $i=1,2$.  Let \[  \theta_{ji}
    = \theta_i - \theta_j + 2\pi k_{ji},
\] 
where integers $k_{ij}$ satisfy $0\le \theta_{ji}< 2\pi$.
Then 
\[ 
\theta_{12} + \theta_{21} = \begin{cases}
2\pi, & \text{ if }\theta_i\ne\theta_j\\
0,    & \text{ if }\theta_i=\theta_j.
\end{cases}
\] 
\end{lemma}
%\indy{Notation}{zzh@$\theta$}%
\indy{Notation}{zzh@$\theta$ (polar, cylindrical, and spherical angle)}%

\begin{proof} The proof is elementary.
\end{proof}

The next lemma gives a precise form to the observation
that given a finite number of rays emanating from the origin
in the plane, the sum of the included angles is $2\pi$.
In precise form, the polar cycle is used to place
a cyclic order on the rays.  There is a degenerate case
when there is at most one ray.


\begin{lemma}[]\guid{ISRTTNZ}\label{lemma:polar-sum}
\formalauthor{Nguyen Quang Truong}
%\formal{thetapq\_wind\_t}
  Let $V\subset\ring{R}^2$ be a finite set of cardinality $n$ that
  does not contain $0$.  Let $\sigma$ be the polar cycle on $V$.  In
  polar coordinates,
\[ 
\v=\left(\, r(\v)\cos\theta(\v),\, r(\v)\sin\theta(\v)\,\right),
\]  
for $\v\in V$, with
$0\le\theta(\v)<2\pi$.
Write
\[ 
\theta(\v,\w) = \theta(\w) - \theta(\v) + 2\pi k_{pq},
\] 
for some integers $k_{pq}$ that satisfy $0\le \theta(\v,\w) < 2\pi$.
Then for all $\v\in V$
and all $0\le i \le j < n$,
\[ 
\theta(\v,\sigma^i(\v)) +\theta(\sigma^i(\v),\sigma^j(\v)) =
\theta(\v,\sigma^j(\v)).
\] 
Moreover, if there exist $\v,\w\in V$ such that $\theta(\v)\ne\theta(\w)$,
\[ 
\sum_{i=0}^{n-1} \theta(\sigma^{i}\v,\sigma^{i+1} \v) = 2\pi.
\] 
(If $\theta(\v)=\theta(\w)$ for all $\v,\w\in V$, then all the
summands are zero.)
\end{lemma}
%\indy{Notation}{zzs@$\sigma$ (permutation)}%
\indy{Notation}{zzs@$\sigma$ (azimuth and polar cycle)}%

\begin{proof}
Fix $\v\in V$.
For $0\le i<n$, define $\theta_i$ by
$\theta_0=\theta(\v)$ and 
\[ \theta_i = \theta(\sigma^i(\v)) + 2\pi \ell_i,
\] 
where  $\ell_i$ satisfies $\theta_0\le \theta_i < \theta_0+2\pi$.
It follows from the definition of the polar cycle that
$\theta_i \le \theta_j$ for $0\le i\le j < n$.  Then
$\theta(\sigma^i \v ,\sigma^j \v) = \theta_j - \theta_i$.
The first conclusion of the lemma reduces to
\[ 
(\theta_i-\theta_0) + (\theta_j-\theta_i) = (\theta_j-\theta_0),
\] 
which is certainly true.
The second conclusion reduces to
\[ 
\sum_{i=0}^{n-2} (\theta_{i+1}-\theta_i) + \theta(\sigma^{n-1}\v,\v)
= \theta(\v,\sigma^{n-1}\v) + \theta(\sigma^{n-1}\v,\v).
\] 
By the previous lemma, this is $0$ or $2\pi$.
\end{proof}


\subsection{azimuth cycle}

As already defined, the polar cycle is a cyclic permutation on a set
of vectors in the plane that traverses them in order of increasing
angle.  What follows is the corresponding construction in three
dimensional space.  There is a cyclic permutation, called the
\newterm{azimuth cycle}, on a set $V$ of vectors in space that
traverses them in order of increasing azimuth angle.  Most of the work
for this construction has already been done in the subsection on polar
cycle, because the azimuth cycle may be constructed as the polar cycle
on the projection of $V$ to a plane.  However, a nondegeneracy
condition must be imposed on $V$ to ensure that the projection to the
plane is one-to-one.  The following definition captures this
nondegeneracy condition.  \indy{Index}{azimuth cycle}%
\indy{Index}{azimuth}%
\indy{Index}{cyclic!permutation}%
\indy{Index}{vector!projection}%


\begin{definition}[cyclic set]\guid{KFKHLWK}
\formaldef{cyclic set}{cyclic\_set} 
Let $(\v_0,\v_1)$ be an ordered pair of
  distinct points in $\ring{R}^3$.  Let $V$ be a finite set of points
  in $\ring{R}^3$.  We say that $V$ is \fullterm{cyclic}{cyclic!set} with respect to
  $(\v_0,\v_1)$ if the following two conditions hold:
\begin{enumerate}
\item If $\u = \w + h (\v_1-\v_0)$, with $\u,\w\in V$ and $h\in \ring{R}$,
then $\u=\w$.  
\item  The line through $\v_0$ and $\v_1$ does not meet $V$.
\end{enumerate}
\end{definition}

A cyclic set $V$ has a well-defined azimuth cycle (Figure~\ref{fig:azimuthcycle}).

\figHOUNZSY % {fig:azimuthcycle}

\begin{definition}[azimuth cycle]\guid{YESEEWW}
\formaldef{$\sigma$}{azim\_cycle}
  Let $\v_0$ and $\v_1$ be distinct points in $\ring{R}^3$.  Let $V$
  be a finite set of points in $\ring{R}^3$ that is cyclic with
  respect to $(\v_0,\v_1)$.  Pick $\p\in\ring{R}^3$ such that
  $\{\v_0,\v_1,\p\}$ is not collinear and let $\{\e_1,\e_2,\e_3\}$ be
  the corresponding positive, adapted, frame.  Let $f$ be the
  projection map:
\[ \v_0 + x\, \e_2 + y\, \e_3 + z\, \e_1 \mapsto
(x,y).\] 
Let $\sigma'$ be the polar cycle on $f(V)$. We define
$\sigma:V\to V$ by $f\sigma(\u) =\sigma'f(\u)$
and call $\sigma$ the \newterm{azimuth cycle}
on $V$ with respect to $(\v_0,\v_1)$.
\indy{Index}{azimuth cycle} %
\indy{Index}{frame}%
\indy{Index}{polar!cycle}%
\indy{Notation}{f@$f$ (function name)}%
\indy{Notation}{zzs@$\sigma$ (azimuth and polar cycle)}%
%\indy{Notation}{zzs@$\sigma$ (polar cycle)}%
%\indy{Notation}{zzs@$\sigma$ (azimuth cycle)}%
\end{definition}

Because facts about the polar cycle lift to facts about the azimuth cycle,
the next few lemmas follow naturally.


\begin{lemma}[]\guid{NLOFMTR} The azimuth cycle $\sigma:V\to V$ on
  a cyclic set $V$ with respect to $(\v_0,\v_1)$ does not depend on
  the choice of $\p\in\ring{R}^3$ such that $\{\v_0,\v_1,\p\}$ is
  noncollinear.
\end{lemma}
\indy{Index}{azimuth cycle}%
\indy{Index}{cyclic!set}%

\begin{proof} The lemma follows from independence of $\sigma\,'$ from
rotations in the $\{\e_2,\e_3\}$ plane  in~\eqref{eqn:rotate}.
\end{proof}


\begin{lemma}[]\guid{YVREJIS} 
\formalauthor{Nguyen Quang Truong}
Let $(\v_0,\v_1)$ be an ordered pair of points in $\ring{R}^3$,
with $\v_0\ne \v_1$.  Assume that $\{\v_2,\v_3\}$ is cyclic
with respect to $(\v_0,\v_1)$.  Then
\[ 
\op{azim}(\v_0,\v_1,\v_2,\v_3) + \op{azim}(\v_0,\v_1,\v_3,\v_2) 
= \begin{cases} 2\pi, & \text{if }\op{azim}(\v_0,\v_1,\v_2,\v_3)\ne 0,\\
0, & \text{if }\op{azim}(\v_0,\v_1,\v_2,\v_3)=0.
\end{cases}
\] 
\end{lemma}


\begin{proof} The lemma follows immediately from Lemma~\ref{lemma:polar2}.
\end{proof}

\begin{lemma}[]\guid{ULEKUUB} \label{lemma:2pi-sum}
Let $(\v_0,\v_1)$ be an ordered pair of points in $\ring{R}^3$,
with $\v_0\ne \v_1$.  Let $V$ be a finite set in $\ring{R}^3$ of
cardinality $n$ that
is cyclic with respect to $(\v_0,\v_1)$,
with azimuth cycle $\sigma$.
Then for all $\u\in V$,
and all $0\le i \le j < n$,
\[ 
\op{azim}(\v_0,\v_1,\u,\sigma^i(\u)) +
\op{azim}(\v_0,\v_1,\sigma^i(\u),\sigma^j(\u)) =
\op{azim}(\v_0,\v_1,\u,\sigma^j(\u)).
\] 
Moreover, if there exists $\w\in V$ such that 
$\op{azim}(\v_0,\v_1,\u,\w)\ne0$,
then
\[ 
\sum_{i=0}^{n-1} \op{azim}(\v_0,\v_1,\sigma^i\u,\sigma^{i+1}\u) = 2\pi.
\] 
(If $\op{azim}(\v_0,\v_1,\u,\w)=0$ for all $\w\in V$, then all the
summands are zero.)
\end{lemma}
\indy{Notation}{azim}%
\indy{Index}{azimuth}%
\indy{Index}{azimuth cycle}%
%\indy{Notation}{zzs@$\sigma$}%
\indy{Notation}{zzs@$\sigma$ (azimuth and polar cycle)}%
\indy{Notation}{n@$n$ (integer variable)}%

\begin{proof} This follows immediately from 
Lemma~\ref{lemma:polar-sum}.
\end{proof}


\subsection{spherical triangle inequality} %%
\indy{Index}{triangle!spherical}%
\indy{Index}{spherical!triangle inequality}%

The geodesic length between two points
$\u,\v$ on a unit sphere centered at $\v_0$ is $\arc_V(\v_0,\{\u,\v\})$.
The following lemma is part of the verification that
the function $d(\u,\v) = \arc_V(\v_0,\{\u,\v\})$ is a metric
on the unit sphere.  The lemma excludes the degenerate case when
points on the sphere are antipodal.
\indy{Notation}{d@$d(\u,\v)$ (metric on $\ring{R}^3$)}%

\begin{lemma}[]\guid{KEITDWB}\label{lemma:sph-tri-ineq}

\formalauthor{Nguyen Quang Truong}
Let $\{\v_0,\v_1,\v_2,\v_3\}$ be a set of four points in $\ring{R}^3$.
Assume that $\v_0$ is not collinear with any of pair of the other points.
Then
\[ 
  \arc_V(\v_0,\{\v_1,\v_3\}) \le \arc_V(\v_0,\{\v_1,\v_2\}) + \arc_V(\v_0,\{\v_2,\v_3\}).
\] 
Equality occurs if and only if $\v_2\in\op{aff}_+(\v_0,\{\v_1,\v_3\})$.
\end{lemma}

\begin{proof} Let $\v_2'$ be the projection of $\v_2$ to the plane
$\op{aff}\{\v_0,\v_1,\v_3\}$.  
By the spherical law of cosines, when the triangle is right
\[ 
\cos\psi = \cos\beta\cos\alpha \le \cos\beta,
\] 
where $\psi = \arc_V(\v_0,\{\v_1,\v_2\})$, $\beta =
\arc_V(\v_0,\{\v_1,\v_2'\})$, $\alpha=\arc_V(\v_0,\{\v_2,\v_2'\})$.  Thus,
$\arc_V(\v_0,\{\v_1,\v_2'\})=\beta\le \psi=\arc_V(\v_0,\{\v_1,\v_2\})$.
Similarly, $\arc_V(\v_0,\{\v_2',\v_3\}) \le \arc_V(\v_0,\{\v_2,\v_3\})$.
Thus, it is enough to show that
\[ 
  \arc_V(\v_0,\{\v_1,\v_3\}) \le \arc_V(\v_0,\{\v_1,\v_2'\}) + \arc_V(\v_0,\{\v_2',\v_3\}).
\] 
The points $\v_0,\v_1,\v_3,\v_2'$ are coplanar.
By the additivity of planar angle (Lemma~\ref{lemma:polar-sum}), if 
$\v_2'\in \op{aff}_+(\v_0,\{\v_1,\v_3\})$, then
\[ 
  \arc_V(\v_0,\{\v_1,\v_3\}) = \arc_V(\v_0,\{\v_1,\v_2'\}) + \arc_V(\v_0,\{\v_2',\v_3\}),   
\] 
and otherwise,
\[ 
  \arc_V(\v_0,\{\v_1,\v_3\}) = \norm{\arc_V(\v_0,\{\v_1,\v_2'\}) }{ \arc_V(\v_0,\{\v_2',\v_3\})}.
\] 
The inequality ensues.

A trace of the argument shows that equality occurs exactly when
$\alpha=0$ and $\v_2'\in \op{aff}_+(\v_0,\{\v_1,\v_3\})$.  Equivalently,
$\v_2'=\v_2\in\op{aff}_+(\v_0,\{\v_1,\v_3\})$.
\end{proof}

\begin{lemma}[]\guid{FGNMPAV}
\formalauthor{Nguyen Quang Truong}
\label{lemma:sph-tri-multi}
Let $\{\v_0,\u_0,\u_1,\u_2,\ldots,\u_r\}$ be a set of points in
$\ring{R}^3$.  Assume that no triple $\{\v_0,\u_i,\u_{i+1}\}$ is
collinear.  Assume that $\{\v_0,\u_0,\u_r\}$ is not collinear.  Then
\[ 
  \arc_V(\v_0,\{\u_0,\u_r\}) \le \sum_{i=0}^{r-1} \arc_V(\v_0,\{\u_i,\u_{i+1}\}).
\] 
\end{lemma}

\begin{proof} The proof is an easy induction on $r$ with base case given by
  Lemma~\ref{lemma:sph-tri-ineq}.
\end{proof}


\section{Chapter Summary}

%\subsection{formal proof}

%Formal proofs of all of the major results in this chapter have already
%been constructed.  In fact, many of them are part of standard
%distribution of the HOL Light proof assistant.  In 2008, Jason Rute
%constructed the formal proof of some of the theorems in this chapter.
%Most of the remaining work was carried out by Nguyen Quang Truong in
%2009.  He also constructed the formal proofs of a large collection of
%lemmas in elementary geometry, including basic facts about the
%functions $\ups$ and $\Delta$.  Finally in 2010, Euler's theorem was
%formalized by Vu Khac Ky and Trieu Thi Diep.
%

%\subsection{summary of notation}

We give a brief chapter summary.  The trigonometric functions $\cos$,
$\sin$, $\arctan$, $\arccos$ are defined in the standard way.  The
function $\atn(x,y)$ is an extension of $\arctan(y/x)$ to every point
$(x,y)$ in the plane.  It is the polar coordinate angle of $(x,y)$.

 $\ring{R}^N$ is the vector space of functions from the finite set $N$ to $\ring{R}$.  If
$n\in\ring{R}$,   then by convention, $\ring{R}^n = \ring{R}^N$, where $N=\{0,\ldots,n-1\}$.
A bold face $\u,\v,\p,\q$ is used for points in $\ring{R}^N$.   Vector space operations, the
dot product $\u\cdot \v$, and the norm $\normo{\u}$ are defined in the standard way.

We write $\op{aff}(S)$ for the affine hull of a set and $\op{conv}(S)$ for the convex hull of a set.
The notation is extended to allow inequality constraints:
	\begin{align*}
\op{aff}_{\pm} (V,V') &= \{t_1 \v_1 +\cdots t_n \v_n \mid
	t_1 +\cdots+t_n = 1, \pm t_j \ge 0, \text{ for } j>k\},\\
\op{aff}^0_{\pm} (V,V') &= \{t_1 \v_1 +\cdots t_n \v_n \mid
	t_1 +\cdots+t_n = 1, \pm t_j > 0, \text{ for } j>k\}.
%\op{aff}\, V &= \op{aff}_\pm(V,\emptyset).\\
\end{align*}
Lines, planes, rays, cones, half-planes, half-spaces, and convex hulls
can all be represented compactly in this notation.

The polynomials $\ups$ and $\Delta$, which appear in
formulas for angle, area, and volume, depend on three and six
variables, respectively.  The function $\arc_V(\u,\{\v,\w\})$ gives
the angle at point $\u$ of a triangle with vertices $\u,\v,\w$.  The
function $\arc(a,b,c)$ is the angle opposite $c$ of a triangle with
sides $a,b,c$.  The function $\dih_V$ is the dihedral angle of a
simplex, expressed as a function of its four vertices.  The function
$\dih$ is the dihedral angle of a simplex, expressed as a function of
its six edges.

The cylindrical coordinates of a point in $\ring{R}^3$ are
$(r,\theta,h)$.  The spherical coordinates are $(r,\theta,\phi)$.  The
angle $\theta$ is called the azimuth angle and is determined by four
points $\v_0,\v_1,\v_2,\v_3$.  The angle $\phi$ is the zenith angle.
This book follows a nonstandard convention for the labeling of the
coordinate axes in cylindrical and spherical coordinates: the central
line of the cylinder and the line through the poles of the coordinate
sphere lie in the direction of the first unit vector $\e_1$.

The cyclic permutation of a finite set of points in the plane, ordered
by increasing angle in polar coordinates is called the polar cycle.
The cyclic permutation $\sigma$ of a finite set of points in
three-dimensional space, ordered by increasing azimuth angle is called
the azimuth cycle.


  
    %\lll
    %%% Quadratic Volumes
%% File Created 3/22/07.

\section{Properties of Measure}

Nowhere do we need a notion
of integration.  Measure alone suffices.  (However, there are a few
volumes described below that I do not see how to calculate without
first writing them as an integral.)

We need a concepts of null set, measurable set, and volume in
three dimensions.  For our purposes, we can take the
the three dimensional Lebesgue measure.   
The null sets can be defined
to be the sets of zero Lebesgue measure. The measurable sets can
be defined as the bounded Lebesgue measurable sets.  The volume of
a measurable set can be defined as its Lebesgue measure.
As we will see in a moment, we need considerably less than Lebesgue measure.



\subsection{properties of null sets}

We assume a notion of null set with the following
properties.

\begin{enumerate}%[Null Set]
\item A finite union of null sets is a null set.\\
 \item A plane is a null set.\\
 \item A sphere is a null set.\\
 \item A circular cone is a null set; that is, a union of all
  lines through a fixed point $P$ and forming fixed
 forming fixed angle with a line through $P$.
\tlabel{enum:null}
\end{enumerate}

We write $A\equiv B$ if sets $A$ and $B$ are equal up to a null set.
That is, there exists a null set $E$ such that
   $(A\setminus B) \cup (B\setminus A) \subset E$.
\index{null set}\index{ZZZequiv@$\equiv$}

\subsection{properties of measure}

We assume a notion of measurability that has the following properties.

\begin{enumerate}%[Measurable set]
 \item The union of two measurable sets is measurable.\\
 \item The intersection of two measurable sets is measurable.\\
 \item The difference of two measurable sets is measurable.
\tlabel{enum:measure}
\end{enumerate}

\subsection{properties of volume}

We assume a notion of volume that has the following properties.

\begin{enumerate}%[Volume]
 \item The volume is defined for every measurable set.  It is
    a non-negative real number.
 \item If $X$ and $Y$ are  measurable, and if
 the symmetric difference of
 $X$ and $Y$ is contained in a null set, then 
    $X$ and $Y$ have the same volume.\\
 \item If $X$ and $Y$ are measurable sets, and if $X\cap
 Y$ is contained in a null set, then
    $$
    \op{vol}(X\cup Y) = \op{vol}(X) + \op{vol}(Y).
    $$
  \item (linear stretch) If $X\subset \ring{R}^3$, $t\in\ring{R}$, 
    $i=1,2,3$, and $e_i\in\ring{R}^3$ is the $i$th standard basis vector,
    set 
      $$T_i(X,t) = \{ u + (t-1) u_i e_i \mid u\in X\}.
      $$
    If $X$ is measurable, then $X'=T_i(X,t)$ is as well,
    and $\op{vol}(X') = |t|\op{vol}(X)$.
  \item (translation) If $X\subset \ring{R}^3$ and $v\in\ring{R}^3$, then let
    $X+v = \{x + v\mid x\in X\}$.  If $X$ is measurable, then $X+v$ is
    as well, and $\op{vol}(X) = \op{vol}(X+v)$.
\tlabel{enum:volume}
\end{enumerate}

In particular, if $X$ is contained in a null set, we may take
$X=Y$ in the preceding to deduce that $\op{vol}(X)=0$.

In addition to these properties, we will also need specific
volume calculations of primitive regions as described in
Lemmas~\ref{lemma:prim-volume} and~\ref{lemma:wedge-vol}.
%% NB Don't need lemma:wedge-sol because solid of a FR is a SC.
%% Don't need lemma:prim-sol, because solds are SC or ST.

\subsection{radial sets and solid angle}\label{sec:solid}


Surface integrals are not required in this book.  Although
the `solid angle' is traditionally defined as a surface integral,
we give an alternative definition based on volume.


\begin{definition}
    A set $C$ is $r$-radial at center $x$ if  $C\subset B(x,r)$
    and if
        $x + u \in C$ implies
        $x + t u \in C$ for all $t$ satisfying $0\le |u| t < r$.
A set $C$ is eventually radial at center $x$ if $C\cap B(x,r)$ is
$r$-radial at center $x$, for some $r>0$.
\end{definition}

\begin{lemma}\tlabel{lemma:r-r'}
Assume that $C$ is measurable and $r$-radial at $x$.  Let $0\le r'<r$,
then $C\cap B(x,r')$ is measurable and
$\op{vol}(C\cap B(x,r')) = \op{vol}(C) (r'/r)^3$.
\end{lemma}

\begin{proof}  We can transform $C$ into $C\cap B(x,r')$ by
a series of translations and stretch transformations.
\end{proof}


\begin{definition}\tlabel{def:sol}
If $C$ is measurable and eventually radial at center $x$, then we
define the solid angle of $C$ at $x$ to be
    $$
    \op{sol}(x,C) = 3 \op{vol}(C\cap B(x,r))/r^3,
    $$
where $r$ is as in the definition of eventually radial. 
By Lemma~\ref{lemma:r-r'}, this
definition is independent of any such $r$.  When the center $x$ is
clear from the context, we write $\op{sol}(C)$ for
$\op{sol}(x,C)$.
\end{definition}



The following properties follow immediately from the definitions.
If $C$ is $r$-radial for some $r>0$ then it is eventually radial.
If $C$ is measurable and $r$-radial, then the volume of $C$
satisfies
    $$
    \op{vol}(C) = \op{sol}(C) r^3/3.
    $$
If $C$ is bounded away from $x$, then $C$ is eventually radial at
$x$, and $\op{sol}(C) = 0$.

\begin{lemma}  If $C$ and $C'$ are  $r$-radial
at $v_0$, then $C\cap C'$ is also $r$-radial at
$x$.
\end{lemma}






\section{Primitive Volumes}

We accept 
certain elementary volume calculations as axioms.  
These regions will be called primitive volumes.  There are only
a few primitive volumes.
All further
volumes calculations will be obtained from these through the basic
properties of measure.   
Our treatment of volume 
is hardly about measure at all.  The focus is rather on
the geometry of the various regions and how to decompose them into
primitives.

We prefer to take the volume of open sets whenever that can be
arranged.  We begin with a description of some of the primitive
regions.






\subsection{ball}

\begin{definition}  The open ball $B(x,r)$ with center $x$ and
radius $r$ is the set
    $$
    \{ y\in\ring{R}^3 \mid |x-y| < r.\}
    $$
\end{definition}



\subsection{wedge}


The set $\op{aff}^0_+(\{v_0,v_1\},\{v_2,v_3\})$ was defined
in Definition~\ref{def:aff}.  We call it a lune.  It is the intersection
of two open half-spaces
    $$
    \op{aff}^0_+(\{v_0,v_1\},\{v_2,v_3\})
    =\op{aff}^0_+(\{v_0,v_1,v_2\},v_3)\cap
    \op{aff}^0_+(\{v_0,v_1,v_3\},v_2)
    $$


A lune has a dihedral angle $\dih(\{v_0,v_1\},\{v_2,v_3\})$ between
$0$ and $\pi$.   For angles that are larger than $\pi$,  we use a wedge
$W(v_0,v_1,w_1,w_2)$.  Assume that $v_0\ne v_1$ and that
$w_1$ and $w_2$ do not lie on
the line $\op{aff}\{v_0,v_1\}$.  Set
$$
W(v_0,v_1,w_1,w_2) = 
  \{x\not\in\op{aff}\{v_0,v_1\} \mid 
  0< \op{azim}(v_0,v_1,w_1,x) < \op{azim}(v_0,v_1,w_1,w_2)\}.
$$
When the angle is less than $\pi$, there is no difference between
a wedge and a lune:

\begin{lemma} Let $\{v_0,v_1,v_2,v_3\}$ be a set of four points
in $\ring{R}^3$.  Assume that the set is not coplanar.
Assume that $\op{azim}(v_0,v_1,v_2,v_3)<\pi$.
Then,
   $$W(v_0,v_1,v_2,v_3) = \op{aff}^0_-(\{v_0,v_1\},\{v_2,v_3\}).$$
\end{lemma}


\subsection{solid triangle}

\begin{definition} The solid triangle $ST(v_0,\{v_1,v_2,v_3\},r)$ is
specified by four points $v_i\in\ring{R}^3$, and a radius $r\ge0$. 
    $$
    ST(v_0,\{v_1,v_2,v_3\},r) = 
    B(v_0,r)\cap \op{cone}(v_0,\{v_1,v_2,v_3\}).
    $$
\end{definition}



\subsection{conic cap}

% renamed from spherical cap.

\begin{definition}
The conic cap $SC(v_0,v_1,r,a)$ is specified by an apex
$v_0\in\ring{R}^3$, a radius $r\ge0$, a non-zero vector $v_1-v_0$ giving
direction, and constant $a$.  The conic cap is the intersection of
the ball $B(v_0,r)$ with a solid right-circular cone:
    $$
    SC(v_0,v_1,r,a)=\{y \in B(v_0,r) \mid (y-v_0)\cdot (v_1-v_0) > |y-v_0|\, |v_1-v_0|\, a\}.
    $$
\end{definition}

\subsection{frustum}

\begin{definition}\tlabel{def:p:rcone}
%\begin{definition}\tlabel{def:rcone} % from tarski..
\index{rcone}\index{right-circular cone}
We define the following collection of right-circular cones.
If $v$ and $w$ are points in $\ring{R}^3$, and
  $h\in\ring{R}$, then set
  $$\begin{array}{lll}
    \op{rcone}(v,w,h) &= \{x\mid (x-v)\cdot (w-v) \ge |x-v|\,|w-v| h\},\\
    \op{rcone}^0(v,w,h) &= \{x\mid (x-v)\cdot (w-v) > |x-v|\,|w-v| h\}.\\
    \op{rcone}_-^0(v,w,h) &= \{x\mid (x-v)\cdot (w-v) < |x-v|\,|w-v| h\}.\\
    \partial\op{rcone}(v,w,h) &= \{x\mid (x-v)\cdot (w-v) = |x-v|\,|w-v| h\}.\\
    \end{array}
    $$
\end{definition}



\begin{definition} The frustum
$FR(v_0,v_1,h',h,a)$ is specified by an apex $v_0\in\ring{R}^3$, heights
$0\le h'\le h$, a vector $v_1-v_0$ giving its direction, and
$a\in[0,1]$. The set $FR$ is given as
    $$
    \{ y \in\op{rcone}^0(v_0,v_1,a) \mid \ 
       h'|v_1-v_0| < (y-v_0) \cdot (v_1-v_0) < h|v_1-v_0| \}.
    $$
\end{definition}

That is, the frustum is the part of a right-circular cone between two
parallel planes that cut the axis of the cone at a right angle.
When $h'=0$, the frustum extends to the apex of the cone, and
we write $FR(v_0,v_1,h,a)=FR(v_0,v_1,h',h,a)$.

\subsection{tetrahedron}

\begin{definition} A tetrahedron is a set of the form
$$\op{conv}^0\{v_1,v_2,v_3,v_4\}.$$
\end{definition}

By Lemma~\ref{tarski:hedra-tope}, this set can also be described
as the intersection of four open half-spaces, with each bounding
plane defined by three of the four points.
Taking this into account, we note that
the sets in this section have all been defined by linear and quadratic
constraints.

\subsection{primitive}

\begin{definition} A primitive region is any of the following.

\begin{enumerate}%%[Primitive Volumes]
 \item A solid triangle $ST$.
 \item A tetrahedron $S$.
 \item A wedge of a frustum (with $h'=0$); 
that is, the intersection of a frustum with
 a wedge:
    $$
     FR(v_0,v_1,h,a) \cap W(v_0,v_1,v_2,v_3).
    $$
\item A wedge of a conic cap; that is, the intersection of a conic cap
with
    a wedge:
    $$
    SC(v_0,v_1,r,c) \cap W(v_0,v_1,v_2,v_3).
    $$
\tlabel{enum:volume-prim}
\end{enumerate}

\end{definition}

\subsection{primitive volume calculation}

\begin{lemma}\tlabel{lemma:prim-volume} 
\begin{enumerate} 
 \item Let $v_1,v_2,v_3$ be unit vectors.
   A solid triangle $ST(v_0,\{v_1,v_2,v_3\},r)$ has volume
   $$
   (\alpha_{123}+\alpha_{231}+\alpha_{312}-\pi)r^3/3,
   $$
   where $\alpha_{ijk} = \dih_V(\{v_0,v_i\},\{v_j,v_k\})$.
  \item The conic cap $SC(v_0,v_1,r,a)$ has volume:
   $$
    2\pi(1-a) r^3/3,
   $$
 \item A frustum $FR(v_0,v_1,h,a)$ has volume:
   $$
   \pi (t^2-h^2) h/3,\quad h = t a.
   $$
 \item A tetrahedron $\op{conv}^0(\{v_1,v_2,v_3,v_4\})$ has volume:
   $$
   \sqrt{\Delta(x_{12},x_{13},x_{14},x_{34},x_{24},x_{23})}/12,
   $$
   where $x_{ij} = |v_i-v_j|^2$.
\end{enumerate}
\end{lemma}

Euler's formula (Lemma~\ref{lemma:euler}) gives an
equivalent expression for $(\alpha_{123}+\alpha_{231}+\alpha_{312}-\pi)$.
Euler's formula will often be used instead of this formula.

\begin{proof}
The formula for the volume of a solid triangle is $r^3/3$ times
its solid angle.  The formula 
   $$\alpha_{123}+\alpha_{231}+\alpha_{312}-\pi$$
for the area of a spherical triangle is classical.    
The conic cap volume is
$r^3/3$ times its solid angle.  
The volume of a right-circular cone is $1/3$ its base times height.
The volume of a tetrahedron is
   $$|\det(v_2-v_1,v_3-v_1,v_4-v_1)|/6.$$
By Lemma~\ref{tarski:cm4}, 
the square of this determinant is given by a formula
$\op{CM}_4(x_{ij})$, which by Lemma~\ref{tarski:cm4} is
$\Delta/4$, with $\Delta\ge0$.  The result follows.
\end{proof}



\subsection{wedge}

If the region is realized by revolution along an axis $\op{aff}\{v_0,v_1\}$, 
then
we can also give the volume of the intersection of the region
with a wedge $W=W(v_0,v_1,v_2,v_3)$.
  In the following
let $\theta = \azim(v_0,v_1,v_2,v_3)$.

\begin{lemma}\tlabel{lemma:wedge-vol}  Let $C$ be either $SC(v_0,v_1,r,a)$ or
   $FR(v_0,v_1,h,a)$.  Let $m$ be the volume of $C$.  
   Then $C\cap W$ has volume $m\,\theta/(2\pi)$.   
\end{lemma}

\begin{lemma}\tlabel{lemma:wedge-sol}  Let $C$ be either $SC(v_0,v_1,r,a)$ or
   $FR(v_0,v_1,h,a)$.  Then $C$ and $C\cap W$ are eventually 
radial at $v_0$. Furthermore,
    $C\cap W$ solid angle 
  $s\,\theta/(2\pi)$, where $s$ is the solid angle of $C$.
\end{lemma}


\begin{proof}
These are elementary integrals.
\end{proof}


\subsection{solid angle of primitives}

All of the primitive sets are eventually radial at the natural
base point $v_0$, so we may take their
solid angle.  By Lemma~\ref{lemma:wedge-sol}, it is enough to compute
the solid angle before intersecting with a wedge.

\begin{lemma} \tlabel{lemma:prim-sol}
\begin{enumerate}
    \item  $ST(v_0,\{v_1,v_2,v_3\})$ is eventually radial at $v_0$
     with solid angle 
     $$
     (\alpha_{123}+\alpha_{231}+\alpha_{312}-\pi),\quad
     \alpha_{ijk}=\dih_V(\{v_0,v_i\},\{v_j,v_k\}).
     $$
    \item $\op{conv}^0\{v_0,v_1,v_2,v_3\}$ is eventually radial at $v_0$
      with solid angle
           $$
     (\alpha_{123}+\alpha_{231}+\alpha_{312}-\pi),\quad
     \alpha_{ijk}=\dih_V(\{v_0,v_i\},\{v_j,v_k\}).
     $$
    \item $SC(v_0,v_1,r,a)$ is eventually radial at $v_0$ with solid
      angle 
      $2\pi(1-a)$.
    \item $FR(v_0,v_1,h,a)$ is eventually radial at $v_0$ with solid
      angle
        $$
        2\pi (1-a).
        $$
\end{enumerate}
\end{lemma}

\begin{proof} In every case, the intersection of 
  the region with $B(v_0,r')$, for $r'>0$ sufficiently small, is
  a conic cap or a solid triangle.  These two volumes have
  already been calculated.  This gives the results as stated.
\end{proof}

\subsection{combining solid angle and volume}

It is often convenient to consider various linear combinations
of the solid angle and volume of eventually radial sets.  

\begin{definition}\label{def:sovo}
With
that in mind, we define the function
  $$
  \op{sovo}(v_0,V,\lambda) = \lambda_v \op{vol}(V) + \lambda_s
  \op{sol}(v_0,V),
  $$
where $V$ is a measurable set that is eventually radial at $v_0$
and $\lambda=(\lambda_v,\lambda_s)$ is a pair of real numbers
determining the linear combination.
\end{definition}

We define some auxiliary functions that will help us express
the value of $\op{sovo}$ on primitive regions.

\begin{definition}\tlabel{def:A}\tlabel{def:phi}
Define the function
 $$
 \phi(h,t,\lambda)=
   \lambda_v  t h (t+h)/6 + \lambda_s 
 $$
Define the function
 $$A(h,t,\lambda) = (1-h/t) (\phi(h,t,\lambda) - \phi(t,t,\lambda)).$$
\index{A}
\index{phi}
\end{definition}

\begin{lemma} If $V$ is measurable and $t$-radial at $v_0$,
then $$\op{sovo}(v_0,V,\lambda) = \op{sol}(v_0,V)\phi(t,t,\lambda).$$
\end{lemma}

\begin{proof} We have $$\vol(V) = \op{sol}(v_0,V)t^3/3,$$
so $$\op{sovo}(v_0,V,\lambda) = 
  \op{sol}(v_0,V)(\lambda_v t^3/3 + \lambda_s) = 
   \op{sol}(v_0,V)\phi(t,t,\lambda).$$
\end{proof}

\begin{lemma}\tlabel{lemma:sovoFR} Let $0 < h < t$.
Set $F  = FR(v_0,v_1,h,h/t)$ and $s = \sol(v_0,F)$.
  We have
  $$
  \begin{array}{lll}
  \op{sovo}(v_0,F,\lambda) 
   &= s
  \phi(h,t,\lambda) \\
   &= 2\pi A(h,t,\lambda) + s\phi(t,t,\lambda).
  \end{array}
  $$
\end{lemma}

\begin{proof}    From the primitive volume calculations,
we have 
  $$
  \begin{array}{lll}
  s &= 2\pi(t-h)/t,\\
  \op{vol}(F) &= \pi(t^2-h^2)h/3\\
      &= s t h (t+h)/6,\\
  \op{sovo}(v_0,F) &= 
     \lambda_v s t h (t+h)/6 + \lambda_s s\\
   &= s \phi(h,t,\lambda).
  \end{array}
  $$
Also,
  $$
  \begin{array}{lll}
  s\phi(h,t,\lambda) &= 2\pi (1-h/t)\phi(h,t,\lambda)\\
  &= 2\pi A(h,t,\lambda) + 2\pi (1-h/t)\phi(t,t,\lambda)\\
  &= 2\pi A(h,t,\lambda) + s \phi(t,t,\lambda).
  \end{array}
  $$
\end{proof}

\section{Scissors and Volumes}
\tlabel{sec:measure-second}

There are many other volumes that can be computed from the
primitive ones enumerated in Definition~\ref{enum:volume-prim}.

\subsection{lune}  

To give a simple example of a derived volume, we consider the
lune $A=\op{aff}_+^0(\{v_0,v_1\},\{v_2,v_3\})$.  It is eventually
radial at $v_0$, so we may compute its solid angle.

\begin{lemma}  $\sol(v_0,A) = 2\dih_V(\{v_0,v_1\},\{v_2,v_3\})$.
\end{lemma}

\begin{proof}
Let $B_{\pm} = \op{aff}_\pm(\{v_0,v_2,v_3\},v_1)$.  $B_- \cap B_+$
is a null set.  The intersections $A\cap B_{\pm}\cap B(v_0,r)$ 
are solid triangles.  This gives the solid angle of $A$ as
follows:
   $$\begin{array}{lll}
   \sol(v_0,A) &= \sol(v_0,A\cap B_+)+\sol(v_0,A\cap B_-) \\
   &= 
   \sol(ST(v_0,\{v_1,v_2,v_3\})) + \sol(ST(v_0,\{v_1',v_2,v_3\})) \\
   &=
   2\dih_V(\{v_0,v_1\},\{v_2,v_3\}).
   \end{array}
   $$
Here, we have used $v_1'= 2 v_0 - v_1$, the reflection of $v_1$
through $v_0$.  The usual calculation of the volume of a solid triangle
inverts this proof, 
and derives the volume from the solid angle of a lune.
\end{proof}



\begin{lemma}\tlabel{lemma:wedge:sol} 
Assume that the sets $\{v_0,v_1,w_1\}$ and
$\{v_0,v_1,w_w\}$ are not collinear. 
$$\sol(v_0,W(v_0,v_1,w_1,w_2)) = 2\op{azim}(v_0,v_1,w_1,w_2).$$
\end{lemma}    

\begin{proof} Every wedge is a union of two lunes, up to a null set.
\end{proof}

\begin{lemma}  
   $$
   \begin{array}{lll}
    \op{sol}(v_0,B(v_0,t)) &= 4\pi\\
    \op{vol}(v_0,B(v_0,t)) &= 4\pi t^3/3\\
    \op{sovo}(v_0,B(v_0,t),\lambda) &= 4 \pi \phi(t,t,\lambda).
   \end{array}
   $$
\end{lemma}

\begin{proof}
The ball is $t$-radial at $v_0$, so the volume is given by
Definition~\ref{def:sol} in terms of solid angle.  It is enough
to check that a hemisphere has solid angle $2\pi$.  This follows
from Lemma~\ref{lemma:wedge:sol}.
\end{proof}  




\subsection{Rogers simplex}

\begin{definition} \tlabel{def:ortho}
An {\it orthosimplex} is a tetrahedron
    $$\op{conv}^0(x,x+v_1,x+v_1+v_2,x+v_1+v_2+v_3),$$
where $v_i\cdot v_j=0$, for $1\le i<j\le 3$.   We write
$\op{orth}^0(x,v_1,v_2,v_3)$ for this orthosimplex.
 \index{orthosimplex}
\end{definition}

\begin{figure}[htb]
  \centering
  \myincludegraphics{\ps/rogers.eps}
  \caption{The Rogers simplex is an orthosimplex.}
  \tlabel{fig:rogers}
\end{figure}


\begin{definition} \tlabel{def:rog}
Let $\{v_0,v_1,v_2,v_3\}$ be a set of four points in $\ring{R}^3$.
Assume that they are not coplanar.  Let $p$ be the circumcenter
of $\{v_0,v_1,v_2\}$ and $r$ its circumradius (see Definition~\ref{def:circumrad2}).  Let $c\ge r$.
By Lemma~\ref{tarski:rog-exist}, there exists a unique
point $p'$ in $A=\op{aff}_+(\{v_0,v_1,v_2\},v_3)$ at equal distance $c$
from $v_0,v_1,v_2$.
Let $$
    \op{rog}^0(v_0,v_1,v_2,v_3,c) = 
    \op{ortho}^0(v_0,w_1,w_2,w_3),
    \quad w_1=(v_0+v_1)/2,\quad w_1+w_2=p,\quad w_1+w_2+w_3=p'.
    $$
(We define $\op{rog}(v_0,\ldots,v_3,c)$ similarly, where we use
$\op{conv}$ instead of $\op{conv}^0$.)
We take $\op{rog}^0$ to be the empty set, if $c< r$.
 \index{rogers simplex}
\end{definition}

\begin{lemma} The vectors $w_1,w_2,w_3$ of Definition~\ref{def:ortho}
are indeed mutually orthogonal.
\end{lemma}

\begin{proof} The orthogonality of $w_1$ and $w_2$ is found in
Lemma~\ref{tarski:eta-ortho}.  The orthogonality of $w_3$ with the
others is found in Lemma~\ref{tarski:rog-ortho}.
\end{proof}

\begin{definition}
Let $\eta(x,y,z)$ be the circumradius of a triangle with sides
$x,y,z$, and let $\eta_V(v_0,v_1,v_2) = \eta(|v_0-v_1|,|v_0-v_2|,|v_1-v_2|)$.
\index{circumradius}
\index{ZZeta@$\eta$}
\end{definition}

\begin{definition}\tlabel{def:abc}
We associate with $\op{rog}^0(v_0,v_1,v_2,v_3,c)$ the constants
$a=|v_1-v_0|/2$, $b=\eta_V(v_0,v_1,v_2)$, and $c$.
We call these the $abc$ parameters of $\op{rog}^0$. 
\end{definition}

The Rogers simplex is a tetrahedron.  Hence it is one of our
primitive regions.  It is eventually radial at $v_0$, hence
it has a solid angle at $v_0$.  When we mention its dihedral
angle, it is understood that it refers to 
   $$
   \dih_V(\{v_0,v_1\},\{v_2,p'\})=\dih_V(\{v_0,(v_0+v_1)/2\},\{p,p'\}),
   $$
where $p$ and $p'$ are the points 
constructed in Definition~\ref{def:rog}.

The squares of the edge lengths of the tetrahedron are
   $$
   (a^2,b^2,c^2,c^2-b^2,c^2-a^2,b^2-a^2).
   $$
Define the functions
   $$
   \begin{array}{lll}
     \op{volR}(a,b,c) &= \begin{cases}
       a\sqrt{(b^2-a^2)(c^2-b^2)}/6& 0 < a < b < c,\\
       0,&\text{otherwise}
       \end{cases}\\
     \op{solR}(a,b,c) &= \begin{cases}
      2 \arctan\left(\sqrt{\frac{(b-a) (c-b)}{(a+b)
   (b+c)}}\right).& 0 < a < b < c,\\
      0,&\text{otherwise}
     \end{cases}\\
     \op{dihR}(a,b,c) &= \begin{cases}
      \arctan\left(\sqrt{\left((c^2 - b^2)/(b^2 - a^2)\right)}\right)
      & 0 < a < b < c,\\
      0,&\text{otherwise}
     \end{cases}
     \end{array}
   $$
Specializing the formulas for dihedral angle, volume, and solid angle to this
setting we get the following expressions for volume and solid angle.
(The calculation of the
 solid angle formula is based on Euler's formula in 
Lemma~\ref{lemma:euler}.)

\begin{lemma}\label{lemma:rog:abc} 
Let $R=\op{rog}^0(v_0,v_1,v_2,v_3,c)$ and let $a$, $b$,
$c$ be the $abc$-parameters of $R$.  Let $\op{dih}(R)$ be the dihedral
angle of $R$ along the edge extending along $\op{aff}\{v_0,v_1\}$.  Then
$$
\begin{array}{lll}
\op{vol}(R) &= \op{volR}(a,b,c)\\
\op{sol}(v_0,R) &= \op{solR}(a,b,c)\\
\op{dih}(R) &= \op{dihR}(a,b,c)\\
\end{array}
$$
\end{lemma}





\begin{remark}
The volume of a unit cube aligned along the coordinate axes is $1$.  
If we want to insist on deriving all volumes from the primitive
volumes, then we can derive the volume of the cube by partitioning
it into six Rogers simplices,
each of volume $\op{vorR}(1,\sqrt2,\sqrt3) = 1/6$, for a total
of $1$, as desired.  Figure~\ref{fig:rogers} shows one of the six
Rogers simplices.
\end{remark}



\subsection{Rogers's lemma}


The following lemma is the key step in the proof of Rogers's
bound on the density of sphere packings \cite{Rog58}.

\begin{lemma} \tlabel{lemma:rogers}
Suppose that $a,b,c$ and $a',b',c'$
are real numbers that satisfy $0 <a \le b \le c$, $0 \le a'\le b'\le c'$,
$a \le a'$, $b \le b'$, $c \le c'$. Then
  $$
  \op{solR}(a',b',c')\op{volR}(a,b,c) \le \op{solR}(a,b,c)\op{volR}(a',b',c').
  $$
\end{lemma}

\begin{proof} If any of the equalities hold: $a=b$, $b=c$, $a'=b'$,
$b'=c'$, then both sides are zero.  We assume $a<b<c$ and $a'<b'<c'$.
Let $w_1=a\,e_1,w_2=\sqrt{b^2-a^2}\, e_2,w_3=\sqrt{c^2-b^2}\, e_3,$
for the standard basis $\{e_1,e_2,e_3\}$.  Each point of the orthosimplex
$S = \op{ortho}^0(0,w_1,w_2,w_3)$ has
the form
   $$s(t_1,t_2,t_3) = t_1 w_1 + t_2 w_2 + t_3 w_3$$
where $t_i>0$ and $t_1+t_2+t_3< 1$.  Similarly,
we define $w'_i$, $S'$, and $s'(t_1,t_2,t_3)$ for the primed objects.

The inequality of the lemma is equivalent to
  $$
  \frac{\sol(0,S')}{\op{vol}(S')} \le \frac{\sol(0,S)}{\op{vol}(S)}.
  $$
By the scaling properties of the measure, $\op{solR}$ and $\op{volR}$ both
scale by the same factor under linear stretching along coordinate axes.
By such a transformation, $S'$ can be transformed to $S$.  The
transformation $T$ is given by 
   $$
   T s'(t_1,t_2,t_3) = s(t_1,t_2,t_3).
   $$
Under this transformation $T$, the volumes become equal.
The desired inequality follows from
   $$
   \sol(0,T S') \le \sol(0,S).
   $$
This follows if the transformation $T$ satisfies
   $T(S'\cap B(0,r))\subset S\cap B(0,r)$.
By Lemma~\ref{tarski:rog-lemma}, we have that 
   $$|s(t_1,t_2,t_3)|\le |s'(t_1,t_2,t_3)|.$$
This means that $T$ carries each point of $S'$ to a point closer to
the origin.  In particular,
  $T(S'\cap B(0,r))\subset S\cap B(0,r)$.
\end{proof}

%% WW Repeated parts of def.
\begin{definition}  Define 
  $$
  \begin{array}{lll}
  \dtet &= \sqrt8 \arctan(\sqrt2/5)\\
  \doct &= \pi/\sqrt8 - \sqrt2 \arctan(\sqrt2/5)\\
  \delta(a,b,c)&= \op{solR}(a,b,c)/(3\op{volR}(a,b,c)),
  \end{array}
  $$
for $a<b<c$.  
\index{ZZdeltatet@$\dtet$}
\index{ZZdeltaoct@$\doct$}
\index{ZZdelta@$\delta$}
\end{definition}

\begin{lemma}\tlabel{lemma:doct-calc}
  $\delta(1,2/\sqrt{3},\sqrt2)=\doct$.
\end{lemma}

\begin{proof}  In this calculation, we do not use Euler's formula
for the solid angle (Lemma~\ref{lemma:euler}).  
Use the dihedral angle formula instead.
A calculation gives
  $$
  \doct(1,2/\sqrt{3},\sqrt2)=
  3 \sqrt{2} \left(\frac{\pi }{4}-\arctan
   \left(\frac{1}{\sqrt{2}}\right)\right).$$
To complete the proof, we need the trig identity
  $$\arctan(\sqrt2/5)  = 3\arctan(\sqrt2/2)-\pi/2.$$
Both sides are between $0$ and $\pi/2$.  Thus, we can prove this
by taking the tangent of both sides. By the addition formula
(Lemma~\ref{lemma:tan-add}),
if $x=\arctan(\sqrt2/2)$, then
   $$\tan(3 x) = \frac{\tan^3(x) - 3\tan(x)}{1-3 \tan^2(x)} = -5/\sqrt2.$$
The result follows.
\end{proof}

\begin{lemma}\tlabel{lemma:dtet-cal}
  $\delta(1,2/\sqrt3,\sqrt6/2)=\dtet$.
\end{lemma}

\begin{proof} Calculating as in Lemma~\ref{lemma:doct-calc}, and using
the same trig identity, we get
$$\begin{array}{lll}
  \delta &=
2\sqrt{2} \left(3\arctan\left(\sqrt2/2\right) - \pi/2)\right),\\
  &=2\sqrt{2}\left(\arctan\left(\sqrt2/5\right)\right),\\
  &=\dtet
\end{array}
$$
\end{proof}

\begin{lemma}\tlabel{lemma:rog-doct}
Suppose $1\le a\le  b\le c$,  $2/\sqrt{3}\le b$, and $\sqrt2\le c$.  Then
$$
\delta(a,b,c) \le \doct.
$$
\end{lemma}

\begin{proof} This follows from Lemma~\ref{lemma:rogers} and
Lemma~\ref{lemma:doct-calc}.
\end{proof}

\begin{lemma}\tlabel{lemma:rog-tet}
Let $1\le a \le b \le c$, $2/\sqrt{3}\le b$ and $\sqrt6/2\le c$.
Then $\delta(a,b,c) \le  \dtet$.
\end{lemma}

\begin{proof}  This follows from Lemma~\ref{lemma:rogers} and
Lemma~\ref{lemma:dtet-cal}.
\end{proof}

By Lemma~\ref{tarski:eta-root3}, the circumradius of a triangle
with sides at least $2$ is always at least $\eta(2,2,2)=2/\sqrt3$.



\subsection{quoin}

Define the function $\op{quovol}$ when $0<a<b<c$ by
    \begin{equation}
    \begin{array}{lll}
    6\,\op{quovol}(a,b,c) &= (a+2c)  %
    % -(a^2+ac-2c^2)
    (c-a)^2\arctan(e)
        +a(b^2-a^2)e\\&-4c^3\arctan(e(b-a)/(b+c)),
    \tlabel{eqn:3.3}
    \end{array}
    \end{equation}
where $e\ge0$ is given by $e^2(b^2-a^2)=(c^2-b^2)$.
Extend the function by $0$, when the condition $0<a<b<c$ fails.


\begin{definition}\label{def:quoin}
Let $\{v_0,v_1,v_2,v_3\}$ be a set of four points in $\ring{R}^3$.
Let $c>  \eta_V(v_0,v_1,v_2)$.  Let $p$ be the circumcenter
of $\{v_0,v_1,v_2\}$.  Let $p'$ be the point 
in $\op{aff}_+(\{v_0,v_1,v_2\},v_3)$ at equal distance $c$
from $v_0,v_1,v_2$ (given by Lemma~\ref{tarski:mk-point}).
We define
$\op{quo}(v_0,v_1,v_2,v_3,c)$ to be the following set 
(Figure~\ref{fig:quoin}):
   $$
   B(v_0,c) \cap \op{aff}_+^0(\{v_0,v_1,v_2\},v_3)
   \cap \op{aff}_-^0(\{v_0,p,p'\},v_1) \cap
   \op{aff}_-^0(\{v_1,p,p'\},v_0).
   $$
We associate with this set the $abc$-parameters, defined
by $a = |v_0-v_1|/2$, $b=|p-v_0|$, $c$ (as given).
\end{definition}

\begin{figure}[htb]
  \centering
  \myincludegraphics{\ps/quoin.eps}
  \caption{The quoin above a Rogers simplex is the part of the
  shaded solid outside
   the illustrated box.  It is bounded by the two
  shaded planes, the plane through
   the front face of the box, and a sphere
   centered at the origin passing through the opposite corner of the box.}
  \tlabel{fig:quoin}
\end{figure}



\begin{lemma}\tlabel{lemma:quo-vol}
Let $Q=\op{quo}(v_0,v_1,v_2,v_3,c)$. Let $(a,b,c)$ be the
$abc$-parameters of $Q$.  
Then $$\op{vol}(Q) = \op{quovol}(a,b,c).$$
%
 \index{quoin}
\end{lemma}

\begin{proof} We give the proof in some detail, because it illustrates
our method of calculating derived volumes from primitive volumes.  Moreover,
this is a key calculation used in several other identities.
The calculations are essentially formal.

Let $\chi_X$ be the characteristic
function of $X$.  If $P = \chi_X$, write $\bar P$ for the characteristic
function of the complement of $X$.  We consider characteristic functions
only up to a null set, and this means that we can ignore issues such
as whether we take open half-spaces or closed half-spaces and so forth.

Set
$$
\begin{array}{llll}
  A &= \chi_X,\quad X = B(v_0,c)&\text{(ball)}\\
  B &= \chi_X,\quad X = \op{aff}_+^0(\{v_0,v_1,v_2\},v_3)&\text{(front)}\\
  C &= \chi_X,\quad X = \op{aff}_-^0(\{(v_0+v_1)/2,p,p'\},v_0)
    &\text{(top)}\\
  D &= \chi_X,\quad X = \op{aff}_-^0(\{v_0,p,p'\},v_1)&\text{(diag)}\\
  E &= \chi_X,\quad X = \op{aff}_+^0(\{v_0,v_1,p'\},p)&\text{(diag)}\\
  F &= \chi_X,\quad X = \op{rcone}^0(v_0,v_1,a/c)\\
\end{array}
$$

We see from Figure~\ref{fig:quoin} that we have the following implications
$(f(x)=1)\implies (g(x)=1)$ when $f,g$ are any of the following characteristic
functions
  $$
   (f,g) = (B\bar C \bar D E,A),\quad
   (B\bar C E F,A),\quad (A B C D, E),\quad (A B C E,F).
  $$
(These implications are justified without pictures in Lemmas~\ref{tarski:BCDE},\ref{tarski:BCEF},\ref{tarski:ABCD}, and~\ref{tarski:ABCE}.)
We recognize $ABEF$ as the characteristic function $[SC]$ of a
conic cap, $B\bar C E F$ as the characteristic function $[WFR]$ of a
wedge of a frustum,  $AB\bar D E$ as the characteristic function $[ST]$
of a solid triangle, $B\bar C\bar D E$ as the characteristic function $[R]$
of a Rogers simplex.  The characteristic function $[Q]$
of the quoin is given
by $A B C D$.  We let $[X]$ be the characteristic function of
$A B C \bar D E$.

We then have formally that
$$
\begin{array}{lllll} \,[SC] &=  A B E F \\
     &= A B \bar C E F &+ A B C E F\\
     &= B \bar C E F &+ A B C D E F &+ A B C \bar D E F\\
     &= [WFR] &+ A B C D &+ A B C \bar D E\\
     &= [WFR] &+ [Q] &+ [X]\\ 
     \,[ST] &= A B \bar D E\\
     &= A B C \bar D E &+ A B \bar C \bar D E\\
     &= [X] &+ B \bar C \bar D E\\
     &= [X] &+ [R].
\end{array}
$$
Solving for $[Q]$, we get
\begin{equation}\tlabel{eqn:qr}
  [Q] = [SC] - [WFR] + [R] - [ST].
\end{equation}
Thus, the volume of a quoin is expressed in terms of primitive volumes.
Substituting the given formulas for the volumes of primitives, we obtain
the result.  (It is necessary to use the Euler formula for solid
angle.)
\end{proof}

\begin{remark}  This type of analysis can be turned into an algorithm
for computing regions described by quadratic constraints in terms
of primitive volumes \cite{quad}.  All the volumes that arise in \cite{DCG}
can be computed by this algorithm.
In particular, the proof of Lemma~\ref{lemma:quo-vol} follows from
the algorithm.
\end{remark}

\begin{remark}\tlabel{rem:RQ}  
We can rewrite Equation~\ref{eqn:qr} as
$$
  [ST]-[R] = ([SC]-[WFR]) - [Q].
$$
That is, as we can see from Figure~\ref{fig:quoin}, the region
in a ball above and outside a Rogers simplex is the same as the
region in a ball above and outside a frustum and outside the quoin.
\end{remark}

\begin{lemma}\tlabel{lemma:solquo}  The quoin
$\quo(v_0,v_1,v_2,v_3,c)$ is eventually radial at $v_0$ and
has solid angle $0$.
\end{lemma}


\begin{proof}  This is trivial, because the quoin is bounded
away from $v_0$.
\end{proof} 

Consider the function
$\op{sovo}(v_0,\quo(v_0,v_1,v_2,v_3,c),\lambda)$, expressed
as a function of the $abc$-parameters of the quoin.  By
Lemma~\ref{lemma:solquo}, the contribution from the solid angle
is zero, so that
$$
\op{sovo}(v_0,\quo(v_0,v_1,v_2,v_3,c),\lambda) = 
 \lambda_v \op{quovol}(a,b,c).
$$


\begin{lemma}\tlabel{lemma:sovo:rog}  Let $R = \op{rog}^0(v_0,
v_1,v_2,v_3,c)$.  Let $(a,b,c)$ be the $abc$-parameters of $R$.
Let $Q=\quo(v_0,v_1,v_2,v_3,c)$.  Let $\op{dih}(R)$ be the
dihedral angle of $R$ along $\{v_0,v_1\}$.
Then
  $$
  \begin{array}{lll}
  \op{sovo}(v_0,R,\lambda) = \sol(v_0,R)\phi(c,c,\lambda) +
  \op{sovo}(v_0,Q) + \op{dih}(R)\,\, A(a,c,\lambda).
  \end{array}
  $$
\end{lemma}

\begin{proof}
By the identity of Remark~\ref{rem:RQ}, we have
  $$
  \op{sovo}(v_0,R,\lambda) - \op{sovo}(v_0,Q) = 
  \op{sovo}(v_0,ST,\lambda) - \op{sovo}(v_0,SC,\lambda) + 
  \op{sovo}(v_0,WFR,\lambda),
  $$
for regions $ST$, $SC$, $WFR$ described in that remark.
We have from previous calculations that
  $$
  \begin{array}{lll}
  \op{sovo}(v_0,ST,\lambda) &= \sol(v_0,ST)\phi(c,c,\lambda) \\ &= 
  \sol(v_0,R)\phi(c,c,\lambda)\\
  \op{sovo}(v_0,WFR,\lambda) &= \op{sol}(v_0,FR)\,\frac{\dih(R)}{2\pi}
   \phi(a,c,\lambda)\\
  \op{sovo}(v_0,SC,\lambda) &= \op{sol}(v_0,FR)\,\frac{\dih(R)}{2\pi}
   \phi(c,c,\lambda)\\ 
  \op{sol}(v_0,FR) (\phi(a,c,\lambda) -\phi(c,c,\lambda)) &=
   2\pi A(a,c,\lambda).
  \end{array}
  $$
The result follows from these equations.
\end{proof}



%\subsection{caps}
%
%\begin{lemma}\tlabel{lemma:cap-rogers}
%Let $B(0,t)$ be a ball of radius $t$ centered at the origin.  Let
%$v_1$ and $v_2$ be vertices.  Assume that $|v_1|< 2t$ and $|v_2|<2
%t$.  Truncate the ball by cutting away the caps
%   $$\op{cap}_i = \{x\in B(0,t) :  |x- v_i| < |x|\}.$$
%Assume that the circumradius of the triangle $\{0,v_1,v_2\}$ is
%less than $t$. Then the intersection of the caps, $\op{cap}_1\cap
%\op{cap}_2$, is the union of four quoins.
%\end{lemma}
%
%\begin{proof} This is true by inspection.  See Figure~\ref{fig:capriquoin}.
%Slice the intersection $\op{cap}_1\cap\op{cap}_2$ into four pieces
%by two perpendicular planes: the plane through $\{0,v_1,v_2\}$,
%and the plane perpendicular to the first and passing through $0$
%and the circumcenter of $\{0,v_1,v_2\}$.  Each of the four pieces
%is a quoin.
%\end{proof}
%
%\begin{figure}[htb]
%  \centering
%  \myincludegraphics{\ps/capriquoin.eps}
%  \caption{The intersection of two caps on the unit ball can
%   be partitioned into four quoins (shaded).}
%  \tlabel{fig:capriquoin}
%\end{figure}
%

\subsection{truncating Rogers}

\begin{lemma}\tlabel{lemma:sovo:truncRog}
Let $R_c=\op{rog}^0(v_0,v_1,v_2,v_3,c)$.  Let $r$ be the
circumradius of $\{v_0,v_1,v_2\}$.  Assume that $r$, $t$,
and $c$ are positive real numbers that satisfy $r, t\le c$.
Let $d$ be the dihedral angle of $R_c$ along the edge $\{v_0,v_1\}$.
Then
   $$
   \op{sovo}(v_0,R_c\cap B(v_0,t),\lambda) = 
   d [(1-\cos\psi)\phi(t,t,\lambda)+A(h,t,\lambda)]
   -s \phi(t,t,\lambda) + \lambda_v \op{quovol}(a,b,t),
   $$
where $s = d (1-\cos\psi) - \sol(v_0,R_c)$,   $h=|v_0-v_1|/2$,
and $\cos\psi = h/c$.
\end{lemma}

\begin{proof}  Let $R_t= \op{rog}^0(v_0,v_1,v_2,v_3,t)$.   Let
$q_t$ (resp. $q_c$) be the point at equidistance $t$ 
(resp. $c$) from $v_0,v_1,v_2$ in
$\op{aff}_+(\{v_0,v_1,v_2\},v_3)$.
If $t\ge r$, then the unique existence of $q_t$ is given by
Lemma~\ref{tarski:rog-exist}.  If $t< r$, set $R_t=\emptyset$ and
$q_t=v_2$.
When $t>r$, the point $q_t$ (resp. $q_c$) is an extreme point of $R_t$ (resp. $R_c$).

Set
  $$
  \begin{array}{lll}
  W_1 &= W(v_0,v_1,v_2,q_t),\\  
  W_2 &= W(v_0,v_1,q_t,q_c).\\
  A &= \op{rcone}^0(v_0,v_1,h/t),\\
  \bar A &= \op{rcone}^0_-(v_0,v_1,h/t).\\
  \end{array}
  $$
We have the following relations by Lemmas~\ref{tarski:wedge-union},
\ref{tarski:rogers-ball}, \ref{tarski:rogers2}, \ref{tarski:rcone-ball}, \ref{tarski:rogers-FR}:
  $$
  \begin{array}{rll}
  R_c \cap (W_1\cup W_2) &\equiv R_c,\\
  W_1\cap W_2 &= \emptyset\\
  R_c \cap B \cap W_1 &= R_t \cap B = R_t\\
  R_c\cap B \cap W_2 \cap A &= FR(v_0,v_1,h,h/t)\cap W_2\\
  \end{array}
  $$
Also, by Lemma~\ref{tarski:rogers-rad},
  $$R_c\cap B\cap W_2\cap \bar A = W_2\cap B\cap \bar A \cap
   \op{aff}_+^0(\{v_0,q_t,q_c\},v_1).$$ 
This is measurable and $t$-radial
at $v_0$.
For simplicity, surpress the parameters $v_0$ and $\lambda$ from
$\op{sovo}$.  It follows from these identities and the
Lemmas~\ref{lemma:sovo:rog} 
and~\ref{lemma:sovoFR} that
  $$
  \begin{array}{lll}
  \op{sovo}(R_c\cap B) &= \op{sovo}(R_c\cap B\cap W_1) + 
  \op{sovo}(R_c\cap B\cap W_2)\\
  &= \op{sovo}(R_t)+ \op{sovo}(R_c\cap B\cap W_2\cap A) +
  \op{sovo}(R_c\cap B\cap W_2\cap \bar A).\\
  \op{sovo}(R_t) &=\sol(R_t)\phi(t,t,\lambda) + \op{sovo}(Q) +
     \dih(R_t) A(h,t,\lambda).\\
  \op{sovo}(R_c\cap B\cap W_2\cap A) &= \op{sovo}(FR(v_0,v_1,h,h/t)\cap W_2)\\
   &= \dih(R_c) A(h,t,\lambda) + \sol(W_2\cap R_c\cap B\cap A)\phi(t,t,\lambda).\\
   \op{sovo}(R_c\cap B\cap W_2\cap \bar A) &= 
        \sol(R_c\cap B\cap W_2\cap \bar A) \phi(t,t,\lambda)\\
   &= (\sol(R_c)-\sol(R_c\cap B\cap W_2\cap A) -\sol(R_t))\phi(t,t,\lambda).
  \end{array}
  $$
Combining these equations, we get the result.
\end{proof}

%\begin{lemma}\label{lemma:truncRog:noQ}
%Let $R=\op{ortho}^0(w_0,v_1,v_2,v_3)$ and $B=B(w_0,t)$.
%Assume that $|v_1|< t < |v_1+v_2|$.  Then
%$$
%  \begin{array}{lll}
%  \op{sovo}(w_0,B\cap R,\lambda) &=
%  \sol(R)\phi(t,t,\lambda) + \dih(R)(1-y/(2t))(\phi(y/2,t,\lambda)
%  -(\phi(t,t,\lambda)))\\
%  &=\sol(R)\phi(t,t,\lambda) + \dih(R) A(y/2,t,\lambda)\\
%  \end{array}
%$$
%\end{lemma}
%
%We note that this formula is that obtained from Lemma~\ref{lemma:sovo:truncRog} by setting the quoin term to zero.
%
%\begin{proof}
%Set
%  $$
%  \begin{array}{lll}
%  A &= \{x \mid (x-v_0)\cdot (v_1-v_0) > |x-v_0| |v_1-v_0| (y/2t)\},\\
%  \bar A &= \{x \mid (x-v_0)\cdot (v_1-v_0) < |x-v_0| |v_1-v_0| (y/2t)\},\\
%  W &= W(w_0,w_0+v_1,w_0+v_1+v_2,w+0+v_1+v_2+v_3\\ 
%  \end{array}
%  $$
%We have that $R\cap (A\cap \bar A)\equiv R$ and $A\cap \bar A=\emptyset$.
%Also, by Lemma~\ref{tarski},
% Put these in Tarski in the Volume section.
%  $$
%  \begin{array}{lll}
%  R\cap B\cap A &= FR(v_0,v_1,h,y/(2t))\cap W\\
%  \end{array}
%  $$
%Also, $R\cap B\cap\bar A$ is measurable and $t$-radial
%at $v_0$.
%
%We have
%  $$
%  \begin{array}{lll}
%  \op{sovo}(B\cap R) &= \op{sovo}(A\cap B\cap R) + \op{sovo}(\bar A\cap B\cap R) \\
%   \op{sovo}(\bar A\cap B\cap R) &= \sol(\bar A\cap B\cap R) \phi(t,t,\lambda)\\
%    &= (\sol(B\cap R)-\sol(A\cap B\cap R))\phi(t,t,\lambda)\\
%    &= (\sol(B\cap R)-\sol(FR\cap W))\phi(t,t,\lambda)\\
%  \op{sovo}(A\cap B\cap R) &= \op{sovo}(FR\cap W)\\
%     &=\sol(FR\cap W)\phi(y/2,t,\lambda)\\
%   \sol(FR\cap W) &= \dih(R)\sol(FR)/(2\pi)\\
%  \sol(FR) = 2\pi (1-y/(2t))\\
%  \end{array}
%  $$
%These identities combine to give the proof.
%\end{proof}
%

\section{Composite Regions}\tlabel{sec:tcc}
    %\oldlabel{4.10}

We will consider in this section several different types
of regions that are composites of pieces that have already been
considered.  The various regions share certain features.

The constructions will depend on points
 $v_0,v_1,w_1,w_2\in\ring{R}^3$. Assume that $\{v_0,v_1,w_1,w_2\}$
is not coplanar.  We also choose constants $c_1,c_2>0$.
Let
$W=W(v_0,v_1,w_1,w_2)$ be the corresponding wedge.  
Let $p_i$ be the circumcenter of
$\{v_0,v_1,w_i\}$ and $b_i$ its circumradius.
Let $u_i$ be the normal to $\{v_0,v_1,w_i\}$, directed so that
$(-1)^i\azim(v_0,v_1,w_i,w_i+u_i) < 0$.  That is, $u_i$ points
from the half-plane $\op{aff}_+^0(\{v_0,v_1\},w_i)$ into the wedge $W$.
For $c_i > b_i$, let $q_i = q(c_i) = p_i + s_i u_i$ (with $s_i>0$) be the
unique point that is equidistant $c_i$ from $v_0,v_1,w_i$.
(The unique existence of $q_i$ is established by Lemma~\showref{tarski:rog-exist}.)
If $c_i\le b_i$, we set $c_i=w_i$.
By the choice of normals $u_i$, we have that
 $$
 0\le \op{azim}(v_0,v_1,w_1,q_1), \text{ and }
 \op{azim}(v_0,v_1,w_1,q_2) \le \op{azim}(v_0,v_1,w_1,w_2).
 $$
Equality occurs exactly when $c_1\le b_1$ (resp. $c_2\le b_2$).
We make the assumption that
\begin{equation}\tlabel{eqn:q1q2}
\op{azim}(v_0,v_1,w_1,q_1) \le \op{azim}(v_0,v_1,w_1,q_2).
\end{equation}
(This assumption will be briefly lifted in Section~\showref{sec:inverted}.)

We define the wedges 
$$
   \begin{array}{lll}
   W &= W(v_0,v_1,w_1,w_2)\\
   W_1 &= W(v_0,v_1,w_1,q_1)\\
   W_2 &= W(v_0,v_1,q_2,w_2)\\
   W'' &= W(v_0,v_1,q_1,q_2)\\
   \end{array}
$$
Under Assumption~\ref{eqn:q1q2}, we have that
$W_1,W_2,W''$ are mutually disjoint, and that
   $$
   W \equiv W_1 \cup W_2 \cup W''.
   $$
Define Rogers simpices for $i=1,2$:
  $$
  \begin{array}{lll}
  R_i &= \op{ortho}^0(v_0,v_1,w_i,w_i+u_i,c_i)\\
  R_i'&= \op{ortho}^0(v_0,w_i,v_1,v_1+u_i,c_i)\\
  \end{array}
  $$
The $abc$-parameters of $R_i$ are $a=|v_1-v_0|/2$, $b_i$, $c_i$.
Those of $R_i$ are $a'_i=|w_i-v_0|/2$, $b_i$, $c_i$.
The dihedral angle of $R_i$ along the edge $\{v_0,v_1\}$ is
equal to the dihedral angle (and azimuth angle) of $W_i$.
We have $R_i,R_i'\subset W_i$.

The composites that we consider will be constructed in various
ways from $R_i,R'_i$ and regions $W''\cap X$, for various regions
$X$ that are rotationally symmetric along the line $\op{aff}\{v_0,v_1\}$.
In the following subsections, we specialize this general
context to various specific composites.

\subsection{plates}

The first composite that we consider will be called a plate.

\begin{definition}\tlabel{def:plate}
Let $v_0,v_1,w_1,w_2$ be points in $\ring{R}^3$ that are not
coplanar.  Let $t > 0$.  Choose $c_1=c_2=t$ and construct
the regions $R_1,R_2,W'',\ldots$ for the parameters 
$v_0,v_1,w_1,w_2,c_1,c_2$.
Define the plate in terms of the regions $R_1,R_2,W''$ as follows
  $$
  PL(v_0,v_1,w_1,w_2,t) = 
  R_1 \cup (W''\cap FR(v_0,v_1,h,h/t))\cup R_2,
  $$
where $h = |v_1-v_0|/2$.
\end{definition}

Under Assumption~\ref{eqn:q1q2}, the regions $R_1,R_2,(W''\cap FR)$
are disjoint and have the appearance of Figure~\ref{fig:plate}.
According to construction, if $t$ is less than the circumradius
of $\{v_0,v_1,w_i\}$, then the corresponding Rogers simplex $R_i$
is empty, the wedge $W_i$ is empty, and that piece can be dropped
from the description of the plate.

% Not yet sketched.
\begin{figure}[htb]
  \centering
  \myincludegraphics{noimage.eps}
  \caption{A plate}
  \tlabel{fig:plate}
\end{figure}


\begin{lemma}
Let $v_0,v_1,w_1,w_2$, $h,t$, be given as in the definition of
the plate. Let $q_i$ be the point constructed above. 
Assume that $h\le t$.  
Let $PL=PL(v_0,v_1,w_1,w_2,t)$.
Let $Q_i=\quo^0(v_0,v_1,w_i,q_i,t)$.
Then for all $\lambda$, we have
  $$
  \op{sovo}(v_0,PL,\lambda) = 
  \sol(v_0,PL)\phi(t,t,\lambda) + 
  \sum_{i=1}^2\op{sovo}(v_0,Q_i,\lambda) +
  \op{azim}(v_0,v_1,w_1,w_2) A(h,t,\lambda).
  $$
\end{lemma}

\begin{proof} We have already calculated the function $\op{sovo}$
on the individual pieces $R_i$ and $W''\cap FR$.  The result
follows by assembling the pieces into the composite and 
Lemmas~\ref{lemma:sovo:rog} and \ref{lemma:sovoFR}:
  $$
  \begin{array}{lll}
  \op{sovo}(PL) &= \op{sovo}(R_1) + \op{sovo}(R_2) + \op{sovo}(W''\cap FR) \\
  \op{sovo}(R_i) &= \op{sol}(R_i) \phi(t,t,\lambda) +\op{sovo}(Q_i)
   + \dih(R_i) A(h,t,\lambda)\\
  \op{sovo}(W''\cap FR) &= \op{sol}(W''\cap FR)\phi(t,t,\lambda) +
   \op{azim}(W'') A(h,t,\lambda)\\
  \sol(PL) &=\sol(R_1)+\sol(R_2)+\sol(W''\cap FR)\\
  \op{azim}(v_0,v_1,w_1,w_2) &= \dih(R_1)+\dih(R_2)+\op{azim}(W'').
  \end{array}
  $$
The result follows immediately from these equations.
\end{proof}



\subsection{corner cells}

Let $\{v_0,v_1,w_1,w_2\}$ be a set of four points in $\ring{R}^3$.  
Assume that $\{v_0,v_1,w_1,w_2\}$ is not coplanar.
We
attach a {\it corner cell} $CC(v_0,v_1,w_1,w_2,t,\mu)$
to these four points and positive
real parameters $t,\mu$.  Let $2h=|v_0-v_1|$
 %$b=\eta(2h,t,\mu)$, 
and $\psi=\arc(2h,t,\mu)$.

Construct the cone $C=\op{rcone}^0(v_0,v_1,\cos\psi)$.
Let $P$ be the half-space containing $v_0$ bounded by
the perpendicular bisector of  $\{v_0,v_1\}$.  Set
$$
  \begin{array}{lll}
  CC_1 &= C\cap P \cap B(v_0,t)\\
  W &=W(v_0,v_1,w_1,w_2) \\
  CC(v_0,v_1,w_1,w_2,t,\mu) &= CC_1 \cap W
  \end{array}
$$

\begin{lemma}\tlabel{lemma:sovo:CC} 
Suppose that $h/t \ge \cos\psi$.  Let the other notation
be as above.   We have
  $$
  \op{sovo}(v_0,CC(v_0,v_1,w_1,w_2,t,\mu),\lambda)=
  \op{azim}(v_0,v_1,w_1,w_2) \left((1-\cos\psi)\phi(t,t,\lambda)+
    A(h,t,\lambda)\right).
  $$
\end{lemma}

\begin{proof}
Set
$$
 \begin{array}{lll}
 A &= \op{rcone}^0(v_0,v_1,h/t),\\
  \bar A &= \op{rcone}^0_-(v_0,v_1,h/t),\\
 W &= W(v_0,v_1,w_1,w_2)\\
 \end{array}
$$
It follows from the definitions and from the
fact that $\partial\op{rcone}(v_0,v_1,h/t)$
is a null set that
we have $A\cap \bar A = \emptyset$ and 
$A\cup \bar A \equiv \ring{R}^3$.
By Lemma~\ref{tarski:CC} and Lemma~\ref{tarski:CCbar}, we have 
$$
  \begin{array}{lll}
    A\cap CC &= W \cap FR(v_0,v_1,h,h/t)\\
    \bar A \cap CC &= \bar A \cap C \cap B(v_0,t).
  \end{array}
$$
This last set is
is $t$-radial and measurable.
It follows from Lemma~\ref{lemma:sovoFR} that
$$
  \begin{array}{lll}
  \op{sovo}(CC) &= \op{sovo}(A\cap CC) + \op{sovo}(\bar A\cap CC)\\
  \op{sovo}(A\cap CC) &= \op{sovo}(W\cap FR) \\
     &= \op{azim}(W) A(h,t,\lambda) + \sol(W\cap FR(v_0,v_1,h,h/t))\phi(t,t,\lambda)\\
  &=\op{azim}(W) A(h,t,\lambda) + (\sol(CC)-\sol(\bar A \cap CC))\phi(t,t,\lambda)\\
  \op{sovo}(\bar A \cap CC) &= \sol(\bar A \cap CC)\phi(t,t,\lambda)\\
  \sol(CC) &= \op{azim}(W) \sol(SC(v_0,v_1,t,\cos\psi))/(2\pi) \\
         &=\op{azim}(W) (1-\cos\psi)\\
  \end{array}
$$
The result follows immediately from these equations.
\end{proof}

\subsection{truncated corner cells}
%\subsection{Formulas for Truncated corner cells}
\tlabel{sec:ftcc}
    %\oldlabel{4.11}



%Starting from the corner cell $CC(v_0,v_1,w_1,w_2,t,\mu)$, 
We define a subset $TCC(v_0,v_1,w_1,w_2,t,\mu)$ of the corner cell
called
the {\it truncated corner cell}.  
We use the construction of $b_i$, $u_i$, $p_i$, $q_i$, $R_i$,
$W_1$, $W_2$, $W''$, $\ldots$ from Section~\ref{sec:tcc},
associated with the parameters $v_0,v_1,w_1,w_2$ and
$c_1=c_2 = h/\cos\psi$, where $h = |v_1-v_0|/2$.  
Assumption~\ref{eqn:q1q2} remains
in force.  Let $B  = B(v_0,t)$.

\begin{definition} We define the truncated corner cell
to be
$$
TCC(v_0,v_1,w_1,w_2,t,\mu) =
(R_1\cap B)\cup (R_2\cap B) \cup (W''\cap CC(v_0,v_1,w_1,w_2,t,\mu)).
$$
\end{definition}

%Let $y=y_1=2h$.
The points $p_i$ and
 $q_i$ are construction in Section~\ref{sec:tcc}.
We also need the Rogers simplices 
$R''_i= \op{rog}^0(v_0,v_1,w_i,q_i,t)$ based on the parameter
$t$, rather than $c_i$.  The Rogers simplex $R''_i$
has $abc$-parameters
$(h,b_i,t)$.  

Let $s(y_1,y_2,y_3,t,\lambda)$ be given by the following formula.
Let $b = \eta(y_1,y_2,y_3)$, $\psi = \arc(y_1,t,\lambda)$,
and $c = y_1/(2\cos\psi)$ in
  $$
  s(y_1,y_2,y_3) = \op{dihR}(h,b,c) (1-\cos\psi) - \op{solR}(h,b,c).
  $$


\begin{lemma}\label{lemma:tcc}  
Let $TCC=TCC(v_0,v_1,w_1,w_2,t,\mu)$ be as constructed.
Let $p_i$ be the circumcenter constructed in Section~\ref{sec:tcc}.
%Let $q_i(t)$  X
As usual, assume Condition~\ref{eqn:q1q2}.  
Let $y = 2h = |v_1-v_0|$.  
Let $s_i = s(y,|w_i-v_0|,|v_1-w_i|,t,\lambda)$, for $i=1,2$.
Assume that
$h/t \ge \cos\psi$.  
Let $Q_i = \op{quo}^0(v_0,v_1,w_i,q_i,t)$.  Then
  \begin{equation}\label{eqn:tcc}
  \begin{array}{lll}
  \op{sovo}(v_0,TCC,\lambda) &= 
  \op{azim}(v_0,v_1,w_1,w_2) \left((1-\cos\psi)\phi(t,t,\lambda)+
    A(y/2,t,\lambda)\right) \\
    &\quad + \op{sovo}(v_0,Q_1,\lambda) - s_1\phi(t,t,\lambda) \\
    &\quad + \op{sovo}(v_0,Q_2,\lambda) - s_2\phi(t,t,\lambda) \\
  \end{array}
  \end{equation}
\end{lemma}

\begin{proof}  The result follows by combining the function
$\op{sovo}$ on each of the pieces of the composite $TCC$.
By Lemmas~\ref{lemma:sovo:CC} and~\ref{lemma:sovo:truncRog},  
% noQ
we have
$$
\begin{array}{lll}
  \op{sovo}(TCC) &= \op{sovo}(W''\cap TCC) + \op{sovo}(W_1\cap TCC)
  + \op{sovo}(W_2\cap TCC)\\
  &=\op{sovo}(W''\cap TCC) + \op{sovo}(R_1\cap B)
  + \op{sovo}(R_2\cap B)\\
  &= \op{azim}(W'') \left((1-\cos\psi)\phi(t,t,\lambda)+
    A(h,t,\lambda)\right) 
   + \op{sovo}(R_1\cap B)
  + \op{sovo}(R_2\cap B)\\
  \op{azim}(W'') &= \op{azim}(v_0,v_1,w_1,w_2)-\dih(R_1)-\dih(R_2)\\
  \op{sovo}(R_i\cap B) &= \dih(R_i) [(1-\cos\psi)\phi(t,t,\lambda)+A(h,t,\lambda)]\\
   &\quad -(\dih(R_i) (1-\cos\psi) - \sol(R_i)) \phi(t,t,\lambda) + 
   \op{sovo}(v_0,Q_i,\lambda).
\end{array}
$$
The result follows from these equations and the definition of
the function $s$.
\end{proof}



We give a second expession for the truncated corner cell.
Let $A_i^\pm$ be the half-spaces
$$
  A_i^\pm = \op{aff}_\pm^0(\{v_0,p_i,q_i\},v_0).
$$
Let $L_i^\pm = CC(v_0,v_1,w_1,w_2,t,\lambda)\cap A_i^\pm$.

\begin{lemma}\label{lemma:LL}
We have
  $$
  TCC(v_0,v_1,w_1,w_2,t,\lambda) = L_1^+\cap L_2^+.
  $$
Moreover, 
$L_1^-\cap L_2^- =\emptyset$.
\end{lemma}

\begin{proof}
The second statement follows from the containment
   $L_i^- \subset W_i$, because, as we have seen,
$W_1$
from $W_2$.
Consider the first statement.  
We have
 $$
 W'' \cap TCC = W'' \cap CC = W'' \cap L_1^+ \cap L_1^-.
 $$
The first equality holds by construction of the truncated corner
cell.  The second equality holds because the parts $L_i^-$ excised
from $CC$ to form $TCC$ are contained in $W_i$.
We also have
  $$
  W_i \cap TCC = R_i \cap B(v_0,t) = W_i  \cap L_i^+.
  $$
%% XX IS THIS A NEW TARSKI??
The result follows.
\end{proof}




\subsection{inverted truncated corner cells}\tlabel{sec:inverted}
%\subsection{Analytic continuation} %DCG 13.3, p144
\oldlabel{5.3}

In this section, we develop a formula for $\op{sovo}$ on 
a truncated corner cell that remains valid if the 
Assumption~\ref{eqn:q1q2} is not in force.
When Assumption~\ref{eqn:q1q2} does not necessarily hold, we
define $TCC(v_0,v_1,w_1,w_2,t,\lambda) = L_1^+ \cap L_2^+$.
By Lemma~\ref{lemma:LL}, 
this is a compatible extension of the definition
of truncated corner cells, up to a null set.
The proof that $L_1^+\cap L_2^- = \emptyset$ 
relies on Assumption~\ref{eqn:q1q2}.


\begin{lemma} Let $\op{sovo}^g(v_0,TCC,\lambda)$ be given
by the right-hand side of Equation~\ref{eqn:tcc}.
If $\phi(t,t,\lambda) > 0$, then
   $$
   \op{sovo}(v_0,TCC,\lambda) > \op{sovo}^g(v_0,TCC,\lambda).
   $$
Furthermore, the sign is reversed if $\phi(t,t,\lambda) < 0$.
\end{lemma}

\begin{proof}
We have by inclusion-exclusion, the formula
$$
\begin{array}{lll}
\op{sovo}(v_0,TCC,\lambda) &=
\op{sovo}(v_0,CC,\lambda) -
\op{sovo}(v_0, L_1^-,\lambda) -
\op{sovo}(v_0, L_2^-,\lambda) \\
 &\quad +
\op{sovo}(v_0, L_1^- \cap L_2^-,\lambda).
\end{array}
$$

If we compare this formula with the calculations for 
$\op{sovo}(v_0,TCC,\lambda)$ in Lemma~\ref{lemma:tcc}, 
we find that, in the notation of that lemma:
$$
\op{sovo}(v_0,CC\cap L_i^-) = \op{sovo}(v_0,Q_i,\lambda) - s_i.
$$

Thus, the formula for $\op{sovo}(TCC)$ in the general case,
differs from the formula under Assumption~\ref{eqn:q1q2} through
the term $\op{sovo}(v_0,L_1^-\cap L_2^-,\lambda)$.

We note that $L_1^-\cap L_2^-$ is $t$-radial 
(Lemma~\ref{tarski:rCCinvert-rad}).  Thus,
$$
\op{sovo}(v_0,L_1^-\cap L_2^-,\lambda) =
\sol(v_0, L_1^-\cap L_2^-)\phi(t,t,\lambda)
$$
and
$$
\op{sovo}(v_0,TCC,\lambda) = \op{sovo}^g(v_0,TCC,\lambda) +
   \sol(v_0,L_1^-\cap L_2^-) \phi(t,t,\lambda).
$$
In particular, the direction of the inequality between
$\op{sovo}$ and $\op{sovo}^g$ is determined by the sign
of $\phi(t,t,\lambda)$.
\end{proof}



\subsection{overlapping truncated corner cells}
%\subsection{More on Truncated Corner Cells}





\begin{lemma}\tlabel{lemma:2tcc} 
For $i=1,2$, let $$CC_i =CC(v_0,v_i,w_i,u_i,t,\mu)$$ and 
be untruncated corner cells, both
with azimuth angle at least $\pi$ and parameters $t=1.255$ and 
$\mu=1.945$.
Assume $|v_1-v_2|\ge 3.2$.  Let $y_i =|v_i-v_0|$ and
Suppose that
$2\le y_i\le 2t$.  Then
$\op{sovo}(v_0,CC_1\cup CC_2,\lambda_{sq}) > 0.8862$ 
\end{lemma}

%% WW Recheck proof. Constant was squander + pimax

\begin{proof}
%Suppose first that $CC_1$ and $CC_2$ are disjoint.
%Following the proof of Lemma~\ref{lemma:CC815}, 
%but with parameter $\mu=1.945$, a lower bound on
%$\op{sovo}$ is obtained when  $y=2t$,
%$\op{azim}=\pi$. The explicit formulas give 
%  $$\op{sovo}(v_0,C,\lambda) > 0.734$$
%for $C=CC,CC'$.
%The result follows in this case.
%
%Suppose that $CC$ meets $CC'$. 
We have
 $$
 \op{sovo}(CC_1\cup CC_2)=
 \op{sovo}(CC_1)+\op{sovo}(CC_2)-\op{sovo}(CC_1\cap CC_2).
 $$
It follows from Lemma~\ref{tarski:2CCrad} that
$CC_1\cap CC_2$ is $t$-radial at $v_0$.  Thus,
$$\op{sovo}(v_0,CC_1\cap CC_2,\lambda) =
  \sol(v_0,CC_1\cap CC_2)\phi(t,t,\lambda).$$

Let $q$ and $q'$ be the two points defined by distances
$t$ from $v_0$, $\mu$ from $v_1$, and $\mu$ from $v_2$.
The existence of such points is given by Lemma~\ref{tarski:mk-point}.
Let $A=\op{aff}_+(v_0,\{q,v_1,v_2\})$ and
$A'=\op{aff}_+(v_0,\{q',v_1,v_2\})$.
Let $rc_i = \op{rcone}^0(v_0,v_i,y_i/(2\cos\psi_i))$.
By Lemma~\ref{tarski:AA'}, we have the relations
$$
\begin{array}{lll}
CC_1\cap CC_2 &\equiv (A\cap CC_1\cap CC_2) \cup (A'\cap CC_1\cap CC_2).\\
\sol(v_0,A\cap CC_1\cap CC_2) &= \sol(v_0,A\cap CC_1) + \sol(v_0,A\cap CC_2)
  -\sol(v_0,A)\\
    &\le \sol(v_0,A\cap rc) + \sol(v_0,A\cap rc') - \sol(v_0,A)\\
    &= \sol(v_0, A \cap rc \cap rc').
\end{array}
$$
The constant $\phi(t,t,\lambda)$ is positive.  Thus, we
get a lower bound on $\op{sovo}(CC_1\cup CC_2)$ by taking 
the intersection $rc \cap rc'$ to be as large as possible.
By Lemma~\ref{tarski:rcone2}, 
this happens when $v_1$ is as close to $v_2$ as possible:
$|v_1-v_2|=\ell=3.2$.  Assume this.

The bound is now easily estimated in terms of primitive
regions.  Adding the similar
term for $A'$, we get
a function $f(y_1,y_2)$ that gives a lower bound on 
$\op{sovo}(CC_1\cup CC_2)$.  Recall that $\lambda=\lambda_{sq}$.
    $$
    \begin{array}{lll}
    \alpha_1 &= \dih(y_1,t,y_2,\mu,\ell,\mu),\\
    \alpha_2 &= \dih(y_2,t,y_1,\mu,\ell,\mu),\\
    \sol &= \sol(y_2,t,y_1,\mu,\ell,\mu),\\
    \phi_i &= \phi(y_i/2,t,\lambda),\quad i=1,2,\\
    f(y_1,y_2)&=
    2\phi(t,t,\lambda_{sq})\sol+
    2\sum_1^2 \alpha_i(1-y_i/(2t))(\phi(t,t,\lambda)-\phi_i)\\
        &\quad +
       \sum_1^2 \op{sovo}(v_0,CC_1(y_i,\pi-2\alpha_i,t,\mu),\lambda).
    \end{array}
    $$
Here $CC(y,\beta,t,\mu)$ is any corner cell
$$CC(v_0,v_1,w_1,w_2,t,\mu)$$ with $|v_1-v_0|=y$ and
$\op{azim}(v_0,v_1,w_1,w_2)=\beta$.
An interval calculation\footnote{\calc{984628285}} %A14
gives $f(y_1,y_2)>0.8862$, for $y_1,y_2\in[2,2t]$.
\end{proof}




%\section{Scores of Simplices and Cones}


%\begin{remark}\tlabel{remark:vor}\index{vor}\index{c-vor}\index{score}
% Deleted function {c-vor} that has been replaced by sovo

%\label{eqn:3.2} deleted. It should be replaced by sovo(FR) formula ref.

%\section{The Function K}
%\tlabel{sec:K} %DCG p105-106.
%% No longer used.  Proof of -1.04 lemma was rewritten.
%%
%
%We define a function $K(S)$ on
%certain simplices $S$ with circumradius at least $\sqr2$. Let
%$S=S(y_1,y_2,\ldots,y_6)$.  Let $R(a,b,c)$ denote a Rogers
%simplex. Set
%    \begin{equation}
%    K(S) = K_0(y_1,y_2,y_6)+K_0(y_1,y_3,y_5)
%    + \dih(S)(1-y_1/\sqr8) \phi(y_1/2,\sqr2),
%    \tlabel{eqn:KS}
%    \end{equation}
%where
%    $$
%    $$
%(If the given Rogers simplices do not exist because the condition
%$0<a<b<c$ is violated, we set the corresponding terms in these
%expressions to 0.) The function $K(S)$ represents the part of the
%score coming from the four Rogers simplices along two of the faces
%of $S$, and the conic region extending out to $\sqr2$ between the
%two Rogers simplices along the edge $y_1$ (Figure~\ref{fig:KS}).
%This region is closely related to the regions $\BigD(v,W)$ of
%Definition~\ref{def:delta-e}, with the difference that the regions
%$\BigD$ lie in a ball of radius $\eta_0(|v|/2)$, but the regions
%here are truncated at $\sqrt2$.
%
%\begin{figure}[htb]
%  \centering
%  \myincludegraphics{\ps/diag43.ps}
%  \caption{The set measured by the function $K(S)$.}
%  \tlabel{fig:KS}
%\end{figure}
%
%

\subsection{crowns}
%\section{The Function anc}
\tlabel{sec:anc} %DCG p 107.

This subsection considers
one final composite region.
Let $\eta(x,y,z)$ be the circumradius of a triangle with
sides $x,y,z$ and let $\eta_0(h,t) = \eta(2,2h,2t)$.

We return to the context established at the beginning
of Section~\ref{sec:tcc}.  We use the parameters $v_0,v_1,w_1,w_2$
and $c=c_1=c_2=\eta_0(h,t)$, where $h =|v_0-v_1|/2 \le t$. 
Assumption~\ref{eqn:q1q2} remains in force.
Let $W_1,W_2,W'',W,R_i,R_i',p_i,q_i$ be as given at the
beginning of Section~\ref{sec:tcc}.
%
Let 
  $$\bar B = \bar B(v_0,t) = \{x \mid |x-v_0| > t\},$$
the complement of a closed ball of radius $t$ at $v_0$.
%
We define the fitted crown to be
$$
FCR(v_0,v_1,w_1,w_2,t) =
  \left((W'' \cap FR(v_0,v_1,h,h/c)) \cup
  R_1 \cup R_2  \cup
  R_1' \cup R_2'\right) \cap \bar B.
$$
As described at the beginning of the section, the 
sets $R_i\cap \bar B$ and $R_i'\cap \bar B$ are empty
(and also $W_i=\emptyset$) when the circumradius of 
$\{v_0,v_1,w_i\}$ is greater than $c=c_i$.

We define some functions that will be used in a formula
for the value of $\op{sovo}$ on a fitted crown.
Define
\begin{equation}\cro(h,t,\lambda) =
2\pi(1-h/\eta_0(h,t))(\phi(h,\eta_0(h,t),\lambda)-\phi(t,t,\lambda)). 
\end{equation} 

\begin{lemma}\label{lemma:sovo:CR} 
Let $t$ and $h$ be real numbers satisfying 
$0 < t \le h$.
Let $b=h/\eta_0(h,t)$.
Let $CR=FR(v_0,v_1,h,b) \setminus B(v_0,t)$.
  Then
$$\op{sovo}(v_0,CR,\lambda) = \cro(h,t,\lambda).$$
\end{lemma}

\begin{proof}  Let $RC=\op{rcone}^0(v_0,v_1,h,b)$.
Then by Lemma~\ref{tarski:RCFR}, we have
$B' = B(v_0,t)\cap RC = FR\cap B(v_0,t)$.
Furthermore, $B'$ is $t$-radial with solid angle equal to that
of $FR$.  Thus, by Lemma~\ref{lemma:sovoFR},
$$
\begin{array}{lll}
\op{sovo}(CR) &= \op{sovo}(FR) - \op{sovo}(B')\\
 &= \op{sol}(FR) (\phi(h,b,\lambda) - \phi(t,t,\lambda))\\
 &= \cro(h,t,\lambda).
\end{array}
$$
\end{proof}

Similarly, if $WCR = W(v_0,v_1,w_1,w_2) \cap CR$, then
$$
\op{sovo}(v_0,WCR,\lambda) = \op{azim}(v_0,v_1,w_1,w_2)\cro(h,t,\lambda)/(2\pi),
$$
because $CR$ 
is rotationally symmetric about the axis $\op{aff}\{v_0,v_1\}$.


%\begin{figure}[htb]
%  \centering
%  \myincludegraphics{\ps/diag44.ps}
%  \caption{An illustration of the terms $\anc$.}
%  \tlabel{fig:anchor}
%\end{figure}

Let $\op{dihR}(a,b,c)$ be the dihedral angle along the edge
$\{v_0,v_1\}$ of a
Rogers simplex $\op{rog}^0(v_0,v_1,v_2,v_3,c)$ with $abc$-parameters
$(a,b,c)$.  Similarly, let $\op{solR}(a,b,c)$ (resp. $\op{sovoR}(a,b,c,\lambda)$)
be the solid angle (resp. value of $\op{sovo}$)
at $v_0$ of such a Rogers simplex.  By Lemma~\ref{lemma:rog:abc},
these values depend only on the $abc$-parameters.
Set
    \begin{equation}
    \begin{array}{lll}
    \anc(y_1,y_2,y_6,t,\lambda) &= 
     -\op{dihR}_1\cro(y_1/2,t,\lambda)/(2\pi)
       %
    -\op{dihR}_2\, A(y_2/2,t,\lambda) \\
        %(1-y_2/(2t))(\phi(y_2/2,t,\lambda)-\phi(t,t,\lambda))
      &+\sum_{i=1}^2 (\op{sovoR}_i - \op{solR}_i \phi(t,t,\lambda))\\
       %-\op{solR}_1\phi(t,t,\lambda)+\op{sovoR}_1\\
       % -\op{solR}_2\phi(t,t,\lambda) + \op{sovoR}_2,
    \tlabel{eqn:4.5}
    \end{array}
    \end{equation}\index{anc@$\anc$}
where $\op{dihR}_i$, $\op{solR}_i$, $\op{sovoR}_i$ are the values
of $\op{dihR}$, $\op{solR}$, and $\op{sovoR}(\cdot,\lambda)$
at $(a_i,b,c) = (y_i/2,\eta(y_1,y_2,y_6),\eta_0(y_1/2,t))$, for $i=1,2$.
Recall that the terms are defined as zero if the inequalities
$0 < a_i \le b\le c$ are violated.  Hence, the function $\anc$ is
zero, except at points in a certain domain.

%% Moved from DCG 11.2 Contexts.
Set
    $$\kappa(y_1,y_2,y_3,y_5,y_6,\alpha,t,\lambda) =
   \alpha\,\cro(y_1/2,t,\lambda)/(2\pi) +
        \anc(y_1,y_2,y_6,t,\lambda)+\anc(y_1,y_3,y_5,t,\lambda).
    $$
    \index{zzkappa@$\kappa$}
We are finally ready to state the main result about fitted crowns.
Assumption~\ref{eqn:q1q2} remains in force.

\begin{lemma}\label{lemma:sovo:FCR}
Let $\{v_0,v_1,w_1,w_2\}$ be a set of four points in $\ring{R}^3$.
Assume the set is not planar.
Let $0 < t < h$, where $h = |v_1-v_0|/2$.
Set $\alpha = \op{azim}(v_0,v_1,w_1,w_2)$.
Let 
 $$(y_1,y_2,z_2,z_1) =
   (|w_1-v_0|,|w_2-v_0|,|w_2-v_1|,|w_1-v_1|).
 $$
Then
$$
\op{sovo}(v_0,FCR(v_0,v_1,w_1,w_2,t),\lambda) =
 \kappa(2h,y_1,y_2,z_2,z_1,\alpha,t,\lambda).
$$
\end{lemma}

\begin{proof}
Let $B = B(v_0,t)$.  We have
$$R\equiv (\bar B\cap R) \cup (B\cap R),\quad B\cap \bar B = \emptyset,
$$
for $R=R_i,R'_i$.  Moreover, $B\cap R_i$ is $t$-radial at $v_0$.
Thus, 
 $$
\begin{array}{lll}
 \op{sovo}(\bar B\cap R_i) &= \op{sovo}(R_i) - \op{sovo}(B\cap R_i)\\
 &= \op{sovo}(R_i) - \sol(R_i)\phi(t,t,\lambda) \\
 &= \op{anc}(2h,y_i,z_i) \\
    &\quad + \dih(R_i)\cro(h,t,\lambda)/(2\pi) + \dih(R'_i) A(y_i/2,t,\lambda)\\
  &\quad -(\op{sovo}(R_i') -\sol(R_i')\phi(t,t,\lambda)). \\
\end{array}
 $$
We have by Lemma~\ref{lemma:sovo:truncRog}, 
$$
\begin{array}{lll}
\op{sovo}(\bar B\cap R_i') &= \op{sovo}(R'_i) - \op{sovo}(B\cap R'_i)\\
\op{sovo}(B\cap R'_i) &= \sol(R'_i)\phi(t,t,\lambda) + \dih(R'_i) A(y_i/2,t,\lambda)\\
\end{array}
$$
We have by Lemma~\ref{lemma:sovo:CR} that
$$
\begin{array}{lll}
\op{sovo}(FCR) &= \op{sovo}(W''\cap FCR) + \op{sovo}(W_1\cap FCR)
 +\op{sovo}(W_2\cap FCR).\\
\op{sovo}(W_i\cap FCR) &= \op{sovo}(\bar B\cap R_i) + \op{sovo}(\bar B\cap R'_i).\\
 \op{sovo}(W''\cap FCR) &= \op{azim}(W'')\cro(h,t,\lambda)/(2\pi),\\
\op{azim}(v_0,v_1,w_1,w_2)&= \op{azim}(W'')+\dih(R_1)+\dih(R_2).\\
\end{array}
$$
These equations give the lemma.
\end{proof}

\begin{lemma}  The solid angle of $FCR(v_0,v_1,w_1,w_2,t,\lambda)$
is zero at $v_0$.
\end{lemma}

\begin{proof}  The region $FCR$ is contained in the complement
of the ball of radius $t>0$ at $v_0$.  It is bounded away from
zero.  Hence, its solid angle is zero.
\end{proof}

It follows that $\op{sovo}(v_0,FCR,\lambda) = \lambda_v\op{vol}(FCR)$,
where $\lambda=(\lambda_v,\lambda_s)$.



\section{Finiteness and Volume}

We have now developed all of the volume calculations that will
be needed in this book.   We finish this chapter with some 
elementary estimates based on the volumes of  cubes and balls.

\begin{lemma}\tlabel{lemma:Zcount}
    For all $p\in\ring{R}^3$ and all $r\ge 0$, the set
    $\ring{Z}^3\cap B(p,r)$ is finite of cardinality at most
    $4\pi (r+\sqrt3)^3/3$.
\end{lemma}

\begin{proof}  If $v\in\ring{Z}^3\cap B(p,r)$, then the ith
coordinate $v_i$ of $v$ must lie in the finite range
    $$
    p_i - r \le v_i \le p_i + r.
    $$
Hence there are only finitely many possibilities for $v$.


Place an open unit cube at each point of $\ring{Z}^3\cap B(p,r)$.
The cubes are measurable, disjoint, and contained in
$B(p,r+\sqrt3)$.  Thus, the combined volume of the cubes, which is
$|\ring{Z}^3\cap B(p,r)|$,  is no greater than the volume of the
containing ball.  The result follows.
\end{proof}

\begin{lemma}\tlabel{lemma:Zlow-count}
  For all $p\in\ring{R}^3$ and all $r\ge\sqrt3$, the set
    $\ring{Z}^3\cap B(p,r)$ is finite of cardinality at least
    $4\pi (r-\sqrt3)^3/3$.
\end{lemma}

\begin{proof} We have already established finiteness in
Lemma~\ref{lemma:Zcount}.  Place a closed unit cube at each point
of $\ring{Z}^3\cap B(p,r)$.  The cubes are measurable and cover
$B(p,r-\sqrt3)$.  Thus, the combined volume of the cubes is at
least the volume of the covered ball.  The result follows.
\end{proof}

\begin{lemma}\tlabel{lemma:Zr2}
For all $p\in\ring{R}^3$, and $k,k'>0$, there exists a $C$ such
that for all $r\ge k'$, we have
    $$
    \ring{Z}^3 \cap (B(p,r+k) \setminus B(p,r-k')) \le C r^2.
    $$
\end{lemma}

\begin{proof}  When $r \ge k'+\sqrt3$, the previous two lemmas show
that the cardinality is at most $4\pi/3$ times
    $$(r + +k + \sqrt3)^3 - (r - k' - \sqrt3)^3 \le C' r^2$$
for some $C'$.  Similarly, if $k'\le r\le k'+\sqrt3$, the
cardinality is at most some fixed constant $C''$.  The result
easily follows.
\end{proof}


    %\lll
    %\chapter{Hypermap}\label{chap:hypermap}
\indy{Index}{hypermap}%

\section{Background on Permutations}

\begin{definition}[permutation]\guid{IFPQAWD}
A \newterm{permutation} $f$ on a set
  $D$ is a bijection $f:D\to D$.
\end{definition}

For example, the identity map $I_D$ on a set $D$,
\begin{displaymath}
I_D(x)=x \text{ for all } x \in D,
\end{displaymath}
 is a permutation.
If $f:D\to D$ is a permutation then there is an inverse function $f^{-1}:D\to D$
that is also a permuation.  
It satisfies
\begin{displaymath}
f f^{-1} = f^{-1} f = I_D.
\end{displaymath}
(This chapter uses product notation $f g$ for the composition of maps
$f\circ g$.)
If $D$ is a finite set, and two maps
$f,g:D\to D$ satisfy $f g = I_D$ on $D$, then $f$ and $g$ are permutations and are
inverses of one another:
\begin{displaymath}
f g = g f = I_D.
\end{displaymath}

Natural number powers  $f^k$ of a permutation $f:D\to D$ are defined
recursively by
\begin{displaymath}
f^0 = I_D,\quad\text{ and } f^{k+1} = f f^k.
\end{displaymath}
Integer powers $f^m$ of a permutation are defined as
$$f^m = f^i (f^{-1})^j,$$ where $m = i -j$.  This is well-defined.
The usual rule of exponents holds:
\begin{displaymath}
f^{a+b} = f^a f^b.
\end{displaymath}

If $f:D\to D$ is a permutation on a finite set $D$, then there is a smallest
positive integer $k$ such that $f^k=I_D$.  The integer $k$ is the \newterm{order}
of the permutation $f$.  If $f^m=I_D$ for any some $m$, then $m = k i$ for some
integer $i$, where $k$ is the order of $f$. The inverse $f^{-1} = f^k f^{-1} = f^{k-1}$ can be written as a
non-negative power of $f$.

A permutation $f$ of a finite set $D$ is \newterm{cyclic}, if the order of $f$ is the cardinality
of $D$.  A permutation $f$ is cyclic if and only if for every $x,y\in D$, there exists an integer $i$
such that $f^i x = y$.

The set of all permutations of the set $\{0,1,2,\ldots,k-1\}$ is written $\op{Sym}(k)$.
The set $\op{Sym}(k)$ is finite and has cardinality $k!$.



\section{Definitions}



\begin{definition}[hypermap,~dart]\guid{ZIHYYRA}\label{def:hypermap}  
  A hypermap is a finite set $D$, together with three functions
  $e,n,f:D\to D$ whose composition is the identity:
  \begin{displaymath}
e\ocirc n\ocirc f = I_D.
\end{displaymath} The
elements of $D$ are called \newterm{darts}.  The functions $e,n$ and
$f$ are called the \newterm{edge map}, the \newterm{node map}, and
the \newterm{face map}, respectively.  \indy{Index}{hypermap}%
\indy{Index}{dart}%
\indy{Index}{edge!map}%
\indy{Index}{node!map}%
\indy{Index}{face!map}%
\indy{Notation}{edgemapz@$e$ (edge map)}%
\indy{Notation}{nodemap@$n$ (node map)}%
\indy{Notation}{facemap@$f$ (face map)}%
\indy{Notation}{D@$D$ (dart)}%
\end{definition}

%\pdf{dart.pdf}{dart}{The arrowhead represents a dart.}
\begin{figure}[htb]
\centering
\szincludegraphics[width=2mm]{\pdfp/dart.eps}
\caption{This symbol represents a dart.}
\label{fig:dart}
\end{figure}

\begin{remark}[planar graphs as hypermaps]\guid{IVPJYAG}\tlabel{rem:hypermap} A hypermap is an abstraction of
the concept of 
planar graph.  Place a dart at each angle of a planar graph.
One function, $f$, 
cycles counterclockwise around the angles of each face.  
Another function, $n$, 
rotates counterclockwise around the angles at each
node.  A third function, $e$, pairs angles at opposite ends of
each edge  (Figure~\ref{fig:hypermap_ex}).   The hypermap extracts
the data $(D,e,n,f)$ from the planar graph and discards the rest.
\indy{Index}{planar graph}%
\end{remark}

\begin{figure}[htb]
\centering
\szincludegraphics[width=80mm]{\pdfp/hypermap-ex.eps}
\caption{Darts mark the angles of a planar graph.  Darts may
be permuted about faces, nodes, and edges.}
\label{fig:hypermap_ex}
\end{figure}

By the background on permutations, $e,n,f$ are all permutations on $D$.
A hypermap satisfies 
\begin{equation}\tlabel{eqn:triality}
e \ocirc n\ocirc f = n\ocirc f\ocirc e = f\ocirc e\ocirc n = I_D.
\end{equation}
Inverted, this triality becomes
\begin{displaymath}
n^{-1} \ocirc e^{-1} \ocirc f^{-1} = (f \ocirc e \ocirc n)^{-1} = I_D.
\end{displaymath}
This inversion is the abstract form of the the duality between nodes
and faces in a planar graph.  Because of these symmetries in the
defining relation, there will be multiple versions of theorems about
hypermaps, all obtained from one proof by symmetry.


\begin{definition}[path,~list,~sublist,~visit,~dart~set]\guid{RRQWGAY} 
Let $D$ be a set (of darts), and let $S$ be a set of permutations of $D$.
A \newterm{path} with \newterm{steps} in $S$
from $x_0$ to $x_{k-1}$ is a \newterm{list}\footnote{The empty path $[]$ seems
to have an ancient origin: ``This is the path made known to me
when I had learned to remove all darts.'' --The Dhammapada} of
darts $[x_0;x_1;\ldots;x_{k-1}]$ such that for each $i$, $x_{i+1} = h_i x_i$,
for some $h_i \in S$.   A \newterm{sublist} of a list is a consecutive
subsequence  $[x_i;x_{i+1};x_{i+1};\ldots;x_j]$, with $0\le i\le j\le k-1$.
A \newterm{unit list} is a list of the form $[x]$.  A
path is \newterm{injective} if $x_i=x_j$ implies $i=j$. 
The \newterm{dart set} of $L$ is $\{x_0,x_1,\ldots,x_{k-1}\}$.  A path \newterm{visits}
a dart $x$, if $x$ is an element of the dart set of $L$.  A set of paths visits a
dart $x$, if some path in the set visits the dart.
\end{definition}

\begin{notation}[$\cooln$]
%Write $P[x_i:x_j]$ for $i<j$ for the sublist $[x_{i+1};\ldots;x_j]$ of
%$P=[x_0;\ldots;x_{k-1}]$.  (The notation is ambiguous when the path is
%not injective.)  
The infix operator $\cooln$ prepends an element $x$ to a list $[x_0;\ldots]$:
\begin{displaymath}
x\cooln[x_0;\ldots] = [x;x_0;\ldots].
\end{displaymath}
%The infix operator $\opat$  \newterm{concatenates} 
%lists:
%\begin{displaymath}
%[a;\ldots;b] \opat [c;\ldots;d]  = [a;\ldots;b;c;\ldots;d].
%\end{displaymath}
\end{notation}
\indy{Notation}{1@$\cooln$ (list operation)}%
\indy{Notation}{P@$P$ (dart path)}%
%\indy{Notation}{1@$[-:-]$ (dart sublist)}%
%\indy{Notation}{Z@$\opat$~(concatenation)}%


\begin{definition}[$\sim_S$]\guid{IENSLJP}
Let $D$ be a set, and let $S$ be a 
set of permutations on $D$.
Define a relation on the set of darts by $x\sim_S y$ when there is a
path from $x$ to $y$ with steps in $S$.
\end{definition}

\begin{lemma}[equal equivalences]\guid{YBGABWW}\rating{50}\label{lemma:er} %\guid{QLPBIKV}
% wording changed by thales Jan 7, 2010.
Let $(D,e,n,f)$ be a hypermap and let $S$ be a  set of permutations.
Then for each $h_1,h_2\in S$, 
the relation $\sim_S$ is the same as the relation $\sim_T$, where
\begin{displaymath}
T = S \cup \{h_1h_2\}.
\end{displaymath}
Moreover, for each $h\in S$, 
the relation $\sim_S$ is the same as the relation $\sim_T$, where
\begin{displaymath}
T = S \cup \{h^{-1}\}.
\end{displaymath}
Also,  the relation $\sim_S$ (that is, $\sim_T$) is an equivalence relation.  
\indy{Index}{equivalence relation}%
\end{lemma}

\begin{proof} If $x\sim_S y$ then clearly $x\sim_T y$.  Conversely,
if $x\sim_T y$, where $T = S\cup\{h_1,h_2\}$, pick a path $P$ from $x$ to $y$ with steps
in $T$ that contains the fewest $h_1h_2$-steps.  

\claim{$P$ does not contain any $h_1h_2$-steps}.  Otherwise, a sublist $[\ldots;u;h_1h_2u;\ldots]$
of $P$ can be expanded to a path $[\ldots;u;h_2u;h_1u;\ldots]$ that contradicts the minimal
property of $P$.

This proves the first conclusion of the lemma.  Fix $h$ in a set of permutations $R$.
By an induction that uses the first conclusion,  for all $i$, $\sim_R$ equals the relation $\sim_{R(h,i)}$,
where $R(h,i) = R \cup \{h,h^2,\ldots,h^i\}$.  If $h\in S$ is an element of order $k$, 
and $T = S\cup\{h^{-1}\}$, then
the second conclusion follows because the following sets give the same relation:
\begin{displaymath}
S,\quad S(h,k-1) = T(h,k-2),\quad T.
\end{displaymath}

By repeated action of the previous conclusion, $\sim_S=\sim_T$, where 
$T = S\cup S^{-1}\cup \{I_D\}$, and where $S^{-1} = \{h^{-1}\mid h\in S\}$.
The unit path $[x]$ yields reflexivity of $\sim_T$.  Also, $T^{-1} = T$ gives the symmetry.  Finally, concatenation of paths gives transitivity.  Thus, $\sim_T$ (i.e., $\sim_S$) is an equivalence relation.
\end{proof}

\begin{definition}[combinatorial~component,~connected]\guid{JVTRXQR}
A \newterm{combinatorial component} of a hypermap $(D,e,n,f)$ is an 
equivalence class of the relation $\sim_S$, where
$S=\{e,f,n\}$. 
(See Lemma~\ref{lemma:er} for other sets that define the same equivalence classes.)  
Write $\#c$ for the
number of combinatorial components.  The hypermap is \newterm{connected} if
$\#c=1$.  \indy{Index}{Dhammapada}%
\indy{Index}{path}%
\indy{Index}{connected}%
\indy{Index}{component!combinatorial}%
\indy{Notation}{1@$\#c$~ (number of components)}%
\end{definition}





\begin{definition}[orbit,~node,~face,~edge]\guid{JIOUCMV}
The \newterm{orbit} of $x\in D$ under a permutation $h$ on
a set $D$ is a set of the form $\{h^i x\mid i\in\ring{N}\}$.  A \newterm{node}
of a hypermap $(D,e,n,f)$ is the orbit of a dart $x\in D$ under $n$.  
A \newterm{face} is an orbit under $f$.  
An \newterm{edge} is an
orbit under $e$.  \indy{Index}{node}%
\indy{Index}{face}%
\indy{Index}{edge}%
\end{definition}

Write $\#h$ for the
number of orbits of a permutation $h$ on $D$.  
\indy{Notation}{h@$h$ (permutation)}%
\indy{Notation}{1@$\#h$~(number of orbits)}%


\begin{lemma}[orbit relation]\guid{PKRQTKP}
Let $D$ be a finite set.  The orbit of $x\in D$ of a permutation $h:D\to D$
is the equivalence class of $x$ under the relation $\sim_S$, when $S=\{h\}$.
\end{lemma}

\begin{definition}[plain]\guid{HFRNMIU}
A hypermap $(D,e,n,f)$ is \newterm{plain}
  (carefully note\footnote{This deliberate play on the homophonous
    {\it plane} privileges writing over speech.  Every plane (hyper)graph may
be planar, but not all plain hypermaps are planar.}
  the  spelling!) when $e$ is an involution on $D$ (that is, $e^2 = I_D$).
  \indy{Index}{planar}%
\end{definition}




\begin{definition}[degenerate]\guid{MKSZLRM}
 A dart in a hypermap $(D,e,n,f)$ 
is degenerate if it is a
fixed point of one of the maps $e,n,f$; otherwise it is nondegenerate.  
%%It is nondegenerate otherwise.
\indy{Index}{dart!degenerate}%
\indy{Index}{dart!nondegenerate}%
\end{definition}

\begin{definition}[simple]\guid{KMHUQNS} 
A hypermap is \newterm{simple} if the intersection of each face with
each node contains at most one dart.  \indy{Index}{simple}%
\end{definition}


% Moved from cup05_tame.tex section on tame plane graphs. 9/5/07:
\begin{lemma}[nodal fixed point]\guid{ZHQCZLX}\rating{50}\tlabel{lemma:nondegen} 
Let $(D,e,n,f)$ be a simple plain hypermap such that every face has
at least three darts.
Then $n$ has no fixed point.
\indy{Index}{fixed point}%
\end{lemma}

\begin{proof} For a contradiction, let $x$ be a fixed point of
$n$. 

\claim{The darts $e x$ and $f x$ lie in the same node and face, so are
equal in the simple hypermap.}  Indeed, they lie in the same node
because $n(f x) = e^{-1} x = e x$. They lie in the same face because
\begin{displaymath}f^2 (e x) = f (f e n x) = f x.\end{displaymath}
So $e x = f x$.

Thus, $f^2 (e x) = f x = e x$, and $e x$ lies on a
face with at most two darts.  This contradicts what is given.
\end{proof}




\section{Walkup}

To focus on a dart $x$ in a
hypermap, it can be useful to draw a hexagon around $x$ and place
the six darts $e x$,
$f x$, $e^{-1} x$, $n x$,  $f^{-1} x$, $e x$, $n^{-1} x$  at its corners
in Figure~\ref{fig:dart+}.  Some of these seven darts may be
equal to one another, even if the figure draws them apart.
Figure~\ref{fig:dart-fix} shows the layout of a degenerate dart.
\indy{Notation}{x@$x$ (dart)}%

\begin{figure}[htb]
\centering
\szincludegraphics[width=40mm]{\pdfp/dart+.eps}
\caption{A dart $x$ and its entourage}
\label{fig:dart+}
\end{figure}

\begin{figure}[htb]
\centering
\szincludegraphics[width=60mm]{\pdfp/dart-fix.eps}
\caption{A dart fixed under a face map.}
\label{fig:dart-fix}
\end{figure}

\subsection{single}

A \newterm{walkup} deletes
a dart from a hypermap and reforms the edge, node, and face
maps to produce a hypermap on the reduced set of darts.  Walkups
come in three flavors: edge walkups, face walkups,
and node walkups.

\begin{definition}[walkup,~degenerate]\guid{DAIZNHD}
The edge \newterm{walkup}
$W_e$ at  a dart $x\in D$ of a hypermap $(D,e,n,f)$ is the hypermap
$(D',e',n',f')$, where $D' = D\setminus\{x\}$ and the
the maps skip over $x$:
\begin{displaymath}
\begin{array}{lll}
f' y &= \text{ if } (f y =  x) \text{ then } f x \text{ else
} f y\\
n' y &= \text{ if } (n y = x) \text{ then } n x \text{ else
} n y\\
e' &= (n'\ocirc f')^{-1}
\end{array}
\end{displaymath}
A walkup at $x$ is said to be \newterm{degenerate} if the dart $x$ is
degenerate.  
\indy{Index}{walkup}%
\indy{Index}{edge!walkup}%
\indy{Index}{face!walkup}%
\indy{Index}{node!walkup}%
\indy{Notation}{Wh@$W_h$ (walkup)}%
\end{definition}

Figure~\ref{fig:walk} shows
the result of an edge walkup on the hexagon around a dart $x$.
The triality symmetry~\ref{eqn:triality}, applied to the definition
of edge walkups, yields the definition of
face walkup $W_f$ and node walkup $W_n$.  
% Figure~\ref{fig:walkfn} shows the result of the face and node
% walkups on the hexagon around a dart $x$.

At a degenerate dart $x$, all three walkups are equal:
$W=W_e=W_n=W_f$ (Figure~\ref{fig:walkdegen}).
\indy{Index}{walkup!degenerate}%
\indy{Notation}{x@$x$ (dart)}%

\begin{figure}[htb]
\centering
\szincludegraphics[width=80mm]{\pdfp/walk.eps}
\caption{The effect of a walkup at $x$}
\label{fig:walk}
\end{figure}


\begin{figure}[htb]
\centering
\szincludegraphics[width=80mm]{\pdfp/walkdegen.eps}
\caption{The effect of a walkup at a degenerate dart}
\label{fig:walkdegen}
\end{figure}


\begin{definition}[merge,~split]\guid{KJIOZBJ}\tlabel{def:merge-split} 
Let $(D,e,n,f)$ be a hypermap, and let $h=n,e$, or $f$.  Let $\op{orbit}(h,x)$
denote the orbit of $x\in D$ under $h$.  Let $(D',e',n',f')$ be the hypermap obtained
from $(D,e,n,f)$ by the walkup $W_h$ at $x\in D$.
Let $h'=e',n',f'$, respectively, according to the choice of $h$.
The walkup $W_h$ at $x$ \newterm{merges} when the walkup joins the
orbit of $h$ through $x$ with another orbit.  That is, there is an orbit $O$ of some
$y\in D'$ under $h':D'\to D'$ of the form
\begin{displaymath}
O \cup\{x\} = \op{orbit}(h,x) \cup \op{orbit}(h,y),
\end{displaymath}
where $y\not\in \op{orbit}(h,x)$.
It \newterm{splits}
when the walkup splits the orbit at $x$ into two orbits.  That is, there are 
distinct orbits $O_1,O_2$ under $h'$ in the hypermap $(D',e',n',f')$ such that
\begin{displaymath}
\{x\}\cup O_1\cup O_2 = \op{orbit}(h,x).
\end{displaymath}
\indy{Index}{split}%
\indy{Index}{merge}%
\indy{Index}{orbit}%
\end{definition}

\begin{lemma}[merge-split]\guid{ZMFKZNH}\rating{150}\tlabel{lemma:merge-split} 
  Let $(D,e,n,f)$ be a hypermap and let $W_h$ be a nondegenerate
  walkup at a dart $x$.  Then $W_h$ merges or splits. Moreover, it merges if
  and only if $x$ and $y$ lie in distinct $h$-orbits, where
  $(h,y)=(f,e x)$,  $(e,n x)$, or $(n,f x)$.
\end{lemma}

\begin{proof} The walkup $W_f$ splits if and only if $f x$ 
(or $x$)
and $e x$ lie in the same $f$-orbit before the split. 
Figure~\ref{fig:split} makes this clear.
The other cases $h=e,n$ hold by triality.
\end{proof}


\begin{figure}[htb]
\centering
\szincludegraphics[height=90mm]{\pdfp/split.eps}
\caption{The face walkup at $x$ mixes $f$-orbits.  If it mixes a
single orbit, the orbit splits. If it mixes two separate orbits, the
orbits merge. }
\label{fig:split}
\end{figure}

The following is a useful way to tell if a walkup merges.


\begin{lemma}[merge criterion]\guid{FKSNTKR}\rating{80}\tlabel{lemma:ng-merge}  
Suppose, in a simple plain hypermap $(D,e,n,f)$, that an edge $\{x,y\}$ consists
of two nondegenerate darts.  Then the walkup $W_f$ 
% (resp. $W_n$)  removed Jan 10, 2009.  Needed? Is it even true?
at $x$ merges.
\end{lemma}
\indy{Index}{merge}%

\begin{proof} 
The darts $f x$ and $e x$ lie in the same node: $n (f x) = e^{-1} x
= e x$. If they are also in the same face of a simple hypermap, then
$f x = e x = y$. So
\begin{displaymath}n y  = n f x = n f e y = y,\end{displaymath}
and $y$ is a fixed
point of $n$, hence degenerate, contrary to assumption.  
Thus, $f x$ and $e x$ are in different faces, and the walkup merges
by Lemma~\ref{lemma:merge-split}.  
\end{proof}


\subsection{double}
\indy{Index}{walkup!double}%

A double walkup is the composite of two walkups of the same type.  The
two darts for the two walkups are to be the members of an orbit of
size two (under $n$, $e$, or $f$).
%%XX?The first walkup is to be chosen so that it merges.  
By choosing the type of the walkups to be different from the type of
the orbit, the first walkup reduces the orbit to a singleton, forcing
the second walkup to be degenerate.

Here are some examples.
\begin{itemize}
\item A double $W_n$ along an edge deletes the edge and 
merges the two endpoints into
a single node (Figure~\ref{fig:doublenode}). 
\item A double $W_f$ along an edge 
deletes the edge and merges the two faces along the edge into
one (Figure~\ref{fig:doubleface}).
\item A double $W_e$ at a node of degree two
deletes the node and merges the two edges at the node into
one (Figure~\ref{fig:doubleedge}).
\end{itemize}


\begin{figure}[htb]
\centering
\szincludegraphics[width=90mm]{\pdfp/double-node-walkup.eps}
\caption{The double node walkup applied to an edge}
\label{fig:doublenode}
\end{figure}


\begin{figure}[htb]
\centering
\szincludegraphics[width=90mm]{\pdfp/double-face-walkup.eps}
\caption{The double face walkup applied to an edge}
\label{fig:doubleface}
\end{figure}


\begin{figure}[htb]
\centering
\szincludegraphics[width=80mm]{\pdfp/double-edge-walkup.eps}
\caption{The double edge walkup applied to a node}
\label{fig:doubleedge}
\end{figure}

\begin{figure}[htb]
\centering
\szincludegraphics[width=80mm]{\pdfp/double_edge.eps}
\caption{The double edge walkup preserves plainness.}
\label{fig:doubleplain}
\end{figure}


\begin{lemma}[plain walkup]\guid{HOZKXVW}\rating{150}\tlabel{lemma:dwalk-planar}  
The three preceding double walkups carry plain
hypermaps into plain hypermaps.
\end{lemma}
\indy{Index}{hypermap!plain}%

\begin{proof} The walkups $W_n$ and $W_f$ preserve the orbit structure
of edges, except for dropping one dart.  By dropping both darts from
the same edge, one edge is lost and all others edges remain
unchanged.

Figure~\ref{fig:doubleplain} illustrates the double $W_e$.  The two
edges $\{x,e x\}$, $\{y, e y\}$ meeting the node are fused by the
double walkup into $\{e x, e y\}$, which is still an edge of size
two.
\end{proof}

\begin{remark}[reverse double walkup]\guid{KPRURND}\label{rem:reverse-double-walkup}
Double walkup transformations can be run in reverse.
Let $H'=(D',e',n',f')$ be
a hypermap and let $D\supset D'$ be a set that contains two additional elements
$x,y$.  Fix distinct elements $x',y'\in D'$.  Define $n,e:D\to D$ as follows.
\begin{displaymath}
\begin{cases} 
e x = y, &\\
e y = x,&\\
e z = e' z,&\text{otherwise.}\\
\end{cases}
\qquad\qquad
\begin{cases} 
n x' = x, &\\
n x =  n' x',&\\
n y' = y,&\\
n y =  n' y',&\\
n z = n' z, &\text{otherwise.}\\
\end{cases}
\end{displaymath}
Define $f$ by forcing the hypermap identity $e n f = I_D$.  
The edge $\{x,y\}$ has been inserted by a reverse double walkup.  The insertion points of the edge
into the hypermap
depend on the data $\{(x',x),(y',y)\}$.   

Reverse double walkup transformations that
insert a node $\{x,y\}$ or a face $\{x,y\}$ into a hypermap are obtained similarly  by triality symmetry.  For example, to insert a node onto an edge $\{x',y'\}$ of cardinality $2$, use
\begin{displaymath}
\begin{cases} 
n x = y, &\\
n y = x,&\\
n z = n' z,&\text{otherwise.}\\
\end{cases}
\qquad\qquad
\begin{cases} 
f x' = x, &\\
f x =  f' x',&\\
\text{etc.}&\\%f
%f y' = y,&\\
%f y =  f' y',&\\
%f z = f' z, &\text{otherwise.}\\
\end{cases}
\end{displaymath}
\indy{Index}{reverse double walkup}%
\end{remark}

\begin{remark}[dart universe]\guid{SCYVYJW}\label{rem:dart-universe}
For reverse double walkups, we need a set from which to draw new darts $x,y$.
We will use a well-ordered set $\Omega$ from which we draw, as needed,
the minimal element of the complement in $\Omega$ of the set of darts
already in play.  We assume that darts can be supplied from $\Omega$,
without mentioning it explicitly.  

For example, if we insert an edge into $(D',e',n',f')$
using the
ordered pair $(x',y')$, we use the data $\{(x',x),(y',y)\}$, where $x$ is the
least element of $\Omega\setminus D'$, and $y$ is the least element of
$\Omega\setminus (D'\cup \{x\})$.  To insert a node into an edge of cardinality
two, it is enough to specify one dart $x'$ in the edge.  Then let $y'$ be the other
dart in the edge, and choose $x,y\in \Omega$ as above.
\indy{Notation}{zzZ@$\Omega$ (well-ordered universe of darts)}
\end{remark}

\begin{definition}[RDW]\guid{TPEZAAM}\label{def:R}  
  Let $H'=(D',e',n',f')$ be a hypermap and let $x'\in D'$.  
%  Let $r$ be the
%  cardinality of the face of $x$, and let $m,p,q$ be integers that
%  satisfy $0\le p$, $0\le m < q < r$, and $m+1 <
%  p+q$.
Let $m,q,p$ be natural numbers.  Assume that $y' =(f')^{m+1}x'$ is not
equal to $z'=(f')^{q+1}x'$.
Construct a hypermap $RDW(H',x',m,p,q)$  as follows.  
First
  add an edge into $H$ using the ordered pair $(y',z')$
(by the reverse double walkup of
  Remark~\ref{rem:reverse-double-walkup}).  Then insert $p$ new nodes
  (of degree $2$) again by reverse double
  walkup transformations, each time at the edge containing $y'$.  
This is $RDW(H,x',m,p,q)$.
\end{definition}
\indy{Notation}{RDW@$RDW$ (reverse double walkup)}


\section{Planarity}
\indy{Index}{walkup}%
\indy{Index}{planarity}%

\begin{definition}[planar]\guid{QVATKMJ}
A hypermap is \newterm{planar} (note the
spelling!) when the Euler relation holds:
\begin{displaymath}\# n + \# e + \# f = \# D + 2\, \#c.\end{displaymath}
\indy{Index}{planar}%
\end{definition}


\begin{remark}[Eulerian relation]\guid{YPVCMHI}\label{rem:Euler}   
The Euler relation for planar graphs can be translated into the
language of hypermaps.  Consider a connected planar graph that
satisfies the Euler relation for the alternating sum of Betti
numbers:
\begin{displaymath}b_0 - b_1 + b_2 = 2\end{displaymath} where $b_0$
is the number of vertices, $b_1$ the number of edges, and $b_2$ the
number of faces (including an unbounded face) of the planar
graph. The hypermap $(D,e,n,f)$, made from the planar graph in
Remark~\ref{rem:hypermap}, is plain, and the involution $e$ has no fixed points.  
Thus, $\# D = 2\#e$, according to the partition of $D$ into edges.  Moreover,
\begin{displaymath}\begin{array}{lll}
b_0 &= \# n\\
b_1 &= \# e\\
b_2 &= \# f\\
2b_1 &= \# D\\
1 &= \#c\\
b_0 - b_1 + b_2  &= \# n + (\#e - \#D) + \# f = 2\,\# c.
\end{array}
\end{displaymath}
Thus, the hypermap is also planar.
\indy{Index}{Euler relation} %
\end{remark}


\begin{lemma}[dart bound]\guid{TGJISOK}\rating{80}\label{lemma:dart-upper} 
Let $H$ be a connected plain planar hypermap such that every edge
has cardinality two.  Assume that there are at least three darts in
every node.  Then
\begin{displaymath}
\# D \le (6\, \#f - 12).
\end{displaymath}
\end{lemma}
\indy{Notation}{H@$H$ (hypermap)}%

\begin{proof}  In a plain planar hypermap, the Euler relation becomes
\begin{displaymath}6\, \#f - 12 = 3\,\#D - 6\,\#n,\end{displaymath}
so it is enough to show that
\begin{displaymath}
\# D \ge 3\,\#n.
\end{displaymath}
This follows directly by assumption: the set of darts can be
partitioned into nodes, with at least three darts per node.
\end{proof}


\begin{definition}[planar~index]\guid{ICAWSNK}
The planar index of a hypermap is
\begin{displaymath}\iota = \# f + \# e + \# n - \# D - 2\,\#
c.\end{displaymath}
(A hypermap with null index is planar.)
\indy{Index}{hypermap!planar index}%
\indy{Notation}{ZZiota@$\iota$ (planar index)}%
\end{definition}

\begin{lemma}[walkup index]\guid{IUCLZYI}\rating{400}\tlabel{lemma:index} 
Let $x$ be a nondegenerate dart of a hypermap $(D,e,n,f)$. Let
$(D',e',n',f')$ be the result of the face walkup $W$ at $x$.  The
walkup changes the size of some orbits.
\begin{displaymath}
\begin{array}{lll}
%\text{\bf Non-degenerate dart $x$: }&\\
\# f' &=\# f +\op{split}_f  \\  
\# e'&=\# e \\
\# n'&=\# n \\
\# D'&=\# D - 1 \\
\#c'&=\# c + \op{split}_c\\
\iota' &= \iota + 1+\op{split}_f - 2\op{split}_c,\\
\end{array}
\end{displaymath}
where
\begin{displaymath}
\op{split}_f = \begin{cases}
1,&\text{if $W$ splits }\\
-1,&\text{if $W$ merges}\\
\end{cases}
\end{displaymath}
and $\op{split}_c=1$ if $e x$ and $f^{-1} x$ belong to different
combinatorial components after the walkup $W$, and $\op{split}_c=0$
otherwise. Moreover, a walkup at a degenerate dart preserves the
planar index.  \indy{Notation}{splitc@$\op{split}_c$}%
\indy{Notation}{splitf@$\op{split}_f$}%
\indy{Notation}{W@$W$ (walkup)}%
\end{lemma}

\begin{proof} The figures make this clear.
\end{proof}

\begin{lemma}[index inequality]\guid{BISHKQW}\rating{100}\tlabel{lemma:planar-index2}
Let $\iota$ be the index of a hypermap $(D,e,n,f)$, and let $\iota'$
be the index after a walkup $W_h$ at a dart $x$.  Then $\iota \le
\iota'$.
\end{lemma} 


\begin{proof} Without loss of generality, by triality symmetry, the
walkup is a face walkup.  If $\op{split}_c=0$, the inequality is
immediate by Lemma~\ref{lemma:index}.  If $\op{split}_c=1$, 
then $e x$ and $f^{-1} x$ lie in
different components after the walkup, hence also in different
faces.  Thus, the walkup splits by Lemma~\ref{lemma:merge-split}.
Hence  $\op{split}_f = 1$.  The result
follows by Lemma~\ref{lemma:index}.
\end{proof}


\begin{lemma}[non-positive index]\guid{FOAGLPA}\rating{50}
\tlabel{lemma:planar-nonpos}  
The planar index
of a hypermap is never positive.
\end{lemma}

\begin{proof}  An face walkup never decreases the index.  A sequence
of face walkups leads to the empty hypermap, which has
index zero.
\end{proof}


\begin{lemma}[planar walkup]\guid{SGCOSXK}\rating{50}
\tlabel{lemma:walkup-planar}
Walkups take planar hypermaps to planar
hypermaps.
\end{lemma}

\begin{proof}  
A planar hypermap has maximum index.  The walkup
can only increase the index, but never beyond its maximum.  
Thus, the index remains at its maximum value.
\end{proof}





\section{Path}

\subsection{contour}

\begin{definition}[cyclic~list]\guid{MYJNYCZ}
A \newterm{cyclic list} $\lp{x_0;\ldots;x_{k-1}}$ is an equivalence class of lists under the transitive closure of the relation:
\begin{displaymath}
[x_0;x_1;x_2\ldots;x_{k-1}] \sim [x_1;x_2;\ldots;x_{k-1};x_0].
\end{displaymath}
A sublist of a cyclic list is a sublist of some representative of the equivalence class.
\end{definition}

\begin{definition}[contour~path,cyclic~list,~contour~loop]\guid{AUIDQRN}
 A \newterm{contour path} from
$x_0$ to $x_{k-1}$ is a path $[x_0;x_1;\ldots;x_{k-1}]$ such that
$x_{i+1} = n^{-1} x_i$ or $f x_i$ for each $i<k$.  (That is, each
step in the path is clockwise step around a node or a
counterclockwise step around a face.)  
A \newterm{contour loop} is an injective cyclic list
$\lp{x_0;x_1;\ldots;x_{k-1}}$ such that
for every $i$, there exists $h_i\in \{f,n^{-1}\}$ such that $x_{i+1} = h_i x_i$, 
where the subscripts are
read modulo $k$.
%$[x_1;\ldots;x_{k-1}]$ is injective and $x_0 = x_{k-1}$, then it is
%a \newterm{contour loop}.  
%A sublist of a contour loop $[x_0;\ldots;x_{k-1}]$ is a path
%$[y_0;
\indy{Index}{contour!path}%
\indy{Index}{contour!loop}%
\indy{Index}{loop}%
\end{definition}



\begin{remark}[contour path illustration]\guid{AWRGIPA}
 Figure~\ref{fig:hypermap_ex}
  constructs a hypermap from a planar graph by drawing darts next to
  each angle.  In this representation, the darts along a contour path
  lie to the left of the corresponding planar graph edges.  For that
  reason, a shaded region to the left of a curve depicts a contour
  path.
\end{remark}

\begin{figure}[htb]
\centering
\szincludegraphics[width=80mm]{\pdfp/shade_dart.eps}
\caption{A contour path as a sequence as dart is represented as a
shaded path.}
\label{fig:shade-dart}
\end{figure}

\begin{lemma}[injective path]\guid{QZTPGJV}\rating{50} 
An injective contour path from
  $x$ to $y$ can be constructed from an arbitrary contour path from
  $x$ to $y$ by dropping some darts from the path.
\end{lemma}

\begin{proof} Repeatedly replace $[\ldots;a;b;\ldots;b;c;\ldots]$ with
$[\ldots;a;b;c;\ldots]$.
\end{proof}





\begin{lemma}[contours-components]\guid{KDAEDEX}\rating{100}\tlabel{lemma:connect-contour}  
Let $H$ be a hypermap.
If $x$ and $y$ are darts in the same combinatorial component of $H$ if and only if
there exists a contour path from $x$ to $y$.
\end{lemma}

\begin{proof} 
Combinatorial components are defined by an equivalence relation $\sim_S$, where
$S = \{e,n,f\}$.  By Lemma~\ref{lemma:er}, this is the same equivalence relation as
$\sim_T$, where $T = \{n^{-1},f\}$.  By the definition of the equivalence relation $T$,
$x\sim_T y$ if and only if there is a contour path from $x$ to $y$.
\end{proof}
\indy{Index}{component!combinatorial}%

\begin{definition}[complement]\guid{GCACAFP} 
Let $(D,e,n,f)$ be a plain hypermap.
Let $P=\lp{x;y;\ldots}$ be a contour loop that does not visit any node
twice in a plain hypermap.   (That is, the dart set of $P$ intersected with a node
is the dart set of a maximal sublist $[z;n^{-1}z;\ldots;n^{-k}z]$ of $n^{-1}$ steps.)
 Replace each maximal sublist of
$n^{-1}$-steps
\begin{displaymath}
[z;n^{-1} z; \ldots; n^{-k} z]
\end{displaymath}
with the sublist
\begin{displaymath}
[n^{-(k+1)} z;n^{-(k+2)} z;\ldots; n z]
\end{displaymath}
Concatenate these new sublists in reverse order.  By the relation $n f = f^{-1} n^{-1}$,
the transitions between the new sublists are $f$-steps.
The resulting contour loop $P^c$
is the \newterm{complement}. 
\end{definition}
\indy{Notation}{1@$*^c$ (complement)}

\begin{figure}[htb]
\centering
\szincludegraphics[width=70mm]{\pdfp/complement.eps}
\caption{The complement contour traces the remaining darts
at the same nodes as the original contour loop. }
\label{fig:contour-comp}
\end{figure}


\subsection{M\"obius}

\begin{definition}[M\"obius~contour]\guid{MBYIEQP}
 A M\"obius contour in a hypermap
$(D,e,n,f)$ is an
injective contour path $P=[x_0;\ldots]$ that satisfies
\begin{equation}
\tlabel{eqn:mobius}
x_j = n x_0\quad x_k = n x_i
\end{equation}
for some $0 < i\le j< k$ (Figure~\ref{fig:mobius}).
\indy{Index}{contour!M\"obius}%
\end{definition}


\begin{remark}[Four-Color theorem]\guid{ROIPZSU}
G. Gonthier devised the notion of M\"obius contour as a way to prove
the Four-Color theorem without appeal to topology.  (The Appel-Haken
proof of the Four-Color theorem relies on the Jordan curve theorem.)
This chapter uses a significant amount of material from ~\cite{Gonthier:2005:FourColor}.
\end{remark}

\begin{figure}[htb]
\centering
\szincludegraphics[width=50mm]{\pdfp/mobius.eps}
\caption{A M\"obius contour}
\label{fig:mobius}
\end{figure}

\begin{figure}[htb]
\centering
\szincludegraphics[width=30mm]{\pdfp/3m.eps}
\caption{The face map on this hypermap gives a M\"obius contour with
three darts}
\label{fig:3m}
\end{figure}

\begin{remark}[M\"obius strip]\guid{NGALZAC}
 Heuristically, a M\"obius contour is a 
combinatorial M\"obius strip that
twists to make 
its left-hand side into
its right-hand side.  A planar hypermap has no such contour.  
Figure~\ref{fig:violate-jct}
redraws a violation of the Jordan curve theorem
as a M\"obius contour.   
\end{remark}

\begin{figure}[htb]
\centering
\szincludegraphics[width=80mm]{\pdfp/violate-jct2.eps}
\caption{A path that tunnels from the interior to the exterior
of a simple closed curve
is analogous to a M\"obius contour.}
\label{fig:violate-jct}
\end{figure}

\begin{figure}[htb]
\centering
\szincludegraphics[width=80mm]{\pdfp/mobius_contour.eps}
\caption{Some M\"obius contours}
\label{fig:mobius-contour}
\end{figure}






\begin{lemma}[planar-non-M\"obius]\guid{LIPYTUI}\rating{300}\tlabel{lemma:no-mobius}
A planar hypermap does not have a M\"obius contour.
\end{lemma}
\indy{Index}{hypermap!planar}%

\begin{proof} For a contradiction, assume that there exist planar
hypermaps with M\"obius contours.  An edge walkup carries
planar hypermaps into planar hypermaps. An edge walkup
at a dart that is not on the M\"obius contour carries the
M\"obius contour to a M\"obius contour 
and reduces the number of darts.  
In the M\"obius Condition~\ref{eqn:mobius},
a walkup at a dart that is not at position $0$, $i$, $j$, $k$
along the contour carries the M\"obius contour to a M\"obius contour
and reduces the number of darts. Thus, a counterexample with
the smallest possible number of darts contains no
darts except those on the M\"obius contour, and its only darts
are at positions $0$, $i=j=1$, $k=2$.

This is a three darted hypermap (Figure~\ref{fig:3m}.)  
The M\"obius condition, the
definition of contours, together with $e\ocirc n\ocirc f=I_D$ force
$e=n=f$, all permutations of order three.  This hypermap is not planar:
\begin{displaymath}\# e + \# n + \# f = 3~~\ne~~ 5 = \# D + 2\,
\#c.\end{displaymath}
\end{proof}



%\subsection{interior}
%\indy{Index}{interior}%
%
%\begin{definition}[interior]\guid{NQXQALU}\label{def:interior} 
%A dart $y$ lies in the \newterm{interior} of a contour
%loop $L$ if there is a an injective contour path
%$x_0,x_1,\ldots,x_k=y$ such that $x_1 = f x_0$ (or $k=0$), and
%such that $x_i$ lies on the loop $L$ if and only if $i=0$.
%\indy{Index}{interior!contour loop}%
%\indy{Notation}{L@$L$ (contour loop)}%
%\indy{Notation}{y@$y$ (dart)}%
%\end{definition}
%

\begin{lemma}[step coherence]\guid{ILTXRQD}\rating{100}\tlabel{lemma:contour-path-type}
Suppose that a hypermap has no M\"obius contours. Let $L$ be a
contour loop.  Let $P$ be any injective contour path with at least
$3$ darts, that starts and ends on $L$, but visits no other darts of
$L$.  Then the first and last steps of $P$ are both of the same type
($n^{-1}$ or $f$).
\end{lemma}
\indy{Notation}{P@$P$ (contour path)}%

\begin{figure}[htb]
\centering
\szincludegraphics[width=80mm]{\pdfp/interior_nf.eps}
\caption{A path must enter and depart from a contour loop with the
same type of step.}
\label{fig:interior_nf}
\end{figure}


\begin{proof} The proof shows the contrapositive.  Suppose $P=[n x;f n
x;\ldots;n y;y]$.  The successor of $n x$ on $L$ is $x$.  Starting
at $x$, follow $L$ to $y$, and on to $n x$.  Follow $P$ back to $n
y$.
%\begin{displaymath}
%x\cooln L[x:n x] \opat P[ n x;ny].
%\end{displaymath}  
This is a M\"obius contour $x\ldots y\ldots n x\ldots n y$.

Suppose $P=[n x;x;\ldots;f^{-1} y;y]$.  Starting at $x$, follow $P$ to
$y$, then follow $L$ to $n x$, and on to $n y$.  This is a M\"obius
contour.\footnote{The second statement can also be deduced from the first statement
by the duality $(D,e,n,f)\leftrightarrow (D,e^{-1},f^{-1},n^{-1})$ that swaps
$f$-steps with $n^{-1}$-steps in a path.}
\end{proof}

%\begin{lemma}[]\guid{UMYSGDB}\rating{80}\tlabel{lemma:dart-interior}
%  Let $L$ be a contour loop on a plain hypermap without M\"obius
%  contours.  Assume a dart $x$ lies in the interior of the loop $L$.
%  Then every dart in its $f$-orbit lies in the interior of the loop.
%  Moreover, if the dart $x$ does not lie on the same node as any dart
%  in $L$, then every dart in the $n$-orbit of $x$ lies in the
%  interior of $L$.
%\end{lemma}

%\begin{proof} Let $P= [x_0;\ldots;x]$ be an injective path that
%  certifies that $x$ lies in the interior of $L$.  If $f x$ lies
%  along this path already or if it lies on $L$, then it is clearly
%  interior.  Otherwise, $[x_0;\ldots,x;f x]$ is a certifying path for
%  $f x$.  Similarly, use the certifying path $[x_0;\ldots;x;n^{-1}
%  x]$ for $n^{-1} x$.
%\end{proof}
%
%%
%\begin{definition}[interior~face,~node]\guid{JUXKJTU}
% A face or a node is interior
%  to a loop in a hypermap if all of its darts are interior.
%  \indy{Index}{interior!face}%
%  \indy{Index}{interior!node}%
%\end{definition}


\begin{lemma}[loop separation]\guid{ICJHAOQ}\rating{180}\tlabel{lemma:contour-f}
Suppose that a hypermap has no M\"obius contours.  Let $L$ be a
contour loop.  Then there does not exist a contour path
$[x_0;\ldots;x_k]$, for $k\ge 1$ with the following properties:
\begin{enumerate}
\item $x_i$ lies on $L$ if and only if $i=0$.
\item $x_1 = f x_0$.
\item $x_0$ and $x_k$ lie in different nodes.
\item Some dart of $L$ is at the node of $x_k$.
\end{enumerate}
\end{lemma}

%\begin{figure}[htb]
%  \centering
%  \szincludegraphics[width=40mm]{\pdfp/no_node_path.eps}
%  \caption{No path exists from a node of $L$ to the interior.}
%  \label{fig:no-node-path}
%\end{figure}

\begin{proof} Assume for a contradiction that the path $P$ exists.
Some sublist is injective and satisfies the same conditions.  Again,
without loss of generality, shrinking the path if needed, $k$ is the
smallest index for which the last two conditions are met.  Append
$n^{-1}$-steps to $P$ to reach a dart of $L$.  This is contrary to
Lemma~\ref{lemma:contour-path-type}.
\end{proof}

\begin{lemma}[three darts]\guid{EUXPBPO}\rating{ZZ}\label{lemma:3dart}  
Assume that each face of a hypermap  has at least three darts.
Then every contour loop that meets at least two nodes has at least
three darts.
\end{lemma}

\begin{proof} Let $P=\lp{x;y}$ be a contour loop meeting two nodes.  Then
$y = f x$ and $x = f y$, so that the face has size two.
\end{proof}


\section{Quotient}
\indy{Index}{quotient}%

\subsection{definition}

\begin{definition}[isomorphism]\guid{GUDUERI}
 Two hypermaps $(D,e,n,f)$ and
$(D',e',n',f')$ are \newterm{isomorphic} when there is a bijection
$G:D\to D'$ such that
\begin{displaymath}h'\circ G = G\circ h\end{displaymath}
for $(h,h')=(e,e'), (f,f'), (n,n')$.
\indy{Index}{isomorphic hypermaps}%
\indy{Notation}{G@$G$ (morphism of hypermaps)}%
\end{definition}


\begin{definition}[normal family]\guid{RQSVFLE}
Let $(D,e,n,f)$ be a hypermap. 
%Assume that 
%there are no darts fixed by $e$ 
%(so that $f x \ne n^{-1} x$ at each dart). 
Let $\cal L$ be a family of contour
loops.  The family $\cal L$ is  normal if the following
conditions hold of its loops. \begin{enumerate}
\item  No dart is visited by two different loops.
\item  Every loop visits at least two nodes.
\item  If a loop visits a node, then every dart at that node is visited
by some loop.
\end{enumerate}
\indy{Index}{normal family}%
\end{definition}

A normal family determines a new hypermap.  A dart in the new set $D'$
of darts is a maximal sublist $[x;n^{-1} x; n^{-2} x;\ldots;n^{-k}
x]$ of $n^{-1}$ steps appearing in some loop in $\cal L$. The map $f'$
takes the maximal path $[x;n^{-1}x;\ldots;y]$ to the maximal path (in
the same contour loop) starting at $f y$. The map ${n'}^{-1}$ takes
the maximal path $[\ldots;y]$ to the maximal sequence (in some other
contour loop) starting $[n^{-1}y;\ldots]$. Equivalently, $n'$ takes
the maximal path $[x;\ldots]$ to the maximal path ending $[\ldots;n
x]$. The map $e'$ is defined by $e'\ocirc n'\ocirc f' = I_{D'}$.
\indy{Index}{path!maximal} %
\indy{Notation}{D@$D$ (dart)}%

\begin{definition}[quotient]\guid{AJENHSB}
 The hypermap $(D',e',n',f')$
constructed from the normal
family $\cal L$ of $H=(D,e,n,f)$ 
is called the \newterm{quotient} of $H$ by $\cal L$, and is denoted
$H/{\cal L}$.  If $x$ is a dart visited by some loop in $\cal L$, then
the maximal path $[\ldots;x;\ldots]$ is called the \newterm{quotient dart} of $x$.
%The hypermap $H$ is said to be a \newterm{cover} of $H/{\cal L}$.  
\indy{Index}{quotient}%
\end{definition}
\indy{Notation}{L@$\cal L$}%
\indy{Notation}{H@$H/{\cal L}$}%
%\indy{Index}{cover}%
\indy{Index}{hypermap!quotient}%
\indy{Index}{normal family}%

Intuitively, the quotient hypermap is represented as a graph whose
cycles under $f'$ are precisely the contour loops in the normal family
(Figure~\ref{fig:quot}).


\begin{figure}[htb]
\centering
\szincludegraphics[width=70mm]{\pdfp/quot.eps}
\caption{The contour loops in a normal family become faces in the
quotient}
\label{fig:quot}
\end{figure}

\subsection{properties}

This subsection explores some of the properties of a quotient hypermap.
The first two lemmas describe the faces and the nodes of the quotient
in terms of the combinatorics of the normal family.

\begin{lemma}[quotient face,~$\F$]\guid{WIRLCNL}\label{lemma:quotient-bijection}
  Let ${\cal L}$ be a normal family of the hypermap $H$.  Then ${\cal
    L}$ is in natural bijection with the set of faces of the quotient
  $H/{\cal L}$.  If $x'=[x;\ldots;n^{-k}x]$ is a maximal path of
  $n^{-1}$ steps in the contour loop $L\in{\cal L}$, then the
  corresponding face $\F(L)$ of $H/{\cal L}$ is the
  one containing the quotient dart $x'$.
\end{lemma}
\indy{Notation}{F@$\F$ (quotient bijection)}

\begin{proof}  This is left as an exercise to the reader.
\end{proof}


\begin{lemma}[quotient node]\guid{UDJNSHH}\label{lemma:quotient-node}
Let $H$ be a hypermap and let ${\cal L}$ be a normal family of $H$.
Then there is a natural bijection between  the set of nodes of
$H/{\cal L}$ and the set of nodes of $H$ that
are visited by some contour loop in ${\cal L}$.   
The bijective function sends the node in $H/{\cal L}$ of 
the dart $x' = [x;n^{-1} x;\ldots;n^{-k}x]$ to the node of $x$ in $H$.
\end{lemma}

\begin{proof}  The proof is an elementary verification.
Let $H/{\cal L} = (D',e',n',f')$.
\claim{This function is well-defined.}  Indeed, 
 \begin{displaymath}
(n')^{-1} x' = [n^{-(k+1)} x;\ldots]
\end{displaymath}
 is also sent to the node of $x$ in $H$.

\claim{This function is onto.}  Indeed,
If $L\in {\cal L}$ visits $x$,  and $x'$ is the image of $x$ in $D'$.
Then the node of $x'$ clearly maps to the node of $x$.  

\claim{Finally, the function is one-to-one.}  Indeed, if the nodes of
two quotient darts $x'$, $y'$ map to the same node of $H$, then
$x'=[x;\ldots]$ and $y'=[y;\ldots]$, where $n^j x = y$ for some $j$.
It follows by the definition of the node map on the quotient, that
$x'$ and $y'$ belong to the same node.
\end{proof}

The next two lemmas look at properties of the quotient that are
inherited from the orignal hypermap.

\begin{lemma}[plain quotient]\guid{JMKRXLA}\rating{280}\tlabel{lemma:quotient-plain}
Let $H$ be a plain hypermap, and let $\cal L$ be a
normal family.  Then $H/{\cal L}$ is a plain hypermap.
\end{lemma}

\begin{proof}  Write $H=(D,e,n,f)$ and $H/{\cal L} = (D',e',n',f')$.  
Write $[\ldots; x]$ for the node in
the quotient ending in dart $x\in D$ and $[x;\ldots]$ for the node
in the quotient starting with dart $x\in D$.  Plainness gives $e^2 x
= x$, so that for any dart $[\ldots x]$ in the quotient:
\begin{displaymath}\begin{array}{lll}
{e'}^{-2} [\ldots; x] &= n' f' n' f' [\ldots; x] = n' f' n' [f x; \ldots] \\&=
n' f' [\ldots; n f x] = n' [f n f; \ldots] = [\ldots; n f n f x]\\ &=
[\ldots; e^{-2} x] = [\ldots; x].
\end{array}\end{displaymath}
Thus, $e'$ has order $2$ on the quotient.
\end{proof}




\begin{definition}[no double joins]\guid{EDUYIEA}
A hypermap $H$ has no \newterm{double joins}, if for every two nodes
in $H$, there is at most one edge that meets both of them.
\end{definition}


\begin{lemma}[quotient-no-double-joins]\guid{KSRDPTZ}
Let $H$ be a plain hypermap with no double joins and let ${\cal L}$ be a normal
family of $H$.  Then $H/{\cal L}$ has no double joins.
\end{lemma}

\begin{proof} By Lemma~\ref{lemma:quotient-plain}, the quotient
  $H/{\cal L}$ is plain.  Let $\{x',e'x'\}$ and $\{y',e'y'\}$ be edges
  with the property that $x'$ and $y'$ lie at one node of $H/{\cal L}$
  and $e'x'$ and $e'y'$ lie at a second (different) node.  Write $x' =
  [\ldots;x]$ and $y' = [\ldots;y]$.  Then $e'x' = [\ldots;e x]$ and
  $e'y' = [\ldots;e y]$.  According to
  Lemma~\ref{lemma:quotient-node}, there is an injective map from
  nodes of $H/{\cal L}$ to nodes of $H$.  It follows that $x$ and $y$
  belong to the same node and that $e x$ and $e y$ belong to the same
  (different) node.  By the assumption that $H$ has no double joins,
  it follows that $x=y$.  Hence also $x' = y'$, and $H/{\cal L}$ has
  no double joins.
\end{proof}


\begin{lemma}[nodal fixed point]\guid{PYOVATA}\label{lemma:nfp}
Let $H=(D,e,n,f)$ be a hypermap in which the edge map has no fixed points.
Let ${\cal  L}$ be a normal family of $H$, with quotient $H/{\cal L} = (D',e',n',f')$.  
Then the following are equivalent conditions:
\begin{itemize}
\item $n'$ has a fixed point in $D'$.
\item The dart set of some $L\in {\cal L}$ contains a node.
\end{itemize}
\end{lemma}

\begin{proof}
If $x'=[x;n^{-1} x;\ldots;n^{-k} x]$ is a dart in $D'$, then $(n')^{-1}x'$ is
$[n^{-(k+1)} x;\ldots]$.  The dart $x'$ is a fixed point if and only if
$x = n^{-(k+1)} x$.  This holds if and only if the dart set of $x'$ is an entire node.
\end{proof}

\subsection{example}

\begin{example}[maximal normal family]\label{ex:Hall} 
  Assume that $H=(D,e,n,f)$ is a
  hypermap. % with no fixed points under $e$.
  Assume that every face meets at least two nodes. Then the set of all
  faces defines a normal family of contour loops: follow $f$ around
  each face $[x;f x;\ldots]$.  If $e$ acts without fixed points, then
  each dart of the quotient is just a unit path consisting of a single
  dart of $H$, and the quotient is isomorphic to $H$ itself.
\end{example}

\begin{example}[minimal normal family]\label{ex:H2} 
  Assume that $H=(D,e,n,f)$ is a plain hypermap.  Let $F = \{x,f x,\ldots\}$ be a face
  that visits at least three nodes and that meets each node in at most
  one dart.  Let $\cal L$ be the family with two contour loops: $\lp{x;f x;\ldots}$ 
and its complement $L^c = \lp{n^{-1} x;\ldots}$.
%\begin{displaymath}
%  [n^{-1} x;n^{-2} x;\ldots;n x;f n x = y;n^{-1} y; n^{-2} y;\ldots; n y; f ny;\ldots]
%\end{displaymath}
The family $\cal L$ is normal. The quotient hypermap $H/{\cal L}$ has
two faces: $F$ and a back side $F'$ of the same cardinality $k$.
\indy{Notation}{F@$F$ (hypermap face)}%
\end{example}

\begin{example}[cyclic]\label{ex:H2k} 
There is a hypermap $H_{2k}$ with two face.  The set of darts is the
disjoint union of two copies of $Z_k$, a cyclic group of order $k$
with generator $1$.  Each cyclic group is a face.  Use the variable
$i$ to index the first cyclic group and $i'$ to index the second.
The face map is $i\mapsto i+1$ and $i'\mapsto (i-1)'$.  The node map
is the involution $i\leftrightarrow i'$.  The edge map is the
involution $i\leftrightarrow (i+1)'$.  The relation $e\ocirc n\ocirc
f = I_D$ is verified:
\begin{displaymath}
\begin{array}{llllllll}
enf(i) &= e n(i+1) &= e(i+1)' &= i\\
e n f (i') & e n (i-1)' & e (i-1) &= i'.\\
\end{array}
\end{displaymath}
If a hypermap is isomorphic to $H_{2k}$ for
some $k$, then it is \newterm{cyclic}.  In particular,
the hypermap constructed in the previous example is cyclic.
\indy{Index}{hypermap!cyclic}%
\indy{Notation}{Z@$Z_k$ (cyclic group)}%
\end{example}


%\begin{lemma}[three darts]\guid{BPAKIYP}
%Let $H$ be a plain hypermap with no double joins in which every
%face has at least three darts.  Let ${\cal L}$ be a node-free normal family of $H$.
%Then every face of $H/{\cal L}$ has at least three darts.
%\end{lemma}
%
%\begin{proof}  By definition, every contour loop in a normal family meets
%at least two nodes.  It follows that every face of $H/{\cal L}$ has at least two
%darts.
%
%Suppose for a contradiction that a face of $H'=H/{\cal L}$ only has two darts $x'$
%and $y'$, exchanged by the face map $f'$ of $H'$:
%\begin{displaymath}
%\begin{array}{lllll}
%y' &= [y;n^{-1} y;\ldots; n^{-k} y] &= f' x' &= [f n^{-\ell} x;\ldots;f x],\\
%x' &= [x;n^{-1}x;\ldots; n^{-\ell} x] &= f' y' &= [f n^{-k} y;\ldots; f y].\\
%\end{array}
%\end{displaymath}
%Using the relation $e^{-1} = n f$, it follows that $\{n^{-\ell} x, n y\}$
%and $\{n^{-k} y, n x\}$ are edges.  Both meet the node of $x$ and the node of $y$.
%In a hypermap with no mutiple joins, this implies that the edges are equal:
%\begin{displaymath}
%n y = n^{-k} y,\qquad n x = n^{-\ell} x.
%\end{displaymath}
%This shows that the dart set of $x'$ is the entire node.  This contradicts
%the assumption that ${\cal L}$ is node-free.
%\end{proof}
%

%\begin{lemma}[]\guid{ZOKKAOI}\tlabel{lemma:quotient-planar}
%Let $H$ be a plain planar hypermap, and let $\cal L$
%be a normal family.  Then $H/{\cal L}$ is a plain planar hypermap.
%\end{lemma}
%
%\begin{proof} Suppose $H/{\cal L}$ is not planar.
%Let $P$ be a M\"obius contour on $H/{\cal L}$.  It lifts uniquely to
%a contour on $H$ with the property that the darts visited on $H$ are
%precisely the darts that belong to a dart in the quotient.  This is
%compatible with the node map $n$.  So the contour path lifts to a
%M\"obius contour on $H$.  Thus, $H$ is not planar.
%\end{proof}

\section{Generation}
\indy{Index}{generation}%


The final section of this chapter presents an algorithm that generates all
simple, plain, planar hypermaps satisfying certain general conditions
(Definition~\ref{def:restricted}).  The algorithm proceeds by adding more and
more
edges and nodes to a cyclic hypermap by a sequence of reverse double walkup
transformations.

\begin{definition}[restricted]\guid{INCRVQC}\label{def:restricted}
A restricted hypermap $H = (D,e,n,f)$ is one with the following
properties.
\begin{enumerate}
\item The hypermap $H$ has no double joins, and is nonempty, planar,
  connected, plain and simple.
\item The edge map $e$ has no fixed points.  % Needed in Lemma:[flag quotient]
\item The node map $n$ has no fixed points.
\item The size of every face is at least $3$.
%%  (All hypoth. Needed?)
\end{enumerate}
\indy{Index}{hypermap!restricted}%
\indy{Notation}{H@$H$ (hypermap)}%
\end{definition}

\begin{remark}[step type]\guid{BMZYWKV}
The assumption that $e x \ne x$ implies that $f x \ne n^{-1} x$ so that $f$-steps of a 
path can be distinguished from $n^{-1}$-steps.
\end{remark}


\subsection{boolean value}
\indy{Index}{flag}%

The algorithm  marks certain faces as `true.'
Roughly, this  means that the the face cannot be modified
at any later stage of the algorithm.   When all of its faces
are true, the hypermap stands in final form.
The function that marks each face as true or false is a
\newterm{flag}.  For the algorithm to work properly, it is necessary
to impose some constraints, as captured in Definition~\ref{def:flag}.
\indy{Index}{hypermap!algorithm}%

%% XX \hat doesn't get used.

Under the bijection $\F$ between a normal family ${\cal L}$ and the set of
faces of a quotient $H/{\cal L}$, any function $\hat\varphi$ on ${\cal L}$
can be identified with a function $\check\varphi$ on the set of faces of $H/{\cal L}$.
(The {\it hat} points up to $H$, and the {\it check} points down to the quotient.)
This identification of functions will be used frequently in this section.

\begin{definition}[canonical function]\guid{CRUDEHU}
 Let $H$ be a hypermap with
  normal family $\cal L$.  The \newterm{canonical function}
  $\hat\varphi_{can}$ is the boolean-valued function on ${\cal L}$
  that is true on $L$ exactly when the dart set of $L$ maps
  bijectively to the face $\F(L)$ of $H/{\cal L}$, under $x\mapsto [x]$.  The
  corresponding function $\check\varphi_{can}$ is also called the
  canonical function.  A contour loop $L$ (or face $\F(L)$)
  is said to be canonically true or false, according to the value of the canonical
  function.  
\indy{Index}{function!canonical boolean}%
\indy{Notation}{zzP@$\check\varphi_{can}$}
\indy{Notation}{zzP@$\hat\varphi_{can}$}
\end{definition}

In other words, the face in the quotient is canonically true, exactly
when the corresponding contour loop $L\in {\cal L}$ has no $n^{-1}$
steps.  The dart set of such a contour loop $L$ is a face of $H$.  
%Based on this observation, we make the following definition.


\begin{definition}[flag]\guid{HFTAHWB}\label{def:flag} 
  Let $S$ be a set of darts in a hypermap $H$.  An $S$-\newterm{flag}
  on $H$ is a boolean-valued function $\check\varphi$ on the set of faces that
  satisfies the following two constraints.
\begin{enumerate}
\item If darts $x,y$ belong to true faces,
then there is a contour path from $x$ to $y$ that remains
in true faces.
\item Each edge of the hypermap meets a true face or $S$.
\end{enumerate}
An $\emptyset$-flag is simply called a flag.
%An isomorphism of flagged hypermaps is an isomorphism of
%hypermaps that respects the flags.
\indy{Index}{flag}%
\indy{Notation}{S@$S$ (set ofdarts)}%
\end{definition}



\begin{example}[cyclic hypermap flag] 
The cyclic hypermap of Example~\ref{ex:H2k}, carries a
flag that marks one face true and the other false.
\end{example}

\begin{example}[maximal quotient flag]\label{ex:Hall-flag} 
Let $H$ be a connected hypermap, and let $\cal L$ be the example of
Example~\ref{ex:Hall}, then the canonical map takes value
$\op{true}$ on every face.  This is a flag. In fact,
Lemma~\ref{lemma:connect-contour} provides the contour paths that
are required in the definition of flag.
\end{example}


%\begin{definition}[canonically true]\guid{PYAOBOS}
%  A contour loop $L$ in a hypermap is \newterm{canonically true} 
%if its
%  dart set is a face of $H$.
%\end{definition}


%\begin{lemma}[quotient isomorphism criterion]\guid{STKBEPH}\rating{100}
%\tlabel{lemma:all-dart}  
%  Let $H$ be a hypermap in which $e$ acts without fixed points, and
%  let ${\cal L}$ be a normal family of $H$. If the canonical boolean
%  function on the set of faces of $H/{\cal L}$ has at least as many
%  true values as there are faces of $H$, then $\cal L$ is the normal
%  family in Example~\ref{ex:Hall}. In particular, $H/{\cal L}$ is
%  isomorphic to $H$.
%\end{lemma}
%
%\begin{proof} If a face of  $H/{\cal L}$ is $\op{true}$,  then
%its darts are unit paths, and the face of $H/{\cal L}$ is in natural
%bijection with a face in $H$.  This is an injective map from the
%set of true faces of $H/{\cal L}$ to the set of faces of $H$.  The
%hypothesis of the lemma implies that this injective map is
%bijective. All of the darts of $H$ are accounted for under this
%bijection. Thus, the quotient has no false faces.  The result
%follows.
%\end{proof}
%


There is a standard way of constructing the sets $S$ of darts that
will be used in $S$-flags.  


\begin{definition}[S]\guid{FDRMSZG}
Let $H$ be a hypermap, $L$ a contour loop of the hypermap,
and $x$ an element of the dart set of $L$.
If  $L$ is  canonically true, then let $S=\emptyset$.
Otherwise,
let $m\ge0$ to be the largest $m$ 
such that 
\begin{displaymath}
[x;f x; f^2 x;\ldots;f^{m+1} x]
\end{displaymath}  
is a sublist of $L$, and
%Set $y = f^{m+1} x$
set $S(H,L,x) = \{f^i x \mid 1 \le i\le m\}$.
\end{definition}
\indy{Notation}{S@$S(H,L,x)$ (flag set)}

\begin{lemma}[flag quotient]\guid{KHGAQRG}\label{lemma:flag-set-quotient}
Let $H$ be a hypermap in which $e$ acts without fixed points, 
$L$ a contour loop, and $x$ and element of the dart set of $L$.
Let ${\cal L}$ be a normal family of $H$ that contains $L$.
Then $S(H,L,x)$ maps bijectively to a set $S'$ of darts in the quotient $H/{\cal L}$.
\end{lemma}

\begin{proof} The darts of the quotient are maximal sublists
  $[y;n^{-1} y;\ldots;n^{-k} y]$ of contour loops $L'\in {\cal L}$
  made entirely of $n^{-1}$ steps.  Each $y\in S(H,L,x)$ is preceded
  by an $f$-step and is followed by an $f$-step in $L$.  Hence the
  maximal sublist of $L$ containing $y$ is a unit path $[y]$.  The
  bijection follows.
\end{proof}


\subsection{markup}\label{sec:face-insert}
\indy{Index}{extension}%


%This section describes the transition from $H/{\cal L}$ to $H/{\cal
%M}$.  The algorithm carries along various auxiliary data satisfying
%various assumptions, enumerated as follows:

%\begin{definition}[node free]\guid{RZNSLOS}
%A  family ${\cal L}$ of contour loops in a hypermap $H$  
%is \newterm{node free}, if no node of $H$ is contained the dart set of any 
%$L\in {\cal L}$.
%\end{definition}



\begin{definition}[marked hypermap]\guid{TUFOKWK}\label{def:marked}
Let $(H,{\cal L},L,x)$ be a tuple, consisting of 
\begin{itemize}
\item a hypermap $H=(D,e,n,f)$ with no M\"obius contours, and in which
  $e$ acts without fixed points.  % Mobius needed for HQY...
\item a normal family ${\cal L}$, 
\item a contour loop $L\in{\cal L}$, and
\item a dart $x$ visited by $L$.
%and 
%\item the canonical boolean-valued function $\varphi'$ on $H/{\cal L}$
%  (identified with a function $\varphi$ on ${\cal L}$ by
%  Lemma~\ref{lemma:quotient-bijection}).
\end{itemize}
Such a tuple is a \newterm{marked hypermap} if
the following conditions hold.
\begin{enumerate}
\item The quotient $H'=H/{\cal L} = (D',e',n',f')$ is simple.  
\item $n'$ has no fixed points on $D'$.
\item $x$ is followed by an $f$-step in the loop: $L = \lp{x;fx;\ldots}$.
%\item $\varphi$ coincides with the natural boolean function on ${\cal L}\setminus \{L\}$.
%\item $\varphi$ is false on the loop $L$.
\item The contour loop $L'\in {\cal L}$ that visits%
\footnote{$L$ visits  $f x$.  By the definition of normality, some contour loop in
${\cal L}$ visits the dart $n f x$ at the same node.} 
$n f x$ is canonically true.
\item 
  $\check\varphi_{can}$ is an $S'$-flag on $H'$, where $S'$ is the image of 
  $S(H,L,x)$ in $D'$.  %Lemma~\ref{lemma:flag-set-quotient}.
  %under the identification of $\varphi$ with a boolean-valued
  %function on $H'$ ().
\end{enumerate}
\end{definition}

%\item $H$ is a hypermap.
%\item $\cal L$ is a normal family of $H$.
%\item $L$ is a contour loop in ${\cal L}$.
%\item $x$ belongs to the dart set of $L$.
%\item $\varphi$ is a boolean function on the faces of $H'$.
%\item The set $S(H,L,x)$ maps bijectively to a set $S'$
%of darts in the quotient.






\begin{example}[illustration]\label{ex:graph-gen}  
  This example illustrates the markup (Figure~\ref{fig:graph-gen}).
  In the figure, the hypermap $H$ is represented as a planar graph.
  The contour loops are represented by left-side shadings of the edges
  of the planar graph.  The shaded edges give the edges of a planar
  graph representing the quotient.  The polygons that are fully shaded
  are true.  Two polygons in the quotient are false.  A dart of $H$ in
  a false contour loop $L$ in $H'$ is marked $x$.  By inspection,
  $S(H,L,x)=\{f x,f^2 x,f^3 x\}$.  By inspection,
  $\check\varphi_{can}$ is an $S'$-flag.  (In fact, the darts in true
  faces form a connected region.  Every edge in the quotient except
  $\{f' x', e' f' x'\}$ meets a true face, and this one edge meets
  $S'$.)
%% XX recheck
\end{example}

\begin{figure}[htb]
\centering
\szincludegraphics[width=90mm]{\pdfp/graph_gen.eps}
\caption{An example of the current situation.}
\label{fig:graph-gen}
\end{figure}

\begin{definition}[$m$,$p$,$q$,$y$,$z$]\guid{BVUFRRE}\label{def:yz}
Let $(H,{\cal L},L,x)$ be a marked hypermap.
Several lemmas use the following natural numbers $m,p,q$ and darts $y,z$.
Set  $y = f^{m+1} x$, where $m = \card(S(H,L,x))$.
  Set
$z=f^{p+1} y$, where $p$ is the smallest natural number 
%Then there is a smallest $p\ge0$,
such that some contour loop in ${\cal L}$ visits $f^{p+1} y$.
Let $x'$ and $z'$ be the images of $x$ and $z$ respectively in $H/{\cal L} = (D',e',n',f')$.
Let $q$ be the smallest natural number such that $z' = (f')^{q+1} x'$.  
\end{definition}
\indy{Notation}{m@$m$ (face map exponent)}
\indy{Notation}{p@$p$ (face map exponent)}
\indy{Notation}{y@$y$ (dart)}
\indy{Notation}{z@$z$ (dart)}

The existence of $q$ follows from the following lemma showing that $x'$ and $z'$
lie in the same face $\F(L)$ of $H/{\cal L}$.  (The existence of $q$ is trivial
when $L$ is canonically true.)

\begin{lemma}[loop confinement]\guid{HQYMRTX}\rating{200} \label{lemma:yz}
Let $(H,{\cal L},L,x)$ be a marked hypermap.
Assume that  $L$ is canonically false. % canonically true assumption not needed.
Let the natural number $m,p$ and darts $y,z$ be given by Definition~\ref{def:yz}.
Then, $L$ visits $z$, and $z\ne f^k x$ when $0 < k \le {m+1}$.
\end{lemma}

\begin{proof} 
%We break the proof into two cases, depending on whether $L$ is canonically true.
%
%\claim{[$L$ is canonically true.]}  In this case $S=\emptyset$,  $m=0$, $p=0$, $k=1$,
%$y = f x$, and $z = f y = f^2 x$.  Also, $L = \lp{x;f x;f^2 x;\ldots}$, and it clearly
%visits $z$.  If $z=f^k x = y$, then $x=y=z$ is a fixed point of $f$.
%
%This completes the first case.
%
%\claim{[$L$ is canonically false.]} In this case,
  For a contradiction, suppose $f^{p+1} y = f^k x$, for some $0<k\le
  m+1$.  Then also, $f^p y = f^{k-1} x$.  If $p>0$, then this
  contradicts the minimality of $p$.   (Note that $L$ visits $f^{k-1}x$ by the
definition of $S$ and $m$.)  So $p=0$, and $y=f^{k-1} x =
  f^{m+1} x$.  Also, $0\le k-1 < {m+1}$, which implies that the face of $x$ has size at
most $m+1$.  This  forces $L$ to be canonically true, which is contrary to assumption.  This proves the second
  conclusion of the lemma.  In particular, $z\not\in S$, where $S =
  S(H,L,x)$.

Let $L'$ be the contour loop of ${\cal L}$ that visits $z$.  For a contradiction,
assume that $L'\ne L$.

\claim{$L'$ is false.}  Otherwise, $L'$ is true with respect to the
canonical flag and it therefore a loop consisting entirely of
$f$-steps.  In particular, $L'$ visits $z,x$, and $y$.  This is
contrary, to the assumption that the contour loop containing $x$ is
false.

Let $H' = (D',e',n',f') = H/{\cal L}$, and let $S'$ be the image of $S$ in $D'$.
Let $z' = [\ldots;u]\in D'$ be the
image of $z$ in $D'$.    
As $z\not\in S$, we also have $z'\not\in S'$.
By the definition of
$S'$-flag, the dart $e'z'$ lies in a true face or $e'z'\in S'$.  
This disjunction splits
splits the proof into two cases.
\begin{enumerate}
\item\claim{[$e'z'$ lies in a true face.]}  In this case, 
\begin{displaymath}
e'z' = f^{-1} n^{-1} [\ldots,u] = f^{-1} [n^{-1}u;\ldots],
\end{displaymath}
so that $n^{-1} u$ is visited by a true contour loop.
Consider the following
path in $H$:
\begin{displaymath}
[y;fy;\ldots;z] @ [n^{-1}z;\ldots;u] @ [n^{-1} u;\ldots;n^{-1} x].
\end{displaymath}
The first segment consists of $f$-steps; the second of $n^{-1}$-steps;
and the third segment exists within true contour loops of ${\cal L}$
by the connectedness of true faces (by properties of flags).  This
path satisfies all the assumptions of Lemma~\ref{lemma:contour-f}.  (In particular,
the node of $n f x$ (or of $f x$)  is distinct from the node of $x$ by the
assumed simplicity of the quotient $H/{\cal L}$.)
The lemma asserts that the path does not exist.
\item 
\claim {[$e'z'\in S$.]}  In this case,  
$f^{-1}n^{-1}u \in S$ and $L$ visits $n^{-1} u$ at the node of $z$.
Consider the following path of $f$-steps in $H$:
\begin{displaymath}
[y;f y;\ldots;z].
\end{displaymath}
This path satisfies all the enumerated conditions of
Lemma~\ref{lemma:contour-f}.  (In particular, $y$ and $z$ are at
different nodes by the assumed simplicity of the quotient $H/{\cal
  L}$.)  The lemma asserts that the path does not exist.
\end{enumerate}
\end{proof}

\begin{lemma}[parameters]\guid{QRDYXYJ}\label{lemma:parameters}
Let $(H,{\cal L},L,x)$ be a
marked hypermap, where $H$ is restricted. Assume that $L$ is canonically false.
Let $m,p,q$ be the natural numbers and let $x,y,z$ be the darts given by
Definition~\ref{def:yz}.  Let $r = \op{card}(\F(L))$.  Then
\begin{displaymath}
0\le p,\quad 0\le m < q < r,\quad m+1 < p+q.
\end{displaymath}
Furthermore, the darts $x$, $y$ belong to different nodes; the darts
$y$ and $z$ belong to different nodes.
\end{lemma}

\begin{proof}
Let $H/{\cal L}=(D',e',n',f')$.  Let $y'$ and $z'$ be the images of $y$ and $z$
in $D'$, respectively.  Let $S'$ be the image of $S(L,x)$ in $D'$.
By definition $m = \op{card}(S(L,x))$.  Both $m$ and $p$ are natural numbers,
so $0\le p$ and $0\le m$. 

\claim{The darts $x$, $y$ belong to different nodes; the darts $y$ and
  $z$ belong to different nodes.}  Indeed, $x$, $y$, and $z$ belong to
the same face.  By the simplicity of $H$, if two of these darts belong
to the same node, then they are equal to each other.  However, $x\ne
y$, for otherwise the subpath $P=[x;f x;\ldots;f^{m+1}x]$ gives a canonically true contour loop, which is contrary to the assumption that $L$ is
canonically false.  Also, $y\ne z$, for otherwise the face of $y$ is
equal to $\{y,f y,f^2 y,\ldots,f^p y\}$.  It follows that $x = f^k y$
is visited by $L$, for some $1<k\le p$.  This contradicts the defining
minimality property of $p$.

\claim{$q<r$.}  Indeed,
by definition, $z' = (f')^{q+1} x'$ and no smaller natural
number has this property.  Also, $x'$ and $y'$ belong to the face $\F(L)$.
If $q\ge r$, then $(f')^{q+1} = (f')^{q-r+1}$, which contradicts the minimality of $q$.

\claim{$m<q$.} Indeed, if $0\le k< m$, then
\begin{displaymath}
(f')^{k+1} x' = [f^{k+1} x]\in S', \quad z' \not\in S',
\end{displaymath}
by Lemma~\ref{lemma:flag-set-quotient} and Lemma~\ref{lemma:yz}.
Thus, $0\le q< m$.  Also, $q\ne m$, for otherwise $z' = (f')^{m+1} x'
= y'$.  This implies that $z$ and $y$ lie in the same node, which has
been proved impossible.  This completes the proof that $m<q$.
 
\claim{$m+1 < p+q$.}  Indeed, the inequalities $0\le p$ and $m<q$ imply
that either $m+1 < p+q$ or $p=0~\land~m+1=q$.  The second disjunct
cannot hold, for otherwise $z' = (f')^{q+1} x' = f' y'$.  Write $y' = [y;\ldots;u]$.
This is not a unit path by the definitions of $S(L,x)$ and $m$, so $y\ne u$, but
they lie at the same node.  Also from $p=0$ it follows that $z= f^{p+1} y = f y$.
So $e y$ and $e u$ both lie at the node of $z'$.  The existence of
 two edges, $\{y, e y\}$ and
$\{u, e u\}$, between two nodes contradicts the hypothesis
on $H$ of no double joins.  This proves the claim and the lemma.
\end{proof}

\subsection{transform}

\begin{definition}[transform]\guid{YQANQNF}
  From one marked hypermap $(H,{\cal L},L,x)$ in which $L$ is
  canonically false, we construct a new tuple
\begin{displaymath}
T(H,{\cal L},L,x) = (H,{\cal M},L_1,x),
\end{displaymath}
 called the \newterm{transform} of
  $(H,{\cal{L}},L,x)$.  
As the notation indicates, the hypermap $H$ and the dart $x$ are the same for both
tuples.  The data ${\cal M}$ and $L_1$ are specified in
the following paragraphs.
\end{definition}

Let $m$, $p$, $y$, and $z$ be  given by
Definition~\ref{def:yz}.
Let $L_1$ be 
%With improvised notation, write
%\begin{displaymath}
%L_1 = \lp{L[x:y] \opat P[y:z] \opat L[z:x]},
%\end{displaymath}
the contour loop in $H$ that follows $L$ from $x$ to $y$, then takes
$f$-steps from $y$ to $z$, then continues along $L$ back to $x$.  
Let $L_2$ be
%\begin{displaymath}
%L_2 = \lp{L[n^{-1}y:n z] \opat P^c[n z:n^{-1}y]},
%\end{displaymath}
the contour loop in $H$ that follows $L$ from $n^{-1} y$ to $n z$,
then complements the path of $L_1$ from $y$ to $z$, traveling instead
from $n z$ to $n^{-1} y$. 


Set
\begin{displaymath}{\cal M} = ({\cal L}\setminus \{L\}) \cup
\{L_1,L_2\}.\end{displaymath}

\begin{remark}[canonical compatibility]\guid{HBLIYVM}
There is a canonical boolean function on ${\cal L}$ and one on ${\cal M}$.
The canonical boolean functions agree on the intersection ${\cal L}\cap {\cal M}$.
This means, there is a well defined boolean-valued function on 
${\cal L}\cup {\cal M}$.  There is no ambiguity.  
\end{remark}
\indy{Index}{function!boolean}%

%  Under the bijection between the set of faces of a quotient $H/{\cal
%    L}$ and the family ${\cal L}$ (of
%  Lemma~\ref{lemma:quotient-bijection}), an $S$-flag on a quotient
%  $H/{\cal L}$ can be identified with a boolean function on ${\cal L}$
%  satisfying appropriate properties.  With this in mind, 
%\begin{displaymath}
%\begin{cases}
%\psi(L) = \varphi(L), &\text{if } L\ne L_1,L_2,\\
%\psi(L_2) = \op{true}, &\text{if $L_2$ is canonically true},\\
%\psi(L) = \op{false}, &\text{otherwise}.\\
%\end{cases}
%\end{displaymath}




\begin{figure}[htb]
\centering
\szincludegraphics[width=80mm]{\pdfp/L1L2.eps}
\caption{The loop $L$ is replaced with two loops $L_1, L_2$.}
\label{fig:L1L2}
\end{figure}



%Each of the two
%loops $L_i$ meets at least two nodes (those
%of $y$ and $z$) and has length at least three by
%Lemma~\ref{lemma:3dart}.  
% $S(H,L_1,x)$ contains the proper subset $S(H,L,x)$.


\begin{lemma}[markup transform]\guid{AQIUNPP}\rating{600}\tlabel{lemma:flag} 
Let $H$ be a restricted hypermap.
If $(H,{\cal L},L,x)$ is a marked hypermap such that $L$
is canonically false,  then the transform
$(H,{\cal M},L_1,x)$ 
is also a marked hypermap.
\end{lemma}

\begin{proof} Let 
\begin{displaymath}
H=(D,e,n,f),\   H' =(D',e',n',f') = H/{\cal
    L},\  H'' = (D'',e'',n'',f'') = H/{\cal M}.   
\end{displaymath}
Let $S'$ be the image of $S(H,L,x)$ in $D'$.  Let $y,z$ be the darts
constructed in Definition~\ref{def:yz} from the marked hypermap $(H,{\cal
  L},L,x)$.  The dart $z$ is not at the same node as $y$ (by
the simplicity of $H$).

  The proof can be organized into independent parts, according to the separate
  properties of a marked hypermap.  The first part of the proof
  establishes that ${\cal M}$ is a normal family.


\case{normal-1} \claim{No dart is visited by two different loops.}
Indeed by construction, the sets of darts of $L_1$ and $L_2$
are disjoint from each other and disjont from the sets of darts of $L'\in
{\cal L}\setminus \{L\}$.  The result now follows from the normality of  ${\cal L}$.

\case{normal-2} \claim{Every loop visits at least two nodes.}  Indeed, this
is true for $L_1$ and $L_2$ because they visit the nodes of $y$ and
$z$.  It is true of the other loops because they belong to the
normal family ${\cal L}$. 

\case{normal-3} \claim{If a loop visits a node, then every dart at
that node is visited by some loop.} 
% Indeed, the new nodes visited on the
%sublist of $L_1$ from $y$ to $z$ do not contain any darts in any
%loop in ${\cal L}$.  (By the definition of normal family, if a loop
%visits a node, then every dart at that node is visited by some loop
%of ${\cal L}$.)  These new nodes contain two darts, with one dart
%along $L_1$ and the other along $L_2$.  The paths $L_1$ and $L_2$
%visit every dart visited by $L$.  They visit every dart at the nodes
%of $y,\ldots,z$.
Indeed, the nodes that are visited by some loop in ${\cal M}$ are
precisely those visited by some loop in ${\cal L}$, together with the
``new'' nodes; that is, the nodes of $f y,\ldots,f^p y$.  The set of
darts that are visited by some loop of ${\cal M}$ is the union of the
set visited by loops in ${\cal L}$, together with two darts at the
new nodes.  As each new nodes has only two darts, and as ${\cal L}$ itself
it normal, the result follows. It follows that ${\cal M}$ is normal.  


\case{simple} To prove the simplicity of the quotient, it is enough to show that
none of the contour
loops in ${\cal M}$  ever return to a node after leaving it.
 (More precisely, the dart set of any $L'\in{\cal M}$ intersected with a node
is the dart set of a maximal sublist $[z;n^{-1}z;\ldots;n^{-k}z]$ of $n^{-1}$ steps.)
This is true of $L'\in{\cal L}\setminus\{L\}$ by assumption and true of
$L_1$ and $L_2$ by construction.  Simplicity follows.

\case{fixed-point free} By Lemma~\ref{lemma:nfp}, to prove that $n''$
does not have a fixed-point, it is enought to show that no loop in
${\cal M}$ has a dart set containing a node.  It is sufficient to
consider the loops $L_1$ and $L_2$.  The set of darts of $L_1$ and
$L_2$ at the old nodes (that is, those not meeting $\{f y,\ldots,f^p
y\}$) are subsets of the set of darts of $L$ at those nodes.  As the
dart set of $L$ does not contain an old node, neither do $L_1$ and
$L_2$.  At the new nodes, $L_1$ and $L_2$ both have at least one dart,
so neither contains the entire node.  It follows that $n''$ is
fixed-point free.

\claim{$e'y'\not\in S'$, where $y'$ is the image of $y$ in $H'$. }
Otherwise, write $y' = [y;n^{-1}y;\ldots;u]$, a sublist of $L$, and
pick $k$ such that $e'y' = [f^k x]\in S'$.  Then
\begin{displaymath}
(n')^{-1} y' = f'e'y' = f'[f^k x] = [f^{k+1}x;\ldots].
\end{displaymath}
By the construction of $S(H,L,x)$, we know that $L$ visits $f^{k+1}x$.
Hence $y'$ and $(n')^{-1}y'$ both lie in the same node and in the same
face $\F(L)$.  By the simplicity of $H'$, it follows that $y' =
(n')^{-1} y'$.  That is, $y'$ is a fixed point of $n'$.  This is contrary to
assumption. The claim follows.


\claim{$e'y'$ lies in a true face of $H'$.}  Indeed,
 since $\check\varphi_{can}$ is
an $S'$-flag on $H'$, the edge $\{y',e'y'\}$ meets a true face or
$S'$.  However, $y'\not\in S'\subset \F(L)$, by the simplicity of $H$.  Also,
$e'y'\not\in S'$, by the previous paragraph.
The dart $y'$ lies in the false face $\F(L)$.  The only
remaining possibility is that $e'y'$ lies in a true face.  



\case{flag-1} \claim{The true faces of $H''$ are
  connected.}  Indeed,  $L_1$ is connected to a true face
by the contour path $[x;n^{-1} x]$, because $n^{-1} x$ lies
in the same face as $n f x$, which is a true face by assumption.
If $L'\in{\cal M}\setminus\{L_1,L_2\}$ is a true, then $L'\in  {\cal L}$, and
it connects with the true faces of ${\cal L}$ as before.
If $L_2$ is true, then the proof requires more
argument.   Write $y'=[y;\ldots;u]$ as above.
The dart $u''=[n^{-1}y;\ldots;u]$ of $H''$ lies in the face $\F(L_2)$.
Also, 
\begin{displaymath}
(n'')^{-1} u''= [n^{-1} u;\ldots] = [f e u;\ldots] = f'' [\ldots; e u]
  = f'' e' y'
\end{displaymath}
Thus, we have a contour path from $u''$ to $e'' y'$, which lies in a
true face, by an earlier claim.  (The dart $e'y'$ is naturally
identified with the dart $e'' y'$ on in $H''$, because the faces are
true.)  Hence a path exists from the true face $\F(L_2)$ into another
true face, and from there any true face may be reached.

\case{flag-2} \claim{Each edge of $H''$ meets a true face or $S''$, where $S''$ is
the image of $S(H,L_1,x)$ in $D''$.}
Indeed, the function $\check\varphi_{can}$ is an $S'$-flag on $H'$.  The edges of
$H'$ can be identified with a subset of the edges of $H''$.  For this
subset, the flag condition on edges is immediate.  Consider the
new  edges  (that is, edges of $H''$ that are not in $H'$).
They all meet $\{y,f y,\ldots,f^p y\}$.  This set is either contained
ini $S(H,L_1,x)$ (when $F_1$ is false), or is contained in the true face
$F_1$ (when $F_1$ is true).  Hence each new  edge meets
a true face or $S''$.


The other verifications are routine.
\end{proof}


\subsection{digraph}

The aim is to prove that every restricted hypermap with a given bound
on the size of the dart set is generated by a particular algorithm.
The proof is a long induction argument.  The proof starts by showing
how to go from one partially constructed hypermap to another more
fully constructed hypermap.  The hypermap $H$ represents the fully
constructed hypermap and two quotients $H/{\cal L}$ and $H/{\cal M}$
represent the partially constructed hypermaps.  The algorithm involves
the transition from the hypermap $H/{\cal L}$ to $H/{\cal M}$.  The
transition from one quotient to another is given by the transform of
marked hypermaps.  The transform thus represents one step of the
algorithm.  \indy{Notation}{M@$\cal M$ (normal family)}%


\begin{definition}[digraph,~vertex,~edge,~head,~tail,~sink,~path]\guid{AVXKIRW}
  A \newterm{digraph} (directed graph) is an ordered pair $(V,E)$
  where $V$ is any set, and $E$ is a set of ordered pairs of vertices.
  An element of $V$ is called a \newterm{vertex}.  An element of $E$
  is called a \newterm{directed edge}.  If $(v,w)\in E$, then $v$ is
  the \newterm{head} and $w$ is the \newterm{tail} of the directed
  edge.  A vertex $v$ is a \newterm{sink} if it is not the head of any
  directed edge.  A path $P=[v_0;v_1;\ldots;v_{k-1}]$ in a digraph is
  a list of vertices, such that $(v_i,v_{i+1})\in E$ for all $i<k-1$.
\end{definition}

\begin{definition}[digraph of a hypermap]\guid{QQHIKFL}
Let $H$ be a restricted hypermap.  Form a directed graph as follows.
The vertex set $V$ of the digraph is the set of all marked hypermaps 
${\cal H}=(H,{\cal L},L,x)$ such that $L$ is canonically false.  
Write $T{\cal H} = (H,{\cal M},L_1,x)$, where $T$ is the transform on $V$.  
The set of tails of directed edges with head ${\cal H}$ is as follows:
\begin{itemize}
\item If every contour loop of ${\cal M}$ is canonically true, then ${\cal H}$ is a sink.
\item If $M$ is canonically false, 
then there is a single tail $T{\cal H}$.
\item If $M$ is canonically true, but not every loop in ${\cal M}$ is canonically true,
then the tails are $(H,{\cal M},M,y)$, where $M$ is a 
canonically false loop in ${\cal M}$ and $y$ is a dart visited by $M$ such
that $y$ is followed by an $f$-step:  $M = \lp{y;f y;\ldots}$.
\end{itemize}
\end{definition}

\begin{lemma}[digraph sink]\guid{XCOXWYJ}\label{lemma:digraph-sink}
Let $(V,E)$ be the digraph of a restricted hypermap $H=(D,e,n,f)$.  Then every path
in $(V,E)$ reaches a sink after at most $\#D$ steps.  Moreover, if ${\cal H}$ is
a sink, and if $T{\cal H} =(H,{\cal M},L_1,x)$ is its transform, then
$H$ is naturally isomorphic to $H/{\cal M}$, under the map that sends
a dart $y$ of $H$ to the unit path $[y]$.
\end{lemma}

\begin{proof} Each step in the path makes one transform.  Each
  transform increases the number of darts visited by the normal
  family.  The number of darts visited by a normal family is bounded
  by $\#D$.  This bounds the length of a path.

  The condition on a sink ${\cal H}$ is that every contour loop of
  ${\cal M}$ is canonically true. 

\claim{${\cal M}$ visits every dart.}  Indeed, let $y$ be any dart of
$H$. Since $H$ is assumed connected, there exists some contour path
$P=[x;\ldots;y]$.  Let $u$ be the last dart on the path that is
visited by ${\cal M}$.  Let $M\in{\cal M}$ be the contour loop that
visits $u$.  Then $u$ is not followed by an $f$-step, because every contour
loop is canonically true: $M = \lp{u;f u;\ldots}$.
Nor is $u$ followed by a $n^{-1}$ because in a normal family every dart
at the node of $u$ is visited by a loop in ${\cal M}$.  Thus, $u$ is the final
dart in the path $P$, which means that $y$ is visited by ${\cal M}$.

It follows that is a bijection between the dart set of $H$ and the
dart set of $H/{\cal M}$, sending a dart $y$ to the unit path $[y]$.
This bijection induces an isomorphism of hypermaps.
\end{proof}


%In the next lemma there are a number of choices to be made.  Let
%$\op{ch}$ be any function on the set of normal families ${\cal L}$ of $H$ containing
%at least one canonically false loop, returning a dart $x'$ of $H/{\cal L}$
%in a false face (with respect to the canonical boolean function).
%Given a normal family ${\cal L}$ (with a canonically false loop),  write
% $x' = [\ldots;x]$ and let $L$ be the the canonically false contour loop of ${\cal L}$ that
%visits $x'$.  Write ${\cal L}\rightsquigarrow (L,x)$.   The dart $x$ of $H$ has the property
%that $L = \lp{x;f x;\ldots}$; that is, $x$ is followed by an $f$-step in $L$.
%
%
%\begin{lemma}[sequence]\guid{YCFWVQJ}\label{lemma:sequence}  
%  Let $H$ be a restricted hypermap.  Fix a choice function $\op{ch}$ as
%  above, and let $F$ be any face of $H$. Then associated with $(\op{ch},F)$, there is a sequence of
%  marked hypermaps ${\cal H}_i = (H,{\cal L}_i,L_i,x_i)$ for
%  $i=0,\ldots,k-1$ such that
%\begin{itemize}
%\item $\varphi'_i$ is the canonical boolean function on $H/{\cal L}_i$, and
%$\varphi_i$ is its lift to ${\cal L}_i$.
%\item ${\cal L}_0$ is the normal family associated with the given face $F$,
%described in Example~\ref{ex:H2}, and ${\cal L}_0 \rightsquigarrow (L_0,x_0)$.
%\item If the contour loop $L$ in the transform $T{\cal H}_i=(H,{\cal M},L,\ldots)$ 
%is canonically false, then ${\cal H}_{i+1} = T{\cal H}_i$.
%\item If the contour loop $L$ in the transform $T{\cal H}_i=(H,{\cal
%    M},L,\ldots)$ is canonically true, and there exists some other
%  contour loop of ${\cal M}$ that is canonically false, then ${\cal
%    H}_{i+1}=(H,{\cal M},M,y,\varphi_{i+1})$, where ${\cal M}\rightsquigarrow (M,y)$. 
%\item The transform $(H,{\cal L}_k,\ldots)$ of $\,{\cal H}_{k-1}$ has
%  quotient $H/{\cal L}_k$ that is naturally isomorphic to $H$.
%\end{itemize}
%\end{lemma}
%
%\begin{proof}
%  Let ${\cal H}_0=(H,{\cal L}_0,L_0,x_0,\varphi_0)$ be given as
%  follows.  Pick any face $F$ of $H$, and construct the normal family
%  ${\cal L}$ of Example~\ref{ex:H2}.
%
%  The construction of ${\cal H}_{i+1}$ depends on the structure of the
%  transform $(H,{\cal M},L,x)$ of ${\cal H}_i$.  We consider
%  three cases.
%\begin{nomerate}
%\item 
%\claim{[Every contour loop of ${\cal M}$ is canonically true.]}   In this case,  by
%Lemma~\ref{lemma:all-dart}, the quotient map $H\to H/{\cal M}$ is an
%isomorphism.  Set $k= i+1$. The sequence terminates.
%\item \claim{[The contour loop $L$ is canonically false.]}  In this
%  case, from the inductive hypothesis that $\varphi'_i$ is the
%  canonical function and the definition of $\psi$, it follows that
%  $\psi'$ is the canonical function.  By Lemma~\ref{lemma:flag}, the
%  transform is a marked hypermap.  Let ${\cal H}_{i+1}$ equal the
%  transform of ${\cal H}_i$.
%\item \claim{[The contour loop $L$ is canonically true, but not all contour loops in
%     ${\cal M}$ are canonically true.]}  In this
%  case, the canonical function on $H/{\cal M}$ is a flag (with
%  $S=\emptyset$).   
%${\cal H}_{i+1}$, as defined, is a marked
%  hypermap.
%\end{nomerate}
%
%The sequence must terminate eventually, because each step
%constructs a quotient of $H$ with more darts than the previous one and
%the number of steps is bounded by the size of the dart set of
%$H$.
%\end{proof}

Next, we wish to describe the directed edges in a way that relies to a
lesser degree on the structural details of the marked hypermaps.
(These details will not be available to us in the algorithm of the
next subsection.)  The next lemma uses reverse doble walkup
transformations to construct a new hypermap from a given hypermap $H'$
that does not require us to represent it first as a quotient $H' =
H/{\cal L}$ for some normal family.
We can immediately relate the to the digraph we
have constructed.   

\begin{lemma}[walkup-digraph]\guid{ISMLATS}\label{lemma:RDW}
Let ${\cal H} =(H,{\cal L},L,x)$ be a marked hypermap,
  with $H$ restricted.  Assume that $L$ is canonically false, and let 
$(H,{\cal{M}},M,x)$ be 
  the transform of $\cal H$.   
Let $m$, $p$, $q$, $y$, and $z$ be
  the constants of Definition~\ref{def:yz}.  Let $x'$ be the
  image of the dart $x$ in $H/{\cal L} = (D',e',n',f')$.
%Then $m,p,q$ satisfy the constraints of
%  Definition~\ref{def:R}, and 
Then $H/{\cal M}$ is isomorphic to $RDW(H/{\cal{L}},x',m,p,q)$.
\end{lemma}


\begin{proof}
%The constraints on $(m,p,q)$ of Definition~\ref{def:R} are satisfied 
%by Lemma~\ref{lemma:parameters}.
%
  By construction, the passage from $H/{\cal M}$ to $H/{\cal L}$
  consists of a double walkups to eliminate the nodes (of size two) at
  $f y$, $f^2 y, \ldots, f^p y$, and then a double walkup to eliminate
  the edge that runs from the the node of $y$ to the node of
  $z$.  %The passage in the other direction from $H/{\cal L}$ to
%$H/{\cal M}$ comes by adding an edge from the node of $y$ to
%the node of $z$ and inserting $p$ new nodes (of degree two)
%along it.
If we play these double walkups in reverse,
one may also pass from $H/{\cal L}$ to $H/{\cal M}$.  
%If
%$m,p,q$ are chosen as above, 
Then $RDW(H/{\cal L},x',m,p,q)$ is isomorphic to
$H/{\cal M}$.  
\end{proof}


\begin{figure}[htb]
\centering
\szincludegraphics[width=80mm]{\pdfp/L1L2dart.eps}
\caption{$H/{\cal L}$ is obtained from $H/{\cal M}$ by double walkup
transformations.}
\label{fig:L1L2dart}
\end{figure}



\subsection{algorithm}

This final section puts the algorithm a precise form, based on
the Knaster-Tarski fixed point theorem.  The Knaster-Tarski fixed
point theorem is a common way to give precise mathematical form to
an algorithm.

\begin{lemma}[Knaster-Tarski]\guid{EAOGWLE}\rating{ZZ}   
Let $X$ be a set.  Let $f:\powerset(X)\to \powerset(X)$ be a
function from the powerset of $X$ to itself.  Assume that $f$ is
monotonic in the sense that whenever $Y\subset Z\subset X$, it
follows that $f(Y) \subset f(Z)$.  Then $f$ has a least fixed point.
That is there exists a set $\op{fix}(f,X)\subset X$ such that
$f(\op{fix}(f,X)) = \op{fix}(f,X)$ and such that the following
minimality condition holds: if $Y\subset X$ is any set such that
$f(Y) \subset Y$, then $Y\subset \op{fix}(f,X)$.
\end{lemma}
\indy{Notation}{X@$X$ (set)}%
\indy{Notation}{f@$f$ (function on powerset)}%
\indy{Notation}{fix@$\op{fix}$~(Knaster-Tarski fixed point)}%
\indy{Notation}{P@$\powerset(\cdot)$ (powerset)}%
\indy{Index}{Knaster-Tarski fixed point theorem}%

\begin{proof} Let $\op{fix}(f,X)$ be the intersection of all subsets
$Y$ of $X$ such that $f(Y)\subset Y$.  It is easily verified that
this set has the required properties.
\end{proof}

Various data are needed for the application of the Knaster-Tarski
fixed-point theorem to the construction of restricted hypermaps.  This
data is presented in a series of definitions.  The first definition
gives a domain $\Omega$ that will contain all the darts of all the
hypermaps that will be constructed by the algorithm.  In practice, we
take $\Omega$ to be a finite subset of the set of natural numbers.

\begin{definition}[$\Omega$,~$\op{ch}$,~$d$]\guid{LVXTTSP}
  Let $\Omega$ be any fixed finite set.  Fix a choice function
  $\op{ch}:\powerset(\Omega)\to \Omega$ that picks an element from
  each nonempty subset:
\begin{displaymath}
X\ne\emptyset\quad  \Rightarrow \quad  \op{ch}(X)\in X.
\end{displaymath}
For example, when
$\Omega$ is a well-ordered set, let $\op{ch}$ choose the least element of a subset
of $\Omega$.  An $(\Omega,d)$-hypermap is a hypermap whose dart set is a
subset of $\Omega$ and such that $d$ is the maximum face size.  A
$(\card(\Omega),d)$-hypermap is one isomorphic to an $(\Omega,d)$-hypermap.
(In applications later in the book, $\card(\Omega)\le14$ and $3\le d\le 6$.) 
\end{definition}
\indy{Notation}{d3@$d$ (upper bound)}%
\indy{Notation}{zzZ@$\Omega$ (set of darts)}%
\indy{Notation}{ch@$\op{ch}$~(choice)}%


The following constructions depend on $\Omega$ and $d$, although the
notation does not reflect this.  One may think of $X_1$ in the next
definition as holding the output of the algorithm and $X_2$ as the
workspace that holds partially constructed hypermaps.

\begin{definition}[$X$,~$X_1$,~$X_2$]\guid{KFPEPWO}
Define a set $X$ as the disjoint union of $X_1$ and $X_2$ as follows.
Let $X_1$ be the set of all $(\Omega,d)$-hypermaps.
Let $X_2$ be the set of tuples $(H,m,\check\varphi,x)$, where 
\begin{itemize}
\item $H$ is an $(\Omega,d)$-hypermap,
\item $\check\varphi$ is an $S$-flag on $H$,
\item  $x$ is a dart in a false face of $H$ (with respect to $\check\varphi$),
\item $m\in\{0,\ldots,d-1\}$, 
and
\item $S = \{f^i x\mid 1 \le i \le m\}$.
\end{itemize}
\end{definition}


The set $A$ in the following definition gives the initialization of the algorithm.

\begin{definition}[A]\guid{JBUOJMF}
Let $H$ be a fixed hypermap isomorphic to $H_{2d}$, with darts in $\Omega$.
Let $\check\varphi$ be the flag on $H$ (with one true face and one
false face).  Let $x$ be the value of the choice function on the false
face.  Set
\begin{displaymath}
A = \{(H,m,\check\varphi,x) \mid 0\le m \le d-1\} \subset X_2.
\end{displaymath}
\end{definition}

The following set gives the indexing set for the iteration of the algorithm.

\begin{definition}[C]\guid{IDDKWYX}
Let $C$ be the set of of $4$-tuples $(m,p,q,r)$ that satisfy the following
constraints:
\begin{itemize}
\item $0\le m < q < r$.
\item $0\le p$.
\item $m+1 < p+q$.
\item $m+p+2 \le d$.
\item if $q+1< r$, then $m+p+3\le d$.
\end{itemize}
\end{definition}

\begin{definition}[extension]\guid{ZMVBANY}  
  Let $(H,m,\check\varphi,x)\in X_2$.  Choose $p,q$ such that $(m,p,q,r)\in
  C$, where $r$ is the cardinality of the face of $x$.  Let $F$ be the
  face of $x$ in $H$.  It follows by construction, that every face
  $F'\ne F$ of $H$ is naturally identified with a face of
  $RDW(H,x,m,p,q)$.  Say that a boolean-valued function $\check\psi$ on the
  set of faces of $RDW(H,x,m,p,q)$ is an \newterm{extension} of
  $\check\varphi$ on $H$ if $\check\psi(F') =\check\varphi(F')$, when $F'\ne F$.
  \indy{Index}{extension}%
\end{definition}


The functions $f$ and $g$ give one iteration of the algorithm.  The
powerset-valued function $g$ is the heart of the algorithm.  It takes
one partially constructed hypermap and modifies it in various ways to
construct further partially constructed hypermaps.  When the flag
$\check\varphi$ permits, some hypermaps are also fully constructed.


\begin{definition}[g]\guid{DMAMRYR}
 Let $g:X_2 \mapsto \powerset(X)$ be given as
  follows.  Let $r$ be the cardinality of the face $F$ of $x$.  The
  subset $g(H,m,\check\varphi,x)\subset X$ is presented as a union of two
  sets:
\begin{displaymath}
   Y_i = X_i \cap g(H,m,\check\varphi,x).
\end{displaymath}
 $H'\in Y_1$ if and only 
\begin{itemize}
\item there exists
$p,q$ with $(m,p,q,r)\in C$ such that $H'=RDW(H,x,m,p,q)$,
and 
\item $\check\varphi(F')$ is true for all $F'\ne F$.
\end{itemize}
 $(H',m',\check\psi,x')\in Y_2$ if and only if
\begin{itemize}
\item there exists $p,q$ with $(m,p,q,r)\in C$, such that $H'=RDW(H,x,m,p,q)$,
\item  $\check\psi$ is an extension of
$\check\varphi$, and 
\item Let $F_1$ be the face of $x$ in $H'$.  One of the following two
  conditions hold:
\begin{itemize}
\item $\check\psi(F_1)$ is false;  $x' = x$; and  $p+m+1 \le m' < r$.
\item $\check\psi(F_1)$ is true; there exists a false face in $H'$; $x'$ is
  the value of the choice function on the union of false faces of
  $H'$; and $0 \le m' < r$.
\end{itemize}
\end{itemize}
\end{definition}


\begin{definition}[f]\guid{YSJTEDX}
Given the function 
$g:X_2 \to \powerset(X)$, set 
\begin{displaymath}f(S) = A \cup (\bigcup \{g(s) \mid s\in S\cap
X_2\}).\end{displaymath}
\indy{Notation}{g@$g$(function)}%
\end{definition}

Any function $f :\powerset(X)\to \powerset(X)$ of this form is
monotonic.  Thus, we have a Knaster-Tarski fixed point set
$\op{fix}(f,X)$.  The main result of this chapter is that a fixed
point construction generates all restricted hypermaps:

\begin{theorem}[hypermap algorithm]\guid{BRGEFNH}\rating{2000}  
\label{lemma:algorithm}
Define $f $ and $X$ as above (depending on $d\ge 3$ and $\Omega\ne
\emptyset$) .    Then every restricted hypermap with at
most $\card(\Omega)$ darts and such that the largest face has size  $d$
is isomorphic to a hypermap in $\op{fix}(f,X)\cap X_1$.
\end{theorem}


In informal terms, by starting with the \newterm{seed} hypermaps in $A$
one may find all restricted hypermaps (for given $\Omega$ and $d$) by
applying the function $f$ repeatedly:
\begin{displaymath}
A_0 = A = f(\emptyset),\quad A_1 = f(A_0),\quad A_2 = f(A_1),\ldots
\end{displaymath}
and by looking at the output $A_i \cap X_1$.
\indy{Index}{seed}%

The proof will be presented below.  The proof is a matter of correlating the Knaster-Tarski fixed point set with the digraph of a restricted hypermap $H$.
Write $\op{fix}_i$ for $\op{fix}(f,X)\cap X_i$.


\begin{definition}[correlation]\guid{SFBFNVW}
  A marked hypermap $(H,{\cal L},L,x)$ is said to be
  \newterm{correlated} to an element $(H',m',\check\varphi,x')\in X_2$ if
  the following conditions hold:
\begin{itemize}
\item $H/{\cal L}$ is isomorphic to $H'$ by some isomorphism $G$.
\item The image of $x$ in $H/{\cal L}$ maps to $x'$ under $G$.
\item The pull back of $\check\varphi$ under $G$ is the canonical function
$\check\varphi_{can}$ on $H/{\cal L}$.
\item $m = \card(S(H,L,x))$.
\end{itemize}
\end{definition}

\begin{lemma}[correlated seed]\guid{NRDWGYQ}\label{lemma:correlated-seed}
  Let $H$ be a restricted $(\card(\Omega),d)$-hypermap with digraph
  $(V,E)$.  There exist a marked hypermap in
  $V$ and an element in $\op{fix}_2$ that are
  correlated.
\end{lemma}

\begin{proof}  From the definition of $f$, it follows
that $A\subset \op{fix}_2$.  Thus, it suffices to correlate a marked
hypermap with an element of $A$.  Let $(H',\cdot,\check\varphi,x')\in A$.

Let $F$ be a face of $H$ of cardinality $d$.  Form the normal family ${\cal L}$ of example~\ref{ex:H2}.  The quotient $H/{\cal L}$ is isomorphic to $H_{2d}$, hence also isomorphic to $H'$.  The isomorphism can be chosen so that
$\check\varphi$ pulls back to $\check\varphi_{can}$ on the faces of $H/{\cal L}$, and so that $x'$ is the image of a dart $x$ in the canonically false 
face of $H/{\cal L}$.  

Let $m = \card(S(H,L,x))$.  Then $(H,{\cal L},L,x)$ is a marked hypermap that
is correlated with $(H',m,\check\varphi,x')\in A$.
\end{proof}

\begin{lemma}[correlated edge]\guid{FDQZOSJ}\label{lemma:correlated-edge}
  Let $H$ be a restricted $(\card(\Omega),d)$-hypermap with digraph $(V,E)$. 
  Assume that ${\cal H}\in V$ is not a
  sink and is correlated with an element in $\op{fix}_2$.
  Then there exists a directed edge  $({\cal H},{\cal H}')\in E$, such that
 $\cal H'$ is correlated with an element of  $\op{fix}_2$.
\end{lemma}

\begin{proof}  Assume that the marked hypermap ${\cal H}=(H,{\cal L},L,x)$
is correlated with the tuple $(H',m,\check\varphi,x')\in \op{fix}_2$,
by means of an isomorphism
\begin{displaymath}
G: H/{\cal L} \to H'.
\end{displaymath}
Let $T{\cal H} = (H,{\cal M},L_1,x)$ be the transform.  Let $m,p,q,x,y,z$
be the parameters of Definition~\ref{def:yz}.  Let $r$ be the cardinality
of $\F(L)$, which is equal to the cardinality of the face of $x'$ in $H'$.  
Then $(m,p,q,r)\in C$ by Lemma~\ref{lemma:yz}.

Let $H'' = RDW(H',x',m,p,q)$.  The isomorphism $G$ and Lemma~\ref{lemma:RDW} combine to give an isomorphism $G':H/{\cal M} \mapsto H''$.
Push the canonical function on the faces of $H/{\cal M}$ to a function
$\check\psi$ on the faces of $H''$.

If $L_1$ is false, then $({\cal H},T{\cal H})$ is a directed edge of the digraph, and
$T{\cal H}$ is correlated with $(H'',m',\check\psi,x')\in g(H',m,\check\varphi,x')\subset \op{fix}_2$, where $m'=\card(S(L_1,x))$.  

If $L_1$ is true, then $H''$ has a false face, and the choice function $\op{ch}$ picks a dart $x''$ in the union of the false faces of $H''$.  Transport this by $G$ to a dart $y'\in H/{\cal M}$ in a false face.  Write $y'=[\ldots;y]$
with $y$ a dart visited by a false contour loop $M$ of ${\cal M}$.  Then
$(H,{\cal L},L,x),(H,{\cal M},M,y))$ is a directed edge of the digraph,
whose tail is correlated with $(H'',m',\check\psi,x'')\in g(H',m,\check\varphi,x')\subset \op{fix}_2$, where $m'=\card(S(M,y))$.
\end{proof}

\begin{lemma}[correlated sink]\guid{RIZGJVS}\label{lemma:correlated-sink}
Let $H$ be a restricted $(\card(\Omega),d)$-hypermap with digraph $(V,E)$.
Then some sink in the digraph is correlated with some element of
$\op{fix}_2$.
\end{lemma}

\begin{proof} Start with any  correlated pair $({\cal
    H}_0,{\cal K}_0)$ with ${\cal H}_0\in V$ and ${\cal K}_0\in \op{fix}_2$ 
  (Lemma~\ref{lemma:correlated-seed}).  Use
  Lemma~\ref{lemma:correlated-edge}, to produce a sequence $({\cal
    H}_i,{\cal K}_i)$ of correlated pairs, where $[{\cal H}_0;{\cal
    H}_1;\ldots]$ is a path in the vertex set $V$.  By Lemma~\ref{lemma:digraph-sink},
  the path reaches a sink within $\#D$ steps.  The final marked
  hypermap ${\cal H}_k$ in the path is a sink that is correlated with
  ${\cal K}_k\in X_2$.
\end{proof}

\begin{proof} Turn to the proof of Theorem~\ref{lemma:algorithm}.  Let
  $(H,{\cal L},L,x)$ be a sink that is correlated with some tuple
  $(H',m,\check\varphi',y')\in \op{fix}_2$
  (Lemma~\ref{lemma:correlated-sink}).  By
  Lemma~\ref{lemma:digraph-sink}, $H$ is isomorphic to the quotient
  $H/{\cal M}$, where $(H,{\cal M},\ldots)$ is the transform of the
  sink.  This quotient is isomorphic to $\op{RDW}(H/{\cal
    L},x',m,p,q)$, where $x'$, $m$, $p$, $q$ are given by
  Definition~\ref{def:yz}.  By the correlatedness of the sink, $H$ is
  isomorphic to $H''=\op{RDW}(H',x',m,p,q)$.  By
  Lemma~\ref{lemma:parameters}, $(m,p,q,r)\in C$, where
  $r=\op{card}(\F(L))$.  Under the isomorphism, $r$ is the cardinality
  of the face of $y$ in $H'$.  By the definition of $f$ and $g$,
  $H''\in \op{fix}_1$.
\end{proof}





%
%
%\begin{proof} Let $H$ be a restricted hypermap whose dart set belongs
%  to $D$ and whose greatest face size is $d$.  It has a quotient
%  $H/{\cal L}_0$ isomorphic to $H_{2d}$.  Let ${\cal L}_i = (H,{\cal
%    L}_i,L_i,x_i,\check\varphi_i)$ be the sequence of marked hypermaps
%  constructed in Lemma~\ref{lemma:sequence}.  XX.
%\end{proof}
%
%\begin{proof} Let $H$ be a restricted hypermap whose dart set belongs
%to $D$ and whose greatest face size is $d$.  It has a quotient
%$H/{\cal L}_0$ isomorphic to $H_{2d}$.  By repeating the
%construction of Section~\ref{sec:face-insert}, one obtains a
%sequence of hypermaps $H_i = H/{\cal L}_i$, $i=0,\ldots,N$,
%terminating with a hypermap $H/{\cal L}_N$, which is isomorphic to
%$H$.  The data $m_i,\check\varphi_i,x_i$ is also obtained for each $H_i$.
%
%The tuple $(H_0,m_0,\check\varphi_0,x_0)$ is isomorphic to an element of
%$A$.  By construction, $A\subset \op{fix}(f,X)$.  If the lemma is
%false, there is a smallest $i>0$ for which
%$(H_i,m_i,\check\varphi_i,x_i)\not\in \op{fix}(f,X)$ (or if $i=N$, for which
%$H\not\in \op{fix}(f,X)$).  There are isomorphisms
%$RDW(H_i,x_i,m_i,p_i,q_i) \simeq H_{i+1}$ for appropriate choices of
%$p_i,q_i$.  By construction, when the data belongs to $\op{fix}(f,X)$
%for $i-1$, the data belongs to $\op{fix}(f,X)$ for $i$.  Thus, $H$
%belongs to $\op{fix}(f,X)\cap X_1$.
%\end{proof}
%
%\subsection{old algorithm}
%
%If a hypermap is restricted and $x$ is any dart, then $x$ and $n x$
% lie on different faces.  In particular, a restricted hypermap has at
% least two faces.  To begin the process, take ${\cal L}$ to be the
% normal family of Example~\ref{ex:H2} with two contour loops, whose
% quotient hypermap is a polygon $H_{2d}$.  When the the initial
% contour loop is chosen on a face of maximal size, the natural number
% $d$ is an upper bound on size of a face.
%
% In summary, a process starts with a single polygon and then adds
% edges and nodes of degree two along the inserted edges, to obtain a
% restricted hypermap $H$.
%
% A modification of the process avoids explicit reference to the
% hypermap $H$ and to the normal family ${\cal L}$.  Let $D$ be a
% finite set that contains all the darts for all of the restricted
% hypermaps to be constructed.  For each $d=3,\ldots,\# D$, the
% process generates all restricted hypermaps with greatest face-size
% $d$, with darts in $D$.
%
% The algorithms performs the following initialization.  The initial
% hypermap is the polygonal hypermap $H_{2d}$.  A flag $\varphi$ marks
% one face true and the other false.  A distinguished dart $x$ is
% selected on the false face.  For each $m<d$, set $S=S_m = \{f^i
% x\mid 1\le i\le m\}$.
%
% Each iteration processes a finite list ${\cal H}$ of quadruples
% $(H,m,\varphi,x)$, where $H$ is a simple hypermap, $\varphi$ is an
% $S$-flag, $x$ lies in a false face, and $S = \{f^i x\mid 1\le i\le
% m\}$ for some $m$.  The algorithm terminates when every face of
% every hypermap in ${\cal H}$ is true.  At every step of the
% algorithm, one quadruple with some false face is removed from ${\cal
%   H}$ and finitely many quadruples are returned to ${\cal H}$.
%
% At each iteration the chosen $(H,m,\varphi,x)$ is modified in the
% following ways and each modification is placed back in the list
% ${\cal H}$.  As the natural number $m$ depends on the unknown normal
% family ${\cal L}$, all possible $m < d$ are used.  Similarly, the
% natural number $p < d$ depends on ${\cal L}$, and all possible $p$
% are used.  Any quadruple that is isomorphic to one previously
% considered is discarded as redundant.
%
% In summary, the algorithm constructs of hypermaps.  The process must
% terminate, because the set $D$ is finite, so there are only finitely
% many quadruples (or quadruples up to isomorphism) that construct
% their dart sets from $D$.
%
%\begin{lemma}[]\guid{BRGEFNH}\rating{2000}  Fix $d\ge 3$ and $D\ne \emptyset$.
%  This process constructs in a finite number of steps all restricted
%  hypermaps, up to isomorphism, such that the dart set belongs to the
%  finite set $D$, and whose greatest face size is $d$.
%\end{lemma}




%%%%%%%%%%%%%%%%%


    %\lll
    %
\chapter{Fan}\label{sec:fan}

A fan is a geometric object that gives a bridge between sphere packings and hypermaps.  A fan is a particular geometric realization of a hypermap.  It is closely related to the notion of planar graph, but relates more directly to sphere packings.  This chapter does not  assume the Jordan curve theorem or Euler formula for planar graphs.  The main result of this chapter gives a simple version of the Euler formula, 
which implies that the hypermap of a fan is planar.

In this chapter, we fix a point $\orgn\in\ring{R}^3$, which serves as the origin.  It does no harm to assume, in fact, that $\orgn =0\in\ring{R}^3$.

If $S$ is a set of points,
set
  $$
  \begin{array}{lll}
  C(S) &= \op{aff}_+(\orgn,S)\\
  C^0(S) &= \op{aff}^0_+(\orgn,S)\\
  \end{array}
  $$

\begin{definition}[fan]  
Let $(V,E)$ be a pair consisting of a set $V\subset \ring{R}^3$ and a set of pairs of elements of $V$.  The pair is said to be
a {\it fan\/} if the following conditions hold.
    \begin{itemize}
    \item $V$ is finite.
    \item $\orgn\not\in V$.
    \item If $\e \in E$, then $\{\orgn\}\cup \e$ is not a collinear set.
    \item Let $\Sigma = E \cup \{\{v\}\mid v\in V\}$.
    For all $\sigma,\sigma'\in \Sigma$, 
 $$C(\sigma)\cap C(\sigma') = C(\sigma\cap \sigma').$$
    \end{itemize}
When $\e\in E$, call $C^0(\e)$ or $C(\e)$ a {\it blade\/} of the fan.
\indy{Index}{blade}
\indy{Index}{fan}
\end{definition}


\begin{note}%XX old fan definition.
We repeat an older version of the definition (in use between May 15 2009 and June 18 2009).
\begin{definition}[retro-fan]  Let $(\orgn,V,E)$ be a triple consisting of a point,
a set of
points, and a set of pairs of elements of $V$.  The triple is said to be
a {\it retro-fan\/} if the following conditions hold.
    \begin{itemize}
    \item $V$ is finite and nonempty.
    \item $\orgn\not\in V$.
    %\item Each element of $E$ has two elements.
    \item For each $v\in V$, the set
        $$
        %% WW changed notation from E_v to E(v) to allow deformations E_t
        E(v) = \{w\in V\mid \{v,w\}\in E\}
        $$
        is cyclic with respect to $(\orgn,v)$.
    \item For each $\e\in E$, $V\cap C^0(\orgn,\e)=\emptyset$.
    \item For sets $\e,\e'\in E$,   we have
        $$C^0(\orgn,\e) \cap C^0(\orgn,\e')\ne\emptyset\ \Rightarrow (\e = \e').$$
    \item For $v,v'\in V$, we have
     $$\op{aff}^0_+(\orgn,v) = \op{aff}^0_+(\orgn,v')\ \Rightarrow (v=v').$$
      %% Added condition May 15, 2009./ killed June 18
    \end{itemize}
\end{definition}
\end{note}

\begin{lemma}  Let $(V,E)$ be a fan.
For each $v\in V$, the set
        $$
        E(v) = \{w\in V\mid \{v,w\}\in E\}
        $$
        is cyclic with respect to $(\orgn,v)$.
\end{lemma}

\begin{proof}  If $w\in E(v)$, then $\{\orgn,v,w\}$ is not collinear.
Also, if $w\ne w'\in E(v)$, then
$$
C(\{v,w\})\cap C(\{v,w'\}) = C(\{v\}).
$$
This implies that $E(v)$ is cyclic.
\end{proof}

\begin{remark}\tlabel{rem:fan}\rating{30}
\begin{itemize}
\item The pair $(V,E)$ is a graph with nodes $V$ and edges $E$.  The set
$\{\{v,w\}\mid w\in E(v)\}$ is the set of edges around a fixed node $v$.
Note that $w\in E(v)$ if and only if $v\in E(w)$.   
%
\item The final condition implies that the sets $C^0(\e)$
do not meet.   This condition will eventually yield planar
hypermaps.
%
\item
Since $E(v)$ is cyclic,
for each $v\in V$, we have an azimuth cycle $\sigma(v):E(v)\to E(v)$.
We allow $E(v) = \{w\}$ to be a
singleton set. If so,
$\sigma(v)$ is the identity map on $E(v)$.
%
Sometimes we write $\sigma(v,w)$ for $\sigma(v)(w)\in E(v)$.
%
\item 
The condition that $\{\orgn,\e\}$ is not a collinear set, when $\e\in
E$, implies no loops: $\{v,v\}\not\in E$.
%
\end{itemize}
\end{remark}


Let $(V,E)$ be a fan.  We define a sets of darts $D$, and
two subsets $D_1,D_2$:
    $$
    \begin{array}{lll}
    D_1 &= \{(v,w)\mid \{v,w\}\in E\}\\
    D_2 &= \{v \mid v\in V,\ \ E(v) = \emptyset\},\\
    D   &= D_1\cup D_2.
    \end{array}
    $$
Darts in $D_2$ are said to be {\it isolated}; and darts in $D_1$ are {\it non-isolated}.
%
We define a permutation $n$ on $D_1$ by
    $$n(v,w) = (v,\sigma(v,w)).$$
We define a permutation $f$ on $D_1$ by
    $$
    f (v,w) = (w,\sigma(w)^{-1} v).
    $$
Define a permutation $e$ on $D_1$ by
    $$
    e (v,w) = (w,v).
    $$
Define permutations $e,n,f$ on $D_2$ by making them degenerate on $D_2$:
    $$
    e (v) = n(v) = f(v) = v.
    $$
Write $\op{hyp}_r(V,E)=(D_1,e,n,f)$ and $\op{hyp}(V,E)=(D,e,n,f)$.  We call them the non-isolated hypermap
and the hypermap associated with $(V,E)$.  The next
lemma justifies this terminology.



\begin{lemma}\guid{AAUHTVE}\rating{70}
Let $(V,E)$ be a fan.  Let $D = D_1\cup D_2$
and $\op{hyp}(V,E) = (D,e,n,f)$, as constructed above.  Then
    \begin{itemize}
    \item $\op{hyp}(V,E)$ is a plain hypermap.
    \item  $e$ has no fixed
points in $D_1$.
    \item  $f$ has no fixed points on $D_1$.
    \item For every pair of distinct nodes, there is at most one
    edge meeting both.
    \item The two darts of an edge (of $D_1$) lie at different nodes.
    \end{itemize}
\end{lemma}

\begin{proof}  We compute
    $$
\begin{array}{lll}
e(n(f(v,w))) &= e(n(w,\sigma(w)^{-1} v))) &=
        e(w,v)\\ 
&= (v,w).
\end{array}
$$
So it is a hypermap. We compute
    $$e(e(v,w)) = e(w,v) = (v,w).$$
So it is plain. A fixed point in $D_1$ under $e$ would force $v = w\in E(v)$,
but by construction $v\not\in E(v)$.  The argument that $f$ has no
fixed points is similar.

   We show that for every pair of distinct nodes, there is at most one edge
meeting both.
That is,
        $$(n^k e x = e n^\ell x)\Rightarrow (n^\ell x = x).$$
Let $x = (v,w)\in D_1$.  Let $\sigma=\sigma(v)$. Then
    $$
    \begin{array}{lllllll}
    n^\ell x &= (v,\sigma^\ell w)\\
    e n^\ell x &= (\sigma^\ell w,*)\\
    e x &= (w,*)\\
    n^k e x &= (w,*)\\
    n^k e x &= e n^\ell x &\ \Rightarrow (w = \sigma^\ell w) &\ \Rightarrow
    (n^\ell x &= (v,w) = x)
    \end{array}
    $$

Finally, we show that each dart of an edge lies on a different node.
That is, $e x \ne n^k x$, for $x\in D_1$.  We have
    $$
    \begin{array}{lll}
        e(v,w) &= (w,*),\quad w\in E(v)\\
        n^k(v,w) &= (v,*),\quad v\not\in E(v).
    \end{array}
    $$
The result follows.
\end{proof}

\section{Topology}\label{sec:topology}

\subsection{basics}

There is hardly any topology that comes up in this book.  Most of
what is needed appears in this chapter.  We make use of some basic
notions in topology such as continuity, connectedness, and compactness.

\begin{remark} The term {\it connected} is now being used in
two different senses: in the topological sense and in a combinatorial
sense for hypermaps.   We will refer to the connected components
of a topological space as topological components and the connected
components of a hypermap as combinatorial components to reduce the confusion.
\end{remark}






The set $\ring{R}^3$ is a metric space under the
Euclidean distance function $d(v,w) = \norm{v}{w}$.  Subsets of
$\ring{R}^3$ are a metric space under the restriction of the metric
$d$ to the subset. Subsets carry the metric space topology. 
If $Y$ is an open set in $\ring{R}^3$, write
$\comp{Y}$ for its set of topological components.
If two
points in $\ring{R}^3$ 
can be joined by a continuous path that avoids $X$,
then they lie in the same topological component of $Y$.
If we produce a family of pairwise disjoint nonempty connected open sets in
$Y$, whose union is all of $Y$, then
this family is $\comp{Y}$.
%embers of the family are the topological components of $Y$.
Let $$S^2(\orgn) = \{ v \mid \norm{ v}{\orgn } = 1\}$$ be the unit sphere in
$\ring{R}^3$, centered at $\orgn$.  






\subsection{topological component and dart}

Let $(V,E)$ be a fan and let $(D,e,n,f) = \op{hyp}(V,E)$
be the associated hypermap.  Write $D = D_1\cup D_2$ as a union of
non-isolated and isolated darts.

\begin{definition}[X,~Y]\label{def:XY}
Let $(V,E)$ be a fan.  Let $X=X(V,E)$ be the union of the
cones
   $$C(\e)$$
as $\e$ ranges over $E$.  Let $Y=Y(V,E)$ be the complement
$Y = \ring{R}^3\setminus X$.
\indy{Index}{X}\indy{Index}{Y}.
\end{definition}


We associate a wedge $\Wdart(x)$, a subset $\Wdart(x,\epsilon)$,
and an azimuth angle $\op{azim}(x)$
with each dart $x=(v,w)\in D$.  Define 
$$
\Wdart(x)=
\begin{cases} 
W(\orgn,v,w,\sigma(v,w)),&\text{if }\card(E(v))>1,\\
\ring{R}^3\setminus \op{aff}_+(\{\orgn,v\},w),&\text{if } E(v) = \{w\},\\
\ring{R}^3\setminus \op{aff}\{\orgn,v\},&\text{if } E(v) = \emptyset.\\
\end{cases}
$$
Define $\op{azim}(x)$ as the azimuth angle of $\Wdart(x)$:
$$
\op{azim}(x)=\begin{cases}
\op{azim}(\orgn,v,w,\sigma(v,w)), &\text{if } \card(E(v)) > 1,\\
2\pi, & \text{otherwise.}\\
\end{cases}
$$
For any $x = (v,\ldots)\in D$, set
    $$
    \Wdart(x,\epsilon) = \Wdart(x) \cap \op{rcone}^0(\orgn,v,\cos\epsilon).
    $$

\begin{note}%XX
All the hypermaps in this book are connected. $D_2$ is not needed.
\end{note}


\begin{lemma}\guid{VBTIKLP}\tlabel{lemma:disjoint}\rating{120}
Let $(D,e,n,f)$ be the hypermap attached to a 
fan $(V,E)$.
Let $N$ be a node of $D$.  There exists $v\in V$
such that the darts of $N$ are precisely
the darts of the form $(v,\ldots)$.  Furthermore, there is a 
disjoint sum decomposition of $\ring{R}^3$ given by
  $$
  \ring{R}^3 = 
  \op{aff}\{\orgn,v\} \cup
  \bigcup_{x\in N} \Wdart(x)  \cup 
  \bigcup_{\{v,w\}\in E} \op{aff}_+^0(\{\orgn,v\},w).
  $$
\end{lemma}

If $x\in D$, write $v(x)\in V$ for the corresponding vertex.  By the lemma,
we may identify $V$ with the set of nodes of $D$.

\begin{proof}
We prove the existence of the disjoint sum decomposition.
First of all, $\ring{R}^3$ is the disjoint union of $\op{aff}\{\orgn,v\}$
and its complement.

The case $\card(E(v))\le 1$ follows immediately from the definitions.  
We may therefore assume  that $\card(E(v)) >1$.
Fix $u$ such that $\{v,u\}\in E$, and let $\sigma$ be the azimuth
cycle on $E(v)$.  Let $a(i)=\op{azim}(\orgn,v,\sigma^i u,\sigma^{i+1}u)$.   See Lemma~\ref{lemma:2pi-sum}.  Every $y\in\ring{R}^3\setminus\op{aff}\{\orgn,v\}$ satisfies
$$
\sum_{i=0}^j a(i) <
\op{azim}(\orgn,v,u,y) < \sum_{i=0}^{j+1} a(i).
$$
or 
$$
\sum_{i=0}^j a(i) = \op{azim}(\orgn,v,u,y)
$$
for a unique $0 \le j < n$, where $n$ is the cardinality of $E(v)$. 
These conditions are exactly the membership conditions for the sets
$
\Wdart(v,\sigma^j u)
$
and $\op{aff}_+^0(\{\orgn,v\},\sigma^j u)$, respectively.
The result follows.
\end{proof}

\begin{corollary}\guid{IBZWFFH}\tlabel{cor:W}\rating{40}
Let $x = (v,\ldots)$ be a node.
We have $\Wdart(x)\cap C(\{v,w\})=\emptyset$, for $w\in E(v)$.
\end{corollary}

\begin{proof} The decomposition of Lemma~\ref{lemma:disjoint} is
disjoint.  It follows directly from the definitions that
   $$C(\e)\subset \op{aff}_+^0(\{\orgn,v\},w) \cup 
    \op{aff}\{\orgn,v\}.$$
\end{proof}

\begin{lemma}\guid{JGIYDLE}\rating{120} 
For each $x$, and $\epsilon$ sufficiently small and positive,
$\Wdart(x,\epsilon)$ is nonempty and lies in a single 
topological component of $Y(V,E)$.
\end{lemma}

\begin{proof}  First we show that $\Wdart(x,\epsilon)$ lies in $Y$,
for $\epsilon$ small.  Let $x=(v,w)\in D_1$.  
Let $S^2(\orgn)$ be the unit sphere centered at $\orgn$.
By making $\epsilon$ small enough,
the sets $\Wdart(x,\epsilon)\cap S^2(\orgn)$
avoid the compact sets $C(\e)\cap S^2(\orgn)$ when $v\not\in e$.
Thus, $\Wdart(x,\epsilon)$ also avoids $C(e)$ when $v\not\in e$.
By Corollary~\ref{cor:W}, $\Wdart(x,\epsilon)$ avoids $C(\e)$, when $v\in \e$.
Thus, $\Wdart(x,\epsilon)\subset Y$, for $\epsilon$ small.

To complete the proof, it is enough to show that each $\Wdart(x,\epsilon)$ is
connected.  
The  set
   $$
   R=\{(r,\theta,\epsilon') \in (0,\infty) \times (\theta_1,\theta_2) \times (0,\epsilon)\}
   $$
is connected.
The set $\Wdart(x,\epsilon)$  is the image of $R$
under a spherical coordinate representation (Definition~\ref{def:sph}).
It is readily verified that the polar coordinate representation is
a continuous map. As the image of a connected set under a continuous map
is connected, $\Wdart(x,\epsilon)$ is connected.
\end{proof}

\begin{definition}[leads~into] For each dart $x$, 
there is then a well-defined connected
component $U_x$ of $Y(V,E)$ 
that contains $\Wdart(x,\epsilon)$ (for all
sufficiently small $\epsilon$). Say the dart {\it leads into}
$U_x$.
\end{definition}


\section{Planarity}


\subsection{face attributes}

For $x\in D$, let $v(x)\in V$ be the node of $x$.

\begin{lemma}\guid{DHVFGBC}[sweep]\rating{400}\label{lemma:sweep}  
Let $(V,E)$ be a fan, with hypermap $(D,e,n,f)$.  
Suppose that $\op{azim}(x)<\pi$
for all darts $x\in D$.  Fix a dart $x\in D$.
Let $v = v(x)$, $v_0 = v(f x)$,
and $v_1 = v(f^2 x)$.  Let $v_t = (1-t) v_0 + t v_1$, for
$0\le t\le 1$.  Let $C^0(t) = \op{aff}_+^0(\orgn,\{v,v_t\})$.
Let $Y = Y(V,E)$ and $X = X(V,E)$.
Then
\begin{itemize}
\item The set $\{\orgn,v,v_t\}$ is not collinear for any $t\in[0,1]$.
\item If $0 < t \le 1$, and $C^0(t)$ meets $X$, then $t=1$, $\{v,v_1\}\in E$, and $C^0(1) = C^0(\{v,v_1\})$.
%\item For $0< t < 1$, we have $C^0(t)\subset Y$.
%item If $\{v,v_1\}\in E$ (that is, if the face of $x$ is a triangle), 
%then $C^0(1)$ is the blade $C^0(\{v,v_1\})$ of the fan.
%\item If $\{v,v_1\}\not\in E$, then $C^0(1)\subset Y$.
\end{itemize}
\end{lemma}


\begin{figure}[htb]
  \centering
  \szincludegraphics[width=50mm]{\pdfp/vt.eps}
  \caption{Adding an edge to the fan.}
  \label{fig:vt}
\end{figure}


\begin{proof}
It follows from the definition of a fan that $\{v,v_0\}\in E$ and
that $\{\orgn,v,v_0\}$ is not collinear.  By continuity, $\{\orgn,v,v_t\}$
is not collinear for $t>0$ sufficiently small.  If it exists, set $t'$ as
the smallest $t>0$ for which $\{\orgn,v,v_t\}$ is a collinear set.  If it exists, let $I=\{t\mid 0< t < t'\}$, otherwise let $I=\{t\mid 0 < t \le 1\}$.  For each $t\in I$, the blade $C^0(t)$ is not a collinear set and is contained in a unique plane $P(t)$ through $\orgn$ and $v$.

We claim that if $t\in I$ and $C^0(t)$ meets $X$, then $t=1$, $\{v,v_1\}\in E$, and $C^0(1)= C^0(\{v,v_1\})$.  By considering possible intersections with vertices $w\in V$ and blades
$C^0(\e)\subset X$, we find that for $t>0$ sufficiently small,
$C^0(t)$ does not meet $X$, hence $C^0(t)\subset Y$.  Assuming 
that it exists, set $t''$
as the smallest $t\in I$ for which $C^0(t)$ meets $X$.   
$C^0(t'')$ cannot meet $X$ at a vertex $w\in V$, because each azimuth angle
is $<\pi$ at $w$, which means that for any $t$ for which  $C^0(t)$ meets $w$ 
there is a smaller $t'''<t$ for which $C^0(t''')$ meets a blade into $w$.
Thus, $C^0(t'')$ first meets $X$ along a blade $C^0(\e)$. If
the intersection is transversal, again we can find a smaller $t$ that
gives an intersection with the blade.  Hence, we may assume that
$C^0(t'')$ and $C^0(\e)$ are coplanar.  From the disjointness
properties of blades of a fan, it follows that $\e = \{v,v_1\}\in E$,
that $t''=1$, and that $C^0(1)$ is a blade of the fan.

Now assume for a contradiction that $t'$ exists.  Pick $0<t''<t'$.  Then
$\{\orgn,v(t''),v(t'),v\}$ lie in a unique plane $P$.  Since all $v(t)$
are collinear,  $v(t)\in P$ all $t\in I$.  In particular $v_0\in C^0(t'')\cap X$,
contradicting the established disjointness of $X$ from $C^0(t'')$.  Thus, $t'$ does not exist.  This proves the first
claim of the lemma.  

Now $I= \{t\mid 0 < t \le 1\}$ and the other claims follow immediately.
\end{proof}

\begin{lemma}\guid{RWXUYZZ}\rating{100} 
Let $(V,E)$ be a fan with hypermap $(D,e,n,f)$. Let $Y=Y(V,E)$. Assume that $\op{azim}(x)<\pi$ for all darts. Then for every face $F$ of the hypermap, there exists a topological component $U$ of $Y$ such that for every $x\in F$, the dart $x$ leads into $U$. 
\end{lemma}

Write $F\mapsto U_F$ for this map from faces to topological components.

\begin{proof}  Fix any dart $x\in F$ and construct the set $C^0(t)$ as
in the previous lemma.  By the previous lemma, the set $C^0(t)$ lies in a single
component $U$ for all $t>0$ sufficiently small.  For all $\epsilon>0$
sufficiently small, there exists $\delta>0$ such that set $C^0(t)$ meets
both $W(x,\epsilon)$ and $W(f x,\epsilon)$ for all $t<\delta$.  Thus,
$x$ and $f x$ lead into the same component $U$.  By induction, for all
$y\in F$, the dart $y$ leads into $U$.
\end{proof}

\begin{lemma}\guid{JUTSTKG}\rating{120}
Let $(V,E)$ be a fan with hypermap $(D,e,n,f)$. Let $Y=Y(V,E)$. Assume that $\op{azim}(x)<\pi$ for all darts.  For every topological component $U$ of $Y$, there is a dart $x\in D$ that leads into $U$.
\end{lemma}

\begin{proof}  
To show the dependence of the sets $C^0(t)$ on the initial dart $x$, write $C^0(t,x)$.

Let $p\in U$.  Choose a path $\gamma:[0,1]\to \ring{R}^3$
such that $\gamma(t)\in U$ for $t<1$ and $\gamma(1)\not\in U$.  Then
$q=\gamma(1)\in X$.  If $q\in\op{aff}^0_+(\orgn,v)$ for some $v\in V$,
then there exists a dart $x$ with node $v = v(x)$ such that for 
all $0\le t < 1$ and all sufficiently
small $\epsilon>0$, we have $\gamma(t)\in W(x,\epsilon)$.  Thus,
$x$ leads into $U$.

The other possibility is that
$q\in C^0(\{v,w\})$ for some $\{v,w\}\in E$.  There is a unique
edge $\{x,y\}$ of the hypermap such that $v=v(x)$ and $w=v(y)$.  
There
is a small neighborhood of $q$ such that every point $q'$ in that neighborhood
takes one of the following forms:
\begin{itemize} \item $q'\in C^0(\{v,w\})$.
\item $q'\in C^0(s,x)$ for some $0<s<1$.
\item $q'\in C^0(s,y)$ for some $0<s<1$.
\end{itemize}
Points of the first form do not meet $Y$.  Thus for some $t<1$ and $s<1$
we have $\gamma(t)\in C^0(s,x)$ or $\gamma(t)\in C^0(s,y)$.  Thus,
$x$ or $y$ leads into $U$.
\end{proof}

\begin{lemma}\guid{KVQWYDL}[triangle attributes]\rating{200} \label{lemma:triangle}
Let $(V,E)$ be a fan with hypermap $(D,e,n,f)$. 
Let $Y=Y(V,E)$.
Assume that $\op{azim}(x)<\pi$
for all darts.  Fix a face $F$ of cardinality three, fix
$x\in F$, and set $x_i = f^i x$. Then
\begin{itemize}  
\item $U_F$ is equal to the intersection of the three half-spaces
$$H^0(i)=\op{aff}_-^0(\{\orgn,v(x_{i+1}),v(x_{i+2})\},v(x_i)),\quad i=0,1,2$$
\item if a dart $y$ leads into $U_F$, then $y\in F$.
\end{itemize}
\end{lemma}

\begin{proof} The intersection of two half-spaces, $H^0(1)\cap H^0(2)$ is
the wedge $W(x)$.   The sets $C^0(t)\subset W(x)$ sweep out precisely
the intersection of $W(x)$ with $H^0(0)$.  The sets $C^0(t)$ belong to
$U_F$.  Hence the intersection $U'$ of the three half-spaces is a subset of $U_F$.

Suppose for a contradiction 
that $p$ is a point of $U_F$ that does not belong to $U'$.  Choose a path $\gamma:[0,1]\to U_F$ with $\gamma(0)\in U'$ and $\gamma(1)=p$.  Let $t>0$ be the first time such that $\gamma(t)\not\in U'$.  Then $q=\gamma(t)$ lies in the set consisting of the closed intersection of half-spaces $H(i)$ corresponding to $H^0(i)$ and lies
in one of the bounding planes.  Let 
$$
X' = \cup C(i),\quad\text{ where } C(i)=C(\{v(x_i),v(x_{i+1})\}).
$$
Then $q\in X'\subset X$.  Thus,
$q\in X\cap Y = \emptyset$, which is impossible.  Thus, $U'=U_F$.

Let $y$ be any dart that leads into $U_F$ at vertex $v(y)$.  Then
$W(y,\epsilon)$ meets $U_F$ for all $\epsilon>0$ sufficiently small.
This implies that $v(y)$ lies in the intersection of the closed half-spaces $H(i)$.  As we have seen, this intersection is the disjoint union of $U_F$ and
$X'$.  As $v(y)\in X$, which does not meet $U_F$, we have $v(y)\in X'$.
The set $X'$ is the disjoint union of the rays $\op{aff}_+(\orgn,v(x_i))$ and
the three blades $C^0(i)$.  These blades do not meet the vertices, hence
$v(y)=v(x_i)$ for some $i$.  Thus, $y$ and $x_i$ belong to the same
node.  The sets $W(y)$ and $W( x_i)$ are disjoint for distinct darts at the same
node, and this implies that $y=x_i\in F$.
\end{proof}

\begin{corollary}\guid{MOZNWEH}\rating{60}\label{lemma:girard-component}
Let $F$ be a face of size $3$ in in the context of Lemma~\ref{lemma:triangle}.  Then $U_F$ is measurable and eventually radial at $\orgn$.
The solid angle of $U_F$ is given by the formula
$$
\sol(U_F) = -\pi + \sum_{x\in F}\op{azim}(x),
$$
\end{corollary}

\begin{proof} An intersection of half-spaces is measurable and
eventually radial.  The solid angle is given by Girard's formula for
a spherical triangle.
\end{proof}

\begin{lemma}\guid{PIIJBJK}[face attributes]\rating{1000}\label{lemma:face}
Let $(V,E)$ be a fan with hypermap $(D,e,n,f)$. 
Let $Y=Y(V,E)$.
Assume that $\op{azim}(x)<\pi$
for all darts $x\in D$.  Then
\begin{itemize}
\item The map $F\mapsto U_F$ is a bijection between faces of the hypermap
and topological components of $Y$.
\item  Each topological component $U_F$ is the intersection of the open
half-spaces $\op{aff}_+^0(\{\orgn,v(x),v({f x})\},v(f^2 x))$, as $x$ runs
over $F$.
\item For every $F$, the topological component $U_F$ is measurable and
eventually radial at $\orgn$.  The solid angle of $U_F$ is given by the
formula
$$
\sol(U_F) = 2\pi + \sum_{x\in F}(\op{azim}(x)-\pi).
$$
\item If $x,y\in F$, with corresponding vertices $v(x),v(y)\in V$, then
$\{\orgn,v(x),v(y)\}$ are not collinear.
Furthermore, 
either $x,y$ are adjacent under the face map, or $C^0(\{v(x),v(y)\})\subset U_F$.  {\it That is, the diagonals of the polygon $U_F$ are all interior.}
\item  Triangulations of $U_F$ exist.
\end{itemize}
\end{lemma}

Before moving to the proof, we give some corollaries.

\begin{corollary}\guid{GINGUAP}\rating{40}
Each topological component $U_F$ is convex.
\end{corollary}

\begin{proof} It is the intersection of half-spaces.
\end{proof}

\begin{corollary}\guid{SRPRNPL}\rating{60}  
The hypermap is simple.
\end{corollary}

\begin{proof}  Let $x\in F$.  By the intersection of half-spaces property, $U_F$ is contained in the wedge $W(x)$ at $x$.  If there is a second dart $y$ at the same node in $F$, then $U_F$ is also contained in $W(y)$. However, by Lemma~\ref{lemma:disjoint}, the wedges at a given node are disjoint.
\end{proof}

\begin{corollary}\guid{WGVWSKE}\rating{150}  
The hypermap is connected.
\end{corollary}

\begin{proof} Assume that $x,y$ be any two darts.  By replacing $x$ with $f x$ if necessary, which lies in the same combinatorial component as $x$, we may
assume that $\{\orgn,v(x),v(y)\}$ is not a collinear set. 
For each blade $C^0(\e)$ of the fan that meets $C=C^0(\{v(x),v(y)\})$
pick one of the two endpoints of $\e$.  Then we get a sequence
$$
v(x)=v_0,v_1,\ldots,v_k=v(y)
$$
such that $C^0(\{v_i,v_{i+1}\})$ lies in a single topological component $U_i$.  Each $U_i$ has the form $U_{F_i}$ for some face $F_i$ of the hypermap.
Thus, we may construct a combinatorial path from $x$ to $y$, by moving by the face map from dart to dart within each $F_i$ and by the node map from dart to dart around a given node $v_j$.
\end{proof}

\begin{corollary}\guid{GGRLKHP}\rating{100}  
The hypermap is planar.
\end{corollary}

\begin{proof}  The solid angle of a sphere is $4\pi$.  The set $X$
has measure zero, so that
$$
4\pi = \sol(Y)= \sum_F \sol(U_F) = 
\sum_F ( 2\pi + \sum_{x\in F} (\op{azim}(x)-\pi) ).
$$
The double sum over faces and darts in a face can be replaced by
a single sum over darts.  
The sum of the azimuth angles of all darts at a node is $2\pi$. Thus,
all the azimuth angle terms give $2\pi\,\#n$.
Thus, the formula becomes
$$
4\pi = 2\pi\, \#f +2\pi\,\#n - \pi\, \#D.
$$
In a plain hypermap in which the edge map has no fixed points, $\#D = 2\,\#e$.
The relation becomes
$$
2 + \#D = \#f + \#e + \#n.
$$
This is the condition of planarity for a connected hypermap.
\end{proof}

\subsection{proof of face attributes}

Now we turn to the proof of the face-attribute lemma (Lemma~\ref{lemma:face}).  We break the proof into a series of small lemmas.  The primary proof method is an induction on the following invariant of a fan $(V,E)$.  If $(V,E)$ is a fan,  let $N(V,E)$ be the natural number
$$
\sum_F (n_F - 3),
$$
where the sum runs over faces of the  hypermap, and $n_F$ is the cardinality of the face $F$.
%We prove the conclusion of the lemma, together with the additional
%conclusion:
%\begin{itemize}
%\item If $C^0(\e)$ is any diagonal of $U_F$ (with $\e\not\in E$), then the %fan $(V,E'')$, where $E'' = E\cup\{\e\}$, satisfies
%$N(V,E'')+1 = N(V,E)$.
%\end{itemize}

\begin{lemma} Lemma~\ref{lemma:face} holds under the additional assumption that $N(V,E) = 0$.
\end{lemma}

\begin{proof}
If $N(V,E)=0$, then the hypermap is a triangulation.  We consider this
case first.  By earlier lemmas, every topological component of $Y$ has
the form $U_F$ for some face $F$.  By Lemma~\ref{lemma:triangle}, $U$ uniquely determines the face $F$.  Thus, there is a bijection between faces of the hypermap and topological components.  By Lemma~\ref{lemma:triangle}, the topological component $U_F$ is the intersection of open half-spaces as asserted.  The solid angle formula is given by Corollary~\ref{lemma:girard-component}.  The facts about diagonals and triangulations are trivial for a hypermap that is already a triangulation. This completes the proof in the base case $N(V,E)=0$.
\end{proof}

Now consider the case $N(V,E)>0$.  There exists a face $F$ of the hypermap that is not a triangle.  By Lemma~\ref{lemma:sweep}, there is a diagonal in $U_F$. We form a new fan $(V,E')$ on the same vertex set with
$E' = E\cup \{\{v,w\}\}$.   

\begin{lemma} 
$N(V,E')<N(V,E)$.
\end{lemma}


\begin{proof}  There is a uniquely determined edge $\{x,z\}$ of $\op{hyp}(V,E')$ such that $v=v(x)$ and $w=v(z)$.   The hypermap $\op{hyp}(V,E)$ is obtained from $\op{hyp}(V,E')$ by a double walkup transformation on the edge $\{x,z\}$.  The darts belong to different faces $F_x$ and $F_z$ (by Lemma~\ref{lemma:triangle}), one of which (say $F_x$ is a triangle.  Thus, the walkup transformation merges two faces.  Let $n$ be the cardinality of $F_z$. Then 
$$N(V,E) - N(V,E') = ((n+1)-3) ~~-~~ ((n-3) + (3-3)) = 1 >0.$$
\end{proof}

Assume for a contradiction that there exists a fan $(V,E)$ 
satisfying the assumptions of the lemma, but not the conclusion.
Among all such counterexamples with fixed $(V)$, we may pick
$E$ to minimize  $N(V,E)$.

By the assumed minimality of $E$, the conclusion of the lemma holds for the
fan $(V,E')$.  Add primes
for quantities associated to the fan $(V,E')$.  The two faces
$F_x$ and $F_z$ merge into a single face $F$ of $\op{hyp}(V,E)$.
Then 
\begin{equation}\label{eqn:U}
U= U_{F_x}\cup U_{F_z}\cup C^0(\e)
\end{equation} 
is a connected open set in $Y$.
If $F'\ne F_x,F_z$ is any other face in $\op{hyp}'$, then $U_{F'}$ is
a connected open set in $Y$.  Moreover, the set $U$ and sets $U_{F'}$
are disjoint and exhaust $Y$, so that the are precisely the topological
components of $Y$.  Some dart of $F$ leads into $U$, so $U=U_F$.  It follows
that the number of faces is equal to the number of topological components, so that the map $F\mapsto U_F$ is a bijection.

\begin{lemma} The area formula holds.
\end{lemma}

\begin{proof} Every topological component of $Y$ 
except $U$ is already a topological component of $Y'$ and the conclusion
holds for components of $Y'$.  The component $U$ is a disjoint union of two
components of $Y'$ and a set $C^0(\e)$ of measure zero.  Thus, it
is also measurable and eventually radial.  The solid angle formula is
additive over the disjoint union, so the formula holds for $U$.
\end{proof}

\begin{lemma}  For any dart $x$ on the face $F$, and dart $y\in F$ that is not adjacent to $x$ under the face map, 
$$
C^0(\{v(x),v(y)\} \subset U_F.
$$
\end{lemma}

\begin{proof}
By the minimality of $E$, it is enough to consider the merged face $F$.  
If $y=f^2x$, then the diagonal is precisely $C^0(\{v,w\})$,
which has already been established.  Otherwise, $y$ can be identified with
a dart of the face $F_z$ in the hypermap $\op{hyp}'$.  The conclusion holds
already for $U_{F_z}\subset U$.  
\end{proof}

%Next we show the extra conclusion added at the beginning of the lemma.
%If the added diagonal is $\{v,w\}$, then $E''=E'$ and we have already
%checked that $N(V,E')+1=N(V,E)$.  Since $x$ was arbitrary, we may assume
%that $v(x)\in e$.  Let $E''' = E \cup \{e\, \{v,w\}\}$.  By the minimality,
%$N(V,E''') + 1 = N(V,E')$.  Also, $E''' = E''\cup \{\{v,w\}\}$.  The
%same argument that shows $N(V,E')+1=N(V,E)$ shows $N(V,E''')+1 = N(V,E'')$.
%Hence,
%$$
%N(V,E'')+1 = N(V,E''')+2 = N(V,E')+1 = N(V,E).
%$$

\begin{lemma} The given intersection of open half-spaces is equal
to the topological component. 
\end{lemma}

\begin{proof}
By the minimality of $E$, it is enough to consider
the merged face $F$ and its topological component $U=U_F$.
We first show that the intersection $U'$ of half-spaces lies in $U$.
Every point in this intersection lies in the plane
$$
\op{aff}(\{\orgn,v,w\})
$$
or in one of the two open half-planes bounded by this plane.  The intersection
of this plane with $U'$ is the blade $C^0(\{v,w\})$, which belongs
to $U$. The intersection of the two half-spaces, by the minimality of $E$,
belong to $U_{F_x}$ and $U_{F_y}$, which are subsets of $U$.  Thus, $U'\subset U$.

To prove the reverse inclusion, let $y$  be a dart of
the face $F$.  we show that $U$ is a subset of the half-space with bounding
plane $\{\orgn,v(y),v(f y)\}$.  We may assume that the edge $\{x,z\}$ of the
hypermap $\op{hyp}'$ satisfies $v(x) = v(y)$, as $x$ can be chosen at an
arbitrary node.  By the minimality of $E$,  
the partition (\ref{eqn:U}) of $U$ gives three pieces
contained respectively in the three parts:
$$
W(x) \cup C^0(\{v,w\}) \cup W(x'),
$$
where $x,x'\in D'$ correspond to the single dart $y\in D$:
$$
\op{azim}(x) + \op{azim}(x') = \op{azim}(y) < \pi.
$$
Thus, $U$ itself is contained in the lune
$$
\op{wedge}(\{\orgn,v(y)\},\{v(f y),v(f^{-1} y)\}),
$$
which is contained in the desired half-space.
\end{proof}

\section{Polygon}

\subsection{cyclic fan}  This section goes into further detail about certain fans that are isomorphic to $H_{2k}$.

If $x$ is a dart in a fan's hypermap, set
$$
\bWdart(x) = \{ x\mid 0\le \op{azim}(\orgn,v,w,x)\le \op{azim}(\orgn,v,w,\sigma(v,w)).
$$
(It is the closure of $\Wdart(x)$, as defined above.)

\begin{definition}[cyclic fan]  A triple $(V,E,F)$ is a {\it cyclic fan} if the following conditions hold.
\begin{itemize} 
\item $(V,E)$ is a fan.
\item Its hypermap $H=\op{hyper}(V,E)$ is isomorphic to $H_{2k}$ for some $k\ge 3$.
\item $F$ is a face of the hypermap $H$ and $\op{azim}(x)\le \pi$ for all $x\in \pi$.
\item $V\subset \bWdart(x)$, for all $x\in F$.
\item If $\{v,w\}\in E$, then $\{0,v,w\}$ is not collinear.
\end{itemize}
\end{definition}

\begin{lemma}  Let $(V,E,F)$ be a cyclic fan.  Let $x$ be a dart in the hypermap of $(V,E,F)$.  If for given $v,w\in V$  the triple $\{0,v,w\}$ is not collinear, then $C(\{v,w\}) \subset \bWdart(x)$.
\end{lemma}

\begin{proof}  This is an elementary consequence of the definitions and the condition $V\subset \bWdart(x)$.
\end{proof}

\begin{lemma}  Let $(V,E,F)$ be a cyclic fan with hypermap $(D,e,n,f)$.  Then there exists a unique cyclic permutation $\phi:V\to V$ such that
\begin{itemize}
\item $(v,\phi v)\in D$, for all $v\in V$.
\item $f(v,w) = (w,\phi w)$, for all $(v,w)\in D$.
\end{itemize}
\end{lemma}

\begin{proof}  This is immediate.
\end{proof}

\begin{definition}[perimeter]\label{lemma:mono}
Let $(V,E,F)$ be a cyclic fan with hypermap $(D,e,n,f)$.  Choose $v\in V$.  We set
$$
P=P(V,E,F) = \sum_{i=0}^{k-1} \arc(0,\phi^i v,\phi^{i+1} v).
$$
This is easily seen to be independent of the choice of $v\in V$.  We call $P$ the perimeter of the cyclic fan.
\end{definition}

\begin{lemma}[monotonicity]  Let $(V,E,F)$ be a cyclic fan with hypermap $(D,e,n,f)$.  Let $k=\card(F)$; $x=(v_0,v_1)\in F$;  $v_r = \phi^r v_0$; $\alpha(r) = \op{azim}(0,v_0,v_1,v_r)$;  $x_r = f^r x = (v_r,v_{r+1})\in F\subset D$, for all $r$.  
We distinguish three cases according to the manner in which the fan meets the ray $R(v_0) = \op{aff}_0^-(\{0\},\{v_0\})$:
\begin{enumerate} 
\item {\bf (great circle)} If for some $i$, $C^0\{v_i,v_{i+1}\}\cap R(v_0)\ne \emptyset$, then the following conditions hold.
\begin{itemize}
\item $0=\alpha(1)=\cdots \alpha(r);\quad \alpha(r+1)=\cdots\alpha(k-1)=\pi$.
\item $\op{azim}(x)=\pi$, for all $x\in D$.
\item The set $V$ lies in a plane through $0$.
\item The perimeter is $P=2\pi$.
\end{itemize}
\item {\bf (lune)} If for some $i$, $\op{aff}_0^+(0,v_i) \cap R\ne\emptyset$, then the following conditions hold.
\begin{itemize}
\item $0=\alpha(1)=\cdots \alpha(r) < \alpha(r+1)=\cdots\alpha(k-1)\le\pi$.
\item $\op{azim}(x)=\pi$, for all $x\ne x_0,x_r\in F$.  Also, $0<\op{azim}(x_0)=\op{azim}(x_r)\le \pi$.
\item $\{0,v_0,v_r\}$ is contained in a line $L$, $\{0,v_0,\ldots,v_r\}$ lie in one half-plane bounded by $L$, and $\{v_r,v_{r+1},\ldots,v_0\}$ lie in another half-plane bounded by $L$.
\item The perimeter is $P=2\pi$.
\end{itemize}
\end{enumerate}

{\bf (generic)~~} Now assume that for every choice of $(v,\phi v)\in F$ the previous conditions fail to be satisfied.  That is, the closed blades do not meet any of the rays $R(v)$, for $v\in V$.  Then for every $x=(v_0,v_1)\in F$, the following conclusions hold.
\begin{itemize}
\item $0=\alpha(1)\le \alpha(2)\le \cdots\le \alpha(k-1)\le\pi$.
\item If $\alpha(j)=\alpha(j-1)$, then by elementary geometry either $v_{j-1}\in C\{v_0,v_j\}$ or $v_j\in C\{v_{j-1},v_0\}$.  In the first case,
$\op{azim}(x_i)=\pi$, for $i=0,\ldots,j-1$.  In the second case,
$\op{azim}(x_i)=\pi$, for $i=j,\ldots,k-1$.
\end{itemize}
\end{lemma}

\begin{proof}  We give the proof according to the case.

{\bf (great circle)} We consider a spherical coordinate system $(r,\theta,\phi)$ with coordinates $(r_j,\theta_j,\phi_j)$ for $v_j$.  Without loss of generality, there is a coordinate system in which $\phi_0=0$, $\theta_r=0$, and $\theta_{r+1}=\pi$.  In these coordinates, we see that $v_0\in\bWdart(x_r)$ implies $\op{azim}(x_r)=\pi$, which in turn implies $\theta_{r-1}=0$.  Continuing by downward induction $v_0\in \bWdart(x_j)$ implies that $\op{azim}(x_j)=\pi$ and $\theta_j=0$.  By a similar induction $\theta_i=\pi$ and $\op{azim}(x_i)=\pi$, for $i=r+1,\ldots,k-1$.  For our choice of coordinates $\alpha(i)=\theta_i$.  This gives the first and second conclusions.  It follows that the points $v_j$ appear in consecutive order around the great circle $\theta=0,\pi$.  The perimeter is thus $2\pi$.

{\bf (lune)}  The proof here is similar.  We may choose coordinates for which $\phi_0=0$, $\phi_r=\pi$, $\theta_{r-1}=0$, $\theta_{r+1}\le\pi$.  We then check that $v_0\in \bWdart(x_{r-1})$ implies that $\op{azim}(x_{r-1})=\pi$ and that $\theta_{r-2}=0$, and continue with a downward induction as before.  Similarly, an upward induction gives that $\theta_j=\theta_{r+1}$ and $\op{azim}(x_j)=\pi$, for $j\ge r+1$.  The first and second conclusions follow as before.  The first set of points are consecutive inside the plane $\theta=0$; the second set of points are consecutive inside the plane $\theta=\theta_{r+1}$.  The perimeter consists of two half-circles, and $P=2\pi$.

{\bf (generic)} Now consider the complementary case in which the closed blades do not meet any of the rays $R(v)$.  For the first claim we may assume by induction that $0\le \alpha(1)\ldots\cdots\ldots \alpha(i)$.  We may pick spherical coordinates for which $\phi_0=0$, $\theta_1=0$, $\theta_i=\alpha(i)$, for all $i$.  We may write the condition
$$
v_0\in \bWdart(x_i)
$$
as a determinant:
$$
\det(v_0,v_i,v_{i+1})\ge 0.
$$
In spherical coordinates, the determinant becomes
$$
\det(\cdots)= r_0r_ir_{i+1}\sin\phi_i\sin\phi_{i+1}\sin(\theta_{i+1}-\theta_i)\ge0.
$$
In this complementary case $\sin\phi_i\ne0$, $\sin\phi_{i+1}\ne0$ (when $i\ne 0,i+1\ne 0$).  From $x_j\in \bWdart(x_0)$, we get $0\le\theta_j\le\theta_{k-1}\le\pi$ for all $j$.  These inequalities give $\theta_i\le\theta_{i+1}$ (with a small extra argument for the degenerate case $(\theta_i,\theta_{i+1})=(0,\pi)$).  The first conclusion follows from $\theta_i=\alpha(i)$.

Assume $\alpha(j)=\alpha(j-1)$.  We give the proof when $v_{j-1}\in C\{v_0,v_j\}$.    In this case, $\alpha(i)=\theta_i=0$, for $i=0,\ldots,j$.  The points $\{0,v_0,\ldots,v_j\}$ are then coplanar and $\op{azim}(x_i)=0$ for $i=0,\ldots,j-1$.  This gives the second conclusion. The other case $v_j\in C\{v_0,v_{j-1}\}$ follows by similar arguments.
\end{proof}


\subsection{perimeter}




\begin{lemma}\guid{WSEWPCH}\tlabel{lemma:convex-hyper}\rating{400}
Let $(V,E,F)$ be a cyclic fan.  Then the perimeter $P$ of the fan is at most $2\pi$.
\end{lemma}

\begin{proof} In the first two cases of Lemma~\ref{lemma:mono}, the perimeter has already been shown to be exactly $2\pi$.  We may assume we are in the third case of that lemma.

XX finish.
\end{proof}

\begin{proof}  A fan does not have any faces of cardinality less than three.
Every blade of the fan has radian measure less than $\pi$.  

Consider the case of a spherical triangle.  If the edges of the
the triangle are $a_i$ and the angles of the polar
triangle are $\alpha'_i$, then $\alpha'_i+a_i=\pi$.
The the perimeter is 
$$a_1+a_2+a_3 = 2\pi - (\alpha'_1 -\alpha'_2 - \alpha'_3-\pi) < 2\pi,$$
because the area of the polar triangle is always strictly positive.

Similarly, if the sides of the faces of the spherical polygon are
$a_i$, then the angles of the polar polygon are $\alpha'_i = \pi-a_i$.
The perimeter is
$$
a_1+\cdots+a_n  = 2pi- A< 2\pi,
$$
where $A = 2\pi-\sum a_i$ is the area of the polar polygon.
%~\cite[p.261]{williamson:2008}.
\end{proof}

\begin{note}%XX
Futher details about polar polygons are to be added, including a definition and a justification of the area formula for the polar.
\end{note}





\section{Polyhedron}

This section shows that a convex polyhedron determines a fan.

\begin{definition}
A polyhedron the intersection of finitely many closed half-spaces.
An interior point $p$ of a polyhedron is a point that contains a neighborhood
entirely contained in the polyhedron. An isolating half-space of
a polyhedron is a closed half-space containing the polyhedron.  A face of a polyhedron
is the intersection of the polyhedron with the bounding plane of an isolating-half space.   The dimension of a face is defined as the dimension of the affine hull of the face.  A vertex is a face of dimension $0$.  An edge is a face of dimension $1$.
\end{definition}

\begin{lemma}\guid{JLIGZGS}\rating{800}\label{lemma:polyhedron}  
Let $P$ be a polyhedron with an interior point $\orgn$.
Let $V$ be the set of vertices of $P$.  Let $E$ be the set of pairs $\{v,w\}$
of vertices such that $\op{conv}\{v,w\}$ is an edge of $P$.
\begin{itemize}
\item $(V,E)$ is a fan.
\item Every dart $x$ of the associated hypermap satisfies $\op{azim}(x)<\pi$.
\end{itemize}
\end{lemma}


\begin{note}%XX
We also need the fact that the circular disks of Lemma~\ref{lemma:13-14} lie in different topological components $Y(V,E)$.  More precisely, no disk meets $X$; and no two disks lie in the same topological component.
\end{note}



    

%%%%%%%%%%%%%%%%%%%%%%%%%%%%%%%%%%%%%%%%%%%%%%%%%%%%
    %\lll
    % file started March 22, 2009

\chapter{Packing}

\begin{summary}
This chapter comprises much of the core material of the book.  This is a book
about dense sphere packings, and at last we take up the topic of dense sphere
packings.
 Associated with a sphere packing $V$ in $\ring{R}^3$
are various subsidiary decompositions of space.
This chapter focuses on three such decompositions: the Voronoi decomposition into
polyhedra,  the Rogers decomposition into simplices, and the Marchal decomposition
into cells.  Each of these decompositions leads to a bound on the density of sphere
packings.  The bounds in the first two cases are not sharp.  The Marchal decomposition
leads to a sharp bound $\pi/\sqrt{18}$ on the density of sphere packing in three
dimensions.  The final sections of this chapter launch a detailed study of the properties
of the Marchal cell decomposition.
\end{summary}

\section{The Primitive State of Our Subject Revealed}


\subsection{definition}



Informally, a \newterm{packing} is an arrangement of congruent
balls in Euclidean three space that are nonoverlapping in the sense
that the interiors of the balls are pairwise disjoint.  By convention,
we take the radius of the congruent balls to be $1$.
%the scale
%invariance of density, without loss of generality, units can be chosen
%so that each ball has radius $1$. 
Let $ V$ be the set of centers of the balls in a
packing. The choice of unit radius for the
balls implies that any two points in $ V$ have distance at
least $2$ from each other. 
 Formally, the packing is identified
with the set of centers $V$.
\indy{Notation}{V@$ V$ (packing)}%

%
%The density of a packing does not decrease when balls are added to the
%packing. Thus, to construct packings of maximal density, one may add
%nonoverlapping balls until there is no room to add further balls.  A
A packing in which no further balls can be added is said to be {\it
saturated}.

\begin{definition}[saturated,~packing]\guid{XASMJUK} 
\formaldef{packing}{packing}
\formaldef{saturated}{saturated}
A \newterm{packing} $ V\subset \ring{R}^3$ is a set such that
\begin{displaymath}
\forall  \u,~\v\in  V.~  \norm{ \u}{\v} < 2 \Rightarrow ( \u=\v).
\end{displaymath} 
A set is \newterm{saturated} if for every $\p\in\ring{R}^3$, there
exists $ \u\in V$ such that $\norm{ \u}{\p}< 2$.
\end{definition}
\indy{Index}{saturated}%
\indy{Index}{packing}%



Let $B(\p,r)$ denote the open ball in
Euclidean three-space at center $\p$ and radius $r$.  The open ball
is measurable with measure $4\pi r^3/3$.
 Set $ V(\p,r) = V \cap
B(\p,r)$. %and $ V^*(\p,r) = V(\p,r)\setminus \{\p\}$.
\formaldef{ball}{ball}
\formaldef{$V(\p,r)$}{V INTER ball(p,r)}
\indy{Index}{measure}%
\indy{Notation}{B@$B(\p,r)$}%

\begin{lemma}[]\guid{KIUMVTC}\rz{0}
\oldrating{80}
\formalauthor{Nguyen Tat Thang}
\label{lemma:V-finite}
Let $ V$ be a packing and let $\p\in\ring{R}^3$.
Then the set $ V(\p,r)$ is finite.
\end{lemma}

\begin{proof}  Let $\p = (p_1,p_2,p_3)$. The floor function gives the map
\begin{displaymath}\v=(v_1,v_2,v_3)\mapsto (\lfloor 2(v_1-p_1)
  \rfloor, \lfloor 2(v_2-p_2) \rfloor, \lfloor 2(v_3-p_3) \rfloor).
\end{displaymath}
It is a one-one map from $ V(\p,r)$ into the set $\ring{Z}^3\cap B(\orz,2r + 1)$.  
By Lemma~\ref{lemma:Zcount} the range of this one-one map is finite. 
Hence the domain $ V(\p,r)$ of the map is also finite.%
\footnote{An alternative proof uses the open cover of the compact ball 
$\bar B(\p,r)$ by the sets $\bar B(\p,r)\setminus V$ and $B(\v,1)$, 
for $\v\in V$. By compactness, the cover is necessarily finite.}
\end{proof}
\indy{Notation}{1@$\lfloor\cdot\rfloor$ (floor)}%






\subsection{Voronoi cell}

Geometric decompositions of space give a way to estimate the density of sphere packings.
One of the most popular decompositions of space is the Voronoi cell decomposition.

\begin{definition}[Voronoi~cell,~$\Omega$]\guid{YGFWXEH}\label{def:voronoi} 
\formaldef{$\Omega$}{voronoi\_closed}
%\indy{Index}{Voronoi cell}% 
Let $V\subset\ring{R}^3$ and $\v\in V$.
The \newterm{Voronoi cell} 
$\Omega(V,\v)$
is the set of points at least as close to $\v$ as to
any other point in $V$. 
% Let $\Omega_t( V,\v) = \Omega( V,\v)
%\cap B(\v,t)$ be the truncated Voronoi cell at radius $t$.
\end{definition}
\indy{Notation}{ZZZomega@$\Omega$ (Voronoi cell)} %

\begin{lemma}[Voronoi partition]\guid{TIWWFYQ}\rz{100}
If $V$ is a saturated packing, then 
\begin{equation}\label{eqn:vor-rn} 
\ring{R}^3 = \bigcup \{\Omega(V,\v)\mid \v \in V\}.
\end{equation}
\end{lemma}

\begin{proof}
  If $V$ is a saturated packing, then every point $\p$ has distance
  less than $2$ from some point of $V$.  The set $V(\p,2)$ is finite
  by Lemma~\ref{lemma:V-finite}.  Hence, $\p$ is at least as close to
  some $\v\in V$ as it is to any other $\w\in V$.  This means that
  $\p\in\Omega(V,\v)$.  
\end{proof}

We use half-spaces to separate one Voronoi cell from another.

\begin{definition}[half-space]\guid{BGXHPKY}
\formaldef{$A$}{bis}
\formaldef{$A_+$}{bis\_le}
\begin{eqnarray*} 
A(\u,\v) &= \{\p\in\ring{R}^3
\mid 2(\v-\u)\cdot \p = \normo{\v}^2 - \normo{\u}^2 \},\\
A_+(\u,\v) &= \{\p\in\ring{R}^3
\mid 2(\v-\u)\cdot \p \le \normo{\v}^2 - \normo{\u}^2 \},
\end{eqnarray*}
when $\u,\v\in\ring{R}^3$.  The plane $A(\u,\v)$ is the \newterm{bisector} of
$\{\u,\v\}$ and $A_+(\u,\v)$ is the half-space of points at least as
close to $\u$ as to $\v$.  
\end{definition}

Each Voronoi cell is a bounded polyhedron.

\begin{lemma}[Voronoi polyhedron]\guid{RHWVGNP}\rz{200}\label{lemma:V4} 
  Let $V\subset\ring{R}^3$ be a saturated packing.  Then
  $\Omega(V,\v)\subset B(\v,2)$.  Also, $\Omega(V,\v)$ is a polyhedron
  defined by the intersection of the finitely many half-spaces
  $A_+(\v,\u)$, for $\u\in V(\v,4)\setminus\{v\}$.
\end{lemma}

\begin{proof} 
The Voronoi cell $\Omega(V,\v)$ is the
intersection of the half-spaces $A_+(\v,\u)$ as $\u$ runs over
$V\setminus \{\v\}$.

Let $\p\not\in B(\v,2)$.  
By saturation, there exists $\u\in V$ such that $\norm{\p}{\u}<2$.
Then 
\begin{displaymath} 
\norm{\p}{\u} < 2 \le \norm{\p}{\v}.
\end{displaymath}
Hence, $\p\not\in\Omega(V,\v)$.  This proves the first conclusion.


Let $\Omega'$ be the intersection of the half-spaces $A_+(\v,\u)$ as
$\u$ runs over $V(\v,4)$.  Clearly, $\Omega(V,\v)\subset \Omega'$.
Assume for a contradiction that $\p\in \Omega'\setminus\Omega(V,\v)$.
The intersection of the ray $\op{aff}_+\{\v,\{\p\}\}$ with
$\Omega(V,\v)$ is a closed and bounded convex subset of the line.  By
general principles of convex sets, this intersection is an interval
$\op{conv}\{\v,\p'\}$, for some $\p'\in\Omega(V,\v)\subset B(\v,2)$.
For some small $t>0$, the point lies beyond the interval but remains
within the ball:
\begin{displaymath} 
\q = (1+t)\p' -t \v\in (B(\v,2)\cap \Omega')\setminus \Omega(V,\v).
\end{displaymath}
Pick $\u\in V\setminus V(\v,4)$ such that $\q\in A_+(\u,\v)$.  By the
triangle inequality,
\begin{displaymath} 
\norm{\u}{\v} \le \norm{\u}{\q} + \norm{\v}{\q} \le 2\norm{\v}{\q} < 4.
\end{displaymath}
This contradicts the assumption $\u\not\in V(\v,4)$.

The number of half-spaces $A_+(\v,\u)$, for $\u\in V(\v,4)$ is finite by
Lemma~\ref{lemma:V-finite}.  A set defined by the intersection of a finite number
of closed half-spaces is a polyhedron.
\end{proof}

\begin{lemma}[Voronoi compact]\guid{DRUQUFE}\rz{0}
\formalauthor{Nguyen Tat Thang}
\oldrating{80} 
Let $ V$ be a saturated packing.  For every $\v\in  V$, 
the Voronoi cell $\Omega( V,\v)$,  is
compact, convex, and measurable.
\end{lemma}

\begin{proof}  By the previous lemma, it is a bounded polyhedron.  Every bounded
polyhedron is compact, convex, and measurable.
\end{proof}




\subsection{reduction to a finite packing}

We finally state the main result of this book, the Kepler conjecture.
The proof fills most of this book. This section describes the
outline of the proof and gives references to the sources of the
details of the proof.


\begin{theorem}[Kepler's conjecture on Dense Packings]\guid{IJEKNGA}\rz{0} 
\label{theorem:kepler}   
\formaldef{Kepler conjecture}{kepler\_conjecture}%
No packing of congruent balls in Euclidean three space has density
greater than that of the face-centered cubic packing.
\end{theorem}
\indy{Index}{Kepler conjecture}

\begin{remark}\guid{LLFORJR}\rz{0}
This density is $\pi/\sqrt{18}\approx 0.74.$  There are other
packings, such as the hexagonal close packing or the face-centered cubic
packing with finitely many balls removed, that attain this
same density.
\end{remark}

The Kepler conjecture is a statement about space-filling packings.  A
space-filling packing is specified by a countable number of real coordinates,
three for the position of each of countably many balls.  The first
task in resolving the conjecture is to reduce the problem to one
involving only a finite number of balls.  This is accomplished by
Lemma~\ref{lemma:deltabound}.

The relevant concepts are {\it negligibility} and {\it
  fcc-compatibility}, given as follows.  Fcc-compatibility means that
the Voronoi cells on average have volume at least that of those in the
fcc-packing.  Negligibility means that the error term is insignificant.


\begin{definition}[negligible,~fcc-compatible]\guid{ZREKEVW}\label{def:negligible}
\formaldef{fcc compatible}{fcc\_compatible}
\formaldef{negligible}{negligible\_fun\_0}
A function $G:V\to \ring{R}$ on a set $V\subset\ring{R}^3$
is \newterm{negligible\/}
if there is a constant $c_1$ such that for all $r\ge1$,
% and all $\p\in\ring{R}^3$,
\begin{displaymath}\sum_{\v\in V(\orz,r)} G(\v) \le c_1
r^2.\end{displaymath}
A function $G: V\to\ring{R}$ is
\newterm{fcc-compatible\/}
if for all $\v\in V$, 
\begin{displaymath}\sqrt{32}\le \op{vol}(\Omega(V,\v)) +
G(\v).\end{displaymath}
\indy{Index}{negligible}%
\indy{Index}{fcc-compatible}%
\indy{Notation}{G (negligible function)}%
\end{definition}


\begin{remark}\guid{RTMZJVG}\rz{0}
The value $\op{vol}(\Omega(V,\v)) + G(\v)$ may be interpreted as an
{\it adjusted\/} volume of the Voronoi cell. The constant
$\sqrt{32}$ that appears in the definition of fcc-compatibility is
the volume of the Voronoi cell in the face-centered cubic and
hexagonal-close packings.  The corrected volume is at least the
volume of these Voronoi cells when the correction term $G$ is
fcc-compatible.  \indy{Index}{corrected volume}%
\end{remark}

% \begin{remark}\guid{GRKKKGN}\rz{0} In \cite{Hales:2006:DCG}, the full Voronoi cell
%   $\Omega(V,\v)$ is used, rather than $\Omega(V,\v)$.  The truncation at
%   radius $2$ is just a matter of convenience to guarantee the
%   boundedness and hence the finite volume of the (truncated) Voronoi
%   cell.  In \cite{Hales:2006:DCG}, the same effect was achieved by
%   requiring all packing%s
%   to be saturated.  Drop the assumption of saturation on $ V$.
%\end{remark}



The density $\delta( V,\p,r)$ of a packing $ V$ within a bounded
region of space is defined as a ratio. The numerator is volume of
$B(V,\p,r)$, defined as the intersection with $B(\p,r)$ of the union
of all balls in the packing.  The denominator is the volume of
$B(\p,r)$. 
%An ordered pair $( V,\v)$ with $\v\in V$ is called a \newterm{centered
%packing}.  \indy{Index}{packing!centered}%
\indy{Notation}{ZZdelta@$\delta( V,\p,r)$}%
\indy{Notation}{V@$ V$ (packing)}%
%\indy{Notation}{V@$ V^*$(packing)}%


\begin{lemma}[reduction to finite dimensions]\guid{JGXZYGW}\rz{0} 
\oldrating{150}
\formalauthor{Nguyen Tat Thang}
\label{lemma:deltabound} % 
If there exists a \newterm{negligible} \newterm{fcc-compatible}
function $G: V\to\ring{R}$ for a saturated packing $ V$, then there
exists a constant $c$ such that for all $r\ge1$,
% and all $\p\in\ring{R}^3$, %%  removed on  Jan 16, 2009 after formalization.
\begin{displaymath} 
\delta( V,\orz,r)
\le \pi/\sqrt{18} + c/r.
\end{displaymath}
%The constant $c$ depends on $ V$ only through the constant
%$c_1$ of Definition~\ref{def:negligible}.
\end{lemma}

%\begin{remark}\label{conj:fcc-neg} 
%For every saturated packing $ V$, there exists a negligible
%fcc-compatible function $G: V\to R$.
%\end{remark}



\begin{proof} 
The volume of $B( V,\orz,r)$ is at most the product of the volume
$4\pi/3$ of a single ball $4\pi/3$ with the number of centers in
$B(\orz,r+1)$.  Hence
\begin{equation} 
\op{vol}\, B( V,\orz,r) \le \card( V(\orz,r+1)) 4\pi/3.
\label{eqn:Abound}
\end{equation}

%In a %saturated packing 
Each truncated Voronoi cell is contained in a ball of
radius $2$ that is concentric with the unit ball in that cell.  The volume
of the large ball $B(\orz,r+3)$ is at least the combined volume of 
all truncated Voronoi
cells centered in $B(\orz,r+1)$. This observation,
combined with fcc-compatibility and negligibility, gives
\begin{equation} 
\begin{split} 
\sqrt{32}\,\,\card( V(\orz,r+1))
&\le \sum_{\v\in V(\orz,r+1)} (G(\v) +
\op{vol}(\Omega(V,\v))) \\
&\le c_1 (r+1)^2 + \op{vol}\,B(\orz,r+3) \\
&\le c_1 (r+1)^2 + (1+3/r)^3 \op{vol}\,B(\orz,r)
\label{eqn:Bbound}
\end{split}.
\end{equation}
\indy{Index}{fcc-compatible}%
Recall that $\delta( V,\orz,r)=
\op{vol}\,B( V,\orz,r)/\op{vol}\,B(\orz,r)$. Divide Inequality
\ref{eqn:Abound} through by $\op{vol}\,B(\orz,r)$.  Use
Inequality~\ref{eqn:Bbound} to eliminate $\card( V(\orz,r+1))$ from the
resulting inequality.  This gives
\begin{displaymath}\delta( V,\orz,r)
\le \frac{\pi}{\sqrt{18}} (1+3/r)^3 + c_1 \frac{(r+1)^2}{r^3\sqrt{32}}.
\end{displaymath}
The result follows for an appropriately chosen constant $c$
(depending on $c_1$).
\end{proof}

\begin{remark}[Kepler conjecture in precise terms]\guid{ZHIQGGN}\rz{0}
\label{remark:precise} 
The precise meaning of the \newterm{sphere packing problem} or the
\newterm{Kepler conjecture} is to prove the bound bound $\delta(
V,\orz,r) \le \pi/\sqrt{18} + c/r$ for every saturated packing $ V$.
The error term $c/r$ comes from the boundary effects of a bounded
container holding the balls.  The error tends to zero as the radius
$r$ of the container tends to infinity.  Thus, by the preceding lemma,
the existence of a negligible fcc-compatible function provides the
solution to the packing problem.  The strategy will be to define a
negligible function and then to solve an optimization problem in
finitely many variables to establish that the function is also
fcc-compatible.
\end{remark}





\section{Rogers Simplex}\label{sec:rogers}



% Think of $ V$ as the set of centers of a packing of congruent balls
% of radius $1$. To be saturated means that there is no room for
% further balls to be added to the packing. There is no loss in
% generality in assuming that the packing is saturated, when searching
% for the greatest possible density of a packing.

Rogers gave a bound on the density of sphere packings in Euclidean
space of arbitrary dimension~\cite{Rogers:1958:Packing}.  His bound
states that the density of a packing in $n$-dimensions cannot exceed
the ratio of the volume of $A \cap T$ to the volume of $T$, where $T$
is a regular tetrahedron of side length $2$ and $A$ is the set of
$n+1$ balls of radius $1$ placed at the extreme points of $T$.  In two
dimensions, the Rogers's bound is sharp and gives a solution to the
sphere packing problem.  In three dimensions the bound is
approximately $0.7797$, which differs significantly from the optimal
value $0.74\ldots$.  Rogers bound is the unattainable density that
would result if regular tetrahedra could tile
space.\footnote{Aristotle erroneously believed that regular tetrahedra
  tile space: ``It is agreed that there are only three plane figures
  which can fill a space, the triangle, the square, and the hexagon,
  and only two solids, the pyramid and the cube.''~\cite{Aristotle}.}

To prove his bound, Rogers gives a partition of Euclidean space into
simplices with extreme points in a packing $V$.   This section develops
the basis properties of Rogers simplices.  In the next section, we
will show how to modify the simplices to obtain a sharp bound on the
density of packings.




\subsection{faces}

Rogers partition is a refinement of the Voronoi cell decomposiiton.
In preparation for this decomposition, this subsection goes into
greater detail about the structure of the faces of a Voronoi cell.  We
parametrize various faces of the Voronoi cell by lists of points in a
saturated packing $V$.

\begin{definition}[$\Omega$ reprise]\guid{BBDTRGC} 
\formaldef{$\Omega(V,W)$}{voronoi\_set}
\formaldef{$\Omega(V,\bu)$}{voronoi\_list}
Let $V$ be a saturated packing.
The notation $\Omega(V,*)$ can be \newterm{overloaded} to denote intersections
of Voronoi cells, when the second argument is a set or list of points.
If $W\subset V$, %(or if $W$ is an ordered tuple of elements of $ V$), 
then the intersection of the family of Voronoi cells is  $\Omega(V,W)$:
\begin{displaymath}\Omega(V,W) = \bigcap \{\Omega(V, \u)\mid \u\in W
\}.\end{displaymath}
Define $\Omega$ on lists % by recursion:
%\begin{displaymath} 
%\begin{array}{lll} 
%\Omega(V,[]) &= \ring{R}^3,\\
%\Omega(V,\u_0\cooln[\u_1;\ldots;\u_k]) &= \Omega(V,\u_0)\cap \Omega(V,[\u_1;\ldots;\u_k]).\\
%\end{array}
%\end{displaymath}
to be the same as its value on point sets: 
\begin{eqnarray*} 
%\Omega(V,[]) &= \ring{R}^3,\\
%\Omega(V,\u_0\cooln\bu) &= \Omega(V,\u_0)\cap \Omega(V,\bu).\\
\Omega(V,[\u_0;\ldots;\u_k]) = \Omega(V,\{\u_0;\ldots;\u_k\}).
\end{eqnarray*}
%Call $\Omega(V,W)$ the $W$-face of $\Omega(V, \u)$ when $ \u\in W$.
\end{definition}

An intersection of Voronoi cells can be written in many equivalent forms:
\begin{displaymath} 
  \Omega(V,\v)\cap \Omega(V,\u) =\Omega(V,\{\u,\v\})= \Omega(V,\v)\cap A_+(\u,\v) 
  = \Omega(V,\v)\cap A(\u,\v) =  \cdots
\end{displaymath}





\begin{definition}[$\bV$]\guid{NOPZSEH} 
  \formaldef{$\bV(k)$}{barV} Let $V$ be a saturated packing.
  When $k=0,1,2,3$, let $ \bV(k)$ be the set of lists
  $\bu=[\u_0;\ldots;\u_k]$ of length $k+1$, with $ \u_i\in V$, such
  that
\begin{equation}\label{eqn:omega-dim} 
\dimaff(\Omega(V,[\u_0;\ldots;\u_j])) = 3-j,
\end{equation}
for all $0<j\le k$.  (Recall that $\dimaff(X)$ is the affine dimension
of $X$ from Definition~\ref{def:affine}.)  Set $\bV(k)=\emptyset$, for
$k>3$.  \indy{Notation}{dimaff@$\dimaff$}%
\end{definition}

In particular, $V$ can be identified with $\bV(0)$ under the natural
bijection $\v\mapsto[\v]$, and $\bV(1)$ is the set of lists $[\u;\v]$
of distinct elements such that the Voronoi cells at $ \u$ and $\v$
have a common facet (Lemma~\ref{lemma:omega-facet}).  

\begin{notation}[underscore]
  % Given a saturated packing $V$, and natural number $k$, the
  % definition gives $\bV(k)$ as a certain set of lists of points in
  % $V$.
  We hope that the use of underscores does not lead to confusion.  The
  underscore in $\bV(k)$ is a function
\begin{displaymath} 
\underline{\phantom V}:\{V \mid \text{$V$ saturated packing} \}
\times \ring{N} \to \ldots
\end{displaymath}
Contrast this with the use of the underscore in $\bu$.  Here the
underscore is not a function, but part of its name, following a
general typographic convention to mark lists of points. The two uses
are coherent in the sense that $\bu\in\bV(k)$.
\end{notation}

\begin{notation}[$\trunc{\bu}{j}$]\guid{JNRJQSM}\rz{0}
%\indy{Notation}{u@$\bu$ (list of points)}% index misformats
\indy{Notation}{d@$d_j$ (truncation of lists)}%
\formaldef{$\trunc{\bu}{j}$} {truncate\_simplex}
When $\bu=[\u_0;\ldots;\u_k]$ and $j\le k$, write
%\footnote{The  notation follows the syntax of Python slices.} 
$\trunc{\bu}{j} = 
[\u_0;\ldots;\u_j]$ for the truncation of the list.  
\indy{Notation}{1@$\trunc{*}{:*}$}%
\end{notation}

Truncation $\bu\mapsto\trunc{\bu}{j}$ maps $\bV(k)$ to $\bV(j)$, when
$j\le k$.  Beware of the index: $k$ is the {\it codimension} of
$\Omega(V,\bu)$ in $\ring{R}^3$, when $\bu\in \bV(k)$; it is not the
{\it length} of the list $\bu$ (which is $k+1$).\footnote{By
  convention aa $d$-simplex is presented as a $d+1$-tuple.  Because of
  this shift by one, the notation $\trunc{\bu}{j}$ also differs by the
  same shift.}


\begin{lemma}[Voronoi face]\guid{KHEJKCI}\rz{150}\label{lemma:omega-face}  
Let $V\subset\ring{R}^3$ be a saturated packing.
Let $\bu=[\u_0;\ldots;\u_{k}]\in \bV(k)$.  
Then $\Omega(V,\bu)$ is a face of $\Omega(V,\u_0)$.
\end{lemma}

\begin{proof} $\Omega(V,\bu)$ is the intersection of $\Omega(V,\u_0)$ with
the planes $A(\u_0,\u_i)$, where $i>0$.  
%Alternatively, it is the
%intersection of $\Omega(V,\u_0)$ with the planes $A(\u_i,\u_0)$, where $i>0$.
It follows directly from the definition of a face that each plane
$A(\u_0,\u_i)$ is a face of the polyhedron $A_+(\u_0,\u_i)$.  Thus, if
the ``open convex hull''
\begin{displaymath} 
\op{aff}^0_+(\emptyset,\{\p,\q\}) = \{ s \p + t \q \mid s>0,\quad t>0,~\quad s + t = 1\}
\end{displaymath} 
meets
$\Omega(V,\bu)$ for some $\p,\q\in \Omega(V,\u_0)\subset
A_+(\u_0,\u_i)$, then it also meets the face $A(\u_i,\u_0)$ and $\p,\q$
must also lie in the face $A(\u_0,\u_i)$ of the polyhedron
$A_+(\u_0,\u_i)$ containing $\p,\q$ (by
the definition of face).  Then $\p,\q$ also lie in
$\Omega(V,\bu)$.  By the definition of face, $\Omega(V,\bu)$ is a face
of $\Omega(V,\u_0)$.
\end{proof}

\begin{lemma}[facets]\guid{IDBEZAL}\rz{250}\label{lemma:omega-facet} 
  Let $V\subset\ring{R}^3$ be a saturated packing.  Let $\bu\in
  \bV(k)$, for some $k<3$.  Then $F$ is a facet of $\Omega(V,\bu)$ if
  and only if there exists $\bv\in \bV(k+1)$ such that $F =
  \Omega(V,\bv)$ and $\trunc{\bv}{k} = \bu$.
\end{lemma}

\begin{proof} 
  Use Lemma~\ref{lemma:V4} to write the polyhedron $\Omega(V,\bu)$ in
  the form of Equation~\ref{eqn:polyrep}:
\begin{displaymath} 
\Omega(V,\bu) = A \cap A_\pm(\v_1,\u_0) \cap \cdots\cap A_\pm(\v_r,\u_0),
\end{displaymath}
where $A$ is the affine hull of $\Omega(V,\bu)$, $\v_i\in V$, where
$A_-(\v,\w) = A_+(\w,\v)$ with the signs $\pm$ chosen as needed, and
$r$ is as small as possible.  By Lemma~\ref{lemma:webster}, if $F$ is
any facet of $\Omega(V,\bu)$, then there exists an $i\le r$ such that
\begin{displaymath} 
F = \Omega(V,\bu) \cap A(\v_i,\u_0) = \Omega(V,\bv),
\end{displaymath}
where $\bv = [\u_0;\cdots;\u_k;\v_i]$ is the list that appends $\v_i$ to $\bu$.
Also, since $F$ is a facet:
\begin{displaymath} 
\dimaff(\Omega(V,\bv)) = \dimaff(F) = \dimaff(\Omega(V,\bu)) - 1 = 3 - k - 1.
\end{displaymath}
So $\bv \in \bV(k+1)$.  This proves the implication in the forward direction.

To prove the converse, let $\bv\in \bV(k+1)$, where $\trunc{\bv}{k} =
\bu$.  Elementary verifications show that $\Omega(V,\bv)\subset
\Omega(V,\bu)$ and that this set is nonempty if $k<3$.  By
Lemma~\ref{lemma:omega-face} and Lemma~\ref{lemma:webster},
$\Omega(V,\bv)$ is a face of $\Omega(V,\bu)$.  By the definition of
$\bV(*)$,
\begin{displaymath} 
\dimaff(\Omega(V,\bv)) = 3 - (k+1) = \dimaff(\Omega(V,\bu)) -1,
\end{displaymath}
so $\Omega(V,\bv)$ is a facet of $\Omega(V,\bu))$.
\end{proof}


\subsection{partitioning space}

Each Rogers simplex is given as the convex hull of  its set of extreme points.  
The extreme points $\omega(d_i\bu)$
are defined by recursion.

\begin{definition}[$\omega$]\guid{JJGTQMN}
  \formaldef{$\omega_k$} {omega\_list\_n} \formaldef{$\omega$}
  {omega\_list} Let $V$ be a saturated packing and let
  $\bu=[\u_0;\ldots]\in \bV(k)$ for some $k$.  Define points
  $\omega_j=\omega_j(V,\bu)\in\ring{R}^3$ by recursion over $j$.
\begin{eqnarray*}
\omega_{0\phantom{+1}} &=& \u_0,\\
\omega_{j+1} &=&\text{the closest point to } \omega_j \text{ on }\Omega(V,\trunc{\bu}{j+1}).
\end{eqnarray*}
Set $\omega(V,\bu) = \omega_{k}(V,\bu)$, when $\bu\in \bV(k)$.  The set $V$ is generally fixed
and is dropped from the notation.
%By induction $\omega_j(\bu) = \
%Define $\omega$ \mid \coprod_{j=0}^3 \bV(j)\to \ring{R}^3$ by recursion over $j$ as follows.
%Let \begin{displaymath} 
%\omega([\u]) = \u,
%\end{displaymath} and
%let $\omega( \bu)$ be the closest point on $\Omega(V, \bu)$ to
%$\omega( \trunc{\bu}{j})$, when $\bu\in \bV(j+1)$.
\end{definition}

\claim{The point $\omega(\bu)$ exists when $\bu\in V(k)$.}  Indeed,
the set $\Omega(V, \bu)$ is nonempty, convex, and compact.  Thus, by
convex analysis, the closest point $\omega( \bu)$ exists uniquely.

The point $\omega_k$ depends on $\bu$ through its projection to
$\trunc{\bu}{k}$ so that
\[
\omega_k(\bu) = \omega_k(\trunc{\bu}{k})=\omega(\trunc{\bu}{k}).
\]
\indy{Notation}{zzomega@$\omega$ (extreme points of Rogers
  simplex)}%

\begin{definition}[R,~Rogers simplex]\guid{PHZVPFY} 
\formaldef{$R$}{rogers}
Let $V\subset\ring{R}^3$ be a saturated packing. For $\bu\in \bV(k)$, let 
 \begin{eqnarray*} 
 R(\bu) &=& \op{conv}\{\omega( \trunc{\bu}{0}), \omega(
 \trunc{\bu}{1}),\ldots,\omega( \trunc{\bu}{k})\}.\\
% R^0(\bu) &=& \op{aff}^0_+(\emptyset,\{\omega( \trunc{\bu}{0}), \omega(
% \trunc{\bu}{1}),\ldots,\omega( \trunc{\bu}{k})\}).
 \end{eqnarray*}  
%Set $R(\bu)=R(0,\bu)$.  
The set $R(\bu)$ is called the Rogers simplex of $\bu$.
\indy{Notation}{R@$R$ (Rogers simplex)}%
\end{definition}

Euclidean space can be partitioned into Rogers simplices.

\begin{lemma}[Rogers decomposition]\guid{GLTVHUM}\rz{700} 
For any saturated packing $V\subset\ring{R}^3$, and any $\u_0\in V$,
\begin{equation} 
\Omega(V,\u_0) = \bigcup \{ R(\bv) \mid \bv\in \bV(3),~\trunc{\bv}{0} =[\u_0]\}.
\end{equation}
Consequently,
\begin{displaymath}
\ring{R}^3 = \bigcup\, \{ R(\bv) \mid \bv\in\bV(3)\}.
\end{displaymath}
\end{lemma}

\begin{proof} 
The proof uses the following standard facts about convex sets and polyhedra from
Section~\ref{sec:poly}.


%
Now turn to the proof.  
By the covering of $\ring{R}^3$  by Voronoi cells by (\ref{eqn:vor-rn}),
%\begin{displaymath}\ring{R}^3 = \bigcup\, \{\Omega(V, \bu)\mid \bu\in
%\bV(0)\}.\end{displaymath}
it is enough to show that each Voronoi cell is covered by Rogers simplices.

Let $\bu\in \bV(j)$, for $j<3$.
Consider the following set:
\begin{displaymath} 
N = \left\{k\in\ring{N}\mid j\le k\le 3, ~~\Omega(V,\bu) 
= \bigcup_{\bv \in \bV(k),~\trunc{\bv}{j}=\bu}
\op{conv}(O_k \cup\Omega(V,\bv)) %\\&\qquad\qquad
%\mid
%\right\}
\right\},
\end{displaymath}
where $O_k = \{\omega(\trunc{\bv}{j}),\ldots,\omega(\trunc{\bv}{k-1})\}$.

\claim{We claim $N = \{j,\ldots,3\}$.}  Indeed, to see that $j\in N$, note that
\begin{displaymath} 
\Omega(V,\bu) = \op{conv}(\Omega(V,\bu)),
\end{displaymath}
which holds by the convexity of the polyhedron $\Omega(V,\bu)$.
Assume that $k\in N$, and consider the membership condition of $N$  for
$k+1$.  We may assume that $k+1\le 3$.
Then
\begin{eqnarray*} 
&\phantom{=}&\bigcup _{\bv \in \bV(k+1),~\trunc{\bv}{j}=\bu}
%\left\{
\op{conv}(O_{k+1} \cup\Omega(V,\bv))
%\mid%
%
%\right\}
\vspace{6pt}\\
&=&\bigcup _{\bv \in \bV(k+1),~\trunc{\bv}{j}=\bu}
%\left\{
\op{conv}(O_{k} \cup\op{conv}(\{\omega(\trunc{\bv}{k})\}\cup\Omega(V,\bv)))
%\mid
%\right\}
\vspace{6pt}\\
&=&\bigcup _{\bw\in \bV(k),~\trunc{\bw}{j}=\bu}
%\left\{
\bigcup_{\bv \in \bV(k+1),~\trunc{\bv}{k}=\bw}
%\left\{
\op{conv}(O_{k} \cup\op{conv}(\{\omega(\trunc{\bv}{k})\}\cup\Omega(V,\bv)))
%\mid
%\right\}\mid 
%\right\}
\vspace{6pt}\\
&=&\bigcup _{\bw\in \bV(k),~\trunc{\bw}{j}=\bu}
%\left\{
\op{conv}(O_{k} \cup \Omega(V,\bw)    )
%\mid 
%\right\}
\vspace{6pt}\\
&=&\quad\Omega(V,\bu).
\end{eqnarray*}
The induction hypothesis is used in the last step.  
This proves $k+1\in N$, and induction gives $N=\{j,\ldots,3\}$.

Consider  the extreme case $j=0$ and $k=3$.  The set $\Omega(V,\bv)$
reduces to $\{\omega(\bv)\}$  and the convex hull becomes
\begin{displaymath} 
\op{conv}(O_{k}\cup \Omega(V,\bv)) = R(\bv),
\end{displaymath}
when $\bv\in \bV(3)$.
This gives
\begin{equation} 
\Omega(V,\u_0) = 
\bigcup \{ R(\bv) \mid \bv\in \bV(3),~\trunc{\bv}{0} =[\u_0]\}.
\end{equation}
This proves the lemma.
\end{proof}


%%%
%\begin{displaymath}\Omega(V,\trunc{\bu}{0}) = 
%\bigcup\, \{ R(\bv) \mid \trunc{\bu}{0}=\trunc{\bv}{0},~\bv\in  \bV(3)\}.
%\end{displaymath}
%By Lemma~\ref{lemma:webster}, the relative boundary of the bounded
%polyhedron $\Omega(V,\trunc{\bu}{0})$ is the union of its facets.  The
%polyhedron can be partition into the cones over these facets.
%\begin{displaymath}\Omega(V,\trunc{\bu}{0}) = \bigcup\, \left\{
%  ~\op{conv}(\{\omega(\trunc{\bu}{0})\}\cup \Omega(V,\trunc{\bu}{1}))   
%   \mid \trunc{\bu}{1}\in  \bV(1)
%   \right\}.
%\end{displaymath}
%It is enough to show that
%\begin{displaymath} 
%\Omega(V,\trunc{\bu}{1}) = \bigcup\, 
%\{ R(1,\bv) \mid \trunc{\bv}{1}=\trunc{\bu}{1},~\bv\in  \bV(3)\}.
%\end{displaymath}
%After successively partitioning each facet into cones over facets of
%facets, it is enough to show that
%\begin{displaymath}\Omega(V,\trunc{\bu}{3}) 
%= \bigcup\,\{R(3,\bv)\mid\trunc{\bv}{3}=\trunc{\bu}{3},~\bv\in \bV(3)\}.\end{displaymath}
%The right-hand side is the singleton $\{\omega(\trunc{\bu}{3})\}$.
% The left-hand side contains this point and is contained in a
% properly decreasing chain of affine sets: $\ring{R}^3$, the bisector
% of $ \u_0$ and $ \u_1$, and so forth.  This determines a unique
% point, so the two sides are equal.
%\end{proof}

The intersection of two different Rogers simplices is a null set.

\begin{lemma}[Rogers disjoint]\guid{DUUNHOR}\rz{300}  \label{lemma:R-inter} 
Let $V$ be a saturated packing, and let $\bu,\bv\in \bV(3)$ be lists such that 
$R(\bu)\ne R(\bv)$.  Then the intersection 
\begin{displaymath} 
R(\bu)\cap R(\bv)
\end{displaymath}
is contained in a plane (and hence has measure zero).
\end{lemma}

This result and the previous lemma show that the simplices $R(\bu)$
partition Euclidean three-space.

\begin{proof} Let $\bu = [\u_0;\ldots]$ and $\bv = [\v_0;\ldots]$.  
Let $k$ be the
first index such that
\begin{displaymath} 
\Omega(V,[\u_0;\ldots;\u_{k}]) \ne \Omega(V,[\v_0;\ldots;\v_{k}]).
%\op{conv}\{\omega( [\u_0]),\ldots,\omega( [\u_0;\ldots; \u_{k}])\}\ne
%\op{conv}\{\omega([\v_0]),\ldots,\omega([\v_0;\ldots;\v_{k}])\}.
\end{displaymath}

\claim{Such an index $k$ exists.}  Indeed, the definition of points
$\omega(\trunc{\bu}{i})$ depends on $\bu$ only through the sets
$\Omega(V,\trunc{\bu}{j})$.  Hence, $R(\u)\ne R(\v)$ implies that the
two sequences $\Omega(V,*)$ must differ at some index.

The intersection $R(\u)\cap R(\v)$ lies in the convex hull $C$ of
$\{\omega([\u_0]),\ldots,\omega( [\u_0;\ldots; \u_{k-1}])\}$ and
$\Omega'=\Omega(V,[\u_0;\ldots;\u_{k};\v_{k}])$.  The set $\Omega'$
lies in a facet of $\Omega(V,\trunc{\bu}{k})$.  Hence the affine
dimension of $\Omega'$ is at most $3-k-1=2-k$.  In general, if a set
$A$ has affine dimension $r$, then the affine dimension of
$\op{conv}(\{\p\}\cup A)$ is at most $r+1$.  It follows that the
affine dimension of $C$ is at most $(2-k)+ k = 2$.  The intersection
is thus contained in a plane.
\end{proof}


\subsection{circumcenter}

The extreme points of a Rogers simplex are closely related to the circumcenter
of various subsets of $V$.  This subsection develops the connection between
Rogers simplices and circumcenters.

\begin{definition}[circumcenter,~circumradius]\guid{IFLFHKT} 
\formaldef{circumcenter}{circumcenter}%
\formaldef{circumradius}{radV}%
Let $S\subset\ring{R}^N$.  
A point $\p$ is a \newterm{circumcenter} of $S$ if it is an element
in the affine hull of $S$ that is equidistant from every $\v\in S$.  If $S$ has
circumcenter $\p$, then the common distance $\norm{\p}{\v}$, for all $\v\in S$,
is the \newterm{circumradius} of  $S$.
\end{definition}

The circumcenter comes as a solution to a system of linear equations.
We pause to review a standard result from the theory of linear
algebra, asserting the existence of a solution to a system of
equations.  Recall that a finite set $S$ is \newterm{affinely
  independent} if $\dimaff(S) = \card(S) -1$.  
%
\formaldef{affinely  independent}{\textasciitilde affine\_dependent s}%

\begin{lemma}[linear systems]\guid{QXSKIIT}\rz{500}\label{lemma:affine-system} 
  Let $S=\{\v_0,\ldots,\v_n\}\subset \ring{R}^N$ be an affinely
  independent set of cardinality $n+1$.  Then every system of
  equations
\begin{displaymath} 
\p \cdot (\v_i - \v_0) = b_i-b_0,\qquad i=1,\ldots,n,
\end{displaymath}
has a unique solution in $\p$ that lies in the affine hull of $S$.
\end{lemma}

\begin{proof} This is a standard result from linear algebra.
We sketch a proof for the sake of completeness.  

Let $\w_i = \v_i-\v_0$ and replace $b_i-b_0$ with $b_i$.  
The lemma reduces to the following claim.
Let $S' = \{\w_1,\ldots,\w_n\}$ be a \newterm{linearly independent} set
of cardinality $n$.  Then every system of equations
\begin{displaymath} 
\p \cdot \w_i = b_i,\qquad i=1,\ldots,n,
\end{displaymath}
has a unique solution in $\p$ that lies in the linear span of $S'$.

\claim{A solution is unique}. Indeed, the difference 
$\p = \p'-\p'' = \sum s_i \w_i$ of two solutions
$\p',\p''$ satisfies
\begin{displaymath} 
\normo{\p}^2=\p\cdot\p = \sum s_i \w_i \cdot (\p' - \p'') =
\sum s_i (b_i-b_i)= 0.
\end{displaymath}
So that $\p=0$ and $\p'=\p''$.  This proves uniqueness.

Let $W$ be the linear span of $\w_1,\ldots,\w_n$.  The image of the
map $W\to\ring{R}^n$, $\p\mapsto (\p\cdot\w_1,\ldots,\p\cdot\w_n)$ is
a linear space, hence an affine set.

\claim{A solution exists; that is, the image is all of $\ring{R}^n$.}
Otherwise, by Lemma~\ref{lemma:aff-u} some equation must hold; that
is, there exists $\u\ne \orz$ such that $\u\cdot \q =b$ for every
point $\q$ in the image.  As $\orz$ lies in the image, $b=0$.  Write
$\p = \sum u_i \w_i\in W$.  Then
\begin{displaymath} 
\normo{\p}^2 = \p\cdot\p = \sum u_i (\p \cdot \w_i) = \u\cdot \q = 0,
\end{displaymath} 
where $\q\in\ring{R}^n$ is the image of $\p\in W$.
Thus $\p=\orz$ so that $\u=0$, and we have reached a contradiction.
\end{proof}

\begin{lemma}[circumcenter exists]\guid{OAPVION}\rz{100} 
  Let $S\subset \ring{R}^N$ be a non-empty affinely independent set.
  Then there exists a unique circumcenter of $S$.
\end{lemma}

\begin{proof} 
  A point $\p$ is a circumcenter if and only if it is a point in the
  affine hull of $S$ that satisfies the system of equations:
\begin{displaymath} 
\norm{\p}{\v_i}^2 = \norm{\p}{\v_0}^2,\qquad i = i,\ldots,n.
\end{displaymath}
Equivalently,
\begin{displaymath} 
\p\cdot (\v_i-\v_0) = b_i-b_0,\qquad i=1,\ldots,n,
\end{displaymath}
where $b_i = \normo{\v_i}^2/2$.  By
Lemma~\ref{lemma:affine-system}, this system of equations has a unique
solution.
\end{proof}

The following lemma describes the structure of the affine hull of a
face of a Voronoi cell.  It describes the affine hull as an intersection of half-spaces,
and shows that it meets $\op{aff}(S)$ orthogonally at the circumcenter of $S$.

\begin{lemma}[]\guid{MHFTTZN}\rz{800}\label{lemma:aff-center} 
Let $V$ be a saturated packing and let $k\le 3$.
Let $\bu=[\u_0,\ldots,\u_k]\in \bV(k)$, and set $S = \{\u_0,\ldots,\u_k\}$.
Then
\begin{itemize} 
\item $\dimaff (S)= k$.  (In particular, $\card\{\u_0,\ldots,\u_k\}=k+1$ and
$S$ is affinely independent.)
\item $\aff{\Omega(V,\bu)}= \cap_{i=1}^k A(\u_0,\u_i).$
\item $\aff{\Omega(V,\bu)} \cap \aff(S) = \{\q\}$, 
where $\q$ is the circumcenter of $S$.
\item $(\aff{\Omega(V,\bu)}-\q) \perp (\aff(S)-\q)$, where
  $X-\q$ denotes the translate of a set $X$ by $-\q$, and $(\perp)$ is
  the orthogonality relation.
\end{itemize}
\end{lemma}
\indy{Notation}{1@$\perp$}


\begin{proof}  The proof is by induction on $k$.  

  \claim{The lemma holds when $k=0$.}  Indeed, $\Omega(V,\u_0)$ is an
  open set, so that its affine hull is $\ring{R}^3$.  This is the
  first conclusion.  The other conclusions reduce to trivial facts:
  $\dimaff\ring{R}^3 = 3$; $\dimaff\{\u_0\}=0$; $\ring{R}^3\cap
  \{\u_0\} = \{\u_0\}$; and $\ring{R}^3\perp \{\orz\}$.

Assume the induction hypothesis for $k$.  We may assume that $k<3$,
for otherwise there is nothing further to prove.  Let $\bu\in
\bV(k+1)$.  Let $\bv = \trunc{\bu}{k}\in \bV(k)$.  Let $\q$ be the
circumcenter of (the point set of) $\bv$.  Write $A_j = \cap_{i\le j}
A(\u_0,\u_i)$; $B_j = \aff(\Omega(V,\trunc{\bu}{j}))$; $C_j =
\aff\{\u_0,\ldots,\u_j\}$.   
%; $S_j = \{\u_0,\ldots,\u_j\}$.
By the induction hypothesis $A_k = B_k$.

\claim{We claim $\dimaff\{\u_0,\ldots,\u_{k+1}\}=k+1$.}
Otherwise, by general background facts about affine sets, $\u_{k+1}\in C_k$.
Write $\u_{k+1}-\q=\sum_{i\le k} t_i (\u_i-\q)$.  If $\p\in A_k$, then
by the orthogonality induction hypothesis:
\begin{eqnarray*} 
(\u_{k+1}-\q)\cdot (\p-\q) &=& \sum t_i (\u_i-\q)\cdot (\p-\q) = 0, \text{ and }\\
\norm{\u_{k+1}}{\p}^2 - \norm{\u_0}{\p}^2 &=&  
\norm{\u_{k+1}}{\q}^2 - \norm{\u_0}{\q}^2.
\end{eqnarray*}
Thus, if $A_k$ meets $A(\u_0,\u_{k+1})$, then $A_k\subset
A(\u_0,\u_{k+1})$.  This is contrary to $0\le \dimaff(A_{k+1}) =
\dimaff(A_k) - 1$, which holds because $\bu\in \bV(k+1)$ with $k<3$.

\claim{We claim that $B_{k+1} = A_{k+1}$.}  Indeed, by definition,
$B_{k+1}\subset A_{k+1}\subset A_k$.  Also,
\begin{displaymath} 
\dimaff B_{k+1} = 3 - (k+1) \le \dimaff{A_{k+1}} \le \dimaff A_k = 3 - k.
\end{displaymath}
Hence, by general background on affine sets, if $A_{k+1}\ne A_k$, then
$B_{k+1}=A_{k+1}$ follows.  Suppose for a contradiction that $A_k =
A_{k+1}$.  Then $\Omega(V,\bv) \subset \Omega(V,\bu) =
\Omega(V,\bv)\cap A(\u_0,\u_{k+1}) \subset \Omega(V,\bv)$, so that
$B_k = B_{k+1}$.  This contradicts the defining conditions of
$\bV(k+1)$.

\claim{We claim that $A_{k+1}\cap C_{k+1} = \{\q_{k+1}\}$, where
  $\q_{k+1}$ is the circumradius of $S_{k+1}$.}  Indeed, by the
definition of $A_{k+1}$, any point in this affine set is equidistant
from every point of $S_{k+1}$.  By the definition of $C_{k+1}$, the
point lies in the affine hull of $S_{k+1}$.  This uniquely
characterizes the circumcenter.

\claim{Finally, $(A_{k+1} -\q)\perp (C_{k+1}-\q)$, where $\q=\q_{k+1}$.}
Indeed, if $\p\in A_{k+1}$, then
\begin{eqnarray*} 
0 &=&\norm{\p}{\u_i}^2 -\norm{\p}{\u_0}^2\\
&=&\norm{(\p-\q)}{(\u_i-\q)}^2 -\norm{(\p-\q)}{(\u_0-\q)}^2\\
&=&-2 (\p-\q)\cdot (\u_i-\u_0).
\end{eqnarray*}
Since the linear span of the points $\u_i-\u_0$ is all of
$C_{k+1}-\q$, the claim follows.

This completes the induction.
\end{proof}

\begin{definition}[h]\guid{CHNGQBD}
\formaldef{h}{hl}% 
%Let $h(\trunc{\bu}{i}) = \norm{\omega(\trunc{\bu}{i})}{ \u_0}$.  
If $\bu=[\u_0;\u_0;\ldots;\u_k]$ is a list of points in $\ring{R}^N$.
let $h(\bu)$ be the
circumradius of its point set $\{ \u_0,\ldots, \u_k\}$.
\end{definition}
\indy{Notation}{h@$h$ (circumradius)}%

The constant $r=\sqrt2$ is the smallest real number $r$ such that
there exist four cocircular points in the plane with circumradius $r$.
The four points are the vertices of a square of side length $2$.
Eight Rogers simplices meet at the circumcenter of the square, but
when $r<\sqrt2$ only six Rogers simplices meet at the circumcenter.
In general, at $r=\sqrt2$, certain degeneracies start to appear that
cannot occur for a smaller radius.  To avoid degeneracies, we often
assume in the following lemmas that the circumradius is less than
$\sqrt2$.

\begin{lemma}[nondegeneracy]\guid{XYOFCGX}\rz{500}\label{lemma:sqrt2-close} 
  Let $V\subset\ring{R}^3$ be a saturated packing.  Let $S\subset V$
  be an affinely independent set with circumcenter $\p$.  Assume that
  the circumradius of $S$ is less than $\sqrt2$.  Then
  $\norm{\v}{\p}>\norm{\u}{\p}$, for all $\u\in S$ and all $\v\in
  V\setminus S$.
\end{lemma}

\begin{proof} 
Otherwise
there is a point $\w\in V\setminus S$ satisfying
\begin{equation}\label{eqn:closest} 
\norm{\w}{\p}\le \norm{ \u}{\p}, \quad\text{for all }  \u\in S.
\end{equation}
The angles $\arc_V(\p,\{\v, \u\})$ are obtuse for distinct elements
$\v,\u$ of $ S$ because of the law of cosines and
\begin{displaymath} 
\norm{\p}{\u} < \sqrt2,\quad \norm{\p}{ \v} <\sqrt2,\quad \norm{\u}{ \v} \ge 2.
\end{displaymath} 
Let $S=\{\u_0,\ldots,\u_k\}$.
A case-by-case argument follows, for each $k\in\{0,1,2,3\}$.

\claim{[$k=0$].}  This case is trivial.

\claim{[$k=1$].}  In this case, the points $\p, \u_0, \u_1$ are collinear and cannot give
two obtuse angles.

\claim{[$k=2$].} In this case, let $\w'$ be the projection of $\w$ to
the plane containing $\p, \u_0, \u_1, \u_2$.  Under orthogonal
projection, the angles remain obtuse:
\begin{displaymath} 
\arc_V(\p,\{\w,\u_i\}) = \arc_V(\p,\{\w',\u_i\}).
\end{displaymath}
The four points $\w', \u_0, \u_1, \u_2$ can
be arranged cyclically around $\p$, according to the polar cycle,
each forming an obtuse angle with
the next.  A circle around $\p$ cannot give four obtuse angles, because the sum is
$2\pi$.

\claim{[$k=3$].}
In this case, assume that $ \u_0,\ldots, \u_3$ are labeled according to the azimuth
cycle
around the line $\op{aff}\{\p,\w\}$.  Consider the dihedral angle
\begin{displaymath} 
\gamma=\gamma_i=\dih(\{\p,\w\},\{ \u_i, \u_{i+1}\})
\end{displaymath}
of the simplex $\{\p,\w,\u_i,\u_{i+1}\}$ along the edge $\{\p,\w\}$.
By the spherical law of cosines, the angle $\gamma$ of the
spherical triangle is given in terms of the edges as
\begin{displaymath} 
\cos c - \cos a \cos b = \sin a \sin b \cos \gamma.
\end{displaymath}
The angles $a,b,c$ are obtuse, so that both terms on the left-hand
side are negative. Thus, $\gamma>\pi/2$.  The azimuth angle
$\op{azim}(\p,\w,\u_i,\u_{i+1})$ is then also greater than $\pi/2$ by
Lemma~\ref{lemma:dih-azim}.  This is impossible, as the sum of the
four azimuth angles $\gamma$ is $2\pi$ by Lemma~\ref{lemma:2pi-sum}.
%This completes the proof that $\omega(\trunc{\bu}{j})$
%is the circumcenter.
\end{proof}

With nondegeneracy established, we can now give further details about
the extreme points of a Rogers simplex and their relationship to the
circumcenter of of a subset $S$ of the packing $V$.

\begin{lemma}[Rogers simplex and circumcenter]\guid{XNHPWAB}\rz{500}\label{lemma:v2} 
Let $V$ be a saturated packing.
Let $\bu=[\u_0;\ldots;\u_k]\in \bV(k)$, for some $k\le 3$,
and let $S=\{\u_0,\ldots,\u_k\}$ be the
point set of $\bu$.
Assume that $h(\bu)<\sqrt2$.
%\begin{displaymath} 
%\norm{\omega(\trunc{\bu}{j}) }{  \u_0} < \sqrt2.
%\end{displaymath}
Then 
%\begin{description} 
\begin{itemize} 
\item%[(circumcenter)]  
$\omega(\bu)$ is the circumcenter of $S$.
\item%[(convex hull)]  
$\omega(\bu)\in\op{conv}(S)$.
\item%[(distinctness)]  
The set $\{\omega(\trunc{\bu}{j})\mid j\le k\}$ has affine dimension $k$.
\item
The sequence $h(\trunc{\bu}{j})$ is
strictly increasing in $j$.
\end{itemize}
%\end{description}
\end{lemma}
\indy{Index}{convex hull}%

\begin{proof} The three conclusions of the lemma will be proved
separately.

\claim{$\omega(\bu)$ is the circumcenter of $S$.} Indeed, by
definition, if $\bu\in \bV(k)$, then
\begin{displaymath} 
\dimaff\Omega(V,[\u_0;\ldots;\u_k]) = 3-k.
\end{displaymath}  
The case $k=0$ of the lemma is trivially satisfied.  Assume by
induction the result holds for natural numbers up to $k$.

Now consider the case $k+1$.  Let $\bu\in \bV(k+1)$, and let $S$ be
the point set of $\bu$.  By the induction hypothesis
$\omega(\trunc{\bu}{k})$ is the circumcenter of the point set of
$\trunc{\bu}{k}$.  Let $\p$ be the point in
$A=\aff(\Omega(V,\bu))$ closest to $\omega(\trunc{\bu}{k})$.  By
Lemma~\ref{lemma:sqrt2-close}, the point $\p\in\Omega(V,S_{j})$.
Thus, $\p=\omega(\bu)$.  By Lemma~\ref{lemma:aff-center}, the
circumcenter of $S$ is the point of intersection of orthogonal affine
sets $\aff(S)$ and $A$.  Thus, the circumcenter equals the unique
point of $A$ closest to any point $\omega(\trunc{\bu}{k})$ in
$\aff(S)$.  The claim follows.

%The circumcenter $\p=\p_{j}$ of the set $S_{j}$ is
%the point on the plane $\op{aff}(\Omega(V,S_{j}))$ closest to
%$\p_{j-1}=\omega(\trunc{\bu}{j-1})$.

% If $j=0$, there is nothing to show.  If $j=1$, the point is the
% midpoint of the convex hull.  If $j=2$, the point is the
% circumcenter of an acute triangle.  If $j=3$, the point is the
% circumcenter of a simplex such that every face has positive
% orientation.  Thus, in every case the point lies in the convex hull.

\claim{We claim $\omega(\bu)\in\op{conv}(S)$.} 
Otherwise, there $\v\in S$ such that $\aff(S')$ separates $\omega(\bu)$ from
$ \v$, where $S'=S\setminus\{v\}$.  Let $\p'$ (resp. $\p=\omega(\bu)$)
be the circumcenter
of $S'$ (resp. $S$).  When $\u\in S'$, the law of cosines gives
\begin{eqnarray*} 
\norm{ \u}{\p}^2 &=& \norm{\u}{\p'}^2 + \norm{\p'}{\p}^2\\ 
\norm{ \v}{\p}^2 &\ge& \norm{\v}{\p'}^2 + \norm{\p'}{\p}^2.
\end{eqnarray*}
This gives $\norm{\v}{\p'}\le \norm{\u}{\p'}$.  This is contrary to
Lemma~\ref{lemma:sqrt2-close}.

\claim{The set $\{\omega(\trunc{\bu}{j})\mid j\le k\}$ has affine dimension $k$.}
%\claim{The points $\omega(\trunc{\bu}{j})$, for
%$j\le k$, are all distinct.}
It follows from  Lemma~\ref{lemma:aff-center} that the vectors 
${\omega(\trunc{\bu}{i+1})}-{\omega(\trunc{\bu}{i})}$ are mutually orthogonal.
Thus, the claim about affine dimension easily follows if we show that these vectors
are nonzero.
%Indeed, by the Pythagorean theorem,
%\begin{equation} 
%\norm{\omega(\trunc{\bu}{j})}{\omega(\trunc{\bu}{0})}^2 =
%\sum_{i=0}^{j-1} \norm{\omega(\trunc{\bu}{i+1})}{\omega(\trunc{\bu}{i})}^2.
%\end{equation}
%so it is enough to show that $\omega(\trunc{\bu}{i})\ne
%\omega(\trunc{\bu}{i+1})$.  
Otherwise, the
circumcenter $\omega(\trunc{\bu}{i})$ of $S_i=\{\u_0,\ldots,\u_i\}$
has an equally close point $ \u_{i+1}\in V\setminus S_i$, which is
impossible by Lemma~\ref{lemma:sqrt2-close}.

\claim{The sequence $h(\trunc{\bu}{j})$ is strictly increasing in
  $j$.}  
Indeed, by the Pythagorean theorem,
\begin{equation} 
\norm{\omega(\trunc{\bu}{j})}{\omega(\trunc{\bu}{0})}^2 =
\sum_{i=1}^{j} \norm{\omega(\trunc{\bu}{i})}{\omega(\trunc{\bu}{i-1})}^2.
\end{equation}
so the result follows from the
previous claim.
\end{proof}


% The concept of {\it positive orientation} is used in the proof.
% This is discussed in the 1998 proof and in {\it Lemmas in Geometry}.
% If a face has circumradius less than $\sqrt2$ it has positive
% orientation.  If every face has positive orientation, then the
% circumcenter of the simplex is contained in its convex hull.

\begin{lemma}\guid{WAUFCHE}\rz{100}
  Let $V$ be a saturated packing.  let $\bu =[\u_0;\ldots]\in
  \bV(k)$, for some $k$.  Then $h(\bu)\le
  \norm{\omega(\bu)}{\u_0}$.  Moreover, if $h(\bu)<\sqrt2$, then
  $h(\bu)=\norm{\omega(\bu)}{\u_0}$.
\end{lemma}

\begin{proof} By construction, the point $\omega(\bu)$ belongs to
  $\Omega(V,\bu)$, and is therefore equidistant to the points in
  $S=\{\u_0,\ldots,\u_k\}$.  The orthogonal projection of
  $\omega(\bu)$ to $\op{aff}(S)$ is the circumcenter of $S$.  The
  orthogonal projection cannot increase distances, so the inequality
  follows.  If $h(\bu)<\sqrt2$, then $\omega(\bu)$ is already the
  circumcenter by Lemma~\ref{lemma:v2}, so that equality holds.
\end{proof}

\subsection{Delaunay simplex}

The Delaunay decomposition of space into simplices is dual to the
Voronoi cell.  Two points in a packing $V$ are joined by an edge if
their Voronoi cells have a common facet.\footnote{The Delaunay
  decomposition may be degenerate if the points of $V$ are not in
  general position.  This book confines itself to the nondegenerate
  situation.}  A $2$-simplex is added with vertices at three points in
$V$ if their Voronoi cells have a common edge.  A $3$-simplex is added
for each common vertex.  A Delaunay $3$-simplex is the convex hull of
four points in the packing $V$.

Under a nondegeneracy condition (on the circumradius of the set of
points), we may construct a Delaunay simplex as a union of Rogers
simplices.  To this end, we examine the set of all Rogers simplices
around a common extreme point.  The convex hull of a nondegenerate set
$S\subset V$ of four points consists of $4!$ Rogers simplices, that
its facets consist of the facets of $3!$ Rogers simplices, and so
forth.  In brief, the Rogers simplices give every nondegenerate
Delaunay simplex an identical simplicial structure.


Recall that $\op{Sym}(k+1)$ be the \newterm{group} of all permutations on the
set $\{0,\ldots,k\}$.  Let $\bu = [\u_0,\ldots, \u_k]$ be a list of
length $k+1$.  For any \newterm{permutation} $\pi\in\op{Sym}(k+1)$,
let $\pi_*(\bu)$ be the \newterm{rearrangement} given by
\begin{displaymath} 
\pi_*(\bu)_i =  \u_{\pi^{-1} i}, % inverse added Feb 16, 2010. left regular action.
\end{displaymath}   
where $\u_i$ denotes the $i$th element of a list $\bu$.
\formaldef{permutation}{permutes}%
\formaldef{$\pi_*$}{left\_action\_list}%
\indy{Notation}{Xpi@$\pi$ (permutation)}
\indy{Notation}{Sym@$\op{Sym}$ (symmetric group)}

The following lemma shows that rearrangements have the same extreme
point of a Rogers simplex.

\begin{lemma}[extreme point rearrangement]\guid{YIFVQDV}\rz{500}   
\label{lemma:perm-Vk} 
  Let $V$ be a saturated packing.  Let $\bu\in \bV(k)$.  Assume that
  $h(\bu)<\sqrt2$. Let $\bv$ be any rearrangement of $\bu$ under a
  permutation.  Then $\bv\in \bV(k)$ and $\omega(\bu) = \omega(\bv)$.
\end{lemma}

\begin{proof} 
Let $\bv = [\v_0;\ldots;\v_k]$.  
Let $S_j = \{\v_0,\ldots,\v_j\}$,  
$\Omega_j = \Omega(V,\trunc{\bv}{j})$, 
$A_j=\cap_{i=1}^j A(\v_0,\v_i)$, $a_j = \dimaff(A_j)$.
By convention, set $A_0 = \ring{R}^3$ so that $a_0=\dimaff(A_0) = 3$.
Also, set $a_{-1} = 4$ by convention.

The set $S_k$ is the point set of $\bu$, which is affinely independent
by Lemma~\ref{lemma:aff-center}.  The set $S_j$ is then also affinely
independent.  Let $p_j$ be the circumcenter of $S_j$.  Then $p_k$ is
the circumcenter of $S_k$.  The circumradius of $S_j$ is at most the
circumradius of $S_k$, which by assumption is less than $\sqrt2$.

\claim{We claim that $\dimaff \Omega_j = a_j$, when $0\le j\le k$.}
By Lemma~\ref{lemma:sqrt2-close}, if $\p=\p_j$, then
\begin{equation}\label{eqn:sqrt2-close} 
\norm{\v}{\p} > \norm{\u}{\p}\text{ for all }\u\in S_j
\text{ and for all }\v\in V\setminus S_j.
\end{equation}   
Pick a small neighborhood $U$ of $\p_j$ such that (\ref{eqn:sqrt2-close}) holds
for all $\p\in U_j$.  By the definition of Voronoi cell, $\Omega_j \cap U=A_j\cap U$.
By background facts on affine sets $\dimaff\Omega_j = \dimaff A_j=a_j$.  This gives
the claim.

To prove the lemma, we prove the following claims by simultaneous induction on $j$.
For all $0\le j\le k$ we have
\begin{itemize} 
\item $a_j \ge a_{j-1} - 1\ge 3-j$.
\item $a_j = 3-j$ if and only if $a_i=3-i$ for all $0\le i\le j$.
\end{itemize}
The base case $j=0$ is trivial.  Assume the induction hypothesis for $j$.

$A_{j+1} = A_{j}\cap A(\v_0,\v_{j+1})$.  The intersection contains $\p_{j+1}$ and is therefore
nonempty.  By general background facts on the intersection of an
affine set with a hyperplane, $a_{j+1} \ge a_{j}-1$.  By the induction hypothesis,
$a_{j}-1\ge 3-(j+1)$.
If $a_{j+1}=3-(j+1)$, then $a_{j}=3-j$ and by the induction hypothesis
$a_{i}=3-i$, for all $0\le i\le j$. This completes the proofs of the claims by induction.

$a_k = \dimaff A_k=\dimaff \Omega_k$.  However, $\Omega_k=
\Omega(V,\bu)$, and since $\bu\in \bV(k)$, it follows that $0=\dimaff\Omega(V,\bu)=a_k$.
By the established claims, $a_i = 3-i$ for all $0\le i\le k$.  This proves $\bv\in\bV(k)$.

Finally $\omega(\bu) = \omega(\bv)$ because both equal the circumcenter of the
point set $S_k$.
%Since the sets $\Omega(V,\trunc{\bu}{j})$ satisfy \eqn{eqn:omega-dim}, it
%follows that $\Omega(V,\bu)\cap \op{conv}\{ \u_0,\ldots, \u_k\}$ is
%the singleton $\{\omega(\bu)\}$, which contains the circumcenter of
%the simplex with extreme points $\{ \u_0,\ldots, \u_k\}$.  This describes
%$\omega(\bu)$ in a way that does not depend on the ordering of $
%\u_0,\ldots, \u_k$.
%
%The condition \eqn{eqn:omega-dim} can be shown to hold for $\bv$.
%The proof of Lemma~\ref{lemma:v2} shows that the midpoint of $\v_0$
%and $\v_1\}$ lies in $\Omega(V,\trunc{\bv}{1})$.  By the distinctness
%conclusion of the same lemma, some neighborhood of this midpoint in
%the bisecting plane of $\{\v_0,\v_1\}$ lies in $\Omega(V,\trunc{\bv}{1})$.
%Thus, $\dimaff(\Omega(V,\trunc{\bv}{1}))=2$.  We continue in this fashion to show
%that
%\begin{displaymath} 
%\dimaff(\Omega(V,\trunc{\bv}{j}))=3-j.\end{displaymath}
\end{proof}

The map from permutations to Rogers simplices is one-one.

\begin{lemma}[permutations one-one]\guid{KSOQKWL}\rz{200} 
  Let $V$ be a saturated packing and let $\bu\in
  \bV(k)$.  Assume that $h(\bu)<\sqrt2$.  Let $\pi\in\op{Sym}(k+1)$ such that
 $R(\bu)= R(\pi_*\bu)$.  Then $\pi= I$.
\end{lemma}

\begin{proof} 
Write $\bv = \pi_*\bu$.  By Lemma~\ref{lemma:v2}, the sets
$\{\omega(\trunc{\bu}{j})\mid j\le k\}$ and 
$\{\omega(\trunc{\bv}{j})\mid j\le k\}$ are each affinely independent of
cardinality $k+1$.  By Lemma~\ref{lemma:simplex-poly}, these are these sets
of extreme points of $R(\bu)$ and $R(\bv)$, respectively.  Thus, it is enough
to show that the sets of extreme points are unequal.

Let $j$ be the largest index such that $\trunc{\bu}{j}=\trunc{\bv}{j}$.
The assumption $\pi\ne I$ implies that $j<k$.  Let $\p$ be the circumcenter
of $\{\u_0,\ldots,\u_{j+1}\}$.  By Lemma~\ref{lemma:sqrt2-close}, 
\begin{displaymath} 
\norm{\u_0}{\p} = \norm{\u_{j+1}}{\p} < \norm{\v_{j+1}}{\p}.
\end{displaymath}
Thus, $\omega(\trunc{\bu}{j+1}) \ne \omega(\trunc{\bv}{j+1})$.  The result follows.
\end{proof}

To prepare for Lemma~\ref{lemma:Rconv}, we need a preliminary lemma that
does some index shuffling for us.  It gives an  explicit representatives of the
cosets of $\op{Sym}(k+1)$ in $\op{Sym}(k+2)$.


\begin{definition}\guid{TSIVSKG}
\formaldef{$\bu^i$}{DROP}
Let $\bu$ be any list.
For each $i$,  let
$\bu^i = [\u_0;\ldots;\hat\u_i;\ldots]$, the list which drops the $i$th entry.
\end{definition}

\begin{lemma}[coset representatives]\guid{IVFICRK}\rz{300}\label{lemma:coset-bijection} 
There is a bijection between the set 
\begin{displaymath} 
\{(i,\sigma)\mid 0\le i\le k+1,\quad \sigma\in \op{Sym}(k+1)\}
\end{displaymath}
and $\op{Sym}(k+2)$ such that for any list $\bu$ of length $k+2$
\[
(\pi_*\bu)_j = \begin{cases} (\sigma_*(\bu^i))_j&0\le j \le k\\
  \bu_i & j=k+1.
\end{cases}
\]
\end{lemma}

\begin{proof} 
The bijection sends $(i,\sigma)$ to the permutation $\pi$, where
\begin{displaymath} 
\pi^{-1} j = \begin{cases} 
\sigma^{-1} j, & \sigma^{-1} j<i\\
(\sigma^{-1}j)+1 & \sigma^{-1} j \ge i\\
i& j=k+1.
\end{cases}
\end{displaymath}
This has the required properties.
\end{proof}

This lemma shows that each (nondegenerate) Delaunay simplex can be
partitioned as a union of Rogers simplices, indexed by the permutation
group.

\begin{lemma}[Delaunay simplex]\guid{WQPRRDY}\rz{450}\label{lemma:Rconv}  
  Let $V$ be a saturated packing and let $\bu = [\u_0;\ldots;\u_k]\in
  \bV(k)$.  Assume that $h(\bu)<\sqrt2$.  
  Then
\begin{displaymath} 
\op{conv}\{ \u_0,\ldots, \u_k\} = \bigcup \,\{ R(\pi_*\bu) \mid \pi\in \op{Sym}(k+1)\}.
\end{displaymath}
\end{lemma}
\indy{Notation}{Sym@$\op{Sym}$ (symmetric group)}%

\begin{proof} The proof is by induction on $k$.  The base case of the induction $k=0$
reduces to the trivial assertion: $\op{conv}\{\u_0\} = \op{conv}\{\u_0\}$.  
%Assume
%the result holds for $k$.


%For each $i$,  let
%$\bu^i = [\u_0;\ldots;\hat\u_i;\ldots;\u_{k+1}]$, which drops the $i$th entry.

\claim{We claim $\bu^i\in \bV(k)$, when $\bu=[\u_0;\ldots;\u_{k+1}]\in \bV(k+1)$.}  Indeed,  some permutation
$\pi\in \op{Sym}(k+2)$ carries $\bu$ to
$\bv=[\u_0;\ldots;\hat\u_i;\ldots;\u_{k+1};\u_i]$.  By
Lemma~\ref{lemma:perm-Vk}, $\bv\in \bV(k+1)$, so that $\bu^i =
\trunc{\bv}{k}\in \bV(k)$.

By the induction hypothesis 
\begin{equation}\label{eqn:sigma} 
\op{conv}(S\setminus\{\u_i\}) = \bigcup \,\{ R(\sigma_*\bu^i) \mid \sigma\in \op{Sym}(k+1)\},
\end{equation}
where $S = \{\u_0,\ldots,\u_{k+1}\}$.
By Lemma~\ref{lemma:simplex-poly}, the facets of the polyhedron
$\op{conv}(S)$ are the sets $\op{conv}(S\setminus\{\u_i\})$.
Lemma~\ref{lemma:facet-partition} gives the partition
\begin{equation} \label{eqn:convS}
\op{conv}(S) = \bigcup_{i=0}^{k+1} \op{conv}(\{\omega(\bu)\}\cup \op{conv}(S\setminus\{\u_i\})).
\end{equation}
Substitute the formula \eqn{eqn:sigma} into \eqn{eqn:convS}, then use the bijection of
Lemma~\ref{lemma:coset-bijection} to replace the double union by a single union
over $\pi\in \op{Sym}(k+2)$.  Background facts in affine
geometry then simplify the expression to the desired formula.  The
induction follows.
%
%
%Let $L = \{ \u_0,\ldots, \u_k\}$.  The proof is by
%induction on $k$.  When $k=0$, the result is trivial.  Now assume
%$k>0$.
%
%The circumcenter $\omega(\bu)$ of $L$ lies in the convex hull of
%these points.  (See the proof of Lemma~\ref{lemma:v2}.)  Thus, the
%left-hand side is the union of cones:
%\begin{displaymath} 
%\op{conv}(W) = \bigcup\,\{ \op{conv}(\omega(\bu),L\setminus \{ \u_i\}\mid i=0,\ldots,k) \}.
%\end{displaymath}
%The sets $L\setminus \{ \u_i\}$ can be identified with cosets of
%$\op{Sym}(k+1) /\op{Sym}(k)$.  By induction $L\setminus \{ \u_i\}$ is
%the union of $R(\bv)$ as $\bv$ runs over all permutations
%$\op{Sym}(k)$ of $L\setminus \{ \u_i\}$.  The result follows by
%induction.
\end{proof}



In summary of this section, by construction, the Rogers simplices
$R(\bu)$ are compatible with the Voronoi decomposition of space.
Under mild restrictions on the circumradius, they can also be
reassembled into simplices (the Delaunay simplices) with extreme
points at the centers of the packing, by Lemma~\ref{lemma:Rconv}.
\indy{Index}{simplex!Delaunay}%
\indy{Index}{partition!Rogers}%
\indy{Index}{decomposition!Voronoi} %




\section{Marchal Cells}



\cite{marchal:2009} has proposed an approach to sphere packings
 that gives significant improvements to the original proof
in~\cite{Hales:2006:DCG}.  The definition of $k$-cells,
conjecture~\ref{conj:m1}, Theorem~\ref{theorem:mk1}, and the method of
Lemma~\ref{lemma:13-14} are all due to him.  
\indy{Index}{Marchal, C.}%

% His articles claim to give a {\it demonstration} of the Kepler
% conjecture \cite{marchal:2009}.  However, the
% mathematically rigorous part of the article only gives a reduction
% of the problem to a difficult optimization problem in a finite
% number of variables.  The method of gradient descent is then used to
% explore the local minima of the optimization problem in finitely
% many variables.

Marchal's partition of space is a variant of Rogers's partition into
the simplices $R(\bu)$.  The main part of construction is the
decomposition obtained by truncating the Voronoi cells by a ball of
radius $\sqrt2$.  In a few carefully chosen situations, he assembles
the simplices $R(\bu)$ into larger convex cells (Delaunay cells), as
suggested by Lemma~\ref{lemma:Rconv}.




\begin{definition}[Marchal cells]\guid{QEEHXUB} 
  \formaldef{$i$-cell}{mcell} \formaldef{$\xi$}{mxi}%
  \indy{Index}{cell}%
  \indy{Index}{Marchal cell}%
  Let $V$ be a saturated packing.  Let $\bu=[\u_0;\ldots;\u_3]\in
  \bV(3)$.  Define $\xi(\bu)$ as follows.  If $\sqrt2\le
  h(\trunc{\bu}{2})$, then let $\xi(\bu)=\omega(\trunc{\bu}{2})$.  If
  $h(\trunc{\bu}{2})<\sqrt2\le h(\bu)$, define $\xi(\u)$
  to be the unique point in
\[
\op{conv}\{\omega(d_2\bu),\omega(\bu)\}
\]
at distance $\sqrt2$ from $\u_0$.  
%\begin{displaymath} 
%h(\trunc{\bu}{2}) <\sqrt2 \le h(\bu).
%\end{displaymath}
%When this inequality holds, there is a a unique point $\xi(\bu)$ in
%$\op{conv}\{\omega(\trunc{\bu}{2}),\omega(\bu)\}$ at distance exactly
%$\sqrt2$ from $ \u_0$.  
%\end{definition}
A set $\cell(\bu,i)\subset\ring{R}^3$ is associated with $\bu$ and $i=0,1,2,3,4$.  
\hfill\break\smallskip 
\case{The $0$-cell} of $\bu$
is
\begin{displaymath} 
\cell(\bu,0) = R(\bu)\setminus B(\u_0,\sqrt2).%
%\{\p\in R(\bu) \mid \norm{\p}{\u_0} \ge \sqrt2\}.
\end{displaymath}
\bigskip
\case{The $1$-cell} of $\bu$ is 
\begin{displaymath} 
\cell(\bu,1) = (R(\bu) \cap  \bar B(\u_0,\sqrt2))\setminus \op{rcone}^0(\u_0,\u_1,a),
\quad a={h(\trunc{\bu}{1})}/{\sqrt2}.
\end{displaymath}
%\op{conv}(\{\omega([ \u_0])\}\cup R_1),\hbox{ where } 
%R_1 = \{\p \in R(\bu) \mid \norm{\omega([ \u_0])}{\p}= \sqrt2\}.
%\end{displaymath}
%\indy{Notation}{R@$R_1$ (Rogers simplex)}%
\bigskip
\case{The $2$-cell} of $\bu$ is
empty unless $h(\trunc{\bu}{1})<\sqrt2\le h(\bu)$.  If nonempty, the $2$-cell is
(with $a$ as above)
\begin{eqnarray*} 
\cell(\bu,2) &=& 
 \op{rcone}(\u_0,\u_1,a)\cap \op{rcone}(\u_1,\u_0,a)\cap 
\op{aff}_+(\{\u_0,\u_1\},\{\xi(\u),\omega(\u)\}).
%\op{conv}( \{\u_0, \u_1\}\cup R_2),\quad\text{where }  \vspace{6pt}\\
%R_2 &=& \{\p \in R(\bu)\cap \Omega(V,\trunc{\bu}{1}) \mid 
%\norm{ \u_0}{\p}=\norm{ \u_1}{\p} =\sqrt2\}.
\end{eqnarray*}
\bigskip
\case{The $3$-cell} of $\bu$ is defined to be empty unless 
$h(\trunc{\bu}{2}) <\sqrt2 \le h(\bu)$.
If nonempty, $\xi(\bu)\in \op{conv}\{\omega(\trunc{\bu}{2}),\omega(\bu)\}$
%\formaldef{$\xi$}{mxi}%
and  the $3$ cell is
\begin{displaymath} 
\cell(\bu,3) = \op{conv}\{ \u_0, \u_1, \u_2,\xi(\bu)\}.
\end{displaymath}
\bigskip
\case{The $4$-cell} of $\bu$ is defined to be empty unless
$h(\bu) <\sqrt2$.
If nonempty, the $4$ cell is
\begin{displaymath} 
\cell(\bu,4) = \op{conv}\{ \u_0, \u_1, \u_2, \u_3\}.
\end{displaymath}
\end{definition}
\indy{Notation}{zzxi@$\xi(\cdot)$ (Marchal cell parameter)}%

The $0$ and $1$-cells are  subsets of a Rogers  simplex
$R$.  Yet, the $2$, $3$, and $4$-cells lie in a union of
simplices.  The index $i$ in  $\cell(\bu,i)$ indicates the number
of points of $V$ that are extreme points of the cell. 

This definition seems unmotivated at first.  Some history might help.
The 1998 proof of the Kepler conjecture partitioned space into a
hybrid of Voronoi cells and Delaunay simplices.  The Delaunay
simplices are fine tuned instruments for detail.  The Voronoi cells
are coarsely tuned, suitable for rough hewing.  Delaunay simplices
articulate the foreground; Voronoi cells recede into background.  The
solution to the problem lies in the right balance between foreground
and background.  Too many Delaunay simplices and the details
overwhelm.  Too many Voronoi cells and the estimates become too weak.
The central geometrical insights of the original proof are expessed as
rules that delineate foreground against background, Delaunay against
Voronoi.

Marchal takes the  hybrid decomposition and improves its
execution.  Marchal's $4$-cell is a Delaunay simplex.  His $0$- and
$1$-cells are parts of a Voronoi cell.  His $2$ and $3$-cells are
gradations between the two.

Examples show the shortcomings of a non-hybrid approach.
Recall that the density of the face centered cubic packing is
$\pi/\sqrt{81}=0.74048\ldots$.  Numerical evidence shows that an
approach based entirely on Delaunay simplices should give a bound of
about $0.740873$, a failure that comes tantalizingly
close~\cite{Hales:1992:JCAM}.  The dodecahedral theorem, asserting that the
Voronoi cell of smallest volume in the regular dodecahedron, gives the
bound of about $0.755$ ~\cite{Hales:2010:Dodec}.  Thus, the pure Voronoi cell
strategy fails as well.  The pure approaches can be modified in ways
that are conjectured to produce sharp bounds. The modified problems
are complex and daunting.
% constant 0.740873 checked 4/2010.
% constant 0.755 cross-checked with dodec.

Starting with L. Fejes T\'oth, it has become a common practice to
truncate Voronoi cells by intersecting them with a ball concentric
with the cell.  Different authors use different radii for the
truncating sphere: $XX$ (L. Fejes T\'oth), $\sqrt2$, $1.385$ and
$1.255$ (Ferguson and H.), $\sqrt2$ (Marchal).  A larger radius retains more
information (and complexity) than a smaller radius.  Marchal's
$0$-cells are the refuse, lying outside the ball of radius $\sqrt2$.
The $0$-cells are inconsequential to the proof.


\bigskip

\begin{definition}[$i$-rearrangement]\guid{BGXEVQU} 
\formaldef{$i$-rearrangement}{\_}
Let $\bu=[\u_0;\ldots;\u_k],\bv=[\v_0;\ldots;\v_k]$ be two lists of the same length.  
One is an
 $i$-\newterm{rearrangement} of the other if
$\pi_*\bu = \bv$ for some $\pi\in\op{Sym}(k+1)$ such
that 
$\pi(j) = j$  when $j \ge i$.
\end{definition}

In particular, if $\bu,\bv$ are $0$- or $1$-rearrangements of one another,
then $\bu = \bv$.


\begin{lemma}[]\guid{EMNWUUS}\rz{100}\label{lemma:M-complement4} 
Let $V$ be a saturated packing.  Let $\bu\in \bV(3)$.
The following are equivalent.
\begin{itemize} 
\item  $\cell(\bu,i)=\emptyset$ for $i=0,1,2,3$.
\item  $\cell(\bu,4)\ne\emptyset$.
\item  $h(\bu)<\sqrt2$.
\end{itemize}
\end{lemma}

\begin{proof} 
The diameter of $R(\bu)$ is easily seen to be $h(\bu)$.  Hence if $h(\bu)<\sqrt2$
all of the defining conditions are empty for $\cell(\bu,i)$, for $i<4$.  The result follows.
\end{proof}

\begin{lemma}[]\guid{SLTSTLO}\rz{200}\label{lemma:M-exhaust} 
Let $V$ be a saturated packing and let $\bu\in \bV(3)$. Then
%there exists a null set $Z$ such that 
every point in $R(\bu)$ 
belongs to $\cell(\bu,i)$  for some $0\le i\le 4$.  Furthermore, there is a null set  
$Z$ such that each point in $R(\bu)\setminus Z$ belongs to a unique
$\cell(\bu,i)$.
\end{lemma}

\begin{proof} 
Explicitly, the null set is the union of $R(\bu)\setminus R^0(\bu)$ (which lies in 
a finite union of planes), the sphere of radius $\sqrt2$ at $\u_0$, 
the difference $\op{rcone}(\u_0,\u_1,a)\setminus \op{rcone}^0(\u_0,\u_1,a)$,
and the
plane $\op{aff}\{\u_0,\u_1,\xi(\bu)\}$ when needed.  Let $p\in R(\bu)$.

\noindent
\claim{$[h(\bu)<\sqrt2]$.} In this case,  $\p\in\cell(\bu,4)$.

\noindent
\claim{$[\norm{\p}{\u_0} \ge\sqrt2]$.} In this case, $\p\in\cell(\bu,0)$.

\noindent
\claim{$[\p\not\in\op{rcone}^0(\u_0,\u_1,\omega(\trunc{\bu}{1})/\sqrt2)]$.} 
In this case, $\p\in \cell(\bu,1)$.

\noindent
\claim{$[\p\in \op{aff}_+(\{\u_0,\u_1\},\{\u_2,\xi(\bu)\})]$.} In this
case, $\p\in \cell(\bu,2)$.

\noindent
\claim{$[\p\in \op{aff}_+(\{\u_0,\u_1\},\{\xi(\bu),\u_3\})]$.} In this
case, $\p\in \cell(\bu,3)$.

When the corresponding strict inequalities are used, we obtain uniqueness for
$R(\bu)\setminus Z$.
\end{proof}




\begin{lemma}[]\guid{RVFXZBU}\rz{600}\label{lemma:marchal-equal} 
Let $V$ be a saturated packing, 
let $\bu,\bv\in \bV(3)$, and let $i,j\in \{0,1,2,3,4\}$.
%Assume that the affine dimension of $\cell(\bu,i)$ is three.
If the intersection, of
$\cell(\bu,i)$ with $\cell(\bv,j)$ has positive measure,
then $i=j$ and $\bu$ is an $i$-rearrangement of $\bv$.
Conversely, if $i=j$ and $\bu$ is an $i$-rearrangement of $\bv$, 
then $\cell(\bu,i)=\cell(\bv,j)$.
\end{lemma}

\begin{proof} 
The converse statement follows directly from the definition of Marchal cells.

Let $\cell(\bu,i)$ and $\cell(\bv,j)$ be two cells whose intersection
$X$ has positive measure.  There exists $\bw\in \bV(3)$ such that
$R(\bw)\cap X$ has positive measure.  In particular, $R(\bw)$ has
affine dimension three.  There are finitely may $R(\bw')$, for $\bw'\in
\bV(3)$ that meet $R(\bw)$.  By Lemma~\ref{lemma:R-inter}, by avoiding
finitely many planes (null sets), there exists subset $X'$ of positive measure
in
$X\cap R(\bw)$ such that $R(\bw)$ is the unique Rogers simplex that
meets $X'$.  Furthermore, by Lemma~\ref{lemma:M-exhaust}, there is a subset $X''$ of
positive measure in $X'$ and a natural number $k$ such that $\cell(\bw,k)$ is
the unique cell meeting $X''$.

Now $\cell(\bu,i)$, which contains $X''$, 
is contained in the union of the sets $R(\bu')$ as $\bu'$ runs over the
$i$-rearrangements of $\bu$.  Hence $\bw$ is an $i$-rearrangement of
$\bu$.  By the converse statement, replacing $\bu$ with an
$i$-rearrangement, we may assume that $\bu=\bw$.  Similarly, we may
assume that $\bv=\bw$.  By the exclusion of null sets as above, $i=j=k$.
\end{proof}



\begin{definition}\guid{LEPJBDJ}
\formaldef{$V(X)$}{VX}
  Let $V$ be a saturated packing and let $\bu=[\u_0;\ldots]\in
  \bV(3)$.  Let $X$ be a $k$-cell.  Define $V(X) =
  \{\u_0,\ldots,\u_{k-1}\}$.  In particular, if $X$ is a $0$-cell
  $V(X)=\emptyset$.
\end{definition}


\begin{lemma}\guid{URRPHBZ}\rz{220}\label{lemma:cell-radial}
  Let $V$ be a saturated packing and let $\bu\in \bV(3)$.  Then
  $X=\op{cell}(\bu,k)$ is measurable and eventually radial at each
  $\v\in V$.  Furthermore the cell $X$ is bounded away from every
  $v\in V\setminus V(X)$, so that the solid angle of $X$ is zero, except
  at $\v\in V(X)$.
\end{lemma}

\begin{proof} This follows from the fact that $R(\bu)$ is a simplex, and
$R(\bu)\cap V = \{\u_0\}$.
\end{proof}


%\begin{definition}\guid{ENBYLCBxo}
%Let $V$ be a saturated packing and let $X$ be a $k$-cell.
%Write $\op{solc}(\bu,k)$ for $\sol(\op{cell}(\bu,k),\u_0)$, where
%$\bu = [\u_0;\ldots]\in \bV(3)$.
%\indy{Notation}{solc@$\op{solc}$ (solid angle of a cell)}
%\end{definition}

\begin{lemma}\guid{QZYZMJC}\rz{200}
Let $V$ be a saturated packing.  For every $\v\in V$, 
\[
\sum_{X\mid \v\in V(X)}  \op{sol}(X,\v) = 4\pi.
\]
where the sum runs over all cells $X$ such that $\v\in V(X)$.
\end{lemma}

\begin{proof}  Indeed, the cells partition $\ring{R}^3$ and $\sol(B(\v,\epsilon)) = 4\pi$.
\end{proof}

\begin{definition}[$\op{tsol}$]\guid{LZYLTFY} 
\formaldef{$\op{tsol}$}{total\_solid}
  %Let $V\cap X$ be the intersection of $ V$ with the set of extreme
  %points of the $k$-cell $X$.  Explicitly, $V\cap X=\emptyset$ if $k=0$;
  %and $V\cap X = \{ \u_0,\ldots, \u_{k-1}\}$ in general.  Each $k$-cell
  %is measurable and eventually radial at each $\u\in V\cap X$.  
  Define
  the \newterm{total solid angle} of a cell $X$ to be
\begin{displaymath} 
\op{tsol}(X) = \sum_{\v \in V(X) } \op{sol}(X,\v).
\end{displaymath}
\end{definition}
\indy{Index}{angle!total solid}%
\indy{Index}{extreme point}%
%\indy{Notation}{VX@$V(X)$ (extreme points of a cell $X$)}
\indy{Notation}{tsol@$\op{tsol}$}%

%By Lemma~\ref{lemma:cell-radial}, the only terms that contribute to the sum are $\u\in V(X)$.

\begin{definition}[edge]\guid{WYORUNK}
\formaldef{$E(X)$}{edgeX}
  \indy{Notation}{EX@$E(X)$}%
  Let $E(X)$ be the set of \newterm{extremal edges} of the $k$-cell
  $X$ in a saturated packing $V$.  More precisely, let
\begin{displaymath}E(X)=\{\{ \u_i, \u_j\}\mid \u_i\ne \u_j\in
V( X)\}.\end{displaymath}
\indy{Notation}{1@$\tbinom{n}{k}$ (binomial coefficient)}%
\end{definition}

In particular, $E(X)$ is empty for $0$ and $1$-cells, and contains
$\tbinom{k}{2}$ pairs when $k\ge 2$.

\begin{definition}[$\op{dih}$]\guid{RSDYMHV} 
\formaldef{$\dih$}{dihX}
Let $V$ be a saturated packing, 
$\bu=[\u_0;\u_1;\u_2;\u_3]\in \bV(3)$, $X=\cell(\bu,k)$,
and $\ee=\{\u_0,\u_1\}\in E(X)$.
Let $\op{dih}(X,\ee)$ be the dihedral angle
along edge $\{\u_0,\u_1\}\in
E(X)$ in the $k$-cell $X = \op{cell}(\bu,k)$.
%be a $k$-cell.
%is defined to be the dihedral angle along that edge of the
%boundary.  
Explicitly, if $X$ is a null set, set $\op{dih}(X,\ee)=0$.  Otherwise,
set $\op{dih}(X,\ee)=\dih(\{\u_0,\u_1\},\{\v,\w\})$,
where 
\[
\{\v,\w\} = 
\begin{cases}
  \{\xi(\u),\omega(\bu) \} &  k=2\\
  \{\u_2,\xi(\u)\} & k=3\\
  \{\u_2,\u_3\} &k=4.
\end{cases}
\]
This is independent of the representation of the cell $X$ in the
form $\cell(\bu,k)$.
\index{Notation}{dih@$\dih$}%
\index{Index}{angle!dihedral}%
\index{Notation}{h@$h$ (half-edge length)} 
\end{definition}

Each
edge $\ee=\{ \u_i, \u_j\}\in E(X)$ determines half-length
$h(\ee) = \norm{ \u_i}{ \u_j}/2$.
This definition of $h$ is compatible with the previous definition for lists in the sense that
$h([\u_0;\u_1]) = h(\{\u_0,\u_1\})$.

\begin{lemma}\guid{GRUTOTI}\rz{500}
Let $V$ be a saturated packing.  Assume that $\v_0,\v_1\in V$ satisfy $\norm{\v_0}{\v_1}<\sqrt2$.  Set $\ee=\{\v_0,\v_1\}$.
Then
\[
\sum_{X\mid \ee\in E(X) } \op{dih}(X,\ee) = 2\pi.
\]
The sum runs over cells $X$ such that $\ee\in E(X)$.
\end{lemma}

\begin{proof} 
The sum of the azimuth angles of the cyclic set 
$S=\{\omega(\bu) \mid \bu =[\v_0;\v_1;\ldots]\in \bV(3)$ is $2\pi$
by Lemma~\ref{lemma:2pi-sum}.   Each azimuth angle is less than $\pi$ and is thus
equal to a dihedral angle.  The sum does not change, if we add additional points
$\xi(\bu)$ and $\omega(\trunc{\bu}{2})$ to $S$.  The result follows.
\end{proof}
%
%The dihedral angle of each $4$-cell $\op{cell}([\u_0;\u_1;\u_2;\u_3],4)$ is the sum of the two corresponding Rogers simplices
%$R([\u_0;\u_1;\u_3;\u_4])$ and $R([\u_0;\u_1;\u_4;\u_3])$.  The dihedral
%angles of $2$- and $3$-cells combine at the point $\xi(\bu)$ to give
%the dihedral angle of the Rogers simplex.  Thus, it is enough to
%show that 
%\[
%\sum_{\bv =[\v_0;\v_1;\ldots]\in \bV(3)} \op{dih}(\{\v_0,\v_1\},\{\omega(\trunc{\bv}{2}),\omega(\bv)\}) = 2\pi.
%\]
%The points $\omega(\trunc{\bv}{2})$ are by construction points on
%the edge of a Voronoi cell between two extreme points
%$\omega(\bv)$ and $\omega(\bv')$.  Thus, the two dihedral terms
%involving $\omega(\trunc{\bv}{2})$ can be combined into a single term
%$\dih(\{\v_0,\v_1\},\{\omega(\bv),\omega(\bv')\}$. The set of points
%\[
%\{\omega(\bv) \mid \bv\in \bV(3), \bv = [\v_0;\v_1;\ldots]\}
%\]
%is a cyclic set with respect to
%  $(\v_0,\v_1)$.   By Lemma~\ref{lemma:2pi-sum} the
%  sum of the azimuth (dihedral) angles is $2\pi$.



%
%
%
%
%First, the proof establishes that cells are either
%disjoint (up to a null set) or equal.  At the same time, it
%determines exactly when two $k$-cells are equal to one another.
%
%By Lemma~\ref{lemma:Rconv}, the $4$-cell $\cell(\bu,4)$ is a union
%of the simplices $R(\bv)$, as $\bv$ runs over rearrangements of $\bu$.
%Two $4$-cells that meet in a set of positive measure are equal.
%Every $4$-cell is a union of simplices $R(\bu)$ with
%$h(\bu)<\sqrt2$.  This condition gives $\cell(\bu,i)=\emptyset$, for
%$i=0,1,2,3$.
%
%Similarly, the $3$-cell is a union of the convex hulls
%$\op{conv}(\{\xi(\bv)\}\cup R(\trunc{\bv}{2}))$ as $\trunc{\bv}{2}$ runs over
%rearrangements of $\trunc{\bu}{2}$.  Note that the point
%$\xi(\bv)=\xi(\bu)$ is independent of the rearrangement, since
%it is determined as the point at distance $\sqrt2$ from $ \u_0$
%along the line through from $\omega(\trunc{\bv}{2})=\omega(\trunc{\bu}{2})$
%perpendicular to the plane $\op{aff}\{ \u_0, \u_1, \u_2\}$ (in the
%half-space of $\trunc{\bu}{3}$). Two $3$-cells that meet in a set of
%positive measure are equal.  Their parameters are equal:
%$\trunc{\bu}{2}=\trunc{\bv}{2}$.  The intersection $\cell(\bu,3)$ cannot meet
%$\cell(\bu,i)$, for $i<3$, in a set of affine dimension three, because the
%plane $\op{aff}\{ \u_0, \u_1,\p\}$ separates them.
%
%The $2$-cell with parameter $\bu$ is contained in the right-circular cone
%\begin{displaymath} 
%\op{rcone}( \u_0, \u_1,\cos(\op{arc}(\norm{ \u_0}{ \u_1},\sqrt2,\sqrt2)))
%\end{displaymath}
%This cone separates the $2$-cell from the $0$- and $1$-cells with the
%same parameter $\bu$.  The $1$-cell is separated from $0$-cells by the
%sphere of radius $\sqrt2$, centered at $ \u_0$.


%\begin{lemma}[]\guid{PDVQTUO}\rz{0} 
%Let $V$ be a saturated packing.  If two Marchal cells of $V$ are not equal, then
%their intersection lies in a null measure set.
%\end{lemma}






\subsection{cells revisited}

The Marchal cells can be described in an alternative intuitive way.
If $S\subset\ring{R}^3$, let
\[
\op{equi}(S,r) = \{ \p \mid \norm{\p}{\v} = r \text{ for all } \v \in S \}.
\]


Let $V$ be a saturated packing.
Define
\[
X(S) = \op{conv} (S\cup\op{equi}(S, \sqrt2) ),
\]
for $S\subset V$.  The set $X(S)$ is empty if the circumradius of $S$
is greater than $\sqrt2$.

The set $X(\emptyset)$ is $\ring{R}^3$.  The set $X(\{\w\})$ is a ball
of radius $\sqrt2$ with center $\w$.  The set $X(\{\v,\w\})$ is a
double cone, $X(\{\u,\v,\w\})$ a bipyramid, and $X(\{{\mathbf
  t},\u,\v,\w\})$ is a simplex.

\begin{lemma}
Let $V$ be a saturated packing.
If $S\subset V$ is not empty, then $X(S)$ is contained in the union of sets
\[
X(S\setminus \{\v\}),  \quad \v\in S.
\]
\end{lemma}

\begin{lemma}
Let $V$ be a saturated packing and let $S\subset V$. 
 If $\op{card}(S)=k$, then
\[Y(S) = X(S) \setminus \bigcup_{S\subset S'\subset V} X(S'),\]
where $S'\subset V$ runs over subsets of cardinality $k+1$ that contain $S$,
equals a finite union of Marchal $k$-cells, up to a null set.
\end{lemma}

It is possible to base a construction of Marchal cells on this lemma,
dispensing entirely with Voronoi cells and Rogers simplices.  This is
essentially what Marchal does.  It is quick and intuitive.  We have
followed a longer path that gives more detail about the structure of
cells.



\subsection{Marchal's conjecture}

This section shows how the existence of a fcc-compatible negligible
function is a consequence of an explicit
inequality related to the the distances $h(\bu)$, where $\bu\in \bV(1)$.

%\begin{note}%
%Replace $\sol_0/\pi =\Delta_1$ to simplify formulas.
%\end{note}


\begin{definition}[$\sol_0$,~$\tau_0$,~$m_1$,~$m_2$,~$h_+$,~$M$]\guid{AOZUTMU} 
\formaldef{$\sol_0$}{sol0}%
\formaldef{$\tau_0$}{tau0}%
\formaldef{$m_1$}{mm1}%
\formaldef{$m_2$}{mm2}%
\formaldef{$h_+$}{hplus}%
\formaldef{$M$}{marchal}%
Define the following constants and functions: 
\begin{eqnarray}\label{eqn:m-def} 
\sol_0 &=& 3\arccos(1/3)-\pi\\
%\Delta_1 &=& (3\arccos(1/3)-\pi)/\pi\\
\tau_0 &=& 4\pi  - 20\sol_0\\
m_1 &=& \sol_0 2\sqrt2/\tau_0 = 1.012\ldots \\ %% K 
m_2  &=&  (6\sol_0- \pi)\sqrt2/(6 \tau_0) = 0.0254\ldots\\ %% M 
h_+ &=& 1.3254 \hbox{~(exact rational value)}
\end{eqnarray}
Let $M:\ring{R}\to\ring{R}$ 
be the following piecewise polynomial function (Figure~\ref{fig:M}):
\begin{equation}\label{eqn:M} 
M(h) =
\begin{cases} 
% (\sqrt2-h) (h-1.3254) (9h^2 - 17 h + 3)/(1.627 (\sqrt2-1))& h\le\sqrt2\\
\dfrac{\sqrt2-h}{\sqrt2-1}~ \dfrac{h_+-h}{h_+-1} ~\dfrac{17 h - 9 h^2 - 3}{5} 
& h \le \sqrt2.\vspace{3pt} \\
0 & h >\sqrt2.
\end{cases}
\\
\end{equation}
\end{definition}

\begin{figure}[htb]
\centering
\szincludegraphics[width=60mm]{\pdfp/Mfun.eps}
% Plot[Mfun[h], {h, 1, Sqrt[2]}]
% copied to Preview, then saved, then converted to eps via pdf2eps.
\caption{The quartic polynomial $M$.}
\label{fig:M}
\end{figure}

The constant $\sol_0$
is the area of a spherical triangle with sides $\pi/3$.
Simple calculations based on the definitions give
\begin{equation}\label{eqn:km}m_1 - 12m_2 = \sqrt{1/2}\end{equation} 
and
\begin{equation}M(1) = 1,\quad M(h_+)=0,\quad M(\sqrt2) =0.\end{equation} 

\begin{definition}[$\gamma$]\guid{KGFDCFM} 
\formaldef{$\gamma$}{gammaX}
For any cell $X$ of a saturated packing, 
define the functional $\gamma(X,\cdot)$ on  $\{f:\ring{R}\to\ring{R}\}$ by
\begin{equation}\label{eqn:gamma-def} 
\gamma(X,f) =  \op{vol}(X)
-\left(\frac{2m_1}{\pi}\right) \op{tsol}(X) + \left(\frac{8m_2}{\pi}\right)
\sum_{\ee\in E(X)} \dih(X,\ee)  f(h(\ee))
\indy{Notation}{ZZddgamma@$\gamma$ (fundamental estimate)}%
\end{equation}
\end{definition}


\begin{theorem}[Marchal's inequality]\guid{HJKDESR}\rz{0}\label{lemma:MI} 
Let $V$ be any saturated packing, and let $X$ be any Marchal cell of $V$.  Then
\begin{equation}\label{eqn:mfe} 
\gamma(X,M)\ge 0,
\end{equation}
where $M$ is the function defined in (\ref{eqn:M}).
\end{theorem}

\begin{proof}  See~\cite{marchal:2009}.  We return to this inequality in the appendix,
to provide our own (computer) proof.
\end{proof}
%% cc:mar are the k-cell estimates for non-cell clusters.
%By Calculation~\ref{calc:marchal}, Marchal's fundamental estimate
%holds for any cell


\begin{conjecture}[Marchal]\guid{PHNFUXP}\rz{0}\label{conj:m1} 
For any packing $ V$, and
any $ \u\in V$,
\begin{displaymath} 
\sum_{\v\in V\setminus\{\u\}} M(h([\u;\v])) \le 12.
\end{displaymath}
\end{conjecture}

Marchal's conjecture is still open.  This book proves a variant of
Marchal's conjecture.

\begin{theorem}\guid{KIZHLTL}\rz{750}\label{theorem:mk1} 
% formalization must follow the statements in pack_concl.hl, 250 each estimate.
Marchal's conjecture~\ref{conj:m1} implies
that for every saturated packing $V$, there exists a negligible fcc-compatible function
$G:V\to \ring{R}$.
\end{theorem}


\begin{proof} 
It is enough to show that $G( \u_0,f) = -\op{vol}(\Omega(V, \u_0)) + 8
m_1 - \sum 8 m_2 f(h([\u_0;\u]))$ is fcc-compatible and negligible, when $f=M$.
The sum runs over $\u\in V\setminus\{\u_0\}$.
%The result then follows from Lemma~\ref{lemma:deltabound}.  
The function $G(\cdot,M)$ is fcc-compatible directly
by equation~\eqn{eqn:km}
and conjecture~\ref{conj:m1}:
\indy{Index}{negligible}%
\indy{Index}{fcc-compatible}%
\begin{eqnarray*} 
\sqrt{32} &=& 8 m_1 - 8\cdot 12 m_2\\
&\le& 8 m_1 - 8 m_2 \sum M(h)\\
&=& \op{vol}(\Omega(V, \u_0)) + G( \u_0,M).
\end{eqnarray*}
The issue is to prove that it is negligible.  In view of future applications, 
we work with a general bounded function $f$ instead
of the function $M$.   More explicitly, we show that there is a
constant $c$ such that for all $r\ge 1$:% and all $\p\in\ring{R}^3$:
\begin{equation}\label{eqn:neg} 
-\sum G( \u,f) = \sum \op{vol}(\Omega(V, \u)) 
-\sum 8m_1 + \sum \sum 8 m_2 f(h) \ge \sum_{X\subset B(\orz,r)} \gamma(X,f)  + c r^2,
\end{equation}
where the outer nested sum runs over $ \u\in  V(\orz,r)$.  

Lemmas~\ref{lemma:Zr2} and \ref{lemma:V-finite} show that the number
of points of $ V$ near the boundary of $B(\orz,r)$ is at most $c r^2$.


The sum of $\gamma(X,f)$ over all cells in a large ball
$B(\orz,r)$ has the form $T_1 + T_2 + T_3$ for
three terms $T_i = T_i(r)$ of \eqn{eqn:gamma-def}.  The desired
equation~\eqn{eqn:neg} consists of three corresponding terms
$T'_i(r)$.  It is enough to show that there exist constants $c_i$ such that
\begin{displaymath} 
T_i'(r) \ge T_i(r) + c_i r^2.
\end{displaymath}

The sum of the volumes of the Voronoi cells $ \u\in B(\orz,r)$ is not
exactly the volume of $B(\orz,r)$, because of the contribution at the
boundary of $B(\orz,r)$ of Voronoi cells that are only partly contained
in $B(\orz,r)$.  Similarly, the sum of the various $k$-cells, for
$X\subset B(\orz,r)$ is not exactly the volume of $B(\orz,r)$, because of
contribution from the boundary. The boundary contributions have order
$r^2$. Thus,
\begin{displaymath} 
T_1'= \sum_{ \u\in  V(\orz,r)} \op{vol}(\Omega(V, \u)) 
\ge \sum_{X\subset B(\orz,r)} \op{vol}(X) + c_1 r^2 = T_1 + c_1 r^2.
\end{displaymath}


The estimates on the other terms are similar.  The solid angles
around each point sum to $4\pi$.
In Landau big O notation, this gives
\begin{eqnarray*} 
\sum_{X\subset B(\orz,r)} \op{tsol}(X) &=& 
\sum_{X\subset B(\orz,r)} \sum_{ \u\in V( X)} \sol(X, \u)\\
&=&\sum_{ \u\in  V(\orz,r)} \sum_{X\mid  \u\in V( X)} \sol(X, \u) + O(r^2)\\
&=&\sum_{ \u\in  V(\orz,r)} 4\pi    + O(r^2).
\end{eqnarray*}
Hence
\begin{displaymath} 
T_2' = -\sum_{ V(\orz,r)} 8 m_1 = 
-\sum_{X\subset B(\orz,r)}\left(\frac{2m_1}\pi\right) \op{tsol(X)} + O(r^2) = T_2 + O(r^2).
\end{displaymath}
Similarly, the dihedral angles around each edge sum to $2\pi$.  A
factor of $2$ enters the following calculation, because there are two
ordered pairs for each unordered pair $\ee=\{ \u_0, \u_1\}$:
\begin{eqnarray*} 
&\phantom{=}&\sum_{X\subset B(\orz,r)} \sum_{~~\ee\in E(X)} \dih(X,\ee)  f(h(\ee)) \vspace{3pt}\\
&=&\sum_{\ee\subset B(\orz,r)} \sum_{~~X\mid \ee\in E(X)} \dih(X,\ee)  f(h(\ee)) +O(r^2)\vspace{3pt}\\
&=&\sum_{\ee\subset B(\orz,r)} 2\pi f(h(\ee)) + O(r^2)\vspace{3pt} \\
&=&\sum_{ \u_0\in  V(\orz,r)} \sum_{~~\u_1\in  V(\orz,r) } \pi f(h( \u_0, \u_1)) + O(r^2).\\
\end{eqnarray*}
Finally,
\begin{eqnarray*} 
T_3' &=& \sum\sum 8 m_2 f(h(\bu)) \\
&\ge& \left(\frac{8m_2}\pi\right)
\sum_{X\subset B(\orz,r)}\sum_{\ee\in E(X)}\dih(X,\ee) f(h(\ee)) + O(r^2) \\
&=& T_3 + O(r^2).
\end{eqnarray*}
\end{proof}




\section{Clusters}

Marchal's conjecture~\ref{conj:m1} is still open.  This section
introduces a variant of Marchal's conjecture.  In this variant,
a piecewise linear function $L$ replaces the piecewise polynomial
function $M$.  More crucially, the support of the function $L$ is
contained in $\leftclosed2,2.52\rightclosed$.  By contrast, the
support of Marchal's function is much larger:
$\leftclosed2,2.6508\rightclosed$.  This small difference in the
support of the function creates an enormous difference in the
difficulty of the conjectures.

The conjecture formulated in this section also implies the existence
of fcc-compatible negligible functions.  To prove this existence result, it is
helpful to group Marchal cells together into new aggregates, called
\newterm{clusters}.  This section makes a detailed study of clusters
in order to produce a negligible function.

The sections and chapters that follow give a proof of the
conjecture in this section.  The proof of this conjecture is the
main intermediate result in the proof of the Kepler
conjecture.


% This section shows how to improve on the estimates of the previous
% section by combining various cells into {\it cell clusters}.
Recall that $M(h_+) = 0$, where   $h_+ = 1.3254$.
\indy{Index}{cell cluster}%

\begin{definition}[$L$,~$h_0$,~$h_-$]\guid{ULZRABY}\label{def:L} 
\formaldef{$h_0$}{h0}
\formaldef{$h_-$}{hminus}
\formaldef{$L$}{lmfun}
Set
\begin{displaymath} 
h_0 = 1.26\\  %%\hm
\end{displaymath}
Let $L:\ring{R}\to\ring{R}$ be the piecewise linear function 
\begin{displaymath} 
L(h) = \begin{cases} 
\dfrac{h_0-h}{h_0-1}, & h \le h_0 \\
0, & h\ge h_0. \\
\end{cases}
\end{displaymath}
It follows from the definition that
\begin{displaymath} 
L(1) = 1\quad L(\hm) = 0.
\end{displaymath}
Let $h_- = 1.23175\ldots$ be the unique root of the quartic polynomial
$M(h)-L(h)$ that lies in the interval $[1.231,1.232]$.
\indy{Notation}{L@$L$ (linear function)}%
\indy{Notation}{h@$h_- = 1.23175\ldots$}%
\indy{Notation}{h@$h_0 = 1.26$}%
\end{definition}

%%
\begin{figure}[htb]
\centering
\szincludegraphics[width=60mm]{\pdfp/Lfun.eps}
% Plot[{Mfun[h],Lfun[h]}, {h, 1.2, 1.35}]
% copied to Preview, then saved, then converted to eps via pdf2eps.
%% WW very big .eps file!
\caption{Detail of the quartic $M$ and linear function $L$.}
\label{fig:L}
\end{figure}

The inequality $L(h)\ge M(h)$ holds except when $h\in [h_-,h_+]$.  The
aim of this section is to prove a variant (Theorem~\ref{theorem:mk2})
of Theorem~\ref{theorem:mk1}
that uses the function $L$ rather than $M$.  For this, one needs to
combine cells into groups called cell clusters.

\begin{definition}[critical edge,~$\op{EC}$,~$\op{wt}$]\guid{MZSRVBC}\label{def:wt} 
A \newterm{critical edge} $\ee$ of a saturated packing $V$ is an unordered pair
that appears as an element of $E(X)$ for some 
$k$-cell $X$ of the packing $V$, and such that
$h(\ee)\in[h_-,h_+]$.  Let $\op{EC}(X)$ 
be the set of critical edges that belong to $E(X)$.  If $X$ is any cell such
that $\op{EC}(X)$ is nonempty, let the \newterm{weight} $\op{wt}(X)$ of $X$ be
$1/\card(\op{EC}(X))$.
\end{definition}
\indy{Index}{critical!edge}%
\indy{Notation}{EC@$\op{EC}$ (critical edges)}%
\indy{Notation}{wt@$\op{wt}$ (weight)}%

\begin{definition}[$\beta_0$,~$\beta$]\guid{PQFEXQN}\label{def:beta} 
\formaldef{$\beta_0$}{bump}
\formaldef{$\beta$}{beta\_bump}
Set 
\begin{displaymath} 
\beta_0(h) = 0.005 (1 - (h-h_0)^2/(h_+-h_0)^2).
\end{displaymath}
If $X$ is a $4$-cell with exactly two critical edges and if those edges
are opposite, then set
\begin{displaymath} 
\beta(\ee,X) = \beta_0(h(\ee)) - \beta_0(h(\ee')), 
\text{ where }\op{EC}(X) = \{\ee,\ee'\} .  
\end{displaymath}
Otherwise, for all other edges in all other cells, set $\beta(\ee,X) = 0$.
\end{definition}
\indy{Notation}{ZZbeta0@$\beta_0$}%
\indy{Notation}{ZZbeta@$\beta$ (bump)}%

\begin{definition}[cell cluster,~$\Gamma$]\guid{YSULGYR}\label{def:gammaL} 
\formaldef{cell cluster}{cell\_cluster}
\formaldef{$\Gamma$}{cluster\_gammaX}
  Let $V$ be a saturated packing.  Let $\ee\in \op{EC}(X)$ be a
  critical edge of a $k$-cell $X$ of $V$ for some $2\le k\le 4$.  A
  \newterm{cell cluster} is the set
\begin{displaymath} 
\op{CL}(\ee) = \{X\mid \ee\in \op{EC}(X)\} 
\end{displaymath}
\indy{Notation}{cluster}%
of all cells around $\ee$. 
If $Z$ is a subset of a cell cluster $\op{CL}(\ee)$, define
\indy{Notation}{CL@$\op{CL}$ (cell cluster)}%
\begin{displaymath} 
\Gamma(\ee,Z) = \sum_{X\in Z} \gamma(X,L) \op{wt}(X) +\beta(\ee,X).
\end{displaymath}
%and where $\op{wt}(X)$ is the weight of $X$.
\indy{Notation}{ZZddGamma@$\Gamma$}%


\end{definition}
\indy{Index}{cell cluster}%

\begin{theorem}[cell cluster estimate]\guid{OXLZLEZ}\rz{1500} 
\label{lemma:cluster}
Let $\op{CL}(\ee)$ be any cell cluster of a critical edge $\ee$ in a saturated packing $V$.  
Then $\Gamma(\ee,\op{CL}(\ee))\ge 0$.
\end{theorem}

The proof of this cell cluster estimate is a numerical calculation
that has been carried out by computer.  Further discussion about the
methods appears in the appendix. The proof of the following assertion
is extremely long and complex.  It relies on many computer
calculations.  The non-computer parts of the proof take up most of the
remainder of the book.%
%
%

\begin{assertion}\guid{BJERBNU}\rz{0}\label{conj:L12} 
  For any  saturated packing $ V$, and any $ \u_0\in V$,
\begin{equation}\label{eqn:L12} 
\sum_{ \u_1\in V\mid h( \u_0, \u_1)\le \hm} L(h\{\u_0, \u_1\}) \le 12.
\end{equation}
\end{assertion}

\begin{lemma}[]\guid{UPFZBZM}\rz{600}\label{theorem:mk2} 
Assertion~\ref{conj:L12} implies
that for every saturated packing $V$, there exists a negligible fcc-compatible function
$G:V\to \ring{R}$.
\end{lemma}

\begin{remark}\label{rem:L12KC}
In light of Lemma~\ref{lemma:deltabound}, assertion~\ref{conj:L12}
implies the Kepler conjecture. 
\end{remark}

\begin{proof} The proof imitates the proof of
Theorem~\ref{theorem:mk1}.  It is enough to show that $G_L( \u_0) =
-\op{vol}(\Omega(V, \u_0)) + 8 m_1 - \sum 8 m_2 L(h(\bu))$ is
fcc-compatible and negligible.  The function $G_L$ is fcc-compatible
directly from equation~\eqn{eqn:km} and Assertion~\ref{conj:L12}.

The theorem follows from a proof that $G_L$ is negligible.  More
precisely, one needs to show there exists a constant $c$ such that
for all $r\ge 1$:% and all $\orz\in\ring{R}^3$:
\begin{equation}\label{eqn:A2neg} 
\sum \op{vol}(\Omega(V, \u)) -\sum 8m_1 + \sum \sum 8 m_2 L(h) \ge c r^2,
\end{equation}

For cells $X$ that do not belong to any cell cluster,
the proof is just as in the proof of Theorem~\ref{theorem:mk1}.
If $\op{EC}(X)=\emptyset$, then 
$L(h(\ee))\ge M(h(\ee))$ for each edge $\ee\in E(X)$, and
\begin{displaymath}\gamma(X,L)\ge \gamma(X,M)\ge 0\end{displaymath} 
by inequality \eqn{eqn:mfe}.

Note that the function $\beta(\ee,X)$ averages to zero for any $4$-cell $X$:
\begin{displaymath} 
\sum_{\ee\in \op{EC}(X)} \beta(\ee,X) = 0.
\end{displaymath}
Hence the terms involving $\beta$ in sums may be disregarded in this proof.
(These terms may be disregarded here, but they are
needed for the proof of Lemma~\ref{lemma:cluster}.)

Theorem~\ref{lemma:cluster} gives the required inequality for cell
clusters.  Again, using big O notation,
\begin{eqnarray*} 
\sum_{X\subset B(\orz,r)} \gamma(X,L) &=&
\sum_{X\subset B(\orz,r)\mid \op{EC}(X)\ne\emptyset} \gamma(X,L) +
\sum_{X\subset B(\orz,r)\mid \op{EC}(X)=\emptyset} \gamma(X,L) \vspace{6pt}\\
&\ge& \sum_{X\subset B(\orz,r)\mid \op{EC}(X)\ne\emptyset} \gamma(X,L)\vspace{6pt} \\
&=&\sum_{X\subset B(\orz,r)}\gamma(X,L)\sum_{\ee \in \op{EC}(X)}\op{wt}(X) + O(r^2)\vspace{6pt}\\
&=&\sum_{\ee\subset B(\orz,r)}\sum_{X\mid \ee \in \op{EC}(X)}\gamma(X,L)\op{wt}(X) + O(r^2)\vspace{6pt}\\
&=&\sum_{\ee\subset B(\orz,r)}\Gamma(\ee,\op{CL}(\ee)) + O(r^2)\vspace{6pt}\\
&\ge& ~~O(r^2).
\end{eqnarray*}


From the definition of $\gamma$, the sum $\sum \gamma(X,L)$ may be
expanded as the sum of three terms, $T_1+T_2+T_3$, and compared term
by term with \eqn{eqn:A2neg}:
\begin{displaymath} 
T_i' \ge T_i + c_i r^2.
\end{displaymath}
This proceeds exactly as in the proof of Theorem~\ref{theorem:mk1}.
\end{proof}

\begin{definition}[$\BB$]\guid{WTKURHK} 
\formaldef{$\BB$}{ball\_annulus}
Let $\BB$ be the
\newterm{annulus} $\bar B(\orz,2h_0)\setminus B(\orz,2)$, where
$\bar B(\orz,r)$ is the closed ball of radius $r$.
\indy{Notation}{BB@$\BB$}
\end{definition}


\begin{corollary}\guid{RDWKARC}\rz{200}\label{cor:CE} 
  If the Kepler conjecture is false, there exists a finite packing
  $V\subset\BB$ such that
\begin{equation}\label{eqn:CE} 
\sum_{ \u\in V} L(h\{\orz, \u\}) > 12.
\end{equation}
\end{corollary}

The proof of the Kepler conjecture proceeds by assuming that there is
a countexample to Assertion~\ref{conj:L12} and then deriving a contradiction.
This corollary formulates the potential counterexample in slightly simpler terms.

\begin{proof} If the Kepler conjecture is false,
Assertion~\ref{conj:L12} is violated for some packing $ V$ and some
$ \u_0\in V$.  After the replacement of $ V$ with $ V - \u_0$ and $
\u_0$ with $\orz$, it follows without loss of generality that $
\u_0=\orz\in V$.  After the replacement of $ V$ with the finite
subset
$V\cap \BB$,
it follows without loss of generality that the packing is a finite subset of $\BB$.
\end{proof}



\section{Counting Spheres}

This section proves two estimates about a packing $V\subset \BB$ that
satisfies the inequality~\ref{eqn:CE}.  The first estimate
(Lemma~\ref{lemma:13-14}) shows that the cardinality of $V$ is thirteen,
fourteen, or fifteen.  The second estimate (See Lemma~\ref{lemma:D'}.)  shows
that no point $\v\in V$ can be strongly isolated from the other points
of $V$.    To prove these two estimates, we need a formula for the smallest
possible area of a spherical polygon that contains a disk.  This formula is
developed in the first subsection.

\subsection{solid angle}
\indy{Index}{polygon}%
\indy{Index}{polygon!regular}%


\begin{lemma}[]\guid{GOTCJAH}\rz{450}\label{lemma:ngon} 
Let $P$ be a bounded polyhedron in $\ring{R}^3$ that contains $\orz$
as an interior point.  Let $F$ be a facet of $P$, given by an
equation
\begin{displaymath} 
F = \{\p \mid \p \cdot \v = b\} \cap P.
\end{displaymath} 
Let $W_F$ be the corresponding topological component of $Y(V_P,E_P)$.  
Assume that $W_F$ contains the right-circular cone 
\begin{displaymath} 
\op{rcone}^0(\orz,\v,h)
\end{displaymath}
for some $h>0$.
Then 
\begin{displaymath} 
\sol(W_F) \ge 
2\pi - 2 k \,\arcsin\left(\,h\sin(\pi/k)\,\right),
\end{displaymath}
where $k$ is the number of edges of $F$.
\end{lemma}

%\begin{lemma}[]\guid{ZZ}\rz{0}\label{lemma:ngon:old} 
%  Let $C$ be a circle on the unit sphere with arcradius $a<\pi/2$.
%  Among all spherical $n$-gons that contain $C$ (that is, among all
%  $n$-fold intersections of hemispheres containing $C$), that of
%  minimal area is the regular $n$-gon.
%\end{lemma}
%
%In other words, the minimal configuration consists of the
% intersection of $n$-hemispheres whose bounding great circles are
% tangent to the circle $C$ at $n$-equally spaced bounds around $C$.

\begin{proof} 
\claim{Without loss of generality, we may assume that each edge of
$F$ meets $\op{rcone}(\orz,\v,h)$ in a unique point.}  Indeed,
each edge of $F$ is the intersection of $F$ with another facet
$F_i$.  Write
\begin{displaymath} 
F_i = \{\p \mid \p \cdot \v_i = b_i\} \cap P.
\end{displaymath}
The region $W_F$ consists of points $\p$ for which there exists a
$t>0$ such that $t\p \in\op{ri}(F)$ (by Lemma~\ref{lemma:WF}).  Hence
if we produce a second polyhedron $P'$ and facet $F'$ with
$\op{ri}(F')\subset \op{ri}(F)$, then
\begin{displaymath} 
\sol(W_F)\ge \sol(W_{F'}).
\end{displaymath}
Shift the facet $F_i$ to
\begin{displaymath} 
\op{aff}(F'_i) = \{\p \mid \p \cdot \v_i = b'_i\}.
\end{displaymath}
where $b'_i$ is chosen so that the point $\p_i\in \op{aff}(F'_i)\cap
\op{aff}(F)$ and lies on the boundary of $\op{rcone}(\orz,\v,h)$.
Define $P'$ by the intersection of $P$ with the half-spaces
\begin{displaymath} 
\{\p \mid \p \cdot \v_i \le b'_i\},
\end{displaymath}
where the signs are chosen so that $b'_i>0$.  Let $F' = P'\cap F$.
Then $F'$ is a facet of $P'$.  The polyhedron $P'$ and facet $F'$
satisfy the assumptions of the lemma with the same constant $k= k_F =
k_{F'}$.  This completes the proof that we may assume that each edge
of $F$ meets $\op{rcone}(\orz,\v,h)$ in a unique point.  That is, the
edge is tangent to the right-circular cone.

Drop the primes from the notation: $P=P'$, $F=F'$, and so forth.  The
Rogers partition gives a partition of the polyhedron $P$ into
simplices.  There are $2k$ simplices in (the closure of) $W_F$.  The
solid angle of each simplex is the area of a spherical triangle.

Consider a spherical triangle with sides $a,b,c$ and opposite angles
$\alpha,\beta,\gamma$.  If $\gamma=\pi/2$, then by Girard's formula,
the area of the triangle is
\begin{displaymath} 
\alpha+\beta-\pi/2,
\end{displaymath}
and by the law of cosines 
\begin{displaymath} 
\cos(\alpha) =\sin(\beta)\cos(a).
\end{displaymath}
This determines the area $g(a,\beta)$ of the triangle 
as a function of $a$ and $\beta$. 
\indy{Index}{Girard's formula}%
\indy{Notation}{g@$g$ (triangle area)}%
\indy{Notation}{ZZalpha@$\alpha$ (angle)}%
\indy{Notation}{ZZbeta@$\beta$ (angle)}%
\indy{Notation}{ZZddgamma@$\gamma$ (angle)}%
\indy{Index}{convex}%
\indy{Index}{Girard's formula}%
\indy{Index}{polygon}%

The solid angle of $W_F$ is the sum of the areas of the triangles:
\begin{displaymath} 
\sum_{i=1}^k g(a,\beta_i) 2,
\end{displaymath}
with angle sum
\begin{displaymath} 
\sum_{i=1}^k \beta_i = 2\pi.
\end{displaymath}
With  $a$ fixed, the second partial of $A$ with respect to $\beta$ is
\begin{displaymath} 
\frac{\partial^2 g(a,\beta)}{\partial \beta^2} = 
\frac{\cos(a)\sin^2(a)\sin(\beta)}{\sin^2(\alpha)} > 0.
\end{displaymath}
The function is convex.
By convexity, the minimum area occurs when all angles are equal
$\beta=\beta_i = \pi/k$.

The solid angle bound of the lemma is equal to 
\begin{displaymath} 
2 k g(a,\beta)
\end{displaymath}
where $\cos(a)=h$.  Alternatively, the polygon breaks into $2k$
triangles, each computed by Girard's formula to have area
\begin{displaymath} 
\beta - (\pi/2 - \alpha)  = \pi/k - \arcsin(\cos(\alpha)) = 
\pi/k - \arcsin(\cos(a)\sin(\beta)).
\end{displaymath}
\end{proof}

%\begin{lemma}[]\guid{BBEVFIC}\rz{0}\label{lemma:ngon-area} 
%  The minimum area of an intersection of $k$-hemispheres containing a
%  circle $C$ of arcradius $a<\pi/2$ is
%\begin{displaymath} 
%2\pi - 2 k \,\arcsin\left(\,\cos(a)\sin(\beta)\,\right),
%\end{displaymath}
%when $\beta = \pi/k$.
%\end{lemma}



\subsection{a polyhedral bound}

\begin{definition}[weakly saturated]\guid{HUCFLEB} 
\label{def:weakly-saturated}
Let $r$ and $r'$ be real numbers such that $2\le r\le r'$.  Define a
set $ V\subset\ring{R}^3\setminus B(\orz,2)$ to be \newterm{weakly saturated} with
parameters $(r,r')$ if for every $\p\in\ring{R}^3$
\begin{displaymath} 
2\le\normo{\p}\le r'~~~\implies~~~ \exists \u\in V.~\norm{ \u}{\p}< r.
\end{displaymath}
\end{definition}

\begin{lemma}[]\guid{TARJJUW}\rz{0}\label{lemma:poly-bounded} 
\oldrating{120}
\formalauthor{Dang Tat Dat}
Fix $r$ and $r'$ such that $2\le r\le r'$.
Let $ V$ be a weakly saturated finite packing with parameters $(r,r')$.
%such that $\orz\in  V$.
%where 
%   $\orz\in V$, and
%   $\normo{ \u}\le r'$ for all $ \u\in V$,
%Set $ V^*= V\setminus\{\orz\}$.
For any $g: V\to\ring{R}$, let $P( V,g)$ be the
polyhedron given by the intersection of half-spaces
\begin{displaymath} 
\{\p \mid  \u\cdot \p \le g( \u)\},\quad \u\in V.
\end{displaymath}
Then $P( V,g)$ is bounded.
\end{lemma}
\indy{Index}{polyhedron}%

\begin{proof} Assume for a contradiction that $P=P( V,g)$ is
  unbounded, and there exists $\p\in P$ such that $\normo{\p} > g( \u)
  r'/2$ for all $ \u\in V$.  Let $\v = r' \p/\normo{\p}$ so that
  $r'=\normo{\v}$.  By the weak saturation of $ V$, there exists $
  \u\in V$ such that $\norm{\v}{ \u}<r$.  Then,
\begin{eqnarray*} 
\normo{\p} &>& g( \u) r'/2 \ge  \u\cdot (r' \p)/2 = \normo{\p}  \u\cdot \v /2\\
&=& \normo{\p} (\normo{ \u}^2 + \normo{\v}^2 - \norm{ \u}{\v}^2)/4\\
&>& \normo{\p}(4+r'^2-r^2)/4\\
&\ge& \normo{\p}.
\end{eqnarray*}
This contradiction shows that $P$ is bounded.
\end{proof}




Since $L(h)\le 1$ when $h\ge1$, it is clear that a finite packing $V$
that satisfies Inequality~\ref{eqn:CE} has cardinality greater than twelve.
The following lemma also gives an upper bound on the cardinality of $V$.

\begin{lemma}[]\guid{DLWCHEM}\rz{500}\label{lemma:13-14}  %%
Let $V\subset \BB$ be a packing that satisfies
inequality~\ref{eqn:CE}.  Then the cardinality of $V$ is thirteen, fourteen, or fifteen.
\end{lemma}


\begin{proof} (Following Marchal)
Consider a finite packing $ V=\{\u_1,\ldots, \u_N\}\subset \BB$ satisfying
inequality~\ref{eqn:CE}.  The packing $V$ contains more than twelve points, because otherwise
the inequality~\ref{eqn:CE} cannot hold, as $L(h)\le 1$.

By adding points
as necessary, the packing becomes weakly saturated in the sense of
Definition~\ref{def:weakly-saturated}, with $r=2$ and $r'=2\hm$.  It is enough to show
that this enlarged set has cardinality less than sixteen.    Let
\begin{displaymath}%
g(h) = \arccos(h/2) - \pi/6,  %
\end{displaymath}%
and let $h_i =
\normo{ \u_i}/2$.  Then $h_i\le h_0=1.26$.
Consider the spherical disks $D_i$ of radii $g(h_i)$,
centered at $ \u_i/\normo{ \u_i}$ on the unit sphere.  These disks do not overlap; this
follows from the easy
Calculation~\ref{calc:cc:disks} %% Marchal disks are disjoint.  % cc:disks
\begin{equation}\label{eqn:disks} 
g(h_i) + g(h_j) \le \op{arc}(2h_i,2h_j,2).%
\end{equation}%
\indy{Notation}{D@$D$ (spherical disks)}%
For each $i$, the plane through the circular boundary of $D_i$ bounds
a half-space containing the origin.  The intersection of these
half-spaces is a polyhedron $P$, which is bounded by
Lemma~\ref{lemma:poly-bounded}.  Lemma~\ref{lemma:polyhedron}
associates a fan $(V_P,E_P)$ with $P$.  (The set $V_P$ is dual to $
V$; the set $V_P$ is in bijection with extreme points of $P$, whereas $
V$ is in bijection with the facets of $P$.)  There are natural
bijections between the following sets:
\begin{itemize} 
\item $ V = \{ \u_1,\ldots, \u_N\}$.
\item The  facets of $P$.
\item The set of  topological components of $Y(V_P,E_P)$.
\item The set of faces in the hypermaps $\op{hyp}(V_P,E_P)$.
\end{itemize}
The bijection of the first two sets follows from the first conclusion
of Lemma~\ref{lemma:webster}.  Lemmas~\ref{lemma:WF} and
~\ref{lemma:face} give the other two bijections.

Let $k_i$ be the cardinality of the face in $\op{hyp}(V_P,E_P)$
corresponding to the facet $i$.  By Lemma~\ref{lemma:ngon}, the solid
angle of the topological component $W_i$ of $Y(V_P,E_P)$ is at least
$\op{reg}(g(h_i),k_i)$, where \indy{Index}{half-plane}%
\indy{Index}{half-space}%
\indy{Notation}{reg (area of regular spherical polygon)}%
\begin{displaymath} 
\op{reg}(a,k) = 2\pi - 2 k (\arcsin(\cos(a)\sin(\pi/k))).
\end{displaymath}
By a computer calculation\footnote{\calc{1965189142}. 
The integer parameter $k$ can be replaced with a real variable.   If $k\ge34$, then
the right-hand-side is negative and the inequality is immediate.} 
%~\ref{calc:cc:alin}, %% Linear lower bound on regular
%% polygon. % cc:alin
\begin{equation}\label{eqn:alin} 
\op{reg}(g(h),k) \ge c_0 + c_1 k + c_2 L(h),\quad
k = 3,4,\ldots,\quad 1\le h\le \hm,
\end{equation}
where
\begin{displaymath}
%c_0 = 0.6327,\quad c_1 = -0.0333,\quad c_2 =0.4754.
c_0=0.591,\quad c_1=-0.0331,\quad c_2 = 0.506.
\end{displaymath} The sum $\sum_i k_i$ is the number of darts
in $\op{hyp}(V_P,E_P)$ by  Lemma~\ref{lemma:polyhedron}.  By
Lemma~\ref{lemma:dart-upper}, $\sum_i k_i \le (6N-12)$.  Summing over
$i$, an estimate on $N$ follows: 
\indy{Index}{polyhedron!convex}%
\indy{Index}{hypermap!planar}%
\begin{eqnarray*} 
4\pi &=& \sum_i\op{sol}(W_i)\\
&\ge& \sum_i \op{reg}(g(h_i),k_i) \\
&\ge& c_0 N +c_1\sum_i k_i + c_2 \sum L(h_i)\\
&\ge& c_0 N +c_1 (6N-12) + c_2 12\\
\end{eqnarray*}
This gives
$16 > N$.
\end{proof} 


\begin{lemma}[]\guid{XULJEPR}\rz{400}\label{lemma:D'}  
Let $ V\subset\BB$ be a finite packing.
Assume that there exists $ \v\in V$ such that $\normo{ \v}=2$ and
$\norm{ \v}{ \u}\ge 2\hm$ whenever $\v\ne\u\in V$.
Then  inequality~(\ref{eqn:CE}) does not hold on $ V$.
\end{lemma}

\begin{proof} Assume for a contradiction that a packing exists that
satisfies the assumptions and the inequality.  Without loss of
generality, assume that $N\ge 13$, since the inequality is known to
hold when $N\le 12$.  Create one large disk $D_1'$ centered at $
\v/2$ and repeat the proof of the previous lemma.  Extend the packing to a the weak
saturation with parameters $r=r'=2\hm$.  This can be done in a way that maintains
the assumptions on $\v$.  By
Lemma~\ref{lemma:poly-bounded}, the polyhedron is
bounded.  By the assumptions of the lemma, we may take
\begin{displaymath}a'=0.797 < \arc(2,2\hm,2\hm)-g(\hm)\end{displaymath} 
for the arcradius of the large disk $D_1'$.  
By a computer calculation\footnote{\calc{6096597438}.
The integer parameter $k$ can be replaced with a real variable.  If $k\ge64$, then
the right-hand-side is negative and the inequality is immediate.} %~\ref{calc:cc:alin2}, 
%% Linear lower bound on regular polygon (large disk) % cc:alin2
\begin{equation}\label{eqn:alin2} 
\op{reg}(a',k) \ge c_0 + c_1 k + c_2 L(1) +
c_3,\quad k=3,4,\ldots\end{equation}
where
$c_3 =  1$.  % was %0.85$.
Then 
\begin{eqnarray*} \label{eqn:D'}
4\pi &=& \sum_i\op{sol}(W_i)\\
&\ge& \op{reg}(a',k_1)+\sum_{i>1} \op{reg}(g(h_i),k_i) \\
&\ge&  c_0 N +c_1\sum_i k + c_2 \sum L(h_i) + c_3\\
&\ge& c_0 N +c_1 (6N-12) + c_2 12 + c_3\\
\end{eqnarray*}
This gives a contradiction
$13 > N \ge 13.$
\end{proof}

    %\lll
    %%

\chapter{Cyclic Fan}\label{sec:cyclic}


\begin{summary}
At various times, it is useful to focus attention on a single face in a fan.  This leads to a construction of a what is called the localization of a fan along a face.  The localization is again a fan.  Special properties of the localized fan are captured in the notion of a cyclic fan.  The upper bound $2\pi$ on the perimeter of a cyclic fan is established.


To focus attention on a particular face $F$ of a fan $(V,E)$, it is useful to disregard all the nodes
and edges of the fan, except for the nodes and edges that constitute $F$.  This pared down fan
is an example of a {\it cyclic fan}, which is defined below.  It can be viewed as a ``localization'' of
the fan along a face $F$.  

A theory of deformations of cyclic fans is developed.  Sufficient conditions are stated for the deformation of a cyclic fan to remain a cyclic fan.
\end{summary}


\section{Localization}


\subsection{basics}


\begin{definition}[cyclic fan]  A triple $(V,E,F)$ is a {\it cyclic fan} if the following conditions hold.
\begin{nomerate} 
\item \case{fan} $(V,E)$ is a fan;
\item \case{face} $F$ is a face of $H = \op{hyp}(V,E)$;
\item \case{cyclic hypermap} $H$ is isomorphic to $H_{2k}$, where $k = \card(F)$;
\item \case{angle} $\op{azim}(x)\le \pi$ for all darts $x\in F$; and
\item \case{wedge} $V\subset \bWdart(x)$ for all $x\in F$.
%\item If $\{\v,\w\}\in E$, then $\{\orz,\v,\w\}$ is not collinear. %% part of def of fan.
\end{nomerate}
\end{definition}
\indy{Index}{fan!cyclic}%

The intersection of $X(V,E)$ with the unit sphere is a spherical polygon, when $(V,E,F)$ is a cyclic fan.  The spherical polygon gives a visual representation of the cyclic fan. The choice of $F$ distinguishes the ``interior'' of the polygon from its exterior.  The final two conditions are convexity constraints.  



\begin{lemma}\guid{WRGCVDR}\rating{ZZ}  
For any cyclic fan $(V,E,F)$, there is a bijection from $F$ onto $V$ given by
$$
 (\v,\w) \mapsto \v
$$
Moreover, write $\v\mapsto(\v,\rho\v)$ for the inverse map. 
Then $\rho:V\to V$ is a cyclic permutation.
%\begin{itemize}
%\item $(\v,\rho \v)\in D$, for all $\v\in V$; and
%\item 
%$$f(\v,\w) = (\w,\sigma(\w)^{-1}\v)= (\w,\rho \w)$$ for all $(\v,\w)\in F$.
%\end{itemize}
\end{lemma}
\indy{Index}{cyclic permutation}%

\begin{proof}  The map from the set of nodes to a face is a bijection for the cyclic hypermap $H_{2k}$, hence it is a bijection for cyclic fans.

For all $(\v,\rho \v)\in F$,
$$f(\v,\rho \v) = (\rho\v,\rho^2\v),$$
so that the order of $\rho$ on $V$ is the order of $f$ on $F$, which is $k=\card(F)$.  Thus, $\rho$ is a cyclic permutation of $V$ of order $k=\card(V)$.
\end{proof}

\begin{definition}[$\rho$,~$\v$]  For any cyclic fan $(V,E,F)$, 
write $\v:F\to V$ and $\rho:V\to V$ for the bijections of the preceding lemma.
\end{definition}
\indy{Notation}{ZZrho@$\rho$}%
\indy{Notation}{v@$\v$}%

\begin{definition}[interior angle,~$\angle$,~$\bWdart$]
For any cyclic fan $(V,E,F)$,
 write
$$
\angle(\v) = \op{azim}((\v,\rho\v)),
$$
for all $\v\in V$.  This is the {\it interior} angle of the cyclic fan at $\v$.
Also, write
$$
\Wdart(F,\v) = \Wdart((\v,\rho \v)),\quad \bWdart(F,\v) =\bWdart((\v,\rho \v)).
$$
\indy{Notation}{1@$\angle$}%
\indy{Notation}{Wdart@$\Wdart(F,\v)$}%
\indy{Notation}{Wdart@$\bWdart(F,\v)$}%
%\indy{Notation}{v@$\v(x)$}%
\end{definition}






\begin{definition}[localization]  Let $(V,E)$ be a fan, and let $F$ be a face of $\op{hyp}(V,E)$.  
Let 
$$
\begin{array}{rll}
V' &= \{\v\in V \mid \exists~\w\in V.~(\v,\w)\in F\}.\\
E' &= \{\{\v,\w\} \in E\mid (\v,\w)\in F\}.
\end{array}
$$
The pair $(V',E')$ is called the {\it localization} of $(V,E)$ along $F$.
\end{definition}
\indy{Index}{localization}%


\begin{lemma}[localization]\guid{LVDUCXU}\rating{ZZ}\label{lemma:localization}  
Let $(V,E)$ be any fan, and let $F$ be a face of its hypermap.  
Then the localization $(V',E')$ is a fan, and $F$ is a face of $\op{hyp}(V',E')$.  Moreover, the hypermap $\op{hyp}(V',E')$ is isomorphic
to $H_{2k}$, where $k= \card(F)$.
 Finally, the values $\op{azim}(x)$ and $\Wdart(x)$  do not depend
whether they computed relative to $\op{hyp}(V,E)$ or to $\op{hyp}(V',E')$, for all $x\in F$.
%That is, 
%$$
%\begin{array}{lll}
%\op{azim}(V,E,x) &=\op{azim}(V',E',x)\\
%\Wdart(V,E,x) &=\Wdart(V',E',x)\\
%\end{array}
%$$
\end{lemma}



\begin{proof}
The proof that  $(V',E')$ is a fan is a sequence of simple verifications based on
the techniques of Remark~\ref{remark:fan-verify}.  The details are left to the reader.

The dart set $D'$ of $\op{hyp}(V',E')$ is naturally identified with the disjoint union $F\cup F'$, where
$F = \{(\v,\w) \in F\}$ and $F'=\{(\w,\v) \mid (\v,\w)\in F\}$.  Under this identification, $F$ is a face of $\op{hyp}(V',E')$.  The edge map $e(\v,\w)= (\w,\v)$ is a bijection of $F$ onto $F'$.   Pick any $x\in F$ and
define a bijection from the disjoint union of two copies of the cyclic group $Z_k$ onto $D'$  by 
the pair of maps
$$
i \mapsto f^i x,\quad\text{ and } i\mapsto e f^i x.
$$
This bijection extends to an isomorphism of hypermaps $H_{2k}$ onto $\op{hyp}(V',E')$.

The proof that the $\op{azim}(x)$ and $\Wdart(x)$ do not depend on the choice of fan is a consequence of their definitions:
$$
\begin{array}{lll}
\op{azim}(x) &= \op{azim}(\orz,\v,\w,\sigma(\v,\w)),\\
\Wdart(x) &= \Wdart(\orz,\v,\w,\sigma(\v,\w)).\\
\end{array}
$$
where $x = (\v,\w)$.    It is enough to check that $\sigma(\v,\w)\in E'(\v)$. 
But $\{\sigma(\v,\w),\v\}\in F$, so this is indeed the case.
%In particular $\op{azim}(x)\le\pi$ follow by the assumptions of the lemma.
\end{proof}



\subsection{geometric types}\label{sec:types}

\begin{definition}[generic,~lunar,~circular]
A cyclic fan $(V,E)$ is {\it generic} if for every $\{\v,\w\}\in E$
and every $\u\in V$, 
$$
C\{\v,\w\}\cap C^0_-\{\u\} = \emptyset.
$$
A cyclic fan is {\it circular} if there exists $\u\in V$ and $\{\v,\w\}\in E$ such that 
$$
C^0\{\v,\w\}\cap C^0_-\{\u\}\ne \emptyset.
$$
A cyclic fan is {\it lunar} with dependency $\v,\w$ if it is not circular and if $\v$ and $\w$ are distinct and dependent elements of $V$.
\end{definition}
\indy{Index}{generic}%
\indy{Index}{lunar}%
\indy{Index}{circular}%


\begin{lemma}\guid{CIZMRRH}\rating{ZZ} Every cyclic fan is
generic, lunar, or circular.  Moreover, these three properties are mutually exclusive.
\end{lemma}
\indy{Index}{fan!cyclic}%
\indy{Index}{cyclic fan}%
\indy{Index}{generic}%
\indy{Index}{lunar}%
\indy{Index}{circular}%

\begin{proof} If $(V,E,F)$ is not generic,  select some $\{\v,\w\}\in E$
and some $\u\in V$ such that
\begin{equation}\label{eqn:non-generic}
C\{\v,\w\}\cap C^0_-\{\u\} \ne \emptyset.
\end{equation}
Now $C\{\v,\w\} = C^0\{\v,\w\} \cup C\{\v\}\cup C\{\w\}$.  
If, for some such triple $(\u,\v,\w)$, the intersection~(\ref{eqn:non-generic}) meets $C^0\{\v,\w\}$, then the cyclic fan is circular.  
Otherwise, the cyclic fan is lunar. 
\end{proof}

\begin{definition}[flat] Let $(V,E,F)$ be a cyclic fan.
If $\angle(\v)=\pi$, then $\v$ is {\it flat}.
\end{definition}
\indy{Index}{flat (node of a fan)}


\begin{lemma}\guid{LDURDPN}\rating{ZZ}  \label{lemma:coplanar}
Assume that $\u$ and $\w\in\ring{R}^3$ are independent, as well as $\u$ and $\v$.  Then $\op{azim}(\orz,\u,\v,\w)=\pi$
if and only if there exists a plane $A$ such that
$S=\{\orz,\u,\v,\w\}\subset A$ and such that the line $\op{aff}\{\orz,\u\}$ separates $\v$ from $\w$ in $A$.
\end{lemma}

\begin{proof}  The given azimuth angle is $\pi$ if and only if $\dih(\{\orz,\u\},\{\v,\w\})=\pi$.  This holds
exactly when $S$ is coplanar and the line $\op{aff}\{\orz,\u\}$ separates $\v$ from $\w$ in $A$.
\end{proof}

\begin{lemma}\guid{KOMWBWC}\rating{ZZ}\label{lemma:kom}
Let $(V,E,F)$ be a cyclic fan.  Let $k=\card(F)$.  Assume that for some $0<r\le k-1$ and some $\v\in V$, the set $U=\{\v,\rho \v,\ldots,\rho^r \v\}$ is contained in a plane $A$ passing through $\orz$.  Let $\e$ be the unit normal to $A$ in the direction $\v\times \rho \v$.  Then 
the set $U$ is cyclic with respect to $(\orz,\e)$, and the azimuth cycle $\sigma$ on $U$ is 
$$
\sigma \u = \begin{cases} \rho \u, & \u\ne \rho^r\v,\\ \v, & \u = \rho^r\v.\end{cases}
$$
\end{lemma}

\begin{proof} 
Write $\v_i = \rho^i \v$, for $i=0,\ldots,r$.

\claim{ 
$(\v_i \times \v_{i+1})\cdot \e > 0$, for all $i\le r-1$. }
Indeed, the base case of an induction  $(\v_0\times \v_1)\cdot \e > 0$ holds by assumption.    
Assume for a contradiction that the inequality holds for $i$, but not for $i+1$.
Then 
$$\op{aff}^0_+(\{\orz,\v_{i+1}\},\v_i) = \op{aff}^0_+(\{\orz,\v_{i+1}\},\v_{i+2}).$$ 
This forces $C^0\{\v_i,\v_{i+1}\}$ to meet $C^0\{\v_{i+1},\v_{i+2}\}$, which
is contrary to the definition of a fan.  Thus, the claim holds.

The fact that $U$ is cyclic follows trivially from the fact that $U$ is contained in a plane $A$ through $\orz$ and that $\e$ is orthogonal to $A$.

\claim{For all $0\le i \le r-1$,  $\sigma \v_i = \v_{i+1}$.}
Otherwise, there is some 
$$\u \in (U\setminus \{\v_i,\v_{i+1}\}) \cap W^0(\orz,\e,\v_i,\v_{i+1}) \cap A.
%  ~~\subset~~ 
$$
However, this intersection is a subset of $C^0\{\v_i,\v_{i+1}\}$,
and $\u\in C^0\{\v_i,\v_{i+1}\}$ is contrary to the property \case{intersection} of fans.  The result follows.
%The membership of $\u$ in the rightmost term is contrary to the definition of fan.  
%Let $\e_3$ be the unit vector in the direction $\v\times \u$.  Let $\e_1$ be the unit vector in the direction $\u$.
%Let $\e_2 = \e_3 \times \e_1$.  The coordinates of $\u,\v,\w$ with respect to the orthonormal frame $(\e_1,\e_2,\e_3)$ take the form
%$$
%\begin{array}{rllrl}
%\v &= * \e_1 + a \e_2,  &\quad & a &< 0\\
%\u &= a' \e_1, &\quad & a' &>0\\
%\w&= * \e_1 + a'' \e_2, &\quad & a'' &>0\\
%\end{array}
%$$
%From this representation it is clear that $\v\times \u$ points in the same direction as $\u\times \w$.
%The set $\{\v,\u,\w\}$ is clearly cyclic and the counterclockwise cycle $\sigma$ in the $\{\e_1,\e_2\}$ plane
%takes $\v$ to $\u$ and $\u$ to $\w$.
\end{proof}

\begin{lemma}\guid{OZQVSFF}\rating{ZZ} \label{lemma:A}  Let $(V,E,F)$ be a cyclic fan and let
$\u,\v,\w\in V$ such that
\begin{itemize}
\item $\{\orz,\u,\v,\w\}$ is contained in a plane $A$; \vspace{3pt}
\item $\u,\w\not\in\op{aff}\{\orz,\v\}$; and \vspace{3pt}
\item $\op{aff}^0_+(\{\orz,\v\},\u) \ne \op{aff}^0_+(\{\orz,\v\},\w)$.
\end{itemize}
Then $\v$ is flat.  Moreover, $\rho \v,\rho^{-1} \v\in A$.
\end{lemma}

\begin{proof} Let $x = (\v,\rho\v)\in F$.  
Order $\u$ and $\w$ so that
$$
\op{azim}(\orz,\v,\rho\v,\u) \le \op{azim}(\orz,\v,\rho\v,\w).
$$
By the definition of cyclic fan, by the conditions $\u,\v\in \bWdart(x)$, 
and by  Lemma~\ref{lemma:coplanar},
%By the assumptions, $\dih(\{\orz,\v\},\{\u,\w\})=\pi$.  Since
%$\u,\w\in \bWdart(x)$, it follows that
%$$\pi = \dih(\{\orz,\v\},\{\u,\w\}) \le \op{azim}(x) = \angle(\v) \le \pi.$$
%The first conclusion follows.
$$
\begin{array}{rll}
0 &\le \op{azim}(\orz,\v,\rho\v,\u) \\
   &= \op{azim}(\orz,\v,\rho\v,\w) - \op{azim}(\orz,\v\,\u,\w)\\
    &= \op{azim}(\orz,\v,\rho\v,\w)-\pi \\
   &\le \op{azim}(\orz,\v,\rho\v,\rho^{-1}\v,\w) - \pi \\
   &=\op{azim}(x) - \pi \\
   &=\angle(\v)-\pi\\
   &\le 0. 
\end{array}
$$
Hence each inequality is equality.  In particular, $\v$ is flat.
In particular, $0 =\op{azim}(\orz,\v,\rho\v,\u)$, so that 
$$\rho\v\in \op{aff}_+(\{\orz,\v\},\u) \subset A.$$
Similarly,
$$
\rho^{-1}\v\in \op{aff}_+(\{\orz,\v\},\w) \subset A.
$$
\end{proof}

If Lemma~\ref{lemma:A} can be applied once, then it can often be applied repeatedly along a chain.  For example, the conclusion of the lemma implies
that $\rho^{-1} \v \in A$.  In fact, by the definition of fan, 
$$\rho^{-1}\v \in A \setminus \op{aff}\{\orz,\v\} = \op{aff}^0_+(\{\orz,\v\},\u) \cup \op{aff}^0_+(\{\orz,\v\},\w).$$
Suppose that $\rho^{-1} \v$ lies in the first term of the union.  If $\u\ne \rho^{-1}\v$, then the assumptions of the lemma
are met for $\{\u,\rho^{-1} \v,\v\}$, giving the conclusions that $\rho^{-1} \v$ is flat, and $\rho^{-2}\v\in A$.  Repeating, we obtain a chain
$$
\pi=\angle(\v) = \angle(\rho^{-1} \v) = \cdots,
$$
with $\v,\rho^{-1}\v,\ldots\in A$.  Another chain can be constructed in the other direction $\v,\rho \v,\ldots$.
This process of chaining gives the following lemma.

\begin{lemma}[circular geometry]\guid{KCHMAMG}\rating{ZZ}\label{lemma:circular}
Let $(V,E,F)$ be a circular fan. Then
\begin{itemize}
\item $\v$ is flat for all $\v\in V$.
\item The set $V$ lies in a plane $A$ through $\orz$.
\item For some choice of unit vector $\e$ orthogonal to $A$, the set $V$ is cyclic with respect to
$(\orz,\e)$, and the azimuth cycle on $V$ coincides with $\rho:V\to V$.
Also,
$$
\op{azim}(\orz,\e,\v,\rho\v) = \op{arc}_V(\orz,\{\v,\rho \v\}).
$$
\end{itemize}
\end{lemma}

\begin{proof}  Let $\v, \u\in V$ be such that $C^0\{\u,\rho \u\}$ meets $C^0_-\{\v\}$.  Apply
Lemma~\ref{lemma:A} to $\{\u,\v,\rho \u\}$ to conclude that $\v$ is flat, and some plane $A$ contains
$\{\orz,\u,\rho\u,\v,\rho \v,\rho^{-1} \v\}$.  If 
$$
\{\orz,\u,\rho \u,\v,\rho \v,\rho^{-1} \v\} \subset V\cap A,
$$
and $\w\in V\cap A$, then there exists $\w_1,\w_2\in (V\cap A)\setminus\{\w\}$ for which
the assumptions of Lemma~\ref{lemma:A} hold.  Then $\w$ is flat, and
$\rho \w \in V\cap A$.  The set $V\cap A$ is therefore preserved by $\rho$.
By observing that $V$ is the only nonempty subset of $V$ that is preserved by $\rho$,
 it follows that $V\subset A$.  


By Lemma~\ref{lemma:coplanar}, $V$ is cyclic with respect to a unit vector $\e$ orthogonal to $A$.  The
azimuth cycle on $V$ is $\v \mapsto \rho \v$.
\end{proof}

\begin{lemma}[lunar geometry]\guid{HKIRPEP}\rating{ZZ}\label{lemma:lunar}
Let $(V,E,F)$ be a lunar fan with dependent vectors $\v,\w\in V$.  
Assume that $\rho^r \v = \w$, for
some $0< r < k$.
Then
\begin{itemize}
\item $\u$ is flat, for all $\u\in V\setminus \{\v,\w\}$; \vspace{3pt}
\item $0< \angle(\v) = \angle(\w)\le \pi$; \vspace{3pt}
\item $V\cap \op{aff}_+(\{\orz,\v\},\rho \v) = \{\v,\rho \v,\ldots, \w\}$; \vspace{3pt}
\item $V \cap \op{aff}_+(\{\orz,\v\},\rho^{-1} \v) = \{\w,\rho \w,\ldots \v\}$;  \vspace{3pt}
\end{itemize}
\end{lemma}

\begin{proof}   Set $V_1 = \{\v,\rho \v,\ldots,\w\}$ and $V_2 = \{\w,\rho \w,\ldots,\v\}$. 
Let $\u\in V\setminus\{\v,\w\}$ be arbitrary.
Apply Lemma~\ref{lemma:A} to the set $\{\u,\v,\w\}$ to find that $\u$ is flat, and
that $\{\orz,\u,\rho \u,\rho^{-1} \u\}$ belongs to a plane $A(\u)$.  Now $A(\u)$ and $A(\rho \u)$ are both the unique plane containing $\{\orz,\u,\rho \u\}$, hence $A(\u) = A(\rho \u)$ 
when $\rho \u\not\in \{\v,\w\}$.  By induction, there are planes $A_1, A_2$ such that $V_i\subset A_i$.  There is
an azimuth cycle $\sigma_i$ on $V_i$ such that $\sigma_i \u = \rho \u$, when $\u \in A_i\setminus \{\v,\w\}$.  

The angles $\angle(\v)$ and $\angle(\w)$ are both equal to the dihedral angle between these two half-planes.  In particular, $0<\angle(\v)=\angle(\w)\le\pi$.
\end{proof}




\begin{lemma}[monotonicity]\guid{EGHNAVX}\rating{ZZ}  \label{lemma:monotone}
Let $(V,E,F)$ be a cyclic fan. Fix $\v_0\in V$.  Assume that $\v_0$ and $\u$ are independent for all $\u\in V\setminus\{\v_0\}$.  For all $i$, set $\v_i = \rho^i \v_0$ and $\beta(i) = \op{azim}(\orz,\v_0,\v_1,\v_i)$.
Then
$$0=\beta(1)\le \beta(2)\le \cdots\le \beta(k-1)\le\pi.$$
Moreover, if $\beta(i)=0$ for some $1<i \le k-1$, then
$$
\angle(\v_1) = \cdots = \angle(\v_{i-1}) = \pi,
$$
and $\{\orz,\v_0,\ldots,\v_i\} \subset \op{aff}^0_+(\{\orz,\v_0\},\v_1)$.
Finally, if $\beta(i)=\beta(k-1)$ for some $1\le i<k-1$ then 
$$
\angle(\v_{i+1}) = \cdots = \angle(\v_{k-1}) = \pi,
$$
and $\{\orz,\v_i,\ldots,\v_{k-1}\} \subset \op{aff}^0_+(\{\orz,\v_0\},\v_{k-1})$.
\end{lemma}

\begin{proof}  
Pick an orthonormal frame and write the points $\v_j$ in spherical coordinates $(r_j,\theta_j,\phi_j)$.  In an appropriate orthonormal frame, $\phi_0=0$, and $\theta_j=\beta(j)$, for all $j$.  From $\v_j\in \bWdart(F,\v_0)$ and $\angle(\v_0)\le\pi$, it follows that $0\le\theta_j\le\theta_{k-1}\le\pi$ when $0\le j\le k-1$.

 One may assume by induction that $0\le \beta(1)\le\cdots\le \beta(i)$.  The condition
$$
\v_0\in \bWdart(F,\v_i)
$$
implies that
$$
0 \le \op{azim}(\orz,\v_i,\v_{i+1},\v_0)\le \op{azim}(\orz,\v_i,\v_{i+1},\v_{i-1})\le\pi.
$$
By Lemma~\ref{lemma:sim}, the resulting inequality
$$
\sin(\op{azim}(\orz,\v_i,\v_{i+1},\v_0))\ge 0
$$
reduces to a triple-product:
$$
(\v_0 \times \v_i)\cdot \v_{i+1}\ge 0.
$$
In spherical coordinates, this inequality becomes
$$
r_0r_ir_{i+1}\sin\phi_i\sin\phi_{i+1}\sin(\theta_{i+1}-\theta_i)\ge0.
$$
Under the independence assumption, $\sin\phi_i\ne0$ and $\sin\phi_{i+1}\ne0$ (when $0< i < k-1$).    These inequalities give $\theta_i\le\theta_{i+1}$ (with a small extra argument to exclude the degenerate case $\theta_{i+1}=0,\theta_i=\pi$).  The conclusion follows by induction.

Assume that $\beta(i)=\theta_i=0$, for some $i>1$.  
Then $\v_1,\ldots,\v_i$ all lie in the
half-plane $\op{aff}^0_+(\{\orz,\v_0\},\v_1)$.  In particular, they are coplanar.  A chaining argument based on Lemma~\ref{lemma:A} gives the result.

The final conclusion is similar.
\end{proof}



\subsection{deformation}\label{sec:deformation}

This section considers deformations of a cyclic fan $(V,E,F)$.

\begin{definition}[deformation]
A {\it deformation} of a cyclic fan $(V,E,F)$ over an interval $I\subset\ring{R}$
is a 
continuous function $\varphi:V\times I \to\ring{R}^3$ (in the discrete topology on $V$ and the product topology on $V\times I$).
\end{definition}
\indy{Index}{deformation}%
\indy{Index}{fan!cyclic}%

Write $\v(t)$ as an abbreviation of $\varphi(\v,t)$, for $t\in I$.  
Also, set
$$
\begin{array}{lll}
V(t)&=\{\v(t) \mid \v\in V\},\\
E(t)&=\{\{\v(t),\w(t)\}\mid \{\v,\w\}\in E\},\\
F(t)&= \{(\v(t),\w(t)) \mid  (\v,\w)\in F\}.
\end{array}
$$


A deformation does not require $(V(t),E(t),F(t))$ to be a cyclic fan for all $t\in I$, although this will often be the case. The permutation $\rho:V\to V$ gives $\varphi(\rho \v,t)\in V(t)$, for every $\v\in V$.  


\begin{example}[lunar deformation]\label{example:lunar}
Consider a lunar cyclic fan $(V,E,F)$ with dependent $\v,\w\in V$.   Pick an orthonormal frame and spherical coordinates such that $\phi(\v)=0$, $\phi(\w)=\pi$.  $\theta(\rho \v)=0$, and $\theta(\rho^{-1} \v)=\theta_{k-1}\le\pi$.  Consider the deformation $\varphi$ over 
$I = \{t \mid 0 \le t < 1\}$
such that radial $r$ and zenith $\phi$ coordinates of $\varphi(\u,t)$ are constant as functions of $t$,
and the azimuth angle $\theta$ of $\varphi(\u,t)$ equals

$$
\begin{cases} 
   (1-t) \theta_{k-1} & \text{if } \u\in \{\rho \w,\rho^2 \w,\ldots, \rho^{-1} \v\};\\
   0 & \text{if } \u\in \{\rho \v,\rho^2 \v,\ldots,\rho^{-1} \w\}.\\
\end{cases}
$$
%This deformation $(V(t),E(t),F(t))$ is a cyclic fan for all $t\in I$.
%The cardinality of $V$ is independent of $t\in I$.
Note that $(V(0),E(0),F(0)) = (V,E,F)$.
\end{example}
\indy{Index}{lunar}%
\indy{Index}{spherical coordinates}%

\begin{lemma}\guid{HZIYFIZ}\rating{ZZ}\label{lemma:lunar-deform} 
Let $(V,E,F)$ be a lunar cylic fan with dependency $\v,\w\in V$.  In the deformation described above, the triple $(V(t),E(t),F(t))$ is a lunar cyclic fan for all $t\in I$.  The cardinality of $V$ is independent of $t\in I$.
\end{lemma}

\begin{proof}  The proof consists of checking  the properties of a lunar cyclic fan, one by one.   The verification of the property \case{V} follows from the methods of Remark~\ref{remark:fan-verify}.

The map $\varphi(\cdot,t)$ is invertible, so that $V(t)$ is in bijection with $V$.  In particular, the cardinality does not depend on $t$.

\case{origin} The radial spherical coordinate is nonzero for every element of $V(t)$.  Hence $V(t)$ does not contain $\orz$.

\case{independence}  For $\{\v,\rho \v\}\in E$, the angle 
$\alpha=\op{arc}_V(\orz,\{\varphi(\v,t),\varphi(\rho \v,t)\})$
is independent of $t$.  The independence property is equivalent to $\alpha\ne0,\pi$.  Thus, the independence property for $E$ implies
the independence for $E(t)$.

\case{intersection} The points $V(t)$ are contained in the union of two half-planes $A_1,A_t$.  The deformation is the identity on $A_1$ and an isometry $A_0\to A_t$ on the second half-plane.  These bijections preserve the incidence relations of blades of the cones $C(\ee)$.  

\case{cyclic hypermap}~\case{face} The combinatorial properties of the hypermap do not depend on $t$.  In particular, the hypermap has face $F(t)$ and the hypermap is isomorphic to $H_{2k}$.

\case{angle} The azimuth angle $\angle(\varphi(\u,t))$ is fixed when $\u\ne \v,\w$ and is decreasing in $t$ when $\u\in \{\v,\w\}$.  Hence, the upper bound on the angle is preserved.

\case{wedge}  $V(t)\subset A_0\cup A_t$, where $A_0$ and $A_t$ are the
half-planes described above.  Now $\bWdart(F,\varphi(\u,t))$ is a half-space containing $A_0$ and $A_t$ when $\u\ne \v,\w$.  Also, $\bWdart(F,\varphi(\u,t))$ is a wedge with bounding half-planes $A_0$ and $A_t$ when $\u\in \{\v,\w\}$.  Hence
$V(t)\subset A_0\cup A_t\subset \bWdart(F,\varphi(\u,t))$ for all $\u\in V$.

\case{lunar} The points $\v,\w\in V(t)$ remain fixed and hence remain dependent.  Thus, the cyclic fan remains lunar.  
\end{proof}
\indy{Index}{angle!azimuth}%

Next  consider a deformation of a cyclic fan.  The following lemma gives a list of conditions that ensure that the deformed fan remains  cyclic throughout the deformation.


\begin{lemma}\guid{XRECQNS}\rating{ZZ}\label{lemma:fan-open}
Let $\varphi$ be a deformation of a  cyclic fan $(V,E,F)$ over an interval $I$.
Assume that $0\in I$ and that $\varphi(\v,0)=\v$ for all $\v\in V$.
For all $\v\in V$ assume that if $\{\orz,\rho^{-1}\v,\v,\rho\v\}$ is coplanar, then 
$$\{\orz,\varphi(\rho^{-1}\v,t),\varphi(\v,t),\varphi(\rho \v,t)\}$$
is coplanar for all sufficiently small $t$. 
If the property \case{wedge} is maintained for sufficiently small $t$, then
 exists $\epsilon>0$ such that $(V(t),E(t),F(t))$
is a  cyclic fan for all $t\in I\cap \leftopen-\epsilon,\epsilon\rightopen$.
Moreover, if $(V,E,F)$ is generic, then the property \case{wedge} is in fact maintained for sufficiently small $t$.  Finally, if $(V,E,F)$ is generic, then the deformed fan is also generic for sufficiently small $t$.
\end{lemma}

\begin{proof} Each of the properties of a generic cyclic fan will be examined in turn.

\case{V}:  The set $V(t)$ is the image of $V$ and is therefore finite and nonempty.  \case{origin} Since $\varphi$ is continuous and $\orz\not\in V$, it follows that $\orz\not\in V(t)$ for sufficiently small $t$.

\case{independence}: If $\v,\w$ are independent, then $\v(t)$ and $\w(t)$ are indepedent for sufficiently small $t$.

\case{intersection}: If $\ee \cap \ee'=\emptyset$, then $C(\ee)\cap S^2$ has a positive distance from $C(\ee')\cap S^2$.  Hence for sufficiently small times, the deformation of these sets remain disjoint.
If $\ee=\{\u,\v\}$ and $\ee'=\{\v,\w\}$ where $\u\ne\w$, then again the deformations of $C(\ee)\cap C(\ee')$ is $C(\{\v(t)\})$ for sufficiently small $t$.  The other cases are similar.

\case{face},~\case{cyclic hypermap}:  The azimuth cycle on $E(\v(t))$ is preserved; hence the combinatorial properties of the hypermap do not change when $t$ is sufficiently small.

\case{angle}: If $\op{azim}(x)<\pi$, then the inequality remains strict for sufficiently small $t$.  If $\op{azim}(x)=\pi$, then the coplanarity assumption of the lemma forces the equality to be preserved for all sufficiently small $t$.

\case{wedge}: The property $\u\in \Wdart(x)$ is an open condition.  It holds for sufficiently small $t$. Consider the case $\u\in \bWdart(x)\setminus \Wdart(x)$, where $x=  (\v_0,\rho\v_0)$.   Write $\v_i = \rho^i\v_0$ and pick $r\le k-1$ such that $\u = \v_r$.  The wedge property holds trivially when $r\in\{0,1,k-1\}$. Assume that $r\not\in\{0,1,k-1\}$. 
By the genericity assumption $\v_r$ and $\v_0$ are independent.  In the notation of Lemma~\ref{lemma:monotone},
$\beta(r) = 0$ or $\beta(r) = \beta(k-1)$.  This proof treats the case $\beta(r)=0$. (The case $\beta(r)=\beta(k-1)$ is similar.)   By Lemma~\ref{lemma:monotone}, 
$$0=\beta(1)=\beta(2)=\cdots=\beta(r),$$
and 
$$
\{\orz,\v_0,\v_1,\ldots,\v_r\} \subset \op{aff}^0_+(\{\orz,\v_0\},\v_1).
$$
In particular, the set is coplanar.   By the coplanarity assumption of the lemma,  $\{\orz,\v_0(t),\v_1(t),\ldots,\v_r(t)\}$ is coplanar for
sufficiently small $t$.  In fact, the condition $\v_r(t)\not\in \op{aff}_+(\{\orz,\v_0(t)\})$ is an open condition, so that $\v_r(t)\in \op{aff}^0_+(\{\orz,\v_0(t)\},\v_1(t))$ for sufficiently small $t$.  This half-plane is the bounding half-plane of $\bWdart(F,\v_0(t))$.  Hence $\u = \v_r\in \bWdart(F,\v_0(t))$ for sufficiently small $t$.

\case{generic}: Genericity is stated as open conditions $\v\not\in C\{\u,\w\}$.  These conditions continue to hold for sufficiently small $t$.
\end{proof}


%%%%%%%%%%%




\section{Perimeter}

%\subsection{perimeter}

\begin{definition}[perimeter]\label{lemma:perim}
Let $(V,E,F)$ be a cyclic fan.    Set
$$
\op{per}(V,E,F) = \sum_{i=0}^{k-1} \arc_V(\orz,\{\rho^i \v,\rho^{i+1} \v\}), 
$$
where  $k=\card(F)$.
This is easily seen to be independent of the choice of $\v\in V$.  Call $\op{per}$ the perimeter of the cyclic fan.
If $\v,\w\in V$ are distinct nodes, define the partial perimeter 
$$
\op{per}(V,E,F,\v,\w) = \sum_{i=0}^{r-1} \arc_V(\orz,\{\rho^i \v,\rho^{i+1} \v\}), 
$$
where $r$ is chosen so that $\w=\rho^r \v$ and $0<r\le k-1$.
\end{definition}
\indy{Index}{perimeter!cyclic fan}%
\indy{Index}{fan!cyclic}%
\indy{Notation}{per@$\op{per}$ (perimeter)}%



\begin{lemma}\guid{WSEWPCH}\tlabel{lemma:convex-hyp}\rating{400}
 The perimeter of every cyclic fan is at most $2\pi$.
\end{lemma}
\indy{Index}{fan!cyclic}%
\indy{Index}{perimeter}%

\begin{proof} 
\claim{If the cyclic fan is circular, then its  perimeter is $\op{per}(V,E,F) =2\pi$.}
Indeed, by Lemma~\ref{lemma:circular}, 
the arcs making up the perimeter all lie in a common plane.   
The azimuth cycle on $V$ coincides with $\rho:V\to V$.
The sum of the terms in the formula defining the perimeter is the sum of the azimuth angles in the azimuth cycle.  The sum is $2\pi$ by Lemma~\ref{lemma:2pi-sum}.


\claim{if the cyclic fan is lunar, then its  perimeter is $\op{per}(V,E,F) =2\pi$.}
Indeed, by Lemma~\ref{lemma:lunar}, the set $V$ is contained in the union of two half-planes.
The perimeter is the sum of arcs in a half-circle in the first half-plane plus
the sum of arcs in a half-circle in the second half-plane. This sum is $2\pi$.

Finally, assume that the cyclic fan is generic.
Suppose for a contradiction that the lemma is false.  Consider all counterexamples
that minimize the cardinality of $V$.  Among all such counterexamples, pick a counterexample with the smallest number of darts $x\in D$ such that $\op{azim}(x) = \pi$.

A cyclic fan $(V,E,F)$ is determined by $V$ and the cyclic permutation $\rho:V\to V$:  $E=\{\{\v,\rho \v\}\mid \v\in V\}$ and $F = \{(\v,\rho \v)\mid \v\in V\}$.

In this particular counterexample, if there is any dart $x=(\v,\w)\in F$ with $\op{azim}(x)=\pi$, then there is a new cyclic fan $(V',E',F')$ with $V' = V\setminus\{\v\}$ and $\rho':V'\to V'$ given by 
$$
\rho'(\u) = \begin{cases}
\rho(\u), & \text{if } \rho(\u)\ne \v;\\
\rho(\v), & \text{if }\rho(\u) = \v.\\
\end{cases}
$$
This is a cyclic fan with the same perimeter, contrary to the presumed minimality of the counterexample.  Thus $\op{azim}(x) <\pi$, for all $x\in F$.

If $\card(V) <3$, then the cyclic fan is circular or lunar, which has aleady been treated.  If $\card(V)=3$, then $V=\{\v_1,\v_2,\v_3\}$.  By the triangle inequality $\arc_V(\orz,\{\v_2,\v_3\}) \le \arc_V(\orz,\{\v_2,-\v_1\})+\arc_V(\orz,\{-\v_1,\v_3\})$.  Thus,
$$
\begin{array}{rll}
\op{per} &=\arc_V(\orz,\{\v_1,\v_2\}) + \arc_V(\orz,\{\v_2,\v_3\}) + \arc_V(\orz,\{\v_1,\v_3\})\\
&\le(\arc_V(\orz,\{\v_1,\v_2\})+\arc(\orz,\{\v_2,-\v_1\}))
   \\&\qquad\qquad+(\arc_V(\orz,\{\v_1,\v_3\})+\arc_V(\orz,\{-\v_1,\v_3\})) \\
   &= \pi+\pi.
\end{array}
$$

Now assume that $\card(V)\ge 4$.  Select $\v\in V$.  Consider a deformation of the cyclic fan $\varphi:V\times I \to \ring{R}^3$ that fixes $V\setminus\{\v\}$, and gives motion to $\v$:
$$
\varphi(\v,t) = \cos(t) \v - \sin(t) \rho \v.
$$
This is increasing in the perimeter.    For sufficiently small $t$,  it remains a generic cyclic fan (Lemma~\ref{lemma:fan-open}).  For sufficiently small $t$, the minimality gives $\angle(\varphi(\u,t))<\pi$.
Eventually, for some smallest $t=t_0$ the deformed value is no longer a generic cyclic fan.
%  or for some $\u\in V$,  $\angle(\varphi(\u,t_0))=\pi$.  In the latter  case,  the minimality condition fails, and the result follows.  




\claim{The \case{independence} property holds at $t=t_0$}.  Indeed, it is enough to check independence when one of the points is $\varphi(\v,t_0)$ and the other is $\rho\v$ or $\rho^{-1}\v$.  The set $\{\orz,\rho\v,\v,\rho^i\v\}$ is not coplanar for $i=-1,-2$, because otherwise some interior angle is $\pi$, by Lemma~\ref{lemma:A}.   
Thus, the plane through $\{\orz,\v,\rho\v\}$ meets the plane through $\{\orz,\rho^{-2}\v,\rho^{-1}\v\}$ along a line $L$ through the origin.  
If the set $\{\orz,\rho^{-1}\v,\varphi(\v,t)\}$ is collinear, then that line is contained in $$\op{aff}\{\orz,\v,\rho\v\}\cap\op{aff}\{\orz,\rho^{-1}\v,\rho^{-2}\v\}=L,$$ which is impossible since $\rho^{-1}\v\not\in \op{aff}\{\orz,\v,\rho\v\}$.  Thus, $\rho^{-1}\v$ and $\varphi(\v,t)$ are independent.

\claim{The line $L$ meets the segment $\op{conv}^0\{\v,-\rho\v\}$.}  Otherwise,
it meets $\op{conv}\{\v,\rho\v\}$.  Then $\v$ and $\rho\v$ lie in distinct half-spaces
bounded by $\op{aff}\{\orz,\rho^{-1}\v,\rho^{-2}\v\}$.  Since $\bWdart(F,\rho^{-1}\v)$ is contained in one of the two half-spaces, the condition $\v,\rho\v\in \bWdart(F,\rho^{-1}\v)$ fails, and $(V,E,F)$ is not a cyclic fan.  This contradiction establishes the claim.

There is a time $t_1$ at which $\{\orz,\rho\v,\varphi(\v,t)\}$ is collinear, and a time $t_2$ at which $\varphi(\v,t)\in L$.  By the previous claim, $t_2<t_1$.
At time $t_2$,  some interior angle is $\pi$.  If $t_1\le t_0$, then $t_2 < t_0$, and this contradicts the observation that all interior angles of the fan are less than $\pi$.  So $t_1> t_0$. Thus, $\varphi(\v,t)$ and $\rho\v$ are independent. This gives independence.

\claim{The \case{intersection} property holds.}  If $\ee\subset V$, write $\ee(t) = \{\varphi(\u,t) \mid \u\in \ee\}$.   Let $W =W(\orz,\v,\rho\v,\rho^{-1}\v)$. The verification of the intersection property is based on the following facts (when $t>0$):
\begin{itemize} 
\item If $\v\not\in \ee$, then $C(\ee(t))=C(\ee)\subset W$.
\item $C^0\{\v(t)\} \cap W = \emptyset$.
\item $C^0\{\v(t),\rho^{-1}\v\}\cap W = \emptyset$.
\item $C\{\v(t),\rho\v\}\cap W = C\{\v,\rho\v\}.$
\item $C\{\v(t),\rho\v\}\cap C\{\v(t),\rho^{-1}\v\} = C\{\v(t)\}$.
\end{itemize}

\claim{At time $t=t_0$, the deformed value is a cyclic fan.}
Otherwise, if the object fails to be a fan at time $t=t_0$, then the \case{independence} property,  \case{intersection} property, or the \case{cyclic} property fails.  
The first two properties have already been checked, so that the deformed value must be a fan.  Furthermore, the conditions for a fan to be cyclic are closed conditions, so they must also hold. 

At $t=t_0$, the deformed fan is cyclic but not generic.   The perimeter bound follows  from previous cases.
\end{proof}

Here is a second proof of the same lemma.  It is conceptually much simpler, but more difficult to formalize.  It is based on polar polygons (a generalization of polar triangles to spherical polygons).

\begin{proof} A fan does not have any faces of cardinality less than three.
Every blade of the fan has radian measure less than $\pi$.  
\indy{Index}{polygon!polar}%

Consider the case of a spherical triangle.  If the edges of the
the triangle are $a_i$ and the angles of the polar
triangle are $\beta_i$, then $\beta_i=\pi-a_i$.
The the perimeter is 
$$a_1+a_2+a_3 = 2\pi - (\beta_1 -\beta_2 - \beta_3-\pi)= 2\pi-\op{sol} < 2\pi,$$
because the solid angle $\op{sol}$ of the polar triangle is always strictly positive.
\indy{Index}{triangle!spherical}%

Similarly, if the edges of the spherical polygon are
$a_i$, then the angles of the polar polygon are $\beta_i = \pi-a_i$.
The perimeter is
$$
a_1+\cdots+a_n  = 2\pi- \op{sol}< 2\pi,
$$
where $\op{sol} = 2\pi-\sum a_i$ is the solid angle of the polar polygon.
%~\cite[\p.261]{williamson:2008}.
\indy{Notation}{sol $\op{sol}$ (solid angle)}%
\end{proof}


\section{Special Fan}\label{sec:weight}  






\subsection{definition}

\begin{definition}[special~fan,~$\hm$]
Let $\hm = 1.26$.
A {\it special fan} is a tuple $(V,E,F,G)$, where
\begin{nomerate}
\item \case{packing} $V$ is a packing.  That is, for every $\v,\w\in V$, if $\norm{\v}{\w}<2$, then $\v=\w$.
\item \case{annulus} $V\subset \BB$.
\item \case{cyclic fan} $(V,E,F)$ is a cyclic fan.
\item \case{subset} $G\subset E$.
\item \case{g norm} If $\{\v,\w\}\in G$, then $\norm{\v}{\w}=2\hm$.
\item \case{e norm} If $\{\v,\w\}\in E$, then $\norm{\v}{\w}\le 2\hm$.
\item \case{diagonal} For all distinct elements $\v,\w\in V$, if
$\{\v,\w\}\not\in E$, then $$\norm{\v}{\w}\ge 2\hm.$$
\item \case{card} %$k=\card(F)$,
 Let      $s=\card(G)$ and $r=\card(E) - s = \card(F)-s$.  Then
$$0\le s \le 3,\quad\text{and}\quad3-s \le r \le 6 - 2s.$$
\end{nomerate}
The constants $r,s$ are called the {\it parameters} of the special fan.
\end{definition}
\indy{Index}{special fan}
\indy{Index}{parameters (of a special fan)}
\indy{Notation}{r (special fan parameter)}
\indy{Notation}{s (special fan parameter)}


\begin{definition}[d]
$$d(r,s) = \begin{cases}
    0.103 (2-s) + 0.2759 (r+2s-4), & r + 2s > 3\\
    0, & r + 2s \le 3.\\
    \end{cases}$$
\end{definition}
%d(3,0)=0, d(4,0)= 0.206; d(5,0)= 0.4819,   .7578
\indy{Notation}{d (lower bound for $\tau$)}

\begin{definition}[$\hm$,~$\tau$,~$\dih_i$]\label{def:tau}
Let $(V,E,F)$ be a cyclic fan.  Set $\hm = 1.26$.  Set
$$
\tau(V,E,F) =\sum_{x\in F} \op{azim}(x)\left(1 + \dfrac{\sol_0}{\pi}  \dfrac{\normo{\v(x)}-2}{2\hm-2}\right) + \left(\pi+{\sol_0}\right) (2- k(F)),
$$
where  $\sol_0=3\arccos(1/3)-\pi\approx0.551$ is the solid angle of a spherical equilateral triangle of side $\pi/3$, and $k(F)$ is the cardinality of $F$.
The function $\tau$ is defined on special fans by disregarding $G$:
$$
\tau(V,E,F,G) = \tau(V,E,F).
$$
Let 
$$
\tau(y_1,y_2,y_3,y_4,y_5,y_6) =
\sum_{i=1}^3 \dih_i(y_1,\ldots,y_6)\left(1 + \dfrac{\sol_0}{\pi}  \dfrac{y_i -2}{2\hm-2}\right) - \left(\pi+{\sol_0}\right),
$$
where
$$
\begin{array}{lll}
\dih_1(y_1,y_2,y_3,y_4,y_5,y_6) &= \dih(y_1,y_2,y_3,y_4,y_5,y_6),\\
\dih_2(y_1,y_2,y_3,y_4,y_5,y_6) &= \dih(y_2,y_3,y_1,y_5,y_6,y_4),\\
\dih_3(y_1,y_2,y_3,y_4,y_5,y_6) &= \dih(y_3,y_1,y_2,y_6,y_4,y_5).\\
\end{array}
$$
\indy{Notation}{h0@$\hm$}
\indy{Notation}{zzt@$\tau$}
\indy{Notation}{sol@$\sol_0$}
\indy{Notation}{dih@$\dih_i$}
\end{definition}



\subsection{compactness}

Let 
$$
\begin{array}{lll}
V(\p) &= \{\p_{i}\mid i=0,\ldots,k-1\},\\
E(\p) &= \{\{\p_{i},\p_{i+1}\}\mid i=0,\ldots,k-1\},\\
F(\p) &= \{(\p_{i},\p_{i+1}) \mid i=0,\ldots,k-1\},\\
\end{array}
$$
where
 $\p:\{0,\ldots,k-1\}\to \ring{R}^3$ is any function, and
$\p_k = \p_0$.  (That is, for purposes of indexing, identify $\{0,\ldots,k-1\}$ with
the cyclic group $Z_k$.)
If $I\subset\{0,\ldots,k-1\}$ then set 
$$G(\p,I) = \{(\p_i ,\p_{i+1}) \mid i\in I\}$$

\begin{definition}[fan~datum]  Let $k\in\ring{N}$ and $I\subset \{0,\ldots,k-1\}$.
 A fan datum of shape $(k,I)$ is a function 
$\p:\{0,\ldots,k-1\}\to \BB$ such that
\begin{itemize}
\item \case{packing} For every $i,j$, if $\norm{\p_i}{\p_j}<2$, then $i=j$.
\item \case{i norm} If $i\in I$, then $\norm{\p_i}{\p_{i+1}}=2\hm$.
\item \case{e norm} For all $i$, $\norm{\p_i}{\p_{i+1}} \le 2\hm$.
\item \case{diagonal}  $\norm{\p_i}{\p_j} \ge 2\hm$ when $|i-j|>1$.
\item \case{card}  Let $s=\card(I)$ and $r=k-s$.  Then 
$$0\le s \le 3,\quad\text{and}\quad3-s \le r \le 6 - 2s.$$
\item \case{angle} $\op{azim}(\orz,\p_i,\p_{i+1},\p_{i-1})\le \pi$ for all $i$.
\item \case{wedge} $\p_j\in W(\orz,\p_i,\p_{i+1},\p_{i-1})$ for all $i,j$.
\end{itemize}
\end{definition}

\begin{lemma}\guid{CKQOWSA}\rating{ZZ}\label{lemma:ctc-fan}
Let $V\subset \BB$ be a packing.  Set 
$$E_{std} = \{\{\v,\w\}\subset V\mid 0 < \norm{\v}{\w} \le 2\hm\}.$$
Then $(V,E_{std})$ is a fan.
\end{lemma}
\indy{Notation}{E@$E_{std}$}%

\begin{proof}
The properties \case{V}, \case{origin}, and \case{independence} follow
by the methods of Remark~\ref{remark:fan-verify}.

\case{intersection}:  Some geometrical reasoning is required to establish the intersection property.   The case
$$
C\{\u\}\cap C\{\v\} = \{\orz\}
$$
follows from the strict triangle inequality 
$$
\normo{\u} \le 2\hm < 4 \le \normo{\v} + \norm{\u}{\v}.
$$
The other cases of the proof are based on the following 
two facts from Tarski arithmetic.
\begin{itemize}
\item Let $\{\v_0,\v_1,\v_2\}\subset \BB$ be a packing of three points.
Assume that $\norm{\v_1}{\v_2}\le 2\hm$.  Then
$C\{\v_0\}\cap C\{\v_1,\v_2\} = \{\orz\}$.
\item Let $\{\v_0,\v_1,\v_2,\v_3\}\subset \BB$ be a packing of four points.
Assume that $\norm{\v_1}{\v_3}\le 2\hm$ and $\norm{\v_2}{\v_4}\le 2\hm$.
Then $C\{\v_1,\v_3\}\cap C\{\v_2,\v_4\} = \{\orz\}$.
\end{itemize}
\end{proof}

\begin{lemma}\guid{VYNCGCO}\rating{ZZ}
If $\p$ is a fan datum of shape $(k,I)$, then
$$
(V(\p),E(\p),F(\p),G(\p,I))
$$
is a special fan.  Moreover, every special fan is equal to
$$
(V(\p),E(\p),F(\p),G(\p,I))
$$
for some fan datum $\p$ of some shape $(k,I)$.
\end{lemma}

\begin{proof}
\claim{Every special fan $(V,E,F,G)$ is equal to $(V(\p),E(\p),F(\p),G(\p,I))$
for some fan datum $\p$ of some shape $(k,I)$.}
Let $k=\card(F)$.  Fix any element $\u\in V$ and set $\p_i = \rho^i \u$.
Let $I = \{i\mid (\p_i,\p_{i+1}) \in G\}$.  Clearly $(V,E,F,G)=(V(\p),E(\p),F(\p),G(\p,I))$, and
$\p$ is a fan datum.

Let $\p$ be any fan datum of shape $(k,I)$.
$(V(\p),E(\p))$
is a fan by Lemmas~\ref{lemma:ctc-fan} and \ref{lemma:subset-fan}.


\claim{$(V(\p),E(\p),F(\p))$ is a cyclic fan.} The properties \case{wedge} and \case{angle} have been built into the definition of a fan datum.  The azmuth cycle $\sigma(\p_i)$ on $E(\p_i) = \{\p_{i+1},\p_{i-1}\}$ interchanges $\p_{i+1}$ with $\p_{i-1}$.  From this fact, the combinatorial properties \case{face} and \case{cyclic hypermap} are easily determined.  The bijection from the disjoint union of two copies of $Z_k$ onto the set of darts is given by the pair of maps
$$
i\mapsto (\p_i,\p_{i+1}),\quad i\mapsto (\p_{i+1},\p_i).
$$
This bijection extends to an isomorphism of hypermap $H_{2k}$ onto
$\op{hyp}(V(\p),E(\p))$.

\claim{$(V(\p),E(\p),F(\p),G(\p,I))$ is a special fan.} The properties of a special fan all follow trivially from the corresponding properties of a fan datum.
\end{proof}

The set of fan data of shape $(k,I)$ is a subspace of the topological space  $\BB^k$.


\begin{lemma}\guid{KFIIPLO}\rating{ZZ}
The space of fan data of shape $(k,I)$ is a compact metric space.  Moreover,
$$
\p \mapsto \tau(V(\p),E(\p),F(\p))
$$
is a continuous function on the space of fan data of shape $(k,I)$.
\end{lemma}

\begin{proof}  $\BB^k$ is a compact metric space and the constraints are all closed conditions.

The function $\tau$ is a polynomial in $\normo{\p_i}$ and $\op{azim}(\orz,\p_i,\p_{i+1},\p_{i-1})$.  The norm and azimuth angle are both continuous functions of $\p$.  
\end{proof}




\subsection{internal blades}


\begin{lemma}\guid{PGSQVBL}\rating{ZZ}  Let $(V,E,F)$ be a cyclic fan.   If $\v,\w\in V$ are independent, then $C\{\v,\w\} \subset \bWdart(x)$ for any dart $x\in F$.
\end{lemma}
\indy{Index}{fan!cyclic}%

\begin{proof}  This is an elementary consequence of the definitions, the cone shape of $\bWdart(x)$,  and the condition $V\subset \bWdart(x)$.
\end{proof}


\begin{lemma}[internal blades]\guid{YOLCBTG}\rating{ZZ}  \label{lemma:internal}
Let $(V,E,F)$ be a cyclic fan. 
Let $\v,\w\in V$ be independent.  Suppose that there exists $\v',\w'$ such
that $\angle(\v'),\angle(\w')<\pi$, where $\v,\v',\w,\w'$ are four distinct elements of
$V$ that appear in cyclic order.
Then $C^0\{\v,\w\}\subset \Wdart(x)$ for all $x\in F$.
%Pick a dart $x=(\v_0,\v_1)\in V$.  Set $\v_j = \rho^j \v_0$.  Assume that there are four darts $(y_1,y_2,y_3,y_4)$, $y_j = x_{j(j)}$, with
%$0\le j(1) < j(2) < j(3) < i(4)\le k-1$ 
%such that $\op{azim}(y_j) < \pi$, for $j=2,4$.  
%Then $C^0\{\v_{i(1)},\v_{i(3)}\} \subset \Wdart(x)$, for all $x\in F$.
\end{lemma}
\indy{Index}{blade!internal}%
\indy{Index}{internal blade}%
\indy{Index}{cyclic order}%

(To say that a sequence $\v_i$ of elements is in {\it cyclic order} means that
$\v_i = \rho^{j (i)}\v_0$, for some increasing function $j$ with range $\{0,\ldots,k-1\}$.)

\begin{proof}  Abbreviate $C^0 = C^0\{\v,\w\}$.  The first case to consider is $\v(x)=\v$.  For all $\p\in C^0\cap \bWdart(x)$, 
$$
0 \le \op{azim}(\orz,\v,\rho \v,\p) \le \op{azim}(\orz,\v,\rho \v,\rho^{-1} \v).
$$  
These inequalities are in fact strict.  If, for example $0 = \op{azim}(\orz,\v,\rho \v,\p)$, then
the set $\{\orz,\v,\rho \v,\w\}$ is coplanar.  Repeated application of Lemma~\ref{lemma:A} gives
$$
\angle(\v) = \angle(\rho \v) = \cdots = \angle(\rho^{-1} \w) = \pi,
$$
which is contrary to $\angle(\v') = \pi$.  
The strict inequalities imply $\p\in \Wdart(x)$ as desired.  The case $\v(x)=\w$ is similar.

Now assume that $\u=\v(x)\ne \v,\w$.  
By Lemma~\ref{lemma:A},  one may assume that $\{\orz,\u,\v,\w\}$ is not coplanar. 
(Otherwise, the contradiction $\angle(\v')=\pi$ or $\angle(\w')=\pi$ is reached.) Then
$$
\op{aff}\{\orz,\u,\v\}\cap C^0 \subset \op{aff}\{\orz,\u,\v\}\cap \op{aff}\{\orz,\v,\w\} \cap C^0 = \op{aff}\{\orz,\v\} \cap C^0 = \emptyset.
$$
Thus, $C^0$ is disjoint from $\op{aff}\{\orz,\u,\v\}$ and is similarly disjoint from $\op{aff}\{\orz,\u,\w\}$.  

We have the following facts:
$$
\v,\w\in W(\orz,\u,\v,\w),\quad C^0\{\v,\w\} \subset W(\orz,\u,\v,\w).
$$
Also,
$$
\begin{array}{rll}
C^0 &= C^0\cap W(\orz,\u,\v,\w) \\
     &\subset C^0 \cap (W^0(\orz,\u,\v,\w) \cup \op{aff}\{\orz,\u,\v\} \cup \op{aff}\{\orz,\u,\w\})\\
     &\subset C^0 \cap W^0(\orz,\u,\v,\w)\\
     &\subset \Wdart(x).
%\v,\w\in \barW &= \{\p \mid 0 \le \op{azim}(\orz,\u,\v,\p) \le \op{azim}(\orz,\u,\v,\w)\}.\\
%C^0 &\subset \barW\\
%\bar W &\subset W(\orz,\u,\v,\w) \cup \op{aff}\{\orz,\u,\v\} \cup \op{aff}\{\orz,\u,\w\}\\
%W(\orz,\u,\v,\w) & \subset \Wdart(x),
\end{array}
$$
%$$
%\Wdart(x) = \bWdart(x)\setminus (\op{aff}\{\orz,\u,\v\}\cup\op{aff}\{\orz,\u,\w\}).
%$$
\end{proof}

\begin{lemma}\guid{TECOXBM}\rating{ZZ}\label{lemma:2hm-slice}
Let $(V,E,F,G)$ be a special fan.  Let $\v,\w\in V$ be distinct elements such that $\norm{\v}{\w}=2\hm$ and such that $\{\v,\w\}\not\in E$.
Then $\v$ and $\w$ are independent, 
and $C^0\{\v,\w\}\subset \Wdart(x)$ for all $x\in F$.
\end{lemma}

\begin{proof} By Remark~\ref{remark:fan-verify}, independence follows from the strict triangle inequality
$$
\norm{\v}{\w} \le 2\hm < 2 + 2 \le \normo{\v} + \normo{\w}.
$$
Assume for a contradiction that the conclusion of the lemma is false.  Then by Lemma~\ref{lemma:internal}, all the intermediate internal angles between $\v$ and $\w$ are equal to $\pi$.  As a result, (after interchanging $\v$ and $\w$ if necessary), the set $\{\orz,\v,\rho\v,\rho^2\v,\ldots,\rho^r\v\}$ is planar, and
$$
\arc_V(\orz,\{\v,\rho^r\v\}) = \sum_{i=0}^{r-1} \arc_V(\orz,\{\rho^i\v,\rho^{i+1}\v\}),
$$
where $\rho^r \v = \w$ and $1 < r \le k-1$.
However, the left-hand side is at most
$$
\arc(2,2,2\hm) < 1.5,
$$
while the right-hand side is at least
$$
2\arc(2\hm,2\hm,2) > 1.5.
$$
This gives the desired contradiction.
\end{proof}

\subsection{slicing}

\begin{definition}[slice] Let $(V,E,F)$ be a cyclic fan.  Assume that $\v,\w\in V$ are
independent  and that
$(V,E')=(V,E\cup \{\{\v,\w\}\})$ is a fan.  Let $F'$ be the face of $\op{hyp}(V,E')$ 
containing the dart $(\w,\v)$.  Write
$$(V[\v,\w],E[\v,\w],F[\v,\w])$$
for the localization of $(V,E')$ along $F'$.  Explicitly,
$$
\begin{array}{lll}
V[\v,\w] &= \{\v,\rho \v,\rho^2 \v,\ldots,\w\};\\
E[\v,\w] &= \{\{\v,\rho \v\},\ldots,\{\rho^{-1}\w,\w\},\{\w,\v\}\};\\
F[\v,\w] &= \{(\v,\rho \v),(\rho \v,\rho^2 \v),\ldots,(\rho^{-1}\w,\w),(\w,\v)\}.
\end{array}
$$
The triple $(V[\v,\w],E[\v,\w],F[\v,\w])$ is called the {\it slice\/} of $(V,E,F)$ along
$(\v,\w)$.
\end{definition}
\indy{Index}{fan!cyclic}%
\indy{Index}{slice}
\indy{Notation}{1@$\cdot[\v,\w]$ (slicing a fan)}

To allow for more than one cyclic fan $(V,E,F)$,  expand the notation, writing $\angle(H,\v)$ for $\angle(\v)$ in the hypermap $H$.  Similarly, write $\Wdart(H,\v)$ for $\Wdart(x)$, and so forth.
\indy{Notation}{azimhv@$\op{azim}(H,\v)$}%
\indy{Notation}{wdart@$\Wdart$}%


\begin{lemma}[slicing]\guid{EJRCFJD}\rating{ZZ}\label{lemma:slice}  Let $(V,E,F)$ be a cyclic fan with hypermap $H$.  Pick $\v,\w\in V$. For each $\u\in \{\v,\w\}$, assume that $\u$ is independent of all elements of $V\setminus\{\u\}$.    Assume that $C^0\{\v,\w\}\subset \Wdart(x)$ for all darts $x\in F$.  Then
\begin{itemize}
\item $(V[\v,\w],E[\v,\w],F[\v,\w])$ and $(V[\w,\v],E[\w,\v],F[\w,\v])$ are cyclic fans.  
\item Let $H[\v,\w]$ and $H[\w,\v]$ be their hypermaps, respectively.  Let $g:V\to\ring{R}$ be any function.  Then
$$
\sum_{\v\in V} g(\v)\angle(H,\v) = \sum_{\v\in V[\v,\w]}g(\v)\angle(H[\v,\w],\v) + \sum_{\v\in V[\w,\v]}g(\v)\angle(H[\w,\v],\v).
$$
\end{itemize}
\end{lemma}
\indy{Index}{slice}%
\indy{Index}{fan!cyclic}%

\begin{proof} 
\claim{$(V,E')$ is a fan, where $E' = E\cup \{\{\v,\w\}\}$.}
 Indeed, except for the intersection property, all of the properties of a fan follow trivially from the fact that $(V,E)$ is a fan and
that $\v$ and $\w$ are independent.  (Note the similarity with Lemma~\ref{lemma:add-edge}.)
The intersection property also is trivial except in the case $\ee=\{\v,\w\}$ and $\ee'\setminus \ee\ne \emptyset$.  Pick $\u\in \ee'\setminus\ee$.  It follows from the node
partition of Lemma~\ref{lemma:disjoint} that
$$
\begin{array}{lll}
C(\ee) \cap C(\ee') &= (C(\v) \cap C(\ee')) \cup (C(\w)\cap C(\ee')) \\
 &= C(\{\v\}\cap \ee') \cup C(\{\w\}\cap \ee') \\
 &= C(\{\v,\w\}\cap \ee').
\end{array}
$$
The intersection property thus holds and $(V,E')$ is a fan.

It follows by Lemma~\ref{lemma:localization} that $(V[\v,\w],E[\v,\w])$ is a fan.

The second conclusion of the lemma follows from the following identities.
If $\u\ne \v,\w$ with $\u\in V[\v,\w]$, then $\u\not\in V[\w,\v]$ and 
\begin{equation}
\Wdart(H,\u)=\Wdart(H[\v,\w],\u),\quad \angle(H,\u) = \angle(H[\v,\w],\u).
\end{equation}
If $\u\in\{\v,\w\}$, then 
$\angle(H,\u)=\angle(H[\v,\w],\u) +\angle(H[\w,\v],\u)$.

Finally, it remains to be seen that the fan is cyclic.  Lemma~\ref{lemma:localization} already shows that the hypermap is isomorphic to $H_{2k}$. and that $F[\v,\w]$ can be identified with a face.  The condition $V[\v,\w]\subset \bWdart(x)$ follows from the fact that the angles $\beta(i)$ are increasing in Lemma~\ref{lemma:monotone}.
\end{proof}



The slicing procedure can also be applied to a special fan $(V,E,F,G)$.
Abbreviate the slices as
$$
\begin{array}{lll}
(V',E',F')&=(V[\v,\w],E[\v,\w],F[\v,\w]),\quad\text{and}\\
 (V'',E'',F'')&= (V[\w,\v],E[\w,\v],F[\w,\v]).
\end{array}
$$
Both edge sets $E'$ and $E''$ contain $\{\v,\w\}$.  The sets $G'$ and $G''$ can be defined
as 
$$
\begin{array}{lll}
G' &= \{\{\v,\w\}\} \cup (E'\cap G).\\
G'' &= \{\{\v,\w\}\} \cup (E''\cap G).\\
\end{array}
$$ 

Let $(r',s')$ and $(r'',s'')$ be the parameters for $(V',E',F',G')$ and $(V'',E'',F'',G'')$, respectively.
Set $k=r+s$, $k'=r'+s'$, and $k''=r''+s''$.

\begin{lemma}\guid{IXZYRSY}\rating{ZZ}\label{lemma:param-add}  
Let $(V,E,F,G)$ be a special fan (with parameters $s,r$).  
Pick distinct elements $\v,\w\in V$  such that $\{\v,\w\}\not\in E$.
Assume that $\norm{\v}{\w}=2\hm$.
For each $\u\in \{\v,\w\}$, assume that $\u$ is independent of all elements of $V\setminus\{\u\}$.    
%Assume that $C^0\{\v,\w\}\subset \Wdart(x)$ for all darts $x\in F$. 
Then $(V',E',F',G')$ and $(V',E',F',G')$ are special fans.  Moreover,
the parameters satisfy the relations
$$
k'+k'' = k + 2,\quad s'+s'' = s + 2,\quad r'+r''=r.
$$
Finally,
$$
\tau(V,E,F)= \tau(V'',E'',F'') +\tau(V',E',F').
$$
\end{lemma}

By interchanging $\v$ and $\w$, the lemma also asserts that $(V'',E'',F'',G'')$ is a special fan.

\begin{proof}  Each of the defining properties of a special fan will be considered in turn.  By Lemma~\ref{lemma:2hm-slice}, if $\{v,w\}\not\in E$, then $C^0\{v,w\}\subset \Wdart(x)$ for all darts $x\in F$.

The properties \case{packing}, \case{annulus}, \case{diagonal}, \case{subset}, \case{g norm}, and \case{e norm} follow directly from definitions and the corresonding properties for $(V,E,F,G)$.
The property \case{cyclic fan} follows from Lemma~\ref{lemma:slice}.

The relations between the constants $k,s,r$ for the various fans follows directly from the construction.
For example, there are two darts more in $F''\cup F'$ than in $F$.  

\claim{Property~\case{card} holds.}  Indeed, recall that each face in the hypermap of a fan has at least three darts.  Hence $3\le k$, $3\le k'$, and $3\le k''$.
The sets $G'$ and $G''$ contain $\{\v,\w\}$.  Hence $1\le s'$ and $1\le s''$.
From $0\le r\le 6 - 2s$, it follows that $s\le 3$.  If $s=3$, then $r=0$, and $k=r+s=3$.  This means that $F$ is a triangle.  For some ordering of the pair, $x=(\v,\w)$ is a dart in $F$.
Hence  $C^0\{\v,\w\} \not\subset \Wdart(x)$.  This is contrary to hypothesis.  Therefore, $s\le 2$.

Now for the verifications.
$$0\le s' = s + 2 - s'' \le s+1\le 3.$$
$$3-s'\le k'-s' = r'.$$
$$k' = k + 2 - k'' \le k-1.$$
$$
r'= k'+s' - 2 s' \le (k-1) + (s+1) - 2s' =k+s - 2s' = r + 2s -2s' \le 6 - 2s'.
$$
This completes the proof of property~\case{card}.

The additivity of $\tau$ follows directly from the azimuth angle estimates in Lemma~\ref{lemma:slice}.
\end{proof}


\section{Minimality}

%\subsection{minimality}


\begin{definition}[minimal~fan,~$k_{min}$,~$\tau^d_{min}$]
Let $k_{min}$ be the minimum of $r+s$ over
all special fans $(V,E,F,G)$ such that 
\begin{equation}\label{eqn:kmin}
\tau(V,E,F) < d (r,s),
\end{equation}
where $(r,s)$ are the parameters of the special fan.
If no special fan exists that satisfies the inequality, then set $k_{min}=0$.
Let $\tau^d_{min}$ be the infimum of $\tau(V,E,F)-d(r,s)$ over all special fans
$(V,E,F,G)$ whose parameters $(r,s)$ satisfy $r+s=k_{min}$.  If no special fans
exist with parameters $r+s=k_{min}$, then set $\tau^d_{min}=0$.
Any special fan $(V,E,F,G)$ with parameter $r+s=k_{min}$ such that
$$
\tau^d_{min}= \tau(V,E,F)-d(r,s)
$$
is a {\it minimal}  fan.
\end{definition}
\indy{Index}{fan!minimal}
\indy{Index}{minimal fan}


\begin{lemma}\guid{ADKOXQY}\rating{ZZ}\label{lemma:c-bound}
There is a constant $c>2\hm$ such that for every minimal fan $(V,E,F,G)$
and every distinct $\v,\w\in V$, either $\{\v,\w\}\in E$ or $\norm{\v}{\w}\ge c$.
\end{lemma}

\begin{proof} Pick a sequence of fan data $\p$ such that the associated
fan is a minimal fan and such that 
$$
\min_{|i-j|>1} \norm{\p_i}{\p_j}
$$
is tending to the minimal value $c$.  By passing to s subsequence, we may
assume without loss of generality that every term in the sequence has the same shape $(k_{min},I)$ for some $I$.  The sequence then lies in a compact metric space.  Passing again to a subsequence, we may assume without loss of generality that the sequence converges.  The limiting value is a fan datum whose associated minimal fan $(V,E,F,G)$ satisfies
$$
\norm{\v}{\w}=c,
$$
for some $\v,\w\in V$ such that $\{\v,\w\}\not\in E$.

Suppose for a contradiction that $c=2\hm$.  Then by Lemma~\ref{lemma:2hm-slice}, $C^0\{\v,\w\}\subset \Wdart(x)$, for all $x\in F$.
The fan can be sliced along the $\{\v,\w\}$.  
The parameters for the new pieces satisfy:
$$
k' = k+2 - k'' \le k-1.
$$
The function $d$ is additive over slices.  For some constants $d_1$ and $d_2$,
\begin{equation}\label{eqn:drs}
\begin{array}{lll}
 d(r,s) &= d_1 (2 - s) + d_2 (r + 2 s-4) \\
&= d_1 (2-s') + d_2 (r'+2 s'-4) + d_1 (2-s'') + d_2 (r''+2s''-4)\\
&= d(r',s') + d(r'',s''). \\
\end{array}
\end{equation}
By Lemma~\ref{lemma:param-add}, one of the two resulting fans satisfies inequality~\ref{eqn:kmin}.  This is contrary to the choice of $k_{min}$ in the definition of minimality.
\end{proof}





\subsection{genericity}

\begin{lemma}\guid{RRAJQBH}\rating{ZZ}\label{lemma:circular-nonmin}
Every minimal fan is generic.
\end{lemma}

\begin{proof}
By the classification of cyclic fans, it is enough to show that the fan is not circular and not lunar.

\claim{In a minimal fan, $\op{azim}(x)<\pi$ for some $x\in F$.}  Otherwise,
the result follows from the following estimates:
$$
\begin{array}{lll}
\tau(V,E,F) &=\sum_{x\in F} \op{azim}(x)\left(1 + \dfrac{\sol_0}{\pi}  \dfrac{\normo{v}-2}{2\hm-2}\right) + \left(\pi+{\sol_0}\right) (2- k(F))\\
 &\ge \sum_{x\in F} \pi + \left(\pi+{\sol_0}\right) (2- k(F))\\
 &= 2\pi + 2\sol_0 - k(F) \sol_0\\
 &\ge 2\pi - 4\sol_0\\
&> 0.7578\\
&=0.103 (2) + 0.2759 (2)\\
&\ge 0.103 (2-s) + 0.2759 (r+2s-4) \\ 
&= d(r,s)\\
\end{array}
$$
This proves the claim.


Section~\ref{sec:deformation} develops a theory of deformations of cyclic fans $(V,E,F)$.  Recall that a deformation is  continuous function $\varphi:V\times I\to\ring{R}^3$.  A deformation determines sets $V(t)$, $E(t)$, $F(t)$, and $G(t)$ for each $t\in I$.


\claim{A minimal fan is not lunar.}  Otherwise,
suppose for a contradiction, that $(V,E,F)$ is lunar, special, and
minimal.  Let $\v,\w\in V$ be dependent.   By the previous claim, we may assume without loss of generality that $\angle(\v)=\angle(\w)<\pi$.   Example~\ref{example:lunar} describes a deformation of the lunar fan $(V,E,F)$
over $I=\leftclosed0,1\rightopen$.  The deformation fixes $\normo{\u}$
and is non-increasing in $\angle(\u(t))$, for $\u\in V$.
From the defining formula for $\tau$, it follows that the function $\tau(V(t),E(t),F(t))$ is a decreasing function of $t$.

\claim{For sufficiently small positive $t$, the deformed value is a special fan.}   The deformed value is a cyclic fan for small positive $t$ by Lemma~\ref{lemma:fan-open}.   The property~\case{diagonal} of special fans holds by Lemma~\ref{lemma:c-bound}.  The other properties of a special fan follow immediately from the construction. 

However, the function $\tau$ attains its minimum at $(V,E,F)$ and hence
$\tau(V(t),E(t),F(t))$ cannot be decreasing in $t$.  This contradiction proves the claim that a minimal fan is not lunar.
\end{proof}

\begin{lemma}\guid{QAGHDMN}\rating{ZZ}
Let $(V,E,F)$ be a cyclic fan.
If $V$ is contained in a plane through the origin, then $(V,E,F)$ is not generic.
\end{lemma}

\begin{lemma}\guid{EAEKYHM}\rating{ZZ}\label{lemma:3-nonflat}
Every non-generic cyclic fan $(V,E,F)$ has at least three nonflat elements in $V$.
\end{lemma}

\begin{proof}
If there is at most one nonflat element $\v\in V$, then there exists a plane $A$ through the origin that contains $V$.  This is not generic.

If there are exactly two elements nonflat elements in $V$, then there exist two half-planes through the origin whose union contains $V$.  The two non-flat points necessarily lie along the intersection of the two planes.  They are dependent elements of a lunar fan.  This is not generic.
\end{proof}

\subsection{irreducibility}

\begin{definition}[extremal,~minimal] Let $(V,E,F,G)$ be a special fan and let $\{\v,\w\}\in E$.
The edge $\{\v,\w\}\in E$ is {\it $G$-extremal} if $\{\v,\w\}\in G$ or
$$
\norm{\v}{\w}\in\{2,2\hm\}.
$$
The edge $\{\v,\w\}\in E$ is {\it $G$-minimal} if $\{\v,\w\}\in G$ or
$$
\norm{\v}{\w}=2.
$$
\end{definition}
\indy{Index}{extremal ($G$-extremal edge)}
\indy{Index}{minimal ($G$-minimal edge)}

\begin{definition}[irreducible]  Let $(V,E,F,G)$ be a special fan and let $\v\in V$.  
Write $\v_i  = \rho^i \v$.
The special fan is {\it irreducible} at $\v$
if the following properties hold. 
\begin{itemize}
\item \case{card} %$k=\card(F)$,
 Let      $s=\card(G)$ and $r=\card(E) - s = \card(F)-s$.  Then
$$0\le s \le 3,\quad\text{and}\quad3-s \le r \le 6 - 2s.$$
\item \case{extreme edge} For every $\w\in E(\v)$,  the edge $\{\v,\w\}\in E$ is $G$-extremal.
\item \case{flat exists}  If none of $\v_1,\v_2,\v_3,\v_4$ is flat, then $r+s\le 4$. 
% $r+s=4$. %, and $\norm{\u}{\w}\in\{ 2,2\hm}$ for all $\{\u,\w\}\in E$.
\item \case{no triple flat} At least one of $\v_0$, $\v_1$, and $\v_2$ is not flat.
\item \case{balance} If $\v$ is flat and if $\{\u,\v\},\{\w,\v\}\in E\setminus G$, then
$$
\norm{\u}{\v} = \norm{\w}{\v}.
$$
\item \case{g flat} If $\v_1$ and $\v_2$ are both flat, then 
$$G\cap \{\{\v_0,\v_1\},\{\v_1,\v_2\},\{\v_2,\v_3\}\} = \emptyset.$$
\item \case{flat middle} If $\v_1$ and $\v_2$ are both flat, then
$$
\norm{\v_1}{\v_2} = 2.
$$
\item \case{minimal node} Either $\normo{\v}=2$ or there exists $\w\in E(\v)$ such
that $\{\v,\w\}$ is $G$-minimal.
\item \case{minimal node flat} If $\v$ is flat, and $\{\u,\v\}\in E\setminus G$, then
$\norm{\u}{\v}=2$ or $\normo{\v}=2$.
\item \case{flat extremal} If $\v_1$ and $\v_2$ are both flat, then either $\v_1$ or $\v_2$
has extremal norm:
$$\normo{\v_1}\in \{2,2\hm\}\quad\text{ or }\quad \normo{\v_2}\in \{2,2\hm\}.$$
\item \case{extremal node}  If $\v_0$, $\v_1$, and $\v_2$ are not flat, then
$$
\normo{\v_1}\in \{2,2\hm\}.
$$
\item \case{flat extremal node}  If $\v_1$ is flat, but none of $\v_0,\v_2,\v_3$ is flat, then
$$
\normo{\v_1}\in \{2,2\hm\}\quad\text{ or }\quad\normo{\v_2}\in \{2,2\hm\}.
$$
\item \case{flat extremal node sym}  If $\v_2$ is flat, but none of $\v_0,\v_1,\v_3$ is flat, then
$$
\normo{\v_1}\in \{2,2\hm\}\quad\text{ or }\quad\normo{\v_2}\in \{2,2\hm\}.
$$
\item \case{flat count} There are at least $3$ nonflat elements of $V$.
\end{itemize}
\end{definition}
\indy{Index}{irreducible}



\begin{lemma}\guid{KDKAGRS}\rating{ZZ}\label{lemma:min-irred}
Every minimal fan $(V,E,F,G)$ is irreducible at every $\v\in V$.
\end{lemma}

The proof appears in the next subsection as a series of small verifications.

\subsection{features}

 If there are $i$ consecutive flat nodes $\v_1,\ldots,\v_i$, then there are  $i+2$ corresponding nodes that lie in a plane $A$ through the origin: $\{\v_0,\ldots,\v_{i+1}\}\subset A$.

\begin{lemma}\guid{FPITROS}\rating{ZZ}
Every minimal fan $(V,E,F,G)$ satisfies property \case{card} of irreducibility.
\end{lemma}

\begin{proof} This holds by the property \case{card} of special fans.
\end{proof}


\begin{lemma}\guid{TESVAFW}\rating{ZZ}
Every minimal fan $(V,E,F,G)$ satisfies property \case{no triple flat} of irreducibility at every $\v\in V$.  
\end{lemma}

\begin{proof}  Three flat nodes produces an arc
$$
\arc_V(0,\{\v_0,\v_4\}) = \op{per}(V,E,F,\v_0,\v_4) \ge4\arc(2\hm,2\hm,2) > \pi,
$$
but the remaining edges  have combined length at most
$$
\arc_V(0,\{\v_0,\v_4\})\le 2\arc(2,2,2\hm) < \pi.
$$
\end{proof}

\begin{lemma}\guid{GOKZLRP}\rating{ZZ}
Every minimal fan $(V,E,F,G)$ satisfies property \case{flat count} of irreducibility for every $\v\in V$.
\end{lemma}

\begin{proof}  
This lemma is a consequence of Lemma~\ref{lemma:3-nonflat}.
\end{proof}



\begin{lemma}\guid{SDCCMGA}\rating{ZZ}
Every minimal fan $(V,E,F,G)$ satisfies property \case{g flat} of irreducibility at every $\v\in V$.
\end{lemma}

\begin{proof}  Argue by contradiction.  From the constraints on $r$ and $s$, if $G\ne\emptyset$, then $s>0$ and $r+s\le 5$. 

By property \case{flat count}, it follows that $r+s> 4$.
%then $V = \{\v,\rho \v,\rho^2\v,\rho^3\v\}$.  A plane $A$ contains $V\cup\{\orz\}$.   This is not generic.

Thus, $r+s=5$.  A plane $A$ contains $\{\orz,\rho^{-1}\v,\v,\rho\v,\rho^2\v\}$. 
We obtain a contradiction by computing $\arc_V(\orz,\rho^{-1}\v,\rho^2\v)$ in two ways.  On the one hand, it is equal to the partial perimeter from $\rho^{-1}\v$ to $\rho^2 \v$.  One the other hand,
it is estimated by the triangle inequality, applied to the other two edges:
$$
\begin{array}{lll}
\arc_V(\rho^{-1}\v,\rho^2\v)
&\le\arc_V(\rho^2\v,\rho^3\v) + \arc_V(\rho^3\v,\rho^{-1}\v) \\
&\le\arc(2,2,2\hm)+\arc(y,2,2\hm)\\
&<\arc(y,2\hm,2)+\arc(2\hm,2\hm,2) +\arc(2\hm,2\hm,2\hm)\\
&\le\arc_V(\rho^{-1}\v,\rho^2\v).
\end{array}
$$
where $y=\normo{\rho^{-1} \v}$.
This is a contradiction.
\end{proof}

\begin{lemma}\guid{BAWWPPB}\rating{ZZ}
Every minimal fan $(V,E,F,G)$ satisfies property \case{extreme edge} of irreducibility for
every $\v\in V$.
\end{lemma}

\begin{proof} 
By Lemma~\ref{lemma:3-nonflat}, there are at least three flat elements of $V$.

Assume for a contradiction that the edge $\{\v,\rho\v\}\in E\setminus G$ of the minimal fan $(V,E,F,G)$ is not extremal.  Let $\u = \rho^i\v$, where $i$ is the smallest positive index for which $\rho^i\v$ is not flat.
Let $r$ be the largest negative index such that
$\rho^r\u$ is non-flat.  Let $s$ be the smallest positive index such that
$\rho^s\u$ is non-flat.

Define a deformation $\varphi$ that satisfies the following conditions.
\begin{itemize}
\item The deformation fixes every element of $V\setminus\{\rho^{r+1}\u,\ldots,\rho^{s-1}\u\}$.  
\item The deformation  fixes the norm $\normo{\w(t)}$ for all $\w\in V$.  
\item The deformation  fixes the norms $\norm{\w(t)}{\varphi(\rho \w,t)}$ for all $\w\in V\setminus\{\v\}$.  
\item The set $\{\orz,\rho^r\u,\varphi(\rho^{r+1}\u,t),\ldots,\varphi(\u,t)\}$ is coplanar.
\item The set $\{\orz,\varphi(\u,t),\varphi(\rho\u,t),\ldots,\rho^s\u\}$ is coplanar.
\item $\norm{\varphi(\v,t)}{\varphi(\rho\v,t)} = \norm{\v}{\rho\v}+t$.
\end{itemize}
There is a unique deformation that satisfies these properties.  The deformation is a special fan for sufficiently small $t$.

\claim{The function $t\mapsto \tau(V(t),E(t),F(t))$ does not have a local minimum at $t=0$.}  
Indeed,  by Definition~\ref{def:tau}, $\tau$ has the form
$$
g(s) = \dih(2,2,2,a+s,b,c) e_1 + \dih(2,2,2,b,c,a+s) e_2 + \dih(2,2,2,c,a+s,b) e_3,
$$
for some $e_i\in\leftclosed1,1+\sol_0/\pi\rightclosed$, some reparametrization $s=s(t)$ of $t$ such that $s(0)=0$, and some parameters $a,b,c$.
The parameters $a,b,c$ lie in $\leftclosed2/\hm,4\rightclosed$ and satisfy $\Delta\ge0$, where  $\Delta = \Delta(4,4,4,a^2,b^2,c^2)$.

If $g$ has a local minimum at $s=0$, then
\begin{equation}\label{eqn:g''}
\Delta g'(0)^2 - 0.01\Delta^{3/2} g''(0) \le 0.
\end{equation}
Indeed, the first derivative is zero and the second derivative is non-negative.
However, Calculation~\ref{calc:Lexell}
shows that the left-hand side of inequality~(\ref{eqn:g''}) is positive for all parameters $a,b,c,e_1,e_2,e_3$.
Thus, there is no local minimum.

The claim contradicts that assumed minimality of $(V,E,F,G)$.  This completes the proof.  
\end{proof}


\begin{lemma}\guid{PZFSEVR}\rating{ZZ}
Every minimal fan $(V,E,F,G)$ satisfies property \case{minimal node} of irreducibility  for
every $\v\in V$.
\end{lemma}

\begin{proof} If the conclusion is false, then for sufficiently small positive $a$,  the deformation $\varphi$ over $[0,a]$ given by
$$
\varphi(\u,t) =
\begin{cases}
\u, & \text{if }\u \ne \v,\\
(1-t) \v, & \text{if }\u = \v\\
\end{cases}
$$
decreases $\tau$.  It is a deformation of special fans.  This shows that $(V,E,F,G)$ is not minimal.
\end{proof}

\begin{lemma}\guid{JLXMEJN}\rating{ZZ}   %case minimal node flat
Every minimal fan $(V,E,F,G)$ satisfies property \case{minimal node flat} of irreducibility  for
every $\v\in V$.
%\case{minimal node flat} If $v$ is flat, and $\{u,v\}\in E\setminus G$, then
%$\norm{\u}{\v}=2$ or $\normo{\v}=2$.
\end{lemma}

\begin{proof} 
Let $A$ be the plane containing $\{\orz,\rho^{-1}\v,\v,\rho\v\}$.  Assume that $\{\u,\v\}\in E\setminus G$.  Let $\u\ne \w\in E(\v)$.  Then $\u,\v,\w\in A$. 

If the conclusion is false, then for sufficiently small positive $a$, let $\varphi$ be the  deformation over $[0,a]$ that fixes all elements of $V$ other than $\v$ and moves $\v$ in the plane $A$ around the circle of radius $\norm{\w}{\v}$ with center $\w$.  The direction of the circular motion can be chosen to decrease $\tau$.  It is a deformation of special fans.  This shows that $(V,E,F,G)$ is not minimal.
\end{proof}

\begin{lemma}\guid{TPKKQOL}\rating{ZZ}
Every minimal fan $(V,E,F,G)$ satisfies property \case{balance}
of irreducibility at every $\v\in V$.
%For every edge $\{\v,\w\}\in E$ in a minimal fan $(V,E,F,G)$, 
%$$\norm{\v}{\w}\in\{2,2\hm\}.$$
\end{lemma}

\begin{proof}  Assume for a contradiction that the property fails.  In view of properties \case{extreme edge} and \case{minimal node flat}, after possibly swapping $\u$ and $\w$, we may assume without loss of generality that
$$
\normo{\v}=2,\quad \norm{\u}{\v}=2,\quad \norm{\w}{\v}=2\hm.
$$
For sufficiently small positive $a$, let $\varphi$ be the deformation over $[0,a]$
that fixes all elements of $V$ other than $\v$ and moves $\v$ in the plane of
$\{\orz,\u,\v,\w\}$ around a circle of radius $2$ with center at the origin.  The direction of the deformation can be chosen to be increasing in $\norm{\u}{\v}$.  The function $\tau$ is constant along the deformation.  The deformation carries minimal fans to minimal fans.  However, the deformed fan does not satisfy property \case{extreme edge}.  This is a contradiction.
\end{proof}




\begin{lemma}\guid{CFJSRQH}  Every minimal fan $(V,E,F,G)$ has property \case{flat middle} at every $\v\in V$.
\end{lemma}

\begin{proof} %Write $\v_i = \rho^{i-2} \v$.  
Assume for a contradiction tht the conclusion is false.  There is a plane $A$ that contains $\{\orz,\v_0,\v_1,\v_2,\v_3\}$.
By property \case{g flat}, none of the edges $\{\v_0,\v_1\}$, $\{\v_1,\v_2\}$, $\{\v_2,\v_3\}$ lies in $G$.  Furthermore, by \case{extreme edge} and \case{balance}, the distances $\norm{\v_i}{\v_{i+1}}$ are equal to one another and take values in $\{2,2\hm\}$, for $i=0,1,2$.  
By \case{minimal node}, if these distances are $2\hm$, then the partial perimeter from $\v_1$ to $\v_4$ is at least
\begin{equation}\label{eqn:3side}
\arc(y_1,2,2\hm)+\arc(2,2,2\hm)+\arc(2,y_4,2\hm),
\end{equation}
where $\normo{\v_i}=y_i$.
However, there are at most two other nodes ($\v_4$ and $\v_5$) and three other edges in the special fan.  By the triangle inequality, the sum of these three lengths is at most \eqn{eqn:3side}.
Thus, equality is obtained in the triangle inequality.  This implies that $\v_4$ and $\v_5$ lie in the plane $A$.  Thus, $V\subset A$.   This is not generic.
\end{proof}

\begin{lemma}\guid{OUCPLRI} 
Every minimal fan $(V,E,F,G)$ satisfies property \case{flat extremal} of irreducibility at every $\v\in V$.
\end{lemma}

\begin{proof}
Assume there are two adjacent flat elements $\v_1,\v_2$, so that there is a plane $A$ that contains $\{\orz,\v_0,\v_1,\v_2,\v_3\}$.
By previously established properties of irreducibility,
\indy{Index}{angle!flat}%
$\norm{\v_i}{\v_{i+1}}=2$, for $i=0,1,2$.
Let $y_i = \normo{\v_i}$.
%

\claim{There exists a deformation that fixes all $\u\ne \v_1,\v_2$, maintains flatness at $\v_1$ and $\v_2$, maintains the coplanarity of $\{\orz,\v_0,\v_1(t),\v_2(t),\v_3\}$, and that keeps $y_1+y_2$, $\norm{\v_0}{\v_1}$, $\norm{\v_2}{\v_3}$,  and $\tau$ constant, while increasing $\norm{\v_1}{\v_2}$.}  Indeed,
the partial perimeter $c$ is given by a sum of three terms:
  $$
  c=\sum_{i=0}^2\arc(y_i,y_{i+1},2).
  $$
Set
$$
g(t_1,t_2) = \arc(y_0,y_1+t_1,2) + \arc(y_1+t_1,y_2-t_1,2+t_2) + \arc(y_2-t_1,y_3,2) - c.
$$
Then $g(0,0)=0$.  

Use subscript notation for partial derivatives of $g$.  If $g_1(0,0) \ne 0$, then by the implicit function theorem there is a function $h$ locally near $0$ such that $h(0)=0$ and $g(h(t_2),t_2)=0$.  Define a deformation that fixes $\u\ne\v_1,\v_2$ and moves $\v_1,\v_2$ in the fixed plane $\op{aff}\{\orz,\v_1,\v_2\}$ subject to the constraints $\norm{\v_1(t)}{\v_2(t)}=2+t$, $\normo{\v_1(t)} = 2+h(t)$, and $\normo{\v_1(t)} = 2-h(t)$.  These conditions uniquely determine the deformation.  The deformed value is a special fan for sufficiently small positive $t$.  It keeps $\tau$ fixed.  Hence the deformed fan is also  minimal.  However, the existence of this deformation contradicts property \case{flat middle}.

On the other hand, if $ g_1(0,0) =0$, then a calculation gives
$$g_2(0,0) = \dfrac{2}{y_1y_2\sqrt{\ldots}} > 0.$$
Again by the implicit function theoreom there is a function $h$ locally near $0$ such that $h(0)=0$ and $g(t_1,h(t_1))=0$.   Define a deformation that fixes $\u\ne \v_1,\v_2$ and moves $\v_1$ and $\v_2$ within the fixed plane $\op{aff}\{\orz,\v_1,\v_2\}$ by the conditions $\normo{\v_1(t)} = 2 + t$, $\normo{\v_2(t)} = 2 - t$, and $\norm{\v_1(t)}{\v_2(t)}=2+h(t)$.  These conditions uniquely determine the deformation.  The deformed fan is special for sufficiently small $t$.  It keeps $\tau$ fixed.  Hence the deformed fan is also a minimal.

By implicit differentiation,  $h'(0) = 0$ and 
$g_2(0,0) h''(0) = -g_{11}(0,0)$.  Thus, $h''(0)$ and $g_{11}(0,0)$ have opposite signs.
The function $t_1\mapsto g(t_1,0)$ equals, up to a constant, the sum of three terms of the form
$\arc(\cdot,\cdot,2)$.
Calculation~\ref{calc:2der} shows that the second derivative of each term with respect to $t_1$ is negative at $t_1=0$.
Thus, $h''(0)>0$ and for sufficiently small nonzero $t$, 
$\norm{\v_1(t)}{\v_2(t)}>2$.
  However, the existence of this deformation contradicts property \case{flat middle}.
\end{proof}







\begin{lemma}\guid{DWXPIHA}\rating{ZZ}
Every minimal fan $(V,E,F,G)$ satisfies property \case{extremal node} for every $\v\in V$.
\end{lemma}

\begin{proof} %\guid{DFSLRHA} 
Assume for a contradiction that the conclusion is false.  By property~\case{extremal edge}, 
$$
\norm{\v_{-1}}{\v_0},~\norm{\v_0}{\v_1}\in \{2,2\hm\}.
$$
Consider the deformation $\varphi:V\times I\to V$ given by
$$
\varphi(\u,t) =
\begin{cases}
\v(t), & \text{if } \u=\v\\
\u, & \text{otherwise},\\
\end{cases}
$$
where $\v(t)$ is the unique point in $\op{aff}_+(\{0,\rho^{-1}\v,\rho \v\},\v)$ determined by the conditions:
$$
\norm{\v(t)}{\u} = \norm{\v}{\u}, \text{~when~} \{\u,\v\}\in E,\quad
\text{~and~}
\normo{\v(t)} = \normo{\v} + t.
$$
For sufficiently small $t$, the deformed value  is a special fan.
The function $t\mapsto\tau(V(t),E(t),F(t))$ has the form
\begin{equation}\label{eqn:tau}
t\mapsto \tau(y_1+t,y_2,y_3,y_4,y_5,y_6) + c
\end{equation}
for some parameters $y_1,\ldots,y_6$ and $c$.
Calculation~\ref{calc:cc:d2a} shows that the function (\ref{eqn:tau})
has no local minimum at $t=0$.
This is contrary to the minimality of $(V,E,F,G)$.
\end{proof}



%The same can be done when there is one flat node forming a linear series of length $2$, with compressed edges:

\begin{lemma} \guid{DCEETTF}\rating{ZZ}
Every minimal fan $(V,E,F,G)$ satisfies properties \case{flat extremal node} and \case{flat extremal node sym} for every $\v\in V$.
\end{lemma}

\begin{proof}
The property \case{flat extremal node sym} is obtained from \case{flat extremal node} by symmetry $v_1\leftrightarrow v_2$, so it is enough to prove \case{flat extremal node}.  Assume for a contradiction that property \case{flat extremal node} fails.
Let $(V,E,F,G)$ be a minimal fan that fails.

Define a deformation $\varphi$ of the minimal fan that fixes $\u\ne \v_1,\v_2$ and that moves $\v_1,\v_2$ subject to the following constraints.
\begin{itemize}
\item $\{\orz,\v_0,\v_1(t),\v_2(t)\}$ is coplanar.
\item $\norm{\v_2(t)}{\v_3}$ is constant.
\item $\norm{\v_0}{\v_1(t)}$, $\norm{\v_1(t)}{\v_2(t)}$, and $\norm{\v_0(t)}{\v_2(t)}$ are constant.
\item $\normo{\v_2(t)} = \normo{\v_2} + t$.
\end{itemize}
These constraints uniquely determine the deformation. The deformed value is a special fan for sufficiently small $t$.

By Calculation~\ref{calc:cc:d2b}, the function does not have a local minimum at $t=0$.
This contradicts the assumption that $(V,E,F,G)$ is minimal.
\end{proof}



\begin{lemma}\guid{CMBZAOZ}\rating{ZZ}
Every minimal fan $(V,E,F,G)$ satisfies property \case{flat exists} of irreducibility for every $\v\in V$.
That is, if none of $\v,\rho \v,\rho^2 \v,\rho^3 \v\in V$ is flat, then $r+s\le 4$.
%then $r+s=4$. %, and $\norm{\v}{\w}= 2$, for all $\{\v,\w\}\in E$.
\end{lemma}

\begin{proof} 
Assume for a contradicton that the property fails for $(V,E,F,G)$.
In particular, the parameters satisfy $r+s>4$.
There are four consecutive nodes $\v_1,\v_2,\v_3,\v_4$ that are not flat.   By property \case{extreme edge},   each edge $\norm{\v_i}{\v_{i+1}}$ is $2$ or $2\hm$. Set $y_i = \normo{\v_i}$ and $y_{ij} = \norm{\v_i}{\v_j}$.   By \case{extremal node}, $y_i$ is $2$ or $2\hm$, for $i=2,3$.   The condition $r+s>4$ implies that $\{\v_1,\v_4\}\not\in E$ and $y_{14}\ge 2\hm$.

Define a deformation of $(V,E,F,G)$ by fixing $\u\ne \v_2,\v_3$ and moving $\v_2,\v_3$ according to the following constraints:
\begin{itemize}
\item $\norm{\v_1}{\v_2(t)}$, $\norm{\v_2(t)}{\v_3(t)}$, and $\norm{\v_3(t)}{\v_4}$ are constant.
\item $\normo{\v_2(t)}$ and $\normo{\v_3(t)}$ are constant.
\item $\norm{\v_1}{\v_3(t)} = y_{13} + t$.
\end{itemize}
These constraints uniquely determine the deformation. The deformed value is a special fan for sufficiently small $t$.  In this context, by Calculation~\ref{calc:cc:qua}, the function $\tau$ as a function of $t$ (with parameters $y_{13},y_1,y_4,y_{14}$) does not have a local minimum at $t=0$.
This is contrary to the assumed minimality of $(V,E,F,G)$.
\end{proof}


\subsection{emptiness}

\begin{lemma}\guid{LPQUDGF}\rating{ZZ}\label{lemma:min-empty}  
The set of minimal fans is empty.
\end{lemma}

\begin{proof}  
Let 
$$X= \{(V,E,F,G,\v_0) \mid  (V,E,F,G) \text{ is minimal and } \v_0\in V\}.
$$  The strategy is to partition $X$ into smaller sets and show that each set is empty.

For each $(V,E,F,G,\v_0)\in X$, there is a unique fan datum $\v$ and shape $(k_{min},I)$ such
that $\v_i = \rho^i\v_0$ and $(V,E,F,G) = (V(\v),E(\v),F(\v),G(\v,I))$.  The index $i$ ranges over the cyclic group $Z_k$ where $k=k_{min}$.  

For each $(V,E,F,G,\v_0)$ with fan datum $\v$, define a function
$$
\op{vlabel}:Z_k \to \{\op{low},\op{mid},\op{high}\} \times \{\op{flat},\op{nonflat}\}
$$
(where $\op{low}$, $\op{mid}$, $\op{high}$, $\op{flat}$, and $\op{nonflat}$ are symbolic labels for elements in sets of cardinality three and two, respectively).
Define
$$
\op{vlabel}(i)_1 = \begin{cases}
   \op{low}, &\text{if } \normo{\v_i} = 2,\\
   \op{high}, &\text{if } \normo{\v_i} = 2\hm,\\
   \op{mid}, &\text{otherwise}
\end{cases}
$$
and
$$
\op{vlabel}(i)_2 = \begin{cases}
   \op{flat}, &\text{if $\v_i$ is flat},\\
   \op{nonflat}, &\text{otherwise}.
\end{cases}
$$

For each $(V,E,F,G,\v_0)$ with fan datum $\v$, define a function
$$
\op{elabel}:Z_k \to \{\op{low},\op{mid},\op{high}\} \times \{\op{g},\op{nong}\}
$$
(where $\op{low}$, $\op{mid}$, $\op{high}$, $\op{g}$, and $\op{nong}$ are symbolic labels for elements in sets of cardinality three and two, respectively).
Define
$$
\op{elabel}(i)_2 = \begin{cases}
   \op{low}, &\text{if } \norm{\v_i}{\v_{i+1}} = 2,\\
   \op{high}, &\text{if } \norm{\v_i}{\v_{i+1}} = 2\hm,\\
   \op{mid}, &\text{otherwise}
\end{cases}
$$
and
$$
\op{elabel}(i)_2 = \begin{cases}
   \op{g}, &\text{if $i\in I$},\\
   \op{nong}, &\text{otherwise}.
\end{cases}
$$

For given functions $\op{vlabel}$ and $\op{elabel}$, 
let $X(\op{vlabel},\op{elabel})$ be the subset of $X$ with the given functions.
There are finitely many labeling functions $\op{vlabel}$ and $\op{elabel}$.
Thus, we may enumerate all cases.

By Lemma~\ref{lemma:min-irred}, every minimal fan is irreducible at every point in $V$.
The labels have been designed in such a way that every fan in a given set
$X(\op{vlabel},\op{elabel})$ is irreducible or no fan in the set is irreducible.
Each set that does not contain an irreducible fan is empty.  We may filter the
set of labeling functions, discarding those that do not contain an irreducible fan.

If $f:Z_k\to Y$ is any function, then define the shift $f[j]:Z_k\to Y$ by
$f[j](i) = f(i+j)$.  It is clear that $X(\op{vlabel},\op{elabel})$ is empty
if and only if $X(\op{vlabel}[j],\op{elabel}[j])$, because $(V,E,F,G,\v_0)$ lies
in the first set if and only if $(V,E,F,G,\rho^j\v_0)$ lies in the second set.  Thus,
we may reduce the list of labeling functions so that they are listed up to a shift.

A short computer calculation shows that every irreducible set of labels is equal
to one of the following up to a shift:

\begin{note}%XX minimal_fan.ml calculation to be supplied.
This list will be supplied in a later version.  The code for the calculation appears in a file \verb!minimal_fan.ml! in the flyspeck svn repository.
\end{note}

The elements of a given set $X(\op{vlabel},\op{elabel})$ can be parametrized
by the lengths 
$$\normo{\v_i} \mid \op{vlabel}(i)_1 = \op{mid},$$
$$\norm{\v_i}{\v_{i+1}} \mid \op{elabel}(i)_1 = \op{mid},$$
and $k_{min}-3$ diagonals.  In every case, the number of such parameters is
at most six.    Calculation~\ref{calc:irred} %cc:par
shows in each case that
$$
\tau(V,E,F) \ge d(r,s),
$$
for every $(V,E,F,G,\v_0)\in X(\op{vlabel},\op{elabel})$.  
This inequality shows that none of the elements $(V,E,F,G,\v_0)$ is minimal.
This completes the proof that $X$ is empty.
{\it Praise to the emptiness that blanks out existence.} %-- Rumi % The essential Rumi page 21.
\indy{Index}{Rumi}%
\end{proof}

\begin{corollary}\guid{JEJTVGB}\rating{ZZ}\label{lemma:empty-d}
$$
\tau(V,E,F) \ge d (r,s)
$$
for every special fan $(V,E,F,G)$, where $(r,s)$ are the parameters of $(V,E,F,G)$.
\end{corollary}

\begin{proof} 
The function $\p\mapsto \tau(V(\p),E(\p),F(\p),G(\p,I))$ is continuous on the space of fan data of shape $(k_{min},I)$

Assume for a contradiction that the conclusion is false.
Then $k_{min}>0$, because it is the cardinality of some nonempty set $V$.
There exists a sequence of fan data $\p_i$ of varying shapes $(k_{min},I_i)$ such that 
$\tau(V(\p_i),E(\p_i),F(\p_i))-d(r,s)$ tends to $\tau^d_{min}$.  By passing to a subsequence, we may assume without loss of generality that $I_i = I$ is independent of $i$.
By passing again to a subsequence in the compact 
metric space of fan data of shape $(k_{min},I)$, the sequence $\p_i$ converges to some fan datum $(\p,I)$.
It follows that $\tau^d_{min} = \tau(V,E,F)-d(r,s)$, where $(V,E,F,G)=(V(\p),E(\p),F(\p),G(\p,I))$.   
This is a minimal fan.
However, the set of minimal fans is empty.
\end{proof}


\section{Appendix: nonlinear inequalities}

\begin{calculation}\guid{2065952723}\rating{ZZ}\label{calc:Lexell}
%See Mathematica numerical calculation.
Let
$$
g(s;a,b,c,e_1,e_2,e_3) = \sum_{i=1}^3 \dih_i(2,2,2,a+s,b,c) e_i,
$$
where $\dih_i$ is given by Definition~\ref{def:tau}.
Let $\Delta = \Delta(4,4,4,a^2,b^2,c^2)$.
Let primes denote derivatives with respect to the variable $s$.
Assume that
$e_i\in\leftclosed1,1+\sol_0/\pi\rightclosed$,  that
$a,b,c\in\leftclosed2/\hm,4\rightclosed$.
Then
\begin{equation}\label{eqn:calc:Lexell}
\Delta (g'(0;a,b,c,e_1,e_2,e_3))^2 - 0.01\Delta^{3/2}g''(0;a,b,c,e_1,e_2,e_3) > 0.
\end{equation}
(The factors of $\Delta$ clear the denominator in (\ref{eqn:calc:Lexell}) to
simplify the inequality to be proved.)
\end{calculation}

\begin{calculation}\guid{2158872499}\rating{ZZ}\label{calc:2der}
%% checked in Mathematica NMaximize
Let $y_1,y_2\in \leftclosed 2,2\hm\rightclosed$.  
\begin{itemize}
\item 
Let $g(t) = \arc(y_1,y_2+t,2)$.  Then $g''(0) < 0$.
Explicitly,
$$
g''(0) = \dfrac{
-64 + 48y_1^2 - 12 y_1^4 + y_1^6 + 80 y_2^2 - 8 y_1^2 y_2^2 - 3 y_1^4 y_2^2
 - 12 y_2 ^4 + 3 y_1^2 y_2^4 - y_2^6
}{y_2^2 \sqrt{\ups(y_1^2,y_2^2,4)}^3}
$$
and the polynomial in the numerator takes negative values on the given domain.
\item
Let $g(t) = \arc(y_1+t,y_2-t,2)$.  Then $g''(0) < 0$.
Explicitly,
$$
g''(0) = \dfrac{\sqrt{\ups(y_1^2,y_2^2,4)} \left(
-4 y_1^2 + y_1^4 - 4y_1^3 y_2 - 4y_2^2 + 6 y_1^2 y_2^2 - 4 y_1 y_2^3 +y_2^4
\right)}{y_1^2 y_2^2 (2+y_1-y_2)^2 (2+y_2-y_1)^2}
$$
and the polynomial in the numerator takes negative values on the given domain.
\end{itemize}
\end{calculation}


\begin{calculation}\guid{5779862781}\rating{ZZ}\label{calc:cc:d2a}
Let $y_5,y_6\in \{ 2,2\hm\}$, $y_1,y_2,y_3\in \leftclosed 2,2\hm\rightclosed$, and $y_4\in \leftclosed 2,4\hm\rightclosed$.
Let $g(t) = \tau(y_1+t,y_2,y_3,y_4,y_5,y_6)$.
If $\Delta(y_1^2,y_2^2,\ldots,y_6^2)> 0$ and if
$
g'(0)=0,
$
then $g''(0)<0$.\footnote{The function $g(t)$ may fail to be differentiable at $t=0$ for parameters $y_1,\ldots,y_6$ for which $\Delta(y_1^2,\ldots,y_6^2)=0$.  Thus, it is necessary to work on the noncompact domain $\Delta>0$ for this inequality.}
%(The proof is an interval arithmetic calculation over a four-dimensional space~cc:d2a.  %%cc:d2a
%The calculation verifies  that the second derivative is negative whenever the derivative is zero.) 
\end{calculation} 


\begin{calculation}\guid{6645853705}\rating{ZZ}\label{calc:cc:d2b}
Let $y_5\in \{2,2\hm\}$,  $y_1,y_2,y_3\in \leftclosed 2,2\hm\rightclosed$, and $y_4\in \leftclosed 2,4\hm\rightclosed$.
Let $g(t) = \tau(y_1+t,y_2,y_3,y_4,y_5,y_6)$.
If $\Delta(y_1^2,y_2^2,\ldots,y_6^2)> 0$, if
$$
\arc(y_1,2\hm,2) + \arc(y_2,2\hm,2) \le y_6 \le \arc(y_1,2,2\hm)+\arc(y_2,2,2\hm)
$$ 
and if
$
g'(0)=0,
$
then $g''(0)<0$.
\end{calculation}

\begin{calculation}\guid{5606476569}\rating{ZZ}\label{calc:cc:qua}
Let $y_{12},y_{23},y_{34},y_2,y_3\in\{2,2\hm\}$.
Let $y_1,y_4\in \leftclosed 2,2\hm\rightclosed$.
Let $y_{14}\in\leftclosed 2\hm,y_1+y_4\rightclosed$.
Let $y_{13}\in\leftclosed 2\hm,y_1+y_3\rightclosed$.
Let $g(t) = \tau(y_1+t,y_2,y_3,y_4,y_5,y_6)$.
If 
$$
\Delta(y_1^2,y_2^2,y_3^2,y_{23}^2,y_{13}^2,y_{12}^2)> 0 \text{ and }
\Delta(y_1^2,y_4^2,y_3^2,y_{43}^2,y_{13}^2,y_{14}^2)> 0,
$$ 
if\footnote{If $\{\v_1,\ldots,\v_4\}$ is a set of vectors such
that $y_i = \normo{\v_i}$ and $y_{ij} = \norm{\v_i}{\v_j}$, then $\op{cross}(y_4,\ldots,y_{12}) = \norm{\v_2}{\v_4}$.}
$$
\op{cross}(y_4,y_1,y_3,y_{13},y_{34},y_{14},y_2,y_{23},y_{12})\ge y_{13},
$$
and if
$
g'(0)=0,
$
then $g''(0)<0$.
\end{calculation}

\begin{calculation}\guid{EFJSUSK}\rating{ZZ}\label{calc:irred} %cc:tau %cc:par
Let $(V,E,F,G)$ be an irreducible minimal fan with parameters $(r,s)$.  Then
$\tau(V,E,F) \ge d(r,s)$.
(A separate calculation
has been made for each of the cases in the list given above.)
%Interval arithmetic calculations~ %% cc:par partition cases for tau[r,s]. 
\end{calculation}
    %\lll
    % Kepler Conjecture.
% Thomas C. Hales
% Starting with Chapter on Tame Hypermaps


%% XX Notation: A vs. v for nodes.
%% XX Notation sigma' for aggregates, sigma'' for the full hypermap.
%% Allow the pentagon-triangle into the definition of tame graph.
%%%% Show there is at most one.  Let it be the seed.
%%




\label{sec:tame}


This chapter defines a class of hypermaps.  Hypermaps in this class
are said to be {\it tame}.  In the next chapter, we give a complete
classification of all tame hypermaps.  This classification of tame
hypermaps was carried out by computer.   This classification is a
major step of the proof of the Kepler Conjecture.

\section{Definition and Classification}


\begin{definition}
Faces of cardinality $3$ are called {\it triangles}, those of
cardinality $4$ are called {\it quadrilaterals}, and so forth. Let
$p_v$ be the number of triangles incident with a node $v$. A face of
cardinality at least $5$ is called an {\it exceptional\/} face.
 %
 \index{triangle}
 \index{exceptional}
 \index{quadrilateral}
 \index{exceptional!face}
 \index{pZ@$p_v$}
\end{definition}

\begin{definition}\label{definition:type}
A face of a hypermap is said to be exceptional if it has at least
five darts.  The {\it type\/} of a node is defined to be a triple of
non-negative integers $(p,q,r)$, where $p$ is the number of
triangles containing the node, $q$ is the number of quadrilaterals
containing it, and $r$ is the number of exceptional faces.
%
 \index{type (of a node)}
\end{definition}


\subsection{weight assignment}\label{sec:wtassign}

We call the constant $\op{tgt}=14.8$, which arises repeatedly in
this section, the {\it target}.  (This constant arises as an
approximation to $4\pi\zeta -8\approx 14.7947$, where $\zeta =
1/(2\arctan(\sqrt{2}/5)$.)
%
 \index{target}\index{tgt@$\op{tgt}=14.8$}
 \index{ZZdzeta@$\zeta= 1/(2\arctan(\sqrt{2}/5))$}

\begin{definition}
  Define $a:\ring{N}\to \ring{R}$ by
  $$a(n) = \begin{cases}
    14.8 &n=0,1,2,\\
    1.4 & n=3,\\
    1.5 & n=4,\\
    0 & \text{otherwise.}
  \end{cases}
  $$
\end{definition}

\begin{definition}
  Define $b:\ring{N}\times \ring{N}\to \ring{R}$ by $b(p,q)=14.8$,
  except for the values in the following table
  (with  $\op{tgt}=14.8$):
  {
  \def\tx{\op{tgt}}
  $$\begin{matrix}  &q=0&1&2&3&4\\
           p=0&\tx&\tx&\tx&7.135&10.649\\
           1&\tx&\tx&6.95&7.135&\tx\\
           2&\tx&8.5&4.756&12.981&\tx\\
           3&\tx&3.642&8.334&\tx&\tx\\
           4&4.139&3.781&\tx&\tx&\tx\\
           5&0.55&11.22&\tx&\tx&\tx\\
           6&6.339&\tx&\tx&\tx&\tx
   \end{matrix}
   $$
   }
\end{definition}

\begin{definition}
  Define $c:\ring{N}\to \ring{R}$ by
  $$c(n) = \begin{cases}
    1 & n=3,\\
    0 & n=4,\\
    -1.03 &n=5,\\
    -2.06 &n=6,\\
    -3.03 &\text{otherwise.}
    \end{cases}
    $$
\end{definition}

\begin{definition}
    Define $d:\ring{N}\to \ring{R}$ by
  $$d(n) = \begin{cases}
    0 & n=3, \\
    2.378 & n=4, \\
    4.896 & n=5, \\
    7.414 & n=6, \\
    9.932 & n=7, \\
    10.916 & n=8,\\
    \op{tgt}=14.8 & \text{otherwise}.
  \end{cases}
  $$
\end{definition}

\begin{definition}
A set $V$ of nodes in a hypermap is called a {\it separated\/} set
of nodes if the following four conditions hold.
%
 \index{separated set}
    \begin{enumerate}
      \item Every node in $V$ is incident with an exceptional face.
      \item No two
        nodes in $V$ are adjacent.  (That is, no edge is incident
        with two different nodes in $V$.)
      \item No quadrilateral in $V$ is incident with two different nodes
        in $V$.
      \item Each node in $V$ has cardinality 5.
    \end{enumerate}
\end{definition}

\begin{definition}
%
A {\it weight assignment\/} of a hypermap $H$ is a function $w$ on
the set of faces of $H$, taking values in the set of non-negative
real numbers. A weight assignment is {\it admissible} if the
following properties hold:
%
 \index{weight assignment}
 \index{admissible (weight assignment)}
\begin{enumerate}
  \item If the face $F$ has cardinality $n$, then
        $w(F) \ge d(n)$
  \item If a node $v$ has type $(p,q,0)$, then
        $$\sum_{F:\,v\cap F\ne\emptyset} w(F) \ge b(p,q).$$
        \label{admissible:b}
  \item Let $V$ be any set of nodes of type $(5,0,0)$, and let $A =\bigcup V$ be
        the set of darts in these nodes.
        If the cardinality of $V$ is $k\le 4$, then
        then
        $$\sum_{F:\,F\cap A\ne\emptyset} w(F) \ge 0.55 k.$$
  \item Let $V$ be any separated set of nodes, and let $A =\bigcup V$ be
        the set of darts in these nodes.
        Then
        $$\sum_{F:\,F\cap A\ne\emptyset} (w(F) -d(\card(F)))
            \ge \sum_{v\in V} a(p_v).$$
        \label{definition:admissible:excess}
\end{enumerate}
The sum $\sum_F w(F)$ is called the {\it total weight} of $w$.
\index{total weight}
\end{definition}





\subsection{hypermap properties}
\label{sec:graphproperty}

We say that a hypermap is {\it tame\/} if it satisfies the following
conditions.
%
 \index{tame}

\begin{enumerate}
    \label{definition:tame}
    %1
    \item The hypermap is plain, planar, and connected.
    \item The edge map $e$ has no fixed points.
    \item The two darts of each edge lie in different nodes.
    \item The set of edges meeting any two given nodes has cardinality at most $1$.
    \item There are at least $2$ faces.
    \item Every face meets every node in at most one
        dart.
    \item There are never two nodes of type $(4,0,0)$ that are
    adjacent to each other.
    \label{definition:tame:40}
    \item The cardinality of each face is at least $3$ and at most $8$.
    \label{definition:tame:length}

    \item If $L$ is a contour loop with $3$ face steps, and if there exists a node in
    the exterior of $L$, then $L$ is a face of the hypermap.
    \label{definition:tame:3-circuit}

    \item If $L$ is a contour loop  $4$ face steps, and there are at least two nodes
    in the exterior of $L$, then the interior of $L$ takes one of the forms
    illustrated in Figure
    \ref{fig:fourcircuit}.  [XX make this more precise.]
    \label{definition:tame:4-circuit}
    \begin{figure}[htb]
        \centering
        \myincludegraphics{\ps/tame4circuit.eps}
        \caption{Tame $4$-circuits}
        \label{fig:fourcircuit}
    \end{figure}

    \item The cardinality of every node is at least $2$ and at most
    $6$.
    \label{definition:tame:degree}

    \item If a node is incident with an exceptional face,
        then the cardinality of the node is at most $5$.
    \label{definition:tame:degreeE}

    \item $$\sum_F c(\card(F)) \ge 8,$$
    \label{definition:tame:score}


    \item There exists an admissible weight assignment
        of total weight less than the target, $\op{tgt}=14.8$.
    \label{definition:tame:squander}



\end{enumerate}
%
Property \ref{definition:tame:score} implies that the hypermap has
at least eight triangles.


\subsection{classification of tame hypermaps}
    \label{sec:proof-classification}

%\section{Statement of the Theorem}
\label{sec:classification}

A list of several thousand hypermaps appears at \cite{web}. The
following theorem is listed as one of the central claims in the
proof in Section~\ref{sec:logic}.

\begin{definition} The opposite of a hypermap $(D,e,n,f)$ is the
hypermap $(D,f n,n^{-1},f^{-1})$.
\end{definition}

\begin{lemma} If a hypermap has properties XXX, then so does its
opposite.
\end{lemma}

\begin{theorem}
\label{theorem:classification} Every tame hypermap is isomorphic to
a hypermap in this list, or is isomorphic to the opposite of a
hypermap in this list.
\end{theorem}

The results of this section are not needed except in the proof of
Theorem \ref{theorem:classification}.

\smallskip

Computers are used to generate a list of all hypermaps and to check
them against the archive of tame hypermaps.  The computer program is
based on the face-insertion construction of Lemma~XX.  There it is
proved that all sufficiently nice hypermaps can be generated by an
elementary face-insertion process.  Tame hypermaps satisfy all the
hypotheses of that lemma.





\section{Contravention is Tame}
    \label{sec:contraproof}

Let $(\Lambda,v_0)$ be a centered packing with
aggregated fan $P=(v_0,V,E)$.  Let  hypermap $H=(D,e,n,f)$
be the planar hypermap attached to $P$.
The hypermap $H$ is connected (Lemma~\ref{XX}).  Each of its
faces is simple (Lemma~\ref{XX}).

The connected components of $Y(v_0,V,E)$ are in bijection with
faces of $H$.  
The fan gives a azimuth angle function
$$
\op{azim} : D \to (0,2\pi).
$$
For each face of $H$, the corresponding component $R$
is eventually radial with solid
angle
  $$
  2\pi + \sum_{x\in F} (\op{azim}(x) -\pi).
  $$
We write $\sol(F)$ for the solid angle of the connected component
of $Y(v_0,V,E)$ associated with a face $F$ of the hypermap.
We have
    $$\sum_{F} \sol(F) = 4\pi.$$


For each face, there is a
real number $\tau(\Lambda,v_0,F)$ such that
$$
  \sum_{F : \text{face}}\tau(\Lambda,v_0,F) = \tau(\Lambda,v_0).
$$
We define a weight function $w(F)$ on the faces of the hypermap
by $w(F) = \sigma(\Lambda,v_0,F)/pt$.  In this way, we attach
a pair $(H,w)$ to each contravening centered packing $(\Lambda,v_0)$.


Let $D_f\subset D$ be the set of darts 
   $x = (v_0,v,u,w)$
such that the face of $x$ is exceptional and $|u-w|<\sqrt8$.
In this case, $\{v_0,v,u,w\}$ are the vertices of a flat quarter,
which is not necessarily in the $Q$-system.

\begin{theorem} \label{theorem:contravene}
Let $(\Lambda,v_0)$ be a contravening centered packing.  Let $(H,w)$ be
the hypermap and function on its faces attached to $(\Lambda,v_0)$ as above.
Then $H$ is a tame hypermap with admissible weight function $w$.
\end{theorem}

\subsection{hypermap is not empty}

%% Proof that the hypermap is not empty.



\begin{lemma}
\label{prop:nonempty} The construction of Section
\ref{sec:stargraph} associates a nonempty hypermap with at least
two faces to every centered packing $(\Lambda,v_0)$ with $\sigma(\Lambda,v_0)>0$.
In particular, the hypermap of a contravening centered packing is not empty.
\end{lemma}

\begin{proof}
First we show that centered packings with $\sigma(D)>0$ have
nonempty vertex sets $U$. (Recall that $U$ is the set of vertices
of distance at most $2t_0$ from the center).  The vertices of $U$
are used in \Chaps~\ref{sec:construction} and \ref{sec:vcells} to
create all of the structural features of the centered packing:
quasi-regular tetrahedra, quarters, and so forth. If $U$ is empty,
the $V$-cell is a solid containing the ball $B(t_0)$ of radius
$t_0$, and $\sigma(D)$ satisfies
    $$
    \begin{array}{lll}
    \sigma(\Lambda,v) &= \op{sovo}(v,VC(\Lambda,v),\lambda_{oct})\\
              &< \op{sovo}(v,B(v,t_0),\lambda_{oct})\\
              &= \sol(B(v,t_0))\phi(t,t,\lambda_{oct})\\
              &< 0.
    \end{array}
    $$
By hypothesis, $\sigma(D)>0$.  So $U$ is not empty.

XX The proof of making each standard region a simple polygon, assumes a
certain amount of nondegeneracy that isn't covered here.

Equation~\ref{eqn:sig-all} shows that the function $\sigma$ can be
expressed as a sum of terms $\sigma_R$ indexed by the standard
regions $R$. It is proved in Theorem~\ref{lemma:quad0} that
$\sigma_R\le0$, unless $R$ is a triangle. Thus, a centered packing
with positive $\sigma(D)$ must have at least one triangle. Its
complement contains a second standard region. Even after we form
aggregates of distinct standard regions to form the simplified
hypermap (Remarks \ref{remark:tri-pent} and \ref{remark:degree6}),
there certainly remain at least two faces.
\end{proof}


\subsection{first properties of hypermaps}
    \label{sec:startame}


Recall that we say that a node $v$ has {\it type\/} $(p,q,r)$ if
there are exactly $p+q+r$ faces that meet the node, of which exactly
$p$ triangles and $q$ quadrilateral faces (see
Definition~\ref{definition:type}).  We write $(p_v,q_v,r_v)$ for the
type of a node $v$.

\begin{lemma} The hypermap $H$ satisfies Conditions XX-XX of tameness.
Explicitly, it is a plain, planar, and connected. The edge map $e$
has no fixed points. There are at least two faces. Every face meets
every node in at most one dart.  There are never two nodes of type
$(4,0,0)$ that are adjacent to each other.  Every face has
cardinality at least $3$ and at most $8$.  If $L$ is a contour loop
with $3$ face steps, and if there exists a node in the exterior of
$L$, then $L$ is a face of the hypermap.
\end{lemma}


\begin{lemma}\dcg{Lemma~21.4}{223} 
Formally contravening hypermaps satisfy Property
\ref{definition:tame:degree} of tameness: The cardinality of every
node is at least $2$ and at most $6$.
\end{lemma}

\begin{proof}  There is no node of cardinality one by
Lemma~\ref{lemma:nodegen}.  There is no node of degree
greater than $6$ by Lemma~\ref{a:6}.
\end{proof}


\subsection{contravening hypermaps}


\begin{lemma} \label{lemma:0.55:bis} %proclaim{Lemma 5.3}
Let $(H,\azim,\flat,\sigma)$ be a formally contravening hypermap.
Let $v_1,\ldots, v_k$, for some $k\le 4$, be distinct nodes of type
$(5,0,0)$.  Let $F_1,\ldots, F_r$ be all the triangles around the
nodes $v_i$, for $i\le k$. Then
    $$
    \sum_{i=1}^r \tau(F_i)> 0.55k\,\pt,
    $$
and
    $$\sum_{i=1}^r \sigma(F_i) < r\,\pt - 0.48k\,\pt.$$
\end{lemma}


\begin{lemma}\label{lemma:no-2}
Let $(H,\azim,\flat,\sigma)$ be a formally contravening hypermap.
Suppose that $L$ is a contour loop with at most four face steps.
Suppose that there are at least two nodes in the exterior of $L$.
Then there at most one node interior to $L$.
\end{lemma}


\begin{lemma} \label{lemma:0.8638}
Let $(H,\azim,\flat,\sigma)$ be a formally contravening
hypermap. For every dart $x$,
    $$0.8638\le \azim(x).$$
For every dart $x$ whose face is not a triangle, we have
    $$1.153\le\azim(x).$$
\end{lemma}
 %
 \index{ZZZZ1.153@$1.153$}
 \index{ZZZZ0.8638@$0.8638$}

\begin{lemma} \label{lemma:excess-1:bis}
Let $(H,\azim,\flat,\sigma)$ be a formally contravening hypermap.
Let $F$ be an exceptional face.  Let $V$ be a set of nodes of $F$.
Let $x(F,v)$ be the dart of $F$ at a node $v$.  Let $(p_v,q_v,r_v)$
be the type of $v\in V$.   Let $a:\ring{N}\to\ring{R}$ be the
function of Section XX. Assume that $V$ has the following
properties:
    \begin{enumerate}
        \item The set $V$ is separated.
        \item If $v\in V$, then there are exactly five faces at
        $v$.
        \item If $v\in V$, then $\flat(x(F,v))$.
        \item If $v\in V$, then $p_v\ge 3$.  That is, at least
        three of the five faces at $v$ are triangles.
        \item If there are two exceptional regions $F$ and $F'$ at
        $v$, then
            $$\azim(x(F,v)) > 1.32 \Rightarrow \azim(x(F',v)) > 1.32.$$
        \item If $(p_v,q_v,r_v)=(3,1,0)$, then $\azim(x(F,v))\le 1.32$.
    \end{enumerate}
Let $A$ be the union of the singleton $\{F\}$, the set of all
triangles with a dart in some $v\in V$, and the set of all
quadrilaterals with a dart in some $v\in V$. Then
    $$\sum_{F\in A}\tau(F) > \sum_{v\in V} (p_v d(3) + q_v d(4) + a
    (p_v))\,\pt.$$
\end{lemma}


\begin{lemma}\label{lemma:nobad4}
Let $(H,\azim,\flat,\sigma)$ be a formally contravening hypermap.
Let $v$ be a node of type $(1,0,1)$ with precisely one triangle and
one pentagon, as show in Figure~\ref{fig:no4circuit:bis}. Let $L$ be
the perimeter contour loop with four face steps having the node $v$
in its interior.  At each of the four nodes $w$ visited by $L$, let
$\azim(w)$ be the sum of the terms $\azim(x)$, with the sum running
over the darts $x$ at the node visited by $L$.  Then
    $\azim(w) > 1.32$
for each of the four nodes $w$ visited by $L$.
\end{lemma}

\begin{lemma} Let $(H,\azim,\flat,\sigma)$ be a formally contravening
hypermap.  Let $v$ be a node of $H$ of type $(1,0,1)$, such that the
exceptional region is a pentagon.  Let $W$ be the set of four nodes
of the pentagon other than $v$.  If there are four triangles
$F_1,\ldots,F_4$ at some node $W$ that do not meet $v$, then
    $$\sum_{i=1}^p \tau(F_i) > a(4)\,\pt.$$
\end{lemma}

\begin{lemma}
Let $(H,\azim,\flat,\sigma)$ be a formally contravening hypermap.
Let $X$ be the set of nodes $v$ with the following properties.
    \begin{enumerate}
    \item The node has type $(5,0,1)$.
    \item The exceptional face at the node is pentagonal.
    \item That pentagonal face has no nodes of type $(1,0,1)$.
    \end{enumerate}
Then $\card(X)\ne 1$.
\end{lemma}


\begin{lemma}  Let $(H,\azim,\flat,\sigma)$ be a formally contravening
hypermap. Assume that $v$ is a node of $H$ whose type is
$(p,q,r)=(3,0,2)$ or $(4,0,1)$.  Assume that $\neg\flat(x)$ for
every dart of $v$.  Let $\tau(F_1),\ldots,\tau(F_p)$ be the
triangles at the node $v$.  Then
    $$
    \sum_{i=1}^p \tau(F_i) > a(p)\,\pt.
    $$
\end{lemma}




\subsection{linear programs} %subsection
\label{sec:2.2}  To continue with the proof that formally
contravening hypermaps are tame, we need to introduce some more
notation and methods.

\begin{lemma} \label{lemma:deg5}
Every formally contravening hypermap satisfies Property
\ref{definition:tame:degreeE} of tameness: If a node meets an
exceptional face, then the cardinality of the node is at most $5$.
\end{lemma}

\begin{proof} Every node of type $(5,0,1)$ meets a face that is a pentagon.
If there are two or more such nodes, then it must be that of Lemma
XX.  However, this has a node of type $(1,0,1)$, which has been
made an aggregate.  Thus, there is at most one node of type $(5,0,1)$.
This arrangement does not appear on a formally contravening hypermap
by Lemma~\ref{lemma:nobad4}.
\end{proof}

\subsection{possible four-circuits}

Every contour loop partitions the faces into the interior and
exterior.  Every contour loop partitions the nodes that do not meet
the loop into exterior and interior nodes.
%
 \index{interior node}

Lemma~\ref{lemma:no-2} asserts that either the interior or the
exterior has at most $1$ enclosed vertex.   When choosing which
aggregate is to be called the interior, we may make our choice so
that the interior has area at most $2\pi$, and hence contains at
most $1$ node. With this choice, we have the following lemma.

\begin{lemma}
Let $(H,\azim,\flat,\sigma)$ be a formally contravening hypermap. If
$L$ is a contour loop with $4$ face steps, and there are at least
two nodes in the exterior of $L$, then the interior of $L$ takes one
of the forms illustrated in Figure XX in Property
    \ref{definition:tame:4-circuit} of tameness.
\end{lemma}

\begin{proof}
By Lemma~XX, the interior of $L$ contains at most one node.

$H$ is a connected plane planar map.  We form a normal family of
contour loops ${\cal L}$ by taking the contour loop $L^{-1}$
reversing $L$ (XX explain) and all the faces in the interior of $L$.
(Check this is a normal family.)  The quotient $H' = H/{\cal L}$ is
a plane planar map.  There is a further quotient of $H'$ with normal
family $\{L,L^{-1}\}$, which is isomorphic to $P_4$ with the natural
flag coming from $H'$.  The niceness conditions of LemmaXX are
satisfied, so we can recover $H'$ from $P_4$ by a sequence of
face-insertions.  Since the interior of $L$ contains at most one
node, this gives restrictions on the partitions that can be used in
face-insertion.

If there are no enclosed vertices, then the only possibilities are
for it to be a single quadrilateral face or a pair of adjacent
triangles.

Assume there is one enclosed vertex $v$.  If $v$ is connected to $3$
or $4$ nodes of the quadrilateral, then that possibility is listed
as part of the conclusion.

If $v$ is connected to $2$ opposite nodes in the $4$-cycle, then the
node $v$ has type $(0,2,0)$ and the bounds of
Lemma~\ref{lemma:pq:bis} show that the hypermap cannot be formally
contravening.

If $v$ is connected to $2$ adjacent nodes in the $4$-cycle, then we
appeal to Lemma~\ref{lemma:nobad4} to conclude that the hypermap
does not contravene.

If $v$ is connected to $0$ or $1$ nodes, then we appeal to
Lemma~\ref{lemma:enclosed:bis}.  This completes the proof.
\end{proof}

\subsection{weight assignments}
    \label{sec:weight}

The purpose of this section is to prove the existence of a good
admissible weight assignment for formally contravening hypermaps.
This will complete the proof that all formally contravening
hypermaps are tame.

\begin{theorem}  Every formally contravening hypermap has an admissible
weight assignment of total weight less than $\op{tgt}=14.8$.
\end{theorem}

Given a formally contravening hypermap $(H,\azim,\flat,\sigma)$, we
define a weight assignment $w$ by
    $$F \mapsto w(F) = \tau(F)/\pt.$$
Since the hypermap is formally contravening,
    $$
    \begin{array}{lll}
    \sum_F w(F) &= \sum_F \tau(F)/\pt \\
            &= \tau^*(H)/\pt\,\le\,\squander/\pt \\
        &< \op{tgt}=14.8.
    \end{array}
    $$
The challenge of the theorem will be to prove that $w$, when
defined by this formula, is admissible.

\subsection{admissibility}
\label{sec:admissibility}

The next three lemmas establish that this definition of $w(F)$ for
formally contravening hypermaps satisfies the first three defining
properties of an admissible weight assignment.

\begin{lemma}  Let $F$ be a face of cardinality $n$ in a formally contravening hypermap.
Define $w(F)$ as above. Then
        $w(F) \ge d(n)$.
\end{lemma}

\begin{proof} This is Lemma~\ref{proposition:wttau}.
\end{proof}

\begin{lemma} Let $v$ be a node of type $(p,q,0)$ in a
formally contravening hypermap.  Define $w(F)$ as above. Then
        $$\sum_{v\in F} w(F) \ge b(p,q).$$
\end{lemma}


\begin{proof} This is Lemma~\ref{lemma:pq:bis}.
\end{proof}

\begin{lemma} Let $V$ be any set of nodes of type $(5,0,0)$ in a
formally contravening hypermap.  Define $w(F)$ as above.
        If the cardinality of $V$ is $k\le 4$,
        then
        $$\sum_{V\cap F\ne\emptyset} w(F) \ge 0.55 k.$$
\end{lemma}

\begin{proof} This is Lemma~\ref{lemma:0.55:bis}.
\end{proof}

The following theorem establishes the final property that $w(F)$
must satisfy to make it admissible.  {\it Separated sets\/} are
defined in Section~\ref{sec:wtassign}.

\begin{theorem}
        \label{proposition:excess}
        Let $V$ be any separated set of nodes in a formally contravening hypermap.
        Define $w(F)$ as above.
        Then
        $$\sum_{V\cap F\ne\emptyset} (w(F) -d(\card(F)))
            \ge \sum_{v\in V} a(p_v),$$
        where $p_v$ denotes the number of triangles containing
        the node $v$.
\end{theorem}

The proof will occupy the rest of this \chap. Since the cardinality
of each node is five, and there is at least one face that is not a
triangle at the node, the only constants $p_v$ that arise are
    $$p_v \in\{0,\ldots,4\}$$
We will prove that in a formally contravening hypermap that the
Properties (1) and (4) of a separated set are incompatible with
$p_v\le 2$.  This will allows us to assume that
$$p_v\in\{3,4\},$$ for all $v\in V$.  These cases will be treated in
Section~\ref{sec:tri34}.

%\section{Proof that $p_v>2$}
%%subsection
%\label{sec:2.4} \label{sec:tri2}
%
%In this subsection $(H,\azim,\flat,\sigma)$ is a formally
%contravening hypermap.  Let $V$ be a separated set of nodes in $H$.
%
%\begin{lemma}  Under these conditions, for every $v\in V$,
%$p_v>1$.
%\end{lemma}
%
%\begin{proof}
%If there are $p$ triangles, $q$ quadrilaterals, and $r$ other
%faces, then
%    $$
%    \begin{array}{lll}
%    \tau^*(H) &\ge\sum_{v\in F}\tau(F)\\
%        &\ge r\, t_5 + \tauLP(p,q,2\pi-r(1.153)).
%    \end{array}
%    $$ If there is a node $w$ that is
%not on any of the faces containing $v$, then the sum of $\tau(F)$
%over the faces containing $w$ yield an additional $0.55\,\pt$ by
%Lemma~\ref{lemma:0.55:bis}. We calculate these constants for each
%$(p,q,r)$ and find that the bound is always greater than
%$\squander$. This implies that $H$ cannot be formally contravening.
%$$\begin{array}{llll}
%    (p,q,r)&\hbox{\it lower bound }&\hbox{\it justification}\\
%    &\\
%    (0,5,0)&22.27\,\pt&\text{Lemma~\ref{lemma:pq:bis}}\\
%    (0,q,r\ge1)& t_5+4 t_4\approx 14.41\,\pt +0.55\,\pt& \\
%    (1,4,0) &17.62\,\pt &\text{Lemma~\ref{lemma:pq:bis}}\\
%    (1,3,1) &t_5 + 12.58\,\pt &(\tauLP)\\
%    (1,2,2) &2t_5 + 7.53\,\pt &(\tauLP)\\
%    (1,q,r\ge3)& 3 t_5 + t_4& \\
%\end{array}
%$$
%\end{proof}
%
%
%\begin{lemma} Under these same conditions, for every $v\in V$,
%$p_v>2$.
%\end{lemma}
%
%\begin{proof}
%Assume that $p_v=2$.  We will show that this implies that $H$ does
%not contravene.  Let $r=r_v$ be the number of exceptional faces at
%$v$. We have $r+p_v\le5$.  We consider various cases, according to
%the value of $r$.
%
%The constants $0.55\,\pt$ and $0.48\,\pt$ used throughout the
%proof come from Lemma~\ref{lemma:0.55:bis}. The constants $t_n$
%comes from Lemma~\ref{lemma:sn-tn}.
%
%($(p,q,r)=(2,0,3)$): First, assume that there are three exceptional
%faces around node $v$. They must all be pentagons
%($2t_5+t_6>\squander$). The aggregate of the five faces is an
%$m$-gon (some $m\le11$).  If there is a node not on this aggregate,
%use $3t_5+0.55\,\pt>\squander$. So there are at most nine triangles
%away from the aggregate, and the Euler relation gives
%    $$
%    \sigma^*(H) \le 9\,\pt + (3 s_5+2\,\pt) < 8\,\pt.
%    $$
%
%($(p,q,r)=(2,1,2)$): The argument if there is a quad, pentagon, and
%hexagon is the same $(t_4+t_6=2t_5,s_4+s_6=2s_5)$.
%
%Assume next that there are two pentagons and a quadrilateral around
%the node. The contour loop around the two pentagons, quadrilateral,
%and two triangles is has $m$ face steps (some $m\le10$). There must
%be a node exterior to this loop, for otherwise the Euler relation
%gives
%    $$
%    \sigma^*(H) \le 8\,\pt+(2s_5+2\,\pt)<8\,\pt.
%    $$
%
%The azimuth angle of one of the pentagons is at most $1.32$.  For
%otherwise, $\tauLP(2,1,2\pi-2(1.32))+2t_5+0.55\,\pt>\squander$.
%
%Lemma~\ref{lemma:1.47} shows that any pentagon $F$ with an azimuth
%angle less than $1.32$ yields $\tau(F)\ge t_5+ (1.47\,\pt)$. If both
%pentagons have an azimuth angle $<1.32$ the lemma follows easily
%from this calculation:
%    $2(t_5+1.47\,\pt)\,\pt+\tauLP(2,1,2\pi-2(1.153))+0.55\,\pt>\squander$.
%If there is one pentagon with angle $>1.32$, we then have
%    $t_5+(1.47\,\pt)+\tauLP(2,1,2\pi-1.153-1.32)+t_5+0.55\,\pt>\squander$.
%
%
%($(p,q,r)=(2,2,1)$): Assume finally that there is one exceptional
%face at the node. If it is a hexagon (or more), we are done
%$t_6+\tauLP(2,2,2\pi-1.153)>\squander$. Assume it is a pentagon. The
%contour loop around the five faces at the node has $m$ face steps
%(some $m\le9$). If there are no more than $9$ triangles exterior to
%the contour loop, then $\sigma^*(H)$ is at most
%$(9-2(0.48))\,\pt+s_5+\sLP(2,2,2\pi-1.153)<8\,\pt$
%(Lemma~\ref{lemma:0.55:bis}). So by the Euler relation, we may
%assume that there are at least three nodes exterior to the contour
%loop.
%
%If the azimuth angle of the dart on the pentagon is greater than
%$1.32$, we have
%  $$\tau^*(H)\ge\tauLP(2,2,2\pi-1.32) +3(0.55)\,\pt +t_5 > \squander;$$
%and if it is less than $1.32$, we have by Lemma~\ref{lemma:1.47}
%    $$
%    \begin{array}{lll}
%        \tau^*(H)\ge\tauLP(2,2,2\pi-1.153)&+3(0.55)\pt+1.47\,\pt+t_5 \\
%            &> \squander.
%    \end{array}
%    $$
%\end{proof}
%

\subsection{separated sets} %when p=3,4 %subsection
\label{sec:2.7} \label{sec:tri34}

In this subsection $(H,\azim,\flat,\sigma)$ is a formally
contravening hypermap.  Let $V$ be a separated set of nodes.  We
assume that there are three or four triangles meeting $v$, for every
$v\in V$.

To prove the Inequality \ref{definition:admissible:excess} in the
definition of admissible weight assignments, we will rely on the
following reductions. Define an equivalence relation on exceptional
faces by $F\sim F'$ if $F=F'$ or if there is a sequence
$F=F_0,\ldots, F_r=F'$ of exceptional faces such that consecutive
faces share a node of type $(3,0,2)$. Let ${\cal F}$ be an
equivalence class of faces.

%% XX GIVE FIGURE HERE with lots of exceptionals.

\begin{lemma} Let $V$ be a separated set of nodes.  For every
equivalence class of exceptional faces $\cal F$, let $V({\cal F})$
be the subset of $V$ whose nodes meet a face in ${\cal F}$. Suppose
that for every equivalence class $\cal F$, the Inequality
\ref{definition:admissible:excess} (in the definition of admissible
weight assignments) holds for $V({\cal F})$. Then the Inequality
holds for $V$.
\end{lemma}

\begin{proof}
By construction, each node in $V$ lies in some $F$, for an
exceptional face.  Moreover, the separating property of $V$ insures
that the triangles and quadrilaterals in the inequality are
associated with a well-defined  ${\cal F}$. Thus, the inequality for
$V$ is a sum of the inequalities for each $V({\cal F})$.
\end{proof}


\begin{lemma}
\label{lemma:split}
 Let $v$ be a node of type $(p,q,r)$ in a separated set $V$.  Suppose that
for some $p'\le p$ and $q'\le q$, we have a lower bound of the form
    $$( p' d(3) + q' d(4) + a(p))\,\pt$$
for what is squandered by $p'$ triangles and $q'$ quadrilaterals at 
a vertex $v$.  Suppose further that the
Inequality~\ref{definition:admissible:excess} (in REFXX) holds for
the separated set $V' = V\setminus \{v\}$. Then the inequality holds
for $V$.
\end{lemma}

\begin{proof}  Let $F_1,\ldots,F_m$, $m={p'+q'}$, be faces corresponding
to the triangles and quadrilaterals in the lemma.  The hypotheses
of the lemma imply that
    $$\sum_{1}^{m} (w(F_i) - d(\card(F_i))) \ge a(p).$$
Clearly, the Inequality for $V$ is the sum of this inequality, the
inequality for $V'$, and $w(F)- d(F)\ge0$.
\end{proof}


\begin{lemma}  Property \ref{definition:admissible:excess}  of
admissibility holds.  That is, let $V$ be any separated set of
nodes. Then
        $$\sum_{F:\,V\cap F\ne\emptyset} (w(F) -d(\card(F)))
            \ge \sum_{v\in V} a(p_v).$$
\end{lemma}

\begin{proof}  Let $V$ be a separated set of nodes.
The results of Section~\ref{sec:tri2} reduce the lemma to the case
where $p_v\in\{3,4\}$ for every node $v\in V$.


One case is easy to deal with.  A node of type $(3,1,1)$ such
that the dart on the exceptional face is at least $1.32$ has a
bound of type Lemma~\ref{lemma:split} by Lemma~\ref{a:311}.
For the rest of the proof, assume that the azimuth angle on
the exceptional face $F$ is less than $1.32$ at nodes of type
$(p,q,r)=(3,1,1)$. This implies in particular by
Lemma~\ref{lemma:1.32:bis} that the dart $x(F,v)$ is flat.

Another case is easy to deal with.  Lemma~\ref{a:no-ef}
shows that a node with no exceptional flat darts also
falls into the situation of Lemma~\ref{lemma:split}.
Thus, we may assume that at each $v\in V$, there is an exceptional
flat dart.

Pick a function $f$ from the set $V$ to the set of exceptional
standard regions as follows. Let $X$ be the set of exceptional faces
$F$ at $v$ for which $x(F,v)$ is flat.  From $X$, let $f(v)$ be the
one with smallest $\azim(x(F,v))$.  We see by construction and
Lemma~\ref{lemma:1.32} that $F = f(v)$ has the properties:
    \begin{itemize}
        \item $\flat(x(F,v))$
        \item $\azim(x(F,v)) > 1.32\ \Rightarrow\ \azim(x(F',v)) >
        1.32$, for any exceptional face $F'$ meeting $v$.
    \end{itemize}

For each exceptional face $F$, let
    $$V_F = \{ v\in V : f(v) = F\}.$$  This set may be empty for
some $F$.  Let $A_F$ be the union of $\{F\}$, and the set of
triangles and quadrilaterals with a dart in some $v\in V_F$.  If
$V_F$ is empty, then $A_F =\{F\}$.  The indexing set $A$ of Property
\ref{definition:admissible:excess} of admissibility is the disjoint
union of $A_F$.  The set $V$ is the disjoint union of the $V_F$ (or
at least of the nonempty ones). So the result follows from
Lemma~\ref{XX} for all the faces.
\end{proof}



The proof that formally contravening hypermaps are tame is complete.

\subsection{more about tame hypermaps}
%% CUT FROM TAME GOOD STUFF.

We have seen that a system of points and arcs on the unit sphere
can be associated with a centered packing $D$.  The points are the
radial projections of the nodes of $U(D)$ (those at distance at
most $2t_0=2.51$ from the origin).  The arcs are the radial
projections of edges between $v,w\in U(D)$, where $|v-w|\le2t_0$.
If we consider this collection of arcs combinatorially as a
hypermap, then it is not always true that these arcs form a
hypermap in the restrictive sense of
\Chap~\ref{sec:def-and-class}.

The purpose of this section is to show that if the original
centered packing contravenes, then minor modifications can be made
to the system of arcs hypermap so that the resulting combinatorial
hypermap has the structure of a hypermap in the sense of
\Chap~\ref{sec:def-and-class}. These hypermaps are called
contravening hypermaps.

A natural number $n(R)$ is associated with each standard region. If
the boundary of that region is a simple polygon, then $n(R)$ is the
number of sides.   If the boundary consists of $k$ disjoint simple
polygons, with $n_1,\ldots,n_k$ sides then
    $$n(R) = n_1+\cdots+n_k + 2(k-1).$$


\begin{lemma}\label{lemma:enclosed:bis} % {Lemma 2.2}
A quadrilateral region does not enclose any vertices of height at
most $2t_0$.
\end{lemma}

%% Summation convention:

%If $F$ is a face of $H$, let
%    $$\sigma_F(D) = \sum \sigma(F),$$
%where the sum runs over the set of standard regions associated with
%$F$.  This sum reduces to a single term unless $F$ is an aggregate
%in the sense of Remarks~\ref{remark:degree6} and
%\ref{remark:tri-pent}.


%% Here is stuff for after the definition of formally contravening hypermap.

%\begin{assumption}  $H$ is a planar hypermap.  $e$ is an
%involution that acts without fixed points.  Every face meets every
%node in at most one dart.  Every face has cardinality at least $3$
%and at most $8$.
%\end{assumption}


%\chapter{The Aggregate Cases}
%    \label{sec:aggregate}

\subsection{weight assignments for aggregates}

\begin{lemma} The bound $tri(v)>2$ holds if $v$ is a node
of an aggregate face.
\end{lemma}

\begin{proof}
The exceptional region enters into the preceding two proofs in a
purely formal way.  Pentagons enter through the bounds
    $$t_5,\ s_5,\ 1.47\,\pt$$
and angles $1.153$, $1.32$.  Hexagons enter through the bounds
    $$t_6,\ s_6$$
and so forth.  These bounds hold for the aggregate faces.  Hence the
proofs hold for aggregates as well.
\end{proof}

\begin{lemma}
Consider a separated set of nodes $V$ on an aggregated face $F$ as
in Remark \ref{remark:tri-pent}.  The Inequality
\ref{definition:admissible:excess} holds (in the definition of
admissible weight assignments):
    $$\sum_{V\cap F\ne\emptyset} (w(F) -d(\card(F)))
            \ge \sum_{v\in V} a(tri(v)).$$
\end{lemma}

\begin{proof}
We may assume that $tri(v)\in\{3,4\}$.

First consider the aggregate of Remark \ref{remark:tri-pent} of a
triangle and eight-sided region, with pentagonal hull $F$. There
is no other exceptional region in a contravening centered packing
with this aggregate:
    $$t_8 + t_5 > \squander.$$
A separated set of nodes $V$ on $F$ has cardinality at most $2$.
This gives the desired bound $$t_8 > t_5 + 2 (1.5)\,\pt.$$

Next, consider the aggregate of a hexagonal hull with an enclosed
node.  Again, there is no other exceptional face. If there are at
most $k\le 2$ nodes in a separated set, then the result follows from
    $$t_8 > t_6 + k (1.5)\,\pt.$$
There are at most three nodes in $V$ on a hexagon, by the
non-adjacency conditions defining $V$. A node $v$ can be removed
from $V$ if it is not the central node of a flat quarter (Lemma
\ref{lemma:split} and Inequalities~\ref{eqn:tau1.32} and
Lemma~\ref{a:no-ef}). If there is an enclosed node $w$, it is
impossible for there to be three nonadjacent nodes, each the central
node of a flat quarter.  In fact, by Lemma~\ref{tarski:node},
any enclosed node must have height greater than $2t_0$.



Finally consider the aggregate of a pentagonal hull with an enclosed
node.  There are at most $k\le2$ nodes in a separated set in $F$.
There is no other exceptional region:
    $$t_7 + t_5 > \squander.$$
The result follows from
    $$t_7 > t_5 + 2(1.5)\,\pt.$$
\end{proof}

\begin{lemma}
Consider a separated set of nodes $V$ on an aggregate face of a
contravening hypermap as in Remark~\ref{remark:degree6}.  The
Inequality~\ref{definition:admissible:excess} holds in the
definition of admissible weight assignments.
\end{lemma}

\begin{proof}
There is at most one exceptional face in the hypermap:
    $$t_8 + t_5 > \squander.$$
Assume first that aggregate face is an octagon (Figure
\ref{fig:degree6}). At each of the nodes of the face that lies on a
triangular standard region in the aggregate, we can remove the node
from $V$ using Lemma \ref {lemma:split} and the estimate
    $$\tauLP(4,0,2\pi-2 (0.8638)) > 1.5\,\pt.$$
This leaves at most one node in $V$, and it lies on a node of $F$
which is ``not aggregated,'' so that there are five standard
regions of the associated centered packing at that node, and one
of those regions is pentagonal.  The value $a(4)=1.5\,\pt$ can be
estimated at this node in the same way it is done for a
non-aggregated case in Section~\ref{sec:tri34}.

Now consider the case of an aggregate face that is a hexagon (Figure
\ref{fig:degree6}).  The argument is the same: we reduce to $V$
containing a single node, and argue that this node can be treated as
in Section~\ref{sec:tri34}.  (Alternatively, use the fact that the
pentagon-triangle combination in this aggregate has been eliminated
by Lemma~\ref{lemma:nobad4}.)
\end{proof}


%% STUFF ABOUT CENTRAL VERTICES, QUARTERS AND SO FORTH.

Recall that the central vertex of a flat quarter is defined to be
the one that does not lie on the triangle formed by the origin and
the diagonal.
%
 \index{central}

%%

XX?  We will say that there is a flat quarter centered at $v$, if
the corner $v'$ over $v$ is the central node of a flat quarter and
that flat quarter lies in the cone over an exceptional region.

%%


%% Table simplified...  The entry (7,0) with 14.76 is relevant here.
%% It needs to be treated.

Define constants $\tlp(p,q)/\pt$ by Table~\ref{eqn:old5.1:bis}. The
entries marked with an asterisk will not be needed.
%
 \index{type (of a node)}
 \index{ZZtauLP@$\tlp(p,q)$}

\begin{equation}
\vbox{\offinterlineskip \hrule
\halign{&\vrule#&\strut\ \hfil#\hfil\ \cr   % "\ " was quad
height 7pt&\omit&&\omit&&\omit&&\omit&&\omit&&\omit&&\omit&\cr
&\hfil $\tlp(p,q)/\pt$\hfil
        &&\hfil $q=0$\hfil
        &&\hfil1\hfil
        &&\hfil2\hfil
        &&\hfil3\hfil
        &&\hfil4\hfil
        &&\hfil5\hfil&
\cr height 7pt&\omit&&\omit&&\omit&&\omit&&\omit&&\omit&&\omit&\cr
\noalign{\hrule}
height7pt&\omit&&\omit&&\omit&&\omit&&\omit&&\omit&&\omit&\cr
&$p=0$&& *&& *&& 15.18&& 7.135&& 10.6497&& 22.27&\cr &1&&    *&& *&&
6.95&& 7.135&&17.62  && 32.3&\cr &2&&    *&&
8.5&&4.756&&12.9814&&*&&*&\cr &3&& *&& 3.6426&&8.334&&20.9&&*&&*&\cr
&4&&4.1396&&3.7812&&16.11&&*&&*&&*&\cr
&5&&0.55&&11.22&&*&&*&&*&&*&\cr &6&&6.339&&*&&*&&*&&*&&*&\cr
&7&&14.76&&*&&*&&*&&*&&*&\cr
height7pt&\omit&&\omit&&\omit&&\omit&&\omit&&\omit&&\omit&\cr}
\hrule }
    %oldtag 5.1
    \label{eqn:old5.1:bis}
\end{equation}


%%  (1,0,1)

\subsection{a non-contravening four-circuit}
\label{sec:impossible-circuit}

This subsection rules out the existence of a particular four-circuit
on a contravening hypermap.  The interior of the circuit consists of
two faces: a triangle and a pentagon.  The circuit and its interior
node are show in Figure~\ref{fig:no4circuit:bis} with nodes marked
$p_1,\ldots,p_5$. The node $p_1$ is the interior node, the triangle
is $(p_1,p_2,p_5)$ and the pentagon is $(p_1,\ldots,p_5)$.


Let $v_1,\ldots,v_4,v_5$ be the corresponding vertices of $U(D)$.
XX?.

The diagonals $\{v_5,v_3\}$ and $\{v_2,v_4\}$ have length at least
$2\sqrt2$ by Lemma~\ref{tarski:2t0-doesnt-pass-through}.  If an
azimuth angle of the  quadrilateral is less than $1.32$, then by
Lemma~\ref{lemma:1.32:bis},  $|v_1-v_3|\le\sqrt{8}$.  Thus, we
assume in the following lemma, that all azimuth angles of the
quadrilateral aggregate are at least $1.32$.

%%

\begin{remark}
We have now fully justified the claim made in
Remark~\ref{remark:degree6}: there is at most one node on six
standard regions, and it is part of an aggregate in such a way that
it does not appear as the node of $H$.
\end{remark}

%%%%%%%%%%%%%%%%%%%%%%

%% AGGREGATE STUFF IN THE PROOF OF \ref{definition:tame:score}
%% Property of Tameness.

We consider three cases for Inequality \ref{eqn:sigma}. In the
first case, assume that the face $F$ corresponds to exactly one
standard region in the centered packing.  XX? In this case,
Inequality \ref{eqn:sigma} follows directly from the bounds of
Lemma~\ref{lemma:sn-tn}:
    $$\sigma(F)\le s_n \le c(n)\,\pt.$$

In the second case, assume we are in the context of a pentagon $F$
formed in Remark~\ref{remark:tri-pent}.  Then, again by
Theorem~\ref{lemma:sn-tn}, we have
$$\sigma(F) \le s_3+s_8\le (c(3)+c(8))\,\pt \le c(5)\,\pt.$$
(Just examine the constants $c(k)$.)

In the third case, we consider the situation of Remark
\ref{remark:degree6}.  The six faces give
$$\sigma(F)\le s_5+\sLP(5,0,2\pi-1.153)< c(8)\,\pt.$$
The constant $1.153$ comes from Lemma~\ref{lemma:0.8638}.


%%%%%%%%%%%%%%%%%%%%%%%%%%%%%

    %\lll
    %% ------------------------------------------------------------ 
% Author:
% Thomas C. Hales 
% Format: LaTeX Book Chapter: Dense Sphere Packings
% ------------------------------------------------------------

\chapter{Further Results}\label{sec:further}


The same methods that have been used to prove the Kepler conjecture
can be used to prove some other longstanding conjectures in discrete
geometry.  This chapter sketches proofs of the strong dodecahedral
conjecture and Fejes T\'oth's full contact conjecture.

Earlier chapters are written in a formal blueprint style.
Complete proofs are provided, even for statements that might be
viewed as geometrically obvious.  In this chapter, we relax our
standards of proof just a bit.  What we write is still a proof by
traditional mathematical standards, but not as detailed as earlier
chapters.

\section{Strong Dodecahedral Theorem}

Bezdek has conjectured that the Voronoi cell of smallest surface is
the regular dodecahedron with inradius $1$.  This is known as the
\newterm{strong dodecahedral conjecture}~\cite{Bezdek00}, \cite{Bezdek05}.  
%
\indy{Index}{Bezdek, K.}%
\indy{Index}{Voronoi cell}%


\begin{theorem}[strong dodecahedral conjecture]\guid{HKJSPQG}
  The surface area of a Voronoi cell in a packing is at least the
  surface area of the regular dodecahedron with inradius $1$.
\end{theorem}
\indy{Index}{dodecahedral conjecture}
\indy{Index}{dodecahedral conjecture!strong}

This section sketches a proof of this theorem.  We
begin with some simple observations.

\begin{remark}
  If a packing is  saturated, then the surface area of a Voronoi
  cell is finite.  \claim{We may assume without loss of 
    generality that the packing is saturated.}
  Indeed, consider a new facet $F$ that is created on a Voronoi cell
  by the addition of a new point to the packing $V$.  Let $X$ be the
  polygonal boundary of the new facet.  The area minimizing surface
  that has $X$ as a boundary is the facet $F$.  Thus the new facet
  replaces a surface of larger area with a surface of smaller area.
  That is, by saturating a packing, the surface area of a Voronoi cell
  can only decrease.
\end{remark}

\begin{remark}
As in the proof of the Kepler conjecture, the truncation of a Voronoi cell is easier
to study than the Voronoi cell itself.  In order to obtain sharp bounds, the truncation
must have no effect on the optimal Voronoi cell.  This constraint forces the
 truncation parameter to be at least
the circumradius $\sqrt{3}\tan{\pi/5}\approx 1.258$ of the regular dodecahedron.   The truncation parameter $\sqrt{2}$
that we use in the proof of the Kepler conjecture satisfies this constraint and is
therefore  well-suited for the strong dodecahedral
conjecture.
\end{remark}

\begin{lemma}[]\guid{JXVEXYV}  % X->Y
  The surface area of the Voronoi cell $\Omega(V, \u_0)$ is at least
  that of $\Omega(V, \u_0)\cap B$, where $B$ is the ball of radius
  $r_0$ centered at $ \u_0$.  
\end{lemma}
\indy{Index}{surface area}%
\indy{Index}{Dodecahedral Conjecture}%

\begin{proof} The surface element for a parameterized surface
  $r(\theta,\phi)$ in spherical coordinates is
\[
  % ff = {r[theta, phi] Cos[theta]Sin[phi], r[theta, phi]
  %   Sin[theta]Sin[phi], r[theta, phi] Cos[phi]}; n = Cross[D[ff,
  % theta], D[ff, phi]]; n.n // Simplify
%
  r \sqrt{r_\theta^2 + (r^2 + r_\phi^2)\sin^2\phi } \,\,d\theta\,d\phi,
\]
which is at least the surface element $r_0^2 \sin\phi\, d\theta\,d\phi$
of a sphere of radius $r_0$, provided
$r(\theta,\phi)\ge r_0$.   Hence, projection of a surface outside
sphere onto the sphere is area decreasing.
\end{proof}


Fejes T\'oth's classical dodecahedral conjecture is the corresponding conjecture
about volumes rather than surface areas, asserting that  the
Voronoi cell of smallest volume is the regular dodecahedron of
inradius $1$.  \indy{Index}{fejestoth@Fejes T\'oth, L.}%
\indy{Index}{Dodecahedral Conjecture}%
The strong dodecahedral conjecture yields the dodecahedral conjecture
as a corollary.

\begin{lemma}[]\guid{QRBKJAW}
  If the surface area of a Voronoi cell is at least the surface area
  of a regular dodecahedron with inradius $1$, then its volume is also
   at least that  of a regular dodecahedron.
\end{lemma}

\begin{proof} Let $A_1,\ldots,A_n$ be the areas of the facets of a
  Voronoi cell.  Let $h_1,\ldots,h_n$ be the distances from the affine
  hulls of the facets to the center of the Voronoi cell.  Then $h_i\ge
  1$.  Assume that $\sum A_i \ge A_D$, where $A_D$ is the surface area
  of a regular dodecahedron.  Then its volume is
\[
\op{vol} = \sum A_i h_i/3 \ge \sum A_i/3 \ge A_D/3 = \op{vol}_D,
\]
where $\op{vol}_D$ is the volume of the regular dodecahedron.
\end{proof}
\indy{Index}{regular dodecahedron!volume}%
\indy{Notation}{A@$A$ (Voronoi cell face area)}%
\indy{Notation}{V@$\op{vol}_D$ (volume of dodecahedron)}%


\subsection{$D$-cells}



The notation follows Section~\ref{sec:rogers}.  Let $V$ be a saturated
packing. If $\bu =[\u_0;\u_1;\u_2;\u_3]\in \bV(3)$, then let $b(\bu) = h([
\u_0; \u_1; \u_2])$.  
Let $\Omega( V,\u_0)$ be a Voronoi cell with  Rogers's partition 
\[
\Omega(V, \u_0) = \bigcup \ \{ R( \bu) \mid { \bu\in  \bV(3), \trunc{\bu}{0}= [\u_0] }\}.
\]
\indy{Notation}{ZZZomega@$\Omega$ (Voronoi cell)}%
\indy{Index}{Rogers's partition}%

\begin{definition}[$D_k$-cell]
  We recall that $B$ is the ball of radius $\sqrt2$ centered at
  $\u_0$.  We define $D_k$-cells for $k=1,2,3,4$ for each $
  \bu=[\u_0;\ldots;\u_3]\in \bV(3)$ by
\[
D_k(\bu) = \Omega(V,\u_0)\cap \cell(\bu,k),
\]
where $\cell(\bu,k)$ is the  Marchal $k$-cell of $\bu$.
\end{definition}
%
\indy{Notation}{D@$D_k$}%
\indy{Index}{D-cell}%


A $D_k$-cell, which is a subset of
$\Omega(V,\u_0)\cap B$, is the adaptation of a Marchal $k$-cell to the
geometry of the strong dodecahedral conjecture.  

\begin{lemma}[]\guid{ZERRZRM}
  Let $V$ be a saturated packing and let $\u_0\in V$.  If the intersection
of a $D_i$-cell with a $D_j$-cell is not a null set, then $i=j$ and the two
cells are equal.   The union
  of all the $D_k$-cells at $\u_0$ is $\Omega(V, \u_0)\cap B$.
\end{lemma}
\indy{Index}{null set}%

\begin{proof} This follows from the corresponding facts for cells in
  Lemma~\ref{lemma:marchal-equal}.  Each null set is in fact a subset
  of a plane.
\end{proof}


Every cell $D_k(\bu)$ is eventually radial at $ \u_0$ and has a
well-defined solid angle $\sol(\u_0,D_k(\bu))$.  Every cell $D_k(\bu)$ has
an \newterm{exposed} surface area $\op{surf}(D_k(\u))$, the area of
the intersection of $D_k(\bu)$ and the boundary of $\Omega(V,
\u_0)\cap B$.  It consists of the sum of the areas of the analytic
facets (linear or spherical surfaces) that do not meet the point $
\u_0$.  The total surface area of $\Omega(V,\u_0)$ is the sum of the
exposed surface areas $\op{surf}(D_k(\u))$.  \indy{Index}{exposed}%

We use the functions $\dih_i$ in
\eqref{eqn:dihi} to introduce a function of six variables $y=(y_1,y_2,\ldots,y_6)$:
\[
\op{soly}(y)=
\dih_1(y)+\dih_2(y)+\dih_3(y)-\pi.
\]
By Girard's formula for the solid angle of a simplex 
(Lemma~\ref{lemma:prim-volume}),
\[
\op{soly}(y_1,y_2,y_3,y_4,y_5,y_6)=
\sol(\u_0,\op{conv}\{\u_0,\u_1,\u_2,\u_3\})
\]
when
\[
y_i = \begin{cases}
\norm{\u_0}{\u_i}, & i\in\{1,2,3\},\\
\norm{\u_j}{\u_k}, & i\in\{4,5,6\}\textand\{i-3,j,k\} = \{1,2,3\}.\\
\end{cases}
\]
\indy{Index}{surface area!exposed}%
\indy{Notation}{surf@$\op{surf}$ (surface area)}%
\indy{Notation}{solid@$\sol$ (solid angle)}%
\indy{Notation}{solid@$\op{soly}$ (solid angle as a function of edges)}%


Every cell $D_k(\bu)$ has a set $E(k,\bu)$ of distinguished edges
(that is, the edges of the cell that have an endpoint at $\u_0$) and a
dihedral angle $\dih(e)$ for $e\in E(k,\bu)$.  Each edge has a length
$h(e)\in\leftclosed1,1.26\rightclosed$.  \indy{Index}{angle!dihedral}%
\indy{Notation}{dih}%
\indy{Index}{edge!length}%
\indy{Notation}{E@$E$ (edge)}%

\subsection{local inequality}

We reduce the strong dodecahedral conjecture to a kissing estimate and a
local inequality.  \indy{Index}{kissing estimate}%


\begin{definition}[$a_D$,~$b_D$,~$y_D$,~$v_D$,~$f$]\guid{TCOSFNQ}
  Define constants $a_D$, $b_D$, $y_D$, and functions $v_D$, $f$ as
  follows.  Let $y_D\approx 2.1029$ be defined by the condition
\[
\op{soly}(2,2,2,y_D,y_D,y_D) = \pi/5.
\]
For any $\u_i\in\ring{R}^3$, let $g(\u_0,\u_1,\u_2,\u_3)$ be the
volume of the intersection of the convex hull of
$S=\{\u_0,\ldots,\u_3\}$ with set of points closer to $ \u_0$ than to
any other point in $S$.  When
\begin{equation}\label{eqn:uy}
  \norm{\u_0}{\u_i}=2\textand \norm{ \u_i}{ \u_j} = y\text{ for }i,j\ge 1,
\end{equation} 
this volume depends only on $y$. Write $v(y) = g(\u_0,\ldots,\u_3)$.
Set
\[
  f(y) = v(y) + a\, \op{soly}(2,2,2,y,y,y) + 3 b \dih(2,2,2,y,y,y).
\]
The linear system
\begin{equation}\label{eqn:fyD}
f(y_D) = 0,\quad f'(y_D) = 0
\end{equation}
has a unique solution in $a,b$ with values $a=a_D\approx -0.581$,
$b=b_D\approx 0.0232$.
\end{definition}
\indy{Notation}{a@$a_D$ (dodecahedral parameter)}%
\indy{Notation}{b@$b_D$ (dodecahedral parameter)}%
\indy{Notation}{yd@$y_D$ (dodecahedral parameter)}%
\indy{Notation}{yd@$v_D$ (dodecahedral function)}%
\indy{Index}{convex hull}%
\indy{Index}{regular dodecahedron!volume}%

Note that the regular dodecahedron has volume $20 v(y_D)$ and surface
area $60 v(y_D)$.  Also,
\begin{equation}
  2\,\op{soly}(2,2,2,y_D,y_D,y_D) =\dih(2,2,2,y_D,y_D,y_D)=2\pi/5.
\end{equation}
\indy{Index}{regular dodecahedron}%
\indy{Index}{regular dodecahedron!surface area}%

\begin{lemma}[local inequality]\guid{PWVDMPT}\label{lemma:D-local}
For any cell $D_k(\bu)$
\[
  \op{surf}(D_k(\bu)) + 3 a_D \sol(D_k(\bu)) 
+ 3 b_D \sum_{e\in E(k,\bu)} L(h(e)) \dih(e) \ge 0,
\]
where $L$ is the function of Definition~\ref{def:L}.  Equality holds
precisely when the cell is a null set or a $4$-cell with edges
$(2,2,2,y_D,y_D,y_D)$.
\end{lemma}
\indy{Index}{local inequality}%



For a cell $D_4(\bu)$ with parameters of the form
  \eqn{eqn:uy}, the local inequality reduces to the inequality
  $f(y)\ge 0$.  The constants $a_D$ and $b_D$ are chosen so that
  $y=y_D$ is a critical point of $f$ with value $f(y_D)=0$.  In
  particular, the local inequality asserts that $f$ has a local
  minimum at $y=y_D$.


\begin{proof} 
  The dimension of a $k$-cell is at most six.  The local cell
  inequality is a \cc{The $D_4$-inequality is 9627800748 and 
$D_3$ is 6938212390.}.  
\end{proof}

\begin{remark}
\end{remark}

\begin{lemma}[]\guid{OIEKCEZ}
  The local inequality and the kissing number estimate \eqn{conj:L12}
\[
\sum L(h) \le 12
\]
imply the strong dodecahedral conjecture.
\end{lemma}

\begin{proof} 
  Sum the local inequality over all the $D_k$-cells in a Voronoi cell.
  The solid angles sum to $4\pi$ and the dihedral angles around each
  edge sum to $2\pi$:
\begin{align*}
  \op{surf}(\Omega) &\ge \op{surf}(\Omega\cap B)\\
  &=\sum_{k,\bu} \op{surf}(D_k(\bu))\\
  &\ge -12\pi a_D - 6\pi\, b_D  \sum L(h)\\
  &\ge -12\pi a_D - 72\pi\, b_D\\
  &= -60 \op{soly}(2,2,2,y_D,y_D,y_D) a_D - 180 \dih(2,2,2,y_D,y_D,y_D) b_D\\
  &= 60 (v(y_D) - f(y_D))\\
  &= 60 v(y_D).\\
\end{align*}
The final term is the surface area of a regular dodecahedron.
\end{proof}

Thus, the strong dodecahedral conjecture follows from the same kissing
number estimate that is used to prove the Kepler conjecture.  The case
of equality occurs only for the regular
dodecahedron.

\newpage
\section{Fejes T\'oth's Full Contact Conjecture}


On Dec 26, 1994, L Fejes T\'oth wrote me a letter stating ``I suppose that you will be
interested in the following conjecture: In $3$-space any packing of equal
balls such that each ball is touched by twelve others consists of hexagonal layers.
In the enclosed papers a strategy is described to prove this conjecture.''
~\cite{Fejes-Toth:89}, ~\cite{Fejes-Toth:69}.

Call a nonempty packing $ V$ in $\ring{R}^3$ in which every point has
distance  $2$ from twelve other points a \newterm{packing with full
  contact}.   We affirm Fejes-T\'oth's conjecture in the following form.
\indy{Index}{packing!full contact}%
\indy{Index}{full contact}%
\indy{Notation}{V@$V$ (packing)}%


\begin{theorem}[Packings with Full Contact]\guid{BDEDUTL}\label{thm:fc} 
  Let $ V$ be a packing with full contact.  Then for every point $
  \u\in V$, the arrangement of twelve around that point is the kissing
  configuration of the HCP or FCC packing.
\end{theorem}
\indy{Index}{HCP}%
\indy{Index}{FCC}%
\indy{Index}{kissing configuration}%

This section  sketches a proof of this result.  


\begin{lemma}[]\guid{LIHVTRE} \label{lemma:gap}
  Let $V$ be any packing with full contact and let $\u\ne\v\in V$.
  Then $\norm{\u}{\v}=2$ or $\norm{\u}{\v} \ge 2.52$.
\end{lemma}
\indy{Index}{packing!full contact}%
\indy{Index}{full contact}%

\begin{proof} Let $ \u_1,\ldots, \u_{12}$ be the twelve kissing points
  around $\u$.  Assume that $\v\ne \u_i$.  By
  Assertion~\ref{conj:L12},
\[
  12 + L(h( \u, \v)) 
  = \sum_{i=1}^{12} L(h( \u, \u_i)) + L(h( \u, \v)) \le 12.
\]
This implies that $L(h( \u, \v))\le 0$, so $\norm{ \u}{ \v}\ge 2.52$.
\end{proof}

A packing $V$ may always be translated so that $\orz\in V$.  We study
the structure of a kissing configuration centered at $\orz$ that
has the separation property of Lemma~\ref{lemma:gap}.

\begin{definition}[$S^2(2)$,~$\CalV$]
  Let $S^2(2)$ be the sphere of radius $2$, centered at $\orz$.  Let
  $\CalV$ be the set of packings $V\subset \ring{R}^3$ such that
\begin{enumerate}\wasitemize 
\item $\card(V) = 12$,
\item $V\subset S^2(2)$.
\item $\norm{\u}{\v} \in \{0,2\}\cup
  \leftclosed2.52,4\rightclosed$ for all $\u,\v\in V$.
\end{enumerate}\wasitemize 
\indy{Notation}{V@$\CalV$ (twelve sphere configurations)}%
\indy{Notation}{S@$S^2(r)$ (sphere of radius $r$)}%
\end{definition}

The strategy of the proof is to classify the hypermaps of contact
fans $(V,E_{ctc})$ for $V\in \CalV$ and to show that there are only two
possibilities: the contact hypermaps of the FCC and the HCP.  From
this, the proof of Fejes T\'oth's conjecture follows.

The classification result is analogous to the one that we have already
obtained for tame hypermaps.  This suggests developing a proof along
exactly the same lines as earlier chapters.  We define a new
collection of hypermaps with properties that are analogous to those
defining a tame hypermap and call them \newterm{hypermaps with tame
  contact}.  A computer generated classification of these hypermaps
shows gives only a few possibilities.  Those other than the FCC and
HCP hypermaps are eliminated by linear programming methods.

\subsection{main estimate}




We use the function $\tau(V,E,F)$ from Definition~\ref{def:tau}.  If
$V$ is a packing, $(V,E)$ is a \newterm{biconnected} graph, and
$V\subset S^2(2)$, then 
\[
\tau(V,E,F) = \sol(U_F) + (2-k(F)) \sol_0.
\]
% \indy{Notation}{ZZtauf@$\tau(V,E,F)$ (contact fans)}%
% \indy{Index}{kissing configuration}%


\begin{remark}[Lexell's theorem]
For any two points $\u,\v\in S^2(2)$,
Lexell's theorem asserts that the locus of points $\w\in S^2(2)$, along which
the spherical triangle with vertices $\u,\v,\w$ has fixed area, is an arc of a circle
with endpoints at the antipodes of $\u$ and $\v$.

Lexell's theorem is a considerable aid in finding the minimum of
  $\sol(U_F)$ or $\tau$.
  It is a consequence of Lexell's theorem that the area of a spherical
  triangle (viewed as a function of its edge lengths) does not have a
  interior point local minimum, when the edge lengths are
  constrained to lie in given intervals.  The minimum occurs at a
  point where each edge length lies at an endpoint of its interval.
%
\indy{Index}{Lexell's theorem}
\end{remark}

The next theorem is main estimate for packings with full contact.
(Compare with Lemma~\ref{lemma:empty-d}.)  It is similar to the main
estimate in Leech's proof of the problem of thirteen spheres
\cite{Leech:1956:MG}.  We recall that $\op{tgt}=1.541$.
%
\indy{Notation}{tgt@$\op{tgt}=1.541$}%


\begin{theorem}\guid{VGJDQJV}\label{lemma:main-estimate-12}
  Let $V$ be a packing in $S^2(2)$, $E$ a set of edges, and $F$ a face
  of $\op{hyp}(V,E)$ such that $(V,E,F)$ is a local fan.  Assume that
  $(V,E)$ is a biconnected graph.  Assume that $F$ has at least three
  darts Assume that every edge in $E$ has length at most $3.2$.  Let
  $S$ be a subset of $E$ such that the length of every edge in $S$ is
  at least $2.52$.  Let $U=U_F$ be the topological component of
  $Y(V,E)$ corresponding to $F$.  Assume that if $\{\u,\v\}\subset V$
  with $C^0\{\u,\v\}\subset U$ and $2\le\norm{\u}{\v} \le 2.52$, then
  $\{\u,\v\}\in E$.  Then
\[\tau(V,E,F) \ge \min(d(r,s),\op{tgt}),\]
where
\[
d(r,s) = 0.103 (2-s) + 0.277 (r+2s-4),\ \ 
s = \card(S),\textand  r = \card(E)-\card(S).
\]
%\indy{Notation}{d@$d$ (contact weight constant)}
\end{theorem}

\begin{proof} This proof imitates the proof of the main estimate from
  \cite{Hales:2006:DCG}.   (Compare
  Chapter~\ref{sec:local}.)

  For a contradiction, let the packing $(V,E,F,S)$ violate the given
  inequality.  Among all counterexamples to the theorem, we may assume
  that $(V,E,F,S)$ is a counterexample that minimizes $k=r+s$.  Let
  $k_{min}$ be the smallest value attained.



  \claim{There exists a counterexample that minimizes
\begin{equation}\label{eqn:td}
\tau(V,E,F)-\min(d(r,s),\op{tgt})
\end{equation}
among all (local) counterexamples that have parameters that satisfy
$k_{min}=k$.}  Indeed, a compactness argument shows that a sequence
tending to the minimum value has a convergent subsequence.  (Compare
Lemma~\ref{lemma:compact-fan} and Lemma~\ref{lemma:c-bound}.)

We may assume that the counterexample $(V,E,F,S)$ is local and minimal
in this sense.

\claim{In a minimal counterexample, all edges of length at least
  $2.52$ belong to $S$.}  Indeed, $d(r-1,s+1)>d(r,s)$.

\claim{A minimal counterexample $(V,E,F,S)$ does not have any edges
  $\ee=\{\u,\v\}\subset V$, satisfying $C^0(\ee)\subset U$ and
  $\norm{\u}{\v}\le 3.2$.}  Otherwise, $\ee$ may be added to the edge
set of the fan.  The face $F$ splits into two faces $F_1$ and $F_2$.
By the additivity of the constants $d(r,s)$ under splitting (analogous
to Equation~\ref{eqn:drs}), one of the two faces $F_1$ or $F_2$ is a
counterexample as well.  This is contrary to the assumed minimality of
$F$: $k(F_1),k(F_2)<k_{min}=k(F)$.


Call a node $\v\in V$ \newterm{concave} or \newterm{convex}, according
to whether $\op{azim}(x)\ge\pi$ or $\op{azim}(x)<\pi$, where $x$ is
the dart of $F$ at $\v$.  \indy{Index}{concave node}%
\indy{Index}{convex node}%

\claim{In a minimal counterexample $(V,E,F)$, if $\v\in V$ is a
  concave node, then both edges at $\v$ have length $3.2$.}  Indeed,%
\footformal{In the original proof of the Kepler conjecture 
  various geometrical subtleties appear in this part of the
  proof.  None of these subtleties appear here.  In fact, the
  deformation remains a packing because
\[
{\mathcal E}(2,2,2,2,2,2,2,2,2)> 3.2.
\]
The deformation remains a fan, because of the calculation
\[
\Delta(4,4,4,3.2^2,4,4)>0.
\]
Once it is established that the edges at a concave node $\v$ have length $3.2$, it
follows that all the vertices in 
\[
W(\orz,\v,\rho\v,\rho^{_1}\v)\cap (V\setminus\{\v\})
\]
have distance at least $3.2$ from $\v$.  It follows that a half-disk
of radius $\arc(2,2,3.2)$ fits inside the region.  Indeed, since
$\arc(2,2,3.2)>\pi/2$, the edges bow away from the concave node, so
that the point on the boundary of $U$ closest to the node is another
node.  A convex node adjacent to the concave node has angle $>\pi/2$
because of dihedral angle estimates $\dih(2,2,2,3.2,3.2,2)>\pi/2$, and
so forth.  } if both edges at $\v$ have length less than $3.2$, there
exists a deformation of the local fan $(V,E,F)$ that fixes every node
except $\v$ and decreases the solid angle $U_F$.  This cannot be a
minimizing counterexample.  Now assume that one of the edges
$\{\v,\u\}$ has length $3.2$.  Let $\{\v,\w\}\in E$ be the other edge.
%By minimality it has extremal length $2,2.52$, or $3.2$.  
Let
$y=\norm{\u}{\w}$.  We check by a \cc{CALC XX  Calculation} that
\[
\dfrac{d\op{soly}(2,2,2,y,3.2,t)}{dt} >0,
\]
when $t\in\leftclosed2.52,3.2\rightclosed$.  Hence, the triangle of
largest area occurs when $\norm{\v}{\w}=3.2$.  (The triangle has
largest area when $U_F$ has smallest solid angle because we are at a
concave node, and the two regions are complementary.)


\claim{In a minimal counterexample, some node is concave.}  Otherwise,
every node is convex.  Every edge has arclength at least
$\arc(2,2,2)=\pi/3$.  By Lemma~\ref{lemma:convex-hyp}, the cardinality
$k$ of $E$ satisfies $(\pi/3)k < 2\pi$, so $k<6$.  (The inequality is
strict for a generic fan.)  By Lexell's theorem, the five edges have
extremal lengths;  that is, every edge in $S$ has length $2.52$ or
$3.2$ and every edge in $E\setminus S$ has length $2$.  The only
remaining degrees of freedom are the lengths of the diagonals.  As the
polygon has at most five sides, we have reduced the proof  to a
finite \cc{CALC XX Calculation} of dimension at
most two.


\claim{In a minimal counterexample some node is convex.}  Otherwise,
the complementary region is convex.  The perimeter estimate for convex
regions gives $k<6$.
One
half of $\op{rcone}(\orz,\v,\cos\theta)$, where
$\theta=\arc(2,2,3.2)\approx 1.854$ fits inside the region.  The
half-cone has solid angle
\[
s=\pi(1-\cos\theta)\approx 4.02.
\]
This gives
\[
\tau(V,E,F) > s + (2-k)\sol_0 > \op{tgt}.
\]
\indy{Index}{arcradius}%



\claim{In a minimal counterexample, there cannot be both a concave
and a convex  node.} 
The convex node adjacent to a concave node $\v$ has interior angle greater
than $\pi/2$ and can be deformed by decreasing the distance between it
and $\v$ in order to decrease the solid angle of $U$.  This shows that the
function $\tau$ has no local minimum among such arrangements.


The various claims show that no minimal counterexample exists.  This
completes the proof.
\end{proof}

\subsection{biconnected fans}

We may create  fans that are biconnected graphs in the same way as in
\cite{Hales:2006:DCG}.  Here is a review
of the construction.



\begin{lemma}\guid{NJFWRPQ}\label{lemma:V'-bi} 
Let $V\in \CalV$.  Then there exists $V'\in \CalV$ with
  the following  properties:
\begin{enumerate}\wasitemize 
\item There is a bijection $\phi:V'\mapsto V$ that induces a bijection
  of contact graphs:
\[
\phi_*:(V,E_{ctc}) \cong (V',E'_{ctc}).
\]
\item Let $E'$ be the set of all pairs $\{\u,\v\}\subset V'$
  such that $2.52\le\norm{\u}{\v} <\sqrt8$.  Set $E =
  E'_{ctc}\cup E'$.  Then $(V',E)$ is a fan.
\item The graph $(V',E)$ is biconnected.
\end{enumerate}\wasitemize 
\end{lemma}

\begin{proof}
  Begin with the contact fan $(V,E_{ctc})$.  Let $E'$ be the set
  of all pairs $\{\u,\v\}\subset V$ such that
  $2.52<\norm{\u}{\v}<\sqrt8$.

  \claim{We claim that $(V,E_{ctc}\cup E')$ is a fan.} Indeed, it is checked by
  \cite[Lemma~4.30]{Hales:2006:DCG} that the blades satisfy the
  intersection property of fans, except possibly when two new blades
  are the diagonals of a quadrilateral face in $(V,E_{ctc})$.  (The
  cited lemma uses the constant $2.51$ instead of $2.52$, but this
  does not affect the outcome.)  The diagonals of a quadrilateral face
  in $(V,E_{ctc})$ is a spherical rhombus and one of its diagonals is
  necessarily at least $\sqrt8$ (with extreme case a square of side
  $2$).  The other fan properties are easily checked.

  If the hypermap $\op{hyp}(V,E_{ctc}\cup E')$ is not connected,
  the set of nodes $V_1\subset V$ in one combinatorial component can
  be moved closer to another combinatorial component until a new edge
  is formed.  This can be done in a way that the deformation of $V$
  remains in $\CalV$ and no new edges of length at most $2.52$ are formed.
  Continuing in this fashion, a connected hypermap is obtained.

  Further deformations within a connected hypermap produce a
  biconnected hypermap with the given properties.
\end{proof}


\begin{lemma}\guid{CTJYKFZ}\label{lemma:dj}
Let $H=(D,e,n,f)$ be a connected hypermap with more than
two darts.  Assume that the hypermap has no loops or double joins. Then
every face has at least three darts.
\end{lemma}

\begin{proof}
  Otherwise, if some face has only two darts, then because of the no
  double join condition, the two edges meeting the face, $\{x, e x\}$
  and $\{ e^{-1} f^{-1} x, f^{-1} x\}$, must actually be equal.  That
  is, $ x = f e x = n^{-1} x$, so that $x$ is a fixed point of the node
  map.  Similarly, $f x$ is a fixed point of the node map.  Then $\{x,
  f x\}$ is a combinatorial component, which is contrary to the
  assumption that the hypermap is connected with more than two darts.
\end{proof}


\begin{definition}[$D_U$,~$m_U$]
  Let $V\in \CalV$.  Let $U$ be a topological component of
  $Y(V,E_{ctc})$ and let $D_U$ be the set of all darts that lead into
  $U$.  It is a union of faces of $\op{hyp}(V,E_{ctc})$.  For each
  $x\in D_U$, let $m(x) >0$ be the smallest natural number such that
  $f^{m} x$ and $x$ lie at the same node.  Let $m_U$ be the maximum of
  $m(x)$  as $x$ runs over $D_U$.  The constant $m_U$ can be viewed as
  a \newterm{simplified face size}.  \indy{Index}{contact!full}%
  \indy{Index}{fan}%
  \indy{Notation}{m@$m$ (simplified face size)}%
  \indy{Notation}{DU@$D_U$ (the set of darts leading into $U$)}
\end{definition}

\begin{lemma}[biconnected]\guid{BTZPFMU}\label{lemma:biconnected}
  Let $V\in \CalV$.  Then $\op{hyp}(V,E_{ctc})$ is biconnected.
\end{lemma}

\begin{proof}
  By Lemma~\ref{lemma:V'-bi}, we may replace $V$ with a new set in
  $\CalV$ if necessary so that $(V,E)$ is a biconnected fan, where
  $E\setminus E_{ctc}$ is the set of pairs $\{\u,\v\}\subset V'$ such
  that $2.52\le\norm{\u}{\v} <\sqrt8$.  We  show that the smaller
  fan $(V,E_{ctc})$ is also biconnected.

  Let $U'$ be a topological component of $Y(V,E_{ctc})$.  Let
  $D'=D_{U'}$.  Let $m'=m(U')$, $r' = \card(D')$, and let $s'$ be
  twice the number of combinatorial components of
  $\op{hyp}(V,E_{ctc})$ that meet $D'$.  Let $k'=r'+s'$.  Set
  $\tau(U') = \sol(U') + (2-k')\sol_0$.

\claim{We claim that $\tau(U')\ge \min(d(r',s'),\op{tgt})$.}  Indeed,
up to a null set (given by the finite union of blades $C^0(\ee)$ for
$\ee\in E\setminus E_{ctc}$), $U'$ is the union of topological
components $U_F$ of $Y(V,E)$, which are in bijection with the faces
$F$ of $\op{hyp}(V,E)$.  The function $\tau(U')$ is additive:
\[
\tau(U') = \sum_{U_F\subset U'} \tau(V,E,F).
\]
By the biconnectedness of $(V,E)$, each value $\tau(V,E,F)$ is the
same before and after localization.
Lemma~\ref{lemma:main-estimate-12} gives a lower bound on the
constants $\tau(V,E,F)$.  The constants $d(r',s')$ are also additive:
\[
d(r',s') = \sum_{U_F\subset U'} d(r(F)),s(F)),
\]
where $s(F)$ is the cardinality of the set of edges of $E\setminus
E_{ctc}$ that meet $F$, and $r(F) = \card(F)-s(F)$.  Thus, the claimed
lower bound on $\tau(U')$ follows from the main estimate
(Lemma~\ref{lemma:main-estimate-12}).


\claim{If $m'\le 5$, then $D'$ is a simple face.} Otherwise, either
the face is not simple or  more than one face  leads into
$U$.  Some node $\v$ lies in the interior to the $m'$-gon.  Let
$\u,\w$ be consecutive nodes around the $m'$-gon.  By a
\cc{CALC XX Calculation}
the angles $\op{azim}(\orz,\v,\u,\w)$ are each less than $2\pi/5$. The
$m'$ angles around $\v$ cannot sum to $2\pi$ as required.

\claim{We claim that $D'$ is a simple face.}  Otherwise, assume for a contradiction
that $D'$ is not simple, $m'\ge 6$, and $d(r',s')<\op{tgt}$.  From the
classification of \cite[p.~126,~Fig.~12.1]{Hales:2006:DCG}, and the
inequalities $d(9,0) > \op{tgt}$, $d(6,2) > \op{tgt}$, it follows that
the set $D'$ has cardinality eight and $m'=6$.  The set $D'$ meets seven
nodes: the six nodes counted by $m'$ and a node in the ``center'' of
the hexagonal arrangement.  As the packing has twelve nodes in all,
 five  nodes remain, each  meeting a
nontriangular topological component of $Y(V,E_{ctc})$.  Thus, the hypermap has
 at least one pentagon or two quadrilaterals.  By \eqn{eqn:delta0},
\[
  \sum_{U'\subset Y(V,E_{ctc})} \tau(U') 
=\sum_F \tau(V,E,F) = (4\pi - 20\sol_0) < \op{tgt}.
\]
However, $\sum_{U'} \tau(U')$ is at least
\begin{align*}
d(8,0) + d(5,0) &> \op{tgt}, \text{ or }\\
d(8,0) + 2 d(4,0) &> \op{tgt}.
\end{align*}
Thus, $D'$ is a simple face.
\indy{Index}{weight!total}%
\indy{Index}{weight}%

\claim{The hypermap is biconnected.}  Otherwise, if the hypermap is
not connected, then we can find two faces of the hypermap that lead
into the same topological component of $Y(V,E_{ctc})$.  If the
hypermap is connected but not biconnected, then some face of the
hypermap is not simple.  Both possibilities contradict the claim that
$D'$ is a simple face.
\end{proof}



\subsection{tame contact}

This subsection modifies the notion of tameness to cover hypermaps
that arise as the contact fan of a packing with full contact.  In the
definition of tame hypermap, two functions $b$ and $d$ are used.  In
this section we define two new functions that are used to define tame
contact.  They are also  called $b$ and $d$ because we have no
further use for the functions in Chapter~\ref{sec:tame}.  
Recall that $\op{tgt}=1.541$.
\indy{Index}{tame}%
\indy{Index}{hypermap!tame}%

\begin{definition}[b]\guid{IHRZTPV}
  Define $b:\ring{N}\times \ring{N}\to \ring{R}$ by
  $b\pqr{(p,q)}=\op{tgt}$, except for the following values:
\[
b(0,3)=b(1,3)=0.618,\quad b(2,2)=0.412.
\]
\end{definition}
\indy{Notation}{b@$b$ (contact weight parameter)}%

\begin{definition}[d]\guid{VUJQZCG}
Define $d:\ring{N}\to \ring{R}$ by
\[d(k) = \begin{cases}
0, & k\le 3, \\
0.206 + 0.277 (k-4),& k=4,\ldots,8,\\
%0.206 & k=4, \\
%0.483 & k=5, \\
%0.760 & k=6, \\
%1.037 & k=7, \\
%1.314 & k=8,\\
\op{tgt},& k>8.\\
\end{cases}
\]
%(In particular, $d(k) = 
\end{definition}
\indy{Notation}{d@$d$ (contact weight parameter)}%

The function $d$ is related to the two-variable function in
Lemma~\ref{lemma:main-estimate-12}: $d(k) = d(k,0)$, when $4\le k\le
8$.

\begin{definition}[weight~assignment]\guid{GLIQSFM}
%
  Recall that a \newterm{weight assignment\/} on a hypermap $H$ is a
  function $\tau$ on the set of faces of $H$ taking values in the set
  of nonnegative real numbers. A weight assignment $\tau$
is a \newterm{contact}
  weight assignment if the following two properties hold:
%
  \indy{Index}{weight assignment}%
  \indy{Index}{contact!weight assignment}%
  \indy{Notation}{ZZtau@$\tau$ (weight assignment)}%
\begin{enumerate}
\item If the face $F$ has cardinality $k$, then
$\tau(F) \ge d(k)$.
\item If a node $\v$ has type $(p,q,0)$, then
  \[\sum_{F:\,\v\cap F\ne\emptyset} \tau(F) \ge
    b{\pqr{(p,q)}}.\]
\end{enumerate}
The sum $\sum_F \tau(F)$ is called the \newterm{total weight} of $\tau$.
\indy{Index}{total weight}%
\end{definition}


\begin{definition}[tame contact]\guid{XJPQTIV}
  A hypermap has \newterm{tame contact\/} if it satisfies the following 
  conditions:
%
\indy{Index}{tame contact}%
\indy{Index}{contact!tame}%
\indy{Index}{planar}%
\indy{Index}{biconnected}%
\indy{Index}{nondegenerate}%
\indy{Index}{loop}%
\indy{Index}{double join}%
%
\begin{enumerate}
%\label{definition:tame}
%1
\item \case{planar} The hypermap is plain and planar.
\item \case{biconnected} The hypermap is biconnected.  In particular,
  each face meets each node in at most one dart.
\item \case{nondegenerate} The edge map $e$ has no fixed points.
\item \case{no loops} The two darts of each edge lie in different
  nodes.
\item \case{no double join} At most one edge meets any two given
  nodes.
\item \case{face count} The hypermaps has at least two faces.
\item \case{face size} The cardinality of each face is at least three
  and at most eight.
%\label{definition:tame:length}
\item \case{node count} The hypermap has twelve nodes.
\item \case{node size} The cardinality of every node is at least two  and at most four.
%\label{definition:tame:degree}
%    \item \case{node} {\tt NO CONDITION}
%\label{definition:tame:degreeE}
\item \case{weights} There exists a contact weight assignment of total
  weight less than $\op{tgt}$.
%\label{definition:tame:squander}
\end{enumerate}
%
\end{definition}


%\subsection{tame contact}

\begin{theorem}\guid{ZXZSVPH} The contact hypermap of a 
  packing $V\in \CalV$ is a hypermap with tame contact.
\end{theorem}
\indy{Index}{hypermap!tame}%
\indy{Index}{hypermap!contact}%
\indy{Index}{hypermap}%
\indy{Index}{contact!full}%

\begin{proof} It is enough to go through the list of properties that
  define a tame contact hypermap and to verify that the contact
  hypermap satisfies each one.  We use the weight assignment $F\mapsto
  \tau(V,E_{ctc},F)$.

\begin{enumerate}
\item \case{planar} The contact hypermap is plain and planar by the
  general properties of fans.\footnote{Earlier chapters give a long
    discussion of planarity.  In this chapter, we are not attempting
    to give a formalizable blueprint, so we relax our standards and
    regard planarity as an obvious feature of fans.}
\item \case{biconnected} The hypermap is biconnected by
  Lemma~\ref{lemma:biconnected}.
\item \case{nondegenerate} The
  edge map has no fixed points by the general properties of fans.
\item \case{no loops}
  There are no loops or multiple joins by the general properties of
  fans.
\item \case{no double join} 
\item \case{face count} Each node has at least two darts by
  biconnectedness. Each face is simple; so the two darts at a node lie
  in different faces.  Thus, the hypermap has at least two faces.
\item \case{face size} The cardinality of each face is at least three
  because the hypermap has no loops or double joins (Lemma~\ref{lemma:dj}).
  The cardinality of a face is at most eight because of the estimate
  $d(9,0)>\op{tgt}$.
\item \case{node count} There are twelve nodes by the definition of a
  packing with full contact.
\item \case{node size} We have already established that the cardinality
  of each node is at least two.  The proof that the cardinality is
  never  greater than four appears in Lemma~\ref{lemma:no-5}.
\item \case{weights} Theorem~\ref{lemma:main-estimate-12} establishes 
 the inequality $\tau(V,E_{ctc},F)\ge d(k)$.
   \indy{Notation}{ZZtauf@$\tau(V,E_{ctc},F)$}%
  The total weight of the weight assignment is given by
  equation~\eqn{eqn:delta0}:
\[
  \sum_F \tau(V,E_{ctc},F) = (4\pi - 20\sol_0) < \op{tgt}.
\]
\indy{Index}{weight!total}%
Let $\v$ be a node of type $(p,q,0)$.  
%Let $A$ be the set of faces
%that meet the node $\v$. 
Then
\[
\sum_{F\mid F\cap \v\ne\emptyset}\tau(V,E_{ctc},F) > d(4)~q.
\]
This gives the nonzero entries in the table of bounds $b(p,q)$.  The
remaining entries follow from Lemma~\ref{lemma:no-5}.
\end{enumerate}
\end{proof}




\begin{lemma}[]\guid{CQRHDZE}\label{lemma:no-5} 
  Let $V\in \CalV$.  Every node of $(V,E_{ctc})$ has degree at most
  four.  Furthermore, suppose the type of a node is $(p,q,0)$.  Then
  $(p,q)$ must be
\[
(0,3),~(1,3),~\text{ or}~~(2,2).
\]
\end{lemma}

\begin{proof} The interior angles of a spherical polygon in the
  contact graph have the following lower $\alpha_k$ and upper bounds
  $\beta_k$, as a function of the number of sides $k$.
\begin{equation}
\begin{array}{lllll}
  \phantom{\ge}k~~&\alpha_k & \beta_k\\
  \phantom{\ge}3~~&\op{azim}(2,2,2,2,2,2)  &\op{azim}(2,2,2,2,2,2)\\
  \phantom{\ge}4~~&\op{azim}(2,2,2,2.52,2,2) &2\,\op{azim}(2,2,2,2,2.52,2)\\
  {\ge}5~~& \op{azim}(2,2,2,2.52,2,2) ~~~& 2\pi.
\end{array}
\end{equation}
Thus,
\[
  p\,\alpha_3 + q\,\alpha_4 +r\, \alpha_5 
\le 2\pi \le p\,\beta_3 + q\,\beta_4 + r \,\beta_5.
\]
There are no solutions for
$(p,q,r)$ in natural numbers when $p+q+r\ge 5$ and
 only the three given solutions in $(p,q,r)$ with $r=0$.
\end{proof}



\subsection{classification}


\begin{lemma}[tame hypermap classification]\guid{AZYOJBE}\rating{ZZ}
  \label{lemma:contact-classification} Every hypermap with tame
  contact is isomorphic to a hypermap in the following list of eight
  hypermaps, or is isomorphic to the opposite of a hypermap in the
  list.  \indy{Index}{isomorphic}%
\end{lemma}

\begin{proof}
  The set of all hypermaps has been classified by the same computer
  algorithm described in Section~\ref{sec:proof-classification}.
  \indy{Index}{contact!tame}%
  \indy{Index}{hypermap}%
  \indy{Index}{hypermap!tame}%
\end{proof}



\begin{lemma}[]\guid{MWWSZTX}\label{lemma:fcc-ft} Let $V\in \CalV$.
  Suppose that $H=\op{hyp}(V,E_{ctc})$ is a hypermap with tame
  contact.  Then $H$ is the FCC or HCP contact hypermap.
\end{lemma}

\begin{proof} The explicit enumeration of hypermaps with tame
  contact has eight cases.  Two are the hypermaps of the
  FCC and HCP.  The remaining six must be eliminated.

There are some linear
  programming constraints that are immediately available to us:
\begin{enumerate}\wasitemize 
\item The angles around each node sum to $2\pi$.
\item Each angle of a triangle is $\alpha_3$.
\item Each angle of each rhombus lies between $\alpha_4$ and $\beta_4$.
\item The opposite angles of each rhombus are equal.
\item The sum of two adjacent angles of a rhombus lies between
\[
  \alpha_4 + \beta_4 \text{~~and~~} 2\,\op{azim}(2,2,2,\sqrt8,2,2).
\]
\end{enumerate}\wasitemize 
By a \cc{CALC XX Calculation},
these linear programs eliminate all but the FCC and HCP hypermaps.
\end{proof}


\begin{lemma}[]\guid{YRTPQXK}\label{lemma:kiss-fcc}
  Let $V\in \CalV$ be a packing such that $\op{hyp}(V,E_{ctc})$ is
  isomorphic to the FCC or HCP contact hypermap.  Then $V$ is
  congruent to the FCC or HCP configuration in $S^2(2)$.
\end{lemma}
\indy{Index}{HCP}%
\indy{Index}{FCC}%
\indy{Index}{kissing configuration}%
\indy{Index}{contact!full}%
\indy{Index}{hypermap}%

\begin{proof} Every face of the hypermap of $(V,E_{ctc})$ is a
  triangle or quadrilateral.  The eight triangles in the FCC or HCP
  contact hypermap determine eight equilateral triangles in $V$ of
  edge length $2$.  The eight triangles rigidly determine $V$ up to
  congruence.
\end{proof}

\begin{proof}[Proof of Theorem~\ref{thm:fc}]  %[Packings with Full
  % Contact]
  The contact hypermap of a packing with full contact has tame
  contact.  By Theorem~\ref{lemma:fcc-ft}, this hypermap is that of
  the FCC or HCP.  By Lemma~\ref{lemma:kiss-fcc}, the kissing
  configuration of the packing is congruent to the FCC or HCP.  As the
  center of the packing may be chosen at an arbitrary point in the
  packing, every point in the packing is congruent to one of these two
  arrangements.  The result follows.
\end{proof}



\section{Musin-Tarasov Theorem}

The Tammes problem, for given $k$ and $n$, asks for the arrangement of
$k$ points on an $n$-dimensional sphere that maximizes the minimum
distance between points.

Recently, Musin and Tarasov solved the Tammes problem for $k=13$
points on a sphere in dimension $n=2$~\cite{Musin-Tarasov}.  Their
proof attaches a planar graph to each candidate solution, derives
various restrictions on the graph, uses a computer to generate all
planar graphs satisfying the restrictions, and then runs linear programs
to eliminate all graphs except for the optimal one.

These methods are similar enough to those that appear in the Kepler
conjecture that it is tempting for us to find another proof along the
lines developed in this book.  We leave this as a challenging series
of exercises (or as a research proposal) for readers who want a
thorough understanding of the methods of this book.  The exercises are
described in the rest of this section.

\subsection{setup}

This section formulates the Tammes problem without assuming 
 Musin and Tarasov's results.

The solution to the Tammes problem when $k=13$ was first proposed by
Sch\"utte and van der Waerden~\cite{vanderWaerden:1951}.  The arrangement of thirteen
points on the unit sphere is uniquely determined by a graph with thirteen
nodes such that two points are joined by an edge exactly when they
realize the smallest possible distance between points.

The Sch\"utte-van der Waerden graph is formed by one point at the north
pole of the sphere, a second ring of four points equally spaced on a
circle of constant latitude, a third and fourth rings of four points
equally spaced on further circles of constant latitude.  Each circle
is rotated $\pi/4$ from the previous ring, so that the points
alternate.

The Sch\"utte-van der Waerden arrangement can be rescaled so that the
$k$ points lie on a sphere at the origin of radius $r$ and 
the minimum distance between points is $2$.  Let $V_{SvdW}$ be this
packing.

\begin{exer}
Calculate the radius $r$ of the Sch\"utte-van der Waerden arrangement $V_{SvdW}$.
\end{exer}

For the rest of this section, $r$ denotes this fixed radius.  An
optimal solution to the Tammes problem for $k=13$ can also be rescaled
so that the points form a packing $V$ on a sphere $S^2(r)$ of radius
$r$.  From now on, a solution to Tammes problem  refers to a set of
thirteen  points on the sphere $S^2(r)$.

Recall that $(V,E_{std})$ is the standard fan.  It has nodes $V$ and edges
\[
E_{std} = E_{std}(V) = \{\{\v,\w\}\subset V\mid 0 <
\norm{\v}{\w} \le 2\hm\}.
\]
By Lemma~\ref{lemma:iso-surround}, any solution $V$ to Tammes problem
can be replaced by another solution in which each node is either
isolated or surrounded.  Call a solution to the Tammes problem
with this additional property an equilibrated arrangements.


We pose the general problem of classifying
all hypermaps $\op{hyp}(V,E_{std})$ up to isomorphism as $V\subset
S^2(r)$ runs over all equilibrated arrangements.

\begin{exer}
  Adapt the proof of Lemma~\ref{lemma:D'} to show that if $V$ is an
  equilibrated arrangement, then $(V,E_{std})$ has at most one
  isolated node.
\end{exer}

Let $V$ be an equilibrated arrangement.   The subset
of surrounded nodes of $V$ has a standard hypermap in which every
azimuth angle is less than $\pi$ and every face is simple.

\begin{exer} 
  Let $V$ be an equilibrated arrangement.  Show that in the standard
  hypermap of the subset of surrounded nodes of $V$ every face has
  cardinality at most six.  (Adapt the proof of
  Lemma~\ref{lemma:face-size}.)
\end{exer}

If $V$ has no isolated node, let $V'=V$.  If $V$ has an isolated node
$\v$, let $V' = \{\v'\}\cup V\setminus\{v\}$, where $\v'$ is a
displacement of $\v$ to make the standard hypermap connected and to
make every face of the hypermap simple.  We may assume that $\v'$ has
distance  at least $2\hm$ from every other node of $V'$ and that it
has distance  exactly $2\hm$ from at least two other nodes of $V'$.  In
displacing $\v$ to $\v'$, the standard hypermap acquires at least two
new edges.  The cardinality of each face of the standard hypermap of
$V'$ is at most seven.


\subsection{Tammes tameness}

For any face $F$ of a hypermap $\op{hyp}(V,E)$, define
\[
\tau_T(V,E,F) = \sol(U_F) + (2- k(F)) \sol_r,
\]
where $k(F)$ is the cardinality of $F$; $U_F$ is the connected
component of $Y(V,E)$ associated with $F$;and $\sol_r =
\op{soly}(r,r,r,2,2,2)$, the minimal solid angle of a spherical triangle on
$S^2(r)$.

\begin{exer}  
  Let $V$ be an equilibrated arrangement and let $V'$ be its
  displacement.  Develop a definition of \newterm{Tammes tame}
  hypermap, analogous to the definition of tame hypermap, with the
  property that $\op{hyp}(V',E_{std}(V'))$ is Tammes tame for the
  weight function $F\mapsto \tau_T(V,E_{std}(V'),F)$.
\end{exer}

\begin{exer}  
  Download and run the hypermap classification software to classify all Tammes tame
  hypermaps, up to isomorphism.
\end{exer}

\begin{exer} Let $V$ be equilibrated with displacement $V'$.  Develop
  a system of linear inequalities, analogous to those used in the
  proof of the Kepler conjecture, that hold for $(V',E_{std}(V'))$.  
\end{exer}

\begin{exer}
  Run the linear programs for each Tammes tame hypermap to show that
  every $V'$ (and $V$) is infeasible unless $V=V'$ and
  $\op{hyp}(V,E_{std})$ is isomorphic to
  $H_{SvdW}=\op{hyp}(V_{SvdW},E_{std}(V_{SvdW}))$.
\end{exer}

At this stage, a single hypermap remains.  The next exercise
establishes the local optimality of the Sch\"utte-van der Waerden
arrangement.  It is the final exercise in the series of exercises to
solve the Tammes problem for thirteen points on a sphere.

\begin{exer} 
  Custom design a system of linear inequalities for equilibrated
  arrangements $V$ with standard hypermap  isomorphic to
  $H_{SvdW}$.  Show that $V_{SvdW}$ is the unique solution to this
  system of inequalities, up to isometry.
\end{exer}






    \end{runninglinenumbers*}
    %\chapter{Appendix}

\begin{note}%XX
The appendix is not for publication.  It contains some further notes about the
proof that may be of interest to a few experts.
\end{note}

Beyond the text, the proof relies on three separate external bodies of
computer code.  These are described in much greater detail at various
places in the book.  These external bodies of code are called
\begin{enumerate}
\item Tame Hypermap Generation,
\item Interval Arithmetic,
\item Linear Programming.
\end{enumerate}
The tame hypermap generation is an stand-alone program that is run
once at a specific point of the proof.  It carries out a combinatorial
classification of planar graphs satisfying a certain restrictive list
of properties.  The reader can safely ignore this computer program
until reaching the relevant point in the proof.  By contrast, the
interval arithmetic is a collection of nearly one thousand
inequalities that have been proved by computer.  These inequalities
are spread throughout the proof and appear throughout this book.  On
first reading, the reader is encouraged to accept these inequalities
as axiomatically given facts.  Detailed documentation about these
inequalities is available for those who wish to follow up later on the
computer-generated proofs of these inequalities.  The final stage of
the proof consists entirely of linear programming.  There are also
several small linear programs that appear in scattered places in the
book.

\section{Computation}

As this project  progressed, the computer  replaced conventional
mathematical arguments more and more, until now
nearly every aspect of the proof relies on
computer verifications.  Many assertions in this book
are results of computer calculations.
To make the solution more accessible, I have
posted extensive resources \cite{web}.

There are three major pieces of computer code that enter into the proof.

\begin{itemize}
\item {\it Combinatorics}.  A computer program classifies all of the
  planar hypermaps that are relevant to the packing problem.
\item {\it  Interval analysis}.  ``Sphere Packings
I'' describes a method of proving various inequalities in a small number
of variables by computer by interval arithmetic.
\item {\it  Linear programming}.  Many of the nonlinear optimization
problems for the scores of sphere packings are replaced by linear
problems that dominate the original score.  They are solved
by linear programming methods by computer.  A typical problem has
between 100 and 200 variables and 1000 and 2000 constraints.  Nearly
100000
such problems enter into the proof.
\end{itemize}

Computers are used in various other ways.  


\begin{itemize}
\item {\it Formal proof}.
Various parts of the solution of the packing problem have now been
formally verified.
\item  {\it Numerical optimization}.  The exploration of the problem
has been substantially
aided by nonlinear optimization and symbolic math packages.
\item {\it Branch and bound methods}.  When linear programming methods
  do not give sufficiently good bounds, they have been combined with
  branch and bound methods from global optimization.
\item {\it Computer Algebra Systems} Mathematica was used for many
  minor calculations, such as the calculation of exact explicit
  formulas for derivatives.
\item {\it Organization of output}.
The organization of the few gigabytes of code and data that
enter into the proof is in itself a nontrivial undertaking.
\end{itemize}



\section{Formal Proof}


\section{Tame Hypermaps}

\clearpage

\section{Interval Analysis}%DCG 8.3, p75
\label{sec:bounds-simplex}

Interval analysis is a method to obtain trustworthy results from a
computer.  This essay gives a basic introduction to this method.

\subsection{deliberate error}

Interval arithmetic can trace its origins to the method of deliberate
error.  This is an ancient method of navigation with imperfect
instruments.  When William the Conqueror crossed the English Channel
in 1066, he deliberately steered to the north of Hastings.

\begin{quote}
  % ``There is no direct evidence of how a twelfth-century pilot found
  % his way across the English Channel $\ldots$
  ``Any pilot even now who had to make that crossing without a chart
  or compass, as William's pilots did, would use the ancient method of
  Deliberate Error: he would not steer directly towards his objective
  but to one side of it, so that when he saw the coast he would know
  which way to turn.'' \cite[p81]{How81}
  % 1066: The Year of the Conquest, Penguin paperback edition
  % 1981. page 148.
\end{quote}

Deliberate error implements ``better safe than sorry.''
To find an address on a one-way street,  a  driver enters the
street too early, to point in the right direction.  
When searching for a familiar landmark
on a two-way street, it is more efficient  
to start the search safely to one side of the landmark, 
rather than search in ever
expanding zigzags in both directions.  
Deliberate error 
is the idea of trapping a target inside a large enough net, in the
way that
Khachiyan traps the optimal solutions of a linear program inside
an ellipsoid.


The method of deliberate error cushions the adverse effects of
imperfect technology.  The method does not aim to minimize the
imperfections.  It works with a faulty chart and compass.
% The method of deliberate error seeks not to minimize the error of
% imperfect technolo

\subsection{arithmetic}

The method of deliberate error, implemented to control for round-off
errors on a computer, leads to interval arithmetic.  To approximate
the circumference of a circle, Archimedes inscribes a polygon and
circumscribes a second polygon, trapping the unknown within a known
interval.  This is interval arithmetic.


Fixed precision floating point numbers on a computer exactly represent
a finite number of rational numbers.  The remaining uncountable set of
real numbers cannot be precisely represented.  For example, 64-bit
floating point numbers can encode at most $2^{64}$ different real
numbers.  This finite set of representable numbers is not closed under
addition, multiplication, subtraction, or division.  For example, the
input $2.0 + 2.0$ returns $4.0$, because all the numbers involved are
precisely representable.  However, through round-off error, my
computer returns
\begin{verbatim}0.0
\end{verbatim} 
in response to
\begin{verbatim}
-1.0 + (1.0 + 1.0e-16)
\end{verbatim}
and 
\begin{verbatim}1.0e-16
\end{verbatim} 
in response to 
\begin{verbatim}
(-1.0 + 1.0) + 1.0e-16.
\end{verbatim}
As this example shows, machine addition is not even associative.  One
is quickly led to absurd conclusions, if one tries to reason about
floating point operations as if they formed a group, a ring, or a
field.  To reason correctly about floating point, one must accept
these operations on their own terms.  To distinguish the imprecise
machine (floating point) operations from standard operations on the
field of real numbers, in the rest of this section (except in computer
code listings), place a dot over the floating point operations $(\dot
+)$, $(\dot -)$, and so forth.

The precise behavior of floating point operations on a computer is
governed by the IEEE-754 floating point standard \cite{Gol}.  By
giving a precise specification of properties of floating point
arithmetic, the standard makes it possible to reason about the
behavior of floating point.  It is possible to prove theorems about
the behavior of IEEE floating point.  For example, it is possible to
prove that a computer that correctly implements the standard should
return the values in reponse to the inputs given above (in the nearest
rounding mode).

Let $F$ be the set of machine-representable floating point numbers.
Assume that $F$ contains two special symbols $\pm\infty$.  The total
order on the real numbers is extended to $\ring{R}\cup\{\pm\infty\}$,
with $-\infty < x < \infty$ for all $x\in\ring{R}$.  Because of these
two special symbols, the floating point floor $x\to\floor{x}_F\in F$
and ceiling $x\mapsto\ceil{x}_F\in F$ functions, with domain
$\ring{R}$ can be defined.  In this section, we drop the subscript $F$
on the floor and ceiling functions.  We map the set of real numbers to
$F^2$, by sending $x$ to $[a,b]$, where $a = \floor{x}$ and
$b=\ceil{x}$.  If $x$ is Hastings, then $b$ is a point on the shore
north of Hastings.  It is the deliberate error to one side of the
target.





On most modern processors, the rounding mode can be set to directed
rounding, as described in the standard.  When the rounding mode is set
upward, the result of any basic arithmetic floating-point operation
$(\dot +)$, $(\dot -)$, $(\dot *)$, $(\dot /)$ applied to two floating
point numbers $(x,y)\mapsto x\dot\diamond y$ is defined by standard to
be the $\ceil{x\diamond y}$.  That is, make the calculation in the
field of real numbers and round up to the next floating point.  When
the rounding mode is set downward, set $x\dot\diamond y =
\floor{x\diamond y}$.  The notation $\dot\diamond$ does not show
rounding mode, but it should, because it is mode dependent.


Interval arithmetic, like the method of deliberate error, does not
seek to eliminate the sources of floating point round off error.
Rather it brings it under scientific control.


Let $I_F = \{[a,b] \in F^2 \mid a \le b\}$ be the set of floating
point intervals.  Basic machine operations extend to intervals.  The
sum $[a_3,b_3]$ of $[a_1,b_1]$ and $[a_2,b_2]$, is defined as
$a_3=\floor{a_1+a_2}$ and $b_3 = \ceil{b_1+b_2}$.  Write $[a_3,b_3] =
[a_1,b_1] \dot+ [a_2,b_2]$.  This addition of intervals is not
associative.  However, addition of intervals satisfy a crucial
inclusion property.  If $x\in[a_1,b_1]$ and $y\in [a_2,b_2]$, and $z =
x+y$, then $z\in [a_3,b_3]$.  The other arithmetic operations can be
extended to machine interval operations in a similar way, so as to
satisfy the corresponding inclusion properties.  Division requires
special treatment when $0$ belongs to an interval denominator.  We
will not go into those details here.

These operations are easily implemented in code.  For example, here is
the actual snippet of {\tt C++} code that implements the addition of
intervals.  The functions {\tt interMath::up()} and {\tt
  interMath::down()} set the rounding modes on the computer.
\begin{verbatim}
inline interval interval::operator+(interval t2) const
{
interval t;
interMath::up(); t.hi = hi+t2.hi;
interMath::down(); t.lo = lo+ t2.lo;
return t;
}
\end{verbatim}

By implementing interval operations on a computer, we can develop a
procedure that takes as input an arbitrary arithmetic expression over
the rational numbers and returns an interval $[a,b]$ with floating
point endpoints (augmented by $\pm\infty$ as usual) that contains the
rational value of that expression.  The true answer, still unknown,
lies trapped between the interval endpoints.  Since the associative
and distributive laws fail, the returned interval $[a,b]$ depends on
the actual syntax of the expresssion, on the placement of parentheses,
and so forth.



\subsection{analysis}

This essay is not a treatise on interval arithmetic.  Its purpose is
to give an introduction, to demonstrate the possibility of rigorously
bounding the error of floating point performed according to IEEE
standards.  We become increasingly sketchy.

The theory of interval analysis progresses step by step, starting with
basic arithmetic and developing through higher levels of analysis.  If
$p$ is a polynomial with rational coefficents
$p\in\ring{Q}[x_1,\ldots,x_n]$, then it has an interval extension
$\bar p : I_F^n \to I_F$.  Just as in the case of rational numbers,
the interval extension $\bar p$ depends on the syntax used to
expressed to polynomial $p$.  The polynomial extension is {\it
  monotonic}: if $z_i\in [a_i,b_i]$ for all $i$, then
$p(z_1,\ldots,z_n) \in \bar p([a_1,b_1],\ldots,[a_n,b_n])$.  Interval
extensions of rational functions is similar.

Consider a function $f:[a,b]\to\ring{R}$ that
has a rational function approximation $r$ with known
error bound:  $|f(x) - r(x)|<\epsilon$ for $x\in[a_1,b_1]$,
with $\pm\epsilon\in F$.
Define a monotonic interval extension $\bar f$ of $f$ by
$\bar f([c,d]) = \bar r([c,d]) \dot + [-\epsilon,\epsilon]$, for
$[c,d]\subset [a,b]$.  
Multivariate functions are similar.   
%
A large class of analytic functions fall within this framework.
We have interval extensions of trigonometric functions, logs, and
exponentials.  Interval extensions of real-valued functions can be
added, multiplied and composed.  

To prove that a function $f$ is positive on a rectangular domain
$[a_1,b_1]\cdots[a_n,b_n]$, compute the
value of an interval extension $[c,d]=\bar f([a_1,b_1]\cdots[a_n,b_n])$.
By the monotonic property of interval extensions, if $c>0$, then $f$
is positive on the domain.  If this is done naively, with the
first interval extension $\bar f$ that comes to mind, the image interval
$[c,d]$ may be so large that no worthwhile information results.
If we naively 
compute $x+(-x)$ by interval arithmetic for $x$ in the interval $[-1,1]$,
we find that $x+(-x)$ lies in the interval sum
$[-1,1]\dot + [-1,1] = [-2,2]$.  Useless!
There is a large mathematical literature describing various 
efficient algorithms
to compute image intervals $[c,d]$ with accuracy.
Kearfott's book is a recommended starting point, because it comes
close to describing the types of algorithms implemented for the
solution to the packing problem \cite{Kea96}. 



\subsection{archive}

Although this book is long, it represents only a fraction of the
solution of the packing problem.  The other resources that are needed
to understand the full solution are available on the internet.

There is an archive of several hundred inequalities that have been
proved by computer.  The list of inequalities can be found at
\cite{web}.\footnote{The archive of interval arithmetic inequalities
  appears at \url{http://flyspeck.googlecode.com/svn/trunk/}} The list
of inequalities are in computer readable form in the rigorous
mathematical syntax of HOL-Light (for Higher Order Logic).

For example, one of the inequalities from that file reads as follows.
\begin{verbatim}
let I_572068135=
all_forall `ineq 
[((square (#2.3)), x1, (#6.3001));
((#4.0), x2, (#6.3001));
((#4.0), x3, (#6.3001));
((#4.0), x4, (#6.3001));
((#4.0), x5, (#6.3001));
((#4.0), x6, (#6.3001))
]
((((tau_sigma_x x1 x2 x3 x4 x5 x6) -.  ((#0.2529) *.  
(dih_x x1 x2 x3 x4 x5 x6))) >. (--. (#0.3442))) \/ 
((dih_x x1 x2 x3 x4 x5 x6) <.  (#1.51)))`;;
\end{verbatim}
The first part of this snippet of code gives lower and upper bounds on
each of the six variables $x_1,\ldots,x_6$.  The final part of this
code gives an inequality (or rather a disjunction of two inequalities)
of nonlinear real-valued functions that holds on the given domain.
The $\#$ symbol marks exact decimal constants.  The functions {\tt
  dih\_x}, {\tt tau\_sigma\_x} are defined rigorously in a separate
file.  They correspond to the functions $\dih$ and $\tau$ in this
book.  The symbols for arithmetic operations are followed by periods
($*.$ and so forth) to distinguish them from the corresponding
operations on natural numbers.

The inequality carries a nine-digit identifier {\tt 572068135}.  This
number is a tracking number that can be used in a search engine to
locate everything known about a given inequality.  For example, if one
googles this number, the search engine returns nine matches related to
this inequality, including a preprint on the arXiv, the website of the
Annals of Mathematics, a Springer Link to the relevant issue of the
journal Discrete and Computational Geometry, a flyspeck discussion
group, the {\it C++} computer code proving the inequality, and the
output file from running that code.  A search on the pdf file of this
book links this inequality to Lemma~\ref{lemma:11.16}.  Every interval
arithmetic calculation in this book carries a nine-digit identifier,
for easy tracking.

Currently, the most convenient way to access the code that proves the
inequality is through the Google's Code Search, a custom search engine
for computer code.\footnote{The interval arithmetic code that proves
  the inequalities is available at
  \url{http://code.google.com/p/flyspeck/}.}


Further information about proving these inequalities by interval
arithmetic can be found in \cite{algorithm} and \cite{part1}.
%\indy{Index}{calc@\calc{123456789}}%

\subsection{code verification}

S. Ferguson and I put considerable effort into developing trustworthy
code.  The two of us made entirely independent implementations of
the code. %automated inequality proving by interval arithmetic. 
(We shared algorithms
but not source code.)  This allowed us to check our answers against each
other to ensure mutual consistency, 
and to eliminate certain sources of bugs.  We also made independent
implementations in Mathematica and {\tt C} of traditional floating point
versions of the functions used for the nonlinear
optimizations.  By requiring these different implementations to give
compatible answers, we eliminated further sources of bugs.

Each nonlinear inequality was also checked independently with the
nonlinear optimization package {\tt cfsqp}.  This is a collection of
{\tt C} routines that searches for the minimum of a smooth function on
a domain described by a system of constraints.  As is the case with
many nonlinear optimization packages, there is no guarantee that the
search will converge to the true global minimum of the function.  By
repeating the search with a large collection of initial values for the
search, it becomes more probable that the true global minimum will be
found.  In practice it works remarkably well.

This package was not used in any proofs.  The numerical testing with
{\tt cfsqp} was used to discover false inequalities before they were
shipped to the interval arithmetic prover for verification.  Those
that failed never shipped.  This extra level of testing adds an extra
level of robustness to this part of the proof.  Testing gives us
(nonrigorous) reasons to believe the inequalities, even if a bug
should appear in our interval arithmetic code.  This gives us some
hope that an undetected bug would be unlikely to affect the overall
design of the solution to the packing problem.


There are certain types of bugs that would be very difficult to
detect.  For example, since floating point arithmetic is not
associative, a misplaced pair of parentheses might throw a calculation
off by a machine epsilon.  This potential source of bugs is evidence
that the entire project was not sufficiently automated: the
parentheses should all have been worked out automatically.  To make
the calculations more robust, we have tried to design the collection
of inequalities so that they hold with a considerable margin of error,
rather than just squeeze by.  (There are only a few inequalities that
are sharp, and they are sharp for clear mathematical reasons related
to the theory of packings.)  In a bug-free environment, such
precautions would not be necessary.  Nonetheless, we take precautions.

As far as I know, no comprehensive efforts were made by the referees
and editors to check the correctness of the computer code before the
publication of the 1998 solution to the packing problem.  The editors'
preface to \cite{DCG} states that during the review process ``some
computer experiments were done in a detailed check.''


The flyspeck project is a long-term project intended to make the
solution to the packing problems one of the most thoroughly checked
computer proofs of all times.  Part of this project calls for a formal
proof of correctness of the computer code used in the interval
verification of inequalities.

Flyspeck is still far from completion in 2008.  Nevertheless, there
are various ongoing projects related to this second-generation
verification of the interval code.  S. McLaughlin has an independent
implementation of the interval arithmetic code used in the packing
problem~\cite{McL08}.  His work has exposed some data-entry bugs.
They are reported in the comprehensive errata to the 1998 proof, which
is maintained at \cite{errata} with additional discussion
at~\cite{flydis}.  Fortunately, no bugs have surfaced over the past
decade in the underlying interval arithmetic inequality proving
algorithms.  The reported errors have been at the data-entry level: a
mismatch between the data as typed into the preprint and data in
computer code.

One of the formal proof assistants under most active development is
the COQ system~\cite{COQ}.  R. Zumkeller has implemented automated
inequality proving with interval arithmetic inside the theorem prover
COQ, with the flyspeck project in mind; although (as of 2008) the
inequalities that are used in this book have not yet been checked in
this way \cite{Zu}.


\subsection{interval analysis and proof}


The editors of the Annals of Mathematics have posted a statement on
computer-assisted proof.  At first, the editors planned to make a
disclaimer directed at the computer solution of the packing problem.
Eventually, they formulated a general policy on computer-assisted
proofs.  The policy mentions interval arithmetic as one way to control
sources of computer error.


\begin{quote}
%Statement by the Editors on Computer-Assisted Proofs

  ``Computer-assisted proofs of exceptionally important mathematical
  theorems will be considered by the Annals.

  ``The human part of the proof, which reduces the original
  mathematical problem to one tractable by the computer, will be
  refereed for correctness in the traditional manner. The computer
  part may not be checked line-by-line, but will be examined for the
  methods by which the authors have eliminated or minimized possible
  sources of error: (e.g., round-off error eliminated by interval
  artihmetic, programming error minimized by transparent surveyable
  code and consistency checks, computer error minimized by redundant
  calculations, etc. [Surveyable means that an interested person can
  readily check that the code is essentially operating as claimed]).

  ``We will print the human part of the paper in an issue of the
  Annals. The authors will provide the computer code, documentation
  necessary to understand it, and the computer output, all of which
  will be maintained on the Annals of Mathematics website online.''
  \cite{Ann06}

%http://annals.princeton.edu/EditorsStatement.html
\end{quote}

A number of proofs in pure and applied mathematics have been based on
interval analysis.  W. Tucker implemented a rigorous ODE solver with
interval arithmetic and used it to prove that the Lorenz equations
have a strange attractor \cite{Tuc02}. The existence of strange
attractors is problem 14 on Smale's list of 18 Centennial Problems
\cite{Sma98}.  Another prominent problem solved by interval methods is
the double bubble conjecture, a generalization of the isoperimetric
problem in three dimensional Euclidean space.  A sphere gives the
solution to the classical isoperimetric problem.  The work of J. Hass,
M. Hutchings, and R. Schlafly shows that the surface area minimizing
way to enclose two regions of equal volume is the double bubble, which
consists of two partial spheres, separated by a flat circular disk
\cite{HHS95}.

Interval arithmetic has also yielded a number of new results on the
problem of packing circles in a square. M. Cs. Mark\'ot and T. Csendes
have obtained optimality proofs for packings of $28$, $29$, and $30$
circles in a square.  See Figure~\ref{fig:optimal-circles}.  This is
an area of active research. See, for example, \cite{Sza07} and
\cite{Mark07}.

%% WW not yet done.
\begin{figure}[htb]
\centering
\myincludegraphics{noimage.eps}
\caption{Optimal circle packings in a square}
\label{fig:optimal-circles}
\end{figure}



\subsection{note}

There are those who have tried to downplay the role of computers and
interval methods in the solution to the packing problem.  In fact,
they play an absolutely central role.  To segregate the computation
destroys the proof.  After writing the paper {\it The Sphere Packing
  Problem}, I had all but given up on solving the problem.  I had an
extremely difficult nonlinear optimization problem on my hands and no
rigorous mathematical method to solve it.  In the summer 1993, I
happened upon book on Pascal-XSC (a language extension of Pascal for
interval analysis) at the Seminary Coop Bookstore next to the
University of Chicago.  This book described the method I had lacked.
With fresh hope, in January 1994, I set aside all else and devoted
full effort to the packing problem.  The interval code was the most
difficult part of the computer code to implement because its speed was
crucial.  Thanks to the improvements of S. Ferguson, eventually the
code could run from beginning to end in about three months.  The
interval verifications were the last part of the proof to be completed
in August 1998.


\clearpage

\section{Inequality Listings}

%% XX MOVE ALL THIS ELSEWHERE.

\subsection{packings, general inequalities}

This appendix gives a summary of the nonlinear inequalities that have
been cited in the chapter on packings.  Information about the computer
verifications can be found at \cite{hales:2009:nonlinear}.


\begin{note}%XX
This book contains a number of nonlinear inequalities that have been
established by interval-arithmetic calculations by computer.  Some
of these interval arithmetic calculations are still in the process
of being verified.  The approach to the proof of the Kepler
conjecture described here is still work in progress.  A description
of the inequalities and their current status can be found
at~\cite{hales:2009:nonlinear}.
\end{note}


(Note that the following is an inequality in at most six variables; the most
difficult case to prove is that of a $4$-cell.)  Formulas for the
volumes and solid angles appear in Chapter~\ref{chapter:volume}.  An
explicit formula for the dihedral angle appears in
Chapter~\ref{part:trig}.


\begin{calculation}\label{calc:marchal}\guid{WJDLOCM}\guid{1025009205}\guid{3564312720}\rating{ZZ}
%% cc:mar are the k-cell estimates for non-cell clusters.
Define the function $M$ by equations (\ref{eqn:M}) and
(\ref{eqn:m-def}).  Define the function $\gamma(X,M)$ by equation
(\ref{eqn:gamma-def}).  If $X$ is a $0$, $1$, $2$, $3$, or $4$-cell,
then
\begin{displaymath}
\gamma(X,M)\ge 0.
\end{displaymath}
\end{calculation}

\begin{calculation}\label{calc:cc:qtr}\guid{GLFVCVK}\guid{4869905472}\guid{2477216213}\guid{8328676778}\rating{ZZ}
Let $\gamma_L$ be given by Definition~\ref{def:gammaL}, $\op{wt}$ by
Definition~\ref{def:wt}, and $\beta$ by Definition~\ref{def:beta}.
If $X$ is any $k$-cell that is not a quater with $k\in\{2,3,4\}$,
then % gammaL is nonneg on quarters. cc:qtr
\begin{displaymath}
\gamma_L(X) \op{wt}(X) + \beta(e,X)\ge 0.
\end{displaymath} 
\end{calculation}

\begin{calculation}\label{calc:cc:2bl}\guid{FHBVYXZ}\guid{1118115412}\rating{ZZ}
Let $\gamma_L$ be given by Definition~\ref{def:gammaL}.  Let $X$ be
any quarter.  Let $Y$ be a $3$-cell that flanks it.  Then
\begin{displaymath}
\gamma_L(X)+\gamma_L(Y)\ge 0,
\end{displaymath}
% 2-leaf calculation, gammaL(fourcell)+gammaL(threecell) >=0. % cc:2bl:
\end{calculation}

\begin{calculation}\label{calc:cc:5bl}\guid{ZTGIJCF}\rating{ZZ}
Let
\begin{displaymath}
a= 0.0560305, \quad\text{and}\quad  b= -0.0445813.
\end{displaymath}
\begin{itemize}
\item \case{1821661595} A $4$-cell $X$ along a spine $e$ satisfies
\begin{displaymath}
\gamma_L(X)\op{wt}(X) + \beta(e,X) \ge a + b\,\op{azim}(X),
\end{displaymath}
\item \case{7907792228} The $2$-cell $X_2$ and two $3$-cells $X_1,X_3$
that flank it along a spine $e$ satisfy
\begin{displaymath}
\sum_{i=1}^3 \left(\gamma_L(X_i)\op{wt}(X_i) + \beta(e,X_i)\right)\ge a + b\,\sum_{i=1}^3\op{azim}(X_i).
\end{displaymath}
\end{itemize}
\end{calculation}

\begin{calculation}\label{calc:cc:disks}\guid{8550443271}\rating{ZZ}
Let
\begin{displaymath}
g(h) = \arccos(h/2) - \pi/6.
\end{displaymath}
If $h_1,h_2\in [1,\hm]$, then
\begin{displaymath}
\op{arc}(2h_1,2h_2,2) - g(h_1) - g(h_2)\ge 0.
\end{displaymath}
\end{calculation}

\begin{calculation}\label{calc:cc:alin}\guid{7991525482}\rating{ZZ}
Let $L$ be given by Definition~\ref{def:L}.
Let
\begin{displaymath}
g(h) = \arccos(h/2) - \pi/6.
\end{displaymath}
Let
\begin{displaymath}
\op{reg}(a,k) = 2\pi - 2 k (\arcsin(\cos(a)\sin(\pi/k))).
\end{displaymath}
Then
\begin{displaymath}
\op{reg}(g(h),k) \ge c_0 + c_1 k + c_2 L(h),\quad
k = 3,4,\ldots,\quad 1\le h\le \hm,
\end{displaymath}
where
\begin{displaymath}c_0 = 0.6327,\quad c_1 = -0.0333,\quad c_2 =
0.4754.\end{displaymath}
\end{calculation}

\begin{calculation}\label{calc:cc:alin2}\guid{8540377696}\rating{ZZ}
Let $L$ be given by Definition~\ref{def:L}.
Let
\begin{displaymath}
g(h) = \arccos(h/2) - \pi/6.
\end{displaymath}
Let
\begin{displaymath}
\op{reg}(a,k) = 2\pi - 2 k (\arcsin(\cos(a)\sin(\pi/k))).
\end{displaymath}
Let
\begin{displaymath}a'=\arc(2,2,2\hm)-g(\hm) \approx
0.797.\end{displaymath} Then for $k=3,4,\ldots$,
\begin{displaymath}\op{reg}(a',k) \ge c_0 + c_1 k + c_2 L(1) +
c_3\end{displaymath}
where 
\begin{displaymath}c_0 = 0.6327,\quad c_1 = -0.0333,\quad c_2 =
0.4754,\quad c_3 = 0.85.\end{displaymath}
\end{calculation}

\begin{calculation}\label{calc:shorts}\rating{ZZ}
The following calculations involve many cases that are enumerated by
computer code.
\begin{itemize}
\item \case{BIXPCGW} Let $Z$ be any cell-cluster along a spine $e$
with three leaves.  Then
\begin{displaymath}
\Gamma(Z)> 0.
\end{displaymath}
\item \case{QITNPEA} Let $Z$ be any cell-cluster along a spine $e$
with four leaves.  Then
\begin{displaymath}
\Gamma(Z)> 0.
\end{displaymath}
\end{itemize}
\end{calculation}




\subsection{local fan: listing}

\begin{calculation}\guid{2065952723}\rating{ZZ}\label{calc:Lexell}
%See Mathematica numerical calculation.
Let
\begin{displaymath}
g(s;a,b,c,e_1,e_2,e_3) = \sum_{i=1}^3 \dih_i(2,2,2,a+s,b,c) e_i,
\end{displaymath}
where $\dih_i$ is given by Definition~\ref{def:tau}.
Let $\Delta = \Delta(4,4,4,a^2,b^2,c^2)$.
Let primes denote derivatives with respect to the variable $s$.
Assume that
$e_i\in\leftclosed1,1+\sol_0/\pi\rightclosed$,  that
$a,b,c\in\leftclosed2/\hm,4\rightclosed$.
Then
\begin{equation}\label{eqn:calc:Lexell}
  \Delta (g'(0;a,b,c,e_1,e_2,e_3))^2 
- 0.01\Delta^{3/2}g''(0;a,b,c,e_1,e_2,e_3) > 0.
\end{equation}
(The factors of $\Delta$ clear the denominator in
(\ref{eqn:calc:Lexell}) to simplify the inequality to be proved.)
\end{calculation}

\begin{calculation}\guid{2158872499}\rating{ZZ}\label{calc:2der}
%% checked in Mathematica NMaximize
Let $y_1,y_2\in \leftclosed 2,2\hm\rightclosed$.  
\begin{itemize}
\item 
Let $g(t) = \arc(y_1,y_2+t,2)$.  Then $g''(0) < 0$.
Explicitly,
\begin{displaymath}
  g''(0) = \dfrac{
    -64 + 48y_1^2 - 12 y_1^4 + y_1^6 
  + 80 y_2^2 - 8 y_1^2 y_2^2 - 3 y_1^4 y_2^2
    - 12 y_2 ^4 + 3 y_1^2 y_2^4 - y_2^6
  }{y_2^2 \sqrt{\ups(y_1^2,y_2^2,4)}^3}
\end{displaymath}
and the polynomial in the numerator takes negative values on the given
domain.
\item
Let $g(t) = \arc(y_1+t,y_2-t,2)$.  Then $g''(0) < 0$.
Explicitly,
\begin{displaymath}
  g''(0) = \dfrac{\sqrt{\ups(y_1^2,y_2^2,4)} \left(
      -4 y_1^2 + y_1^4 - 4y_1^3 y_2 - 4y_2^2 
   + 6 y_1^2 y_2^2 - 4 y_1 y_2^3 +y_2^4
    \right)}{y_1^2 y_2^2 (2+y_1-y_2)^2 (2+y_2-y_1)^2}
\end{displaymath}
and the polynomial in the numerator takes negative values on the given
domain.
\end{itemize}
\end{calculation}

\begin{calculation}\guid{2986512815}\rating{ZZ}\label{calc:cc:qua}  %m11
Let $y_1y_2,y_3,y_7\in \leftclosed 2,2\hm\rightclosed$,
$y_5,y_8,y_9\in \{2,2\hm\}$, $y_4,y_6\ge 2\hm$.
Let $x_i = y_i^2$.
Assume that
\begin{displaymath}
\Delta(x_1,x_2,x_3,x_4,x_5,x_6)>0,\quad{ and }
\Delta(x_3,x_2,x_7,x_9,x_8,x_4)>0.
\end{displaymath}
Assume that
\begin{displaymath}
\dih(y_3,y_1,y_2,y_6,y_4,y_5)+\dih(y_3,y_2,y_7,y_9,y_8,y_4) < \pi
\end{displaymath}
and\footnote{If $\{\v_1,\ldots,\v_4\}$ is a set of vectors such that
$y_i = \normo{\v_i}$ and $y_{ij} = \norm{\v_i}{\v_j}$, then
$\op{cross}(y_4,\ldots,y_{12}) = \norm{\v_2}{\v_4}$.}
\begin{displaymath}
\op{cross}(y_1,y_2,y_3,y_4,\ldots,y_9) \ge y_4.
\end{displaymath}
Let 
\begin{displaymath}g(t;y_1,\ldots,y_9) =
  \tau_{tri}(y_1,y_2,y_3,y_4+t,y_5,y_6)+\tau_{tri}(y_3,y_2,y_7,y_9,y_8,y_4+t).
\end{displaymath}
Then \begin{displaymath}g'(0)^2 - 0.01 g''(0) > 0.\end{displaymath}
\end{calculation}


\begin{calculation}\guid{EFJSUSK}\rating{ZZ}\label{calc:irred} %cc:tau
%%cc:par
Let $(V,E,F,G)$ be an irreducible minimal fan with parameters
$(r,s)$.  Then $\tau(V,E,F) \ge d(r,s)$.  (A separate calculation
has been made for each of the cases in the list given above.)
% Interval arithmetic
% calculations~ %% cc:par partition cases for tau[r,s].
\end{calculation}










%\begin{calculation}\guid{5779862781}\rating{ZZ}\label{calc:cc:d2a}
%  Let $y_5,y_6\in \{ 2,2\hm\}$, $y_1,y_2,y_3\in \leftclosed
%  2,2\hm\rightclosed$, and $y_4\in \leftclosed 2,4\hm\rightclosed$.
%  Let $g(t) = \tau_{tri}(y_1+t,y_2,y_3,y_4,y_5,y_6)$.  If
%  $\Delta(y_1^2,y_2^2,\ldots,y_6^2)> 0$ and if $ g'(0)=0, $ then
%  $g''(0)<0$.\footnote{The function $g(t)$ may fail to be
%    differentiable at $t=0$ for parameters $y_1,\ldots,y_6$ for which
%    $\Delta(y_1^2,\ldots,y_6^2)=0$.  Thus, it is necessary to work on
%    the noncompact domain $\Delta>0$ for this inequality.}
%%  (The proof is an interval arithmetic calculation over a
%%  four-dimensional space~cc:d2a.  %%cc:d2a
%%  The calculation verifies that the second derivative is negative
%%  whenever the derivative is zero.)
%\end{calculation} 
%

%\begin{calculation}\guid{6645853705}\rating{ZZ}\label{calc:cc:d2b}
%  Let $y_5\in \{2,2\hm\}$, $y_1,y_2,y_3\in \leftclosed
%  2,2\hm\rightclosed$, and $y_4\in \leftclosed 2,4\hm\rightclosed$.
%  Let $g(t) = \tau_{tri}(y_1+t,y_2,y_3,y_4,y_5,y_6)$.  If
%  $\Delta(y_1^2,y_2^2,\ldots,y_6^2)> 0$, if
%\begin{displaymath}
%\arc(y_1,2\hm,2) + \arc(y_2,2\hm,2) 
%\le y_6 \le \arc(y_1,2,2\hm)+\arc(y_2,2,2\hm)
%\end{displaymath} 
%and if
%$
%g'(0)=0,
%$
%then $g''(0)<0$.
%\end{calculation}
%
%\begin{calculation}\guid{5606476569}\rating{ZZ}\label{calc:cc:qua}
%Let $y_{12},y_{23},y_{34},y_2,y_3\in\{2,2\hm\}$.
%Let $y_1,y_4\in \leftclosed 2,2\hm\rightclosed$.
%Let $y_{14}\in\leftclosed 2\hm,y_1+y_4\rightclosed$.
%Let $y_{13}\in\leftclosed 2\hm,y_1+y_3\rightclosed$.
%Let $g(t) = \tau_{tri}(y_1+t,y_2,y_3,y_4,y_5,y_6)$.
%If 
%\begin{displaymath}
%\Delta(y_1^2,y_2^2,y_3^2,y_{23}^2,y_{13}^2,y_{12}^2)> 0 \text{ and }
%\Delta(y_1^2,y_4^2,y_3^2,y_{43}^2,y_{13}^2,y_{14}^2)> 0,
%\end{displaymath} 
%if\footnote{If $\{\v_1,\ldots,\v_4\}$ is a set of vectors such that
%  $y_i = \normo{\v_i}$ and $y_{ij} = \norm{\v_i}{\v_j}$, then
%  $\op{cross}(y_4,\ldots,y_{12}) = \norm{\v_2}{\v_4}$.}
%\begin{displaymath}
%\op{cross}(y_4,y_1,y_3,y_{13},y_{34},y_{14},y_2,y_{23},y_{12})\ge y_{13},
%\end{displaymath}
%and if
%$
%g'(0)=0,
%$
%then $g''(0)<0$.
%\end{calculation}
%

%%%%%%%%%%%%%%%%%%%%%%%%%%%%%%%%%%%%%%%%%%%%%%%%%%%%

\bibliographystyle{plainnat}
\bibliography{../tex/bibliography/all}

% shell:>makeindex index/Index
% shell:>makeindex index/Notation
%\printindex{index/Index}{General index}
%\printindex{index/Notation}{Notation index}

\end{document}
