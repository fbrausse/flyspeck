%%
% Author Thomas C. Hales
% LaTeX Format


%!TEX TS-program = latex    
%% This line is for TexShop. 

% Revision history. See svn.
% Document created Dec 6, 2002
% Revision started Jan 2007, from published DCG.

% Revisions Sept 2009: dependency on Tarski removed. 2 enclosed over quad removed. hypermap algorithm rewritten.  

%% XX unable to make the \part command work.
%% In pdf sidebar table of contents get stray markings in Appendix.
%% XX unable to get endnotes to work

\documentclass[spanningrule]{cambridge7A}
%% Cambridge University Press Macros from
%% https://authornet.cambridge.org/information/productionguide/laTex_files/

% required by CUP
\usepackage[numbers]{natbib}
\usepackage{rotating}
\usepackage{floatpag}
 \rotfloatpagestyle{empty}
\usepackage{amsthm}
\usepackage{graphicx}
\usepackage{multind}\ProvidesPackage{multind}
\usepackage{times}

% my additions
\usepackage{verbatim}
\usepackage{latexsym}
\usepackage{amsfonts}
\usepackage{amsmath}
\usepackage{crop}
\usepackage{txfonts}
\usepackage[hyphens]{url}
\usepackage{setspace}
\usepackage{ellipsis} % 
% http://www.ctan.org/tex-archive/macros/latex/contrib/ellipsis/ellipsis.pdf 
%\setstretch{2}  % for double spacing

% fonts
\usepackage[mathscr]{euscript} % powerset.
\usepackage{pifont} %ding
\usepackage[displaymath]{lineno}

%%\usepackage{fancyhdr}
%%\usepackage{mparhack}
%%\usepackage{edmargin} %endnotes

%% hyperref interferes with the \part macro of cambridge
\usepackage[letterpaper,colorlinks=true,%
  citecolor=red,%
  %breaklinks=true,%
  %pdftex,
  ps2pdf,%
  hyperindex=true]{hyperref}

% Also interferes with \part macro of CUP.
% http://tug.ctan.org/tex-archive/macros/latex/contrib/pdfcomment/doc/pdfcomment.pdf
\usepackage{pdfcomment}

%tikz graphics
\usepackage{xcolor}
\usepackage{tikz}
\usetikzlibrary{chains,shapes,arrows,trees,matrix,positioning,decorations,backgrounds,fit}

% wrapping graphics
% http://en.wikibooks.org/wiki/LaTeX/Floats,_Figures_and_Captions
\usepackage{wrapfig}





%\usepackage{tikz}

%
%\usepackage[paper size={90mm, 120mm},left=2mm,right=2mm,top=2mm,bottom=2mm,nohead]{geometry}
%\usepackage{amsthm}
%\usepackage{mathidx}
%%\usepackage{makeidx}
%%\usepackage{multicol}
%\usepackage{pdfsync}  %for TexShop sync.
%\usepackage{mparhack} %http://www.tex.ac.uk/cgi-bin/texfaq2html?label=marginparside
%%\usepackage{multind}
%\usepackage{hanging} creates a memory leak, don't use!
%\usepackage{zref}
%\usepackage{xkeyval}
%\usepackage{ifpdf}
%\usepackage{ifthen}
%\usepackage{calc}
%\usepackage{marginnote}


%%%%%%%%%%%%%%%%%%%%%%%%%%%%%%%%%%%%%%%%%

% This file contains local settings and system dependencies



% Auxiliary directories
\def\dsp{/Users/thomashales/Pictures/mathFigures/DenseSpherePackings}  % flypaper graphics
%\def\pdf{/Users/thomashales/Pictures/collect_geom} % tarski graphics
\def\pdfp{/Users/thomashales/Pictures/mathFigures/collect_geom} % kepler graphics

\def\showgraphics{t}  
% t: display graphics (there are none to show yet)
% f (default): print a "no graphics logo" where graphics would normally go.


\def\displayallproof{t} 
% t (default): display all proofs.
% f: print documents without the proofs-- theorem statements only

\def\displayrating{f}
% t (default): display all ratings (verbose is also true)
% f : don't show them.

\def\verbose{t}
% f (default): do not display debugging information,
% t : display debug information and information about the formalization.

     
%-%
% --Repository--
%-%
% generate revision number by
% svn propset svn:keywords "LastChangedRevision" kepmacros.tex
\def\svninfo{%
  Document TeXed on \today. \hfill\break
  Repository Root: https://flyspeck.googlecode.com/svn \hfill\break
  SVN $LastChangedRevision$
  }

%-%
% --Fonts--
%-%
\font\twrm=cmr8

%-%
% --Graphics--
%-%
%set \showgraphics option in flag.tex
% flypaper graphics
 \def\szincludegraphics[#1]#2{%
      \if\showgraphics t{\includegraphics[#1]{#2}}%
      \else{\includegraphics{noimage.eps}}\fi}

% % kepler graphics
% \def\pdffigtemplatex[#1]#2#3#4{%   
% % usage: \pdffigtemplatex[width=80mm]{file.eps}{labelname}{caption}
% \begin{figure}[htb]%
%   \centering
%  \szincludegraphics[#1]{\pdfp/#2}
%  \caption{#4}
%  \label{fig:#3}%
% \end{figure}%
% }

\def\tikzfig#1#2#3{%
\begin{figure}[htb]%
  \centering
\begin{tikzpicture}#3
\end{tikzpicture}
  \caption{#2}
  \label{fig:#1}%
\end{figure}%
}

\def\tikzwrap#1#2#3#4{%
\begin{wrapfigure}{r}{#4\textwidth}
  \begin{center}
\begin{tikzpicture}#3
\end{tikzpicture}
\end{center}
  \caption{#2}
  \label{fig:#1}%
\end{wrapfigure}%
}


%\def\pdfg#1#2#3#4{\if\showgraphics t{\pdffigtemplatex[#1]{#2}{#3}{#4}}\else{}\fi}
%\def\myincludegraphics#1{%
%      \if\showgraphics t{\includegraphics{#1}}%
%      \else{\includegraphics{noimage.eps}}\fi}


%-%
% --Footnotes and Endnotes--
%-%
% http://help-csli.stanford.edu/tex/latex-footnotes.shtml
%\long\def\symbolfootnote[#1]#2{\begingroup%
%\def\thefootnote{\ensuremath{\fnsymbol{footnote}}}\footnote[#1]{#2}\endgroup}

%-%
% --Special Formatting--
%-%
% http://en.wikibooks.org/wiki/LaTeX/Formatting#List_Structures
%\renewcommand{\labelitemii}{$\star$}
\renewcommand{\labelitemii}{$\circ$}
\renewcommand{\labelenumii}{\alph{enumii}}
\newenvironment{summary}
  {\begingroup\bigskip\narrower\noindent{\bf Summary.~}\it}
%  {~\ding{98}\par\phantom{!}\endgroup\bigskip}
  {~\par\phantom{!}\endgroup\bigskip}
\newenvironment{tidbit}{\smallskip\begingroup}{\endgroup\smallskip}
%\newenvironment{enumerate}
%  {\renewcommand{\labelitemi}{}\begin{itemize}}
%  {\end{itemize}}
\def\wasitemize{\relax}
\def\uncase#1{{\sc #1}}
\def\case#1{{\sc (#1)}}
\def\claim#1{{\it  #1}}
\def\calcentry#1#2#3#4{{\smallskip{\bf #1}\quad{\tt [#2]}\quad{(#3)}\quad {#4}}} % computer calc entry
\def\id#1{\ensuremath{\text{\tt #1}}}



%-%
% --Indexing, References, Citations--
%-%
\def\indy#1#2{\index{index/#1}{#2}}
%\def\eqn#1{{\bf (\ref{#1})}}   % deprecated, use \eqref.
\def\newterm#1{\indy{Index}{#1}{\it #1}\relax}
\def\oldterm#1{\indy{Index}{#1}{#1}\relax}
\def\cc#1#2{%
  \indy{Index}{computer calculation!{#1}}{\it computer calculation}%
  \ifverbose{\footnote{\guid{#1}  #2}}~\cite{website:FlyspeckProject}} % arg dropped.


%-%
% --Endnotes
%-%
%\renewcommand{\maketextnotes}{\global\textnotesontrue
%  \newwrite\textnotes
%  \immediate\openout\textnotes=\jobname.ent
% \literaltextnote{
%\notesheadername={\the\textnotesheadername}
%%\pagestyle{endnotesstyle}
%\mark{3}
%\label{textualnotes}
%\normalfont \backmattertextfont}
%}
%\newcommand{\shipnotes}{
%   \iftextnoteson
%   \theendnotes
%   \immediate\closeout\textnotes
%   \input \jobname.ent
%   \else
%   \relax
%   \fi
%}

%-%
% --Proof Display--
%-%
% set with \displayallproof in flag_fly. If f, then proofs are swallowed.
%% "proved" environment. toggle with \displayallproof
%
\def\hide#1{}
\def\swallowed{\relax}
\def\swallow#1\swallowed{}
\newenvironment{iproved}{}{}
\newenvironment{proved}{\resetproved\begin{iproved}}{\end{iproved}}
\def\hideproof{\renewenvironment{iproved}{%
   \centerline{\it -- Proof Proofed --}
  \renewenvironment{itemize}{}{}
  \renewenvironment{enumerate}{}{}
  \def\item{\relax}
  \catcode13=12
  \swallow
}{}}
\def\showproof{\renewenvironment{iproved}{\begin{proof}}{\end{proof}}}
\def\resetproved{\if\displayallproof t\showproof\else\hideproof\fi}



%-%
% --Debugging Information--
%-%
%% verbose:
\def\rating#1{\if\displayrating t%
  {{\textsc {[rating={\ensuremath {#1}}].\ }}}\else{}\fi}
\def\rz#1{\rating{#1}}
\def\cutrate{}
\def\oldrating#1{\if\displayrating t%
  {{\textsc {[former rating={\ensuremath {#1}}].\ }}}\else{}\fi}

\def\formalauthor#1{\if\verbose t{{\tt [formal proof by #1].\ }}\else{}\fi}
\def\dcg#1#2{{\if\verbose t%
  {{\tt{[DCG-#1]}}\indy{References}{ZC{#2 #1}@{DCG-#1}|page{#2}}}\else{}\fi}}
\def\tlabel#1{\label{#1}\if\verbose t{{\tt [#1].\ }%
   \indy{References}{#1|itt}}\else{}\fi}
\def\ifcverbose#1#2{\if\verbose t{{#1}}\else{#2}\fi}
\def\ifverbose#1{\ifcverbose{#1}{}}  %\verbose t{{#1}}\else{}\fi}
%\def\formal#1{\ifverbose{[{#1}]}}
\def\formal#1{\relax }
\def\formaldef#1#2{\ifverbose{\texttt{[{#1} $\leftrightsquigarrow$ {#2}]}}}
\def\footformal#1{\if\verbose t{\footnote{#1}}\else{}\fi}
\def\guid#1{{\tt[#1].\ }\indy{References}{ZA{#1}@{#1}|itt}}
\def\guid#1{\ifverbose{{\tt [#1]}}}
\def\guid#1{{{\tt [#1]}}}
\def\ineq#1{{{\tt  [#1]}}}
%\def\guid#1{{{\tt [#1]}}}

%\def\calc#1{{\textsc{calc-#1}}\indy{Interval}{{#1}@{#1}}}
%\def\xfootnote#1{\if\verbose t{\endnote{#1}}\else{\footnote{#1}}\fi}
%\def\xfootnote#1{\footnote{#1}}
%\def\xendnote#1{\if\verbose t{\endnote{#1}}\else{}\fi}


% margin notes
\setlength{\marginparwidth}{1.2in}
\def\mar#1{}
 %\ifverbose{\marginpar{\text{\raggedright\footnotesize #1}}}}
%\def\hypermark[#1]#2{\ifcverbose{\hyperref[#1]{#2}}{#2}}


%-%
%--Formatting--
%-%
\def\dfrac#1#2{\frac{\displaystyle #1}{\displaystyle #2}}
\def\textand{\text{ \ and \ }}  % for math eqns.

%-%
%--Redefining--
%-%
\def\emptyset{\varnothing}
\def\ups{\upsilonup} % Needs txfonts; else use \upsilon


%-%
% --Symbols--
%-%
% norm and brackets
\def\|{{\hskip0.1em|\hskip-0.15em|\hskip0.1em}}
\def\mid{\ :\ }
\def\tc{\hbox{:}}
\def\cooln{:\hskip-0.02em:}
\def\norm#1#2{\hbox{\ensuremath{\|#1\unskip-\unskip{#2}\|}}}
\def\normo#1{{\|#1\|}}
\def\sland{\ \land\ }
\def\abs#1{|#1|}
% brackets
\def\leftopen{(}
\def\leftclosed{[}
\def\rightopen{)}
\def\rightclosed{]}
\def\lp#1{{\llbracket{#1}\rrbracket}} 
%\def\comp#1{\llbracket #1 \rrbracket}
\def\comp#1{[#1]}
\def\tangle#1{\langle #1\rangle}
\def\ceil#1{\lceil #1\rceil}
\def\floor#1{\lfloor #1\rfloor}
%accents:
\def\=#1{\accent"16 #1}
\def\ast{\ensuremath{{}^*}}

% mathcal
\def\CalV{{\mathcal V}}
\def\CalL{{\mathcal L}}
\def\BB{{\mathcal B}}
\def\powerset{{\mathscr P}}

% mathbb
\newcommand{\ring}[1]{\mathbb{#1}}
%\def\N{{\mathbb N}}
%\def\Rp{\ring{R}^{3\,\prime}}
%\def\A{{\mathbf A}}
\def\F{{\mathbf F}} % map on faces H to H/{cal L}

% vector notation
\def\v{{\mathbf v}}
\def\u{{\mathbf u}}
\def\w{{\mathbf w}}
\def\e{{\mathbf e} }  
\def\p{{\mathbf p}}
\def\q{{\mathbf q}}

% operatorname
\def\op#1{{\operatorname{#1}}}
\def\optt#1{{\operatorname{{\texttt{#1}}}}}

%\def\opat{{\op{@}}}
\def\atn{\op{arctan\ensuremath{_2}}}
\def\azim{\op{azim}}
\def\nd{\op{node}}
\def\sol{\operatorname{sol}}
\def\vol{\op{vol}}
\def\dih{\operatorname{dih}}
\def\Adih{\operatorname{Adih}}
\def\arc{\operatorname{arc}}
\def\rad{\operatorname{rad}}
\def\bool{\operatorname{bool}}
\def\true{\op{true}}
\def\false{\op{false}}


%\def\orz{\varthetaup} % center of packing
\def\orz{{\mathbf 0}} % center of packing
\def\Wdarto{W^0_{\text{dart}}}
\def\Wdart{W_{\text{dart}}}
%\def\Wedge{W_{\text{edge}}}
\def\cell{\operatorname{cell}}
\def\dimaff{\operatorname{dim\,aff}}
\def\aff{\operatorname{aff}}
\def\card{\op{card}}

\def\del{\partial}
\def\doct{\delta_{oct}}
\def\dtet{\delta_{tet}}
\def\hm{{h_0}} % 1.26
\def\stab{c_{{\scriptstyle \text{stab}}}} % 3.01
\def\tgt{\operatorname{\it{target}}}
\def\pqr#1{#1} % marks type (p,q,r).
\def\trunc#1#2{#1\hbox{\ensuremath{[:\hskip-0.25em plus 0em minus 0em{#2}]}}}
\def\trunc#1#2{#1[\text{:}\hskip0em plus 0em minus 0em{#2}]}
\def\trunc#1#2{d_{#2}{#1}}
%\def\trunc#1#2{#1{[\le\hskip-0.25em{ #2}]}}

%% HYPERMAP macros:
% avoid e for both hypermap edge and edge {v,w}
\def\ee{\varepsilonup}
\def\ocirc{}
\def\wild{{*}}  % wildcard char.

%% PACKNG macros:
%\def\lam{\lambda}
%\def\Lam{\Lambda}
\def\bu{{\underline{\u}}}
\def\bv{{\underline{\v}}}
\def\bw{{\underline{\w}}}
\def\bV{{\underline{V}}}
%\def\arcs#1#2#3{{\arcV(#1,\{#2,#3\})}}

%% LOCAL FAN macros:
\def\smain{S_{\scriptstyle\text{main}}} 
 
 
% test file for tikz figures 

\tikzset{help lines/.style={very thin,gray}}

% for pdf version:
% XX remove background for final version.
\tikzset{background rectangle/.style={fill=gray!7,rounded corners=1ex}} % was blue!7, now bw
\tikzset{every picture/.style={show background rectangle}}
%\tikzset{background rectangle/.style={fill=blue!20,rounded corners=1ex}}




% line numbers
\def\lll{\resetlinenumber[1]}
\def\linenumberfont{\normalfont\small\sffamily}

\def\tocpart#1{
  \addcontentsline{toc}{part}{\Large{#1}}}
% (or even \LARGE)
  
\crop
%\makeindex
\makeindex{index/Notation}
\makeindex{index/Index}

\def\linput#1{\lll\input{#1}}


\raggedbottom  % for now.
%\raggedright  % don't worry for now.

%%% end notes -kill this.
%%\maketextnotes

%%%%%%%%%%%%%%%%%%%%%%%%%%%%%%%%%%%%%%%%

% new theorems

\theoremstyle{plain}
\newtheorem{theorem}[equation]{Theorem}
\newtheorem{lemma}[equation]{Lemma}

\newtheorem{background}[equation]{Background}
\newtheorem{corollary}[equation]{Corollary}
\newtheorem{example}[equation]{Example}
\newtheorem{assumption}[equation]{Assumption}
\newtheorem{interpretation}[equation]{Interpretation}
\newtheorem{conjecture}[equation]{Conjecture}

\theoremstyle{definition}
\newtheorem{definition}[equation]{Definition}

\theoremstyle{remark}
\newtheorem{remark}[equation]{Remark}
\newtheorem{notation}[equation]{Notation}
%\newtheorem{note}[equation]{Author's Note}
\newtheorem*{note}{Author's Note.}
\newtheorem{calculation}[equation]{Calculation}
\newtheorem{assertion}[equation]{Assertion}
\newtheorem{exer}[equation]{Exercise}

%%%%%%%%%%%%%%%%%%%%%%%%%%%%%%%%%%%%%%%%%

\begin{document}
\title[a blueprint for formal proofs]
    {%Flyspeck :
      Dense Sphere Packings}
\author{Thomas C. Hales}
    
%%%%%%%%%%%%%%%%%%%%%%%%%%%%%%%%%%%%%%%%%
%%% FRONT
    %\frontmatter
    \maketitle
    \tableofcontents
    %\thanks

%%%%%%%%%%%%%%%%%%%%%%%%%%%%%%%%%%%%%%%%%

   %\noindent



\bigskip




\begin{note}%XX
This manuscript is not for ready general distribution.  Please do not circulate it.
\begin{enumerate}\wasitemize 
\item The computer calculations that back up various claims have not
  been completed.  In particular, various nonlinear inequalities
  remain to be proved.  The linear programs for one of the hypermaps
  have still not terminated.
\item The figures are missing.
\end{enumerate}\wasitemize 
\end{note}

\bigskip\noindent %
This research has been supported by the National Science Foundation
under Grants 0503447 and 0804189 as well as a grant from the Benter
Foundation.

\bigskip\noindent\svninfo 

\newpage


   \newpage
   %\setcounter{chapter}{-1}
   %%------------------------------------------------------------
% Author: Thomas C. Hales
% Format: LaTeX
% Book Chapter: Dense Sphere Packings
%------------------------------------------------------------

\chapter*{Preface}

% Believe not everything, but only what is proven: the former is foolish, the latter the act of a sensible man. -- Democritus.

%{
%
%\narrower
%
%{\it ``A personality which has been veiled by a formal method
%  throughout many chapters is suddenly seen face to face in the
%  Preface.'' }
%% Introductory note, p3, Famous Prefaces, Harvard Classics, Vol
%% 39. P.F. Collier & Son, 1910.
%
%}

%\bigskip

%\centerline{\it ``Those who justify themselves do not convince.''
%  --Lao-Tzu}
%% quoted in A. Watts's essay in "Modern Buddhism" ed. Donald Lopez,
%% page 160


{

\narrower

{\it

  ``I think there's a revolution in mathematics around the corner. I
  think that $\ldots$ %in later times
  people will look back on the fin-de-siecle of the twentieth century
  and say `then is when it happened' (just like we look back at the
  Greeks for inventing the concept of proof and at the nineteenth
  century for making analysis rigorous). I really believe that. And it
  amazes me that no one seems to notice $\ldots$

  ``Never before have the platonic mathematical world and the physical
  world been this similar, this close. Is it strange that I expect
  leakage between these two worlds? That I think the proof strings
  will find their way to the computer memories?$\ldots$

  ``What I expect is that some kind of computer system will be
  created, a proof checker, that all mathematicians will start using
  to check their work, their proofs, their mathematics. I have no idea
  what shape such a system will $\ldots$ take. But I expect some
  system to come into being that is past some threshold so that it is
  practical enough for real work, and then quite suddenly some kind of
  `phase transition' will occur and everyone will be using that
  system.''

{\hfill--Freek Wiedijk \cite{FWR}} % http://www.cs.ru.nl/~freek/jordan/index.html

}

}

\newpage

{

\narrower\parindent=0pt
\parskip=0.4\baselineskip

{\it

Alecos: Christos has a problem with the `foundational quest'!

Christos:  Wrong!  I have two problems with your  {\rm {version}} of it!  One, it
didn't fail and, two, it wasn't a tragedy!  Granted, there are some tragic
parts!  But the ending is happy, as in the `Oresteia'!  

Apostolos:  Happy for whom?  Cantor, going insane?  G\"odel starving himself to
death out of paranoia? Hilbert or Russell and their psychotic sons? Or Frege with--

Christos: `The meaning
is in the ending!' you said so yourself!  So, follow the quest for ten more years and
you get a brand-new triumphant finale with the creation of the computer, which is the
quest's real hero!   Your problem is, simply, that you see it as a story of people!

Apostolos: Well, stories do tend to be about people!

Christos:  So, choose the right people!  And show what they really did!  All we we learn
of the great von Neumann is he said `It's over'  when he heard G\"odel!

Alecos: But it was over in a sense, wasn't it?  Pop went Hilbert's `no ignorabimus'!

Christos:  But then came the quest's jeune premier, its parsifal $\ldots$ Alan Turing!
He said `Ok, we can't prove everything! So, let's see what we can prove!' and to define
proof, he invented, in 1936, a theoretical machine which contains all the ideas of the
computer! $\ldots$ which, after the war, he and von Neumann, the quest's proudest sons,
brought to full life!

{\hfill--Logicomix} % page 303.

}

}




\newpage

{

\narrower\parindent=0pt
\parskip=0.4\baselineskip

{\it

``Ever hear of the Kepler Conjecture?''

``Nope.''

I laid the notebook on the table and flipped through the pages. ``It was first stated
in 1611 by Johannes Kepler,'' I said.  ``Kepler becomae interested in the problem while
he was corresponding with an Englishman named Thomas Harriot, who was trying
to help his friend Sir Walter Raleigh figure out the best way to stack cannonballs on ship
decks.  The goal was to find the densest possible spherical arrangements, $\ldots$ basically,
the way grocers stack oranges''

``Okay,'' he said, nodding.

``Kepler's conjecture {\rm{seems}} perfectly sound,'' I said.

``That is does,'' Ben said.

``But here's the thing. $\ldots$  I looked it up and discovered that, in 1998, a proof
had finally been put forward by an American mathematician named Thomas Hales.  In 2003,
a committee that had been assigned to verify Hales's work confirmed that they were ninety-nine
percent certain of the proof's correctness.  But that one percent was key.  The mathematical
world is still waiting for the publication of the data that will prove the Kepler Conjecture definitively.''

``Sucks for Thomas Hales,'' Ben said.

``I agree.  But it makes sense that they have to be certain, doesn't it?''


{\hfill--Michelle Richmond, No One You Know} % page 177.

}

}

%%DD Figure of cannonballs.

\bigskip

{

\narrower\parindent=0pt
\parskip=0.4\baselineskip

{\it

Sometimes fixing a $1$ percent defect takes $500$ percent effort.

{\hfill-- Joel Spolsky, Joel on Software} % page 122

}

}

\bigskip

{

\narrower\parindent=0pt
\parskip=0.4\baselineskip

{\it

Every one fully persuaded is a fool.

{\hfill-- Balthasar Graci\'an, the Art of Worldly Wisdon} % p110

}

}



\newpage

\section*{Blueprint for Formal Proofs}

In 1611, Kepler wrote a booklet in which he asserted that the familiar
cannonball arrangement of congruent balls in space achieves the
highest possible density.  No other arrangement fills a larger
fraction of space.  This assertion is the Kepler conjecture.  In 1900,
Hilbert made this conjecture part of his eighteenth problem.  This
book presents a proof of this assertion.

This assertion has become a test of the capability of computers to
deliver a reliable mathematical proof.  The original proof by Sam
Ferguson and me involved many long computer calculations that
exhausted the efforts of a team of referees.  This book represents my
efforts to redesign the proof in a way that makes the correctness of
the computer proof as transparent as possible.

After all is said and done, a proof is only as reliable as the
processes that are used to verify its correctness.  The ultimate
standard of proof is a formal proof.  A formal proof is nothing other
than an unbroken chain of logical inferences from an explicit set of
axioms.  While this may be the mathematical ideal of proof, actual
mathematical practice generally deviates significantly from the ideal.



%More than ten years have passed since a proof was first
%obtained. Why give a new presentation of the proof?
%
%The original proof was not widely understood.  The complexity was not
%because of conceptual challenges.  In fact, the proof makes only
%modest demands on the theoretical training of the reader.  It is
%possible to read and understand the proof with a knowledge of a
%limited body of mathematics, such as basic calculus and elementary
%Euclidean geometry.
%
%Nevertheless, the proof involves many long calculations. Even worse,
%it it relies on computer calculations.  An error in any calculation or
%a bug in the computer code has the potential to topple the entire
%proof.
%
%The referees were conscientious and checked many of the calculations.
%However, for the most part, the computer code lay beyond the scope of
%referee review, and even careful quality control can let a
%occasional bug slip through undetected.
%
%After all is said and done, no proof is more reliable than the
%reliability of the processes that are used to verify its
%correctness.  These processes include the checking that the author
%makes before releasing the proof for public scrutiny, the checking
%of the referees, and the checking done by readers after publication.

In recent years, as part of this project, I have been increasingly preoccupied by the
processes that mathematicians rely on to insure the correctness of complex
proofs. Researchers from Frege to G\"odel, who solved a problem of
rigor in mathematics, found a theoretical solution but did not
extinguish the burning fire at the foundations of mathematics
because they omitted the practical implementation. Some, such as
Bourbaki, have even gone so far as to claim that ``formalized
mathematics cannot in practice be written down in full'' and call
such a project
``absolutely unrealizable'' \cite[p 10,11]{Bour:68:Sets}. % Theory of
                                                          % Sets, page
                                                          % 10,11.

While it is true that formal proofs may be too long to print,
computers -- which do not have the same limitations as paper -- have
become the natural host of formal mathematics. In recent decades,
logicians and computer scientists have reworked the foundations of
mathematics, putting them in an efficient form designed for real use
on real computers.

For the first time in history, it is possible to generate and verify
every single logical inference of a major mathematical theorem.  This
has now been done for the four-color theorem, the prime number
theorem, the Jordan curve theorem, the Brouwer fixed point theorem,
and the fundamental theorem of calculus, among others.  Freek Wiedijk
reports that 82\% of a list of 100 famous theorems have now been
checked formally \cite{wiedijk:100}.  The list of 18 remaining
theorems contains two particular challenges: the independence of the
Continuum Hypothesis and Fermat's Last theorem.

Some mathematicians remain skeptical of the process because computers
have been used to generate and verify the logical inferences.
Computers are notoriously imperfect, with flaws ranging from software
bugs to defective chips.  Even if a computer verifies the inferences,
who will verify the verifier, or then verify the verifier of the
verifier?  Indeed, it would be unscientific of us to place an
unmerited trust in computers.

The choice comes down to two competing verification processes.  The
first is the traditional process of referees, which depends largely on
the luck of the draw -- some referees are meticulous, others are
careless.  The second process is formal computer verification, which
is less dependent on the whims of a particular referee.  In my view,
the choice between the conventional referee process and computer
verification is as evident as the choice between a sundial and an atomic
clock in contemporary science.

The boundary that separates an ``easy'' proof from a ``difficult''
proof shifts with current technology.  The introduction of steel in
architecture is not a mere reinforcement of wood and stone, it changes
the architect's world of possibilities.  There will no longer be any
reason to limit ourselves to ten-thousand-page proofs when our
technology supports million-page proofs.

The standard of proof I have adopted is the highest scientific standard
available by current technology.  That 
standard is formal verification by computer.  This standard
continues to evolve with the advancement of technology.

I dream of a fully formally verified solution to the
packing problem.  This project is still unfinished, but significant
progress is being made.  In this book, I rearrange the proof with
formal verification in mind .  The book is {\it a blueprint for formal
  proofs} because it gives the design of the formal proof to be
constructed.  My decisions about what to include in this book has been
shaped by the list of theorems already available in the library the
proof assistant {\tt HOL Light}.  For example, this book assumes basic
point-set topology and measure theory, which have been formalized by
John Harrison~\cite{HOLL}.

The style of formal proofs is different from that of conventional
proofs.  It is better to have a large number of short snappy proofs,
rather than a few intricate ones.  Humans enjoy surprising new
perspectives, but computers benefit from repetition and
standardization.  Despite these differences, I have worked to make
proofs that will bring pleasure to the human reader while providing
precise instructions for the implementation in silicon.



\section*{Structure of this Book}

The book is divided into parts.
The introductory part describe the major ideas, methods, and
organization of the proof.  

%There is an essay on each major computer
%component of the proof. The purpose is to provide a panoramic view of
%proof, to provide intuition about proof strategies.  After reading
%this part of the book, the reading should understand what the proof is
%all about, without yet dipping into technical details.
The part on foundations provides background material about
constructions in discrete geometry.    The first
of these chapters
covers trigonometric identities and basic vector geometry.  The second
treats volume from an elementary point of view.  The third chapter
covers planar graph theory from a purely combinatorial point of view.
The fourth chapter continues with planar graphs, now from a 
geometric point of view.

The next part of the book gives the solution to the packing problem.
The first chapter in this part gives a top-level overview of the major
steps of the proof.  It describes how the problem can be reduced from
a problem in infinitely many variables to a problem in finitely many
variables.  The remaining chapters in this part flesh out that
skeleton.

The final part of the book resolves some other longstanding conjectures in
discrete geometry: K. Bezdek's strong dodecahedral conjecture and Fejes
T\'oth's full contact conjecture.

Many simplifications of the original proof have been found over the past
several years.  The simplified proof is published here for the first time.
G. Gonthier expresses his formal proof of the four-color
theorem in terms of hypermaps.  He reworks the proofs of the
four-color theorem to avoid the use of the Jordan curve theorem, using
instead the much simpler notion of M\"obius contour.  I have followed
Gonthier's lead in these respects and also avoid the use of the Jordan curve theorem.

The optimality of the face-centered cubic packing is an assertion
about infinite space-filling packings.  For computational purposes, it is
useful to reduce the sphere packing problem to finite packings.  A
{\it correction term} is associated with each different reduction from
infinite packings to finite packings.  S. Ferguson and I considered a
large number of different correction terms.  We searched for one that
would simplify the computations as much as possible.  In a discussion
of the solution of the packing problem, I wrote that ``correction
terms are extremely flexible and easy to construct, and soon Samuel
Ferguson and I realized that every time we encountered difficulties in
solving the minimization problem, we could adjust $f$ [the correction
term] to skirt the difficulty. $\ldots$ If I were to revise the proof
to produce a simpler one, the first thing I would do would be to
change the correction term once again.  It is the key to a simpler
proof.''  C. Marchal has recently found a very simple 
correction term, that is, a simple way  to make the reduction from infinite packings
to finite packings.~\cite{marchal:2009}.  We use his reduction in this book.

There are many other improvements of the proof that are not visible in
the book, because they are implemented in computer code.  We have been
able to reduce the number of lines of computer code from over 187,000
to well under 10,000.  Needless to say, this significantly simplifies the 
formalization project.





\bigskip
\hbox{}



\bigskip
\hbox{}

{
\parindent=0pt
\obeylines

Thomas C. Hales
Pittsburgh, PA
May 2010

}








 
%%%%%%%%%%%%%%%%%%%%%%%%%%%%%%%%%%%%%%%%%
  \mainmatter
  \begin{runninglinenumbers*}
  
    \tocpart{Overview}
    %%------------------------------------------------------------
% Author: Thomas C. Hales
% Format: LaTeX
% Book Chapter: Dense Sphere Packings
%------------------------------------------------------------


\chapter{Close Packing}\label{sec:close}

\section{History}\label{sec:history}

This section gives a brief history of the study of dense sphere
packings.  Further details appear at \cite{Szpiro} and
\cite{Hales:2006:overview}.
The early history of sphere packings is concerned with the
face-centered cubic (FCC) packing, a familiar pyramid arrangement
of congruent balls used to stack cannonballs at war memorials and
oranges at fruit stands (Figure~\ref{fig:fcc-packing}).

\figDHQRILO % fig:fcc-packing


\subsection{Sanskrit sources}



The study of the mathematical properties of the FCC
packing can be traced\footnote{I am obliged to Plofker~\cite{Plo00}.} to a Sanskrit work (the \=Aryabha\d t\={\i}ya
 of \=Aryabha\d ta) composed around 499 CE.  The following passage gives
the formula for the number of balls in a pyramid pile with triangular base as
a function of the number of balls along an edge of the pyramid~\cite{Ary}:

% {\it \=Aryabha\d t\={\i}ya}, Ga\d nitap\=ada 21:

\bigskip

{\narrower\it\font\ninerm=cmr9

For a series [lit. ``heap''] with a common difference and
  first term of 1, the product of three [terms successively] increased
  by 1 from the total, or else the cube of [the total] plus 1
  diminished by [its] root, divided by 6, is the total of the pile
  [lit. ``solid heap''].  

}

\bigskip

 In modern notation, the passage gives two formulas for the number of
 balls in a pyramid with $n$ balls along an edge (Figure~\ref{fig:sanskrit}):
\begin{equation}\label{eqn:sanskrit}
\dfrac{n(n+1)(n+2)}{6} =  \dfrac{(n+1)^3 - (n+1)}{6}
\end{equation}

\figKSOEMIZ % fig:sanskrit

\subsection{Harriot and Kepler}

The modern mathematical study of spheres and their close packings can
be traced to Harriot.  His work -- unpublished, unedited, and largely
undated -- shows a preoccupation with sphere packings.  He seems to
have first taken an interest in packings at the prompting of Sir
Walter Raleigh.  At the time, Harriot was Raleigh's mathematical
assistant, and Raleigh gave him the problem of determining formulas
for the number of cannonballs in regularly stacked piles.  Harriot
interpreted the number of balls in a pyramid as an entry in Pascal's
triangle\footnote{Harriot was well-versed in Pascal's triangle long
  before Pascal.} (Figure~\ref{fig:pascal}). Through his study of
triangular and pyramidal numbers, Harriot later discovered finite
difference interpolation~\cite{BeS08}.  Shirley, Harriot's biographer,
writes that it was his study of cannonball arrangements in the late
sixteenth century that ``led him inevitably to the corpuscular or
atomic theory of matter originally deriving from Lucretius and
Epicurus'' \cite[p.~242]{Shi83}.

\figBDCABIA % fig:pascal

Kepler became involved in sphere packings through his correspondence
with Harriot around 1606--1607 on the topic of optics.
Harriot, the atomist, attempted to understand reflection and refraction
of light in atomic terms.  Kepler favored a more classical explanation of
reflection and refraction in terms of what Kargon describes as ``the union of two opposing
qualities -- transparence and opacity''~\cite[p.26]{Kar66}.  
Harriot was stunned that
Kepler would be satisfied by such reasons.

Despite Kepler's initial reluctance to adopt an atomic
theory, he was eventually swayed and  published an essay in  1611
that explores the consequences of a theory of matter composed of small
spherical particles. 
Kepler's essay describes the FCC packing and asserts
that ``the packing will be the tightest possible, so that in no other
arrangement could more pellets be stuffed into the same
container''~\cite{Kep66}.  This assertion has come to be known as the
Kepler conjecture.  This book
gives a proof of this conjecture.

\subsection{Newton and Gregory}

The next episode in the history of this problem,  a debate between
Isaac Newton and David Gregory,  centered on the
question of how many congruent balls  can be arranged to touch
a given ball.  The analogous question in two dimensions is readily answered;
six pennies, but no more, can be arranged
to touch a central penny.  In three dimensions, Newton said that the maximum was
twelve balls, but Gregory claimed that thirteen might be possible.

The Newton-Gregory problem was not solved until centuries later
(Figure~\ref{fig:musin}).  The first proper proof was obtained by van
der Waerden and Sch\"utte in 1953 \cite{Sch53}.  An elementary proof
appears in Leech \cite{Leech:1956:MG}.  Although a connection between
the Newton-Gregory problem and Kepler's problem is not obvious, Fejes
T\'oth successfully linked the problems in 1953~\cite{Fej53}.

\figPTFTWZM % fig:musin

\section{Face-Centered Cubic}



The FCC packing is the familiar pyramid arrangement of
balls on a square base as well as a pyramid arrangement on a
triangular base.  The two packings
differ only in their orientation in space.
Figure~\ref{fig:tri-square} shows how the triangular base
packing fits between the peaks of two adjacent square based pyramids.

\figNTNKMGO % fig:tri-square

Density, defined as a ratio of volumes, is insensitive to changes of
scale.  For convenience, it is sufficient to consider balls of unit
radius. This means that the distance between centers of balls in a
packing is always  at least $2$.  We identify a packing with its set $V$
of centers.   For our purposes, a packing is just a set of points
in $\ring{R}^3$ in which the elements are separated by distances of at least
$2$.



The density of a packing is the ratio of the volume occupied by the
balls to the volume of a large container.  The
purpose of a finite container is to prevent the volumes from becoming
infinite.  To eliminate the distortion of the packing caused by the
shape of the its boundary, we take the limit of the densities within an increasing
sequence of spherically shaped containers, as the diameter tends to infinity.

The FCC packing is obtained from a cubic lattice, by inserting a ball
at each of the eight extreme points of each cube and then inserting a
another ball at the center of each of the six facets of each cube
(Figure~\ref{fig:face-centered-cubic}).  The name
\fullterm{face-centered cubic}{FCC} comes from this construction.  The edge
of each cube is $\sqrt8$, and the diagonal of each facet is $4$.  The
density of the packing as a whole is equal to the density within a
single cube.  The cube has volume $\sqrt8^3$ and contains a total of
four balls: half a ball along each of six facets and one eighth a ball
at each of eight corners.  Thus, the density within one cube is
   \[ 
   \frac{   4 (4\pi/3)}{\sqrt8^3} = \frac{\pi}{\sqrt{18}}.
   \] 


\figTCFVGTS % fig:face-centered-cubic


%The tiling of regular tetrahedra and octahedra can be
%superimposed on the picture of the cube.  Each tetrahedron has an extreme point
%in common with the cube and three other extreme points at centers of facets
%of the cube.   One octahedron is concentric with the cube and has an extreme
%point at the center of each facet.  There is an
%additional quarter of an octahedron along each edge of the cube, extending to the
%midpoints of the two adjacent facets, making a total of eight
%tetrahedra and four octahedra.  As each octahedron has the volume of
%four tetrahedra, exactly $1/3$ of the cube is filled with tetrahedra,
%the other $2/3$ with octahedra.  This decomposition shows that the
%volume of a tetrahedron is $2\sqrt2/3$.
%%(pretend ignorance). The volume $16\sqrt2$ 
%%of the cube equals $24$ tetrahedra \dots, giving each a volume
%%of 
%%$2\sqrt{2}/3$.

The density $\pi/\sqrt{18}$ of the packing is the ratio of the volume
$4\pi/3$ of a ball to the volume of a fundamental domain of the FCC
lattice.  The volume of the fundamental domain is therefore
$4\sqrt{2}$.  A fundamental domain of the FCC lattice is a
parallelepiped that can be dissected into two regular tetrahedra and
one regular octahedron (Figure~\ref{fig:fcc-fun-domain}).  The FCC
packing is then an alternating tiling by tetrahedra and octahedra in
2:1 ratio.  A tetrahedron scaled by a factor of two consists of one
tetrahedron at each extreme point and one octahedron in the center
(Figure~\ref{fig:tet-oct-ratio}). By similarity, the total volume is
$8 = 2^3$ times the volume of each smaller tetrahedron. This
dissection exhibits the volume of a regular octahedron as exactly four
times the volume of a regular tetrahedron of the same edge length.  As
a result, the volume of a regular tetrahedron of side $2$ is $1/6$ the
volume of the fundamental domain, or $2\sqrt{2}/3$.

\figSEYIMIE % fig:fcc-fun-domain

\figAZGXQWC % fig:tet-oct-ratio

The density of the FCC packing is the weighted density
of the densities of the tetrahedron and octahedron.  Write $\dtet$ and
$\doct$ for these densities.  Explicitly, $\dtet$ is the ratio of the
volume of the part within the tetrahedron of the unit balls (at the
four extreme points) to the full volume of the tetrahedron.  As tetrahedra fill
$1/3$ of volume of the fundamental domain and an octahedron fills
the other $2/3$,
\[ 
  \frac{\pi}{\sqrt{18}} = \frac{1}{3}\dtet + \frac{2}{3}\doct.
\] 

As above, we identify a packing with the set $V$ of centers of the
balls.  The \fullterm{Voronoi cell}{decomposition!Voronoi} 
of a point $v$ in a packing $V$ is
defined as the set of all points in $\ring{R}^3$ (or more generally in
$\ring{R}^n$) that are at least as close to $v$ as to any other point
of $V$ (Figure~\ref{fig:voronoix}).  Each Voronoi cell of the FCC
packing is a rhombic dodecahedron
(Figure~\ref{fig:rhombic-dodec}),
which is constructed from an inscribed cube by placing a square based pyramid
(with height half as great as an edge of its square base) on each of
the six facets.

\indy{Index}{Voronoi|see{decomposition}}%

\figEVIAIQPx % fig:voronoix

\figPQJIJGE % fig:rhombic-dodec

%Recall that the cubes under discussion  have an edge length
%$\sqrt{2}$.
Rhombic dodecahedra, being the Voronoi cells of the FCC packing, tile space.
In each rhombic dodecahedron, we 
may color the inscribed cube black and the six square-based pyramids
white.  In the tiling, 
the black cubes fill the black spaces of an infinite three-dimensional
checkerboard, and the white pyramids fill the white spaces.

A Voronoi cell contains an inscribed black cube of side $\sqrt2$ and a total
of one white cube, for a total volume of $4\sqrt2$, which is
again the volume of the fundamental domain.  The density of the
FCC packing is the ratio of the volume of a ball to the volume
of its Voronoi cell, which gives $\pi/\sqrt{18}$ yet again.



\section{Hexagonal-Close Packing}\label{sec:hcp}

There is a popular and persistent misconception that the FCC
 packing is the only packing with density $\pi/\sqrt{18}$.
The hexagonal-closed packing (HCP) has the same density.
\indy{Index}{HCP}%
\indy{Index}{FCC}%


In the FCC packing, each ball is tangent to twelve others in the same
fixed arrangement.  We call it the \fullterm{FCC
  pattern}{FCC!pattern}.  Likewise, in the HCP, each ball is tangent
to twelve others in the same arrangement
(Figure~\ref{fig:fcc-hcp-pattern}).  We call it the \fullterm{HCP
  pattern}{HCP!pattern}.  The FCC pattern and HCP patterns are
different from each other.  In the FCC pattern, four different planes
through the center give a regular hexagonal cross section, while the
HCP pattern has only one such plane.

\figSGIWBEN % fcc-hcp-pattern

There are, in fact, uncountably many packings of density
$\pi/\sqrt{18}$ in which the tangent arrangement around each ball is
either the FCC pattern or the HCP pattern.

A \newterm{hexagonal layer} is a translate of the
two-dimensional hexagonal lattice (also known as the triangular
lattice). That is, it is a translate of the planar lattice generated
by two vectors of length $2$ and angle $2\pi/3$.  The FCC
 packing is an example of a packing built from hexagonal layers.

 If $L$ is a hexagonal layer, then a second hexagonal layer $L'$ can be
 placed parallel to the first so that each lattice point of $L'$ has
 distance $2$ from three different points of $L$,
%When the second
% layer is placed in this manner, 
 which is the smallest possible distance from first layer.  A choice
 of a unit normal vector $\e$ to the plane of $L$ determines an upward
 direction.  There are two different positions in which $L'$ can be
 closely placed above $L$
% and two different positions in which
% $L'$ can be placed closely below $L$ 
(Figure~\ref{fig:hex-layers}).  Each successive layer 
  ($L$, $L'$, $L''$, and so
 forth) offers two further choices  for the placement of
 that layer. Running through different 
 sequences of choices gives uncountably many packings.  In each of
 these packings the tangent arrangement around each ball is the FCC or HCP arrangement.

\figCCQCYWU % fig:hex-layers

As a packing is constructed, each layer may be labeled
$A$, $B$, or $C$ depending on three possible orthogonal projections to a fixed plane
with normal vector $\e$.
Each layer carries a different label from the layers immediately
above and below it.  In the FCC packing, the successive layers are
$A,B,C,A,B,C$, and so forth.  In the HCP packing, the successive layers are
$A,B,A,B$, and so forth.  If the vertices of a triangle are labeled $A$, $B$, and $C$,
then the succession of labels is a
walk along the vertices of the triangle, and inequivalent walks through the
triangle describe different packings.


The different walks through a triangle give all possible packings of
infinitely many congruent balls in which each tangent arrangement is
either the FCC pattern or the HCP pattern~\cite{CoSl95}.  To see that
there are no other possibilities, we first assume that every ball of
$V$ is surrounded by the FCC pattern.  Adjacent FCC patterns interlock
in a unique way that forces $V$ itself to crystallize into the FCC
packing.  This completes the proof in this case.

Now we assume that a packing $V$ contains some ball (centered at $\u$)
in the HCP pattern. Its uniquely determined
plane of reflectional symmetry contains $\u$ and the centers of six
others arranged in a regular hexagon. If $\v$ is the center of one of
the six other balls in the plane of symmetry, its  tangent arrangement
of twelve balls must include $\u$ and an additional four of the
twelve balls around $\u$. These five centers around $\v$ are not a
subset of the FCC pattern, but  extend uniquely to
a HCP pattern.   Around $\u$ and $\v$, the HCP patterns  have the same
plane of symmetry. In this way, as
soon as some center has the HCP pattern, the pattern
propagates along the plane of symmetry to create a hexagonal layer
$L$.

Once a packing $V$ contains a single hexagonal layer, the condition
that each ball be tangent to twelve others forces a hexagonal layer
$L'$ above $L$ and another hexagonal layer below $L$.  Thus, a single
hexagonal layer forces an infinite sequence of close-packed hexagonal
layers.  The position of each layer over the previous layer is described 
by the labels $A$, $B$, and $C$ of the triangle.
This completes the proof that the different walks through a triangle give
all possibilities.



\section{Gauss}

Gauss proved that the FCC packing has the greatest density
of any lattice packing in three-dimensional Euclidean space.  There is a
short proof that does not require any calculations.

\begin{proof}
Start with an arbitrary lattice $V$ in which every point has distance 
at least $2$ from every other.  Center a unit ball at each point in
the lattice.  In a lattice of greatest density, some pair of balls
touch.  The lattice property then forces the balls into parallel
infinite linear strings like beads on a string.  Two of these infinite
parallel strings touch if the lattice is  optimal.  The
lattice property then constrains the strings in parallel sheets.  On
each sheet the touching parallel strings form a rhombic tiling.  Each
parallel sheet sits as snugly as possible on the sheet below in an optimal
lattice.  In such an arrangement, a ball (centered at
$\v_0$) of one sheet touches three balls (centered at
$\v_1,\v_2,\v_3$) on a the next layer down (Figure~\ref{fig:rhombus}).

\figAFRJFRK % fig:rhombus


As the balls on each sheet form a rhombic tile, two of the distances
between $\v_1,\v_2,\v_3$, corresponding to two edges of the rhombus, are
equal to $2$.  This means that $\v_0$ together with two of
$\v_1,\v_2,\v_3$ form an equilateral triangle.  

From the perspective of the plane containing this equilateral
triangle, the lattice property forces this entire plane, as well as
parallel planes, to be tiled with equilateral triangles.  From the
earlier argument, each of these planes sits as snugly as possible on
the sheet below.  A ball of one sheet touches the three balls in an
equilateral triangle on the layer below.  These four balls form a
regular tetrahedron, which uniquely identifies the lattice as the FCC.
\end{proof}






\section{Thue}\label{sec:thue}


As mentioned in the preface, Thue solved the packing problem for
congruent disks in the plane.  The optimal packing is the hexagonal
packing (Figure~\ref{fig:2D-hex}).  The density of this packing is
$\pi/\sqrt{12}$, that is, the ratio of the area of a unit disk to the
area of a hexagon of inradius one.  Thue's theorem admits an
elementary proof that we sketch.  Casselman has an 
interactive demo of this solution \cite{casselman:pennies}.

\figOCULYIA % fig:2D-hex

\begin{proof}
Let $V$ be the set of centers of a collection of unit disks in
$\ring{R}^2$.  Take the Voronoi cell around each
disk.\footnote{Voronoi cells of packings in any dimension $\ring{R}^n$
  are defined by the same rule as we gave above for $\ring{R}^3$.}   It
is enough to show that each Voronoi cell has density at most
$\pi/\sqrt{12}$ because the limiting density of the packing in the entire plane cannot exceed
a bound on the density within a Voronoi cell.  


Truncate the Voronoi cell by intersecting it with a disk of radius
$r=2/\sqrt3$.   The density increases as the volume of the cell is made smaller,
so if the truncated Voronoi cell
has density at most $\pi/\sqrt{12}$, then so does the untruncated Voronoi cell.

There is not a point $\w$ in the plane that has distance  less than $r$
from three disk centers $\v_1,\v_2,\v_3$.  Otherwise, one of the three
angles $\gamma$ at $\w$ formed by pairs $(\v_i,\v_j)$ of points
 is at most $2\pi/3$, and $\cos\gamma\ge -0.5$.
The \newterm{law of cosines} applied to the triangle $\w,\v_i,\v_j$ with angle
$\gamma$ and sides $a$, $b$, and $c$ gives the contradiction
   \[ 
   4 \le c^2 = a^2 + b^2 - 2 a b \cos\gamma 
   \le a^2 + b^2 + a b < 3r^2 = 4.
   \] 
Thus, the boundary of the truncated Voronoi cell consists of circular
arcs and chords of the circle of radius $r$, as shown in Figure~\ref{fig:2D-proof}.

\figSENQMWT % fig:2D-proof

The parts of the Voronoi cell that lie within a circular sector have
density $1/r^2 = 3/4 < \pi/\sqrt{12}$.  A simple calculation shows
that the part of a Voronoi cell that lies within a triangle has
density
   \begin{equation}\label{eqn:rog2d}
   \frac{\theta}{r^2 \cos\theta\sin\theta}
   \end{equation}
% checked 3/31/2008.
for some $0 \le \theta\le \pi/6$.  An easy optimization gives the maximum
at $\theta=\pi/6$ with value $\pi/\sqrt{12}$.  This completes the proof of Thue's theorem.
\end{proof}

In some ways it it unfortunate that the problem in two dimensions is
so elementary.  It gives only meager hints about how to solve the problem
in three dimensions such as the value of Voronoi cells
and the usefulness of truncation.  The optimization problem on
triangles in Equation~\ref{eqn:rog2d} generalizes to $n$-dimensions.
But beyond these simple observations,  little from the proof of Thue's
theorem prepares us for higher dimensions.

%\subsection{two dimensions}

\bigskip

\indy{Index}{decomposition!Delaunay}%

There are other proofs of Thue's theorem, including one by Fejes
T\'oth that uses the \fullterm{Delaunay triangulation}{decomposition!Delaunay} 
of a packing $V$
in the plane (or in $n$-dimensions).  A Delaunay triangulation of $V$
is a triangulation of Euclidean space into simplices with extreme
points in $V$ such that no point of $V$ lies in the interior of any
circumscribing circle of any of the simplices (Figure~\ref{fig:delaunay}).  
If $V$ is
\newterm{saturated},\footnote{A packing $V$ is saturated if it is not
  a proper subset of any other packing $V'$.  To maximize density, it
  is useful to increase the density by saturating the packing with
  additional points.} then a Delaunay triangulation of
$V$ exists.  Each Delaunay triangle in a saturated packing $V$ has
circumradius at most $2$ because otherwise an additional point can be
placed at the center of the circumscribing circle, contrary to saturation.

\figANNTKZP % fig:delaunay

\begin{proof}
  By admitting the existence of a Delaunay triangulation, the proof of
  the packing problem for saturated packings $V$ in two dimensions becomes
  elementary.  Each Delaunay triangle contains a portion of a disk at each of
  its three vertices.  The three interior angles of a triangle sum to
  $\pi$, giving half a disk per triangle.  If we show that each triangle has area
  at least $\sqrt{3}$, then it follows that the density of the packing is at most
  $(\pi/2)/\sqrt{3} = \pi/\sqrt{12}$.  The problem thus reduces to
  an area minimization problem.  To decrease the area of a triangle
  $\{\v_0,\v_1,\v_2\}$, we first replace it with a smaller similar
  triangle with shortest edge (say $\v_1\v_2$) of length $2$.  The
  third vertex $\v_0$ is constrained to have distance at least $2$
  from $\v_1$ and $\v_2$, and to have circumradius at most $2$.  The
  constraints on $\v_0$ form three circular arcs as shown in
  Figure~\ref{fig:delaunay-proof}.

\figCCKQLLH  % fig:delaunay-proof


The minimizing triangle is determined by the point $\v_0$ closest to
the line through $\v_1$ and $\v_2$.  There are three such triangles,
each with area exactly $\sqrt3$.  This completes the proof.
\end{proof}

\section{Dense Packings in a Nutshell}

This section describes the proof of the Kepler conjecture in general,
without
getting embroiled in detail.  The entire book
is a blueprint with all the electrical schematics, plumbing, and
ventilation systems.  This section is the tourist brochure.

The Kepler conjecture asserts that no packing of congruent balls in
three-dimensional Euclidean space has density greater than the density
$\pi/\sqrt{18} \approx 0.74048$ of the FCC packing.  For a contradiction, we suppose that an
explicit counterexample exists to the Kepler conjecture in the form of a
packing of balls of radius $1$ with density  greater than
$\pi/\sqrt{18}$.  Additional balls may be added to this packing until
saturation is reached.  The saturation of a counterexample may push its
density even higher.

We present the proof in four stages.  Undefined terms are clarified in the
discussion that follows.

\begin{enumerate}
\item A geometric partition of space, adapted to a saturated
  counterexample $V$, reduces the problem to finite packing $W$ that
  gives a counterexample to a particular inequality.  In notation
  established below, the particular inequality is $\CalL(W,\orz)\le
  12$ for every finite packing $W\subset B(\orz,2.52)$.  The
  counterexample satisfies $\CalL(W,\orz)>12$.
\item The finite packing $W$ is transformed into another finite packing that violates the same
inequality and that has a few additional properties that make it a \newterm{contravening} packing.
\item The combinatorial structure of $W$ is encoded as a hypermap.  A list is
  made of the purely combinatorial properties of $W$.  A hypermap with
  these properties is said to be \newterm{tame}.
\item A computer generates an explicit list, enumerating
  tame hypermaps up to isomorphism.  Linear programs, which are
  adapted to each tame hypermap in the enumeration, certify that
  none of the combinatorial possibilities can be  realized geometrically as a finite packing
  $W\subset \ring{R}^3$.  
\end{enumerate}
\indy{Notation}{L1@$\CalL(V)$ (estimation of a packing)}%

From the nonexistence of a counterexample $W$, 
it follows that there is no saturated
counterexample $V$ to the Kepler conjecture.



\subsection{geometric partition}

The first stage of the proof defines a geometric partition of space
and uses it to reduce the Kepler conjecture to an optimization problem
in a finite number of variables.

We recall that a saturated packing is
identified with the discrete set $V$ of centers of the
congruent balls.  Also, as above, the Voronoi cell $\Omega(V,\v)$
associated with $\v\in V$ is the polyhedron formed by all points of
$\ring{R}^3$ that are at least as close to $\v$ as to any other $\w\in
V$.   

The Voronoi cell at $\v$ can be further partitioned into Rogers
simplices, each of which is determined by a facet of the Voronoi cell, an edge of
the facet, and an extreme point of the edge.  The Rogers simplex is defined to be the
convex hull of four points: $\v\in V$, the closest point $\v_1$ to $\v$ on the given facet, the closest point $\v_2$ to $\v_1$ on the edge, and the
extreme point $\v_3$ of the edge (Figure~\ref{fig:rogers-intro}).

\figORQISJR % fig:rogers-intro

We dissect and combine the Rogers simplices somewhat further to make
them into \fullterm{Marchal cells}{Marchal cell} (Figure~\ref{fig:marchal-intro}).  
The exact rules for the
construction of Marchal cells do not concern us here.  The rules
depend on which of the points $\v_1,\ldots,\v_3$ have distance less
than $\sqrt2$ from $\v$.
\indy{Index}{decomposition!Marchal}%

\figODGBUWK % fig:marchal-intro

The function
 $\CalL(V,\v)$ is defined as
\begin{equation}\label{eqn:LV0}
\CalL(V,\v) = \sum_{\w\in V} L(\norm{\w}{\v}/2),
\end{equation}
where $L$ is the piecewise linear function that has a linear graph from
$(x,y)=(1,1)$ to $(0,1.26)$ and is equal to zero for $x\ge 1.26$.  (The
constants $1.26$ and $2.52=2(1.26)$ appear throughout the proof as
parameters used in truncation.)   
The sum in the definition of $\CalL$ is actually finite for every packing $V$ because only finitely many terms
lie in the support of $L$. 

Next, a function $G:V\to \ring{R}$ is defined geometrically in terms
of the volumes, solid angles, and dihedral angles of Marchal cells.
We do not give the definition here because it is rather
complex.  The function $G$ has the following two fundamental
properties:
\begin{enumerate}
\item If $\CalL(V,\v)\le 12$, then 
\[
4\sqrt{2}\le \Omega(V,\v) +G(\v).
\]
\item There exists $C>0$ such that the points of $V$ in a ball $B(\orz,r)$
of radius $r\ge 1$ satisfy
\[
\sum_{\v\in V \cap B(\orz,r)} G(\v) < C r^2.
\]
\end{enumerate}
The constant $4\sqrt{2}$ is the volume of the Voronoi cell of the FCC packing.

From these fundamental properties and from the assumption that $V$ is a saturated counterexample,
it follows that $\CalL(V,\v)>12$ for some $\v\in V$.  Indeed, if $\CalL(V,\v)\le 12$ for all
$\v\in V$, then the fundamental properties
imply that on average the Voronoi cells of $V$ have volume at least that of the FCC packing, up to a negligible error term $C r^2$.  From this, it follows that the density
of the packing $V$ is at most that of the FCC packing.



Returning to the counterexample $V$, we pick $\v\in V$ such that
$\CalL(V,\v)>12$.  By the translational invariance of the problem, we
may assume that $\v=\orz$.  Then
\begin{equation}\label{eqn:LW}
\CalL(W,\orz) = \sum_{\w\in W} L(\normo{\w}/2)  > 12,
\end{equation}
where $W$ is the finite set  $\{\w\in V\mid 0 < \normo{\w} \le 2.52\}$.

This completes the first stage of the proof.
%, by reducing the Kepler conjecture to an optimization
%problem $\max_W \CalL(W,\orz) \le 12$ in a finite number of variables $W\subset B(\orz,2.52)$.
The counterexample $V$ to the Kepler conjecture leads to a finite packing $W$
that satisfies~\eqref{eqn:LW}.

\subsection{contravening packing}

We assume that $V$ is a counterexample to the Kepler conjecture and
that $W\subset V$ is a finite subset that satisfies \eqref{eqn:LW}.
The second stage of the proof shows that the finite packing $W$ can be
enhanced in various ways.  The result of the enhancement is a new
finite packing that is a \fullterm{contravening
  packing}{contravening}.  At this stage, we also make $W$ into a
graph by defining a set of edges $E$ with nodes in $W$.

For example, the value of $\CalL$
depends only on the norms $\normo{\w}$, and $L$ is a decreasing
function, so that any rearrangement of the points of $W$ that does not
increase the norms strengthens the inequality \eqref{eqn:LW}.

The finite packing $W$ determines a graph $(W,E)$ with node set $W$.  The set
of edges is defined by $\{\v,\w\}\in E$ if 
\[2\le\norm{\v}{\w}\le 2.52.\] This graph is called the \newterm{standard
  fan} of $W$.

We can get a crude idea about what $W$ must look like by studying the
set of normalized points $\w/\normo{\w}$ in the unit sphere.  These
points can be used to partition the unit sphere into spherical
polygons.  As we know that the sum of the areas of the polygons equals
the area $4\pi$ of the sphere, we can extract bits of information
about $W$ from estimates of the areas of the polygons.  Analysis along
these lines leads to the conclusion that some finite packing $W$
has the following 
properties:
\begin{enumerate}\wasitemize 
\item $W\subset B(\orz,2.52)$.
\item $\CalL(W) > 12$.
\item The cardinality of $W$ is thirteen, fourteen, or fifteen.
\item $W$ maximizes the function $\CalL$.
\item Join points $\v/\normo{\v}$ and $\w/\normo{\w}$ with a geodesic arc on the
unit sphere if $\{\v,\w\}\in E$.  Then the arcs do not meet except at the endpoints and
give a planar graph.  Moreover, the angle between each pair of consecutive arcs at a vertex is less
that $\pi$.  In particular, the spherical polygons cut out by the arcs are convex.
\end{enumerate}\wasitemize 
A finite packing $W$ with these properties is called a \newterm{contravening} packing.


\subsection{tame hypermap}

The starting point of the third stage of the proof is a contravening
packing $W$ and the corresponding planar graph $(W,E)$.  The result of
this stage is a \fullterm{tame hypermap}{tame!hypermap} (described below).

By definition, a \fullterm{planar graph}{planar!graph} is a graph that admits a
\newterm{planar} embedding.  On the other hand, a graph, endowed with
a fixed embedding into the plane, is a \fullterm{plane graph}{plane!graph}.  A
planar graph has too little structure for our purposes because it
does not single out a particular embedding and the plane graph has too
much structure because it gives a topological object where combinatorics
alone should suffice.  A hypermap gives just the right amount of
structure.  It is a purely combinatorial notion, yet encodes the
relations among nodes, edges, and faces determined by the embedding.
An entire chapter of this book is about hypermaps.


The graph $(W,E)$ of a contravening packing $W$ determines a planar hypermap
$\op{hyp}(W,E)$. We study the following question: what  purely
combinatorial properties of the hypermap $\op{hyp}(W,E)$ can be derived from
the assumption  that
$W$ is a contravening packing?  For example, the cardinality of a
contravening packing $W$ is thirteen, fourteen, or fifteen.  Hence, the hypermap
has thirteen, fourteen, or fifteen nodes.  Much of the later chapters of the book
revolve around the question of the combinatorial properties of the
hypermap.

The final chapter of the proof compiles all of these combinatorial
properties into a long list.  
Although the exact details of the list are not significant,
%The list of properties is every bit as
%artificial as a top ten list of world wonders or unsolved mysteries.
%This The idea is to produce -- with minimal effort -- any list of
the list of combinatorial properties severely constrains the set of possible
hypermaps.  

Any hypermap satisfying all of these properties is said to be
\newterm{tame}.  This list of properties appears in
Definition~\ref{def:tame}.

%There is little to be gained by extra efforts at this stage because
%the final stage gives more efficient means to constrain the set of
%possibilities.

\subsection{linear programming}

The fourth and final stage completes the proof the nonexistence of
the contravening packing $W$.  At the beginning of this stage,
$\op{hyp}(W,E)$ is a tame hypermap.  The list of defining properties
of a tame hypermap are sufficiently restrictive that an explicit finite
list can be generated of every tame hypermap, up to isomorphism.  
This list is generated by computer.  The details of the algorithm are
described in the chapter on hypermaps.

Equipped with an explicit list of possible combinatorial structures,
we move to the proof's end game.  At this stage, because of the computer
generated list of tame hypermaps, the cardinality and
combinatorial structure of $W$ are explicit.

A list is made of the properties of $W$ (and its associated hypermap)
that can be described by linear inequalities.  For each tame hypermap,
a computer solves one or more linear programs that test for feasible
solutions to the system of linear inequalities.  In each case, the
computer produces a certificate that shows that no feasible solution
exists.  It follows that no tame hypermap can be realized in the form
$\op{hyp}(W,E)$.  Each tame hypermap, which represents a
combinatorially feasible arrangement, is geometrical infeasible.  It
follows that $W$, and hence also $V$, do not exist.

As no counterexample exists, the proof of the Kepler conjecture ensues.

%\section{Gallery}

%This section explores the history of packings and coverings through a
%series of figures.

%Harriot (Pascal's triangle) -- Marchal 2D -- Marchal 3D -- Rogers's
%proof (2D) -- Roger's (3D) -- Fejes T\'oth's proof 2D -- Fejes
%T\'oth's proposal 3D -- Hsiang 3D -- dodecahedral conjecture --
%Delaunay simplices 3D (conjectured best) -- Hales 3D (superposition)
%-- Hales 3D hybrid -- Beth Chen (tetrahedra) -- covering problem 2D --
%covering 3D -- heptagons (Kuperberg) -- atom packings -- circle
%packings -- Tammes problem -- van der Waerden 13 (Musin)

    
    \tocpart{Foundations}
    %%


\def\odpcvgh{
\tikzfig{trig}{\guid{ODPCVGH} Trigonometric and inverse trigonometric
functions}
%
%arctangent function on the domain \leftopen -4,4\rightopen\ 
%and the $\arccos$ function on $\leftclosed-1,1\rightclosed$.}}
{
[scale=0.5]
\draw (-2*1.57,0) sin (-1.57,-1) cos (0,0) sin (1.57,1) cos (3.14,0) sin (3*1.57,-1);
\draw   (-2*1.57,-1) cos (-1.57,0) sin  (0,1) cos (1.57,0) sin (2*1.57,-1) cos (3*1.57,0); 
\draw[help lines,<->] (-3.3,0) -- (3*1.57 + 0.2,0);
\draw[help lines,<->] (0,-1) -- (0,2.0);
\draw plot[smooth] file {tikz/tan.table};
\node at (-0.5,-1.8) {$\tan$};
\node at (2,0.5) {$\sin$};
\node at (1.1,-0.3) {$\cos$};
% GG need axis labels and ticks, base points of labels should be precisely aligned.
\begin{scope}[xshift=10cm]
\draw plot[smooth] file {tikz/arctan.table} node[above] {$\arctan$};
\draw plot[smooth] file {tikz/arccos.table} node[right] {$\arccos$};
%\draw[gray,->,very thin] (-1.2,0) -- (1.4,0);
\draw[help lines,<->] (0,-1.6) -- (0,3.2);
\draw[help lines,<->] (-4,0) -- (4,0);
\end{scope}
}}

%
\def\sample{
\tikzfig{circle}{\guid{HGMTQFG} Lemma~\ref{lemma:circle} as a special case of the Pythagorean theorem}
{
[scale=0.1]
\draw (0,0)  --(12,0) --  (12,5) --  cycle;
\draw[very thin] (11,0) -- (11,1) -- (12.0,1);
\path (5,-1.5) node {$\cos x$};
\path (16,2.5) node {$\sin x$};
\path (6,5)  node {$1$};
}
}

%
\def\sample{
\tikzfig{tan}{\guid{GQQAKYI} 
The tangent function on $\leftopen-\pi/2,\pi/2\rightopen$}
{
[scale=0.2]
\draw plot[smooth] file {tikz/tan.table};
\draw[help lines,->] (-1.57,0) -- (1.57,0);
\draw[help lines,->] (0,-6.0) -- (0,6.0);
}
}

\def\sample{
\tikzfig{arctrig}{\guid{RUJPPWJ} 
The arctangent function on the domain \leftopen -4,4\rightopen\ 
and the $\arccos$ function on $\leftclosed-1,1\rightclosed$.}
{
[scale=0.4]
\draw plot[smooth] file {tikz/arctan.table};
\draw plot[smooth] file {tikz/arccos.table};
%\draw[gray,->,very thin] (-1.2,0) -- (1.4,0);
\draw[help lines,->] (0,-0.2) -- (0,3.2);
\draw[help lines,->] (-4,0) -- (4,0);
}
}

\def\sample{
\tikzfig{atn-polar}{\guid{YOXQFUB} 
The function $\atn$ gives the polar angle $\theta$ of $(x,y)$.}
{
[scale=0.15]
\draw[gray,->,very thin] (-4,0) -- (14,0);
\draw[gray,->,very thin] (0,-2) -- (0,5);
\draw (0,0)  --(12,0) --  (12,5) --  cycle;
\draw[very thin] (11,0) -- (11,1) -- (12.0,1);
\path (6,-1.5) node {$x$};
\draw[very thin] (4,0) arc (0:22.62:4);
\path (14,2.5) node {$y$};
\path (2,3) node {$\theta$};
}
}

\def\tuligly{
\tikzfig{M}{\guid{TULIGLY} The quartic polynomial $M$}
{
[scale=2.0]
\draw[help lines,<->] (1.0,0) -- (1.5,0);
\draw[help lines,<->] (1,-0.2) -- (1,1.2);
\draw plot[smooth] file {tikz/TULIGLY.table};
}}

\def\bjliekb{
\tikzfig{L}{\guid{BJLIEKB} Detail of the quartic $M$ and piecewise linear function $L$ on
the domain $\leftclosed1.2,1.35\rightclosed$}
{
[scale=12.0]
\draw plot[smooth] file {tikz/BJLIEKB.table};
\draw[help lines,<->] (1.18,0) -- (1.37,0);
\draw[help lines,<->] (1.2,-0.01) -- (1.2,0.25);
\draw[help lines] (1.23175,0.25) -- (1.23175,-0.01) node[anchor=north,black] {$h_-$};
\draw[help lines] (1.26,0.25) -- (1.26,-0.01) node[anchor=north,black] {$~~\hm$};
\draw[help lines] (1.3254,0.25) -- (1.3254,-0.01) node[anchor=north,black] {$h_+$};
\draw (1.2,0.230769 ) -- (1.26,0);  %
\draw (1.26,0) -- (1.35,0);
}}
%




    %\linput{trig}  
    %\linput{volume}
    %\linput{hypermap}
    %\linput{fan}

%%%%%%%%%%%%%%%%%%%%%%%%%%%%%%%%%%%%%%%%%
    
    \tocpart{The Kepler Conjecture}
    \linput{packing}
    %\linput{local}
    %\linput{tame}
    
    \tocpart{And Beyond}
    \linput{further}
    %% Branched made from latex/notices... on 5/5/2010.

\chapter{Formal Proof}

{\narrower\it 

  ``There remains but one course for the recovery of a sound and healthy
  condition -- namely, that the entire work of the understanding be
  commenced afresh, and the mind itself be from the very outset not
  left to take its own course, but guided at every step; and the
  business be done as if by machinery.'' --  F. Bacon, 1620, Novum Organum
% preface.

}

\bigskip


%%%%%%%%%%%%%%%%%%%%%%

\bigskip

\section{Bugs}

Daily,\footnote{This chapter was first published in the Notices of the AMS, December
  2008.} we confront the errors of computers.  They crash, hang,
succumb to viruses, run buggy software, and harbor spyware.  Our
tabloids report bizarre computer glitches: the library patron who is
fined \$40 trillion for an overdue book because a barcode is scanned
as the size of the fine; or the dentist in San Diego who was delivered
over sixteen thousand tax forms to his doorstep when he abbreviated
``suite'' in his address as ``su.''

% http://amartester.blogspot.com/2007/04/bugs-per-lines-of-code.html
  On average,
a programmer introduces 1.5 bugs per line while typing.
% The "private bug rate" of 1.5 per line refers to 
% "Beizer 1984" referred to on page 23 of "Testing Computer software" by Kaner et al.
Most are typing errors that are spotted at once.
About one bug per hundred lines of computer code ships  to
market without detection.  Bugs are an accepted
part of programming culture.
The book that describes itself as the ``bestselling software testing
book of all time'' states that ``testers shouldn't want to verify
that a program runs correctly''~\cite{KFN}.
Another book on software
testing states, ``Don't insist that every bug be fixed \dots.
When the programmer fixes a minor bug, he might create
a more serious one.''  Corporations may keep critical bugs
off the books to
limit legal liability.
 Only those bugs should be corrected
that affect  profit.
The tools designed to root out bugs are themselves
full of bugs. ``Indeed, test tools are often buggier than
comparable (but cheaper) development tools''
\cite{KBP}.
% Lessons Learned in Software Testing, Kaner, Bach, Pettichord 2001.
As for hardware reliability, former 
Intel President Andy Grove himself said, 
``I have come to the conclusion that no microprocessor is ever
perfect; they just come closer to perfection \dots.''
\cite[p.~221]{Mac}.
% quoted p221. MacKenzie, Mechanizing Proof.
%(at the time of the famous pentium bug) 


Bugs can be far-reaching.
The bug causing the 
explosion of the Ariane 5 rocket cost hundreds of millions
of dollars.  As long ago as 1854, Thoreau wrote that 
``by the error of some calculator
the vessel often splits upon a rock that should have reached
a friendly pier.''  % Walden, Economy, page 13.
In 2007, the {\it New York Times\/} reported Shamir's warning that
even a small math error in a widely used computer chip could 
be exploited to defeat cryptography and would
place
``the security of the global electronic commerce system at risk
\dots{}''~\cite{NYT}.
% Adding Math to List of Security Threats, New York Times, November 17, 2007.





\section{Mathematical Certainty}

By contrast, philosophers tell us that
mathematics consists of analytic truths,
free of all imperfection.  We prove that $1+1=2$ by
recalling the definition of $1$ as the successor of $0$,
$2$ as the successor of $1$, and then invoking twice the recursive
definition
of addition: 
  $$1+1 = 1 + S(0) = S(1 + 0) = S(1) = 2.$$

If only all proofs were so simple.  
Mathematical error is as old as mathematics itself.
Euclid's very first proposition asks, ``On a given straight line
to construct an equilateral triangle.''  Euclid's construction
makes the implicit assumption -- not justified by the axioms -- that
two circles, each passing through the other's center, must intersect.
We revere Euclid, not because he got everything right, but because
he set us on the right path.

We have entered an era of proofs of extraordinary complexity.
Take, for example, Almgren's masterpiece in geometric measure
theory, called appropriately enough the {\it ``Big Paper.'' }
%http://www.worldscibooks.com/mathematics/4253.html
The preprint is 1728 pages long. Each line is a chore. He spent over a
decade writing it in the 1970s and early 80s.  It was not published
until 2000.  Yet the theorem is fundamental.  It establishes the
regularity of minimizing rectifiable currents, up to codimension two;
in basic terms, it shows that higher dimensional soap bubbles are
smooth rather than jagged -- just as one would naturally expect.  How
am I to develop enough confidence in the proof that I am willing to
cite it in my own research?  Do the stellar reputations of the author
and editors suffice, or should I try to understand the details of the
proof?  I would consider myself very fortunate if I could work through
the proof in a year.

Computer proofs, which are sprouting up in many fields of mathematics,
compound the complexity: 
the non-existence of a projective plane of order ten,
the proof that the Lorenz equations have a strange attractor,
the double-bubble problem for minimizing soap bubbles enclosing
two equal volumes, the optimality of the Leech lattice among
24-dimensional lattice packings, hyperbolic three-manifolds,
and the one that got it all started: the four-color theorem.
% W. Tucker solved Smale's 14th problem by computer, establishing
%that the Lorenz equations have a strange attractor. 
%A. Kumar and H. Cohn
%The Search for a Finite Projective Plane of Order 10
%http://www.cecm.sfu.ca/organics/papers/lam/paper/html/node5.html
%http://arxiv.org/abs/math/9609207 Homotopy 3-manifolds.
What assurance of correctness do complex computer proofs provide?



\section{Formal Proof}

Traditional mathematical proofs are written in a way to make them
easily understood by mathematicians. Routine logical steps are
omitted. An enormous amount of context is assumed on the part of the
reader. Proofs, especially in topology and geometry, rely on intuitive
arguments that a trained mathematician would be capable
of translating into a more rigorous
argument.


A formal proof is a proof in which every logical inference has been
checked all the way back to the fundamental axioms of mathematics.
All the intermediate logical steps are supplied, without exception. No
appeal is made to intuition, even if the translation from intuition to
logic is routine. Thus, a formal proof is less intuitive, and yet less
susceptible to logical errors.

There is a wide gulf that separates traditional proof from formal proof.
For example, Bourbaki's Theory of Sets was designed as a purely theoretical
edifice that was never intended to be used in the proof of actual theorems.
Indeed, Bourbaki declares that ``formalized mathematics cannot
in practice be written down in full'' and calls such a project
``absolutely unrealizable.''  % Bourbaki, Elements of Sets, Addison-Wesley, Reading, MA, 1968,  pp.10,11.
The basic trouble with various foundational systems is that meta-mathematical arguments (for
example, abbreviations that are external to the system or
inductions over the syntactical form of an expression) 
are usually introduced early on, and without these simplifying meta-arguments,
the vehicle stalls, never making it up the steep incline from primitive notions to 
high-level concepts.   The gulf can be extreme.  In fact, Matthias has calculated
that to expand the definition of the number $1$ fully in terms of Bourbaki primitives requires
over four trillion symbols.
% A Term of Length 4,523,659,424,929 Synthese 133 (2002) 75--86.
In Bourbaki's view, 
the foundations of mathematics are roped-off museum pieces
to be silently appreciated, 
but not handled directly.

There is an opposing view that regards the 
foundational enterprise
as unfinished until it is realized in practice and written down in full.
This chapter sketches the current state of this endeavor.
It has been necessary to commence afresh and to retool the foundations
of mathematics for practical efficiency, while preserving
its reliability and austere beauty.  For anything beyond a trivial
proof, the number of logical inferences is so large that a computer is
used to ensure that no steps are omitted.   This endeavor raises basic questions
about trust in computers.  This chapter also places formal proofs within
a broader context of automating more general mathematical tasks.

As the art is currently practiced, each formal proof starts with a traditional
mathematical proof, which is rewritten in a greatly expanded form, in which all the
assumptions are made explicit and all cases are treated in full.
For example, a traditional mathematical proof might show that a graph is
planar by drawing the graph on a sheet of paper.  The expanded form of
the proof  replaces the picture by careful argument.  From the
expanded text, a computer script is prepared, which generates all
the logical inferences of the proof.  The transcription of a single traditional
proof into a formal proof is a major undertaking.

\bigskip
\noindent
\framebox{\parbox{4.2in}{
\smallskip
\centerline{\it Three Early Milestones}
\smallskip

1954 -- Davis programs the Presburger
algorithm for additive arithmetic into the
``Johniac'' computer at the Institute
for Advanced Study.  
Johniac proves that the sum of two
even numbers is even, to usher in the era of computer proof.

\smallskip

1956 -- The automation of Russell and Whitehead's
{\it Principia Mathematica\/} begins~\cite{NSS}.
By the end of 1959, Wang's procedure has generated proofs
of every theorem of the Principia in the predicate calculus~\cite{Wang}.

\smallskip

1968 -- De Bruijn designs
the first computer program to check the validity of general mathematical
proofs.  His program Automath eventually checks
every proposition in a primer that Landau writes for his
daughter on the construction
of real numbers as Dedekind cuts.  
% daughter: see http://www.cs.ru.nl/~freek/aut/aut-4.1-manual.pdf
% Landau as google book: http://books.google.com/books?id=U9B5FKvx3pYC
% Landau as pdf: http://www.cs.ru.nl/~freek/aut/
% The culminating result was the proof that $ii = - 1$.
\smallskip

}}
\bigskip

\bigskip
\noindent
\framebox{\parbox{4.2in}{ \smallskip \centerline{\it N. G. de Bruijn}

%% Mar 23, 2010. removed.
%    \smallskip On April 24, 2008, F. Wiedijk and I visited N. G. de
%    Bruijn at his home in Nuenen, shortly before his ninetieth
%    birthday. (Nuenen is the Dutch town where Vincent van Gogh lived
%    when he painted the {\it Potato Eaters}.)  We discussed Automath,
%    Brouwer, Heyting, and some of his coauthors (Knuth and Erd\"os).
%    De Bruijn has contributed to many fields of mathematics, including
%    analytic number theory, Penrose tilings, quasicrystals, and
%    optimal control.


\smallskip
De Bruijn indices give a notation that
eliminates all dummy  variables from formulas with
quantifiers: $\forall\,x.~P(x)$ becomes
$(\forall~P~1)$.  This notation solves the problem of
free variable capture.

\smallskip
De Bruijn observed that the ratio of lengths of a formal proof to
the corresponding conventional proof is remarkably
stable across different proofs.  The ratio, called the de Bruijn factor,
has become the standard benchmark to measure the overhead of a formal proof.



\smallskip
}}
\bigskip

\subsection{examples}

Computer proof assistants have been under development for decades (see
Box~``Early Milestones''), 
but only recently has it become practical to prove major
theorems formally.  The most spectacular example is Gonthier's formal
proof of the four-color theorem.  His starting point is the
second-generation proof by Robertson et al.  Although both the traditional
proof and Gonthier use a computer, the two computer
processes differ from one another in the same way that 
adding $1+1=2$ on a calculator differs from the mathematical
justification of $1+1=2$ by definitions, recursion, and a rigorous
construction of the natural numbers.  In short, a large logical gulf
separates them.  As a result of Gonthier's formalization, the proof of
the four-color theorem has become one of the most meticulously
verified proofs in history.

In recent years, several other significant theorems have been formally
verified. (See Table~\ref{table}.)  The table lists the theorems, which
proof assistant was used (there are many to choose from), the person
who produced a formal proof, and the mathematicians who produced the
original proof.  The Prime Number Theorem, which asserts that the number
of primes less than $n$ is asymptotic to $n/\log\,n$, has two
essentially different proofs: the elementary proof of Selberg and
Erd\"os and the analytic proof of Hadamard and de la Vall\'ee Poussin.
Formal versions of both proofs have been produced.  More ambitious
projects are in store: Gonthier's team is now formalizing the
Feit-Thompson odd order theorem, and the leading problem of the
document {\it Ten Challenging Research Problems for Computer Science}
is the formalization of the proof of Fermat's Last
Theorem~\cite{Berg}.
% J. Bergstra, 5 July 2005.






\smallskip

\begin{table}[ht]
\caption{Examples of Formal Proofs}
\centering
\begin{tabular}{l l l l l}
\hline
Year\hspace{0.5em} &Theorem\hspace{8em} &Proof System\hspace{2em}  &Formalizer\hspace{3em} &Traditional Proof\\ [0.5ex]
\hline \\
1986 &First Incompleteness &Boyer-Moore   &Shankar &G\"odel \\
1990 &Quadratic Reciprocity&Boyer-Moore &Russinoff &Eisenstein\\
1996 &Fundamental - of Calculus &HOL Light &Harrison &Henstock\\
2000 &Fundamental - of Algebra &Mizar &Milewski    &Brynski\\ % email tchales@gamil.com, Aug 17 from Milewski mentions  Brynski.
2000 &Fundamental - of Algebra &Coq &Geuvers et al.   &Kneser\\
2004 &Four Color &Coq &Gonthier &Robertson et al.\\
2004 &Prime Number &Isabelle &Avigad et al. &Selberg-Erd\"os\\
2005 &Jordan Curve  &HOL Light &Hales &Thomassen \\
2005 &Brouwer Fixed Point &HOL Light &Harrison &Kuhn \\
2006 &Flyspeck I &Isabelle &Bauer-Nipkow &Hales \\
2007 &Cauchy Residue &HOL Light &Harrison &classical \\
2008 &Prime Number &HOL Light &Harrison &analytic proof \\
% 2008 &Flyspeck II &Isabelle &Obua &Hales \\
 [1ex]
\hline
\end{tabular}
\label{table}
\end{table}
% Shankar 1986 dissertation:
% 1986 asserted in Shankar's book, Metamathematics, Machines and G�del's Proof (available as Google book, page xi.

% Quadratic Reciprocity: David M. Russinoff, A Mechanical Proof of Quadratic Reciprocity. J. Autom. Reasoning 8(1): 3-21 (1992).
% http://www.russinoff.com/papers/gauss.pdf
% 

% Fund. Th. of Alg: http://www.cs.ru.nl/~freek/100/
% Aug 21, 2000 for Mizar: http://mizar.uwb.edu.pl/JFM/pdf/polynom5.pdf
% Isabelle: 2008; http://isabelle.in.tum.de/library/HOL/HOL-Complex/Fundamental_Theorem_Algebra.html
% 
%
% A Constructive Proof of the Fundamental Theorem of Algebra without Using the Rationals
% See google book: http://books.google.com/books?id=wbHgOnNZNdYC
%
%Source	Lecture Notes In Computer Science; Vol. 2277 archive
%Selected papers from the International Workshop on Types for Proofs and Programs table of contents
%Pages: 96 - 111  
%Year of Publication: 2000
%ISBN:3-540-43287-6
%Authors	
%Herman Geuvers	
%Freek Wiedijk	
%Jan Zwanenburg	
%Publisher	
%Springer-Verlag  London, UK


The box~``Formal Jordan Curve Theorem'' 
displays the statement of the Jordan Curve theorem in computer
readable form as it appears in the formal proof.  The complete
specification of the theorem should also list all definitions, all the
way back to the primitives.  Without giving the detailed definitions
here, we note that {\it top2} refers to the standard topology on the
plane; {\it top2}~$A$ indicates that $A$ is an open set in the plane;
$\hbox{\it euclid}\,\,2$ is the Euclidean plane; and {\it connected
  top2}~$A$ means that $A$ is a connected set in the plane.

\bigskip
\noindent
\framebox{\parbox{4.2in}{
\parindent=0pt
\smallskip

\centerline{\it Formal Jordan Curve Theorem}
\smallskip

\vbox{
\def\hb{\hfill\break}
\def\h{\hbox{}}
\def\s#1{\hskip#1}
\def\w{\hskip0.65em}

\obeylines

  {\tt %let JORDAN\_CURVE\_THEOREM = prove\_by\_refinement(
  \h~~~$\forall C.\s0.4em \hbox{\it simple\_closed\_curve}\s0.3em\hbox{\it top2} \s0.3em C\w\Rightarrow$\hb
  \h~~~~~( $\exists A\, B.\s0.4em\hbox  {\it top2}\s0.3em A\w\wedge\w{\it top2}\s0.3em B\w\wedge$\hb
  \h~~~~~~~$\hbox{\it connected}\s0.3em\hbox{\it top2}\s0.3em A\w\wedge\w\hbox{\it connected top2}\s0.3em B\w\wedge$\hb
  \h~~~~~~~$A \ne \emptyset\w\wedge\w B \ne \emptyset\w\wedge$\hb
  \h~~~~~~~$A \cap B = \emptyset\w\wedge\w A \cap C = \emptyset\w \wedge\w B \cap C = \emptyset\w \wedge$\hb
  \h~~~~~~~$A \cup B \cup C =\hbox{\it euclid}\hskip0.3em 2$ )
  \h~~~%$\cdots$);;


}
\smallskip
}


}}
\medskip




A large library is maintained of all previously established proofs in
the system, and anyone may use any result that has been previously
established.  Although every step of every proof is always checked, as
researchers contribute to the system, interaction with the system
gradually moves away from the primitive foundations towards something
more closely resembling the high-level practice of mathematicians.
The hope is that proof assistants will eventually become sufficiently user-friendly to
become a familiar part of the mathematical workplace, much as email,
\TeX\relax, computer algebra systems, and web browsers are today.






\section{HOL Light}


This section gives a brief introduction to one foundational system
designed for doing mathematical proofs on a computer.  The system is
called \newterm{HOL Light}, an acronym for a lightweight implementation of
Higher Order Logic.  I have singled it out because of its simple
design and because it is the system that I understand the best.  Some
understanding of the design of a simple system is helpful before
turning to questions of soundness in the next section.  HOL Light by
itself is only a small part of the overall formal-theorem-proving
landscape.  There are several competing systems to choose from, built
on various logical foundations with their own powerful features.
People argue about the relative merits of the different systems much
in the same way that people argue about the relative merits of
operating systems, political loyalties, or programming languages.  To
some extent, preferences show a geographical bias: HOL in the UK,
Mizar in Poland, Coq in France, and Isabelle in Germany and the UK.


The basic components of the HOL Light system are its types, terms,
theorems, rules of inference, and axioms.  Each is briefly described
in turn.  The Box~``The HOL Light System'' % The HOL Light System
gives a summary of the entire system.

\subsection{types}

Much day-to-day mathematics is written at a level of abstraction that
is indifferent to its exact representation as sets.  For example, it
does not matter how an ordered pair is encoded as a set, as long as
the ordered pair has the characteristic property
 $$
 (x,y) = (x',y') \quad \Leftrightarrow\quad  x = x' \hbox{ and } y=y'.
 $$
 It is bad style to break the abstraction to write $2\in(0,1)$.  This
 layer of abstraction is good news because it allows us to shift from
 Zermelo-Fraenkel-Choice (ZFC) set theory to a different foundational
 system with equanimity and ease.

Many proof assistants are based on types.  Types are familiar to
computer programmers.  In a typed computer language, $3$ is an integer
and $[1.0;2.0;3.0]$ is an array of floating point numbers. An attempt
to add $3$ to this array results in a type mismatch error, and the
computer program will not compile.  The type checking mechanism of
programming languages conveniently detects many bugs at the time of
compilation.


ZFC set theory has no such type checking mechanism.  As de Bruijn
puts it,
``Theoretically, it seems perfectly legitimate
to ask whether the union of the cosine function
and the number $e$ (the basis of natural
logarithms) contains a finite geometry''~\cite{dbXY}.
% - N. G. de Bruijn, Types in Mathematics, page 29
Mathematicians have the good sense not to ask such questions.
However, when moving mathematics to a computer, which is lacking in
common sense, it is useful to introduce types into the foundations to
prevent this kind of nonsense.  By convention, a colon is written
before the name of a type.  For instance, we write the type of the
real number $e$ as $\tc\ring{R}$, or simply $e:\ring{R}$, to indicate
that $e$ is a real number.  The cosine function has a different type
$\tc\ring{R}\to\ring{R}$, or $\cos:\ring{R}\to\ring{R}$.  The type of
the union operator forces its two arguments to have the same type, so
that an attempt to take the union of the cosine function with $e$ is
then flat out rejected.

HOL Light is a new axiomatic foundation with types, different from the
usual ZFC.  The types are presented in
Box~``The HOL Light System.'' % The HOL Light System.
There are only two primitive types: the boolean type {\it \tc bool}
and an infinite type {\it \tc ind}. The rest are formed with type
variables joined by arrows.  A mechanism is also provided for creating
a new type that is in bijection with a nonempty subset of an existing
type, allowing the system to be extended with types for ordered pairs,
integers, rational numbers, real numbers, and so forth.

\subsection{terms}

Terms are the basic mathematical objects of the HOL Light system.  The syntax is based
on Church's $\lambda$-calculus, which uses the notation
   $$
   \lambda x.\ f (x)
   $$
   to represent the function that takes $x$ to $f(x)$, which a
   mathematician would write as $f:\ring{N}\to\ring{N}$, $x\mapsto
   f(x)$.  The name $\lambda$-calculus is derived from the use of the
   letter $\lambda$ to mark function arguments.
   The Box~``The HOL Light System''
   lays out the construction of terms.



In ZFC set theory, there is a bijection of sets
  $$
  Z^{X \times Y} \simeq (Z^Y)^X.
  $$
  In other words, a function $(x,y)\mapsto f(x,y)$ from the Cartesian
  product $X \times Y$ to $Z$ can be viewed as a function on $X$ that
  maps $x$ to a function $f(x,\cdot):Y\to Z$.  The right-hand side of
  this bijection is called the \newterm{curried form} of the function (named
  after the logician Haskell Curry).  In typed systems, the curried
  form of multivariate functions is generally preferred.  Treating
  $X,Y,Z$ as types, we write the type of the curried function as
  $f:X\to (Y \to Z)$ or simply $f:X\to Y\to Z$.

The system has only two primitive constants.  One of them\footnote{The
  second constant is the Hilbert choice operator $(\varepsilon)$,
  discussed below.  Recall that every term that is not a variable, a
  function application, or $\lambda$-abstraction is a constant.
  {\it Constancy'\/} is thus a broader notion here than in first-order logic
  and includes terms such as equality that take arguments.
  Parentheses are drawn around the equality symbol $( = )$ to denote
  the prefixed curried form, with $( = )\, x\, x$ as an alternative
  syntax for $x = x$.} is the equality symbol $( = )$ of type $\tc
A\to A\to bool$.  That is, equality is a curried function that takes
two arguments of the same type and returns the boolean type.




\bigskip
\noindent
\framebox{\parbox{4.2in}{ 
\parindent=0pt 
\smallskip 
\centerline{\it
      The HOL Light
      System%\footnote{\tt Note to editor: typeset the entire HOL Light system on a single page.}
    } % EDITOR
   %

    \smallskip 
{\bf HOL Light} (Lightweight Higher Order Logic) is a
    foundational system designed for doing mathematical proofs on a
    computer.  The notation is based on a typed $\lambda$-calculus.

   %
    \bigskip {\bf 1. Types:} The collection of types is freely
    generated from {\it type variables} $\tc A, \tc B,\ldots$ and {\it
      type constants} $\tc bool$ (boolean), $\tc ind$ (infinite type),
    joined by {\it arrows} $( \to )$.  The colon is used as a
    notational device to indicate a type.  For example, $\tc bool$,
    $\tc bool\to A$, and $\tc(bool\to A)\to (ind \to B)$ are types.

   %
    \bigskip {\bf 2. Terms:} The collection of terms is freely
    generated from {\it variables} $x,y$,\dots{} and {\it constants}
    $0,$\dots, using {\it abstraction} ($\lambda x. t$ where $x$ is a
    variable and $t$ a term) and {\it application} ($f(x)$ for
    compatibly typed terms $x$ and $f$).  Each term has a type.  The
    notation $x\tc A$ indicates that the type of term $x$ is $\tc A$.
    Variables and constants are assigned a type at the moment of
    creation; the types of abstractions and applications are defined
    recursively: the type of $\lambda x. t$ is $\tc A\to B$ when $x\tc
    A$ and $t\tc B$; the type of $f(x)$ is $\tc B$ if $f\tc A\to B$
    and $x\tc A$.  

    %
   \bigskip {\bf 3. Theorems:} A theorem is a {\it
      sequent} $\{p_1,\ldots,p_k\} \vdash q$, where $p_1,\ldots,p_k,q$
    are terms of type $\tc bool$.  The terms $p_1,\ldots,p_k$ are
    called the assumptions and $q$ is called the conclusion of the
    sequent.  The design of the system prevents the construction of
    theorems except through inferences from existing theorems, new
    definitions, and axioms.

}}

\bigskip
\noindent
\framebox{\parbox{4.2in}{ \parindent=0pt 

  %
 \smallskip
 {\bf 4. Inference
      Rules:} The system has ten inference rules and a mechanism for
    defining new constants and types. Each inference rule is depicted
    as a fraction; the inputs to the rule are listed in the numerator,
    and the output in the denominator.  The inputs to the rules may be
    terms or other theorems.  In the following rules, we assume that
    $p$ and $p'$ are equal, up to a renaming of bound variables, and
    similarly for $b$ and $b'$.  (Such terms are called
    $\alpha$-equivalent.)  

\quad On first reading, ignore the
    assumption lists $\Gamma$ and $\Delta$. They propagate silently
    through the inference rules but are really not what the rules are
    about.  When taking the union $\Gamma\cup\Delta$,
    $\alpha$-equivalent assumptions should be considered as equal.

    \smallskip 
%\protect\twocolumn %\framebox{ %\vbox{ 

\smallskip 

Equality is reflexive:  %\footnote{\tt Note to editor: Typeset rules in a double column format, four per column. } % EDITOR
    $$ \frac{a}{\vdash a=a} $$ 

Equality is transitive: 
$$ \frac{\Gamma
      \vdash a=b;~~~\Delta\vdash b'=c} {\Gamma\cup\Delta \vdash
      a=c} $$ 

Equal functions applied to equals are equal: $$
    \frac{\Gamma\vdash f=g;~~~\Delta\vdash a=b}
    {\Gamma\cup\Delta\vdash f\hskip0.1em a = g\hskip0.1em b} $$ 

The
    rule of abstraction holds. Equal terms give equal functions: $$
    \frac{x;~~~\Gamma\vdash a=b} {\Gamma \vdash \lambda x.\ a~=\lambda
      x.\ b} ~\hbox{\ (if $x$ is not free in $\Gamma$)} $$ 

The
    application of the function $x\mapsto a$ to $x$ gives $a$: $$
    \frac{(\lambda x.~a)\, x} {\vdash (\lambda x.\ a)\, x = a}
    $$ 

%}} %\pagebreak %\framebox{\vbox{ %%ASSUME 

Assume $p$, then conclude $p$: $$ \frac{p\tc bool} {p \vdash p} $$ 

An equality-based rule of modus ponens holds: %% EQ_MP 
$$ \frac{\Gamma\vdash p;~~~\Delta \vdash p'=q} {\Gamma\cup \Delta \vdash q} $$ 
% equivalent Harrison's but order reversed.  

If the assumption $q$ gives conclusion $p$ and the assumption $p$ gives $q$, then they are equivalent: $$ \frac{\Gamma \vdash p;~~~\Delta\vdash q} {(\Gamma\setminus q)\cup (\Delta\setminus p) \vdash p=q} $$ 

Type variable substitution holds.  If arbitrary types are substituted in parallel for type variables in a sequent, a theorem results.  

Term variable substitution holds.  If arbitrary terms are substituted in parallel for term variables in a sequent, a theorem results.  %}} %\protect\onecolumn 

}} % FRAMEBOX

\bigskip
\noindent
\framebox{\parbox{4.2in}{
\parindent=0pt


\bigskip {\bf 5. Mathematical Axioms:} There are only three mathematical axioms.  
$$\begin{array}{lll}
    \hbox{Axiom of Extensionality:} &\quad\forall f.\hskip1em(\lambda x.\, f\, x) = f.\\
    \hbox{Axiom of Infinity:} &\quad\exists f\tc ind\to ind.~~(\op{ONE\_ONE}\,f) \land \neg(\op{ONTO}\, f).\\
    \hbox{Axiom of Choice:}&\quad
    \forall P\,x.\hskip1em P x \Rightarrow  P(\varepsilon P).\\
  \end{array} $$ 

Extensionality asserts that every function is
  determined by its input-output relation. Dedekind's axiom of
  infinity asserts the existence of a function that is one-to-one but
  not onto.  The Hilbert choice operator $\varepsilon$ applied to a
  predicate $P$ chooses a term that satisfies the predicate, provided
  the predicate is satisfiable.



}} % FRAMEBOX
\bigskip



\subsection{axioms, inference, and theorems}


There are three mathematical axioms: an axiom of extensionality that
asserts that a function is determined by the values that it takes on
all inputs, an axiom of infinity that asserts that the type $\tc ind$
is not finite, and an axiom of choice.  The system has ten rules of
inference, as described in Box~``The HOL Light System.'' % The HOL Light System
For example, the first two state that equality is reflexive and
transitive.  The final two rules of inference allow one to substitute
new terms for the free variables in a theorem and allow one to
substitute new types for the type variables in a theorem.  Beyond
these ten rules of inference are mechanisms for defining new constants
and new types.  A theorem is expressed in {\it sequent} form; that is,
as a set of assumptions, followed by a conclusion.


\subsection{extending the primitive system}

This primitive system lacks the customary logical operators.  There
are no symbols for ``and'', ``or'', ``not'', and ``implies.''  There are no
universal or existential quantifiers.  The set membership operator is
absent.  It is remarkable none of this is needed to express the rules
of inference.

Logical operators are defined later.  For example, the boolean
constant $T$ (true) can be defined as the conclusion of any theorem
that has no assumptions.  The most accessible yet jarringly iconoclast
theorem comes from the reflexive law applied to equality itself:
$$
\vdash ( = ) ( = ) ( = ).
$$
Each new definition becomes a theorem.
So then $\vdash T = ((=) (=) (=))$.  Conjunction $( \land )$
is roundaboutly defined as the
curried function
 that on boolean inputs $p$ and $q$
returns $(\lambda f.\ f\, p\, q) = (\lambda f.\ f\, T\, T)$; that is,
conjunction yields  true exactly when no curried function $f$ is able to
distinguish $(p,q)$ from
$(T,T)$.
The other logical operations are built with similar tricks.

The inference rules and axioms
become bits of data that are processed by other computer procedures.
For example, to give a formal proof that 
$$
%% CHECKED May 31, 2008; Sep2.
2682440^4 + 15365639^4 + 18796760^4 = 20615673^4
$$
a human is not required to type each primitive inference.  An
automated procedure takes any arithmetic identity as input, generates
the inferences, and produces the theorem as output.  A large number of
such small decision procedures have been programmed into the system to
handle routine tasks such as polynomial simplification, basic
tautologies in logic, and decidable fragments of arithmetic.
Procedures that automatically search for steps in a proof are also
programmed into the computer.  New procedures may be contributed by
any user at any time to automate further tasks.  The design of the
kernel of the system prevents a rogue user from writing computer code
that could compromise the soundness of the system.


All the basic theorems of mathematics up through the Fundamental
Theorem of Calculus are proved from scratch on the user's laptop in
about two minutes every time the system loads, so that the casual user
does not need to be concerned with the low-level details.  Basic facts
of logic and elementary mathematics are simply there in the system to
be used as needed.


\section{Soundness}

HOL Light is both an axiomatic system for doing mathematics and a
computer program that implements the system.  How trustworthy is it?

If the computer is set aside for a moment, and the axiomatic system
alone analyzed, it is known to be consistent relative to ZFC.  That
is, an inconsistency in the HOL Light system would imply the
inconsistency of ZFC.



\subsection{computer implementation}

{\narrower\it  

  You've got to prove the theorem-proving program correct. You're in a
  regression aren't you?  --A. Robinson~\cite[p.~288]{Mac}.
% page 288, Chapter 8, MacKenzie.

}

\smallskip

The more pressing question is the soundness and reliability of the
computer program that implements the logic.  An earlier section
reported that a typical software program has approximately one bug per
100 lines of computer code.  The most reliable software ever created,
for example mission-critical software written for the space shuttle,
has fewer than one bug per 10,000 lines of computer code.  Various
proof assistants vary widely in reliability, ranging from some of the
world's most carefully crafted code at the upper end, to rubbish at
the lower end.  I confine my attention to the upper-end.
% http://amartester.blogspot.com/2007/04/bugs-per-lines-of-code.html



The computer code that implements the axioms and rules of inference is
referred to as the kernel of the system.  It takes fewer than 500
lines of computer code to implement the kernel of HOL Light.  (By
contrast, a Linux distribution contains approximately 283 million
lines of computer code.)
% http://en.wikipedia.org/wiki/Linux (Code size) In a later study, the
% same analysis was performed for Debian GNU/Linux version 4.0.[55]
% This distribution contained over 283 million source lines of code.
A bug anywhere in the kernel of this system might have fatal consequences.  For example,
if one of the axioms  is incorrectly typed, it might lead to an inconsistent system.




Yes, it is a regress, but a rather manageable one.  The kernel is
a tiny amount of computer code, but it verifies hundreds of thousands of lines of
code.  Eventually, it may verify
millions.  The same
kernel verifies everything from the prime number theorem to the
correctness of hardware designs.

Since the kernel is so small, it can be checked on many different
levels.  The code has been written in a friendly programming style for
the benefit of a human audience.  The source code is available for
public scrutiny.  Indeed, the code has been studied by eminent
logicians.  By design, the mathematical system is spartan and clean.
The computer code has these same attributes.  A powerful type-checking
mechanism within the programming language prevents a user from
creating a theorem by any means except through this small fixed
kernel.  Through type-checking, soundness is ensured, even after a
large community of users contributes further theorems and computer
code.  I imagine a poster\footnote{A T-shirt has already been
  made!}
% T-shirt Wiedijk email April 16, 2008 tchales@gmail.com
of the lines of the kernel,  taught in undergraduate courses, and
published throughout the world, as the bedrock of mathematics.  It is
math commenced afresh as executable code.

Experience from other top-tier theorem-proving systems has been that
about three to five bugs are found in each system over a period
of 15-20 years of use.  After decades of use on many different
systems, to my knowledge, only one proof has ever had to be retracted
as a result of bug in a theorem-proving system, and this in a system
that I do not rank in the top-tier: in 1995 a heap overflow error led
to the false claim that the theorem-prover REVEAL had solved the
Robbins conjecture. %% page 289, MacKenzie.
We can assert with utmost confidence that the error rates of top-tier
theorem-proving systems are orders of magnitude lower than error rates
in the most prestigious mathematical journals.  Indeed, since a formal
proof starts with a traditional proof, then does more
checking even at the human level, it would be hard for the outcome to
be otherwise.

As an extra check, Harrison gave what can almost be described as a
formal proof in HOL Light of its own soundness~\cite{HaSelf}.  To get
around the self-referential limitations imposed by G\"odel, he gave
two separate proofs.  In the first proof, a weakened version of HOL
Light is created, without the axiom of infinity.  The standard version
is used to give a formal proof of the soundness of the weakened
version.  In the second proof, a strengthened version of HOL Light is
created, with an additional axiom giving a large cardinal.  The
strengthened version then proves the standard version sound.  These
proofs go beyond traditional relative consistency proofs in logic in
two respects.  First, they are formal proofs, rather than
conventional proofs.  Second, the proofs establishthe
soundness not only of the logic, but also the underlying soundness of the
computer code implementing the logic.\footnote{The soundness of the
  computer code is considered relative to a semantic model of the
  underlying programming language.  This model may differ from the
  real-world behavior of the programming language, a reminder that the
  task of verification is never complete.}

\subsection{export}

In the past few years, a number of programs have been written to
automatically translate a proof written in one system into a proof in
another system.  If a proof in one system is incorrect because of an
underlying flaw in the theorem-proving program itself, then the export
to a different system fails, and the underlying flaw is exposed.
(Except of course, unless the second theorem-proving program also has
a bug that is perfectly aligned with the bug in the first system.
Since these systems are largely independently designed and
implemented, the events of failure in different systems are treated as
nearly independent, so that the probability of a perfect alignment of
failures across $n$ systems, goes to zero roughly as $p^n$, where $p$
is the individual failure rate.)

The soundness of HOL Light has been
exported by Adams~\cite{Adams}.  The
export is a formal proof within a second theorem-prover that
the HOL Light logic and implementation are sound.  It will soon be
within reach for several systems to give proofs of one another's
soundness.  When this is achieved, the probability of a false
certification of a pseudo-proof is pushed an order of magnitude closer
to zero.  With a computer -- indeed with any physical artifact,
whether a codex, transistor, or a flash drive made of proteins from
salt-marsh bacteria --
% bug proteins 
% http://www.getusb.info/50-terabyte-flash-drive-made-of-bug-protein/
% http://www.tomshardware.com/news/bacteria-drives-store-terabytes,3125.html
it is never a matter of achieving philosophical certainty.  It is a
scientific knowledge of the regularity of nature and human technology,
akin to the scientific evidence that Planck's constant $\hbar$ lies
reliably within its experimental range.  Technology can push the
probability of a false certification ever closer to zero: $10^{-6}$,
$10^{-9}$, $10^{-12}$\dots. The intent is that one day a system will
store a million proofs without so much as a misplaced semicolon.

A bug in the compiler, operating system, or underlying hardware has
the potential to compromise a formal proof.  To minimize such bugs,
formal proofs can be made about the correctness of the ambient
computational environment.  Indeed, verification of hardware design,
compilers, and computer languages has long been one of the principal
aims of formal methods.  HOL itself was initially created for hardware
verification.  As early as 1989, a simple computer system from
high-level language down to microprocessor was ``formally specified
and mechanically verified''~\cite{BHMY}.
%% quoted in MacKenzie page 243, ref 77.
Today, the semantics of various high-level programming languages have
been defined with complete mathematical rigor~\cite{Harper}.  In
recent work that is nothing short of spectacular, Leroy has
developed a formally verified compiler for the C programming
language~\cite{CC}.  (When the target of a formal verification is a
piece of computer code rather than a standard mathematical text, the
formalization checks that the computer code conforms to a precise
specification of the algorithm, certifying that the computer code is
bug free.)


\section{Full Automation}

Formal proofs are part of a larger project of automating all
mechanizable mathematical tasks, from conjecture making to concept
formation.  This section touches on the problem of fully automated
proofs -- the discovery of proofs entirely by computer without any
human intervention.  The next section briefly describes the ultimate
challenge of producing an automated mathematician.  Progress has been
gradual.  Fifty years ago, it was famously predicted that within a
decade ``a digital computer will discover and prove an important new
mathematical theorem.''
%  MacKenzie, page 89.  H. Simon and A. Newell 
This did not happen as scheduled.


Most success has been with the development of algorithms to solve
special classes of problems.  The WZ algorithm gives automated proofs
of identities of hypergeometric sums.  Gr\"obner basis methods solve
ideal membership problems.  Wu's geometry algorithm proves theorems
such as Pappus' theorem and Pascal's theorem on the ellipse.
% See Chou's article "Proving Elementary Geometry Theorems Using Wu's Algorithm" in *25 years.*
Tarski's algorithm  solves
problems that can be formulated in the first-order language of the real numbers.
The list of specialized algorithms is in fact enormous.

The most widely acclaimed example of a fully automated computer proof
is the solution of the Robbins conjecture in 1996.  The conjecture
asserts that an alternative definition is equivalent to the usual
definition of a Boolean algebra.  Remarkably, the
solution does not involve any human assistance, specialized
algorithms, or software designed with this particular problem in mind.
Just type the problem into W. McCune's general purpose theorem prover
{\it EQN}, hit return, and wait eight days for the solution to
appear~\cite{Mc1}, \cite{Mc2}.

Yet the story is only a qualified success.  It has remained almost an
isolated example, rather than the first in a torrent of results.  The
conjecture itself has the rather special form of a word problem in an
abstractly defined algebraic system -- a type of problem particularly
suited for computer search.  The proof that was found by computer can
be expressed as a short yet non-obvious sequence of
substitutions. (See box.) % EDITOR: Full Automation of the Robbins..


\bigskip
\noindent
\framebox{\parbox{4.2in}{ \smallskip \centerline{\it Full Automation
      of the \newterm{Robbins Conjecture}} \smallskip Let $S$ be a nonempty set
    with an associative commutative binary operation $(x,y)\mapsto xy$
    and a unary operation $x\mapsto[x]$ which, for convenience, we
    write synonymously as $x\mapsto \bar x$.  The Robbins conjecture
    (in Winker form) asserts that the general Robbins identity
   $$
   [[ab][a\bar b]] = a
   $$
   implies the existence of $c,d\in S$ such that $[cd]=\bar c$.  Here
   is the original proof that EQN discovered, as reconstructed
   in~\cite{fit}.
\begin{proof}  A solution is $c=x^3u$, $d=x u$, where $u=[x\bar x]$ and $x$ is arbitrary.
Abbreviate $j=[cd]$,  $e=u[x^2]\bar c$.  Over the equality sign, 
a prime indicates a direct application of the Robbins identity; a superscript
indicates a substitution of the numbered line; no superscript indicates a rewriting of abbreviations $c,d,e,j,u$.
$$
\begin{array}{lll}
%zero
 0: [u [x^2]] &= [[x\bar x][xx]] =' x.\\
%two
 1: [x u [x u [x^2]\bar c]] &=' [   [[x u x^2] [x u [x^2]]]  [x u [x^2]\bar c]] = 
       [  [\bar c [x u [x^2]]]   [\bar c x u [x^2]] ] =' \bar c.\\ 
%four
 2: [u\bar c]&= [u[x^2 u x]]=^0 [u[x^2 u[u[x^2]]]] =' [ [[u x^2][u [x^2]]] [x^2 u [u [x^2]]] ] 
  \\&='
   [u [x^2]] =^0 x.\\ 
%seven:
 3: [j u]&= [[xcu]u]=' [[xcu][[uc][u\bar c]]] =^2 [[xcu][x[cu]]] =' x\\
%eight
 4: [x[x[x^2]u\bar c]] &=' [  [[x[u\bar c]][x u \bar c]] [x [x^2]u \bar c]] =^2 [ [[x^2][x u \bar c]] [[x^2] x u\bar c]] =' [x^2]\\
%ten:
5: [x\bar c] &=^1 [x [x u [x u [x^2]\bar c]]] =^0  [[u [x^2] ] [x u [x u [x^2]\bar c]]]  
  \\&= [[u [x^2]] [u x[ x e]]] =^4 [  [u[x[xe]]][u x[xe]] ] ='     u\\
%thirteen:
6: [j x]&=' [j[[xc][x\bar c]]] =^{5} [j[[xc]u]] =[[uxc][u[xc]]]=' u\\
[1ex]
%
7:  [cd]&= j =' [[j[x\bar c]][jx\bar c]] =^{5} [[ju][jx\bar c]] =^3 [x[j x \bar c]] 
  =^2[[\bar c u][\bar c j x]]
  \\& =^{6} [ [\bar c [jx]][\bar c j x]] =' \bar c.
 \end{array}
$$
\end{proof}
}} 
\bigskip


Overall, the level today of fully automated computer proof (lying
outside special purpose algorithms) remains that of undergraduate
homework exercises: a group in which every element has order two is
necessarily abelian; a set is not in
bijection with its powerset (Cantor's theorem);
%
% JSTOR: 2004 Annual Meeting of the Association for Symbolic Logic
% E-mail: cebrown(andrew. cmu. edu. The Theorem Proving System TPS can
% be ... TPS can prove automatically are: THM 1 5B: If some iterate of
% function f has a ...  links.jstor.org/
% sici?sici=1079-8986(200503)11%3A1%3C92%3A2AMOTA%3E2.0.CO%3B2-V - Similar pages
if some iterate of a function has a unique fixed point, then the
function has a fixed point; the base $e$ for natural logarithms is
irrational~\cite{TPS},~\cite{Bee}.  Because of current limitations,
fully automated proof tools generally serve to fill in intermediate
steps of a larger formal proof.  They are not ready to take on the
Riemann hypothesis.


\section{Automated Discovery}

What happens if one sets aside rigor and lets a computer explore?  A
groundbreaking project was D. Lenat's 1976 Stanford thesis.  His
computer program \newterm{AM}, which is short for Automated Mathematician, was designed to
discover new mathematical concepts.  When AM was set loose to explore
in the wild, it discovered the concepts of natural number, addition,
multiplication, prime numbers, Pythagorean triples, and even the
fundamental theorem of arithmetic.  The thesis touched off a firestorm
of criticism and praise.

To put AM in context, consider a hypothetical program that is
instructed to discover new concepts by deleting conditions from the
list of axioms defining a finite abelian group.  The computer would
then immediately discover the concepts of infinite group, nonabelian
group, monoid, and so forth because these concepts all arise as
subsets of the axioms.  We might hear exaggerated claims that
%These discoveries could be sensationalized:
a program in Artificial Intelligence has made the ultimate leap
  from the finite to infinite, and from the abelian to the nonabelian,
  rediscovering fundamental concepts in seconds that mathematicians
  have grappled with for centuries.  There are nagging questions
about the emptiness of AM's discoveries; a suggestive representation
of the problem gives the answer away.

More recent projects stir the imagination, even if the field is still
young.  Computer programs have generated over one thousand conjectures
in graph theory, expressing numerical relationships between different
graph invariants.  One open conjecture is described in
Box~``An Open Computer-Generated Conjecture.'' % An Open Computer-Generated Conjecture
No technological barriers prevent us from unleashing conjecturing
machines in all branches of mathematics to see what moonshine they
reveal.


\bigskip
\noindent
\framebox{\parbox{4.2in}{ \smallskip \centerline{\it An Open
      Computer-Generated Conjecture} \smallskip Let $G$ be a finite
    graph with the following properties: \begin{enumerate} \item It
      has at least two vertices.  \item The graph is simple; that is,
      there are no loops or multiple joins.  \item It is regular; that
      is, every vertex has the same degree.  \item The graph is
      connected.  \end{enumerate} For example, the complete graph (the
    graph with an edge between every two vertices) on $n$ vertices has
    these properties, when $n\ge 2$.  Define the {\it total domination
      number} of $G$ to be the size of the smallest subset of vertices
    such that every vertex of $G$ is adjacent to some vertex in the
    subset.  The {\it path covering number} is the size of the
    smallest partition of the vertices into subsets, such that there
    exists a path confined to each subset $S$ that steps through each
    vertex of S exactly once, that is, the induced graph on $S$ has a
    Hamiltonian path.  \smallskip The computer program Graffiti.pc
    conjectures that {\it the total domination number of $G$ is at
      least twice the path covering number of $G$}.  For example, the
    complete graph on $n$ vertices has path covering number one
  because it has a Hamiltonian path.  Its total domination number is
    two (take any two vertices).  The conjecture is sharp in this case
    by these direct observations~\cite{DLPWW}.

\smallskip
}} 


\section{Flyspeck}

My interest in formal proofs grows out of a practical desire for a
thorough verification of my own research that goes beyond what the
traditional peer review process has been able to provide.  A few years
ago, I launched a project called \newterm{Flyspeck} to give a formal proof
of the Kepler conjecture, asserting that no packing of congruent balls
in three-dimensional Euclidean space can have density greater than the
density of the FCC packing (also known as the
cannonball arrangement).  The name Flyspeck, which quite appropriately
can mean to scrutinize, is derived from the acronym FPK for the
Formal Proof of the Kepler conjecture.


The original proof of this theorem was unusually difficult to check.
In a letter of qualified acceptance for publication in the {\it Annals
  of Mathematics}, an editor described the process, ``The referees put
a level of energy into this that is, in my experience,
unprecedented. They ran a seminar on it for a long time. A number of
people were involved, and they worked hard. They checked many local
statements in the proof, and each time they found that what you
claimed was in fact correct. Some of these local checks were highly
non-obvious at first, and required weeks to see that they worked
out\dots.{} They have not been able to certify the correctness of the
proof, and will not be able to certify it in the future, because they
have run out of energy to devote to the problem.''  In addition to a
three hundred page text, the proof relies on about forty thousand lines of
custom computer code.  To the best of my knowledge, the computer code
was never carefully examined by the referees.  The policy of the {\it
  Annals of Mathematics} states, ``The human part of the proof, which
reduces the original mathematical problem to one tractable by the
computer, will be refereed for correctness in the traditional
manner. The computer part may not be checked line-by-line, but will be
examined for the methods by which the authors have eliminated or
minimized possible sources of error\dots.''

Ultimately, the mathematical corpus is no more reliable than the
processes that assure its quality.  A formal proof attains a much
higher level of quality control than can be achieved by ``local
checks'' and an ``examination of methods.''


Flyspeck may take as many as twenty work-years to complete. Bauer,
Obua, and Zumkeller have already defended Ph.D. theses on the
project~\cite{Bauer:2006:Thesis}, \cite{obua:phd},
\cite{Zumkeller:2008:Thesis}.  Together with the work of their advisor
Nipkow, who is one of the principal architects of the Isabelle proof
assistant, nearly half of the computer code used in the proof of the
Kepler conjecture is now certified.

\section{Quod Erat Demonstrandum}


The Flyspeck project is a minute speck in the overarching QED project,
an anonymous manifesto declaring that all significant mathematical
results should be preserved in a vast library of formal proofs.  The
labor required to realize such a library would be staggering.  In
1991, de Bruijn proposed an assembly line to turn mathematical ideas
into formally verified proofs~\cite{dB91}.  The standard benchmark for
the human labor to transcribe one printed page of textbook mathematics
into machine verified formal text is one week.  To undertake the
formalization of just one hundred thousand pages of core mathematics
would be one of the most ambitious collaborative projects ever
undertaken in pure mathematics, the sequencing of a mathematical
genome.  One might imagine a massive wiki collaboration that settles
the text of the most significant theorems in contemporary mathematics
from Poincar\'e to Sato-Tate.

Outsourcing is the brute force solution to the QED manifesto.  Most
researchers, however, prefer beauty over brute force; we may hope for
advances in our understanding that will permit us someday to convert a
printed page of textbook mathematics into machine verified formal text
in a matter of hours rather than after a full week's labor.  As long
as transcription from traditional proof into formal proof is based on
human labor rather than automation, formalization remains an art
rather than a science.  Until that day of automation, we fall short of
a practical understanding of the foundations of mathematics.



\section{Recommended Reading and Software}

By far the best overview of the subject is the book {\it Mechanizing
  Proof,} winner of the 2003 Merton Book Award of the American
Sociological Association~\cite{Mac}.
% Review by Hayes: http://www.americanscientist.org/template/BookReviewTypeDetail/assetid/12866.
The QED manifesto can be found at~\cite{QED}.
Historical surveys include~\cite{Bled},
%``Automated Theorem Proving after 25 Years,'' 
\cite{Ha07},~\cite{Gor}, and
% A short survey of automated reasoning.
\cite{Mu}.
% Present State of Mechanical Deduction
For something more comprehensive, see~\cite{Ha09}.
% Harrison's book


Several theorem proving systems are extensively documented and are available for download,
including HOL Light~\cite{HOLL}, Isabelle~\cite{Isa}, Coq~\cite{COQ},
 Mizar~\cite{Mizar}, TPS, 
PVS,
ACL2, 
NuPRL, and MetaPRL.
A web-browser version
of Coq allows one to experiment with a proof assistant without
downloading any software~\cite{PW}.

    
    \end{runninglinenumbers*}
    \appendix
    %%\addtocontents{toc}{%
    %%   \protect\setcounter{tocdepth}{1}}
    %% File started Jan 24, 2010.
%
\chapter{Numerical Methods}

Computers have been used at several significant places in the book.
The calculations can be sorted into three general categories:
nonlinear optimization, linear programming, and the combinatorics of
hypermaps. This appendix goes into further detail about these general
categories.


\section{Nonlinear Optimization}

\subsection{interval analysis}%DCG 8.3, p75
\label{sec:bounds-simplex}

Interval analysis is a method to obtain trustworthy computations over
the real numbers from a computer.  This subsection gives a basic
introduction to this method.  Interval arithmetic can trace its
origins to the method of deliberate error.  This is an ancient method
of navigation with imperfect instruments.  When William the Conqueror
crossed the English Channel in 1066, he deliberately steered to the
north of Hastings.

\begin{quote}
  % ``There is no direct evidence of how a twelfth-century pilot found
  % his way across the English Channel $\ldots$
  ``Any pilot even now who had to make that crossing without a chart
  or compass, as William's pilots did, would use the ancient method of
  Deliberate Error: he would not steer directly towards his objective
  but to one side of it, so that when he saw the coast he would know
  which way to turn.'' \cite[p81]{How81}
  % 1066: The Year of the Conquest, Penguin paperback edition
  % 1981. page 148.
\end{quote}

Deliberate error implements ``better safe than sorry.''
%To find an address on a one-way street,  a  driver enters the
%street too early, to point in the right direction.  
When searching for a familiar landmark
on a two-way street, it is more efficient  
to start the search safely to one side of the landmark, 
rather than search in ever
expanding zigzags in both directions.  
%Deliberate error 
%is the idea of trapping a target inside a large enough net, in the
%way that
%Khachiyan traps the optimal solutions of a linear program inside
%an ellipsoid.


The method of deliberate error cushions the adverse effects of
imperfect technology.  The method does not aim to minimize the
imperfections.  It aims to reliably contain the error.
% The method of deliberate error seeks not to minimize the error of
% imperfect technolo


Let $F$ be the linearly ordered set of machine-representable floating
point numbers.  Our presentation will be general enough to allow
various implementations of the set $F$.  For example, $F$ may be a set of
 double precision floating
point numbers, exact rational numbers, or constructible real numbers.
Assume that $F$ contains two special symbols $\pm\infty$,
representing a floating-point number larger than all real numbers and
another that is smaller than all real numbers.  

The floating point floor function $x\to\floor{x}_F\in
F$ and ceiling function $x\mapsto\ceil{x}_F\in F$, with domain
$\ring{R}$ can be defined.  In interval arithmetic, each real number
$x$ is represented by a pair of floating-point numbers:
$a=\floor{x}_F$ and $b=\ceil{x}_F$. The real number $x$ lies in the
closed interval $\leftclosed a,b\rightclosed$.

On most modern processors, the rounding mode can be set to directed
rounding.  For example, when the rounding mode is set upward, the
computer approximation of any basic arithmetic operation $\diamond$
(addition, subtraction, multiplication, or division) on two floating
point numbers $x$ and $y$ is $\ceil{x\diamond y}_F$.  That is, the
floating-point operation is the smallest element of $F$ that is no
greater than the actual value $x\diamond y$.

As exact arithmetic operations are performed on real numbers, the
computer follows along with corresponding intervals $\leftclosed
a,b\rightclosed$ that contain the result.  As the computation
progresses, intervals increase in width as needed so that the real
number remains sandwiched between two floating point numbers in $F$.
The net result is an interval that reliably contains the result of a
real-number calculation.

Interval arithmetic, like the method of deliberate error, does not
seek to eliminate the sources of floating point round off error.
Rather it controls it through a strict scientific standard.


A number of proofs in pure and applied mathematics have been based on
interval analysis.  W. Tucker implemented a rigorous ODE solver with
interval arithmetic and used it to prove that the Lorenz equations
have a strange attractor \cite{Tuc02}. The existence of strange
attractors is problem 14 on Smale's list of 18 Centennial Problems
\cite{Sma98}.  Another prominent problem solved by interval methods is
the double bubble conjecture, a generalization of the isoperimetric
problem in three dimensional Euclidean space.  A sphere gives the
solution to the classical isoperimetric problem.  The work of J. Hass,
M. Hutchings, and R. Schlafly shows that the surface area minimizing
way to enclose two regions of equal volume is the double bubble, which
consists of two partial spheres, separated by a flat circular disk
\cite{HHS95}.

Interval arithmetic has also yielded a number of new results on the
problem of packing circles in a square. M. Cs. Mark\'ot and T. Csendes
have obtained optimality proofs for packings of $28$, $29$, and $30$
circles in a square.  See Figure~\ref{fig:optimal-circles}.  This is
an area of active research. See, for example, \cite{Sza07} and
\cite{Mark07}.


\subsection{Bernstein polynomials}

The \newterm{Bernstein polynomials} $B_{i,n}\in\ring{Z}[x]$ are defined as
\begin{equation}
B_{i,n}(x) = {n\choose i} x^i (1-x)^{n-i},\quad i=0,\ldots,n.
\end{equation}
The set $\{B_{i,n}\mid i=0,\ldots,n\}$ is a basis of the vector space
of polynomials $\ring{R}[x]_n$ of degree at most $n$.

The polynomials are manifestly nonnegative on the unit interval:
$B_{i,n}(x)\ge 0$, when $0\le x\le 1$.  The polynomials form a
partition of unity:
\[
\sum_{i=0}^n B_{i,n}(x) = \sum_{i=0}^n {n\choose i} x^i (1-x)^{n-i} = 
(x + (1-x))^n = 1.
\]

Any polynomial $p$ of degree at most $n$, 
when expressed in terms of the Bernstein basis:
\[
p(x) =  \sum_{i=0}^n a_i B_{i,n}(x),
\]
can be easily bounded from below and above:
\[
\min_i a_i \le p(x) \le \max_i a_i,\quad\text{ when } 0 \le x \le 1.
\]
Indeed, if $a = \min_i a_i$, then by nonnegativity and the partition
of unity properties
\[
a  =  \sum_{i=0}^n a B_{i,n}(x) \le \sum_{i=0}^n a_i B_{i,n}(x) = p(x).
\]
The upper bound is similar.

These bounds lead to a simple algorithm to prove a polynomial bound
$p(x) < M$ for $a\le x\le b$.  First, rescale the polynomial $p$ so
that the domain is $\leftclosed 0,1\rightclosed$:
\[
p_1(x) = p(a (1-x) + b x), \quad 0\le x \le 1.
\]
Next, express $p_1(x)$ as a linear combination $\sum a_i B_{i,n}(x)$.
If $\max_i a_i < M$, then the desired bound is proved and the algorithm
terminates.  Otherwise,
 partition the domain
\[
\leftclosed 0,1\rightclosed = \leftclosed  0,\frac{1}{2}\rightclosed
\cup \leftclosed  \frac{1}{2},1\rightclosed,
\]
and then recursively apply the algorithm to prove $p_1(x)<M$ on each
subinterval.   (If ever $\min_i a_i > M$, then the desired inequality is false.)


\subsection{multivariate polynomials}

A multivariate polynomial $p(x_1,\ldots,x_n)\in \ring{R}[x_1,\ldots,x_k]$ that
has degree at most $n$ in each variable $x_i$ can be written as a linear
combination of the polynomials
\[
B_{\alpha,n}(x_1,\ldots,x_n)=B_{i_1,n}(x_1)\cdots B_{i_k,n}(x_k),\qquad
 i_1,\ldots,i_k\in \{0,\ldots,n\},
\]
In multi-index notation $\alpha=(i_1,\ldots,i_k)$, 
\[
p = \sum_{\alpha\mid \max{\alpha}\,\le\, n} a_\alpha B_{\alpha,n}.
\]
On the unit cube $[0,1]^k$, we have
\[
\min a_\alpha \le p(x_1,\ldots,x_k) \le \max a_\alpha.
\]
As with univariate polynomials, this gives an algorithm to prove
polynomial bounds $p(x_1,\ldots,x_k)<M$ on rectangular domains:
rescale so that the domain is the unit cube; check whether
\[
\max a_\alpha < M;
\]
if not, subdivide the rectangle and repeat recursively.



\subsection{nonpolynomial optimization}

The nonlinear inequalities that are required in this book are close to
being polynomial.  In fact, many  of the inequalities can be put into a
general form:
\begin{equation}\label{eqn:ABCD}
\sum_{i=0}^k A_i \atn(\sqrt{B_i},C_i) < D
\end{equation}
for some multivariate polynomials $A_i,B_i,C_i,D$.  There are a few
variations on this general form, but generally speaking, the only
nonpolynomial functions that are needed in this book are the square
root, division $(/)$, and the arctangent.  These
are functions of one or two variables.  Polynomial upper and lower
approximations are readily obtained.

For example, if $A_i\ge 0$, the inequality \eqn{eqn:ABCD} is a
consequence of two simpler inequalities: a polynomial inequality
\[
\sum_{i=0}^k A_i L_i(B_i,C_i) < D.
\]
and polynomial upper bounds to nonpolynomial functions in two variables:
\begin{equation}\label{eqn:sL}
\atn(\sqrt{x},y) < L_i(x,y),
\end{equation}
for all $(x,y)$ in the range of the polynomials $B_i,C_i$, for some
polynomials $L_i$.  The polynomial
inequality can be proved by Bernstein polynomial methods.  The
inequality \eqn{eqn:sL} can be proved by various methods.  After all,
the analytic properties of $\atn(\sqrt{x},y)$ are thoroughly
understood.

If two simpler inequalities of the given form cannot be produced over
the entire $k$-dimensional rectangular domain, one can recursively
cover the domain by smaller rectangles until two such inequalities can
be found on each smaller rectangle.

The material in this section is based on the work of R. Zumkeller, who
has developed Bernstein polynomial methods in the context of the
Flyspeck project~\cite{roland-thesis}, \cite{zumkeller-nonlinear}.


\section{Linear Programming}

\subsection{primal program}

Starting with Dantzig's famous simplex algorithm in 1947, 
a vast mathematical literature describes algorithms to
solve the constrained optimization problem of finding a column vector
$x\in\ring{R}^n$ that maximizes the objective function $ x$
\begin{equation}\label{eqn:lp1}
\max_{x}  c x
\end{equation}
where $c\in\ring{R}^n$ is a fixed row vector, subject to a system of
linear inequalities,
\begin{equation}\label{eqn:lp2}
A x\le b
\end{equation} 
for some matrix $A$ and vector $b\in \ring{R}^m$.  (An inequality $A
x\le b$ of vectors means that each component of the vectors satisfies
the inequality.)  Such a maximization problem is called a
\newterm{linear program}.  Surveys of algorithms appear in
\cite{Wri05} and \cite{Tod02}.

% , an unusual name for what it is, not a program in the sense of
% computer code, rather a program in the sense of military logistics,
% growing out of G. Dantzig's war experience at the Pentagon
% \cite{Dan91}.  G. Dantzig proposed the simplex method to solve such
% problems in 1947.  The system of constraints $A x \le b$ defines a
% polyhedron, the vector $c$ gives a preferred direction in space, and
% the simplex method walks along edges of the polyhedron, from extreme
% points to extreme point, until reaching a extreme point where no
% further progress can be made.  The simplex method has been named one
% of the top ten algorithms of the twentieth century \cite{Cip00}.
%

A linear program is \newterm{feasible} if there exists an $x$ satisfying
all the constraints $A x \le b$.  A linear program is \newterm{bounded}
if the set 
\[
\{c x \mid A x \le b\}
\]
is bounded from above.

\subsection{duality}


Given the linear program of Equations~\ref{eqn:lp1},\ref{eqn:lp2}
defined from $A,b,c$, there is another linear program defined by
the same data, but in \newterm{dual} form
\begin{equation}
\min_y {y b}
\end{equation}
such that
\begin{equation}
y A = c \text{ and } y \ge 0,
\end{equation}
where $y\in\ring{R}^m$ is a row vector.  To distinguish the two linear
programs, the one given by Equations~\ref{eqn:lp1} and \ref{eqn:lp2}
is called the \newterm{primal}.  If $x$ is any \newterm{feasible
  solution} to the primal and if $y$ is any feasible solution to the
dual (meaning that they satisfies the constraints but are not
necessarily optimal), then by the inequality constraints, we have
trivially that
$$y b \ge y A x = c x.$$
That is, any feasible solution $y$ to the dual linear program gives an
upper bound to the primal, and a feasible solution $x$ to the primal
linear program gives a lower bound to the dual.  If the optimal dual
solution is $y=y^*$ and the optimal primal solution is $x=x^*$, then
$$y^* b \ge c x^*.$$
The \newterm{linear programming duality theorem} asserts that if the
primal is feasible and bounded, then the dual is also feasible and
bounded, and moreover $y^* b = c x^*$.  In summary, we can bound the
primal with any feasible solution to the dual, and find the maximum of
the primal by minimizing the dual.

%Linear programming duality produces certificates of a bound
%on the primal.  Suppose we treat a linear programming package as
%an untrusted 
%black box without any knowledge of its internal algorithms.
%To handle round-off errors, we now also assume that we have
%a priori bounds $|x_i| \le m$ on the variables $x_i$.
%The untrusted package produces a vector $y$, which it claims
%is (approximately) a feasible solution to the dual problem.  
%First, replacing
%any negative coefficients in $y$ with $0$, we may assume that
%it satisfies the constraint $y\ge 0$.  The quantity $\delta = c - y A$
%should be exactly zero for a feasible solution to the dual program, but
%because of round-off errors, it may be  off.  
%Then for any feasible $x$ to the primal problem
%$$
%c x = (\delta + y A)  x \le \delta  x + y b.
%$$
%Using the a priori bounds on $x$, we get $\delta x\le \sum 
%m|\delta_i|=D$.  An upper bound on the primal is the value
%$D + y b$, which is computed without any knowledge of the
%algorithms used by the software package.  By the duality theorem
%for linear programming, there exists a certificate for which this
%computed value is exactly the maximum of the primal.
%
%In the packing problem, in practice we have a priori bounds
%on the variables $x$, and this method works extremely well.  Note
%that the a priori bound $m$ is allowed to be an extremely crude estimate,
%because in producing the estimate $D$, it is multiplied by $\delta$,
%which typically has the magnitude of a machine epsilon.
%


\subsection{infeasibility}

The problem we consider in this subsection is to verify that the
maximum of $c x$ is less than a given constant $M$, when subject to
\eqn{eqn:lp2} and to given bounds on the entries of $x$.
That is, we wish to verify that the system of inequalities
\begin{equation}\label{eqn:empty}
A x \le b,\quad \ell \le x\le u,\quad c x \ge M
\end{equation}
has no solutions in $x$. Any vector $y$ with the properties
\begin{equation}\label{eqn:y}
  y A = c,\quad y\ge 0,\quad y b < M
\end{equation}
is a \newterm{certificate} that the system \eqn{eqn:empty} is infeasible.  Indeed,
if $y$ has these properties, then for any $x$ satisfying
$A x \le b$, it follows that
\begin{equation}\label{eqn:cxM}
  c x = y A x \le y b < M
\end{equation}
as desired.


\subsection{numerical solution}


The theory of duality for linear programming can be used to show that if a
system is infeasible, then a certificate of infeasibility exists.  A certificate
may be found by solving the given linear programming problem \eqn{eqn:ci}.
It is not necessary to know how a
certificate of infeasibility is produced.  It can be produced 
by an untrusted algorithm.  

The concept of certificate can be
adapted to a context that allows for small  errors, produced by
numerical approximations on a computer.
Because of inexact computer arithmetic, the
equality in \eqn{eqn:y} will only be approximately correct. The imprecision in the
dual certificate $y$ can be readily eliminated as follows. If $u$ is any
vector, let $u^+$ be the vector obtained by replacing the negative
entries of $u$ with $0$, and let $u^-$ be the vector obtained by
replacing the positive entries of $u$ with $0$.  In the
following lemma, $\epsilon_1$ and $\epsilon_2$ are small error terms
that result from machine approximation. By including them in the
bounds on $c x$, a rigorous bound can be recovered.  


\begin{lemma}  Suppose that the real-valued vectors and matrices
$A,A_1,A_2$, $c,c_1,c_2$, $x,b,\ell,u$, $y$ satisfy the following
relations
  $$
  A x\le b, \quad A_1 \le A \le A_2,
  \quad c_1 \le c \le c_2,\quad \ell\le x\le u,\quad
  0\le y.
  $$
Define residuals
  $$
   \epsilon_1 = c_1 - y A_2,\quad \epsilon_2 = c_2  - y A_1.
  $$
If
$$
y b + \epsilon_2^+ u^+ + \epsilon_1^+ u^- + \epsilon_2^- l^+ + \epsilon_1^- l^- < M,
$$
then $c x < M$.
\end{lemma}

\begin{proof} S. Obua has given a formal proof of this lemma in the
Isabelle/HOL system \cite[3.7.2]{Obua:2008:Thesis}.  The proof
is a simple embellishment of Inequality~\ref{eqn:cxM}:
\begin{eqnarray*}
c x -y b &\le& c_2 x^+ + c_1 x^- -y A x\\
              &=& (\epsilon_2 x^+ + \epsilon_1 x^-)
  + y (A_1-A) x^+ + y (A_2 - A) x^- \\
              &\le& (\epsilon_2^+ +\epsilon_2^-) x^+ 
       + (\epsilon_1^++\epsilon_1^-) x^-\\
              &\le& \epsilon_2^+ u^+ +  \epsilon_2^- l^+ 
           + \epsilon_1^+ u^- + \epsilon_1^- l^-.
\end{eqnarray*}
\end{proof}

The numerical data $A_1,A_2,c_1,c_2,\ell,u,y,b$ are all explicitly given,
so that the method yields explicit bounds.
It is not necessary to trust the software
that produces the certificate $y$, because
all of the assumptions of the lemma can be checked directly.
The conditions $A x \le b$ and $\ell\le x\le u$ hold by the assumption
of the feasibility of $x$; the condition $0\le y$ is produced by replacing the
negative entries of $y$ with $0$; the other assumptions of the lemma are
checked by simple matrix operations by computer.  Interval arithmetic 
(or exact rational arithmetic) guarantees
the reliability of these matrix
operations.

In practice, it is necessary for us to solve hundreds of thousands of
linear program feasibility problems\eqn{eqn:empty}, each involving
hundreds of variables and thousands of linear constraints.

\subsection{linear relaxation}

\newterm{Nonlinear optimization} searches for the maximum of a
continuous function $f$ over a compact domain $X\subset \ring{R}^n$.
A \newterm{linear relaxation} of the domain is a polyhedron defined by
constraints $A x \le b$ such that the polyhedron contains the domain
$X$.  That is, $x\in X$ implies $A x \le b$.  A \newterm{linear
  relaxation} of the objective function $f$ is a linear function
$x\mapsto c x$ such that $f(x) \le c x$ for all $x\in X$.  It then
follows trivially that the maximum of $f$ over $X$ is at most the
value of the linear program that maximizes $c x$ such that $A x \le
b$.

Relaxation gives a mathematically sound linearization of a
nonlinear optimization problem.  It should be expected that the bounds
obtained by this method will often be crude.  Nevertheless, linear
programs can be solved efficiently, while general nonlinear
programming problems cannot.

\begin{remark}[popularization]
A popular article about the proof of the packing problem
illustrated the method by depicting the nonlinear function $f$
as the rolling hills in the countryside, cows grazing in the background.  
A helicopter,
carrying a large roof, was slowing
descending over the hilltops, bounding the height of a hilltop
with the piecewise linear roof.
\end{remark}

\begin{remark}[vanishing box trick]
Relaxation can also be seen as a {\it vanishing
box trick.}  Consider the set of counterexamples
\[
\{x\in X\mid f(x)\ge c x\}
\]
to a desired nonlinear inequality ($f(x)< c x$ on $X$).  At the
outset, the set of counterexamples to the problem has an unknown size
and structure.  These counterexamples are all placed in a large box.
The width of the box is measured and is found to be negative.
Therefore the box is empty, it contains no counterexamples, and the
nonlinear inequality is proved.  It is remarkable that a method based
on such a simple idea is sufficiently powerful to prove many difficult
nonlinear inequalities.  In this intuitive picture, the box represents
the polyhedron obtained by linear relaxation.  The negative width
represents the inequality produced by the dual certificate $y$.
\end{remark}

\subsection{linear assembly}

The nonlinear optimization problems that we solve in this book have
the structure of \newterm{linear assembly problems}.  These are
problems whose nonlinearities can confined to small dimensions.

The linear assembly problem asks for a proof of an upper bound
\begin{equation}\label{eqn:cxM}
c x < M
\end{equation}
for all $x=(x_1,\ldots,x_s)\in \ring{R}^{m_1}\times\cdots
\times\ring{R}^{m_s}=\ring{R}^n$
subject to constraints
\begin{eqnarray}
A x &\le& b,\notag\\
%x &=& (x_1,x_2,\ldots,x_s),\\
x_i &\in& D_i\notag
\end{eqnarray}
for some matrices $A,b$, natural numbers $m_i$, and sets $D_i\subset
\ring{R}^{m_i}$, where $m_1+\cdots+m_s = n$.  The only nonlinearities
are those appearing in the constraints $x_i\in D_i$.

An \newterm{assembly certificate} of the upper bound \eqn{eqn:cxM}
consists of the following data
\[
P_{i,j}\mid i=1,\ldots,s,~~j\in J_i,\quad\text{ and }\quad 
y_k\mid~~k=(j_1,\ldots,j_s),~~j_i\in J_i.
\]
where $P_{i,j}$ are polyhedra in $\ring{R}^{m_i}$ such that 
\begin{equation}\label{eqn:DP}
D_i \subset\cup_{j\in J_i} P_{i,j},
\end{equation} and $y_k$ are linear programming
dual certificates of the infeasibility of the linear programs
\[
c x \ge M,\quad A x \le b,\quad x_i\in P_{i,j_i}.
\]

An assembly certificate provides an immediate proof of the upper bound
of the linear assembly problem.  Indeed, If $x$ satisfies the
constraints of the linear assembly problem, then $x_i\in D_i$ for all
$i$.  Hence $x_i\in P_{i,j_i}$ for some indices $j_i$.  The dual
certificate $y_k$, $k=(j_1,\ldots)$, then certifies that $c x < M$.

The nonlinear part of the certificate consists in the verification
of \eqn{eqn:DP}.  In practice, the dimensions $m_i$
may be small, making this nonlinear part computationally feasible,
even when $n=\sum m_i$ is large.  (In practice, we may have $m_i\le 6$
and $n > 100$.)

%\bigskip
%For the problem to be a true linear relaxation, there must be rigorous
%proofs that each $x\in X$ satisfies the linear constraints $A x\le b$.
%This again, is a nonlinear optimization problem on the domain $X$ with
%objective function $A_i x$. 
%
%The packing problem has a special structure that avoids this
%problem.  Even though the nonlinear optimization problem involves
%around 150 variables, 
%this special structure allows all the nonlinearities to 
%be confined to small dimensions (around 6 variables).  In other words,
%the packing problem can be
%described as a coupled system of small nonlinear subsystems,
%and all the coupling comes through linear systems of equations. 
%We give a toy model of this in a moment to describe how this works.
%
%This special structure is one of the key points of the entire proof.
%If you want to know why this hard nonlinear optimization problem has
%been solved, but others have not, you now know the fundamental reason.
%By partitioning the linear relaxation of the domain $X$ according
%to the small nonlinear subsystems, we can arrange for each inequality
%$A_i x\le b_i$ to be sparsely populated, and for each relaxation
%constraint to be a statement about a nonlinear inequality on a domain
%of low dimension.  This is a tractible problem:
%we  summon interval arithmetic to prove these
%low dimensional inequalities by computer.  In this way we are able
%to give a rigorous proof that we have a true linear relaxation of a
%difficult nonlinear optimization problem.  In practice, the linear
%relaxation  bounds  are good enough to solve the packing
%problem.
%
%
%\subsection{infeasibility}

%It may seem that such a strategy is too hopelessly naive to work.
%In fact, precisely this strategy is pursued, and it works beautifully.
%The act of placing a counterexample in a box is called linear 
%relaxation.  The counterexamples form an unknown nonlinear set.
%The box that contains them is defined by linear constraints.  
%Relaxation refers to a relaxation of constraints, so that the
%box contains the counterexamples -- it is not merely a linear
%approximation to the set of counterexamples.  The box needn't be
%rectangular.  Any polyhedron will do.  To determine that a box
%has negative width is to say that it gives an {\it infeasible} linear
%program.  With the strategy, there is no need to limit ourselves
%to a single box, we can use a finite collection of boxes that
%contain the set of counterexamples.  This leads to branch and bound
%methods.
%
%
%In this appendix, we will discuss these methods in further detail,
%including linear programming, linear relaxation,  infeasibility,
%and branch and bound methods.
%
%The introduction of linear programming methods to the packing
%problem was gradual and tranquil.  
%Many of the
%inequalities in the packing problem have an obvious linear form.
%As an engineering student
%at Stanford, I had studied
%linear and nonlinear optimization.
%Given my background, 
%it was natural for me to express them as a linear program.  
%Once linear programming had made an appearance, the 
%repeated successes of the method led me to rely on it more and more.
%In the end, approximately $10^5$ linear programs appear in the solution,
%each involving some $200$ variables and around $2000$ constraints.
%This is a inconsequential task for modern algorithms.  
%
%The consequential part is to organize the output of this many linear
%programs into a comprehensible proof narrative.  How do I convince you
%the reader that this vast collection of linear programs constitutes
%the proof of a theorem?  What is the precise statement of that
%theorem, expressed in a way that does not refer to $10^5$ separate
%problems? There is an entire chapter of this book devoted to these
%questions.  This introductory essay gives enough background to carry
%us through to that chapter.
%
%
%
%
%Later in 1947, within minutes of first hearing about
%linear programming, von Neumann introduced linear programming
%duality, as an outgrowth of his theory of games.
%In 1979, L. Khachiyan found a way to solve linear programs in
%polynomial time \cite{Kha79}.  Still later, 
%N. Karmarkar found a polynomial
%time algorithm of practical value \cite{Kar84}.  
%M. Wright gives a recent survey
%of interior-point algorithms (algorithms that search for the 
%optimal solution by moving through the interior of the polyhedron
%rather than following the simplex method's strategy of searching
%along the boundary of the polyhedron) in \cite{Wri05}. 
%M. Todd also gives an excellent survey of algorithms \cite{Tod02}.
%These algorithms have been implemented in various
%software packages.
%Because of this extensive and well-developed literature on the
%subject of linear programming, we will take it as given that we
%can solve large-scale linear programs.
%
%
%
%The basic linear program admits many variations.  We can minimize
%the objective function
%\[x \mapsto c x\] 
%rather than maximize it.
%Any linear equation
%$a x = b$ can be written as a system of inequalities
%\[a  x \le b \text{ and } -a x \le -b.\]
%This allows us to reduce constrained linear optimization with
%equality constraints to a constrained linear optimization with
%inequality constraints.
%
%
%
%\subsection{infeasibility}
%
%Consider a general system of linear inequalities
%\begin{alignat}{1}
%\label{eqn:lpsys}
%A x &\le c\notag\\
%a &\le x  \,\le b,
%\end{alignat}
%with given matrix coefficients $A,c$ and given 
%lower and upper bounds $(a\le x\le b)$ on a
%vector $x$ of unknowns.  
%
%
%
%
%%In practice, many of the primal linear programs that appear
%%in the packing problem are infeasible.  This requires
%%some small adjustments to the previous discussion.   In the
%%linear programs that we encounter in this book, it is not
%%necessary to determine the exact upper bound of a the linear program.
%%We have an explicit constant $K$, and we wish to prove that
%%the maximum of the objective function is less than $K$.
%%Rather
%%than consider two separate
%%types of problems, feasible and infeasible, we recast all the linear
%%programs in this book in terms of infeasibility.
%%When the maximum of $c x$ is less than $K$, by
%%adding the constraint $c x\ge K$ to the system of constraints
%%$A x\le b$,  we get an infeasible linear program.
%%This is the vanishing box trick:  the counterexamples to a nonlinear
%%problem are constrained to lie inside an infeasible system of linear
%%%inequalities.
%%
%%
%%We now drop the objective function, and work
%%with a system of inequalities  of the form
%%  By
%%translating dual linear program certificates into this new context
%%we are led to the following definition.
%%
%
%\begin{definition}[certificate~of~infeasibility]
%A \newterm{certificate of infeasibility} for the system \ref{eqn:lpsys},
%is a pair $(u,v)$ satisfying
%\begin{eqnarray*}
%0&\le& u\\
%0&\le& v\\
%0&\le& u A + v\\
%0& >& u(c-Aa) + v(b-a).
%\end{eqnarray*}
%\end{definition}
%
%\begin{example}
%If there is an index $i$ for which $b_i < a_i$, then the system is
%clearly infeasible.  In this case, we have an obvious certificate
%of infeasibility given by $(u,v)=(0,e_i)$, where $e_i$ is the
%standard basis vector at index $i$.
%\end{example}
%
%\begin{lemma}
%If a certificate of infeasibility $(u,v)$ exists for the system
%\ref{eqn:lpsys}, then the system is infeasible.
%\end{lemma}
%
%\begin{proof}
%Suppose for a contradiction that $x$ is a feasible solution.
%Then
%\begin{eqnarray*}
%0 &\le& (u A + v)(x-a) \\
%&=& [u (c- A a) + v (b- a)] - [u (c - A x)] - [v (b - x)]\\
%&<& 0.\qedhere
%\end{eqnarray*}
%\end{proof}
%
%%The set of certificates form a cone: if $(u,v)$ is a certificate and $t>0$,
%%then $(t u, t v)$ is also a certificate.  
%
%Since the equations defining a certificate of infeasibility are
%linear in $u$ and $v$, linear programming algorithms may be used
%to produce certificates:
%\begin{align}\label{eqn:ci}
%\min_{u,v} &\, u (c-A a) + v(b-a)
%\intertext{such that }
%0 &\le u\notag\\
%0 &\le v\notag\\
%0& \le u A + v\notag
%\end{align}
%Trivial manipulations transform this minimization problem into the standard
%format \eqn{eqn:lp1} \eqn{eqn:lp2}.
%The system of constraints has an obvious feasible solution $u=v=0$.
%If we add the constraint that the objective function is at least $-1$,
%then there exists a bounded feasible solution.  
%
%\begin{lemma}  Let $A_1,A_2,A$.
%\end{lemma}
%
%If an approximation $(u',v')$ to a certificate of infeasibility is
%produced (say by computer), it can often be adjusted to give a
%true certificate.  Given an approximation $(u',v')$, define
%$(u,v)$ by
%\begin{eqnarray*}
%u &=& \max(0,u')\\
%v &=& \max(0,v',-u A).
%\end{eqnarray*}
%If $(u,v)$ satisfies
%$$u( c-A a) + v(b-a) <0,$$
%then it is a certificate of infeasibility.
%







\section{Hypermap Generation}

Recall that a restricted hypermap  $H = (D,e,n,f)$ is one with the following
properties.
\begin{enumerate}
\item The hypermap $H$ has no double joins, and is nonempty, planar,
  connected, plain and simple.
\item The edge map $e$ has no fixed points.  % Needed in Lemma:[flag quotient]
\item The node map $n$ has no fixed points.
\item The size of every face is at least $3$.
%%  (All hypoth. Needed?)
\end{enumerate}

Section~\ref{sec:generation} develops an algorithm to generate all
restricted hypermap up to isomorphism (or rather all that satisfy
prescribed bounds on number of nodes and faces).

The set of tame hypermaps is a subset of the a finite set of
restricted hypermaps (satisfying prescribed bounds).  We can generate
all tame hypermaps up to isomorphism: generate all the restricted
hypermaps and filter out those that are not tame.

These algorithms have been implemented in software.  The computer code
has been executed to obtain an explicit list of all tame hypermaps, up
to isomorphism.  By setting different initial parameters, the program
can also generate various other classes such as hypermaps with tame
contact, Tammes-tame hypermaps, and the class of hypermaps needed for
the proof of the dodecahedral conjecture
~\cite{unknown}. % McLaughlin.
G. Bauer's thesis makes an exhaustive analysis of the computer
program, and reimplements the code~\cite{unknown}.  T. Nipkow and G.
Bauer subject the computer code to formal proof~\cite{unknown}.




    %\newpage
    %

\section*{Acknowledgments}


I am grateful to Sam Ferguson for the years that he spent working on
this problem with me.  I also thank the early editors Robert
MacPherson, Gabor Fejes T\'oth, and Jeff Lagarias for the many
improvements they brought to the original proof.

Many have worked tirelessly to make the Flyspeck formalization project
a reality.  I wish to thank Nguyen Quang Truong, Nguyen Tat Thang,
Tran Nam Trung, Trieu Thi Diep, Dang Tat Dat, Nguyen Duc Tam, Vu Quang
Thanh, Nguyen Duc Thinh, Hoang Le Truong, Ta Thi Hoai An, Erin Susick,
Laurel Beth Martin, Mary Johnston, Catalin Anghel, Alexey Solovyev,
Jeremy Avigad, Mark Adams, Freek Wiedijk, Tobias Nipkow, John
Harrison, Sean McLaughlin, Gertrud Bauer, Matthew Wampler-Doty, Steven
Obua, and Roland Zumkeller.


I would like to thank John Harrison for designing and implementing the
most amazing piece of software I have ever encountered.  He has
contributed large bodies of formal mathematics to the libraries of the
HOL Light proof assistant in order to make the Flyspeck project
possible.  Tobias Nipkow became involved in the project at an early
stage and directed the graduate work of two students, Bauer and Obua,
on the formal verification of computer code for the Flyspeck
project. Mark Adams has taught me about how large-scale project should
be organized and has skillfully managed Flyspeck group in Hanoi.

When I learned that my NSF proposal had been funded, I considerd
turning down the funding because I worried that the deliverables I had
put into the proposal were beyond my capacity.  It was one thing to
grandstand and quite another to implement the largest formalization
project that had ever been attempted.  In retrospect, I am indebted to
NSF for its generous support of this project.  It has supported this
project in all of its aspects, including a two-month workshop on
formal proof and the Flyspeck project at the Math Institute in Hanoi
during the summer of 2009.  This workshop and all that has grown out
of it have been crucial for the success of this project.

I am also indebted to the Benter Foundation for a generous grant to
rework the original proof along lines proposed by Marchal.

Much of the material from this book was covered in a course on
discrete geometry and computers at the University of Pittsburgh and
then later in Hanoi.  I would like to thank the members of these
groups for assisting in the preparation of the book.

A draft of this book was written during a sabbatical leave from the
University of Pittsburgh, 2007--2008.  I wish to thank the many
institutions that supported me during this period: the Max Planck
Institute in Bonn, the \'Ecole Normale Sup\'erieure in Paris, the
Institute of Math in Hanoi, Radboud University in Nijmegen, and the
University of Strasbourg.  These visits would not have been possible
without the assistance of Grigori Mints, Fran\c{c}ois Loeser, Florence
Lecomte, Henk Barendregt, Ha Huy Khoai and Ng\^o Vi\d{\^e}t Trung.

\bigskip

This book is dedicated to the memory of my grandfather Wayne B. Hales.




%%%%%%%%%%%%%%%%%%%%%%%%%%%%%%%%%%%%%%%%%%

%\raggedright
%\bibliographystyle{plainnat}
%\bibliography{../latex/bibliography/all}

%% shell:>makeindex index/Index
%% shell:>makeindex index/Notation
%\printindex{index/Index}{General index}
%\printindex{index/Notation}{Notation index}

 

\end{document}
