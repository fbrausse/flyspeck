%------------------------------------------------------------
% Author: Thomas C. Hales
% Format: LaTeX
% Book Chapter: Dense Sphere Packings
%------------------------------------------------------------

%% XX Write essay on Formal Proof.
%% XX Finish Mathematica calc on dodec.

\chapter{Close Packing}
\section{Face Centered Cubic}



The face-centered cubic packing is the familiar pyramid arrangement of
balls on a square base.  It is also the pyramid arrangement on a
triangular base.  The square base packing and triangular base packing
differ only in their orientation in space.
Figure~\ref{fig:tri-square-cannonballs} shows how the triangular base
packing fits between the peaks of two adjacent square base pyramids.

%% WW not yet done.
\begin{figure}[htb]
  \centering
  \myincludegraphics{noimage.eps}
  \caption{The square and triangular based packings of cannonballs}
  \label{fig:tri-square-cannonballs}
\end{figure}

Density, defined as a ratio of volumes, is insensitive to changes of
scale.  For convenience, it is sufficient to with balls of unit
radius. This means that the distance between centers of balls in a
packing will always be at least $2$.  Identify a packing with its set
of centers.  Thus, for our purposes a packing is just a set of points
in $\ring{R}^3$, whose elements are separated by distances at least
$2$.


% based on page 28 of kepler research scan/scan0001.tif
The fcc packing is an alternating tiling of regular tetrahedra and
regular octahedra.  See Figure~\ref{fig:tet-oct}.  The figure has one
tetrahedron at each extreme point and one octahedron in the center. By
similarity, the total volume is $8 = 2^3$ times the volume of each
smaller tetrahedron. This dissection exhibits the volume of a regular
octahedron as exactly four times the volume of a regular tetrahedron
of the same edge length.

%% WW not yet done.
\begin{figure}[htb]
  \centering
  \myincludegraphics{noimage.eps}
  \caption{The fcc as a tiling of tetrahedra and octahedra}
  \label{fig:tet-oct}
\end{figure}

The face-centered cubic packing is the packing obtained from a cubic
lattice, with a ball at each extreme point of each cube and adding a
ball at the center of each face of each cube.  The name `face-centered
cubic' comes from this construction.  The edge of each cube is
$2\sqrt2$, and the diagonal of each face $4$, for balls of unit
diameter.  The density of the packing is equal to the density within
each cube.  The cube has volume $(2\sqrt2)^3 = 16\sqrt2$.  It contains
a total of four balls: half a ball along each of six faces and one
eighth a ball at each of eight corners.  Thus, the density is
   \begin{displaymath}
   \frac{   4 (4\pi/3)}{16\sqrt2} = \frac{\pi}{\sqrt{18}}.
   \end{displaymath}
See Figure~\ref{fig:fcc-cube}.

%% WW not yet done.
\begin{figure}[htb]
  \centering
  \myincludegraphics{noimage.eps}
  \caption{The face-centered cubic structure}
  \label{fig:fcc-cube}
\end{figure}

The alternating arrangement of regular tetrahedra and octahedra can be
superimposed on the cubic picture.  There is one tetrahedron at each
extreme point of the cube, extending to the centers of the three
adjacent faces.  An octahedron sits the center of the cube, an extreme
point of the octahedron at the center of each face.  There is an
additional quarter of an octahedron along each edge, extending to the
midpoints of the two adjacent faces.  This gives a total of eight
tetrahedra and four octahedra.  As each octahedron has the volume of
four tetrahedra, exactly $1/3$ of the cube is filled with tetrahedra,
the other $2/3$ with octahedra.  This decomposition determines the
volume $2\sqrt2/3$ of a tetrahedron.
%(pretend ignorance). The volume $16\sqrt2$ 
%of the cube equals $24$ tetrahedra $\ldots$, giving each a volume
%of 
%$2\sqrt{2}/3$.

The density of the face-centered cubic packing is the weighted density
of the densities of the tetrahedron and octahedron.  Write $\dtet$ and
$\doct$ for these densities.  For example, $\dtet$ is the ratio of the
volume of the part within the tetrahedron of the unit balls (at the
four extreme points) to the volume of the tetrahedron.  As the cube is
$1/3$ filled with tetrahedra and $2/3$ filled with octahedra,
\begin{displaymath}
  \frac{\pi}{\sqrt{18}} = \frac{1}{3}\dtet + \frac{2}{3}\doct.
\end{displaymath}

The Voronoi cell of a point $v$ in a packing $V$ consists of all
points in $\ring{R}^3$ that are closer to $v$ than to any other point
of $V$.  Each Voronoi cell of the face-centered cubic packing is a
rhombic dodecahedron (Figure~\ref{fig:rhombic}).  %% not drawn.
The rhombic dodecahedron can be constructed from a cube by placing a
square based pyramid (with height half as great as an edge of its
square base) on each of the six faces
(Figure~\ref{fig:rhombic-cube}).  %% not drawn.
For the scale to be correct, the initial cube should have edge
$\sqrt{2}$.

The Voronoi cells can be superimposed on the cube of side $2\sqrt2$.
Center a cube of side $\sqrt2$ at each ball in the packing. These
cubes fill the black grid of an infinite three dimensional
checkerboard, leaving a second grid of white cubes unfilled.  Each
white cube can be partitioned into pyramids along its faces with apex
at the center of the cube.  Attaching these pyramids along their bases
to the adjoining black cubes gives the Voronoi cell.

%% WW not yet done.
\begin{figure}[htb]
  \centering
  \myincludegraphics{noimage.eps}
  \caption{The rhombic dodecahedron as Voronoi cell}
  \label{fig:rhombic}
\end{figure}

%% WW not yet done.
\begin{figure}[htb]
  \centering
  \myincludegraphics{noimage.eps}
  \caption{The rhombic dodecahedron constructed from a cube}
  \label{fig:rhombic-cube}
\end{figure}

Each Voronoi cell contains one black cube of side $\sqrt2$ and a total
of one white cube, for a total volume of $4\sqrt2$.  This constant is
one of the fundamental constants in this book.  The volume of every
Voronoi cell is compared against the volume of the rhombic
dodecahedron.  The density of the face-centered cubic is the ratio of
the volume of a ball to the volume of its Voronoi cell, which gives
$\pi/\sqrt{18}$, yet again.





\section{Hexagonal-close packing}

There is a popular and persistent misconception in the popular press
that the face-centered cubic packing is the only packing with density
$\pi/\sqrt{18}$.

In the face-centered cubic packing, each ball is tangent to twelve
others.  For each ball in the packing, this arrangement of twelve
tangent balls is the same.  Call it the fcc pattern.  In the
hexagonal-close packing, each ball is tangent to twelve others
(Figure~\ref{fig:hcp}).  For each ball in the packing, the arrangement
of twelve tangent balls is again the same.  Call it the hcp pattern.
The fcc pattern is different from the hcp pattern.  In the fcc
pattern, there are four different planes through the center of the
central ball that contain the centers of six other balls at the
vertices of a regular hexagon.  In the hcp pattern, there is only one
such plane.  Call the arrangement of balls tangent to a given ball the
\newterm{local tangent arrangement} of the ball.

%% WW not yet done.
\begin{figure}[htb]
  \centering
  \myincludegraphics{noimage.eps}
  \caption{The fcc and hcp patterns}
  \label{fig:hcp}
\end{figure}

There are uncountably many packings of density $\pi/\sqrt{18}$ that
have the property that every ball is tangent to twelve others and such
that the tangent arrangement around each ball is either the fcc
pattern or the hcp pattern.

Define a \newterm{hexagonal layer} to be a translate of the
two-dimensional hexagonal lattice (also known as the triangular
lattice). That is, it is a translate of the planar lattice generated
by two vectors of length $2$ and angle $2\pi/3$.  The face-centered
cubic packing is an example of a packing built from hexagonal layers.

If $L$ is a hexagonal layer, a second hexagonal layer $L'$ can be
placed parallel to the first so that each lattice point of $L'$ has
distance $2$ from three different points of $L$.  When the second
layer is placed in the manner, it is as close to the first layer as
possible. Fix $L$ and a unit normal to the plane of $L$. The normal
allows us to speak of the second layer $L'$ as being ``above'' or
``below'' the layer $L$. There are two different positions in which
$L'$ can be placed closely above $L$ and two different positions in
which $L'$ can be placed closely below $L$
(Figure~\ref{fig:two-holes}). As packing is constructed layer by
layer, ($L$, $L'$, $L''$, and so forth), there are two choices at each
stage of the close placement of the layer above the previous
layer. Running through different sequences of choices gives
uncountably many packings.  In each of these packings the tangent
arrangement around each ball is that of the twelve balls in the
face-centered cubic or the twelve balls in the hexagonal-close
packing.

%% WW not yet done.
\begin{figure}[htb]
  \centering
  \myincludegraphics{noimage.eps}
  \caption{The two positions of hexagonal layers}
  \label{fig:two-holes}
\end{figure}

Let $V$ be a packing built as a sequence of close-packed hexagonal
layers in this fashion.  If $A$ is any plane parallel to the hexagonal
layers, then there are at most three different orthogonal projections
of the layers $L$ to $A$.  Call these projections $A$, $B$, $C$.  Each
hexagonal layer has a different projection than the layers immediately
above and below it.  In the fcc packing, the successive layers are
$A,B,C,A,B,C,\ldots$.  In the hcp packing, the successive layers are
$A,B,A,B,\ldots$.  When $A$, $B$, and $C$ represent the vertices of a
triangle, the succession of hexagonal layers can be described by a
walk along the vertices of the triangle. Different walks through the
triangle describe different packings (Figure~\ref{fig:fcc-tri}).

%% WW not yet done.
\begin{figure}[htb]
  \centering
  \myincludegraphics{noimage.eps}
  \caption{Describing layers by walks through a triangle}
  \label{fig:fcc-tri}
\end{figure}

The different walks through a triangle give all packings of infinitely
many equal balls in which the tangent arrangement around every ball is
either the fcc pattern of twelve balls or the hcp pattern of twelve
balls.

Different walks through a triangle give all such packings.  To see
this, assume first that $V$ does not contain any balls whose local
tangent arrangement is the hcp pattern.  Then every local tangent
arrangement is the fcc pattern, and $V$ itself is then the
face-centered cubic packing.  This completes the proof in this case.

Now assume that a packing $V$ contains some ball (centered at $v_0$)
in the hcp pattern. The hcp pattern contains a uniquely determined
plane of symmetry. This plane contains $v_0$ and the centers of six
others arranged in a regular hexagonal. If $v$ is the center of one of
the six others in the plane of symmetry, its local tangent arrangement
of twelve balls must include $v_0$ and an additional four of the
twelve balls around $v_0$. These five centers around $v$ are not a
subset of the fcc pattern. They can be uniquely extended to twelve
centers arranged in the hcp pattern. This hcp pattern has the same
plane of symmetry as the hcp pattern around $v_0$. In this way, as
soon as there is a single center with the hcp pattern, the pattern
propagates along the plane of symmetry to create a hexagonal layer
$L$.

Once a packing $V$ contains a single hexagonal layer, the condition
that each ball be tangent to twelve others forces a hexagonal layer
$L'$ above $L$ and another hexagonal layer below $L$.  Thus, a single
hexagonal layer forces an infinite sequence of close-packed hexagonal
layers.  This completes the proof.



\section{Gauss}

Gauss proved that the face-centered packing has the greatest density
of any lattice packing in three dimensional Eulidan space.  There is a
short proof that does not require any calculations.

Start with an arbitrary lattice $V$ in which every point has distance
at least two from every other.  Center a unit ball at each point in
the lattice.  A lattice of greatest density has the property that some
pair of balls touch.  The lattice property then forces the the balls
into parallel infinite linear strings like beads on a string.  The
lattice of greatest density has the property that two of these
infinite parallel strings touch.  The lattice property then arranges
the strings into parallel sheets.  On each sheet the touching parallel
strings form a rhombic tiling.  The lattice of greatest density has
the property that each parallel sheet should sit as snugly as possible
on the sheet below.  That means that some ball (centered at $v_0$) of
one sheet touches three balls (centered at $v_1,v_2,v_3$) on a the
next layer down (Figure~\ref{fig:rhombus}).

%% WW not yet done.
\begin{figure}[htb]
  \centering
  \myincludegraphics{noimage.eps}
  \caption{A ball from one sheet touches three balls centered at points of a rhombus.}
  \label{fig:rhombus}
\end{figure}

As the balls on each sheet form a rhombic tile, two of the distances
between $v_1,v_2,v_3$, corresponding to two edges of the rhombus, are
equal to $2$.  This means that $v_0$ together with two of
$v_1,v_2,v_3$ form an equilateral triangle.  From the perspective of
the plane containing this equilateral triangle, the lattice property
forces this entire plane, as well as parallel planes, to be tiled with
equilateral triangles.  From the earlier argument, each plane sits as
snugly as possible on the sheet below.  Some ball of one sheet touches
the three balls forming an equilateral triangle in the layer below.
These four balls form a regular tetrahedron.  This tetrahedron
uniquely identifies the lattice as the face-centered cubic.






\section{Thue}\label{sec:thue}


As mentioned, Thue solved the packing problem for congruent disks in
the plane.  The optimal packing is the hexagonal packing
(Figure~\ref{fig:2D-hex}).  The density of this packing is
$\pi/\sqrt{12}$: the ratio of the area of a unit disk to the area of a
hexagon of inradius one.  It admits an elementary solution to be
sketched here.  B. Casselman has an animated interactive demo of this
solution \cite{casselman:pennies}.

%% WW not yet done.
\begin{figure}[htb]
  \centering
  \myincludegraphics{noimage.eps}
  \caption{The optimal packing in two dimensions}
  \label{fig:2D-hex}
\end{figure}

Let $V$ be the set of centers of a collection of unit disks.  Take the
Voronoi cell around each disk.  It is enough to show that each Voronoi
cell has area at least $\sqrt{12}$.  For simplicity, assume that $0\in
V$ is the center of our Voronoi cell.

Truncate the Voronoi cell by intersecting it with a disk of radius
$r=2/\sqrt3$.  It is enough to show that the truncated Voronoi cell
has density at most $\pi/\sqrt{12}$.

There is not a point $w$ in the plane that has distance less than $r$
from three disk centers $v_1,v_2,v_3$.  Otherwise, one of the three
angles $\gamma$ at $w$ is at most $2\pi/3$; and $\cos\gamma\ge -0.5$
The law of cosines applied to the triangle $w,v_i,v_j$ with angle
$\gamma$ gives the contradiction:
   \begin{displaymath}
   4 \le c^2 = a^2 + b^2 - 2 a b \cos\gamma 
   \le a^2 + b^2 + a b < 3r^2 = 4.
   \end{displaymath}
Thus, the boundary of the truncated Voronoi cell consists of circular
arcs and chords of the circle of radius $r$ (see Figure~\ref{fig:2D-proof}).

%% WW not yet done.
\begin{figure}[htb]
  \centering
  \myincludegraphics{noimage.eps}
  \caption{The sectors and triangles of a truncated Voronoi cell}
  \label{fig:2D-proof}
\end{figure}

The parts of the Voronoi cell that lie within a circular sector have
density $1/r^2 = 3/4 < \pi/\sqrt{12}$.  The parts of the Voronoi cell
that lie within a triangle have density
   \begin{equation}\label{eqn:rog2d}
   \frac{\theta}{r^2 \cos\theta\sin\theta}
   \end{equation}
% checked 3/31/2008.
where $0 \le \theta\le \pi/6$.  An easy optimization gives the maximum
at $\theta=\pi/6$ with value $\pi/\sqrt{12}$.

In some ways it it unfortunate that the problem in two dimensions is
so elementary.  It gives meager hints about how to solve the problem
in three dimensions.  It suggests the introduction of Voronoi cells
and the usefulness of truncation.  The little optimization problem on
triangles in Equation~\ref{eqn:rog2d} generalizes to $n$-dimensions.
This is Rogers's lemma.  It is a tremendously useful lemma in the
study of packing densities.

%\subsection{two dimensions}

\bigskip

There are many proofs of Thue's theorem.  Here is a second proof that
is due to L. Fejes T\'oth.  There is a \newterm{Delaunay
  triangulation} for infinite sets $V$ of points in the plane (or more
generally in $n$-dimensions).  Assume that $V$ is the set of centers
of packing so that distances between points of $V$ are at least two.
A enlargement of the packing $V\subset V'$ increases density.  In
issues of density, it does no harm to assume that the packing is
\newterm{saturated}; that is, it is not a proper subset of another
packing.  A saturated packing of the plane (or $n$-space) has a
Delaunay triangulation.  It is a triangulation in which the
circumscribing circle of each triangle contains no points of $V$ in
its interior.  Each Delaunay triangle has radius at most $2$, for
otherwise the packing is not saturated: an additional point can be
placed at the center of the circumscribing circle.

\begin{proof}
  Admitting the existence of the Delaunay triangulation, the proof of
  the packing problem in two dimensions is elementary.  Each Delaunay
  triangle contains a portion of a disk at each of its three vertices.
  The interior angles of a triangle sum to $\pi$, giving half a disk
  per triangle.  If each triangle has area at least $\sqrt{3}$, then
  the density of the packing is at most $(\pi/2)/(\sqrt{3}) =
  \pi/\sqrt{12}$.  To minimize the area of Delaunay triangle
  $\{v_0,v_1,v_2\}$, first replace it with a smaller similar triangle
  whose shortest edge (say $v_1v_2$) has length $2$.  The third vertex
  $v_0$ is constrained to have distance at least $2$ from $v_1$ and
  $v_2$, and to give circumradius at most $2$.  The constraints on
  $v_0$ form three circular arcs as shown in
  Figure~\ref{fig:2D-FT} % not drawn.

%% WW not yet done.
\begin{figure}[htb]
  \centering
  \myincludegraphics{noimage.eps}
  \caption{The constraints on the point $A$, vertex of a Delaunay triangle}
  \label{fig:2D-FT}
\end{figure}

The minimizing triangle is determined by the point $v_0$ closest to
the segment $\{v_1,v_2\}$.  There are three such points, all giving
triangles of area exactly $\sqrt3$.  This completes the proof.
\end{proof}



\clearpage






