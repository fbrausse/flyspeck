% (c) Christoph Lange 2007
\documentclass{llncs}

% Draft?
\newif\ifdraft
\drafttrue
%\draftfalse

% \usepackage[english]{babel}
\usepackage[T1]{fontenc}
\usepackage[utf8]{inputenc}
\usepackage{lmodern}
\usepackage{textcomp}

\ifdraft
\usepackage[show]{ed}
%\usepackage{pdfsync}
\else
\usepackage[hide]{ed}
\usepackage{microtype}
\fi


% \usepackage{a4wide}
% \usepackage{amsmath}
% \usepackage{amsfonts}
% \usepackage{amstext}
% \usepackage{array}
% \usepackage{graphicx}
% \usepackage{ifthen}
% \usepackage[savemem]{listings}
% \usepackage{lstpatch}
% \usepackage{lstomdoc}
% \usepackage{makeidx}
% \usepackage{scrpage2}
% \usepackage[binary,squaren]{SIunits}
% \usepackage{supertabular}
% \usepackage{tabularx}
% \usepackage{thm2e}
% \usepackage[normalem]{ulem}
% \usepackage{wrapfig}
% \usepackage[svgnames]{xcolor}

% \usepackage{tikz}

% % Symbol fonts
% \let\RealRightarrow=\Rightarrow
% \usepackage{marvosym}
% \renewcommand{\Rightarrow}{\RealRightarrow}
% \usepackage{wasysym}

% KWARC packages
\usepackage{acronyms,myindex,semantic-markup}
% \let\Realstex=\stex
% \usepackage{paths}
% \renewcommand{\stex}{\Realstex}

% ... and adjustments
\def\omdocni{{\sc OMDoc}} % non-indexed OMDoc
\def\swimni{{\sc SWiM}} % non-indexed SWiM

\def\thetitle{Flyspeck in a Wiki -- Collaborating on Mankind's Largest Proof}

% load this last
% \definecolor{NavyBlue}{cmyk}{0.94,0.54,0,0.3}
% \usepackage[pdftex,pdfstartview=FitV,plainpages=false,pdfpagelabels,colorlinks=true,linkcolor=NavyBlue,citecolor=NavyBlue,urlcolor=NavyBlue,hypertexnames=true]{hyperref}
% \hypersetup{
%     pdfauthor = {Christoph Lange},
%     pdftitle = {\thetitle},
%     pdfkeywords = {Semantic Wiki OMDoc Ontology Services Science Mathematical
% Knowledge Management Mathematics}
% }
\usepackage{url}

% local packages (my legacy)
% \input{../macros/mathe}

\hyphenation{name-space}
\hyphenation{Me-dia-Wi-ki}

% \hypersetup{bookmarksdepth=4}

% % Page styles
% \pagestyle{scrheadings}
% \clearscrheadfoot
% \ohead{\headmark}
% \ofoot[\pagemark]{\pagemark}
% \setheadsepline{0.3pt}[\color{gray}]
% \setkomafont{pagehead}{\normalfont\small\sffamily\slshape}
% \setkomafont{pagenumber}{\normalfont\small\sffamily\slshape}
%
% % Listing styles
% \lstset{float=htb,columns=flexible,frame=lines,language=[omdoc]XML,basicstyle=\scriptsize,
%         indexstyle=\indextt,indexstyle=[1]\indexelement,indexstyle=[2]\indexattribute,
%         numbers=left,stepnumber=5,numberstyle=\tiny,showstringspaces=false}
% \lstset{basicstyle=\ttfamily,basewidth=.5em}
%
% % Array setup
% \newcolumntype{v}[1]{>{\raggedright\arraybackslash\hspace{0pt}}p{#1}}

% % TikZ setup
% \usetikzlibrary{arrows}
% \tikzstyle{default}=[font=\sffamily,>=triangle 60]
% \tikzstyle concept=[font=\sffamily\bfseries,draw,minimum height=3.5ex,rounded corners]

% Local abbreviations
\def\abSMW{\product{Semantic MediaWiki}}

\title{\thetitle}
\author{Christoph Lange\inst{1} \and Sean McLaughlin\inst{2} \and Florian Rabe\inst{3}}
\institute{Computer Science, Jacobs University Bremen\thanks{formerly
International University Bremen}, \email{\{ch.lange,f.rabe\}@jacobs-university.de} \and
School of Computer Science, Carnegie Mellon University, Pittsburgh}

\begin{document}

\maketitle

\begin{abstract}
  \begin{todo}{@Sean: In one sentence, explain that this is a HUGE proof, also touching
      many areas of maths.}
    The purpose of the Flyspeck project is to develop a formally verifiable proof of
    Kepler's century-old conjecture about packing balls in three-dimensional space.
    Hales' original proof from 1998 heavily relies on computer calculations and thus
    requires more formalization in order to be verifiable.
  \end{todo}
  
  Flyspeck is scheduled to run for around twenty years\ednote{@Sean: right? --CL} and will
  require a lot of manpower.  In order to get a community involved with formalizing
  sub-problems, we have started to publish and document them in a semantic wiki,
  exploiting the inherent structure of the proof for browsing and collaboration services.

  This paper introduces the use case and establishes requirements for a system that
  supports collaboration on the Kepler proof, and it presents a first system
  implementation based on Semantic MediaWiki.  With lessons learned from this pre-study,
  we develop ideas how the project can be supported even better by a semantic wiki
  specifically tailored to the needs of mathematicians.
\end{abstract}

@Sean, I inserted this auto-generated message. Have you ever written something together
with Florian? Then you probably know our ednote macros. Anyway, this is just FYI, and
after having read it, you may delete it :-) --Christoph
\hrule
\edexplanation
\hrule
\section{The Flyspeck Project (Sean)}
\label{sec:flyspeck}

Kepler conjecture (probably with a nice drawing?) -- one of the greatest problems of
mankind

Hales' original proof -- its enormous size (mention number of areas of mathematics
covered) and why it's not been accepted

Flyspeck project\cite{hales:DSP:2006:432}

So far on Google code (see section~\ref{sec:req})

Idea to get the community involved -- give them a wiki (leading the way to Christoph's
part: As we want to exploit certain structural properties of the proof, we'd like to have
a \emph{semantic} wiki that is capable of handling them)

\begin{enumerate}
\item make the whole extent of the proof graspable
\item make the workflows manageable
\item later maybe support proof development/proof validation right in the
  wiki?\ednote{@Sean, is this an issue? --CL}
\end{enumerate}

\begin{todo}{@Christoph}
  Motivate this with the principles of wikinomics \url{http://www.wikinomics.com}: If a
  smart but poor boy in Africa with his OLPC accesses our homepage\ldots
\end{todo}

\section{State of the Art}
\label{sec:sota}

\subsection{Mathematical Knowledge Management (Christoph/Florian?)}
\label{sec:mkm}

Project before the age of computer-supported MKM: classification of the final groups

Theorem provers and their libraries (all with a central repository and a repository
viewer): Mizar, Coq, PVS

formal and informal aspects of mathematical knowledge, content markup; OMDoc covering both
formal and informal aspects.  Actually, we're now interested in semi-formal
knowledge:
\begin{itemize}
\item documenting formal knowledge (the proof, or the theorems to be proved, resp.)
\item from informal ideas (how to prove sth.) to formal proofs
\end{itemize}

\subsection{Wikis for Mathematics (Christoph)}
\label{sec:math-wiki}

\subsubsection{Semantic Wikis}
\label{sec:semwiki}

Semantic wikis~\cite{semwiki06} are wikis enhanced by semantic annotations.  Although many
ways of semantically enhancing wikis have been investigated, a modelling approach prevails
where one resource (in the RDF\ednote{FYI: RDF can be assumed as known in this community;
  for the MKM, we'd have to explain it. --CL} sense) --- e.\,g.\ one mathematical theorem
--- is represented by one wiki page and relations between resources by links between
pages.  Both pages and links can be typed with terms from
ontologies~\cite{OrDeMoVoHa06:annotation-navigation-semwiki}, which are either preloaded
into the wiki or modelled ad-hoc~\cite{KrSchVr:semwiki-reasoning07}\ednote{@Christoph: Use
  this later: The Flyspeck wiki will support both kinds of ontologies: The OMDoc document
  ontology is preloaded, while other annotations can be added at will.}.  Semantic wikis
commonly offer enhanced navigation capabilities by displaying a summary of all typed
links, grouped by type, with each page.  Most of them allow to search for pages by type or
by them being subject or object of any RDF triple (= typed link), while it depends on the
reasoner used by the wiki whether only explicit RDF triples or also inferred ones are
considered~\cite{KrSchVr:semwiki-reasoning07}.  Such queries can usually be executed
interactively via a special search form, or in an automated way as \emph{inline} queries
embedded into the content of a page.

\subsubsection{Informal Knowledge Collections}
\label{sec:math-knowledge-collections}

Current collaborative projects for managing \emph{informal} mathematical knowledge range
from comprehensive encyclopediæ like the mathematical sections of
\product{Wikipedia}~\cite{wikipedia} or the courseware repository and content management
system \product{Connexions}~\cite{CNX:whitepaper} to projects specially focused on
mathematics like \product{PlanetMath}~\cite{krowne03:collaborative-math-libraries}, which
is powered by a highly customized wiki-like system.  The pages in these systems are
categorized and searchable in full-text, with additional metadata records in the case of
\product{PlanetMath}.  Neither of these systems is a \emph{semantic} wiki, and thus they
fail to solve the following two problems, which are essential for MKM:

\begin{enumerate}
\item\label{item:formula-search-usecase} In \product{Wikipedia} and \product{PlanetMath},
  formulæ are given in presentation-oriented {\LaTeX}.  Imagine a wiki page about the
  Pythagorean Theorem, stated as $a^2 + b^2 = c^2$, and a user searching for the
  equivalent formula $x^2 + y^2 = z^2$ (or even $c=\sqrt{a^2+b^2}$!) --- The system would
  not find the theorem.
\item Neither could ``all theorems about triangles for which a
  proof exists'' be searched for, as the link from a proof to the theorem it proves is not
  typed.
\end{enumerate}

\product{Connexions}, on the other hand, could in principle cope with these two problems,
but in practice it does not: Formulæ are written in the content-oriented sublanguage of
{\mathml}~\cite{CarlisleEd:MathML07}, and the CNXML markup language used for larger
structures allows for annotating texts as mathematical statements like lemmas, but this
structural information is not yet \emph{used} by the system.  Moreover, none of the
systems mentioned so far supports an easy navigation from the occurrence of a mathematical
symbol in a formula to the declaration or definition of this symbol, if it is defined in
some other place of the wiki; instead, the author has to provide links he considers
relevant in the text surrounding the formula.

\subsubsection{Domain-Specific Semantics}
\label{sec:domain-semantics}

Note that general-purpose semantic wikis do not support the above-mentioned use case
(\ref{item:formula-search-usecase}) either, as they neither have a sufficient notion of
equality nor understand mathematical content markup.  If we assume ``semantic'' not just
to mean RDF or description logics, but any kind of (higher-order) logic required for
specific domains\ednote{@Florian: This is quite superficial, can we write it in a more
  sophisticated way?} and employ domain-specific ways of knowledge representation we can
imagine semantic wikis specifically supporting mathematics.  For use case
(\ref{item:formula-search-usecase}), we could have the wiki pages crawled by a formula
search engine like MathWebSearch~\cite{KohSuc:asemf06}, which applies substitution tree
indexing to mathematical formulae.  Even more formal approaches integrate automated
theorem provers into wikis (see section~\ref{sec:related}).

\section{Supporting Flyspeck in a Semantic Wiki (Christoph)}

Starting from requirements how collaborating on formalizing the Kepler proof should be
supported, we have conducted a pre-study in a general-purpose semantic wiki in order to
see what possibilities this technology can offer Flyspeck.  Based on the results of this
study, we present domain-specific technologies developed in our group and establish a work
plan for tailoring a semantic wiki to the Flyspeck project.

\subsection{Requirements (Christoph, Sean)}
\label{sec:req}

So far, the Flyspeck project has had three\ednote{@Sean, fix the number!} core members who
collaborated via Google Code~\cite{flyspeck:web}.  While the services offered by Google
Code (a Subversion repository, a mailing list, and others) were found to be sufficient for
the core development team, we were not satisfied with the wiki integrated into the Google
Code web interface.  Lacking support for mathematical formulae, it would not even allow
for presenting the theorems and lemmas to be formally proved in a human-readable fashion.
Secondly, besides untyped links and adding labels to pages it does not offer any further
\emph{structuring} support, which is essential for browsing and querying a large
knowledge collection.

In this early phase of ``crowdsourcing'' Flyspeck, the focus is not yet on developing and
checking formal proofs collaboratively, but on making its extent and structure
comprehensible and on communicating where work needs to be done.  For this the outline of
the whole proof\ednote{@Sean, what's the best way to introduce the book? --CL} needs to be
represented in the wiki, where the mathematical statements (including definitions, lemmas,
and theorems) are available in a human-readable way (with formulae in \LaTeX\ or
presentational MathML) as well as a machine-readable presentation suitable for downloading
into a theorem prover.  In order to obtain a well-structured network of knowledge items,
each mathematical statement should be presented on one wiki page, which shows its
human-readable representation from the book, offers additional space for annotation, and
allows for downloading a formal representation\ednote{@Sean: Will the structure of the
  wiki mirror the structure of the book only in the beginning, or forever? Will the proof
  be completely rearranged (e.g. because of a new axiomatization), or will the book be
  worked off section after section? --CL}.  The following kinds of annotations are
desirable:

\begin{description}
\item[Categorization by topic:] In the beginning, one would mirror the sectioning structure
  of the book (e.\,g.\ ``ball'' being a subsection of ``primitive volumes'', which in turn
  is a section of the chapter ``volume calculations'').  Standardized ways of classifying
  mathematical topics, such as the Mathematical Subject Classification
  (MSC)\ednote{reference}, could be added later.
\item[Project-organization metadata] such as the information whether a lemma has already
  been proven formally.\ednote{@Sean: more?}
\item[Dependency links:] These can be links from individual symbols in mathematical
  formulae to the place where they are defined, or from any page $p$ to other pages
  containing knowledge that is required for understanding $p$ --- either pages in the same
  wiki, or external resources like PlanetMath or Wikipedia articles.\ednote{Note: If we
    import from a formal source (like Twelf), we can auto-generate some of them.}
\item[Discussion posts] should be strongly tied to the topic being discussed, and they
  should be classified into categories like question, answer, explanation,
  etc.
\end{description}

To the visitor and potential collaborator, an impression of the extent and structure of
the project --- its enourmous size and its specialization into diverse fields of
mathematics --- must be given, as well as tools for browsing and querying the knowledge.
The topical structure as well as the dependencies must be browsable via links.  Not only
should it be possible to query knowledge items by their annotations, but important query
results must also be available as dynamically generated lists.  Examples for queries are:

\begin{itemize}
\item ``Which lemmas about composite regions have not been formally proven so far?''
\item ``What do I need to read in order to understand Jordan's curve
  theorem?''\ednote{internal dependency graph, tutorial, planetmath}
\item ``What lemmas are difficult to prove?''
  \begin{itemize}
  \item \ldots in the sense that many invalid proofs have already been submitted
  \item \ldots in the sense that many people have asked questions in the related discussion
  \end{itemize}
\item \ldots\ednote{@Christoph: develop some ideas about which dynamic queries are
    relevant. Check Matthias' BSc thesis. --CL}
\end{itemize}

A volunteer who is willing to work out and contribute a formal proof for some lemma should
be able to download a self-contained formal representation of this lemma and all symbol
declarations it depends on\ednote{@Florian/Sean: Can a lemma depend on anything
  else?}\issue{Is a dependency graph like this sufficient, or could it also be of
  advantage to download more? E.\,g.\ other lemmas that have already been proved and could
  be utilized? Would that ultimately require a download of \emph{everything}, or can we be
  smarter? --CL}\ednote{@Florian: Do we want to convert between theorem proving languages?
  Is this feasible with OMDoc?}.

While a small group of maintainers will integrate submitted formal proofs into one central
proof script outside of the wiki\ednote{and check it}, metadata \emph{about} the progress
of the proof and a human-readable outline of the proof\ednote{@Sean: This as well?} will
be added to the wiki.

\ednote{@Christoph, create a diagram!}:

\begin{enumerate}
\item They work out a proof and submit it back\issue{@Sean: How??? Upload to the wiki,
    auto-conversion back to Twelf, or to a human maintainer who manually integrates it
    into the ``one big Twelf file''? --CL}
\end{enumerate}

In a later phase, it has to be investigated whether formal proofs can also be
collaboratively developed and validated in the wiki (cf.\ section~\ref{sec:related}).

\subsection{Pre-Study in Semantic MediaWiki (Christoph)}
\label{sec:smw-study}

For reasons of rapid prototyping, we chose Semantic
MediaWiki~\cite{KrSchVr:semwiki-reasoning07} for a pre-study in order to gain intuition
about how the project could be managed.  Semantic MediaWiki offers rich text formatting
and structuring capabilities, above all \LaTeX\ formulae, typing of pages and links,
ad-hoc prototyping of ontologies, and inline queries.  Extensions and plugins can easily
be developed using a well-documented API.

\begin{todo}{@Christoph: describe the current implementation with a few
    screenshots/mockups}
  \begin{itemize}
  \item import the one large Twelf file
  \item break it down into declarations, add annotations gained from the symbol IDs
    (e.\,g.\ ``instanceOf Lemma'')\issue{@Sean, can we also gain sth.\ valuable from the
      Twelf declarations themselves, without too much effort? --CL}
  \item declarations go to individual wiki pages
  \item Twelf presented as \LaTeX, also broken down into fragments
  \item Aggregate each pair of Twelf/\LaTeX\ fragment to one wiki page that includes both
    and offers space for annotations
  \item Users can make formal or informal annotations: e.\,g.\ plain-text comments, or
    further semantic links or categorizations
  \item Show the potential of inline queries: all unproven lemmas in the area of graph
    theory
  \end{itemize}
\end{todo}

\subsection{SWiM, a Semantic Wiki for Mathematical Knowledge Management (Christoph)}
\label{sec:swim}

\begin{todo}{@Christoph: Elaborate on this discussion with Florian}
  OMDoc is agnostic towards logics -- that could be a benefit as long as we do not yet
  have a proof object. Work in progress, nodes without content, loosely coupled: Ideal
  setting for a wiki!  Wiki as a means of information for the collaborators about the
  progress of the \emph{whole} project.  Make a proof that's \emph{so} complex
  comprehensible in some way, show open issues/problems to potential collaborators.
  There's a whole book containing the informal outline of the proof; link the wiki to the
  book.  Probably also import the book via sTeX to the wiki.\\
  Make the whole project publicly viewable (e.g. for a progress report), make it
  manageable.\\
  Analogy: from software documentation to literate programming
\end{todo}

short intro: OMDoc~\cite{Kohlhase:omdoc1.2}, SWiM~\cite{Lange:swmkm-tr07}, document
ontology, envisaged services

We have developed {\swim}, a semantic wiki for mathematics~\cite{Lange:swmkm-tr07}, which
is an enhanced \product{IkeWiki}~\cite{KrSchVr:semwiki-reasoning07} with {\omdoc} as its
page format.

\subsection{Outlook: Towards Full Flyspeck Support in SWiM (mostly Christoph)}
\label{sec:flyspeck-swim}

first lessons learned from the pre-study

use OMDoc as universal exchange format between ATP languages; every ATP system so far is
an island; OMDoc makes ATP scale to the web

Features that rely on \emph{mathematical} structures are quite kludgy in a PHP/text/regexp
way.  Mathematical content markup for formulae (OMDoc/OpenMath) would be better.  Example:
Linking symbols to their declarations has already been done for OMDoc.

\issue{@Florian, do you think we could make use of MathWebSearch for certain services?
  (@Sean, that's our semantic math formula search engine.)  Now that I've mentioned the
  Pythagoras example, we could think about it. --CL}

\section{Related Work (all)}
\label{sec:related}

\ednote{Again, I'll concentrate on wiki-like collaborative systems.  Is there any related
  work from the pure theorem-proving point of view?  Is there any proof comparable to
  Kepler's? --CL}

\begin{todo}{It will be some effort to distance our approach from ProofWiki or Logiweb,
  to explain why we're not doing it their way. Actually, the Semantic Web community might
  not care, as they do not know our competitors, but the MKM community will
  care.\\
  @Christoph: check the claims made here!  Find out whether such systems have already
  been used in Flyspeck-like scenarios, i.\,e.\ collaboratively proving something.\\
  @Florian: You've also heard these talks at MKM -- I might need your help to justify why
  we're going a different way.}  Recent wikis with integrated proof assistance do not yet
exploit structural semantics of the mathematical knowledge other than what they feed to
the proof assistant or the theorem prover.

  Logiweb~\cite{Grue:Logiweb07} is an open source, distributed system for publication of
  machine checked mathematics.  While the author does not call his system a ``wiki'', it
  is actually similar in spirit\ednote{@CL: in what way? --CL}.

  Arguments against\issue{@Sean/Florian: I need your help with this.  Actually, I just
    started doing the Flyspeck wiki my way, without considering such possible
    alternatives. --CL}:
  \begin{itemize}
  \item uses presentational {\TeX} for output (but hyperlinked symbols are at least nice
    to have for Flyspeck)
  \item no internal support for hyperlinks (required for browsing the Flyspeck)
  \item no query/search/retrieval (essential for Flyspeck)
  \item no import/export features $\Rightarrow$ not connectable to external theorem
    provers (essential for Flyspeck!), calculi and proof tactics need to be defined in the
    system itself.  Would be a nice idea to run Flyspeck this way, but we're not going to
    do that\issue{@Sean/Florian: Why? Or should we? --CL}
  \end{itemize}

  ProofWiki~\cite{CorKal:CoopReposFormalProofs07}\ednote{FYI, see also
    \url{http://homepages.inf.ed.ac.uk/rpollack/mathWiki/slides/Corbineau.pdf}}:
  wiki-aware web interface for proof assistants

  Pros:
  \begin{itemize}
  \item symbols linked to their definition
  \item import/export is possible
  \item auto-generates index pages (e.\,g.\ all definitions, all theorems)
  \item Could probably be combined with Semantic MediaWiki extension?  (Because it's also
    based on MediaWiki)
  \end{itemize}

  Cons:
  \begin{itemize}
  \item not a semantic wiki in the sense that the wiki system \emph{itself} uses a
    semantic representation of the wiki pages: just text search and rather static index
    pages, more flexible queries not supported (at least they have identified this as a
    problem)
  \item except for hyperlinks, no advanced browsing support.  With a semantic wiki
    (e.\,g.\ flexibly inferred dependencies), we can do better.  They just use
    dependencies for exporting proof scripts.
  \item no place for free-form wiki text \emph{on the same page} with a proof script, as
    wiki pages are entirely replaced by proof script editor $\Rightarrow$ can only use
    proof script comments for annotations, or separate wiki pages for informal-only text
    (Is this actually a disadvantage?)
  \item Only supports Coq so far.  They suggest how other provers could be integrated, but
    this looks like a \emph{lot} of work.  We can do better, with OMDoc as a universal
    exchange language.
  \end{itemize}
\end{todo}

\section{Conclusion}
\label{sec:conc}

\ednote{Is there anything we could already conclude? --CL}

\paragraph{Acknowledgments}
\label{sec:ack}

\begin{itemize}
\item Michael Kohlhase
\item Immanuel Normann
\item Robert J.\ Simmons
\item Stefan Decker?
\end{itemize}

\bibliographystyle{abbrv}
% load crossrefs last when using modular bib files
\bibliography{kwarc}

% \printindex

\ednotemessage
\end{document}

% vim:tw=80:autoindent: 
%%% Local Variables: 
%%% mode: latex
%%% fill-column: 90
%%% End: 
