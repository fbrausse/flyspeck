% (c) Christoph Lange 2007
\documentclass{llncs}

% Draft?
\newif\ifdraft
\drafttrue
%\draftfalse

% \usepackage[english]{babel}
\usepackage[T1]{fontenc}
\usepackage[utf8]{inputenc}
\usepackage{lmodern}
\usepackage{textcomp}

\ifdraft
\usepackage[show]{ed}
\usepackage{pdfsync}
\else
\usepackage[hide]{ed}
\usepackage{microtype}
\fi


% \usepackage{a4wide}
% \usepackage{amsmath}
% \usepackage{amsfonts}
% \usepackage{amstext}
% \usepackage{array}
% \usepackage{graphicx}
% \usepackage{ifthen}
% \usepackage[savemem]{listings}
% \usepackage{lstpatch}
% \usepackage{lstomdoc}
% \usepackage{makeidx}
% \usepackage{scrpage2}
% \usepackage[binary,squaren]{SIunits}
% \usepackage{supertabular}
% \usepackage{tabularx}
% \usepackage{thm2e}
% \usepackage[normalem]{ulem}
% \usepackage{wrapfig}
% \usepackage[svgnames]{xcolor}

% \usepackage{tikz}

% % Symbol fonts
% \let\RealRightarrow=\Rightarrow
% \usepackage{marvosym}
% \renewcommand{\Rightarrow}{\RealRightarrow}
% \usepackage{wasysym}

% KWARC packages
\usepackage{acronyms,myindex,semantic-markup}
% \let\Realstex=\stex
% \usepackage{paths}
% \renewcommand{\stex}{\Realstex}

% ... and adjustments
\def\omdocni{{\sc OMDoc}} % non-indexed OMDoc
\def\swimni{{\sc SWiM}} % non-indexed SWiM

\def\thetitle{Flyspeck in a Wiki -- Collaborating on a Huge Mathematical Proof}

% load this last
% \definecolor{NavyBlue}{cmyk}{0.94,0.54,0,0.3}
% \usepackage[pdftex,pdfstartview=FitV,plainpages=false,pdfpagelabels,colorlinks=true,linkcolor=NavyBlue,citecolor=NavyBlue,urlcolor=NavyBlue,hypertexnames=true]{hyperref}
% \hypersetup{
%     pdfauthor = {Christoph Lange},
%     pdftitle = {\thetitle},
%     pdfkeywords = {Semantic Wiki OMDoc Ontology Services Science Mathematical
% Knowledge Management Mathematics}
% }
\usepackage{url}

% local packages (my legacy)
% \input{../macros/mathe}

\hyphenation{name-space}
\hyphenation{Me-dia-Wi-ki}

% \hypersetup{bookmarksdepth=4}

% % Page styles
% \pagestyle{scrheadings}
% \clearscrheadfoot
% \ohead{\headmark}
% \ofoot[\pagemark]{\pagemark}
% \setheadsepline{0.3pt}[\color{gray}]
% \setkomafont{pagehead}{\normalfont\small\sffamily\slshape}
% \setkomafont{pagenumber}{\normalfont\small\sffamily\slshape}
%
% % Listing styles
% \lstset{float=htb,columns=flexible,frame=lines,language=[omdoc]XML,basicstyle=\scriptsize,
%         indexstyle=\indextt,indexstyle=[1]\indexelement,indexstyle=[2]\indexattribute,
%         numbers=left,stepnumber=5,numberstyle=\tiny,showstringspaces=false}
% \lstset{basicstyle=\ttfamily,basewidth=.5em}
%
% % Array setup
% \newcolumntype{v}[1]{>{\raggedright\arraybackslash\hspace{0pt}}p{#1}}

% % TikZ setup
% \usetikzlibrary{arrows}
% \tikzstyle{default}=[font=\sffamily,>=triangle 60]
% \tikzstyle concept=[font=\sffamily\bfseries,draw,minimum height=3.5ex,rounded corners]

% Local abbreviations
\def\abSMW{\product{Semantic MediaWiki}}

\title{\thetitle}
\author{Christoph Lange\inst{1} \and Sean McLaughlin\inst{2} \and Florian Rabe\inst{3}}
\institute{Computer Science, Jacobs University Bremen\thanks{formerly
International University Bremen}, \email{\{ch.lange,f.rabe\}@jacobs-university.de} \and
School of Computer Science, Carnegie Mellon University, Pittsburgh}

\begin{document}

\maketitle

\begin{abstract}
  \begin{todo}[Sean]{In one sentence, explain that this is a HUGE proof, also touching
      many areas of maths.}
    The purpose of the Flyspeck project is to develop a formally verifiable proof of
    Kepler's century-old conjecture about packing balls in three-dimensional space.
    Hales' original proof from 1998 heavily relies on computer calculations and thus
    requires more formalization in order to be verifiable.
  \end{todo}
  
  Flyspeck is scheduled as a long-term project\ednote{@Sean: How many years do you
    estimate? --CL} that will require a lot of manpower.  In order to get a community
  involved with formalizing sub-problems, we have started to publish them in a semantic
  wiki, exploiting the inherent structure of the proof for browsing and collaboration
  services.

  This paper introduces the use case and establishes requirements for a system that
  supports collaboration on the Kepler proof, and it presents a first system
  implementation based on Semantic MediaWiki.  With lessons learned from this pre-study,
  we develop ideas how the project can be supported even better by a semantic wiki
  specifically tailored to the needs of mathematicians.
\end{abstract}

\section{The Flyspeck Project (Sean)}
\label{sec:flyspeck}

Kepler conjecture (probably with a nice drawing?)

Hales' original proof

Flyspeck project\cite{hales:DSP:2006:432}

Idea to get the community involved

\section{State of the Art}
\label{sec:sota}

\subsection{Automated Theorem Proving (Sean/Florian?)}
\label{sec:atp}

\ednote{What kind of logics/proving techniques will be used? --CL}

\subsection{Mathematical Knowledge Management (Christoph/Florian?)}
\label{sec:mkm}

managing large collections of proofs\ednote{@Florian: Does it make sense to mention things
  like Mizar here? --CL}

formal and informal aspects of mathematical knowledge\ednote{Already mention OMDoc, which
  covers both, here? --CL}


\subsection{Semantic Wikis (Christoph)}
\label{sec:semwiki}

\begin{itemize}
\item Very short intro to wikis, problems that wikis have with formal knowledge (the
  ``Pythagoras'' example again: formula search and statement-level queries)
\item Short intro to semantic web, RDF, ontologies
\item Semantic Wikis: predominant model page=resource, common ways of annotating and
  navigating
\end{itemize}

\section{Supporting Flyspeck in a Semantic Wiki (Christoph)}

\subsection{Requirements}
\label{sec:req}

\ednote{@Sean: What problem do we want to solve/What does Flyspeck need? --CL}

\begin{todo}[Christoph]{Contribution to the semantic web community}
  Where's the ``semantics'' in Flyspeck+Wiki? (Utilize certain \emph{structures} of the
  given knowledge for navigation and other services!)
\end{todo}

\begin{todo}[Christoph/Florian]{Contribution to the MKM community; but a bit less of that
    for the semantic web paper}
  Stress the fact that we also want to collect informal annotations/comments about the
  parts of the proof.  Otherwise, our implementation wouldn't offer anything beyond the
  state of the art defined by other collaborative/wikified proof assistants (see related
  work).
\end{todo}

\begin{newpart}{Let me try to describe the requirements as I understood them}
  We envisage the work flow as follows\ednote{@Christoph, create a diagram!}:
  \begin{enumerate}
  \item The proof is maintained in one large Twelf file, which allows for checking
    everything at once.\ednote{@Sean, what's the deeper reason? Are there just so many
      interdependencies that we could not handle otherwise? --CL}
  \item The individual sections of this file (technically, all these are type
    declarations) are imported into the wiki to form separate wiki pages.
  \item On each page, a human-readable rendering is given (e.\,g.\ \LaTeX\ or MathML)
  \item There is a space for annotations.  Annotations can be:
    \begin{enumerate}
    \item Categorizations (e.\,g.\ field of mathematics\ednote{Would MSC help here?})
    \item Project-organization metadata (proven/worked off)
    \item Linking (what kinds of links make sense?)
    \item Discussion posts
    \end{enumerate}
  \item Note that some annotations (e.\,g.\ logical dependencies) can be auto-generated
    from the Twelf source upon import!
  \item There are some overview pages, like a list of theorems unproven so far
  \item Volunteers/collaborators can download/export single theorems (and their
    dependencies, to make it self-contained!) in their favorite theorem-proving language
    (e.\,g.\ Isabelle\ednote{@Sean/Florian: What others are likely to be used? --CL})
  \item They work out a proof and submit it back\ednote{@Sean: How??? Upload to the wiki,
      auto-conversion back to Twelf, or to a human maintainer who manually integrates it
      into the ``one big Twelf file''? --CL}
  \end{enumerate}
\end{newpart}

\section{Pre-Study in Semantic MediaWiki (Christoph)}
\label{sec:smw-study}

Why Semantic MediaWiki\cite{kroetzsch06:semantic-mediawiki}?
\begin{itemize}
\item rapid prototyping (PHP)
\item well-documented extension/plugin API
\item math support (well, \TeX)
\item powerful and flexible templates/includes
\item Pragmatic reasons\ednote{Mention, or rather conceal?}:
  \begin{itemize}
  \item SWiM not yet ready for Flyspeck requirements
  \item we had an installation available (MathWeb)
  \end{itemize}
\end{itemize}

\begin{todo}[Christoph]{describe the current implementation with a few
    screenshots/mockups}
  \begin{itemize}
  \item import the one large Twelf file
  \item break it down into declarations, add annotations gained from the symbol IDs
    (e.\,g.\ ``instanceOf Lemma'')\ednote{@Sean, can we also gain sth.\ valuable from the
      Twelf declarations themselves, without too much effort? --CL}
  \item declarations go to individual wiki pages
  \item Twelf presented as \LaTeX, also broken down into fragments
  \item Aggregate each pair of Twelf/\LaTeX\ fragment to one wiki page that includes both
    and offers space for annotations
  \item Users can make formal or informal annotations: e.\,g.\ plain-text comments, or
    further semantic links or categorizations
  \item Show the potential of inline queries: all unproven lemmas in the area of graph
    theory
  \end{itemize}
\end{todo}

\section{SWiM, a Semantic Wiki for Mathematical Knowledge Management (Christoph)}
\label{sec:swim}

short intro: OMDoc~\cite{Kohlhase:omdoc1.2}, SWiM~\cite{Lange:swmkm-tr07}, document
ontology, envisaged services

\section{Outlook: Towards Full Flyspeck Support in SWiM (mostly Christoph)}
\label{sec:flyspeck-swim}

first lessons learned from the pre-study

use OMDoc as universal exchange format between ATP languages

Features that rely on \emph{mathematical} structures are quity kludgy in a PHP/text/regexp
way.  Mathematical content markup for formulae (OMDoc/OpenMath) would be better.  Example:
Linking symbols to their declarations has already been done for OMDoc.

\ednote{@Florian, do you think we could make use of MathWebSearch for certain services?
  --CL}

\section{Related Work (all)}
\label{sec:related}

ProofWeb (wiki-aware web interface for proof assistants), see
\url{http://www.cs.ru.nl/~cek/proofweb/}

LogiWeb: interactive ``content management system'' for mathematical proofs: from proof
verification (programmable in the system, or by external tools) to \LaTeX\ publishing; see
\url{http://logiweb.eu}.

\paragraph{Acknowledgments}
\label{sec:ack}

\bibliographystyle{abbrv}
% load crossrefs last when using modular bib files
\bibliography{kwarc}

% \printindex

\ednotemessage
\end{document}

% vim:tw=80:autoindent: 
%%% Local Variables: 
%%% mode: latex
%%% fill-column: 90
%%% End: 
