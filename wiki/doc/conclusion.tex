
\section{Conclusion and Further Work}
\label{sec:conc}

Our preliminary experiments lead us to believe that, due to its rich
semantic web and OMDoc infrastructure, future work toward supporting
Flyspeck should continue in the SWiM infrastructure.  For the text-based
page format of MediaWiki, features that rely on structures like the
linking of symbols could only be realized in an ad hoc way
using, say, regular expressions.  Relying on the XML infrastructure of
OMDoc, these features are either already available or easier to develop.
However, rapidly \emph{prototyping} our first ideas about the wiki
support required for Flyspeck was easier in Semantic MediaWiki due to
its ability to design ad hoc ontologies and its implementation in the
interpreted language PHP.

For this case study, we created OMDoc from Twelf. OMDoc also offers
support for the alternative workflow of stepwise formalization as
well.  One could either start by converting the Flyspeck book from
{\LaTeX} to HTML with Presentation MathML formulæ and formalize the
presentation markup into content markup step by step, or one could
start the formalization on the {\TeX} side.  There, one would
formalize the book to s\TeX{}, a content-oriented {\TeX} notation for
OMDoc, which can then be converted to OMDoc\cite{Kohlhase:albwo06}.
Either way involves a {\TeX}-to-XML transformation, which has been
tested in large scale in our group\cite{URL:arXMLiv}.

If several parts of the proof are done in different theorem provers,
highly non-trivial and mostly novel translations become necessary to
provide one single proof object. Here OMDoc could be used as an
exchange format between theorem prover languages, and formal
translations could be specified in OMDoc itself.  While this line of
research is interesting, it is difficult for us to forsee what kinds
of translations, if any, will be needed.

Using the module system of OMDoc could help to simplify
the structure of the proof. In any case it can be used to explicate
the dependencies between components of the proof. Additionally,
proof search will be greatly simplified if the semantic-aware search engine
MathWebSearch\cite{KohSuc:asemf06} is used, which applies substitution
tree indexing to mathematical formulae.

\ednote{@Christoph: Needs to support both preloaded and ad-hoc
  ontologies: The OMDoc document ontology is preloaded, while other
  annotations can be added at will. (Make SWiM more ``wiki''!)}

More ontology-powered services, particularly ones that facilitate editing
documents, are planned for version 0.3\cite{swim-roadmap,Lange:SWiMSciColl07}

\subsubsection{Browse: }

\ednote{@Christoph: narrative structure: do not use ad-hoc categories, but OMDoc's native
  NarCons\cite{KohMueMue:dfncimk07}; need to be added to document ontology.}

\subsection{Download:}

Dependencies can partly be inferred by a DL reasoner, but for a complete support
of OMDoc's notion of dependency, an OMDoc-specific calculus will have to be
applied, which is currently in development.

\subsection{Upload/reintegrate}: Needs investigation; see related work.

%%% Local Variables: 
%%% mode: latex
%%% TeX-master: "flyspeck-wiki-eswc08"
%%% End: 
