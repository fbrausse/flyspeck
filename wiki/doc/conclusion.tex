
\section{Conclusion and Further Work}
\label{sec:conc}

The experiments with Semantic MediaWiki and SWiM led to the conclusion to
integrate further support for Flyspeck in SWiM, using its rich semantic web and
OMDoc infrastructure.  Features that rely on structures like the linking
of symbols could only be realized in a very prototypical way for the text-based
page format of MediaWiki, using regular expressions.  Relying on the XML
infrastructure of OMDoc, these features are easier to develop, or already
available.  However, rapidly \emph{prototyping} our first ideas about the wiki
support required for Flyspeck was easier in Semantic MediaWiki due to its
ability to design ad-hoc ontologies and its implementation in the interpreted
language PHP.

For this case study, we created import-ready OMDoc from Twelf, but OMDoc also
offers excellent support for the alternative workflow of stepwise formalization
as well\cite[chap.\ 4]{Kohlhase:omdoc1.2}.  One could either start with
converting the Flyspeck book from {\LaTeX} to HTML with Presentation MathML
formulæ and step by step formalize the presentation markup into content markup,
or one could start the formalization on the {\TeX} side.  There, one would
formalize the book to s\TeX{}, a content-oriented {\TeX} notation for OMDoc,
which can then be converted to OMDoc\cite{Kohlhase:albwo06}.  Either way
involves a {\TeX}-to-XML transformation, which has been tested in large scale in
our group\cite{URL:arXMLiv}.

\subsection{Annotate}

(Christoph about the informal stuff)

\ednote{@Christoph: Needs to support both preloaded and ad-hoc ontologies: The OMDoc
  document ontology is preloaded, while other annotations can be added at will. (Make SWiM
  more ``wiki''!)}


\subsubsection{Browse: }


\ednote{@Christoph: narrative structure: do not use ad-hoc categories, but OMDoc's native
  NarCons\cite{KohMueMue:dfncimk07}; need to be added to document ontology.}


%%%%%%%%%%%%%%%%%%%

\textbf{Florian, your theory refactoring goes here}

%%%%%%%%%%%%%%%%%%%

\subsection{Search/Query (Christoph)}



  In \product{Wikipedia} and \product{PlanetMath}, formulæ are given in
  presentational {\LaTeX}.  If the Pythagorean Theorem were represented as $a^2
  + b^2 = c^2$, and a user searched for the equivalent formula $x^2 + y^2 = z^2$
  (or even $c=\sqrt{a^2+b^2}$!), the system would not find the theorem.


\begin{oldpart}{Use some words of this somehow}
  Note that general-purpose semantic wikis do not support the above-mentioned
  use case either, as they neither have a sufficient notion of equality nor
  understand mathematical content markup.  If we assume ``semantic'' not just to
  mean RDF or description logics, but any kind of (higher-order) logic required
  for specific domains\ednote{@Florian: This is quite superficial, can we write
    it in a more sophisticated way?} and employ domain-specific ways of
  knowledge representation we can imagine semantic wikis specifically supporting
  mathematics.  For the Pythagoras use case, we could have the wiki pages
  crawled by a formula search engine like MathWebSearch\cite{KohSuc:asemf06},
  which applies substitution tree indexing to mathematical formulae.
\end{oldpart}

\issue{@Florian, do you think we could make use of MathWebSearch for certain services?
  (@Sean, that's our semantic math formula search engine.)  Now that I've mentioned the
  Pythagoras example, we could think about it. --CL}

\subsection{Download:}

(dependency inference? Move over from SWiM eval)

%%%%%%%%%%%%%%%%%%%

\textbf{Florian, please some words about prover language conversion}

use OMDoc as universal exchange format between ATP languages; every ATP system so far is
an island; OMDoc makes ATP scale to the web.  At least definitions.  However, we're not
yet sure\ednote{@Florian: right?}, whether we can actually rely on OMDoc, as translation
of \emph{proofs} is not that trivial.\ednote{@Sean/Florian: why?}

  OMDoc is agnostic towards logics -- that could be a benefit as long as we do
  not yet have a proof object.


%%%%%%%%%%%%%%%%%%%

\subsection{Upload/reintegrate}: Needs investigation; see related work.

\subsubsection{Snippets/Thoughts}



%%% Local Variables: 
%%% mode: latex
%%% TeX-master: "flyspeck-wiki-eswc08"
%%% End: 
