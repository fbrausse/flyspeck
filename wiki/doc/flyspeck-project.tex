
\section{The Flyspeck Project}
\label{sec:flyspeck}
  The target of our case study is the Flyspeck Project, which seeks
to formally verify Thomas Hales' 2005 proof of the Kepler Conjecture.  
The Kepler Conjecture asserts that the density of a packing of unit spheres
is at most $\pi/(3\sqrt{2})$, the density of the face centered cubic and 
hexagonal close packings.  The conjecture, posed by Kepler in 1611, formed part
of Hilbert's 18th problem, and was recognized as one of the most famous unsolved
problems of mathematics.    

  Hales' proof, completed in 1995, was not accepted immediately by the mathematical community. 
Besides its considerable length,  the proof relies essentially on computer calculations.  
The 300 pages of text and many thousands of lines of computer code made
checking the proof for errors in the referee process unusually difficult,
leading to a publication delay of nearly 10 years.  
In 2003, Hales proposed using computers to rigorously check the entire proof in
detail, including the computer code.  He dubbed this
effort the \textit{Flyspeck Project}.  
The software systems used in such formalizations are called \textit{theorem provers}
or \textit{proof assistants}.
Beginning with a set of axioms and inference rules, they can,
with adequate human guidance, verify that a purported proof follows from the axioms.  
Examples of proof assistants are Isabelle\cite{Isabelle}, Coq\cite{Coq}, 
and HOL Light\cite{HOLL}. (In the rest of this paper, we use ``formalize'' to mean
that the theorem, proof or definition has been expressed in one of these systems.) 

  Unfortunately, modern proof assistants are still far from being able to check
proofs at the level given in most journals and mathematics textbooks.  A rough
estimate is that it takes about a week to formalize a single page of mathematical
text.  Based on such estimates, Hales' expects that it will take 
around 20 man-years to complete the Flyspeck project.  

The first steps have already been taken.  Nipkow and
Bauer\cite{FlyspeckI:Nipkow} proved a fundamental algorithm in 
Isabelle\cite{Isabelle:definition}, and other parts of the code are
currently being investigated.  Hales is currently compiling a book\cite{FlyspeckBook}
of lemmas from different areas of mathematics that need to be formalized.    
The book is currently 450 pages.  
There is a project page\cite{GoogleCode:Flyspeck}
for Flyspeck hosted at GoogleCode\cite{GoogleCode}.  That page has a source repository
containing the book of lemmas and the definitions, in HOL Light, of some important functions
and inequalities.  Overall, though, the project is still in its infancy.

The Flyspeck project has garnered significant enthusiasm in the theorem proving
community: Hales has given talks about the project at a number of
prestigious conferences, and aspects of Flyspeck are currently the topic of at least
3 PhD dissertations.  Despite the interest, however, it is now somewhat difficult to 
determine the current state of the project or to get involved.  

%Flyspeck project\cite{hales:DSP:2006:432}

% So far on Google code (see section~\ref{sec:req})

\subsection{Requirements}
In this paper, we seek to remedy this situation.  We propose the use of a 
semantic wiki to organize and present details about the current state of Flyspeck.
An example scenario is as follows.  

\begin{enumerate}  %
\item A user wishes to contribute to Flyspeck. %
\item She looks at our wiki main page, which shows her what still needs to be done. %
\item Prefering analysis to geometry, she searches for open problems involoving analysis.  %
  This returns a list of lemmas related to analysis from which she can choose one that %
  seems possible given her time constraints %
\item She downloads the relevant formal definitions and lemmas, along with the text %
  of a paper proof culled from Hales' book. %
\item She uses a proof assistant to begin formalizing the paper proof (or some variant thereof) %
\item She needs clarification on some definition and additionally has an idea on how to generalize this %
  lemma.  She thus asks for help and makes comments on the wiki forum.   %
\item She completes her proof, and uploads the proof assistant file to the wiki. %
  The wiki checks it for correctness and then adds it to the database.  That lemma will %
  now not appear in the list of unfinished lemmas. %
\end{enumerate}  %


!%\ \ednote{@Christoph !%\ %
!%\   Motivate this with the principles of wikinomics \url{http://www.wikinomics.com}: If a !%\ %
!%\   smart but poor boy in Africa with his OLPC accesses our homepage\ldots !%\ %
!%\   (Sean) not sure if this wikinomics stuff should go here or in the introduction or the conclusion...} !%\ %


With this scenario in mind, we propose that the wiki should minimally offer: %

\begin{enumerate}  %
\item A database of the theory, constant, and lemma definitions %
\item A way to browse the database by category, or search with keywords %
\item A way to download the definitions and lemmas statements and proofs (when they exist).   %
\item A way to upload new proofs %
\item A forum to discuss issues involved in the formalizations %
\end{enumerate}  %
