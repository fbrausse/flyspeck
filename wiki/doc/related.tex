\section{Related Work}
\label{sec:related}

\begin{background}
Outside of wikis, the combination of computerized proofs and human-readable text
has been investigated in Isar~\cite{Wenzel:isar99}, an alternative literate
programming language for Isabelle, and in Mizar~\cite{MizarKB}, whose language
of Mizar is close to mathematical vernacular.  In contrast to Isar, there is a
large web-based library of Mizar proofs. It is browsable and searchable on the
web but managed in a centralized and hierarchical way, which is not comparable
to wiki collaboration.\ednote{ESWC: You might also want to make references to
  Linda Spaces / Blackboard systems, as AI based collaborative environments.}

%%%
% SAVE THIS FOR MKM!
% \subsection{Management of Mathematical Knowledge}
% \label{sec:mkm}

% Projects that manage mathematical knowledge can be classified into
% three groups.  First, there are projects that do not systematically
% employ computer support for their management needs.  An example for
% such a project is the classification of the finite simple
% groups~\cite{Gorenstein-Lyons-Salomon:1994}.  This group could be
% dubbed the ``informal'' group because the produced knowledge, while
% formal to a large degree, does not exist in a machine-understandable
% form.

% Projects in the second, fully formal group use computer systems to manipulate
% the mathematical knowledge.  The most important examples for such systems are
% automated and interactive reasoning tools that permit to construct and search
% mathematical proofs, like the proof assistants mentioned in
% section~\ref{sec:flyspeck}.  For these tools, large libraries, usually with a
% central repository and a repository viewer, have been developed.  For example,
% the standard Isabelle library covers elementary number theory, analysis,
% algebra, and set theory.  It was used for the fully formal proof of the four
% color theorem~\cite{Gonthier:2005:FourColor}. However, as a rough estimate, all
% libraries taken together (ignoring the non-trivial translation issues) cover
% only 10-20\% of the background theory needed for Flyspeck. For example the fact
% that the volume of a sphere of radius $r$ is $\pi r^3$ is not proven in any
% library.  Thus, currently it is not possible to prove anything about a packing
% density.

% Finally, semi-formal projects try to combine the advantages of the
% other two groups by formalizing only parts of or only certain aspects
% of the mathematical knowledge.  For example, OMDoc and SWiM allow one to
% combine formal mathematical content with natural language in a way
% that keeps the structure and interfaces of the knowledge
% machine-readable.  This is particularly suited for projects where an
% informal document is incrementally transformed into a formal one, such
% as in Flyspeck.

% \subsection{Wikis for Mathematical Knowledge}
\label{sec:math-wiki}

Informal mathematical knowledge is currently managed in comprehensive
encyclopediæ like the mathematical sections of Wikipedia\footnote{See
  \url{http://www.wikipedia.org} %, \url{http://cnx.org}
  or \url{http://www.planetmath.org}, respectively, and~\cite{Lange:swmkm-tr07}
  for a more comprehensive evaluation.} 
% or the courseware repository and content
%management system Connexions\footnotemark[\value{footnote}], but also in
or in PlanetMath\footnotemark[\value{footnote}], which focuses on mathematics
and is powered by a highly customized wiki-like system.  The pages in these
systems are categorized and searchable in full-text, with additional metadata
records in PlanetMath.  Neither of the systems is a \emph{semantic} wiki, and
\claim{for lacking typed links they fail to answer queries essential for Flyspeck}, such
as query~\ref{item:proven-lemma} from section~\ref{sec:req}, and they do not
link mathematical symbols to their declarations; instead, the author has to
provide links he considers relevant in the text surrounding the formula.
% Connexions could cope with these two problems, but in practice it does not:
% formulæ are written in Content {\mathml}~\cite{CarlisleEd:MathML07}, and the
% CNXML markup language used for larger structures allows for annotating texts as
% mathematical statements like lemmas~\cite{connexions05:cnxml}, but this
% structural information is not yet \emph{used} by the system.

Recently, there is a growing interest in integrating proof assistants with
wikis%
%\footnote{See \url{http://homepages.inf.ed.ac.uk/da/mathwiki/} for relevant
%  activities.}
.  \emph{Logiweb} is not a wiki but a distributed system for publishing machine
checked mathematics in high-quality PDF that shares part of the key wiki
principles~\cite{Grue:Logiweb07}.  Anybody can contribute to a Logiweb site and
edit new pages in a simple text syntax.  On the other hand, Logiweb does not
offer other essential features.  For example, browsing by traversing links is
supported neither in the editor nor in the generated PDF, and a built-in search
or query facility is not offered.  Logiweb does not allow for \emph{exchanging}
knowledge as required for Flyspeck: Documents can only be exported in
presentational formats like PDF or \TeX{}, but their semantic structures
%can be exported
%as low-level XML or Lisp data structures but not
cannot be exported in mathematical markup or theorem proving languages.  The way
Logiweb checks proofs is not compatible with other theorem provers, as all
calculi and proof tactics need to be defined in the Logiweb system itself.
\emph{ProofWiki} is an integration of the ProofWeb Coq frontend into
MediaWiki~\cite{CorKal:CoopReposFormalProofs07}.  Coq's export tools are used to
generate browsable HTML or {\LaTeX} with linked symbols from the proof scripts.
Generating index pages, such as lists of all definitions or all theorems, is
planned, but not yet in a way that could be customized by users.  So far, there
is just text search, and dependencies among knowledge items are only computed
for exporting proofs but not used for browsing inside the system.  Pages can
either be formal proof scripts (with restricted possibilities to include
informal comments) or informal wiki pages.  Semi-formal documents or stepwise
formalizing of knowledge are not supported.  Importing and exporting Coq proof
scripts to and from the wiki is possible.  While the authors provide
instructions on how to integrate other theorem provers, doing so would be a lot
of work, as there is no abstraction layer or metalanguage for exchanging or
converting data.  Both Logiweb and ProofWiki are ``semantic'' in the sense that
the integrated proof checker utilizes the mathematical knowledge in the wiki
pages.  But the semantics is not utilized for anything else, such as
facilitating browsing or editing, or connecting to semantic web services.
Developing and verifying formal proofs in the wiki is not yet the focus of
Flyspeck in this early stage, but it may be required later.
% if the central
% maintainer approach does not turn out to work.
\end{background}

%%% Local Variables: 
%%% mode: latex
%%% TeX-master: "flyspeck-wiki-eswc08.tex"
%%% End: 
