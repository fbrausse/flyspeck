\section{Related Work}
% \label{sec:related}

%%%
% SAVE THIS FOR MKM!
% \subsection{Management of Mathematical Knowledge}
% \label{sec:mkm}

% Projects that manage mathematical knowledge can be classified into
% three groups.  First, there are projects that do not systematically
% employ computer support for their management needs.  An example for
% such a project is the classification of the finite simple
% groups\cite{Gorenstein-Lyons-Salomon:1994}.  This group could be
% dubbed the ``informal'' group because the produced knowledge, while
% formal to a large degree, does not exist in a machine-understandable
% form.

% Projects in the second, fully formal group use computer systems to manipulate
% the mathematical knowledge.  The most important examples for such systems are
% automated and interactive reasoning tools that permit to construct and search
% mathematical proofs, like the proof assistants mentioned in
% section~\ref{sec:flyspeck}.  For these tools, large libraries, usually with a
% central repository and a repository viewer, have been developed.  For example,
% the standard Isabelle library covers elementary number theory, analysis,
% algebra, and set theory.  It was used for the fully formal proof of the four
% color theorem\cite{Gonthier:2005:FourColor}. However, as a rough estimate, all
% libraries taken together (ignoring the non-trivial translation issues) cover
% only 10-20\% of the background theory needed for Flyspeck. For example the fact
% that the volume of a sphere of radius $r$ is $\pi r^3$ is not proven in any
% library.  Thus, currently it is not possible to prove anything about a packing
% density.

% Finally, semi-formal projects try to combine the advantages of the
% other two groups by formalizing only parts of or only certain aspects
% of the mathematical knowledge.  For example, OMDoc and SWiM allow one to
% combine formal mathematical content with natural language in a way
% that keeps the structure and interfaces of the knowledge
% machine-readable.  This is particularly suited for projects where an
% informal document is incrementally transformed into a formal one, such
% as in Flyspeck.

% \subsection{Wikis for Mathematical Knowledge}
\label{sec:math-wiki}

Wikis have been used to manage informal as well as formal mathematical knowledge, whereas
semi-formal mathematical knowledge it not yet supported widely.
Informal mathematical knowledge is currently managed in comprehensive encyclopediæ like
the mathematical sections of \product{Wikipedia}\footnote{See
  \url{http://www.wikipedia.org}, \url{http://cnx.org} or \url{http://www.planetmath.org},
  respectively, and\cite{Lange:swmkm-tr07} for a more comprehensive evaluation.} or the
courseware repository and content management system
\product{Connexions}\footnotemark[\value{footnote}], but also in
\product{PlanetMath}\footnotemark[\value{footnote}], which focuses on mathematics and is
powered by a highly customized wiki-like system.  The pages in these systems are
categorized and searchable in full-text, with additional metadata records in
\product{PlanetMath}.  Neither of the systems is a \emph{semantic} wiki, and for lacking
typed links they fail to answer queries essential for Flyspeck, such as
query~\ref{item:proven-lemma} from section~\ref{sec:req}, and they do not link
mathematical symbols to their declarations; instead, the author has to provide links he
considers relevant in the text surrounding the formula.  \product{Connexions} could cope
with these two problems, but in practice it does not: formulæ are written in Content
{\mathml}\cite{CarlisleEd:MathML07}, and the CNXML markup language used for larger
structures allows for annotating texts as mathematical statements like
lemmas\cite{connexions05:cnxml}, but this structural information is not yet \emph{used}
by the system.

Recently, there is a growing interest in integrating proof assistants
with wikis\footnote{See
  \url{http://homepages.inf.ed.ac.uk/da/mathwiki/} for relevant
  activities.}.  \product{Logiweb} and \product{ProofWiki} are
wiki-like systems that support checking or interactive development of
proofs. They are ``semantic'' in the sense that the integrated proof
checker utilizes the mathematical knowledge in the wiki pages.  But
the semantics is not utilized for anything else, such as facilitating
browsing or editing, or connecting to semantic web services.
Developing and verifying formal proofs in the wiki is not yet the
focus of Flyspeck in this early stage, but it may be required later if
the central maintainer approach does not turn out to
work.
%% \ednote{@Christoph: Find out whether such systems have already
%%   been used in Flyspeck-like scenarios, i.\,e.\ collaboratively
%%   proving something.}

\paragraph{Logiweb} is a distributed system for publishing machine checked mathematics in
high-quality PDF\cite{Grue:Logiweb07}.  While the author does not call
it a ``wiki'', it shares part of the key wiki principles: anybody can
contribute to a Logiweb site and edit new pages in a simple text
syntax with a browser.  On the other hand, Logiweb does not offer
other features that would be essential for Flyspeck.  For example,
browsing by traversing links is supported neither in the editor nor in
the generated PDF, and a built-in search or query facility is not
offered.  Logiweb does not allow for \emph{exchanging} knowledge as
required for Flyspeck: Documents can be exported in presentational
formats like PDF or \TeX{}, and their internal data structures can be
exported as low-level XML or Lisp data structures but not in languages
for mathematical markup or theorem proving.  Finally, the way Logiweb
checks proofs is not compatible with other theorem provers, as all
calculi and proof tactics need to be defined in the Logiweb system
itself.

\paragraph{ProofWiki} is an integration of ProofWeb, a web frontend to Coq, into
MediaWiki\cite{CorKal:CoopReposFormalProofs07}.  Coq's conversion
tools are used to generate a human-readable and browsable HTML or
{\LaTeX} presentation with linked symbols from the proof scripts.
Generating index pages, such as lists of all definitions or all
theorems, is planned, but not yet in a way that could be influenced or
customized through the wiki interface.  So far, there is just text
search, and dependencies among knowledge items are only computed for
exporting proof scripts but not used for browsing inside the system.
Another disadvantage of ProofWiki in the context of Flyspeck is that
pages can either be formal proof scripts (with restricted
possibilities to include informal comments), or informal wiki pages.
Semi-formal documents or stepwise formalizing of knowledge is not
supported.  Importing and exporting Coq proof scripts to and from the
wiki is possible, but other formats are not yet supported.  While the
authors do provide instructions on how to integrate other theorem
provers, doing so would be a lot of work, as there is no abstraction
layer or metalanguage for exchanging or converting data.

%%% Local Variables: 
%%% mode: latex
%%% TeX-master: "flyspeck-wiki-eswc08.tex"
%%% End: 
