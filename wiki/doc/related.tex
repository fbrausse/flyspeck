\section{Related Work (all)}
\label{sec:related}


\subsection{Management of Mathematical Knowledge}
\label{sec:mkm}

Projects that manage mathematical knowledge can be classified into three groups. The first group comprises projects that do not systematically employ computer support for their management needs. An example for such a projects is the classification of the finite simple groups~\cite{Gorenstein-Lyons-Salomon:1994}. This group could be dubbed the informal group because the produced knowledge, while formal to a large degree, does not exist in a machine-readable form.

The second group is the fully formal group. Projects in this group use computer systems to manipulate the mathematical knowledge. The most important examples for such systems are automated and interactive reasoning tools that permit to construct and search for mathematical proofs. Several such systems are in use, such as Mizar~\cite{mizarmanual}, Coq~\cite{Coq}, or Isabelle~\cite{Isabelle:definition}. For these tools, large libraries, usually with a central repository and a repository viewer, have been developed. For example, the standard Isabelle library covers elementary number theory, analysis, algebra, and set theory. An example or such a project is the fully formal proof of the four color theorem~\cite{Gonthier:FourColor}.

The third group is the semi-formal group. Semi-formal projects try to combine the advantages of the other two groups by formalizing only parts of or only certain aspects of the mathematical. For example, OMDoc and SWiM permit to combine formal mathematical content with natural language in a way that keeps the structure and interfaces of the knowledge machine-readable. This is particularly suited for projects where an informal document is incrementally transformed into a formal one, such as in Flyspeck.

\subsection{Wikis for Mathematics (Christoph)}
\label{sec:math-wiki}

\subsubsection{Semantic Wikis}
\label{sec:semwiki}

Semantic wikis~\cite{semwiki06} are wikis enhanced by semantic annotations.  Although many
ways of semantically enhancing wikis have been investigated, a modeling approach prevails
where one resource (in the RDF sense) --- e.\,g.\ one mathematical theorem --- is
represented by one wiki page and relations between resources by links between pages.  Both
pages and links can be typed with terms from
ontologies~\cite{OrDeMoVoHa06:annotation-navigation-semwiki}, which are either preloaded
into the wiki or modelled ad-hoc~\cite{KrSchVr:semwiki-reasoning07}.  Semantic wikis
commonly offer enhanced navigation capabilities by displaying a summary of all typed
links, grouped by type, with each page.  Most of them allow to search for pages by type or
by them being subject or object of any RDF triple (= typed link), while it depends on the
reasoner used by the wiki whether only explicit RDF triples or also inferred ones are
considered~\cite{KrSchVr:semwiki-reasoning07}.  Such queries can usually be executed
interactively via a special search form, or in an automated way as \emph{inline} queries
embedded into the content of a page.

\subsubsection{Informal Knowledge Collections}
\label{sec:math-knowledge-collections}

Current collaborative projects for managing \emph{informal} mathematical knowledge range
from comprehensive encyclopediæ like the mathematical sections of \product{Wikipedia}\ednote{reference} or
the courseware repository and content management system \product{Connexions}\ednote{reference} to projects
specially focused on mathematics like \product{PlanetMath}\ednote{reference}\footnote{See
  \url{http://www.wikipedia.org}, \url{http://cnx.org} or \url{http://www.planetmath.org},
  respectively.}, which is powered by a highly customized wiki-like system.  The pages in
these systems are categorized and searchable in full-text, with additional metadata
records in the case of \product{PlanetMath}.  Neither of these systems is a
\emph{semantic} wiki, and thus they fail to solve the following two problems, which are
essential for MKM:

\begin{enumerate}
\item\label{item:formula-search-usecase} In \product{Wikipedia} and \product{PlanetMath},
  formulæ are given in presentation-oriented {\LaTeX}.  Imagine a wiki page about the
  Pythagorean Theorem, stated as $a^2 + b^2 = c^2$, and a user searching for the
  equivalent formula $x^2 + y^2 = z^2$ (or even $c=\sqrt{a^2+b^2}$!) --- The system would
  not find the theorem.
\item Neither could ``all theorems about triangles for which a
  proof exists'' be searched for, as the link from a proof to the theorem it proves is not
  typed.
\end{enumerate}

\product{Connexions}, on the other hand, could in principle cope with these two problems,
but in practice it does not: Formulæ are written in the content-oriented sublanguage of
{\mathml}~\cite{CarlisleEd:MathML07}, and the CNXML markup language used for larger
structures allows for annotating texts as mathematical statements like lemmas, but this
structural information is not yet \emph{used} by the system.  Moreover, none of the
systems mentioned so far supports an easy navigation from the occurrence of a mathematical
symbol in a formula to the declaration or definition of this symbol, if it is defined in
some other place of the wiki; instead, the author has to provide links he considers
relevant in the text surrounding the formula.

\subsubsection{Domain-Specific Semantics}
\label{sec:domain-semantics}

Note that general-purpose semantic wikis do not support the above-mentioned use case
(\ref{item:formula-search-usecase}) either, as they neither have a sufficient notion of
equality nor understand mathematical content markup.  If we assume ``semantic'' not just
to mean RDF or description logics, but any kind of (higher-order) logic required for
specific domains\ednote{@Florian: This is quite superficial, can we write it in a more
  sophisticated way?} and employ domain-specific ways of knowledge representation we can
imagine semantic wikis specifically supporting mathematics.  For use case
(\ref{item:formula-search-usecase}), we could have the wiki pages crawled by a formula
search engine like MathWebSearch~\cite{KohSuc:asemf06}, which applies substitution tree
indexing to mathematical formulae.  Even more formal approaches integrate automated
theorem provers into wikis.  Two of these systems are discussed in
section~\ref{sec:wiki-pa}.


\subsection{Wikis with Integrated Proof Assistants}
\label{sec:wiki-pa}

Recently, there is a growing interest in integrating automated theorem provers or proof
assistants with wikis\footnote{See \url{http://homepages.inf.ed.ac.uk/da/mathwiki/} for
  relevant activities.}.  Both Logiweb and ProofWiki are wiki-like systems that support
checking or interactive development of proofs.  Both are ``semantic'' in the sense that
the integrated proof checker can utilize the mathematical knowledge in the wiki pages.
But the semantics is not utilized for purposes other than that, such as facilitating
browsing or editing, or connecting to services on the semantic web.  Developing and
verifying formal proofs in the wiki is not yet the focus of Flyspeck in this early stage,
but it may be required later if the central maintainer approach does not turn out to work.

\paragraph{Logiweb} is a distributed system for publishing machine checked mathematics in
high-quality PDF~\cite{Grue:Logiweb07}.  While the author does not call it a ``wiki'', it
shares part of the key wiki principles: Anybody can contribute to a Logiweb site and edit
new pages in a simple text syntax with a browser.  On the other hand, Logiweb does not
offer other features that would be essential for Flyspeck: browsing by traversing links is
supported neither in the editor nor in the generated PDF, and Logiweb does not offer a
built-in search or query facility.  Logiweb does not allow for \emph{exchanging} knowledge
as required for Flyspeck: Documents can be exported in presentational formats like PDF or
\TeX{}, and their internal, low-level data structures can be exported as XML or Lisp
S-expressions, but currently there is no easy way to convert these representations to
other languages for mathematical markup or theorem proving.  Finally, the way Logiweb
checks proofs is not compatible with other theorem provers, as all calculi and proof
tactics need to be defined in the Logiweb system itself.

\paragraph{ProofWiki} is an integration of ProofWeb, a web frontend to the Coq proof
assistant, into MediaWiki~\cite{CorKal:CoopReposFormalProofs07}.  Coq's converting tools
are used to generate human-readable and browsable HTML or {\LaTeX} presentation from the proof
scripts.  In the HTML generated that way, symbols are linked to their declaration.  Index
pages, such as lists of all definitions or all theorems, are generated, but their
generation cannot be influenced or customized through the wiki
interface\ednote{@Christoph: check!}.  So far, there is just text search, and dependencies
among knowledge items are only computed for exporting proof scripts but not used for
browsing inside the system.  Another disadvantage of ProofWiki in the context of Flyspeck
is that pages can either be formal proof scripts (with restricted possibilities to include
informal comments), or informal wiki pages.  Semi-formal documents or stepwise formalizing
of knowledge is not supported.  Importing and exporting Coq proof scripts to and from the
wiki is possible, but other formats are not yet supported.  While the authors do provide
instructions on how to integrate other theorem provers, doing so would be a lot of work,
as there is no abstraction layer or metalanguage for exchanging or converting data.

\ednote{@Christoph: Find out whether such systems have already been used in Flyspeck-like
  scenarios, i.\,e.\ collaboratively proving something.}

%%% Local Variables: 
%%% mode: latex
%%% fill-column: 90
%%% TeX-master: "flyspeck-wiki-eswc08"
%%% End: 
