
\section{Supporting Flyspeck in a Semantic Wiki}

Starting from a minimal set of requirements, we have evaluated the
applicability of two concrete semantic wikis to Flyspeck.  We used two
different frameworks: a prototype developed on top of Semantic
MediaWiki, and our own semantic wiki SWiM.  Based on the results of
this pre-study, we establish a roadmap for tailoring SWiM to
specifically meet the needs of the Flyspeck collaborators.

\subsection{Motivation}
\label{sec:req}

The Flyspeck project has garnered significant enthusiasm in the theorem proving
community: Hales has given talks about the project at a number of prestigious
conferences, and aspects of Flyspeck are currently a motivation behind at least
two PhD dissertations.  Currently, though, Flyspeck is not ready for
crowdsourcing.  Because definitions rely so heavily on each other, and the lemma
statements rely on the definitions, Hales needs to oversee the computerization
of the definitions so that the mathematical constants are correct\footnote{For
  example, you can represent a vector as a function from the integers to the
  reals, or as a tuple of reals.  The operations of vector spaces will depend on
  this representation, etc.}.  Once the formal definitions are complete, the
rest of the project can proceed practically in parallel, being worked off by
many independent collaborators\ednote{@Sean, I wanted to make this more
  explicit. --CL}.

To build a prototype, we computerized the definitions and lemma statements of
the first chapter of the book (Trigonometry) in the
Twelf\cite{Schurmann:1999:Twelf} proof assistant.  Once that was complete, we
listed a minimal set of behaviors the wiki should support.

\subsection{Requirements}

Our focus in this work is on making the extent and structure of
Flyspeck comprehensible and on communicating where work needs to be
done.  For this the outline of the whole proof from the
book\cite{Hales:2007:FlyspeckBook} needs to be represented in the
wiki, where the mathematical statements (including definitions,
lemmas, and theorems) are available in a human-readable way (with
formulae in \LaTeX\ or presentational MathML) as well as a
computerized presentation suitable for downloading into a theorem
prover.  In order to obtain a well-structured network of knowledge
items, each mathematical statement should be presented on one wiki
page, which shows its human-readable representation from the book,
offers additional space for annotation, and allows for downloading a
formal representation.  

An example usage scenario is as follows; A user wishes to contribute to
Flyspeck.  She looks at our wiki main page, which shows her what still needs to
be done.  Preferring analysis to geometry, she searches for open problems
involving analysis.  This returns a list of lemmas related to analysis from
which she can choose one that seems possible given her time constraints. She
downloads the relevant formal definitions and lemmas, along with the text of a
paper proof culled from Hales' book.  She uses a proof assistant to begin
formalizing the paper proof (or some variant thereof).  At some point, she needs
clarification on some definition and additionally has an idea on how to
generalize this lemma.  She thus asks for help and makes comments in a forum
space of the wiki.  She completes her proof, and uploads the proof assistant
file to the wiki.  The wiki checks the proof for correctness and then adds it to
the database.

With this scenario in mind, we propose that the wiki should minimally offer: 

\begin{enumerate} 
\item A \textbf{knowledge base} of the theory, constant, and lemma definitions 
\item A way to \textbf{browse} the knowledge by category, or search with keywords 
\item A way to \textbf{download} the definitions, lemmas statements and existing
  proofs
\item A way to \textbf{upload} new proofs 
\item A \textbf{forum} to discuss issues involved in the formalizations 
\end{enumerate} 

To support this minimal infrastructure, the wiki should offer 

\begin{description}
\item[Categorization by topic:] In the beginning, one would mirror the narrative structure
  of the book (e.\,g.\ ``ball'' being a subsection of ``primitive volumes'', which in turn
  is a section of the chapter ``volume calculations'').  Standardized ways of classifying
  mathematical topics, such as the Mathematical Subject Classification
  (MSC)\cite{AMS:MSC2000}, could be added later.
\item[Project-organization metadata] such as whether the proof
  of a lemma has already been computerized, or if someone is currently 
  attempting a proof.  This is essential so that two people don't duplicate
  work.
\item[Dependency links:] These can be links from individual symbols in
  mathematical formulae to the place where they are declared, or from any page
  $p$ to other pages containing knowledge that is required for understanding $p$---either pages in the same wiki, or external resources like
  \product{PlanetMath} or \product{Wikipedia} articles.
\item[Discussion posts] should be strongly tied to the topic being discussed,
  and classified into categories like question, answer, explanation, etc.
\end{description}

To the visitor and potential collaborator, an impression of the extent and
structure of the project---its size and its specialization into diverse fields
of mathematics---must be given, as well as tools for browsing and querying the
knowledge.  Not only should it be possible to query knowledge items by their
annotations, but important query results must also be available as dynamically
generated lists.  Examples for queries are:

\begin{enumerate}
\item\label{item:proven-lemma} ``Which lemmas about composite regions have not
  been computerized so far?''
\item ``Are there textual resources I can read in order to understand Jordan's
  curve theorem?''
\item ``What lemmas are difficult to prove?''
  \begin{enumerate}
  \item \ldots in the sense that many people have already attempted them, but given up
  \item\label{item:question-count} \ldots in the sense that many people have asked
    questions in the related discussion
  \end{enumerate}
\item \ldots\ednote{What else makes sense?\\
    If we have to prove that A has property B, would a (similar) proof of C
    having
    property B help us?\\
    For proving sth. about a concept C, would it help to use lemmas proven about
    specializations or generalizations of C?}
\end{enumerate}

A volunteer who is willing to work out and contribute a computerized proof for
some lemma should be able to download a self-contained computerized
representation of this lemma and everything it depends on, with different
notions of ``dependency'': The strongest one is that a lemma depends on the
declarations and definitions of all symbols it uses and on the transitive
closure of all symbols used by the latter.  Additionally, other related lemmas
could be downloaded and assumed as axioms, disburdening the user from building
the new proof from scratch.  Assuming that the Flyspeck book is written in a
linear order, all definitions and lemmas \emph{before} the current one in the
narrative order could be useful.

% \newcommand{\wikipage}[5]{\node[draw,text width=5cm,font=\tiny\sffamily] (#1) at #2 {
%     {\footnotesize\bfseries #3}\\
%     #4
%     ~\\[1em]
%     [Download Isabelle representation]\\
%     #5
%   };}
% \begin{figure}
%   \centering
%   \begin{tikzpicture}[set style={{default}+=[scale=1.5,font=\sffamily]},default,xscale=.8]
%     \wikipage{lemma}{(0,0)}{Lemma 1.3}{The cosine is an even function.  The sine is an odd function.\\
%       $\cos(-x)=\cos(x),\qquad\sin(-x)=-\sin(x)$}{Page type: Lemma\\
%       Topic: Trigonometry\\
%       Proven: no (3 unsuccessful attempts)}
%     \wikipage{cos}{(6,0)}{Cosine}{$cos\colon\mathbb{R}\to\mathbb{R},x\mapsto\ldots$}{
%       Page type: Definition\\
%       Topic: Trigonometry
%     }
%     \wikipage{todo}{(0,3)}{To do}{Unproven lemmas:
%       \begin{itemize}
%       \item Lemma 1.3
%       \item \ldots
%       \end{itemize}
%     }{
%       Page type: Overview
%     }
%     \draw[->] (lemma) -- node[above] {usesSymbol} (cos);
%     \draw[->] (todo) -- node[left] {references} (lemma);
%   \end{tikzpicture}
%   \caption{Page structure}
%   \label{fig:pagestructure}
% \end{figure}
During the formalization of the knowledge, we anticipate that the
definitions will undergo refactoring in order to facilitate the
actual development of the proofs.  (Historically, this has been the 
case with many large computerized proofs, cf. \cite{Gonthier:2005:FourColor}.) 
Refactoring support by the wiki would thus be advantageous.

%%% Local Variables: 
%%% mode: latex
%%% TeX-master: "flyspeck-wiki-eswc08"
%%% End: 
