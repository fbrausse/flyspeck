\chapter{Linear Assembly Problems}

\section{Linear Assembly Problems} \label{linear}

In this section we define a class of nonlinear optimization
problems that we call {\it linear assembly problems}.

Assume given a topological space $X$, and a finite collection of
topological spaces, called {\it local domains}.  For each local
domain $D$ there is a map $\pi_D:X\to D$.  There are functions
$u_i$, $i=1,\ldots,N$, each defined on some local domain $D_i =
\op{dom}(u_i)$, and we let $x_i$ denote the composite $x_i =
\pi_{D_i}\circ u_i$.

On each local domain $D$, the functions $u_i$ are related by a
finite set of nonlinear relations
\begin{equation}\phi(u_i : \op{dom}(u_i) = D) \ge0, \quad \phi \in \Phi_D.
    \label{phi}
\end{equation}

We use vector notation $x = (x_1,\ldots,x_N)$, with constant
vectors $c$, $b$, and matrix $A$ given.

The problem is to maximize $c\cdot x$ subject to the constraints
    \begin{equation}\label{Ax}A\, x \le b,
    \end{equation}
and to the nonlinear relations~\ref{phi}.  A problem of this form
is called a linear assembly problem.  (Intuitively, there are a
number of nonlinear objects $D$, that form the pieces of a jigsaw
puzzle that fit together according to the linear
conditions~\ref{Ax}.)

\begin{example}
Assume a single local domain $D$, and let $\pi_D:X=D$ be the
identity map.  The function $f = c\cdot x $ is nonlinear. The
problem is to maximize $f$ over $D$ subject to the nonlinear
relations $\Phi_D$.  This is a general constrained nonlinear
optimization problem.
\end{example}

\begin{example}  Assume that each $u_i$ has a distinct local domain $D_i = \ring{R}$.
Let $X = \ring{R}^N$, let $\pi_D$ be the projection onto the $i$th
coordinate, and let $x_i$ be the $i$th coordinate function on
$\ring{R}^n$. Assume that $\Phi_D$ is empty for each $D$.  The
problem becomes the general linear programming problem
    $$\max c\cdot x$$
such that $A x\le b$.
\end{example}

These two examples give the nonlinear and linear extremes in
linear assembly problems. The more interesting cases are the mixed
cases which combine nonlinear and linear programming.
Example~\ref{pr:third} gives one such case.

\begin{figure}[htb]
  \centering
  \myincludegraphics{\ps/vor.eps} %% CORRECT GRAPHIC?
  \caption{A truncated Voronoi cell and a subset of the cell lying in a sector}
  \label{voronoi}
\end{figure}

\begin{example} (2D Voronoi cell minimization). Take a packing of disks of
radius $1$ in the plane.  Let $\Lambda$ be the set of centers of
the disks.  Assume that the origin $0\in\Lambda$ is one of the
centers. The truncated Voronoi cell at $0$ is the set of all
$x\in\ring{R}^2$ such that $|x|\le t$, and $x$ is closer to the
origin than to any other center in $\Lambda$.  We assume
$t\in(1,\sqrt2)$.

Only the centers of distance at most $2t$ affect the shape and
area of the truncated Voronoi cell.  For each $n=0,1,2,\ldots$, we
have a topological space of all truncated Voronoi cells with $n$
nonzero disk centers $v_i$ at distance at most $2t$.  Fix $n$, and
let $X$ be the topological space.

Let $D=D_i$, $i=1,\ldots,n$,  be the sectors lying between
consecutive segments $(0,v_i)$.  Each sector is characterized by
its angle $\alpha$ and the lengths $y_a$ and $y_b$ of the two
segments $(0,v_i)$, $(0,v_j)$ between which the sector lies.  The
part $A$ in $D$ of the area of the truncated Voronoi cell is a
function of the variables $\alpha$, $y_a$, $y_b$.  A nonlinear
implicit equation $\phi=0$ relates $A$, $\alpha$, $y_a$, and $y_b$
on $D$. The variables $u_i$ of the linear assembly problem for the
local domain $D$ are $A$, $y_a$, $y_b$, $\alpha$.


We have a linear assembly problem.  The function $c\cdot x$ is the
area of the truncated Voronoi cell, viewed as a sum of variables
$A$, for each sector $D$ (or rather, their pullbacks to $X$ under
the natural projections $X\to D$).

The assembly constraints are all linear. One linear relation
imposes that the angles of the $n$ different sectors must sum to
$2\pi$. Other linear relations impose that the variable $y_a$ on
$D$ equals the variable $y_b$ on $D'$ if the two variables
represent the length of the same segment $(0,v_i)$ in $X$.
\end{example}


\subsection{solving linear assembly problems}

In this section we describe how various linear assembly problems
are solved in the proof of the Kepler conjecture in terms
sufficiently general to apply to other linear assembly problems as
well.

Let us introduce some general notation.  Let $x_D = (x_i:
\op{dom}(u_i)=D)$ be the vector of variables with local domain
$D$. Write $c\cdot x$ in the form $\sum_D c_D\cdot x_D$ and the
assembly conditions as
$$A \,x =\sum_D A_D x_D,$$
according to the local domain of the variable.


\subsubsection{Linear relaxation}
The first general technique is {\it linear relaxation}. We replace
the nonlinear relations $\phi(x_D)\ge0, \phi\in\Phi_D$ with a
collection of linear inequalities that are true whenever the
constraints $\Phi_D$ are satisfied: $A'_D x_D \le b_D$.  A linear
program is obtained by replacing the nonlinear constraints
$\Phi_D$ with the linear constraints. Its solution dominates the
nonlinear optimization problem.  In this way, the nonlinear
maximization problem can be bounded from above.

Let us review some constructions that insure rigor in linear
programming solutions. We assume general familiarity with the
basic theory and terminology of linear programming. It is
well-known that the primal has a feasible solution iff the dual is
bounded.  We will formulate our linear programs in such a way that
both the primal and the dual problems are feasible and bounded.

We use vector notation to formulate a primal problem as
    \begin{equation}
        \max\, c\cdot x
        \label{cx}
    \end{equation}
such that $A x \le b$, where $x$ is a column vector of free
variables (no positivity constraints), $A$ is a matrix, $c$ is a
row vector, and $b$ is a column vector.

We can insure that this primal problem is bounded by bounding each
of the variables $x_i$.  (This is easily achieved considering the
geometric origins of our problem, which provides interpretations
of variables as particular dihedral angles, edge lengths, and
volumes.) We assume that these bounds form part of the constraints
$A x\le b$.

The linear programs we consider have the property that if the
maximum is less than a constant $K$, the solution does not
interest us.  (For instance, in the dodecahedral conjecture,
Voronoi cell volumes are of interest only if the volume is less
than the volume of the regular dodecahedron.) This observation
allows us to replace the primal problem with one having an
additional variable $t$:
    %%


\subsection{nonlinear duality}
The second general technique is nonlinear duality.  Suppose that
we wish to show that the maximum of the primal problem~\ref{cx} is
at most $M$.

Let $x^* = (x^*_D)$ be a guess of the solution to the problem,
obtained for example, by numerical nonlinear optimization. We
relax the nonlinear optimization by dropping from the matrix $A$
and the vector $b$ those inequalities that are not binding at
$x^*$. With this modification, we may assume that $A\,x^*=b$.  Let
$m$ be the size of the vector $b$, that is, the number of binding
linear conditions. Let $d$ be the number of local domains $D$.

We introduce a linear dual problem with real variables $t$,
$r_\phi: \phi\in\Phi_D$, and $w\in\ring{R}^m$. The variables
$r_\phi$ and $w$ are constrained to be non-negative.

We consider the linear problem of maximizing $t$ such that
    \begin{equation}
        M + d\, t - c\cdot x^* \ge 0
        \label{Mx}
    \end{equation}
and such that for each $x_D$ in each $D$ the linear inequality
    \begin{equation}
        c_D\cdot (x_D-x^*_D) + \sum_{\Phi_D} r_\phi \phi(x) +
                    w A_D (x^*_D-x_D) + t
            < 0
        \label{xD}
    \end{equation}
is satisfied.

There is no guarantee that a feasible solution exists to this
system of inequalities.  However, any feasible solution gives an
upper bound $M$. Indeed, let $x=(x_D)$ be any feasible argument to
the primal, and let $t,r_\phi,w$ be a feasible solution to the
dual. Taking the sum of the linear inequalities~\ref{xD}, over $D$
at $x$, we have (recall $\phi\ge0$ and $A x\le b$):
$$
\begin{array}{lll}
M &\ge M + c\cdot (x-x^*) + \sum_D\sum_{\Phi_D} r_\phi \phi(x)
    + w A (x^*-x) + d\, t,\\
    &\ge c\cdot x + (M + d\, t - c\cdot x^*) + w (b-A x),\\
    &\ge c\cdot x.
\end{array}
$$

Since the dual problem has infinitely many constraints (because of
constraints for each $x\in D$), we solve the dual problem in two
stages. First, we approximate each $D$ by a finite set of test
points, and solve the finitely constrained linear programming
problem for $t, r_\phi$, and $w$.

We replace $t$ with $t_0 = (-M +c\cdot x^*)/d$ (to make the
constraint \ref{Mx} bind).  It follows from the feasibility of $t$
that $t\ge t_0$, and that $t_0,r_\phi,w$ is also feasible on the
finitely constrained problem. To show that $t_0,r_\phi,w$
satisfies all the inequalities~\ref{xD} (under the substitution
$t\mapsto t_0$), we use interval arithmetic to show that each of
these inequalities hold. (To make these interval arithmetic
verifications as easy as possible, we have chosen the solution
$t_0,r,w$ to make the closest inequality hold by as large a margin
$t-t_0$ as possible. This is the meaning of the maximization over
$t$ in the dual problem.)  The next section will give further
details about interval arithmetic verifications.

\subsection{branch and bound}
The third technique is branch and bound.  When no feasible
solution is found in step (2), it may still be possible to
partition $X$ into finitely many sets $X = \coprod X_i$, on which
feasible solutions to the dual may be found.  Although this is an
essential part of the solution, the rules for branching in the
Kepler conjecture follow the structure of that problem, and we do
not give a general branching algorithm.


\section{Definitions and Interpretations}

\begin{definition}
By $\optt{quarter}(\alpha)$ we mean that at dart $\alpha$, we have
an upright quarter.
\end{definition}

\begin{definition}
By $\optt{slice}(\alpha)$ we mean that at dart $\alpha$ we have an upright
diagonal and a slice: $\optt{azim}\slt \pi$ and
the opposite edge length is at most $3.2$ and at least $2$.
\end{definition}

\begin{definition}
By $\optt{gap}(\alpha)$ we mean that at dart $\alpha$, we have an
upright diagonal $\optt{azim}\slt \pi$ and the opposite edge length
is greater than $3.2$.
\end{definition}

\begin{definition}
By $\optt{upright}(\alpha)$ we mean that at $\alpha$ there is an upright
diagonal.
\end{definition}


If $\optt{f}$ is any of the
functions
    $$\optt{vor0},\optt{gamma}, \optt{nu},$$
we set $\optt{tau0}$, $\optt{tau\_gamma}$,
$\optt{tau\_nu}$, respectively,
to
    $$\optt{tau\_*} = -f(\alpha) +\optt{sol}(\alpha)\zeta\pt.$$
We set
    $$
    \optt{tau}(\alpha,t) = -
    \optt{sovo}(S,t,\lambda_{sq})+\optt{sol}(\alpha)\zeta\pt.
    $$
We say that $\alpha$ is compressed or decompressed 
according to the scoring of $\optt{mu}(\alpha)$.  (See
Section~\ref{sec:rules}.)

We  measure what is squandered by a flat quarter by $\hat\tau =
\sol\zeta\pt - \hat\sigma$.

XX Define width $y_4$ of a (geometric) dart,
types fitted crown, type C, enclosed masking dart, masking dart, etc. 


\section{Basic Relations}

\begin{lemma} If $\optt{upright}(\alpha)$ then $\optt{gap}(\alpha)$ or
$\optt{slice}(\alpha)$ or $\optt{azim}(\alpha) \sge \pi$.
\end{lemma}


\begin{lemma}  If $N$ is a node that is upright, then at least
one dart at $N$ is a quarter.
\end{lemma}

\section{Listing of Assembly Problems}

\subsection{Triangles}


\begin{lemma} \label{lemma:1pt}
%\proclaim{Lemma 3.13}
A quasi-regular tetrahedron $S$ satisfies $\sigma(S)\le 1\,\pt$.
Equality occurs if and only if the quasi-regular tetrahedron is
regular of edge length $2$.\index{quasi-regular!tetrahedron}
%
\end{lemma}

\begin{proof}
This is \calc{586468779}.
\end{proof}

\subsection{Triangle-Quad Types}

%% DCG 10.2, p100

%If more than $\squander$ are squandered at a vertex of a given type,
%then that type of vertex cannot be part of a centered packing scoring
%more than $8\,\pt$.  These relations between scores and vertex types
%will allow us to reduce the feasible planar maps to an explicit finite
%list. For each of the planar maps on this list, we calculate a second,
%more refined linear programming bound on the score. Often, the refined
%linear programming bound is less than $8\,\pt$.

This section derives the bounds on the scores of the clusters
around a given vertex as a function of the type of the vertex.
Define constants $\tlp(p,q)/\pt$ by Table~\ref{eqn:old5.1}.  The
entries marked with an asterisk will not be needed.

\bigskip
% Table eqn:old5.1 of constants.


% page 246 of TeXBook
%\def\pt{\hbox{\it pt}}

\begin{equation}
\vbox{\offinterlineskip \hrule
\halign{&\vrule#&\strut\ \hfil#\hfil\ \cr   % "\ " was quad
height 7pt&\omit&&\omit&&\omit&&\omit&&\omit&&\omit&&\omit&\cr
&\hfil $\tlp(p,q)/\pt$\hfil
        &&\hfil $q=0$\hfil
        &&\hfil1\hfil
        &&\hfil2\hfil
        &&\hfil3\hfil
        &&\hfil4\hfil
        &&\hfil5\hfil&
\cr height 7pt&\omit&&\omit&&\omit&&\omit&&\omit&&\omit&&\omit&\cr
\noalign{\hrule}
height7pt&\omit&&\omit&&\omit&&\omit&&\omit&&\omit&&\omit&\cr
&$p=0$&& *&& *&& 15.18&& 7.135&& 10.6497&& 22.27&\cr &1&&    *&&
*&&  6.95&& 7.135&&17.62  && 32.3&\cr &2&&    *&&
8.5&&4.756&&12.9814&&*&&*&\cr &3&& *&&
3.6426&&8.334&&20.9&&*&&*&\cr
&4&&4.1396&&3.7812&&16.11&&*&&*&&*&\cr
&5&&0.55&&11.22&&*&&*&&*&&*&\cr &6&&6.339&&*&&*&&*&&*&&*&\cr
&7&&14.76&&*&&*&&*&&*&&*&\cr
height7pt&\omit&&\omit&&\omit&&\omit&&\omit&&\omit&&\omit&\cr}
\hrule }
    %oldtag 5.1
    \label{eqn:old5.1}
\end{equation}
% based on sp in more.m



\begin{lemma}
    \label{lemma:pq}
    %{Proposition 5.2}
Let $S_1,\ldots,S_p$ and $R_1,\ldots,R_q$ be the tetrahedra and quad
clusters around a vertex of type $(p,q)$. Consider the constants of
Table~\ref{eqn:old5.1}.               Now,
    $$
    \begin{array}{lll}
    &\sum^p\tau(S_i) + \sum^q\tau_{R_i}(v,\Lambda) \ge \tlp(p,q),\\
    \end{array}
    $$
\end{lemma}

\begin{proof} Set
    $$
    (d_i^0,t_i^0)=(\dih(S_i),\tau(S_i)),\qquad
    (d_i^1,t_i^1)=(\dih(R_i),\tau(R_i)).
    $$
The linear combination $\sum^p\tau(S_i)+\sum^q\tau_{R_i}(v,\Lambda)$ is at
least the minimum of $\sum^p t_i^0+\sum^q t_i^1$ subject to
$\sum^p d_i^0+\sum^q d_i^1 = 2\pi$ and to the system of linear
inequalities \calc{830854305} and the system of linear
inequalities \calc{940884472} (obtained by replacing $\tau$ and
dihedral angles by $t_i^j$ and $d_i^j$). The constant $\tlp(p,q)$
was chosen to be slightly smaller than the actual minimum of this
linear programming problem.

The entry $\tlp(5,0)$ is based on Lemma~\ref{lemma:0.55}, $k=1$.
\end{proof}

\subsection{Five triangles around a node}

\begin{lemma}
    \label{lemma:0.55}
    %proclaim{Lemma 5.3}
Let $v_1,\ldots, v_k$, for some $k\le 4$, be distinct vertices of
a centered packing of type $(5,0)$.  Let $S_1,\ldots, S_r$ be
quasi-regular tetrahedra around the edges $\{v_0,v_i\}$, for $i\le
k$. Then
    $$\sum_{i=1}^r \tau(S_i)> 0.55k\,\pt,$$
and
    $$\sum_{i=1}^r \sigma(S_i) < r\,\pt - 0.48k\,\pt.$$
\end{lemma}


\begin{proof}
We have $\tau(S)\ge 0$, for any quasi-regular tetrahedron $S$.  We
refer to the edges $y_4,y_5,y_6$ of a simplex $S(y_1,\ldots,y_6)$
as its top edges. Set $\xi=2.1773$.

The proof of the first inequalities relies on seven
calculations\footnote{\calc{636208429}}. Throughout the proof, we
will refer to these inequalities simply as Inequality~$i$, for
$i=1,\ldots,7$.

We claim (Claim~1) that if $S_1,\ldots,S_5$ are quasi-regular
tetrahedra around an edge $\{v_0,v\}$ and if $S_1=S(y_1,\ldots,y_6)$,
where $y_5\ge\xi$ is the length of a top edge $e$ on $S_1$ shared
with $S_2$, then $\sum_1^5\tau(S_i) > 3(0.55)\,\pt$.  This claim
follows from Inequalities~1 and~2 if some other top edge in this
group of quasi-regular tetrahedra has length greater than $\xi$.
Assuming all the top edges other than $e$ have length at most
$\xi$, the estimate follows from $\sum_1^5\dih(S_i)=2\pi$ and
Inequalities~3, ~4.

Now let $S_1,\ldots,S_8$ be the eight quasi-regular tetrahedra
around two edges $\{v_0,v_1\}$, $\{v_0,v_2\}$ of type $(5,0)$. Let $S_1$
and $S_2$ be the simplices along the face $\{v_0,v_1,v_2\}$. Suppose
that the top edge $\{v_1,v_2\}$ has length at least $\xi$. We claim
(Claim 2) that $\sum_1^8\tau(S_i)> 4(0.55)\,\pt$.  If there is a
top edge of length at least $\xi$ that does not lie on $S_1$ or
$S_2$, then this claim reduces to Inequality~1 and Claim 1. If any
of the top edges of $S_1$ or $S_2$ other than $\{v_1,v_2\}$ has
length at least $\xi$, then the claim follows from Inequalities~1
and ~2. We assume all top edges other than $\{v_1,v_2\}$ have length
at most $\xi$. The claim now follows from Inequalities~3 and ~5,
since the dihedral angles around each vertex sum to $2\pi$.

We prove the bounds for $\tau$.  The proof for $\sigma$ is
entirely similar, but uses the constant $\xi=2.177303$ and seven
new calculations\footnote{\calc{129662166}} rather than the seven
given above. Claims analogous to Claims~1 and 2 hold for the
$\sigma$ bound by this new group of seven inequalities.


Consider $\tau$ for $k=1$.  If a top edge has length at least
$\xi$, this is Inequality~1.  If all top edges have length less
than $\xi$, this is Inequality~3, since dihedral angles sum to
$2\pi$.

We say that a top edge lies around a vertex $v$ if it is an edge
of a quasi-regular tetrahedron with vertex $v$. We do not require
$v$ to be the endpoint of the edge.

Take $k=2$. If there is an edge of length at least $\xi$ that lies
around only one of $v_1$ and $v_2$, then Inequality~1 reduces us
to the case $k=1$.  Any other edge of length at least $\xi$ is
covered by Claim 1.  So we may assume that all top edges have
length less than $\xi$.  And then the result follows easily from
Inequalities~3 and ~6.

Take $k=3$. If there is an edge of length at least $\xi$ lying
around only one of the $v_i$, then Inequality~1 reduces us to the
case $k=2$. If an edge of length at least $\xi$ lies around
exactly two of the $v_i$, then it is an edge of two of the
quasi-regular tetrahedra. These quasi-regular tetrahedra give
$2(0.55)\,\pt$, and the quasi-regular tetrahedra around the third
vertex $v_i$ give $0.55\,\pt$ more. If a top edge of length at
least $\xi$ lies around all three vertices, then one of the
endpoints of the edge lies in $\{v_1,v_2,v_3\}$, so the result
follows from Claim 1. Finally, if all top edges have length at
most $\xi$, we use Inequalities~3, ~6, ~7.

Take $k=4$.  Suppose there is a top edge $e$ of length at least
$\xi$. If $e$ lies around only one of the $v_i$, we reduce to the
case $k=3$. If it lies around two of them, then the two
quasi-regular tetrahedra along this edge give $2(0.55)\,\pt$ and
the quasi-regular tetrahedra around the other two vertices $v_i$
give another $2(0.55)\,\pt$.  If both endpoints of $e$ are among
the vertices $v_i$, the result follows from Claim 2.  This happens
in particular if $e$ lies around four vertices.  If $e$ lies
around only three vertices, one of its endpoints is one of the
vertices $v_i$, say $v_1$.  Assume $e$ is not around $v_2$. If
$v_2$ is not adjacent to $v_1$, then Claim 1 gives the result. So
taking $v_1$ adjacent to $v_2$, we adapt Claim 1, by using all
seven Inequalities, to show that the eight quasi-regular
tetrahedra around $v_1$ and $v_2$ give $4(0.55)\,\pt$. Finally, if
all top edges have length at most $\xi$, we use Inequalities~3,
~6, ~7.
\end{proof}

In a special case, the constant of Lemma~\ref{lemma:0.55} can be
improved by a small amount.

\begin{lemma}
    \label{lemma:0.55A}
    %proclaim{Lemma 5.3}
Let $v$ be a vertex of a centered packing of type $(5,0)$.  Let
$S_1,\ldots, S_5$ be quasi-regular tetrahedra around the edge
$\{v_0,v\}$. Then
    $$\sum_{i=1}^5 \sigma(S_i) < 4.52\,\pt - 10^{-8}.$$
\end{lemma}

\begin{proof}
If any of the top edges has length greater than $\xi$, we use a
slightly improved calculation\footnote{\calc{241241504-1}} that
yields this constant. Otherwise, the same
calculation\footnote{\calc{82950290}} that was used in the previous
lemma gives the desired estimate
  $$
  \sum\sigma < 5(0.31023815) - 2\pi(0.207045) < 4.52\,\pt - 10^{-8}
  $$
\end{proof}

\subsection{Limitations on Types}%DCG 10.3, p103
    %\heads{6. Limitations on types}

Recall that a vertex of a planar map has type $(p,q)$ if it is the
vertex of exactly $p$ triangles and $q$ quadrilaterals. This
section restricts the possible types that appear in a centered
packing.

Let $t_4$ denote the constant $0.1317\approx 2.37838774\,\pt$.

\begin{lemma}\label{lemma:0.1317} If $R$ is a quad cluster, then
   $$\tau_R(v,\Lambda) \ge t_4.$$
\end{lemma}

\begin{proof}
A calculation\footnote{\calc{996268658}} asserts precisely this.
\end{proof}

\begin{lemma} \label{lemma:pq-impossible}
    %\proclaim{Lemma 6.1}
The following eight types $(p,q)$ are impossible:
    (1)  $p\ge 8$,
    (2)  $p\ge 6$ and $q\ge 1$,
    (3)  $p \ge 5$ and $q\ge 2$,
    (4)  $p \ge 4$ and $q\ge 3$,
    (5)  $p \ge 2$ and $q\ge 4$,
    (6)  $p \ge 0$ and $q\ge 6$,
    (7)  $p \le 3$ and $q=0$,
    (8) $p \le 1$ and $q=1$.
\end{lemma}

\begin{proof}
Calculations\footnote{\calc{657406669}, \calc{208809199},
\calc{984463800}, and \calc{277330628}} give a lower bound on the
dihedral angle of $p$ simplices and $q$ quadrilaterals at
$0.8638p+1.153 q$ and an upper bound of $1.874445 p + 3.247 q$. If
the type exists, these constants must straddle $2\pi$. One readily
verifies in Cases 1--8 that these constants do not straddle
$2\pi$.
\end{proof}

\begin{lemma}
    \label{lemma:pq-types}
    %\proclaim{Lemma 6.2}
If the type of any vertex of a centered packing is one of
$(4,2)$, $(3,3)$, $(1,4)$, $(1,5)$, $(0,5)$, $(0,2)$, %$(7,0)$,
then the centered packing does not contravene.
\end{lemma}


\begin{proof}  According to Table~\ref{eqn:old5.1},
we have $\tlp(p,q)> \squander$, for $(p,q) = (4,2)$, $(3,3)$,
$(1,4)$, $(1,5)$, $(0,5)$, or $(0,2)$. By
Lemma~\ref{lemma:sigma-tau}, the result follows in these cases.
\end{proof}



\begin{remark} \label{rem:pq-list}
In summary of the preceding two lemmas, we find that we may
restrict our attention to the following types of vertices.
    $$
    \begin{matrix}
   (7,0)&      &       &       &       \\
   (6,0)&      &       &       &       \\
   (5,0)&(5,1) &       &       &       \\
   (4,0)&(4,1) &       &       &       \\
        &(3,1) &(3,2)  &       &       \\
        &(2,1) &(2,2)  &(2,3)  &       \\
        &      &(1,2)  &(1,3)  &       \\
        &      &       &(0,3)  &(0,4)  \\
    \end{matrix}
    $$
It will be shown in Lemma~\ref{lemma:70}, that the type $(7,0)$
does not occur in a contravening centered packing.
\end{remark}


\subsection{Two darts at an upright node}


\begin{lemma}
Let $H$ be a geometric hypermap.  Let $N$ be an upright node
of cardinality $2$.  Then exactly one dart at $N$ is a quarter.
\end{lemma}

\begin{proof}
An upright node has at least one dart that is a quarter.
The dihedral angle of a quarter is
less than\footnote{\calc{971555266}} $\pi$, so it is
impossible for both darts to be quarters.
\end{proof}

\begin{lemma}\label{a:context21} %\label{eqn:4.9}
Let $H$ be a geometric hypermap.  Let $N$ be an upright node
of cardinality $2$.  Assume that exactly one dart $\alpha$ 
at $N$ is a quarter.  Let $\alpha'$ be the other dart at $N$.
Then
 $$
 \optt{mu}(\alpha) + \optt{kappa}(\alpha') < \optt{vor0}(\alpha).
 $$
\end{lemma}

\begin{proof}
This follows from
calculations\footnote{\calc{906566422}, \calc{703457064}, and
\calc{175514843}}.
\end{proof}

\subsection{Three darts at an upright node}
%\section{Three anchors} %DCG 11.3,p.114
    \oldlabel{3.4}


\begin{lemma}\dcg{Lemma~11.2}{114}
    \oldlabel{3.4.1}
Let $H$ be a geometric hypermap. Assume that $N$ is a node
of cardinality three and that a unique dart $\alpha_0$  of $N$
is an upright quarter.  Let $\alpha_1,\alpha_2$ be the other
two darts of $N$.
Let 
  $$\optt{s}(\alpha)=\begin{cases}
  \optt{kappa}(\alpha),&\text{if $\alpha$ fitted crown}\\
  \optt{vor\_anal}(\alpha)-\optt{vor0}(\alpha),&\text{if $\alpha$ type C}\\
  0,&\text{otherwise}
  \end{cases}
  $$
Then 
  $$
  s(\alpha_1) + s(\alpha_2) + \optt{nu}(\alpha_0) < \optt{vor0}(\alpha_0).
  $$
% The upright diagonal can be erased in the context $\x(3,2)$.
\end{lemma}


\begin{proof}
Let $v_1$ and $v_2$ be the two anchors of the upright diagonal $\{0,v\}$
along the quarter. Let the third anchor be $v_3$.

Assume first that $|v|\ge 2.696$. If $Q$ is compressed,
then\footnote{\calc{73974037}} %A10  by $\A_{10}$,
the score is dominated by the truncated
function $\op{sv}_0$.  Assume $Q$ is decompressed. If $|v_1|$,
$|v_2|\le 2.45$, then a calculation\footnote{\calc{764978100}} %A11
gives the result. Take $|v_2|\ge
2.45$.  By symmetry, $|v-v_1|$ or $|v-v_2|\ge 2.45$. The case
$|v-v_1|\ge2.45$ is treated by another calculation.%
\footnote{\calc{764978100}} %A11
We take
$|v-v_2|\ge2.45$. Let $S=\{0,v,v_2,v_3\}$. If $S$ is of type $\SC$,
the result follows.\footnote{\calc{764978100}} %A11
$S$ is of type $\SC$, if and only if $y_4\le 2.77$, (because
by Lemma~\ref{tarski:eta245}, $\eta_{456}>\sqrt2$).
If $S$ is  not of type $\SC$, then by Lemma~\ref{tarski:eta696} and
Lemma~\ref{tarski:eta-rad},
we have $\rad(S) \ge\eta(|v|,2.45,24.5)\ge \eta_0(|v|/2)$.
This justifies the use of $\kappa$ (see Section~\ref{x-2.3}
Case (2)). That the truncated function dominates the score now
follows from a calculation.\footnote{\calc{618205535}} %A9

Now assume that $|v|\le 2.696$. If the simplices $\{0,v,v_1,v_3\}$
and $\{0,v,v_2,v_3\}$ are of type $\SC$, the bound follows from a
calculation.\footnote{\calc{73974037}} %A10
\footnote{\calc{764978100}} %A11
%$\A_{10},\A_{11}$.
If say  $S=\{0,v,v_2,v_3\}$ is not of type $\SC$,
then
    $$\rad(S)\ge\sqrt2>  \eta_0(2.696/2)\ge\eta_0(h),$$
justifying the use of $\kappa$. The bound follows from further
calculations.\footnote{\calc{618205535}} %A9
\footnote{\calc{73974037}} %A10
\footnote{\calc{764978100}} %A11
%    $\A_9,\A_{10},\A_{11}$.
($\Gamma+\kappa <\octavor_0$,
etc.)
\end{proof}


\begin{lemma}\dcg{Lemma~11.3}{115}
    \oldlabel{3.4.2}
    \label{lemma:unerased}
Let $H$ be a geometric hypermap.  Let $N$ be a node
of cardinality three.  Assume that exactly two darts at $N$
are quarters.  Assume that the width of the third dart $\alpha_0$
is at least $2\sqrt2$.  (XX By the rules of Definition~\ref{def:q-system},
this is equivalent to saying that the node is not enclosed over
a masked flat quarter.)
Then 
% The upright diagonal can be erased in the context $\x(3,1)$, provided
% the three anchors do not form a flat quarter at the origin.
\end{lemma}

\begin{proof}
In the absence of a flat quarter, truncate, score, and remove the
vertex $v$ as in the context $\x(3,1)$ of
Lemma~\ref{lemma:mixed-vor0}. 
\end{proof}

\subsection{Six darts at an upright node}
%\section{Six anchors} %DCG 11.4, p.115
    \oldlabel{3.5}

\begin{lemma}\dcg{Lemma~11.4}{115}  
Let $H$ be a geometric hypermap.
Let $N$ be a node of cardinality at least six.
Let the darts at $N$ that are quarters be
$\alpha_1,\ldots,\alpha_k$.
Then 
  $$
  \sum_{i=1}^k\optt{tau\_nu}(\alpha_i) > \squander.
  $$
%An upright diagonal has at most five anchors.
\end{lemma}

\begin{proof}
The proof relies on constants and inequalities from two
calculations.\footnote{\calc{729988292}} %A3
\footnote{\calc{83777706}} %A8
%$\A_3$ and $\A_8$.
If between two anchors there is a quarter, then the angle is
greater than $0.956$, but if there is not,  the angle is greater than
$1.23$.  So if there are $k$ quarters and at least six anchors, they
squander more than
    $$ k (1.01104) - [2\pi-(6-k)1.23]0.78701 > \squander,$$
for $k\ge0$.
\end{proof}

\subsection{Five darts at an upright node}
\label{sec:5updart}

\begin{lemma}\dcg{Sec~11.7,intro}{118}\label{a:5dart:concave}  
Let $H=(D,n,e,f)$ be a geometric
hypermap.  Let $N$ 
be a node of $H$ of cardinality 5.    Assume that for all $\alpha\in N$
  $$
  \optt{upright}(\alpha).
  $$
Then $\optt{azim}(\alpha)\slt \pi$ for all $\alpha\in N$.
Hence $\optt{gap}(\alpha)$ or $\optt{slice}(\alpha)$.
\end{lemma}

\begin{proof}  The angle is at most $2\pi - 4(0.956) < \pi$.
\end{proof}

\begin{lemma}\dcg{Sec~11.7,Rem~11.3}{118}  An upright diagonal with
five darts has at most $2$ gaps.  More precisely, let $H$ be a geometric
hypermap.  Let $N$ 
be a node of $H$ of cardinality 5.   Assume that $\optt{upright}(\alpha)$
for $\alpha\in N$.  There is a set $S\subset N$ of cardinality at most
$2$ such that $\optt{gap}(\alpha) \Rightarrow \alpha\in S$.
\end{lemma}

\begin{proof}
There are at most
two gaps by the calculation\footnote{\calc{83777706}} %A8
    $$3(1.65)+2(0.956)>2\pi.$$
\end{proof}

\begin{lemma}\label{a:5dart:3q}\dcg{Sec~11.7,in Lemma~11.14}{119}
Let $H$ be a geometric hypermap.
Let $N$ be a node of $H$ of  of cardinality $5$.  Assume that
two of the darts at $N$ are gaps.  Then the other three darts
at $N$ are quarters.
\end{lemma}

\begin{proof}
By a calculation,\footnote{\calc{83777706}} %A8 $\A_8$,
the slices are all quarters,
    $1.23+2(1.65)+2(0.956)>2\pi$.
\end{proof}

\begin{lemma}\dcg{Lemma~11.14}{119}  Let $H$ be a geometric
hypermap.  Let $N$ be a node of $H$ of cardinality $5$.  Assume
that $\optt{upright}(\alpha)$ for $\alpha\in N$.  Assume that there
is a set $S\subset N$ of cardinality $3$ such that
  $$\optt{quarter}(\alpha)\Leftrightarrow \alpha\in S.$$
Then 
  $$
  \sum_{\alpha\in S} \optt{tau\_nu}(\alpha) \sgt \squander.
  $$
\end{lemma}

\begin{proof}
The dihedral angle of the quarters combined is less than $2\pi-2(1.65)$.  
The linear programming
bound based on various inequalities\footnote{\calc{729988292}} %A3 $\A_3$
is greater than $0.859>\squander$.
\end{proof}



\begin{lemma}\label{lemma:4-crowdedq}\dcg{Lemma~11.18}{119}
    %\oldlabel{3.8.3}
Let $H$ be a geometric hypergraph.
Suppose a node of cardinality five is an upright diagonal.
Suppose that exactly one of the five darts is a gap.
 If
any of the four slices is not an upright quarter then
the centered packing does not contravene.
\end{lemma}

\begin{proof}
We use a series of inequalities.\footnote{\calc{628964355}} %A5
\footnote{\calc{187932932}} %A7
\end{proof}

\begin{lemma}
\label{lemma:4-crowded}\dcg{Lemma~11.18}{119}
Let $H$ be a geometric hypermap.
Suppose a node of cardinality five is an upright diagonal.
Suppose that exactly one of the five darts is a gap.
The sum of $\optt{nu}$ over the four slices is at most
$-0.25$. The sum of $\optt{tau\_nu}$ over the
four slices is at least $0.4$.
\end{lemma}

\begin{proof}
A list of inequalities\footnote{\calc{815492935}} %A2 $\A_2$
together with\footnote{\calc{83777706}} %A8
$\dih>1.65$ give the bound $-0.25$.
Further inequalities \footnote{\calc{729988292}} %A3 $\A_3$
give the bound $0.4$.  
\end{proof}


\begin{lemma}\dcg{Cor~11.9}{120}  
Let $H$ be a geometric hypermap.
Suupose that the node $N$ is a $4$-crowded upright
diagonal.  Let $\alpha_1,\ldots,\alpha_4$ be the quarters
and let $\alpha_0$ be a gap at the node $N$.  Then
  $$
  \optt{kappa}(\alpha_0) + \sum_{i=1}^4 \optt{tau\_nu}(\alpha_i)
  > 0.42274.
  $$
\end{lemma}

\begin{proof}  The crown along the gap,
with the bound of Lemma~\ref{lemma:4-crowded}, 
gives\footnote{\calc{618205535}} %A9
    $0.4-\kappa \ge 0.4+0.02274$
squandered by the upright quarters around a $4$-crowded upright
diagonal.
\end{proof}


\begin{lemma} \label{a:min0-vor} 
Let $H$ be a geometric hypermap.
Let $N$ be a node of degree three.
Assume that darts $\alpha_1$ and $\alpha_2$ are quarters,
and that $\alpha_0$ is a dart with azimuth angle at most $\pi$ 
and width less than $2\sqrt2$.
Then $$\optt{vu}(\alpha_1) + \optt{nu}(\alpha_2) <
     \optt{vor0}(\alpha_1) + \optt{vor0}(\alpha_2).$$
\end{lemma}

\begin{proof}
By a calculation\footnote{\calc{855677395}}, if $|v|\ge 2.69$,
then the upright quarters satisfy
    $$\nu < \op{sv}_0 + 0.01 (\pi/2-\dih)$$
so the upright quarters can be erased.  Thus we assume without
loss of generality that $|v|\le 2.69$.

By Lemma~\ref{tarski:E:part4:2}, we have $|v|>2.6$.
If $|v_1-v_2|\le 2.1$,  or $|v_1-v_3|\le 2.1$, then
Lemma~\ref{tarski:E:part4:3}, gives $|v|>2.72$, 
 contrary to assumption.  So take $|v_1-v_2|\ge 2.1$ and
$|v_1-v_3|\ge2.1$. Under these conditions we have the interval
calculation\footnote{\calc{148776243}} %A13
  $\nu(Q) < \op{sv}_0(Q)$ where $Q$ is the upright quarter.
\end{proof}


\subsection{Four darts at an upright node}


\begin{lemma}\dcg{Remark~11.28}{125}
\label{remark:3rd-quarter} Let $H$ be a geometric hypermap.
Let $N$ be an upright node of degree four at which there are
exactly three darts $\alpha_1$, $\alpha_2$, $\alpha_3$
that are upright quarters.  Assume that the node is enclosed
over a masked flat quarter.  
Then
 $$
 \sum_{i=1}^3\optt{vu}(\alpha_i) <
     \sum_{i=1}^3\optt{vor0}(\alpha_i) + \xiV.
 $$
\end{lemma}

\begin{proof}
 If we have an upright diagonal enclosed
over a masked flat quarter in the context $\x(4,1)$, then there are
three upright quarters.  By the same argument as in Lemma~\ref{a:min0-vor}, 
the two quarters over the masked flat quarter score $\le\op{sv}_0$. The
third quarter is dominated by $\op{sv}_0 + \xiV$.
\end{proof}


\begin{lemma}[Erasing four darts, no masked]
\dcg{Lemma~11.21}{120}
\oldlabel{3.9.1}
Let $H$ be a geometric hypermap.  Let $N$ be a node
whose darts are upright.  Assume that $N$ has cardinality four.
Assume that there are at least as many non-slices as quarters at $N$.
Let $f(\alpha)$ be given as $\optt{nu}$ if $\alpha$ is an upright
quarter, $\optt{vor0}$ if it is another slice, and
$\optt{kappa}$ at darts that are not slices.  Then
  $$
  \sum_{\alpha\in N} f(\alpha) < \sum_{\alpha\in N} \optt{vor0}(\alpha).
  $$
\end{lemma}

\begin{proof}
By assumption, there are at least as non-slices as upright quarters. Each
non-slice drops us by $\xik$ and each quarter lifts us by at most%
\footnote{\calc{618205535}} %A9
\footnote{\calc{73974037}} %A10
\footnote{\calc{764978100}} %A11
$\xiG$. We have $\xikG<0$.
\end{proof}

\begin{lemma}[Erasing four darts, masked]\dcg{After Rem~11.22}{120}
Let $H$ be a geometric hypermap.  Let $N$ be a node
whose darts are upright.  Assume that $N$ has cardinality four.
Assume that ``$N$ is enclosed over a flat quarter'' at dart $\beta$.
Assume that there are at least as many non-slices as quarters at $N$.
Let $f(\alpha)$ be given as $\optt{nu}$ if $\alpha$ is an upright
quarter, $\optt{vor0}$ if it is another slice, and
$\optt{kappa}$ at darts that are not slices.  Then
  $$
  \sum_{\alpha\in N} f(\alpha) < -0.0114 + 
  \sum_{\alpha\in N} \optt{vor0}(\alpha).
  $$
\end{lemma}

\begin{proof}
The azimuth angle at a non-slice is $>1.65$. 
We have
$0.0114< -2\xikG$.
XX This proof seems incomplete.  Don't we need $\optt{nu}(\alpha) <
\optt{vor0}$ based on something like DCG-Remark~11.22?
\end{proof}

XX We remark that the preceding lemma is designed for use with 
the scoring of DCG-page123-2(c).




\begin{lemma}\dcg{Lemma~11.23}{121}
    \oldlabel{3.9.2}
    \label{lemma:0.008}
Let $H$ be a geometric hypermap.  Let
Let $N$ be a node that is an upright diagonal with four darts.  
Assume that one of the darts $\alpha_0$ is a gap and that the other
three $\alpha_1,\alpha_2,\alpha_3$ are slices.  
\begin{itemize}
\item If all of the slices are upright quarters, then
  $$
  \optt{kappa}(\alpha_0) + \sum_{i=1}^3 \optt{nu}(\alpha_i) <
  \sum_{i=1}^3 \optt{vor0}(\alpha_i) + 0.008.
  $$
%The slices can be erased with penalty  $\pi_0=0.008$. 
\item Assume that one of the slices is not an upright quarter.
Let 
$\optt{s}(\alpha)$ equal $\optt{nu}$ at slices that
are quarters and $\optt{vor0}$ at slices that are not.
Then
  $$
  \optt{kappa}(\alpha_0) + \sum_{i=1}^3 \optt{s}(\alpha) <
  \sum_{i=1}^3 \optt{vor0}(\alpha_i) + 0.00222.
  $$
% we can erase with penalty $\pi_0=0.00222$.
\end{itemize}
\end{lemma}


\begin{proof}
The constants and inequalities used in this proof can be found in
a series of calculations.%
\footnote{\calc{618205535}} %A9
\footnote{\calc{73974037}} %A10
\footnote{\calc{764978100}} %A11


First we establish the penalty $0.008$.   
By these inequalities, the result follows
if the diagonal satisfies $y_1\ge 2.57$.

Take $y_1\le 2.57$. If any of the upright quarters are decompressed, 
the result follows from $(\xikG+\xiG<0.008)$. If the edges
along the gap are less than $2.25$, the result follows from
$(-0.03883+3\xiG = 0.008)$. If all but one edge along the
gap are  less than 2.25, the result follows from $(-0.0325 + 2\xiG
+ 0.00928 = 0.008)$.

If there are at least two edges along the gap of length at
least $2.25$, we consider two cases according to whether they lie
on a common face of an upright quarter.  The same group of
inequalities gives the result. The bound $0.008$ is now fully
established.

\smallskip
Next we prove the bound involving $0.00222$, when one
of the slices is not a quarter.  If $|v|\ge2.57$, then
we use
    $$2\xiG + \xiV + \xik \le 0.00935+0.003521 -0.2274\le 0.$$
If $|v|\le2.57$, we use
    $$2(0.01561)-0.029 \le 0.00222.$$
\end{proof}


\begin{lemma}\dcg{Lemma~11.23}{121}
Let $H$ be a geometric hypermap.  Let
Let $N$ be a node that is an upright diagonal with four darts.  
Assume that one of the darts $\alpha_0$ is a gap and that the other
three $\alpha_1,\alpha_2,\alpha_3$ are slices.  
Let 
$\optt{s}(\alpha)$ equal $\optt{nu}$ at slices that
are quarters and $\optt{vor0}$ at slices that are not.
Assume 
some upright quarter along this diagonal masks a flat quarter.
Then (1) or (2) holds.
   \begin{enumerate}
    \item 
  $$
  \optt{kappa}(\alpha_0) + \sum_{i=1}^3 \optt{s}(\alpha) <
  \sum_{i=1}^3 \optt{vor0}(\alpha_i) -0.0063.
  $$
  The diagonal of the flat is at least $2.6$, and the edge
    opposite the diagonal is at least $2.2$.
    \item 
    $$
  \optt{kappa}(\alpha_0) + \sum_{i=1}^3 \optt{s}(\alpha) <
  \sum_{i=1}^3 \optt{vor0}(\alpha_i) -0.0114.
  $$
   The diagonal of the flat is at least $2.7$, and the edge
    opposite the diagonal is at most $2.2$.
    \end{enumerate}
\end{lemma}



\begin{proof}
\smallskip
Let $v_1\ldots,v_4$ be the consecutive anchors of
the upright diagonal $\{0,v\}$ with $\{v_1,v_4\}$ the gap.
Suppose $|v_1-v_3|\le 2\sqrt{2}$.

By Lemma~\ref{tarski:dcg-p122}, 
the upright diagonal $\{0,v\}$ is not enclosed over
$\{0,v_1,v_2,v_3\}$.   
Thus, $\op{conv}^0\{v_1,v_3\}$ meets $\op{conv}\{0,v,v_2\}$ so that the
simplices $\{0,v,v_1,v_2\}$
and $\{0,v,v_2,v_3\}$ are decompressed.

To complete the proof of the lemma, we show that when
some upright quarter along this diagonal masks a flat quarter, 
 either (1) or (2) holds.
Suppose we mask a flat quarter $Q'=\{0,v_1,v_2,v_3\}$.
We have established that $\op{conv}^0\{v_1,v_3\}$ meets 
$\op{conv}\{0,v,v_2\}$.
To establish (1) assume that $|v_2|\ge 2.2$.  Lemma~\ref{remark:2.6} 
gives
    $$|v_1-v_3|>2.6.$$
The bound $0.0063$ comes from
    $$\xikG + 2\xiV < -0.0063$$

To establish (2) assume that $|v_2|\le 2.2$. Lemma~\ref{remark:2.6} gives
    $$|v_1-v_3|>2.7.$$
  If the simplex
$\{0,v,v_3,v_4\}$ is decompressed, then $$\xik + 3\xiV  < -0.0114$$
Assume that $\{0,v,v_3,v_4\}$ is compressed. We have
    $$-0.004131 +\xikG + \xiV \le -0.0114.$$
\end{proof}

\subsection{Flat quarters} %

XX The following is a direct application of interval arithmetic.
Why repeat it here?


\begin{lemma}\dcg{Lemma~11.29}{125}
    \oldlabel{3.11.3}
$\mu < \op{sv}_0 +0.0268$ for all flat quarters. If the central
vertex has height $\le2.17$, then $\mu<\op{sv}_0+0.02$.
\end{lemma}

\begin{proof}
This is an interval calculation.\footnote{\calc{148776243}} %A13
\end{proof}




\begin{lemma}\dcg{Lemma~11.30}{125}\label{lemma:1.32}
    \oldlabel{3.11.4}
Let $H$ be a geometric hypermap.  Let $\alpha$ be a dart that
is standard (XX meaning not upright and edges of length $2$ to $2.51$).
Then the width of $\alpha$ is less than $2\sqrt2$.
\end{lemma}


\begin{proof} Let $S=S(y_1,\ldots,y_6)$ be the simplex inside the exceptional
cluster centered at $v$, with $y_1=|v|$. The inequality $\dih\le 1.32$
gives the interval calculation $y_4< 2\sqrt{2}$., so $S$ is a quarter.
\end{proof}


XX The following is a direct application of an interval
arithmetic calculation.  Why repeat it here?

\begin{lemma}
Let $v$ be a corner of a flat quarter at which the
dihedral angle is at most $1.32$. 
Then $\hat\tau(Q)>3.07\,\pt$. Moreover, if $\hat\sigma=\op{sv}_0$ and if
$\eta_{456}\ge\sqrt2$, 
we may use the stronger constant
$\tau_0(Q)> 3.07\,\pt+\xi_V+2\xiG'$.
\end{lemma}


\begin{proof}
The result follows by
interval arithmetic.\footnote{\calc{148776243}} %A13
\end{proof}

\subsection{Tame Plane Graphs}


\begin{lemma}\label{a:6}\dcg{Lemma~21.4}{223} 
Formally contravening hypermaps satisfy Property
\ref{definition:tame:degree} of tameness: The cardinality of every
node is at least $2$ and at most $6$.
\end{lemma}

\begin{proof}
Let the type of the node be $(p,q,r)$.  If $r=0$, then the
impossibility of a node of cardinality $7$ or more is found in the
table entry $b(7,0)$ (Lemma~\ref{lemma:pq-types:bis}). If $r\ge1$,
then Lemma~\ref{lemma:0.8638} shows that the azimuth angles of the
darts at the node cannot sum to $2\pi$:
    $$6 (0.8638) + 1.153 > 2\pi.$$
\end{proof}




\subsection{Computer Calculations and Their Consequences}
\label{sec:ccc}

We have the following linear program. There are many different
choices of objective function and constraint {\it Csum} depending on
the particular constants $\sLP$, $\tauLP$, or $\tlp/\pt$ that need
to be computed.  In the linear program the constants $\pi$ and $\pt$
are replaced by numerical approximations.  Section~\ref{XX} explains
how the output from the numerical routines can be adjusted to yield
perfectly rigorous results.  The listing is in a format that can be
read by the program {\it LPSolve}.  See \cite{lpsolve}.

The origin of these inequalities is interval arithmetic.  They are
listed in nonlinear form at \cite{XX}.  The numeric labels of the
equations here is consistent with the labels in that archive.

The correspondence between linear program variables in the program
listing and the variables in use elsewhere in the book is the
following.  Here $F$ is a face, and $x$ is a dart in that face.
    $$
    \begin{array}{lll}
    \card(F)=3 &\Rightarrow\  \azim(x)=\azim_3,\ \tau(F)=\op{tau}_3,\ \sigma(F)=\op{sigma}_3\\
    \card(F)=4 &\Rightarrow\  \azim(x)=\azim_4,\ \tau(F)=\op{tau}_4,\ \sigma(F)=\op{sigma}_4\\
    \end{array}
    $$

{ \obeylines\tt
  \hbox{}\parindent=4pt

 /* Change  "min/max" and "Csum", according to the objective */
 \ \hbox{}
 /* This example computes b(2,2) */
 // min: 2 tau3\_s + 2 tau4\_s;
 // Csum: 2 azim3 + 2 azim4 - twopi = 0;
 \ \hbox{}
 /* This example computes tauLP(2,2,5.0) */
 //min: 2 tau3 + 2 tau4;
 //Csum: 2 azim3 + 2 tau4 <= 5.0;
 \ \hbox{}
 /* This example computes sigmaLP(5,0,2pi-1.153) */
 max: 5 sigma3 + 0 sigma4;
 Csum: 5 azim3 + 0 sigma4 - twopi <= -1.153;
 \ \hbox{}
 /* Variable bounds */
 twopi: twopi =  6.2831853071795862;
 \ \hbox{}
 // pt = 0.055373645668464144;
 Ctaup: 0.055373645668464144 tau3\_s - tau3 = 0;
 Ctauq: 0.055373645668464144 tau4\_s - tau4 = 0;
 \ \hbox{}
 /* assumed conditions: */
 /* triangle tau */
 J927432550: 0.3897 azim3 + tau3 > 0.4666;
 J221945658: 0.2993 azim3 + tau3 > 0.3683;
 J53415898:  tau3 > 0.0;
 J106537269: -0.1689 azim3 + tau3 > -0.208;
 J254527291: -0.2529 azim3 + tau3 > -0.3442;
 \ \hbox{}
 /* triangle sigma */
 J539256862: sigma3 - 0.37898 azim3 < -0.4111;
 J864218323: sigma3 + 0.142 azim3 < 0.23021;
 Jsigma\_1pt: -Infinity <= sigma3 <= 1.0;
 J776305271: sigma3 + 0.3302 azim3 < 0.5353;
 \ \hbox{}
 /* quad  tau */
 J539320075: 4.49461 azim4 + tau4 > 5.81446 ;
 J122375455: 2.1406 azim4 + tau4 > 2.955;
 J408478278: 0.316 azim4 + tau4 > 0.6438;
 J996268658: tau4 > 0.1317;
 J393682353: -0.2365 azim4 + tau4 > -0.3825;
 J775642319: -0.4747 azim4 + tau4 > -1.071;
 \ \hbox{}
 /* quad sigma */
 J310151857: sigma4 - 4.56766 azim4 < -5.7906;
 J655029773: sigma4 - 1.5094 azim4 < -2.0749;
 J\_73283761:  sigma4 - 0.5301 azim4 < -0.8341;
 JLemm14\_11: -Infinity <= sigma4 <= 0;
 J\_15141595:  sigma4 - 0.3878 azim4 < -0.6284;
 J574391221: sigma4 + 0.1897 azim4 < 0.4124;
 J396281725: sigma4 + 0.5905 azim4 < 1.5707;
 \ \hbox{}

 /* all vars have lower bound 0, except sigma3, sigma4  */


}

\bigskip

We let $\tauLP(p,q,\alpha)$ denote the solution to this linear
program with objective $$\min: p\, \tau_3 + q\, \tau_4$$ and
constraint
$$\op{Csum}: p\, \azim_3 + q\,\azim_4 \le d.$$

We let $\tlp(p,q)$ denote the solution to this linear program with
objective $$\min: p \tau_3 + q \tau_4$$ and constraint
$$\op{Csum}: p\, \azim_3 + q\,\azim_4 =2\pi.$$  The constants $b(p,q)$
are computed as lower bounds satisfying $\tlp(p,q) > b(p,q)\,\pt$,
which the exception of the constants $b(5,0)$ and $b(7,0)$, which
are slight improvements on the linear programs.

We let $\sLP(p,q,\alpha)$ denote the solution to this linear program
with objective $$\max: p\, \sigma_3 + q\, \sigma_4$$ and constraint
$$\op{Csum}: p\, \azim_3 + q\,\azim_4 \le d.$$



\begin{lemma} We have the following estimates:
    $$
    \begin{array}{lll}
    &s_5+\sLP(5,0,2\pi-1.153)< c(8)\,\pt\\
    &s_6+\sLP(5,0,2\pi-1.153) < s_9\\
    &s_5+\sLP(5,0,2\pi-1.153)<s_8\\
    &(9-2(0.48))\,\pt+s_5+\sLP(2,2,2\pi-1.153)<8\,\pt\\
    &2t_5+\tauLP(4,0,2\pi-2(1.153))>\squander\\
    \end{array}
    $$
\end{lemma}

\begin{proof} Run the linear programs and see what you get.
\end{proof}

These are just a few of a long list of inequalities such as these
that will appear in the pages that follow.  They all come from the
same basic linear program with varying objective function and angle
sum constraint.



\begin{lemma}  If $v$  is a node of an exceptional face,
and if there are $6$ faces meeting at $v$, then the exceptional face
is a pentagon and the other $5$ faces are triangles.  In particular,
the node has type $(5,0,1)$.
\end{lemma}

\begin{proof}  Let $(p,q,r)$ be the type of the node.  We consider
several cases, according to the value of $p$.

{\bf($p\le2$)} If there are at least four non-triangular regions at
the node, then the sum of azimuth angles around the node is at least
$4(1.153)+2(0.8638)>2\pi$, which is impossible.  (See
Lemma~\ref{lemma:0.8638}.)

{\bf($p=3$)} If there are three non-triangular regions at the node,
then $\tau^*(H)$ is at least
$2t_4+t_5+\tauLP(3,0,2\pi-3(1.153))>\squander$.

{\bf($p=4$)} If there are two exceptional regions at the node, then
$\tau^*(H)$ is at least $2t_5+\tauLP(4,0,2\pi-2(1.153))>\squander$.

If there are two non-triangular regions at the node, then
$\tau^*(H)$ is at least  $t_5+\tauLP(4,1,2\pi-1.153)>\squander$.

{\bf($p=5$)} We are left with the case of five triangles and one
exceptional face.

When there is an exceptional face at a node of cardinality six, we
claim that the exceptional face must be a pentagon. If the face is a
heptagon or more, then $\tau^*(H)$ is at least
$t_7+\tauLP(5,0,2\pi-1.153) > \squander$.

If the face is a hexagon, then $\tau^*(H)$ is at least $t_6 +
\tauLP(5,0,2\pi-1.153) > t_9$. Also, $s_6+\sLP(5,0,2\pi-1.153) <
s_9$. The contour loop around the six faces has at most $9$ face
steps. Lemma~\ref{lemma:s9-t9:bis} gives the bound of $8\,\pt$.
\end{proof}




\begin{lemma}
    \label{lemma:aggregate6}
    Let $(H,\azim,\flat,\sigma)$ be formally contravening.
    \begin{enumerate}
    \item The aggregate $F$ of the six faces at a node of type
    $(5,0,1)$ satisfies
            $$
            \begin{array}{lll}
            \sigma(F) < s_8,\\
            \tau(F) > t_8.
            \end{array}
            $$
    \item There are at most two nodes of type $(5,0,1)$.  If
        there are two, then they are non-adjacent vertices on a
        pentagon, as shown in Figure \ref{fig:doubledegree6}.  (The
        pentagon has a node of type $(1,0,1)$.)
    \end{enumerate}
\end{lemma}
\begin{figure}[htb]
  \centering
  \myincludegraphics{\ps/doubledegree6.eps}
  \caption{Non-adjacent nodes of cardinality $6$ on a pentagon}
  \label{fig:doubledegree6}
\end{figure}

\begin{proof}
We begin with the first part of the lemma. The sum  $\tau(F)$ over
these six standard regions is at least
    $$t_5+\tauLP(5,0,2\pi-1.153)> t_8.$$
Similarly,
    $$s_5+\sLP(5,0,2\pi-1.153)<s_8.$$
%
We note that there can be at most one exceptional face with a node
of cardinality six.  Indeed, if there are two, then they must both
be nodes of the same pentagon:
    $$t_8+t_5>\squander.$$
Such a second node on the octagonal aggregate leads to one of the
follow constants greater than $\squander$.  These same constants
show that such a second node on a hexagonal aggregate must share two
triangular faces with the first node of cardinality six.
$$\begin{array}{lll}
    t_8 &+\tauLP(4,0,2\pi-1.32-0.8638),\quad\text{or}\\
    t_8 &+1.47\,\pt+\tauLP(4,0,2\pi-1.153-0.8638),\quad\text{or}\\
    t_8 &+\tauLP(5,0,2\pi-1.153) .
\end{array}
$$
(The relevant constants are found at Lemma~\ref{lemma:1.47} and
Lemma~\ref{lemma:0.8638}.)
\end{proof}

\begin{lemma}\label{a:311}  % Used in separation props of tame graphs.
Assume the node has type $(3,1,1)$.
 Assume the azimuth angle of the dart on the exceptional face is
least $1.32$, then
    \begin{equation}
    \tauLP(3,1,2\pi-1.32)>1.4\,\pt + t_4.
    \label{eqn:tau1.32}
    \end{equation}
This gives the bound in the sense of Lemma~\ref{lemma:split} at such
a node. 
\end{lemma}

\begin{lemma}\label{a:no-ef}  Let $v$ be a node of type
$(p,q,r)=(4,0,1)$, $(3,1,1)$, or $(3,0,2)$.  Assume that at $v$ the
exceptional darts are not flat.  Then
    $$\tauLP(p,q,\alpha) > ( p d(3) + q d(4) + a(p))\,\pt.$$
\end{lemma}

\begin{proof}
By Lemma~\ref{lemma:1.32:bis},
the azimuth angles of the
exceptional regions at $v$ are at least $1.32$.   The conclusion%
\footnote{\calc{551665569}, \calc{824762926}, and
\calc{325738864}}
%% K.C.-2002-version: 17.20 Group 20, 17.21 Group 21. (page 49).
follows.
\end{proof}

\section{Assembly Problems Cut from Elsewhere}

\subsection{Quarters score nonpositive}

The following lemma appears as Lemma~\ref{lemma:quarter0}.
It has been implemented as assembly problem \assembly{JPOMPNK}.

\begin{lemma} %\label{lemma:quarter0}\dcg{Lemma 8.12}
Let $Q$ be a quarter in the $Q$-system (either flat or upright).
Then $\sigma(Q)\le 0$. 
\end{lemma}


\begin{proof}  
We make use
of the definition of $\sigma$ on quarters from
Definition~\ref{def:sigma}. The general context (that is, contexts
other than $(2,1)$ and $(4,0)$) of upright quarters is established
by the inequalities\footnote{\calc{522528841} and
\calc{892806084}} that hold for all upright quarters $Q$ with
distinguished vertex $v$:
    $$
    \begin{array}{lll}
    &2\Gamma(Q) + \op{sv}_0(v,Q) - \op{sv}_0(\hat v,Q,\lambda_{goct}) \le 0\\
    &\op{svan}(v,Q) + \op{svan}(\hat v,Q) 
  +\op{sv}_0(v,Q) - \op{sv}_0(\hat v,Q)\le0.
    \end{array}
    $$
For the remaining upright quarters (that is, contexts $(2,1)$ and $(4,0)$)
and for all flat quarters,
it is enough to show that $\Gamma(Q)\le0$, if $\eta^+\le\sqrt2$ and
$\op{svan}(Q,v)\le0$, if $\eta^+\ge\sqrt2$.

Consider the case $\eta^+\le\sqrt2$.  If $Q$ is a quarter such that
every face has circumradius at most $\sqrt2$,
then\footnote{\calc{346093004}} $\Gamma(Q)\le0$.  
Because of this, we may assume that the circumradius of $Q$ is
greater than $\sqr2$. 
Since
(Definition~\ref{def:svor})
    $$4\Gamma(Q)=\sum_{i=1}^4 \op{svan}(Q,v_i),$$
it is enough to show that $\op{svan}(Q)<0$.  Since $\eta^+\le\sqrt2$ 
and the circumradius is greater than
$\sqrt2$, $\op{sv}(Q,\sqrt2)$ is a strict truncation of the $V$-cell
in $Q$, so that
    $$\op{svan}(Q)<\op{svan}(Q,\sqrt2).$$
We show the right hand side is nonpositive.  Let $v$ be the
distinguished vertex of $Q$.  Let $A$ be $1/3$ the solid angle of
$Q$ at $v$ . By the definition of $\op{svan}(Q,\sqrt2)$, it is
nonpositive if and only if
    \begin{equation}
        A\le \doct \,\op{vol}(\op{VC}(Q,v)\cap B(v,\sqrt2)).
        \label{eqn:Adoct}
    \end{equation}
($\op{VC}(Q,v_0)$ is defined in Section~\ref{sec:rules}.) The
intersection $\op{VC}(Q,v)\cap B(v,\sqrt2)$ consists of six Rogers
simplices $R(a,b,\sqrt2)$, three conic wedges (extending out to
$\sqrt2$), and the intersection of $B(v,\sqrt2)$ with a cone over
$v$. By Lemmas~\ref{lemma:rogers-app}, \ref{lemma:wedge}, and
\ref{lemma:cone}, these three types of solids give inequalities like
that of Equation~\ref{eqn:Adoct}. Summing the inequalities from
these lemmas, we get Equation~\ref{eqn:Adoct}.

Consider the case $\eta^+\ge\sqrt2$ and $\sigma=\op{svan}$. If the
quarter is upright, then\footnote{\calc{40003553}} $\op{svan}(Q)\le0$.
Thus, we may assume the quarter is flat.  
The
analytic continuation defining $\op{svan}(Q)$ is the same
\footnote{This claim is justified by \calc{5901405}, which
shows that $\op{svan}(Q)\le0$ when the two functions differ.} as
    $$4(-\doct\op{vol}(X) + \sol(X)/3),$$
where $X$ is the subset of the cone at $v$ over $Q$ consisting of
points in that cone closer to $v$ than to any other vertex of $Q$.
The extreme point of $X$ has distance at least $\sqrt2$ from $v$
(since $\eta^+$ and hence the circumradius of $Q$ are at least
$\sqrt2$).  Thus,
    $$\op{svan}(Q) \le \op{svan}(Q,\sqrt2).$$
We have $\op{svan}(Q,\sqrt2)\le0$ as in the previous paragraph, by
Lemma~\ref{lemma:rogers-app}, \ref{lemma:wedge}, and
\ref{lemma:cone}.
\end{proof}



\subsection{A particular 4-circuit} %DCG 14.2, p158

This subsection bounds the score of a particular $4$-circuit on a
contravening planar hypermap.  The interior of the circuit
consists of two faces: a triangle and a pentagon.  The circuit and
its enclosed vertex are show in Figure \ref{fig:no4circuit} with
vertices marked $p_1,\ldots,p_5$.  The vertex $p_1$ is the
enclosed vertex, the triangle is $(p_1,p_2,p_5)$ and the pentagon
is $(p_1,\ldots,p_5)$.

\begin{figure}[htb]
  \centering
  \myincludegraphics{\ps/no4circuit.eps}
  \caption{A $4$-circuit}
  \label{fig:no4circuit}
\end{figure}

Suppose that $(v,\Lambda)$ is a centered packing whose associated hypermap
contains such triangular and pentagonal standard regions. Recall
that $(v,\Lambda)$ determines a set $U(v,\Lambda)$ of vertices in Euclidean
$3$-space of distance at most $2t_0$ from $v_0$, and that
each vertex $p_i$ can be realized geometrically as a point on the
unit sphere at $v_0$, obtained as the radial projection of
some $v_i\in U(v,\Lambda)$.

\begin{lemma}\dcg{Lemma~14.3}{158}  
One of the edges $\{v_1,v_3\}$, $\{v_1,v_4\}$ has
length less than $2\sqrt{2}$.  Both of the them have lengths less
than $3.02$. Also, $|v_1|\ge2.3$.
\end{lemma}

\begin{proof} This is Lemma~\ref{tarski:4circuit}.
\end{proof}


There are restrictive bounds on the dihedral angles of the
simplices $\{v_0,v_1,v_i,v_j\}$ along the edge $\{v_0,v_1\}$. The
quasi-regular tetrahedron has a dihedral angle of at most%
\footnote{\calc{984463800}} $1.875$.  The dihedral angles of the
simplices $\{v_0,v_1,v_2,v_3\}$, $\{v_0,v_1,v_5,v_4\}$
adjacent to it are at most%
\footnote{\calc{821707685}}  $1.63$. The dihedral angle of the
remaining simplex $\{v_0,v_1,v_3,v_4\}$ is at most%
\footnote{\calc{115383627}} $1.51$.   This leads to lower bounds
as well. The quasi-regular tetrahedron has a dihedral angle that
is at least $2\pi - 2(1.63)-1.51 > 1.51$.  The dihedral angles
adjacent to the quasi-regular tetrahedron is at least $2\pi-
1.63-1.51-1.875> 1.26$. The remaining dihedral angle is at least
$2\pi-1.875-2(1.63) > 1.14$.

A centered packing $(v,\Lambda)$ determines a set of vertices $U(v,\Lambda)$ that
are of distance at most $2t_0$ from $v$.  Three
consecutive vertices $p_1$, $p_2$, and $p_3$ of a standard region
are determined as the projections to the unit sphere of three
corners $v_1$, $v_2$, and $v_3$, respectively in $U(v,\Lambda)$. By
Lemma~\ref{lemma:1.32}, if the interior angle of the standard
region is less than $1.32$, then $|v_1-v_3|\le\sqrt{8}$.

\begin{lemma}\dcg{Lemma~14.4}{159} \label{lemma:11.16}
These two standard regions $F=\{R_1,R_2\}$ give
    $\tau_F(v,\Lambda) \ge 11.16\,\pt$.
\end{lemma}

\begin{proof}
Let $\dih$ denote the dihedral angle of a simplex along a given
edge. Let $S_{ij}$ be the simplex $\{v_0,v_1,v_i,v_j\}$, for
$(i,j)=(2,3),(3,4), (4,5),(2,5)$. We have $\sum_{(4)}\dih(S_{ij})
= 2\pi$. Suppose one of the edges $\{v_1,v_3\}$ or $\{v_1,v_4\}$ has
length $\ge2\sqrt2$. Say $\{v_1,v_3\}$.

We have\footnote{\calc{572068135}, \calc{723700608},
\calc{560470084}, and \calc{535502975}}
    $$
    \begin{array}{lll}
    \tau(S_{25}) &- 0.2529\dih(S_{25}) > -0.3442,\\
    \tau_0(S_{23}) &- 0.2529\dih(S_{23}) > -0.1787,\\
    \hat\tau(S_{45}) &- 0.2529\dih(S_{45}) > -0.2137,\\
    \tau_0(S_{34}) &- 0.2529\dih(S_{34}) > -0.1371.\\
    \end{array}
    $$
We have a penalty $\xiG$ for erasing, so that
    $$
    \begin{array}{lll}
        \tau(v,\Lambda) &\ge \sum_{(4)}\tau_x(S_{ij}) - 5\xiG\\
                &>2\pi(0.2529)-0.3442\\
                &\qquad -0.1787-0.2137-0.1371-5\xiG\\
                &>11.16\,\pt,
    \end{array}
    $$
where $\tau_x=\tau,\hat\tau,\tau_0$ as appropriate.

Now suppose $\{v_1,v_3\}$ and $\{v_1,v_4\}$ have length $\le2\sqrt2$.
If there is an upright diagonal that is not enclosed over either
flat quarter, the penalty is at most $3\xiG+2\xiV$. Otherwise, the
penalty is smaller: $4\xiG'+\xiV$. We have
    $$
    \begin{array}{lll}
    \tau(v,\Lambda)
    &\ge \sum_{(4)}\tau(S_{ij})-(3\xiG+2\xiV)\\
    &>2\pi(0.2529)-0.3442\\
    &\qquad -2(0.2137)-0.1371 -(3\xiG+2\xiV)\\
    &>11.16\,\pt.\\
    \end{array}
    $$
\end{proof}




\subsection{mixed bound} %DCG 10.14, p105 (moved -1.04 bound)
% Rewritten.

\begin{lemma}\dcg{Lemma~10.14}{105} \label{lemma:1.04}
%\proclaim{Proposition 4.1}
The score of a mixed quad cluster is less than $-1.04\,\pt$.
\end{lemma}

\begin{proof}
In a mixed quad cluster there is at least one enclosed vertex.
Any enclosed vertex in a quad cluster has length at least $2t_0$
by Lemma~\ref{lemma:enclosed}. In particular, the anchors of an
enclosed vertex are corners of the quad cluster. There are no flat
quarters.

We erase all of the enclosed vertices except for one, which
we can do with the estimate of Lemma~\ref{lemma:mixed-vor0}.
The enclosed vertex has zero, one, or two anchors.  The upright
quarters around that vertex are scored with the function appropriate
to its context.
The rest of the quad cluster is estimated by the function $\op{svR}(\cdot,t_0)$.

If the enclosed vertex has zero anchors, then the entire quad
cluster $(R,D)$ satisfies $$\sigma(v,R,\Lambda)\le \op{svR}(v,R,\Lambda,t_0).$$
The right-hand side of this equation is independent of the enclosed
vertex $v$.  In particular, we can move it until 
two consecutive  corners of the quad cluster are anchors of $\{v_0,v\}$
and one of the distances $|v-v_i|=2.51$ for one of those two corners.

A calculation shows%
\footnote{\calc{XX}. This is a new interval calculation that
needs to be verified: If $(R,D)$ is any quad cluster with both
diagonals greater than $\sqrt8$, and some distance $|v_i-v_{i+1}|>2.38$,
then $\op{svR}(v,R,\Lambda,t_0) < -1.04\,\pt$.} 
that if $\sigma_R(v,\Lambda) \ge -1.04\,\pt$, then 
$|v_i-v_{i+1}|\le 2.38$ for $i=1,2,3,4$.  We assume these
constraints.

We may use the deformation of Lemma~\ref{x-4.9.2}
at each vertex $v_i$ that is not an anchor of $\{v_0,v\}$ so either
it becomes an anchor (with $|v-v_i|=2t_0$), or it satisfies
  $$|v_i-v_{i+1}|=|v_i-v_{i-1}|=2;\quad |v_i|\in\{2,2t_0\}.$$ 

In the course of deformation (say of corner $v_1$),  
the diagonal $\{v_1,v_3\}$ may reach length $\sqrt8$.  
In this case, we stop
the deformation at that corner and continue with another
corner (say $v_2$, if it exists)
that is not an anchor until its deformation
is complete, and then return to the complete the deformation
at $v_1$.  We cannot have both diagonals $\{v_1,v_3\}$ and $\{v_2,v_4\}$
drop to $\sqrt8$, by Lemma~\ref{XX:tarski}, unless $\{v_0,v\}$ has four
anchors.  The result, after the deformations are complete, may
have a diagonal of length less than $\sqrt8$.  At this point,
that constraint is no longer needed.  Its only purpose was to
fulfill a hypothesis of Lemma~\ref{x-4.9.2}.

We claim that if corner $v_i$ is not an anchor of $\{v_0,v\}$,
then $2t_0 < |v-v_i| \le 2.95$.  In fact, if $|v-v_i| > 2.96$,
then the sum of the dihedral angles around $\{v_0,v\}$ satisfies%
\footnote{\calc{XX}.  These are new interval calculations 
  that need to be verified.  For an upright quarter in
  $[2.51,\sqrt8][2,2.51]^2[2,2.38][2,2.51]^2$, $\dih < 2.01$.
For a simplex in $[2.51,\sqrt8][2,2.51]^2 [2] [2.96,++][2,++]$,
$\dih < 1.13$.}
 $$
 2\pi = \sum_{i=1}^4 \dih_V(\{v_0,v\},\{v_i,v_{i+1}\}) < 2(2.01)+2(1.13) < 2\pi.
 $$

We write the upper bound on $\sigma_R(v,\Lambda)$ 
as a sum of contributions from the four simplices
$\{v_0,v,v_i,v_{i+1}\}$.    That contribution is the score of the
upright quarter, if the simplex is an upright quarter in the $Q$-system.
Otherwise, we use the upper bound $\op{svR}(\cdot,t_0)$.

If $\{v_0,v\}$ has just two anchors (say at adjacent corners $v_1,v_4$),
then the contribution from $\{v_0,v,v_1,v_4\}$ is at most $0$.  In
fact, if the simplex is an upright quarter in the $Q$-system, this
follows from Lemma~\ref{XX}.  If the simplex is scored by $\op{svR}(\cdot,t_0)$,
then (say) $|v-v_1|=2t_0$ and the result is a calculation.%
\footnote{\calc{XX}.  This is a new calculation.  
It needs to be verified.  If an upright
simplex satisfies $[2.51,\sqrt8][2,2.51]^2[2,2.38][2.51][2,2.51]$,
then $\op{svR}(v,S,t_0) < 0$.}  Moreover, the contributions from the other
three simplices give%
\footnote{\calc{XX}. These are new calculations.  They need
to be verified.  On
$[2.51,\sqrt8]\{2,2.51\}^2[2][2.51,2.96]^2$, $\op{svR}(\cdot,t_0) < -0.0475$; and
on $[2.51,\sqrt8]\{2,2.51\}[2,2.51][2][2.51,2.96][2,2.51]$,
   $\op{svR}(\cdot,t_0) < -0.0055$.
}
 $$\sum_{i=1}^3\op{svR}_{0,R,V}(v_0,\{v,v_i,v_{i+1}\}) <
   -0.0055 - 0.0475 - 0.0055 < -1.04\,\pt.$$

If $\{v_0,v\}$ has three anchors (say $v_1,v_2,v_4$), then
writing $\sigma'_i$ for the upper bound on the contribution
from $\{v_0,v,v_i,v_{i+1}\}$, we have%
\footnote{\calc{XX}.  These are new calculations that need to
be verified.  For an upright quad with $y_4\in[2,2.38]$, we
have $\sigma(Q) < \epsilon_1 -0.08(\dih(Q)-\pi/2)$.  I added
this small $\epsilon_1$ to make it more likely that it will follow
by a linear program for the other inequalities for $\sigma(Q)$.
Note that there are various cases, according to the context; we
haven't erased anything here.  We have the similar
  $\op{svR}_0 < \epsilon_1 -0.08(\dih(Q)-\pi/2)$,
when $[2.51,\sqrt8][2,2.51]^2[2,2.38][2,2.51][2.51]$.  Then we
have $\op{svR}_0 < \epsilon_2  -0.08(\dih(Q)-\pi/2)$,
when $[2.51,\sqrt8][2,2.51]^2[2][2,2.51][2.51,2.96]$.
}
  $$
  \sum_{i=1}^4\sigma'_i < 
  \sum (\epsilon_i -0.08 (\dih_V(\{v_0,v\},\{v_i,v_{i+1}\})-\pi/2))
  = \sum\epsilon_i = -1.04\,\pt.
  $$
where $\epsilon_2=\epsilon_3 = -0.54\,\pt$ and $\epsilon_1=\epsilon_4 =
0.02\,\pt$.
\end{proof}

%% OLD PROOF. 
%
%We generally truncate the $V$-cell at $\sqr2$ as in the proof of
%Theorem~\ref{lemma:quad0}.  By that lemma, it breaks the $V$-cell
%into pieces whose score is nonpositive. Thus, if we identify
%certain pieces that score less than $-1.04\,\pt$, the result
%follows. Nevertheless, a few simplices will be left untruncated in
%the following argument. We will leave a simplex untruncated only
%if we are certain that this is justified.
%%% Avoid mention of orien-tation.
%% Each of its faces has positive orien-tation
%% and that the simplices sharing a face $F$ with $S$ either lie in
%% the $Q$-system or have positive orien-tation along $F$.  
%If so, we may use\footnote{\calc{185703487},
%\calc{69785808}, and \calc{104677697}} the function $\op{svan}$ on $S$
%rather than truncation $\op{sv}_0$.
%
%In this proof, by enclosed vertex, we mean one of height at most
%$2\sqrt2$. Let $v$ be an enclosed vertex with the fewest anchors.
%If there are no anchors, the right circular cone $C(h,\eta_0(h))$
%(aligned along $\{v_0,v\}$; see Definition~\ref{def:cone}) belongs
%to $\op{VC}(v_0)$, where $\eta_0(h)=\eta(2h,2,2t_0)$ as in
%Definition~\ref{def:eta0} and $|v|=2h$. In fact, if such a point
%lies in $\op{VC}(u)$, with $u \ne v$, then $u$ must be a corner of
%the quad cluster or an enclosed vertex of height at least $2t_0$.
%In either case, the right circular cone belongs to $\op{VC} (v_0)$.
%By Formula~\ref{lemma:sovoFR}, the score of this cone is
%$2\pi(1-h/\eta_0(h))\phi(h,\eta_0(h))$. An optimization in one
%variable gives an upper bound of $-4.52\,\pt$, for $t_0\le h\le
%\sqr2$.   This gives the bound of $-1.04\,\pt$ in this case.
%
%If there is one anchor,  we cut the cone in half along the plane
%through $\{v_0,v\}$ perpendicular to the plane containing the anchor
%and $\{v_0,v\}$. The half of the cone on the far side of the anchor
%lies under the face at $v$ of the $V$-cell.  We get a bound of
%$-4.52\,\pt/2 < -1.04\,\pt$.
%
%In the remaining cases, each enclosed vertex has at least two
%anchors.  Each anchor is a corner of the quad cluster.  Fix an
%enclosed vertex $v$. Suppose that $v_1$, a corner, is an anchor of
%$v$. Assume that the face $\{v_0,v,v_1\}$ bounds at most one upright
%quarter. We sweep around the edge $\{v_0,v_1\}$, away from the
%upright quarter if there is one,  until we come to another
%enclosed vertex $v'$ such that $\{v_0,v_1,v'\}$ has circumradius
%less than $\sqr2$ or such that $v_1$ is an anchor of $\{v_0,v'\}$.
%If such a vertex $v'$ does not exist, we sweep all the way to
%$v_2$ a corner of the quad cluster adjacent to $v_1$.
%
%Section~\ref{sec:K} defines a function $K$ that we use in
%this proof.
%
%If $v'$ exists, then various
%calculations\footnote{\calc{104677697}, \calc{69785808},
%\calc{586706757}, and \calc{87690094}} give the bound
%$-1.04\,\pt$, depending on the size of the circumradius of
%$\{v_0,v,v'\}$. This allows us to assume that we do not encounter
%such an enclosed vertex $v'$ whenever we sweep away, as above,
%from the face formed by an anchor.
%
%Now consider the simplex $S=\{v_0,v_1,v_2,v\}$, where $v_1$ is an
%anchor of $\{v_0,v\}$.  We assume that it is not an upright quarter.
%There are three alternatives. The first is that $S$ decreases the
%score of the quarter by at least $v_0.52\,\pt$.
%Calculations\footnote{\calc{185703487} and \calc{441195992}} show
%that this occurs if the circumradius of the face $\{v_0,v,v_2\}$ is
%less than $\sqr2$, or if the circumradius of the face is greater
%than $\sqr2$, provided that the length of $\{v,v_1\}$ is at most
%$2.2$. The second alternative\footnote{\calc{848147403},
%\calc{969320489}, and \calc{975496332}.} is that the face
%$\{v_0,v,v_1\}$ of $S$ is shared with a quarter $Q$ and that $S$ and
%$Q$ taken together bring the score down by $0.52\,\pt$. In fact,
%if there are two such simplices $S$ and $S'$ along $Q$, then the
%three simplices $Q$, $S$, and $S'$ pull the
%score\footnote{\calc{766771911}} below $-1.04\,\pt$. The third
%alternative is that there is a simplex $S'=\{v_0,v,v,v_3\}$ sharing
%the face $\{v_0,v,v_1\}$, which, like $S$, scores less than
%$-0.31\,\pt$.  In each case, $S$ and the adjacent simplex through
%$\{v_0,v,v_1\}$ score less than $-0.52\,\pt$. Since $v$ has at least
%two anchors, the quad cluster scores less than $2(-0.52)\,\pt
%=-1.04\,\pt$.
%%
%
%



\subsection{A particular 5-circuit} %DCG 14.3, p160

\begin{lemma}\dcg{Lemma~14.5}{160}\label{lemma:6079}  
Assume that $R$ is a pentagonal standard region
    with an enclosed vertex $v$ of height at most $2t_0$.
    %(See Figure~\ref{fig:pent-tri1}.)
    Assume further that
    \begin{itemize}
        \item $|v_i|\le 2.168$ for each of the five corners.
        \item Each interior angle of the pentagon is at most
        $2.89$.
        \item If $v_1$, $v_2$, $v_3$ are consecutive corners over
        the pentagonal region, then $$|v_1-v_2|+|v_2-v_3|<4.804.$$
        \item $\sum_5 |v_i-v_{i+1}|\le 11.407.$
    \end{itemize}
    Then $\sigma_R(v,\Lambda)< -0.2345$ or $\tau_R(v,\Lambda) > 0.6079.$
\end{lemma}

\begin{proof}
Since $-0.4339$ is less than this the lower bound, a $3$-crowded
upright diagonal does not occur. Similarly, since $-0.25$ is less
than the lower bound, a $4$-crowded upright diagonal does not
occur (Lemma~\ref{lemma:4-crowded} and Lemma~\ref{x-3.8}).

Suppose that there is a loop in context $(n,k)=(4,2)$. Again by
Lemma~\ref{lemma:loop} (with $n(R)=7$),
$$\sigma_R(v,\Lambda)  < -0.2345.$$
%The constants come from
%Table~\ref{x-5.11} and  Theorem~\ref{thm:the-main-theorem}.

%If we branch and bound on the triangular faces, this LP-derived
%inequality can be improved to
%    $$\tau[F] < 0.6079.$$

%If there is a loop other than $(4,2)$ and $(4,1)$, the linear
%program becomes infeasible:
%    $$\tau[F] < 0.644 < t_7 + \dloop(n,k) < \tau[F].$$
We conclude that all loops have context $(n,k)=(4,1)$.


{\bf Case 1.}  {\it The vertex $v=v_{12}$ has distance at least
$2t_0$ from the five corners of $U(v,\Lambda)$ over the pentagon.}

%The interval calculations relevant in Case 1 appear in
%~\ref{A.3.8}.

The penalty to switch the pentagon to a pure $\op{svR}_0$ score is at
most $5\xiG$ (see Section~\ref{sec:prep-cluster}).  There cannot
be two flat quarters because then Lemma~\ref{tarski:E:part4:5} gives
$$|v_{12}|>2t_0.$$


{\bf (Case 1-a)} Suppose there is one flat quarter,
$|v_1-v_4|\le2\sqrt2$. There is a lower bound of 1.2 on the
dihedral angles of the simplices $\{v_0,v_{12},v_i,v_{i+1}\}$.  This
is obtained as follows.  The proof relies on the convexity of the
quadrilateral region.  We leave it to the reader to verify that
the following pivots can be made to preserve convexity.  Disregard
all vertices except $v_1,v_2,v_3,v_4,v_{12}$.  We give the
argument that $\dih(v_0,v_{12},v_1,v_4)>1.2$.  The others are
similar. Disregard the length $|v_1-v_4|$.  We show that
    $$
    \begin{array}{lll}
        sd &:=\dih(v_0,v_{12},v_1,v_2)+\dih(v_0,v_{12},v_2,v_3)\\
           &+\dih(v_0,v_{12},v_3,v_4) < 2\pi-1.2.
    \end{array}
    $$
Lift $v_{12}$ so $|v_{12}|=2t_0$. Maximize $sd$ by taking
$|v_1-v_2|=|v_2-v_3|=|v_3-v_4|=2t_0$.  Fixing $v_3$ and $v_4$,
pivot $v_1$ around $\{v_0,v_{12}\}$ toward $v_4$, dragging $v_2$
toward $v_{12}$ until $|v_2-v_{12}|=2t_0$.  Similarly, we obtain
$|v_3-v_{12}|=2t_0$. We now have $sd\le 3(1.63)< 2\pi-1.2$, by a
calculation.\footnote{\calc{821707685}}

Return to the original figure and move $v_{12}$ without increasing
$|v_{12}|$ until each simplex $\{v_0,v_{12},v_i,v_{i+1}\}$ has an edge
$(v_{12},v_j)$ of length $2t_0$. Interval
calculations\footnote{\calc{467530297} and \calc{135427691}} show
that the four simplices around $v_{12}$ squander
    $$2\pi(0.2529)-3(0.1376)-0.12 > \squander + 5\xiG.$$

{\bf (Case 1-b)} Assume there are no flat quarters. By hypothesis,
the perimeter satisfies $$\sum|v_i-v_{i+1}|\le 11.407.$$ We have
$\arc(2,2,x)'' = 2x/(16-x^2)^{3/2} >0$. The arclength of the
perimeter is therefore at most
$$2\arc(2,2,2t_0) + 2\arc(2,2,2) + \arc(2,2,2.387) <  2\pi.$$
There is a well-defined interior of the spherical pentagon, a
component of area $<2\pi$.  If we deform by decreasing the
perimeter, the component of area $<2\pi$ does not get swapped with
the other component.

Disregard all vertices but $v_1,\ldots,v_5,v_{12}$.  If a vertex
$v_i$ satisfies  $|v_i-v_{12}|>2t_0$, deform $v_i$ as in
Section~\ref{x-4.9} until $|v_{i-1}-v_{i}|=|v_i-v_{i+1}|=2$, or
$|v_i-v_{12}|=2t_0$. If at any time, four of the edges realize the
bound $|v_i-v_{i+1}|=2$, we have reached an impossible situation,
because it leads to the contradiction\footnote{\calc{115383627}
and \calc{603145528}}
    $$2\pi = \sum^{(5)}\dih < 1.51 + 4 (1.16) < 2\pi.$$
(This inequality relies on the observation, which we leave to the
reader, that in any such assembly, pivots can by applied to bring
$|v_{12}-v_i|=2t_0$ for at least one edge of each of the five
simplices.)



The vertex $v_{12}$ may be moved without increasing $|v_{12}|$ so
that eventually by these deformations (and reindexing if
necessary) we have $|v_{12}-v_i|=2t_0$, $i=1,3,4$. (If we have
$i=1,2,3$, the two dihedral angles along $\{v_0,v_2\}$
satisfy\footnote{\calc{115383627}} $<2(1.51)<\pi$, so the
deformations can continue.)



There are two cases. In both cases $|v_i-v_{12}|=2t_0$, for
$i=1,3,4$.
$$
\begin{array}{lll}
(i)\quad &|v_{12}-v_2|=|v_{12}-v_5|=2t_0,\\
(ii)\quad &|v_{12}-v_2|=2t_0,\quad |v_4-v_5|=|v_5-v_1|=2,\\
\end{array}
$$
Case (i) follows from interval
calculations\footnote{\calc{312132053}}
$$
\sum\tau_0 \ge 2\pi(0.2529) - 5 (0.1453) > 0.644+7\xiG.
$$
In case (ii), we have again
    $$2\pi(0.2529)-5 (0.1453).$$
In this interval calculation we have assumed that
$|v_{12}-v_5|<3.488$. Otherwise, setting $S=(v_{12},v_4,v_5,v_1)$, Lemma~\ref{tarski:3488}
shows the simplex does not exist.
($|v_4-v_1|\ge2\sqrt2$ because
there are no flat quarters.)
This completes Case 1.

\medskip

{\bf Case 2.} {\it The vertex $v_{12}$ has distance at most $2t_0$
from the vertex $v_1$ and distance at least $2t_0$ from the
others.}

Let $\{v_0,v_{13}\}$ be the upright diagonal of a loop $(4,1)$.  The
vertices of the loop are not $\{v_2,v_3,v_4,v_5\}$ with $v_{12}$
enclosed over $\{v_0,v_2,v_5,v_{13}\}$ by
Lemma~\ref{lemma:anc-simplex-not-enc}. The vertices of the loop
are not $\{v_2,v_3,v_4,v_5\}$ with $v_{12}$ enclosed over
$\{v_0,v_1,v_2,v_5\}$ because this and Lemmas~\ref{tarski:E:part4:6}
and \ref{tarski:E:part4:7} would lead to a contradiction
$y_{12}>2t_0$. 
We get a contradiction for the same reasons
 unless $\{v_1,v_{12}\}$ is an edge of some
upright quarter of every loop of type $(4,1)$.

We consider two cases.  (2-a) There is a flat quarter along an
edge other than $\{v_1,v_{12}\}$.  That is, the central vertex is
$v_2$, $v_3$, $v_4$, or $v_5$.  (Recall that the {\it central
vertex} of a flat quarter is the vertex other than $v_0$ that
is not an endpoint of the diagonal.) (2-b) Every flat quarter has
central vertex $v_1$.

{\bf Case 2-a.}  We erase all upright quarters including those in
loops, taking penalties as required. There cannot be two flat
quarters because then Lemmas~\ref{tarski:E:part4:8} and
\ref{tarski:E:part4:9} would imply $|v_{12}|>2t_0$.

The penalty is at most $7\xiG$.  We show that the region (with
upright quarters erased) squanders $>7\xiG+0.644$.  We assume that
the central vertex is $v_2$ (case 2-a-i) or $v_3$ (case 2-a-ii).
In case 2-a-i, we have three types of simplices around $v_{12}$,
characterized by the bounds on their edge lengths.  Let
$\{v_0,v_{12},v_1,v_5\}$ have type A, $\{v_0,v_{12},v_5,v_4\}$ and
$\{v_0,v_{12},v_4,v_3\}$ have type B, and let $\{v_0,v_{12},v_3,v_1\}$
have type C.  In case 2-a-ii there are also three types.  Let
$\{v_0,v_{12},v_1,v_2\}$ and $\{v_0,v_{12},v_1,v_5\}$ have type A,
$\{v_0,v_{12},v_5,v_4\}$ type B, and $\{v_0,v_{12},v_2,v_4\}$ type D.
(There is no relation here between these types and the types of
simplices $A$, $B$, $C$ defined in \Chap~\ref{sec:fine}.) Upper
bounds on the dihedral angles along the edge $\{v_0,v_{12}\}$ are
given as calculations\footnote{\calc{821707685}, \calc{115383627},
\calc{576221766}, and \calc{122081309}}. These upper bounds come
as a result of a pivot argument similar to that establishing the
bound 1.2 in Case 1-a.

These upper bounds imply the following lower bounds.  In case
2-a-i,
$$
\begin{array}{lll}
\dih &> 1.33 \quad(A),\\
\dih &> 1.21 \quad(B),\\
\dih &> 1.63 \quad(C),\\
\end{array}
$$
and in case 2-a-ii,
$$
\begin{array}{lll}
\dih &> 1.37 \quad(A),\\
\dih &> 1.25 \quad(B),\\
\dih &> 1.51 \quad(v,\Lambda),\\
\end{array}
$$
In every case the dihedral angle is at least $1.21$. In case
2-a-i, the inequalities give a lower bound on what is squandered
by the four simplices around $\{v_0,v_{12}\}$. Again, we move $v_{12}$
without decreasing the score until each simplex
$\{v_0,v_{12},v_i,v_{i+1}\}$ has an edge satisfying
$|v_{12}-v_j|\le2t_0$. Interval
calculations\footnote{\calc{644534985}, \calc{467530297}, and
\calc{603910880}} give
    $$
    \begin{array}{lll}
    \sum_{(4)}\tau_0 &> 2\pi (0.2529) - 0.2391-2(0.1376)-0.266\\
        &>0.808.
    \end{array}
    $$
In case 2-a-ii, we have\footnote{\calc{135427691}}
    $$
    \begin{array}{lll}
    \sum_{(4)}\tau_0 &> 2\pi (0.2529) - 2(0.2391)-0.1376-0.12\\
        &>0.853.
    \end{array}
    $$
So we squander more than $7\xiG+0.644$, as claimed.

{\bf Case 2-b.}  We now assume that there are no flat quarters
with central vertex $v_2,\ldots,v_5$. We claim
 that $v_{12}$ is not enclosed over $\{v_0,v_1,v_2,v_3\}$ or
$\{v_0,v_1,v_5,v_4\}$. In fact, if $v_{12}$ is enclosed over
$\{v_0,v_1,v_2,v_3\}$, then we reach the
contradiction\footnote{\calc{821707685} and \calc{115383627}}
    $$
    \begin{array}{lll}
    \pi&<\dih(v_0,v_{12},v_1,v_2)+\dih(v_0,v_{12},v_2,v_3)\\
        &< 1.63+1.51 < \pi.
    \end{array}
    $$

We claim
 that $v_{12}$ is not enclosed over $\{v_0,v_5,v_1,v_2\}$.
Let $S_1=\{v_0,v_{12},v_1,v_2\}$, and $S_2=\{v_0,v_{12},v_1,v_5\}$.  We
have by hypothesis,
$$y_4(S_1)+y_4(S_2) = |v_1-v_2|+|v_1-v_5|< 4.804.$$
An interval calculation\footnote{\calc{69064028}} gives
    $$
    \begin{array}{lll}
    \sum_{(2)}\dih(S_i) &\le \sum_{(2)}
    \left(\dih(S_i)+0.5(0.4804/2-y_4(S_i))\right)\\
    &<\pi.
    \end{array}
    $$
So $v_{12}$ is not enclosed over $\{v_0,v_1,v_2,v_5\}$.

Erase all upright quarters, taking penalties as required.  Replace
all flat quarters with $\op{sv}_0$-scoring taking penalties as
required. (Any flat quarter has $v_1$ as its central vertex.) We
move $v_{12}$ keeping $|v_{12}|$ fixed and not decreasing
$|v_{12}-v_1|$.  The only effect this has on the score comes
through the quoins along $\{v_0,v_1,v_{12}\}$. Stretching
$|v_{12}-v_1|$ shrinks the quoins and increases the score. (The
sign of the derivative of the quoin with respect to the top edge
is computed in the proof of Lemma~\ref{x-4.9.1}.)

If we stretch $|v_{12}-v_1|$ to length $2t_0$, we are done by case
1 and case 2-a. (If deformations produce a flat quarter, use case
2-a, otherwise use case 1.) By the claims, we can eventually
arrange (reindexing if necessary) so that
$$
\begin{array}{lll}
(i)&\quad |v_{12}-v_3|=|v_{12}-v_4|=2t_0,\quad\text{or}\\
(ii)&\quad |v_{12}-v_3|=|v_{12}-v_5|=2t_0.
\end{array}
$$
We combine this with the deformations of Section~\ref{x-4.9} so
that in case (i) we may also assume that if $|v_5-v_{12}|>2t_0$,
then $|v_4-v_5|=|v_5-v_1|=2$ and that if $|v_2-v_{12}|>2t_0$, then
$|v_1-v_2|=|v_2-v_3|=2$. In case (ii) we may also assume that if
$|v_4-v_{12}|>2t_0$, then $|v_3-v_4|=|v_4-v_5|=2$ and that if
$|v_2-v_{12}|>2t_0$, then $|v_1-v_2|=|v_2-v_3|=2$.

Break the pentagon into subregions by cutting along the edges
$(v_{12},v_i)$ that satisfy $|v_{12}-v_i|\le2t_0$. So for example
in case (i), we cut along $(v_{12},v_3)$, $(v_{12},v_4)$,
$(v_{12},v_1)$, and possibly along $(v_{12},v_2)$ and
$(v_{12},v_5)$.  This breaks the pentagon into triangular and
quadrilateral regions.

In case (ii), if $|v_4-v_{12}|>2t_0$, then the argument used in
Case 1 to show that $|v_4-v_{12}|<3.488$ applies here as well.
%% Keep comment: DCG p164.  I commented out to avoid mention of Delta.
%% But it is used eventually in the argument.
%% Deelta doesn't explicitly get mentioned in the two footnoted calcs,
%% so what I'm commented out isn't that essential.
% In
%case (i) or (ii), if $|v_{12}-v_2|>2t_0$, then for similar
%reasons, we may assume
%    $$\Deelta(|v_{12}-v_2|^2,4,4,8,(2t_0)^2,|v_{12}-v_1|^2)\ge0.$$
We use
calculations\footnote{\calc{312132053} and \calc{644534985}} to
conclude that
    $$\sum\tau_0 \ge 2\pi (0.2529) -3 (0.1453) -2 (0.2391) > 0.6749.$$
If the penalty is less than $0.067=0.6749-0.6079$, we are done.

We have ruled out the existence of all loops except $(4,1)$. Note
that a flat quarter with central vertex $v_1$ gives penalty at
most $0.02$ by Lemma~\ref{x-3.11.3}.
  If there is at most one
such a flat quarter and at most one loop, we are done:
$$3\xiG + 0.02 < 0.067.$$
Assume there are two loops of context $(n,k)=(4,1)$.  They both
lie along the edge $\{v_1,v_{12}\}$, which precludes any unmasked
flat quarters. If one of the upright diagonals has height
$\ge2.696$, then the penalty is at most $3\xiG+3\xiV< 0.067$.
Assume both heights are at most $2.696$. The total interior angle
of the exceptional face at $v_1$ is at least four times the
dihedral angle of one of the flat quarters along $\{v_0,v_1\}$, or
$4(0.74)$ by an interval calculation\footnote{\calc{751442360}}. This is
contrary to the hypothesis of an interior angle $<2.89$.   This
completes Case 2. This shows that heptagons with pentagonal hulls
do not occur.
\end{proof}



\subsection{truncated corner cell calculations}

In this section, we calculate several different bounds
on the function $\op{sovo}$ on truncated corner cells for
various choices of the parameters $t$, $\mu$, and $\lambda$.

The dependence of $\op{sovo}$ on the azimuth angle
is linear with coefficient $(1-\cos\psi)\phi(t,t,\lambda)$.
For fixed, $\psi$, $t$, and $\lambda$, this coefficient
has fixed sign.  Also, if $\op{azim}<\pi$, the azimuth
angle depends monotonically on $|w_1-w_2|$.  We thus get
an bound on $\op{sovo}$ 
when $|w_1-w_2|$ is chosen to be as small as possible.

\medskip

\begin{lemma}\tlabel{lemma:tcc-est} 
Let $TCC=TCC(v_0,v_1,w_1,w_2,t,\mu)$ 
be a truncated corner cell with
parameters $t=1.255$, $\mu=1.6$ and azimuth angle at least $\pi$. 
Assume that $|v_1-w_i|\ge 3.07$ for $i=1,2$.
Let $y=|v_1-v_0|$.  Assume that $2\le y\le 2t$.
Assume that $2\le |v_1-w_i|\le 2t$, for $i=1,2$.
Then $\op{sovo}(v_0,TCC,\lambda_{sq}) > 0.297$.
\end{lemma}

\begin{proof}
Let $z_i = |v_1-w_i|$, for $i=1,2$.
Our estimate is based on Lemma~\ref{lemma:tcc}.  From that lemma,
we have
  $$
  \begin{array}{lll}
  \op{sovo}(v_0,TCC,\lambda) &= T_0 + T_1 + T_2 + T_3 \\
  T_0 &= \op{sovo}(v_0,Q_1,\lambda) +  \op{sovo}(v_0,Q_2,\lambda)\\
  T_1  &=  - s_1\phi(t,t,\lambda) \\
  T_2  &= - s_2\phi(t,t,\lambda) \\
  T_3 &= 
  \op{azim}(v_0,v_1,w_1,w_2) \left((1-\cos\psi)\phi(t,t,\lambda)+
    A(y/2,t,\lambda)\right) \\
  \end{array}
  $$
The condition $z_i\ge3.07$
forces the circumradius of $\{v_0,v_1,w_i\}$ to be greater
than $t$.  This implies that $Q_i=\emptyset$.  The corresponding
term $T_0$ is zero.

The coefficient of $\op{azim}$ in the term $T_3$
is an explicit function of a single variable $y\in[2,2t]$.
It is minimized and takes a positive value
when $y=2t$.
In particular,
the coefficient of $\op{azim}$ is
positive. We obtain a lower bound on the $T_3$ by taking
$\op{azim}(v_0,v_1,w_1,w_2)\ge \pi$ and $y=2t$.
This gives $T_3 > 0.32$.

For the given choices of $t,\lambda$, we have
$0 < \phi(t,t,\lambda) < 0.6671$.  The term $s_i$ is
maximized when $y_3=2t_0$, $y_5=3.07$,
so that $s_i < 0.017$.  (This was checked with interval arithmetic in
Mathematica.) Thus,
    $$\op{sovo}(v_0,TCC,\lambda)  = T_0 + T_1 + T_2 + T_3 >
   0 - 2 (0.0017)(0.6671) +0.32 > 0.297.$$
\end{proof}



\begin{lemma}\label{lemma:CC815}  
Let $CC=CC(v_0,v_1,w_1,w_2,t,\mu)$
be an untruncated corner
cell with parameters $\mu=1.815$, $t=1.255$,
$y=|v_1-v_0|$, with $2\le y\le 2.2$, 
and azimuth angle at least $\pi$.  Then
 $$\op{sovo}(v_0,CC,\lambda_{sq}) > 0.8862.$$ %WW check: was \squander +\maxpi.$$
\end{lemma}

% WW Recheck proof

\begin{proof}  According to Lemma~\ref{lemma:sovo:CC},
the function $\op{sovo}(CC)$ has the form
$\op{azim}(v_0,v_1,w_1,w_2) f(y)$, for
some explicit rational function $f$ of the
variable $y$. 
The minimum, which occurs at $y=2.2$, is positive.
A lower bound is then $\pi f(2.2)$. 
We evaluate this constant to get the result.
\end{proof}

