%% historical shell

% Author Thomas C. Hales
% Copyright 2007




The series of papers in this volume gives a proof of the Kepler
conjecture, which asserts that the density of a packing of
congruent spheres in three dimensions is never greater than
$\pi/\sqrt{18}\approx 0.74048\ldots$. This is the oldest problem
in discrete geometry and is an important part of Hilbert's 18th
problem. An example of a packing achieving this density is the
face-centered cubic packing.


\label{sec:intro-review}

\section{The face-centered cubic packing}

A packing of spheres is an arrangement of nonoverlapping spheres
of radius 1 in Euclidean space. Each sphere is determined by its
center, so equivalently it is a collection of points in Euclidean
space separated by distances of at least 2.

The uniqueness established by this work is as strong as can be
hoped for. It shows that certain local structures (decomposition
stars) attached to the face-centered cubic (fcc) and
hexagonal-close packings (hcp) are the only structures that
maximize a local density function.


\section{Early History, Hariot, and Kepler}
\label{sec:early}



The modern mathematical study of spheres and their close packings
can be traced to T. Hariot.  Hariot's work -- unpublished,
unedited, and largely undated -- shows a preoccupation with sphere
packings. He seems to have first taken an interest in packings at
the prompting of Sir Walter Raleigh.  At the time, Hariot was
Raleigh's mathematical assistant,  and Raleigh gave him the
problem of determining formulas for the number of cannonballs in
regularly stacked piles. Kepler became involved in sphere packings
through his correspondence with Hariot in the early years of the
17th century.

Kepler's essay describes the face-centered cubic packing and
asserts that ``the packing will be the tightest possible, so that
in no other arrangement  could more pellets be stuffed into the
same container.''  This assertion has come to be known as the
Kepler conjecture.   The purpose of this collection of papers is
to give a proof of this conjecture.

\section{History}
\label{sec:history}

The next episode in the history of this problem is a debate
between Isaac Newton and David Gregory.  Newton and Gregory
discussed the question of how many spheres of equal radius can be
arranged to touch a given sphere.  This is the three-dimensional
analogue of the simple fact that in two dimensions six pennies,
but no more, can be arranged to touch a central penny.  This is
the kissing-number problem in $n$-dimensions. In three dimensions,
Newton said that the maximum was twelve spheres, but Gregory
claimed that thirteen might be possible.



\chapter{Overview of the proof}


The following two sections (added Jan 2003)  describe some of the
motivation behind the partitions of space that have been used in
the proof of the Kepler conjecture.  This discussion includes
various ideas that were tried, found wanting, and discarded.
However, this discussion provides motivation for some of the
choices that appear in the proof of the Kepler conjecture.

Let $S$ be a regular tetrahedron of side length $2$.  If we place
a unit ball at each of the four vertices, the fraction of the
tetrahedral solid occupied by the part of the four balls within
the tetrahedron is $\dtet\approx 0.7797$. Let $O$ be a regular
octahedron of side length $2$.  If we place a unit ball at each of
the four vertices, the fraction of the octahedral solid occupied
by the four balls is $\doct\approx 0.72$. The face-centered cubic
packing can be obtained by packing eight regular tetrahedra and
six regular octahedra around each vertex. The density
$\pi/\sqrt{18}$ of this packing is a weighted average of $\dtet$
and $\doct$:
    $$\frac\pi{\sqrt{18}} = \frac13\dtet + \frac23\doct.$$
