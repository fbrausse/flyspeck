\begin{abstract}
This article describes a formal proof of the Kepler conjecture in a combination of the HOL Light and Isabelle proof
assistants.
This paper constitutes the official published account of the now completed Flyspeck project.
\end{abstract}


\section{Introduction}


In 1611, Kepler published the booklet {\it Six-Cornered Snowflake}, which contains a statement that has
become  known as the Kepler conjecture.  The assertion is that no packing of congruent balls in Euclidean three-space has
density greater than that of the face-centered cubic packing.  It is the oldest problem in discrete geometry.
In 1900, Hilbert included the Kepler conjecture in his influential list of mathematical problems.  In his book \cite{XX},
L.\ Fejes T\'oth 
gave a coherent proof strategy and later suggested that computers might be used to study of the problem.
The truth of the Kepler conjecture was established 
by Ferguson and Hales in 1998, but their proof was not published in full until 2006~\cite{DCG}.

The delay in publication was caused by the difficulties of the referees in  verifying a complex computer proof.
In the end, the proof was published without a complete certification of correctness from the referees.  Many details about the
review process for this theorem appear in [Lagarias].  Lagarias eventually became the mathematician the most actively
involved in the checking of the original proof.  He  writes, ``The nature of this proof $\ldots$ makes it hard for
humans to check every step reliably. $\ldots$  [D]etailed checking of many specific assertions found them
to be essentially correct in every case.  The result of the reviewing process produced in these reviewers a 
strong degree of conviction of the essential correctness of this proof approach, and that the reduction method
led to nonlinear programming problems of tractable size.''   We note the lack of complete conviction.

At the Joint Math meetings in Baltimore in
2003(?), Hales announced
a project to give a formal proof of the Kepler conjecture,
and later he published a project description \cite{MAPS XX}.
%  A published announcement of the project later appeared in
%in [MAPS]. 
% {Introduction to the Flyspeck Project, MAPS 05021. http://drops.dagstuhl.de/opus/volltexte/2006/432}.
%That announcement states ``In truth, my motivations for the project are far more complex than a simple hope of 
%removing residual doubt from the minds of few referees. 
%Indeed, I see formal methods as fundamental to the long-term growth of mathematics.''
The project came to be called Flyspeck, an expansion of the the acronym FPK, for the Formal Proof of the Kepler conjecture.
This paper constitutes the official published account of the now completed Flyspeck project.


The first  definite contribution to the project was the formal verification of a major piece of
computer code that was used in the proof, in work at TU Munich by Bauer under the direction of T. Nipkow. (See section X.)
Major work on the project started when NSF funded
the project in 200(?).   An international conference on the Flyspeck project and formal proofs sponsored by the NSF and the
Hanoi Math Institute in 200(?) transformed the project into a large international collaboration.
The book ``Dense Sphere Packings'' describes the mathematical details of the proof that was formalized.  This article
focuses on the formalization itself.



\section{The statement}

As mentioned in the introduction, 
the Kepler conjecture asserts that no packing of congruent balls in Euclidean three-space can have density exceeding
that of the face-centered cubic packing.  That density is $\pi/\sqrt{18}$, or approximately $0.74$.    The face-centered
cubic packing is not the only packing that realizes the bound.   The hexagonal-close packing and various packings
combining layers from the hexagaon-close packing and the face-centered cubic packing all achieve this same bound.
The density is insensitive to trivial modifications, such as removing a single ball from the packing.   Such modifications
will also achieve the maximum density.  The theorem that has been formalized does not make any
uniqueness claims.

The density of a packing is defined as a limit of the density obtained within finite containers, as the size of the container
tends to infinity.  To make the statement in the formal proof as simple as possible, we formalize a statement about the
density of a packing inside a finite spherical container.  This statement contains an error term.  The ratio
of the error term to the volume of the container  tends to zero as the volume of
the container tends to infinity.  Thus in the limit, we obtain the Kepler conjecture in its traditional form.


We will display various terms that are represented in HOL syntax.  For the convenience of the reader, appendix X lists
some of the syntactic conventions of HOL Light.  In particular, the universal quantifier $\forall$ is written as (!), the
existential quantifier is written (?),  the embedding of natural numbers into the real numbers is denoted (\&).



As a ratio of volumes, the density of a packing is scale invariant.   There is no loss of generality in assuming that
the balls in the packing are normalized
to have unit length.
We identify a packing of balls in $\ring{R}^3$ with the set $V$ of centers of the balls,
so that the distance between distinct elements of $V$ is at least $2$, the diameter of a ball.

More formally,

\begin{obeylines}

\begin{verbatim}

`(packing V <=> 
  (!u v. u IN V /\ v IN V /\ dist(u,v) < &2 ==> u = v))`

\end{verbatim}
\end{obeylines}
This definition states that $V$ is a packing if and only if for every $\u, \v \in V$, if
the distance from $\u$ to $\v$ is less than $2$, then $\u=\v$.

We define the constant {\tt the\_kepler\_conjecture} to be the term

\begin{obeylines}

\begin{verbatim}
`the_kepler_conjecture <=>
  (!V. packing V
    ==> (?c. !r. &1 <= r
        ==> &(CARD(V INTER ball(vec 0,r))) <=
            pi * r pow 3 / sqrt(&18) + c * r pow 2))`
\end{verbatim}
\end{obeylines}

In words, we define the Kepler conjecture to be the following claim:
for every packing $V$, there exists a real number $c$ such that for every real number $r\ge 1$, the number
of elements $V$ contained in an open spherical container of radius $r$ centered a the origin is at most
\[
  \frac{\pi\, r^3}{\sqrt{18}} + c\, r^2.
\]
An analysis of our proof of the theorem shows that there exists a small computable constant $c$ that works uniformly
for all packings $V$,
but we only formalize the weaker statement that allows $c$ to depend on $V$.
The restriction $r\ge 1$, which bounds $r$ away from $0$,
is needed because there can be arbitrarily small containers whose intersection with $V$ is nonempty.

 The proof of the Kepler conjecture relies on a combination of traditional mathematical argument
and three separate bodies of computer calculation.   The results of the computer calculations have been expressed
in precise mathematical terms and specified formally in HOL Light.  The computer calculations are as follows.
\begin{enumerate}
\item The proof of the Kepler conjecture relies on nearly a thousand nonlinear inequalities.  These inequalities can
be expressed in terms of trigonometric and inverse trigonometric functions, the square root function, and arithmetic over
the real numbers.  The term \verb!the_nonlinear_inequalities! in HOL Light is the conjunction of these nonlinear
inequalities.  See Section~XX.
\item The combinatorial structure of each possible counterexample to the Kepler conjecture is encoded as a planar graph
satisfying a number of restrictive conditions.  Any graph satisfying these conditions is said to be {\it tame}.  A list of
all tame graphs up to isomorphism has been generated by an exhaustive computer search.  The formal statement  that every
tame graph is isomorphic to one of these cases can be expressed in HOL Light as 
\verb!import_tame_classification!.  See Section~XX.
\item The final body of computer code  is a large collection of linear programs.  The results have
been formally specified as \verb!linear_programming_results! in HOL Light.
This will be discussed in Section~XX.
\end{enumerate}

It is then natural to break the formal proof of the Kepler conjecture
into four parts: the formalization of the text part (that is, the
traditional non-computer portions of the proof), and the three
separate bodies of computer calculations.  Because of the size of the
formal proof, the full proof of the Kepler conjecture has not been
obtained in a single session of HOL Light.  What we formalize in a
single session is a theorem

\begin{obeylines}

\begin{verbatim}
|-  the_nonlinear_inequalities /\
    import_tame_classification
    ==> the_kepler_conjecture
\end{verbatim}

\end{obeylines}

This theorem represents the formalization of two of the four parts of
the proof: the text part of the proof and the linear programming.  It
leaves the other two parts (nonlinear inequalities and tame
classification) as assumptions.  The formal proof of
\verb!the_nonlinear_inequalities! was obtained by dividing the
computation among numerous computers and recombining the results, as
described in Section~XX.  The formal proof of
\verb!import_tame_classification! was obtained in the Isabelle proof
assisant and translated into a term in HOL Light, as described in
Section~XX.  Thus, combining all of these results from various
sessions of HOL Light and Isabelle, we have obtained a formalization
of every part of the proof of the Kepler conjecture.

\section{The nonlinear inequalities}

The term \verb!the_nonlinear_inequalities! is defined as a conjunction of
several hundred nonlinear inequalities. The domains of these
inequalities have been partitioned to create more than 23,000
inequalities. The verification of all nonlinear inequalities in HOL
Light on the Microsoft Azure cloud took approximately 5000
processor-hours. Almost all verifications were made in parallel with
32 cores, hence the real time is about 5000 / 32 = 156.25
hours. Nonlinear inequalities were verified with compiled versions of
HOL Light and the verification tool developed in Solovyev's 2012
thesis.

To check that no pieces were overlooked in the distribution of
inequalities to various cores, the pieces have been reassembled in a
specially modified version of HOL Light that allows the import of
theorems from other sessions of HOL light. In that version, we obtain
a formal proof of the theorem

\begin{verbatim}
|- the_nonlinear_inequalities
\end{verbatim}


\section{Checking the formalization}

A proof assistant largely cuts the mathematical referees out of the
verification process.  This is not to say that human oversight is no
longer needed.  Rather, the nature of the oversight is such that
specialized mathematical expertise is only needed for a matter of minutes
during the process.  The rest of the audit of a formal proof can be
performed by any trained user of the HOL Light and Isabelle proof
assistants.

The article \cite{XX-Adams} describes the steps involved in the
auditing of the formal proof.  The proofs scripts must be executed to
see that they produce the claimed theorem as output.  The definitions
must be examined to see that the meaning of the final theorem (the
Kepler conjecture) agrees with the common understanding of the
theorem.  In other words, did the right theorem get formalized?  Were
any unapproved axioms added to the system?  The formal proof system
itself should be audited to make sure there is no foul play in the
syntax, visual display, and underlying internals.  A fradulent user of
a proof assistant might ``exploit a flaw to get the project completed
on time or on budget.  In their review, the auditor must assume
malicious intent, rather than use arguments about the improbability of
innocent error.''

\subsection{mathematical definitions}

Of all the constants appearing in the statement of the Kepler
conjecture, $\pi$ is the most difficult to specify.  The following
theorem in HOL reduces the correctness of the definition of $\pi$
to the correctness of more basic definitions in real and natural
number arithmetic.

\begin{obeylines}

\begin{verbatim}
|- abs (pi / &4 - sum (0..n) (\i. (-- &1) pow i / &(2 * i + 1))) 
     <= &1 / &(2 * n + 3)
\end{verbatim}

\end{obeylines}

In familiar notation, this asserts the well-known approximation for $\pi$ coming from the conditionally convergent
series expansion of  $\op{atan}(1)$:
\[
   \left| \frac{\pi}{4} - \sum_{i=0}^n \frac { (-1)^i } { 2 i + 1} \right | \le \frac {1} {2 n + 3}.
\]
As each definition in the Kepler conjecture is unfolded, ultimately there are hundreds of constants involved.
Fortunately, if we are willing to trust that HOL Light libraries correctly define the basic logical operations (such as quantifiers and
standard boolean functions such as {\it and} and {\it or}),  set theoretic, and arithmetic for the natural and real numbers, then
there are very few additional definitions to be checked.  Errors in basic definition are quickly detected because a large
web of theorems constrain them.
The essential definitions (real absolute value, integer powers, the square root function, finite sums,
the Euclidean distance on $\ring{R}^3$,  the characterization of finite sets and their cardinalities) are listed in Appendix XX.

\subsection{Auditing a distributed formal proof}

This formal proof has several special features that call for more careful auditing.
The most serious issue is that the full formal proof was not obtained in a single session of HOL Light.
The audit should check that the statement of the tame classification theorem in Isabelle has been faithfully translated
into HOL Light.  (It seems to us that our greatest vulnerability to error lies in the hand translation of this statement from
Isabelle to HOL Light.)  In particular, the audit should check that the long list of tame graphs that is used in Isabelle is
identical to the list that is used in HOL Light.

The nonlinear inequalities were also combined from various sessions of HOL Light.  A specially modified version of 
HOL Light combines
these inequalities into a single theorem.  The auditor should check the design of this modification of HOL Light.




\section{Library development} 

As part of the Flyspeck project, various libraries of theorems have been development that may be useful in other formalization
projects.  These include the libraries for multivariate integral and differential calculus, extensions to the trigonometry
library including spherical trigonometry, list processing,
the properties of polyhedra,  and tools for the verification of linear programs and nonlinear
inequalities over the real numbers.

Numerous other research projects in the formal proofs have made use of the Flyspeck project in some way or have been inspired by
the needs of Flyspeck.  These projects include the automated translation of proofs between formal proof systems
[Obua-S], [McLaughlin], [Adams]; the refactoring of formal proofs [Adams]; machine learning applied to proofs and
proof automation [Urban-Kaliszyk]; a mechanism to execute trusted external arithmetic 
from Isabelle/HOL [Obua]; the visual presentation of proofs [X]; the verification of linear programs [X]; and the
verification of nonlinear inequalities [X].

\section{the text portion of the proof}

The original proof of the Kepler conjecture, as presented in 1998, consists of about 300 pages of text together with supplemental
computer verifications.    The formalization of the text part of the proof of the Kepler conjecture follows the blueprint presented
in the book {\it Dense Sphere Packings: A Blueprint for Formal Proofs} \cite{XX}.  This book modifies the original proof in several
significant ways to make it more suitable for formalization.   

In the modified proof,  topological results concerning planar graphs are replaced with combinatorial results about hypermaps.
(This is inspired by G. Gonthier's formal proof of the four-color theorem, which is also done in terms of hypermaps).
The modified proof is based on a different geometric partition of space than that originally used.  C. Marchal
introduced this partition and first observed its
relevance to the Kepler conjecture.  

The modified proof also arranges material according to a small number of significant concepts
(volume, trigonometry, hypermaps, fans, polyhedra and Voronoi partitions, and so forth), which
gave the organizational structure to the formal proof libraries.

The actual formalization follows the book {\it Dense Sphere Packings} rather closely.  In fact, the book and the formalization
happened together, with numerous revisions of the text in response to issues that arose in the formalization.  The TeX files
of the book label each definition and theorem with an identifier, 
consisting of a random string of seven letters.  For example, the definition of the dihedral angle (Def 2.66 of DSP) has been randomly
assigned the identifier YMHELNF. These identifiers remained
fixed over time, even as major reorganizations of the book took place, making it possible to maintain a tight  correlation between
 passages in the text with
the corresponding formal proof scripts.  The names of many of the computer files in the project are named according to
these identifier, linking each file to the theorem in the text it formalizes.

As the formalization of the text continued after the publication of the book, a few unpublished appendices were prepared
to describe remaining details of the formal proof project.  The TeX files for these appendices are publicly available.
The first of these appendices on hypermaps and planar graphs
gives details about how to relate the results of the Isabelle project on tame graphs with the material in HOL Light.
The presentation on hypermaps in this appendix differs from the presentation in Section 4.7.4 of DSP, which covers
roughly the same material.  
The second appendix describes the theorem OXLZLEZ (Theorem 6.93, the cell cluster inequality) of DSP.   The book DSP
does not prove this result, but states, ``The proof of this cell cluster inequality is a computer calculation, which is the most
delicate computer estimate in the book.  It reduces the cell cluster inequality to hundreds of nonlinear inequalities in at most
six variables.''    A third unpublished appendix gives further details about the Main Estimate (Section 7.4 of DSP).  Here again,
the appendix differs from the published account.  Other than these appendices, the differences between the published blueprint
and the actual formalization are relatively minor.




\section{combining HOL Light sessions}

Hash.


\section{Importing Flyspeck results from Isabelle}

The first major success of the Flyspeck project was the formalization of the classification of tame plane graphs.
In the original proof of the Kepler conjecture, this classification was done by computer, using software written by Hales
to generate planar graphs satisfying given properties.   This formalization project
thus involved the verification of computer code.  Work on the formal verification of the code was started by Gertrud Bauer,
and became the subject of her PhD thesis.  The work was completed by Bauer and Nipkow in \cite{XX}.

This work was done in the Isabelle/HOL proof assistant, while all the rest of the project has been carried out in HOL Light.
It seems that it would be feasible to translate the Isabelle code to HOL Light.  This would be a desirable project and would
unite the entire project under a single roof.  We have not carried this out.   We suggest that this would make a good graduate
student thesis for an industrious student.

There are reasons why the classification of the tame graphs cannot be done automatically imported from Isabelle to HOL Light
using readily available technology.   A tool that automates the import from Isabelle to HOL Light was written by
S. McLaughlin with precisely this application in minde \cite{XX}.  Unfortunately, this tool has not been maintained to make it
compatible with the most recent releases of Isabelle and HOL Light.   A more serious issue is that the proof in Isabelle/HOL uses
reflection:  the code is verified then reflection is used to produce a theorem asserting the output of execution of the code as
a theorem.   The HOL Light kernel does not permit reflection.  Thus, the reflected portions of the formal proof would have to be
modified as part of the import.

Instead, we leave the formalization of the Kepler conjecture distributed between two different proof assistants.
In HOL Light, the Isabelle work appears as an assumption
\verb!import_tame_classification!.  This constant is 


\begin{obeylines}

\begin{verbatim}
|- import_tame_classification <=>
     (!g. g IN PlaneGraphs /\ tame g 
        ==> fgraph g IN_simeq archive)
\end{verbatim}

\end{obeylines}

which was translated from the Isabelle (stripped of its usual pretty printing):

\begin{obeylines}

\begin{verbatim}
|- "g \<in> PlaneGraphs" and "tame g" 
       shows "fgraph g \<in>\<^isub>\<simeq> Archive"
\end{verbatim}

\end{obeylines}


In informal terms, the assumption asserts that every tame graph is isomorphic to a graph appearing in a certain
long explict list (the archive) of graphs.  All of the HOL terms \verb!PlaneGraphs!, \verb!tame!, \verb!archive!,
\verb!iso_fgraph!, \verb!fgraph! are verbatim translations of the corresponding definitions in Isabelle (extended
recursively to the constants appearing in the definitions).  The types are similarly translated (lists to lists, natural numbers
to natural numbers, and so forth).  The archive of graphs is generated from the same ML file for
both the HOL Light and the Isabelle statements.  

It would be desirable (again this is a plea to an industrious graduate student) to have all of the formalization under a single roof.
The verbatim definitions between the two systems were translated by hand, and this becomes the single most important potential
source of an error in our work.  Also,  there is no guarantee that
the formal proofs in two different systems will remain compatible future versions of Isabelle and HOL Light.

Since the formal proof is distributed between two different systems with two different logics, we briefly indicate why
this theorem in Isabelle must also be a proof in HOL (assuming the consistency of Isabelle).  Briefly, this particular statement
can be expressed as a tautology in first-order propositional logic, and tautologies in first-order propositional logic that hold
in Isabelle also hold in HOL Light.  The point is that all quantifiers in the theorem are actually bounded and can thus be
expanded as finitely many cases in propositional logic.   Each of the finitely many possible graphs can be specified by a finite number of
boolean conditions.  Similarly, graph isomorphism can be replaced by an enumeration of finitely
many possible bijections of vertices.  
Tameness involves some conditions in integer arithmetic, but again everything is bounded, 
and arithmetic operations can be expanded into boolean gates using the usual tricks.    Thus, it is only a propositional tautology
that we share between systems.



\section{imported libraries from Coq}

Various libraries of theorems that were developed in Coq have turned out to be very useful for the Flyspeck project, 
especially in the final stages of Flyspeck.  This has been made possible by the package HOL-SSReflect developed by
 A. Solovyev as part of thesis \cite{XX}.  HOL-SSReflect is a proof scripting language modeled on the SSReflect 
package for Coq \cite{XX}.  Although the underlying logics of Coq and HOL Light are quite different, 
many of the proofs take place at a high level where the details about the underlying logic are abstracted away.  
In this way, the syntactic presentation of a proof in Coq can be nearly identical to the syntactic 
presentation of a proof in HOL Light.  Solovyev has been able to import entire libraries of results from 
Coq to HOL, making only minor changes in the proof scripts.  
Approximately 5\% of the proof scripts in the project have been written in SSReflect-HOL.

In particular, the ssreflect libraries for booleans (ssrbool), natural numbers (ssrnat), and sequences (ssreflect.seq.html)
have been imported to HOL light and
have been used extensively in our project.

See http://ssr2.msr-inria.inria.fr/doc/ssreflect-1.4/Ssreflect.seq.html

Our approach here is remarkably different from our approach to Isabelle, since the results from Coq have been fully
formalized within HOL Light. 


\section{Verifying nonlinear inequalities}

\section{Verifying linear programs}

\section{Requirements}

- How a user can run everything.
- Software requirements: MONO, Ocaml, HOL Light, Java, etc.


\section{On trustworthiness}

We have already discussed above the need for the reader to check the definitions involved in the statement of the theorem.

A frequent question has been whether mathematicians (and in particular, the referees of the original proof of the Kepler conjecture)
will finally accept the proof of the Kepler conjecture without reservation.  This misses the point of the formalization entirely.
The point of the formalization is to render their judgement irrelevant.  The proof verification aspects of the referees 
have been bypassed, replaced by technology
more reliable than a human referee has ever been.

This does not mean that human judgement is no longer needed.  What it means is that the element of human evaluation has shifted from
the logical correctness of a particular proof (in this case the Kepler conjecture) to the trustworthiness of the proof assistant
(in this case HOL Light) and the broader computer environment in which the software runs.


\section{Acknowledgements}

The various roles of project members appears in the statement announcing the completion of the project.

We wish to acknowledge the help, support, influence, and various contributions of the following individuals:
%Dang Tat Dat, 
%Hoang Le Truong,
Nguyen Duc Thinh,  
Nguyen Duc Tam, 
%Nguyen Tat Thang,
%Nguyen Quang Truong, 
%Ta Thi Hoai An, 
%Tran Nam Trung, 
%Trieu Thi Diep, 
%Vu Khac Ky, 
Vu Quang Thanh,
%Vuong Anh Quyen,
% 
%Mark Adams,
Catalin Anghel, 
Jeremy Avigad, 
Henk Barendregt,
%Gertrud Bauer, 
%
Herman Geuvers,
Georges Gonthier,
Daron Green,
Mary Johnston,
%Thomas Hales,
%John Harrison, 
%Cezary Kaliszyk,
%Victor Magron,
Christian Marchal,
Laurel %Beth 
Martin, 
%Sean McLaughlin, 
%Tobias Nipkow, 
%Steven Obua, 
%Joe Pleso, 
%
%Jason Rute,
Robert Solovay,
%Alexey Solovyev,
Erin Susick,
Dan Synek,
%Josef Urban,
Nicholas Volker, 
Matthew Wampler-Doty, 
Freek Wiedijk, 
Carl Witty,
Wenming Ye,
%and Roland Zumkeller.

We wish to thank the following sources of institutional support:
NSF grant 0503447 on the "Formal Foundations of Discrete Geometry" and NSF grant 0804189 on the "Formal Proof of the Kepler Conjecture", 
Microsoft Azure Research, William Benter Foundation, University of Pittsburgh, Radboud University, Institute of Math (VAST), VIASM.


\newpage
\section{Appendix on definitions}

The following theorem provides evidence that key definitions in the statement of the Kepler conjecture
are the expected ones.

\begin{obeylines}

\begin{verbatim}
|-  
// real absolute value:
   (&0 <= x ==> abs x = x) /\ (x < &0 ==> abs x = -- x) /\   

// powers:
    x pow 0 = &1 /\ x pow SUC n = x * x pow n /\

// square root:
   (&0 <= x ==> &0 <= sqrt x /\ sqrt x pow 2 = x) /\ 

// finite sums:
   sum (0..0) f = f 0 /\ sum (0..SUC n) f =  
     sum (0..n) f + f (SUC n) /\ 

// pi:
   abs (pi / &4 - sum (0..n) (\i. (-- &1) pow i / &(2 * i + 1))) 
     <= &1 / &(2 * n + 3) /\

// finite sets and their cardinalities:
   (A HAS_SIZE n <=> FINITE A /\ CARD A = n) /\
   {} HAS_SIZE 0 /\ {a} HAS_SIZE 1 /\ 
   (A HAS_SIZE m /\ B HAS_SIZE n /\ (A INTER B) HAS_SIZE p 
     ==> (A UNION B) HAS_SIZE (m+n - p)) /\

// bijection between R^3 and ordered triples of reals:
   triple_of_real3 o r3 = (\w:real#real#real. w) /\
   r3 o triple_of_real3 = (\v:real^3. v) /\ 

// the origin:
   vec 0 = r3(&0,&0,&0) /\

// the metric on R^3:
   dist(r3(x,y,z),r3(x',y',z')) = 
     sqrt((x - x') pow 2 + (y - y') pow 2 + (z - z') pow 2) /\

// a packing:
   (packing V <=> 
     (!u v. u IN V /\ v IN V /\ dist(u,v) < &2 ==> u = v))
\end{verbatim}

\end{obeylines}



References.