\begin{abstract}
This article describes a formal proof of the Kepler conjecture in a combination of the HOL Light and Isabelle proof
assistants.
This paper constitutes the official published account of the now completed Flyspeck project.
\end{abstract}


\section{Introduction}


In 1611, Kepler published the booklet {\it Six-Cornered Snowflake},
which contains a statement that has become known as the Kepler
conjecture.  The assertion is that no packing of congruent balls in
Euclidean three-space has density greater than that of the
face-centered cubic packing.  It is the oldest problem in discrete
geometry.  In 1900, Hilbert included the Kepler conjecture in his
influential list of mathematical problems.  In his book \cite{XX}, L.\
Fejes T\'oth gave a coherent proof strategy and later suggested that
computers might be used to study of the problem.  The truth of the
Kepler conjecture was established by Ferguson and Hales in 1998, but
their proof was not published in full until 2006~\cite{DCG}.

The delay in publication was caused by the difficulties of the
referees in verifying a complex computer proof.  In the end, the proof
was published without a complete certification of correctness from the
referees.  Many details about the review process for this theorem
appear in [Lagarias].  Lagarias eventually became the mathematician
the most actively involved in the checking of the original proof.  He
writes, ``The nature of this proof $\ldots$ makes it hard for humans
to check every step reliably. $\ldots$ [D]etailed checking of many
specific assertions found them to be essentially correct in every
case.  The result of the reviewing process produced in these reviewers
a strong degree of conviction of the essential correctness of this
proof approach, and that the reduction method led to nonlinear
programming problems of tractable size.''  We note the lack of
complete conviction.

At the Joint Math meetings in Baltimore in 2003(?), Hales announced a
project to give a formal proof of the Kepler conjecture, and later he
published a project description \cite{MAPS XX}.
%  A published announcement of the project later appeared in
%in [MAPS]. 
% {Introduction to the Flyspeck Project, MAPS 05021. http://drops.dagstuhl.de/opus/volltexte/2006/432}.
%That announcement states ``In truth, my motivations for the project are far more complex than a simple hope of 
%removing residual doubt from the minds of few referees. 
%Indeed, I see formal methods as fundamental to the long-term growth of mathematics.''
The project came to be called Flyspeck, an expansion of the the
acronym FPK, for the Formal Proof of the Kepler conjecture.  This
paper constitutes the official published account of the now completed
Flyspeck project.


The first definite contribution to the project was the formal
verification of a major piece of computer code that was used in the
proof, in work at TU Munich by Bauer under the direction of
T. Nipkow. (See section X.)  Major work on the project started when
NSF funded the project in 200(?).  An international conference on the
Flyspeck project and formal proofs sponsored by the NSF and the Hanoi
Math Institute in 200(?) transformed the project into a large
international collaboration.  The roles of the individual projet members
have been spelled out in the announcement of the proof~\cite{XX}.
The book ``Dense Sphere Packings''
describes the mathematical details of the proof that was formalized.
This article focuses on the formalization itself.



\section{The statement}

As mentioned in the introduction, 
the Kepler conjecture asserts that no packing of congruent balls in Euclidean three-space can have density exceeding
that of the face-centered cubic packing.  That density is $\pi/\sqrt{18}$, or approximately $0.74$.    The face-centered
cubic packing is not the only packing that realizes the bound.   The hexagonal-close packing and various packings
combining layers from the hexagaon-close packing and the face-centered cubic packing all achieve this same bound.
The density is insensitive to trivial modifications, such as removing a single ball from the packing.   Such modifications
will also achieve the maximum density.  The theorem that has been formalized does not make any
uniqueness claims.

The density of a packing is defined as a limit of the density obtained within finite containers, as the size of the container
tends to infinity.  To make the statement in the formal proof as simple as possible, we formalize a statement about the
density of a packing inside a finite spherical container.  This statement contains an error term.  The ratio
of the error term to the volume of the container  tends to zero as the volume of
the container tends to infinity.  Thus in the limit, we obtain the Kepler conjecture in its traditional form.


We will display various terms that are represented in HOL syntax.  For the convenience of the reader, appendix X lists
some of the syntactic conventions of HOL Light.  In particular, the universal quantifier $\forall$ is written as (!), the
existential quantifier is written (?),  the embedding of natural numbers into the real numbers is denoted (\&).



As a ratio of volumes, the density of a packing is scale invariant.   There is no loss of generality in assuming that
the balls in the packing are normalized
to have unit length.
We identify a packing of balls in $\ring{R}^3$ with the set $V$ of centers of the balls,
so that the distance between distinct elements of $V$ is at least $2$, the diameter of a ball.

More formally,

\begin{obeylines}

\begin{verbatim}

`(packing V <=> 
  (!u v. u IN V /\ v IN V /\ dist(u,v) < &2 ==> u = v))`

\end{verbatim}
\end{obeylines}
This definition states that $V$ is a packing if and only if for every $\u, \v \in V$, if
the distance from $\u$ to $\v$ is less than $2$, then $\u=\v$.

We define the constant {\tt the\_kepler\_conjecture} to be the term

\begin{obeylines}

\begin{verbatim}
`the_kepler_conjecture <=>
  (!V. packing V
    ==> (?c. !r. &1 <= r
        ==> &(CARD(V INTER ball(vec 0,r))) <=
            pi * r pow 3 / sqrt(&18) + c * r pow 2))`
\end{verbatim}
\end{obeylines}

In words, we define the Kepler conjecture to be the following claim:
for every packing $V$, there exists a real number $c$ such that for every real number $r\ge 1$, the number
of elements $V$ contained in an open spherical container of radius $r$ centered a the origin is at most
\[
  \frac{\pi\, r^3}{\sqrt{18}} + c\, r^2.
\]
An analysis of our proof of the theorem shows that there exists a small computable constant $c$ that works uniformly
for all packings $V$,
but we only formalize the weaker statement that allows $c$ to depend on $V$.
The restriction $r\ge 1$, which bounds $r$ away from $0$,
is needed because there can be arbitrarily small containers whose intersection with $V$ is nonempty.

 The proof of the Kepler conjecture relies on a combination of traditional mathematical argument
and three separate bodies of computer calculation.   The results of the computer calculations have been expressed
in precise mathematical terms and specified formally in HOL Light.  The computer calculations are as follows.
\begin{enumerate}
\item The proof of the Kepler conjecture relies on nearly a thousand nonlinear inequalities.  These inequalities can
be expressed in terms of trigonometric and inverse trigonometric functions, the square root function, and arithmetic over
the real numbers.  The term \verb!the_nonlinear_inequalities! in HOL Light is the conjunction of these nonlinear
inequalities.  See Section~XX.
\item The combinatorial structure of each possible counterexample to the Kepler conjecture is encoded as a planar graph
satisfying a number of restrictive conditions.  Any graph satisfying these conditions is said to be {\it tame}.  A list of
all tame graphs up to isomorphism has been generated by an exhaustive computer search.  The formal statement  that every
tame graph is isomorphic to one of these cases can be expressed in HOL Light as 
\verb!import_tame_classification!.  See Section~XX.
\item The final body of computer code  is a large collection of linear programs.  The results have
been formally specified as \verb!linear_programming_results! in HOL Light.
This will be discussed in Section~XX.
\end{enumerate}

It is then natural to break the formal proof of the Kepler conjecture
into four parts: the formalization of the text part (that is, the
traditional non-computer portions of the proof), and the three
separate bodies of computer calculations.  Because of the size of the
formal proof, the full proof of the Kepler conjecture has not been
obtained in a single session of HOL Light.  What we formalize in a
single session is a theorem

\begin{obeylines}

\begin{verbatim}
|-  the_nonlinear_inequalities /\
    import_tame_classification
    ==> the_kepler_conjecture
\end{verbatim}

\end{obeylines}

This theorem represents the formalization of two of the four parts of
the proof: the text part of the proof and the linear programming.  It
leaves the other two parts (nonlinear inequalities and tame
classification) as assumptions.  The formal proof of
\verb!the_nonlinear_inequalities! was obtained by dividing the
computation among numerous computers and recombining the results, as
described in Section~XX.  The formal proof of
\verb!import_tame_classification! was obtained in the Isabelle proof
assisant and translated into a term in HOL Light, as described in
Section~XX.  Thus, combining all of these results from various
sessions of HOL Light and Isabelle, we have obtained a formalization
of every part of the proof of the Kepler conjecture.



\section{The text formalization}


The next sections of this article turn to each of the four parts of the proof of the Kepler
conjecture, starting wih the text part of the formalization in this section.  In the remainder
of this article, we will call the proof of the Kepler conjecture
 as it appears in \cite{DCG} the {\it original proof}.
The formalization of the text part of the proof of the Kepler conjecture follows the plan presented
in the book {\it Dense Sphere Packings: A Blueprint for Formal Proofs} \cite{XX}.  
The proof that appears there will be called the {\it blueprint proof}.

The actual formalization follows the blueprint proof
rather closely.  In fact, the book and the formalization were developed
together, with numerous revisions of the text in response to issues
that arose in the formalization.


\subsection{differences between the original proof and the blueprint proof}

The blueprint proof follows the same general outline as the original
proof.  However, many changes have been made to make it more suitable
for formalization.  We list some of the primary differences between
the two proofs

\begin{enumerate}
\item In the blueprint proof, topological results concerning planar
  graphs are replaced with purely combinatorial results about
  hypermaps.  (This is inspired by G. Gonthier's formal proof of the
  four-color theorem, which is based on hypermaps).
\item The blueprint proof is based on a different geometric partition
  of space than that originally used.  Marchal introduced this
  partition and first observed its relevance to the Kepler conjecture.
  Visually, this partition of space looks similar to the partition
  used in the original proof.  However, it is described by a cleaner
  set of rules that are better adapted to formalization.
\item In a formal proof, every new concept comes at a cost: libraries
  of lemma must be developed to support the concept.  Therefore, we
  have organize the blueprint proof around a small number of major
  concepts such as trigonometry, volume, hypermap, fan, polyhedra, and
  Voronoi partitions.
\item The statements of the blueprint proof are more precise and take
  greater care than the original in making all hypotheses explicit.
\item To facilitate a large collaboration, the chapters of the blueprint have been made as
independent from one another as possible.
\item Long proofs have been broken up into a series of shorter lemmas.
\item The blueprint proof is more computational than the original.  In
  the original, every effort was made to prove as much as possible by
  hand.  Computer calculations were a last resort.  (This was done out
  of fear of negative reception to a computer proof.)  In the
  blueprint, the use of computer has been fully embraced.  As a
  result, many technical lemmas of the original proof can be
  eliminated.
\end{enumerate}


Because the original proof was not used for the formalization, we
cannot assert that the original proof has been formally verified to be
error free.  Similarly, we cannot assert that the computer code for
the original proof is free of bugs.  The detection and correction of
small errors is a routine part of any formalization project.  Overall,
hundreds of small errors in the proof of the Kepler conjecture were
corrected during formalization.  We claim is that the proof has been
transformed over a period of years into flawless form.

\subsection{appendices to the blueprint}

As part of the Flyspeck project, the book \cite{DSP} has been
supplemented with $84$ pages of unpublished appendices that give
additional details about the the blueprint proof \cite{XX}.

The first $34$ pages of those appendices
give details about how to formalize the main estimate~\cite[Sec~7.4]{DSP}.
The most serious error that was detected in the original text is described and corrected
in \cite{XX Revision}.  The error concerns the deformation arguments that appear in
the proof of the main estimate.  The main estimate is the most technically challenging
part of the original text, and its formalization was the most technically challenging part of the
blueprint text.    



A second major $14$-page  appendix is devoted to the proof of the
Cell Cluter Inequality~\cite[Thm~6.93]{DSP}.  (The theorem is stated below,
in Section~\ref{sec:rt}.)  In the blueprint text,
the proof is skipped, with the following comment: ``The proof of this
cell cluster inequality is a computer calculation, which is the most
delicate computer estimate in the book.  It reduces the cell cluster
inequality to hundreds of nonlinear inequalities in at most six
variables.''  



A final $19$-page appendix deals with the final integration of the various libraries in
 the project.  As a large collaborative effort that used two different proof assistants there
were many small differences between libraries that had to be reconciled to obtain a 
clean final theorem.   The most significant difference to reconcile was the notion
of planarity as used in classification of tame graphs and the notion of planarity 
that appears in the hypermap library.  Tame graph planarity is
defined by an algorithm that starts with a polygon and successively adds loops to
it in a way that intuitively preserves planarity.  (The algorithm is an implementation
in code of a process of drawing a sequence of loops, each connected to the previous
without crossings.)  By contrast,  in the hypermap library, the Euler formula is turned
into a definition of hypermap planarity.  The appendix describes how to relate
the two notions.  This appendex can be viewed as an expanded version of
Section~4.7.4 of \cite{DSP}.


\subsection{relating text to formal proof scripts}\label{sec:rt}

Since the blueprint and formal proof were developed at the same time, the 
numbering of lemmas and theorems continued to change as project 
took shape.

We needed a stable system of links between passages in the blueprint
with corresponding passages in the formal proof scripts that was
preserved from one version to the next.  To establish these links, the
TeX files of the blueprint label each definition and theorem with an
identifier, consisting of a random string of seven letters.  The
formal proof scripts are marked with the same random string to link
the blueprint to the proof script.  These random strings have remained
fixed over time, even as major reorganizations of the book took place.
These random strings were stripped out of the published version of the
\cite{DCG}, but appear in the pdf documents.

For example, the Cell Cluster Inequality (Theorem 6.93 of the
blueprint) has been randomly assigned the identifier OXLZLEZ.


\begin{theorem*}[cell cluster inequality]\guid{OXLZLEZ} 
\label{lemma:cluster}
Let $\op{CL}(\ee)$ be any cell cluster of a critical edge $\ee$ in a
saturated packing $V$.  Then $\Gamma(\ee)\ge 0$.
\end{theorem*}


The file OXLZLEZ.hl in the formal proof project contains a thm with identifier
OXLZLEZ:

\begin{verbatim}
let OXLZLEZ = prove_by_refinement(
  `!V. pack_nonlinear_non_ox3q1h /\ ox3q1h /\ 
    packing V /\ saturated V   ==>
    cell_cluster_estimate_v1 V`,
    ...
    );;
\end{verbatim}


\begin{verbatim}
|- !V. cell_cluster_estimate_v1 V <=>
            (!e. cluster_gamma_v1 V e (cell_cluster V e) >= &0)
\end{verbatim}

Here \verb!pack_nonlinear_non_ox3q1h! and \verb!ox3q1h! are some of the
nonlinear inequalities.   The formal theorem asserts that if the appropriate nonlinear inequalities
hold, then every saturated packing satisfies the cell cluster estimate.

The work described in \cite{Urban-K} uses these links between the informal and
formal texts to create a hyperlinked html document that allows a user to click
on an informal text to view the corresponding formal script.



\section{Nonlinear Inequalities}

[XX Insert about 2 pages of summary of Solovyev's thesis.]

The term \verb!the_nonlinear_inequalities! is defined as a conjunction of
several hundred nonlinear inequalities. The domains of these
inequalities have been partitioned to create more than 23,000
inequalities. The verification of all nonlinear inequalities in HOL
Light on the Microsoft Azure cloud took approximately 5000
processor-hours. Almost all verifications were made in parallel with
32 cores, hence the real time is about 5000 / 32 = 156.25
hours. Nonlinear inequalities were verified with compiled versions of
HOL Light and the verification tool developed in Solovyev's 2012
thesis.

To check that no pieces were overlooked in the distribution of
inequalities to various cores, the pieces have been reassembled in a
specially modified version of HOL Light that allows the import of
theorems from other sessions of HOL light. In that version, we obtain
a formal proof of the theorem

\begin{verbatim}
|- the_nonlinear_inequalities
\end{verbatim}

\subsection{combining HOL Light sessions}

Hash.


\section{Tame Classification}

[XX insert about 2 pages of summary of Bauer-Nipkow project.]

The first major success of the Flyspeck project was the formalization
of the classification of tame plane graphs.  In the original proof of
the Kepler conjecture, this classification was done by computer, using
custom software to generate planar graphs satisfying given
properties.  This formalization project thus involved the verification
of computer code.  Work on the formal verification of the code was
started by Gertrud Bauer, and became the subject of her PhD thesis.
The work was completed by Bauer and Nipkow in \cite{XX}.


\subsection{importing flyspeck results from Isabelle}


This work was done in the Isabelle/HOL proof assistant, while all the
rest of the project has been carried out in HOL Light.  It seems that
it would be feasible to translate the Isabelle code to HOL Light to
have the entire project under the same roof, but this lies beyond the
scope of the Flyspeck project.

Current tools do not readily allow the automatic import of this result
from Isabelle to HOL Light.  A tool that automates the import from
Isabelle to HOL Light was written by S. McLaughlin with precisely this
application in minde \cite{XX}.  Unfortunately, this tool has not been
maintained.   A more serious issue is that the proof in
Isabelle uses computational reflection: 
the code is verified then translated into ML code, which is executed,
and its output is then used as a theorem in Isabelle.
The HOL Light kernel does not permit reflection.
Thus, the reflected portions of the formal proof would have to be
modified as part of the import.

Instead, we leave the formalization of the Kepler conjecture
distributed between two different proof assistants.  In HOL Light, the
Isabelle work appears as an assumption

\verb!import_tame_classification!.  This constant is 


\begin{obeylines}

\begin{verbatim}
|- import_tame_classification <=>
     (!g. g IN PlaneGraphs /\ tame g 
        ==> fgraph g IN_simeq archive)
\end{verbatim}

\end{obeylines}

which was translated from the Isabelle (stripped of its usual pretty printing):

\begin{obeylines}

\begin{verbatim}
|- "g \<in> PlaneGraphs" and "tame g" 
       shows "fgraph g \<in>\<^isub>\<simeq> Archive"
\end{verbatim}

\end{obeylines}


In informal terms, the assumption asserts that every tame graph is
isomorphic to a graph appearing in a certain long explict list (the
archive) of graphs.  All of the HOL terms \verb!PlaneGraphs!,
\verb!tame!, \verb!archive!, \verb!iso_fgraph!, \verb!fgraph! are
verbatim translations of the corresponding definitions in Isabelle
(extended recursively to the constants appearing in the definitions).
The types are similarly translated (lists to lists, natural numbers to
natural numbers, and so forth).  These definitions and types were
translated by hand.  The archive of graphs is generated from the same
ML file for both the HOL Light and the Isabelle statements.


Since the formal proof is distributed between two different systems
with two different logics, we briefly indicate why this theorem in
Isabelle must also be a proof in HOL (assuming the consistency of
Isabelle).  Briefly, this particular statement can be expressed as a
SAT problem in first-order propositional logic.  SAT problems 
translate directly between systems (and are satisfiable in one system
if and only if they are satisfiable in the other, assuming the consistency
of both systems).

In translating the classification theorem into a SAT problem,
 the point is that all quantifiers in the theorem are actually
bounded and can thus be expanded as finitely many cases in
propositional logic.  Each of the finitely many possible graphs can be
specified by a finite number of boolean conditions.  Similarly, graph
isomorphism can be replaced by an enumeration of finitely many
possible bijections of vertices.  Tameness involves some conditions in
integer arithmetic, but again everything is bounded, and a bounded arithmetic
table can be drawn up in boolean gates using the usual tricks.
Thus, at a theoretical level, there is no difficulty in sharing this
theorem between systems.

\section{Linear Programs}

[XX insert about 2 pages here on the linear programming part of the proof.]

\section{Checking the formalization}

A proof assistant largely cuts the mathematical referees out of the
verification process.  This is not to say that human oversight is no
longer needed.  Rather, the nature of the oversight is such that
specialized mathematical expertise is only needed for a matter of minutes
during the process.  The rest of the audit of a formal proof can be
performed by any trained user of the HOL Light and Isabelle proof
assistants.

The article \cite{XX-Adams} describes the steps involved in the
auditing of the formal proof.  The proofs scripts must be executed to
see that they produce the claimed theorem as output.  The definitions
must be examined to see that the meaning of the final theorem (the
Kepler conjecture) agrees with the common understanding of the
theorem.  In other words, did the right theorem get formalized?  Were
any unapproved axioms added to the system?  The formal proof system
itself should be audited to make sure there is no foul play in the
syntax, visual display, and underlying internals.  A fradulent user of
a proof assistant might ``exploit a flaw to get the project completed
on time or on budget.  In their review, the auditor must assume
malicious intent, rather than use arguments about the improbability of
innocent error.''

\subsection{mathematical definitions}

Of all the constants appearing in the statement of the Kepler
conjecture, $\pi$ is the most difficult to specify.  The following
theorem in HOL reduces the correctness of the definition of $\pi$
to the correctness of more basic definitions in real and natural
number arithmetic.

\begin{obeylines}

\begin{verbatim}
|- abs (pi / &4 - sum (0..n) (\i. (-- &1) pow i / &(2 * i + 1))) 
     <= &1 / &(2 * n + 3)
\end{verbatim}

\end{obeylines}

In familiar notation, this asserts the well-known approximation for
$\pi$ coming from the conditionally convergent series expansion of
$\op{atan}(1)$:

\[
   \left| \frac{\pi}{4} - \sum_{i=0}^n \frac { (-1)^i } { 2 i + 1} \right | \le \frac {1} {2 n + 3}.
\]

As each definition in the Kepler conjecture is unfolded, ultimately
there are hundreds of constants involved.  Fortunately, if we are
willing to trust that HOL Light libraries correctly define the basic
logical operations (such as quantifiers and standard boolean functions
such as {\it and} and {\it or}), set theoretic, and arithmetic for the
natural and real numbers, then there are very few additional
definitions to be checked.  Errors in basic definition are quickly
detected because a large web of theorems constrain them.  The
essential definitions (real absolute value, integer powers, the square
root function, finite sums, the Euclidean distance on $\ring{R}^3$,
the characterization of finite sets and their cardinalities) are
listed in Appendix XX.

\subsection{auditing a distributed formal proof}

This formal proof has several special features that call for more
careful auditing.  The most serious issue is that the full formal
proof was not obtained in a single session of HOL Light.  The audit
should check that the statement of the tame classification theorem in
Isabelle has been faithfully translated into HOL Light.  (It seems to
us that our greatest vulnerability to error lies in the hand
translation of this statement from Isabelle to HOL Light.)  In
particular, the audit should check that the long list of tame graphs
that is used in Isabelle is identical to the list that is used in HOL
Light.

The nonlinear inequalities were also combined from various sessions of
HOL Light.  A specially modified version of HOL Light combines these
inequalities into a single theorem.  The auditor should check the
design of this modification of HOL Light.


\section{Final Remarks}


\subsection{installing and running the project}
- How a user can run everything.
- Software requirements: MONO, Ocaml, HOL Light, Java, etc.



\subsection{extensions}

Numerous other research projects in the formal proofs have made use of the Flyspeck project in some way or have been inspired by
the needs of Flyspeck.  These projects include the automated translation of proofs between formal proof systems
[Obua-S], [McLaughlin], [Adams]; the refactoring of formal proofs [Adams]; machine learning applied to proofs and
proof automation [Urban-Kaliszyk]; a mechanism to execute trusted external arithmetic 
from Isabelle/HOL [Obua]; the visual presentation of proofs [X]; the verification of linear programs [X]; and the
verification of nonlinear inequalities [X].


\section{Acknowledgements}

The various roles of project members appears in the statement announcing the completion of the project.

We wish to acknowledge the help, support, influence, and various contributions of the following individuals:
%Dang Tat Dat, 
%Hoang Le Truong,
Nguyen Duc Thinh,  
Nguyen Duc Tam, 
%Nguyen Tat Thang,
%Nguyen Quang Truong, 
%Ta Thi Hoai An, 
%Tran Nam Trung, 
%Trieu Thi Diep, 
%Vu Khac Ky, 
Vu Quang Thanh,
%Vuong Anh Quyen,
% 
%Mark Adams,
Catalin Anghel, 
Jeremy Avigad, 
Henk Barendregt,
%Gertrud Bauer, 
%
Herman Geuvers,
Georges Gonthier,
Daron Green,
Mary Johnston,
%Thomas Hales,
%John Harrison, 
%Cezary Kaliszyk,
%Victor Magron,
Christian Marchal,
Laurel %Beth 
Martin, 
%Sean McLaughlin, 
%Tobias Nipkow, 
%Steven Obua, 
%Joe Pleso, 
%
%Jason Rute,
Robert Solovay,
%Alexey Solovyev,
Erin Susick,
Dan Synek,
%Josef Urban,
Nicholas Volker, 
Matthew Wampler-Doty, 
Freek Wiedijk, 
Carl Witty,
Wenming Ye,
%and Roland Zumkeller.

We wish to thank the following sources of institutional support:
NSF grant 0503447 on the "Formal Foundations of Discrete Geometry" and NSF grant 0804189 on the "Formal Proof of the Kepler Conjecture", 
Microsoft Azure Research, William Benter Foundation, University of Pittsburgh, Radboud University, Institute of Math (VAST), VIASM.


\newpage
\section{Appendix on definitions}

The following theorem provides evidence that key definitions in the statement of the Kepler conjecture
are the expected ones.

\begin{obeylines}

\begin{verbatim}
|-  
// real absolute value:
   (&0 <= x ==> abs x = x) /\ (x < &0 ==> abs x = -- x) /\   

// powers:
    x pow 0 = &1 /\ x pow SUC n = x * x pow n /\

// square root:
   (&0 <= x ==> &0 <= sqrt x /\ sqrt x pow 2 = x) /\ 

// finite sums:
   sum (0..0) f = f 0 /\ sum (0..SUC n) f =  
     sum (0..n) f + f (SUC n) /\ 

// pi:
   abs (pi / &4 - sum (0..n) (\i. (-- &1) pow i / &(2 * i + 1))) 
     <= &1 / &(2 * n + 3) /\

// finite sets and their cardinalities:
   (A HAS_SIZE n <=> FINITE A /\ CARD A = n) /\
   {} HAS_SIZE 0 /\ {a} HAS_SIZE 1 /\ 
   (A HAS_SIZE m /\ B HAS_SIZE n /\ (A INTER B) HAS_SIZE p 
     ==> (A UNION B) HAS_SIZE (m+n - p)) /\

// bijection between R^3 and ordered triples of reals:
   triple_of_real3 o r3 = (\w:real#real#real. w) /\
   r3 o triple_of_real3 = (\v:real^3. v) /\ 

// the origin:
   vec 0 = r3(&0,&0,&0) /\

// the metric on R^3:
   dist(r3(x,y,z),r3(x',y',z')) = 
     sqrt((x - x') pow 2 + (y - y') pow 2 + (z - z') pow 2) /\

// a packing:
   (packing V <=> 
     (!u v. u IN V /\ v IN V /\ dist(u,v) < &2 ==> u = v))
\end{verbatim}

\end{obeylines}



References.