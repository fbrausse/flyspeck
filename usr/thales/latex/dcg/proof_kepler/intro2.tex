%% INTRO TO FORM

The following papers give a proof of the Kepler conjecture, which
asserts that no packing of congruent balls in three dimensional
Euclidean space has density exceeding that of the face-centered
cubic packing.

%\def\historical{\Part~\ref{part:intro}}

A historical overview of the Kepler conjecture is found in the first
paper in this series. Since the history of this problem is treated
there, this paper does not go into the details of the extensive
literature on this problem. We mention that Hilbert included the
Kepler conjecture as part of his eighteenth problem \cite{hilbert}.
L. Fejes T\'oth was the first to formulate a plausible strategy for
a proof \cite{FT}. He also suggested that computers might play a
role in the solution of this problem.  The historical account also
discusses the development of some of the key concepts of this paper.

An expository account of the proof is contained in \cite{CH}.  A
general reference on sphere packings is \cite{CS}.  A general
discussion of the computer algorithms that are used in the proof can
be found in \cite{algorithm}.  Some speculations on the structure of
a second-generation proof can be found in \cite{arbeitstagung}.
Details of computer calculations can be found on the internet at
\cite{web}.

The first section of this paper gives the top level structure of the
proof of the Kepler conjecture.  The next two sections describe the
fundamental decompositions of space that are needed in the proof.
The first decomposition, which is called the $Q$-system, is a
collection of simplices that do not overlap. This decomposition was
originally inspired by the Delaunay decomposition of space.  The
other decomposition, which is called the $V$-cell decomposition, is
closely related to the Voronoi decomposition of space.  In the
following section, these two decompositions of space are combined
into geometrical objects called {\it decomposition stars}. The
decomposition star is the fundamental geometrical object in the
proof of the Kepler conjecture.

The final section of this paper, which was coauthored with Samuel P.
Ferguson, describes a particular nonlinear function  on the set of
all decomposition stars, called the scoring function.  The Kepler
conjecture reduces to an optimization problem involving this
nonlinear function on the set of all decomposition stars. This is an
optimization problem in a finite number of variables. The subsequent
papers (Papers~\ref{part:iii} -- \ref{part:tame}) solve that
optimization problem.

The choice of the particular scoring function to use was arrived at
jointly with Samuel P. Ferguson.  He has contributed to this project
in many important ways, including the results in
Section~\ref{sec:scoring}.

\smallskip

Some history of the proof and this paper is as follows.  The
original proof, as envisioned in 1994 and accomplished in 1998, was
divided into a five-step program. As a result, the original papers
were called ``Sphere Packings I,'' ``Sphere Packings II,'' and so
forth. The first two papers in the series were published in an
earlier volume of DCG. As it turned out, the fourth step ``Sphere
Packings IV'' is considerably more difficult than the other steps in
the program. It became clear that a single paper would not suffice,
and the fourth step of the proof was divided into two parts ``Sphere
Packings IV'' and ``Kepler Conjecture (Sphere Packings VI).'' Samuel
Ferguson's thesis ``Sphere Packings V'' solved one of the five major
steps in the proof. (Although ``Sphere Packings IV'' and ``Sphere
Packings VI'' belonged together, because of the numbering scheme,
Ferguson's theses ``Sphere Packings V'' was inserted between these
two papers.)

The proof that is contained in this volume is a rewritten version of
the proof. For historical reasons, the papers in this volume have
retained the original titles, but because of extensive revisions
over the past several years, the proof is no longer arranged
according to the five steps of the 1994 program.

In addition to the $5+1$ papers corresponding to the five steps of
the original program, there is the current paper. It has the
following origin.  In 1996, it became clear that progress on the
problem required some adjustments in the main nonlinear optimization
problem of ``Sphere Packings I'' and ``II.''  As the original 1996
manuscript put it, ``There are infinitely many scoring schemes that
should lead to a proof of the Kepler conjecture.  The problem is to
formulate the scheme that makes the Kepler conjecture as accessible
as possible'' \cite{reform}.
%% From `A reformulation of the Kepler Conjecture' by Thomas C. Hales
%% Version 1, dated 11/16/96.
The original purpose of this paper was to make some useful
improvements in the scoring function from ``Sphere Packings I'' and
``II'' and to make the changes in such a way that the main results
of those papers would still hold true.

Over the past years, this paper has grown considerably in scope to
the point that it is now lays the foundation for all of the papers
in the series.  In fact, all of the foundational material from
``Sphere Packings I,'' and ``II,'' and the 1998 preprint series has
been collected together in this article.  The scoring function is no
longer the same as the one presented in ``Sphere Packings I,'' and
``II.''  This paper adapts the relevant material from these earlier
papers to the current scoring function.   This paper has expanded to
the point that it is now possible to understand the entire proof of
the Kepler conjecture without reading ``I'' and ``II.''
