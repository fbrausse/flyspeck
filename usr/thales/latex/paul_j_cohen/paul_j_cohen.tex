% Paul J. Cohen: A student's tribute.
% Author: Thomas C. Hales
% Affiliation: University of Pittsburgh
% email: hales@pitt.edu
%
% latex format

% History.  File started November 29, 2008
% First draft completed Dec 2, 2008.
%% 

\documentclass{llncs}
\usepackage{verbatim}
\usepackage{graphicx}
\usepackage{amsfonts}
\usepackage{amscd}
\usepackage{amssymb}
\usepackage{alltt}
% automatically generate revision number by
% svn propset svn:keywords "LastChangedRevision" paul_j_cohen.tex
\def\svninfo{{\tt
  filename: paul\_j\_cohen.tex\hfill\break
  PDF generated from LaTeX sources on \today; \hfill\break
  Repository Root: https://flyspeck.googlecode.com/svn \hfill\break
  SVN $LastChangedRevision$
  }
  }
%-%

% Math notation.
\def\op#1{{\hbox{#1}}} 
\def\tc{\hbox{:}}
\newcommand{\ring}[1]{\mathbb{#1}}

% Flags


%%%%%%%%%%%%%%%%%%%%%%%%%%%%%%%%%%

\begin{document}

\title{My teacher Paul J. Cohen}
\author{Thomas C. Hales}
\institute{University of Pittsburgh\\
\email{hales@pitt.edu}}
\maketitle


\section*{}

A few other
students and I met weekly with Paul Cohen throughout my undergraduate
years at Stanford.  
During the Putnam season, these meetings became coaching sessions
for the competition.  In the off season, he taught us a vast amount of
mathematics.
To name just a few topics, Paul Cohen gave me my first significant lessons in
sheaf theory, Lie theory, Riemann surfaces, Hilbert spaces, 
homotopy, and homology.


Galois theory, which he had read from the original works
as a teenager at Stuyvesant, was one of his favorite subjects.  He even carried the analogy of field extensions into model theory: the adjunction of a generic set to a model of set theory is ``akin to a variable adjunction to a field;'' and he considered a set in the resulting extension
as a function of the adjoined set, just as a rational function is a function of
the adjoined variable in the theory of fields.  If an axiom is not satisfied, adjoin
a solution!
This is the man who guided me through basic Galois theory, adjoined $\sqrt{17}$ to the rationals to compute for me the Gauss sum
 $
 \sum_{i=1}^8 \zeta^{n^2}$  (with $\zeta^{17}=1$) in the construction of the $17$-gon, 
and derived for me Cardano's formula for the cubic.  


He had no patience for proofs that failed to dazzle.
Once he barged in on James MacGregor's lecture after peering through the window at a proof of H\"older's inequality that was not to his liking.  He took control of the blackboard,  gave his own proof, then went on his way.


He sparred with me on almost any topic.  If I conjugated German verbs, he coached my accent with his Yiddish.  If I studied the Schr\"odinger equation, he countered with the Dirac equation.  If I read John Stuart Mill's {\it Political Economy}, he argued the logical flaws of {\it Das Kapital}.



His influence on my life has been profound.
My admiration for him bordered on worship (and still does).
I was constantly aware of being in the presence of genius and hung onto every word of his,
thinking for days about proofs that came to him in a flash.
When I first arrived at Stanford, I was the
kid from Utah with a rather limited background.  When I left,  I imagined myself Paul Cohen's
youngest prot\'eg\'e.  Cohen's interests in the early 80s, especially his graduate courses on Lie theory and the trace formula, helped to shape
my decision to seek  Langlands as my graduate advisor.
More than any others, I have him and Bob Langlands to thank for my mathematical education.

\smallskip



As an undergraduate reading about Jacobi's power series in Hardy and Wright, Cohen dreamed there should be 
a decision procedure for power series.   Years later, a decision procedure took 
shape to decompose a domain (in a field of characteristic zero, complete under a discrete valuation) into 
cells on which elementary questions are trivially decided.
One of his original applications of the
cell decomposition was to prove that the truth of certain statements is independent of the characteristic of the underlying field, in the spirit of Ax and Kochen.
Motivic integration builds on Cohen's concept of cell decomposition in the work of Pas, Cluckers, Denef, and Loeser.  In motivic integration, the measure is first defined on cells, then shown to be independent of the decomposition of the domain into cells.  
Motivic integration develops the theme of field independence even further: as a corollary of general results about the field independence of $p$-adic integrals, it transfers Ng\^o's proof of the Fundamental Lemma  from positive characteristic to characteristic zero, where it has profound applications to the theory of automorphic representations.

Paul Cohen's proof of quantifier elimination for the elementary theory of the real numbers has been influential
in formal proofs.  The
Cohen-Hormander decision procedure has been implemented by
Harrison and McLaughlin
in the formal proof assistant HOL-Light.  There are other algorithms for
quantifier elimination that have faster execution, but none that can compete with his, in ease of implementation and in simplicity of formal expression. 


He held that great
mathematics is simple and can come in a flash.
He disliked the rise of ponderous research programs with multitudes contributing
small steps.
Trying to sum up his thought, I think of Zarathustra's aphorism on aphorisms, ``In the mountains, the shortest way is from peak to peak: but for that one must have long legs.''  His proofs are aphorisms that span some of math's most
majestic peaks.  

\bigskip
\noindent
\svninfo
\end{document}

